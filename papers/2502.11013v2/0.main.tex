%%
%% This is file `sample-sigconf.tex',
%% generated with the docstrip utility.
%%
%% The original source files were:
%%
%% samples.dtx  (with options: `sigconf')
%% 
%% IMPORTANT NOTICE:
%% 
%% For the copyright see the source file.
%% 
%% Any modified versions of this file must be renamed
%% with new filenames distinct from sample-sigconf.tex.
%% 
%% For distribution of the original source see the terms
%% for copying and modification in the file samples.dtx.
%% 
%% This generated file may be distributed as long as the
%% original source files, as listed above, are part of the
%% same distribution. (The sources need not necessarily be
%% in the same archive or directory.)
%%
%%
%% Commands for TeXCount
%TC:macro \cite [option:text,text]
%TC:macro \citep [option:text,text]
%TC:macro \citet [option:text,text]
%TC:envir table 0 1
%TC:envir table* 0 1
%TC:envir tabular [ignore] word
%TC:envir displaymath 0 word
%TC:envir math 0 word
%TC:envir comment 0 0f
%%
%%
%% The first command in your LaTeX source must be the \documentclass
%% command.
%%
%% For submission and review of your manuscript please change the
%% command to \documentclass[manuscript, screen, review]{acmart}.
%%
%% When submitting camera ready or to TAPS, please change the command
%% to \documentclass[sigconf]{acmart} or whichever template is required
%% for your publication.
%%
%%
\documentclass[sigconf]{acmart}
\settopmatter{printacmref=false}
\renewcommand\footnotetextcopyrightpermission[1]{}

\usepackage{hyperref}
\usepackage{siunitx} % 表格精度
\usepackage{dsfont} 
\usepackage{subfigure}
\usepackage{multirow}
\usepackage{threeparttable}
\usepackage{graphicx}
\usepackage{textcomp}
\usepackage{xcolor}
\usepackage{enumitem}
\usepackage{comment}
\usepackage{pifont}
\usepackage{amsthm}
%\usepackage{algorithmic}
\usepackage{algorithm}
\usepackage{amsmath}
\usepackage{bm}
\usepackage{bbding}
\usepackage{amsthm}
\usepackage{algpseudocode}
\usepackage{fontawesome}

%\usepackage{amssymb}
\usepackage{amsmath}
\usepackage{amsthm}



\newtheoremstyle{boldassumption}% 用于Assumption的样式
  {\topsep}% 空白的上方
  {\topsep}% 空白的下方
  {\itshape}% 字体样式
  {}% 缩进
  {\bfseries}% 标题字体样式
  {.}% 标题后的标点
  {.5em}% 标题和正文之间的空格
  {}% 设置定理的头部和正文的间距

\theoremstyle{boldassumption}
\newtheorem{assumption}{Assumption}

\newtheorem{definition}{Definition}

% \DeclareUnicodeCharacter{2217}{}

%\usepackage{algpseudocode}

%%
%% \BibTeX command to typeset BibTeX logo in the docs
\AtBeginDocument{%
  \providecommand\BibTeX{{%
    Bib\TeX}}}


\author{Zhi~Sheng, Yuan~Yuan, Yudi~Zhang, Depeng~Jin, Yong~Li}
\affiliation{
\institution{Tsinghua University}
\country{}
}
\email{y-yuan20@mails.tsinghua.edu.cn, liyong07@tsinghua.edu.cn}


% \renewcommand{\shortauthors}{Zhi Sheng, Yuan Yuan, Yudi Zhang, Depeng Jin, \& Yong Li}


%% Rights management information.  This information is sent to you
%% when you complete the rights form.  These commands have SAMPLE
%% values in them; it is your responsibility as an author to replace
%% the commands and values with those provided to you when you
%% complete the rights form.


% \copyrightyear{2024}
% \acmYear{2024}
% \setcopyright{rightsretained}
% \acmConference[KDD '24]{Proceedings of the 30th ACM SIGKDD Conference on Knowledge Discovery and Data Mining}{August 25--29, 2024}{Barcelona, Spain}
% \acmBooktitle{Proceedings of the 30th ACM SIGKDD Conference on Knowledge Discovery and Data Mining (KDD '24), August 25--29, 2024, Barcelona, Spain}\acmDOI{10.1145/3637528.3671662}
% \acmISBN{979-8-4007-0490-1/24/08}


%%
%% Submission ID.
%% Use this when submitting an article to a sponsored event. You'll
%% receive a unique submission ID from the organizers
%% of the event, and this ID should be used as the parameter to this command.
%%\acmSubmissionID{123-A56-BU3}

%%
%% For managing citations, it is recommended to use bibliography
%% files in BibTeX format.
%%
%% You can then either use BibTeX with the ACM-Reference-Format style,
%% or BibLaTeX with the acmnumeric or acmauthoryear sytles, that include
%% support for advanced citation of software artefact from the
%% biblatex-software package, also separately available on CTAN.
%%
%% Look at the sample-*-biblatex.tex files for templates showcasing
%% the biblatex styles.
%%

%%
%% The majority of ACM publications use numbered citations and
%% references.  The command \citestyle{authoryear} switches to the
%% "author year" style.
%%
%% If you are preparing content for an event
%% sponsored by ACM SIGGRAPH, you must use the "author year" style of
%% citations and references.
%% Uncommenting
%% the next command will enable that style.
%%\citestyle{acmauthoryear}

\makeatletter
\gdef\@copyrightpermission{
  \begin{minipage}{0.3\columnwidth}
   \href{https://creativecommons.org/licenses/by/4.0/}{\includegraphics[width=0.90\textwidth]{Figures/4ACM-CC-by-88x31.pdf}}
  \end{minipage}\hfill
  \begin{minipage}{0.7\columnwidth}
   \href{https://creativecommons.org/licenses/by/4.0/}{This work is licensed under a Creative Commons Attribution International 4.0 License.}
  \end{minipage}
  \vspace{5pt}
}
\makeatother
%


%% end of the preamble, start of the body of the document source.
\begin{document}

%%
%% The "title" command has an optional parameter,
%% allowing the author to define a "short title" to be used in page headers.
%\title{Residual Corrective Diffusion Modeling}
\title{Collaborative Deterministic-Diffusion Model for Probabilistic Urban Spatiotemporal Prediction}

%%
%% The "author" command and its associated commands are used to define
%% the authors and their affiliations.
%% Of note is the shared affiliation of the first two authors, and the
%% "authornote" and "authornotemark" commands
%% used to denote shared contribution to the research.

%%
%% By default, the full list of authors will be used in the page
%% headers. Often, this list is too long, and will overlap
%% other information printed in the page headers. This command allows
%% the author to define a more concise list
%% of authors' names for this purpose.

%%
%% The abstract is a short summary of the work to be presented in the
%% article.
\begin{abstract}
Accurate prediction of urban spatiotemporal dynamics is essential for enhancing urban management and decision-making.
Existing spatiotemporal prediction models are predominantly deterministic, focusing on primary spatiotemporal patterns. However, those dynamics are highly complex, exhibiting multi-modal distributions that are challenging for deterministic models to capture.
In this paper, we highlight the critical role of probabilistic prediction in capturing the uncertainties and complexities inherent in spatiotemporal data.
While mainstream probabilistic models can capture uncertainty, they struggle with accurately learning primary patterns and often suffer from computational inefficiency.
To address these challenges, we propose CoST, which collaborates deterministic and probabilistic models to improve both predictive accuracy and the ability to handle uncertainty.
To achieve this, we design a mean-residual decomposition framework, where the mean value is modeled by a deterministic model, and the residual variations are learned by a probabilistic model, specifically diffusion models.
Moreover, we introduce a scale-aware diffusion process, which better accounts for spatially heterogeneous dynamics across different regions.
Extensive experiments on eight real-world datasets demonstrate that CoST significantly outperforms existing methods in both deterministic and probabilistic metrics, achieving a 20\% improvement with low computational cost.
CoST bridges the gap between deterministic precision and probabilistic uncertainty, making a significant advancement in the field of urban spatiotemporal prediction.




\end{abstract}
% 

%%
%% The code below is generated by the tool at http://dl.acm.org/ccs.cfm.
%% Please copy and paste the code instead of the example below.
%%


% \begin{CCSXML}
% <ccs2012>
%    <concept>
%        <concept_id>10010147.10010257.10010293</concept_id>
%        <concept_desc>Computing methodologies~Machine learning approaches</concept_desc>
%        <concept_significance>500</concept_significance>
%        </concept>
%  </ccs2012>
% \end{CCSXML}

% \ccsdesc[500]{Computing methodologies~Machine learning approaches}


%%
%% Keywords. The author(s) should pick words that accurately describe
%% the work being presented. Separate the keywords with commas.
%\keywords{datasets, neural networks, gaze detection, text tagging}
%% A "teaser" image appears between the author and affiliation
%% information and the body of the document, and typically spans the
%% page.
\keywords{Probabilistic prediction, deterministic models, diffusion models}

%%
%% This command processes the author and affiliation and title
%% information and builds the first part of the formatted document.
\maketitle

\section{Introduction}
% Large Language Models~(LLMs) represent a transformative advancement in the field of language processing, demonstrating an unparalleled capacity for text generation and comprehension, which can be further applied in a wide variety of applications.  
% %Large language models (LLMs) have risen to prominence in various fields, offering endless possibilities for artificial intelligence applications. 
% Despite their significant prevalence in recent years, LLMs are frequently challenged with security and privacy issues, such as poor explainability~\cite{}, poor robustness~\cite{}, data dependency~\cite{}, etc. Among them, a specific and notable concern that has garnered increasing attention is the phenomenon of `hallucination', where models generate plausible but factually inaccurate or irrelevant content when employed for specific tasks such as problem-solving.  
% %In particular, the hallucination issue is when these large models are employed for problem-solving, users frequently voice concerns regarding being misled or deceived by the models' nonsensical and erratic outputs. 
% The tendency of these models to produce inaccurate outputs and fabricate facts has severely undermined the safety and usability of LLM applications, which calls for immediate attention in LLM research. 
% %Hallucination in large language models (LLMs) is a critical issue that needs immediate attention in LLM research. The tendency of these models to produce inaccurate outputs and fabricate facts has severely undermined the safety and usability of LLM applications. 
%exceptional 
%including limited explainability, compromised robustness, and a heavy reliance on data, each 
%However, d
Large Language Models (LLMs) have revolutionized language processing, demonstrating impressive text generation and comprehension capabilities with diverse applications. However, despite their growing use, LLMs face significant security and privacy challenges~\cite{siddiq2023generate, hou2023large, kaddour2023challenges, li2024model, 10.1145/3691620.3695510}, which affect their overall effectiveness and reliability. A critical issue is the phenomenon of \emph{hallucination}, where LLMs generate outputs that are coherent but factually incorrect or irrelevant. This tendency to produce misleading information compromises the safety and usability of LLM-based systems. This paper focuses on \emph{fact-conflicting hallucina}tion (FCH), the most prominent form of hallucination in LLMs. FCH occurs when LLMs generate content that directly contradicts established facts. For instance, as illustrated in \figref{fig:example1}, an LLM incorrectly asserts that ``\emph{Haruki Murakami won the Nobel Prize in Literature in 2016}'', whereas the fact is that ``\emph{Haruki Murakami has not won the Nobel Prize, though he has received numerous other literary awards}''. 
Such inaccuracies can significantly lead to user confusion and undermine the trust and reliability that are crucial for LLM applications.

% Large Language Models~(LLMs) have brought transformative advancements to language processing and beyond, showcasing text generation and comprehension abilities with wide-ranging applications. 
% Despite the increasing prevalence, LLMs face critical challenges in security and privacy aspects~\cite{siddiq2023generate, hou2023large, kaddour2023challenges}, heavily impacting their effectiveness and reliability. 
% One notable issue is the phenomenon of \emph{hallucination}, where LLMs produce coherent but factually inaccurate or irrelevant outputs during problem-solving. 
% Such a tendency to generate misleading information jeopardizes the safety and usability of LLM-based applications. 
% This paper concerns the \emph{fact-conflicting hallucination}~(FCH), which is the primary form of hallucinations in LLMs. 
% FCH occurs when LLMs generate content that directly contradicts the well-established facts, as exemplified in \figref{fig:example1}, where an LLM incorrectly believes 
% ``\emph{Haruki Murakami won the Nobel Prize in Literature in 2016}'', deviating from the fact that ``\emph{Haruki Murakami has not won the Nobel Prize but other numerous awards for his work in Literature}''. Such misinformation can cause significant user confusion and undermine the trust and reliability that are essential in various LLM applications. 

%correct answer of 

%is manifested by
%Such misinformation dissemination leads to significant user confusion, eroding the trust and reliability that are crucial in various LLM applications. 

%Large Language Models~(LLMs) represent a transformative advancement in the field of language processing, demonstrating an unparalleled capacity for text generation and comprehension, which can be further applied in a wide variety of applications. Despite their growing prevalence, LLMs encounter critical challenges, particularly in aspects of security and privacy. These include concerns such as limited explainability~\cite{}, compromised robustness~\cite{}, and heavy reliance on data~\cite{}, each posing distinct challenges to their efficacy and reliability. Among these, the phenomenon of ``hallucination'' stands out as a notable concern. This occurs when LLMs, while employed in tasks like problem-solving, generate outputs that are coherent yet factually inaccurate or irrelevant. Such a tendency to produce misleading information not only compromises the safety of LLM applications but also raises urgent questions regarding their usability. 

% Hallucinations in LLMs manifest in several distinct forms, each contributing differently to the challenges identified in their growing applications. 
% %The first, known as ``Input-conflicting hallucination'', arises when there is a discrepancy between the model's output and the user's initial input, reflecting a potential misunderstanding of the task at hand. On the other hand, ``Context-conflicting hallucination'' represents the second type, occurring when LLMs produce inconsistent responses in prolonged or multi-turn interactions, indicative of their limitations in maintaining coherent context. 
% Among the three types categorized in the literature~\cite{huang2023survey,zhang2023hallucination}, ``Fact-conflicting hallucination~(FCH)'' poses a particularly serious concern which is the primary focus of this paper. This phenomenon generates content in direct opposition to established factual knowledge. As illustrated in the example shown in Figure~\ref{fig:example1}, when an LLM was asked about the first person to walk on the moon, it incorrectly answered ``Charles Lindbergh in 1951'', a clear deviation from the factual answer of Neil Armstrong in 1969. This type of hallucination can lead to the dissemination of incorrect information and cause significant confusion among users, undermining the trust and reliability critical in various LLM applications. %Addressing fact-conflicting hallucinations is therefore essential for the advancement of LLMs, ensuring they not only function effectively but also responsibly in their diverse roles.


% According to \cite{huang2023survey} and \cite{zhang2023hallucination}, hallucinations in large language models can be categorized into types such as factual hallucinations and contextual hallucinations. Current benchmark assessments tend to focus on evaluating the propensity of LLMs to generate erroneous facts. The origin of these issues can be traced back to multiple deficiencies, including flaws in training data, training algorithms, and the inference process.

% \begin{figure}[t]
%     \centering
%     \includegraphics[width=0.95\linewidth]{fig/example1-cropped.pdf}\\
%     \caption{A Hallucination Output Example.}
%     %\vspace{-0.5cm}
%     \label{fig:example1}
% \end{figure}

\begin{figure}[t]
\centering
\vspace{3mm}
\hspace{-3mm}
\includegraphics[width=\linewidth]{fig/drowzee-example.pdf}
\\[0.5em]
\caption{A Hallucination Output Example}
\label{fig:example1}
\vspace{-4mm}
\end{figure}
%\lnk{Factual Hallucination and LLM inference current status}

Recent studies have introduced various methods to detect LLM hallucinations. A common approach involves developing specialized benchmarks, such as TruthfulQA~\cite{lin-etal-2022-truthfulqa}, HaluEval~\cite{HaluEval}, and KoLA~\cite{yu2023kola}, to assess hallucinations in tasks like question-answering, summarization, and knowledge graphs. 
While manually labeled datasets provide valuable insights, current methods often rely on simplistic or semi-automated techniques such as string matching, manual validation, or verification through another language model. These approaches reveal significant gaps in automatically and effectively detecting fact-conflicting hallucinations (FCH). 
The primary challenges in FCH detection arise from the lack of dedicated ground truth datasets, the absence of comprehensive test cases designed to trigger FCH, and the lack of a robust testing framework.  
Unlike other types of hallucinations, such as input-conflicting or context-conflicting hallucinations~\cite{ji-etal-2023-rho, shi2023large}, which can often be identified through semantic consistency checks, detecting FCH requires the verification of factual accuracy against external knowledge sources/databases. This process is particularly challenging and resource-intensive, especially for tasks that involve complex logical relationships~\cite{zhang2024fusion}. We identify three primary challenges in addressing this research gap:


% Recent studies have introduced a range of methods for detecting 
% hallucinations. One common approach involves creating comprehensive benchmarks tailored for this purpose. 
% Datasets such as TruthfulQA~\cite{lin-etal-2022-truthfulqa}, HaluEval~\cite{HaluEval}, and KoLA~\cite{yu2023kola} have been designed to evaluate hallucinations across different contexts, including question-answering, summarization, and knowledge graphs. 
% Despite the value of these manually labeled datasets, the current techniques heavily rely on naive and semi-automatic methods, such as string matching, manual validation, or utilizing another LLM for confirmation. 
% Therefore, there is a gap 
% in automatically and effectively testing FCHs, and the primary obstacle in testing FCH is the absence of dedicated ground truth datasets and an extensive testing framework.  
% Unlike other types of hallucinations, e.g., input-conflicting or context-conflicting 
% \cite{ji-etal-2023-rho, shi2023large}, 
% which can be identified through checks for semantic consistency, 
% detecting FCH
% requires the verification of the content's factual accuracy against external sources of knowledge or databases. This makes the process particularly arduous and resource-intensive, especially for tasks processing content with complex logical connections. 
% Here, we highlight three concrete challenges in filling up the identified research gap: 




%The main obstacle in testing for FCH is the absence of dedicated ground truth datasets and specific testing frameworks. Unlike other types of hallucinations~(e.g., input-conflicting and context-conflicting hallucinations, to be detailed in Section~\ref{subsec:cat}) which can be identified through checks for semantic consistency, FCH demands the verification of the content's factual accuracy against external sources of knowledge or databases. This requirement makes the process particularly challenging and resource-intensive, especially for tasks processing contents with inherent logical connections.

% \shil{(I feel the transition is not smooth, we first introducing datasets, and not explaining how they use these datasets to test llm. after these, we can state these methods are not automatic.)}


% To tackle FCH, recent works have developed various techniques for testing and detecting hallucination~\citep{yu2023kola,HaluEval}. The typical and intuitive solution is to develop comprehensive benchmarks for detection. This is done through a process of sampling, filtering, and enhancing ground-truth answers to identify the best and correct answers from given candidates. For example, a well-known hallucination evaluation benchmark HaluEval~\cite{HaluEval} constructs scenarios where LLMs are tested on their ability to select the most factually accurate answers from a set of provided options, with a focus on filtering out hallucinated responses. %\yi{ also talk about the construction of benchmark?}
% Additionally, human annotation plays a critical role in identifying hallucinations in LLM outputs~\cite{Alpaca}. This involves humans determining whether responses contain hallucinated information and considering aspects such as unverifiability, non-factuality, and irrelevance. 



% \lnk{Key challenge: lack of hallucination testing when faced with logic reasoning related problems}
%Bridging the identified research gap in the literature necessitates exploring the inherent challenges faced in detecting FCHs, which are crucial for advancing and enhancing the reliability of LLMs. 

\begin{enumerate}[itemsep=1mm, wide,  labelindent=9pt]
%[itemsep=0ex,leftmargin=0.35cm]
%Challenge\#1: 
%While these benchmarks effectively detect certain hallucinations, they 
\item {\textbf{Automatically constructing and updating benchmark datasets.}} Existing methodologies mainly rely on manually curated benchmarks for detecting specific hallucinations, which fail to encompass the broad and dynamic spectrum of fact-conflicting scenarios in LLMs. 
Meanwhile, due to the ever-evolving nature of knowledge, the need for frequent updates to benchmark data imposes a substantial and continuous maintenance effort.
The reliance on benchmark datasets thus restricts the FCH detection techniques' adaptability, scalability, and  %more importantly, 
detection capability;  
%Challenge\#2:
% in existing test cases. 
\item {\textbf{Efficiently generating FCH test cases.}}
LLMs often answer correctly to simple, straightforward questions due to their extensive training on vast datasets. However, to effectively assess their reasoning capabilities, it is important to generate more complex questions, such as those involving intricate temporal characteristics, that require reasoning rather than just recalling facts. However, constructing such test cases is non-trivial. The challenge lies in designing questions that use familiar knowledge but involve reasoning patterns the LLM may not have been explicitly trained on. Creating such test cases efficiently while ensuring they probe reasoning skills in ways the model has not previously encountered is essential to uncovering latent hallucinations;
% queries that involve temporal concepts, such as ``\emph{Does the human population finally reach six billion by the year 2000?}'' may often be used in applications. However, the correctness of the LLM outputs cannot be guaranteed, potentially leading to misleading information. Currently, there are no satisfactory approaches to automatically verify LLM outputs in such test cases; 
%errors even before the occurrence of large model hallucinations; 
%However, it is known that 
%Another critical issue lies in the verification of temporal logic in existing test cases. 
%It is well known that test cases involving temporal-related questions often face difficulties in automatically verifying the soundness and completeness of these issues. Consequently, the correctness of these test cases cannot be guaranteed, potentially introducing errors even before the occurrence of large model hallucinations;
%Challenge\#3: 
\item {\textbf{Validating the reasoning steps from LLM outputs.}} Even when LLMs finally produce correct answers, the outputs may not indicate an accurate reasoning process, potentially masking false understanding -- a source of FCH. Additionally, the quality of manual validation can differ based on human expertise. As a result, automatically validating reasoning processes, particularly those involving complex logical relationships, is inherently challenging. 
\vspace{1mm}
\end{enumerate}







% \lnk{Key challenge: factual knowledge exploring and new facts generation}
%\yi{we should focus on testing, addressing is a little bit vague.}
% The current research landscape in LLM presents a critical gap in automatically testing FCHs. Predominantly, existing methodologies are anchored to manual benchmarks. %\yi{this sentence is quite chinglish.}
% While these benchmarks are effective in detecting certain types of hallucinations, such as those in Figure~\ref{fig:example1}, they fall short in encompassing the broad and dynamic spectrum of fact-conflicting scenarios inherent to LLMs. %\yi{again, this sentence is not very clear}
% Meanwhile, the need for frequent updates to benchmark data, due to the ever-evolving nature of knowledge, imposes a significant and continuous maintenance effort.
% The reliance on benchmark datasets thus restricts the detection techniques’ adaptability, scalability, and worse, detection capability. 
% From a second perspective, the consistency in the quality of benchmark questions can vary, reflecting the differing levels of experience and skill among the human experts responsible for creating them. This is particularly reflected in the stages such as data labeling and results validation. Additionally, it is important to acknowledge that humans are prone to errors.
% %the scalability and the deof these existing methods are also significantly challenged by their dependency on extensive human intervention, particularly in stages such as data labeling and results validation. %This heavy reliance on manual efforts not only limits the scalability of such approaches but also questions their feasibility in efficiently handling the extensive and intricate datasets characteristic of LLMs.
% Thus, the development of more autonomous, agile, and scalable testing techniques is imperative to effectively identify and tackle FCHs in LLMs.%\yi{in this paper, we focus on testing, but until this paragraph, no terms about ``testing'' explicitly occur.}

% \lnk{Solution to Challenge1: comprehensive logic reasoning based testing framework}

% \lnk{Solution to Challenge2: wiki factual knowledge extraction and prolog rules inference for scalability.}
% \lnk{Key challenge: }

%\textbf{Our Work.}
%To address limitations in the existing techniques, 
%we are the first, to the best of our knowledge, to introduce 
To address the problems outlined above, this paper presents a novel automatic end-to-end metamorphic testing technique based on temporal logic for detecting FCH. To the best of our knowledge, we are the first to create a comprehensive FCH testing framework that utilizes factual knowledge reasoning and metamorphic testing, all seamlessly integrated into the fully automated tool, \tool. 

%\shil{(which four methods?)}
\tool begins by establishing a comprehensive factual knowledge base sourced through crawling information from accessible knowledge bases such as Wikipedia. Each piece of this knowledge acts as a ``seed'' for subsequent transformations. Leveraging logical operators to automatically generate temporal reasoning rules, we transform and augment these seeds and expand factual knowledge into a well-established set of question-answer pairs.
%\yi{into xx}. 
Using the questions and answers in the knowledge set as test cases and ground truth, respectively, we construct a reliable and robust FCH testing benchmark. 


The experiment uses a series of carefully designed template-based prompts to test for FCHs in LLMs. To thoroughly evaluate the reasoning behind the responses, we instruct the LLMs not only to generate answers to the test cases but also to provide detailed justifications for their answers. To reliably identify FCH, we introduce two semantic-aware, similarity-based metamorphic oracles. These oracles extract the key semantic elements from each sentence and map out the logical relationships between them. By comparing the logical and semantic structures of the LLM's responses with the ground truth, the oracles can detect FCH by identifying significant deviations in the LLM's answers from the correct information.




%well-crafted prompts\yi{how prompts generated?} to engage LLMs, testing the alignment of their generated content with our enhanced ground truth. Disparities between LLM outputs and the ground truth signal potential hallucinations. 
%Additionally, in our commitment to fostering collaborative research, we have released our constructed dataset as a benchmark~\cite{drowzee}.

%Our approach directly addresses the need for a comprehensive and flexible testing method by transforming structural factual data into a diverse range of scenarios that LLMs may encounter. This method not only improves the reliability of detection but also enhances its adaptability to various factual contexts.
%Furthermore, we address the scalability challenge by automating the transformation and enlargement of our knowledge base, significantly reducing the dependency on human effort. The well-designed prompts used to test LLMs further streamline the process, making it more efficient in identifying potential hallucinations by comparing LLM outputs with our extended ground truth.

%\textbf{Results and Findings.}
%In evaluating our proposed FCH testing framework and \tool, 
%we undertake 
%to evaluate their effectiveness 
We demonstrate the effectiveness of our approach through comprehensive experiments in multiple contexts. First, our evaluation involves deploying \tool across a wide range of topics drawn from a diverse selection of Wikipedia articles. Second, we test our framework on various open-source and commercial LLMs, thoroughly assessing its applicability and performance across different model architectures. 
Our key findings indicate that \tool succeeds in automatically generating practical test cases and identifying hallucination issues of nine LLMs across nine domains. 
Using these test sets, our experiments show that the rate of hallucination responses produced by various LLMs ranges from 24.7\% to 59.8\% for cases unrelated to temporal reasoning and 16.7\% to 39.2\% for cases requiring temporal reasoning. 
%\shil{shall we differentiate the number for non-temporal and temporal one?}.  
We then categorize these hallucination responses into \emph{erroneous knowledge hallucination} and \emph{erroneous inference hallucination}. 
%\syh{four types?}. 
Through an in-depth analysis, we unveil that the lack of logical reasoning capabilities contributes the most to the FCH issues in LLMs. 
Additionally, we observe that LLMs are particularly prone to generating hallucinations in test cases involving temporal concepts and out-of-distribution knowledge. 
Such an evaluation demonstrates that the 
%Furthermore, we confirm that 
test cases generated using %our 
logical reasoning rules can effectively trigger and detect LLM hallucinations.  %issues in . 


This paper builds upon the earlier version~\cite{DBLP:journals/pacmpl/LiL0SW024} by incorporating hallucination detection through temporal-logic-guided test generation. It includes additional motivational examples (\secref{sec:motivating}), a comprehensive set of reasoning rules for encoding \emph{Metric Temporal Logic} (MTL)~\cite{DBLP:conf/lics/OuaknineW05} formulae (\secref{sec:encoding_MTL}) and automatically generating temporal-logic-related question-answer pairs (\secref{prompt}), and further experimental studies (the {RQ4} at \secref{sec:eval}) that detect hallucinations due to insufficient temporal reasoning capabilities. The main contributions of this work are summarized as follows: 
%We summarize the main contributions of this paper below:
\begin{itemize}[itemsep=1mm,leftmargin=0.35cm]
\item 
%Development of 
\textbf{A novel FCH testing framework.} 
To the best of our knowledge, 
we are the first to develop a novel testing framework based on logic programming and metamorphic testing to automatically detect FCH issues in LLMs. %\yi{hanging sentence}This framework represents a significant advancement over current methodologies, providing a more systematic, comprehensive approach to detection.
%Construction and Release of
\item \textbf{An extensive benchmark based on factual knowledge.} 
To facilitate collaborative efforts and future advances in identifying FCH, 
the source code of \tool and constructed benchmark dataset are publicly available  \cite{drowzee}. 
\item \textbf{Test generation via temporal reasoning.} 
Our tool automatically generates test cases that provide a more comprehensive evaluation of LLMs in handling reasoning tasks and identifying factual inconsistencies. By applying temporal logic-based reasoning rules, we expand the initial seed data from our knowledge base, enhancing the diversity and complexity of the test scenarios. 

\item \textbf{Semantic-aware oracles for LLM answer validation.} We propose and implement two automated verification mechanisms, i.e., the oracles, that analyze the semantic structure similarity between sentences. These oracles are designed to validate the reasoning logic behind the answers generated by LLMs, hereby reliably detecting the occurrence of FCHs. 

\end{itemize}



\section{Related Works}
\label{sec:related_works}
% Our 





% Introduce several speech enhancement and dereverberation methods based on DNN and Bayesian inference.

% VAE-NMF~\cite{baby2021speech}

% DVAE-VEM~\cite{bie2022unsupervised}

RVAE-EM~\cite{wang2024rvae}
The cubical increase in computation with input length addresses a major limitation observed in our previous publication RVAE-EM~\cite{wang2024rvae}.
% StoRM and its reference

% Introduce RIR and T60 estimation 

% BUDDy

% \cite{mateljan2003comparison}

% Fourier analyzer

% MLS-based system~\cite{rife1989transfer}

% swept sine system~\cite{farina2000simultaneous}



% 加一节,怎么区分确定性和不确定性

\section{Methodology}


To achieve effective probabilistic predictions, we propose CoST that simultaneously leverages the advantages of both deterministic and probabilistic models. Our approach involves two stages. In the first stage, the deterministic model is pretrained to predict the conditional mean that captures the primary patterns. In the second stage, the parameters of the deterministic model are frozen, and the scale-aware diffusion model, constrained by a customized fluctuation scale, is jointly trained to model residual distributions that reflect random fluctuations.   
Figure~\ref{fig:model} illustrates an overview of our model.


\subsection{Mean-Residual Decomposition}

For urban spatiotemporal probabilistic prediction, current approaches typically employ a single probabilistic model to capture the full distribution of data, incorporating both the primary spatiotemporal patterns and the random fluctuations. However, it is challenging to model both of these distributions. Inspired by~\cite{mardani2023residual} and the Reynolds decomposition in fluid dynamics~\cite{pope2001turbulent}, we propose to decompose the target data \(\mathbf{x}^{ta}\) as follows:
\begin{equation}
 \mathbf{x}^{ta} = \underbrace{\mathbb{E}[\mathbf{x}^{ta}|\mathbf{x}^{co}]}_{\substack{:=\boldsymbol{\mu}(Deterministic)}} + \underbrace{(\mathbf{x}^{ta}-\mathbb{E}[\mathbf{x}^{ta}|\mathbf{x}^{co}])}_{\substack{:=\mathbf{r}(Probabilistic)}},
\end{equation}
where \(\boldsymbol{\mu}\) is the conditional mean representing the primary patterns, and \(\mathbf{r}\) is the residual representing the random variations. Our core idea is that if a deterministic model can accurately predict the conditional mean, that is, \(\boldsymbol{\mu}\approx\mathbb{E}_{\theta}[\mathbf{x}^{ta}|\mathbf{x}]\), then the probabilistic model only needs to focus on learning the simpler residual distribution, thus combining the strengths of both models to enhance the probabilistic prediction capability.









\subsection{Mean Prediction via Deterministic Model}

We require a deterministic model that can accurately predict the conditional mean to align with our research hypothesis, and also maintain high predictive efficiency to avoid additional increases in training and inference time. Therefore, we select the MLP-based STID model as our mean prediction module.
In the first stage of training, we pretrain the model for 50 epochs to effectively capture the primary spatiotemporal patterns. Specifically, we input historical conditional data \(\mathbf{x}^{co}\) into the STID model to obtain the conditional mean output \(\mathbb{E}_{\theta}[\mathbf{x}^{ta}|\mathbf{x}^{co}]\).

The STID model is pretrained by optimizing the following loss function:

\begin{equation}
\label{eq:loss2}
   \mathcal{L}_{2}  = \left\| \mathbb{E}_{\theta}[\mathbf{x}^{ta}|\mathbf{x}^{co}] - \mathbf{x}^{ta} \right\|_2^2 .
\end{equation}

\subsection{Residual Learning via Diffusion Model}
Diffusion models have achieved significant success in probabilistic modeling. In this work, we employ a diffusion model for probabilistic prediction, training it to learn the residual distribution:
\begin{equation}
\label{eq:one-setp-forward}
    \mathbf{r}_{ta}=\mathbf{x}^{ta}-\mathbb{E}_{\theta}[\mathbf{x}^{ta}|\mathbf{x}^{co}].
\end{equation}
Consequently, the target data \(\mathbf{x}^{ta}\) for diffusion models in Eqs.~\eqref{eq:one-setp-forward}, \eqref{eq:inference}, and \eqref{eq:loss1} is replaced by \(\mathbf{r}_{ta}\).
The residual distribution of urban spatiotemporal data is not independently and identically distributed (i.i.d.) nor does it follow a fixed distribution, such as \(\mathcal{N}(0, \mathbf{\sigma})\). Instead, it often exhibits complex spatiotemporal dependence and heterogeneity. So we consider both temporal residual learning and spatial residual learning. 




\subsubsection*{\textbf{Temporal Residual Learning.}} 
For urban spatiotemporal data, the residual distribution varies at different time points. For example, fluctuations are lower at night and higher during the day, with uncertainty varying between weekends and weekdays. To model this, we incorporate the timestamp information as the condition for the denoising process. Meanwhile, the residual distribution can also be affected by sudden weather changes or public events. To capture these real-time changes, we concatenate the context data $\mathbf{x}^{co}_0$ with noised target residual $\mathbf{r}^{ta}_n$ as the input. The noise is not added to $\mathbf{x}^{co}_0$ and $\mathbf{r}^{ta}_n$ during the diffusion training and inference.




\subsubsection*{\textbf{Spatial Residual Learning.}}
In areas with frequent traffic accidents, fluctuations tend to be more pronounced and may induce anomalous variations in adjacent regions, thus affecting their distributions.
For spatial dependence modeling, the model learns a spatial embedding for each location, following the STID approach. Additionally, we propose a scale-aware diffusion process to further distinguish the heterogeneity for different regions. In this section, we detail the calculation of \(\mathbf{Q}\) and how it is integrated into the scale-aware diffusion process.

\noindent\textbf{(i) Customized Fluctuation Scale.} Specifically, we apply the Fast Fourier Transform (FFT) to spatiotemporal sequences in the training set to quantify fluctuation levels in different regions and use the custom scale \(\mathbf{Q}\) as input to account for spatial differences in residual. Specifically, we first employ FFT to extract the fluctuation components for each region within the training set. The detailed steps are as follows:









\begin{equation}
    \begin{aligned}
    & \mathbf{A}_{\mathrm{k}} = \left| \text{FFT}(\mathbf{x})_\mathrm{k} \right|, \quad \mathbf{{\phi}}_{\mathrm{k}} = \mathbf{\phi} \left( \text{FFT}(\mathbf{x})_\mathrm{k} \right), \\
    & \mathbf{A}_{\text{max}}=\max_{\mathrm{k}\in\left\{1,\cdots,\left\lfloor\frac{\mathbf{L}}{2}\right\rfloor + 1\right\}}\mathbf{A}_{\mathrm{k}}, \\
    & \mathcal{K} = \left\{ \mathrm{k} \in \left\{ 1, \cdots, \left\lfloor \frac{{L}}{2} \right\rfloor + 1 \right\} : \mathbf{A}_{\mathrm{k}} < 0.1 \times \mathbf{A}_{\text{max}} \right\}, \\
    & \mathbf{x}_{\mathbf{r}}[i] = \sum_{\mathrm{k} \in \mathcal{K}} \mathbf{A}_{\mathrm{k}} \Big[ \cos \left( 2\pi \mathbf{f}_{\mathrm{k}} i + \mathbf{\phi}_{\mathrm{k}} \right) \\
    & \qquad \qquad + \cos \left( 2\pi \bar{\mathbf{f}}_{\mathrm{k}} i + \bar{\mathbf{\phi}}_{\mathrm{k}} \right) \Big],
    \end{aligned}
\end{equation}
where \(\mathbf{A}_{\mathrm{k}},\mathbf{\phi}_{\mathrm{k}}\) reprent the amplitude and phase of the $\mathrm{k}-$th frequency component. $L$ is the temporal length of the training set. \(\mathbf{A}_{\text{max}}\) is the maximum amplitude among the components, obtained using the $\max$ operator. $\mathcal{K}$ represents the set of indices for the selected residual components. \(\mathbf{f}_{\mathrm{k}}\) is the frequency of the \(\mathrm{k}\)-th component. $\bar{\mathbf{f}}_{\mathrm{k}}, \bar{\mathbf{\phi}}_{\mathrm{k}}$ represent the conjugate components. \(\mathbf{x}_{\mathbf{r}}\) ref to the extracted residual component of the training set. We then compute the variance $\sigma^2_k$ of the residual sequence for each location $k$ and expand it to match the shape as 
\(\mathbf{r}^{ta}_0 \in \mathbb{R}^{B \times K \times P}\) , where $B$ represents the batch size. And we can get the variance tensor \(\mathcal{M}\): 
\begin{equation}
\begin{aligned}
    &\mathcal{M}_{b,k,p}=\sigma_{k}^2,\\
&\forall b\in\{1,\cdots,B\}, \forall k\in\{1,\cdots,K\}, \forall p\in\{1,\cdots,P\}.
\end{aligned}
\end{equation}
The residual fluctuations are bidirectional, encompassing both positive and negative variations, so we generate a random sign tensor \(\mathbf{S}\in\mathbb{R}^{B\times K\times P}\) for \(\mathcal{M}\), where each element \(S_{b,k,p}\) of \(\mathbf{S}\) is sampled from a Bernoulli distribution with \(p = 0.5\). 
%That is, \(r_{b,k,p}\) takes the value $1$ with probability $0.5$ and $-1$ with probability $0.5$. 
The customized fluctuation scale \(\mathbf{Q}\) is then defined as:
\begin{equation}
\begin{aligned}
&\mathbf{Q}_{b,k,p}=S_{b,k,p}\times\mathcal{M}_{b,k,p},\\
&\forall b\in\{1,\cdots,B\}, \forall k\in\{1,\cdots,K\}, \forall p\in\{1,\cdots,P\}.
\end{aligned}
\end{equation}
Then \(\mathbf{Q}\) is used as the input of the denoising network. 





\noindent\textbf{(ii) Scale-aware Diffusion Process.}

The vanilla diffusion models typically model all regions as the same \(\mathcal{N}(0, I)\) distribution at the end of the diffusion process, failing to distinguish the differences among regions. To further model the differences of residual distribution across various regions, we adopt the technique proposed by~\cite{han2022card} to make the residual learning region-specific conditioned on \({\mathbf{Q}}\). Specially, we have calculated the customized fluctuation scale \({\mathbf{Q}}\), and We redefined the noise distribution at the endpoint of the diffusion process as follows:
\begin{equation}
    p(\mathbf{r}^{ta}_N)=\mathcal{N}({\mathbf{Q}},I),
\end{equation}
Accordingly, the Eq~\ref{eq:new one-step} in the forward process is rewritten as:
\begin{equation}
\label{eq:new one-step}
    \mathbf{r}_n^{ta} = \sqrt{\bar{\alpha}_n} \mathbf{r}_0^{ta}+(1-\sqrt{\bar{\alpha}_n})\mathbf{Q} + \sqrt{1 - \bar{\alpha}_n} \mathbf{\epsilon}, \quad \mathbf{\epsilon} \sim \mathcal{N}(0, I).
\end{equation}
And in the denoising process, we sample \(\mathbf{r}_N^{ta}\) from $\mathcal{N}({\mathbf{Q}},I)$, and denoise it use Eq~(\ref{eq:inference}), the computation of \(\mu_{\theta}(\mathbf{r}_n^{ta}, n| \mathbf{x}_0^{co})\) in Eq~\ref{eq:inference} is modified as:
\begin{equation}
\label{eq: mu}
    \mu_{\theta}(\mathbf{r}_n^{ta}, n| \mathbf{x}_0^{co})=\frac{1}{\sqrt{\bar{\alpha}_n}} \left( \mathbf{r}_n^{ta} - \frac{\beta_n}{\sqrt{1 - \bar{\alpha}_n}} \mathbf{\epsilon}_{\theta}(\mathbf{r}_n^{ta}, n| \mathbf{x}_0^{co}) \right)+(1-\frac{1}{\sqrt{\bar{\alpha}_n}})\mathbf{Q}.
\end{equation}
This approach enables the diffusion process to be governed by the \(\mathbf{Q}\) values at each region, leading to more effective utilization of the customized scale conditions.


\subsection{Training and Inference}
\begin{algorithm}
\caption{\methodname{} Training}
\KwIn{Coarse-to-fine Autoencoder $\text{Enc}$, $\text{Dec}$}
\KwOut{}
\For{$i \gets 1$ \textbf{to} $n-1$}{
    \For{$j \gets 1$ \textbf{to} $n-i$}{
        \If{$L[j] > L[j+1]$}{
            Swap $L[j]$ and $L[j+1]$
        }
    }
}
\Return $L$
\end{algorithm}
\begin{algorithm}[!t]
\caption{Inference}
\label{al: sampling}
\begin{algorithmic}[1]
    \State \textbf{Input:} Context data $\mathbf{x}_0^{co}$, customized fluctuation scale $\mathbf{Q}$, trained diffusion model $\epsilon_{\theta}$, trained deterministic model $\mathbb{E}_{\theta}$
    \State \textbf{Output:} Target data $\mathbf{x}_0^{ta}$
    \State Estimate the conditional mean \(\mathbb{E}_{\theta}[\mathbf{x}^{ta}_0|\mathbf{x}^{co}_0]\)
    \State Sample $\mathbf{r}_N^{ta}$ from $\epsilon \sim \mathcal{N}(\mathbf{S},I)$
    \For{$n = N$ to $1$} 
        \State Estimate the noise $\mathbf{\epsilon}_{\theta}(\mathbf{r}_n^{ta}, n| \mathbf{x}_0^{co})$
        \State Calculate the $\mu_{\theta}(\mathbf{r}_n^{ta}, n| \mathbf{x}_0^{co})$ using Eq.~(\ref{eq: mu})
        \State Sample $\mathbf{r}_{n-1}^{ta}$ using Eq.~(\ref{eq:inference})
    \EndFor
    \State \textbf{Return:} $\mathbf{x}_0^{ta}=\mathbb{E}_{\theta}[\mathbf{x}^{ta}_0|\mathbf{x}^{co}_0]+\mathbf{r}_0^{ta}$
\end{algorithmic}

\end{algorithm}

\subsubsection*{\textbf{Training}}
Our training process is a two-stage procedure. We first pretrain the deterministic model STID to enable it to predict the conditional mean. Subsequently, we train the diffusion mode to learn the distribution of residuals, where the residuals are calculated as the difference between the true values and the conditional mean predicted by the pretrained STID model with frozen parameters. The detailed training procedure is presented in Algorithm~\ref{al: train}.
\subsubsection*{\textbf{Inference}}
The inference process of the model consists of two paths: one utilizes the pretrained STID model to predict the conditional mean, while the other uses the diffusion model to predict the residuals. The final sample is obtained by summing the results of both paths. This process is detailed in Algorithm~\ref{al: sampling}.
\section{Experiments}\label{sec:exp}

\begin{table}[t]
\centering
\caption{\textbf{Quantitative results on OpenCompass~\cite{2023opencompass} multimodal leaderboard.}
$^{\ddag}$ denotes closed-source models. Hall denotes HallusionBench.
}
\label{tab:exp_it_oc}
\setlength{\tabcolsep}{1pt}
\begin{tabular}{l|c|c|cccccccc}
\toprule
Models   & Params & Avg. & MM- & MM- & MM- & Math- & Hall & AI2D  & OCR- & MMVet \\
   &  &  & Bench & Star & MU & Vista &  &  & Bench & \\
\midrule
Step-1o$^{\ddag}$   & N/A   & \textbf{77.7}  & 87.3  & 69.3  & 69.9 & 74.7  & 55.8 & 89.1 & 926 & \textbf{82.8}  \\
SenseNova$^{\ddag}$  & N/A   & 77.4  & 85.7  & \textbf{72.7}  & 69.6 & \textbf{78.4}  & 57.4 & 87.8 & 894 & 78.2  \\
InternVL2.5-78B-MPO~\cite{wang2024mpo}  & 78B  & 77.0   & 87.7  & 72.1  & 68.2  & 76.6  & 58.1  & 89.2 & 909 & 73.5  \\
Qwen2.5-VL-72B~\cite{bai2025qwen25vltechnicalreport}   & 73.4B  & 76.2  & \textbf{87.8}  & 71.1  & 67.9  & 70.8  & 58.8  & 88.2  & 881  & 76.7  \\
TeleMM$^{\ddag}$   & N/A   & 75.9  & 79.9 & 70.8 & 66.6 & 75.7  & \textbf{60.6}  & 88.5 & 891 & 75.7  \\
InternVL2.5-38B-MPO~\cite{wang2024mpo}  & 38B  & 75.3  & 85.4  & 70.1 & 63.8 & 73.6 & 59.7 & 87.9 & 894 & 72.6  \\
InternVL2.5-78B~\cite{chen2024expanding}  & 78B  & 75.2 & 87.5  & 69.5 & 70 & 71.4 & 57.4 & 89.1 & 853 & 71.8   \\
Qwen2-VL-72B~\cite{qwen2-vl_2024}   & 73.4B  & 74.8  & 85.9  & 68.6  & 64.3  & 69.7  & 58.7  & 88.3  & 888  & 73.9  \\
InternVL2.5-38B~\cite{chen2024expanding}  & 38B  & 73.5  & 85.4  & 68.5  & 64.6  & 72.4  & 57.9  & 87.6  & 841  & 67.2  \\
JT-VL-Chat-V3.0$^{\ddag}$  & N/A   & 73.4  & 81.7  & 67.5  & 59.3  & 71.9  & 53.9  & 87.2  & \textbf{967}  & 69.2  \\
Taiyi$^{\ddag}$  & N/A   & 73.0  & 84.8  & 69  & 60.4  & 72.3  & 56.8  & \textbf{90.8}  & 820  & 67.9  \\
Step-1.5V$^{\ddag}$  & N/A   & 72.5 & 82.0  & 65.1  & 61.2  & 69.7  & 54.3  & 87.5  & 886  & 71.3  \\
Gemini-1.5-Pro-002$^{\ddag}$~\cite{geminiteam2024gemini15unlockingmultimodal}   & N/A   & 72.1 & 82.8  & 67.1  & 68.6  & 67.8  & 55.9  & 83.3  & 770  & 74.6  \\
InternVL2.5-26B-MPO~\cite{wang2024mpo}  & 26B  & 72.1  & 84.2  & 67.7  & 56.4  & 71.5  & 52.4  & 86.2  & 905  & 68.1  \\
GPT-4o-20241120$^{\ddag}$~\cite{openai2024gpt4ocard}  & NA   & 72.0   & 84.3  & 65.1  & \textbf{70.7}  & 59.9  & 56.2  & 84.9  & 806  & 74.5  \\
LLaVA-OneVision-72B~\cite{li2024llavaonevision}  & 73B  & 68.0  & 84.5  & 65.8  & 56.6  & 68.4  & 47.9  & 86.2  & 741  & 60.6  \\
NVLM-D-72B~\cite{nvlm2024}   & 79.4B  & 67.6  & 78.5  & 63.7  & 60.8  & 63.9  & 49.7  & 80.1  & 849  & 58.9  \\
Molmo-72B~\cite{deitke2024molmo}  & 73.3B  & 64.1  & 79.5  & 63.3  & 52.8  & 55.8  & 46.6  & 83.4  & 701  & 61.1  \\
\rowcolor{Gray} \textbf{\method-72B}   & 71.8B  & 75.1  & 86.3  & 70.7  & 57.6  & 73.3  & 56.4  & 87.6  & 889   & 79.8  \\
\midrule
\multicolumn{11}{l}{\textit{Models smaller than 20B}} \\
\midrule
Ola-7b~\cite{ola_2025}   & 8.88B   & \textbf{72.6}  & \textbf{84.3}  & \textbf{70.8}  & \textbf{57.0}  & 68.4  & \textbf{53.5}  & \textbf{86.1}  & 822  & \textbf{78.6}  \\
Qwen2.5-VL-7B~\cite{bai2025qwen25vltechnicalreport}   & 8.29B   & 70.4  & 82.6  & 64.1  & 56.2  & 65.8  & 56.3  & 84.1  & 877  & 66.6  \\
InternVL2.5-8B-MPO~\cite{wang2024mpo}   & 8B   & 70.3  & 82  & 65.2  & 54.8  & 67.9  & 51.7  & 84.5  & \textbf{882}  & 68.1  \\
MiniCPM-o-2.6~\cite{yao2024minicpm}   & 8.67B   & 70.2  & 80.6  & 63.3  & 50.9  & \textbf{73.3}  & 51.1  & 86.1  & 889  & 67.2  \\
Ovis1.6-Gemma2-9B~\cite{lu2024ovis}  & 10.2B  & 68.8  & 80.5  & 62.9  & 55.0  & 67.2  & 52.2  & 84.4  & 830  & 65.0  \\
InternVL2.5-8B~\cite{chen2024expanding}   & 8B   & 68.1  & 82.5  & 63.2  & 56.2  & 64.5  & 49.0  & 84.6  & 821  & 62.8  \\
POINTS1.5-Qwen2.5-7B~\cite{points1.5_2024} & 8.3B   & 67.4  & 80.7  & 61.1  & 53.8  & 66.4  & 50.0  & 81.4  & 832  & 62.2  \\
Valley-Eagle$^{\ddag}$   & 8.9B   & 67.4  & 80.7  & 60.9  & \textbf{57.0}  & 64.6  & 48.0  & 82.5  & 842  & 61.3  \\
Qwen2-VL-7B~\cite{qwen2-vl_2024}  & 8B   & 67.0  & 81.0 & 60.7 & 53.7 & 61.4  & 50.4 & 83 & 843 & 61.8 \\
DeepSeek-VL2~\cite{wu2024deepseekvl2}   & 16.1B  & 66.4  & 81.2  & 61.0  & 50.7  & 59.4  & 51.5  & 84.5  & 825  & 60.0  \\
VITA-1.5~\cite{fu2025vita}   & 8.3B   & 63.3  & 76.8  & 60.2  & 52.6  & 66.2  & 44.6  & 79.2  & 741  & 52.7  \\
Baichuan-Omni~\cite{baichuan-omni}   & 7B   & -  & 75.6  & -  & 47.3  & 51.9  & 47.8  & -  & 700  & 65.4  \\
LLaVA-OneVision-7B~\cite{li2024llavaonevision}   & 8B   & 61.2  & 76.8  & 56.7  & 46.8  & 58.5  & 47.5  & 82.8  & 697  & 50.6  \\
Molmo-7B-D~\cite{deitke2024molmo}   & 8B   & 58.9  & 70.9  & 54.4  & 48.7  & 47.3  & 47.7  & 79.6  & 694  & 53.3  \\
% MiniCPM-o 2.6~\cite{yao2024minicpm}   & 8B   & 70.2  & 80.5  & 64.0  & 50.4  & 71.9  & 51.9  & 85.8  & 897  & 67.5  \\
\rowcolor{Gray} \textbf{\method-9B}  & 8.8B   & 69.7  & 80.7  & 60.5  & 51.2  & 68.3  & 51.8  & 84.5  & 883 & 72.3 \\
\bottomrule
\end{tabular}
\end{table}

\begin{table}[t]
  \caption{\textbf{Performance comparison on video and Interleave benchmarks} compared with existing approaches. $^*$ indicates officially released checkpoints evaluated by us. Best performance is marked \textbf{bold}. }
  \label{tab: video_n_interleave}
  \centering
  \setlength{\tabcolsep}{7.5pt}
  \begin{tabular}{lccccc}
    \toprule
       & \multicolumn{2}{c}{\textbf{VideoMME}} & \multicolumn{1}{c}{\textbf{MVBench}} & \multicolumn{2}{c}{\textbf{Llava-Interleave}}\\
    \cmidrule(r){2-3} \cmidrule(r){4-4} \cmidrule(r){5-6}
    Model & w/o subs & w subs & avg & in-domain & out-domain \\
    \midrule
     MiniCPM-V-2.6~\cite{yao2024minicpm} &  60.9 &  63.6 &  - &  - &  - \\
     LLaVA-OneVision-7B~\cite{li2024llavaonevision} &  58.2 &  - &  - &  - &  - \\
     Qwen2-VL-7B~\cite{qwen2-vl_2024} &  63.3 &  69.0 &  67.0 &  49.5$^*$ &  51.0$^*$ \\
     InternVL2-8B~\cite{chen2024far} &  56.3 & 59.3 &  65.8 &  - &  - \\
     VITA-1.5~\cite{fu2025vita} &  56.1 & 58.7 &  55.4 &  - &  - \\
     Baichuan-Omni~\cite{baichuan-omni} &  58.2 & - &  60.9 &  - &  - \\
     MiniCPM-o-2.6~\cite{yao2024minicpm} & 63.0$^*$ & 65.3$^*$ & 58.1$^*$ &  43.5$^*$ &  36.8$^*$ \\
     \rowcolor{Gray} \textbf{\method-9B} &  60.4  & 65.0 &  66.3 &  59.8 &  87.8 \\
    \midrule
    VideoLLaMA2-72B~\cite{cheng2024videollama2} & 61.4 & 63.1 & 62.0 & - & - \\
    LLaVA-OneVision-72B~\cite{li2024llavaonevision} &  66.2 &  69.5 &  59.4 &  - &  - \\
    Qwen2-VL-72B~\cite{qwen2-vl_2024} &  71.2 &  77.8 &  \textbf{73.6} &  - &  - \\
    InternVL2-Llama3-76B~\cite{chen2024far} &  64.7  & 67.8 &  69.6 &  - &  - \\
    \rowcolor{Gray} \textbf{\method-72B} &  65.2  & 67.7 &  69.6 &  \textbf{63.5} &  \textbf{89.9} \\
    \midrule
    GPT-4v~\cite{GPT4VisionSystemCard} & 59.9 & 63.3 & 43.7 & 39.2 & 57.78 \\
    GPT-4o-20240513~\cite{openai2024gpt4ocard} & 71.9 & 77.2 & - & - & - \\
    Gemini-1.5-Pro~\cite{geminiteam2024gemini15unlockingmultimodal} & \textbf{75.0} & \textbf{81.3} & - & - & - \\
    \bottomrule
\end{tabular}
\end{table}


In this section, we present a comprehensive evaluation of our \method model, comprising both quantitative and qualitative analyses of its performance. Furthermore, we conduct ablation studies to analyze the contributions of several key design components to the performance of our \method model, providing insights into their distinct impacts.

% In this section, we first evaluate the model’s performance on a variety of mainstream benchmarks, demonstrating the advantages of \method.
% Then, a series of qualitative results are presented to show the model’s specific capabilities, including multimodal understanding and free-form image generation.
% Finally, we conduct an ablation study to analyze several key components in \method.

\subsection{Quantitative Results}\label{subsec:exp_quantitative_results}

\subsubsection{Image-Text Understanding}
To evaluate the effectiveness of our \method in image-text understanding, we benchmark it against state-of-the-art MLLMs on the OpenCompass~\cite{2023opencompass} multimodal leaderboard, a widely recognized platform for multimodal evaluation. This leaderboard contains 8 different multimodal benchmarks, including complex VQA (MMBench~\cite{liu2025mmbench}, MMStar~\cite{chen2024we}, MMMU~\cite{yue2023mmmu}, AI2D~\cite{kembhavi2016diagram}, and MMVet~\cite{yu2024mm}), multimodal reasoning (MathVista~\cite{lu2024mathvista}), hallucination evaluation (Hallusionbench~\cite{Guan_2024_hallusionbench}), and OCR (OCRBench~\cite{Liu_2024}).
\cref{tab:exp_it_oc} shows the overall results. Our \method-72B model achieves top-tier performance on most benchmarks, surpassing closed-source models like GPT-4o and Gemini-1.5-Pro. Furthermore, our \method-9B model exhibits competitive performance among models of similar size, showcasing its robust capabilities in image-text understanding tasks.

% In this section, we compare our \method with leading MLLMs on the mainstream OpenCompass~\cite{2023opencompass} multimodal leaderboard to demonstrate its advancement on image-text understanding.
% This leaderboard contains 8 different multimodal benchmarks, including complex VQA (MMBench~\cite{liu2025mmbench}, MMStar~\cite{chen2024we}, MMMU~\cite{yue2023mmmu}, AI2D~\cite{kembhavi2016diagram}, and MMVet~\cite{yu2024mm}), multimodal reasoning (MathVista~\cite{lu2024mathvista}), hallucination evaluation (Hallusionbench~\cite{Guan_2024_hallusionbench}), and OCR (OCRBench~\cite{Liu_2024}).
% \cref{tab:exp_it_oc} shows the overall results.
% \method exhibits competitive performance compared with other MLLMs.
% Our \method-71B model achieves top-tier performance on most benchmarks.
% It outperforms closed-source models such as GPT-4o and Gemini-1.5-Pro.
% The \method-9B model also achieves competitive performance among other vision-language specific MLLM models smaller than 20B.
% Notably, it achieves excellent performance on MathVista, AI2D, and MMVet, demonstrating its comprehensive ability on multimodal reasoning and complex VQA.




\subsubsection{Video \& Interleaved Image-Text Understanding}

We evaluate our model's video and interleaved image-text understanding abilities on three mainstream benchmarks.

\textbf{Video-MME}~\cite{fu2024video}: Video-MME is a benchmark designed to evaluate MLLMs in full-spectrum video analysis. It encompasses a wide variety of video types across multiple domains and durations, featuring multimodal inputs such as video, subtitles, and audio. For this benchmark, testing is conducted with under 96 frames, and results are reported for both "with subtitles" and "without subtitles" settings.

\textbf{MVBench}~\cite{li2024mvbench}: MVBench serves as a video understanding benchmark aimed at thoroughly evaluating the temporal awareness of MLLMs in an open-world context. It includes 20 challenging video tasks that range from perception to cognition, which cannot be adequately addressed using a single frame. Testing for this benchmark utilizes dynamic sampling frames.

\textbf{LLaVA-Interleave Bench}~\cite{llava-next_2024}: LLaVA-Interleave Bench comprises a comprehensive suite of multi-image benchmarks collected from public datasets or generated via the GPT-4V API. It is created to assess the interleaved multi-image reasoning capabilities of MLLMs, with reported results for both "in-domain" and "out-domain" subsets.

As shown in Table~\ref{tab: video_n_interleave}, \method-9B achieves the second-best results across VideoMME and MVBench (outperformed only by Qwen2-VL-7B but requiring significantly fewer frames). However, the performance gains do not scale up to \method-72B due to limitations in the quantity of instruction-tuned video data. Moreover, both our \method-9B and \method-72B greatly surpass all other baselines in multi-image benchmarks, both in-domain and out-of-domain, highlighting their potential as strong competitors for complex tasks.


\subsubsection{Audio Understanding}

We evaluate our M2-omni model's audio understanding abilities on four mainstream benchmarks.

\textbf{Multilingual LibriSpeech (MLS)}~\cite{MLS_English}: The Multilingual LibriSpeech dataset is an extensive collection of read audiobooks sourced from Librivox, available in eight different languages. We utilize the English test set from this dataset to assess the model's speech comprehension capabilities. The latest version of this corpus comprises approximately 50,000 hours.

\textbf{Librispeech}~\cite{Librispeech}: The Librispeech corpus comprises approximately 1,000 hours of transcribed speech audio data derived from read English audiobooks. The entire dataset is categorized into three training sets (100 hours of clean, 360 hours of clean, and 500 hours of other), two validation sets (clean and other), and two test sets (clean and other). In this study, we assess our model's audio comprehension capabilities using both the clean and other testsets.

\textbf{Aishell1}~\cite{AISHELL1}:  The Aishell1 dataset comprises 178 hours of speech data, recorded by 400 speakers from various accent regions across China. It is organized into three subsets: a training set consisting of 340 speakers, a validation set with 40 speakers, and a test set featuring 20 speakers.

\textbf{AudioCaps}~\cite{AudioCaps}: AudioCaps is a comprehensive dataset featuring audio event descriptions specifically curated for the purpose of audio captioning. The sounds within this collection are derived from the AudioSet dataset. We utilize this dataset to assess the audio captioning capabilities of our \method.
 % To facilitate accurate captioning, annotators were supplied with audio tracks and corresponding categorical hints, with additional video hints provided as necessary.

The results are presented in Table~\ref{tab:exp_audio_understand}, and our \method-9B demonstrates competitive performance in speech recognition and audio captioning tasks. 
Specifically, our \method-9B is comparable to GPT-4o-Realtime~\cite{openai2024gpt4ocard}.
In addition, \method-9B significantly outperforms all other baselines on AudioCaps benchmarks, while achieving the second-best results for the MLS English, Librispeech other, Librispeech-clean and Aishell1 benchmarks.

\begin{table}[]
\centering
\caption{\textbf{Quantitative results on speech recognition and audio captioning.}
 $^*$ indicates results from \cite{yao2024minicpm}.
}
\label{tab:exp_audio_understand}
\setlength{\tabcolsep}{7pt}
\begin{tabular}{l|cccccccccc}
\toprule
Models   & MLS- & Librispeech- & Librispeech- & Aishell1 & AudioCaps \\
                & English & other & clean &  & \\
                & WER$\downarrow$ & WER$\downarrow$ & WER$\downarrow$ & WER$\downarrow$ & CIDER$\uparrow$ \\
\midrule
UIO2-L-1.1B~\cite{lu2023uio2}   & - & - & - & - & 45.7   \\
UIO2-XL-3.2B~\cite{lu2023uio2}  & - & - & - & - & 45.7   \\
UIO2-XXL-6.8B~\cite{lu2023uio2} & - & - & - & - & 48.9  \\
Whisper-large-v2~\cite{Whisper}  & \textbf{6.83} & \textbf{5.16} & 2.87 & - & - \\
Paraformer-cn~\cite{gao2022paraformer} & - & - & - & 2.12 & - \\
VITA-1.5~\cite{VITA_1.5} & - & 7.5 & 3.4 & 2.2 & - \\
Mini-Omini2~\cite{mini_omni2} & - & 9.8 & 4.8 & - & - \\
Freeze-Omini~\cite{Freeze_Omni} & - & 10.5 & 4.1 & 2.8 & - \\
MiniCPM-o-2.6~\cite{yao2024minicpm} & - & - & \textbf{1.7} & \textbf{1.6} & - \\
GPT-4o-Realtime~\cite{openai2024gpt4ocard} & - & - & 2.6$^*$ & 7.3$^*$ & - \\
\rowcolor{Gray} \textbf{\method-9B}   & 7.19 & 5.29 & 2.07 & 1.99 & \textbf{49.2} \\
\bottomrule
\end{tabular}
\end{table}

\begin{table}[t]
\centering
\caption{\textbf{Quantitative results on language benchmarks.} $^*$ indicates officially released checkpoints evaluated using the tools provided by OpenCompass~\cite{2023opencompass}.
}
\label{tab:exp_language}
\setlength{\tabcolsep}{5pt}
\begin{tabular}{cccccccc}
\hline
Tasks & MMLU & AGIEVAL & ARC-C & GPQA & MATH & HellaSwag & \begin{tabular}[c]{@{}l@{}}Avg.\\ Accuracy\end{tabular} \\ \hline
LLama3.1-8B & 69.4 & 41.2$^*$ & 83.4 & 30.4 & 51.9 & 75.1$^*$ & 58.6  \\
\rowcolor{Gray} \textbf{\method-9B} & 68.5 & 43.7 & 78.7 & 32.3 & 51.8 & 80.1 & 59.2  \\ \hline
\end{tabular}
\end{table}

\subsubsection{Audio Generation}
In this section, we also evaluated our model on the commonly-used test set: SEED-TTS test-zh. \textbf{SEED-TTS}~\cite{SEED_TTS} serves as an out-of-domain evaluation test set, comprising diverse input texts and reference speeches from various domains. We present the experimental results for \method-9B and the baseline models in Table~\ref{tab:exp_audio_generation}. As shown in Table~\ref{tab:exp_audio_generation}, our model outperforms MiniCPM-o-2.6~\cite{yao2024minicpm} in speech generation capability, achieving significant improvements in both evaluation metrics. However, our \method-9B still lags behind traditional vertical speech generation models, highlighting the need for further research and development to bridge this gap.


\subsubsection{Text-only Performance}
In this section, we assess the performance of our proposed \method-9B model and its initial counterpart, Llama3.1-8B~\cite{llama3_2024}. To evaluate the models' knowledge and examination capabilities, we employ a range of benchmarks, including AGIEVAL~\cite{zhong2023agievalhumancentricbenchmarkevaluating} and MMLU~\cite{hendrycks2021measuringmassivemultitasklanguage}. Furthermore, we utilize a diverse set of benchmarks to evaluate the models' multi-step problem-solving capabilities, including MATH~\cite{hendrycks2021measuringmathematicalproblemsolving} for mathematical derivation, HellaSwag~\cite{zellers2019hellaswagmachinereallyfinish} for commonsense reasoning in real-world contexts, ARC-C~\cite{allenai:arc} for scientific logical chains, and GPQA~\cite{rein2023gpqagraduatelevelgoogleproofqa} for critical analysis in expert-level domains. For all evaluation datasets, we adopt a generation-based assessment approach with greedy decoding.

Our experimental results, presented in \cref{tab:exp_language}, demonstrate that the performance of our proposed \method-9B model outperforms its initial counterpart, Llama3.1-8B across most evaluation datasets,   which is attributed to our multi-stage language preservation strategy and the high-quality instruction tuning data used in our training process.

% In this section, we evaluate the performance of our \method-9B and its initial Llama3.1~\cite{llama3_2024} models. To assess the models' knowledge and examination capabilities, we utilize the AGIEVAL~\cite{zhong2023agievalhumancentricbenchmarkevaluating},  MMLU~\cite{hendrycks2021measuringmassivemultitasklanguage} benchmarks. Additionally, we employ  MATH~\cite{hendrycks2021measuringmathematicalproblemsolving}, HellaSwag~\cite{zellers2019hellaswagmachinereallyfinish}, ARC-C~\cite{allenai:arc} and GPQA~\cite{rein2023gpqagraduatelevelgoogleproofqa} to evaluate the models' multi-step problem-solving ability, including mathematical derivation, commonsense reasoning in real-world contexts, scientific logical chains, and critical analysis in expert-level domains. For all evaluation datasets, we adopt a generation-based assessment approach with greedy decoding. The overall results are in \cref{tab:exp_language}.It can be observed that in most of the evaluation datasets, the performance of our \method-9B and Llama3.1~\cite{llama3_2024} models is comparable, maintaining their linguistic capabilities. Furthermore, in some rankings, our models exhibit superior performance in certain aspects compared to their text-only baseline models. This improvement is attributed to our multi-stage language preservation strategy and the high-quality instruction tuning data used in our training process.

\begin{table}[t]
\centering
\caption{
\textbf{Free-form dialogue generation evaluation results.}
}
% \vspace{3pt}
\setlength{\tabcolsep}{8pt}
\begin{tabular}{c|c|c|c}
\toprule
Model & Relevance & Fluency & Informativeness\\
\midrule
TextBind~\cite{li2023textbind} & 3.85 & 4.30 & 3.25\\
\rowcolor{Gray} \textbf{\method-9B} & 4.60 & 4.80 & 3.80\\
\bottomrule
\end{tabular}
\label{tab-model_freeform_results}
% \vspace{-12pt}
\end{table}




% For a evaluation of open-world multi-turn multimodal instruction following, we collect a test set comprising 50 conversations from realistic scenarios and utilize \method-9B to generate arbitrarily interleaved text and images in proper conversation contexts. For quantitative results, we ask GPT-4o~\cite{openai2024gpt4ocard} to rate each conversation ranging from 0 to 5 considering relevance, fluency and informativeness. We carry out our quantitative results against recent work TextBind~\cite{li2023textbind}. As shown in \cref{tab-model_freeform_results}, \method-9B exhibits overall better understanding and generating ability of multi-turn multimodal conversations. More qualitative cases can be found in \cref{fig-IT-Freeform-Result}.



\begin{table}[t]
  \caption{\textbf{Quantitative results on audio generation.} $^*$ indicates officially released checkpoints evaluated by us.}
  \label{tab:exp_audio_generation}
  \centering
  \setlength{\tabcolsep}{14pt}
  \begin{tabular}{lccccc}
    \toprule
       & \multicolumn{2}{c}{\textbf{SEED test-zh}}\\
    \cmidrule(r){2-3}
    Model & CER(\%)$\downarrow$ & SS$\uparrow$  \\
    \midrule

     Human & 1.26 &0.755 \\
     Vocoder Resyn. & 1.27 & 0.720 \\
     \midrule
     Seed-TTS~\cite{SEED_TTS} & 1.12 & 0.796 \\
     FireRedTTS~\cite{FireRedTTS} & 1.51 &0.635 \\
     MaskGCT~\cite{MaskGCT} & 2.27 & 0.774 \\
     E2-TTS(32 NFE)~\cite{E2_TTS} & 1.97 & 0.730 \\
     F5-TTS(32 NFE)~\cite{F5_TTS} & 1.56 & 0.741 \\
     CosyVoice~\cite{CosyVoice} &3.63 &0.723 \\
     CosyVoice2~\cite{CosyVoice2} &1.45 &0.748 \\
     CosyVoice2-S~\cite{CosyVoice2} &1.45 &0.753 \\
     CosyVoice2-S~\cite{CosyVoice2} &1.45 &0.753 \\
     \midrule
     MiniCPM-o-2.6~\cite{yao2024minicpm} &8.03$^*$ &0.474$^*$ \\
     \rowcolor{Gray} \textbf{\method-9B} &  6.36  & 0.604 \\
    \bottomrule
\end{tabular}
\end{table}


\subsubsection{User Experience Evaluation}\label{sec:human_evaluation}
\textbf{Evaluation Metric}:
Current benchmarks such as MMBench~\cite{liu2025mmbench}, MMStar~\cite{chen2024we}, and MMMU~\cite{yue2023mmmu} primarily focus on assessment through judgment-style questions. However, this assessment does not align with the users' actual interactive experience with MLLMs. To address this limitation, drawing inspiration from SuperclueV~\cite{supercluev}, we develop evaluation criteria specifically for assessing the models' performance on user experience, which contains four key dimensions: relevance, fluency, informativeness, and format rationality. \textit{Relevance} assesses the extent to which the model's responses align with both the provided prompts and the multimodal inputs.
\textit{Fluency} evaluates the naturalness, smoothness, clarity, comprehensibility, and anthropomorphic quality of the model's responses.
\textit{Informativeness} measures the extent to which the model's responses provide relevant information, knowledge, and analytical reasoning, enhancing their utility, detail, depth, and innovation.
\textit{Format rationality} examines the model's ability to adaptively generate appropriately structured and clear formats, for presenting results based on varying prompt types.



% Current benchmarks such as MMBench~\cite{liu2025mmbench}, MMStar~\cite{chen2024we}, and MMMU~\cite{yue2023mmmu} primarily focus on assessment through judgment-style questions. However, this assessment does not align with the users' actual interactive experience with MLLMs. Drawing inspiration from SuperclueV~\cite{supercluev}, we develop evaluation criteria specifically for assessing the models' experience performance, which contains four key dimensions: relevance, fluency, content richness, and format rationality. \textbf{Relevance} assesses the extent to which the model's responses align with both the provided prompts and the multi-modal inputs.
% \textbf{Fluency} evaluates the naturalness, smoothness, clarity, comprehensibility, and anthropomorphic quality of the model's responses.
% \textbf{Content richness} gauges the degree to which the model's responses are enriched with supplementary information, knowledge, and analytical reasoning, enhancing their utility, detail, depth, and innovation.
% \textbf{Format rationality} examines the model's ability to adaptively generate appropriately structured and clear formats for presenting results based on varying prompt types.


\begin{table}[t]
\centering
\caption{
\textbf{Detailed model experience evaluation standards.}
}
% \vspace{3pt}
\setlength{\tabcolsep}{4pt}
\begin{tabular}{c|c}
\toprule
Score & Description\\
\midrule
1 & Totally unsatisfied, totally unacceptable \\
2 & Basically not satisfied, with many obvious problems \\
3 & Generally satisfied, with a few obvious problems \\
4 & Basically satisfied, minor flaws allowed \\
5 & Completely satisfied, almost perfect \\
\bottomrule
\end{tabular}
\label{tab-model_expr_standards}
% \vspace{-12pt}
\end{table}

\textbf{Evaluation Dataset}: We collect chat samples from the actual users' multi-turn interaction dialogues, which cover a variety of tasks, including visual question answering (VQA), conversational interactions, chart interpretation, mathematical problem-solving, optical character recognition (OCR), and other related tasks. GPT-4o~\cite{openai2024gpt4ocard} is instructed to follow the evaluation criteria to generate initial reference answers for these collected samples. To ensure accuracy, human annotators refine the initial responses generated by GPT-4o. This process yields an evaluation dataset with nearly 300 samples, each with a corresponding ground truth.

We utilize GPT-4o to evaluate the model's responses against the ground truth, adhering to the standards outlined in  \cref{tab-model_expr_standards}.  As shown in \cref{tab-user_experience},  our M2-omni model, after undergoing  alignment tuning,  demonstrates an average increase of 5.7\%-23.4\% in user experience performance, which is further validated by human annotations on selected cases. Meanwhile, our model's performance on the OC benchmark across other modalities remains relatively consistent, thereby demonstrating the effectiveness of our unified training strategy, which integrates DPO and instruction tuning in the alignment tuning stage.

% We employ GPT-4o to score the models' responses compared with ground truth according to the standards of \cref{tab-model_expr_standards}.  \cref{tab-user_experience} shows the model after alignment tuning demonstrates an average increase of 5.7\% in performance. This enhancement is corroborated by human annotations on selected cases. Simultaneously, the general capabilities on OC benchmark across other modalities remain nearly the same, with a decrease in average evaluation scores of less than 1\%. This demostrates the effectiveness of our unified training strategy that integrates DPO and
% instruction tuning in alignment tuning stage.


\subsubsection{Free-Form Dialogue Generation}
To evaluate the open-world multi-turn multimodal instruction following capabilities of our model, we create a test set consisting of 50 conversations derived from realistic scenarios. We utilize \method-9B to generate arbitrarily interleaved text and images in proper conversation contexts.
For quantitative results, following our user experience evaluation metric, we employ GPT-4o to rate each conversation on a scale of 0 to 5 across three evaluation dimensions: relevance, fluency, and informativeness.
We carry out our quantitative results against recent work TextBind~\cite{li2023textbind}. As shown in \cref{tab-model_freeform_results}, \method-9B exhibits overall better understanding and generating ability of multi-turn multimodal conversations. More qualitative cases can be found in \cref{fig-IT-Freeform-Result}.





\begin{table}[t]
\centering\footnotesize
\caption{
\textbf{Detailed evaluation on user experience benchmark and OC benchmark. OC is short for the OpenCompass image-text understanding benchmark.}
}
% \vspace{3pt}
\setlength{\tabcolsep}{3pt}
\begin{tabular}{c|c|c|c|c|c|c}
\toprule
Model & Relevance & Fluency & Informativeness & Format Rationality & Expr. Avg($\Delta$\%) & OC Avg($\Delta$)\\
\midrule
\method-9B & 4.556 & 4.036 & 2.742 & 3.573 & 3.726 & -\\
\rowcolor{Gray} \method-9B-Align & 4.893 & 4.735 & 4.118 & 4.644 & 4.598(+23.4\%) & -0.3\\
\method-72B & 4.942 & 4.689 & 3.267 & 4.265 & 4.351 & -\\
\rowcolor{Gray} \method-72B-Align & 4.946 & 4.875 & 3.961 & 4.615 & 4.598(+5.7\%) & -0.2\\
InternVL2-26B~\cite{internvl_2024} & 4.886 & 4.76 & 4.15 & 4.52 & 4.577 & -\\
GPT-4o~\cite{openai2024gpt4ocard} & 5 & 4.878 & 3.854 & 4.831 & 4.64 & -\\
\bottomrule
\end{tabular}
\label{tab-user_experience}
% \vspace{-12pt}
\end{table}



\subsection{Qualitative Results}\label{subsec:exp_qualitative_results}

In this section, we qualitatively assess the capabilities of our \method, presenting examples of each modality and different tasks.

We show multimodal understanding abilities of our \method in \cref{fig-exp_case_all}. \method demonstrates promising capabilities in processing cross-modal problems, encompassing image understanding, video understanding, interleaved image-text understanding, and image-audio understanding. More examples can be found in the appendix, provided in \cref{subsec:appendix_cases}.

\cref{fig-IT-Freeform-Result} illustrates the model's ability to generate free-form dialogue, where our \method can create images based on the conversation context without explicit user input, useful for explaining ideas to users.




\begin{figure}[t]
    \centering
    \includegraphics[width=0.9\linewidth]{figures/case_exp.pdf}
    \caption{
    \textbf{Cases for multimodal understanding.}
    \method shows great potential to solve various multimodal problems.
    }
    \label{fig-exp_case_all}
\end{figure}




\begin{figure}[t]
    \centering
    \includegraphics[width=0.9\linewidth]{figures/free_form_gen.pdf}
    \caption{
    \textbf{Cases for Free-Form Dialogue Generation.}
    }
    \label{fig-IT-Freeform-Result}
\end{figure}


\subsection{Ablation Study}\label{subsec:exp_ablation}

\begin{table}[t]
\centering
\caption{\textbf{Ablation studies on step balancing strategy.} The loss weight setting [1,1,1] corresponds to the uniform weighting of the loss functions for image-text pairs, interleaved image-text, and video datasets.  * and \# represent the loss weight settings. * is obtained through experimental trials and parameter tuning. \# is obtained by normalizing the loss weights using the inverse of the loss at convergence, as described in Section \cref{subsubsec-Step Balancing Strategy}. We evaluate the few-shot performance on VQA tasks and the zero-shot performance on the captioning task of our pre-trained model.}
\label{tab:ablation_step_balance_pretrain}
\setlength{\tabcolsep}{4pt}
\begin{tabular}{c|c|ccc}
\toprule
\multicolumn{1}{l|}{Data Sample Balance} & Loss Weight Balance & \multicolumn{1}{l}{OK-VQA(4-shot)} & \multicolumn{1}{l}{VQAv2(4-shot)} & \multicolumn{1}{l}{Flickr30k(0-shot)} \\ \hline
Random Sample                        & {[}1,1,1{]}          & 40.5                             & 54.3                             & 87.0                                 \\
Round-robin                          & {[}1,1,1{]}          & 41.6                             & 54.4                             & 88.1                                 \\
Accumulation                         & {[}1,1,1{]}          & 41.7                             & 54.6                             & 88.2                                 \\
Accumulation                         & ${[}0.2,1.0,0.03{]}^{*}$   & 39.7                             & 52.5                             & 87.1                                 \\
Accumulation                         & ${[}0.45,0.36,1.09{]}^{\#}$ & \textbf{42.1}                             & \textbf{55.4}                             & \textbf{88.2}                                 \\
\bottomrule
\end{tabular}
\end{table}


In this section, we conduct ablation studies to investigate the effectiveness of our step balance strategy and dynamic adaptive balance strategy in our M2-omni model. These experiments aim to provide insights into the impact of these key components on our M2-omni’s performance.

\subsubsection{Step Balance Strategy}\label{subsubsec:step_balance_ablaton}

As described in \cref{subsubsec-Step Balancing Strategy} ,  we investigate the impact of various data sample balancing strategies and loss weight balancing schemes on the multimodal joint training stage of pre-training. We evaluate the performance of candidate strategies on two VQA benchmarks, OK-VQA~\cite{marino2019ok} and VQAv2~\cite{goyal2017making}, and assess its image captioning performance using the Flickr30k~\cite{young2014image} benchmark.

For pretrained models lacking in instruction following ability, to assess the effectiveness of our approach, we evaluate the performance of these models on VQA tasks using a few-shot approach and on image caption tasks using a zero-shot approach. \cref{tab:ablation_step_balance_pretrain} presents the results of our M2-omni pretrained models, which demonstrate the effectiveness of our step balance strategy.

% Besides, three task weighting manner are compared: [1,1,1], which means all data shares the same optimization step size; [0.2,1.0,0.03], which is consistent with that proposed in \cite{alayrac2022flamingo}; [0.45,0.36,1.09], the inverse of the loss at convergence state, as \cref{subsubsec-Step Balancing Strategy} described. Note that the three values in the ratio correspond to image-text pairs, interleaved image-text and video datasets.

%  We directly evaluate the pre-trained model's performance on VQA tasks using a few-shot approach and on image caption tasks using a zero-shot approach. For VQA tasks, we use two benchmarks: OK-VQA~\cite{marino2019ok} and VQAv2~\cite{goyal2017making}, while for image captioning, we use the Flickr30k~\cite{young2014image} benchmark. \cref{tab:ablation_step_balance_pretrain} shows the results of training models on the combined datasets using three different merging regimes. It can be observed that the accumulation strategies and setting the task weights to the inverse of the loss achieve the best performance.







\begin{table}[t]
\centering
\caption{
\textbf{Ablation results of the dynamic adaptive balance strategy}. Results for unimodal baselines are derived from the following single-modal models: \textsuperscript{$\dagger$} Image-Text Model, \textsuperscript{$\ddagger$} Video-Text Model, and \textsuperscript{
$\mathsection$} Audio-Text Model. The best result for each benchmark is \textbf{bolded}, while the best result for each model across all epochs is \underline{underlined}.
}
% \vspace{3pt}
\setlength{\tabcolsep}{3pt}
\begin{tabular}{l|l|ccccc|cc|cc}
\toprule
Models &  & MM- & OK- & VQAv2 & Text- & GQA & MSVD- & MSRVTT & Audio & MLS- \\
& & Bench & VQA &&VQA&& QA & QA & Caps & English($\downarrow$) \\
\midrule
\multirow{3}{*}{\makecell[l]{Single-modal\\Baselines}} & ep1 & 68.0\textsuperscript{$\dagger$} & 56.4\textsuperscript{$\dagger$} & 74.8\textsuperscript{$\dagger$} & \underline{70.4}\textsuperscript{$\dagger$} & 58.4\textsuperscript{$\dagger$} & 72.3\textsuperscript{$\ddagger$} & 59.3\textsuperscript{$\ddagger$} & 29.0\textsuperscript{$\mathsection$} & 11.4\textsuperscript{$\mathsection$} \\
& ep2 & \underline{\textbf{77.8}}\textsuperscript{$\dagger$} & \underline{59.8}\textsuperscript{$\dagger$} & \underline{76.9}\textsuperscript{$\dagger$} & 69.8\textsuperscript{$\dagger$} & 60.6\textsuperscript{$\dagger$} & \underline{\textbf{76.5}}\textsuperscript{$\ddagger$} & \underline{60.1}\textsuperscript{$\ddagger$} & \underline{39.9}\textsuperscript{$\mathsection$} & 9.33\textsuperscript{$\mathsection$} \\
& ep3 & 77.3\textsuperscript{$\dagger$} & 58.0\textsuperscript{$\dagger$} & 76.8\textsuperscript{$\dagger$} & 69.1\textsuperscript{$\dagger$} & \underline{60.8}\textsuperscript{$\dagger$} & 74.4\textsuperscript{$\ddagger$} & 58.6\textsuperscript{$\ddagger$} & 39.5\textsuperscript{$\mathsection$} & \underline{8.96}\textsuperscript{$\mathsection$} \\
\midrule
\multirow{3}{*}{\makecell[l]{Mixture \\w/o MM-Bal.}}
& ep1 & 70.5 & 55.9 & 75.7 & 70.2 & 57.7 & \underline{75.1} & \underline{59.6} & 27.5 & 12.1 \\
& ep2 & \underline{75.8} & \underline{58.8} & \underline{77.0} & \underline{70.5} & \underline{\textbf{61.1}} & 73.4 & 58.5 & 33.5 & 9.45 \\
& ep3 & 75.6 & 58.4 & 76.5 & 69.5 & 60.1 & 70.2 & 56.9 & \underline{39.6} & \underline{8.98} \\
\midrule
\multirow{3}{*}{\makecell[l]{Mixture \\w/ MM-Bal.}}
& ep1 & 74.7 & 59.6 & 76.0 & 71.2 & 59.0 & 73.1 & 58.7 & 35.5 & 9.27 \\
& ep2 & \underline{\textbf{77.8}} & \underline{\textbf{61.7}} & \underline{\textbf{77.2}} & \underline{\textbf{71.8}} & 60.5 & \underline{74.8} & 58.5 & 41.2 & 8.31 \\
& ep3 & 77.1 & 60.5 & 77.0 & 69.8 & \underline{60.7} & 74.6 & \underline{\textbf{60.2}} & \underline{\textbf{44.1}} & \underline{\textbf{8.04}} \\

\bottomrule
\end{tabular}
\label{tab-multi_task_balanced_ablation}
% \vspace{-12pt}
\end{table}

\subsubsection{Dynamic Adaptive Balance Strategy}

We conducted a evaluation of our dynamic adaptive balance strategy across text-image, video, and audio modalities using constrained datasets. The evaluation was conducted on benchmark datasets specific to each modality: for text-image tasks, MMbench~\cite{liu2025mmbench}, OK-VQA~\cite{marino2019ok}, VQAv2~\cite{goyal2017making}, TextVQA~\cite{singh2019towards}, and GQA~\cite{hudson2019gqa} were employed; for video, MSVD-QA~\cite{xu2017video} and MSRVTT-QA~\cite{xu2017video} benchmarks were utilized; and for audio, we assessed performance on the AudioCaps~\cite{kim2019audiocaps} (AAC) and MLS~\cite{Pratap2020MLSAL}-English (ASR) tasks. The experimental outcomes are detailed in Table~\ref{tab-multi_task_balanced_ablation}.

In contrast to actual training pipeline, our evaluation involved instruction tuning starting from pre-trained models. Specifically, for each modality, we initially trained single-modality baseline models (the 'Sinle-modal Baselines' in Table~\ref{tab-multi_task_balanced_ablation}) individually over three epochs to establish the maximum achievable performance per modality. The results indicate that optimal performance was predominantly observed by the second epoch. However, the ASR task, due to its more complex patterns, had not fully converged even by the third epoch. Subsequently, we combined data from all three modalities to train a unified model (the 'Mixture w/o MM-Bal.' in Table~\ref{tab-multi_task_balanced_ablation}). Under this multimodal training regimen, the image-text modality reached its optimal performance at the second epoch, while the video modality achieved peak performance as early as the first epoch and with performance consistently decreasing in subsequent epochs. In contrast, the audio modality demonstrated continuous improvement, attaining its best performance by the third epoch. These observations underscore the imbalance in training progress among different modalities when engaged in multimodal training.

To address this imbalance, we introduced the dynamic adaptive balance strategy within our M2-omni training framework. This strategy dynamically adjusts the loss weights for each modality based on their respective training progress. In the context of this evaluation, it accelerates the training of the audio modality while appropriately reducing the learning weights for the image-text and video modalities to prevent overfitting. The evaluation results for this balanced training approach are denoted as 'Mixture w/ MM-Bal.' in Table~\ref{tab-multi_task_balanced_ablation}. The results demonstrate that, although some degree of imbalance among modalities persists, the balanced training strategy significantly alleviates the issues observed with simple mixed training: optimal performances across benchmarks are now concentrated around the second and third epochs, and performance across all modalities has been markedly enhanced. Moreover, under the balanced training strategy, the model achieved single-modality optimal performance in 7 out of 9 benchmarks. The best-performing model (at epoch 2) surpassed the optimal performance of each single-modality baseline in 6 out of 9 benchmarks (MMBench, OK-VQA, VQAv2, TextVQA, AudioCaps, MLS-English). Additionally, for the audio modality, the model at epoch 3 outperformed the single-modality baselines in 5 out of 9 benchmarks (OK-VQA, VQAv2, MSRVTT-QA, AudioCaps, MLS-English), with significant improvements in audio performance. These experimental results highlight the effectiveness of our dynamic adaptive balance strategy.

\section{Conclusion}
In this paper, we introduced Atom of Thoughts (\our), a novel framework that transforms complex reasoning processes into a Markov process of atomic questions. By implementing a two-phase transition mechanism of decomposition and contraction, \our eliminates the need to maintain historical dependencies during reasoning, allowing models to focus computational resources on the current question state. Our extensive evaluation across diverse benchmarks demonstrates that \our serves effectively both as a standalone framework and as a plug-in enhancement for existing test-time scaling methods. These results validate \our's ability to enhance LLMs' reasoning capabilities while optimizing computational efficiency through its Markov-style approach to question decomposition and atomic state transitions.


% \section*{Acknowledgments}
% This work was supported in part by the National Key Research and Development Program of China under grant 2020YFA0711403  and the National Natural Science Foundation of China under 62171260 and 62272260.

%\clearpage

%% the bibliography file.
\bibliographystyle{ACM-Reference-Format}
\bibliography{8.reference}

% \begin{acks}
%  This work was supported in part by the National Key Research and Development Program of China under grant 2020YFA0711403, the National Nature Science Foundation of China under U22B2057, 61971267, and U1936217, and BNRist. 
% \end{acks}



%\clearpage

\appendix
\section*{Appendix}
\begin{table*}[h!]
\caption{The basic information of grid-based spatio-temporal data.}
\label{tbl:append_data}
\begin{threeparttable}
\resizebox{1.9\columnwidth}{!}{
\begin{tabular}{cccccccc}
\toprule
Dataset & City & Type & Temporal Period & Spatial partition & Interval & Mean & Std \\
\hline
TaxiBJ & Beijing & Taxi flow&  2014/03/01 - 2014/06/30 & $32 \times 32$ & Half an hour & 111.5 & 139.3 \\
BikeDC & Washington, D.C. & Bike flow&  2010/09/20 - 2010/10/20 & $20 \times 20$ & Half an hour & 0.924 & 4.88 \\



CellularSH & Shanghai & Cellular traffic &  2014/08/01 - 2014/08/21 & $32\times28$ & One hour & 0.175 & 0.212 \\
CellularNJ & Nanjing & Cellular traffic &  2021/02/02 - 2021/02/22 & $20\times28$ & One hour & 0.842 & 1.30 \\
CrowdBJ & Beijing & Crowd flow &  2018/01/01 - 2018/01/31 & $1010$ & One hour & 7.07 & 11.1 \\
CrowdBM & Baltimore & Crowd flow &  2019/01/01 - 2019/05/31 & $403$ & One hour & 14.4 & 29.3 \\
Los-Speed & Los Angeles & Traffic speed&  2012/03/01 - 2012/03/07 & $207$ & Five minutes & 59.0 & 12.5 \\

\bottomrule
\end{tabular}}
\end{threeparttable}
\end{table*}

% \begin{table*}[t!]
% \caption{The basic information of Graph-based spatio-temporal data.}
% \label{tbl:append_data_graph}
% \begin{threeparttable}
% \resizebox{1.8\columnwidth}{!}{
% \begin{tabular}{ccccccccc}
% \toprule
% Dataset & City & Type & Temporal Period & Interval & \#Nodes & \#Edges & Mean & Std \\
% \hline

% TrafficBJ & Beijing & Traffic speed & 2022/03/05 - 2022/04/05 & 15min& 13675& 24444& 6.837&  3.412\\
% TrafficSH & Shanghai & Traffic speed & 2022/01/27 - 2022/02/27 & 15min & 21099& 39065& 7.815&  4.044\\
% TrafficNJ & Nanjing & Traffic speed  & 2022/03/05 - 2022/04/05 & 15min & 13419& 25100& 6.699&  4.253\\

% \bottomrule
% \end{tabular}}
% \end{threeparttable}
% \end{table*}
\begin{table*}[h!]
\caption{Short-term prediction results on two additional datasets in terms of both deterministic and probabilistic metrics. \textbf{Bold} indicates the best performance, while \underline{underlining} denotes the second-best.}
\label{tbl:short1-app}
\begin{threeparttable}
% \resizebox{2.0\columnwidth}

\resizebox{1.5\columnwidth}{!}{
\begin{tabular}{ccccccccccc}
\toprule
\multirow{2}{*}{\textbf{Model}}
& \multicolumn{5}{c}{\textbf{CellularNJ}} & \multicolumn{5}{c}{\textbf{CrowdBM}}   \\
\cmidrule(lr){2-6} \cmidrule(lr){7-11} 
 &\textbf{MAE} & \textbf{RMSE}  &\textbf{CRPS} & \textbf{QICE} & \textbf{IS} & 
\textbf{MAE} & \textbf{RMSE} & \textbf{CRPS} & \textbf{QICE} & \textbf{IS} \\


\midrule
D3VAE& 0.580 & 1.135   &	0.565&	0.096&6.03	&	11.0&24.7	&	0.593& 0.110	&136.4\\


DiffSTG& 0.317& 0.649&	0.291&	0.071&3.11	&	8.88&21.3	&0.453&0.047	&68.5\\

TimeGrad& 0.340&  0.357  &	0.432&	0.162&5.87	&10.1	& \underline{12.4}	&\textbf{0.240}&\underline{0.085}	&\underline{46.9}\\


CSDI&0.129 & 0.237   &	\underline{0.111}&	 \underline{0.039}&	\underline{0.80}&	7.31& 19.3&	0.390& 0.054&61.1\\

NPDiff& \underline{0.123}&  \underline{0.175}  &0.128	&	0.133&2.22	&	\underline{5.42}& 13.7	&0.331&0.119	&91.2\\

DyDiffusion&0.222&   0.357 &0.196	&0.080	&	1.80&-	&-	&-	&-&-\\




\cmidrule(lr){1-1} \cmidrule(lr){2-6} \cmidrule(lr){7-11}
\textbf{CoST}&\textbf{0.102} &\textbf{0.172}    &\textbf{0.090}	&\textbf{0.037}	&\textbf{0.682}	&\textbf{5.04}	&\textbf{12.1}	&\underline{0.256}	&\textbf{0.027}& \textbf{37.8}\\
%\textbf{Reduction}& &    &	&	&	&	&	&	&\\
\bottomrule
\end{tabular}}
\end{threeparttable}
\end{table*}


% \input{Tables/short-term2-append}

%%
%% The acknowledgments section is defined using the "acks" environment
%% (and NOT an unnumbered section). This ensures the proper
%% identification of the section in the article metadata, and the
%% consistent spelling of the heading.


%%
%% The next two lines define the bibliography style to be used, and



%%
%% If your work has an appendix, this is the place to put it.


\end{document}
\endinput
%%
%% End of file `sample-sigconf.tex'.
