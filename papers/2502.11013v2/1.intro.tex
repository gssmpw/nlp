\section{Introduction}


\begin{figure}[t]
    \centering
    \includegraphics[width=\linewidth]{Figures/intro.pdf}
    \caption{Comparison of existing models with our mean-residual decomposition framework.}
    \label{fig:intro}
\end{figure}




Accurate prediction of urban spatiotemporal dynamics is crucial for urban management, efficiency optimization, and the development of smart cities~\cite{zhang2017deep,xie2020urban,jin2023spatio,yuan2024foundation}. These urban spatiotemporal data encapsulate complex patterns while also exhibiting randomness~\cite{sheng2025unveiling,chirigati2016data,li2020just,behnisch2009urban,zhang2017deep}. Taking traffic flow data as an example, it displays clear temporal and spatial patterns: it peaks during rush hours and is influenced by neighboring regions, while also being prone to random fluctuations caused by events like accidents or extreme weather.



Mainstream urban spatiotemporal prediction methods mainly rely on deterministic models~\cite{shao2022spatial,yan2018spatial,yuan2024unist,yuan2024foundation,zhang2017deep}. These models typically use mean absolute error ($\mathcal{L}_1$) or squared error ($\mathcal{L}_2$) between predicted and true values as the loss function. This allows to learn the expected value and capture primary spatiotemporal patterns. However, the data often follow a multi-modal distribution~\cite{gao2024enhancing,han2022card}, where multiple states can coexist at a given spatiotemporal point.
For example, in a scenario where a road segment with frequent accidents has an accident rate \(p_1\), with peak traffic volumes $y_1$ during normal hours and $y_2$ during accidents, deterministic models learn a conditional mean: \(\mathbb{E}[y|x]=p_1\times y_2+(1 - p_1)\times y_1\). However, this underestimates the resources needed in congested areas and over-allocates them to free-flowing ones. In contrast, probabilistic methods can address this by modeling multi-modal distributions.



Existing probabilistic models, such as Generative Adversarial Networks (GANs)~\cite{goodfellow2020generative,gao2022generative}, Variational Autoencoders (VAEs)~\cite{kingma2013auto,li2022generative}, and Diffusion Models~\cite{ho2020denoising,tashiro2021csdi}, have shown promising performance. 
These methods aim to learn the full data distribution, capturing both primary patterns (e.g., long-term temporal trends and global spatial patterns) and non-linear variations (e.g., random fluctuations caused by incidental events). However, their focus on modeling uncertainty often limits their ability to accurately capture the primary patterns.
Additionally, these models often suffer from inefficiency in real-world deployment. For example, GANs struggle with training challenges and mode collapse~\cite{goodfellow2020generative,brophy2021generative}, while diffusion models require multiple denoising steps for sampling, increasing computational costs~\cite{tashiro2021csdi,ruhling2023dyffusion,yuan2023spatio}.


Therefore, we aim to leverage the strengths of both deterministic and probabilistic models to enhance predictive accuracy while effectively capturing uncertainty. However, this is not trivial and faces three challenges: (i) Spatiotemporal data not only exhibits primary spatiotemporal patterns but also involves random fluctuations. (ii) The multi-modal distributions across different spatiotemporal locations are inherently heterogeneous. (iii) Large-scale urban computing and decision-making applications require both computationally efficient and scalable prediction models.



To address the aforementioned challenges, we propose \textbf{CoST} that \underline{\textbf{Co}}llaborates the deterministic model and probabilistic model for urban \underline{\textbf{S}}patio\underline{\textbf{T}}emporal prediction. As illustrated in Figure~\ref{fig:intro}, we leverage an advanced deterministic spatiotemporal prediction model to capture the primary patterns and predict the conditional mean \( \mathbb{E}(y|x) \). On this basis, we employ the diffusion model to model the residual distribution \( p(r|x) = p(y - \mathbb{E}(y|x)|x) \), compensating for the limitations of the mean prediction. Additionally, we propose a scale-aware diffusion process to handle the spatial heterogeneity. To thoroughly evaluate the performance of probabilistic predictions, we conduct experiments on eight real-world datasets, encompassing crowd flows, cellular network traffic, vehicle flow, and traffic speed. We use both deterministic metrics (MAE, RMSE) and probabilistic metrics (CRPS, QICE, IS) for evaluation.
Experimental results demonstrate that our model outperforms existing approaches in both predictive performance and computational efficiency.
In summary, our main contributions are as follows:  
\begin{itemize}[leftmargin=*] 
    \item We propose to collaborate deterministic and probabilistic models for  probabilistic urban spatiotemporal prediction, leveraging their strengths in both accuracy and uncertainty handling.
    \item We propose CoST, a mean-residual decomposition framework where the mean is modeled deterministically and residual variations are learned probabilistically using diffusion models. Additionally, we introduce a scale-aware diffusion process to capture spatial variability in residual distributions.
    \item We conduct experiments on 8 real-world datasets using comprehensive metrics for evaluation. Results demonstrate that CoST outperforms existing approaches across multiple key metrics, achieving  20\% performance improvement with high efficiency.
    
\end{itemize}
 

