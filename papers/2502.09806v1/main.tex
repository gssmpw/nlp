\documentclass[twocolumn]{article}

\usepackage[utf8]{inputenc}
\usepackage[T1]{fontenc}
\usepackage{amsmath,amssymb}
\usepackage{graphicx}
\usepackage{hyperref}
\hypersetup{breaklinks=true}  
\usepackage{url}
\usepackage{xurl} 
\usepackage{algorithm}
\usepackage{algorithmic}
\usepackage{float}
\usepackage{booktabs}
\usepackage{lipsum}  
\usepackage{verbatim}
\usepackage{xcolor}
\usepackage{enumitem}
\setlist[itemize]{itemsep=0.5pt} 
\usepackage{setspace} 
\usepackage{tikz}
\usepackage{subcaption}
\usepackage{amsthm}  
\newtheorem{assumption}{Assumption}
\usepackage{mdframed}
\usepackage{tcolorbox}
\usepackage{booktabs}
\usepackage{array}
\usepackage{enumitem}
\usepackage{authblk}  
\usepackage[authoryear]{natbib}
\usepackage{tabularx}
\renewcommand{\arraystretch}{0.85} 
\setlength{\tabcolsep}{1pt}



\title{Prioritized Ranking Experimental Design Using Recommender Systems in Two-Sided Platforms}

\author[1]{Mahyar Habibi}
\author[2]{Zahra Khanalizadeh}
\author[3]{Negar Ziaeian}

\affil[1]{Bocconi University}
\affil[2]{University of Washington}
\affil[3]{University of Warwick}

\date{February 2025}

\begin{document}
\raggedbottom

\maketitle

\begin{abstract}
\vspace{-10pt}
\noindent Interdependencies between units in online two-sided marketplaces complicate estimating causal effects in experimental settings. We propose a novel experimental design to mitigate the interference bias in estimating the total average treatment effect (TATE) of item-side interventions in online two-sided marketplaces. Our Two-Sided Prioritized Ranking (TSPR) design uses the recommender system as an instrument for experimentation. TSPR strategically prioritizes items based on their treatment status in the listings displayed to users. We designed TSPR to provide users with a coherent platform experience by ensuring access to all items and a consistent realization of their treatment by all users. We evaluate our experimental design through simulations using a search impression dataset from an online travel agency. Our methodology closely estimates the true simulated TATE, while a baseline item-side estimator significantly overestimates TATE.
\end{abstract}

\vspace{0.5em}
\noindent\textbf{Keywords:} Experimental design, recommender systems, two-sided platforms, causal inference, online marketplaces, interference

\section{Introduction} 
Online platforms, such as e-commerce sites and online marketplaces, frequently conduct randomized controlled experiments (e.g., A/B tests) to optimize user experience, boost engagement, and drive sales. These experiments help mitigate risks associated with changes and innovations while providing rapid product feedback \citep{kohavi2020trustworthy, bojinov2022online, xia2019safe, xu2018sqr, kohavi2009online}. 

Standard experimental designs for estimating unbiased treatment effects rely on the Stable Unit Treatment Value Assumption (SUTVA), which states that the treatment assigned to one unit does not influence the outcomes of other units \citep{rubin1974estimating, imbens2015causal}. In online marketplaces, this assumption is often violated due to the interconnected nature of users and items. For instance, in item-side experiments, modifying features of treated items---such as offering discounts---can influence demand for non-treated items due to substitution or complementary effects. Such interference, spillover, or network effects have been observed in various platforms, including ridesharing (\cite{chamandy2016experimentation}) and online pricing experiments (\cite{choi2019monetizing}). Failing to account for interference in randomized experiments can introduce substantial bias, leading to the overestimation or underestimation of the intervention’s true impact \citep{blake2014marketplace, fradkin2019simulation}.

Under interference, treatment effects depend on how interventions are distributed across units,  making the Total Average Treatment Effect (TATE) a central measure in such settings. TATE quantifies the impact of treating all units compared to treating none \citep{manski2013identification, munro2024treatmenteffectsmarketequilibrium}, and it has been widely adopted in the literature as it captures the full extent of both direct effects and spillovers across units. 

We propose the Two-Sided Prioritized Ranking (TSPR) experimental design that leverages recommender systems in two-sided marketplaces by strategically reordering items in ranked listings shown to users. Our approach builds on the well-documented phenomenon of position bias, where items appearing at the top of a list have greater influence on user behavior than those positioned lower \citep{craswell2008experimental, friedberg2022causal}. Recommender systems, which match items to user queries, are the primary ranking mechanism in online marketplaces. Our design leverages this structure to provide a systematic approach for estimating the TATE of item-side interventions on user-level outcomes. This framework applies broadly to two-sided marketplaces such as Expedia, Airbnb, Amazon, and many others.

In the TSPR design, users are randomized into two groups, and items are partitioned into three subsets: the Treated group, which receives the intervention, and two distinct groups that do not, called Untreated and Placebo. For one group of users, the recommender system prioritizes Untreated items at the top of search results, while for the other group, it prioritizes Treated items. Having both Untreated and Placebo groups---with Untreated matching the size of the Treated group---ensures the balanced quality of top-ranked items across user groups. We then estimate TATE by comparing partial outcomes between the two groups of users, where partial outcomes are defined as the cumulative outcome of the prioritized Treated or Untreated items placed at the top of the listings.

To evaluate our methodology, we use an open-source dataset of hotel search impressions from Expedia containing consumer queries, clicks, and booking outcomes. We develop a model of click and booking behavior to generate semi-synthetic data for our Monte Carlo simulations. We simulate a treatment resembling a platform-wide price increase on listed items, which, if applied to all items, reduces the user conversion rate by 0.050. We then compare two estimators across 500 simulation runs. Across simulations, our proposed method estimates TATE with an average of -0.047 and an average bootstrapped standard error of 0.016. In contrast, a naive estimator, which compares the average booking rate between treated and non-treated items, significantly overestimates TATE, producing an average estimated effect of -0.091 and an average standard error of 0.014.

A key advantage of our design is its ability to preserve a coherent user experience during experimentation. It ensures that no user loses access to any items, regardless of their randomized group assignment, and provides a consistent realization of item treatment across all users. A coherent user experience is crucial for online platforms, fostering trust, satisfaction, and long-term engagement \citep{veliz2023oxford, kahneman1986fairness, kohavi2020trustworthy}. However, many existing methods for estimating TATE in two-sided platforms disrupt user experience, limiting their practicality for real-world deployment. Switchback testing \citep{robins1986new, sneider2019experiment, bojinov2023design} alternates treatment assignments over time for the same units, enabling individual-level causal estimation but at the cost of user experience coherency. Frequent treatment fluctuations, such as dynamic pricing changes, may confuse users and distort engagement patterns.

Our proposed randomization framework is similar to two-sided randomization (TSR) methods \citep{johari2022experimental, bajari2023experimental}. In TSR, users and items are partitioned into mutually exclusive treatment-control subpopulations, and each group of users interacts with either treated or non-treated items. Consequently, TSR creates a fragmented marketplace which comes at the expense of user experience by restricting access to a subset of items and may lead to lost revenue, reduced customer satisfaction, and increased churn rates \citep{liu2013website}. In contrast, our proposed two-sided randomization framework guarantees users' access to all available items despite their randomized assignment.  Despite growing attention to ethical and responsible experimentation in online platforms \citep[e.g.,][]{polonioli2023ethics, saintjacques2020fairness}, the importance of maintaining a coherent user experience within experimental design has received limited attention. By incorporating user experience coherency as a core design objective, our work is the first to bridge this gap in the causal inference literature for two-sided platforms while maintaining the statistical advantages of TSR approaches.

Maintaining statistical power is another key strength of our design, as it randomizes at the user level and avoids the limitations imposed by cluster-based approaches. Cluster-based randomization groups related users or items to minimize spillover effects, but suffers from reduced power \citep{ugander2013graphclusterrandomizationnetwork, eckles2017design, holtz2024reducing}. Moreover, defining appropriate clusters in dynamic marketplace environments is often infeasible, and poor cluster definitions can lead to severe power loss and unreliable TATE estimates. Even when suitable clusters can be defined, implementing cluster-based randomization can be computationally expensive and operationally complex \citep{candogan2023correlated}.

Although mitigating interference in ranking experiments has been studied \citep[e.g.,][]{goli2024bias, zhan2024estimating, nandy2021b, ursu2018power}, existing experimental designs have not leveraged the ranking system itself as a tool for experimentation. By integrating recommender systems into experimental design, we bridge the gap between ranking mechanisms and causal inference methods. To the best of our knowledge, this study is the first to propose using recommender systems as an instrument for experimentation in online platforms.

The remainder of this paper is organized as follows: Section \ref{sec:methodology} formally defines the Two-Sided Prioritized Ranking experimental design and outlines our estimation methodology. Section \ref{sec:data_simulation} details the data and simulation setup, followed by the results in Section \ref{sec:results}. Finally, Section \ref{sec:conclusion} concludes the paper.

\section{Methodology} \label{sec:methodology}

\subsection{Two-Sided Prioritized Ranking (TSPR) Experimentation Setup}

We model a two-sided platform as a matching mechanism between a set of queries $q \in Q$, representing user inputs, and a set of items $i \in I$, representing available options. The platform uses a recommender system to compute relevance scores $r_{q,i} \in \mathbb{R}$ for each query-item pair based on the attributes of the query $q$ and the item $i$, such as user preferences and item features. Once query $q$ is submitted, the platform ranks the available items in descending order of their relevance scores $r_{q,i}$ and presents the ranked list to the user. When a user views the listed items, their interactions determine the outcomes $y_{q,i}$ for each displayed item. For simplicity, we assume all items initially have outcome values of $0$, and post-interaction, $y_{q,i}$ takes non-negative real values, representing outcomes such as clicks, bookings, or revenue. Additionally, since we assume each user submits exactly one query, we use the terms ``user'' and ``query'' interchangeably.

In the aforementioned setting, standard A/B testing with randomized item-level intervention assignment fails to produce unbiased treatment effect estimates due to the interference between items in the same query, which violates SUTVA. Furthermore, the proposed experimental design must satisfy two key constraints: preserving universal user access to all items throughout the experiment and maintaining consistent item treatment status across users. 

Due to item-side interference, the effect of a binary treatment $T \in \{0, 1\}$ on item-query outcomes $y_{q,i}$ depends on the distribution of interventions across units. This motivates our focus on the Total Average Treatment Effect (TATE), which captures both direct effects and spillovers by measuring the difference in expected outcomes between full treatment and no treatment conditions. Given our interest in TATE, we focus on query-level outcomes $Y_q = \sum_i y_{q,i}$, which aggregate individual item-query outcomes across all items displayed in response to query $q$. For notational simplicity, we omit the query subscript $q$ and denote query-level outcomes as $Y$.  Formally, we denote TATE by $\theta$ and define it as: 
\begin{equation}
    \label{eq:tate}
    \theta = \mathbb{E}\left[Y | \forall i \in \mathcal{I}: i \in \mathcal{I}^1  \right] - \mathbb{E}\left[Y|  \forall i \in \mathcal{I}: i \in \mathcal{I}^0\right]
\end{equation}
where $\mathcal{I}^1= \{i \in I \mid T_i = 1\}$  and $\mathcal{I}^0= \{i \in I \mid T_i = 0\}$  denote the sets of treated and non-treated items, respectively. Since each item can only be assigned to one treatment condition ($T_i = 1$ or $T_i = 0$) at a time, only one of the two terms on the right-hand side of the equation is observable at any given point. 

The proposed method rests on a few implicit assumptions. First, we assume that items at the top of the listings are significantly more influential on users' behavior (\cite{craswell2008experimental}), and the influence of items gradually vanishes as we move further down the list. Consequently, the effective exposure of a user to the treatment depends on the extent to which treated items appear near the top of the ranked list, as these items receive a disproportionate share of the user's attention. By strategically changing the ordering of items, we alter the \emph{effective exposure} of a user to the treatment. Second, our method requires the average number of relevant items per query to be sufficiently large to ensure that the repositioning scheme is effective in maximizing the exposure to treated items for one group of queries while minimizing it for the other. Third, we assume that user-side interference is negligible (e.g., slack supply) and the primary source of interference is the interdependence among outcomes of the items displayed under the same listing. 

Our proposed experimental design for estimating TATE is summarized in Table \ref{tab:experiment_setup}, with Figure \ref{fig:setup} illustrating the two-sided randomization scheme and group-specific listing priorities for query results.

\begin{table}[t]
    \centering
    \small  
    \caption{\centering Two-Sided Prioritized Ranking (TSPR) Experimental Design}
    \label{tab:experiment_setup}
    \begin{tabularx}{\linewidth}{>{\raggedright\arraybackslash}X}  
        \toprule
        \textbf{Experiment Setup} \\  
        \midrule
        \begin{minipage}{\linewidth} 
        \begin{enumerate}[leftmargin=*, itemsep=1pt, topsep=0pt, partopsep=0pt]
            \item Set the probability of receiving treatment for an item $p < 0.5$, and minimum relevance threshold $\underline{r}$. 

            \item Randomize items into Treated, Untreated, and Placebo subsets with probabilities $p$, $p$, and $1-2p$, respectively. Apply the treatment only to the Treated group.

            \item For each incoming query $q$:
            \begin{enumerate}[label=3.\arabic*., leftmargin=*]
                \item Randomly assign $q$ to $Q^A$ or $Q^B$ and set the item priorities as follows:
                \begin{itemize}[leftmargin=*]
                    \item If $q \in Q^A$: 1-Untreated, 2-Placebo, and 3-Treated.
                    \item If $q \in Q^B$: 1-Treated, 2-Placebo, and 3-Untreated.
                \end{itemize}
                
                \item Filter the set of relevant items with $r_{q,i} > \underline{r}$.
                \item Rank items primarily by priority (ascending) and secondarily by relevance score (descending).  
            \end{enumerate}
        \end{enumerate}
        \end{minipage}  
        \\
        \bottomrule
    \end{tabularx}  
\end{table}

\begin{figure}[htbp]
    \centering
    \caption{\centering Two-Sided Prioritized Ranking (TSPR) Experimental Design}
    \includegraphics[width=0.45\textwidth]{fig/Figure_1.pdf}
    \parbox{0.45\textwidth}{\footnotesize{\textit{Notes:} The figure illustrates the TSPR experiment setup. Items are partitioned into three groups, and queries are divided into two subsets. The relevant items for each query are first ordered based on their group-specific priority and then by their relevance score.}}
    \label{fig:setup}
\end{figure}

As outlined in Table \ref{tab:experiment_setup}, after specifying the global parameters $p$ and $\underline{r}$, we begin by partitioning items into three subsets: Treated, Untreated, and Placebo, with probabilities $p$, $p$, and $1-2p$, respectively. The intervention is only applied to the items in the Treated subset. Incorporating the Placebo subset serves a crucial purpose in maintaining experimental balance. Without a Placebo subset, when $p < 0.5$, the Untreated subset would be larger than the Treated subset, creating an asymmetric effect in step 3 of our design. Specifically, for queries in $Q^A$ for which we prioritize non-treated items, the larger Untreated pool would yield top-ranked items of higher average quality compared to the top-ranked items from the smaller Treated pool shown to $Q^B$. This imbalance would cause the recommender system's modification to impact $Q^A$ and $Q^B$ differently, confounding our ability to isolate the intervention's effect. The  Placebo subset ensures that the Treated and Untreated are roughly equal in size, making the expected match quality per rank comparable between the two groups.

In the next step, the stream of incoming queries are randomized into $Q^A$ or $Q^B$ with equal probability. Then, depending on the randomized allocation, item priorities are assigned such that queries in $Q^A$ will face items in the following order: Untreated, Placebo, and Treated, while the queries in $Q^B$, will be given items in the order of Treated, Placebo, and Untreated. Before returning the listings, items are filtered to ensure that only items of sufficient relevance are displayed to users. Modifying the recommender system may potentially cause poor-quality matches to be included in the listing, degrade user experience, and create large distortions in user behavior. To address this concern, we introduce a filtering parameter, denoted $\underline{r}$, and only items with a relevance score $r_{q,i} > \underline{r}$ are included in the final listing. The choice of $\underline{r}$ involves a trade-off and should be chosen carefully: setting it too low risks including irrelevant items that could deteriorate user experience, while setting it too high might result in queries receiving an insufficient number of items to display.

\subsection{TATE Estimation}

Building on the observations from the experimental method described above, we propose an estimator for the Total Average Treatment Effect (TATE) that does not require specifying an exposure mapping or imposing strong assumptions about substitution patterns among items.

The experimental design produces data with two important characteristics. First, because the recommender system prioritizes items rather than exclusively displaying those from one subset, queries in $Q^B$ may see some non-treated items, while queries in $Q^A$ may encounter some treated items. Second, the number of Treated (or Untreated) items appearing at the top of listings varies across queries based on item availability and relevance. These characteristics necessitate an assumption about how the intervention's expected impact on query-level outcomes relates to the number of treated items placed at the top of listings.

We define \emph{partial outcome}, denoted by $Y^{l} = \sum_{i=1}^{l} y^{i}$, as the cumulative outcome of the first $l$ listed items. Given the assumption that individual item-level outcomes are non-negative ($y_i \geq 0$), the expected value of the partial outcome, $\mathbb{E}[Y^{l}]$, is non-decreasing in $l$, which reflects the fact that including additional items in the calculation of outcome can only increase or maintain the total observed outcome. Similarly, we define the partial treatment effect, $\theta^l$, as the expected effect on partial outcome, $Y^l$, from applying the intervention only to the first $l$ listed items. That is,
\begin{equation}
    \label{eq:partial_treatment}
    \resizebox{\linewidth}{!}{$
     \theta^l = \mathbb{E}[Y^{l}  | \forall i \leq l: i \in \mathcal{I}^{1} \land \forall i > l: i \in \mathcal{I}^{0} ] - \mathbb{E}[Y^{l}  | \forall i : i \in \mathcal{I}^0].
    $}
\end{equation}


\begin{assumption}
    For all values of $l$, the partial treatment effect, $\theta^l$, is a fraction of the total treatment effect, $\theta$, by a factor of the ratio of the expected partial outcome to the expected total outcome in the absence of treatment. Formally,
   \begin{equation}
    \label{eq:assumption}
    \begin{aligned}
        \theta^l = \frac{\mathbb{E}[Y^{l}  | \forall i: i \in \mathcal{I}^0]}{\mathbb{E}[Y | \forall i : i \in \mathcal{I}^0]} \theta, \quad \forall l.
    \end{aligned}
\end{equation}
\end{assumption}

The assumption above allows us to aggregate the outcomes of queries faced with different numbers of Treated or Untreated units at the top of their corresponding listings. Due to the declining item influence down the listing position, as $l$ increases, the marginal contribution to the expected partial outcome $\mathbb{E}[Y^{l}]$ decreases. Consequently, the marginal effect of treating an additional item at rank $l$ gradually approaches zero as we move further down the listing, reflecting the diminishing impact of lower-ranked items on the query-level outcome.

In the TSPR design, the first term in the RHS of Equation \ref{eq:partial_treatment} corresponds to queries in $Q^B$ where the Treated block contains $l$ items. The second term represents the expected partial outcome $Y^l$ when no items in the list are treated. Under TSPR, this quantity is only observed when there are no relevant Treated units for a query in $Q^A$, a scenario that may be rare in practice, making direct estimation difficult. Given our assumptions that each query has a relatively large number of available items and that lower-listed items have negligible impact on user behavior, we propose the following approximation:
\begin{equation}
\label{eq:approximation}
\mathbb{E}[Y^{l}  | \forall i : i \in \mathcal{I}^0] \approx \mathbb{E}[Y^{l}  | \forall i \leq l : i \in \mathcal{I}^{0}]
\end{equation}
where $l$ denotes the number of Untreated items. This term can be represented by a query in $Q^A$ with a block of $l$ Untreated items. The introduction of Placebo items in our design further supports the validity of this approximation.

Using Equations \ref{eq:tate}, \ref{eq:partial_treatment}, \ref{eq:assumption}, \ref{eq:approximation}, and assuming for all $l$, $\mathbb{E}[Y^{l}  | \forall i \leq l: i \in \mathcal{I}^{0}] > 0$, we have:
\begin{equation}
\label{eq:theta}
\resizebox{\linewidth}{!}{$
\begin{aligned}
\theta &\approx \mathbb{E}[Y | \forall i : i \in \mathcal{I}^0] \\
&\times\frac{\mathbb{E}[Y^{l}  | \forall i \leq l: i \in \mathcal{I}^{1} \land \forall i > l: i \in \mathcal{I}^{0}] -  \mathbb{E}[Y^{l}  | \forall i \leq l : i \in \mathcal{I}^{0}]}{\mathbb{E}[Y^{l}  | \forall i \leq l: i \in \mathcal{I}^{0}]} 
\end{aligned}
$}
\end{equation}

for any choice of $l$.

Equation \ref{eq:theta} establishes a connection between the observable differences in partial outcomes from experimental data and the TATE. The first term on the RHS of the equation can be estimated by running a \emph{pre-experiment} phase in which the modified recommender system is deployed with the same composition of Treated, Placebo, and Untreated items, \emph{before} applying any treatment to the items in the Treated group. The second term on the RHS of the equation can be estimated \emph{during the experiment} using the relative difference in partial outcomes between queries in $Q^A$ and $Q^B$ that have exactly $l$ Untreated and Treated items at the top of their listings.

To provide a single estimate of TATE, we need to combine the estimates resulted from all observed values of $l$ having at least one correspondent query in $Q^A$ and $Q^B$. Here, we use frequency weighting and estimate TATE as,
\begin{align}
    \label{eq:theta_hat}
    \hat{\theta} =  \sum_{l=1}^{L} w_l \, \bar{Y}_{0} \left( \frac{\frac{1}{|Q^{B}_{l}|} \sum_{Q^{B}_{l}}Y^{l}_{B} -  \frac{1}{|Q^{A}_{l}|} \sum_{Q^{A}_{l}}Y^{l}_{A}}{\frac{1}{|Q^{A}_{l}|} \sum_{Q^{A}_{l}}Y^{l}_{A}} \right); \notag \\
    w_l = \frac{|Q^{B}_{l}| + |Q^{A}_{l}| }{ |Q^{B}| + |Q^{A}|}
\end{align}

where $Q^{B}_{l}$ ($Q^{A}_{l}$) denote queries in $Q^{B}$ ($Q^{A}$) for which the number of Treated (Untreated) items was equal to $l$, and $\bar{Y}_{0}$ is the expected outcome estimated using pre-experiment phase with the modified recommender system. Standard errors can be obtained via bootstrapping.

\section{Data and Simulation} \label{sec:data_simulation}
To illustrate our methodology, we use an open-source dataset of hotel search impressions from Expedia \citep{expedia-personalized-sort}, capturing consumer queries and their corresponding search behaviors---specifically, clicks and booking outcomes---over an eight-month period spanning 2012 and 2013. The dataset encompasses nearly 10 million observations derived from approximately 400,000 unique search impressions. Each search impression represents the result of a consumer query, providing a list of hotels along with their observable characteristics.

Consumers interact with the platform in three stages. First, consumers initiate queries by specifying trip details (destination, travel dates, booking window, etc.). Second, they receive a ranked list of hotel results through an experimental setup: two-thirds of users see listings ranked by the platform's original recommender system, while the remaining one-third encounter randomly sorted results. This experimental variation in ranking mechanisms allows us to model how item positions influence click and booking behavior. Finally, users engage by clicking on hotels to view details and may either complete a booking or leave without purchasing.

\begin{table}[ht]
\caption{Summary Statistics of Search Impressions}
\centering
\resizebox{\linewidth}{!}{  
\begin{tabular}{lrrrr}
\hline
                        & Mean  & Median & Min & Max \\
\hline
Randomized Ranking (Yes=1)   & 0.30  & 0   & 0   & 1   \\
Total Hotels per Impression  & 24.56 & 29  & 4   & 33  \\
Clicks per Impression        & 1.11  & 1   & 1   & 30  \\
Bookings per Impression      & 0.69  & 1   & 0   & 1   \\
\hline
\end{tabular}
}  
\label{tab:summary_stats}
\end{table}


To evaluate our experimental design, we implement a series of Monte Carlo simulations that replicate consumer interactions in an online two-sided marketplace, incorporating query-driven item ranking, click behavior, and booking decisions. We assume that the platform maintains a pool of available items, denoted as $N$, and displays a subset $n_q$ in response to each query.

To model user interactions, we assume that each item displayed to a consumer has a net (hidden) utility, denoted as $v$. The relevance score $r$, which represents the recommender system's match score between a consumer's query and an item, is modeled as $r = v + \epsilon$ with $\epsilon$ following a normal distribution $N(0, \sigma^2)$. We assume that the original ranking system is decreasing in $r$. However, for randomly ranked search impressions, the sorting order is determined randomly.

Click probabilities are modeled as a logistic function of the raw and quadratic rank values, hidden utilities, and prior user clicks on lower-ranked options. Booking decisions are modeled as a logit choice among clicked items, depending solely on net utility $v$. To ensure the simulation aligns with real-world behavior, hyperparameters $\sigma_e$ and $n_q$ are selected to match simulated conversion rates with observed data. This is achieved through an iterative process, where click and booking parameters are first estimated using the data-generating process, followed by user action simulations. The simulated conversion rates are then compared with empirical rates, and hyperparameters are adjusted to minimize discrepancies. Figure \ref{fig:subplot1} shows that the simulated click-through rate closely matches the observed data, demonstrating the convergence of the simulation to real-world behavior.

\begin{figure}[htbp]
    \centering
    \caption{\centering Click-Through Rate by Item Rank}
    \includegraphics[width=0.45\textwidth]{fig/ctr_actual_vs_sim.pdf}
    \label{fig:subplot1}
    \parbox{0.45\textwidth}{\footnotesize{\textit{Notes:} This figure presents the actual click-through rate (CTR) and the simulated CTR as a function of item position in the query results from a hold-out sample not used in the estimation of the click and booking models.}}
\end{figure}

Table \ref{tab:summary_stats} presents summary statistics at the search impression level, highlighting key patterns in click and booking behaviors across random and relevance-based rankings. 


\section{Results} \label{sec:results}
We conduct counterfactual simulations for 20,000 queries using our estimated models of click and booking behavior. First, to establish a simulated ground truth for TATE, we simulate the marketplace under two extreme scenarios: one where no items receive treatment and another where all items are treated. The treatment is implemented as a constant reduction in users' hidden utility from booking an item, resembling the effect of a platform-wide price or markup increase, which translates to a 0.05 decrease in the conversion rate. In these simulations, we maintain the recommender system without any modifications.

We then implement our Two-Sided Prioritized Ranking (TSPR) experimental design to estimate TATE in a setting where treatment is applied to 25\% ($p=0.25$) of the items. Following our methodology, we randomly assign each query to either group A or B with equal probability. For one group, the recommender system is modified to prioritize Treated items in the ranking, while for the other group, it prioritizes Untreated items. The remaining items are positioned according to the experimental design outlined in Table \ref{tab:experiment_setup}, maintaining access to all items while creating the necessary variation in exposure to treatment.

Additionally, to form a baseline for comparison, we simulate an item-side randomized experiment, which is an extension of the Horvitz-Thompson estimator \citep{horvitz1952generalization} to two-sided platforms and commonly used as a baseline in the literature \citep[e.g.,][]{johari2022experimental, bajari2023experimental}. In this setup, items are randomly assigned to the treatment group ($T$) with probability $p$ or the control group ($C$) with probability $1-p$. However, no randomization at the query level takes place. TATE is estimated as the difference in means of total outcomes per query, adjusted by the inverse of the inclusion probability of treated and control items:
\begin{equation}
    \hat{\theta}_{IS} =  \frac{\sum_{i \in T} \sum_{q \in Q} y_{q,i}}{p|Q|} - \frac{\sum_{i \in C} \sum_{q \in Q} y_{q,i}}{(1-p)|Q|}
\end{equation}
where $Q$ denotes the set of all queries, and $y_{q,i}$ represents the outcome for item $i$ in query $q$. 

\begin{figure}[htbp]
    \centering
     \caption{\centering TATE Estimates}
    \includegraphics[width=0.45\textwidth]{fig/hist_tate_naive_CIs.pdf}
    \parbox{0.45\textwidth}{\footnotesize{\textit{Notes:} The figure presents Total Average Treatment Effect (TATE) estimates from 500 simulated experiments. The vertical dashed line represents the ground truth TATE, corresponding to a 0.05 reduction in booking rate.  Panel (a) plots the histogram of TATE estimates from the TSPR method with 25\% treatment coverage (\(p=0.25\)), yielding an average estimate of -0.047 (average bootstrapped SE: 0.016).  Panel (b) plots the histogram of TATE estimates from the naive estimator under the same conditions, yielding an average estimate of -0.091 (average bootstrapped SE: 0.014). TATE estimates with 95\% confidence intervals that contain the ground truth are shown in bold color.
    }}
    \label{fig:tate_est}
    
\end{figure}

Figure \ref{fig:tate_est} presents the distribution of the TATE estimates from the TSPR setup and contrasts it with the distribution of $\hat{\theta}_{IS}$ across 500 runs. TSPR estimates TATE with an average of -0.047 (average bootstrapped SE: 0.016). In contrast, the baseline estimator significantly overestimates the effect with an average of -0.091 (average bootstrapped SE: 0.014), roughly double the true value, despite using a similar treatment group size. This overestimation occurs because the baseline approach fails to account for interference between treated and non-treated items within the same listing. The proportion of the TATE estimates with 95\% confidence intervals that contain the ground truth TATE (-0.05) is considerably larger for TSPR compared to the baseline estimator.

The relevance threshold parameter $\underline{r}$ serves a dual methodological and practical purpose: maintaining partial outcomes close to the original recommender system while preserving user experience through quality control of top-positioned items. In practice, platforms can determine an appropriate value for $\underline{r}$ using historical data or pre-intervention experiments with the modified recommender system, allowing them to balance the trade-off between listing quality and estimates' accuracy. In our simulations, we set $\underline{r} = 1.7$ to keep partial outcomes under the modified recommender system close to the baseline case, as illustrated in Figure \ref{fig:par_y}. The figure demonstrates that marginal contributions to partial outcomes are substantial for small values of $l$ but rapidly diminish with increasing position. This supports our approximation (\ref{eq:approximation}) by suggesting minimal contributions from items down the list.
\begin{figure}[htbp]
    \centering
     \caption{\centering Partial Outcomes Across Ranks}
    \includegraphics[width=0.45\textwidth]{fig/partial_outcome.pdf}
    \parbox{0.45\textwidth}{\footnotesize{\textit{Notes:} The figure plots the partial outcomes $Y^l$ for rank $l$, in four scenarios across 100 simulations. The first two scenarios are under the unmodified recommender system with no treatment ($p=0.0$) and full treatment ($p=1.0$). The other two scenarios illustrate the partial outcomes for $Q^A$ and $Q^B$ in the simulated experiments when the probability of assignment to both the Treated and Untreated group is $p=0.25$.}}
    \label{fig:par_y}
\end{figure}

Figure \ref{fig:sensitivity} examines the robustness of our estimates to the choice of relevance threshold $\underline{r}$. The TATE estimates remain relatively stable under moderate changes in $\underline{r}$, suggesting that our methodology is robust to the specific choice of this parameter. However, as illustrated, higher values of $\underline{r}$ restrict the number of Treated or Untreated items that qualify for top positions and reduce the range of block sizes available for estimation in Equation \ref{eq:theta}, leading to larger standard errors in our estimates.

\begin{figure}[htbp]
    \centering
     \caption{\centering Sensitivity Analysis of the TATE Estimates to $\underline{r}$}
    \includegraphics[width=0.45\textwidth]{fig/r_values_sensitivity_results.pdf}
    \parbox{0.45\textwidth}{\footnotesize{\textit{Notes:} The figure illustrates the TATE estimates and their 95\% bootstrapped confidence intervals for different choices of $\underline{r}$ from 100 simulated experiments in the TSPR setup when the probability of assignment to both the Treated and Untreated group is $p=0.25$.}}
    \label{fig:sensitivity}
\end{figure}

\section{Conclusion} \label{sec:conclusion}
This paper introduces a novel experimental design for two-sided marketplaces that leverages recommender systems to estimate the Total Average Treatment Effect (TATE) while addressing interference and maintaining a coherent user experience. By reordering items in query listings, our Two-Sided Prioritized Ranking (TSPR) design minimizes bias from network spillovers and ensures equal access to items for all users. Using a semi-synthetic dataset of hotel search impressions, we demonstrate that our design provides reliable TATE estimates while a standard item-side estimator significantly overestimates TATE.

The TSPR method is particularly designed for platforms relying on ranking algorithms, such as e-commerce sites and online marketplaces. Our design enables such platforms to estimate treatment effects while maintaining a coherent user experience and preserving platform functionality. It addresses concerns regarding unintended consequences of randomized experiments, such as user disengagement or inequitable access to items. 

Our paper opens promising avenues for future research. First, although we evaluate our method using semi-synthetic data, validating it with real-world data from diverse platforms would enhance its generalizability and provide deeper insights into its practical implementation. Second, our design focuses on item-side interventions and assumes negligible cross-user interference. Extending it to user-side interventions or settings with significant cross-user interference remains an open challenge and would greatly expand its applicability.


\section*{Acknowledgments}
We would like to express our gratitude to Manuel Bagues, Ludovica Gazze, Avi Goldfarb, Alan Griffith, Jason Kerwin, Mohammad H. Seyedsalehi, Sadegh Shirani, and participants at the 2024 Conference in Digital Experimentation (CODE@MIT), whose valuable insights were instrumental in shaping this work.


\bibliographystyle{agsm}
% This must be in the first 5 lines to tell arXiv to use pdfLaTeX, which is strongly recommended.
\pdfoutput=1
% In particular, the hyperref package requires pdfLaTeX in order to break URLs across lines.

\documentclass[11pt]{article}

% Change "review" to "final" to generate the final (sometimes called camera-ready) version.
% Change to "preprint" to generate a non-anonymous version with page numbers.
\usepackage{acl}

% Standard package includes
\usepackage{times}
\usepackage{latexsym}

% Draw tables
\usepackage{booktabs}
\usepackage{multirow}
\usepackage{xcolor}
\usepackage{colortbl}
\usepackage{array} 
\usepackage{amsmath}

\newcolumntype{C}{>{\centering\arraybackslash}p{0.07\textwidth}}
% For proper rendering and hyphenation of words containing Latin characters (including in bib files)
\usepackage[T1]{fontenc}
% For Vietnamese characters
% \usepackage[T5]{fontenc}
% See https://www.latex-project.org/help/documentation/encguide.pdf for other character sets
% This assumes your files are encoded as UTF8
\usepackage[utf8]{inputenc}

% This is not strictly necessary, and may be commented out,
% but it will improve the layout of the manuscript,
% and will typically save some space.
\usepackage{microtype}
\DeclareMathOperator*{\argmax}{arg\,max}
% This is also not strictly necessary, and may be commented out.
% However, it will improve the aesthetics of text in
% the typewriter font.
\usepackage{inconsolata}

%Including images in your LaTeX document requires adding
%additional package(s)
\usepackage{graphicx}
% If the title and author information does not fit in the area allocated, uncomment the following
%
%\setlength\titlebox{<dim>}
%
% and set <dim> to something 5cm or larger.

\title{Wi-Chat: Large Language Model Powered Wi-Fi Sensing}

% Author information can be set in various styles:
% For several authors from the same institution:
% \author{Author 1 \and ... \and Author n \\
%         Address line \\ ... \\ Address line}
% if the names do not fit well on one line use
%         Author 1 \\ {\bf Author 2} \\ ... \\ {\bf Author n} \\
% For authors from different institutions:
% \author{Author 1 \\ Address line \\  ... \\ Address line
%         \And  ... \And
%         Author n \\ Address line \\ ... \\ Address line}
% To start a separate ``row'' of authors use \AND, as in
% \author{Author 1 \\ Address line \\  ... \\ Address line
%         \AND
%         Author 2 \\ Address line \\ ... \\ Address line \And
%         Author 3 \\ Address line \\ ... \\ Address line}

% \author{First Author \\
%   Affiliation / Address line 1 \\
%   Affiliation / Address line 2 \\
%   Affiliation / Address line 3 \\
%   \texttt{email@domain} \\\And
%   Second Author \\
%   Affiliation / Address line 1 \\
%   Affiliation / Address line 2 \\
%   Affiliation / Address line 3 \\
%   \texttt{email@domain} \\}
% \author{Haohan Yuan \qquad Haopeng Zhang\thanks{corresponding author} \\ 
%   ALOHA Lab, University of Hawaii at Manoa \\
%   % Affiliation / Address line 2 \\
%   % Affiliation / Address line 3 \\
%   \texttt{\{haohany,haopengz\}@hawaii.edu}}
  
\author{
{Haopeng Zhang$\dag$\thanks{These authors contributed equally to this work.}, Yili Ren$\ddagger$\footnotemark[1], Haohan Yuan$\dag$, Jingzhe Zhang$\ddagger$, Yitong Shen$\ddagger$} \\
ALOHA Lab, University of Hawaii at Manoa$\dag$, University of South Florida$\ddagger$ \\
\{haopengz, haohany\}@hawaii.edu\\
\{yiliren, jingzhe, shen202\}@usf.edu\\}



  
%\author{
%  \textbf{First Author\textsuperscript{1}},
%  \textbf{Second Author\textsuperscript{1,2}},
%  \textbf{Third T. Author\textsuperscript{1}},
%  \textbf{Fourth Author\textsuperscript{1}},
%\\
%  \textbf{Fifth Author\textsuperscript{1,2}},
%  \textbf{Sixth Author\textsuperscript{1}},
%  \textbf{Seventh Author\textsuperscript{1}},
%  \textbf{Eighth Author \textsuperscript{1,2,3,4}},
%\\
%  \textbf{Ninth Author\textsuperscript{1}},
%  \textbf{Tenth Author\textsuperscript{1}},
%  \textbf{Eleventh E. Author\textsuperscript{1,2,3,4,5}},
%  \textbf{Twelfth Author\textsuperscript{1}},
%\\
%  \textbf{Thirteenth Author\textsuperscript{3}},
%  \textbf{Fourteenth F. Author\textsuperscript{2,4}},
%  \textbf{Fifteenth Author\textsuperscript{1}},
%  \textbf{Sixteenth Author\textsuperscript{1}},
%\\
%  \textbf{Seventeenth S. Author\textsuperscript{4,5}},
%  \textbf{Eighteenth Author\textsuperscript{3,4}},
%  \textbf{Nineteenth N. Author\textsuperscript{2,5}},
%  \textbf{Twentieth Author\textsuperscript{1}}
%\\
%\\
%  \textsuperscript{1}Affiliation 1,
%  \textsuperscript{2}Affiliation 2,
%  \textsuperscript{3}Affiliation 3,
%  \textsuperscript{4}Affiliation 4,
%  \textsuperscript{5}Affiliation 5
%\\
%  \small{
%    \textbf{Correspondence:} \href{mailto:email@domain}{email@domain}
%  }
%}

\begin{document}
\maketitle
\begin{abstract}
Recent advancements in Large Language Models (LLMs) have demonstrated remarkable capabilities across diverse tasks. However, their potential to integrate physical model knowledge for real-world signal interpretation remains largely unexplored. In this work, we introduce Wi-Chat, the first LLM-powered Wi-Fi-based human activity recognition system. We demonstrate that LLMs can process raw Wi-Fi signals and infer human activities by incorporating Wi-Fi sensing principles into prompts. Our approach leverages physical model insights to guide LLMs in interpreting Channel State Information (CSI) data without traditional signal processing techniques. Through experiments on real-world Wi-Fi datasets, we show that LLMs exhibit strong reasoning capabilities, achieving zero-shot activity recognition. These findings highlight a new paradigm for Wi-Fi sensing, expanding LLM applications beyond conventional language tasks and enhancing the accessibility of wireless sensing for real-world deployments.
\end{abstract}

\section{Introduction}

In today’s rapidly evolving digital landscape, the transformative power of web technologies has redefined not only how services are delivered but also how complex tasks are approached. Web-based systems have become increasingly prevalent in risk control across various domains. This widespread adoption is due their accessibility, scalability, and ability to remotely connect various types of users. For example, these systems are used for process safety management in industry~\cite{kannan2016web}, safety risk early warning in urban construction~\cite{ding2013development}, and safe monitoring of infrastructural systems~\cite{repetto2018web}. Within these web-based risk management systems, the source search problem presents a huge challenge. Source search refers to the task of identifying the origin of a risky event, such as a gas leak and the emission point of toxic substances. This source search capability is crucial for effective risk management and decision-making.

Traditional approaches to implementing source search capabilities into the web systems often rely on solely algorithmic solutions~\cite{ristic2016study}. These methods, while relatively straightforward to implement, often struggle to achieve acceptable performances due to algorithmic local optima and complex unknown environments~\cite{zhao2020searching}. More recently, web crowdsourcing has emerged as a promising alternative for tackling the source search problem by incorporating human efforts in these web systems on-the-fly~\cite{zhao2024user}. This approach outsources the task of addressing issues encountered during the source search process to human workers, leveraging their capabilities to enhance system performance.

These solutions often employ a human-AI collaborative way~\cite{zhao2023leveraging} where algorithms handle exploration-exploitation and report the encountered problems while human workers resolve complex decision-making bottlenecks to help the algorithms getting rid of local deadlocks~\cite{zhao2022crowd}. Although effective, this paradigm suffers from two inherent limitations: increased operational costs from continuous human intervention, and slow response times of human workers due to sequential decision-making. These challenges motivate our investigation into developing autonomous systems that preserve human-like reasoning capabilities while reducing dependency on massive crowdsourced labor.

Furthermore, recent advancements in large language models (LLMs)~\cite{chang2024survey} and multi-modal LLMs (MLLMs)~\cite{huang2023chatgpt} have unveiled promising avenues for addressing these challenges. One clear opportunity involves the seamless integration of visual understanding and linguistic reasoning for robust decision-making in search tasks. However, whether large models-assisted source search is really effective and efficient for improving the current source search algorithms~\cite{ji2022source} remains unknown. \textit{To address the research gap, we are particularly interested in answering the following two research questions in this work:}

\textbf{\textit{RQ1: }}How can source search capabilities be integrated into web-based systems to support decision-making in time-sensitive risk management scenarios? 
% \sq{I mention ``time-sensitive'' here because I feel like we shall say something about the response time -- LLM has to be faster than humans}

\textbf{\textit{RQ2: }}How can MLLMs and LLMs enhance the effectiveness and efficiency of existing source search algorithms? 

% \textit{\textbf{RQ2:}} To what extent does the performance of large models-assisted search align with or approach the effectiveness of human-AI collaborative search? 

To answer the research questions, we propose a novel framework called Auto-\
S$^2$earch (\textbf{Auto}nomous \textbf{S}ource \textbf{Search}) and implement a prototype system that leverages advanced web technologies to simulate real-world conditions for zero-shot source search. Unlike traditional methods that rely on pre-defined heuristics or extensive human intervention, AutoS$^2$earch employs a carefully designed prompt that encapsulates human rationales, thereby guiding the MLLM to generate coherent and accurate scene descriptions from visual inputs about four directional choices. Based on these language-based descriptions, the LLM is enabled to determine the optimal directional choice through chain-of-thought (CoT) reasoning. Comprehensive empirical validation demonstrates that AutoS$^2$-\ 
earch achieves a success rate of 95–98\%, closely approaching the performance of human-AI collaborative search across 20 benchmark scenarios~\cite{zhao2023leveraging}. 

Our work indicates that the role of humans in future web crowdsourcing tasks may evolve from executors to validators or supervisors. Furthermore, incorporating explanations of LLM decisions into web-based system interfaces has the potential to help humans enhance task performance in risk control.






\section{Related Work}
\label{sec:relatedworks}

% \begin{table*}[t]
% \centering 
% \renewcommand\arraystretch{0.98}
% \fontsize{8}{10}\selectfont \setlength{\tabcolsep}{0.4em}
% \begin{tabular}{@{}lc|cc|cc|cc@{}}
% \toprule
% \textbf{Methods}           & \begin{tabular}[c]{@{}c@{}}\textbf{Training}\\ \textbf{Paradigm}\end{tabular} & \begin{tabular}[c]{@{}c@{}}\textbf{$\#$ PT Data}\\ \textbf{(Tokens)}\end{tabular} & \begin{tabular}[c]{@{}c@{}}\textbf{$\#$ IFT Data}\\ \textbf{(Samples)}\end{tabular} & \textbf{Code}  & \begin{tabular}[c]{@{}c@{}}\textbf{Natural}\\ \textbf{Language}\end{tabular} & \begin{tabular}[c]{@{}c@{}}\textbf{Action}\\ \textbf{Trajectories}\end{tabular} & \begin{tabular}[c]{@{}c@{}}\textbf{API}\\ \textbf{Documentation}\end{tabular}\\ \midrule 
% NexusRaven~\citep{srinivasan2023nexusraven} & IFT & - & - & \textcolor{green}{\CheckmarkBold} & \textcolor{green}{\CheckmarkBold} &\textcolor{red}{\XSolidBrush}&\textcolor{red}{\XSolidBrush}\\
% AgentInstruct~\citep{zeng2023agenttuning} & IFT & - & 2k & \textcolor{green}{\CheckmarkBold} & \textcolor{green}{\CheckmarkBold} &\textcolor{red}{\XSolidBrush}&\textcolor{red}{\XSolidBrush} \\
% AgentEvol~\citep{xi2024agentgym} & IFT & - & 14.5k & \textcolor{green}{\CheckmarkBold} & \textcolor{green}{\CheckmarkBold} &\textcolor{green}{\CheckmarkBold}&\textcolor{red}{\XSolidBrush} \\
% Gorilla~\citep{patil2023gorilla}& IFT & - & 16k & \textcolor{green}{\CheckmarkBold} & \textcolor{green}{\CheckmarkBold} &\textcolor{red}{\XSolidBrush}&\textcolor{green}{\CheckmarkBold}\\
% OpenFunctions-v2~\citep{patil2023gorilla} & IFT & - & 65k & \textcolor{green}{\CheckmarkBold} & \textcolor{green}{\CheckmarkBold} &\textcolor{red}{\XSolidBrush}&\textcolor{green}{\CheckmarkBold}\\
% LAM~\citep{zhang2024agentohana} & IFT & - & 42.6k & \textcolor{green}{\CheckmarkBold} & \textcolor{green}{\CheckmarkBold} &\textcolor{green}{\CheckmarkBold}&\textcolor{red}{\XSolidBrush} \\
% xLAM~\citep{liu2024apigen} & IFT & - & 60k & \textcolor{green}{\CheckmarkBold} & \textcolor{green}{\CheckmarkBold} &\textcolor{green}{\CheckmarkBold}&\textcolor{red}{\XSolidBrush} \\\midrule
% LEMUR~\citep{xu2024lemur} & PT & 90B & 300k & \textcolor{green}{\CheckmarkBold} & \textcolor{green}{\CheckmarkBold} &\textcolor{green}{\CheckmarkBold}&\textcolor{red}{\XSolidBrush}\\
% \rowcolor{teal!12} \method & PT & 103B & 95k & \textcolor{green}{\CheckmarkBold} & \textcolor{green}{\CheckmarkBold} & \textcolor{green}{\CheckmarkBold} & \textcolor{green}{\CheckmarkBold} \\
% \bottomrule
% \end{tabular}
% \caption{Summary of existing tuning- and pretraining-based LLM agents with their training sample sizes. "PT" and "IFT" denote "Pre-Training" and "Instruction Fine-Tuning", respectively. }
% \label{tab:related}
% \end{table*}

\begin{table*}[ht]
\begin{threeparttable}
\centering 
\renewcommand\arraystretch{0.98}
\fontsize{7}{9}\selectfont \setlength{\tabcolsep}{0.2em}
\begin{tabular}{@{}l|c|c|ccc|cc|cc|cccc@{}}
\toprule
\textbf{Methods} & \textbf{Datasets}           & \begin{tabular}[c]{@{}c@{}}\textbf{Training}\\ \textbf{Paradigm}\end{tabular} & \begin{tabular}[c]{@{}c@{}}\textbf{\# PT Data}\\ \textbf{(Tokens)}\end{tabular} & \begin{tabular}[c]{@{}c@{}}\textbf{\# IFT Data}\\ \textbf{(Samples)}\end{tabular} & \textbf{\# APIs} & \textbf{Code}  & \begin{tabular}[c]{@{}c@{}}\textbf{Nat.}\\ \textbf{Lang.}\end{tabular} & \begin{tabular}[c]{@{}c@{}}\textbf{Action}\\ \textbf{Traj.}\end{tabular} & \begin{tabular}[c]{@{}c@{}}\textbf{API}\\ \textbf{Doc.}\end{tabular} & \begin{tabular}[c]{@{}c@{}}\textbf{Func.}\\ \textbf{Call}\end{tabular} & \begin{tabular}[c]{@{}c@{}}\textbf{Multi.}\\ \textbf{Step}\end{tabular}  & \begin{tabular}[c]{@{}c@{}}\textbf{Plan}\\ \textbf{Refine}\end{tabular}  & \begin{tabular}[c]{@{}c@{}}\textbf{Multi.}\\ \textbf{Turn}\end{tabular}\\ \midrule 
\multicolumn{13}{l}{\emph{Instruction Finetuning-based LLM Agents for Intrinsic Reasoning}}  \\ \midrule
FireAct~\cite{chen2023fireact} & FireAct & IFT & - & 2.1K & 10 & \textcolor{red}{\XSolidBrush} &\textcolor{green}{\CheckmarkBold} &\textcolor{green}{\CheckmarkBold}  & \textcolor{red}{\XSolidBrush} &\textcolor{green}{\CheckmarkBold} & \textcolor{red}{\XSolidBrush} &\textcolor{green}{\CheckmarkBold} & \textcolor{red}{\XSolidBrush} \\
ToolAlpaca~\cite{tang2023toolalpaca} & ToolAlpaca & IFT & - & 4.0K & 400 & \textcolor{red}{\XSolidBrush} &\textcolor{green}{\CheckmarkBold} &\textcolor{green}{\CheckmarkBold} & \textcolor{red}{\XSolidBrush} &\textcolor{green}{\CheckmarkBold} & \textcolor{red}{\XSolidBrush}  &\textcolor{green}{\CheckmarkBold} & \textcolor{red}{\XSolidBrush}  \\
ToolLLaMA~\cite{qin2023toolllm} & ToolBench & IFT & - & 12.7K & 16,464 & \textcolor{red}{\XSolidBrush} &\textcolor{green}{\CheckmarkBold} &\textcolor{green}{\CheckmarkBold} &\textcolor{red}{\XSolidBrush} &\textcolor{green}{\CheckmarkBold}&\textcolor{green}{\CheckmarkBold}&\textcolor{green}{\CheckmarkBold} &\textcolor{green}{\CheckmarkBold}\\
AgentEvol~\citep{xi2024agentgym} & AgentTraj-L & IFT & - & 14.5K & 24 &\textcolor{red}{\XSolidBrush} & \textcolor{green}{\CheckmarkBold} &\textcolor{green}{\CheckmarkBold}&\textcolor{red}{\XSolidBrush} &\textcolor{green}{\CheckmarkBold}&\textcolor{red}{\XSolidBrush} &\textcolor{red}{\XSolidBrush} &\textcolor{green}{\CheckmarkBold}\\
Lumos~\cite{yin2024agent} & Lumos & IFT  & - & 20.0K & 16 &\textcolor{red}{\XSolidBrush} & \textcolor{green}{\CheckmarkBold} & \textcolor{green}{\CheckmarkBold} &\textcolor{red}{\XSolidBrush} & \textcolor{green}{\CheckmarkBold} & \textcolor{green}{\CheckmarkBold} &\textcolor{red}{\XSolidBrush} & \textcolor{green}{\CheckmarkBold}\\
Agent-FLAN~\cite{chen2024agent} & Agent-FLAN & IFT & - & 24.7K & 20 &\textcolor{red}{\XSolidBrush} & \textcolor{green}{\CheckmarkBold} & \textcolor{green}{\CheckmarkBold} &\textcolor{red}{\XSolidBrush} & \textcolor{green}{\CheckmarkBold}& \textcolor{green}{\CheckmarkBold}&\textcolor{red}{\XSolidBrush} & \textcolor{green}{\CheckmarkBold}\\
AgentTuning~\citep{zeng2023agenttuning} & AgentInstruct & IFT & - & 35.0K & - &\textcolor{red}{\XSolidBrush} & \textcolor{green}{\CheckmarkBold} & \textcolor{green}{\CheckmarkBold} &\textcolor{red}{\XSolidBrush} & \textcolor{green}{\CheckmarkBold} &\textcolor{red}{\XSolidBrush} &\textcolor{red}{\XSolidBrush} & \textcolor{green}{\CheckmarkBold}\\\midrule
\multicolumn{13}{l}{\emph{Instruction Finetuning-based LLM Agents for Function Calling}} \\\midrule
NexusRaven~\citep{srinivasan2023nexusraven} & NexusRaven & IFT & - & - & 116 & \textcolor{green}{\CheckmarkBold} & \textcolor{green}{\CheckmarkBold}  & \textcolor{green}{\CheckmarkBold} &\textcolor{red}{\XSolidBrush} & \textcolor{green}{\CheckmarkBold} &\textcolor{red}{\XSolidBrush} &\textcolor{red}{\XSolidBrush}&\textcolor{red}{\XSolidBrush}\\
Gorilla~\citep{patil2023gorilla} & Gorilla & IFT & - & 16.0K & 1,645 & \textcolor{green}{\CheckmarkBold} &\textcolor{red}{\XSolidBrush} &\textcolor{red}{\XSolidBrush}&\textcolor{green}{\CheckmarkBold} &\textcolor{green}{\CheckmarkBold} &\textcolor{red}{\XSolidBrush} &\textcolor{red}{\XSolidBrush} &\textcolor{red}{\XSolidBrush}\\
OpenFunctions-v2~\citep{patil2023gorilla} & OpenFunctions-v2 & IFT & - & 65.0K & - & \textcolor{green}{\CheckmarkBold} & \textcolor{green}{\CheckmarkBold} &\textcolor{red}{\XSolidBrush} &\textcolor{green}{\CheckmarkBold} &\textcolor{green}{\CheckmarkBold} &\textcolor{red}{\XSolidBrush} &\textcolor{red}{\XSolidBrush} &\textcolor{red}{\XSolidBrush}\\
API Pack~\cite{guo2024api} & API Pack & IFT & - & 1.1M & 11,213 &\textcolor{green}{\CheckmarkBold} &\textcolor{red}{\XSolidBrush} &\textcolor{green}{\CheckmarkBold} &\textcolor{red}{\XSolidBrush} &\textcolor{green}{\CheckmarkBold} &\textcolor{red}{\XSolidBrush}&\textcolor{red}{\XSolidBrush}&\textcolor{red}{\XSolidBrush}\\ 
LAM~\citep{zhang2024agentohana} & AgentOhana & IFT & - & 42.6K & - & \textcolor{green}{\CheckmarkBold} & \textcolor{green}{\CheckmarkBold} &\textcolor{green}{\CheckmarkBold}&\textcolor{red}{\XSolidBrush} &\textcolor{green}{\CheckmarkBold}&\textcolor{red}{\XSolidBrush}&\textcolor{green}{\CheckmarkBold}&\textcolor{green}{\CheckmarkBold}\\
xLAM~\citep{liu2024apigen} & APIGen & IFT & - & 60.0K & 3,673 & \textcolor{green}{\CheckmarkBold} & \textcolor{green}{\CheckmarkBold} &\textcolor{green}{\CheckmarkBold}&\textcolor{red}{\XSolidBrush} &\textcolor{green}{\CheckmarkBold}&\textcolor{red}{\XSolidBrush}&\textcolor{green}{\CheckmarkBold}&\textcolor{green}{\CheckmarkBold}\\\midrule
\multicolumn{13}{l}{\emph{Pretraining-based LLM Agents}}  \\\midrule
% LEMUR~\citep{xu2024lemur} & PT & 90B & 300.0K & - & \textcolor{green}{\CheckmarkBold} & \textcolor{green}{\CheckmarkBold} &\textcolor{green}{\CheckmarkBold}&\textcolor{red}{\XSolidBrush} & \textcolor{red}{\XSolidBrush} &\textcolor{green}{\CheckmarkBold} &\textcolor{red}{\XSolidBrush}&\textcolor{red}{\XSolidBrush}\\
\rowcolor{teal!12} \method & \dataset & PT & 103B & 95.0K  & 76,537  & \textcolor{green}{\CheckmarkBold} & \textcolor{green}{\CheckmarkBold} & \textcolor{green}{\CheckmarkBold} & \textcolor{green}{\CheckmarkBold} & \textcolor{green}{\CheckmarkBold} & \textcolor{green}{\CheckmarkBold} & \textcolor{green}{\CheckmarkBold} & \textcolor{green}{\CheckmarkBold}\\
\bottomrule
\end{tabular}
% \begin{tablenotes}
%     \item $^*$ In addition, the StarCoder-API can offer 4.77M more APIs.
% \end{tablenotes}
\caption{Summary of existing instruction finetuning-based LLM agents for intrinsic reasoning and function calling, along with their training resources and sample sizes. "PT" and "IFT" denote "Pre-Training" and "Instruction Fine-Tuning", respectively.}
\vspace{-2ex}
\label{tab:related}
\end{threeparttable}
\end{table*}

\noindent \textbf{Prompting-based LLM Agents.} Due to the lack of agent-specific pre-training corpus, existing LLM agents rely on either prompt engineering~\cite{hsieh2023tool,lu2024chameleon,yao2022react,wang2023voyager} or instruction fine-tuning~\cite{chen2023fireact,zeng2023agenttuning} to understand human instructions, decompose high-level tasks, generate grounded plans, and execute multi-step actions. 
However, prompting-based methods mainly depend on the capabilities of backbone LLMs (usually commercial LLMs), failing to introduce new knowledge and struggling to generalize to unseen tasks~\cite{sun2024adaplanner,zhuang2023toolchain}. 

\noindent \textbf{Instruction Finetuning-based LLM Agents.} Considering the extensive diversity of APIs and the complexity of multi-tool instructions, tool learning inherently presents greater challenges than natural language tasks, such as text generation~\cite{qin2023toolllm}.
Post-training techniques focus more on instruction following and aligning output with specific formats~\cite{patil2023gorilla,hao2024toolkengpt,qin2023toolllm,schick2024toolformer}, rather than fundamentally improving model knowledge or capabilities. 
Moreover, heavy fine-tuning can hinder generalization or even degrade performance in non-agent use cases, potentially suppressing the original base model capabilities~\cite{ghosh2024a}.

\noindent \textbf{Pretraining-based LLM Agents.} While pre-training serves as an essential alternative, prior works~\cite{nijkamp2023codegen,roziere2023code,xu2024lemur,patil2023gorilla} have primarily focused on improving task-specific capabilities (\eg, code generation) instead of general-domain LLM agents, due to single-source, uni-type, small-scale, and poor-quality pre-training data. 
Existing tool documentation data for agent training either lacks diverse real-world APIs~\cite{patil2023gorilla, tang2023toolalpaca} or is constrained to single-tool or single-round tool execution. 
Furthermore, trajectory data mostly imitate expert behavior or follow function-calling rules with inferior planning and reasoning, failing to fully elicit LLMs' capabilities and handle complex instructions~\cite{qin2023toolllm}. 
Given a wide range of candidate API functions, each comprising various function names and parameters available at every planning step, identifying globally optimal solutions and generalizing across tasks remains highly challenging.



\section{Preliminaries}
\label{Preliminaries}
\begin{figure*}[t]
    \centering
    \includegraphics[width=0.95\linewidth]{fig/HealthGPT_Framework.png}
    \caption{The \ourmethod{} architecture integrates hierarchical visual perception and H-LoRA, employing a task-specific hard router to select visual features and H-LoRA plugins, ultimately generating outputs with an autoregressive manner.}
    \label{fig:architecture}
\end{figure*}
\noindent\textbf{Large Vision-Language Models.} 
The input to a LVLM typically consists of an image $x^{\text{img}}$ and a discrete text sequence $x^{\text{txt}}$. The visual encoder $\mathcal{E}^{\text{img}}$ converts the input image $x^{\text{img}}$ into a sequence of visual tokens $\mathcal{V} = [v_i]_{i=1}^{N_v}$, while the text sequence $x^{\text{txt}}$ is mapped into a sequence of text tokens $\mathcal{T} = [t_i]_{i=1}^{N_t}$ using an embedding function $\mathcal{E}^{\text{txt}}$. The LLM $\mathcal{M_\text{LLM}}(\cdot|\theta)$ models the joint probability of the token sequence $\mathcal{U} = \{\mathcal{V},\mathcal{T}\}$, which is expressed as:
\begin{equation}
    P_\theta(R | \mathcal{U}) = \prod_{i=1}^{N_r} P_\theta(r_i | \{\mathcal{U}, r_{<i}\}),
\end{equation}
where $R = [r_i]_{i=1}^{N_r}$ is the text response sequence. The LVLM iteratively generates the next token $r_i$ based on $r_{<i}$. The optimization objective is to minimize the cross-entropy loss of the response $\mathcal{R}$.
% \begin{equation}
%     \mathcal{L}_{\text{VLM}} = \mathbb{E}_{R|\mathcal{U}}\left[-\log P_\theta(R | \mathcal{U})\right]
% \end{equation}
It is worth noting that most LVLMs adopt a design paradigm based on ViT, alignment adapters, and pre-trained LLMs\cite{liu2023llava,liu2024improved}, enabling quick adaptation to downstream tasks.


\noindent\textbf{VQGAN.}
VQGAN~\cite{esser2021taming} employs latent space compression and indexing mechanisms to effectively learn a complete discrete representation of images. VQGAN first maps the input image $x^{\text{img}}$ to a latent representation $z = \mathcal{E}(x)$ through a encoder $\mathcal{E}$. Then, the latent representation is quantized using a codebook $\mathcal{Z} = \{z_k\}_{k=1}^K$, generating a discrete index sequence $\mathcal{I} = [i_m]_{m=1}^N$, where $i_m \in \mathcal{Z}$ represents the quantized code index:
\begin{equation}
    \mathcal{I} = \text{Quantize}(z|\mathcal{Z}) = \arg\min_{z_k \in \mathcal{Z}} \| z - z_k \|_2.
\end{equation}
In our approach, the discrete index sequence $\mathcal{I}$ serves as a supervisory signal for the generation task, enabling the model to predict the index sequence $\hat{\mathcal{I}}$ from input conditions such as text or other modality signals.  
Finally, the predicted index sequence $\hat{\mathcal{I}}$ is upsampled by the VQGAN decoder $G$, generating the high-quality image $\hat{x}^\text{img} = G(\hat{\mathcal{I}})$.



\noindent\textbf{Low Rank Adaptation.} 
LoRA\cite{hu2021lora} effectively captures the characteristics of downstream tasks by introducing low-rank adapters. The core idea is to decompose the bypass weight matrix $\Delta W\in\mathbb{R}^{d^{\text{in}} \times d^{\text{out}}}$ into two low-rank matrices $ \{A \in \mathbb{R}^{d^{\text{in}} \times r}, B \in \mathbb{R}^{r \times d^{\text{out}}} \}$, where $ r \ll \min\{d^{\text{in}}, d^{\text{out}}\} $, significantly reducing learnable parameters. The output with the LoRA adapter for the input $x$ is then given by:
\begin{equation}
    h = x W_0 + \alpha x \Delta W/r = x W_0 + \alpha xAB/r,
\end{equation}
where matrix $ A $ is initialized with a Gaussian distribution, while the matrix $ B $ is initialized as a zero matrix. The scaling factor $ \alpha/r $ controls the impact of $ \Delta W $ on the model.

\section{HealthGPT}
\label{Method}


\subsection{Unified Autoregressive Generation.}  
% As shown in Figure~\ref{fig:architecture}, 
\ourmethod{} (Figure~\ref{fig:architecture}) utilizes a discrete token representation that covers both text and visual outputs, unifying visual comprehension and generation as an autoregressive task. 
For comprehension, $\mathcal{M}_\text{llm}$ receives the input joint sequence $\mathcal{U}$ and outputs a series of text token $\mathcal{R} = [r_1, r_2, \dots, r_{N_r}]$, where $r_i \in \mathcal{V}_{\text{txt}}$, and $\mathcal{V}_{\text{txt}}$ represents the LLM's vocabulary:
\begin{equation}
    P_\theta(\mathcal{R} \mid \mathcal{U}) = \prod_{i=1}^{N_r} P_\theta(r_i \mid \mathcal{U}, r_{<i}).
\end{equation}
For generation, $\mathcal{M}_\text{llm}$ first receives a special start token $\langle \text{START\_IMG} \rangle$, then generates a series of tokens corresponding to the VQGAN indices $\mathcal{I} = [i_1, i_2, \dots, i_{N_i}]$, where $i_j \in \mathcal{V}_{\text{vq}}$, and $\mathcal{V}_{\text{vq}}$ represents the index range of VQGAN. Upon completion of generation, the LLM outputs an end token $\langle \text{END\_IMG} \rangle$:
\begin{equation}
    P_\theta(\mathcal{I} \mid \mathcal{U}) = \prod_{j=1}^{N_i} P_\theta(i_j \mid \mathcal{U}, i_{<j}).
\end{equation}
Finally, the generated index sequence $\mathcal{I}$ is fed into the decoder $G$, which reconstructs the target image $\hat{x}^{\text{img}} = G(\mathcal{I})$.

\subsection{Hierarchical Visual Perception}  
Given the differences in visual perception between comprehension and generation tasks—where the former focuses on abstract semantics and the latter emphasizes complete semantics—we employ ViT to compress the image into discrete visual tokens at multiple hierarchical levels.
Specifically, the image is converted into a series of features $\{f_1, f_2, \dots, f_L\}$ as it passes through $L$ ViT blocks.

To address the needs of various tasks, the hidden states are divided into two types: (i) \textit{Concrete-grained features} $\mathcal{F}^{\text{Con}} = \{f_1, f_2, \dots, f_k\}, k < L$, derived from the shallower layers of ViT, containing sufficient global features, suitable for generation tasks; 
(ii) \textit{Abstract-grained features} $\mathcal{F}^{\text{Abs}} = \{f_{k+1}, f_{k+2}, \dots, f_L\}$, derived from the deeper layers of ViT, which contain abstract semantic information closer to the text space, suitable for comprehension tasks.

The task type $T$ (comprehension or generation) determines which set of features is selected as the input for the downstream large language model:
\begin{equation}
    \mathcal{F}^{\text{img}}_T =
    \begin{cases}
        \mathcal{F}^{\text{Con}}, & \text{if } T = \text{generation task} \\
        \mathcal{F}^{\text{Abs}}, & \text{if } T = \text{comprehension task}
    \end{cases}
\end{equation}
We integrate the image features $\mathcal{F}^{\text{img}}_T$ and text features $\mathcal{T}$ into a joint sequence through simple concatenation, which is then fed into the LLM $\mathcal{M}_{\text{llm}}$ for autoregressive generation.
% :
% \begin{equation}
%     \mathcal{R} = \mathcal{M}_{\text{llm}}(\mathcal{U}|\theta), \quad \mathcal{U} = [\mathcal{F}^{\text{img}}_T; \mathcal{T}]
% \end{equation}
\subsection{Heterogeneous Knowledge Adaptation}
We devise H-LoRA, which stores heterogeneous knowledge from comprehension and generation tasks in separate modules and dynamically routes to extract task-relevant knowledge from these modules. 
At the task level, for each task type $ T $, we dynamically assign a dedicated H-LoRA submodule $ \theta^T $, which is expressed as:
\begin{equation}
    \mathcal{R} = \mathcal{M}_\text{LLM}(\mathcal{U}|\theta, \theta^T), \quad \theta^T = \{A^T, B^T, \mathcal{R}^T_\text{outer}\}.
\end{equation}
At the feature level for a single task, H-LoRA integrates the idea of Mixture of Experts (MoE)~\cite{masoudnia2014mixture} and designs an efficient matrix merging and routing weight allocation mechanism, thus avoiding the significant computational delay introduced by matrix splitting in existing MoELoRA~\cite{luo2024moelora}. Specifically, we first merge the low-rank matrices (rank = r) of $ k $ LoRA experts into a unified matrix:
\begin{equation}
    \mathbf{A}^{\text{merged}}, \mathbf{B}^{\text{merged}} = \text{Concat}(\{A_i\}_1^k), \text{Concat}(\{B_i\}_1^k),
\end{equation}
where $ \mathbf{A}^{\text{merged}} \in \mathbb{R}^{d^\text{in} \times rk} $ and $ \mathbf{B}^{\text{merged}} \in \mathbb{R}^{rk \times d^\text{out}} $. The $k$-dimension routing layer generates expert weights $ \mathcal{W} \in \mathbb{R}^{\text{token\_num} \times k} $ based on the input hidden state $ x $, and these are expanded to $ \mathbb{R}^{\text{token\_num} \times rk} $ as follows:
\begin{equation}
    \mathcal{W}^\text{expanded} = \alpha k \mathcal{W} / r \otimes \mathbf{1}_r,
\end{equation}
where $ \otimes $ denotes the replication operation.
The overall output of H-LoRA is computed as:
\begin{equation}
    \mathcal{O}^\text{H-LoRA} = (x \mathbf{A}^{\text{merged}} \odot \mathcal{W}^\text{expanded}) \mathbf{B}^{\text{merged}},
\end{equation}
where $ \odot $ represents element-wise multiplication. Finally, the output of H-LoRA is added to the frozen pre-trained weights to produce the final output:
\begin{equation}
    \mathcal{O} = x W_0 + \mathcal{O}^\text{H-LoRA}.
\end{equation}
% In summary, H-LoRA is a task-based dynamic PEFT method that achieves high efficiency in single-task fine-tuning.

\subsection{Training Pipeline}

\begin{figure}[t]
    \centering
    \hspace{-4mm}
    \includegraphics[width=0.94\linewidth]{fig/data.pdf}
    \caption{Data statistics of \texttt{VL-Health}. }
    \label{fig:data}
\end{figure}
\noindent \textbf{1st Stage: Multi-modal Alignment.} 
In the first stage, we design separate visual adapters and H-LoRA submodules for medical unified tasks. For the medical comprehension task, we train abstract-grained visual adapters using high-quality image-text pairs to align visual embeddings with textual embeddings, thereby enabling the model to accurately describe medical visual content. During this process, the pre-trained LLM and its corresponding H-LoRA submodules remain frozen. In contrast, the medical generation task requires training concrete-grained adapters and H-LoRA submodules while keeping the LLM frozen. Meanwhile, we extend the textual vocabulary to include multimodal tokens, enabling the support of additional VQGAN vector quantization indices. The model trains on image-VQ pairs, endowing the pre-trained LLM with the capability for image reconstruction. This design ensures pixel-level consistency of pre- and post-LVLM. The processes establish the initial alignment between the LLM’s outputs and the visual inputs.

\noindent \textbf{2nd Stage: Heterogeneous H-LoRA Plugin Adaptation.}  
The submodules of H-LoRA share the word embedding layer and output head but may encounter issues such as bias and scale inconsistencies during training across different tasks. To ensure that the multiple H-LoRA plugins seamlessly interface with the LLMs and form a unified base, we fine-tune the word embedding layer and output head using a small amount of mixed data to maintain consistency in the model weights. Specifically, during this stage, all H-LoRA submodules for different tasks are kept frozen, with only the word embedding layer and output head being optimized. Through this stage, the model accumulates foundational knowledge for unified tasks by adapting H-LoRA plugins.

\begin{table*}[!t]
\centering
\caption{Comparison of \ourmethod{} with other LVLMs and unified multi-modal models on medical visual comprehension tasks. \textbf{Bold} and \underline{underlined} text indicates the best performance and second-best performance, respectively.}
\resizebox{\textwidth}{!}{
\begin{tabular}{c|lcc|cccccccc|c}
\toprule
\rowcolor[HTML]{E9F3FE} &  &  &  & \multicolumn{2}{c}{\textbf{VQA-RAD \textuparrow}} & \multicolumn{2}{c}{\textbf{SLAKE \textuparrow}} & \multicolumn{2}{c}{\textbf{PathVQA \textuparrow}} &  &  &  \\ 
\cline{5-10}
\rowcolor[HTML]{E9F3FE}\multirow{-2}{*}{\textbf{Type}} & \multirow{-2}{*}{\textbf{Model}} & \multirow{-2}{*}{\textbf{\# Params}} & \multirow{-2}{*}{\makecell{\textbf{Medical} \\ \textbf{LVLM}}} & \textbf{close} & \textbf{all} & \textbf{close} & \textbf{all} & \textbf{close} & \textbf{all} & \multirow{-2}{*}{\makecell{\textbf{MMMU} \\ \textbf{-Med}}\textuparrow} & \multirow{-2}{*}{\textbf{OMVQA}\textuparrow} & \multirow{-2}{*}{\textbf{Avg. \textuparrow}} \\ 
\midrule \midrule
\multirow{9}{*}{\textbf{Comp. Only}} 
& Med-Flamingo & 8.3B & \Large \ding{51} & 58.6 & 43.0 & 47.0 & 25.5 & 61.9 & 31.3 & 28.7 & 34.9 & 41.4 \\
& LLaVA-Med & 7B & \Large \ding{51} & 60.2 & 48.1 & 58.4 & 44.8 & 62.3 & 35.7 & 30.0 & 41.3 & 47.6 \\
& HuatuoGPT-Vision & 7B & \Large \ding{51} & 66.9 & 53.0 & 59.8 & 49.1 & 52.9 & 32.0 & 42.0 & 50.0 & 50.7 \\
& BLIP-2 & 6.7B & \Large \ding{55} & 43.4 & 36.8 & 41.6 & 35.3 & 48.5 & 28.8 & 27.3 & 26.9 & 36.1 \\
& LLaVA-v1.5 & 7B & \Large \ding{55} & 51.8 & 42.8 & 37.1 & 37.7 & 53.5 & 31.4 & 32.7 & 44.7 & 41.5 \\
& InstructBLIP & 7B & \Large \ding{55} & 61.0 & 44.8 & 66.8 & 43.3 & 56.0 & 32.3 & 25.3 & 29.0 & 44.8 \\
& Yi-VL & 6B & \Large \ding{55} & 52.6 & 42.1 & 52.4 & 38.4 & 54.9 & 30.9 & 38.0 & 50.2 & 44.9 \\
& InternVL2 & 8B & \Large \ding{55} & 64.9 & 49.0 & 66.6 & 50.1 & 60.0 & 31.9 & \underline{43.3} & 54.5 & 52.5\\
& Llama-3.2 & 11B & \Large \ding{55} & 68.9 & 45.5 & 72.4 & 52.1 & 62.8 & 33.6 & 39.3 & 63.2 & 54.7 \\
\midrule
\multirow{5}{*}{\textbf{Comp. \& Gen.}} 
& Show-o & 1.3B & \Large \ding{55} & 50.6 & 33.9 & 31.5 & 17.9 & 52.9 & 28.2 & 22.7 & 45.7 & 42.6 \\
& Unified-IO 2 & 7B & \Large \ding{55} & 46.2 & 32.6 & 35.9 & 21.9 & 52.5 & 27.0 & 25.3 & 33.0 & 33.8 \\
& Janus & 1.3B & \Large \ding{55} & 70.9 & 52.8 & 34.7 & 26.9 & 51.9 & 27.9 & 30.0 & 26.8 & 33.5 \\
& \cellcolor[HTML]{DAE0FB}HealthGPT-M3 & \cellcolor[HTML]{DAE0FB}3.8B & \cellcolor[HTML]{DAE0FB}\Large \ding{51} & \cellcolor[HTML]{DAE0FB}\underline{73.7} & \cellcolor[HTML]{DAE0FB}\underline{55.9} & \cellcolor[HTML]{DAE0FB}\underline{74.6} & \cellcolor[HTML]{DAE0FB}\underline{56.4} & \cellcolor[HTML]{DAE0FB}\underline{78.7} & \cellcolor[HTML]{DAE0FB}\underline{39.7} & \cellcolor[HTML]{DAE0FB}\underline{43.3} & \cellcolor[HTML]{DAE0FB}\underline{68.5} & \cellcolor[HTML]{DAE0FB}\underline{61.3} \\
& \cellcolor[HTML]{DAE0FB}HealthGPT-L14 & \cellcolor[HTML]{DAE0FB}14B & \cellcolor[HTML]{DAE0FB}\Large \ding{51} & \cellcolor[HTML]{DAE0FB}\textbf{77.7} & \cellcolor[HTML]{DAE0FB}\textbf{58.3} & \cellcolor[HTML]{DAE0FB}\textbf{76.4} & \cellcolor[HTML]{DAE0FB}\textbf{64.5} & \cellcolor[HTML]{DAE0FB}\textbf{85.9} & \cellcolor[HTML]{DAE0FB}\textbf{44.4} & \cellcolor[HTML]{DAE0FB}\textbf{49.2} & \cellcolor[HTML]{DAE0FB}\textbf{74.4} & \cellcolor[HTML]{DAE0FB}\textbf{66.4} \\
\bottomrule
\end{tabular}
}
\label{tab:results}
\end{table*}
\begin{table*}[ht]
    \centering
    \caption{The experimental results for the four modality conversion tasks.}
    \resizebox{\textwidth}{!}{
    \begin{tabular}{l|ccc|ccc|ccc|ccc}
        \toprule
        \rowcolor[HTML]{E9F3FE} & \multicolumn{3}{c}{\textbf{CT to MRI (Brain)}} & \multicolumn{3}{c}{\textbf{CT to MRI (Pelvis)}} & \multicolumn{3}{c}{\textbf{MRI to CT (Brain)}} & \multicolumn{3}{c}{\textbf{MRI to CT (Pelvis)}} \\
        \cline{2-13}
        \rowcolor[HTML]{E9F3FE}\multirow{-2}{*}{\textbf{Model}}& \textbf{SSIM $\uparrow$} & \textbf{PSNR $\uparrow$} & \textbf{MSE $\downarrow$} & \textbf{SSIM $\uparrow$} & \textbf{PSNR $\uparrow$} & \textbf{MSE $\downarrow$} & \textbf{SSIM $\uparrow$} & \textbf{PSNR $\uparrow$} & \textbf{MSE $\downarrow$} & \textbf{SSIM $\uparrow$} & \textbf{PSNR $\uparrow$} & \textbf{MSE $\downarrow$} \\
        \midrule \midrule
        pix2pix & 71.09 & 32.65 & 36.85 & 59.17 & 31.02 & 51.91 & 78.79 & 33.85 & 28.33 & 72.31 & 32.98 & 36.19 \\
        CycleGAN & 54.76 & 32.23 & 40.56 & 54.54 & 30.77 & 55.00 & 63.75 & 31.02 & 52.78 & 50.54 & 29.89 & 67.78 \\
        BBDM & {71.69} & {32.91} & {34.44} & 57.37 & 31.37 & 48.06 & \textbf{86.40} & 34.12 & 26.61 & {79.26} & 33.15 & 33.60 \\
        Vmanba & 69.54 & 32.67 & 36.42 & {63.01} & {31.47} & {46.99} & 79.63 & 34.12 & 26.49 & 77.45 & 33.53 & 31.85 \\
        DiffMa & 71.47 & 32.74 & 35.77 & 62.56 & 31.43 & 47.38 & 79.00 & {34.13} & {26.45} & 78.53 & {33.68} & {30.51} \\
        \rowcolor[HTML]{DAE0FB}HealthGPT-M3 & \underline{79.38} & \underline{33.03} & \underline{33.48} & \underline{71.81} & \underline{31.83} & \underline{43.45} & {85.06} & \textbf{34.40} & \textbf{25.49} & \underline{84.23} & \textbf{34.29} & \textbf{27.99} \\
        \rowcolor[HTML]{DAE0FB}HealthGPT-L14 & \textbf{79.73} & \textbf{33.10} & \textbf{32.96} & \textbf{71.92} & \textbf{31.87} & \textbf{43.09} & \underline{85.31} & \underline{34.29} & \underline{26.20} & \textbf{84.96} & \underline{34.14} & \underline{28.13} \\
        \bottomrule
    \end{tabular}
    }
    \label{tab:conversion}
\end{table*}

\noindent \textbf{3rd Stage: Visual Instruction Fine-Tuning.}  
In the third stage, we introduce additional task-specific data to further optimize the model and enhance its adaptability to downstream tasks such as medical visual comprehension (e.g., medical QA, medical dialogues, and report generation) or generation tasks (e.g., super-resolution, denoising, and modality conversion). Notably, by this stage, the word embedding layer and output head have been fine-tuned, only the H-LoRA modules and adapter modules need to be trained. This strategy significantly improves the model's adaptability and flexibility across different tasks.


\section{Experiment}
\label{s:experiment}

\subsection{Data Description}
We evaluate our method on FI~\cite{you2016building}, Twitter\_LDL~\cite{yang2017learning} and Artphoto~\cite{machajdik2010affective}.
FI is a public dataset built from Flickr and Instagram, with 23,308 images and eight emotion categories, namely \textit{amusement}, \textit{anger}, \textit{awe},  \textit{contentment}, \textit{disgust}, \textit{excitement},  \textit{fear}, and \textit{sadness}. 
% Since images in FI are all copyrighted by law, some images are corrupted now, so we remove these samples and retain 21,828 images.
% T4SA contains images from Twitter, which are classified into three categories: \textit{positive}, \textit{neutral}, and \textit{negative}. In this paper, we adopt the base version of B-T4SA, which contains 470,586 images and provides text descriptions of the corresponding tweets.
Twitter\_LDL contains 10,045 images from Twitter, with the same eight categories as the FI dataset.
% 。
For these two datasets, they are randomly split into 80\%
training and 20\% testing set.
Artphoto contains 806 artistic photos from the DeviantArt website, which we use to further evaluate the zero-shot capability of our model.
% on the small-scale dataset.
% We construct and publicly release the first image sentiment analysis dataset containing metadata.
% 。

% Based on these datasets, we are the first to construct and publicly release metadata-enhanced image sentiment analysis datasets. These datasets include scenes, tags, descriptions, and corresponding confidence scores, and are available at this link for future research purposes.


% 
\begin{table}[t]
\centering
% \begin{center}
\caption{Overall performance of different models on FI and Twitter\_LDL datasets.}
\label{tab:cap1}
% \resizebox{\linewidth}{!}
{
\begin{tabular}{l|c|c|c|c}
\hline
\multirow{2}{*}{\textbf{Model}} & \multicolumn{2}{c|}{\textbf{FI}}  & \multicolumn{2}{c}{\textbf{Twitter\_LDL}} \\ \cline{2-5} 
  & \textbf{Accuracy} & \textbf{F1} & \textbf{Accuracy} & \textbf{F1}  \\ \hline
% (\rownumber)~AlexNet~\cite{krizhevsky2017imagenet}  & 58.13\% & 56.35\%  & 56.24\%& 55.02\%  \\ 
% (\rownumber)~VGG16~\cite{simonyan2014very}  & 63.75\%& 63.08\%  & 59.34\%& 59.02\%  \\ 
(\rownumber)~ResNet101~\cite{he2016deep} & 66.16\%& 65.56\%  & 62.02\% & 61.34\%  \\ 
(\rownumber)~CDA~\cite{han2023boosting} & 66.71\%& 65.37\%  & 64.14\% & 62.85\%  \\ 
(\rownumber)~CECCN~\cite{ruan2024color} & 67.96\%& 66.74\%  & 64.59\%& 64.72\% \\ 
(\rownumber)~EmoVIT~\cite{xie2024emovit} & 68.09\%& 67.45\%  & 63.12\% & 61.97\%  \\ 
(\rownumber)~ComLDL~\cite{zhang2022compound} & 68.83\%& 67.28\%  & 65.29\% & 63.12\%  \\ 
(\rownumber)~WSDEN~\cite{li2023weakly} & 69.78\%& 69.61\%  & 67.04\% & 65.49\% \\ 
(\rownumber)~ECWA~\cite{deng2021emotion} & 70.87\%& 69.08\%  & 67.81\% & 66.87\%  \\ 
(\rownumber)~EECon~\cite{yang2023exploiting} & 71.13\%& 68.34\%  & 64.27\%& 63.16\%  \\ 
(\rownumber)~MAM~\cite{zhang2024affective} & 71.44\%  & 70.83\% & 67.18\%  & 65.01\%\\ 
(\rownumber)~TGCA-PVT~\cite{chen2024tgca}   & 73.05\%  & 71.46\% & 69.87\%  & 68.32\% \\ 
(\rownumber)~OEAN~\cite{zhang2024object}   & 73.40\%  & 72.63\% & 70.52\%  & 69.47\% \\ \hline
(\rownumber)~\shortname  & \textbf{79.48\%} & \textbf{79.22\%} & \textbf{74.12\%} & \textbf{73.09\%} \\ \hline
\end{tabular}
}
\vspace{-6mm}
% \end{center}
\end{table}
% 

\subsection{Experiment Setting}
% \subsubsection{Model Setting.}
% 
\textbf{Model Setting:}
For feature representation, we set $k=10$ to select object tags, and adopt clip-vit-base-patch32 as the pre-trained model for unified feature representation.
Moreover, we empirically set $(d_e, d_h, d_k, d_s) = (512, 128, 16, 64)$, and set the classification class $L$ to 8.

% 

\textbf{Training Setting:}
To initialize the model, we set all weights such as $\boldsymbol{W}$ following the truncated normal distribution, and use AdamW optimizer with the learning rate of $1 \times 10^{-4}$.
% warmup scheduler of cosine, warmup steps of 2000.
Furthermore, we set the batch size to 32 and the epoch of the training process to 200.
During the implementation, we utilize \textit{PyTorch} to build our entire model.
% , and our project codes are publicly available at https://github.com/zzmyrep/MESN.
% Our project codes as well as data are all publicly available on GitHub\footnote{https://github.com/zzmyrep/KBCEN}.
% Code is available at \href{https://github.com/zzmyrep/KBCEN}{https://github.com/zzmyrep/KBCEN}.

\textbf{Evaluation Metrics:}
Following~\cite{zhang2024affective, chen2024tgca, zhang2024object}, we adopt \textit{accuracy} and \textit{F1} as our evaluation metrics to measure the performance of different methods for image sentiment analysis. 



\subsection{Experiment Result}
% We compare our model against the following baselines: AlexNet~\cite{krizhevsky2017imagenet}, VGG16~\cite{simonyan2014very}, ResNet101~\cite{he2016deep}, CECCN~\cite{ruan2024color}, EmoVIT~\cite{xie2024emovit}, WSCNet~\cite{yang2018weakly}, ECWA~\cite{deng2021emotion}, EECon~\cite{yang2023exploiting}, MAM~\cite{zhang2024affective} and TGCA-PVT~\cite{chen2024tgca}, and the overall results are summarized in Table~\ref{tab:cap1}.
We compare our model against several baselines, and the overall results are summarized in Table~\ref{tab:cap1}.
We observe that our model achieves the best performance in both accuracy and F1 metrics, significantly outperforming the previous models. 
This superior performance is mainly attributed to our effective utilization of metadata to enhance image sentiment analysis, as well as the exceptional capability of the unified sentiment transformer framework we developed. These results strongly demonstrate that our proposed method can bring encouraging performance for image sentiment analysis.

\setcounter{magicrownumbers}{0} 
\begin{table}[t]
\begin{center}
\caption{Ablation study of~\shortname~on FI dataset.} 
% \vspace{1mm}
\label{tab:cap2}
\resizebox{.9\linewidth}{!}
{
\begin{tabular}{lcc}
  \hline
  \textbf{Model} & \textbf{Accuracy} & \textbf{F1} \\
  \hline
  (\rownumber)~Ours (w/o vision) & 65.72\% & 64.54\% \\
  (\rownumber)~Ours (w/o text description) & 74.05\% & 72.58\% \\
  (\rownumber)~Ours (w/o object tag) & 77.45\% & 76.84\% \\
  (\rownumber)~Ours (w/o scene tag) & 78.47\% & 78.21\% \\
  \hline
  (\rownumber)~Ours (w/o unified embedding) & 76.41\% & 76.23\% \\
  (\rownumber)~Ours (w/o adaptive learning) & 76.83\% & 76.56\% \\
  (\rownumber)~Ours (w/o cross-modal fusion) & 76.85\% & 76.49\% \\
  \hline
  (\rownumber)~Ours  & \textbf{79.48\%} & \textbf{79.22\%} \\
  \hline
\end{tabular}
}
\end{center}
\vspace{-5mm}
\end{table}


\begin{figure}[t]
\centering
% \vspace{-2mm}
\includegraphics[width=0.42\textwidth]{fig/2dvisual-linux4-paper2.pdf}
\caption{Visualization of feature distribution on eight categories before (left) and after (right) model processing.}
% 
\label{fig:visualization}
\vspace{-5mm}
\end{figure}

\subsection{Ablation Performance}
In this subsection, we conduct an ablation study to examine which component is really important for performance improvement. The results are reported in Table~\ref{tab:cap2}.

For information utilization, we observe a significant decline in model performance when visual features are removed. Additionally, the performance of \shortname~decreases when different metadata are removed separately, which means that text description, object tag, and scene tag are all critical for image sentiment analysis.
Recalling the model architecture, we separately remove transformer layers of the unified representation module, the adaptive learning module, and the cross-modal fusion module, replacing them with MLPs of the same parameter scale.
In this way, we can observe varying degrees of decline in model performance, indicating that these modules are indispensable for our model to achieve better performance.

\subsection{Visualization}
% 


% % 开始使用minipage进行左右排列
% \begin{minipage}[t]{0.45\textwidth}  % 子图1宽度为45%
%     \centering
%     \includegraphics[width=\textwidth]{2dvisual.pdf}  % 插入图片
%     \captionof{figure}{Visualization of feature distribution.}  % 使用captionof添加图片标题
%     \label{fig:visualization}
% \end{minipage}


% \begin{figure}[t]
% \centering
% \vspace{-2mm}
% \includegraphics[width=0.45\textwidth]{fig/2dvisual.pdf}
% \caption{Visualization of feature distribution.}
% \label{fig:visualization}
% % \vspace{-4mm}
% \end{figure}

% \begin{figure}[t]
% \centering
% \vspace{-2mm}
% \includegraphics[width=0.45\textwidth]{fig/2dvisual-linux3-paper.pdf}
% \caption{Visualization of feature distribution.}
% \label{fig:visualization}
% % \vspace{-4mm}
% \end{figure}



\begin{figure}[tbp]   
\vspace{-4mm}
  \centering            
  \subfloat[Depth of adaptive learning layers]   
  {
    \label{fig:subfig1}\includegraphics[width=0.22\textwidth]{fig/fig_sensitivity-a5}
  }
  \subfloat[Depth of fusion layers]
  {
    % \label{fig:subfig2}\includegraphics[width=0.22\textwidth]{fig/fig_sensitivity-b2}
    \label{fig:subfig2}\includegraphics[width=0.22\textwidth]{fig/fig_sensitivity-b2-num.pdf}
  }
  \caption{Sensitivity study of \shortname~on different depth. }   
  \label{fig:fig_sensitivity}  
\vspace{-2mm}
\end{figure}

% \begin{figure}[htbp]
% \centerline{\includegraphics{2dvisual.pdf}}
% \caption{Visualization of feature distribution.}
% \label{fig:visualization}
% \end{figure}

% In Fig.~\ref{fig:visualization}, we use t-SNE~\cite{van2008visualizing} to reduce the dimension of data features for visualization, Figure in left represents the metadata features before model processing, the features are obtained by embedding through the CLIP model, and figure in right shows the features of the data after model processing, it can be observed that after the model processing, the data with different label categories fall in different regions in the space, therefore, we can conclude that the Therefore, we can conclude that the model can effectively utilize the information contained in the metadata and use it to guide the model for classification.

In Fig.~\ref{fig:visualization}, we use t-SNE~\cite{van2008visualizing} to reduce the dimension of data features for visualization.
The left figure shows metadata features before being processed by our model (\textit{i.e.}, embedded by CLIP), while the right shows the distribution of features after being processed by our model.
We can observe that after the model processing, data with the same label are closer to each other, while others are farther away.
Therefore, it shows that the model can effectively utilize the information contained in the metadata and use it to guide the classification process.

\subsection{Sensitivity Analysis}
% 
In this subsection, we conduct a sensitivity analysis to figure out the effect of different depth settings of adaptive learning layers and fusion layers. 
% In this subsection, we conduct a sensitivity analysis to figure out the effect of different depth settings on the model. 
% Fig.~\ref{fig:fig_sensitivity} presents the effect of different depth settings of adaptive learning layers and fusion layers. 
Taking Fig.~\ref{fig:fig_sensitivity} (a) as an example, the model performance improves with increasing depth, reaching the best performance at a depth of 4.
% Taking Fig.~\ref{fig:fig_sensitivity} (a) as an example, the performance of \shortname~improves with the increase of depth at first, reaching the best performance at a depth of 4.
When the depth continues to increase, the accuracy decreases to varying degrees.
Similar results can be observed in Fig.~\ref{fig:fig_sensitivity} (b).
Therefore, we set their depths to 4 and 6 respectively to achieve the best results.

% Through our experiments, we can observe that the effect of modifying these hyperparameters on the results of the experiments is very weak, and the surface model is not sensitive to the hyperparameters.


\subsection{Zero-shot Capability}
% 

% (1)~GCH~\cite{2010Analyzing} & 21.78\% & (5)~RA-DLNet~\cite{2020A} & 34.01\% \\ \hline
% (2)~WSCNet~\cite{2019WSCNet}  & 30.25\% & (6)~CECCN~\cite{ruan2024color} & 43.83\% \\ \hline
% (3)~PCNN~\cite{2015Robust} & 31.68\%  & (7)~EmoVIT~\cite{xie2024emovit} & 44.90\% \\ \hline
% (4)~AR~\cite{2018Visual} & 32.67\% & (8)~Ours (Zero-shot) & 47.83\% \\ \hline


\begin{table}[t]
\centering
\caption{Zero-shot capability of \shortname.}
\label{tab:cap3}
\resizebox{1\linewidth}{!}
{
\begin{tabular}{lc|lc}
\hline
\textbf{Model} & \textbf{Accuracy} & \textbf{Model} & \textbf{Accuracy} \\ \hline
(1)~WSCNet~\cite{2019WSCNet}  & 30.25\% & (5)~MAM~\cite{zhang2024affective} & 39.56\%  \\ \hline
(2)~AR~\cite{2018Visual} & 32.67\% & (6)~CECCN~\cite{ruan2024color} & 43.83\% \\ \hline
(3)~RA-DLNet~\cite{2020A} & 34.01\%  & (7)~EmoVIT~\cite{xie2024emovit} & 44.90\% \\ \hline
(4)~CDA~\cite{han2023boosting} & 38.64\% & (8)~Ours (Zero-shot) & 47.83\% \\ \hline
\end{tabular}
}
\vspace{-5mm}
\end{table}

% We use the model trained on the FI dataset to test on the artphoto dataset to verify the model's generalization ability as well as robustness to other distributed datasets.
% We can observe that the MESN model shows strong competitiveness in terms of accuracy when compared to other trained models, which suggests that the model has a good generalization ability in the OOD task.

To validate the model's generalization ability and robustness to other distributed datasets, we directly test the model trained on the FI dataset, without training on Artphoto. 
% As observed in Table 3, compared to other models trained on Artphoto, we achieve highly competitive zero-shot performance, indicating that the model has good generalization ability in out-of-distribution tasks.
From Table~\ref{tab:cap3}, we can observe that compared with other models trained on Artphoto, we achieve competitive zero-shot performance, which shows that the model has good generalization ability in out-of-distribution tasks.


%%%%%%%%%%%%
%  E2E     %
%%%%%%%%%%%%


\section{Conclusion}
In this paper, we introduced Wi-Chat, the first LLM-powered Wi-Fi-based human activity recognition system that integrates the reasoning capabilities of large language models with the sensing potential of wireless signals. Our experimental results on a self-collected Wi-Fi CSI dataset demonstrate the promising potential of LLMs in enabling zero-shot Wi-Fi sensing. These findings suggest a new paradigm for human activity recognition that does not rely on extensive labeled data. We hope future research will build upon this direction, further exploring the applications of LLMs in signal processing domains such as IoT, mobile sensing, and radar-based systems.

\section*{Limitations}
While our work represents the first attempt to leverage LLMs for processing Wi-Fi signals, it is a preliminary study focused on a relatively simple task: Wi-Fi-based human activity recognition. This choice allows us to explore the feasibility of LLMs in wireless sensing but also comes with certain limitations.

Our approach primarily evaluates zero-shot performance, which, while promising, may still lag behind traditional supervised learning methods in highly complex or fine-grained recognition tasks. Besides, our study is limited to a controlled environment with a self-collected dataset, and the generalizability of LLMs to diverse real-world scenarios with varying Wi-Fi conditions, environmental interference, and device heterogeneity remains an open question.

Additionally, we have yet to explore the full potential of LLMs in more advanced Wi-Fi sensing applications, such as fine-grained gesture recognition, occupancy detection, and passive health monitoring. Future work should investigate the scalability of LLM-based approaches, their robustness to domain shifts, and their integration with multimodal sensing techniques in broader IoT applications.


% Bibliography entries for the entire Anthology, followed by custom entries
%\bibliography{anthology,custom}
% Custom bibliography entries only
\bibliography{main}
\newpage
\appendix

\section{Experiment prompts}
\label{sec:prompt}
The prompts used in the LLM experiments are shown in the following Table~\ref{tab:prompts}.

\definecolor{titlecolor}{rgb}{0.9, 0.5, 0.1}
\definecolor{anscolor}{rgb}{0.2, 0.5, 0.8}
\definecolor{labelcolor}{HTML}{48a07e}
\begin{table*}[h]
	\centering
	
 % \vspace{-0.2cm}
	
	\begin{center}
		\begin{tikzpicture}[
				chatbox_inner/.style={rectangle, rounded corners, opacity=0, text opacity=1, font=\sffamily\scriptsize, text width=5in, text height=9pt, inner xsep=6pt, inner ysep=6pt},
				chatbox_prompt_inner/.style={chatbox_inner, align=flush left, xshift=0pt, text height=11pt},
				chatbox_user_inner/.style={chatbox_inner, align=flush left, xshift=0pt},
				chatbox_gpt_inner/.style={chatbox_inner, align=flush left, xshift=0pt},
				chatbox/.style={chatbox_inner, draw=black!25, fill=gray!7, opacity=1, text opacity=0},
				chatbox_prompt/.style={chatbox, align=flush left, fill=gray!1.5, draw=black!30, text height=10pt},
				chatbox_user/.style={chatbox, align=flush left},
				chatbox_gpt/.style={chatbox, align=flush left},
				chatbox2/.style={chatbox_gpt, fill=green!25},
				chatbox3/.style={chatbox_gpt, fill=red!20, draw=black!20},
				chatbox4/.style={chatbox_gpt, fill=yellow!30},
				labelbox/.style={rectangle, rounded corners, draw=black!50, font=\sffamily\scriptsize\bfseries, fill=gray!5, inner sep=3pt},
			]
											
			\node[chatbox_user] (q1) {
				\textbf{System prompt}
				\newline
				\newline
				You are a helpful and precise assistant for segmenting and labeling sentences. We would like to request your help on curating a dataset for entity-level hallucination detection.
				\newline \newline
                We will give you a machine generated biography and a list of checked facts about the biography. Each fact consists of a sentence and a label (True/False). Please do the following process. First, breaking down the biography into words. Second, by referring to the provided list of facts, merging some broken down words in the previous step to form meaningful entities. For example, ``strategic thinking'' should be one entity instead of two. Third, according to the labels in the list of facts, labeling each entity as True or False. Specifically, for facts that share a similar sentence structure (\eg, \textit{``He was born on Mach 9, 1941.''} (\texttt{True}) and \textit{``He was born in Ramos Mejia.''} (\texttt{False})), please first assign labels to entities that differ across atomic facts. For example, first labeling ``Mach 9, 1941'' (\texttt{True}) and ``Ramos Mejia'' (\texttt{False}) in the above case. For those entities that are the same across atomic facts (\eg, ``was born'') or are neutral (\eg, ``he,'' ``in,'' and ``on''), please label them as \texttt{True}. For the cases that there is no atomic fact that shares the same sentence structure, please identify the most informative entities in the sentence and label them with the same label as the atomic fact while treating the rest of the entities as \texttt{True}. In the end, output the entities and labels in the following format:
                \begin{itemize}[nosep]
                    \item Entity 1 (Label 1)
                    \item Entity 2 (Label 2)
                    \item ...
                    \item Entity N (Label N)
                \end{itemize}
                % \newline \newline
                Here are two examples:
                \newline\newline
                \textbf{[Example 1]}
                \newline
                [The start of the biography]
                \newline
                \textcolor{titlecolor}{Marianne McAndrew is an American actress and singer, born on November 21, 1942, in Cleveland, Ohio. She began her acting career in the late 1960s, appearing in various television shows and films.}
                \newline
                [The end of the biography]
                \newline \newline
                [The start of the list of checked facts]
                \newline
                \textcolor{anscolor}{[Marianne McAndrew is an American. (False); Marianne McAndrew is an actress. (True); Marianne McAndrew is a singer. (False); Marianne McAndrew was born on November 21, 1942. (False); Marianne McAndrew was born in Cleveland, Ohio. (False); She began her acting career in the late 1960s. (True); She has appeared in various television shows. (True); She has appeared in various films. (True)]}
                \newline
                [The end of the list of checked facts]
                \newline \newline
                [The start of the ideal output]
                \newline
                \textcolor{labelcolor}{[Marianne McAndrew (True); is (True); an (True); American (False); actress (True); and (True); singer (False); , (True); born (True); on (True); November 21, 1942 (False); , (True); in (True); Cleveland, Ohio (False); . (True); She (True); began (True); her (True); acting career (True); in (True); the late 1960s (True); , (True); appearing (True); in (True); various (True); television shows (True); and (True); films (True); . (True)]}
                \newline
                [The end of the ideal output]
				\newline \newline
                \textbf{[Example 2]}
                \newline
                [The start of the biography]
                \newline
                \textcolor{titlecolor}{Doug Sheehan is an American actor who was born on April 27, 1949, in Santa Monica, California. He is best known for his roles in soap operas, including his portrayal of Joe Kelly on ``General Hospital'' and Ben Gibson on ``Knots Landing.''}
                \newline
                [The end of the biography]
                \newline \newline
                [The start of the list of checked facts]
                \newline
                \textcolor{anscolor}{[Doug Sheehan is an American. (True); Doug Sheehan is an actor. (True); Doug Sheehan was born on April 27, 1949. (True); Doug Sheehan was born in Santa Monica, California. (False); He is best known for his roles in soap operas. (True); He portrayed Joe Kelly. (True); Joe Kelly was in General Hospital. (True); General Hospital is a soap opera. (True); He portrayed Ben Gibson. (True); Ben Gibson was in Knots Landing. (True); Knots Landing is a soap opera. (True)]}
                \newline
                [The end of the list of checked facts]
                \newline \newline
                [The start of the ideal output]
                \newline
                \textcolor{labelcolor}{[Doug Sheehan (True); is (True); an (True); American (True); actor (True); who (True); was born (True); on (True); April 27, 1949 (True); in (True); Santa Monica, California (False); . (True); He (True); is (True); best known (True); for (True); his roles in soap operas (True); , (True); including (True); in (True); his portrayal (True); of (True); Joe Kelly (True); on (True); ``General Hospital'' (True); and (True); Ben Gibson (True); on (True); ``Knots Landing.'' (True)]}
                \newline
                [The end of the ideal output]
				\newline \newline
				\textbf{User prompt}
				\newline
				\newline
				[The start of the biography]
				\newline
				\textcolor{magenta}{\texttt{\{BIOGRAPHY\}}}
				\newline
				[The ebd of the biography]
				\newline \newline
				[The start of the list of checked facts]
				\newline
				\textcolor{magenta}{\texttt{\{LIST OF CHECKED FACTS\}}}
				\newline
				[The end of the list of checked facts]
			};
			\node[chatbox_user_inner] (q1_text) at (q1) {
				\textbf{System prompt}
				\newline
				\newline
				You are a helpful and precise assistant for segmenting and labeling sentences. We would like to request your help on curating a dataset for entity-level hallucination detection.
				\newline \newline
                We will give you a machine generated biography and a list of checked facts about the biography. Each fact consists of a sentence and a label (True/False). Please do the following process. First, breaking down the biography into words. Second, by referring to the provided list of facts, merging some broken down words in the previous step to form meaningful entities. For example, ``strategic thinking'' should be one entity instead of two. Third, according to the labels in the list of facts, labeling each entity as True or False. Specifically, for facts that share a similar sentence structure (\eg, \textit{``He was born on Mach 9, 1941.''} (\texttt{True}) and \textit{``He was born in Ramos Mejia.''} (\texttt{False})), please first assign labels to entities that differ across atomic facts. For example, first labeling ``Mach 9, 1941'' (\texttt{True}) and ``Ramos Mejia'' (\texttt{False}) in the above case. For those entities that are the same across atomic facts (\eg, ``was born'') or are neutral (\eg, ``he,'' ``in,'' and ``on''), please label them as \texttt{True}. For the cases that there is no atomic fact that shares the same sentence structure, please identify the most informative entities in the sentence and label them with the same label as the atomic fact while treating the rest of the entities as \texttt{True}. In the end, output the entities and labels in the following format:
                \begin{itemize}[nosep]
                    \item Entity 1 (Label 1)
                    \item Entity 2 (Label 2)
                    \item ...
                    \item Entity N (Label N)
                \end{itemize}
                % \newline \newline
                Here are two examples:
                \newline\newline
                \textbf{[Example 1]}
                \newline
                [The start of the biography]
                \newline
                \textcolor{titlecolor}{Marianne McAndrew is an American actress and singer, born on November 21, 1942, in Cleveland, Ohio. She began her acting career in the late 1960s, appearing in various television shows and films.}
                \newline
                [The end of the biography]
                \newline \newline
                [The start of the list of checked facts]
                \newline
                \textcolor{anscolor}{[Marianne McAndrew is an American. (False); Marianne McAndrew is an actress. (True); Marianne McAndrew is a singer. (False); Marianne McAndrew was born on November 21, 1942. (False); Marianne McAndrew was born in Cleveland, Ohio. (False); She began her acting career in the late 1960s. (True); She has appeared in various television shows. (True); She has appeared in various films. (True)]}
                \newline
                [The end of the list of checked facts]
                \newline \newline
                [The start of the ideal output]
                \newline
                \textcolor{labelcolor}{[Marianne McAndrew (True); is (True); an (True); American (False); actress (True); and (True); singer (False); , (True); born (True); on (True); November 21, 1942 (False); , (True); in (True); Cleveland, Ohio (False); . (True); She (True); began (True); her (True); acting career (True); in (True); the late 1960s (True); , (True); appearing (True); in (True); various (True); television shows (True); and (True); films (True); . (True)]}
                \newline
                [The end of the ideal output]
				\newline \newline
                \textbf{[Example 2]}
                \newline
                [The start of the biography]
                \newline
                \textcolor{titlecolor}{Doug Sheehan is an American actor who was born on April 27, 1949, in Santa Monica, California. He is best known for his roles in soap operas, including his portrayal of Joe Kelly on ``General Hospital'' and Ben Gibson on ``Knots Landing.''}
                \newline
                [The end of the biography]
                \newline \newline
                [The start of the list of checked facts]
                \newline
                \textcolor{anscolor}{[Doug Sheehan is an American. (True); Doug Sheehan is an actor. (True); Doug Sheehan was born on April 27, 1949. (True); Doug Sheehan was born in Santa Monica, California. (False); He is best known for his roles in soap operas. (True); He portrayed Joe Kelly. (True); Joe Kelly was in General Hospital. (True); General Hospital is a soap opera. (True); He portrayed Ben Gibson. (True); Ben Gibson was in Knots Landing. (True); Knots Landing is a soap opera. (True)]}
                \newline
                [The end of the list of checked facts]
                \newline \newline
                [The start of the ideal output]
                \newline
                \textcolor{labelcolor}{[Doug Sheehan (True); is (True); an (True); American (True); actor (True); who (True); was born (True); on (True); April 27, 1949 (True); in (True); Santa Monica, California (False); . (True); He (True); is (True); best known (True); for (True); his roles in soap operas (True); , (True); including (True); in (True); his portrayal (True); of (True); Joe Kelly (True); on (True); ``General Hospital'' (True); and (True); Ben Gibson (True); on (True); ``Knots Landing.'' (True)]}
                \newline
                [The end of the ideal output]
				\newline \newline
				\textbf{User prompt}
				\newline
				\newline
				[The start of the biography]
				\newline
				\textcolor{magenta}{\texttt{\{BIOGRAPHY\}}}
				\newline
				[The ebd of the biography]
				\newline \newline
				[The start of the list of checked facts]
				\newline
				\textcolor{magenta}{\texttt{\{LIST OF CHECKED FACTS\}}}
				\newline
				[The end of the list of checked facts]
			};
		\end{tikzpicture}
        \caption{GPT-4o prompt for labeling hallucinated entities.}\label{tb:gpt-4-prompt}
	\end{center}
\vspace{-0cm}
\end{table*}
% \section{Full Experiment Results}
% \begin{table*}[th]
    \centering
    \small
    \caption{Classification Results}
    \begin{tabular}{lcccc}
        \toprule
        \textbf{Method} & \textbf{Accuracy} & \textbf{Precision} & \textbf{Recall} & \textbf{F1-score} \\
        \midrule
        \multicolumn{5}{c}{\textbf{Zero Shot}} \\
                Zero-shot E-eyes & 0.26 & 0.26 & 0.27 & 0.26 \\
        Zero-shot CARM & 0.24 & 0.24 & 0.24 & 0.24 \\
                Zero-shot SVM & 0.27 & 0.28 & 0.28 & 0.27 \\
        Zero-shot CNN & 0.23 & 0.24 & 0.23 & 0.23 \\
        Zero-shot RNN & 0.26 & 0.26 & 0.26 & 0.26 \\
DeepSeek-0shot & 0.54 & 0.61 & 0.54 & 0.52 \\
DeepSeek-0shot-COT & 0.33 & 0.24 & 0.33 & 0.23 \\
DeepSeek-0shot-Knowledge & 0.45 & 0.46 & 0.45 & 0.44 \\
Gemma2-0shot & 0.35 & 0.22 & 0.38 & 0.27 \\
Gemma2-0shot-COT & 0.36 & 0.22 & 0.36 & 0.27 \\
Gemma2-0shot-Knowledge & 0.32 & 0.18 & 0.34 & 0.20 \\
GPT-4o-mini-0shot & 0.48 & 0.53 & 0.48 & 0.41 \\
GPT-4o-mini-0shot-COT & 0.33 & 0.50 & 0.33 & 0.38 \\
GPT-4o-mini-0shot-Knowledge & 0.49 & 0.31 & 0.49 & 0.36 \\
GPT-4o-0shot & 0.62 & 0.62 & 0.47 & 0.42 \\
GPT-4o-0shot-COT & 0.29 & 0.45 & 0.29 & 0.21 \\
GPT-4o-0shot-Knowledge & 0.44 & 0.52 & 0.44 & 0.39 \\
LLaMA-0shot & 0.32 & 0.25 & 0.32 & 0.24 \\
LLaMA-0shot-COT & 0.12 & 0.25 & 0.12 & 0.09 \\
LLaMA-0shot-Knowledge & 0.32 & 0.25 & 0.32 & 0.28 \\
Mistral-0shot & 0.19 & 0.23 & 0.19 & 0.10 \\
Mistral-0shot-Knowledge & 0.21 & 0.40 & 0.21 & 0.11 \\
        \midrule
        \multicolumn{5}{c}{\textbf{4 Shot}} \\
GPT-4o-mini-4shot & 0.58 & 0.59 & 0.58 & 0.53 \\
GPT-4o-mini-4shot-COT & 0.57 & 0.53 & 0.57 & 0.50 \\
GPT-4o-mini-4shot-Knowledge & 0.56 & 0.51 & 0.56 & 0.47 \\
GPT-4o-4shot & 0.77 & 0.84 & 0.77 & 0.73 \\
GPT-4o-4shot-COT & 0.63 & 0.76 & 0.63 & 0.53 \\
GPT-4o-4shot-Knowledge & 0.72 & 0.82 & 0.71 & 0.66 \\
LLaMA-4shot & 0.29 & 0.24 & 0.29 & 0.21 \\
LLaMA-4shot-COT & 0.20 & 0.30 & 0.20 & 0.13 \\
LLaMA-4shot-Knowledge & 0.15 & 0.23 & 0.13 & 0.13 \\
Mistral-4shot & 0.02 & 0.02 & 0.02 & 0.02 \\
Mistral-4shot-Knowledge & 0.21 & 0.27 & 0.21 & 0.20 \\
        \midrule
        
        \multicolumn{5}{c}{\textbf{Suprevised}} \\
        SVM & 0.94 & 0.92 & 0.91 & 0.91 \\
        CNN & 0.98 & 0.98 & 0.97 & 0.97 \\
        RNN & 0.99 & 0.99 & 0.99 & 0.99 \\
        % \midrule
        % \multicolumn{5}{c}{\textbf{Conventional Wi-Fi-based Human Activity Recognition Systems}} \\
        E-eyes & 1.00 & 1.00 & 1.00 & 1.00 \\
        CARM & 0.98 & 0.98 & 0.98 & 0.98 \\
\midrule
 \multicolumn{5}{c}{\textbf{Vision Models}} \\
           Zero-shot SVM & 0.26 & 0.25 & 0.25 & 0.25 \\
        Zero-shot CNN & 0.26 & 0.25 & 0.26 & 0.26 \\
        Zero-shot RNN & 0.28 & 0.28 & 0.29 & 0.28 \\
        SVM & 0.99 & 0.99 & 0.99 & 0.99 \\
        CNN & 0.98 & 0.99 & 0.98 & 0.98 \\
        RNN & 0.98 & 0.99 & 0.98 & 0.98 \\
GPT-4o-mini-Vision & 0.84 & 0.85 & 0.84 & 0.84 \\
GPT-4o-mini-Vision-COT & 0.90 & 0.91 & 0.90 & 0.90 \\
GPT-4o-Vision & 0.74 & 0.82 & 0.74 & 0.73 \\
GPT-4o-Vision-COT & 0.70 & 0.83 & 0.70 & 0.68 \\
LLaMA-Vision & 0.20 & 0.23 & 0.20 & 0.09 \\
LLaMA-Vision-Knowledge & 0.22 & 0.05 & 0.22 & 0.08 \\

        \bottomrule
    \end{tabular}
    \label{full}
\end{table*}




\end{document}


\end{document}