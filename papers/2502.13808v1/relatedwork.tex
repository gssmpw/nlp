\section{Related Work}
In recent years, deep learning has rapidly advanced, particularly with the development of Convolutional Neural Networks (CNN) and their variants, significantly improving medical image segmentation task\cite{ZHANG2023109020}. 

% \subsection{Medical Image Segmentation Models}
 One of the influential models in this domain is U-Net\cite{unet}, which uses skip connections to preserve detailed information during downsampling, making it highly effective for medical image segmentation. However, U-Net primarily focuses on local features and struggles with capturing broader contextual information, limiting its performance in images with complex structures. Several models have been proposed to improve both edge detection and contextual understanding in medical image segmentation. Attention U-Net\cite{atunet}, introduced to refine U-Net, incorporates an attention mechanism that focuses on relevant regions such as organ boundaries or lesions, improving segmentation precision. However, it still faces challenges in preserving edge details while capturing broader contextual information. CE-Net\cite{ce} improves multi-scale feature extraction by using dilated convolutions and a context encoder, making it suitable for handling complex anatomical structures. Despite this, CE-Net faces challenges with accurately segmenting unclear or blurred boundaries. KiU-Net\cite{DBLP:journals/tmi/ValanarasuSHP22} tackles edge preservation by using an overcomplete convolutional structure that enhances detail retention, particularly in edge-aware tasks. However, this design comes at the cost of higher computational complexity, making it less efficient for large-scale datasets.