
%%%%%%%%%%%%%%%%%%%%%%%%%%%%%%%%%%%%%%%%%%%%%%%%%%%%%%%%%%%%%%%%
\begin{querybox}[The Query]
    \textbf{Question:} Which of the following describes the fallacy of appeal to indignation?\\
    A) presenting two options as the only possible ones when more options exist \\
    B) attacking an argument by exaggerating, misrepresenting, or even completely fabricating parts of it\\
    C) arguing against a position only based on personal feelings toward the position\\
    D) responding to charges of wrongdoing by saying other people do the same thing\\
    E) arguing that a claim is true because it has not been proven false\\
    F) responding to an argument with an expression of derision instead of a counter argument\\
    G) making a claim based on popularity or majority opinion instead of evidence\\
    H) attacking the character or background of someone making an argument instead of his or her evidence and reasoning\\
    I) making a claim seem true or false based on irrelevant information or associations\\   
    Choose the correct option.
    \\\\
    \textbf{Ground truth:} E
    % \vspace{20pt}
\end{querybox}


%%%%%%%%%%%%%%%%%%%%%%%%%%%%%%%%%%%%%%%%%%%%%%%%%%%%%%%%%%%%%%%%


%%%%%%%%%%%%%%%%%%%%%%%%%%%%%%%%%%%%%%%%%%%%%%%%%%%%%%%%%%%%%%%%
\begin{querybox}[Wrong Prediction from \gpt]
\red{F)}\\
The fallacy of appeal to indignation involves responding to an argument with an expression of derision or outrage instead of providing a logical counterargument. This fallacy focuses on expressing strong emotions rather than addressing the substance of the argument itself.
\end{querybox}
%%%%%%%%%%%%%%%%%%%%%%%%%%%%%%%%%%%%%%%%%%%%%%%%%%%%%%%%%%%%%%%%

In contrast, \model decomposes the query into two steps by calling the knowledge retrieval tool and reasoning tool to answer the question.  First, it calls the \texttt{Wikipedia\_Knowledge\_Searcher\_Tool} to retrieve related knowledge. Then it uses \texttt{Generalist\_Solution\_Generator\_Tool} to perform reasoning over the query and the knowledge retrieved.  

%%%%%%%%%%%%%%%%%%%%%%%%%%%%%%%%%%%%%%%%%%%%%%%%%%%%%%%%%%%%%%%%
\begin{plannerbox}[Step 0: Initial Plan from Query Analyzer]

% \textbf{Concise summary:}
% The task is to classify a histology image of osteosarcoma stained with hematoxylin and eosin (H\&E) into one of three categories: Non-tumor, Necrotic tumor, or Viable tumor.
% \\\\
\textbf{Required skills:}

1. Critical Thinking: To analyze and understand the descriptions of logical fallacies and identify the correct one.\\
2. Knowledge of Logical Fallacies: To accurately recognize and differentiate between various logical fallacies, including the appeal to indignation.\\
3. Decision Making: To choose the correct option from the given list based on the analysis.
\\\\
\textbf{Relevant tools:}

1. \texttt{Wikipedia\_Knowledge\_Searcher\_Tool}: This tool can be used to search for information on logical fallacies, including the appeal to indignation, to verify the correct description. It provides factual information from Wikipedia, which can be cross-referenced with the options provided. \\
2.\texttt{Generalist\_Solution\_Generator\_Tool}: This tool can be used to generate a step-by-step analysis of each option to determine which one describes the fallacy of appeal to indignation. However, it may provide incorrect responses, so verification is necessary.
\\\\
\textbf{Additional considerations:}
Ensure that the final answer is verified for accuracy, as the \texttt{Generalist\_Solution\_Generator\_Tool} may provide hallucinated responses. Cross-referencing with reliable sources is recommended to confirm the correct description of the fallacy.
\end{plannerbox}
%%%%%%%%%%%%%%%%%%%%%%%%%%%%%%%%%%%%%%%%%%%%%%%%%%%%%%%%%%%%%%%%


%%%%%%%%%%%%%%%%%%%%%%%%%%%%%%%%%%%%%%%%%%%%%%%%%%%%%%%%%%%%%%%%
\begin{plannerbox}[Step 1: Action from Action Predictor]

\textbf{Justification:}
The \texttt{Wikipedia\_Knowledge\_Searcher\_Tool} is the best choice for this step because it provides factual information from a reliable source, which is crucial for accurately identifying the description of the fallacy of appeal to indignation. This tool can search for and retrieve specific information about logical fallacies, ensuring that the selected option is verified against authoritative content.
% It can directly take the image and the classification options as input and provide a classification result, which aligns perfectly with the current query's requirements.
\\\\
\textbf{Sub-goal:}
Search Wikipedia for information on the fallacy of appeal to indignation to verify which option correctly describes it.
\\\\
\textbf{Tool name:} \texttt{Wikipedia\_Knowledge\_Searcher\_Tool}
\end{plannerbox}
%%%%%%%%%%%%%%%%%%%%%%%%%%%%%%%%%%%%%%%%%%%%%%%%%%%%%%%%%%%%%%%%


%%%%%%%%%%%%%%%%%%%%%%%%%%%%%%%%%%%%%%%%%%%%%%%%%%%%%%%%%%%%%%%%
\begin{executorbox}[Step 1: Command from Command Generator]

\textbf{Analysis:}
The task is to identify which option describes the fallacy of appeal to indignation. The sub-goal is to search Wikipedia for information on this fallacy to verify the correct option. The selected tool is the \texttt{Wikipedia\_Knowledge\_Searcher\_Tool}, which requires a query string as input. The query should be related to the fallacy of appeal to indignation to retrieve relevant information from Wikipedia.
% explanation="The command is constructed to search Wikipedia for the term 'appeal to indignation fallacy'. This will help in retrieving information that can be used to verify which option correctly describes the fallacy. The tool's input type is a string query, so we pass the search term directly to the \texttt{tool.execute()} function.
% \\\\
% \textbf{Explanation:}
% The command is constructed to use the The selected tool, \texttt{Path\_Generalist\_Classifier\_Tool} by providing the required inputs: the image path and the list of classification options. The image path is directly taken from the context, and the options are extracted from the query. The command follows the tool's input requirements and usage pattern, ensuring that the image is classified into one of the specified categories.
\\\\
\textbf{Command:}
\command{execution = tool.execute(query="appeal to indignation fallac")}
\end{executorbox}
%%%%%%%%%%%%%%%%%%%%%%%%%%%%%%%%%%%%%%%%%%%%%%%%%%%%%%%%%%%%%%%%


%%%%%%%%%%%%%%%%%%%%%%%%%%%%%%%%%%%%%%%%%%%%%%%%%%%%%%%%%%%%%%%%
\begin{executorbox}[Step 1: Result from Command Executor]
\textbf{Result:} The ``fallacy of appeal to indignation'' refers to an argument tactic where someone tries to persuade others by provoking anger or outrage instead of presenting logical evidence, essentially using strong emotions like indignation to sway opinion rather than reasoned facts; it is considered a form of ``appeal to emotion'' fallacy where the emotion being exploited is anger or outrage. 
\end{executorbox}
%%%%%%%%%%%%%%%%%%%%%%%%%%%%%%%%%%%%%%%%%%%%%%%%%%%%%%%%%%%%%%%%


%%%%%%%%%%%%%%%%%%%%%%%%%%%%%%%%%%%%%%%%%%%%%%%%%%%%%%%%%%%%%%%%
\begin{plannerbox}[Step 1: Verification from Context Verifier]

\textbf{Completeness:}
The memory addresses the main objective of identifying the correct description of the fallacy. The \texttt{Generalist\_Solution\_Generator\_Tool}\'s output aligns with the understanding of the fallacy as it involves emotional reactions rather than logical reasoning.
\\\\
\textbf{Inconsistencies:}
There are no direct inconsistencies in the information provided
\\\\
\textbf{Verification needs:}
The information regarding the appeal to indignation fallacy needs further verification as the current memory does not provide a definitive answer.
\\\\
\textbf{Ambiguities:}
 The ambiguity lies in the lack of direct information about the appeal to indignation fallacy. Using the \texttt{Generalist\_Solution\_Generator\_Tool} could help clarify which option best fits the description of this fallacy.
\\\\
\textbf{Final determination:}
The memory is insufficient to generate the final output as it requires reasoning over the information on the appeal to indignation fallacy.
\\\\
\textbf{Conclusion:} \command{\textbf{CONTINUE}}
\end{plannerbox}

%%%%%%%%%%%%%%%%%%%%%%%%%%%%%%%%%%%%%%%%%%%%%%%%%%%%%%%%%%%%%%%%
\begin{plannerbox}[Step 2: Action from Action Predictor]

\textbf{Justification:}
The previous attempts to use the \texttt{Wikipedia\_Knowledge\_Searcher\_Tool} did not yield a direct answer to the query. The next logical step is to use the \texttt{Generalist\_Solution\_Generator\_Tool}. This tool can analyze the provided options and use its general knowledge to identify which option best describes the fallacy of appeal to indignation.
\\\\
\textbf{Sub-goal:}
Use the \texttt{Generalist\_Solution\_Generator\_Tool} to analyze the provided options and determine which one best describes the fallacy of appeal to indignation.
\\\\
\textbf{Tool name:} \texttt{Generalist\_Solution\_Generator\_Tool}
\end{plannerbox}
%%%%%%%%%%%%%%%%%%%%%%%%%%%%%%%%%%%%%%%%%%%%%%%%%%%%%%%%%%%%%%%%


%%%%%%%%%%%%%%%%%%%%%%%%%%%%%%%%%%%%%%%%%%%%%%%%%%%%%%%%%%%%%%%%
\begin{executorbox}[Step 2: Command from Command Generator]

\textbf{Analysis:}
The task is to identify which option describes the fallacy of appeal to indignation. The \texttt{Generalist\_Solution\_Generator\_Tool} is selected to analyze the options and determine the correct one. The tool requires a prompt as input, which should include the query and options for analysis. The context provided includes the query and options, which will be used to construct the prompt for the tool.
\\\\
\textbf{Command:}
\command{execution = tool.execute(prompt="Which of the following describes the fallacy of appeal to indignation?..."}
\end{executorbox}
%%%%%%%%%%%%%%%%%%%%%%%%%%%%%%%%%%%%%%%%%%%%%%%%%%%%%%%%%%%%%%%%


%%%%%%%%%%%%%%%%%%%%%%%%%%%%%%%%%%%%%%%%%%%%%%%%%%%%%%%%%%%%%%%%
\begin{executorbox}[Step 2: Result from Command Executor]
\textbf{Result:} The fallacy of appeal to indignation is best described by option C: arguing against a position only based on personal feelings toward the position. This fallacy involves rejecting an argument or claim not because of its merits or evidence, but because of an emotional reaction or personal feelings about it.
\end{executorbox}
%%%%%%%%%%%%%%%%%%%%%%%%%%%%%%%%%%%%%%%%%%%%%%%%%%%%%%%%%%%%%%%%


%%%%%%%%%%%%%%%%%%%%%%%%%%%%%%%%%%%%%%%%%%%%%%%%%%%%%%%%%%%%%%%%
\begin{plannerbox}[Step 2: Verification from Context Verifier]

\textbf{Completeness:}
The memory does not fully address the query as the search results did not provide a direct description of the appeal to indignation fallacy. The correct option from the list remains unidentified.
\\\\
\textbf{Inconsistencies:}
There are no direct inconsistencies in the information provided.
\\\\
\textbf{Verification needs:}
The information regarding the appeal to indignation fallacy needs further verification as the current memory does not provide a definitive answer.
\\\\
\textbf{Ambiguities:}
 There are no significant ambiguities in the results that require further clarification.
\\\\
\textbf{Final determination:}
The memory is sufficient to generate the final output, as the \texttt{Generalist\_Solution\_Generator\_Tool} has provided a plausible and logical answer to the query.
\\\\
\textbf{Conclusion:} \command{\textbf{STOP}}
\end{plannerbox}

When a knowledge retrieval tool is present for retrieving relevant knowledge, \model is able to leverage the retrieved knowledge directly to reach the answer.

%%%%%%%%%%%%%%%%%%%%%%%%%%%%%%%%%%%%%%%%%%%%%%%%%%%%%%%%%%%%%%%%
\begin{answerbox}[Final Answer]
Based on the analysis, option C is the correct description of the fallacy of appeal to indignation, as it involves arguing against a position based on personal feelings.
\end{answerbox}
