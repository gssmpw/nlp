
%%%%%%%%%%%%%%%%%%%%%%%%%%%%%%%%%%%%%%%%%%%%%%%%%%%%%%%%%%%%%%%%
\begin{querybox}[The Query]
\textbf{Question:}
Using the numbers [1, 1, 6, 9], create an expression that equals 24. You must use basic arithmetic operations (+, -, ×, /) and parentheses. 

\textbf{Example:} for [1, 2, 3, 4], one solution is (1+2+3)×4.
\end{querybox}
%%%%%%%%%%%%%%%%%%%%%%%%%%%%%%%%%%%%%%%%%%%%%%%%%%%%%%%%%%%%%%%%


%%%%%%%%%%%%%%%%%%%%%%%%%%%%%%%%%%%%%%%%%%%%%%%%%%%%%%%%%%%%%%%%
\begin{querybox}[Wrong Prediction from \gpt]
One possible expression using the numbers [1, 1, 6, 9] to make 24 is:
\[
\red{(9 + 1) \times (6 / 2)}
\]
Breaking it down:

\quad - \( 9 + 1 = 10 \)

\quad - \( 6 / 2 = 3 \)

\quad - \red{\( 10 \times 3 = 24 \)}

This satisfies the conditions using basic arithmetic operations and parentheses. 
\end{querybox}
%%%%%%%%%%%%%%%%%%%%%%%%%%%%%%%%%%%%%%%%%%%%%%%%%%%%%%%%%%%%%%%%


%%%%%%%%%%%%%%%%%%%%%%%%%%%%%%%%%%%%%%%%%%%%%%%%%%%%%%%%%%%%%%%%
\begin{querybox}[Wrong Prediction from \gpt with another attempt]
One possible solution using the numbers [1, 1, 6, 9] to make 24 is:
\[
\red{(6 / (1 - 1) + 9) \times 1}
\]
However, (1 - 1) equals 0, which makes division by zero undefined. So we must try another approach.

A correct solution is:

\[
\red{(6 / 1) \times (9 / 1) = 6 \times 9 = 54}
\]

This exceeds 24. Let's adjust:

A valid expression that equals 24 is:

\[
\red{(6 \times 9) / (1 + 1) = 54 / 2 = 24}
\]

\red{This correctly uses all four numbers and results in 24.} 
\end{querybox}
%%%%%%%%%%%%%%%%%%%%%%%%%%%%%%%%%%%%%%%%%%%%%%%%%%%%%%%%%%%%%%%%

%%%%%%%%%%%%%%%%%%%%%%%%%%%%%%%%%%%%%%%%%%%%%%%%%%%%%%%%%%%%%%%%
\begin{plannerbox}[Step 0: Initial Plan from Query Analyzer]
\textbf{Required skills:}

1. Mathematical Problem Solving: Ability to manipulate numbers and operations to achieve a specific result.\\
2. Arithmetic Operations: Proficiency in using addition, subtraction, multiplication, and division.\\
3. Logical Reasoning: Skill in applying logical steps to combine numbers and operations effectively.
\\\\
\textbf{Relevant tools:}

1. \texttt{Python\_Code\_Generator\_Tool}: This tool can be used to generate and test different combinations of arithmetic operations on the given numbers to find an expression that equals 24. It is suitable for simple arithmetic calculations and can help automate the trial-and-error process.\\
2. \texttt{Generalist\_Solution\_Generator\_Tool}: Although not specifically designed for arithmetic problems, it can provide a step-by-step approach to solving the problem by suggesting possible combinations and reasoning through them.
\end{plannerbox}
%%%%%%%%%%%%%%%%%%%%%%%%%%%%%%%%%%%%%%%%%%%%%%%%%%%%%%%%%%%%%%%%


%%%%%%%%%%%%%%%%%%%%%%%%%%%%%%%%%%%%%%%%%%%%%%%%%%%%%%%%%%%%%%%%
\begin{plannerbox}[Step 1: Action from Action Predictor]
\textbf{Justification:}
The \texttt{Python\_Code\_Generator\_Tool} is the most suitable choice for this task because it is specifically designed to handle arithmetic calculations and can automate the process of testing different combinations of the given numbers and operations to achieve the target result of 24. ...
\\\\
\textbf{Sub-goal:}
Generate and test different arithmetic expressions using the numbers [1, 1, 6, 9] to find a combination that equals 24.
\\\\
\textbf{Tool name:} \texttt{Python\_Code\_Generator\_Tool}
\end{plannerbox}
%%%%%%%%%%%%%%%%%%%%%%%%%%%%%%%%%%%%%%%%%%%%%%%%%%%%%%%%%%%%%%%%


%%%%%%%%%%%%%%%%%%%%%%%%%%%%%%%%%%%%%%%%%%%%%%%%%%%%%%%%%%%%%%%%
\begin{executorbox}[Step 1: Command from Command Generator]
\textbf{Command:}
\command{execution = tool.execute(query="Using the numbers [1, 1, 6, 9], create an expression that equals 24 using basic arithmetic operations (+, -, *, /) and parentheses.")}
\end{executorbox}
%%%%%%%%%%%%%%%%%%%%%%%%%%%%%%%%%%%%%%%%%%%%%%%%%%%%%%%%%%%%%%%%


%%%%%%%%%%%%%%%%%%%%%%%%%%%%%%%%%%%%%%%%%%%%%%%%%%%%%%%%%%%%%%%%
\begin{executorbox}[Step 1: Result from Command Executor]
\textbf{Generated Python code:}
\begin{codebox}
# Define the numbers
numbers = [1, 1, 6, 9]

# Calculate the expression
# Using the expression: 6 / (1 - (9 / 1)) = 24
result = 6 / (1 - (9 / 1))

# Print the result with a descriptive message
print(f"The result of the expression using the numbers {numbers} is: {result}")
\end{codebox}

\textbf{Execution:}
The result of the expression using the numbers [1, 1, 6, 9] is: -0.75
\end{executorbox}
%%%%%%%%%%%%%%%%%%%%%%%%%%%%%%%%%%%%%%%%%%%%%%%%%%%%%%%%%%%%%%%%


%%%%%%%%%%%%%%%%%%%%%%%%%%%%%%%%%%%%%%%%%%%%%%%%%%%%%%%%%%%%%%%%
\begin{plannerbox}[Step 1: Verification from Context Verifier]
\textbf{Justification:}
The memory does not fully address the query as the expression found does not equal 24. The task requires finding a valid arithmetic expression using the numbers [1, 1, 6, 9] that results in 24, which has not been achieved.
There are no contradictions in the information provided, but the result is incorrect, indicating a need for further exploration of possible solutions.
The incorrect result from the \texttt{Python\_Code\_Generator\_Tool} suggests that further verification and exploration of different expressions are necessary. This could involve manually testing different combinations or using the \texttt{Generalist\_Solution\_Generator\_Tool} to explore potential solutions.
\\\\
\textbf{Conclusion:} \command{\textbf{CONTINUE}}
\end{plannerbox}
%%%%%%%%%%%%%%%%%%%%%%%%%%%%%%%%%%%%%%%%%%%%%%%%%%%%%%%%%%%%%%%%


%%%%%%%%%%%%%%%%%%%%%%%%%%%%%%%%%%%%%%%%%%%%%%%%%%%%%%%%%%%%%%%%
\begin{plannerbox}[Step 2: Action from Action Predictor]
\textbf{Justification:}
The previous attempt using this tool resulted in an incorrect expression, indicating that further exploration of possible combinations is needed. By refining the query and focusing on generating valid expressions, we can efficiently utilize the tool's capabilities to find a solution that equals 24.
\\\\
\textbf{Sub-goal:}
Generate and test additional arithmetic expressions using the numbers [1, 1, 6, 9] to find a valid combination that equals 24, ensuring the use of parentheses to alter operation precedence.
\\\\
\textbf{Tool name:} \texttt{Python\_Code\_Generator\_Tool}
\end{plannerbox}
%%%%%%%%%%%%%%%%%%%%%%%%%%%%%%%%%%%%%%%%%%%%%%%%%%%%%%%%%%%%%%%%


%%%%%%%%%%%%%%%%%%%%%%%%%%%%%%%%%%%%%%%%%%%%%%%%%%%%%%%%%%%%%%%%
\begin{executorbox}[Step 2: Command from Command Generator]
\textbf{Command:} 
\command{execution = tool.execute(query="Using the numbers [1, 1, 6, 9], create an expression that equals 24 using basic arithmetic operations (+, -, *, /) and parentheses.")}
\end{executorbox}
%%%%%%%%%%%%%%%%%%%%%%%%%%%%%%%%%%%%%%%%%%%%%%%%%%%%%%%%%%%%%%%%


%%%%%%%%%%%%%%%%%%%%%%%%%%%%%%%%%%%%%%%%%%%%%%%%%%%%%%%%%%%%%%%%
\begin{executorbox}[Step 2: Result from Command Executor]
\textbf{Generated Python code:}
\begin{codebox}
# Step 1: Calculate the inner division
inner_division = 9 / 1

# Step 2: Subtract the result from 1
subtraction_result = 1 - inner_division

# Step 3: Divide 6 by the result of step 2
final_result = 6 / subtraction_result

# Print the final result with a descriptive message
print("The result of the expression (6 / (1 - (9 / 1))) is:", final_result)
\end{codebox}

\textbf{Execution:}
The result of the expression (6 / (1 - (9 / 1))) is: -0.75
\end{executorbox}
%%%%%%%%%%%%%%%%%%%%%%%%%%%%%%%%%%%%%%%%%%%%%%%%%%%%%%%%%%%%%%%%


%%%%%%%%%%%%%%%%%%%%%%%%%%%%%%%%%%%%%%%%%%%%%%%%%%%%%%%%%%%%%%%%
\begin{plannerbox}[Step 2: Verification from Context Verifier]
\textbf{Justification:}
The memory is insufficient to generate the final output as it does not provide a correct expression that equals 24. Additional tool usage, specifically the \texttt{Generalist\_Solution\_Generator\_Tool}, is necessary to explore other possible solutions and fulfill the query.
\\\\
\textbf{Conclusion:} \command{\textbf{Continue}}
\end{plannerbox}
%%%%%%%%%%%%%%%%%%%%%%%%%%%%%%%%%%%%%%%%%%%%%%%%%%%%%%%%%%%%%%%%



%%%%%%%%%%%%%%%%%%%%%%%%%%%%%%%%%%%%%%%%%%%%%%%%%%%%%%%%%%%%%%%%
\begin{plannerbox}[Step 3: Action from Action Predictor]
\textbf{Justification:}
The \texttt{Python\_Code\_Generator\_Tool} is the most suitable choice for this step because it is specifically designed to handle arithmetic calculations and can automate the process of testing different combinations of operations and parentheses. Previous attempts using this tool have not yet found a solution, but it remains the best option for systematically exploring possible expressions. The \texttt{Generalist\_Solution\_Generator\_Tool}, while capable of providing step-by-step reasoning, is not optimized for generating and testing multiple arithmetic expressions efficiently. Given the need to find a valid expression that equals 24, the \texttt{Python\_Code\_Generator\_Tool} can be used to continue testing new combinations, leveraging its ability to quickly execute and evaluate arithmetic expressions.
\\\\
\textbf{Sub-goal:}
Generate and test new arithmetic expressions using the numbers [1, 1, 6, 9] with different combinations of operations and parentheses to find a valid expression that equals 24.
\\\\
\textbf{Tool name:} \texttt{Python\_Code\_Generator\_Tool}
\end{plannerbox}
%%%%%%%%%%%%%%%%%%%%%%%%%%%%%%%%%%%%%%%%%%%%%%%%%%%%%%%%%%%%%%%%


%%%%%%%%%%%%%%%%%%%%%%%%%%%%%%%%%%%%%%%%%%%%%%%%%%%%%%%%%%%%%%%%
\begin{executorbox}[Step 3: Command from Command Generator]
\textbf{Command:} 
\command{execution = tool.execute(query="Find an expression using the numbers [1, 1, 6, 9] with operations +, -, *, / and parentheses to equal 24.")}
\end{executorbox}
%%%%%%%%%%%%%%%%%%%%%%%%%%%%%%%%%%%%%%%%%%%%%%%%%%%%%%%%%%%%%%%%


%%%%%%%%%%%%%%%%%%%%%%%%%%%%%%%%%%%%%%%%%%%%%%%%%%%%%%%%%%%%%%%%
\begin{executorbox}[Step 3: Result from Command Executor]
\textbf{Generated Python code:}
\begin{codebox}
from itertools import permutations, product

# Define the numbers and target
numbers = [1, 1, 6, 9]
target = 24
operations = ['+', '-', '*', '/']

# Function to evaluate an expression safely
def evaluate_expression(expr):
    try:
        return eval(expr)
    except ZeroDivisionError:
        return None

# Generate all permutations of numbers
for num_perm in permutations(numbers):
    # Generate all combinations of operations
    for ops in product(operations, repeat=3):
        # Generate different parenthesized expressions
        expressions = [
            f"({num_perm[0]} {ops[0]} {num_perm[1]}) {ops[1]} ({num_perm[2]} {ops[2]} {num_perm[3]})",
            f"(({num_perm[0]} {ops[0]} {num_perm[1]}) {ops[1]} {num_perm[2]}) {ops[2]} {num_perm[3]}",
            f"({num_perm[0]} {ops[0]} ({num_perm[1]} {ops[1]} {num_perm[2]})) {ops[2]} {num_perm[3]}",
            f"{num_perm[0]} {ops[0]} (({num_perm[1]} {ops[1]} {num_perm[2]}) {ops[2]} {num_perm[3]})",
            f"{num_perm[0]} {ops[0]} ({num_perm[1]} {ops[1]} ({num_perm[2]} {ops[2]} {num_perm[3]}))"
        ]
        
        # Evaluate each expression
        for expr in expressions:
            if evaluate_expression(expr) == target:
                print(f"Expression that equals {target}: {expr}")
                break
\end{codebox}
\textbf{Execution:}\\
Expression that equals 24: ((1 + 1) * 9) + 6\\
...\\
Expression that equals 24: 6 + ((1 + 1) * 9)\\
...
\end{executorbox}
%%%%%%%%%%%%%%%%%%%%%%%%%%%%%%%%%%%%%%%%%%%%%%%%%%%%%%%%%%%%%%%%


%%%%%%%%%%%%%%%%%%%%%%%%%%%%%%%%%%%%%%%%%%%%%%%%%%%%%%%%%%%%%%%%
\begin{plannerbox}[Step 3: Verification from Context Verifier]
\textbf{Justification:}
The memory provides a comprehensive solution to the query of creating an expression using the numbers [1, 1, 6, 9] that equals 24. The \texttt{Python\_Code\_Generator\_Tool} was used effectively to explore various combinations of arithmetic operations and parentheses. The final result includes multiple valid expressions that satisfy the query, such as: ((1 + 1) * 9) + 6, 6 + ((1 + 1) * 9).
\textbf{Conclusion:} \command{\textbf{STOP}}
\end{plannerbox}
%%%%%%%%%%%%%%%%%%%%%%%%%%%%%%%%%%%%%%%%%%%%%%%%%%%%%%%%%%%%%%%%


%%%%%%%%%%%%%%%%%%%%%%%%%%%%%%%%%%%%%%%%%%%%%%%%%%%%%%%%%%%%%%%%
\begin{answerbox}[Final Answer]
The expression \((1 + 1) \times 9 + 6\) successfully uses the numbers [1, 1, 6, 9] with basic arithmetic operations and parentheses to equal 24.
\end{answerbox}
