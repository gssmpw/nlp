
%%%%%%%%%%%%%%%%%%%%%%%%%%%%%%%%%%%%%%%%%%%%%%%%%%%%%%%%%%%%%%%%
\begin{querybox}[The Query]
\begin{minipage}{0.3\linewidth}
    \includegraphics[width=1\linewidth]{examples/example_pathcls_286.jpg}
\end{minipage}
\hspace{5pt}
\begin{minipage}{0.55\linewidth}
    \vspace{-20pt}
    \textbf{Image path:} \texttt{tissue.jpg}
    
    \textbf{Question:} This is a hematoxylin and eosin (H\&E) stained osteosarcoma histology image of:\\
    A) Non-tumor\\
    B) Necrotic tumor\\
    C) Viable tumor\\
    Choose the correct option.
    \\\\
    \textbf{Ground truth:} B) Necrotic tumor
\end{minipage}
\end{querybox}
%%%%%%%%%%%%%%%%%%%%%%%%%%%%%%%%%%%%%%%%%%%%%%%%%%%%%%%%%%%%%%%%


%%%%%%%%%%%%%%%%%%%%%%%%%%%%%%%%%%%%%%%%%%%%%%%%%%%%%%%%%%%%%%%%
\begin{querybox}[Wrong Prediction from \gpt]
\red{C) Viable tumor}\\
The image shows a dense cellular area with viable tumor cells, typical of a viable osteosarcoma.
\end{querybox}
%%%%%%%%%%%%%%%%%%%%%%%%%%%%%%%%%%%%%%%%%%%%%%%%%%%%%%%%%%%%%%%%

In contrast, \model decomposes the query into two steps by calling two vision tools to enhance visual perception. First, it calls the \texttt{Image\_Captioner\_Tool} to provide a general description of the associated image. Then, it uses the \texttt{Object\_Detector\_Tool} to detect and count the baseballs. These two steps provide consistent information about the number of baseballs, thus resulting in a correct answer.

%%%%%%%%%%%%%%%%%%%%%%%%%%%%%%%%%%%%%%%%%%%%%%%%%%%%%%%%%%%%%%%%
\begin{plannerbox}[Step 0: Initial Plan from Query Analyzer]
\textbf{Required skills:}

1. Histopathology Knowledge: Understanding of osteosarcoma and its histological features.\\
2. Image Analysis: Ability to interpret H\&E stained images.\\
3. Classification Skills: Ability to categorize images based on visual features.
\\\\
\textbf{Relevant tools:}

1. \texttt{Path\_Generalist\_Classifier\_Tool}: This tool is suitable for classifying histopathology images into predefined categories. It can be used to determine whether the image is of a non-tumor, necrotic tumor, or viable tumor.
\\\\
\textbf{Additional considerations:}
Ensure the options provided for classification are clear and specific. Cross-reference the tool's output with expert knowledge if possible, as the tool may not always be reliable.
\end{plannerbox}
%%%%%%%%%%%%%%%%%%%%%%%%%%%%%%%%%%%%%%%%%%%%%%%%%%%%%%%%%%%%%%%%


%%%%%%%%%%%%%%%%%%%%%%%%%%%%%%%%%%%%%%%%%%%%%%%%%%%%%%%%%%%%%%%%
\begin{plannerbox}[Step 1: Action from Action Predictor]

\textbf{Justification:}
The task is to classify a histology image of osteosarcoma into one of three categories: Non-tumor, Necrotic tumor, or Viable tumor. The \texttt{Path\_Generalist\_Classifier\_Tool} is specifically designed for classifying histopathology images into predefined categories, making it the most suitable tool for this task. 
\\\\
\textbf{Context:}
Image path: \texttt{tissue.jpg}
\\\\
\textbf{Sub-goal:}
Classify the histology image into one of the categories: Non-tumor, Necrotic tumor, or Viable tumor using the provided image path.
\\\\
\textbf{Tool name:} \texttt{Path\_Generalist\_Classifier\_Tool}
\end{plannerbox}
%%%%%%%%%%%%%%%%%%%%%%%%%%%%%%%%%%%%%%%%%%%%%%%%%%%%%%%%%%%%%%%%


%%%%%%%%%%%%%%%%%%%%%%%%%%%%%%%%%%%%%%%%%%%%%%%%%%%%%%%%%%%%%%%%
\begin{executorbox}[Step 1: Command from Command Generator]

\textbf{Analysis:}
The task is to classify a histology image of osteosarcoma into one of three categories: Non-tumor, Necrotic tumor, or Viable tumor. The selected tool, \texttt{Path\_Generalist\_Classifier\_Tool}, is designed to classify histopathology images based on provided options. The tool requires two inputs: the path to the image and a list of classification options. The image path is provided in the context, and the classification options are specified in the query. The tool's metadata confirms that it is suitable for this type of classification task, as it is designed for histopathology images and multiple-choice questions.
\\\\
\textbf{Command:}
\command{execution = tool.execute(image="tissue.jpg", options=["Non-tumor", "Necrotic tumor", "Viable tumor"])}
\end{executorbox}
%%%%%%%%%%%%%%%%%%%%%%%%%%%%%%%%%%%%%%%%%%%%%%%%%%%%%%%%%%%%%%%%


%%%%%%%%%%%%%%%%%%%%%%%%%%%%%%%%%%%%%%%%%%%%%%%%%%%%%%%%%%%%%%%%
\begin{executorbox}[Step 1: Result from Command Executor]
\textbf{Result:}
Necrotic tumor
\end{executorbox}
%%%%%%%%%%%%%%%%%%%%%%%%%%%%%%%%%%%%%%%%%%%%%%%%%%%%%%%%%%%%%%%%


%%%%%%%%%%%%%%%%%%%%%%%%%%%%%%%%%%%%%%%%%%%%%%%%%%%%%%%%%%%%%%%%
\begin{plannerbox}[Step 1: Verification from Context Verifier]

\textbf{Completeness:}
The memory addresses the query by using the \texttt{Path\_Generalist\_Classifier\_Tool} to classify the histology image into one of the specified categories: Non-tumor, Necrotic tumor, or Viable tumor. The tool's output indicates that the image is classified as a ``Necrotic tumor''. This directly answers the query, fulfilling the main objective of classifying the image.
\\\\
\textbf{Inconsistencies:}
There are no inconsistencies in the information provided. The classification result is clear and aligns with the query's requirements.
\\\\
\textbf{Verification needs:}
The \texttt{Path\_Generalist\_Classifier\_Tool} has limitations and may not always be reliable. However, the classification of ``Necrotic tumor'' seems plausible given the context. If further verification is needed, consulting a histopathology expert or using additional histopathology-specific tools could be considered.
\\\\
\textbf{Ambiguities:}
There are no ambiguities in the result. The classification is straightforward and matches one of the provided options.
\\\\
\textbf{Final determination:}
The memory is complete and accurate enough to generate the final output. The classification tool has provided a clear answer to the query, and no additional tool usage is necessary.
\\\\
\textbf{Conclusion:} \command{\textbf{STOP}}
\end{plannerbox}
%%%%%%%%%%%%%%%%%%%%%%%%%%%%%%%%%%%%%%%%%%%%%%%%%%%%%%%%%%%%%%%%

The final answer is summarized by the solution summarizer in the planner, given the full trajectory in the context described above. When a highly specialized tool is present for its corresponding domain, in this case histopathology classification, \model is able to leverage the specialized tool directly to reach the answer.

%%%%%%%%%%%%%%%%%%%%%%%%%%%%%%%%%%%%%%%%%%%%%%%%%%%%%%%%%%%%%%%%
\begin{answerbox}[Final Answer]
The histology image of osteosarcoma stained with hematoxylin and eosin (H\&E) has been classified using the \texttt{Path\_Generalist\_Classifier\_Tool}. Based on the analysis, the image is identified as a:

B) Necrotic tumor
\end{answerbox}
