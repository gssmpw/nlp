
%%%%%%%%%%%%%%%%%%%%%%%%%%%%%%%%%%%%%%%%%%%%%%%%%%%%%%%%%%%%%%%%
\begin{querybox}[The Query]
\begin{minipage}{0.3\linewidth}
    \includegraphics[width=1\linewidth]{examples/rotting_kiwi_0016.jpg}
\end{minipage}
\hspace{5pt}
\begin{minipage}{0.55\linewidth}
    \textbf{Image path:} \texttt{rotting\_kiwi.jpg}
    
    \textbf{Question:} You are given a 3 x 3 grid in which each cell can contain either no kiwi, one fresh kiwi, or one rotten kiwi. Every minute, any fresh kiwi that is 4-directionally adjacent to a rotten kiwi also becomes rotten. What is the minimum number of minutes that must elapse until no cell has a fresh kiwi?
    \\\\
    \textbf{Ground truth:} 4 minutes
    % \vspace{20pt}
\end{minipage}
\end{querybox}
%%%%%%%%%%%%%%%%%%%%%%%%%%%%%%%%%%%%%%%%%%%%%%%%%%%%%%%%%%%%%%%%


%%%%%%%%%%%%%%%%%%%%%%%%%%%%%%%%%%%%%%%%%%%%%%%%%%%%%%%%%%%%%%%%
\begin{querybox}[Wrong Prediction from \gpt]
To solve this, we need to track how the rot spreads from the rotten kiwi to the fresh ones:

\begin{enumerate}
    \item \textbf{Minute 1:} The rotten kiwi in the bottom left corner will rot the fresh kiwi directly above it (middle left cell).

    \item \textbf{Minute 2:} The newly rotten kiwi in the middle left cell will rot the fresh kiwi directly \red{above it (top left cell) and the fresh kiwi to its right (middle center cell)}.

    \item \red{\textbf{Minute 3:} The rotten kiwi in the middle center cell will rot the fresh kiwi directly above it (top center cell).}
\end{enumerate}

\red{After 3 minutes}, all fresh kiwis will be rotten.
\end{querybox}
%%%%%%%%%%%%%%%%%%%%%%%%%%%%%%%%%%%%%%%%%%%%%%%%%%%%%%%%%%%%%%%%

In contrast, \model addresses the query by first calling \texttt{Image\_Captioner\_Tool} to provide a general description of the associated image. This step is essential as the mistake from the \gpt is because of the wrong perception of the image, with the \texttt{Image\_Captioner\_Tool}, the reasoning over the rotting process is also based on the detailed description of the image, resulting in a correct answer.

%%%%%%%%%%%%%%%%%%%%%%%%%%%%%%%%%%%%%%%%%%%%%%%%%%%%%%%%%%%%%%%%
\begin{plannerbox}[Step 0: Initial Plan from Query Analyzer]
\textbf{Required skills:}

1. Image Analysis: To interpret the grid and identify the state of each kiwi (fresh or rotten).\\
2. Logical Reasoning: To simulate the process of fresh kiwis becoming rotten over time and determine the minimum time required.\\
3. Problem Solving: To apply the rules and constraints effectively to reach the correct answer.
\\\\
\textbf{Relevant tools:}

1. \texttt{Image\_Captioner\_Tool}: To generate a description of the image, identifying the state of each kiwi, which aids in understanding the initial setup.\\
2. \texttt{Generalist\_Solution\_Generator\_Tool}: To simulate the process of kiwis turning rotten and calculate the minimum time required, using logical reasoning based on the image description.
\\\\
\textbf{Additional considerations:}
Ensure the image is accurately interpreted to correctly identify the initial state of each kiwi. Logical reasoning should be applied carefully to simulate the process over time. Verify the final answer against the provided choices.
\end{plannerbox}
%%%%%%%%%%%%%%%%%%%%%%%%%%%%%%%%%%%%%%%%%%%%%%%%%%%%%%%%%%%%%%%%


%%%%%%%%%%%%%%%%%%%%%%%%%%%%%%%%%%%%%%%%%%%%%%%%%%%%%%%%%%%%%%%%
\begin{plannerbox}[Step 1: Action from Action Predictor]

\textbf{Justification:}
The \texttt{Image\_Captioner\_Tool} is the most appropriate choice for the next step because it can generate a detailed description of the image, which is crucial for understanding the initial state of the kiwis in the grid. This understanding is necessary to simulate the process of fresh kiwis becoming rotten and to determine the minimum time required. The tool's ability to describe the image will provide the foundational information needed for logical reasoning in subsequent steps.
\\\\
\textbf{Context:}
Image path: \texttt{rotting\_kiwi.jpg}
\\\\
\textbf{Sub-goal:}
Generate a detailed description of the image 'rotting\_kiwi.jpg' to identify the initial state of each kiwi in the 3x3 grid.
\\\\
\textbf{Tool name:} \texttt{Image\_Captioner\_Tool}
\end{plannerbox}
%%%%%%%%%%%%%%%%%%%%%%%%%%%%%%%%%%%%%%%%%%%%%%%%%%%%%%%%%%%%%%%%


%%%%%%%%%%%%%%%%%%%%%%%%%%%%%%%%%%%%%%%%%%%%%%%%%%%%%%%%%%%%%%%%
\begin{executorbox}[Step 1: Command from Command Generator]

\textbf{Analysis:}
The task requires generating a detailed description of an image to identify the initial state of each kiwi in a 3x3 grid. The selected tool, \texttt{Image\_Captioner\_Tool}, is designed to generate captions for images. The tool's input requires an image path and an optional prompt. The context provides the image path, which is 'rotting\_kiwi.jpg'. The sub-goal is to generate a detailed description of this image, which aligns with the tool's capability to describe images in detail. Therefore, the command will use the image path and a prompt to guide the captioning process.\\\\
\textbf{Explanation:}
The command uses the \texttt{Image\_Captioner\_Tool} to generate a detailed description of the image. The image path is provided directly from the context, and a prompt is used to ensure the tool focuses on describing the initial state of each kiwi in the grid. This setup aligns with the tool's input requirements and the task's sub-goal.
\\\\
\textbf{Command:}
\command{execution = tool.execute(image='rotting\_kiwi.jpg', prompt='Describe the initial state of each kiwi in the 3x3 grid in detail.')}
\end{executorbox}
%%%%%%%%%%%%%%%%%%%%%%%%%%%%%%%%%%%%%%%%%%%%%%%%%%%%%%%%%%%%%%%%


%%%%%%%%%%%%%%%%%%%%%%%%%%%%%%%%%%%%%%%%%%%%%%%%%%%%%%%%%%%%%%%%
\begin{executorbox}[Step 1: Result from Command Executor]
\textbf{Result:}
The 3x3 grid contains the following kiwis:

\begin{itemize}
    \item \textbf{Top Row:}
    \begin{itemize}
        \item Left: Empty
        \item Middle: Empty
        \item Right: A kiwi with a brown outer skin and a green interior with black seeds.
    \end{itemize}

    \item \textbf{Middle Row:}
    \begin{itemize}
        \item Left: A kiwi with a brown outer skin and a green interior with black seeds.
        \item Middle: A kiwi with a brown outer skin and a green interior with black seeds.
        \item Right: A kiwi with a brown outer skin and a green interior with black seeds.
    \end{itemize}

    \item \textbf{Bottom Row:}
    \begin{itemize}
        \item Left: A kiwi with a brown outer skin and a green interior with black seeds, marked with a red "X."
        \item Middle: Empty
        \item Right: Empty
    \end{itemize}
\end{itemize}


\end{executorbox}
%%%%%%%%%%%%%%%%%%%%%%%%%%%%%%%%%%%%%%%%%%%%%%%%%%%%%%%%%%%%%%%%


%%%%%%%%%%%%%%%%%%%%%%%%%%%%%%%%%%%%%%%%%%%%%%%%%%%%%%%%%%%%%%%%
\begin{plannerbox}[Step 1: Verification from Context Verifier]
The memory provides a detailed description of the initial state of each kiwi in the 3x3 grid. The top row has one fresh kiwi, the middle row has three fresh kiwis, and the bottom row has one rotten kiwi. The task is to determine how many minutes it will take for all fresh kiwis to become rotten.The memory is complete and accurate, and no additional tool usage is necessary to determine the answer.
\\\\
\textbf{Conclusion:} \command{\textbf{STOP}}
\end{plannerbox}
%%%%%%%%%%%%%%%%%%%%%%%%%%%%%%%%%%%%%%%%%%%%%%%%%%%%%%%%%%%%%%%%

%%%%%%%%%%%%%%%%%%%%%%%%%%%%%%%%%%%%%%%%%%%%%%%%%%%%%%%%%%%%%%%%

The final answer is summarized by the solution summarizer in the planner, given the full trajectory in the context described above. This two-step flow overcomes the base \gpt model’s limitation in visual perception.

%%%%%%%%%%%%%%%%%%%%%%%%%%%%%%%%%%%%%%%%%%%%%%%%%%%%%%%%%%%%%%%%
\begin{answerbox}[Final Answer]

\textbf{Detailed analysis:}
To solve the problem, we need to determine how many minutes it takes for all fresh kiwis to become rotten, given the initial state of the grid.
\\\\
\textbf{Initial Grid Analysis}
\begin{itemize}
    \item \textbf{Top Row:} 
    \begin{itemize}
        \item Left: Empty
        \item Middle: Empty
        \item Right: Fresh Kiwi
    \end{itemize}

    \item \textbf{Middle Row:}
    \begin{itemize}
        \item Left: Fresh Kiwi
        \item Middle: Fresh Kiwi
        \item Right: Fresh Kiwi
    \end{itemize}

    \item \textbf{Bottom Row:}
    \begin{itemize}
        \item Left: Rotten Kiwi
        \item Middle: Empty
        \item Right: Empty
    \end{itemize}
\end{itemize}

\textbf{Process}
\begin{enumerate}
    \item \textbf{Minute 1:}  
    The rotten kiwi at the bottom left will cause the fresh kiwi directly above it (middle row, left) to become rotten.

    \item \textbf{Minute 2:}  
    The newly rotten kiwi (middle row, left) will cause the adjacent fresh kiwis (middle row, middle) to become rotten.  
    The fresh kiwi in the top row, right, is not adjacent to any rotten kiwi yet.

    \item \textbf{Minute 3:}  
    The rotten kiwi (middle row, middle) will cause the adjacent fresh kiwi (middle row, right) to become rotten.  
    The fresh kiwi in the top row, right, is still not adjacent to any rotten kiwi.

    \item \textbf{Minute 4:}  
    The rotten kiwi (middle row, right) will cause the fresh kiwi in the top row, right, to become rotten.
\end{enumerate}

\textbf{Conclusion:}
It takes 4 minutes for all fresh kiwis to become rotten.

\end{answerbox}
