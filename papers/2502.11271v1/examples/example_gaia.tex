
Below is a challenging example from GAIA-Text \cite{mialon2023gaia} that involves multiple steps, specialized domain knowledge, and rigorous reasoning. Solving it requires understanding the Babylonian number system, recognizing the cuneiform symbols, and performing careful calculations and conversions. The incorrect prediction illustrates a limitation of the base \gpt model. 

%%%%%%%%%%%%%%%%%%%%%%%%%%%%%%%%%%%%%%%%%%%%%%%%%%%%%%%%%%%%%%%%
\begin{querybox}[The Query]
\textbf{Question:}
Consider the following symbols: \eight~\fiftysix

This is a number written using the Mesopotamian/Babylonian number system and represented with Sumerian cuneiform. Convert this number into Arabic numerals as a decimal number.
\\\\
\textbf{Ground truth:} 536
\\
\textbf{Annotated solution (for response validation only):} \\
\textbf{Annotated solution (for response validation only):} \\
1. Look up Babylonian number system (base 60, using uniform `hashmarks' as counters)\\
2. Converted the Cuniform to Arabic (\eight 8 \fiftysix~56)\\
3. Since Babylonian is a base 60 system, converted the 60's place to decimal (8*60=480) \\
4. Added 56 to 480 (536).
\end{querybox}
%%%%%%%%%%%%%%%%%%%%%%%%%%%%%%%%%%%%%%%%%%%%%%%%%%%%%%%%%%%%%%%%


%%%%%%%%%%%%%%%%%%%%%%%%%%%%%%%%%%%%%%%%%%%%%%%%%%%%%%%%%%%%%%%%
\begin{querybox}[Wrong Prediction from \gpt]
The given symbols are written in Sumerian cuneiform and follow the Babylonian number system, which is a \fiftysix.
\\\\
Let's analyze the symbols:
\\
1. \eight: \red{This symbol represents 60 in the Babylonian numeral system.}\\
2. \fiftysix:

\quad - \fifty \red{represents 10}.

\quad - \six \red{represents 2}.

\quad - Together, \fiftysix~\red{represents 12}.
\\\\
Now, interpreting the full number:
\\
\quad - In a positional system, the leftmost symbols represent higher powers of 60.

\quad -  \eight = \red{60} (this is in the $60^1$ place).

\quad - \fiftysix~= \red{12} (this is in the $60^0$ place).
\\\\
Compute the decimal equivalent:
\[
\red{(60 \times 1) + (12 \times 1) = 60 + 12 = 72}
\]

\red{Thus, the number written in Arabic numerals is \textbf{72}.}
\end{querybox}
%%%%%%%%%%%%%%%%%%%%%%%%%%%%%%%%%%%%%%%%%%%%%%%%%%%%%%%%%%%%%%%%

By contrast, in its initial three steps, \model attempts to gather information on the Babylonian system and Sumerian cuneiform symbols via Wikipedia and Google searches, although these initial efforts do not yield useful results. The planner in \model then suggests a more general search approach; at Step 4, \model conducts a broader Google search for general resources, and at Step 5, it obtains the crucial information by visiting one of the returned URLs.
%%%%%%%%%%%%%%%%%%%%%%%%%%%%%%%%%%%%%%%%%%%%%%%%%%%%%%%%%%%%%%%%


\begin{plannerbox}[Step 0: Initial Plan from Query Analyzer]
\textbf{Required skills:}
\\
1. Knowledge of Ancient Number Systems: Understanding the Mesopotamian/Babylonian number system and Sumerian cuneiform symbols is crucial for interpreting the given symbols.\\
2. Historical Linguistics: Ability to translate ancient scripts into modern numerical systems.\\
3. Research Skills: Ability to find reliable sources or references that explain the conversion process from cuneiform to Arabic numerals.
\\\\
\textbf{Relevant tools:}
\\
1. \texttt{Google\_Search\_Tool}: This tool can be used to search for resources or references on the Mesopotamian/Babylonian number system and Sumerian cuneiform symbols, which are necessary for understanding and converting the symbols.\\
2. \texttt{Wikipedia\_Knowledge\_Searcher\_Tool}: This tool can be used to find detailed articles on the Babylonian number system and Sumerian cuneiform, providing background information and conversion methods.
\\\\
\textbf{Additional considerations:}
The task requires a deep understanding of ancient numeral systems, which may not be directly supported by the available tools. Therefore, the tools should be used to gather information and resources that can aid in manual conversion. Additionally, verifying the accuracy of the conversion with multiple sources is recommended to ensure correctness.
\end{plannerbox}
%%%%%%%%%%%%%%%%%%%%%%%%%%%%%%%%%%%%%%%%%%%%%%%%%%%%%%%%%%%%%%%%


%%%%%%%%%%%%%%%%%%%%%%%%%%%%%%%%%%%%%%%%%%%%%%%%%%%%%%%%%%%%%%%%
\begin{plannerbox}[Step 1: Action from Action Predictor]
\textbf{Justification:}
The query requires converting Sumerian cuneiform symbols into Arabic numerals, which involves understanding the Babylonian number system. The \texttt{Wikipedia\_Knowledge\_Searcher\_Tool} is ideal for this task as it can provide detailed articles and resources on ancient number systems, including the Babylonian system and Sumerian cuneiform.
\\\\
\textbf{Context:}
Convert the Sumerian cuneiform symbols \eight \fiftysix~into Arabic numerals as a decimal number.
\\\\
\textbf{Sub-goal:}
Search for information on the Babylonian number system and Sumerian cuneiform symbols to understand their numerical values and conversion methods.
\\\\
\textbf{Tool name:} \texttt{Wikipedia\_Knowledge\_Searcher\_Tool}
\end{plannerbox}
%%%%%%%%%%%%%%%%%%%%%%%%%%%%%%%%%%%%%%%%%%%%%%%%%%%%%%%%%%%%%%%%


%%%%%%%%%%%%%%%%%%%%%%%%%%%%%%%%%%%%%%%%%%%%%%%%%%%%%%%%%%%%%%%%
\begin{executorbox}[Step 1: Command from Command Generator]
\textbf{Command:}
\command{execution = tool.execute(query = "Babylonian number system Sumerian cuneiform symbols \eight \fiftysix")}
\end{executorbox}
%%%%%%%%%%%%%%%%%%%%%%%%%%%%%%%%%%%%%%%%%%%%%%%%%%%%%%%%%%%%%%%%


%%%%%%%%%%%%%%%%%%%%%%%%%%%%%%%%%%%%%%%%%%%%%%%%%%%%%%%%%%%%%%%%
\begin{executorbox}[Step 1: Result from Command Executor]
\textbf{Execution:} No results found for the given query.
\end{executorbox}
%%%%%%%%%%%%%%%%%%%%%%%%%%%%%%%%%%%%%%%%%%%%%%%%%%%%%%%%%%%%%%%%


%%%%%%%%%%%%%%%%%%%%%%%%%%%%%%%%%%%%%%%%%%%%%%%%%%%%%%%%%%%%%%%%
\begin{plannerbox}[Step 1: Verification from Context Verifier]
\textbf{Justification:}
The memory is insufficient to generate the final output. Additional tool usage, specifically the \texttt{Google\_Search\_Tool}, is necessary to gather more information on the Babylonian number system and Sumerian cuneiform symbols to perform the conversion accurately.
\\\\
\textbf{Conclusion:} \command{\textbf{CONTINUE}}
\end{plannerbox}
%%%%%%%%%%%%%%%%%%%%%%%%%%%%%%%%%%%%%%%%%%%%%%%%%%%%%%%%%%%%%%%%


%%%%%%%%%%%%%%%%%%%%%%%%%%%%%%%%%%%%%%%%%%%%%%%%%%%%%%%%%%%%%%%%
\begin{plannerbox}[Step 2: Action from Action Predictor]
\textbf{Justification:}
The previous attempt to use the \texttt{Wikipedia\_Knowledge\_Searcher\_Tool} did not yield any results for the specific cuneiform symbols. To progress, we need a broader search that can provide resources or references on the Mesopotamian/Babylonian number system and Sumerian cuneiform symbols. The \texttt{Google\_Search\_Tool} is well-suited for this task as it can perform a wide-ranging search across the internet, potentially finding more diverse sources that explain the conversion process from cuneiform to Arabic numerals.
% \\\\
% \textbf{Context:}
% Previous search attempt with Wikipedia_Knowledge_Searcher_Tool did not find results for the query 'Babylonian number system Sumerian cuneiform symbols \eight \fiftysix'.
\\\\
\textbf{Sub-goal:}
Perform a Google search to find resources or references on the Mesopotamian/Babylonian number system and Sumerian cuneiform symbols, focusing on conversion methods to Arabic numerals.
\\\\
\textbf{Tool name:} \texttt{Google\_Search\_Tool}
\end{plannerbox}
%%%%%%%%%%%%%%%%%%%%%%%%%%%%%%%%%%%%%%%%%%%%%%%%%%%%%%%%%%%%%%%%


%%%%%%%%%%%%%%%%%%%%%%%%%%%%%%%%%%%%%%%%%%%%%%%%%%%%%%%%%%%%%%%%
\begin{executorbox}[Step 2: Command from Command Generator]
\textbf{Command:} 
\command{execution = tool.execute(query="Mesopotamian Babylonian number system Sumerian cuneiform conversion \eight \fiftysix")}
\end{executorbox}
%%%%%%%%%%%%%%%%%%%%%%%%%%%%%%%%%%%%%%%%%%%%%%%%%%%%%%%%%%%%%%%%


%%%%%%%%%%%%%%%%%%%%%%%%%%%%%%%%%%%%%%%%%%%%%%%%%%%%%%%%%%%%%%%%
\begin{executorbox}[Step 2: Result from Command Executor]
\textbf{Execution:}
\begin{codebox}
{
    "title": "New Capabilities, New Risks? - Evaluating Agentic General ...",
    "link": "https://www.lesswrong.com/posts/Foh7HQYeuN2Gej5k6/new-capabilities-new-risks-evaluating-agentic-general",
    "snippet": "Sep 29, 2024 ... ... \ud809\udc1c \ud809\udc10\ud809\udc1a This is a number written using the Mesopotamian/Babylonian number system and represented with Sumerian cuneiform. Convert this number\u00a0..."
}
\end{codebox}
\end{executorbox}
%%%%%%%%%%%%%%%%%%%%%%%%%%%%%%%%%%%%%%%%%%%%%%%%%%%%%%%%%%%%%%%%


%%%%%%%%%%%%%%%%%%%%%%%%%%%%%%%%%%%%%%%%%%%%%%%%%%%%%%%%%%%%%%%%
\begin{plannerbox}[Step 2: Verification from Context Verifier]
\textbf{Justification:}
The Wikipedia and Google searches did not provide relevant results or conversion methods. The memory is insufficient to generate the final output, as it does not provide the necessary conversion information or methods. Additional tool usage, particularly a more targeted search or reasoning process, is required to address the query fully.
\\\\
\textbf{Conclusion:} \command{\textbf{Continue}}
\end{plannerbox}
%%%%%%%%%%%%%%%%%%%%%%%%%%%%%%%%%%%%%%%%%%%%%%%%%%%%%%%%%%%%%%%%


%%%%%%%%%%%%%%%%%%%%%%%%%%%%%%%%%%%%%%%%%%%%%%%%%%%%%%%%%%%%%%%%
\begin{plannerbox}[Step 3: Action from Action Predictor]
\textbf{Justification:}
The previous steps attempted to gather information on the Babylonian number system and Sumerian cuneiform symbols using Wikipedia and Google searches, but they did not yield useful results. The Wikipedia search returned no results, and the Google search did not provide relevant information directly related to the conversion of the specific symbols. Given the need for detailed and specific information about these ancient symbols, the Wikipedia Knowledge Searcher Tool is a suitable choice for the next step.
\\\\
\textbf{Sub-goal:}
Search Wikipedia for detailed information on the Babylonian number system and Sumerian cuneiform, focusing on the conversion of the symbols \eight and \fiftysix~to Arabic numerals.
\\\\
\textbf{Tool name:} \texttt{Wikipedia\_Knowledge\_Searcher\_Tool}
\end{plannerbox}
%%%%%%%%%%%%%%%%%%%%%%%%%%%%%%%%%%%%%%%%%%%%%%%%%%%%%%%%%%%%%%%%


%%%%%%%%%%%%%%%%%%%%%%%%%%%%%%%%%%%%%%%%%%%%%%%%%%%%%%%%%%%%%%%%
\begin{executorbox}[Step 3: Command from Command Generator]
\textbf{Command:} 
\command{execution = tool.execute(query="Babylonian number system Sumerian cuneiform \eight \fiftysix")}
\end{executorbox}
%%%%%%%%%%%%%%%%%%%%%%%%%%%%%%%%%%%%%%%%%%%%%%%%%%%%%%%%%%%%%%%%


%%%%%%%%%%%%%%%%%%%%%%%%%%%%%%%%%%%%%%%%%%%%%%%%%%%%%%%%%%%%%%%%
\begin{executorbox}[Step 3: Result from Command Executor]
\textbf{Execution:}
No results found for the given query.
\end{executorbox}
%%%%%%%%%%%%%%%%%%%%%%%%%%%%%%%%%%%%%%%%%%%%%%%%%%%%%%%%%%%%%%%%


%%%%%%%%%%%%%%%%%%%%%%%%%%%%%%%%%%%%%%%%%%%%%%%%%%%%%%%%%%%%%%%%
\begin{plannerbox}[Step 3: Verification from Context Verifier]
\textbf{Justification:}
The memory is insufficient to generate the final output as it does not provide the necessary conversion of the cuneiform symbols to Arabic numerals. Additional tool usage, particularly a more focused search or an attempt to reason through the problem with the Generalist Solution Generator Tool, is necessary to address the query.
\\\\
\textbf{Conclusion:} \command{\textbf{Continue}}
\end{plannerbox}
%%%%%%%%%%%%%%%%%%%%%%%%%%%%%%%%%%%%%%%%%%%%%%%%%%%%%%%%%%%%%%%%


%%%%%%%%%%%%%%%%%%%%%%%%%%%%%%%%%%%%%%%%%%%%%%%%%%%%%%%%%%%%%%%%
\begin{plannerbox}[Step 4: Action from Action Predictor]
\textbf{Justification:}
Given the lack of direct results from previous searches using the Wikipedia Knowledge Searcher Tool and the Google Search Tool, the next logical step is to use the Google Search Tool again. This tool can provide a broader range of resources and references that might not be available on Wikipedia. The goal is to find specific resources or articles that explain the conversion of Sumerian cuneiform symbols to Arabic numerals, which is crucial for solving the query. The previous Google search did not yield useful results, possibly due to the specificity of the query. A more general search might yield better results.
\\\\
\textbf{Context:}
Previous Google search query: `Mesopotamian Babylonian number system Sumerian cuneiform conversion \eight \fiftysix'.
\\\\
\textbf{Sub-goal:}
Perform a Google search to find general resources or articles on converting Sumerian cuneiform symbols to Arabic numerals, focusing on understanding the Babylonian number system.
\\\\
\textbf{Tool name:} \texttt{Google\_Search\_Tool}
\end{plannerbox}
%%%%%%%%%%%%%%%%%%%%%%%%%%%%%%%%%%%%%%%%%%%%%%%%%%%%%%%%%%%%%%%%


%%%%%%%%%%%%%%%%%%%%%%%%%%%%%%%%%%%%%%%%%%%%%%%%%%%%%%%%%%%%%%%%
\begin{executorbox}[Step 4: Command from Command Generator]
\textbf{Command:} 
\command{execution = tool.execute(query="Mesopotamian Babylonian number system Sumerian cuneiform conversion")}
\end{executorbox}
%%%%%%%%%%%%%%%%%%%%%%%%%%%%%%%%%%%%%%%%%%%%%%%%%%%%%%%%%%%%%%%%


%%%%%%%%%%%%%%%%%%%%%%%%%%%%%%%%%%%%%%%%%%%%%%%%%%%%%%%%%%%%%%%%
\begin{executorbox}[Step 4: Result from Command Executor]
\textbf{Execution:}
\begin{codebox}
[
    {
        "title": "Babylonian Numerals Converter - Online Number System Calculator",
        "link": "https://www.dcode.fr/babylonian-numbers",
        "snippet": "babylonian, mesopotamian, sumerian, numeral, 60, sexagesimal, babylon, cuneiform, writing, civilization, tablet, clay, wedge, bracket, pipe, bar. Source code."
    },
    ...
]
\end{codebox}
\end{executorbox}
%%%%%%%%%%%%%%%%%%%%%%%%%%%%%%%%%%%%%%%%%%%%%%%%%%%%%%%%%%%%%%%%


%%%%%%%%%%%%%%%%%%%%%%%%%%%%%%%%%%%%%%%%%%%%%%%%%%%%%%%%%%%%%%%%
\begin{plannerbox}[Step 4: Verification from Context Verifier]
\textbf{Justification:}
The memory is insufficient to generate a final output as it lacks the specific conversion information needed. Additional tool usage, particularly more targeted searches or expert consultation, is necessary to obtain the required conversion details.
\\\\
\textbf{Conclusion:} \command{\textbf{Continue}}
\end{plannerbox}
%%%%%%%%%%%%%%%%%%%%%%%%%%%%%%%%%%%%%%%%%%%%%%%%%%%%%%%%%%%%%%%%

%%%%%%%%%%%%%%%%%%%%%%%%%%%%%%%%%%%%%%%%%%%%%%%%%%%%%%%%%%%%%%%%
\begin{plannerbox}[Step 5: Action from Action Predictor]
\textbf{Justification:}
The previous steps attempted to find information on the Babylonian number system and Sumerian cuneiform symbols using both Google and Wikipedia searches, but they did not yield specific results for the symbols \eight and \fiftysix. However, the Google search did return links to resources that might contain the necessary conversion information. The 'Babylonian Numerals Converter - Online Number System Calculator' and 'Babylonian Numbers Converter' links seem promising for directly converting the symbols to Arabic numerals. Therefore, the next logical step is to use the URL\_Text\_Extractor\_Tool to extract text from these URLs, which may contain the conversion information needed.
\\\\
\textbf{Context:}
URL: \url{https://www.dcode.fr/babylonian-numbers}
\\\\
\textbf{Sub-goal:}
Extract text from the URL 'https://www.dcode.fr/babylonian-numbers' to find information on converting Babylonian cuneiform symbols to Arabic numerals.
\\\\
\textbf{Tool name:} \texttt{URL\_Text\_Extractor\_Tool}
\end{plannerbox}
%%%%%%%%%%%%%%%%%%%%%%%%%%%%%%%%%%%%%%%%%%%%%%%%%%%%%%%%%%%%%%%%


%%%%%%%%%%%%%%%%%%%%%%%%%%%%%%%%%%%%%%%%%%%%%%%%%%%%%%%%%%%%%%%%
\begin{executorbox}[Step 5: Command from Command Generator]
\textbf{Command:} 
\command{execution = tool.execute(url="https://www.dcode.fr/babylonian-numbers")}
\end{executorbox}
%%%%%%%%%%%%%%%%%%%%%%%%%%%%%%%%%%%%%%%%%%%%%%%%%%%%%%%%%%%%%%%%


%%%%%%%%%%%%%%%%%%%%%%%%%%%%%%%%%%%%%%%%%%%%%%%%%%%%%%%%%%%%%%%%
\begin{executorbox}[Step 5: Result from Command Executor]
\textbf{Execution:}

...

\textbf{What are babylonian numbers? (Definition)}

Babylonian numeration is a numbering system used by the ancient Babylonians/Sumerians in Mesopotamia to represent numbers. In mesopotamian/babylonian/sumerian number system, numbers are written in a cuneiform style with $|$ (pipe or nail) and $<$ (corner wedge or bracket), written in base 60 (sexagesimal).
\\\\
\textbf{How to write babylonian numbers?}

The number is written in base 60, the 60 digits are broken down into vertical bars (often noted $|$) which are worth one unit (1) and chevrons (often noted $<$) which are worth ten (10) in base 10.

...

Since Unicode 5 (2006) cuneiform symbols can be represented on compatible browsers, here is the table of characters used by dCode:

\includegraphics[width=0.4\textwidth]{examples/dcode.png}

...
\end{executorbox}
%%%%%%%%%%%%%%%%%%%%%%%%%%%%%%%%%%%%%%%%%%%%%%%%%%%%%%%%%%%%%%%%


%%%%%%%%%%%%%%%%%%%%%%%%%%%%%%%%%%%%%%%%%%%%%%%%%%%%%%%%%%%%%%%%
\begin{plannerbox}[Step 5: Verification from Context Verifier]
% \textbf{Completeness:}
% The memory does not fully address the query as it fails to find an expression using the numbers [1, 1, 6, 9] that equals 24. The attempts made with the \texttt{Python\_Code\_Generator\_Tool} resulted in incorrect outputs (-0.75), indicating that the expressions tested were not suitable
% \\\\
% \textbf{Unused Tools:}
% The \texttt{Generalist\_Solution\_Generator\_Tool} has not been used. This tool could potentially provide a step-by-step reasoning approach to explore different combinations of the numbers and operations, which might lead to a correct solution.
% \\\\
% \textbf{Inconsistencies:}
% There are no direct inconsistencies in the memory, but the results obtained do not meet the query's requirement of achieving a result of 24.
% \\\\
% \textbf{Verification Needs:}
% The results from the \texttt{Python\_Code\_Generator\_Tool} need verification because they do not satisfy the query. Further exploration of different expressions is necessary to find a valid solution.
% \\\\
% \textbf{Ambiguities:}
% The current results are clear in their failure to meet the query's requirements, but the path to a correct solution remains ambiguous. Using the \texttt{Generalist\_Solution\_Generator\_Tool} could help clarify potential solutions by suggesting different combinations and logical steps.
% \\\\
\textbf{Justification:}
The memory provides a comprehensive analysis of the query, which involves converting a number written in the Mesopotamian/Babylonian number system using Sumerian cuneiform symbols into Arabic numerals. The initial analysis correctly identifies the need for understanding ancient numeral systems and suggests using the \texttt{Google\_Search\_Tool} and \texttt{Wikipedia\_Knowledge\_Searcher\_Tool} to gather relevant information.
\\\\
The memory shows that multiple attempts were made to find information on the Babylonian number system and Sumerian cuneiform symbols using both the \texttt{Wikipedia\_Knowledge\_Searcher\_Tool} and \texttt{Google\_Search\_Tool}. However, the specific query with the symbols \eight and \fiftysix~did not yield direct results.
\\\\
Despite this, a successful extraction from the URL `\url{https://www.dcode.fr/babylonian-numbers}' provided detailed information on Babylonian numerals, including the values of the symbols \eight (8) and \fifty (50), which are crucial for conversion.
\\\\
\textbf{Conclusion:} \command{\textbf{Continue}}
\end{plannerbox}
%%%%%%%%%%%%%%%%%%%%%%%%%%%%%%%%%%%%%%%%%%%%%%%%%%%%%%%%%%%%%%%%

%%%%%%%%%%%%%%%%%%%%%%%%%%%%%%%%%%%%%%%%%%%%%%%%%%%%%%%%%%%%%%%%
\begin{answerbox}[Final Answer]
To convert the given Sumerian cuneiform symbols \eight and \fiftysix~into Arabic numerals, we need to understand the Babylonian number system, which is a base-60 (sexagesimal) system. Here's a step-by-step breakdown of the conversion process:
\\\\
\textbf{1. Identify the Symbols:}

\quad - \eight represents the number 8.

\quad - \fifty represents the number 50.

\quad - \six represents the number 6. 
\\\\
\textbf{2. Understand the Structure:}

\quad - Babylonian numbers are written in a positional system similar to our decimal system but based on 60. Each position represents a power of 60.
\\\\
\textbf{3. Analyze the Given Symbols:}

\quad - The symbols are written as \eight \fiftysix. This suggests two separate groupings, which can be interpreted as two different positional values.
\\\\
\textbf{4. Convert Each Group:}

\quad - The first symbol \eight (8) is in the higher position, so it represents \(8 \times 60 = 480\).

\quad - The second group \fiftysix~(50 + 6) represents \(56\).
\\\\
\textbf{5. Calculate the Total:}

\quad - Add the values from each group: \(480 + 56 = 536\).
\\\\
Therefore, the Sumerian cuneiform symbols \eight \fiftysix~convert to the Arabic numeral 536.
\end{answerbox}
