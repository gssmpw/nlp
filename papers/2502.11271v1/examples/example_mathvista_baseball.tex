
%%%%%%%%%%%%%%%%%%%%%%%%%%%%%%%%%%%%%%%%%%%%%%%%%%%%%%%%%%%%%%%%
\begin{querybox}[The Query]
\begin{minipage}{0.3\linewidth}
    \includegraphics[width=1\linewidth]{tools/examples/mathvista_113.png}
\end{minipage}
\hspace{5pt}
\begin{minipage}{0.55\linewidth}
    \textbf{Image path:} \texttt{baseball.png}
    
    \textbf{Question:} How many baseballs are there?
    \\\\
    \textbf{Ground truth:} 20
\end{minipage}
\end{querybox}
%%%%%%%%%%%%%%%%%%%%%%%%%%%%%%%%%%%%%%%%%%%%%%%%%%%%%%%%%%%%%%%%


%%%%%%%%%%%%%%%%%%%%%%%%%%%%%%%%%%%%%%%%%%%%%%%%%%%%%%%%%%%%%%%%
\begin{querybox}[Wrong Prediction from \gpt]
\red{There are a total of 19 baseballs across the four buckets:}

- \red{The first bucket (top left) contains 4 baseballs.}

- The second bucket (top right) contains 5 baseballs.

- \red{The third bucket (bottom left) contains 4 baseballs.}

- \red{The fourth bucket (bottom right) contains 6 baseballs.}
\\\\
\red{Adding them together: 4 + 5 + 4 + 6 = 19 baseballs.}
\end{querybox}
%%%%%%%%%%%%%%%%%%%%%%%%%%%%%%%%%%%%%%%%%%%%%%%%%%%%%%%%%%%%%%%%

In contrast, \model decomposes the query into two steps by calling two vision tools to enhance visual perception. First, it calls the \texttt{Image\_Captioner\_Tool} to provide a general description of the associated image. Then, it uses the \texttt{Object\_Detector\_Tool} to detect and count the baseballs. These two steps provide consistent information about the number of baseballs, thus resulting in a correct answer.

%%%%%%%%%%%%%%%%%%%%%%%%%%%%%%%%%%%%%%%%%%%%%%%%%%%%%%%%%%%%%%%%
\begin{plannerbox}[Step 0: Initial Plan from Query Analyzer]

\textbf{Concise summary:}
The query asks, ``How many baseballs are there?'' accompanied by an image showing several buckets containing baseballs. The objective is to determine the total number of baseballs present in the image.
\\\\
\textbf{Required skills:}

1. Image Analysis: Understanding and interpreting visual content. \\
2. Tool Operation: Ability to operate and execute commands using the provided tools. \\
3. Critical Thinking: Evaluating tool outputs and making decisions based on them.
\\\\
\textbf{Relevant tools:}

1. \texttt{Image\_Captioner\_Tool}: Used to generate a description of the image, which can provide context and identify objects present. \\
2. \texttt{Object\_Detector\_Tool}: Used to detect and count the number of baseballs in the image, providing specific object identification and quantification.
\\\\
\textbf{Additional considerations:}
Consider the limitations of each tool, such as potential inaccuracies in complex scenes or object detection. Verify results if possible, and be aware of the need for potential manual verification or supplementary tools for precise counting.
\end{plannerbox}
%%%%%%%%%%%%%%%%%%%%%%%%%%%%%%%%%%%%%%%%%%%%%%%%%%%%%%%%%%%%%%%%


%%%%%%%%%%%%%%%%%%%%%%%%%%%%%%%%%%%%%%%%%%%%%%%%%%%%%%%%%%%%%%%%
\begin{plannerbox}[Step 1: Action from Action Predictor]

\textbf{Justification:}
The \texttt{Image\_Captioner\_Tool} is the best choice for the first step because it provides a general description of the image, which can help identify the presence of baseballs and other relevant objects. This initial step is crucial for understanding the context of the image before proceeding to count specific objects using the \texttt{Object\_Detector\_Tool}.
\\\\
\textbf{Context:}
Image path: \texttt{baseball.png}
\\\\
\textbf{Sub-goal:}
Generate a detailed description of the image located at \texttt{"baseball.png"} to identify the presence of baseballs and other relevant objects.
\\\\
\textbf{Tool name:} \texttt{Image\_Captioner\_Tool}
\end{plannerbox}
%%%%%%%%%%%%%%%%%%%%%%%%%%%%%%%%%%%%%%%%%%%%%%%%%%%%%%%%%%%%%%%%


%%%%%%%%%%%%%%%%%%%%%%%%%%%%%%%%%%%%%%%%%%%%%%%%%%%%%%%%%%%%%%%%
\begin{executorbox}[Step 1: Command from Command Generator]

\textbf{Analysis:}
The task requires two steps: first, using the \texttt{Image\_Captioner\_Tool} to generate a caption for the image, and second, using the \texttt{Object\_Detector\_Tool} to count the number of baseballs. The current focus is on the first step, which involves generating a descriptive caption for the image located at \texttt{'baseball.png'}. The \texttt{Image\_Captioner\_Tool} requires an image path and optionally a prompt. The default prompt is sufficient for generating a general description of the image.
\\\\
\textbf{Explanation:}
The command uses the \texttt{Image\_Captioner\_Tool} to generate a caption for the image. The image path is provided as \texttt{'baseball.png'}, and the default prompt is used to describe the image. This step is necessary to provide context for the subsequent object detection task.
\\\\
\textbf{Command:}
\command{execution = tool.execute(image="baseball.png")}
\end{executorbox}
%%%%%%%%%%%%%%%%%%%%%%%%%%%%%%%%%%%%%%%%%%%%%%%%%%%%%%%%%%%%%%%%


%%%%%%%%%%%%%%%%%%%%%%%%%%%%%%%%%%%%%%%%%%%%%%%%%%%%%%%%%%%%%%%%
\begin{executorbox}[Step 1: Result from Command Executor]
\textbf{Result:}
The image shows four blue buckets, each containing five baseballs. The buckets are arranged in a grid with three on the top row and one on the bottom left. Each bucket is depicted from a top-down perspective, and the baseballs inside are white with red stitching. The buckets have handles on the sides. The background is plain white, emphasizing the buckets and baseballs.
\end{executorbox}
%%%%%%%%%%%%%%%%%%%%%%%%%%%%%%%%%%%%%%%%%%%%%%%%%%%%%%%%%%%%%%%%


%%%%%%%%%%%%%%%%%%%%%%%%%%%%%%%%%%%%%%%%%%%%%%%%%%%%%%%%%%%%%%%%
\begin{plannerbox}[Step 1: Verification from Context Verifier]
\textbf{Completeness:}
The memory addresses the query by using the \texttt{Image\_Captioner\_Tool} to describe the image, which mentions that there are four buckets, each containing five baseballs. This provides a total count of 20 baseballs. However, the \texttt{Object\_Detector\_Tool} has not been used yet to verify this count, which is a requirement of the task.
\\\\
\textbf{Unused Tools:}
The \texttt{Object\_Detector\_Tool} is available and should be used to count the baseballs as per the task description. This tool is specifically designed to detect and count objects, which is necessary to fulfill the query accurately.
\\\\
\textbf{Inconsistencies:}
There are no inconsistencies in the information provided by the \texttt{Image\_Captioner\_Tool}. However, the task requires verification using the \texttt{Object\_Detector\_Tool}, which has not been done.
\\\\
\textbf{Verification needs:}
The count of baseballs needs verification using the \texttt{Object\_Detector\_Tool} due to the task's explicit requirement to use this tool for counting.
\end{plannerbox}
%%%%%%%%%%%%%%%%%%%%%%%%%%%%%%%%%%%%%%%%%%%%%%%%%%%%%%%%%%%%%%%%

%%%%%%%%%%%%%%%%%%%%%%%%%%%%%%%%%%%%%%%%%%%%%%%%%%%%%%%%%%%%%%%%
\begin{plannerbox}[Step 1: Verification from Context Verifier (Continued)]
\textbf{Ambiguities:}
There are no ambiguities in the current description, but the task's requirement to use the \texttt{Object\_Detector\_Tool} suggests that the count should be confirmed using this tool.
\\\\
\textbf{Final determination:}
The memory is not complete as it lacks the use of the \texttt{Object\_Detector\_Tool} to count the baseballs, which is a critical part of the task. Therefore, additional tool usage is necessary to fulfill the query accurately.
\\\\
\textbf{Conclusion:} \command{\textbf{CONTINUE}}
\end{plannerbox}
%%%%%%%%%%%%%%%%%%%%%%%%%%%%%%%%%%%%%%%%%%%%%%%%%%%%%%%%%%%%%%%%


%%%%%%%%%%%%%%%%%%%%%%%%%%%%%%%%%%%%%%%%%%%%%%%%%%%%%%%%%%%%%%%%
\begin{plannerbox}[Step 2: Action from Action Predictor]

\textbf{Justification:}
The next logical step is to use the \texttt{Object\_Detector\_Tool} to count the number of baseballs in the image. The \texttt{Image\_Captioner\_Tool} has already provided a detailed description indicating the presence of baseballs in buckets. The \texttt{Object\_Detector\_Tool} is specifically designed to detect and quantify objects, making it the most suitable choice for counting the baseballs accurately. This step directly addresses the query's requirement to determine the number of baseballs, leveraging the tool's capability to identify and count specific objects within an image.
\\\\
\textbf{Context:}
Image path: \texttt{baseball.png}
\\\\
\textbf{Previous description:}
The image shows four blue buckets, each containing five baseballs. The buckets are arranged in a grid with two on the top row and two on the bottom row. Each bucket has a handle on the side, and the baseballs inside are white with red stitching. The background is plain white, emphasizing the buckets and baseballs.
\\\\
\textbf{Sub-goal:}
Use the \texttt{Object\_Detector\_Tool} to detect and count the number of baseballs in the image located at \texttt{"baseball.png"}.
\\\\
\textbf{Tool name:} \texttt{Object\_Detector\_Tool}
\end{plannerbox}
%%%%%%%%%%%%%%%%%%%%%%%%%%%%%%%%%%%%%%%%%%%%%%%%%%%%%%%%%%%%%%%%


%%%%%%%%%%%%%%%%%%%%%%%%%%%%%%%%%%%%%%%%%%%%%%%%%%%%%%%%%%%%%%%%
\begin{executorbox}[Step 2: Command from Command Generator]

\textbf{Analysis:}
The task requires detecting and counting the number of baseballs in a given image using the \texttt{Object\_Detector\_Tool}. The image path is provided, and the context suggests that the image contains baseballs. The tool's metadata indicates that it requires an image path and a list of labels to detect. The label relevant to our task is `baseball'. The tool also allows setting a confidence threshold, model size, and padding, but these are optional. For this task, we will use the default values for these optional parameters.
\\\\
\textbf{Explanation:}
The command is constructed to use the \texttt{Object\_Detector\_Tool} to detect baseballs in the specified image. We provide the image path and a list containing the label `baseball' to focus the detection on baseballs. The default values for threshold, model size, and padding are used, as they are not specified in the task requirements. This setup will allow the tool to detect and count the baseballs in the image.
\\\\
\textbf{Command:}
\command{execution = tool.execute(image="baseball.png", labels=["baseball"])}
\end{executorbox}
%%%%%%%%%%%%%%%%%%%%%%%%%%%%%%%%%%%%%%%%%%%%%%%%%%%%%%%%%%%%%%%%


%%%%%%%%%%%%%%%%%%%%%%%%%%%%%%%%%%%%%%%%%%%%%%%%%%%%%%%%%%%%%%%%
\begin{executorbox}[Step 2: Result from Command Executor]
\begin{codebox}
[
    {
        "label": "baseball",
        "confidence score": 0.69,
        "box": "(558, 48, 615, 107)",
        "saved_image_path": "solver_cache/baseball_1.png"
    },
    ...
    {
        "label": "baseball",
        "confidence score": 0.63,
        "box": "(336, 95, 393, 153)",
        "saved_image_path": "solver_cache/baseball_20.png"
    }
]
\end{codebox}
\end{executorbox}
%%%%%%%%%%%%%%%%%%%%%%%%%%%%%%%%%%%%%%%%%%%%%%%%%%%%%%%%%%%%%%%%


%%%%%%%%%%%%%%%%%%%%%%%%%%%%%%%%%%%%%%%%%%%%%%%%%%%%%%%%%%%%%%%%
\begin{plannerbox}[Step 2: Verification from Context Verifier]

\textbf{Completeness:}
The memory addresses the query by using both the \texttt{Image\_Captioner\_Tool} and the \texttt{Object\_Detector\_Tool}. The \texttt{Image\_Captioner\_Tool} provided a detailed description of the image, stating that there are four blue buckets, each containing five baseballs, which totals to 20 baseballs. This description aligns with the task's requirement to identify the presence of baseballs.
The \texttt{Object\_Detector\_Tool} was then used to count the baseballs, detecting 20 baseballs in the image. This matches the description provided by the \texttt{Image\_Captioner\_Tool}, confirming the count.
\\\\
\textbf{Unused Tools:}
All relevant tools have been used. The \texttt{Image\_Captioner\_Tool} provided context, and the \texttt{Object\_Detector\_Tool} confirmed the count of baseballs.
\\\\
\textbf{Inconsistencies:}
There are no inconsistencies between the outputs of the two tools. Both tools indicate the presence of 20 baseballs.
\\\\
\textbf{Verification Needs:}
The results from both tools are consistent, reducing the need for further verification. The \texttt{Object\_Detector\_Tool}'s output aligns with the \texttt{Image\_Captioner\_Tool}'s description, providing confidence in the accuracy of the count.
\\\\
\textbf{Ambiguities:}
There are no ambiguities in the results. The tools have provided clear and consistent information regarding the number of baseballs.
\\\\
\textbf{Final Determination:}
The memory is complete and accurate enough to generate the final output. The use of both tools has provided a consistent and verified count of the baseballs, fulfilling the query's requirements.
\\\\
\textbf{Conclusion:} \command{\textbf{STOP}}
\end{plannerbox}
%%%%%%%%%%%%%%%%%%%%%%%%%%%%%%%%%%%%%%%%%%%%%%%%%%%%%%%%%%%%%%%%

%%%%%%%%%%%%%%%%%%%%%%%%%%%%%%%%%%%%%%%%%%%%%%%%%%%%%%%%%%%%%%%%
\begin{answerbox}[Full Trajectory in the Context after Two Steps]
\begin{answercodebox}
"Action Step 1": {
    "tool_name": "Image_Captioner_Tool",
    "sub_goal": "Generate a descriptive caption for the image located at \"baseball.png\" to provide context for subsequent object detection.",
    "command": "execution = tool.execute(image=\"baseball.png\")",
    "result": [
        "The image shows four blue buckets, each containing five baseballs. The buckets are arranged in a grid with three on the top row and one on the bottom left. Each bucket is depicted from a top-down perspective, and the baseballs inside are white with red stitching. The buckets have handles on the sides. The background is plain white, emphasizing the buckets and baseballs."
    ]
},
"Action Step 2": {
    "tool_name": "Object_Detector_Tool",
    "sub_goal": "Use the Object_Detector_Tool to detect and count the number of baseballs in the image located at \"baseball.png\".",
    "command": "execution = tool.execute(image=\"baseball_113.png\", labels=[\"baseball\"])",
    "result": [
        [
            {
                "label": "baseball",
                "confidence score": 0.69,
                "box": "(558, 48, 615, 107)",
                "saved_image_path": "solver_cache/baseball_1.png"
            },
            ...
            {
                "label": "baseball",
                "confidence score": 0.63,
                "box": "(336, 95, 393, 153)",
                "saved_image_path": "solver_cache/0/baseball_20.png"
            }
        ]
    ]
}
\end{answercodebox}
\end{answerbox}
%%%%%%%%%%%%%%%%%%%%%%%%%%%%%%%%%%%%%%%%%%%%%%%%%%%%%%%%%%%%%%%%

The final answer is summarized by the solution summarizer in the planner, given the full trajectory in the context described above. This two-step flow overcomes the base \gpt model’s limitation in visual perception.

%%%%%%%%%%%%%%%%%%%%%%%%%%%%%%%%%%%%%%%%%%%%%%%%%%%%%%%%%%%%%%%%
\begin{answerbox}[Final Answer]
\textbf{Detailed analysis:}

\quad\textbf{Step 1:} \texttt{Image\_Captioner\_Tool}

\quad\quad\quad\quad Result: The image contains four blue buckets, each with five baseballs, arranged in a grid pattern.
   
\quad\textbf{Step 2}: \texttt{Object\_Detector\_Tool}

\quad\quad\quad\quad Result: Detected 20 baseballs with varying confidence scores.
\\\\
\textbf{Key findings:}
The image contains a total of 20 baseballs, distributed evenly across four buckets.
Each bucket contains five baseballs, as confirmed by both tools.
\\\\
\textbf{Answer to the query:}
The image shows four blue buckets, each containing five baseballs. Therefore, \textbf{there are a total of 20 baseballs.}
\\\\
\textbf{Additional insights:}
The consistent results from both tools reinforce the accuracy of the analysis.
The arrangement of the buckets and baseballs is clear and well-organized, aiding in accurate detection.
\end{answerbox}
