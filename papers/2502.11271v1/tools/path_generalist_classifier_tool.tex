\begin{codecolorbox}[Path Generalist Classifier Tool: Metadata]{python}
tool_name="Path_Generalist_Classifier_Tool",

tool_description="A tool for answering multiple choice questions about H&E microscopy images. Do NOT use for open-ended questions. Do NOT use for images that are not H&E-stained.",

input_types={
    "image": "str - The path to the histopathology image.",
    "options": "list[str] - A list of options to classify the image against."
},

output_type="str - The classification result.",

demo_commands=[
    {
        "command": 'execution = tool.execute(image="path/to/image.jpg", options=["lung adenocarcinoma", "lung squamous cell carcinoma"])',
        "description": "Classify the image into one of the given options."
    },
    {
        "command": 'execution = tool.execute(image="path/to/image.png", options=["debris" "cancer-associated stroma", "adipose", "normal colon mucosa", "colorectal adenocarcinoma epithelium", "none of the above"])',
        "description": "Classify the image into one of the given options."
    }
],

user_metadata={
    "limitations": "This tool is designed for answering classification questions about H&E-stained microscopy images. This tool is not suitable for open ended questions. Do NOT use this tool if the input is a natural image, a medical image of other domains (such as IHC, CT, MRI, or X-ray images), or a raw whole slide image (i.e., svs, ndpi, czi, etc). This tool is not always reliable and the result should be cross-referenced by other tools or your own knowledge.",
    "best_practice": "Provide clear and specific options for classification. This tool is ideal for classification tasks where the options are well-defined and specific to histopathology (H&E) images."
}
\end{codecolorbox}


\begin{textcolorbox}[Path Generalist Classifier Tool: Example 1]

\toolinput{query} \texttt{["Non-tumor", "Necrotic tumor", "Viable tumor"]}

\toolinput{image} \texttt{"tissue.png"}

\vspace{0.2cm}
\includegraphics[width=0.4\linewidth]{tools/examples/tissue.jpg}
\\\\
\tooloutput{type}
\begin{codebox}
Viable tumor
\end{codebox}
\tooloutput{structured result}
\begin{codebox}
{
    "result": "Viable tumor"
}
\end{codebox}

\end{textcolorbox}