\begin{textcolorbox}[Prompt for Command Predictor]
\textbf{Task:} Generate a precise command to execute the selected tool based on the given information.
\\\\
\textbf{Query:} \texttt{\{question\}}
\\
\textbf{Image:} \texttt{\{image\}}
\\
\textbf{Context:} \texttt{\{context\}}
\\
\textbf{Sub-Goal:} \texttt{\{sub\_goal\}}
\\
\textbf{Selected Tool:} \texttt{\{tool\_name\}}
\\
\textbf{Tool Metadata:} \texttt{\{tool\_metadata\}}
\\\\
\textbf{Instructions:}
\\\\
\textbf{Instructions:}

1. Carefully review all provided information: the query, image path, context, sub-goal, selected tool, and tool metadata.

2. Analyze the tool's \texttt{input\_types} from the metadata to understand required and optional parameters.

3. Construct a command or series of commands that aligns with the tool's usage pattern and addresses the sub-goal.

4. Ensure all required parameters are included and properly formatted.

5. Use appropriate values for parameters based on the given context, particularly the \texttt{Context} field which may contain relevant information from previous steps.

6. If multiple steps are needed to prepare data for the tool, include them in the command construction.
\\\\
\textbf{Output Format:}

\textless{}\texttt{analysis}\textgreater{}: a step-by-step analysis of the context, sub-goal, and selected tool to guide the command construction.

\textless{}\texttt{explanation}\textgreater{}: a detailed explanation of the constructed command(s) and their parameters.

\textless{}\texttt{command}\textgreater{}: the Python code to execute the tool, which can be one of the following types:

\quad a. A single line command with \texttt{execution = tool.execute()}.

\quad b. A multi-line command with complex data preparation, ending with \texttt{execution = tool.execute()}.

\quad  c. Multiple lines of \texttt{execution = tool.execute()} calls for processing multiple items.
\begin{codebox}
```
python
<your command here>
```
\end{codebox}
\textbf{Rules:}

1. The command MUST be valid Python code and include at least one call to \texttt{tool.execute()}.

2. Each \texttt{tool.execute()} call MUST be assigned to the \texttt{execution} variable in the format \texttt{execution = tool.execute(...)}.

3. For multiple executions, use separate \texttt{execution = tool.execute()} calls for each execution.

4. The final output MUST be assigned to the \texttt{execution} variable, either directly from \texttt{tool.execute()} or as a processed form of multiple executions.

5. Use the exact parameter names as specified in the tool's \texttt{input\_types}.

6. Enclose string values in quotes, use appropriate data types for other values (e.g., lists, numbers).

7. Do not include any code or text that is not part of the actual command.

8. Ensure the command directly addresses the sub-goal and query.

9. Include ALL required parameters, data, and paths to execute the tool in the command itself.

10. If preparation steps are needed, include them as separate Python statements before the \texttt{tool.execute()} calls.
\end{textcolorbox}


\begin{textcolorbox}[Prompt for Command Prediction (Continued)]

\textbf{Examples (Not to use directly unless relevant):}
\\\\
\textbf{Example 1 (Single line command):}

 \textless{}\texttt{analysis}\textgreater{}: The tool requires an image path and a list of labels for object detection.
 
\textless{}\texttt{explanation}\textgreater{}: We pass the image path and a list containing ``baseball'' as the label to detect.

\textless{}\texttt{command}\textgreater{}:
\begin{codebox}
```
python
execution = tool.execute(image="path/to/image", labels=["baseball"])
```
\end{codebox}

\textbf{Example 2 (Multi-line command with data preparation):}

\textless{}\texttt{analysis}\textgreater{}: The tool requires an image path, multiple labels, and a threshold for object detection.

\textless{}\texttt{explanation}\textgreater{}: We prepare the data by defining variables for the image path, labels, and threshold, then pass these to the \texttt{tool.execute()} function.

\textless{}\texttt{command}\textgreater{}:
\begin{codebox}
```
python
image = "path/to/image"
labels = ["baseball", "football", "basketball"]
threshold = 0.5
execution = tool.execute(image=image, labels=labels, threshold=threshold)
```
\end{codebox}


\textbf{Example 3 (Multiple executions):}

\textless{}\texttt{analysis}\textgreater{}: We need to process multiple images for baseball detection.

\textless{}\texttt{explanation}\textgreater{}: We call the tool for each image path, using the same label and threshold for all.

\textless{}\texttt{command}\textgreater{}:
\begin{codebox}
```
python
execution = tool.execute(image="path/to/image1", labels=["baseball"], threshold=0.5)
execution = tool.execute(image="path/to/image2", labels=["baseball"], threshold=0.5)
execution = tool.execute(image="path/to/image3", labels=["baseball"], threshold=0.5)
```
\end{codebox}
\end{textcolorbox}


\begin{textcolorbox}[Prompt for Command Predictor (Continued)]
\textbf{Some Wrong Examples:}

\textless{}\texttt{command}\textgreater{}:
\begin{codebox}
```
python
execution1 = tool.execute(query="...")
execution2 = tool.execute(query="...")
```
\end{codebox}


\textbf{Reason:} only \texttt{execution = tool.execute} is allowed, not \texttt{execution1} or \texttt{execution2}.
\\\\
\textless{}\texttt{command}\textgreater{}:
\begin{codebox}
```
python
urls = [
    "https://example.com/article1",
    "https://example.com/article2"
]

execution = tool.execute(url=urls[0])
execution = tool.execute(url=urls[1])
```
\end{codebox}

\textbf{Reason:} The command should process multiple items in a single execution, not separate executions for each item.
\\\\
\textbf{Remember:} Your \textless{}\texttt{command}\textgreater{} field MUST be valid Python code including any necessary data preparation steps and one or more \texttt{execution = tool.execute()} calls, without any additional explanatory text. The format \texttt{execution = tool.execute} must be strictly followed, and the last line must begin with \texttt{execution = tool.execute} to capture the final output.
\end{textcolorbox}