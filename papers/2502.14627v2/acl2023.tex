% This must be in the first 5 lines to tell arXiv to use pdfLaTeX, which is strongly recommended.
\pdfoutput=1
% In particular, the hyperref package requires pdfLaTeX in order to break URLs across lines.

\documentclass[11pt]{article}

% Remove the "review" option to generate the final version.
\usepackage[]{ACL2023}

% Standard package includes
\usepackage{times}
\usepackage{latexsym}

% For proper rendering and hyphenation of words containing Latin characters (including in bib files)
\usepackage[T1]{fontenc}
\usepackage{amsmath}
\usepackage{amssymb}
\usepackage{subfigure}
\usepackage{graphicx}
\usepackage{subfigure}
\usepackage{multirow}
\usepackage{hyperref} % 参考文献链接颜色为蓝色
% For Vietnamese characters
% \usepackage[T5]{fontenc}
% See https://www.latex-project.org/help/documentation/encguide.pdf for other character sets

% This assumes your files are encoded as UTF8
\usepackage[utf8]{inputenc}

% This is not strictly necessary, and may be commented out.
% However, it will improve the layout of the manuscript,
% and will typically save some space.
\usepackage{microtype}

% This is also not strictly necessary, and may be commented out.
% However, it will improve the aesthetics of text in
% the typewriter font.
\usepackage{inconsolata}


% If the title and author information does not fit in the area allocated, uncomment the following
%
%\setlength\titlebox{<dim>}
%
% and set <dim> to something 5cm or larger.

\title{ATRI: Mitigating Multilingual Audio Text Retrieval Inconsistencies by Reducing Data Distribution Errors}

% Author information can be set in various styles:
% For several authors from the same institution:
\author{Yuguo Yin\textsuperscript{1}, \ Yuxin Xie\textsuperscript{1}, \ Wenyuan Yang\textsuperscript{2}, \ Dongchao Yang\textsuperscript{3}, \ Jinghan Ru\textsuperscript{1},\\ \ \textbf{Xianwei Zhuang\textsuperscript{1}, \ Liming Liang\textsuperscript{1}, \ Yuexian Zou\textsuperscript{1}}\thanks{* Yuexian Zou is the corresponding author.}\\
  \textsuperscript{1}School of ECE, Peking University, China \\
  \textsuperscript{2}School of Cyber Science and Technology, Sun Yat-sen University, China\\  
  \textsuperscript{3} The Chinese University of Hong Kong, China \\
  \texttt{\{ygyin,yuxinxie,jinghanru,xwzhuang,limingliang\}@stu.pku.edu.cn}, \\
  \texttt{yangwy56@mail.sysu.edu.cn},\ \ \ \texttt{dcyang@se.cuhk.edu.hk},\ \ \ \texttt{zouyx@.pku.edu.cn}\\
  }
% \author{Author 1 \and ... \and Author n \\
%         Address line \\ ... \\ Address line}
% if the names do not fit well on one line use
%         Author 1 \\ {\bf Author 2} \\ ... \\ {\bf Author n} \\
% For authors from different institutions:
% \author{Author 1 \\ Address line \\  ... \\ Address line
%         \And  ... \And
%         Author n \\ Address line \\ ... \\ Address line}
% To start a seperate ``row'' of authors use \AND, as in
% \author{Author 1 \\ Address line \\  ... \\ Address line
%         \AND
%         Author 2 \\ Address line \\ ... \\ Address line \And
%         Author 3 \\ Address line \\ ... \\ Address line}

% \author{First Author \\
%   Affiliation / Address line 1 \\
%   Affiliation / Address line 2 \\
%   Affiliation / Address line 3 \\
%   \texttt{email@domain} \\\And
%   Second Author \\
%   Affiliation / Address line 1 \\
%   Affiliation / Address line 2 \\
%   Affiliation / Address line 3 \\
%   \texttt{email@domain} \\}

\begin{document}
\maketitle
\begin{abstract}
Multilingual audio-text retrieval (ML-ATR) is a challenging task that aims to retrieve audio clips or multilingual texts from databases. However, existing ML-ATR schemes suffer from inconsistencies for instance similarity matching across languages. 
We theoretically analyze the inconsistency in terms of both multilingual modal alignment direction error and weight error, and propose the theoretical weight error upper bound for quantifying the inconsistency. Based on the analysis of the weight error upper bound, we find that the inconsistency problem stems from the data distribution error caused by random sampling of languages. We propose a consistent ML-ATR scheme using 1-to-k contrastive learning and audio-English co-anchor contrastive learning, aiming to mitigate the negative impact of data distribution error on recall and consistency in ML-ATR. Experimental results on the translated AudioCaps and Clotho datasets show that our scheme achieves state-of-the-art performance on recall and consistency metrics for eight mainstream languages, including English. 
Our code will be available at \href{https://github.com/ATRI-ACL/ATRI-ACL}{https://github.com/ATRI-ACL/ATRI-ACL}.

% \textcolor{red}{Our code will be available at [URL].}
\end{abstract}

\section{Introduction}

\begin{figure}[h]
    \centering
    \begin{overpic}[trim=0cm 0cm 0cm 0cm,clip,angle=0,origin=c,width=.4\linewidth]{images/teaser_absolute.png}
        %  trim={<left> <lower> <right> <upper>}
        %  \put(horiz, vert)
        %  \put(horiz, vert){\rotatebox{90}{Text}}
        %
        \put(107, 32){$\mathbf{\to}$}
    \end{overpic}\hspace{1cm}
    \begin{overpic}[trim=0cm 0cm 0cm 0cm,clip,angle=0,origin=c,width=.4\linewidth]{images/teaser_translated_yellow.png}
        %  trim={<left> <lower> <right> <upper>}
        %  \put(horiz, vert)
        %  \put(horiz, vert){\rotatebox{90}{Text}}
        %
    \end{overpic}
    \caption{Using translation methods, a controller trained on an environment with a given visual variation \textit{(left)} can be reused without any training or fine-tuning on a different environment (\textit{right}) with comparable performance. In red we see the trajectory of a car driven by the same controller when connected to two different encoders, one for each visual variation.
    }
    \label{fig:teaser}
\end{figure}

Deep Reinforcement Learning (RL) has enabled agents to achieve remarkable performance in complex decision-making tasks, from robotic manipulation to high-dimensional games (Mnih et al., 2015; Silver et al., 2017). 
Although recent RL techniques achieved strong improvements over sample efficiency \citep{yarats2021drqv2, kostrikov2020image}, training new agents remains a costly process, both in computational and temporal terms.
Despite these advances, most methods still require at least partial retraining when dealing with domain shifts such as visual appearance, reward functions, or action spaces \citep{pmlr-v97-cobbe19a, zhang2020learning}. These domain changes typically require expensive retraining, which can be prohibitive for real-world settings that require millions of interactions.

A variety of approaches have been proposed to address these shifting conditions. Domain randomization \citep{tobin2017domain, sadeghi2016cad2rl} trains agents across diverse visual styles or physics settings, promoting invariant features but demanding broader coverage of possible variations. Multi-task RL \citep{parisotto2015actor, teh2017distral} attempts to learn shared representations across multiple tasks.

In the supervised setting, recent representation learning techniques \citep{Moschella2022-yf,maiorca2023latent, norelli2022b, cannistraci2023bricks}, show that it is possible to zero-shot recombine encoders and decoders to perform new tasks across different modalities (images, text..) and tasks (classification, reconstruction) and even architectures.
In RL, methods adopting the relative representation framework \citep{Moschella2022-yf} have shown promising results in adapting encoders to different controllers with zero or few-shots adaptation, for robotic control from proprioceptive states \citep{jian2021adversarial} or for playing games in the Gymnasium suite \citep{towers2024gymnasium} from pixels \citep{ricciardi2025r3lrelativerepresentationsreinforcement}.
These methods, however, still require training models to use the new relative representations.

By contrast, \cite{maiorca2023latent} suggest that modules from independently trained neural networks can be connected via a simple linear or affine transformation, with no training constraint or fine-tuning required, if such transformations can be reliably estimated from a small set of “anchor” samples, pairs of states or observations deemed semantically equivalent.

Our main contribution is the implementation of a RL method based on semantic alignment to map between latent spaces of different neural models, so that their encoders and controllers can be stitched with the goal of creating new agents that can act on visual-task combinations never seen together in training. This includes the use of the transformations to map modules from different networks, and the collection of anchor samples used to estimate these transformations. We call our method Semantic Alignment for Policy Stitching (\textbf{SAPS}).
We perform analyses and empirical tests on the CarRacing and LunarLander environments to show the performance of new agents created via zero-shot stitching of encoders and controllers trained on different visual-task variations, demonstrating significant gains compared to existing zero-shot methods.
\section{Related Work}
\label{sec:related}


\noindentbold{2D visual foundation models}
In recent years, we have witnessed the emergence of large pretrained models—so-called foundation models that are trained on large-scale datasets and serve as a \textit{foundation} for many downstream tasks.
These models demonstrate remarkable versatility across multiple modalities, including language~\cite{team2023gemini,touvron2023llama,touvron2023llama2,dubey2024llama3,vicuna2023,radford2019language,brown2020language,chung2024scaling,achiam2023gpt,bai2023qwen,yang2024qwen2,jiang2023mistral,jiang2024mixtral}, vision~\cite{sam,ravi2024sam,dino_v1,oquab2023dinov2,zou2024segment,rombach2022high,ho2020denoising,nichol2021improved,songdenoising,songscore}, audio~\cite{deshmukh2023pengi,zhang2023speechgpt,rubenstein2023audiopalm,borsos2023audiolm}. 
Furthermore, they enable multi-modal reasoning capabilities that bridge across different modalities~\cite{girdharImageBindOneEmbedding2023,Qwen-VL,llava,radfordLearningTransferableVisual2021,jia2021scaling,team2024gemini}.
Among these models, those that operate on visual modalities are known as visual foundation models (VFM).
VFMs excel in various computer vision tasks such as image segmentation~\cite{sam,ravi2024sam,zou2024segment,zou2023generalized,cheng2021per,cheng2022masked,jain2023oneformer,li2024semantic}, object detection~\cite{liu2023grounding,carion2020end}, representation learning~\cite{dino_v1,oquab2023dinov2}, and open-vocabulary understanding~\cite{radfordLearningTransferableVisual2021,li2022language,ghiasi2022scaling,ram,ram_pp,yu2023convolutions,kang2024defense,naeem2024silc,cho2024cat}.
When integrated with large language models, they enable sophisticated visual reasoning and natural language interactions~\cite{llava,Qwen-VL,girdharImageBindOneEmbedding2023,team2024gemini,guo2024regiongpt,yuan2024osprey,you2023ferret}.
We use such vision language models to construct open vocabulary segmentation and captions for point clouds based on multiview images.







\noindentbold{Open-vocabulary 3D segmentation}
Building on the success of 2D VFMs, recent work have extended open-vocabulary capabilities to 3D scene understanding.
OpenScene~\cite{Peng2023OpenScene} first introduced zero-shot 3D semantic segmentation by distilling knowledge from language-aligned image encoders~\cite{li2022language,ghiasi2022scaling}.
Subsequent methods~\cite{ding2022pla,yang2024regionplc,jiang2024open} leverage multiview images to generate textual captions, which then serve as training supervision.
However, these methods face challenges in generating high-quality 3D mask-text pairs at scale.
For open-vocabulary 3D instance segmentation, existing methods~\cite{takmaz2023openmask3d,nguyen2024open3dis,huang2024openins3d} typically rely on closed-vocabulary proposal networks such as Mask3D~\cite{schult2023mask3d}, which inherently constrains their ability to detect novel object categories. 
Moreover, these methods leverage 2D VFMs like CLIP~\cite{radfordLearningTransferableVisual2021} for region classification by projecting 3D regions onto multiple 2D views.
This approach requires both 2D images and 3D point clouds during inference. Additionally, it necessitates multiple inferences of large 2D models on projected masks, resulting in high computational costs. 
We address these limitations by developing the first single-stage open-vocabulary 3D instance segmentation model that operates directly in 3D without ground truth labels, using our \dataname dataset and Segment3D~\cite{huang2024segment3d} proposals.

\noindentbold{3D vision-language datasets}
Several datasets align 3D scenes with textual annotations to facilitate language-driven 3D understanding. 
ScanRefer~\cite{chen2020scanrefer}, ReferIt3D~\cite{achlioptas2020referit_3d} and EmbodiedScan~\cite{wangEmbodiedScanHolisticMultiModal2023} provide fine-grained object-level localization through detailed referential phrases, while ScanQA~\cite{azuma2022scanqa} targets spatially grounded question-answering. 
In contrast, SceneVerse~\cite{jiaSceneVerseScaling3D2024} and MMScan~\cite{lyu2024mmscan} employ large-language models or vision-language models to partially automate annotation.
Despite leveraging advanced models, these datasets depend significantly on costly human annotations derived from closed-vocabulary sources, limiting their support for open-vocabulary and scalability for large-scale 3D segmentation tasks.

\section{Definition and Inconsistency Analysis}
\label{Sect:Mathematical Demonstration about Inconsistency}
\subsection{Formal Definition of ML-ATR}
Audio-text retrieval is the task of learning cross-modality alignment between audio and multilingual text captions. Contrastive learning \cite{ru2023imbalanced,zhuang2025vargpt} has become the most effective method for learning expressive cross-modality embedding spaces.

Denote a dataset $D=\{(a_i, t_{i1},...t_{iK})\}_{i=1}^N$ as a multilingual audio text retrieval dataset, where $N$ denotes the size of dataset, $K$ refers the total language number in the dataset, $a_i$ denotes the audio in $i$-th data, $t_{ik}$ denotes the $k$-th language in $i$-th data. Given an audio encoder $f_\theta (\cdot)$ and a multilingual text encoder $g_\phi(\cdot)$, we denote the joint probability distribution as:

\begin{equation}
\label{Eq:origin distribution}
\small
    \begin{aligned}
        p(a_i,t_{ik})= \frac{\exp\left(s(f_\theta(a_i), g_\phi(t_{ik})) / \tau \right)}
  {\sum_{j=1}^N\sum_{l=1}^K \exp\left(s(f_\theta(a_j), g_\phi(t_{jl})) / \tau \right)},
    \end{aligned}
\end{equation}

\begin{equation}
\label{Eq:origin distribution}
\small
    \begin{aligned}
        p(a_i,t_{i})= \frac{\exp\left(s(f_\theta(a_i), g_\phi(t_{i})) / \tau \right)}
  {\sum_{j=1}^N \exp\left(s(f_\theta(a_j), g_\phi(t_{j})) / \tau \right)},
    \end{aligned}
\end{equation}

$s(\cdot)$ denotes the cosine similarity between audio and text embedding. The ideal optimization function of learning the embedding space is

\begin{equation}
\small
\begin{aligned}
\max_{\theta, \phi}\sum^{N}_{i=1}\sum^{K}_{k=1}p(a_i,t_{ik}) \mathbb{E}_{(a_i,t_{ik})}[log\ p(a_i,t_{ik})].
\end{aligned}
\end{equation}

However, instead of training all the languages of a piece of data in an epoch, the existing ML-ATR scheme randomly selects the text of a language to do the training. For each epoch $e$, a set of random numbers $Q=\{q_1,... .q_N\},q_i\stackrel{R}{\leftarrow}\{1,...K\}$. The optimization function they used is formalized as:

\begin{equation}
\label{Eq:error distribution}
\small
    \begin{aligned}
        p_e'(a_i,t_{iq_i})= \frac{\exp\left(s(f_\theta(a_i), g_\phi(t_{iq_i})) / \tau \right)}
  {\sum_{j=1}^N \exp\left(s(f_\theta(a_j), g_\phi(t_{jq_j})) / \tau \right)},
    \end{aligned}
\end{equation}

\begin{equation}
\small
\begin{aligned}
\max_{\theta, \phi}\sum^{N}_{i=1}p_e'(a_i,t_{iq_i}) \mathbb{E}_{(a_i,t_{iq_i})}[log\ p_e'(a_i,t_{iq_i})].
\end{aligned}
\end{equation}

The probability distribution $p_e'(a_i,t_{iq_i})$ of their scheme is not the same as the original probability distribution $p(a_i,t_{ik})$. This results in a model that does not fit the training data perfectly, making modality alignment ineffective, which in turn results in reduced recall and inconsistency problems.

\begin{figure}[htbp]
    \centering
    \includegraphics[scale=0.12]{fig/inconsistency_modality_alignment.png}
    \caption{\textbf{A visual illustration of inconsistency due to modality alignment errors}.}
    \label{Fig:modality alignment error}
\end{figure}

\subsection{Analysis of the Inconsistency Issue}
\label{Sect:Description of the Inconsistency Issue}
We first analyze the issue of inconsistency from the perspective of modality alignment directional errors. As shown in Fig. \ref{Fig:modality alignment error}, an intuitive example of modality alignment error is illustrated. Consider a simple case of bilingual audio-text retrieval, let the embedding of an audio sample be $\vec{a}$, and the embeddings of the corresponding texts in two languages be $\vec{t_1}$ and $\vec{t_2}$. Ideally, the audio embedding $\vec{a}$ should be aligned with the combined representation of both text embeddings $\frac{1}{2}(\vec{t_1} + \vec{t_2})$ (indicated by the green arrow). However, in existing ML-ATR schemes, the audio embedding is only aligned with the text embedding of a randomly selected language within each epoch. For instance, if the selected language is $t_2$, the audio embedding $\vec{a}$ will be aligned solely towards $\vec{t_2}$ (indicated by the red arrow). The angle between the red and green arrows is the modality alignment direction error, which makes the audio and multilingual text modes not well aligned.

It's obvious that incorrect alignment introduces noise to the gradient, leading to errors between the model weights and their optimal values, making the model's retrieval recall and consistency metrics degrade. We give a theoretical weight error upper bound and analyze its composition to mitigate the inconsistency problem and improve retrieval recall. The detailed proof can be found in Appendix \ref{Appe:Proof of Weight Error Upper Bound}.

% The incorrect alignment introduces noise into the gradient, resulting in error between the model weights and their optimal values. The performance degradation caused by the current training paradigm can thus be quantified regarding model weight error. Through theoretical analysis, we derive an upper bound for weight error, influenced by data distribution, learning rate, and training dynamics.

We assume that the optimization algorithm is stochastic gradient descent (SGD) \cite{ru2025we} to heuristically analyse the upper bound of the weight error. Given that the number of training steps per epoch $T$, the data distribution obtained by randomly sampling the language according to the existing ATR scheme is denoted as $p_e'$, and the original data distribution is denoted as $p$. $\mathbf w’_{eT}$ denotes the model weight in the $T$-th step under the $e$-th epoch trained with the data distribution $p'_e$, whereas $\mathbf w_{eT}$ denotes the weight that is trained with the data distribution $p$. If the gradient $\nabla_\mathbf w\mathbb{E}_{(a,t)}[log\ p(a,t)]$ is $\lambda_{(x,y)}$-Lipschitz \cite{bethune2023dp}, then we have the following inequality for weight error upper bound:
\begin{equation}
\label{Eq:weight error}
\small
\begin{aligned}
&||\mathbf w_{eT}-\mathbf w'_{eT}||\\
\leq & a^T||\mathbf w_{(e-1)T}-\mathbf w'_{(e-1)T}||+\\
&\eta \sum_{(a,t)}||p(a,t)-p'_e(a,t)||\sum^{T-1}_{j=1}(a^jg_{max}(\mathbf w_{eT-1-j})),
\end{aligned}
\end{equation}

\begin{equation}
\small
\begin{aligned}
g_{max}(\mathbf w)=max_{(a,t)}||\nabla_\mathbf w\mathbb{E}_{(a,t)}[log\ p(a,t)]||,
\end{aligned}
\end{equation}

\begin{equation}
\small
\begin{aligned}
a=1+\eta\sum_{(a,t)}p'_e(a,t)\lambda_{(x,y)}.
\end{aligned}
\end{equation}

\textbf{Note}: The weight $\mathbf w$ consists of the parameter $\theta$ for the audio encoder $f_\theta$ and the parameter $\phi$ for the multilingual text encoder $g_\phi$ in ML-ATR. The data distributions $p$ and $p'_e$ correspond to the Eq. \eqref{Eq:origin distribution} and \eqref{Eq:error distribution}, respectively. For simplicity, we denote $(a,t)$ as all audio-text pairs in the batch of the $T$-th step, where the text $t$ can be in any one of the languages. $\sum_{(a,t)}||p(a,t)-p'_e(a,t)||$ denotes the data distribution error in the batch at step $T$.

Detailed proof of Eq \eqref{Eq:weight error} can be found in Appendix \ref{Appe:Proof of Weight Error Upper Bound}. Based on Eq. \eqref{Eq:weight error}, we have the following results:

\begin{itemize}
    \item Intuitively, the weight error $||\mathbf w_{eT}-\mathbf w'_{eT}||$ comes from two main sources. One is the weight error after the $(e-1)$-th epoch, i.e. $||\mathbf w'_{(e-1)T}-\mathbf w_{(e-1)T}||$. The other is caused by the probabilistic distances of the data distributions, i.e. $\sum_{(a,t)}||p'_e(a,t)-p(a,t)||$. Since $a\geq 1$, the error from both sources increases with epoch and step. In addition, the weight error is also affected by the learning rate $\eta$, the number of training steps $T$ and the maximum gradient $g_{max}(\mathbf w_{eT-1-j})$.
    \item Further expansion of Eq. \eqref{Eq:weight error} shows that the weighting error arises from the data distribution error of each epoch. Expanding $||\mathbf w_{(e-1)T}-\mathbf w'_{(e-1)T}||$ in Eq. \eqref{Eq:weight error}, we find it consist of $||\mathbf w_{(e-2)T}-\mathbf w'_{(e-2)T}||$ and $||p(a,t)-p'_{e-1}(a,t)||$. Further expanding Eq. \eqref{Eq:weight error} to the weight error in $1$-th epoch, it can be concluded that the weight error of the existing ML-ATR scheme comes from the data distribution error $\sum^e_{i=1}\sum_{(a,t)}||p(a,t)-p'_i(a,t)||$ due to the randomly selected languages in each epoch. We can mitigate the inconsistency problem and improve the recall by reducing the weight error upper bound by reducing the data distribution error for each epoch.
\end{itemize}
\section{Proposed ML-ATR Scheme}
We propose two methods to reduce the data distribution error during training. One is 1-to-K contrastive learning, which has a higher memory overhead. The other is audio-English co-anchor contrastive learning, which achieves performance close to 1-to-K Contrastive Learning while approximating the memory overhead to the existing ML-ATR scheme. Here are the details of the two methods.

\subsection{1-to-K Contrastive Learning}
Building on our theoretical analyses, we conclude that reducing data distribution error is critical for addressing the inconsistency problem in multilingual audio-text retrieval. To achieve this, we propose 1-to-K Contrastive Learning (KCL), a training strategy that replaces random language sampling with the simultaneous use of all $K$ linguistic texts corresponding to each audio instance. This approach theoretically eliminates data distribution error, corrects modal alignment direction, and significantly enhances both the recall and consistency of retrieval performance. The loss function $\mathcal{L}^{at}_{kcl}$ for the proposed 1-to-K Contrastive Learning in ML-ATR is defined as follows:

\begin{equation}
\small
\begin{aligned}
\mathcal{L}_{kcl}=\frac{1}{2NK}(\mathcal{L}^{a2t}_{kcl}+\mathcal{L}^{t2a}_{kcl}).
\end{aligned}
\end{equation}

The loss function $\mathcal{L}^{at}_{kcl}$ consists of two parts, $\mathcal{L}^{a2t}_{kcl}$ and $\mathcal{L}^{t2a}_{kcl}$, and they are calculated as follows:

\begin{equation}
\small
\begin{aligned}
\mathcal{L}^{a2t}_{kcl}=-\sum^K_{k=1}\sum^N_{i=1}log\frac{\exp(s(f_\theta(a_i),g_\phi(t_{ik}))/\tau)}{\sum^N_{j=1}\exp(s(f_\theta(a_i),g_\phi(t_{jk}))/\tau)},
\end{aligned}
\end{equation}

$\mathcal{L}^{a2t}_{kcl}$ denotes the contrastive learning loss function from audio to multilingual text.

\begin{equation}
\small
\begin{aligned}
\mathcal{L}^{t2a}_{kcl}=-\sum^K_{k=1}\sum^N_{i=1}log\frac{\exp(s(g_\phi(t_{ik}),f_\theta(a_i))/\tau)}{\sum^N_{j=1}\exp(s(g_\phi(t_{ik}),f_\theta(a_j))/\tau)},
\end{aligned}
\end{equation}

$\mathcal{L}^{t2a}_{kcl}$ denotes the contrastive learning loss function from multilingual text to audio.


$K$ is the number of languages and $N$ is the number of data instances. As shown in Tab. \ref{Tab:overhead}, including multiple multilingual texts in 1-to-K contrastive learning increases GPU memory usage and training time. In practical ML-ATR applications, supporting more languages amplifies these overheads compared to existing schemes.

To address this, we further propose CACL, which improves retrieval consistency and recall without significantly increasing overhead.

\subsection{Audio-English Co-Anchor Contrastive Learning}
To reduce the weighting error with as little increase in training time and GPU memory consumption as possible, we propose audio-English co-anchor contrastive learning (CACL). During the training process, each data takes its audio, English text, and text in other random languages and does contrastive learning with each other. 

For each epoch, given a set of random numbers $Q=\{q_1,...q_N\},q_i\stackrel{R}{\leftarrow}\{2,...K\}$, get the triplet of the training data $(a_i,t_{i1},t_{iq_i})$, where $a_i$ denotes $i$-th audio, $t_{i1}$ denotes the English text, and $t_{iq_i}$ denotes the text of $q_i$-th language. We have the training loss $\mathcal{L}_{cacl}$ shown below:

\begin{equation}
\small
\begin{aligned}
\mathcal{L}_{cacl}=\frac{1}{6N}(\mathcal{L}^{ae}_{cacl}+\mathcal{L}^{at}_{cacl}+\mathcal{L}^{et}_{cacl}).
\end{aligned}
\end{equation}

The loss function $\mathcal{L}_{cacl}$ consists of three components $\mathcal{L}^{ae}_{cacl},\mathcal{L}^{at}_{cacl},\mathcal{L}^{et}_{cacl}$. All three components are based on the following general contrastive learning loss formulation:

\begin{equation}
\small
\begin{aligned}
\mathcal{L}^{uv}_{cacl}=&-\sum^N_{i=1}log\frac{\exp(s(u_i,v_i)/\tau)}{\sum^N_{j=1}\exp(s(u_i,v_j)/\tau)}\\
&-\sum^N_{i=1}log\frac{\exp(s(v_i,u_i)/\tau)}{\sum^N_{j=1}\exp(s(v_i,u_j)/\tau)},
\end{aligned}
\end{equation}
where $u_i$ and $v_i$ represent input embeddings from different modalities or languages. The three components are defined as follows:

\begin{itemize}
    \item \textbf{Audio-English Alignment} ($\mathcal{L}^{ae}_{cacl}$): 
    
    $u_i=f_\theta(a_i)$ represents audio embeddings, and $v_i=g_\phi(t_{i1})$ represents English text embeddings.
    \item \textbf{Audio-Multilingual Alignment} ($\mathcal{L}^{at}_{cacl}$): $u_i=f_\theta(a_i)$ represents audio embeddings, and $v_i=g_\phi(t_{iq_i})$ represents text embeddings in a randomly selected language.
    \item \textbf{English-Multilingual Alignment} ($\mathcal{L}^{et}_{cacl}$): $u_i=g_\phi(t_{i1})$ represents English text embeddings, and $v_i=g_\phi(t_{iq_i})$ represents text embeddings in a randomly selected language.
\end{itemize}

% \begin{equation}
% \small
% \begin{aligned}
% \mathcal{L}^{ae}_{cacl}=&-\sum^N_{i=1}log\frac{\exp(s(f_\theta(a_i),g_\phi(t_{i1}))/\tau)}{\sum^N_{j=1}\exp(s(f_\theta(a_i),g_\phi(t_{j1}))/\tau)}\\
% &-\sum^N_{i=1}log\frac{\exp(s(g_\phi(t_{i1}),f_\theta(a_i))/\tau)}{\sum^N_{j=1}\exp(s(g_\phi(t_{i1}),f_\theta(a_j))/\tau)},
% \end{aligned}
% \end{equation}

% $\mathcal{L}^{ae}_{cacl}$ denotes the contrastive learning loss for modality alignment between audios and English texts.

% \begin{equation}
% \small
% \begin{aligned}
% \mathcal{L}^{at}_{cacl}=&-\sum^N_{i=1}log\frac{\exp(s(f_\theta(a_i),g_\phi(t_{iq_i}))/\tau)}{\sum^N_{j=1}\exp(s(f_\theta(a_i),g_\phi(t_{jq_j}))/\tau)}\\
% &-\sum^N_{i=1}log\frac{\exp(s(g_\phi(t_{iq_i}),f_\theta(a_i))/\tau)}{\sum^N_{j=1}\exp(s(g_\phi(t_{iq_i}),f_\theta(a_j))/\tau)},
% \end{aligned}
% \end{equation}

% $\mathcal{L}^{at}_{cacl}$ denotes the contrastive learning loss for modal alignment between audios and texts in other randomly selected languages.

% \begin{equation}
% \small
% \begin{aligned}
% \mathcal{L}^{et}_{cacl}=&-\sum^N_{i=1}log\frac{\exp(s(g_\phi(t_{i1}),g_\phi(t_{iq_i}))/\tau)}{\sum^N_{j=1}\exp(s(g_\phi(t_{i1}),g_\phi(t_{jq_j}))/\tau)}\\
% &-\sum^N_{i=1}log\frac{\exp(s(g_\phi(t_{iq_i}),g_\phi(t_{i1}))/\tau)}{\sum^N_{j=1}\exp(s(g_\phi(t_{iq_i}),g_\phi(t_{j1}))/\tau)},
% \end{aligned}
% \end{equation}

% $\mathcal{L}^{et}_{cacl}$ denotes the contrastive learning loss for modal alignment between English texts and texts in other randomly selected languages.

The effectiveness of audio-English CACL can be explained from two perspectives:
\begin{itemize}
    \item From the perspective of modality alignment (Fig. \ref{Fig:modality alignment error}), the loss function $\mathcal{L}^{et}_{cacl}$ in CACL brings embeddings of English and other languages closer, reducing the distance between the text embedding $\vec{t_1},\vec{t_2}$ and the mean $\frac{1}{2}(\vec{t_1}+\vec{t_2})$ and minimizing the deviation in the modality alignment direction of audio and text.
    \item From the perspective of data distribution error $\sum_{(a,t)}||p(a,t)-p'_e(a,t)||$ in Eq. \eqref{Eq:weight error}, CACL's loss functions $\mathcal L^{ae}_{cacl}, \mathcal L^{at}_{cacl}$ ensures that the model learns more pairs of audio texts in an epoch. The text in them also contains a large percentage of high-quality English text. It makes the data distribution in CACL closer to the original one, and reduces the weight error of the model.
    %\item The English text in the ML-ATR dataset is manually labeled and less noisy than the text in other languages obtained via translation models. Letting the embedding space of other languages to align with both audio and English text can further mitigate the noise from translated text, improving cross-language audio-text alignment. Additionally, existing English-oriented ATR models already have good pre-trained weights, and CACL can help the alignment of audio with other languages using this pre-existing knowledge.
\end{itemize}

Note that in CACL, the number of texts used for training in each epoch does not increase with the number of languages, which effectively reduces both GPU memory and time overhead in ML-ATR scenarios with a large number of languages. Our experimental results illustrate that CACL approximates the training time and explicit memory overhead of existing ML-ATR schemes, yet achieves recall and consistency metrics close to those of 1-to-K comparative learning.
\section{Experiments}
\subsection{Dataset}
We employ the AudioCaps \cite{kim2019audiocaps}, and Clotho \cite{drossos2020clotho} for our experiments. AudioCaps includes around 49,000 audio samples, each lasting about 10 seconds. Each audio is paired with a single sentence in the training set, while in both the validation and test sets, each audio has five associated sentences. The Clotho dataset consists of 6,974 audio samples, each ranging from 15 to 30 seconds long and annotated with five sentences. It is split into 3,839 training samples, 1,045 validation samples, and 1,045 test samples. 

Additionally, to assess our scheme's performance in the ML-ATR task, we use the Deepseek \cite{bi2024deepseek} API to translate the text from AudioCaps and Clotho into seven widely spoken languages, including French (fra), German (deu), Spanish (spa), Dutch (nld), Catalan (cat), Japanese (jpn), and Chinese (zho).

\subsection{Models}
\textbf{Audio Encoder}: 
We utilize the recently proposed CED-Base model \cite{dinkel2024ced}, a vision transformer with 86 million parameters for the Audio Encoder. Trained on Audioset through knowledge distillation from a large teacher ensemble, the model processes 64-dimensional Mel-spectrograms derived from a 16 kHz signal. It then extracts non-overlapping 16 × 16 patches from the spectrogram, resulting in 248 patches over a 10-second input (4 × 62).
\\
\textbf{Text Encoder}:
The key to multilingual audio-text retrieval is the text encoder's ability to handle texts in multiple languages. In this work, we focus solely on the SONAR-TE model \cite{duquenne2023sonar}. SONAR-TE generates a single vector bottleneck to encapsulate the entire text, avoiding the token-level cross-attention typically employed in conventional sequence-to-sequence machine translation models. The fixed-size text representation is derived by pooling the token-level outputs from the encoder. In the following sections, SONAR refers specifically to the text encoder.

\subsection{Setup}
We use ML-CLAP \cite{yan2024bridging} as the baseline, which is the state-of-the-art for ML-ATR tasks. To have a fair comparison, the model is initialized using the pre-trained weights of ML-CLAP and is further fine-tuned on our multilingual Audiocaps and Clotho datasets using three training methods: ML-CLAP, proposed CACL, and proposed KCL.

All models were fine-tuned for 10 epochs on a single A100 80GB PCIe GPU with a batch size of 24, a learning rate of $5 \times 10^{-6}$, using the Adam optimizer. The temperature hyperparameter $\tau$ was set to 0.07 for all configurations. The audio was sampled at $1.6\times 10^{4}$. We selected the model with the best recall performance during the fine-tuning period for each scheme to perform the experiments.

\subsection{Evaluation Metric}
We use the recall of rank k (R@k) and the average precision of rank 10 (mAP10) as the metrics for the retrieval performance of the model to show that reducing data distribution errors improves the retrieval performance in each language. R@k refers to the fact that for a query, R@k is 1 if the target-value item occurs in the first k retrieved items, and 0 otherwise. mAP10 calculates the average precision of all the queries among the first 10 retrieved results. With these two metrics, we can comprehensively evaluate the retrieval performance of the model on multilingual datasets. 

To assess the consistency of the embedding space across languages, we use three metrics: embedding space gap $\vec{\triangle}_{gap,k}$ \cite{liang2022mind}, average embedding distance $\vec{\triangle}_{dis,k}$, mean rank variance (MRV). The computation of $\vec{\triangle}_{gap,k}$, $\vec{\triangle}_{dis,k}$ and MRV is shown below:


\begin{equation}
\small
    \begin{aligned}
        \vec{\triangle}_{gap,k}=\frac 1 N \sum^N_{i=1}g_\phi(t_{i1})-\frac 1 N \sum^N_{i=1}g_\phi(t_{ik}),
    \end{aligned}
\end{equation}

\begin{equation}
\small
    \begin{aligned}
        \vec{\triangle}_{dis,k}=\frac 1 N \sum^N_{i=1}||g_\phi(t_{i1})-g_\phi(t_{ik})||,
    \end{aligned}
\end{equation}

\begin{equation}
\small
    \begin{aligned}
        MRV=\frac 1 {NK} \sum^N_{i=1}\sum^K_{k=1}|Rank_{ik}-\overline{Rank_j}|^2.
    \end{aligned}
\end{equation}

$\vec{\triangle}_{gap,k}$ and $\vec{\triangle}_{dis,k}$ denotes the embedding space gap and average embedding distance between English and $k$-th language respectively. $Rank_{ik}$ denotes the similarity ranking of the $k$-th language under the $i$-th data, and $\overline{Rank_i}$ denotes the average similarity ranking under the $i$-th data.

\begin{table*}[ht]
\caption{Recall and precision results for baseline and our method under multilingual AudioCaps and Clotho dataset}
\small
\centering
\begin{tabular}{c|c|ccc|ccc|ccc|ccc}
\hline
\multirow{3}{*}{\rotatebox{90}{\textbf{Scheme}}} & \multirow{3}{*}{\textbf{Lang}} & \multicolumn{6}{c|}{\textbf{AudioCaps}} & \multicolumn{6}{c}{\textbf{Clotho}}\\ 
\cline{3-14} & & \multicolumn{3}{c|}{T2A} & \multicolumn{3}{c|}{A2T} & \multicolumn{3}{c|}{T2A} & \multicolumn{3}{c}{A2T}\\
\cline{3-14}
 & & R@1 & R@5 & mAP10 & R@1 & R@5 & mAP10 & R@1 & R@5 & mAP10 & R@1 & R@5 & mAP10 \\ \cline{1-14}
\multirow{9}{*}{\rotatebox{90}{ML-CLAP}} & eng & 47.31 & 80.65 & 61.44 & 64.91 &	90.54 &	38.62 &	25.98 &	54.5 & 38.15 & 34.03 &	61.05 &	21.19 \\ 
& fra & 45.88 &	78.92 &	60.01 &	61.65 &	89.39 &	37.90 &	24.42 &	52.51 &	36.24 &	30.95 &	57.59 &	19.66 \\ 
& deu & 45.60 &	79.49 &	59.93 &	62.65 &	88.76 &	37.88 &	24.08 &	52.61 &	36.40 &	31.62 &	57.40 &	19.39 \\ 
& spa & 45.00 &	79.32 &	59.62 &	63.04 &	88.86 &	37.38 &	24.05 &	52.75 &	36.22 &	31.43 &	57.98 &	19.65 \\ 
& nld & 45.88 &	79.64 &	59.92 &	62.50 &	90.33 &	37.72 &	23.88 &	51.53 &	35.73 &	31.40 &	57.98 &	19.58 \\ 
& cat & 44.36 &	77.89 &	58.58 &	61.65 &	87.60 &	36.43 &	22.83 &	50.84 &	34.80 &	30.91 &	56.43 &	18.26 \\ 
& jpn & 43.04 &	76.86 &	57.54 &	59.45 &	87.81 &	35.20 &	23.04 &	50.34 &	34.89 &	31.43 &	56.55 &	18.77 \\ 
& zho & 41.70 &	74.72 &	55.74 &	53.67 &	84.76 &	33.38 &	21.65 &	48.84 &	33.53 &	28.41 &	56.14 &	17.26 \\ \cline{2-14}
& avg & 44.84 &	78.43 &	59.09 &	61.19 &	88.50 &	36.81 &	23.84 &	51.74 &	35.74 &	31.27 &	57.64 &	19.22 \\ \hline

\multirow{9}{*}{\rotatebox{90}{our CACL}} & eng & 49.05 &	82.14 &	63.07 &	66.31 &	\textbf{91.49} &	39.41 &	26.36 &	55.19 &	38.68 &	34.71 &	61.34 &	\textbf{21.57} \\ 
& fra & 46.86 &	79.97 &	60.83 &	63.23 &	89.48 &	37.92 &	\textbf{24.90} &	\textbf{53.09} &	36.67 &	\textbf{32.40} &	58.55 &	19.85 \\ 
& deu & 46.21 &	80.08 &	60.62 &	63.13 &	\textbf{89.91} &	38.14 &	24.51 &	52.86 &	36.52 &	\textbf{33.36} &	58.07 &	19.49 \\ 
& spa & 46.68 &	80.52 &	60.90 &	63.23 &	\textbf{90.12} &	37.45 &	\textbf{24.59} &	52.71 &	\textbf{36.72} &	32.40 &	58.17 &	19.75 \\ 
& nld & 47.41 &	80.23 &	61.22 &	63.23 &	\textbf{90.86} &	37.95 &	24.15 &	51.75 &	36.05 &	32.21 &	58.65 &	19.5 \\ 
& cat & 45.27 &	78.61 &	59.43 &	61.23 &	88.44 &	36.49 &	23.28 &	51.42 &	35.17 &	30.67 &	56.05 &	18.67 \\ 
& jpn & 44.76 &	78.50 &	58.97 &	61.55 &	88.67 &	34.91 &	23.36 &	51.53 &	35.28 &	\textbf{31.82} &	\textbf{58.26} &	18.99 \\ 
& zho & 42.01 &	76.02 &	56.23 &	56.40 &	86.65 &	33.93 &	22.50 &	49.42 &	34.01 &	27.69 &	\textbf{57.59} &	17.48 \\ \cline{2-14}
& avg & 46.03 &	79.50 &	60.15 &	62.28 &	89.45 &	37.02 &	24.20 &	52.24 &	36.27 &	31.90 &	58.33 &	19.41
 \\ \hline


\multirow{9}{*}{\rotatebox{90}{our KCL}} & eng & \textbf{49.68} &	\textbf{82.44} &	\textbf{63.34} &	\textbf{66.59} &	91.34 &	\textbf{40.52} &	\textbf{26.67} &	\textbf{55.46} &	\textbf{38.97} &	\textbf{36.34} &	\textbf{64.13} &	21.36 \\
& fra & \textbf{47.79} &	\textbf{80.52} &	\textbf{61.53} &	\textbf{63.41} &	\textbf{}\textbf{89.57} &	\textbf{39.21} &	24.61 &	52.73 &	\textbf{36.79} &	31.82 &	\textbf{60.76} &	\textbf{20.02} \\ 
& deu & \textbf{47.81} &	\textbf{80.81} &	\textbf{61.78} &	\textbf{63.34} &	89.28 &	\textbf{39.02} &	\textbf{24.90} &	\textbf{53.25} &	\textbf{37.02} &	33.17 &	\textbf{59.61} &	\textbf{19.90} \\ 
& spa & \textbf{47.33} &	\textbf{80.67} &	\textbf{61.49} &	\textbf{63.76} &	89.39 &	\textbf{38.73} &	24.31 &	\textbf{52.96} &	36.55 &	\textbf{33.36} &	\textbf{61.25} &	\textbf{20.27} \\ 
& nld & \textbf{47.92} &	\textbf{}\textbf{80.76} &	\textbf{61.70} &	\textbf{63.55} &	90.52 &	\textbf{39.14} &	\textbf{24.53} &	\textbf{52.51} &	\textbf{36.61} &	\textbf{33.55} &	\textbf{62.30} &	\textbf{19.98} \\ 
& cat & \textbf{46.44} &	\textbf{79.62} &	\textbf{60.42} &	\textbf{62.71} &	\textbf{89.49} &	\textbf{37.65} &	\textbf{23.67} &	\textbf{51.86} &	\textbf{35.70} &	\textbf{31.53} &	\textbf{57.98} &	\textbf{1}\textbf{8.90} \\ 
& jpn & \textbf{45.27} &	\textbf{78.86} &	\textbf{59.49} &	\textbf{62.28} &	\textbf{89.16} &	\textbf{36.81} &	\textbf{23.65} &	\textbf{52.17} &	\textbf{35.68} &	31.25 &	57.50 &	\textbf{19.49} \\ 
& zho & \textbf{42.25} &	\textbf{76.38} &	\textbf{56.75} &	\textbf{57.66} &	\textbf{87.28} &	\textbf{34.79} &	\textbf{23.09} &	\textbf{49.90} &	\textbf{34.60} &	\textbf{30.48} &	56.34 &	\textbf{17.85} \\ \cline{2-14}
& avg & \textbf{46.81} &	\textbf{80.00} &	\textbf{60.81} &	\textbf{62.91} &	\textbf{89.50} &	\textbf{38.23} &	\textbf{24.42} &	\textbf{52.60} &	\textbf{36.49} &	\textbf{32.68} &	\textbf{59.98} &	\textbf{19.72} \\ \hline
\end{tabular}
\label{Tab:recall and meanr}
\end{table*}

\subsection{Evaluation Result of Recall and Precision}
We present a detailed numerical comparison analysis of the experiment results in Tab \ref{Tab:recall and meanr}, focusing on the performance improvements of our proposed methods, CACL and KCL, over the baseline ML-CLAP across various languages and datasets.

\subsubsection{Analysis of Evaluation Results}
Overall, the proposed CACL and KCL consistently outperform ML-CLAP across the majority of languages and datasets in terms of recall at 1 (R@1), recall at 5 (R@5), and mean average precision at the top 10 results (mAP10) for both Text-to-Audio (T2A) and Audio-to-Text (A2T) tasks. Notably, our proposed KCL achieves state-of-the-art performance, delivering a 5\% improvement in R@1 for the English-oriented monolingual ATR task and a 4.3\% improvement in R@1 for the multilingual ATR task compared to ML-CLAP. This experimental result corroborates our theoretical analysis of the weighting error in Sect. \ref{Sect:Mathematical Demonstration about Inconsistency}. Here is the detailed analysis:

CACL's average metrics across languages are higher than ML-CLAP, while KCL's average metrics across languages have further improvement compared to CACL. Our theoretical analyses in Sect can explain this phenomenon. \ref{Sect:Mathematical Demonstration about Inconsistency}:
\begin{itemize}
    \item CACL uses audio and text together as the anchor point for modality alignment in other languages, which can effectively reduce the data distribution error and modality alignment error, thus achieving better modality alignment results and improved metrics compared to ML-CLAP.
    \item Compared to CACL, which mitigates data distribution errors, KCL theoretically eliminates these errors. As a result, KCL achieves superior modality alignment compared to CACL, leading to further improvements in both recall and precision.
\end{itemize}


\subsubsection{Analysis of Special Situations}

\textbf{Occasional Metric Anomalies}: We observed occasional anomalies where a small proportion of KCL metrics were lower than CACL metrics, and some CACL metrics were lower than ML-CLAP metrics. We attribute these discrepancies to noise in the dataset. Specifically, the weight error in Eq. \eqref{Eq:weight error} represents the difference between the current and optimal model weights for fitting the training data. If the dataset is too noisy, the optimal weights may not improve the test set's performance. As a result, KCL and CACL, which have lower weight errors, may still underperform ML-CLAP on certain metrics. The higher frequency of such anomalies in the noisier Clotho dataset, compared to Audiocaps, supports this explanation. Given that these anomalies are rare among the 108 evaluated metrics, we consider them acceptable and conclude that they do not impact the overall performance advantage of CACL and KCL in the ML-ATR task.

\textbf{Performance Gaps Across Languages}: The lower metrics for Japanese and Chinese in Tab. \ref{Tab:recall and meanr} are mainly due to their significant syntactic differences from other languages, making them harder for the model to learn. Expanding the dataset for these languages could improve the model's performance by providing more representative data.

\textbf{Better Replicated Performance}: Compared to the original paper, our replicated ML-CLAP model achieves significant improvements across all metrics, mainly due to differences in data quality. Compared to the SONAR-translated text used by baseline, the multilingual text we translated with LLM is of higher quality, which in turn can improve the retrieval performance of the model.

\begin{table}[ht]
\caption{Results of spatial differences in the embedding of other languages and English}
\small
\centering
\begin{tabular}{c|c|cc|cc}
\hline
\multirow{3}{*}{\rotatebox{90}{\textbf{Scheme}}} & \multirow{3}{*}{\textbf{Lang}} & \multicolumn{2}{c|}{\textbf{AudioCaps}} & \multicolumn{2}{c}{\textbf{Clotho}}\\ 
\cline{3-6} & & \multicolumn{2}{c|}{E2T} & \multicolumn{2}{c}{E2T}\\
\cline{3-6}
 & & Gap & Dis & Gap & Dis \\ \cline{1-6}
\multirow{8}{*}{\rotatebox{90}{ML-CLAP}} & fra & 0.199 & 0.094 & 0.120 & 0.301\\ 
& deu & 0.210 & 0.370 & 0.124 & 0.289\\ 
& spa & 0.147 & 0.290 & 0.117 & 0.284\\ 
& nld & 0.204 & 0.346 & 0.121 & 0.274\\ 
& cat & 0.151 & 0.357 & 0.121 & 0.307\\ 
& jpn & 0.237 & 0.445 & 0.123 & 0.353\\ 
& zho & 0.181 & 0.414 & 0.177 & 0.323\\ \cline{2-6}
& avg & 0.189 & 0.330 & 0.129 & 0.304\\ \hline

\multirow{8}{*}{\rotatebox{90}{our CACL}} & fra & 0.160 &  0.281 & 0.112 & 0.288\\ 
& deu & 0.194 & 0.334 & 0.103 & 0.261\\ 
& spa & 0.090 & 0.210 & 0.099 & 0.265\\ 
& nld & 0.172 & 0.325 & 0.106 & 0.255\\ 
& cat & 0.104 & 0.252 & 0.108 & 0.280\\ 
& jpn & 0.217 & 0.402 & 0.122 & 0.359\\ 
& zho & 0.192 & 0.381 & 0.159 & 0.352\\ \cline{2-6}
& avg & 0.161 & 0.312 & 0.115 & 0.294\\ \hline

\multirow{8}{*}{\rotatebox{90}{our KCL}} & fra & \textbf{0.145} & \textbf{0.274} & \textbf{0.094} & \textbf{0.261}\\ 
& deu & \textbf{0.155} & \textbf{0.290} & \textbf{0.084} & \textbf{0.231}\\ 
& spa & \textbf{0.081} & \textbf{0.192} & \textbf{0.084} & \textbf{0.230}\\ 
& nld & \textbf{0.148} & \textbf{0.285} & \textbf{0.072} & \textbf{0.204}\\ 
& cat & \textbf{0.092} & \textbf{0.245} & \textbf{0.087} & \textbf{0.243}\\ 
& jpn & \textbf{0.188} & \textbf{0.356} & \textbf{0.106} & \textbf{0.310}\\ 
& zho & \textbf{0.181} & \textbf{0.379} & \textbf{0.123} & \textbf{0.312}\\ \cline{2-6}
& avg & \textbf{0.141} & \textbf{0.288} & \textbf{0.092} & \textbf{0.255}\\ \hline
\end{tabular}
\label{Tab:embeddings gap}
\end{table}


\subsection{Evaluation Result of Consistency}
\subsubsection{Analysis of Embedding Space Consistency}
The results of the consistency metrics embedding space gap $\vec{\triangle}_{gap,k}$ and average embedding distance $\vec{\triangle}_{dis,k}$ are shown in Tab. \ref{Tab:embeddings gap}. In addition, we give a visualization of the embedding space in Appendix \ref{Appe:Embedding Space} and case analysis in Appendix \ref{Appe:Case Analysis} to further illustrate the effectiveness of ATRI in solving the consistency problem.

Smaller values of $\vec{\triangle}_{gap,k}$ and $\vec{\triangle}_{dis,k}$ indicate better alignment of a language's embedding space with English, leading to more consistent retrieval in the ML-ATR task. Compared to the baseline ML-CLAP, CACL achieves an average reduction of 12.9\% in Gap and 4.4\% in Dis, while KCL reduces Gap by 19.1\% and Dis by 14.3\%, demonstrating improved cross-language retrieval consistency.

\begin{table}[ht]
\caption{Results of Mean Rank Variance}
\small
\centering
\begin{tabular}{c|c|c}
\hline
\multirow{2}{*}{\textbf{Scheme}} & \textbf{AudioCaps} & \textbf{Clotho}\\ \cline{2-3}
 & MRV & MRV \\ \hline
ML-CLAP & 10.38 & 347.34 \\ \hline
CACL & 8.71 & 274.87 \\ \hline
KCL & \textbf{7.52} & \textbf{263.15} \\ \hline
\end{tabular}
\label{Tab:MRV}
\end{table}

\subsubsection{Analysis of Rank Consistency}
MRV quantifies the consistency of search rankings across languages, with lower values indicating more consistent results across languages. Unlike metrics based on embedding space, MRV offers a more direct assessment of model consistency in the ML-ATR task. As shown in Tab. \ref{Tab:MRV}, KCL achieves the lowest MRV, representing a 25.9\% reduction compared to ML-CLAP, while CACL achieves a 22.3\% reduction. This effectively shows that the inconsistency issue can be effectively mitigated by reducing the data distribution error.

We note that the MRV metrics under the Audiocaps dataset are significantly lower than Clotho's. This is due to the fact that the Clotho dataset is much noisier and more difficult to get consistent retrieval results across languages.

\begin{table}[ht]
\caption{Evaluation results in GPU memory overheads and time overheads}
\small
\centering
\begin{tabular}{c|cc|cc}
\hline
\multirow{2}{*}{\textbf{Scheme}} & \multicolumn{2}{c|}{\textbf{AudioCaps}} & \multicolumn{2}{c}{\textbf{Clotho}} \\ \cline{2-5}
 & GMO(MB) & TO(s) & GMO(MB) & TO(s)\\ \hline
ML-CLAP & 22172 & 3349 & 30912 & 1592\\ \hline
our CACL & 26788 & 3745 & 31528 & 1714\\ \hline
our KCL & 68256 & 4209 & 79480 & 1884\\ \hline
\end{tabular}
\label{Tab:overhead}
\end{table}

\subsection{Evaluation Results about Training Overhead}
Tab.\ref{Tab:overhead} summarises the GPU memory overhead (GMO) and time overhead (TO) during training for three scenarios: ML-CLAP, CACL, and KCL. KCL training requires simultaneous input of text in eight languages, which significantly increases overhead, resulting in a higher GMO of about 2.8 times and a 27\% increase in TO compared to ML-CLAP. In contrast, CACL inputs just twice as much text as ML-CLAP, resulting in a modest increase of about 10\% in both overheads. This makes CACL more suitable for scenarios that prioritize lower training overheads, while KCL is more suitable for applications that emphasize retrieval performance.
\section{Conclusion}
In this paper, we introduced Atom of Thoughts (\our), a novel framework that transforms complex reasoning processes into a Markov process of atomic questions. By implementing a two-phase transition mechanism of decomposition and contraction, \our eliminates the need to maintain historical dependencies during reasoning, allowing models to focus computational resources on the current question state. Our extensive evaluation across diverse benchmarks demonstrates that \our serves effectively both as a standalone framework and as a plug-in enhancement for existing test-time scaling methods. These results validate \our's ability to enhance LLMs' reasoning capabilities while optimizing computational efficiency through its Markov-style approach to question decomposition and atomic state transitions.


\bibliographystyle{acl_natbib}
\bibliography{reference}
% \bibliographystyle{acl_natbib}
% \bibliography{custom}

\appendix

%\clearpage

\appendix
\section*{Appendix}
\begin{table*}[h!]
\caption{The basic information of grid-based spatio-temporal data.}
\label{tbl:append_data}
\begin{threeparttable}
\resizebox{1.9\columnwidth}{!}{
\begin{tabular}{cccccccc}
\toprule
Dataset & City & Type & Temporal Period & Spatial partition & Interval & Mean & Std \\
\hline
TaxiBJ & Beijing & Taxi flow&  2014/03/01 - 2014/06/30 & $32 \times 32$ & Half an hour & 111.5 & 139.3 \\
BikeDC & Washington, D.C. & Bike flow&  2010/09/20 - 2010/10/20 & $20 \times 20$ & Half an hour & 0.924 & 4.88 \\



CellularSH & Shanghai & Cellular traffic &  2014/08/01 - 2014/08/21 & $32\times28$ & One hour & 0.175 & 0.212 \\
CellularNJ & Nanjing & Cellular traffic &  2021/02/02 - 2021/02/22 & $20\times28$ & One hour & 0.842 & 1.30 \\
CrowdBJ & Beijing & Crowd flow &  2018/01/01 - 2018/01/31 & $1010$ & One hour & 7.07 & 11.1 \\
CrowdBM & Baltimore & Crowd flow &  2019/01/01 - 2019/05/31 & $403$ & One hour & 14.4 & 29.3 \\
Los-Speed & Los Angeles & Traffic speed&  2012/03/01 - 2012/03/07 & $207$ & Five minutes & 59.0 & 12.5 \\

\bottomrule
\end{tabular}}
\end{threeparttable}
\end{table*}

% \begin{table*}[t!]
% \caption{The basic information of Graph-based spatio-temporal data.}
% \label{tbl:append_data_graph}
% \begin{threeparttable}
% \resizebox{1.8\columnwidth}{!}{
% \begin{tabular}{ccccccccc}
% \toprule
% Dataset & City & Type & Temporal Period & Interval & \#Nodes & \#Edges & Mean & Std \\
% \hline

% TrafficBJ & Beijing & Traffic speed & 2022/03/05 - 2022/04/05 & 15min& 13675& 24444& 6.837&  3.412\\
% TrafficSH & Shanghai & Traffic speed & 2022/01/27 - 2022/02/27 & 15min & 21099& 39065& 7.815&  4.044\\
% TrafficNJ & Nanjing & Traffic speed  & 2022/03/05 - 2022/04/05 & 15min & 13419& 25100& 6.699&  4.253\\

% \bottomrule
% \end{tabular}}
% \end{threeparttable}
% \end{table*}
\begin{table*}[h!]
\caption{Short-term prediction results on two additional datasets in terms of both deterministic and probabilistic metrics. \textbf{Bold} indicates the best performance, while \underline{underlining} denotes the second-best.}
\label{tbl:short1-app}
\begin{threeparttable}
% \resizebox{2.0\columnwidth}

\resizebox{1.5\columnwidth}{!}{
\begin{tabular}{ccccccccccc}
\toprule
\multirow{2}{*}{\textbf{Model}}
& \multicolumn{5}{c}{\textbf{CellularNJ}} & \multicolumn{5}{c}{\textbf{CrowdBM}}   \\
\cmidrule(lr){2-6} \cmidrule(lr){7-11} 
 &\textbf{MAE} & \textbf{RMSE}  &\textbf{CRPS} & \textbf{QICE} & \textbf{IS} & 
\textbf{MAE} & \textbf{RMSE} & \textbf{CRPS} & \textbf{QICE} & \textbf{IS} \\


\midrule
D3VAE& 0.580 & 1.135   &	0.565&	0.096&6.03	&	11.0&24.7	&	0.593& 0.110	&136.4\\


DiffSTG& 0.317& 0.649&	0.291&	0.071&3.11	&	8.88&21.3	&0.453&0.047	&68.5\\

TimeGrad& 0.340&  0.357  &	0.432&	0.162&5.87	&10.1	& \underline{12.4}	&\textbf{0.240}&\underline{0.085}	&\underline{46.9}\\


CSDI&0.129 & 0.237   &	\underline{0.111}&	 \underline{0.039}&	\underline{0.80}&	7.31& 19.3&	0.390& 0.054&61.1\\

NPDiff& \underline{0.123}&  \underline{0.175}  &0.128	&	0.133&2.22	&	\underline{5.42}& 13.7	&0.331&0.119	&91.2\\

DyDiffusion&0.222&   0.357 &0.196	&0.080	&	1.80&-	&-	&-	&-&-\\




\cmidrule(lr){1-1} \cmidrule(lr){2-6} \cmidrule(lr){7-11}
\textbf{CoST}&\textbf{0.102} &\textbf{0.172}    &\textbf{0.090}	&\textbf{0.037}	&\textbf{0.682}	&\textbf{5.04}	&\textbf{12.1}	&\underline{0.256}	&\textbf{0.027}& \textbf{37.8}\\
%\textbf{Reduction}& &    &	&	&	&	&	&	&\\
\bottomrule
\end{tabular}}
\end{threeparttable}
\end{table*}


% \input{Tables/short-term2-append}

\end{document}
