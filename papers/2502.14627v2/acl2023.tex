% This must be in the first 5 lines to tell arXiv to use pdfLaTeX, which is strongly recommended.
\pdfoutput=1
% In particular, the hyperref package requires pdfLaTeX in order to break URLs across lines.

\documentclass[11pt]{article}

% Remove the "review" option to generate the final version.
\usepackage[]{ACL2023}

% Standard package includes
\usepackage{times}
\usepackage{latexsym}

% For proper rendering and hyphenation of words containing Latin characters (including in bib files)
\usepackage[T1]{fontenc}
\usepackage{amsmath}
\usepackage{amssymb}
\usepackage{subfigure}
\usepackage{graphicx}
\usepackage{subfigure}
\usepackage{multirow}
\usepackage{hyperref} % 参考文献链接颜色为蓝色
% For Vietnamese characters
% \usepackage[T5]{fontenc}
% See https://www.latex-project.org/help/documentation/encguide.pdf for other character sets

% This assumes your files are encoded as UTF8
\usepackage[utf8]{inputenc}

% This is not strictly necessary, and may be commented out.
% However, it will improve the layout of the manuscript,
% and will typically save some space.
\usepackage{microtype}

% This is also not strictly necessary, and may be commented out.
% However, it will improve the aesthetics of text in
% the typewriter font.
\usepackage{inconsolata}


% If the title and author information does not fit in the area allocated, uncomment the following
%
%\setlength\titlebox{<dim>}
%
% and set <dim> to something 5cm or larger.

\title{ATRI: Mitigating Multilingual Audio Text Retrieval Inconsistencies by Reducing Data Distribution Errors}

% Author information can be set in various styles:
% For several authors from the same institution:
\author{Yuguo Yin\textsuperscript{1}, \ Yuxin Xie\textsuperscript{1}, \ Wenyuan Yang\textsuperscript{2}, \ Dongchao Yang\textsuperscript{3}, \ Jinghan Ru\textsuperscript{1},\\ \ \textbf{Xianwei Zhuang\textsuperscript{1}, \ Liming Liang\textsuperscript{1}, \ Yuexian Zou\textsuperscript{1}}\thanks{* Yuexian Zou is the corresponding author.}\\
  \textsuperscript{1}School of ECE, Peking University, China \\
  \textsuperscript{2}School of Cyber Science and Technology, Sun Yat-sen University, China\\  
  \textsuperscript{3} The Chinese University of Hong Kong, China \\
  \texttt{\{ygyin,yuxinxie,jinghanru,xwzhuang,limingliang\}@stu.pku.edu.cn}, \\
  \texttt{yangwy56@mail.sysu.edu.cn},\ \ \ \texttt{dcyang@se.cuhk.edu.hk},\ \ \ \texttt{zouyx@.pku.edu.cn}\\
  }
% \author{Author 1 \and ... \and Author n \\
%         Address line \\ ... \\ Address line}
% if the names do not fit well on one line use
%         Author 1 \\ {\bf Author 2} \\ ... \\ {\bf Author n} \\
% For authors from different institutions:
% \author{Author 1 \\ Address line \\  ... \\ Address line
%         \And  ... \And
%         Author n \\ Address line \\ ... \\ Address line}
% To start a seperate ``row'' of authors use \AND, as in
% \author{Author 1 \\ Address line \\  ... \\ Address line
%         \AND
%         Author 2 \\ Address line \\ ... \\ Address line \And
%         Author 3 \\ Address line \\ ... \\ Address line}

% \author{First Author \\
%   Affiliation / Address line 1 \\
%   Affiliation / Address line 2 \\
%   Affiliation / Address line 3 \\
%   \texttt{email@domain} \\\And
%   Second Author \\
%   Affiliation / Address line 1 \\
%   Affiliation / Address line 2 \\
%   Affiliation / Address line 3 \\
%   \texttt{email@domain} \\}

\begin{document}
\maketitle
\begin{abstract}
Multilingual audio-text retrieval (ML-ATR) is a challenging task that aims to retrieve audio clips or multilingual texts from databases. However, existing ML-ATR schemes suffer from inconsistencies for instance similarity matching across languages. 
We theoretically analyze the inconsistency in terms of both multilingual modal alignment direction error and weight error, and propose the theoretical weight error upper bound for quantifying the inconsistency. Based on the analysis of the weight error upper bound, we find that the inconsistency problem stems from the data distribution error caused by random sampling of languages. We propose a consistent ML-ATR scheme using 1-to-k contrastive learning and audio-English co-anchor contrastive learning, aiming to mitigate the negative impact of data distribution error on recall and consistency in ML-ATR. Experimental results on the translated AudioCaps and Clotho datasets show that our scheme achieves state-of-the-art performance on recall and consistency metrics for eight mainstream languages, including English. 
Our code will be available at \href{https://github.com/ATRI-ACL/ATRI-ACL}{https://github.com/ATRI-ACL/ATRI-ACL}.

% \textcolor{red}{Our code will be available at [URL].}
\end{abstract}



\section{Introduction}
\IEEEPARstart{I}{n} recent years, flourishing of Artificial Intelligence Generated Content (AIGC) has sparked significant advancements in modalities such as text, image, audio, and even video. 
Among these, AI-Generated Image (AGI) has garnered considerable interest from both researchers and the public.
Plenty of remarkable AGI models and online services, such as StableDiffusion\footnote{\url{https://stability.ai/}}, Midjourney\footnote{\url{https://www.midjourney.com/}}, and FLUX\footnote{\url{https://blackforestlabs.ai/}}, offer users an excellent creative experience.
However, users often remain critical of the quality of the AGI due to image distortions or mismatches with user intentions.
Consequently, methods for assessing the quality of AGI are becoming increasingly crucial to help improve the generative capabilities of these models.

Unlike Natural Scene Image (NSI) quality assessment, which focuses primarily on perception aspects such as sharpness, color, and brightness, AI-Generated Image Quality Assessment (AGIQA) encompasses additional aspects like correspondence and authenticity. 
Since AGI is generated on the basis of user text prompts, it may fail to capture key user intentions, resulting in misalignment with the prompt.
Furthermore, authenticity refers to how closely the generated image resembles real-world artworks, as AGI can sometimes exhibit logical inconsistencies.
While traditional IQA models may effectively evaluate perceptual quality, they are often less capable of adequately assessing aspects such as correspondence and authenticity.

\begin{figure}\label{fig:radar}
    \centering
    \includegraphics[width=1.0\linewidth]{figures/radar_plot.pdf}
    \caption{A comparison on quality, correspondence, and authenticity aspects of AIGCIQA2023~\cite{wang2023aigciqa2023} dataset illustrates the superior performance of our method.}
\end{figure}

Several methods have been proposed specifically for the AGIQA task, including metrics designed to evaluate the authenticity and diversity of generated images~\cite{gulrajani2017improved,heusel2017gans}. 
Nevertheless, these methods tend to compare and evaluate grouped images rather than single instances, which limits their utility for single image assessment.
Beginning with AGIQA-1k~\cite{zhang2023perceptual}, a series of AGIQA databases have been introduced, including AGIQA-3k~\cite{li2023agiqa}, AIGCIQA-20k~\cite{li2024aigiqa}, etc.
Concurrently, there has been a surge in research utilizing deep learning methods~\cite{zhou2024adaptive,peng2024aigc,yu2024sf}, which have significantly benefited from pre-trained models such as CLIP~\cite{radford2021learning}. 
These approaches enhance the analysis by leveraging the correlations between images and their descriptive texts.
While these models are effective in capturing general text-image alignments, they may not effectively detect subtle inconsistencies or mismatches between the generated image content and the detailed nuances of the textual description.
Moreover, as these models are pre-trained on large-scale datasets for broad tasks, they might not fully exploit the textual information pertinent to the specific context of AGIQA without task-specific fine-tuning.
To overcome these limitations, methods that leverage Multimodal Large Language Models (MLLMs)~\cite{wang2024large,wang2024understanding} have been proposed.
These methods aim to fully exploit the synergies of image captioning and textual analysis for AGIQA.
Although they benefit from advanced prompt understanding, instruction following, and generation capabilities, they often do not utilize MLLMs as encoders capable of producing a sequence of logits that integrate both image and text context.

In conclusion, the field of AI-Generated Image Quality Assessment (AGIQA) continues to face significant challenges: 
(1) Developing comprehensive methods to assess AGIs from multiple dimensions, including quality, correspondence, and authenticity; 
(2) Enhancing assessment techniques to more accurately reflect human perception and the nuanced intentions embedded within prompts; 
(3) Optimizing the use of Multimodal Large Language Models (MLLMs) to fully exploit their multimodal encoding capabilities.

To address these challenges, we propose a novel method M3-AGIQA (\textbf{M}ultimodal, \textbf{M}ulti-Round, \textbf{M}ulti-Aspect AI-Generated Image Quality Assessment) which leverages MLLMs as both image and text encoders. 
This approach incorporates an additional network to align human perception and intentions, aiming to enhance assessment accuracy. 
Specially, we distill the rich image captioning capability from online MLLMs into a local MLLM through Low-Rank Adaption (LoRA) fine-tuning, and train this model with human-labeled data. The key contributions of this paper are as follows:
\begin{itemize}
    \item We propose a novel AGIQA method that distills multi-aspect image captioning capabilities to enable comprehensive evaluation. Specifically, we use an online MLLM service to generate aspect-specific image descriptions and fine-tune a local MLLM with these descriptions in a structured two-round conversational format.
    \item We investigate the encoding potential of MLLMs to better align with human perceptual judgments and intentions, uncovering previously underestimated capabilities of MLLMs in the AGIQA domain. To leverage sequential information, we append an xLSTM feature extractor and a regression head to the encoding output.
    \item Extensive experiments across multiple datasets demonstrate that our method achieves superior performance, setting a new state-of-the-art (SOTA) benchmark in AGIQA.
\end{itemize}

In this work, we present related works in Sec.~\ref{sec:related}, followed by the details of our M3-AGIQA method in Sec.~\ref{sec:method}. Sec.~\ref{sec:exp} outlines our experimental design and presents the results. Sec.~\ref{sec:limit},~\ref{sec:ethics} and~\ref{sec:conclusion} discuss the limitations, ethical concerns, future directions and conclusions of our study.
\section{Related Work}
\label{sec:related}


\noindentbold{2D visual foundation models}
In recent years, we have witnessed the emergence of large pretrained models—so-called foundation models that are trained on large-scale datasets and serve as a \textit{foundation} for many downstream tasks.
These models demonstrate remarkable versatility across multiple modalities, including language~\cite{team2023gemini,touvron2023llama,touvron2023llama2,dubey2024llama3,vicuna2023,radford2019language,brown2020language,chung2024scaling,achiam2023gpt,bai2023qwen,yang2024qwen2,jiang2023mistral,jiang2024mixtral}, vision~\cite{sam,ravi2024sam,dino_v1,oquab2023dinov2,zou2024segment,rombach2022high,ho2020denoising,nichol2021improved,songdenoising,songscore}, audio~\cite{deshmukh2023pengi,zhang2023speechgpt,rubenstein2023audiopalm,borsos2023audiolm}. 
Furthermore, they enable multi-modal reasoning capabilities that bridge across different modalities~\cite{girdharImageBindOneEmbedding2023,Qwen-VL,llava,radfordLearningTransferableVisual2021,jia2021scaling,team2024gemini}.
Among these models, those that operate on visual modalities are known as visual foundation models (VFM).
VFMs excel in various computer vision tasks such as image segmentation~\cite{sam,ravi2024sam,zou2024segment,zou2023generalized,cheng2021per,cheng2022masked,jain2023oneformer,li2024semantic}, object detection~\cite{liu2023grounding,carion2020end}, representation learning~\cite{dino_v1,oquab2023dinov2}, and open-vocabulary understanding~\cite{radfordLearningTransferableVisual2021,li2022language,ghiasi2022scaling,ram,ram_pp,yu2023convolutions,kang2024defense,naeem2024silc,cho2024cat}.
When integrated with large language models, they enable sophisticated visual reasoning and natural language interactions~\cite{llava,Qwen-VL,girdharImageBindOneEmbedding2023,team2024gemini,guo2024regiongpt,yuan2024osprey,you2023ferret}.
We use such vision language models to construct open vocabulary segmentation and captions for point clouds based on multiview images.







\noindentbold{Open-vocabulary 3D segmentation}
Building on the success of 2D VFMs, recent work have extended open-vocabulary capabilities to 3D scene understanding.
OpenScene~\cite{Peng2023OpenScene} first introduced zero-shot 3D semantic segmentation by distilling knowledge from language-aligned image encoders~\cite{li2022language,ghiasi2022scaling}.
Subsequent methods~\cite{ding2022pla,yang2024regionplc,jiang2024open} leverage multiview images to generate textual captions, which then serve as training supervision.
However, these methods face challenges in generating high-quality 3D mask-text pairs at scale.
For open-vocabulary 3D instance segmentation, existing methods~\cite{takmaz2023openmask3d,nguyen2024open3dis,huang2024openins3d} typically rely on closed-vocabulary proposal networks such as Mask3D~\cite{schult2023mask3d}, which inherently constrains their ability to detect novel object categories. 
Moreover, these methods leverage 2D VFMs like CLIP~\cite{radfordLearningTransferableVisual2021} for region classification by projecting 3D regions onto multiple 2D views.
This approach requires both 2D images and 3D point clouds during inference. Additionally, it necessitates multiple inferences of large 2D models on projected masks, resulting in high computational costs. 
We address these limitations by developing the first single-stage open-vocabulary 3D instance segmentation model that operates directly in 3D without ground truth labels, using our \dataname dataset and Segment3D~\cite{huang2024segment3d} proposals.

\noindentbold{3D vision-language datasets}
Several datasets align 3D scenes with textual annotations to facilitate language-driven 3D understanding. 
ScanRefer~\cite{chen2020scanrefer}, ReferIt3D~\cite{achlioptas2020referit_3d} and EmbodiedScan~\cite{wangEmbodiedScanHolisticMultiModal2023} provide fine-grained object-level localization through detailed referential phrases, while ScanQA~\cite{azuma2022scanqa} targets spatially grounded question-answering. 
In contrast, SceneVerse~\cite{jiaSceneVerseScaling3D2024} and MMScan~\cite{lyu2024mmscan} employ large-language models or vision-language models to partially automate annotation.
Despite leveraging advanced models, these datasets depend significantly on costly human annotations derived from closed-vocabulary sources, limiting their support for open-vocabulary and scalability for large-scale 3D segmentation tasks.

\section{Definition and Inconsistency Analysis}
\label{Sect:Mathematical Demonstration about Inconsistency}
\subsection{Formal Definition of ML-ATR}
Audio-text retrieval is the task of learning cross-modality alignment between audio and multilingual text captions. Contrastive learning \cite{ru2023imbalanced,zhuang2025vargpt} has become the most effective method for learning expressive cross-modality embedding spaces.

Denote a dataset $D=\{(a_i, t_{i1},...t_{iK})\}_{i=1}^N$ as a multilingual audio text retrieval dataset, where $N$ denotes the size of dataset, $K$ refers the total language number in the dataset, $a_i$ denotes the audio in $i$-th data, $t_{ik}$ denotes the $k$-th language in $i$-th data. Given an audio encoder $f_\theta (\cdot)$ and a multilingual text encoder $g_\phi(\cdot)$, we denote the joint probability distribution as:

\begin{equation}
\label{Eq:origin distribution}
\small
    \begin{aligned}
        p(a_i,t_{ik})= \frac{\exp\left(s(f_\theta(a_i), g_\phi(t_{ik})) / \tau \right)}
  {\sum_{j=1}^N\sum_{l=1}^K \exp\left(s(f_\theta(a_j), g_\phi(t_{jl})) / \tau \right)},
    \end{aligned}
\end{equation}

\begin{equation}
\label{Eq:origin distribution}
\small
    \begin{aligned}
        p(a_i,t_{i})= \frac{\exp\left(s(f_\theta(a_i), g_\phi(t_{i})) / \tau \right)}
  {\sum_{j=1}^N \exp\left(s(f_\theta(a_j), g_\phi(t_{j})) / \tau \right)},
    \end{aligned}
\end{equation}

$s(\cdot)$ denotes the cosine similarity between audio and text embedding. The ideal optimization function of learning the embedding space is

\begin{equation}
\small
\begin{aligned}
\max_{\theta, \phi}\sum^{N}_{i=1}\sum^{K}_{k=1}p(a_i,t_{ik}) \mathbb{E}_{(a_i,t_{ik})}[log\ p(a_i,t_{ik})].
\end{aligned}
\end{equation}

However, instead of training all the languages of a piece of data in an epoch, the existing ML-ATR scheme randomly selects the text of a language to do the training. For each epoch $e$, a set of random numbers $Q=\{q_1,... .q_N\},q_i\stackrel{R}{\leftarrow}\{1,...K\}$. The optimization function they used is formalized as:

\begin{equation}
\label{Eq:error distribution}
\small
    \begin{aligned}
        p_e'(a_i,t_{iq_i})= \frac{\exp\left(s(f_\theta(a_i), g_\phi(t_{iq_i})) / \tau \right)}
  {\sum_{j=1}^N \exp\left(s(f_\theta(a_j), g_\phi(t_{jq_j})) / \tau \right)},
    \end{aligned}
\end{equation}

\begin{equation}
\small
\begin{aligned}
\max_{\theta, \phi}\sum^{N}_{i=1}p_e'(a_i,t_{iq_i}) \mathbb{E}_{(a_i,t_{iq_i})}[log\ p_e'(a_i,t_{iq_i})].
\end{aligned}
\end{equation}

The probability distribution $p_e'(a_i,t_{iq_i})$ of their scheme is not the same as the original probability distribution $p(a_i,t_{ik})$. This results in a model that does not fit the training data perfectly, making modality alignment ineffective, which in turn results in reduced recall and inconsistency problems.

\begin{figure}[htbp]
    \centering
    \includegraphics[scale=0.12]{fig/inconsistency_modality_alignment.png}
    \caption{\textbf{A visual illustration of inconsistency due to modality alignment errors}.}
    \label{Fig:modality alignment error}
\end{figure}

\subsection{Analysis of the Inconsistency Issue}
\label{Sect:Description of the Inconsistency Issue}
We first analyze the issue of inconsistency from the perspective of modality alignment directional errors. As shown in Fig. \ref{Fig:modality alignment error}, an intuitive example of modality alignment error is illustrated. Consider a simple case of bilingual audio-text retrieval, let the embedding of an audio sample be $\vec{a}$, and the embeddings of the corresponding texts in two languages be $\vec{t_1}$ and $\vec{t_2}$. Ideally, the audio embedding $\vec{a}$ should be aligned with the combined representation of both text embeddings $\frac{1}{2}(\vec{t_1} + \vec{t_2})$ (indicated by the green arrow). However, in existing ML-ATR schemes, the audio embedding is only aligned with the text embedding of a randomly selected language within each epoch. For instance, if the selected language is $t_2$, the audio embedding $\vec{a}$ will be aligned solely towards $\vec{t_2}$ (indicated by the red arrow). The angle between the red and green arrows is the modality alignment direction error, which makes the audio and multilingual text modes not well aligned.

It's obvious that incorrect alignment introduces noise to the gradient, leading to errors between the model weights and their optimal values, making the model's retrieval recall and consistency metrics degrade. We give a theoretical weight error upper bound and analyze its composition to mitigate the inconsistency problem and improve retrieval recall. The detailed proof can be found in Appendix \ref{Appe:Proof of Weight Error Upper Bound}.

% The incorrect alignment introduces noise into the gradient, resulting in error between the model weights and their optimal values. The performance degradation caused by the current training paradigm can thus be quantified regarding model weight error. Through theoretical analysis, we derive an upper bound for weight error, influenced by data distribution, learning rate, and training dynamics.

We assume that the optimization algorithm is stochastic gradient descent (SGD) \cite{ru2025we} to heuristically analyse the upper bound of the weight error. Given that the number of training steps per epoch $T$, the data distribution obtained by randomly sampling the language according to the existing ATR scheme is denoted as $p_e'$, and the original data distribution is denoted as $p$. $\mathbf w’_{eT}$ denotes the model weight in the $T$-th step under the $e$-th epoch trained with the data distribution $p'_e$, whereas $\mathbf w_{eT}$ denotes the weight that is trained with the data distribution $p$. If the gradient $\nabla_\mathbf w\mathbb{E}_{(a,t)}[log\ p(a,t)]$ is $\lambda_{(x,y)}$-Lipschitz \cite{bethune2023dp}, then we have the following inequality for weight error upper bound:
\begin{equation}
\label{Eq:weight error}
\small
\begin{aligned}
&||\mathbf w_{eT}-\mathbf w'_{eT}||\\
\leq & a^T||\mathbf w_{(e-1)T}-\mathbf w'_{(e-1)T}||+\\
&\eta \sum_{(a,t)}||p(a,t)-p'_e(a,t)||\sum^{T-1}_{j=1}(a^jg_{max}(\mathbf w_{eT-1-j})),
\end{aligned}
\end{equation}

\begin{equation}
\small
\begin{aligned}
g_{max}(\mathbf w)=max_{(a,t)}||\nabla_\mathbf w\mathbb{E}_{(a,t)}[log\ p(a,t)]||,
\end{aligned}
\end{equation}

\begin{equation}
\small
\begin{aligned}
a=1+\eta\sum_{(a,t)}p'_e(a,t)\lambda_{(x,y)}.
\end{aligned}
\end{equation}

\textbf{Note}: The weight $\mathbf w$ consists of the parameter $\theta$ for the audio encoder $f_\theta$ and the parameter $\phi$ for the multilingual text encoder $g_\phi$ in ML-ATR. The data distributions $p$ and $p'_e$ correspond to the Eq. \eqref{Eq:origin distribution} and \eqref{Eq:error distribution}, respectively. For simplicity, we denote $(a,t)$ as all audio-text pairs in the batch of the $T$-th step, where the text $t$ can be in any one of the languages. $\sum_{(a,t)}||p(a,t)-p'_e(a,t)||$ denotes the data distribution error in the batch at step $T$.

Detailed proof of Eq \eqref{Eq:weight error} can be found in Appendix \ref{Appe:Proof of Weight Error Upper Bound}. Based on Eq. \eqref{Eq:weight error}, we have the following results:

\begin{itemize}
    \item Intuitively, the weight error $||\mathbf w_{eT}-\mathbf w'_{eT}||$ comes from two main sources. One is the weight error after the $(e-1)$-th epoch, i.e. $||\mathbf w'_{(e-1)T}-\mathbf w_{(e-1)T}||$. The other is caused by the probabilistic distances of the data distributions, i.e. $\sum_{(a,t)}||p'_e(a,t)-p(a,t)||$. Since $a\geq 1$, the error from both sources increases with epoch and step. In addition, the weight error is also affected by the learning rate $\eta$, the number of training steps $T$ and the maximum gradient $g_{max}(\mathbf w_{eT-1-j})$.
    \item Further expansion of Eq. \eqref{Eq:weight error} shows that the weighting error arises from the data distribution error of each epoch. Expanding $||\mathbf w_{(e-1)T}-\mathbf w'_{(e-1)T}||$ in Eq. \eqref{Eq:weight error}, we find it consist of $||\mathbf w_{(e-2)T}-\mathbf w'_{(e-2)T}||$ and $||p(a,t)-p'_{e-1}(a,t)||$. Further expanding Eq. \eqref{Eq:weight error} to the weight error in $1$-th epoch, it can be concluded that the weight error of the existing ML-ATR scheme comes from the data distribution error $\sum^e_{i=1}\sum_{(a,t)}||p(a,t)-p'_i(a,t)||$ due to the randomly selected languages in each epoch. We can mitigate the inconsistency problem and improve the recall by reducing the weight error upper bound by reducing the data distribution error for each epoch.
\end{itemize}
\section{Proposed ML-ATR Scheme}
We propose two methods to reduce the data distribution error during training. One is 1-to-K contrastive learning, which has a higher memory overhead. The other is audio-English co-anchor contrastive learning, which achieves performance close to 1-to-K Contrastive Learning while approximating the memory overhead to the existing ML-ATR scheme. Here are the details of the two methods.

\subsection{1-to-K Contrastive Learning}
Building on our theoretical analyses, we conclude that reducing data distribution error is critical for addressing the inconsistency problem in multilingual audio-text retrieval. To achieve this, we propose 1-to-K Contrastive Learning (KCL), a training strategy that replaces random language sampling with the simultaneous use of all $K$ linguistic texts corresponding to each audio instance. This approach theoretically eliminates data distribution error, corrects modal alignment direction, and significantly enhances both the recall and consistency of retrieval performance. The loss function $\mathcal{L}^{at}_{kcl}$ for the proposed 1-to-K Contrastive Learning in ML-ATR is defined as follows:

\begin{equation}
\small
\begin{aligned}
\mathcal{L}_{kcl}=\frac{1}{2NK}(\mathcal{L}^{a2t}_{kcl}+\mathcal{L}^{t2a}_{kcl}).
\end{aligned}
\end{equation}

The loss function $\mathcal{L}^{at}_{kcl}$ consists of two parts, $\mathcal{L}^{a2t}_{kcl}$ and $\mathcal{L}^{t2a}_{kcl}$, and they are calculated as follows:

\begin{equation}
\small
\begin{aligned}
\mathcal{L}^{a2t}_{kcl}=-\sum^K_{k=1}\sum^N_{i=1}log\frac{\exp(s(f_\theta(a_i),g_\phi(t_{ik}))/\tau)}{\sum^N_{j=1}\exp(s(f_\theta(a_i),g_\phi(t_{jk}))/\tau)},
\end{aligned}
\end{equation}

$\mathcal{L}^{a2t}_{kcl}$ denotes the contrastive learning loss function from audio to multilingual text.

\begin{equation}
\small
\begin{aligned}
\mathcal{L}^{t2a}_{kcl}=-\sum^K_{k=1}\sum^N_{i=1}log\frac{\exp(s(g_\phi(t_{ik}),f_\theta(a_i))/\tau)}{\sum^N_{j=1}\exp(s(g_\phi(t_{ik}),f_\theta(a_j))/\tau)},
\end{aligned}
\end{equation}

$\mathcal{L}^{t2a}_{kcl}$ denotes the contrastive learning loss function from multilingual text to audio.


$K$ is the number of languages and $N$ is the number of data instances. As shown in Tab. \ref{Tab:overhead}, including multiple multilingual texts in 1-to-K contrastive learning increases GPU memory usage and training time. In practical ML-ATR applications, supporting more languages amplifies these overheads compared to existing schemes.

To address this, we further propose CACL, which improves retrieval consistency and recall without significantly increasing overhead.

\subsection{Audio-English Co-Anchor Contrastive Learning}
To reduce the weighting error with as little increase in training time and GPU memory consumption as possible, we propose audio-English co-anchor contrastive learning (CACL). During the training process, each data takes its audio, English text, and text in other random languages and does contrastive learning with each other. 

For each epoch, given a set of random numbers $Q=\{q_1,...q_N\},q_i\stackrel{R}{\leftarrow}\{2,...K\}$, get the triplet of the training data $(a_i,t_{i1},t_{iq_i})$, where $a_i$ denotes $i$-th audio, $t_{i1}$ denotes the English text, and $t_{iq_i}$ denotes the text of $q_i$-th language. We have the training loss $\mathcal{L}_{cacl}$ shown below:

\begin{equation}
\small
\begin{aligned}
\mathcal{L}_{cacl}=\frac{1}{6N}(\mathcal{L}^{ae}_{cacl}+\mathcal{L}^{at}_{cacl}+\mathcal{L}^{et}_{cacl}).
\end{aligned}
\end{equation}

The loss function $\mathcal{L}_{cacl}$ consists of three components $\mathcal{L}^{ae}_{cacl},\mathcal{L}^{at}_{cacl},\mathcal{L}^{et}_{cacl}$. All three components are based on the following general contrastive learning loss formulation:

\begin{equation}
\small
\begin{aligned}
\mathcal{L}^{uv}_{cacl}=&-\sum^N_{i=1}log\frac{\exp(s(u_i,v_i)/\tau)}{\sum^N_{j=1}\exp(s(u_i,v_j)/\tau)}\\
&-\sum^N_{i=1}log\frac{\exp(s(v_i,u_i)/\tau)}{\sum^N_{j=1}\exp(s(v_i,u_j)/\tau)},
\end{aligned}
\end{equation}
where $u_i$ and $v_i$ represent input embeddings from different modalities or languages. The three components are defined as follows:

\begin{itemize}
    \item \textbf{Audio-English Alignment} ($\mathcal{L}^{ae}_{cacl}$): 
    
    $u_i=f_\theta(a_i)$ represents audio embeddings, and $v_i=g_\phi(t_{i1})$ represents English text embeddings.
    \item \textbf{Audio-Multilingual Alignment} ($\mathcal{L}^{at}_{cacl}$): $u_i=f_\theta(a_i)$ represents audio embeddings, and $v_i=g_\phi(t_{iq_i})$ represents text embeddings in a randomly selected language.
    \item \textbf{English-Multilingual Alignment} ($\mathcal{L}^{et}_{cacl}$): $u_i=g_\phi(t_{i1})$ represents English text embeddings, and $v_i=g_\phi(t_{iq_i})$ represents text embeddings in a randomly selected language.
\end{itemize}

% \begin{equation}
% \small
% \begin{aligned}
% \mathcal{L}^{ae}_{cacl}=&-\sum^N_{i=1}log\frac{\exp(s(f_\theta(a_i),g_\phi(t_{i1}))/\tau)}{\sum^N_{j=1}\exp(s(f_\theta(a_i),g_\phi(t_{j1}))/\tau)}\\
% &-\sum^N_{i=1}log\frac{\exp(s(g_\phi(t_{i1}),f_\theta(a_i))/\tau)}{\sum^N_{j=1}\exp(s(g_\phi(t_{i1}),f_\theta(a_j))/\tau)},
% \end{aligned}
% \end{equation}

% $\mathcal{L}^{ae}_{cacl}$ denotes the contrastive learning loss for modality alignment between audios and English texts.

% \begin{equation}
% \small
% \begin{aligned}
% \mathcal{L}^{at}_{cacl}=&-\sum^N_{i=1}log\frac{\exp(s(f_\theta(a_i),g_\phi(t_{iq_i}))/\tau)}{\sum^N_{j=1}\exp(s(f_\theta(a_i),g_\phi(t_{jq_j}))/\tau)}\\
% &-\sum^N_{i=1}log\frac{\exp(s(g_\phi(t_{iq_i}),f_\theta(a_i))/\tau)}{\sum^N_{j=1}\exp(s(g_\phi(t_{iq_i}),f_\theta(a_j))/\tau)},
% \end{aligned}
% \end{equation}

% $\mathcal{L}^{at}_{cacl}$ denotes the contrastive learning loss for modal alignment between audios and texts in other randomly selected languages.

% \begin{equation}
% \small
% \begin{aligned}
% \mathcal{L}^{et}_{cacl}=&-\sum^N_{i=1}log\frac{\exp(s(g_\phi(t_{i1}),g_\phi(t_{iq_i}))/\tau)}{\sum^N_{j=1}\exp(s(g_\phi(t_{i1}),g_\phi(t_{jq_j}))/\tau)}\\
% &-\sum^N_{i=1}log\frac{\exp(s(g_\phi(t_{iq_i}),g_\phi(t_{i1}))/\tau)}{\sum^N_{j=1}\exp(s(g_\phi(t_{iq_i}),g_\phi(t_{j1}))/\tau)},
% \end{aligned}
% \end{equation}

% $\mathcal{L}^{et}_{cacl}$ denotes the contrastive learning loss for modal alignment between English texts and texts in other randomly selected languages.

The effectiveness of audio-English CACL can be explained from two perspectives:
\begin{itemize}
    \item From the perspective of modality alignment (Fig. \ref{Fig:modality alignment error}), the loss function $\mathcal{L}^{et}_{cacl}$ in CACL brings embeddings of English and other languages closer, reducing the distance between the text embedding $\vec{t_1},\vec{t_2}$ and the mean $\frac{1}{2}(\vec{t_1}+\vec{t_2})$ and minimizing the deviation in the modality alignment direction of audio and text.
    \item From the perspective of data distribution error $\sum_{(a,t)}||p(a,t)-p'_e(a,t)||$ in Eq. \eqref{Eq:weight error}, CACL's loss functions $\mathcal L^{ae}_{cacl}, \mathcal L^{at}_{cacl}$ ensures that the model learns more pairs of audio texts in an epoch. The text in them also contains a large percentage of high-quality English text. It makes the data distribution in CACL closer to the original one, and reduces the weight error of the model.
    %\item The English text in the ML-ATR dataset is manually labeled and less noisy than the text in other languages obtained via translation models. Letting the embedding space of other languages to align with both audio and English text can further mitigate the noise from translated text, improving cross-language audio-text alignment. Additionally, existing English-oriented ATR models already have good pre-trained weights, and CACL can help the alignment of audio with other languages using this pre-existing knowledge.
\end{itemize}

Note that in CACL, the number of texts used for training in each epoch does not increase with the number of languages, which effectively reduces both GPU memory and time overhead in ML-ATR scenarios with a large number of languages. Our experimental results illustrate that CACL approximates the training time and explicit memory overhead of existing ML-ATR schemes, yet achieves recall and consistency metrics close to those of 1-to-K comparative learning.

\section{Experiment}
\textbf{Datasets.} 
We assess the robustness of \ModelName\footnote{Our code is available at \url{https://github.com/RingBDStack/DiffSP}} in graph and node classification tasks. 
For graph classification, we use MUTAG~\cite{ivanov2019understanding}, IMDB-BINARY~\cite{ivanov2019understanding}, IMDB-MULTI~\cite{ivanov2019understanding}, REDDIT-BINARY~\cite{ivanov2019understanding}, and COLLAB~\cite{ivanov2019understanding}. For node classification, we test on Cora~\cite{yang2016revisiting}, CiteSeer~\cite{yang2016revisiting}, Polblogs~\cite{adamic2005political}, and Photo~\cite{shchur2018pitfalls}. Details are in Appendix~\ref{appendix:datasets}.

\noindent\textbf{Baselines.} 
Due to the limited research on robust GNNs targeting graph classification, we compare \ModelName\ with robust representation learning and structure learning methods designed for graph classification, including IDGL~\cite{chen2020iterative}, GraphCL~\cite{you2020graph}, VIB-GSL~\cite{sun2022graph}, G-Mixup~\cite{han2022g}, SEP~\cite{wu2022structural}, MGRL~\cite{ma2023multi}, SCGCN~\cite{zhao2024graph}, HSP-SL~\cite{zhang2019hierarchical}, SubGattPool~\cite{bandyopadhyay2020hierarchically} DIR~\cite{wu2022discovering}, and VGIB~\cite{yu2022improving}.
For node classification, we choose baselines from: 1) \textit{Structure Learning Based} methods, including GSR~\cite{zhao2023self}, GARNET~\cite{deng2022garnet}, and GUARD~\cite{li2023guard}; 2) \textit{Preprocessing Based} methods, including SVDGCN~\cite{entezari2020all} and JaccardGCN~\cite{wu2019adversarial}; 3) \textit{Robust Aggregation Based} methods, including RGCN~\cite{zhu2019robust}, Median-GCN~\cite{chen2021understanding}, GNNGuard~\cite{zhang2020gnnguard}, SoftMedian~\cite{geisler2021robustness}, and ElasticGCN~\cite{liu2021elastic}; and 4) \textit{Adversarial Training Based} methods, represented by the GraphADV~\cite{xu2019topology}.
Details of baselines can be found in Appendix~\ref{appendix:baselines}.
% #######

\noindent\textbf{Adversarial Attack Settings.}
For graph classification, we evaluate the performance against three strong evasion attacks: PR-BCD~\cite{geisler2021robustness}, GradArgmax~\cite{dai2018adversarial}, and CAMA-subgraph~\cite{wang2023revisiting}. 
For node classification, we evaluate six evasion attacks: 1) \textit{Targeted Attacks}: PR-BCD~\cite{geisler2021robustness}, Nettack~\cite{zugner2018adversarial}, and GR-BCD~\cite{geisler2021robustness}; 2) \textit{Non-targeted Attacks}: MinMax~\cite{li2020deeprobust}, DICE~\cite{zugner2018metalearningu}, and Random~\cite{li2020deeprobust}.
Further details on the attack methods and budget settings are provided in Appendix~\ref{appendix:attacks}.

\noindent\textbf{Hyperparameter Settings.} Details are provided in Appendix~\ref{appendix:implements}.
% \vspace{-0.2\baselineskip}

\begin{table*}[!tp]
  \captionsetup{skip=5pt}
  \centering
  \caption{Accuracy score (\% ± standard deviation) of \textit{node classification} task on real-world datasets against \textit{targeted attack}. 
  % The best results are shown in bold type and the runner-ups are \underline{underlined}.
  }
  \label{table:node_classification_targeted}
  \tabcolsep=0.1cm
  \resizebox{\linewidth}{!}{ 
\begin{tabular}{c|c|cccccccccccc|c}
    \toprule
    \textbf{Dataset} & \textbf{Attack} & GCN   & GSR   & GARNET & GUARD & SVD   & Jaccard & RGCN  & MedianGCN & GNNGuard & SoftMedian & ElasticGCN & GraphAT & \textbf{DiffSP} \\
    \midrule
    \multirow{4}[0]{*}{\textbf{Cora}} & PR-BCD & 55.59\scalebox{0.8}{±1.47} & \underline{74.75\scalebox{0.8}{±0.53}} & 66.80\scalebox{0.8}{±0.46} & 65.71\scalebox{0.8}{±0.79} & 64.66\scalebox{0.8}{±0.35} & 60.49\scalebox{0.8}{±1.00} & 55.91\scalebox{0.8}{±0.65} & 61.77\scalebox{0.8}{±0.68} & 65.14\scalebox{0.8}{±1.07} & 59.36\scalebox{0.8}{±0.63} & 63.86\scalebox{0.8}{±1.38} & 63.74\scalebox{0.8}{±0.99} & \textbf{75.13\scalebox{0.8}{±1.27}} \\
          & Nettack & 49.25\scalebox{0.8}{±5.28} & 67.25\scalebox{0.8}{±5.20} & 62.95\scalebox{0.8}{±4.75} & 52.50\scalebox{0.8}{±4.08} & 70.25\scalebox{0.8}{±0.79} & 56.75\scalebox{0.8}{±2.65} & 47.50\scalebox{0.8}{±1.67} & \underline{76.25\scalebox{0.8}{±5.17}} & 76.00\scalebox{0.8}{±5.03} & 67.50\scalebox{0.8}{±4.25} & 65.25\scalebox{0.8}{±3.22} & 73.50\scalebox{0.8}{±9.14} & \textbf{77.75\scalebox{0.8}{±3.62}} \\
          & GR-BCD & 66.34\scalebox{0.8}{±1.45} & \textbf{78.86\scalebox{0.8}{±0.53}} & 72.35\scalebox{0.8}{±0.91} & 72.08\scalebox{0.8}{±1.23} & 65.34\scalebox{0.8}{±0.72} & 71.88\scalebox{0.8}{±0.76} & 69.74\scalebox{0.8}{±2.08} & 72.90\scalebox{0.8}{±1.06} & 70.45\scalebox{0.8}{±1.20} & 75.52\scalebox{0.8}{±0.86} & 78.44\scalebox{0.8}{±1.42} & 77.06\scalebox{0.8}{±1.24} & \underline{76.83\scalebox{0.8}{±0.65}} \\
          & \cellcolor{gray!20}Average & \cellcolor{gray!20}{57.06} & \cellcolor{gray!20}\underline{73.62} & \cellcolor{gray!20}{67.37} & \cellcolor{gray!20}{63.43} & \cellcolor{gray!20}{66.75} & \cellcolor{gray!20}{63.04} & \cellcolor{gray!20}{57.72} & \cellcolor{gray!20}{70.31} & \cellcolor{gray!20}{70.53} & \cellcolor{gray!20}{67.46} & \cellcolor{gray!20}{69.18} & \cellcolor{gray!20}{71.43} & \cellcolor{gray!20}\textbf{76.57} \\
    \midrule
    \multirow{4}[0]{*}{\textbf{CiteSeer}} & PR-BCD & 45.06\scalebox{0.8}{±1.83} & \underline{63.33\scalebox{0.8}{±0.60}} & 55.75\scalebox{0.8}{±1.71} & 54.48\scalebox{0.8}{±0.96} & 59.61\scalebox{0.8}{±0.51} & 48.72\scalebox{0.8}{±1.20} & 41.08\scalebox{0.8}{±1.55} & 49.72\scalebox{0.8}{±0.71} & 49.78\scalebox{0.8}{±2.33} & 49.20\scalebox{0.8}{±0.89} & 48.79\scalebox{0.8}{±1.41} & 61.54\scalebox{0.8}{±1.01} & \textbf{64.35\scalebox{0.8}{±0.89}} \\
          & Nettack & 60.75\scalebox{0.8}{±8.34} & 75.25\scalebox{0.8}{±2.65} & 72.00\scalebox{0.8}{±2.84} & 59.25\scalebox{0.8}{±3.92} & \underline{77.25\scalebox{0.8}{±1.84}} & 71.50\scalebox{0.8}{±3.16} & 42.25\scalebox{0.8}{±4.78} & 74.00\scalebox{0.8}{±2.93} & 77.00\scalebox{0.8}{±3.50} & 59.00\scalebox{0.8}{±2.11} & 63.50\scalebox{0.8}{±3.76} & 73.25\scalebox{0.8}{±5.14} & \textbf{78.80\scalebox{0.8}{±4.53}} \\
          & GR-BCD & 50.56\scalebox{0.8}{±2.17} & \underline{65.50\scalebox{0.8}{±0.57}} & 57.04\scalebox{0.8}{±2.57} & 54.74\scalebox{0.8}{±1.82} & 60.40\scalebox{0.8}{±0.59} & 59.83\scalebox{0.8}{±1.17} & 44.82\scalebox{0.8}{±1.60} & 55.17\scalebox{0.8}{±1.31} & 58.88\scalebox{0.8}{±3.38} & 55.65\scalebox{0.8}{±0.93} & 60.37\scalebox{0.8}{±2.91} & 62.25\scalebox{0.8}{±1.25} & \textbf{65.63\scalebox{0.8}{±1.30}} \\
          & \cellcolor{gray!20}Average & \cellcolor{gray!20}{52.12} & \cellcolor{gray!20}\underline{68.02} & \cellcolor{gray!20}{61.60} & \cellcolor{gray!20}{56.16} & \cellcolor{gray!20}{65.75} & \cellcolor{gray!20}{60.02} & \cellcolor{gray!20}{42.72} & \cellcolor{gray!20}{59.63} & \cellcolor{gray!20}{61.89} & \cellcolor{gray!20}{54.62} & \cellcolor{gray!20}{57.55} & \cellcolor{gray!20}{65.68} & \cellcolor{gray!20}\textbf{69.59} \\
    \midrule
    \multirow{4}[0]{*}{\textbf{PolBlogs}} & PR-BCD & 73.73\scalebox{0.8}{±1.19} & 86.50\scalebox{0.8}{±0.52} & 75.52\scalebox{0.8}{±0.50} & 81.82\scalebox{0.8}{±1.06} & 78.02\scalebox{0.8}{±0.16} & 51.45\scalebox{0.8}{±1.23} & 74.01\scalebox{0.8}{±0.32} & 65.07\scalebox{0.8}{±4.21} & 51.93\scalebox{0.8}{±2.54} & \underline{87.88\scalebox{0.8}{±1.29}} & 74.71\scalebox{0.8}{±2.89} & 80.67\scalebox{0.8}{±0.85} & \textbf{90.24\scalebox{0.8}{±0.92}} \\
          & Nettack & 74.75\scalebox{0.8}{±4.92} & 75.75\scalebox{0.8}{±1.69} & 83.75\scalebox{0.8}{±3.77} & 76.75\scalebox{0.8}{±3.13} & 80.75\scalebox{0.8}{±1.69} & 47.75\scalebox{0.8}{±6.06} & 76.50\scalebox{0.8}{±1.75} & 46.00\scalebox{0.8}{±2.11} & 50.24\scalebox{0.8}{±6.52} & 83.50\scalebox{0.8}{±3.37} & \textbf{86.00\scalebox{0.8}{±4.12}} & 83.95\scalebox{0.8}{±2.72} & \underline{84.55\scalebox{0.8}{±5.90}} \\
          & GR-BCD & 71.31\scalebox{0.8}{±3.41} & 84.75\scalebox{0.8}{±0.66} & 75.49\scalebox{0.8}{±0.77} & 87.13\scalebox{0.8}{±3.63} & 90.27\scalebox{0.8}{±0.36} & 50.71\scalebox{0.8}{±1.98} & 79.13\scalebox{0.8}{±0.54} & 56.95\scalebox{0.8}{±5.15} & 51.26\scalebox{0.8}{±1.78} & 87.50\scalebox{0.8}{±0.81} & 91.12\scalebox{0.8}{±2.71} & \underline{92.70\scalebox{0.8}{±0.18}} & \textbf{92.75\scalebox{0.8}{±0.38}} \\
          & \cellcolor{gray!20}Average & \cellcolor{gray!20}{73.26} & \cellcolor{gray!20}{82.33} & \cellcolor{gray!20}{78.25} & \cellcolor{gray!20}{81.90} & \cellcolor{gray!20}{83.01} & \cellcolor{gray!20}{49.97} & \cellcolor{gray!20}{76.55} & \cellcolor{gray!20}{56.01} & \cellcolor{gray!20}{51.14} & \cellcolor{gray!20}\underline{86.29} & \cellcolor{gray!20}{83.94} & \cellcolor{gray!20}{85.77} & \cellcolor{gray!20}\textbf{89.18} \\
    \midrule
    \multirow{4}[0]{*}{\textbf{Photo}} & PR-BCD & 65.35\scalebox{0.8}{±2.48} & 73.81\scalebox{0.8}{±1.90} & 77.58\scalebox{0.8}{±1.93} & \underline{84.14\scalebox{0.8}{±3.75}} & 80.04\scalebox{0.8}{±1.13} & 66.13\scalebox{0.8}{±2.82} & 63.79\scalebox{0.8}{±11.99} & 79.75\scalebox{0.8}{±0.96} & 65.62\scalebox{0.8}{±2.63} & 76.84\scalebox{0.8}{±1.46} & 76.21\scalebox{0.8}{±1.89} & 78.72\scalebox{0.8}{±2.13} & \textbf{84.78\scalebox{0.8}{±1.82}} \\
          & Nettack & 83.70\scalebox{0.8}{±5.16} & 83.75\scalebox{0.8}{±4.12} & 88.00\scalebox{0.8}{±3.07} & 84.25\scalebox{0.8}{±2.65} & 82.75\scalebox{0.8}{±5.45} & 84.00\scalebox{0.8}{±5.43} & 75.50\scalebox{0.8}{±3.07} & 86.50\scalebox{0.8}{±3.16} & 87.50\scalebox{0.8}{±5.77} & \textbf{88.75\scalebox{0.8}{±1.32}} & 83.00\scalebox{0.8}{±3.29} & 87.25\scalebox{0.8}{±12.30} & \underline{87.75\scalebox{0.8}{±4.32}} \\
          & GR-BCD & 69.11\scalebox{0.8}{±7.85} & \underline{84.84\scalebox{0.8}{±2.29}} & 85.27\scalebox{0.8}{±1.57} & 82.15\scalebox{0.8}{±2.24} & 83.74\scalebox{0.8}{±1.11} & 76.24\scalebox{0.8}{±2.98} & 68.60\scalebox{0.8}{±7.28} & 84.23\scalebox{0.8}{±1.49} & 79.20\scalebox{0.8}{±1.80} & 79.69\scalebox{0.8}{±1.19} & 83.94\scalebox{0.8}{±0.95} & 87.49\scalebox{0.8}{±1.26} & \textbf{87.58\scalebox{0.8}{±0.58}} \\
          & \cellcolor{gray!20}Average & \cellcolor{gray!20}{72.72} & \cellcolor{gray!20}{80.80} & \cellcolor{gray!20}{83.62} & \cellcolor{gray!20}{83.51} & \cellcolor{gray!20}{82.18} & \cellcolor{gray!20}{75.46} & \cellcolor{gray!20}{69.30} & \cellcolor{gray!20}{83.49} & \cellcolor{gray!20}{77.44} & \cellcolor{gray!20}{81.76} & \cellcolor{gray!20}{81.05} & \cellcolor{gray!20}\underline{84.48} & \cellcolor{gray!20}\textbf{86.70} \\
    \bottomrule
\end{tabular}
}

\vspace{-0.5em}
\end{table*}


\begin{table*}[htbp]
  \centering
  \captionsetup{skip=5pt}
  \caption{Accuracy score (\% ± standard deviation) of \textit{node classification} task on real-world datasets against \textit{non-targeted attack}. 
  % The best results are shown in bold type and the runner-ups are \underline{underlined}.
  }
  \label{table:node_classification_non_targeted}
  \tabcolsep=0.1cm
  \resizebox{\linewidth}{!}{ 
   \begin{tabular}{c|c|cccccccccccc|c}
    \toprule
    \textbf{Dataset} & \textbf{Attack} & GCN   & GSR   & GARNET & GUARD & SVD   & Jaccard & RGCN  & MedianGCN & GNNGuard & SoftMedian & ElasticGCN & GraphAT & \textbf{DiffSP} \\
    \midrule
    \multirow{4}[0]{*}{\textbf{Cora}} & MinMax & 59.91\scalebox{0.8}{±2.60} & 67.80\scalebox{0.8}{±2.18} & 65.68\scalebox{0.8}{±0.58} & 61.62\scalebox{0.8}{±2.85} & 64.75\scalebox{0.8}{±0.96} & 64.43\scalebox{0.8}{±2.48} & 62.49\scalebox{0.8}{±2.19} & 56.35\scalebox{0.8}{±3.34} & 63.63\scalebox{0.8}{±2.40} & \underline{74.53\scalebox{0.8}{±0.70}} & 17.05\scalebox{0.8}{±5.33} & 63.35\scalebox{0.8}{±2.60} & \textbf{75.00\scalebox{0.8}{±1.12}} \\ 
          & DICE  & 69.58\scalebox{0.8}{±2.17} & 74.55\scalebox{0.8}{±0.74} & 68.88\scalebox{0.8}{±1.08} & 71.50\scalebox{0.8}{±2.68} & 59.52\scalebox{0.8}{±0.39} & 71.89\scalebox{0.8}{±0.56} & 69.92\scalebox{0.8}{±0.97} & 71.61\scalebox{0.8}{±0.72} & 68.82\scalebox{0.8}{±0.95} & 73.38\scalebox{0.8}{±0.68} & 74.11\scalebox{0.8}{±1.28} & \underline{75.84\scalebox{0.8}{±0.57}} & \textbf{75.96\scalebox{0.8}{±0.87}} \\ 
          & Random & 70.43\scalebox{0.8}{±2.22} & 77.37\scalebox{0.8}{±0.88} & 75.63\scalebox{0.8}{±0.93} & 74.96\scalebox{0.8}{±0.51} & 62.54\scalebox{0.8}{±0.65} & 73.74\scalebox{0.8}{±0.60} & 72.74\scalebox{0.8}{±1.00} & 74.31\scalebox{0.8}{±0.95} & 68.33\scalebox{0.8}{±1.72} & 77.52\scalebox{0.8}{±0.65} & 74.06\scalebox{0.8}{±3.87} & \underline{77.39\scalebox{0.8}{±0.91}} & \textbf{77.63\scalebox{0.8}{±0.80}} \\ 
          & \cellcolor{gray!20}Average & \cellcolor{gray!20}{66.64}  & \cellcolor{gray!20}\underline{73.24}  & \cellcolor{gray!20}{70.06}  & \cellcolor{gray!20}{69.36}  & \cellcolor{gray!20}{62.27}  & \cellcolor{gray!20}{70.02}  & \cellcolor{gray!20}{68.38}  & \cellcolor{gray!20}{67.42}  & \cellcolor{gray!20}{66.93}  & \cellcolor{gray!20}{75.14}  & \cellcolor{gray!20}{55.07}  & \cellcolor{gray!20}{72.19}  & \cellcolor{gray!20}\textbf{76.20} \\
    \midrule
    \multirow{4}[0]{*}{\textbf{CiteSeer}} & MinMax & 52.07\scalebox{0.8}{±6.63} & 54.74\scalebox{0.8}{±4.92} & 59.00\scalebox{0.8}{±2.35} & 58.02\scalebox{0.8}{±1.44} & 35.83\scalebox{0.8}{±1.89} & 56.65\scalebox{0.8}{±3.81} & 42.85\scalebox{0.8}{±7.72} & 53.39\scalebox{0.8}{±3.44} & 57.98\scalebox{0.8}{±2.97} & 60.84\scalebox{0.8}{±1.40} & 17.05\scalebox{0.8}{±5.33} & \underline{61.54\scalebox{0.8}{±3.70}} & \textbf{61.59\scalebox{0.8}{±1.10}} \\
          & DICE  & 57.46\scalebox{0.8}{±1.63} & 62.48\scalebox{0.8}{±1.08} & 55.59\scalebox{0.8}{±3.01} & 62.19\scalebox{0.8}{±0.99} & 57.33\scalebox{0.8}{±0.49} & 63.00\scalebox{0.8}{±0.87} & 50.88\scalebox{0.8}{±1.59} & 59.95\scalebox{0.8}{±0.97} & 58.85\scalebox{0.8}{±3.22} & 59.85\scalebox{0.8}{±0.81} & 60.30\scalebox{0.8}{±1.46} & \underline{65.28\scalebox{0.8}{±0.81}} & \textbf{65.43\scalebox{0.8}{±0.70}} \\
          & Random & 56.19\scalebox{0.8}{±3.08} & 64.01\scalebox{0.8}{±1.08} & 56.34\scalebox{0.8}{±3.70} & 62.47\scalebox{0.8}{±0.88} & 54.54\scalebox{0.8}{±0.62} & 64.20\scalebox{0.8}{±0.46} & 50.13\scalebox{0.8}{±1.95} & 60.60\scalebox{0.8}{±0.81} & 61.51\scalebox{0.8}{±3.32} & 58.66\scalebox{0.8}{±1.49} & 58.00\scalebox{0.8}{±3.61} & \underline{64.94\scalebox{0.8}{±1.12}} & \textbf{66.78\scalebox{0.8}{±0.54}} \\
          & \cellcolor{gray!20}Average & \cellcolor{gray!20}{55.24}  & \cellcolor{gray!20}{60.41}  & \cellcolor{gray!20}{56.98}  & \cellcolor{gray!20}{60.89}  & \cellcolor{gray!20}{49.23}  & \cellcolor{gray!20}{61.28}  & \cellcolor{gray!20}{47.95}  & \cellcolor{gray!20}{57.98}  & \cellcolor{gray!20}{59.45}  & \cellcolor{gray!20}{59.78}  & \cellcolor{gray!20}{45.12}  & \cellcolor{gray!20}\underline{63.92}  & \cellcolor{gray!20}\textbf{64.60} \\
    \midrule
    \multirow{4}[0]{*}{\textbf{PolBlogs}} & MinMax & 86.96\scalebox{0.8}{±0.43} & 88.56\scalebox{0.8}{±0.82} & 87.85\scalebox{0.8}{±0.19} & \underline{89.51\scalebox{0.8}{±0.85}} & 87.11\scalebox{0.8}{±0.32} & 51.01\scalebox{0.8}{±1.75} & 87.04\scalebox{0.8}{±0.19} & 87.95\scalebox{0.8}{±4.81} & 50.32\scalebox{0.8}{±1.19} & 88.76\scalebox{0.8}{±0.37} & 87.33\scalebox{0.8}{±0.62} & 88.32\scalebox{0.8}{±0.35} & \textbf{89.52\scalebox{0.8}{±3.08}} \\
          & DICE  & 76.52\scalebox{0.8}{±2.76} & 80.75\scalebox{0.8}{±4.72} & 85.05\scalebox{0.8}{±1.01} & 83.76\scalebox{0.8}{±0.78} & 82.84\scalebox{0.8}{±0.20} & 50.27\scalebox{0.8}{±1.91} & 81.50\scalebox{0.8}{±0.44} & 74.19\scalebox{0.8}{±3.02} & 50.79\scalebox{0.8}{±1.59} & 86.47\scalebox{0.8}{±0.45} & 82.40\scalebox{0.8}{±2.24} & \underline{87.39\scalebox{0.8}{±0.44}} & \textbf{88.85\scalebox{0.8}{±1.32}} \\
          & Random & 83.24\scalebox{0.8}{±5.81} & 87.81\scalebox{0.8}{±1.03} & 83.42\scalebox{0.8}{±1.59} & 87.48\scalebox{0.8}{±1.51} & 85.59\scalebox{0.8}{±0.31} & 51.02\scalebox{0.8}{±1.75} & 85.46\scalebox{0.8}{±0.40} & 83.57\scalebox{0.8}{±2.71} & 50.28\scalebox{0.8}{±1.13} & 90.35\scalebox{0.8}{±0.56} & 49.50\scalebox{0.8}{±2.20} & \underline{90.50\scalebox{0.8}{±0.56}} & \textbf{92.61\scalebox{0.8}{±0.93}} \\
          & \cellcolor{gray!20}Average & \cellcolor{gray!20}{82.24}  & \cellcolor{gray!20}{85.71}  & \cellcolor{gray!20}{85.44}  & \cellcolor{gray!20}{86.92}  & \cellcolor{gray!20}{85.18}  & \cellcolor{gray!20}{50.77}  & \cellcolor{gray!20}{84.67}  & \cellcolor{gray!20}{81.90}  & \cellcolor{gray!20}{50.46}  & \cellcolor{gray!20}{88.53}  & \cellcolor{gray!20}{73.08}  & \cellcolor{gray!20}\underline{88.74}  & \cellcolor{gray!20}\textbf{90.33} \\
    \midrule 
    \multirow{4}[0]{*}{\textbf{Photo}} & MinMax & 73.12\scalebox{0.8}{±3.17} & 76.36\scalebox{0.8}{±3.09} & 81.75\scalebox{0.8}{±1.91} & 75.89\scalebox{0.8}{±3.28} & 69.92\scalebox{0.8}{±5.50} & 74.20\scalebox{0.8}{±3.94} & \underline{87.04\scalebox{0.8}{±0.19}} & 67.43\scalebox{0.8}{±4.31} & 71.44\scalebox{0.8}{±6.66} & 85.23\scalebox{0.8}{±2.12} & 8.56\scalebox{0.8}{±3.24} & 81.70\scalebox{0.8}{±2.48} & \textbf{88.51\scalebox{0.8}{±0.61}} \\ 
          & DICE  & 84.60\scalebox{0.8}{±1.17} & 82.52\scalebox{0.8}{±1.66} & \underline{85.43\scalebox{0.8}{±0.92}} & 82.92\scalebox{0.8}{±1.27} & 76.42\scalebox{0.8}{±1.39} & 83.20\scalebox{0.8}{±1.44} & 81.57\scalebox{0.8}{±0.44} & 82.83\scalebox{0.8}{±2.45} & 83.87\scalebox{0.8}{±1.19} & 84.72\scalebox{0.8}{±0.90} & 81.86\scalebox{0.8}{±3.61} & \textbf{87.22\scalebox{0.8}{±1.13}} & 83.52\scalebox{0.8}{±1.19} \\ 
          & Random & 85.38\scalebox{0.8}{±1.76} & 83.62\scalebox{0.8}{±2.91} & 84.12\scalebox{0.8}{±3.95} & 85.49\scalebox{0.8}{±1.55} & 79.13\scalebox{0.8}{±2.84} & 83.37\scalebox{0.8}{±1.93} & \textbf{86.87\scalebox{0.8}{±2.89}} & 84.07\scalebox{0.8}{±2.52} & 83.24\scalebox{0.8}{±4.83} & 85.95\scalebox{0.8}{±1.06} & 75.32\scalebox{0.8}{±2.38} & \underline{86.23\scalebox{0.8}{±2.26}} & 84.60\scalebox{0.8}{±0.46} \\ 
          & \cellcolor{gray!20}Average & \cellcolor{gray!20}{81.03}  & \cellcolor{gray!20}{80.83}  & \cellcolor{gray!20}{83.77}  & \cellcolor{gray!20}{81.43}  & \cellcolor{gray!20}{75.16}  & \cellcolor{gray!20}{80.26}  & \cellcolor{gray!20}{85.16}  & \cellcolor{gray!20}{78.11}  & \cellcolor{gray!20}{79.52}  & \cellcolor{gray!20}\underline{85.30}  & \cellcolor{gray!20}{55.25}  & \cellcolor{gray!20}{85.05}  & \cellcolor{gray!20}\textbf{85.54} \\
    \bottomrule 
    \end{tabular}
}

% \vspace{-0.3em}
\end{table*}



\subsection{Adversarial Robustness}
\textbf{Graph Classification Robustness.}
We evaluated the robustness of the graph classification task under three adversarial attacks across five datasets. Since the choice of classifier affects attack effectiveness, especially in graph classification due to pooling operations, it is crucial to standardize the model architecture. Simple changes like adding a linear layer can reduce the impact of attacks. To ensure a fair comparison, we used a two-layer GCN with a linear layer and mean pooling for both the baselines and our proposed \ModelName. Each experiment was repeated 10 times, with results shown in Table~\ref{table:graph_classification}.

\textit{Result.} 1) \ModelName~ consistently outperforms all baselines under the PR-BCD attack and achieves the highest average robustness across all attacks on five datasets, with a notable 4.80\% average improvement on the IMDB-BINARY dataset.
2) It's important to note that while baselines may excel against specific attacks, they often struggle with others. In contrast, \ModelName\ maintains consistent robustness across both datasets and attacks, thanks to its ability to learn clean distributions and purify adversarial graphs without relying on specific priors about the dataset or attack strategies.


\begin{figure*}[!t]
    \begin{minipage}{0.33\textwidth}
        \centering
        \includegraphics[width=\textwidth]{figure/ablation.pdf}
    \vspace{-2.5em}
        
        \caption{Ablation Study}
        \label{fig_ablation}
    \end{minipage} 
    \begin{minipage}{0.33\textwidth}
    \centering
    \includegraphics[width=\textwidth]{figure/time_influence.pdf}
    \vspace{-2.5em}
    
    \caption{Purification Steps Study}
    \label{fig_diffusion_steps}
\end{minipage}
\begin{minipage}{0.33\textwidth}
    \centering
    \includegraphics[width=\textwidth]{figure/tg_influence.pdf}
    \vspace{-2.5em}
    
    \caption{Guide Scale Study}
    \label{fig_graph_transfer_entropy}
\end{minipage}\\
\begin{minipage}{1.0\textwidth}
    \centering
    \includegraphics[width=\textwidth]{figure/visualize.pdf}
    \vspace{-2.5em}
    \caption{Visualization Study}
    \label{fig_visualization}
    \vspace{-1em}
\end{minipage}
\end{figure*}

\noindent\textbf{Node Classification Robustness.}
We evaluate the robustness of \ModelName\ on the node classification task against six attacks across four datasets, using the same other settings as in the graph classification experiments. The results are presented in Table \ref{table:node_classification_targeted} and Table \ref{table:node_classification_non_targeted}.

\textit{Result.} We have two key observations: 
1) \ModelName\ achieves the best average performance across both targeted and non-targeted attacks on all datasets, demonstrating its robust adaptability across diverse scenarios. 
2) \ModelName\ performs particularly well under stronger attacks but is less effective against weaker ones like Random and DICE. This is because these attacks introduce numerous noisy edges, many of which do not exhibit distinctly adversarial characteristics. Instead, these edges are often plausible within the graph. Consequently, these additional perturbations can mislead \ModelName, making it harder to discern the correct information within the graph, leading the generated graph to deviate from the target clean graph.

% \subsection{Model Analyses}

\vspace{-0.5em}

\subsection{Ablation Study}
In this subsection, we analyze the effectiveness of \ModelName's two core components: 1) \ModelName~ (w/o LN), which excludes the LID-Driven Non-Isotropic Diffusion Mechanism; and 2) \ModelName\ (w/o TG), which excludes the Graph Transfer Entropy Guided Denoising Mechanism. We evaluate variants on the IMDB-BINARY and COLLAB datasets under PR-BCD and GradArgmax attacks for graph classification and on the Cora and CiteSeer dataset under PR-BCD and MinMax attacks for node classification. Results are shown in Figure~\ref{fig_ablation}.

\textit{Result.} \ModelName~ consistently outperforms the other variants. \ModelName\ (w/o LN) over-perturbs the valuable parts of the graph leading to degraded performance. Similarly, \ModelName~ (w/o TG) increases the uncertainty of generation, causing deviations from the target clean graph. These reduce the robustness against evasion attacks.

\vspace{-1em}
\subsection{Study on Cross-Dataset Generalization}
We assess \ModelName's generalization ability. The goal is to determine whether \ModelName~ effectively learns the predictive patterns of clean graphs. We train \ModelName\ on IMDB-BINARY and use the trained model to purify graphs on IMDB-MULTI, and vice versa.

\textit{Result.} As shown in Table~\ref{table:transfer}, \ModelName~ trained on different datasets, still demonstrates strong robustness compared to GCN trained and tested on the same dataset. Furthermore, \ModelName~ exhibits only a small performance gap compared to when it is trained and tested on the same dataset directly.
These results highlight \ModelName's ability to learn the underlying clean distribution of a category of data and capture predictive patterns that generalize across diverse datasets.

\begin{figure*}[!ht]
    \centering
    \includegraphics[width=.95\linewidth]{figures-src/transfer.pdf}
    \caption{SliderSpace directions for the ``person'' concept successfully generalize to related ``police'' and ``athlete'' concepts. They also transfer to out-of-domain concepts like ``dog''}
    \label{fig:transfer}
\end{figure*}



\subsection{Study on Purification Steps}
We evaluate the performance as the number of diffusion steps varies. For graph classification on the IMDB-BINARY dataset, we adjust the diffusion steps from 1 to 9 under GradArgMax, PR-BCD, and CAMA-Subgraph attacks. For node classification on the Cora dataset, we vary the diffusion steps from 1 to 12 under the GR-BCD, PR-BCD, and MinMax attacks.
The results are shown in Figure~\ref{fig_diffusion_steps}.


\textit{Result.}
We observe that all-time step settings demonstrate the ability to effectively purify adversarial graphs. At smaller time steps, the overall trend shows increasing accuracy as the number of diffusion steps increases. This is likely because fewer time steps do not introduce enough noise to sufficiently suppress the adversarial information in the graph. As the diffusion steps increase, we do not see a significant decline in performance. This stability can be attributed to our LID-Driven Non-Isotropic Diffusion Mechanism, which minimizes over-perturbation of the clean graph parts. Additionally, we found that the time required for purifying increased linearly.




\subsection{Study on Scale of Graph Transfer Entropy}
To analyze the impact of the guidance scale $\lambda$, we vary $\lambda$ from $\text{1e}^{-1}$ to $\text{1e}^{5}$. The results are presented in Figure~\ref{fig_graph_transfer_entropy}. For graph classification, experiments are conducted on the IMDB-BINARY dataset under the GradArgmax, PR-BCD, and CAMA-Subgraph attacks. For node classification, experiments are performed on the Cora dataset under the PR-BCD, GR-BCD, and MinMax attacks.

\textit{Result.} The results show that smaller values of $\lambda$ have minimal effect on accuracy. However, they reduce the stability of the purification during the reverse denoising process, leading to a higher standard deviation. This instability arises because the model is less effective at reducing uncertainty and guiding the generation process when $\lambda$ is too small.  On the other hand, large $\lambda$ values decrease accuracy by overemphasizing guidance, causing the model to reintroduce adversarial information into the generated graph.



\subsection{Graph Purification Visualization}
We visualize snapshots of different purification time steps on the IMDB-BINARY dataset using NetworkX~\cite{hagberg2008exploring}, as shown in Figure~\ref{fig_visualization}. The visualization process demonstrates that \ModelName\ has mastered the ability to generate clean graphs, achieving graph purification.

% More experiments and analyses are provided in Appendix~\ref{appendix:additional_analysis}.

\vspace{-5pt}
\section{Discussion}
\textbf{Conclusion.}
In this work, we propose the \textit{\methodname{}} metric, $M_{AP}$, to evaluate preference data quality in alignment.
By measuring the gap from the model's current implicit reward margin to the target explicit reward margin, $M_{AP}$ quantifies the discrepancy between the current model and the aligned optimum, thereby indicating the potential for alignment enhancement.
Extensive experiments validate the efficacy of $M_{AP}$ across various training settings under offline and self-play preference learning scenarios.

\textbf{Limitations and future work}. 
Despite the performance improvements, $M_{AP}$ requires tuning a parameter $\beta$ to combine the explicit and implicit margins; future work could explore how to set this ratio automatically.
Additionally, while our experiments focus on the widely applied DPO and SimPO objectives, a broader investigation with alternative preference learning methods is crucial in future works.

% \section{Conclusion}
% In this paper, we introduce the \methodname{} metric to evaluate preference data quality in LLM alignment.
% By measuring the discrepancy between the model's current implicit reward margin to the target explicit reward margin, this metric quantifies the gap between the current model and the aligned optimum, thereby indicating the potential for alignment enhancement.
% Empirical results demonstrate that training on data selected by our metric consistently improves alignment performance, outperforming existing metrics across different base models and training objectives.
% Moreover, this metric extends to data generation scenarios (\ie self-play alignment): by identifying high-quality data from the intrinsic self-generated context, our metric yields superior results across various training settings, providing a comprehensive solution for enhancing LLM alignment through optimized
% preference data generation, selection, and utilization.


\section*{Impact Statement}
This paper presents work whose goal is to advance the field of Machine Learning. There are many potential societal consequences of our work, none which we feel must be specifically highlighted here.


\bibliographystyle{acl_natbib}
\bibliography{reference}
% \bibliographystyle{acl_natbib}
% \bibliography{custom}

\appendix

\newpage
\appendix
% \onecolumn

% \section{Pseudo-code for \Ours}

\section{LLM inference strategy and IR pipelines}

\begin{table}[h]
\caption{Correspondence between LLM inference and IR pipelines.}
  \label{tb:llm-retriever-reranker}
  \centering
%   \small
  \scalebox{0.8}{\begin{tabular}{lccc}
    \toprule
    Method & Retriever & Reranker & Pipeline       \\
    \midrule
    Greedy decoding     & LLM &  $\emptyset$ & Retriever-only  \\
    \midrule
    Best-of-N \citep{stiennon2020learning} & LLM & Reward model & Retriever-reranker  \\
    \midrule
    Majority voting  \citep{wang2022self}  & LLM & Majority & Retriever-reranker  \\
    \midrule
    Iterative refinement \citep{madaan2024self} & LLM & $\emptyset$ & Iterative retrieval  w. query rewriting \\
    \bottomrule
  \end{tabular}}
\end{table}


\section{How can SFT and preference optimization help the LLM from an IR perspective?}\label{apx:sft-rlhf-empirical}


We assess how well LLMs perform at two tasks: fine-grained reranking (using greedy decoding accuracy) and coarse-grained retrieval (using Recall@$N$).  
We focus on how SFT and DPO, affect these abilities.  
Using the Mistral-7b model, we evaluate on the GSM8k and MATH datasets with two approaches: SFT-only, and SFT followed by DPO (SFT $\rightarrow$ DPO).

In the SFT phase, the model is trained directly on correct answers. 
For DPO, we generate 20 responses per prompt and created preference pairs by randomly selecting one correct and one incorrect response.  
We use hyperparameter tuning and early stopping to find the best model checkpoints (see Appendix \ref{apx:sec:sft-rlhf} for details).


\begin{table}[h]
\caption{Retrieval (Recall@N) and reranking (greedy accuracy) metrics across dataset and training strategies, with Mistral-7b as the LLM. 0.7 is used as the temperature. Recall@N can also be denoted as pass@N.}\label{tb:sft-rlhf-result}
\vskip 1em
\centering
\small
\begin{tabular}{llcccc}
    \toprule
     & Metric & \textbf{init model} & \textbf{SFT} & \textbf{SFT $\rightarrow$ DPO} \\
    \midrule
    \multirow{4}{*}{\rotatebox{90}{GSM8K}} 
    & Greedy Acc & 0.4663 & 0.7680 & 0.7991  \\
    & Recall@20 & 0.8347 & 0.9462 & 0.9545  \\
    & Recall@50 & 0.9090 & 0.9629 & 0.9727  \\
    & Recall@100 & 0.9477 & 0.9735 & 0.9826   \\
    \midrule
    \multirow{4}{*}{\rotatebox{90}{Math}} 
    & Greedy Acc & 0.1004 & 0.2334 & 0.2502 \\
    & Recall@20 & 0.2600 & 0.5340 & 0.5416  \\
    & Recall@50 & 0.3354 & 0.6190 & 0.6258  \\
    & Recall@100 & 0.4036 & 0.6780 & 0.6846  \\
    \bottomrule
\end{tabular}
\end{table}

The results are shown in Table \ref{tb:sft-rlhf-result}.  
We observe that both SFT and DPO improve both retrieval and reranking, with SFT being more effective. Adding DPO after SFT further improves performance on both tasks.  
This is consistent with information retrieval principles that both direct retriever optimization and reranker-retrieval distillation can enhance the retriever performance, while the latter on top of the former can further improve the performance. Further discussions can be found in Appendices \ref{apx:discuss1} and \ref{apx:discuss2}.


\section{Discussion on the connection and difference between SFT and direct retriever optimization}\label{apx:discuss1}

As discussed in Section \ref{sec:llm-tuning-retriever}, the direct retriever optimization goal with InfoNCE is shown as:
\begin{gather*}
    \max \log P(d_{\text{gold}}|q) = \max \log \frac{\text{Enc}_d(d_{\text{gold}}) \cdot\text{Enc}_q(q)}{\sum^{|C|}_{j=1} \text{Enc}_d(d_j) \cdot\text{Enc}_q(q)},
\end{gather*}
while the SFT optimization goal is shown as:
\begin{gather}
    \max \log P(y_{\text{gold}}|x) = \max \log \prod^{|y_{\text{gold}}|}_i P(y_{\text{gold}}(i)|z_i) 
    = \max \sum^{|y_{\text{gold}}|}_i \log \frac{\text{Emb}(y_{\text{gold}}(i)) \cdot\text{LLM}(z_i)}{\sum^{|V|}_{j=1} \text{Emb}(v_j) \cdot\text{LLM}(z_i)}. \label{apx:eq:sft}
\end{gather}

As a result, the SFT objective can be seen as a summation of multiple retrieval optimization objectives, where $\text{LLM}(\cdot)$ and word embedding $\text{Emb}(\cdot)$ are query encoder and passage encoder respectively.

However, for direct retriever optimization with InfoNCE, $\text{Enc}_d(\cdot)$ is usually a large-scale pretrained language model which is computationally expensive on both time and memory.
In this case, it is unrealistic to calculate the $\text{Enc}_d(d_j)$ for all $d_j\in C$, when $C$ is large, because of the time constrain and GPU memory constrain.
As a result, a widely-adopted technique is to adopt ``in-batch negatives'' with ``hard negatives'' to estimate the $\log P(d_{\text{gold}}|q)$ function:
\begin{gather*}
    \max \log P(d_{\text{gold}}|q) = \max \log \frac{\text{Enc}_d(d_{\text{gold}}) \cdot\text{Enc}_q(q)}{\sum^{|C|}_{j=1} \text{Enc}_d(d_j) \cdot\text{Enc}_q(q)} \\
    \sim \max \log \frac{\text{Enc}_d(d_{\text{gold}}) \cdot\text{Enc}_q(q)}{\sum^{|B|}_{i=1} \text{Enc}_d(d_i) \cdot\text{Enc}_q(q) + \sum^{|H|}_{j=1} \text{Enc}_d(d_j) \cdot\text{Enc}_q(q)},
\end{gather*}
where $B$ is the in-batch negative set and $H$ is the hard negative set.
Note that $B\bigcup H \subset C$.
This objective is more efficient to optimize but is not the original optimization goal. As a result, the learned model after direct retriever optimization is not optimal.
It is also found that the hard negatives $H$ is the key to estimate the original optimization goal \citep{zhan2021optimizing}.
Thus, reranker-retriever distillation can further improve the retriever by introducing more hard negatives.

On the other hand, LLM optimization, as shown in Eq. (\ref{apx:eq:sft}), can be seen as a summation of multiple retrieval optimization function.
In each retrieval step, the passage can be seen as a token and the corpus is the vocabulary space $V$.
Given that the passage encoder $\text{Emb}(\cdot)$ (word embedding) here is cheap to compute and the vocabulary space $V$ ($<$100k) is usually not as large as $C$ ($>$1M) in IR, the objective in Eq. (\ref{apx:eq:sft}) can be directly optimized without any estimation.
In this case, the LLM as a retriever is more sufficiently trained compared with the retriever training in IR.


\section{Discussion on the connection and difference between preference optimization and reranker-retriever distillation}\label{apx:discuss2}

As discussed in Section \ref{sec:llm-tuning-retriever}, preference optimization with an online reward model $f_{\text{reward-model}}(\cdot) \overset{r}{\rightarrow} \text{data} \overset{g(\cdot)}{\rightarrow}  f_{\text{LLM}}(\cdot)$ can be seen as a reranker to retriever distillation process $f_{\text{rerank}}(\cdot) \overset{r}{\rightarrow} \text{data}\overset{g(\cdot)}{\rightarrow}   f_{\text{retrieval}}(\cdot)$, where the reward model is the reranker (\textit{i.e.}, cross-encoder) and the LLM is the retriever (\textit{i.e.}, bi-encoder).

However, there are two slight differences here:
\begin{itemize}[leftmargin=*]
\item The LLM after SFT is more sufficiently trained compared to a retriever after direct optimization. As discussed in Appendix \ref{apx:discuss1}, the SFT optimization function is not an estimated retriever optimization goal compared with the direct retrieval optimization. As a result, the LLM after SFT is suffienctly trained. In this case, if the reward model (reranker) cannot provide information other than that already in the SFT set (\textit{e.g.}, using the SFT prompts), this step may not contribute to significant LLM capability improvement.
\item The reward model may introduce auxiliary information than the reranker in IR. For a reranker in IR, it captures a same semantic with the retriever: semantic similarity between the query and the passage. However, in LLM post-training, the goal and data in SFT and preference optimization can be different. For example, the SFT phase could have query/response pairs which enable basic chat-based retrieval capability for the LLM. While the reward model may contain some style preference information or safety information which do not exist in SFT data. In this case, the preference optimization which is the reranker to retriever distillation step could also contribution to performance improvement.
\end{itemize}


\section{Evaluate LLMs as retrievers}\label{apx:llm-as-retriever}

In addition to Mathstral-7b-it on GSM8K in Figure \ref{fig:mathstral-gsm8k-infer}, we conduct extensive experiments to both Mistral-7b-it and Mathstral-7b-it on GSM8K and MATH. The results are shown in Figure \ref{apx:fig:empirical-llm-retriever}.
We have similar findings as in Figure \ref{fig:mathstral-gsm8k-infer} that:
(1) As $N$ increases, Recall@$N$ improves significantly, indicating that retrieving a larger number of documents increases the likelihood of including a correct one within the set.
(2) For smaller values of $N$ (e.g., $N=1$), lower temperatures yield higher Recall@$N$. This is because lower temperatures reduce response randomness, favoring the selection of the most relevant result.
(3) Conversely, for larger $N$ (e.g., $N>10$), higher temperatures enhance Recall@$N$. Increased temperature promotes greater response diversity, which, when combined with a larger retrieval set, improves the chances of capturing the correct answer within the results.

\begin{figure*}[h]
    \centering
    \subfigure[Mistral-7b-it on GSM8k]{\includegraphics[width=0.45\textwidth]{figure/LLM_alignment_gsm8k_mathstral7b_infer.pdf}}
    \subfigure[Mistral-7b-it on GSM8k]{\includegraphics[width=0.45\textwidth]{figure/LLM_alignment_gsm8k_mistral7b_infer.pdf}}
    \subfigure[Mathstral-7b-it on MATH]{\includegraphics[width=0.45\textwidth]{figure/LLM_alignment_math_mathstral7b_infer.pdf}} 
    \subfigure[Mistral-7b-it on MATH]{\includegraphics[width=0.45\textwidth]{figure/LLM_alignment_math_mistral7b_infer.pdf}}
    % \vspace{-0.1in}
    \vskip -1em
    \caption{Evaluate the LLM as a retriever with Recall@N (Pass@N). As the number (N) of retrieved responses increases, the retrieval recall increases. The higher the temperature is, the broader spectrum the retrieved responses are, and thus the higher the recall is.}\label{apx:fig:empirical-llm-retriever}
\end{figure*}


% \subsection{How SFT and RLHF benefit the LLM retriever?}\label{apx:sft-rlhf}

% In addition to the experiments with Gemma-1-7b-it in Table \ref{tb:sft-rlhf-result}, we also conduct experiments to study the effect of SFT and DPO on Deepseek-math-7b-base model \citep{shao2024deepseekmath}.
% The results on MATH dataset are shown in Table \ref{apx:tb:sft-rlhf-result}, where we have similar discovery with that in Table \ref{tb:sft-rlhf-result}:
% (1) Both SFT and DPO can improve the retrieval capability of the LLM, while SFT is more effective.
% (2) On top of SFT, DPO can slightly improve the reranking capability (greedy accuracy) but not the general retrieval capability.

% \begin{table}[h]
% \caption{Retrieval (Recall@N) and reranking (greedy accuracy) metrics across dataset and training strategies. LLM: Deepseek-math-7b. Temperature: 0.7. Recall@N can also be denoted as pass@N.}\label{apx:tb:sft-rlhf-result}
% \vskip 1em
% \centering
% \scalebox{0.8}{
% \begin{tabular}{llcccc}
%     \toprule
%      & Metric & \textbf{init model} & \textbf{DPO} & \textbf{SFT} & \textbf{SFT $\rightarrow$ DPO} \\
%     \midrule
%     \multirow{4}{*}{\rotatebox{90}{Math}} 
%     & Greedy Acc & 0.0972 & 0.1164 & 0.3078 & 0.312 \\
%     & Recall@20 & 0.4914 & 0.5136 & 0.6524 & 0.6558 \\
%     & Recall@50 & 0.6058 & 0.6278 & 0.7332 & 0.736 \\
%     & Recall@100 & 0.6728 & 0.6976 & 0.7844 & 0.7828 \\
%     \bottomrule
% \end{tabular}}
% \end{table}



\section{\Ours retriever optimization objective}\label{apx:proofs}

We provide the proof for different variants of \Ours's objective functions.

\subsection{Contrastive ranking}\label{apx:proof:contrastive}

\begin{theorem}
Let \( x \) be a prompt and \( (y_w, y^{(1)}_l, ..., y^{(m)}_l)  \) be the responses for \( x \) under the contrastive assumption (Eq.(\ref{eq:contrastive-assumption})).
Then the objective function to learn the LLM \( \pi_\theta \):
\end{theorem}

\begin{equation}
    \begin{aligned}
    \mathcal{L}_{\text{con}} = -\mathbb{E} & \biggl[
    \log \frac{\exp\bigl(\gamma(y_w \mid x)\bigr)}{
        \exp\bigl(\gamma(y_w \mid x)\bigr) + \sum_{i=1}^m \exp\bigl(\gamma(y_l^{(i)} \mid x)\bigr)}
    \biggr], \\
    \text{where } &\quad \gamma(y \mid x) = \beta \log \frac{\pi_\theta(y \mid x)}{\pi_{\mathrm{ref}}(y \mid x)}.
\end{aligned}\label{eq:contrastive}
\end{equation}

\textit{Proof.}
From \citep{rafailov2024direct}, we know that
\begin{gather}
    r(x, y) = \beta \text{log} \frac{\pi_{\text{llm}}(y|x)}{\pi_{\text{ref}}(y|x)} + \beta \text{log} Z,
\end{gather}
where $Z = \sum_{y'} \pi_{\text{ref}}(y'|x) \text{exp}(\frac{1}{\beta} r(x, y'))$.

Then,
\begin{equation}\label{eq:1-n}
\begin{aligned}
\mathbb{P}\text{r}(y_w & \succeq y^{(1)}_l, ..., y_w \succeq y^{(m)}_l) 
= \text{softmax}(r(x, y_w)) \\
&= \frac{\text{exp}(r(x,y_w))}{\text{exp}(r(x,y_w)) + \sum^m_{i=1}\text{exp}(r(x,y^{(i)}_l))} \\
&= \frac{1}{1 + \sum^m_{i=1}\text{exp}(r(x,y^{(i)}_l)-r(x,y_w))} \\
&= \frac{1}{1 + \sum^m_{i=1}\text{exp}(\gamma(y^{(i)}_l \mid x) + \beta \text{log} Z - \gamma(y_w \mid x) - \beta \text{log} Z)} \\
&= \frac{1}{1 + \sum^m_{i=1}\text{exp}(\gamma(y^{(i)}_l \mid x) - \gamma(y_w \mid x))} \\
&= \frac{\exp\bigl(\gamma(y_w \mid x)\bigr)}{
        \exp\bigl(\gamma(y_w \mid x)\bigr) + \sum_{i=1}^m \exp\bigl(\gamma(y_l^{(i)} \mid x)\bigr)}
\end{aligned}
\end{equation}

We can learn $\pi_\theta$ by maximizing the logarithm-likelihood: 
\begin{gather}
\max \log \mathbb{P}\text{r}(y_w \succeq y^{(1)}_l, \dots, y_w \succeq y^{(m)}_l) \Leftrightarrow 
\min - \log \mathbb{P}\text{r}(y_w \succeq y^{(1)}_l, \dots, y_w \succeq y^{(m)}_l) = \mathcal{L}, \\
 \therefore \mathcal{L}_{\text{con}} = -\mathbb{E} \biggl[
    \log \frac{\exp\bigl(\gamma(y_w \mid x)\bigr)}{
        \exp\bigl(\gamma(y_w \mid x)\bigr) + \sum_{i=1}^m \exp\bigl(\gamma(y_l^{(i)} \mid x)\bigr)}
    \biggr], \\
\text{where} \quad \gamma(y \mid x) = \beta \log \frac{\pi_\theta(y \mid x)}{\pi_{\mathrm{ref}}(y \mid x)}.
\end{gather}



\subsection{LambdaRank ranking}\label{apx:proof:lambdarank}

\begin{theorem}
Let \( x \) be a prompt and \( (y_1, ..., y_m)  \) be the responses for \( x \) under the LambdaRank assumption (Eq.(\ref{eq:lambdarank-assumption})).
Then the objective function to learn the LLM \( \pi_\theta \):
\end{theorem}

% \begin{gather}
%     \mathcal{L}_{\text{lamb}}=-\mathbb{E}\;\biggl[ \sum_{1<i<j<m}
%   w_{ij}\log \sigma\Bigl(
%      \gamma(y_i \mid x)-
%      \gamma(y_j \mid x)
%   \Bigr)
% \biggr]
% \end{gather}
\begin{gather}
    \mathcal{L}_{\text{lamb}}=-\mathbb{E}\;\biggl[ \sum_{1<i<j<m}
   \log \sigma\Bigl(
     \gamma(y_i \mid x)-
     \gamma(y_j \mid x)
   \Bigr)
\biggr].
\end{gather}
% where $w_{ij}$ is an adjustable weight.

\textit{Proof.}
\begin{equation}
\begin{aligned}
\mathbb{P}\text{r}(y_1 & \succeq ... \succeq y_m)
= \prod_{1<i<j<m} \sigma(r(x,y_i) - r(x,y_j)) \\
&= \prod_{1<i<j<m} \sigma(\gamma(x,y_i) + \beta \text{log} Z - \gamma(x,y_j) - \beta \text{log} Z)  \\
&= \prod_{1<i<j<m} \sigma(\gamma(y_i \mid x)-
     \gamma(y_j \mid x)).
\end{aligned}
\end{equation}

We can learn $\pi_\theta$ by maximizing the logarithm-likelihood: 
\begin{gather}
\max \log \mathbb{P}\text{r}(y_w \succeq y^{(1)}_l, \dots, y_w \succeq y^{(m)}_l) \Leftrightarrow 
\min - \log \mathbb{P}\text{r}(y_w \succeq y^{(1)}_l, \dots, y_w \succeq y^{(m)}_l) = \mathcal{L}, \\
 \therefore \mathcal{L}_{\text{lamb}}=-\mathbb{E}\;\biggl[ \sum_{1<i<j<m}
   \log \sigma\Bigl(
     \gamma(y_i \mid x)-
     \gamma(y_j \mid x)
   \Bigr)
\biggr], \\
\text{where} \quad \gamma(y \mid x) = \beta \log \frac{\pi_\theta(y \mid x)}{\pi_{\mathrm{ref}}(y \mid x)}.
\end{gather}
% $w_{ij}$ can be added to control the weight of each pair in the candidate list.


\subsection{ListMLE ranking}\label{apx:proof:listmle}

\begin{theorem}
Let \( x \) be a prompt and \( (y_1, ..., y_m)  \) be the responses for \( x \) under the ListMLE assumption (Eq.(\ref{eq:listmle-assumption})).
Then the objective function to learn the LLM \( \pi_\theta \):
\end{theorem}

\begin{equation}
\begin{aligned}
    \mathcal{L}_{\text{lmle}} &= -\mathbb{E} \biggl[
    \sum^m_{i=1} \log \frac{\exp\bigl(\gamma(y_i \mid x)\bigr)}{
        \exp\bigl(\gamma(y_i \mid x)\bigr) + \sum_{j=i}^m \exp\bigl(\gamma(y_j \mid x)\bigr)}
    \biggr].
\end{aligned}
\end{equation}

\textit{Proof.}
From Eq.(\ref{eq:1-n}),
\begin{gather}
\begin{aligned}
    \mathbb{P}\text{r}(y_1 & \succeq ... \succeq y_m) = \prod^m_{i=1} \mathbb{P}\text{r}(y_i \succeq y_{i+1}, ..., y_i \succeq y_m)  \\
    & = \prod^m_{i=1} \frac{\text{exp}(\gamma(y_i \mid x))}{\text{exp}(\gamma(y_i \mid x)) + \sum^m_{j=i+1}\text{exp}(\gamma(y_j \mid x))}
\end{aligned}.
\end{gather}
% The derivation above uses the result from Eq.(\ref{eq:1-n}).

We can learn $\pi_\theta$ by maximizing the logarithm-likelihood: 
\begin{gather}
\max \log \mathbb{P}\text{r}(y_w \succeq y^{(1)}_l, \dots, y_w \succeq y^{(m)}_l) \Leftrightarrow 
\min - \log \mathbb{P}\text{r}(y_w \succeq y^{(1)}_l, \dots, y_w \succeq y^{(m)}_l) = \mathcal{L}, \\
 \therefore \mathcal{L}_{\text{lmle}} = -\mathbb{E} \biggl[
    \sum^m_{i=1} \log \frac{\exp\bigl(\gamma(y_i \mid x)\bigr)}{
        \exp\bigl(\gamma(y_i \mid x)\bigr) + \sum_{j=i}^m \exp\bigl(\gamma(y_j \mid x)\bigr)}
    \biggr], \\
\text{where} \quad \gamma(y \mid x) = \beta \log \frac{\pi_\theta(y \mid x)}{\pi_{\mathrm{ref}}(y \mid x)}.
\end{gather}


\section{Baselines}\label{apx:sec:baselines}

We conduct detailed illustrations on the baselines compared with \Ours in Section \ref{sec:main-result} below.

\begin{itemize}[leftmargin=*]
  \item RRHF \citep{yuan2023rrhf} scores responses via a logarithm of conditional probabilities and learns to align these probabilities with human preferences through ranking loss.
  \item SLiC-HF \citep{zhao2023slic} proposes a sequence likelihood calibration method which can learn from human preference data.
  \item DPO \citep{guo2024direct} simplifies the PPO \citep{ouyang2022training} algorithms into an offline direct optimization objective with the pairwise Bradley-Terry assumption.
  \item IPO \citep{azar2024general} theoretically grounds pairwise assumption in DPO into a pointwise reward.
  \item CPO \citep{xu2024contrastive} adds a reward objective with sequence likelihood along with the SFT objective.
  \item KTO \citep{ethayarajh2024kto} adopts the Kahneman-Tversky model and proposes a method which directly maximizes the utility of generation instead of the likelihood of the preferences.
  \item RDPO \citep{park2024disentangling} modifies DPO by including an additional regularization term to disentangle the influence of length.
  \item SimPO \citep{meng2024simpo} further simplifies the DPO objective by using the average log probability of a sequence as the implicit reward and adding a target reward margin to the Bradley-Terry objective.
  \item Iterative DPO \citep{xiong2024iterative} identifies the challenge of offline preference optimization and proposes an iterative learning framework.
\end{itemize}


\section{Experiment settings}\label{apx:sec:main-result-setting}

\subsection{Table \ref{tab:main-performance}}\label{apx:sec:main}

We conduct evaluation on two widely used benchmark: AlpacaEval2 \citep{dubois2024length} and MixEval \citep{ni2024mixeval}.
We consider two base models: Mistral-7b-base and Mistral-7b-it. For Mistral-7b-base, we first conduct supervised finetuning following \citet{meng2024simpo} before the preference optimization.

The performance scores for offline preference optimization baselines are from SimPO \citep{meng2024simpo}.
To have a fair comparison with these baselines, we adopt the same off-the-shelf reward model \citep{jiang2023llm} as in SimPO for the iterative DPO baseline and \Ours.

For the iterative DPO baseline, we generate 2 responses for each prompt, score them with the off-the-shelf reward model and construct the preference pair data to tune the model.

For \Ours (contrastive $\mathcal{L}_{\text{con}}$), we generate 10 responses each iteration and score them with the reward model. The top-1 ranked response and the bottom-3 ranked responses are adopted as the chose response and rejected responses respectively.
Generation temperature is selected as 1 and 0.8 for Mistral-7b-base and Mistral-7b-it respectively (we search it among 0.8, 0.9, 1.0, 1.1, 1.2).

For \Ours (LambdaRank $\mathcal{L}_{\text{lamb}}$), we generate 10 responses each iteration and score them with the reward model. The top-2 ranked response and the bottom-2 ranked responses are adopted as the chose response and rejected responses respectively.
Generation temperature is selected as 1 and 0.8 for Mistral-7b-base and Mistral-7b-it respectively (we search it among 0.8, 0.9, 1.0, 1.1, 1.2).

For \Ours (ListMLE $\mathcal{L}_{\text{lmle}}$), we generate 10 responses each iteration and score them with the reward model. The top-2 ranked response and the bottom-2 ranked responses are adopted as the chose response and rejected responses respectively.
Generation temperature is selected as 1 and 0.8 for Mistral-7b-base and Mistral-7b-it respectively (we search it among 0.8, 0.9, 1.0, 1.1, 1.2).

\Ours can achieve even stronger performance with stronger off-the-shelf reward model \citep{dong2024rlhf}.
% Results with stronger a reward model can be found in Appendix \ref{apx:sec:stronger-rm}.

\subsection{Table \ref{tab:objective}}\label{apx:sec-objective-setting}

We conduct experiments on both Gemma2-2b-it \citep{team2024gemma} and Mistral-7b-it \citep{jiang2023mistral}.
Following \citet{Tunstall_The_Alignment_Handbook} and \citet{dong2024rlhf}, we perform training on UltraFeedback dataset for 3 iterations and show the performance of the final model checkpoint.
We use the pretrained reward model from \citet{dong2024rlhf}.
The learning rate is set as 5e-7 and we train the LLM for 2 epochs per iteration.

For the pairwise objective, we generate 2 responses for each prompt and construct the preference pair data with the reward model.
For the others, we generate 4 responses per prompt and rank them with the reward model.
For the contrastive objective, we construct the 1-vs-N data with the top-1 ranked response and the other responses.
For the listMLE and lambdarank objective, we take the top-2 as positives and the last-2 as the negatives.
Experiments with opensource LLM as the evaluator (\texttt{alpaca\_eval\_llama3\_70b\_fn}) can be found in Table \ref{tab:objective2}.



\begin{table*}[t]
    \centering
    % \renewcommand{\arraystretch}{1.2}
    \caption{Preference optimization objective study on AlpacaEval2 and MixEval. For AlpacaEval2, we report the result with both opensource LLM evaluator \texttt{alpaca\_eval\_llama3\_70b\_fn} and GPT4 evaluator \texttt{alpaca\_eval\_gpt4\_turbo\_fn}. SFT corresponds to the initial chat model.}\label{tab:objective2}
    \small
    \begin{tabular}{llccccccccc}
        \toprule
        & & \multicolumn{2}{c}{AlpacaEval 2 (opensource LLM)} & \multicolumn{2}{c}{AlpacaEval 2 (GPT-4)} & \multicolumn{1}{c}{MixEval} & \multicolumn{1}{c}{MixEval-Hard} \\
         \cmidrule(r){3-4} \cmidrule(r){5-6} \cmidrule(r){7-7} \cmidrule(r){8-8}
        & Method & LC Winrate & Winrate & LC Winrate & Winrate & Score & Score \\
        \midrule
        \multirow{6}{*}{\rotatebox{90}{Gemma2-2b-it}} & SFT & 47.03 & 48.38 & 36.39 & 38.26 & 0.6545 & 0.2980 \\
        \cmidrule{2-8}
        & pairwise & 55.06 & 66.56 & 41.39 & 54.60 & 0.6740 & 0.3375 \\
        & contrastive & 60.44 & 72.35 & 43.41 & 56.83 & 0.6745 & 0.3315 \\
        & ListMLE & 63.05 & 76.09 & 49.77 & 62.05 & 0.6715 & 0.3560 \\
        & LambdaRank & 58.73 & 74.09 & 43.76 & 60.56 & 0.6750 & 0.3560 \\
        \midrule
        \midrule
        \multirow{6}{*}{\rotatebox{90}{Mistral-7b-it}} & SFT & 27.04 & 17.41 & 21.14 & 14.22 & 0.7070 & 0.3610 \\
        \cmidrule{2-8}
        & pairwise & 49.75 & 55.07 & 36.43 & 41.86 & 0.7175 & 0.4105 \\
        & contrastive & 52.03 & 60.15 & 38.44 & 42.61 & 0.7260 & 0.4340 \\
        & ListMLE & 48.84 & 56.73 & 38.02 & 43.03 & 0.7360 & 0.4200 \\
        & LambdaRank & 51.98 & 59.73 & 40.29 & 46.21 & 0.7370 & 0.4400 \\
        \bottomrule
    \end{tabular}
\end{table*}


\subsection{Table \ref{fig:list-study}}\label{apx:sec-list-setting}

We adopt Gemma2-2b-it as the initial model. All the models are trained with iterative DPO for 3 iterations. We use the off-the-shelf reward model \citep{dong2024rlhf}.
We generate 2 responses for each prompt in each iteration.
For ``w. current'', we only use the scored responses in the current iteration for preference optimization data construction.
For ``w. current + prev'', we rank the responses in the current iteration and the previous one iteration, and construct the preference pair data with the top-1 and bottom-1 ranked responses.
For ``w. current + all prev'', we rank all the responses for the prompt in the current and previous iterations and construct the preference pair data.
For ``single temperature'', we only adopt temperature 1 and generate 2 responses for reward model scoring.
For ``diverse temperature'', we generate 2 responses with temperature 1 and 0.5 respective and rank the 4 responses to construct the preference data with the reward model.

\subsection{Table \ref{tb:sft-rlhf-result}}\label{apx:sec:sft-rlhf}

We use mistral-7b-it \citep{jiang2023mistral} as the initial model to alleviate the influence of the math related post-training data of the original model.
% For SFT, we conduct training on the training set of MATH \citep{hendrycks2021measuring} and GSM8K \citep{cobbe2021training} respectively.
For SFT, we conduct training on the meta-math dataset \citep{yu2023metamath}.
For DPO, we use the prompts in the training set of the two dataset and conduct online iterative preference optimization with the binary rule-based reward (measure if the final answer is correct or not with string match). 
The evaluation is performed on the test set of MATH and GSM8K respectively.
% For both SFT and DPO, we conduct careful hyper-parameter search.
For SFT, we follow the same training setting with \citet{yu2023metamath}.
For DPO, we search the learning rate in 1e-7, 2e-7, 5e-7, 2e-8, 5e-8 and train the LLM for 5 iterations with early stop (1 epoch per iteration for MATH and 2 epoch per iteration for GSM8K). The learning rate is set as 1e-7 and we select the checkpoint after the first and fourth iteration for GSM8K and MATH respectively.

\subsection{Figure \ref{fig:merge-study}(a)}\label{apx:sec-hard-neg-setting}

We conduct training with the prompts in the training set of GSM8K and perform evaluation on GSM8K testing set.
We conduct learning rate search and finalize it to be 2e-7.
The learning is performed for 3 iterations.

We make explanations of how we construct the four types of negative settings:
For (1) a random response not related to the given prompt, we select a response for a random prompt in Ultrafeedback.
For (2) a response to a related prompt, we pick up a response for a different prompt in the GSM8K training set.
For (3) an incorrect response to the given prompt with high temperature, we select the temperature to be 1.
For (4) an incorrect response to the given prompt with low temperature, we select the temperature to be 0.7.

\begin{figure}[t]
\centering
\includegraphics[scale=0.4]{figure/LLM_alignment_gemma_temperature_study.pdf}
\vskip -1em
\caption{Training temperature study with $\mathcal{L}_{\text{pair}}$ on Gemma2-2b-it and Alpaca Eval 2. Within a specific range ($>$ 0.9), lower temperature leads to harder negative and benefit the trained LLM. However, temperature lower than this range can cause preferred and rejected responses non-distinguishable and lead to degrade training.}\label{apx:tab:temp-hard}
\end{figure}

\subsection{Figure \ref{fig:merge-study}(b)}\label{apx:sec-hard-neg-setting-temp}

We conduct experiments on both Gemma2-2b-it and Mistral-7B-it models.
For both LLMs, we conduct iterative DPO for 3 iterations and report the performance of the final model.
We perform evaluation on Alpaca Eval2 with \texttt{alpaca\_eval\_llama3\_70b\_fn} as the evaluator.

For temperature study, we find that under a specific temperature threshold, repeatedly generated responses will be large identical for all LLMs and cannot be used to construct preference data, while the threshold varies for different LLMs.
% As a result, we select temperatures above the threshold for robust experiments.
The ``low'' and ``high'' refer to the value of those selected temperatures.
% For Gemma2-2b-it, we use temperature as 0.2, 0.5 and 0.7 to generate the responses, score the responses by the reward model and train the LLM with the newly labeled data.
% For Mistral-7b-it, we set the temperature as 1, 1.1 and 1.2 respectively.
We also conduct experiments on Gemma2-2b-it model and show the results in Figure \ref{apx:tab:temp-hard}.


\subsection{Figure \ref{fig:merge-study}(c)}\label{apx:sec-length-setting}

We adopt Mistral-7b-it as the initial LLM and the contrastive objective (Eq. \ref{eq:contrastive}) in iterative preference optimization.
We generate 4/6/8/10 responses with the LLM and score the responses with the off-the-shelf reward model \citep{dong2024rlhf}.
The top-1 scored response is adopted as the positive response and the other responses are treated as the negative responses to construct the 1-vs-N training data.
The temperature is set as 1 to generate the responses.


% \newpage
% \section{\Ours with a stronger reward model}\label{apx:sec:stronger-rm}

% In Section \ref{sec:lrpo}, we show the results with LLM-Blender \citep{jiang2023llm} as the reward model to have a fair comparison with the baseline methods.
% In this section, we would like to show that \Ours can achieve even stronger performance with stronger off-the-shelf reward model \citep{dong2024rlhf}.
% The results are shown in Table \ref{apx:tab:main-performance}, where we can find that a stronger reward model can further improve the performance of \Ours.


% \begin{table*}[h]
%     \centering
%     \caption{Method evaluation on AlpacaEval 2 and MixEval. LC WR and WR denote length-controlled win rate and win rate respectively. Offline baseline performances on AlpacaEval 2 are from \citept{meng2024simpo} with LLM-Blender reward model \citep{jiang2023llm}.}\label{apx:tab:main-performance}
%     \scalebox{0.78}{
%     \begin{tabular}{lcccccccccc}
%         \toprule
%         Model & \multicolumn{4}{c}{Mistral-Base (7B)} & \multicolumn{4}{c}{Mistral-Instruct (7B)} \\
%         \cmidrule(lr){2-5} \cmidrule(lr){6-9}
%         & \multicolumn{2}{c}{Alpaca Eval 2}  & \multirow{1}{*}{MixEval} & \multirow{1}{*}{MixEval-Hard} & \multicolumn{2}{c}{Alpaca Eval 2}  & \multirow{1}{*}{MixEval} & \multirow{1}{*}{MixEval-Hard} \\
%         \cmidrule(lr){2-3} \cmidrule(lr){4-4} \cmidrule(lr){5-5}  \cmidrule(lr){6-7} \cmidrule(lr){8-8} \cmidrule(lr){9-9}
%         & LC WR & WR  & Score & Score & LC WR  & WR & Score & Score \\
%         \midrule
%         SFT    & 8.4  & 6.2   &  0.602  & 0.279  & 17.1 & 14.7  & 0.707 & 0.361 \\
%         RRHF   & 11.6 & 10.2   &  0.600  & 0.312  & 25.3 & 24.8  &   0.700    & 0.380 \\
%         DPO    & 15.1 & 12.5  &  0.686  &  0.341 & 26.8 & 24.9  & 0.702 & 0.355 \\
%         KTO    & 13.1 & 9.1    & \textbf{0.704}  & 0.351   & 24.5 & 23.6  &   0.692    & 0.358 \\
%         RDPO   & 17.4 & 12.8  & 0.693  & 0.355   & 27.3 & 24.5  &   0.695    & 0.364 \\
%         SimPO  & 21.5 & 20.8  &  0.672  &  0.347 & 32.1 & 34.8  & 0.702  & 0.363 \\
%         Iterative DPO  & 18.9  & 16.7  & 0.660   & 0.341  & 20.4 & 24.84  & 0.719  & 0.389 \\
%         \midrule
%         \multicolumn{9}{c}{Reward model: LLM-Blender \citep{jiang2023llm}}  \\
%         \midrule
%         \Ours (contrastive) & 31.6 & 30.8  &   0.703 & 0.409  & 32.7 & 38.6  &  0.718 & \textbf{0.418} \\
%         \Ours (LambdaRank) &  \textbf{34.9} & \textbf{37.2} & 0.695 &  \textbf{0.452}  & \textbf{32.9} & \textbf{38.9}   & \textbf{0.720} & 0.417  \\
%         \Ours (ListMLE) & 31.1  &  32.1   &  0.669  & 0.390  &  29.7 & 36.2    & 0.709  & 0.397 \\
%         \midrule
%         \multicolumn{9}{c}{Reward model: FsfairX \citep{dong2024rlhf}}  \\
%         \midrule
%         \Ours (contrastive) & \textbf{41.5} & \textbf{42.9} & 0.718 & 0.417    & \textbf{43.0}  & \textbf{53.8} & 0.718 & 0.425   \\
%         \Ours (LambdaRank) & 35.8 & 34.1 & 0.717 & 0.431   & 41.9  & 48.1 & \textbf{0.740} & \textbf{0.440}  \\
%         \Ours (ListMLE) & 36.6 & 37.8 & \textbf{0.730} & \textbf{0.423}   & 39.6  & 48.1 & 0.717 & 0.397   \\
%         \bottomrule
%     \end{tabular}}
% \end{table*}


\end{document}
