

The proliferation of misinformation on social media \citep{arnold2020onlineFC,diakopoulos2020computational}, along with the rise of generative AI \citep{augenstein2024factualityLLMs} have led to increasing concerns about its current and future potential harms, (e.g., to health \citep{ijerph19095321}) and threats to democracy and political stability 
% \citep{McKay2021misinfodemocracy,reglitz2022fake}.
\citep{reglitz2022fake}. 

Fact-checkers play a crucial role in combatting misinformation \citep{graves2017anatomy}, and in recent years, have partnered with social media platforms, e.g., Meta, YouTube, and TikTok, to tackle its spread on these platforms. However, due to the scale of misleading content shared online, community moderation (e.g., options to flag potential misinformation, group/server moderators) is often employed in parallel \citep{morrow2022emerging}, as a complementary approach (e.g., \citep{YouTubeMisinfo}; see also the practice of \textit{snoping} \citep{pilarski_community_2024}). 

\begin{figure}[t]
    \centering
    \includegraphics[width=0.95\columnwidth]{Figures/note_pan.png}
    \caption{An example of a community note. Notice the fact-checking link and rating.}
    \label{fig:notes_sample}
\end{figure}

%% put in implications section
% In addition to verifying claims, in recent years many fact-checking organisations have also assumed a wider role in combatting misinformation spread, conducting long-term investigative journalism projects and citizen media literacy programs \citep{juneja2022human}. 

The expansion of fact-checking projects in the last decade \citep{lauer2024growfactcheck}, alongside their broader initiatives to curb misinformation (e.g., citizen media literacy programmes \citep{juneja2022human}) have been aided by partnerships with social media platforms such as Meta and Google \citep{graves2020infrastructure}, which fund independent fact-checking agencies to fact-check potentially false claims on their platform.\footnote{Fact-checkers provide a judgment of claim veracity and exert no influence on the platforms' content moderation policies \citep{PoynterMeta2025}.} 

However, political pressure and accusations of bias and censorship, %from politicians and social media platforms, 
and most recently, Meta's announcement of its plans to end its partnerships with fact-checkers in the U.S. and implement a community moderation model \citep{MetaFactChecking2025}, threatens the financial stability of fact-checking organisations, and hence, their ability to keep up with the increasing volume and sophistication of misinformation spread \citep{Duke2024FactCheckingSputters,IFCN2024report}.


% Community moderation has been proposed as a means of scaling up fact-checking \citep{martel2024crowdmisinfo}, and partly addressing the challenges of establishing cross-partisan trust in fact-checking \citep{poynter2019republicans}. One of the most notable examples of such a system is Twitter/X, which commenced its pilot Community Notes programme (then called Birdwatch) in January 2021, and later launched it to the public in October 2022 \citep{TwitterBirdwatch2021}. 


Meta's recent policy shift also implies that these two strategies (fact-checking and community notes) are independent and in opposition, rather than two complementary strategies of tackling online misinformation.
In this paper, \textit{we examine Twitter/X community notes as a case study to understand how fact-checking is used in community notes}. Specifically, we investigate the following two questions:  \textbf{(RQ1) To what extent do community notes rely on the work of professional fact-checkers? } and \textbf{(RQ2) What are the traits of posts and notes that rely on fact-checking sources?} 
Studying the relationship between fact-checking and community notes is vital for understanding the shared role of expert and community-driven fact-checking in the global information ecosystem.

% Answering these questions can shed light on the long-term implications of this move. 

We find that at least 1 in 20 community notes rely explicitly on the work of professional fact-checkers, while this reliance is higher still for high-stakes topics such as health and politics. Our experiments also show that fact-checking is vital for debunking misleading content linked to broader narratives or conspiracy theories. These findings imply that high-quality community notes cannot be produced independently of professional fact-checking. They further suggest that the pressure on fact-checkers exerted by platforms and politicians by defunding and discrediting fact-checking organisations will have corrosive effects on the quality of notes and destructive implications for information integrity more widely.
    
    
    % in turn, can hint into the fact that Meta's move will for sure, and this is not a speculation, increase the spread of such cases of misinformation in the long run.
    
    
% We ask the following questions:

% \begin{itemize}
%     \item RQ1: To what degree do community notes rely on fact-checkers?
%     \item RQ2: Are there types of misinformation for which community notes are particularly reliant on fact-checking for?  
% \end{itemize}

