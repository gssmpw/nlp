

Two commonly-employed strategies to combat the rise of misinformation on social media are (i) fact-checking by professional organisations and (ii) community moderation by platform users. Policy changes by Twitter/X and, more recently, Meta, signal a shift away from partnerships with fact-checking organisations and towards an increased reliance on crowdsourced community notes. However, the extent and nature of dependencies between fact-checking and \emph{helpful} community notes remain unclear. To address these questions, we use language models to annotate a large corpus of Twitter/X community notes with attributes such as topic, cited sources, and whether they refute claims tied to broader misinformation narratives. Our analysis reveals that community notes cite fact-checking sources up to five times more than previously reported. Fact-checking is especially crucial for notes on posts linked to broader narratives, which are \textit{twice} as likely to reference fact-checking sources compared to other sources. In conclusion, our results show that successful community moderation heavily relies on professional fact-checking. 


% We discuss the implications of these findings for the future of fact-checking.

% We discover that effective community moderation depends on professional fact-checking

% The increasing prevalence of misinformation on social media platforms has motivated various approaches to debunking false information.  Policy changes by Twitter/X and more recently, Meta, have signalled a shift away from partnerships with fact-checking organisations and increased reliance on crowdsourced community notes. However, it is unclear (i) whether community notes represent an adequate stand-alone solution and (ii) whether notes targeting certain types of misinformation are more reliant on fact-checkers than others. 
% To answer these questions, we analyse and annotate a large corpus of Twitter/X community notes.
% Our findings indicate that successful community moderation is reliant on professional fact-checking: community notes that link to fact-checking articles are deemed more helpful than those that do not. \textcolor{red}{Say that they rely on these sources at least 4 times more than was previously reported.}
% We also find that professional fact-checking is particularly vital for community notes attached to posts belonging to broader misinformation narratives (e.g., conspiracy theories).
% We discuss the implications of these findings for the future of fact-checking.
