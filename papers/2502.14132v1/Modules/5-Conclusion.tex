In this work, we annotate a large corpus of Twitter community notes with attributes such as topic, cited sources, and whether they refute claims tied to broader misinformation narratives. We find that effective community moderation depends on professional fact-checking to an extent far greater than previously reported. We find that community notes linked to broader narratives or conspiracy theories are particularly reliant on fact-checking.

Our results reveal that community notes and professional fact-checking are deeply interconnected—fact-checkers conduct in-depth research beyond the reach of amateur platform users, while community notes publicise their work. The move by platforms to end their partnerships and funding for fact-checking organisations will hinder their ability to fact-check and pursue investigative journalism, which community note writers rely on. This, in turn, will limit the efficacy of community notes, especially for high-stakes claims tied to broader narratives or conspiracies.

% This indicates that community notes and professional fact-checking are intrinsically linked and their relationship is symbiotic -- professional fact-checkers perform in-depth research that amateur platform users can't, while community notes help publicise the work of fact-checkers. Therefore, we posit, the move by platforms to end their partnerships, and therefore funding, of fact-checking organisations will significantly hinder their ability to perform their basic function of fact-checking as well as their more complex investigative journalism projects that are especially necessary for community note writers. This, in turn, will significantly limit the effectiveness of community notes, specifically for high-stakes claims such as these related to broader narratives or conspiracies.