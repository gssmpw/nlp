\section{Related Work}
\noindent
\textbf{Hopfield Networks.}
Hopfield networks~\cite{hopfield82, hopfield88} are a type of artificial neural network with recurrent structures that model associative memory.
Their development laid the foundation for later models such as Boltzmann machines~\cite{Ackley1985boltzmannmachines} and long short-term memory~\cite{lstm} in the latter part of the 20th century.

In recent years, MHNs, also known as dense associative memory~\cite{krotov16dense}, have been attracting attention because they can have an exponentially large memory capacity \cite{demircigil17hugecapacity}.
Numerous studies have demonstrated the effectiveness of MHNs on various tasks including image classification~\cite{furst2021cloob, ota2023learning}, immune repertoire classification~\cite{widrich20modern},
tabular data classification~\cite{schafl2021hopular},
reaction template prediction~\cite{seidl2022improving},
predictive coding~\cite{NEURIPS2021_1fb36c4c} and reinforcement learning~\cite{widrich2021modern}.

MHNs are formulated as dynamical systems described by analytical differential equations.
Specifically, \citet{ramsauer21hopfieldallyouneed} generalized the energy function from discrete states to continuous states, and then \citet{krotov2021large} formulated the dynamical system of MHNs with two-body differential equations.
Follow-up studies, such as work on universal Hopfield networks~\cite{millidge2022universal}, have further generalized the dynamical system.

\noindent \textbf{OOD Detection.}
OOD detection aims to identify data samples that deviate from the distribution of training data samples.
This paper focuses on post hoc approaches, where the detection mechanism is applied after the model has been trained.
One of the most well-known approaches is maximum softmax probability (MSP) scoring~\cite{hendrycks2017msp}, which uses the highest softmax output score to identify OOD samples, under the assumption that ID samples yield higher MSP scores compared to OOD samples.
To more precisely estimate the distribution of OOD samples, various enhancements and alternative post hoc methods have been proposed~\cite{liang2018odin,liu2020energy,sun2021react,sun2022dice,shen2023posthoc,chen2024tagfog}.
Among them, energy-based OOD detection approaches~\cite{liu2020energy, sun2021react} are related to this study in the sense that MHNs have a scalar-valued function associated with the network states, the so-called the Hopfield energy.

Most recently, several pioneering studies have demonstrated the effectiveness of Hopfield energy for OOD detection~\cite{zhang23she, Hofmann2024boosting}.
Their methods identify data samples with high Hopfield energy as OOD samples and achieve superior performance among energy-based OOD detection methods.
However, from a theoretical perspective, every test sample, including an OOD sample, falls into one of the attractors representing a memory pattern associated with an ID data sample as the dynamical system of MHNs evolves over time.
To address this problem, this paper explores MHNs that explicitly have an attractor for OOD samples.