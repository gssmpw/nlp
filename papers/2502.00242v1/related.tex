\section{Relevant Works}
\label{sec:related}

Exploring the existing discussions on NES and energy efficiency (EE), the authors in~\cite{10121451} investigate various techniques for reducing network power consumption by adapting time, frequency, spatial, and power domain resources. They also provide an analysis of the trade-offs between EE and throughput to identify which domain offers the highest potential for power savings.
An extensive EE study on multiple-antenna cellular networks is conducted in~\cite{8454484}, introducing a more robust concept of area EE instead of spatial EE.
An overview of general industry practices for RAN energy saving, including both 3GPP and open RAN, is provided in~\cite{kundu2024energyefficientranindustry}, highlighting the trend towards AI/ML-based designs in 6G.
{A deep reinforcement learning based solution for NES has been proposed in~\cite{low_traffic_nes_samsung}, where base stations develop local energy saving solutions.}
The authors in~\cite{8922617} survey new and anticipated architectural changes in 6G networks aimed at network sustainability, ubiquitous coverage, pervasive AI/ML application, and enhanced protocols.
An analytical and system-level study of various densification technologies for 6G networks is performed in~\cite{azzino2024energycostefficient6gnetworks}, focusing on EE analysis for traffic-loaded scenarios.
The authors in~\cite{deployement_paper} showcase a DT-assisted network planning and deployment optimization problem, minimizing network CAPEX, which is unique among existing works. 
{None of the previous studies explore the DT-assisted network wide energy optimization for low-traffic and idle scenarios, and at different levels of granularity.}

%\note{Authors in~\cite{nes_cellless_ran} proposes an access point switching algorithm for cell-less RAN, although with limited visibility of the network, it may lead to suboptimal resource utilization. A deep reinforcement learning based solution for NES has been proposed in~\cite{low_traffic_nes_samsung}, where base stations develop local energy saving solutions. However these methods do not have the flexibility optimize network resources at different levels of granularity and a question of their time complexity can be difficult to answer}.
%To the best of our knowledge, this work is the first to demonstrate DT-assisted network wide energy optimization for low-traffic and idle scenarios.
