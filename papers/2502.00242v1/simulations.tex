\section{Digital Twin for System-level Analysis}
\label{sec:sim}

We develop a detailed digital twin of an area (0.34 km $\times$ 0.28 km) in downtown Philadelphia (shown in Figure~\ref{fig:network}), accurately representing a cellular network operating at millimeter wave frequencies. 
To analyze the network's capacity and coverage, we model only outdoor UE locations, which are uniformly distributed across 1 m $\times$ 1 m grids in outdoor areas.
The radio frequency channels are generated based on ray tracing using parameters provided in Table~\ref{tab:rf_parameters}, along with realistic foliage modeling in the DT and the associated losses. 

The network under consideration consists of 49876 UE locations and $N_C=169$ active cells distributed across 69 sites, meeting the \emph{throughput} target of 50 Mbps for 80\% of UEs in the initial deployment. The same set of cells provides SSB \emph{coverage} to 94\% ($N_{UE}=46884$) of all UEs.

\begin{figure}
\centering
\includegraphics[width=0.9\linewidth]{figs/network.png}
\caption{28GHz network based on Downtown Philadelphia Digital Twin \small{(Map data \copyright  OpenStreetMap contributors, Microsoft, Esri community Maps contributors. Map layers by Esri, markers added for more visibility, license: https://creativecommons.org/licenses/by-sa/2.0/legalcode)}}
    \label{fig:network}
    \vspace{-15pt}
\end{figure}

\begin{table}[]
\begin{center}
%\large
\vspace{10pt}
\begin{tabular}{|c|c|c|}
\cline{1-3} 
\textbf{Parameters}        & \textbf{Cell}         & \textbf{UE}      \\ \cline{1-3} 
{Carrier Frequency}        &  \multicolumn{2}{|c|}{28 GHz}                \\ \cline{1-3}    
{Bandwidth}        &  \multicolumn{2}{|c|}{800 MHz}                \\ \cline{1-3}  
{Antenna array } & 8 row x 24 col  & 2 row x 2 col  \\
{arrangement} & x 2 polarization & x 2 polarization \\ \cline{1-3}
%{Codebook SSB} &  \multicolumn{2}{|c|}{8 Az beams x 4 El beams}  \\ \cline{1-3}
%{SU-MIMO order} & \multicolumn{2}{|c|}{2 }\\    \cline{1-3}  
%{Number of sectors} & 3 & N/A \\   \cline{1-3}  
{Maximum array gain} & 28.15 dBi & 10 dBi \\   \cline{1-3}  
{Body Loss} & N/A & 8 dB \\  \cline{1-3}  
{Implementation margin} & N/A  & 1.9 dB  \\ \cline{1-3}  
{Noise Figure}  & 10 dB & 6.7 dB  \\ \cline{1-3}  
{Cell-edge reliability margin}  & \multicolumn{2}{|c|}{13.2 dB}                \\ \cline{1-3}  
{Foliage loss} &  \multicolumn{2}{|c|}{4 dB/m}                \\ \cline{1-3}  
{Building reflection loss} & \multicolumn{2}{|c|}{6.4 dB}                \\ \cline{1-3} 
\end{tabular}
\caption{Digital-twin parameters}
\label{tab:rf_parameters}
\end{center}
\vspace{-25pt}
\end{table}

\begin{figure*}%[h]
\centering
\begin{subfigure}[b]{0.34\linewidth}
    \includegraphics[width=\linewidth]{figs/baseline_codebook.png}
    \caption{Baseline SSB codebook arrangement.}
    \label{fig:baseline_codebook}
\end{subfigure}
\hfill
\begin{subfigure}[b]{0.31\linewidth}
    \includegraphics[width=\linewidth]{figs/min_snr.png}
    \caption{Minimum SNR along beam directions.}
    \label{fig:min_snr}
\end{subfigure}
\hfill
\begin{subfigure}[b]{0.31\linewidth}
   \includegraphics[width=\linewidth]{figs/sample_hybrid_beams.png}
\caption{Sample of an optimized SSB codebook.}
    \label{fig:sample_hybrid_beam}
\end{subfigure}
\caption{Baseline SSB codebook (a), and opportunities for SSB codebook optimization in (c) for a given cell with a non-uniform coverage (b).}
\label{fig:beam_per_sector}
\vspace{-15pt}
\end{figure*}

The baseline SSB codebook comprises 32 SSB beams, each with a width of 15$^{\circ}$ in azimuth and 7.5$^{\circ}$ in elevation, covering an azimuth span of   120$^{\circ}$ (-60$^{\circ}$ to 60$^{\circ}$) and an elevation span of 30$^{\circ}$ (-30$^{\circ}$ to 0$^{\circ}$), wherein 0$^{\circ}$ elevation is associated with the horizon, and negative values represent angles below the horizon. Figure~\ref{fig:baseline_codebook} shows the arrangement of the beams in the baseline codebook.

For the beam-level optimizations (both local and global), we consider four different types of beams with varying beamwidths in azimuth (BW$_{Az}$) and elevation (BW$_{El}$); Type-1: (BW$_{Az} =$ 15$^{\circ}$, BW$_{El} =$ 7.5$^{\circ}$), 
Type-2: (BW$_{Az} =$ 15$^{\circ}$, BW$_{El} =$ 15$^{\circ}$),
Type-3: (BW$_{Az} =$ 30$^{\circ}$, BW$_{El} =$ 7.5$^{\circ}$), and
Type-4: (BW$_{Az} =$ 30$^{\circ}$, BW$_{El} =$ 15$^{\circ}$).
Figure~\ref{fig:sample_hybrid_beam} shows an example of an SSB codebook comprising beams of different types, optimally selected for a cell with a non-uniform coverage region (as shown in Figure~\ref{fig:min_snr}).
The NES optimizations presented in~\S\ref{sec:system} are solved within the integer solution space using a mixed integer linear program (MILP)~\cite{10.1287/ijoc.2018.0857}.