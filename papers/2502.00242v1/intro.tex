\section{Introduction}

The global proliferation and deployment of 5G New Radio (NR) have remarkably met the ever-increasing demand from users and applications with unprecedented data rates~\cite{5G_survey}. To achieve quality of service (QoS) requirements such as reliability, low latency, and enhanced capacity, the total cost of ownership, including capital expenditure (CAPEX) and operational expenditure (OPEX), has significantly increased in modern wireless networks.
According to recent reports by the Global System for Mobile Communications Association (GSMA)~\cite{gsma_report} and the Next Generation Mobile Networks (NGMN) Alliance~\cite{9349624}, on average, 90\% of the energy use for a telecommunications operator — one of the main contributors to OPEX — comes from the network side, with the radio access network (RAN) accounting for 50-80\% of the total network energy use. Therefore, optimizing the energy efficiency of the RAN is paramount for the telecommunications industry as we move forward to 6G. 

{Statistical analysis on typical traffic distribution across the network, such as the one presented in~\cite{nokia_report}, reveals a very non-uniform traffic load distribution across time and base stations.}
It is observed that around 70\% of base stations (gNB) handle only 20\% of the traffic, resulting in very low utilization. Additionally, traffic during peak hours is approximately 60-70\% higher than the average and significantly greater than during low-traffic hours at night. {Consequently, many base stations, during few hours per day, are in no or low-traffic state. During this period, the base stations are still active and consume energy, mostly for providing system broadcasts (non-data traffic) and maintaining idle resources.}
This provides a significant opportunity to design and implement NES features targeting low-traffic and idle scenarios.

3GPP is currently exploring idle-mode NES strategies for standardization in 5G Advanced and 6G. However, the practical implications and system-level implementation aspects have not been thoroughly examined. Optimizing network operations for idle-mode requires information about the idle UEs that are not connected to the network. Traditional methods allow the network to track the idle UEs only at the granularity of tracking areas, which typically encompass multiple cells. To enable efficient network optimization at the cell or sub-cell levels, advanced technologies such as digital twins (DT)~\cite{DT} are needed. { %A digital twin is a combination of a physical system or asset, a virtual representation of that asset and a bi-directional communications between the two for the purpose of services such as monitoring, optimization, predictive maintenance, etc~\cite{DT_survey_definition}. 
In the context of wireless communication networks, a DT is a high fidelity digital representation of the network that provides an accurate representation of the radio frequency and physical environments using computer vision based methods~\cite{jiang2024learnablewirelessdigitaltwins} and ray tracing techniques~\cite{ray_tracing}, as well as the network nodes' behaviors and operations.}

This work presents a comprehensive system-level analysis aimed at minimizing the OPEX related to energy consumption during idle mode. The optimization focuses on network resources, specifically the set of active cells and the associated communication beams.
The key contributions of this work are as follows:
\begin{enumerate}[leftmargin=*, nosep, topsep=0pt]
    \item We propose system-level NES strategies in off-peak hours by selecting optimal configurations for cells and beams.

    \item We develop a simulation framework based on a highly accurate digital twin of an urban network deployment.
        
    \item Using the developed DT, we analyze the proposed strategies in terms of their impact on the network OPEX and demonstrate NES gains of up to 46.4\% in the mentioned network.
    
    \item Finally, we provide practical considerations for implementing the proposed NES strategies and analyze their impact on UE operation.
\end{enumerate}

The remainder of the paper is organized as follows. \S\ref{sec:related} reviews previous work related to this paradigm. \S\ref{sec:prelim} introduces essential topics and context necessary for the core system design. 
\S\ref{sec:system} details the  optimization problem formulations.
\S\ref{sec:sim} presents the developed digital twin framework. 
~\S\ref{sec:results} evaluates the proposed strategies. 
Finally, \S\ref{sec:conclusion} concludes the paper. 
