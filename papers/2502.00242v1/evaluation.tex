\section{Evaluation}
\label{sec:results}
In this section, we present the performance results of the proposed system-level optimizations using the developed DT. We also discuss practical aspects of implementing these strategies, and evaluate their impacts on the UE operation.

%\vspace{-10pt}

\begin{table*}[]
\begin{center}
%\large
\vspace{10pt}
\begin{tabular}{|c|c|c|c|c|c|}
\cline{1-5} 
\textbf{Strategy}  & \textbf{Number of Active Cells} & \textbf{Number of Active Beams}  & \textbf{Energy Consumption (relative unit)} & \textbf{Energy Saving (\%)} \\ \cline{1-5} 
Baseline    &212           & 6784             &9.494$\times 10^3$   & N/A   \\ \cline{1-5}  
Local beam-level       &212          & 1002             &7.031$\times 10^3$   & 25.9   \\ 
optimization~\eqref{eq:beam_per_sector}         &            &           &         &   \\ \cline{1-5}
Global cell-level       &163          & 5216             &7.300$\times 10^3$   & 23.0   \\ 
optimization~\eqref{eq:problem_sc}         &            &           &         &   \\ \cline{1-5}
Global joint beam- and      &163          & 490             &5.286$\times 10^3$   & 44.0   \\ 
cell-level optimization~\eqref{eq:main_problem}      &            &           &         &  \\ \cline{1-5}
\end{tabular} 
\vspace{5pt}
\caption{Network energy savings with system-level optimizations at different level of granularity.}
\label{tab:results}
\end{center}
\vspace{-10pt}
\end{table*}
\vspace{-5pt}
\subsection{Network Energy Saving} 
A comprehensive set of results is presented in Table~\ref{tab:results}, indicating optimizations at different levels of granularity.
The table also references the equations corresponding to different optimization formulations.

We observe that locally minimizing the number of active beams for each cell can reduce total number of beams by 85.2\% %along with 5.1\% cells turned OFF (no associated UE based on the SINR criterion) 
leading to 25.9\% network energy saving.
A globally optimal cell ON-OFF strategy deactivates 23\% of cells, offering proportional network energy saving. 
However, a global strategy that optimally selects the set of active cells, and their associated active beams guarantees the most energy savings, up to 44\%. This global strategy suggests
shutting down 23\% of the cells and deactivating up to 92.8\% of the total SSB beams in the network.
As expected, increasing the level of granularity and solving the optimization problem globally offers significantly greater gains than coarse and/or local optimizations. Next, we will discuss the practicality of implementing such strategies. 

%{Regarding the computational complexity of solving the proposed optimizations, we note that this should not be a major concern, since these problems are not expected to be solved in real time. For the sake of completeness, further discussions are provided in~\cite{nesFulldocument}}
%Additionally, generalizability and practicality aspects are also evaluated in~\cite{nesFulldocument}

\subsection{Practicality of the Proposed Strategies}
Global optimization of system-level resources relies on a key assumption that an accurate DT is available, specially for the scenarios involving idle or inactive UEs where the network does not have sufficient knowledge about their location, population, and link qualities. A well-calibrated DT can be used to provide a reliable representation/expectation of such missing information.  
Depending on the use-case, and the operator's choice, the level of fidelity of a DT can be selected to balance the complexity of creating the DT and the achievable gains. 
For example, a simple radio environment map of the network with potential gNB locations and assuming uniform user distribution can allow a cell or even beam level optimization. However, If the network has access to a DT calibrated based on the historical user distribution, beam, scheduling, and other network behavior information, higher NES gains are possible.

%{Although, a highest fidelity of the DT may not always be necessary and that flexibility is enabled by different level (granularity) of energy optimization proposed in this work. For example, a simple radio environment map of the network with potential gNB locations and assuming uniform user distribution can allow a cell or even beam level optimization. However, If the network has access to historical user distribution, beam, scheduling, and other network behavior information, a higher NES gain is possible.} %Nonetheless, some gain can be achieved due to the dynamic fidelity of the proposed solutions.}

Implementation of the proposed optimizations can be based on a central network management entity, such as a Service Management and Orchestration (SMO) system. This allows for the optimal selection of cells and beams for (de)activation, to achieve the highest NES. 
In practice, such an entity may produce a centralized action to be implemented, or a policy to be followed by gNBs. The policy can be expressed in terms of a set of time periods (within a day) and/or thresholds (e.g., traffic load or number of connected UEs), along with associated actions (e.g., cell/beam (de)activation) for each cell.

Another aspect related to the centralized optimization is the scale of the optimization problem. Although referred to as global optimization, in reality, the entire operator's network should be divided into clusters for performing such optimization. A cluster can range in size from a whole metropolitan area to a local neighborhood. The choice should balance the trade-off between complexity and the ultimate reward of the network optimization.

Alternatively, the optimizations can be performed in more localized areas and in a distributed manner with local information sharing across neighboring gNBs. Additionally, simpler heuristics for network optimizations may be pursued, where gNBs take iterative actions. The implementation and analysis of such scenarios are beyond the scope of this work and are left for future research.
\begin{figure*}%[h]
\centering
\begin{subfigure}[b]{0.325\linewidth}
    \includegraphics[width=\linewidth]{figs/ssb_snr_single_ray.png}
    \caption{Distribution of SSB SNR.}
    \label{fig:dl_ssb_snr}
\end{subfigure}
\begin{subfigure}[b]{0.325\linewidth}
    \includegraphics[width=\linewidth]{figs/ue_cell_connect_singleray.png}
    \caption{Coverage diversity.}
    \label{fig:coverage_div}
\end{subfigure}
\begin{subfigure}[b]{0.325\linewidth}
    \includegraphics[width=\linewidth]{figs/beam_per_cell_singleray.png}
    \caption{Number of active beams per cell.}
    \label{fig:beam_per_sector}
\end{subfigure}
\caption{Impact of the proposed NES strategies on UE operation.}
\label{fig:UE_imapct}
\vspace{-15pt}
\end{figure*}


\vspace{-5pt}

%\subsection{On Complexity of the Solution}
\subsection{On Complexity of the Solution}
It is important to note that the decisions for (de)activation of cells and beams at low or no traffic scenarios should essentially be made in a slow time scale (to avoid frequent changes in the system), and are based on historical or statistical assumptions about the possible idle UE locations and their link qualities. 
Hence, the nature of the problem is not dynamic (unlike problems like scheduling that depend on connected UEs' random and dynamic locations, mobility and traffic). As such, the proposed optimizations, in this work, are to be performed in non-real-time. This makes the concerns about associated complexities less relevant. However, for the sake of completeness a time complexity discussion of the proposed methods are provided below.
The optimization variables being in integer spaces, the problems are solved as MILP. Treating MILP as an optimization method, its complexity is NP-hard. However, in numerical solutions, if branch and bound method is applied, each branch can be considered to be solved in polynomial time as they are deterministic decision problems in each branch.
With 234 active cell in initial deployment along with 49876 potential UE locations in an area spanning 1.512 square kilometer, the time required to find optimal solutions, using the same machine, for the three NES solutions are provided in Table \ref{tab:results}. We refrain from reporting time complexity for the local beam-level optimization, as the scale of the problem in local level and network-wide optimization is different.


\begin{table}[]
\begin{center}
%\large
\vspace{10pt}
\begin{tabular}{|c|c|c|c|c|c|}
\cline{1-2} 
\textbf{Strategy} &\textbf{Computation time}\\ 
 &\textbf{(seconds)}\\ \cline{1-2} 
Baseline   & N/A  \\ \cline{1-2}  
%Local beam-level        & 275.10  \\ 
%optimization~(6)       &\\ \cline{1-2}
Global cell-level        & 2.53 \\ 
optimization~(7)          & \\ \cline{1-2}
Global joint beam- and     & 34.25 \\ 
cell-level optimization~(9)       &\\ \cline{1-2}
\end{tabular} 
\vspace{5pt}
\caption{Computation time for different NES solutions with the network under consideration.}
\label{tab:results}
\end{center}
\vspace{-10pt}
\end{table}
\vspace{-5pt}


\section{Impact on UE Operation}
Although the optimization problems are formulated with a coverage constraint to ensure that no coverage hole are created as a results of cell/beam deactivations, they may still impact the operation of idle UEs, as elaborated further below.

\subsubsection{Link SNR}
Deactivating some cells/beams forces the associated UEs to be served by neighboring cells/beams with lower signal quality and strength. This may degrade the operation of the idle UEs in practice, as they may need multiple attempts to successfully receive a DL signal (such as SSB, SIB, or the paging message), or transmit an UL signal (such as PRACH). 
Figure~\ref{fig:dl_ssb_snr} shows the reduction in the SSB SNR of the strongest link associated with each UE after different local/global optimizations. 

\subsubsection{Coverage Diversity} 
To maintain reliability, coverage diversity is a desirable feature where a given UE can be covered by more than one cells and/or beams. However, the proposed strategies, which involve deactivating a subset of cells and/or beams, reduce the level of coverage diversity. 
Figure \ref{fig:coverage_div} shows the reduction in the number of cells each UE can potentially connect to after different optimization strategies discussed in this work. 
The global joint optimization leads to least overall link SNR and coverage diversity owing to most cell/beam deactivation.
%We did not consider diversity in our problem formulations in this paper. 
Devising schemes to find an optimal balance between network energy savings and diversity is left for future work.


\subsubsection{Cell Search}
In idle mode, UEs need to periodically search for and measure neighboring cells and execute cell reselection when necessary to ensure they always have a strong and reliable candidate cell for upcoming connections. 
The cell search process is the main factor in UE energy consumption during idle mode. A positive consequence of cell/beam deactivation is the reduction in the number of candidate cells/beams that idle UEs need to search for. 
Figure~\ref{fig:beam_per_sector} shows a distribution of the number of active beams per cell, which is reduced from the baseline of 32 beams to at most 9 beams. This translates to up to 3X reduction in the UEs' search and measurement window, providing an opportunity for UE energy savings. 
\vspace{-2pt}

%\subsection{Impact on UE Operation}



%\vspace{-10pt}