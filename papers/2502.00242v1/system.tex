\subsection{Local Beam-level Optimization}
As discussed in Section~\ref{sec: SSB prelim}, each cell may adopt a baseline SSB codebook for the transmission and reception of its common channels. 
By leveraging an accurate DT, we can determine the radio coverage region of each cell, and identify the potential location and signal strength of the UEs associated with them, based on a UE-cell association criterion such as the strongest received power or SINR.
This analysis reveals that the coverage region of cells can be highly non-uniform, meaning that that minimum SINR among the associated UEs in different beam directions may vary. For example, there may be blockage in certain directions, preventing UEs from being associated beyond a certain distance, or above a path-loss threshold. Non-uniform coverage can also result from signal strength-based UE-cell association in overlapping coverage areas of neighboring cells. Figure~\ref{fig:min_snr} illustrates such a spatially non-uniform coverage area of a cell from our DT. 

SSB codebook for each cell $c$ can be optimally selected to minimize energy cost while meeting the cell's target coverage region. We assume each cell $c$ has a pool of $N_{B,c}$ beams, including beams of different shapes (and hence different gains) and directions to serve the entire coverage area of its associated $N_{UE,c}$ user locations.
 Given that the energy cost in (\ref{eq:cell_opex}) is an increasing function of the number of beams, minimizing the number of beams for a given cell is sufficient to minimize the energy cost of the associated cell. Considering a beam-level connectivity matrix $\bm{A}_{beam, c}$ of dimension ($N_{B,c}, N_{UE,c}$) for a given cell $c$, the local beam optimization problem is formulated as:
\begin{equation}\label{eq:beam_per_sector}
    \min_{\bm{x}_{b}} \bm{1}^T_{N_{B,c}}\bm{x}_b, ~s.t. ~ {\bm{A}^T_{beam,c}}\bm{x}_b \ge \bm{1}_{N_{UE,c}}
\end{equation}
wherein $\bm{x}_b$ is the binary decision vector, of length $N_{B,c}$, indicating the set of active beams. 
It is important to note that in this variation of the optimization problem, we assume the UE-cell association information is given (e.g., determined based on the DT and highest SINR based cell selection criterion). This means that the same set of UE locations will be covered by a cell, after the beam optimization. 

\subsection{Global Optimization}
Optimizing network energy consumption globally for the idle mode can be performed at two different levels of granularity: cell-level and joint cell- and beam-level. One of the main differences between global and local optimization problems is that, for global optimization, we do not assume any prior UE-cell association; instead, all possible connections are considered as part of the optimization.

\subsubsection{Cell-level}
Let $N_C$ indicate the number of originally deployed and active cells. The global cell-level optimization identifies a subset of these cells that can collectively cover all $N_{UE}$ user locations with minimum total energy cost.

For simplicity, we assume each cell uses a baseline SSB codebook comprising $N_{B,base}$ beams. However, the problem formulation can be straightforwardly extended to support different numbers of baseline beams for different cells. According to~\eqref{eq:cell_opex}, the energy cost of each active cell is $C(N_{B,base})$. Therefore, minimizing the total energy cost is equivalent to minimizing the number of active cells. Using~\eqref{eq:connectivity}, we consider a connectivity matrix ($\bm{A}_{cell}$) of dimension $(N_C, N_{UE})$ between cells and UEs. The global cell-level optimization problem is formulated as:
\begin{equation}\label{eq:problem_sc}
    \min_{\bm{x}} \bm{1}_{N_C}^T\bm{x}, ~s.t. ~ {\bm{A}_{cell}^T}\bm{x} \ge \bm{1}_{N_{UE}}
\end{equation}
wherein  $\bm{x}$ is the binary decision vector, of length $N_C$, indicating the set of active cells.
 
\subsubsection{Joint Cell- and Beam-level} 
The problem formulations so far aim to minimize the total energy costs globally at cell-level and locally by optimizing the beam selection per cell. 
By combining both approaches, a complete solution can be provided by selecting an optimal set of beams (and associated cells) from a network-wide pool of beams, of size $N_C N_B$, that can cover all $N_{UE}$ locations while minimizing total energy cost.
Following~\eqref{eq:connectivity}, we consider a connectivity matrix ($\bm{A}_{beam}$) of dimension $(N_C N_B, N_{UE})$ between all beams and UEs. 
We further define a new binary matrix $\bm{B}$ of dimension ($N_C N_B, N_C$) to indicate the association between beams and cells. Specifically,  $B_{ij}=1$, if the $i^{th}$ beam in the network-wide pool of beams is associated with cell $j$, and zero otherwise.  
The binary vector $\bm{x}$, of length $N_C N_B$, is used to indicate the set of selected active beams. Given these parameters, the total number of active beams in the network is $\bm{1}_{N_CN_B}^T\bm{x}$, and the set of active cells (i.e., cells with at least one active beam) can be represented by the indicator vector $\bm{1}_{\{\bm{B}^T\bm{x}>0\}}$. The number of active cells is thus $\bm{1}_{N_C}^T\bm{1}_{\{\bm{B}^T\bm{x}>0\}}$. 

Using the approximated energy cost function in~\eqref{eq:cell_opex_approx}, the global joint optimization can be formulated as:
\begin{align}\label{eq:nonlin_prob}
    &  \min_{\bm{x}}~ c_{static}\bm{1}_{N_C}^T\bm{1}_{\{\bm{B}^T\bm{x}>0\}}+ m\bm{1}_{N_CN_B}^T\bm{x}, \\
    &s.t. ~ {\bm{A}_{beam}^T}\bm{x} \ge \bm{1}_{N_{UE}} \nonumber
\end{align}

This objective function is not a linear functions of $\bm{x}$ due to the indicator function in the first term of the energy cost.
Theoretically, we can solve this as a non-linear problem. However, given the scale of the problem in large, densified networks, finding an optimal solution may be impractical.

{To linearize the problem, we utilize a relaxation approach from partial set cover problems~\cite{chekuri2019approximatingpartialsetcover}. Typically, such relaxation is applied to linear or integer linear programming where an auxiliary variable is introduced to satisfy additional constraints for the solution space~\cite{Bragin2022}. We can apply the same concept here to replace} the non-linear indicator function $\bm{1}_{\{\bm{B}^T\bm{x}>0\}}$ with an auxiliary binary decision vector $\bm{x}_c$ that indicates which cells are active. We further add a new linear constraint relating $\bm{x}_c$ and $\bm{x}$ as follows. Let $N_b(i)\in[0,N_B]$ denote the number of active beams for cell $i$, i.e., $N_b(i) := \bm{B}^T\bm{x}$. If $N_b(i)=0$ (or $>0$), then we require $\bm{x}_c(i)=0$ (or $1$). These conditions will be met by the optimal $\bm{x}_c$ using the following constraint: $N_B\bm{x}_c \ge \bm{B}^T\bm{x}$. 
 The modified problem is formulated as:

\begin{align}\label{eq:main_problem}   
    &\min_{\bm{x},\bm{x}_c}~ c_{static}\bm{1}_{N_C}^T\bm{x_c} + m\bm{1}_{N_CN_B}^T\bm{x}, \\
    &s.t. ~ {\bm{A}_{beam}^T}\bm{x} \ge \bm{1}_{N_{UE}}, N_b\bm{x}_c \ge \bm{B}^T\bm{x} \nonumber
\end{align}
%\note{Note, that if optimality is not achieved, the second constraint will still act as a loose boundary condition. }