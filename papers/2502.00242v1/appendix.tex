\section{Initial Network Deployment}

{Say, the network, the potential traffic points ($N_{TP}$) are distributed uniformly with a resolution of 1 square meter area for each. Assuming a target throughput of $t$ for each user and a cell can serve $K$ users (termed multiplexing factor) with equal resource scheduling (duty cycle), each user must have a throughput of $tK$. %(assuming SISO, for MIMO the spatial gain must be considered). 
The link SINR are estimated using ray-tracing to develop the DT. Based on this target throughput an SINR threshold (SINR$_{th}$) is calculated. Thus for a given $K$, the target throughput and SINR$_{th}$ is different indicating different number of traffic points with link SINR exceeding SINR$_{th}$ defined by the function $P(K)$.  The activity factor $a$ is the percentage of traffic points that have an active UE. The number of traffic points that can be supported with activity factor $a$ and multiplexing factor $K$ is given by $Q(K) = K/a$. 
The optimal multiplexing factor that yields the highest number of supported traffic points for a cell is given by$K^{*} =  \max_{k} (\min(P(K), Q(K)))$, from which we obtain the updated SINR threshold based on $K^{*}$ for a given cell. 
Based on this, a binary connectivity matrix $\bm{A}_{dep}$ of dimension $(N_{TP},N_p)$, where $A_{ij} = 1$ if cell $j$ and TP $i$ has a link exceeding the SINR threshold estimated for that cell.}
The general optimization problem is formulated as: 
\begin{align}\label{eq:deployment}
   \min_{\bm{x}_p} {\bm{1}^T_{N_p}}{\bm{x}_p} , ~s.t. ~ \bm{A}_{dep}\bm{x}_p \ge \bm{y}, ~ {\bm{1}^T_{N_{TP}}}\bm{y} \ge \alpha
\end{align}
{where $\bm{y}$ is a binary vector of length $N_{TP}$ with $y_i$ indicating TP $i$ is covered and $\alpha$ is a predetermined \% target throughput coverage for the network deployment.
We encourage the reader to review ~\cite{deployement_paper} for a more in-depth discussion and analysis optimization method adopted to solve this deployment problem.}

\section{Impact on UE Operation}
Although the optimization problems are formulated with a coverage constraint to ensure that no coverage hole are created as a results of cell/beam deactivations, they may still impact the operation of idle UEs, as elaborated further below.

\subsubsection{Link SNR}
Deactivating some cells/beams forces the associated UEs to be served by neighboring cells/beams with lower signal quality and strength. This may degrade the operation of the idle UEs in practice, as they may need multiple attempts to successfully receive a DL signal (such as SSB, SIB, or the paging message), or transmit an UL signal (such as PRACH). 
Figure~\ref{fig:dl_ssb_snr} shows the reduction in the SSB SNR of the strongest link associated with each UE after different local/global optimizations. 

\subsubsection{Coverage Diversity} 
To maintain reliability, coverage diversity is a desirable feature where a given UE can be covered by more than one cells and/or beams. However, the proposed strategies, which involve deactivating a subset of cells and/or beams, reduce the level of coverage diversity. 
Figure \ref{fig:coverage_div} shows the reduction in the number of cells each UE can potentially connect to after different optimization strategies discussed in this work. 
We did not consider diversity in our problem formulations in this paper. Devising schemes to find an optimal balance between network energy savings and diversity is left for future work.

The global joint optimization leads to least overall link SNR and coverage diversity owing to most cell/beam deactivation.
\subsubsection{Cell Search}
In idle mode, UEs need to periodically search for and measure neighboring cells and execute cell reselection when necessary to ensure they always have a strong and reliable candidate cell for upcoming connections. 
The cell search process is the main factor in UE energy consumption during idle mode. A positive consequence of cell/beam deactivation is the reduction in the number of candidate cells/beams that idle UEs need to search for. 
Figure~\ref{fig:beam_per_sector} shows a distribution of the number of active beams per cell, which is reduced from the baseline of 32 beams to at most 9 beams. This translates to up to 3X reduction in the UEs' search and measurement window, providing an opportunity for UE energy savings. 



\subsection{On Complexity of the Solution}
It is important to note that the DT being available, the optimization can be performed at the deployment stage or periodically and no real-time optimization is required making the concern of complexity moot. However, for the sake of completeness a time complexity discussion of the proposed methods are provided below.
The optimization variables being in integer spaces, the problems are solved as MILP. Treating MILP as an optimization method, its complexity is NP-hard. However, in numerical solutions, if branch and bound method is applied, each branch can be considered to be solved in polynomial time as they are deterministic decision problems in each branch.
With 234 active cell in initial deployment along with 49876 potential UE locations in an area spanning 1.512 square kilometer, the time required to find optimal solutions using the three NES solutions are provided in~\S\ref{tab:results}.


\begin{table}[]
\begin{center}
%\large
\vspace{10pt}
\begin{tabular}{|c|c|c|c|c|c|}
\cline{1-2} 
\textbf{Strategy} &\textbf{Computation time}\\ 
 &\textbf{(seconds)}\\ \cline{1-2} 
Baseline   & N/A  \\ \cline{1-2}  
Local beam-level        & 275.10  \\ 
optimization~\eqref{eq:beam_per_sector}         &\\ \cline{1-2}
Global cell-level        & 2.53 \\ 
optimization~\eqref{eq:problem_sc}          & \\ \cline{1-2}
Global joint beam- and     & 34.25 \\ 
cell-level optimization~\eqref{eq:main_problem}       &\\ \cline{1-2}
\end{tabular} 
\vspace{5pt}
\caption{Computation time for different NES solutions with the network under consideration.}
\label{tab:results}
\end{center}
\vspace{-10pt}
\end{table}
\vspace{-5pt}