\section{Preliminaries}
\label{sec:prelim}
\subsection{SSB Codebook} \label{sec: SSB prelim}
In 5G, signal transmission and reception can be directed towards specific beam directions. Broadcast control channels, such as the Synchronization Signal Block (SSB), System Information Blocks (SIB), and Physical Random Access Channel (PRACH), may be supported at multiple instances with different beam directions to ensure coverage of entire cell area.

The set of beams associated with broadcast control channels is sometimes referred to as the SSB codebook. According to 3GPP specifications, it is possible to configure up to 8 SSBs in frequency ranges below 6 GHz and up to 64 SSBs in the millimeter-wave frequency range. However, the design of the SSB codebook is determined by the network implementation, which decides the number of SSB beams (within the specified limits), as well as their azimuth and elevation spans and beam directions.
Figure~\ref{fig:baseline_codebook} illustrates example of an SSB codebook comprising 32 beams that uniformly divide an angular range of 120 degrees in azimuth and 30 degrees in elevation.
\vspace{-10pt}
\subsection{Energy Consumption Model}
In this work, the energy consumption model adopted for gNB is based on the 3GPP specifications \cite{3gpp.38.864}. The model includes various inactive and active modes, featuring three sleep levels (micro, light, and deep) for the inactive mode, as well as active downlink (DL) and active uplink (UL) states. For the active states, the power expenditure (P) of the gNB consists of two distinct parts:
\smallskip
\begin{equation}\label{eq:power_model}
    P = P_{static} + P_{dynamic} 
\end{equation}
The static component ($P_{static}$) is present regardless of the gNB's configuration, and is assumed to have the same value as the micro-sleep power consumption. The dynamic component ($P_{dynamic}$) 
 depends on the gNB's configuration, such as antenna configuration, transmit power, and occupied bandwidth. In reality, a gNB can comprise multiple cells. However, for simplicity in this work, we assume each gNB has one cell and hence may use the term cell and gNB interchangeably. 
 
Based on this model, we calculate a cell's energy cost in idle mode, where it transmits SSBs and SIBs, and monitors for PRACH with a given periodicity (e.g., 20 msec).
For a given cell, with $N_b$ number of active beams, the energy cost can be represented as:
\begin{equation}\label{eq:cell_opex}
    C(N_b) = 1_{\{N_b>0\}}c_{\textrm{static}} + f(N_b)
\end{equation}
where, $1_{\{X\}} = 1 ~\textrm{if}~ X$ is TRUE and $0$ otherwise; $c_{\textrm{static}}$ is a constant energy cost per cell and $f(x)$ is the additional energy cost of supporting $x \ge 1$ beams in the SSB codebook.

Figure~\ref{fig:sample_ssb_period} shows energy cost $C(N_b)$ for various number of beams $N_b$, and assuming 20 msec periodicity of the control channels. 
More information about the other assumptions for this analysis is provided in~\S\ref{sec:sim}.
From this figure, we can approximate $f(N_b)$ as a linear function. Hence:
\begin{equation}\label{eq:cell_opex_approx}
    C(N_b) \approx 1_{\{N_b>0\}}c_{\textrm{static}} + mN_b
\end{equation}
where $m$ is the approximate cost of each active beam.

\begin{figure}
\centering
\includegraphics[width=0.8\linewidth]{figs/beam_cost.png}
\caption{Cost of operation for a cell in idle mode as a function of number of SSB beams.}
    \label{fig:sample_ssb_period}
    \vspace{-20pt}
\end{figure}


\section{System-level Network Energy Optimization}
\label{sec:system}

\subsection{Baseline Network Deployment}
An initial deployment of the cell sites is selected to meet a \textit{target throughput} for a peak UE load.
Specifically, we assume that in the geographical area under consideration, there are potential active UE locations (associated with all possible outdoor locations) 
and candidate cell sites (associated with available poles in the area). 
The goal of the initial network deployment problem is to choose the optimal cell sites 
that satisfy the \textit{target throughput} requirement during peak load time, while minimizing the total deployment cost. 
Each cell site is assumed to comprise three sectors (or cells) that divide the site's coverage region. 
This problem is solved based on an a priori signal-to-interference-plus-noise ratio (SINR) estimation, which dictates the datarate a serving cell can provide to a user. %The UE-cell association is based the strongest link with highest SINR. 
%Additional details of the network deployment problem formulation and solution is provided in~\cite{nesFulldocument}

{For the distribution of the users, we assume $N_{TP}$ potential traffic points (user locations) are distributed uniformly with a resolution of 1 square meter area for each in the outdoor areas and $N_p$ number of pole locations are identified as potential cell sites. To meet a target throughput of $r_t$ %\navid{NAVID: please change the notation for t} 
for each user and assuming a cell can serve $K$ users (termed multiplexing factor) with equal resource (duty cycle) scheduling, each user must have an achievable channel rate of $r_tK$. Based on the target throughput, an SINR threshold SINR$_{th}(K)$ can be obtained as a function of $K$.  %(assuming SISO, for MIMO the spatial gain must be considered). 
The link SINR between each cell and traffic point is estimated using ray-tracing as well as an estimated interference margin. 
For a given candidate cell and multiplexing factor $K$, the number of surrounding traffic points with link SINR to the cell exceeding SINR$_{th}(K)$ is denoted by $P(K)$. 
The activity factor $a$ is defined as the percentage of traffic points with an active UE. The number of traffic points that can be supported with activity factor $a$ and multiplexing factor $K$ is given by $Q(K) = K/a$. 
Therefore, the optimal multiplexing factor that yields the highest number of supported traffic points for a cell is given by $K^{*} =  \arg\max_{k} (\min(P(K), Q(K)))$, from which we obtain the selected SINR threshold SINR$_{th}(K^{*})$. 
Based on selected SINR thresholds for all candidate cell locations, a binary connectivity matrix $\bm{A}_{dep}$ of dimension $(N_{TP},N_p)$ is defined, %\wenjun{we may also consider changing $N_p$ notation, since it is not clear it refers to pole} 
where $A_{ij} = 1$ if candidate cell $j$ and TP $i$ has a link exceeding the SINR threshold estimated for that cell.}
The general optimization problem is formulated as minimizing the number of deployed cells such that a given percentage $\alpha$ of traffic points (under the assumed activity factor) achieve the target throughput: 
\begin{align}\label{eq:deployment}
   \min_{\bm{x}_p} {\bm{1}^T_{N_p}}{\bm{x}_p} , ~s.t. ~ \bm{A}_{dep}\bm{x}_p \ge \bm{y}, ~ {\bm{1}^T_{N_{TP}}}\bm{y} \ge \alpha
\end{align}
{where $\bm{y}$ is a binary vector of length $N_{TP}$ with $y_i$ indicating TP $i$ is covered and $\alpha$ is a predetermined \% target throughput coverage for the network deployment. 
We encourage the reader to review ~\cite{deployement_paper} for a more in-depth discussion on the optimization method adopted to solve this deployment problem.}

While the selection of deployed cell sites 
aims to satisfy \textit{capacity} requirements during the peak times, the network should primarily ensure a minimum \textit{coverage} for UEs during off-peak hours. 
The \textit{coverage} requirement is defined in terms of the necessary link budget for broadcast control channels, such as SSB, SIB, PRACH, which is typically less stringent than the SINR requirements for data traffic (associated with \textit{capacity} needs).
Therefore, although the initially selected cell sites can provide coverage to $N_{UE}$ users (or more accurately, $N_{UE}$ user locations in the area), fewer network resources (e.g., cells, or beams) may be sufficient to maintain the same \textit{coverage} during off-peak hours. This presents an opportunity for network energy savings by activating only subset of cells and/or beams during these times.

A UE $j$ is considered to be covered by a cell $i$ (or a beam $i$), if the associated SINR meets an SSB SINR threshold: $\textrm{SINR}_{ij} \ge \textrm{SINR}_{th}$. 
Accordingly, we construct a connectivity matrix ($\bm{A}$) to represent the coverage: 
\begin{align}\label{eq:connectivity}
    A_{ij} &= 1~~\textrm{if}~ \textrm{SINR}_{ij}>\textrm{SINR}_{th} \\
             &= 0~~\textrm{otherwise.} \nonumber 
\end{align}
The general network energy saving problem in idle mode can be formulated as follows,
\begin{equation}\label{eq:high-level}
    \min_{\bm{x}} cost(\bm{x}), ~s.t. ~ {\bm{A}^T}\bm{x} \ge \bm{1}_{N_{UE}}
\end{equation}
in which, $\bm{x}$ is a binary vector whose $i^{th}$ element indicates whether a cell $i$ (or beam $i$) is active, $\bm{1}_{N}$ is an all-one vector of length $N$, and $cost(\bm{x})$ is the energy cost associated with the decision vector $\bm{x}$.

The optimization problem (\ref{eq:high-level}) can be formulated and solved at different levels of granularity: 1) local beam (de)activation (i.e., selecting a set of active beams for each cell independently), 2) global cell (de)activation (i.e., selecting a set of active cells), and 3) global cell and beam (de)activation (i.e., selecting a set of active cells and the active beams within them). Next, we elaborate on each of these optimization strategies, which are also outlined in Table~\ref{tab:nes_strat}.  

\begin{table}[]
\begin{center}
%\large
\vspace{10pt}
\begin{tabular}{|c|c|}
\cline{1-2} 
\textbf{Strategy}        & \textbf{Description}      \\ \cline{1-2} 
  \multicolumn{2}{|c|}{\textbf{Local Solution}}                \\ \cline{1-2}  
Beam Level          & Optimizing SSB codebook per cell \\
\cline{1-2} 
  \multicolumn{2}{|c|}{\textbf{Global Solution}}                \\ \cline{1-2}  
Cell level &   Optimizing the set of active  
  cells, \\
  & each cell using the baseline SSB codebook \\
\cline{1-2}
 Beam and cell level   &   Jointly optimizing the set of active cells    \\
&  and SSB codebook for each active cell  \\
\cline{1-2}
\end{tabular}
\vspace{5pt}
\caption{NES strategies in idle mode.}
\label{tab:nes_strat}
\end{center}
\vspace{-25pt}
\end{table}
