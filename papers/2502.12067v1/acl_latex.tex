% This must be in the first 5 lines to tell arXiv to use pdfLaTeX, which is strongly recommended.
\pdfoutput=1
% In particular, the hyperref package requires pdfLaTeX in order to break URLs across lines.

\documentclass[11pt]{article}
\usepackage[dvipsnames]{xcolor}
% Change "review" to "final" to generate the final (sometimes called camera-ready) version.
% Change to "preprint" to generate a non-anonymous version with page numbers.
\usepackage[preprint]{acl}

% Standard package includes
\usepackage{times}
\usepackage{latexsym}

% For proper rendering and hyphenation of words containing Latin characters (including in bib files)
\usepackage[T1]{fontenc}
% For Vietnamese characters
% \usepackage[T5]{fontenc}
% See https://www.latex-project.org/help/documentation/encguide.pdf for other character sets

% This assumes your files are encoded as UTF8
\usepackage[utf8]{inputenc}

% This is not strictly necessary, and may be commented out,
% but it will improve the layout of the manuscript,
% and will typically save some space.
\usepackage{microtype}

% This is also not strictly necessary, and may be commented out.
% However, it will improve the aesthetics of text in
% the typewriter font.
\usepackage{inconsolata}

\usepackage{hyperref}
\usepackage{url}
\newcommand{\explain}[2]{\underbrace{#1}_{{\footnotesize\raggedright #2}}}

% for text examples
\newcommand{\delete}[1]{\{\textit{\sout{#1}}\}}
\usepackage[most]{tcolorbox}
\newcommand{\sdelete}[1]{\textit{\sout{#1}}}
\colorlet{lightSalmon}{Salmon!80}
\newcommand{\colorize}[2]{\colorbox{lightSalmon!#1!white}{\strut #2}}

% for figures
\usepackage{graphicx}
\usepackage{subfigure}
\usepackage{wrapfig}

% for equation
\usepackage{amsmath}
\usepackage{amstext}
\usepackage{amsfonts}
\usepackage{bm}

\usepackage{bbm}
% for table
\usepackage{multirow}
\usepackage{booktabs}
\usepackage{array}
\usepackage{caption}
\usepackage{multirow}
\usepackage{booktabs}
\usepackage{array}
\usepackage{caption}
\usepackage{color}
\usepackage{colortbl}
\usepackage{tablefootnote}
\usepackage{adjustbox}
\newcommand{\cred}[1]{\textcolor{red}{$_{#1}$}}

% for algorithm
\usepackage{caption}
\usepackage{algorithm}
\usepackage{algpseudocode}
\newcommand{\mycolor}[1]{\textcolor[RGB]{64,101,149}{#1}}
\newcommand{\mydarkcolor}[1]{\textcolor[RGB]{64,101,149}{#1}}
\algnewcommand{\LineComment}[1]{\Statex ~~~~~~\textsc{//}~\textit{#1}}

%for itemize
\usepackage{enumitem}
\setenumerate[1]{itemsep=0pt,partopsep=0pt,parsep=\parskip,topsep=5pt}
\setitemize[1]{itemsep=0pt,partopsep=0pt,parsep=\parskip,topsep=5pt}
\setdescription{itemsep=0pt,partopsep=0pt,parsep=\parskip,topsep=5pt}

% for highlight
\usepackage{soul}

% for taxonomy
\usepackage{tikz}
\usepackage[edges]{forest}
\definecolor{hidden-draw}{RGB}{64,101,149}
% \definecolor{hidden-pink}{RGB}{255,245,247}
\definecolor{hidden-pink}{RGB}{231,239,250}

% for notation
\usepackage[mathscr]{euscript}
\newcommand{\M}{\mathcal{M}}

%for itemize
\usepackage{amssymb}  
\usepackage{pifont}
\newcommand{\cmark}{\ding{51}}
\newcommand{\xmark}{\ding{55}}
\newcommand{\greenyes}{\textcolor{green}{\ding{51}}}
\newcommand{\redno}{\textcolor{red}{\ding{55}}}

% for title
\usepackage{xspace}

\newcommand{\method}{\texttt{TokenSkip}\xspace}

%% === commands for comments ===
\usepackage{ulem}

% If the title and author information does not fit in the area allocated, uncomment the following
%
%\setlength\titlebox{<dim>}
%
% and set <dim> to something 5cm or larger.

\title{
\texorpdfstring{\includegraphics[width=18pt]{fig/arrow.png}}{}
\method: Controllable Chain-of-Thought Compression in LLMs
}

\author{{Heming Xia}\textsuperscript{\rm 1}, {\textbf{Yongqi Li}}\textsuperscript{\rm 1}\thanks{Corresponding Author}, {\textbf{Chak Tou Leong}}\textsuperscript{\rm 1}, {\textbf{Wenjie Wang}}\textsuperscript{\rm 2}, {\textbf{Wenjie Li}}\textsuperscript{\rm 1}\\
  \textsuperscript{\rm 1}Department of Computing, The Hong Kong Polytechnic University \\
  \textsuperscript{\rm 2}University of Science and Technology of China \\
  {\tt \{he-ming.xia, chak-tou.leong\}@connect.polyu.hk}
}

\begin{document}
\maketitle
\begin{abstract}
Chain-of-Thought (CoT) has been proven effective in enhancing the reasoning capabilities of large language models (LLMs). Recent advancements, such as OpenAI's o1 and DeepSeek-R1, suggest that scaling up the length of CoT sequences during inference could further boost LLM reasoning performance. However, due to the autoregressive nature of LLM decoding, longer CoT outputs lead to a linear increase in inference latency, adversely affecting user experience, particularly when the CoT exceeds 10,000 tokens. To address this limitation, we analyze the semantic importance of tokens within CoT outputs and reveal that their contributions to reasoning vary. Building on this insight, we propose \method, a simple yet effective approach that enables LLMs to selectively skip less important tokens, allowing for controllable CoT compression. Extensive experiments across various models and tasks demonstrate the effectiveness of \method in reducing CoT token usage while preserving strong reasoning performance. Notably, when applied to Qwen2.5-14B-Instruct, \method reduces reasoning tokens by $40\%$ (from 313 to 181) on GSM8K, with less than a $0.4\%$ performance drop. We release our code and checkpoints in \url{https://github.com/hemingkx/TokenSkip}.
\end{abstract}

\section{Introduction}

Deep Reinforcement Learning (DRL) has emerged as a transformative paradigm for solving complex sequential decision-making problems. By enabling autonomous agents to interact with an environment, receive feedback in the form of rewards, and iteratively refine their policies, DRL has demonstrated remarkable success across a diverse range of domains including games (\eg Atari~\citep{mnih2013playing,kaiser2020model}, Go~\citep{silver2018general,silver2017mastering}, and StarCraft II~\citep{vinyals2019grandmaster,vinyals2017starcraft}), robotics~\citep{kalashnikov2018scalable}, communication networks~\citep{feriani2021single}, and finance~\citep{liu2024dynamic}. These successes underscore DRL's capability to surpass traditional rule-based systems, particularly in high-dimensional and dynamically evolving environments.

Despite these advances, a fundamental challenge remains: DRL agents typically rely on deep neural networks, which operate as black-box models, obscuring the rationale behind their decision-making processes. This opacity poses significant barriers to adoption in safety-critical and high-stakes applications, where interpretability is crucial for trust, compliance, and debugging. The lack of transparency in DRL can lead to unreliable decision-making, rendering it unsuitable for domains where explainability is a prerequisite, such as healthcare, autonomous driving, and financial risk assessment.

To address these concerns, the field of Explainable Deep Reinforcement Learning (XRL) has emerged, aiming to develop techniques that enhance the interpretability of DRL policies. XRL seeks to provide insights into an agent’s decision-making process, enabling researchers, practitioners, and end-users to understand, validate, and refine learned policies. By facilitating greater transparency, XRL contributes to the development of safer, more robust, and ethically aligned AI systems.

Furthermore, the increasing integration of Reinforcement Learning (RL) with Large Language Models (LLMs) has placed RL at the forefront of natural language processing (NLP) advancements. Methods such as Reinforcement Learning from Human Feedback (RLHF)~\citep{bai2022training,ouyang2022training} have become essential for aligning LLM outputs with human preferences and ethical guidelines. By treating language generation as a sequential decision-making process, RL-based fine-tuning enables LLMs to optimize for attributes such as factual accuracy, coherence, and user satisfaction, surpassing conventional supervised learning techniques. However, the application of RL in LLM alignment further amplifies the explainability challenge, as the complex interactions between RL updates and neural representations remain poorly understood.

This survey provides a systematic review of explainability methods in DRL, with a particular focus on their integration with LLMs and human-in-the-loop systems. We first introduce fundamental RL concepts and highlight key advances in DRL. We then categorize and analyze existing explanation techniques, encompassing feature-level, state-level, dataset-level, and model-level approaches. Additionally, we discuss methods for evaluating XRL techniques, considering both qualitative and quantitative assessment criteria. Finally, we explore real-world applications of XRL, including policy refinement, adversarial attack mitigation, and emerging challenges in ensuring interpretability in modern AI systems. Through this survey, we aim to provide a comprehensive perspective on the current state of XRL and outline future research directions to advance the development of interpretable and trustworthy DRL models.
%!TEX root = Article.tex

% Begin of file 2-Preliminaries.tex

\section{Foundations}
\label{sec:prl}

In this section we present some material that we will need in the subsequent
sections, and define a data model that consists of common aspects of RDF and
Property Graphs.


\subsection{A Common Data Model}

When developing a common framework for SHACL, ShEx, and PG-Schema, the first
challenge is establishing  a \emph{common data model}, since SHACL and ShEx work
on RDF, whereas PG-Schema works on Property Graphs.
Rather than using a model that generalises  both RDF and Property Graphs, we
propose a simple model, called \emph{common graphs}, which we obtained by asking
what, fundamentally, are the \emph{common aspects} of RDF and Property Graphs
(Appendix~\ref{sec:appendix-foundations} gives more details on the distilling of
common graphs).

Let us assume disjoint countable sets of nodes $\Nodes$, values $\Values$,
predicates $\Predicates$, and keys $\Keys$ (sometimes called properties).

% We sometimes say \emph{element} for a node or a value, and \emph{label} for a predicate or key. \todo{Drop if not used.}

\begin{definition}
  A \emph{common graph} is a pair $\graph = (E, \rho)$ where
  \begin{itemize}[\textbullet]
  \item
    $E \subseteq_{\mathit{fin}} \Nodes \times \Predicates \times \Nodes$ is its
    set of edges (which carry predicates), and
  \item
    $\rho \colon \Nodes \times \Keys \pto \Values$ is a finite-domain partial
    function mapping node-key pairs to values.
  \end{itemize}
  The set of nodes of a common graph $\graph$, written $\nodes(\graph)$,
  consists of all elements of $\Nodes$ that occur in $E$ or in the domain of
  $\rho$.
  Similarly, $\keys(\graph)$ is the subset of $\Keys$ that is used in $\rho$,
  and $\values(\graph)$ is the subset of $\Values$ that is used in $\rho$ (that
  is, the range of $\rho$).
\end{definition}

% \begin{example}[Media Service Common Graph] \label{ex:sharedScenario}
% To illustrate the common graphs, we introduce the following scenario. We assume a data model that has users, who can access and own accounts and invite other users to their accounts. Users have keys, such as email and credit-card. An example for this can be seen in~\Cref{fig}.
% % The nodes correspond to  conceptual classes, which will be identified by their available properties and keys. Properties are depicted as directed arrows, and keys are shown inside the conceptual classes.
% % The boxes inform about the available categories of nodes, with the keys they may have available (such as the key $\Exkey{plan}$ for nodes of category ``Account''), and properties connect nodes via directed arrows (such as $\Exprop{buyer}$, which connects nodes of category ``Sale'' and ``Account'').
% \end{example}

\begin{example}
  \label{ex:common-graph}
  Consider Figure~\ref{fig:common-graph}, containing a graph to store
  information about \emph{users} who may have access to (possibly multiple)
  \emph{accounts} in, \eg, a media streaming service.
  In this example, we have six nodes describing four persons ($u_1,...,u_4$) and
  two accounts ($a_1$, $a_2$).
  As a common graph $\graph = (E, \rho)$, the nodes are $a_1$, $u_1$, etc.
  Examples of edges in $E$ are $(u_2, \exaccess,a_1)$ and $(u_3, \exinvited,
  u_2)$.
  Furthermore, we have $\rho(u_2, \exemail) =$ d@d.d and $\rho(a_1,card) =
  1234$.
  So, $E$ captures the arrows in the figure (labelled with predicates) and
  $\rho$ captures the key/value information for each node.
  %
% Moreover, 3 predicates are used, appearing in Figure~\ref{fig:common-graph} as labels on links between nodes, representing the relation~$E$. Nodes are further associated with some key-value pairs, representing the function $\rho$.
  %
  Notice that a person may be the owner of an account, and may potentially have
  access to other accounts.
  This is captured using the predicates $\exowns$ and $\exaccess$, respectively.
  In addition, the system implements an invitation functionality, where users
  may invite other people to join the platform.
  The previous invitations are recorded using the predicate $\exinvited$.
  Both accounts and users may be privileged, which is stored via a Boolean value
  of the key~$\exprivileged$.
  We note that the presence of the key $\exemail$ (\resp, of the key (credit)
  $\excard$) is associated with, and indeed identifies users (\resp, accounts).
\end{example}

% \todo[inline]{In the example, worth noting that the graph node names are names, and not identities. Maybe it would be better to name them A, B, C, D to avoid misunderstanding?}

\begin{figure}[t]
\resizebox{1\linewidth}{!}{
  \includegraphics{example-common.pdf}
}
\Description{A diagram of the user common graph.}
\caption{The media service common graph. }
\label{fig:common-graph}
\end{figure}

It is easy to see that every common graph is a property graph (as per the formal
definition of property graphs~\cite{ABDF23}).
A common graph can also be seen as a set of triples, as in RDF.
Let
\[
  \Triples
=
  \left( \Nodes \times \Predicates \times \Nodes \right)
\;\cup\;
  \left( \Nodes \times \Keys \times \Values \right)\,.
\]
Then, a common graph can be seen as a finite set $\graph \subseteq \Triples$
such that for each $u \in \Nodes$ and $k \in \Keys$ there is at most one
$v \in \Values$ such that $(u, k, v) \in \graph$.
Indeed, a common graph $(E, \rho)$ corresponds to
\[
  E \;\cup\; \{ (u, k, v) \mid \rho(u,k) = v\}\;.
\]
When we write $\rho(u, k) = v$ we assume that $\rho$ is defined on $(u, k)$.

\medskip

\noindent\emph{Throughout the paper we see property graph $\graph$
simultaneously as a pair $(E, \rho)$ and as a set of triples from $\Triples$,
switching between these perspectives depending on what is most convenient at a
given moment.}


\subsection{Node Contents and  Neighbourhoods}

Let $\Records$ be the set of all \emph{records}, \ie, finite-domain partial
functions $r \colon \Keys \pto \Values$.
We write records as sets of pairs $\left\{ (k_1, w_1), \dots (k_n, w_n)
\right\}$ where $k_1, \dots, k_n$ are all different, meaning that $k_i$ is
mapped to $w_i$.

For a common graph $\graph = (E,\rho)$ and node $v$ in $\graph$, by a slight
abuse of notation we write $\rho(v)$ for the record $\left\{ (k, w) \mid
\rho(v,k) = w \right\}$ that collects all key-value pairs associated with node
$v$ in $\graph$.
We call $\rho(v)$ the \emph{content} of node $v$ in $\graph$.
This is how PG-Schema interprets common graphs: it views key-value pairs in
$\rho(v)$ as \emph{properties} of the node $v$, rather than independent,
navigable objects in the graph.

SHACL and ShEx, on the other hand, view common graphs as sets of triples and
make little distinction between keys and predicates.
The following notion---when applied to a node---uniformly captures the local
context of this node from that perspective: the content of the node and all
edges incident with the node.

%\begin{definition}[Neighbourhood]
%Given a common graph $\graph = (E,\rho)$ and a node $v\in\Nodes$, we write $\neigh_\graph(v)$ for the common graph $(E',\rho')$ where $E' = \left \{ (u_1,p,u_2) \in  E \mid u_1 = v \text{ or } u_2 = v\right\}$ and $\rho'$ is obtained by restricting $\rho$ so that $\rho'(v) = \rho(v)$ and $\rho'(u)$ is empty for all $u\neq v$. Similarly, for $w\in\Values$, we let $\neigh_\graph(w)$ be the common graph $(\emptyset,\rho')$ where $\rho'(u) = \left\{(k,w')\in\rho(u)\mid w'=w\right\}$ for all $u\in\Nodes$.
%Given a common graph $\graph$ and a node or value $v\in\Nodes\cup\Values$, the \emph{neighbourhood of $v$ in $\graph$}, written $\neigh_\graph(v)$, is the common graph consisting of triples $(u_1, p, u_2)$ from $\graph$ such that $p\in\Predicates\cup\Keys$ and either $u_1=v$ or $u_2=v$.
%\end{definition}

%That is, for $v\in\Nodes$,  $\neigh_\graph(v)$ is a star-shaped graph where only the central node has non-empty content.  For $w\in\Values$, $\neigh_\graph(w)$ is a graph with no edges and only a single value occurring in the contents of nodes.

%If we view common graphs as sets of triples, $\neigh_\graph(v)$ for $v\in\Nodes\cup\Values$ is simply the set of all triples from $\graph$ that mention $v$.

%We will also use the notion of \emph{partial neighbourhoods}, where only specified subsets of keys and predicates are taken into account.

%It is easiest to define it seeing common graphs as sets of triples.

\begin{definition}[Neighbourhood]
  Given a common graph $\graph$ and a node or value $v \in \Nodes \cup \Values$,
  the \emph{neighbourhood} of $v$ in $\graph$ is $\neigh_\graph(v) = \left\{
  (u_1, p, u_2) \in \graph \mid u_1 = v \text{ or } u_2 = v \right\}$.
  %
% \todo[inline]{Wim: This is ill-defined. We do say before that a common graph can be viewed as a set of triples if we want to think about it as RDF. But this definition should also apply to the PG view. We should be clearer about what we mean with the key/value pairs and only use ingredients from Def 1. In fact, if we take the RDF view, the definition is inconsistent with text below that says that, if $v$ is a value, then the neighborhood has no edges.}
% \todo[inline]{Suggestion to rephrase: introduce $\graph = (E,\rho)$ and say $\neigh_\graph(v) = \{(u_1,p,u_2) \in E \mid ... \} \cup \{???\}$ (Actually I don't understand yet what we want wrt $\rho$.)}
% \todo[inline]{Filip: In many places in the paper we treat $\graph$ as a pair $(E,\rho)$ or as a subset of $\Triples$, whatever is more convenient. It should suffice to warn the reader that we do this. We could write the definition in terms of $(E,\rho)$, but it would be clumsy. I really think it is fine as written.  On the other hand, if this is not helping, we can probably just skip this definition entirely and introduce only the $\pm$ variant of neighbourhoods in the section on ShEx.}
% \todo[inline]{Wim: OK, I understand better now what's intended and clarified below.}
\end{definition}

\todo{JH: Is this actually used anywhere?}

When $v \in \Nodes$, then $\neigh_\graph(v)$ is a star-shaped graph
where only the central node has non-empty content.
When $v \in \Values$, then $\neigh_\graph(v)$ consists of all the nodes in
$\graph$ that have some key with value $v$, which is a common graph with no
edges and a restricted function $\rho$.

%\todo[inline]{Maybe move to respective sections. Could also save space.}


\subsection{Value Types}

We assume an enumerable set of \emph{value types} $\ValueTypes$.
The reader should think of value types as \texttt{integer}, \texttt{boolean},
\texttt{date}, \etc
Formally, for each value type $\vtype \in \ValueTypes$, we assume that there is
a set $\sem{\vtype} \subseteq \Values$ of all values of that type and that each
value $v \in \Values$ belongs to some type, \ie, there is at least one $\vtype
\in \ValueTypes$ such that $v \in \sem{\vtype}$.
Finally, we assume that there is a type $\any \in \ValueTypes$ such that
$\sem{\any} = \Values$.


\subsection{Shapes and Schemas}
\label{ssec:shapes}

We formulate all three schema languages using \emph{shapes}, which are unary
formulas describing the graph's structure around a \emph{focus} node or a value.
Shapes will be expressed in different formalisms, specific to the schema
language; for each of these formalisms we will define when a focus node or value
$v \in \Nodes \cup \Values$ \emph{satisfies} shape $\varphi$ in a common graph
$\graph$, written $\graph, v \models \varphi$.

Inspired by ShEx \emph{shape maps}, we abstract a schema $\schema$ as a set of
pairs $(\sel,\varphi)$, where $\varphi$ is a shape and $\sel$ is a
\emph{selector}.
A selector is also a shape, but usually a very simple one, typically checking
the presence of an incident edge with a given predicate, or a property with a
given key.
A graph $\graph$ is \emph{valid} \wrt $\schema$, in symbols $\graph \models
\schema$, if
\[
  \graph, v \models \sel
\quad \text{implies} \quad
  \graph, v \models \varphi,
\]
for all $v \in \Nodes \cup \Values$ and $(\mathit{sel}, \varphi) \in \schema$.
That is, for each focus node or value satisfying the selector, the graph around
it looks as specified by the shape.
We call schemas $\schema$ and $\schema'$ \emph{equivalent} if $\graph \models
\schema$ \iff $\graph \models \schema'$, for all $\graph$.
In what follows, we may use $\mathit{sel} \Rightarrow \varphi$ to indicate a
pair $(\mathit{sel}, \varphi)$ from a schema $\SHACLSchema$.

% \begin{example}[Schemas over Media Service Common Graph]
%     \label{ex:ShapeExample}

% We stay in the same scenario introduced in \Cref{ex:sharedScenario}. We list here illustrative examples for requirements on common graphs that can be imposed via schemas.  To give an intuitive idea about the selector and the shape, we indicate this informally by splitting the sentences into an initial part that selects nodes or values, and the second part which must hold for these elements:\\
% \noindent
% \emph{For every account}, there must exist a primary credit card ; \\
% \noindent \emph{For every account}, there are  five users of it or less;\\
% \emph{Every owner of an account}, has a unique email address.
% \end{example}

\begin{example}
  \label{ex:constraint-desc}
  We next describe some constraints one may want to express in the domain of
  Example~\ref{ex:common-graph}.
  \begin{enumerate}[(C1)]
  \item
    We may want the values associated to certain keys to belong to concrete
    datatypes, like strings or Boolean values.
    In our example, we want to state that the value of the key $\excard$ is
    always an integer.
  \item
    We may expect the existence of a value associated to a key, an outgoing
    edge, or even a complex path for a given source node.
    For our example, we require that all owners of an account have an email
    address defined.
  \item
    We may want to express database-like uniqueness constraints.
    For instance, we may wish to ensure that the email address of an account
    owner uniquely identifies them.
  \item
    We may want to ensure that all paths of a certain kind end in nodes with
    some desired properties. For example, if an account is privileged, then all
    users that have access to it should also be privileged.
  \item
    We may want to put an upper bound on the number of nodes reached from a
    given node by certain paths. For instance, every user may have access to at
    most 5 accounts.
\end{enumerate}

% \todo[inline]{Wim: Reminder to self. I'd like to illustrate some open/closed things here. (There's no time anymore for this.)}
% \todo[inline]{Wim: More urgently though, we should explain better about how we model things. Let's say that ``users'' are those nodes that have an email key and ``accounts'' are those that have a card key?}
% \todo[inline]{Iovka: I support the need to make this precise. Then, should we use these two selectors in all examples?\\
% Also, we might say that we need this trick because we do not have rdf:type nor labels on nodes.}
% \todo[inline]{Cem: After discussion with Filip, I fixed the setting such that it is keys that identify users and accounts. Problem: this makes C2 awkward. }

\end{example}

% End of file 2-Preliminaries.tex

\section{\method}
\label{sec:tokenskip}
\begin{figure*}[t]
\centering
\includegraphics[width=0.95\textwidth]{fig/tokenskip.pdf}
\caption{Illustration of \method. During the training phase, \method first generates CoT trajectories from the target LLM. These CoTs are then compressed to a specified ratio, $\gamma$, based on the semantic importance of tokens. \method fine-tunes the target LLM using compressed CoTs, enabling controllable CoT inference at the desired $\gamma$.}
\label{fig:tokenskip}
\end{figure*}

We introduce \method, a simple yet effective approach that enables LLMs to skip less important tokens, enabling controllable CoT compression with adjustable ratios. This section demonstrates the details of our methodology, including token pruning~(\S\ref{sec:token-pruning}), training~(\S\ref{sec:training}), and inference~(\S\ref{sec:inference}).

\subsection{Token Pruning}
\label{sec:token-pruning}
The key insight behind \method is that ``\textit{each reasoning token contributes differently to deriving the answer.}'' To enhance CoT efficiency, we propose to trim redundant tokens from LLM CoT outputs and fine-tune LLMs using these trimmed CoT trajectories. The token pruning process is guided by the concept of \textit{token importance}, as detailed in Section~\ref{sec:token-importance}. 

Specifically, given a target LLM $\M$, one of its CoT trajectories $\boldsymbol{c}=\left\{c_i\right\}_{i=1}^{m}$, and a desired compression ratio $\gamma \in \left[0,1\right]$, \method first calculates the semantic importance of each CoT token $I\left(c\right)$, as defined in Eq~(\ref{eq:llmlingua2}). The tokens are then ranked in descending order based on their importance values. Next, the $\gamma$-th percentile of these importance values is computed, representing the threshold for token pruning:
\begin{equation}
I_\gamma=\mathrm{np.percentile}\left(\left[I\left(c_1\right), . ., I\left(c_m\right)\right], \gamma\right).
\end{equation}
Finally, CoT tokens with an importance value greater than or equal to $I_\gamma$ are retained in the compressed CoT trajectory:
\begin{equation}
\widetilde{\boldsymbol{c}}=\left\{c_i \mid I\left(c_i\right) \geq I_\gamma\right\}, 1 \leq i \leq m.
\end{equation}

\subsection{Training}
\label{sec:training}
Given a training dataset $\mathcal{D}$ with $N$ samples and a target LLM $\M$, we first obtain $N$ CoT trajectories with $\M$. Then, we filter out trajectories with incorrect answers to ensure the high quality of training data. For the remaining CoT trajectories, we prune each CoT with a randomly selected compression ratio $\gamma$, as demonstrated in Section~\ref{sec:token-pruning}. For each $\langle\text{question}, \text{compressed CoT}, \text{answer}\rangle$, we inserted the compression ratio $\gamma$ after the question. Finally, each training sample is formatted as follows: 
\begin{equation}
\nonumber
    \mathcal{Q} \ \mathrm{[EOS]} \ \gamma \ \mathrm{[EOS]} \ \mathrm{Compressed\ CoT} \ \mathcal{A},
\end{equation}
where $\langle\mathcal{Q}, \mathcal{A}\rangle$ indicates the $\langle\text{question}, \text{answer}\rangle$ pair. Formally, given a question $\boldsymbol{x}$, compression ratio $\gamma$, and the output sequence $\boldsymbol{y}=\left\{y_i\right\}_{i=1}^{l}$, which includes the compressed CoT $\widetilde{\boldsymbol{c}}$ and the answer $\boldsymbol{a}$, we fine-tunes the target LLM $\M$, enabling it to perform chain-of-thought in a compressed pattern by minimizing
\begin{equation}
\mathcal{L}=\sum_{i=1}^{l} \log P\left(y_{i} \mid \bm{x}, \gamma, \bm{y}_{<i}; \bm{\theta}_{\M}\right),
\end{equation}
where $\bm{y} =\left\{\widetilde{c}_1, \cdots,\widetilde{c}_{m^{\prime}}, a_1, \cdots, a_t  \right\}$. Note that the compression is performed solely on CoT sequences, and we keep the answer $\boldsymbol{a}=\left\{a_i\right\}_{i=1}^{t}$ unchanged. To preserve LLMs' reasoning capabilities, we also include a portion of the original CoT trajectories in the training data, with $\gamma$ set to 1.

\subsection{Inference}
\label{sec:inference}
The inference of \method follows autoregressive decoding. Compared to original CoT outputs that may contain redundancy, \method facilitates LLMs to skip \textit{unimportant} tokens during the chain-of-thought process, thereby enhancing reasoning efficiency. Formally, given a question $\boldsymbol{x}$ and the compression ratio $\gamma$, the input prompt of \method follows the same format adopted in fine-tuning, which is $\mathcal{Q} \ \mathrm{[EOS]} \ \gamma \ \mathrm{[EOS]}$. The LLM $\M$ sequentially predicts the output sequence $\hat{\bm{y}}$:
\begin{equation}
\nonumber
\hat{\boldsymbol{y}}=\arg \max _{\boldsymbol{y}^*} \sum_{j=1}^{l^{\prime}} \log P\left(y_j \mid \boldsymbol{x}, \gamma, \boldsymbol{y}_{<j}; \bm{\theta}_{\M}\right),
\end{equation}
where $\hat{\bm{y}} =\left\{\hat{c}_1, \cdots,\hat{c}_{m^{\prime\prime}}, \hat{a}_1, \cdots, \hat{a}_{t^{\prime}}  \right\}$ denotes the output sequence, which includes CoT tokens $\hat{\bm{c}}$ and the answer $\bm{\hat{a}}$. We illustrate the training and inference process of \method in Figure~\ref{fig:tokenskip}. 

\section{Experiments}
\label{sec:experiments}

\subsection{Next K-mer Prediction}
\label{sec:kmer_predition}
\begin{figure}[t]
    \centering
    \includegraphics[width=0.5\textwidth]{figures/pdf/kmer_prediction_main_text.pdf}
    \caption{Evaluation of next K-mer prediction. (A) Accuracy of the next K-mer prediction task across various tokenizers and input token lengths. (B) Comparison of the \textbf{Gener}\textit{ator} against baseline models on a dataset comprised exclusively mammalian DNA.}
    \label{fig:kmer_main}
\end{figure}

As mentioned in \textit{Sec.} \ref{sec:tokenization}, we conducted extensive experiments to explore the most suitable tokenizer for training causal DNA language models. This was achieved by training multiple models on identical datasets, each employing a different tokenizer. All models share the same architecture as the \textbf{Gener}\textit{ator} and are uniformly compared at 32,000 training steps. We employed the accuracy of the next K-mer prediction task as our evaluation metric. This zero-shot task facilitates a direct assessment of the pre-trained model quality, ensuring equitable comparisons across various tokenizers. As depicted in \textit{Fig.} \ref{fig:kmer_main}A, the tested tokenizers include BPE tokenizers with vocabulary sizes ranging from 512 to 8192, and K-mer tokenizers with K values from 1 to 8 (noting that the single nucleotide tokenizer corresponds to a K-mer tokenizer with K=1). Overall, K-mer tokenizers demonstrate superior performance compared to BPE tokenizers. Among the K-mer tokenizers, the 6-mer tokenizer is selected for its robust performance with limited input tokens and its ability to maintain top-tier performance as the number of input tokens increases.

Moreover, we evaluated the performance of Mamba \cite{Mamba,Mamba-2}, recognized for its capacity in handling long-context pre-training. To adequately assess its capabilities, we configured a Mamba model utilizing the single nucleotide tokenizer with 1.2B parameters and a context length of 98k bp. The Mamba model is compared to the 1-mer and 6-mer models under varied configurations. The comparison with the 1-mer model is straightforward; the Mamba model (denoted as Mamba\texttimes1 in \textit{Fig.} \ref{fig:kmer_main}A) exhibits slightly better performance with fewer input tokens but underperforms as the token count increases. Despite Mamba's context length being six times that of the 1-mer model, this feature does not translate into improved performance. This might suggest that Mamba's renowned ability to handle long-context pre-training primarily refers to cost-effective training rather than enhanced model performance \cite{Empirical, DeciMamba}. To compare against the 6-mer model, we adjust the input token count for the Mamba model by a factor of six (denoted as Mamba\texttimes6) to compare the models on the same base-pair basis. In this context, Mamba\texttimes6 shows slightly better performance with fewer input tokens; however, it rapidly lags as the token count increases. These findings collectively indicate that a transformer decoder architecture paired with a 6-mer tokenizer provides the most effective approach for training causal DNA language models, aligning with the configuration of the \textbf{Gener}\textit{ator}.

We further compared the \textbf{Gener}\textit{ator} model with other baseline models to evaluate their generative capabilities. As illustrated in \textit{Fig.} \ref{fig:kmer_main}B, we assess model performance using a dataset composed exclusively of mammalian DNA, given that HyenaDNA and GROVER are trained solely on human genomes. The \textbf{Gener}\textit{ator} significantly outperforms other baseline models, including its variant, \textbf{Gener}\textit{ator}-All, which incorporates pre-training on non-gene regions. This suggests that the gene sequence training strategy, which emphasizes semantically rich regions, provides a more effective training scheme compared to the conventional whole sequence training. This effectiveness is likely due to the sparsity of gene segments in the whole genome (less than 10\%) and the disproportionate importance of these segments. Among the other baseline models, NT-multi demonstrates the best performance, likely attributable to its extensive model scale (2.5B parameters), yet it still lags significantly behind the \textbf{Gener}\textit{ator}. This result aligns with expectations, as the MLM training paradigm is recognized for its limitations in generative capabilities. Meanwhile, HyenaDNA, despite utilizing the NTP training paradigm, does not show improved performance compared to other masked language models, likely due to its overly small model size (55M parameters), insufficient for exhibiting robust generative abilities. This comparison underscores the critical role of the \textbf{Gener}\textit{ator} in bridging the gap for large-scale generative DNA language models within the eukaryotic domain.

Due to space constraints, we have chosen only to demonstrate specific examples with mammalian DNA data and a fixed K-mer prediction length of 16 bp in \textit{Fig.} \ref{fig:kmer_main}. A more comprehensive analysis across various taxonomic groups and K-mer lengths is provided in the appendix.

\subsection{Benchmark Evaluations}
In this section, we compare the \textbf{Gener}\textit{ator} with state-of-the-art genomic foundation models: Enformer~\cite{enformer}, DNABERT-2, HyenaDNA, Nucleotide Transformer, Caduceus, and GROVER, across various benchmark tasks. To ensure a fair comparison, we uniformly fine-tune each model and perform a 10-fold cross-validation on all datasets. For each model on each dataset, we conduct a hyperparameter search, exhaustively tuning learning rates in $\{1e^{-5}, 2e^{-5}, 5e^{-5}, \ldots, 1e^{-3}, 2e^{-3}, 5e^{-3}\}$ and batch sizes in $\{64, 128, 256, 512\}$. Detailed hyperparameter settings and implementation specifics are provided in the appendix.

\paragraph{Nucleotide Transformer Tasks}
Since the NT task dataset was revised recently~\cite{nucleotide-transformer}, we conducted experiments on both the original and revised datasets. The results for the revised NT tasks are provided in Table~\ref{tab:nucleotide_transformer_tasks_revised}, and the results for the original NT tasks are provided in Table~\ref{tab:nucleotide_transformer_tasks}. Overall, the \textbf{Gener}\textit{ator} outperforms other baseline models. However, the \textbf{Gener}\textit{ator}-All variant shows some performance decline. Notably, despite its earlier release, Enformer continues to deliver competitive results in chromatin profile and regulatory element tasks. This performance could be attributed to its original training in a supervised manner specifically for chromatin and gene expression tasks. The latest release of Nucleotide Transformer, NT-v2, although smaller in size (500M), demonstrates enhanced performance compared to NT-multi (2.5B). In contrast, DNABERT-2 and GROVER, which utilize BPE tokenizers, along with HyenaDNA and Caduceus, which employ the finer-grained single nucleotide tokenizer, do not show distinct performance advantages, likely due to the limited model scope and data scale.

\paragraph{Genomic Benchmarks}
We also conducted a comparative analysis on the Genomic Benchmarks~\cite{genomic-benchmarks}, which primarily focus on the human genome. The evaluation results are provided in Table~\ref{tab:genomic_benchmarks}. Overall, the \textbf{Gener}\textit{ator} still outperforms other models. However, it is worth noting that the Caduceus models also exhibit comparable performance while being significantly smaller (8M). This is likely due to the fact that Caduceus models are trained exclusively on the human genome, making them efficient and compact. Nevertheless, this exclusivity may limit their generalizability to other genomic contexts.

\paragraph{Gener Tasks} 
Lastly, we evaluated the newly proposed Gener tasks, which focus on assessing genomic context comprehension across various sequence lengths and organisms. As shown in Table~\ref{tab:gener_tasks}, the \textbf{Gener}\textit{ator} achieves the best performance on both gene and taxonomic classification tasks, with NT-v2 also demonstrating similar results. Further details on the evaluation of Gener tasks, including visualizations of confusion matrices, are provided in the appendix. The superior performance of the \textbf{Gener}\textit{ator} and NT-v2 is likely due to their pre-training on multispecies datasets. In contrast, despite also being trained on multispecies data, DNABERT-2 exhibits noticeable performance degradation. This may be attributed to its limited model size (117M for DNABERT-2, 500M for NT-v2, and 1.2B for \textbf{Gener}\textit{ator}) and shorter context length (3k for DNABERT-2, 12k for NT-v2, and 98k for \textbf{Gener}\textit{ator}). Other models, such as HyenaDNA and Caduceus, although trained exclusively on the human genome, still exhibit relevant generalizability on both tasks after fine-tuning, attributable to their long-context capacity (\textgreater 100k). GROVER, on the other hand, significantly lags behind in taxonomic classification due to its limited context length (3k).

\begin{table*}[!htb]
\small
\renewcommand{\arraystretch}{1}
\centering
\caption{Evaluation of the revised Nucleotide Transformer tasks. The reported values represent the Matthews correlation coefficient (MCC) averaged over 10-fold cross-validation, with the standard error in parentheses.}
\resizebox{\textwidth}{!}{
\begin{tabular}{lcccccccccc}
\toprule
& Enformer & DNABERT-2 & HyenaDNA & NT-multi & NT-v2 & Caduceus-Ph & Caduceus-PS & GROVER & \textbf{Gener}\textit{ator} & \textbf{Gener}\textit{ator}-All \\
& (252M) & (117M) & (55M) & (2.5B) & (500M) & (8M) & (8M) & (87M) & (1.2B) & (1.2B) \\
\midrule
H2AFZ          & 0.522 (0.019) & 0.490 (0.013) & 0.455 (0.015) & 0.503 (0.010) & \underline{0.524 (0.008)} & 0.417 (0.016) & 0.501 (0.013) & 0.509 (0.013) & \textbf{0.529 (0.009)} & 0.506 (0.019) \\
H3K27ac        & \underline{0.520 (0.015)} & 0.491 (0.010) & 0.423 (0.017) & 0.481 (0.020) & 0.488 (0.013) & 0.464 (0.018) & 0.464 (0.022) & 0.489 (0.023) & \textbf{0.546 (0.015)} & 0.496 (0.014) \\
H3K27me3       & 0.552 (0.007) & 0.599 (0.010) & 0.541 (0.018) & 0.593 (0.016) & \underline{0.610 (0.006)} & 0.547 (0.010) & 0.561 (0.036) & 0.600 (0.008) & \textbf{0.619 (0.008)} & 0.590 (0.014) \\
H3K36me3       & 0.567 (0.017) & \underline{0.637 (0.007)} & 0.543 (0.010) & 0.635 (0.016) & 0.633 (0.015) & 0.543 (0.009) & 0.602 (0.008) & 0.585 (0.008) & \textbf{0.650 (0.006)} & 0.621 (0.013) \\
H3K4me1        & \textbf{0.504 (0.021)} & \underline{0.490 (0.008)} & 0.430 (0.014) & 0.481 (0.012) & \underline{0.490 (0.017)} & 0.411 (0.012) & 0.434 (0.030) & 0.468 (0.011) & \textbf{0.504 (0.010)} & \underline{0.490 (0.016)} \\
H3K4me2        & \textbf{0.626 (0.015)} & 0.558 (0.013) & 0.521 (0.024) & 0.552 (0.022) & 0.552 (0.013) & 0.480 (0.013) & 0.526 (0.035) & 0.558 (0.012) & \underline{0.607 (0.010)} & 0.569 (0.012) \\
H3K4me3        & 0.635 (0.019) & \underline{0.646 (0.008)} & 0.596 (0.015) & 0.618 (0.015) & 0.627 (0.020) & 0.588 (0.020) & 0.611 (0.015) & 0.634 (0.011) & \textbf{0.653 (0.008)} & 0.628 (0.018) \\
H3K9ac         & \textbf{0.593 (0.020)} & 0.564 (0.013) & 0.484 (0.022) & 0.527 (0.017) & 0.551 (0.016) & 0.514 (0.014) & 0.518 (0.018) & 0.531 (0.014) & \underline{0.570 (0.017)} & 0.556 (0.018) \\
H3K9me3        & 0.453 (0.016) & 0.443 (0.025) & 0.375 (0.026) & 0.447 (0.018) & 0.467 (0.044) & 0.435 (0.019) & 0.455 (0.019) & 0.441 (0.017) & \textbf{0.509 (0.013)} & \underline{0.480 (0.037)} \\
H4K20me1       & 0.606 (0.016) & \underline{0.655 (0.011)} & 0.580 (0.009) & 0.650 (0.014) & 0.654 (0.011) & 0.572 (0.012) & 0.590 (0.020) & 0.634 (0.006) & \textbf{0.670 (0.006)} & 0.652 (0.010) \\
Enhancer       & \textbf{0.614 (0.010)} & 0.517 (0.011) & 0.475 (0.006) & 0.527 (0.012) & 0.575 (0.023) & 0.480 (0.008) & 0.490 (0.009) & 0.519 (0.009) & \underline{0.594 (0.013)} & 0.553 (0.020) \\
Enhancer type & \textbf{0.573 (0.013)} & 0.476 (0.009) & 0.441 (0.010) & 0.484 (0.012) & 0.541 (0.013) & 0.461 (0.009) & 0.459 (0.011) & 0.481 (0.009) & \underline{0.547 (0.017)} & 0.510 (0.022) \\
Promoter all   & 0.745 (0.012) & 0.754 (0.009) & 0.693 (0.016) & 0.761 (0.009) & \underline{0.780 (0.012)} & 0.707 (0.017) & 0.722 (0.014) & 0.721 (0.011) & \textbf{0.795 (0.005)} & 0.765 (0.009) \\
Promoter non-TATA & 0.763 (0.012) & 0.769 (0.009) & 0.723 (0.013) & 0.773 (0.010) & 0.785 (0.009) & 0.740 (0.012) & 0.746 (0.009) & 0.739 (0.018) & \textbf{0.801 (0.005)} & \underline{0.786 (0.007)} \\
Promoter TATA  & 0.793 (0.026) & 0.784 (0.036) & 0.648 (0.044) & \underline{0.944 (0.016)} & 0.919 (0.028) & 0.868 (0.023) & 0.853 (0.034) & 0.891 (0.041) & \textbf{0.950 (0.009)} & 0.862 (0.024) \\
Splice acceptor & 0.749 (0.007) & 0.837 (0.006) & 0.815 (0.049) & 0.958 (0.003) & \textbf{0.965 (0.004)} & 0.906 (0.015) & 0.939 (0.012) & 0.812 (0.012) & \underline{0.964 (0.003)} & 0.951 (0.006) \\
Splice site all & 0.739 (0.011) & 0.855 (0.005) & 0.854 (0.053) & 0.964 (0.003) & \textbf{0.968 (0.003)} & 0.941 (0.006) & 0.942 (0.012) & 0.849 (0.015) & \underline{0.966 (0.003)} & 0.959 (0.003) \\
Splice donor   & 0.780 (0.007) & 0.861 (0.004) & 0.943 (0.024) & 0.970 (0.002) & \underline{0.976 (0.003)} & 0.944 (0.026) & 0.964 (0.010) & 0.842 (0.009) & \textbf{0.977 (0.002)} & 0.971 (0.002) \\
\bottomrule
\end{tabular}
}
\label{tab:nucleotide_transformer_tasks_revised}
\end{table*}
\begin{table*}[!htb]
\small
\renewcommand{\arraystretch}{1.2}
\centering
\caption{Evaluation of the original Nucleotide Transformer tasks. The reported values represent the Matthews correlation coefficient (MCC) averaged over 10-fold cross-validation, with the standard error in parentheses.}
\resizebox{\textwidth}{!}{%
\begin{tabular}{lcccccccccc}
\toprule
& Enformer & DNABERT-2 & HyenaDNA & NT-multi & NT-v2 & Caduceus-Ph & Caduceus-PS & GROVER & \textbf{Gener}\textit{ator} & \textbf{Gener}\textit{ator}-All \\
& (252M) & (117M) & (55M) & (2.5B) & (500M) & (8M) & (8M) & (87M) & (1.2B) & (1.2B) \\
\midrule
H3 & 0.724 (0.018) & 0.785 (0.012) & 0.781 (0.015) & 0.793 (0.013) & 0.788 (0.010) & 0.794 (0.012) & 0.772 (0.022) & 0.768 (0.008) & \textbf{0.806 (0.005)} & \underline{0.803 (0.007)} \\
H3K14ac & 0.284 (0.024) & 0.515 (0.009) & \textbf{0.608 (0.020)} & 0.538 (0.009) & 0.538 (0.015) & 0.564 (0.033) & 0.596 (0.038) & 0.548 (0.020) & \underline{0.605 (0.008)} & 0.580 (0.038) \\
H3K36me3 & 0.345 (0.019) & 0.591 (0.005) & 0.614 (0.014) & 0.618 (0.011) & 0.618 (0.015) & 0.590 (0.018) & 0.611 (0.048) & 0.563 (0.017) & \textbf{0.657 (0.007)} & \underline{0.631 (0.013)} \\
H3K4me1 & 0.291 (0.016) & 0.512 (0.008) & 0.512 (0.008) & 0.541 (0.005) & 0.544 (0.009) & 0.468 (0.015) & 0.487 (0.029) & 0.461 (0.018) & \textbf{0.553 (0.009)} & \underline{0.549 (0.018)} \\
H3K4me2 & 0.207 (0.021) & 0.333 (0.013) & \textbf{0.455 (0.028)} & 0.324 (0.014) & 0.302 (0.020) & 0.332 (0.034) & \underline{0.431 (0.016)} & 0.403 (0.042) & 0.424 (0.013) & 0.400 (0.015) \\
H3K4me3 & 0.156 (0.022) & 0.353 (0.021) & \textbf{0.550 (0.015)} & 0.408 (0.011) & 0.437 (0.028) & 0.490 (0.042) & \underline{0.528 (0.033)} & 0.458 (0.022) & 0.512 (0.009) & 0.473 (0.047) \\
H3K79me3 & 0.498 (0.013) & 0.615 (0.010) & 0.669 (0.014) & 0.623 (0.010) & 0.621 (0.012) & 0.641 (0.028) & \textbf{0.682 (0.018)} & 0.626 (0.026) & \underline{0.670 (0.011)} & 0.631 (0.021) \\
H3K9ac & 0.415 (0.020) & 0.545 (0.009) & 0.586 (0.021) & 0.547 (0.011) & 0.567 (0.020) & 0.575 (0.024) & 0.564 (0.018) & 0.581 (0.015) & \textbf{0.612 (0.006)} & \underline{0.603 (0.019)} \\
H4 & 0.735 (0.023) & 0.797 (0.008) & 0.763 (0.012) & \underline{0.808 (0.007)} & 0.795 (0.008) & 0.788 (0.010) & 0.799 (0.010) & 0.769 (0.017) & \textbf{0.815 (0.008)} & \underline{0.808 (0.010)} \\
H4ac & 0.275 (0.022) & 0.465 (0.013) & 0.564 (0.011) & 0.492 (0.014) & 0.502 (0.025) & 0.548 (0.027) & \underline{0.585 (0.018)} & 0.530 (0.017) & \textbf{0.592 (0.015)} & 0.565 (0.035) \\
Enhancer & 0.454 (0.029) & 0.525 (0.026) & 0.520 (0.031) & 0.545 (0.028) & \underline{0.561 (0.029)} & 0.522 (0.024) & 0.511 (0.026) & 0.516 (0.018) & \textbf{0.580 (0.015)} & 0.540 (0.026) \\
Enhancer type & 0.312 (0.043) & 0.423 (0.018) & 0.403 (0.056) & 0.444 (0.022) & 0.444 (0.036) & 0.403 (0.028) & 0.410 (0.026) & 0.433 (0.029) & \textbf{0.477 (0.017)} & \underline{0.463 (0.023)} \\
Promoter all & 0.910 (0.004) & 0.945 (0.003) & 0.919 (0.003) & 0.951 (0.004) & 0.952 (0.002) & 0.937 (0.002) & 0.941 (0.003) & 0.926 (0.004) & \textbf{0.962 (0.002)} & \underline{0.955 (0.002)} \\
Promoter non-TATA & 0.910 (0.006) & 0.944 (0.003) & 0.919 (0.004) & \underline{0.955 (0.003)} & 0.952 (0.003) & 0.935 (0.007) & 0.940 (0.002) & 0.925 (0.006) & \textbf{0.962 (0.001)} & \underline{0.955 (0.002)} \\
Promoter TATA & 0.920 (0.012) & 0.911 (0.011) & 0.881 (0.020) & 0.919 (0.008) & \underline{0.933 (0.009)} & 0.895 (0.010) & 0.903 (0.010) & 0.891 (0.009) & \textbf{0.948 (0.008)} & 0.931 (0.007) \\
Splice acceptor & 0.772 (0.007) & 0.909 (0.004) & 0.935 (0.005) & \underline{0.973 (0.002)} & \underline{0.973 (0.004)} & 0.918 (0.017) & 0.907 (0.015) & 0.912 (0.010) & \textbf{0.981 (0.002)} & 0.957 (0.009) \\
Splice site all & 0.831 (0.012) & 0.950 (0.003) & 0.917 (0.006) & 0.974 (0.004) & \underline{0.975 (0.002)} & 0.935 (0.011) & 0.953 (0.005) & 0.919 (0.005) & \textbf{0.978 (0.001)} & 0.973 (0.002) \\
Splice donor & 0.813 (0.015) & 0.927 (0.003) & 0.894 (0.013) & 0.974 (0.002) & \underline{0.977 (0.007)} & 0.912 (0.009) & 0.930 (0.010) & 0.888 (0.012) & \textbf{0.978 (0.002)} & 0.967 (0.005) \\
\bottomrule
\end{tabular}
}
\label{tab:nucleotide_transformer_tasks}
\end{table*}
\begin{table*}[!htb]
\small
\renewcommand{\arraystretch}{1.2}
\centering
\caption{Evaluation of the Genomic Benchmarks. The reported values represent the accuracy averaged over 10-fold cross-validation, with the standard error in parentheses.}
\resizebox{\textwidth}{!}{
\begin{tabular}{lcccccccc}
\toprule
& DNABERT-2 & HyenaDNA & NT-v2 & Caduceus-Ph & Caduceus-PS & GROVER & \textbf{Gener}\textit{ator} & \textbf{Gener}\textit{ator}-All \\
& (117M) & (55M) & (500M) & (8M) & (8M) & (87M) & (1.2B) & (1.2B) \\
\midrule
Coding vs. Intergenomic & 0.951 (0.002) & 0.902 (0.004) & 0.955 (0.001) & 0.933 (0.001) & 0.944 (0.002) & 0.919 (0.002) & \textbf{0.963 (0.000)} & \underline{0.959 (0.001)} \\
Drosophila Enhancers Stark & 0.774 (0.011) & 0.770 (0.016) & 0.797 (0.009) & \textbf{0.827 (0.010)} & 0.816 (0.015) & 0.761 (0.011) & \underline{0.821 (0.005)} & 0.768 (0.015) \\
Human Enhancers Cohn & \underline{0.758 (0.005)} & 0.725 (0.009) & 0.756 (0.006) & 0.747 (0.003) & 0.749 (0.003) & 0.738 (0.003) & \textbf{0.763 (0.002)} & 0.754 (0.006) \\
Human Enhancers Ensembl & 0.918 (0.003) & 0.901 (0.003) & 0.921 (0.004) & \textbf{0.924 (0.002)} & \underline{0.923 (0.002)} & 0.911 (0.004) & 0.917 (0.002) & 0.912 (0.002) \\
Human Ensembl Regulatory & 0.874 (0.007) & 0.932 (0.001) & \textbf{0.941 (0.001)} & \underline{0.938 (0.004)} & \textbf{0.941 (0.002)} & 0.897 (0.001) & 0.928 (0.001) & 0.926 (0.001) \\
Human non-TATA Promoters & 0.957 (0.008) & 0.894 (0.023) & 0.932 (0.006) & \textbf{0.961 (0.003)} & \textbf{0.961 (0.002)} & 0.950 (0.005) & \underline{0.958 (0.001)} & 0.955 (0.005) \\
Human OCR Ensembl & 0.806 (0.003) & 0.774 (0.004) & 0.813 (0.001) & \underline{0.825 (0.004)} & \textbf{0.826 (0.003)} & 0.789 (0.002) & 0.823 (0.002) & 0.812 (0.003) \\
Human vs. Worm & 0.977 (0.001) & 0.958 (0.004) & 0.976 (0.001) & 0.975 (0.001) & 0.976 (0.001) & 0.966 (0.001) & \textbf{0.980 (0.000)} & \underline{0.978 (0.001)} \\
Mouse Enhancers Ensembl & \underline{0.865 (0.014)} & 0.756 (0.030) & 0.855 (0.018) & 0.788 (0.028) & 0.826 (0.021) & 0.742 (0.025) & \textbf{0.871 (0.015)} & 0.784 (0.027) \\
\bottomrule
\end{tabular}
}
\label{tab:genomic_benchmarks}
\end{table*}
\begin{table*}[!htb]
\small
\renewcommand{\arraystretch}{1}
\centering
\caption{Evaluation of the Gener tasks. The reported values represent the weighted F1 score averaged over 10-fold cross-validation, with the standard error in parentheses.}
\resizebox{\textwidth}{!}{
\begin{tabular}{lcccccccc}
\toprule
& DNABERT-2 & HyenaDNA & NT-v2 & Caduceus-Ph & Caduceus-PS & GROVER & \textbf{Gener}\textit{ator} & \textbf{Gener}\textit{ator}-All \\
& (117M) & (55M) & (500M) & (8M) & (8M) & (87M) & (1.2B) & (1.2B) \\
\midrule
Gene & 0.660 (0.002) & 0.610 (0.007) & \underline{0.692 (0.005)} & 0.629 (0.005) & 0.644 (0.007) & 0.630 (0.003) & \textbf{0.700 (0.002)} & 0.687 (0.003) \\
Taxonomic & 0.922 (0.003) & 0.970 (0.024) & 0.981 (0.001) & 0.958 (0.021) & 0.968 (0.006) & 0.843 (0.006) & \textbf{0.999 (0.000)} & \underline{0.998 (0.001)} \\
\bottomrule
\end{tabular}
}
\label{tab:gener_tasks}
\end{table*}

\subsection{Central Dogma}

In our experimental setup, we selected two target protein families from the UniProt~\cite{UniProt} database: the Histone and Cytochrome P450 families. By cross-referencing gene IDs and protein IDs, we extracted the corresponding protein-coding DNA sequences from RefSeq~\cite{RefSeq}. These sequences served as training data for fine-tuning the \textbf{Gener}\textit{ator} model, directing it to generate analogous protein-coding DNA sequences.

To assess the quality of generation, we compared several summary statistics. The results for the Histone family are provided in \textit{Fig.} \ref{fig:histone_generation}, while the evaluation results for the Cytochrome P450 family are provided in \textit{Fig.} \ref{fig:cytochrome_generation}.  After deduplication, the lengths of the generated DNA sequences and their translated protein sequences, using a codon table, closely resemble the distributions observed in the target families. This preliminary validation suggests that our generated DNA sequences maintain stable codon structures that are translatable into proteins. We conducted a more in-depth structural and functional analysis of these translated protein sequences. First, we assessed whether protein language models `recognize' these generated protein sequences by calculating their perplexity (PPL) using Progen2~\cite{progen2}. The results show that the PPL distribution of generated sequences closely matches that of the natural families and significantly differs from the shuffled sequences.

Furthermore, we used AlphaFold3 to predict the folded structures of the generated protein sequences and employed Foldseek~\cite{Foldseek} to find analogous proteins in the Protein Data Bank (PDB)~\cite{RCSBPDB}. Remarkably, we identified numerous instances where the conformations of the generated sequences exhibited high similarity to established structures in the PDB ($\text{TM-score}>0.8$). This structural congruence is observed despite substantial divergence in sequence composition, as indicated by sequence identities less than $0.3$. This low sequence identity positively suggests that the model is not merely replicating existing protein sequences but has learned the underlying principles to design new molecules with similar structures. This highlights the capability of the \textbf{Gener}\textit{ator} in generating biologically relevant sequences. 

\subsection{Enhancer Design}
We employed the enhancer activity data from DeepSTARR~\cite{DeepSTARR}, following the dataset split initially proposed by DeepSTARR and later adopted by NT. Using this data, we developed an enhancer activity predictor by fine-tuning the \textbf{Gener}\textit{ator}. This predictor surpasses the accuracy of DeepSTARR and NT-multi (Table \ref{tab:enhancer_benchmark}), establishing itself as the current state-of-the-art predictor. By applying our refined SFT approach as outlined in \textit{Sec.} \ref{sec:sequence_design}, we generated a collection of candidate enhancer sequences with specific activity profiles. As illustrated in \textit{Fig.} \ref{fig:enhancer_design}, the predicted activities of these candidates exhibit significant differentiation between the generated high/low-activity enhancers and natural samples.

To our knowledge, this represents one of the first attempts to use LLMs for prompt-guided design of DNA sequences, highlighting the capability of the \textbf{Gener}\textit{ator} in this domain. These generated sequences, and more broadly, this sequence design paradigm using the \textbf{Gener}\textit{ator}, merit further exploration. Our approach underscores the potential of the \textbf{Gener}\textit{ator} model to transform DNA sequence design methodologies, providing a novel pathway for the conditional design of DNA sequences using LLMs. In our subsequent research, we plan to extend our evaluations through further validations in wet lab conditions to explore the real-world applicability of these designed sequences.

\begin{figure}[!htb]
    \centering
    \includegraphics[width=0.6\textwidth]{figures/pdf/histone_generation.pdf}
    \caption{Histone generation. (A) Distribution densities of the protein sequence lengths for generated and natural samples. (B) Distribution densities of Progen2 PPL for generated and natural samples, along with randomly shuffled sequences. (C) Scatter plot of TM-score and AlphaFold3 prediction confidence (pLDDT) with marginal distributions. (D) Two folded structures of generated samples displaying structural congruence with natural samples.}
    \label{fig:histone_generation}
\end{figure}
\begin{figure}[!htb]
    \centering
    \includegraphics[width=0.6\textwidth]{figures/pdf/cytochrome_generation.pdf}
    \caption{Cytochrome P450 generation. (A) Distribution densities of the protein sequence lengths for generated and natural samples. (B) Distribution densities of Progen2 PPL for generated and natural samples, along with randomly shuffled sequences. (C) Scatter plot of TM-score and AlphaFold3 prediction confidence (pLDDT) with marginal distributions. (D) Two folded structures of generated samples displaying structural congruence with natural samples.}
    \label{fig:cytochrome_generation}
\end{figure}

\begin{table}[!htb]
\small
\renewcommand{\arraystretch}{1}
\centering
\caption{Evaluation of the DeepSTARR dataset. The reported values represent the Pearson correlation coefficient.}
\begin{tabular}{lccc}
\toprule
 & DeepSTARR & NT-multi & \textbf{Gener}\textit{ator} \\
\midrule
Developmental & \underline{0.68} & 0.64 & \textbf{0.70} \\
Housekeeping & 0.74 & \underline{0.75} & \textbf{0.79} \\
\bottomrule
\end{tabular}
\label{tab:enhancer_benchmark}
\end{table}
\begin{figure}[!htb]
    \centering
    \includegraphics[width=0.6\textwidth]{figures/pdf/enhancer_design.pdf}
    \caption{Enhancer design. (A-B) Pearson correlation between the predicted enhancer activity and the measured activity. (C-D) Distribution densities of the predicted activity of generated enhancer sequences with distinct activity profiles.}
    \label{fig:enhancer_design}
\end{figure}

\subsection{Analysis}
\label{sec:analysis}
\begin{figure}[t]
\centering
    \includegraphics[width=0.95\columnwidth]{fig/allratio.pdf}
    \caption{Comparison of ratio adherence across different compression ratio settings. The experimental results are obtained with LLaMA-3.1-8B-Instruct on GSM8K.}
    \label{fig:allratio}
\end{figure}

\begin{figure}[t]
\centering
    \includegraphics[width=0.95\columnwidth]{fig/compressor.pdf}
    \caption{Performance comparison of \method using different token importance metrics, evaluated with LLaMA-3.1-8B-Instruct on GSM8K.}
    \label{fig:compressor}
\end{figure}

\paragraph{Compression Ratio} 
In our main results, we focus on compression ratios greater than 0.5. To further investigate the performance of \method at lower compression ratios, we train an additional variant, denoted as \texttt{More Ratio}, with extra compression ratios of 0.3 and 0.4. As shown in Figure~\ref{fig:allratio}, the ratio adherence of models largely degrades at these lower ratios. We attribute this decline to the excessive trimming of reasoning tokens, which likely causes a loss of critical information in the completions, hindering the effective training of LLMs to learn CoT compression. Furthermore, we observe that the overall adherence of \texttt{More Ratio} is not as good as \method with the default settings, which further supports our hypothesis.

\paragraph{Importance Metric} 
Figure~\ref{fig:compressor} presents a performance comparison of \method across different token importance metrics. In addition to the metrics discussed in Section \ref{sec:token-importance}, we include \texttt{GPT-4o}\footnote{We use the \texttt{gpt-4o-2024-08-06} version for experiments.} as a strong token importance metric for comparison. Specifically, for a given CoT trajectory, we prompt \texttt{GPT-4o} to trim redundant tokens according to a specified compression ratio, without adding any additional tokens. Additionally, we ask \texttt{GPT-4o} to suggest the \textit{optimal} compression format of the CoT trajectory, referred to as \texttt{GPT-4o-Optimal} in Figure~\ref{fig:compressor}. We incorporate all training data generated by \texttt{GPT-4o} to train a variant of \method. We use the ``\texttt{[optimal]}'' token to prompt the model, obtaining the results of \texttt{GPT-4o-Optimal}.

As illustrated in Figure~\ref{fig:compressor}, \method utilizing LLMLingua-2~\cite{pan:2024llmlingua2} outperforms the variant with Selective Context~\cite{li:2023selective}, which aligns with our demonstrations in Section \ref{sec:token-importance}. Additionally, incorporating \texttt{GPT-4o} for token importance measurement further enhances compression performance, suggesting that a more robust CoT compressor could improve \method even further. However, the API costs associated with \texttt{GPT-4o} make it impractical for processing large datasets. In contrast, LLMLingua-2, which includes a BERT-size model, offers a cost-effective and efficient alternative for training \method. Furthermore, \texttt{GPT-4o-Optimal} achieves a better balance between reasoning accuracy and CoT token reduction, emphasizing the potential of flexible compression ratios in CoT generation --- an avenue we plan to explore in future work.

\paragraph{Length Budget} 
As outlined in Section~\ref{sec:exp_setup}, we adjust the maximum length budget to \texttt{max\_len}$\times\gamma$ when evaluating \method on MATH-500, ensuring a fair comparison of compression ratios. However, this brute-force length truncation inevitably impacts the reasoning performance of LLMs, as LLMs are unable to complete the full generation. In this analysis, we explore whether LLMs can ``\textit{think}'' more effectively using a compressed CoT format. Specifically, we evaluate \method under the same length budget as the original LLM (e.g., 1024 for MATH-500). The experimental results, shown in Figure~\ref{fig:budget}, demonstrate a significant performance improvement of \method under this length budget, compared to those adjusted by compression ratios. Notably, with compression ratios of 0.7, 0.8, and 0.9, \method outperforms the original LLM, yielding an absolute performance increase of 1.3 to 2.6 points. These findings highlight \method's potential to enhance the reasoning capabilities of LLMs within the same length budget.

\begin{figure}[t]
\centering
    \includegraphics[width=0.95\columnwidth]{fig/budget.pdf}
    \caption{Performance comparison of \method with varying maximum length constraints, evaluated with LLaMA-3.1-8B-Instruct on the MATH-500 dataset.}
    \label{fig:budget}
\end{figure}

\begin{figure*}[t]
\centering
\includegraphics[width=1.0\textwidth]{fig/cases.pdf}
\caption{Three CoT compression examples from \method. For each sample, we list the question, original CoT outputs from corresponding LLMs, and the compressed CoT by \method. The tokens that appear in both the original CoT and the compressed CoT are highlighted in \sethlcolor{pink}\hl{red}.}
\label{fig:cases}
\end{figure*}

\paragraph{Case Study} 
Figure~\ref{fig:cases} presents several examples of \method, derived from the test sets of GSM8K and MATH-500. These examples clearly illustrate that \method allows LLMs to learn shortcuts between critical reasoning tokens, rather than generating shorter CoTs from scratch. For instance, in the first case, \method facilitates LLaMA-3.1-8B-Instruct to skip semantic connectors such as ``\textit{of}'' and ``\textit{the}'', as well as expressions that contribute minimally to the reasoning, such as the first sentence. Notably, we observe that numeric values and mathematical equations are prioritized for retention in most cases. This finding aligns with recent research~\cite{Ma:2024mathmatters}, which suggests that mathematical expressions may contribute more significantly to reasoning than CoT in natural language. Furthermore, we find that \method does not reduce the number of reasoning steps but instead trims redundant tokens within those steps.
\section{Related Work}

\paragraph{Efficient CoT} While Chain-of-Thought (CoT) enhances task performance by simulating human-like reasoning patterns, its reasoning steps introduce significant computational overhead. As a result, researchers have sought methods to reduce this overhead while retaining the benefits of CoT. One intuitive approach is to simplify, skip~\cite{marconato2024not, Ding:2024cotshortcut, liu2024skipstep}, or generate thinking steps in parallel~\cite{ning2023skeleton} to improve efficiency. Another strategy involves compressing reasoning steps into continuous latent representations~\cite{goyal2024think, deng2024explicit, hao2024training, cheng2024compressed}, allowing LLMs to reason without explicitly generating discrete word tokens. To minimize the generation of redundant natural language information that has minimal impact on reasoning, \citet{hu2023chain} implements structured syntax and symbols, while \citet{han2024token} guides token consumption through dynamic token budget estimation. Similarly, \citet{kang2024c3ot} prompts larger LLMs (i.e., \texttt{GPT-4}) to directly compress CoT, then fine-tunes LLMs to reason using these compressed CoTs. In contrast, this work focuses on pruning CoT tokens based on their semantic importance. Additionally, \method leverages a small LM for token pruning, significantly reducing computational overhead.

\paragraph{Prompt Compression} As LLMs advance in their zero-shot capabilities, the growing demand for complex instructions and long-context prompts has led to substantial computational and memory challenges in processing lengthy inputs. To address this bottleneck, researchers have explored various prompt compression techniques. One intuitive approach involves using a lightweight LM to generate more concise, semantically similar prompts~\cite{chuang-etal-2024-learning}. However, given that explicit natural language representations often contain redundant information, some researchers have turned to implicit continuous tokens to represent task prompts~\cite{wingate-etal-2022-prompt, mu2024learning} and long-context inputs~\cite{chevalier-etal-2023-adapting, ge2024incontext, mohtashami2023randomaccess}. Other approaches focus on directly compressing input prompts by filtering and retaining high-informative tokens~\cite{li:2023selective, jiang2023:llmlingua, pan:2024llmlingua2}. For instance, Selective Context uses the perplexity of LLMs to measure token importance and removes tokens deemed less important. LLMLingua-2~\cite{pan:2024llmlingua2} introduces a small bidirectional language model for token importance measurement and trains this LM with \texttt{GPT-4} compression data, which serves as the token importance metric in this work.
% \section{Discussion and Future Work}

% %Onscad is insample data cause these LLMS have seen openscad

% The decision space of language design is enormous, so we had to make some decisions about what to explore in the language design of AIDL. In particular, we did not build a new constraint system from scratch and instead developed ours based on an open-source constraint solver. This limited the types of primitives we allow, e.g. ellipses are not currently supported. \jz{Additionally, rectangles in AIDL are constrained to be axis-aligned by default because we found that in most use cases, a rectangle being rotated by the solver was unintuitive, and we included a parameter in the language allows rectangles to be marked as rotatable. While this feature was included in the prompts to the LLM, it was never used by the model. We hope to explore prompt-engineering techniques to rectify this issue in the future. Similarly, we hope to reduce the frequency of solver errors by providing better prompts for explaining the available constraints.} \adriana{Add two other limitations to this paragraph: that we typically noticed that things are axis alignment, say why we use this as default and in the future could try to get the gpt to not use default more often. Mention that we still have Solver failures that could be addressed by better engineering in future. }

% In testing our front-end, we observed that repeated instances of feedback tends to reduce the complexity of models as the LLM would frequently address the errors by removing the offending entity. This leads to unnecessarily removed details. More extensive prompt-engineering could be employed in future work to encourage the LLM to more frequently modify, rather than remove, to fix these errors. \adriana{no idea what this paragraph is trying to say}


% \adriana{This seems  like a future work paragraph so maybe start by saying that in the future you could do other front end or fine tune a model with aidl, we just tested the few shot.  } \jz{In the future, we hope to improve our front-end generation pipeline by finetuning a pretrained LLM on example AIDL programs.} In addition, multi-modal vision-langauge model development has exploded in recent months. Visual modalities are an obvious fit for CAD modeling -- in fact, most procedural CAD models are produced in visual editors -- but we decided not to explore visual inputs yet based on reports ([PH] cite OPENAIs own GPT4V paper) that current vision-language models suffter from the same spatial reasoning issues as purely textual models do (identifying relative positions like above, left of, etc.). This also informed our decision to omit spline curves which are difficult to describe in natural language. This deficit is being addressed by the development of new spatial reasoning datasets ([PH] cite visual math reasoning paper), so allowing visual user input as well as visual feedback in future work with the next generation of models seems promising.

% The decision space of language design is enormous, so it was impossible to explore it all here. We had to make some decisions about what to explore, guided by experience, conjecture, technical limitations, and anecdotal experience. Since we primarily explore the interaction between language design and language models in order to overcome the shortcomings in the latter, we did not wish to focus effort on building new constraint systems. This led us to use an open-source constraint solver to build our solver off of. This limited the types of primitives we allow; in particular, most commercial geometric solvers also support ellipses.

% In testing our generation frontend, we observed that repeated instances of feedback tended to reduce the complexity of models as the LLM would frequently address the errors by removing the offending entity. This is a fine strategy for over-constrained systems, but can unnecessarily remove detail when done in response to a syntax or validation error. More extensive prompt-engineering could be employed to encourage the LLM to more frequently modify, rather than remove, to fix these errors


% In recent months, multi-modal vision-language model development has exploded. Visual modalities are an obvious fit for CAD modeling -- in fact, most procedural CAD models are produced in visual editors -- but we decided not to explore visual input yet based on reports (cite OpenAIs own GPT4V paper) that current vision-language models suffer from the same spatial reasoning issues as purely textual models do (identifying relative positions like above, left of, etc.). This also informed our decision to omit spline curves; they are not easily described in natural language. This deficit is being addressed by the development of new spatial reasoning datasets (cite visual math reasoning paper), so allowing visual user input as well as visual feedback in future work with the next generation of models seems promising.



\section{Conclusion}

AIDL is an experiment in a new way of building graphics systems for language models; what if, instead of tuning a model for a graphics system, we build a graphics system tailored for language models? By taking this approach, we are able to draw on the rich literature of programming languages, crafting a language that supports language-based dependency reasoning through semantically meaningful references, separation of concerns with a modular, hierarchical structure, and that compliments the shortcomings of LLMs with a solver assistance. Taking this neurosymbolic, procedural approach allows our system to tap into the general knowledge of LLMs as well as being more applicable to CAD by promoting precision, accuracy, and editability. Framing AI CAD generation as a language design problem is a complementary approach to model training and prompt engineering, and we are excited to see how advance in these fields will synergize with AIDL and its successors, especially as the capabilities of multi-modal vision-language models improve. AI-driven, procedural design coming to CAD, and AIDL provides a template for that future.

% Using procedural generation instead of direct geometric generation enables greater editability, accuracy, and precision
% Using a general language model allows for generalizability beyond existing CAD datasets and control via common language.
% Approaches code gen in LLMs through language design rather than training the model or constructing complexing querying algorithms. This could be a complimentary approach
% Embedding as a DSL in a popular language allows us to leverage the LLMs syntactic knowledge while exploiting our domain knowledge in the language design
% LLM-CAD languages should hierarchical, semantic, support constraints and dependencies




%In this paper, we proposed AIDL, a language designed specifically for LLM-driven CAD design. The AIDL language simultaneously supports 1) references to constructed geometry (\dgone{}), 2) geometric constraints between components (\dgtwo{}), 3) naturally named operators (\dgthree{}), and 4) first-class hierarchical design (\dgfour{}), while none of the existing languages supports all the above. These novel designs in AIDL allow users to tap into LLMs' knowledge about objects and their compositionalities and generate complex geometry in a hierarchical and constrained fashion. Specifically, the solver for AIDL supports iterative editing by the LLM by providing intermediate feedback, and remedies the LLM's weakness of providing explicit positions for geometries.

%\adriana{This seems  like a future work paragraph so maybe start by saying that in the future you could do other front end or fine tune a model with aidl, we just tested the few shot.  }
%\paragraph{Future work} In recent months, multi-modal vision-language model development has exploded. Visual modalities are an obvious fit for CAD modeling -- in fact, most procedural CAD models are produced in visual editors -- but we decided not to explore visual input yet based on reports (cite OpenAIs own GPT4V paper) that current vision-language models suffer from the same spatial reasoning issues as purely textual models do (identifying relative positions like above, left of, etc.). This also informed our decision to omit spline curves; they are not easily described in natural language. This deficit is being addressed by the development of new spatial reasoning datasets (cite visual math reasoning paper), so allowing visual user input as well as visual feedback in future work with the next generation of models seems promising. 

\section*{Limitations}

While Drift contributes promising advances in implicit personal preferences, several limitations remain that should be addressed in future research.

\paragraph{Needs of Online Human Evaluation Benchmarks.}  
A major challenge in personal preference research is the absence of reproducible human evaluations. Even if future benchmarks collect more user-specific annotations beyond PRISM, evaluating personalized generation outputs requires \textit{re-engaging with the same users for feedback}. Although we designed the Perspective dataset to align the label construction and test set evaluation pipelines, it still relies on virtual personas. Therefore, to advance this field, there is a need for online evaluation benchmarks that can reproducibly assess personalized generation using real user feedback.

\paragraph{Limited Analysis Between Drift Attributes and Actual Users.}  
Due to practical and ethical issues, we do not have full access to the backgrounds of actual users. While the PRISM dataset provides basic information (e.g., the intended use of LLMs and brief self-introductions), our analysis (as seen in Figure~\ref{fig:activated-attributes-prism}) is limited in explaining why certain attributes are activated and how these relate to user characteristics. A more in-depth investigation into the correlation between Drift attributes and real user profiles should be studied with future benchmarks.

\paragraph{Biases in Differential Prompting.}  
Our study does not thoroughly analyze the limitations of the zero-shot rewarding mechanism used for each attribute. It is possible that differential prompting may fail to capture certain attributes accurately, and methods like those employed in Helpsteer2—where data is explicitly constructed—could offer more precise evaluations. Nevertheless, given the vast diversity of personal preferences, a zero-shot approach remains essential. As shown in Figure~\ref{fig:rm-results}, this approach yields significantly higher performance, and Figure~\ref{fig:attributes_num} demonstrates that even when the number of attributes is reduced to levels comparable to those used in Helpsteer2, performance remains robust. In essence, unreliable attributes are unlikely to be used during decoding, which mitigates this limitation. Moreover, as future research develops to enable LLM to follow system prompts more precisely, these advances will directly enhance Drift.

\paragraph{Tokenizer Dependency.}  
Drift Decoding adjusts the next-token distribution at each step, which requires that the LLM and the sLM share the same support—that is, they must use the same tokenizer. 

\paragraph{Limited Baselines.}  
Due to the scarcity of datasets for implicit personal preferences, this domain is far less mature compared to explicit preferences. As highlighted in Table~\ref{tab:method_comparison}, the limited availability of extensive baselines forced us to concentrate primarily on analyzing the unique characteristics of Drift.

\section*{Ethical Statement}

While Drift effectively integrates users’ implicit preferences into generated outputs, it also introduces several ethical risks that must be carefully managed. Notably, the Drift Approximation stage allows us to directly assess the activation levels of each attribute, which provides an opportunity to identify and preemptively block system prompts that could lead to harmful or undesirable content before they are incorporated into the decoding process. This capability underscores the importance of further research into combining Drift with diverse system prompts, ensuring that the generation of undesirable content is minimized while still delivering personalized services.

Additionally, considering that existing research~\citep{kim2024guaranteed} indicates it is impossible to obtain filtered autoregressive distributions under certain conditions, it is necessary to combine rejection sampling on final outputs using safeguards~\citep{lifetox} such as LlamaGuard~\citep{llamaguard} and ShieldGemma~\citep{shieldgemma}. This approach can further enhance the safety of the final generated content.


% Bibliography entries for the entire Anthology, followed by custom entries
%\bibliography{anthology,custom}
% Custom bibliography entries only
\bibliography{custom}

\clearpage

\appendix

\section*{Appendix}
\section{CoT Recovery}
\label{appendix:recovery}

In this section, we provide the detailed prompt for our recovery experiments, which is illustrated in Figure~\ref{fig:recovery-prompt}. Besides, we present the CoT recovery result from \texttt{GPT-4o}~\cite{gpt-4} in Figure~\ref{fig:recovery-gpt}. The recovered results demonstrate that \texttt{GPT-4o} could understand the compressed CoT content and correctly restore the original CoT process.

\begin{figure}[htbp]
\begin{tcolorbox}[colback=blue!5!white,colframe=blue!75!black,title=Input Prompt for CoT Recovery,fontupper=\footnotesize,fonttitle=\scriptsize]
Could you please recover the following compressed Chain-of-Thought output of a mathematical question to its original full content?

\vskip 0.1in

The question is: 

Marcus is half of Leo’s age and five years younger than Deanna. Deanna is 26. How old is Leo?

\vskip 0.1in

The compressed Chain-of-Thought: 

break down Deanna 26 Marcus five years younger 26 - 5 21 Marcus half Leo's age Leo twice Marcus's age Marcus 21, Leo's age 2 x 21 = 42.

\vskip 0.1in

Original Full Chain-of-Thought:
\end{tcolorbox}
\vskip -0.1in
\caption{Input prompt for LLaMA-3.1-8B-Instruct designed to recover the compressed CoT from a GSM8K math problem.}
\vskip -0.1in
\label{fig:recovery-prompt}
\end{figure}

\begin{figure}[htbp]
\begin{tcolorbox}[colback=blue!5!white,colframe=blue!75!black,title=Revovering the Compressed Chain-of-Thought,fontupper=\footnotesize,fonttitle=\scriptsize]
\textbf{Compressed CoT}: break down Deanna 26 Marcus five younger 26 - 5 21 Marcus half Leo's age twice Marcus Marcus 21, Leo's age 2 x 21 = 42.

\vskip 0.1in
        
\textbf{Recovered CoT}: 1. We know that Deanna is 26 years old. 2. Marcus is five years younger than Deanna. So, Marcus’s age is \( 26 - 5 = 21 \). 3. Marcus is also half of Leo’s age, which means Leo’s age is twice Marcus’s age. 4. Since Marcus is 21 years old, Leo’s age is \( 2 \times 21 = 42 \). So, Leo is 42 years old.
\end{tcolorbox}
\vskip -0.1in
\caption{Recovering the compressed CoT for GSM8K math word problem using GPT-4o.}
\vskip -0.1in
\label{fig:recovery-gpt}
\end{figure}

\section{Experimental Details}
\label{appendix:exp_details}

\subsection{Implementation Details}
\label{appendix:training_details}
We utilize LLMLingua-2~\cite{pan:2024llmlingua2} as the token importance metric to generate our compressed CoT training data. The compression ratio $\gamma$ is randomly selected from $\{0.5, 0.6, 0.7, 0.8, 0.9, 1.0\}$ for each training sample. We adopt LoRA~\cite{lora} to train our models. The rank $r$ is set to 8, and the scaling parameter $\alpha$ is set to 16. We train the models for 3 epochs on both datasets. The peak learning rate is set to 5e-5, following a cosine decay schedule. We use AdamW~\cite{AdamW} for optimization, with a warmup ratio of 0.1. We implement our training process using the \texttt{LLaMA-Factory}~\cite{llamafactory} library. Inference for both our method and all baselines is performed using the Huggingface transformers package. During inference, the maximum number of tokens \texttt{max\_len} is set to 512 for GSM8K and 1024 for MATH. All experiments are conducted using Pytorch 2.1.0 on 2$\times$NVIDIA GeForce RTX 3090 GPU (24GB) with CUDA 12.1, and an Intel(R) Xeon(R) Platinum 8370C CPU with 32 cores.

\subsection{Detailed Results with Qwen}
\label{appendix:qwen_detail}
We provide detailed experimental results of the Qwen2.5-Instruct series evaluated on GSM8K in Table~\ref{tab:qwen}. As the model scale increases, there is less performance degradation at higher compression ratios, indicating that larger LLMs are better at identifying shortcuts between critical reasoning tokens, enabling more efficient CoT generation.

\begin{table}[htbp]
\centering
\small
\setlength{\tabcolsep}{1.4mm}
%\resizebox{\linewidth}{!}{
\begin{tabular}{@{}llcrrc@{}}
\toprule
\textbf{Scale} &\textbf{Methods} &\textbf{Ratio} &Accuracy &Tokens &\textit{Act}Ratio \\ \midrule
\multirow{7}{*}{\texttt{3B}} & \texttt{Original}  & - & 83.7\cred{(0.0 \downarrow)} &314.87 &- \\ \cmidrule{2-6}
&\multirow{6}{*}{\method}  &1.0 & 83.4\cred{(0.3 \downarrow)} &318.79 & 1.00  \\
& &0.9 & 83.2\cred{(0.5 \downarrow)} &262.99 & 0.83\\
& &0.8 & 81.6\cred{(2.1 \downarrow)} &250.71 & 0.79\\
& &0.7 & 80.1\cred{(3.6 \downarrow)} &233.03 & 0.73\\
& &0.6 & 77.3\cred{(6.4 \downarrow)} &199.55 & 0.63\\
& &0.5 & 74.4\cred{(9.3 \downarrow)} &170.55 & 0.54 \\\midrule
\multirow{7}{*}{\texttt{7B}} & \texttt{Original}  & - & 91.4\cred{(0.0 \downarrow)} &297.83 &- \\ \cmidrule{2-6}
&\multirow{6}{*}{\method}  &1.0 & 91.7\cred{(0.3 \uparrow)} &295.78 & 1.00  \\
& &0.9 & 91.1\cred{(0.3 \downarrow)} &254.77 & 0.86\\
& &0.8 & 90.1\cred{(1.3 \downarrow)} &237.27 & 0.80\\
& &0.7 & 89.9\cred{(1.5 \downarrow)} &216.73 & 0.73\\
& &0.6 & 87.9\cred{(3.5 \downarrow)} &178.07 & 0.60\\
& &0.5 & 86.0\cred{(5.4 \downarrow)} &151.44 & 0.51 \\\midrule
\multirow{7}{*}{\texttt{14B}} & \texttt{Original}  & - & 93.1\cred{(0.0 \downarrow)} &313.11 &- \\ \cmidrule{2-6}
&\multirow{6}{*}{\method}  &1.0 & 93.0\cred{(0.1 \downarrow)} &314.55 & 1.00  \\
& &0.9 & 93.3\cred{(0.2 \uparrow)} &269.22 & 0.86\\
& &0.8 & 93.2\cred{(0.1 \uparrow)} &247.24 & 0.79\\
& &0.7 & 93.4\cred{(0.3 \uparrow)} &218.62 & 0.70\\
& &0.6 & 92.7\cred{(0.4 \downarrow)} &180.68 & 0.57\\
& &0.5 & 91.4\cred{(1.7 \downarrow)} &156.85 & 0.50 \\
\bottomrule
\end{tabular}%}
\caption{Experimental results on the Qwen2.5-Instruct series. We report accuracy, average CoT token count, and actual compression ratio (\textit{Act}Ratio) for comparison.}
\label{tab:qwen}
\end{table}

\end{document}
