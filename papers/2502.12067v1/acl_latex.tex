% This must be in the first 5 lines to tell arXiv to use pdfLaTeX, which is strongly recommended.
\pdfoutput=1
% In particular, the hyperref package requires pdfLaTeX in order to break URLs across lines.

\documentclass[11pt]{article}
\usepackage[dvipsnames]{xcolor}
% Change "review" to "final" to generate the final (sometimes called camera-ready) version.
% Change to "preprint" to generate a non-anonymous version with page numbers.
\usepackage[preprint]{acl}

% Standard package includes
\usepackage{times}
\usepackage{latexsym}

% For proper rendering and hyphenation of words containing Latin characters (including in bib files)
\usepackage[T1]{fontenc}
% For Vietnamese characters
% \usepackage[T5]{fontenc}
% See https://www.latex-project.org/help/documentation/encguide.pdf for other character sets

% This assumes your files are encoded as UTF8
\usepackage[utf8]{inputenc}

% This is not strictly necessary, and may be commented out,
% but it will improve the layout of the manuscript,
% and will typically save some space.
\usepackage{microtype}

% This is also not strictly necessary, and may be commented out.
% However, it will improve the aesthetics of text in
% the typewriter font.
\usepackage{inconsolata}

\usepackage{hyperref}
\usepackage{url}
\newcommand{\explain}[2]{\underbrace{#1}_{{\footnotesize\raggedright #2}}}

% for text examples
\newcommand{\delete}[1]{\{\textit{\sout{#1}}\}}
\usepackage[most]{tcolorbox}
\newcommand{\sdelete}[1]{\textit{\sout{#1}}}
\colorlet{lightSalmon}{Salmon!80}
\newcommand{\colorize}[2]{\colorbox{lightSalmon!#1!white}{\strut #2}}

% for figures
\usepackage{graphicx}
\usepackage{subfigure}
\usepackage{wrapfig}

% for equation
\usepackage{amsmath}
\usepackage{amstext}
\usepackage{amsfonts}
\usepackage{bm}

\usepackage{bbm}
% for table
\usepackage{multirow}
\usepackage{booktabs}
\usepackage{array}
\usepackage{caption}
\usepackage{multirow}
\usepackage{booktabs}
\usepackage{array}
\usepackage{caption}
\usepackage{color}
\usepackage{colortbl}
\usepackage{tablefootnote}
\usepackage{adjustbox}
\newcommand{\cred}[1]{\textcolor{red}{$_{#1}$}}

% for algorithm
\usepackage{caption}
\usepackage{algorithm}
\usepackage{algpseudocode}
\newcommand{\mycolor}[1]{\textcolor[RGB]{64,101,149}{#1}}
\newcommand{\mydarkcolor}[1]{\textcolor[RGB]{64,101,149}{#1}}
\algnewcommand{\LineComment}[1]{\Statex ~~~~~~\textsc{//}~\textit{#1}}

%for itemize
\usepackage{enumitem}
\setenumerate[1]{itemsep=0pt,partopsep=0pt,parsep=\parskip,topsep=5pt}
\setitemize[1]{itemsep=0pt,partopsep=0pt,parsep=\parskip,topsep=5pt}
\setdescription{itemsep=0pt,partopsep=0pt,parsep=\parskip,topsep=5pt}

% for highlight
\usepackage{soul}

% for taxonomy
\usepackage{tikz}
\usepackage[edges]{forest}
\definecolor{hidden-draw}{RGB}{64,101,149}
% \definecolor{hidden-pink}{RGB}{255,245,247}
\definecolor{hidden-pink}{RGB}{231,239,250}

% for notation
\usepackage[mathscr]{euscript}
\newcommand{\M}{\mathcal{M}}

%for itemize
\usepackage{amssymb}  
\usepackage{pifont}
\newcommand{\cmark}{\ding{51}}
\newcommand{\xmark}{\ding{55}}
\newcommand{\greenyes}{\textcolor{green}{\ding{51}}}
\newcommand{\redno}{\textcolor{red}{\ding{55}}}

% for title
\usepackage{xspace}

\newcommand{\method}{\texttt{TokenSkip}\xspace}

%% === commands for comments ===
\usepackage{ulem}

% If the title and author information does not fit in the area allocated, uncomment the following
%
%\setlength\titlebox{<dim>}
%
% and set <dim> to something 5cm or larger.

\title{
\texorpdfstring{\includegraphics[width=18pt]{fig/arrow.png}}{}
\method: Controllable Chain-of-Thought Compression in LLMs
}

\author{{Heming Xia}\textsuperscript{\rm 1}, {\textbf{Yongqi Li}}\textsuperscript{\rm 1}\thanks{Corresponding Author}, {\textbf{Chak Tou Leong}}\textsuperscript{\rm 1}, {\textbf{Wenjie Wang}}\textsuperscript{\rm 2}, {\textbf{Wenjie Li}}\textsuperscript{\rm 1}\\
  \textsuperscript{\rm 1}Department of Computing, The Hong Kong Polytechnic University \\
  \textsuperscript{\rm 2}University of Science and Technology of China \\
  {\tt \{he-ming.xia, chak-tou.leong\}@connect.polyu.hk}
}

\begin{document}
\maketitle
\begin{abstract}
Chain-of-Thought (CoT) has been proven effective in enhancing the reasoning capabilities of large language models (LLMs). Recent advancements, such as OpenAI's o1 and DeepSeek-R1, suggest that scaling up the length of CoT sequences during inference could further boost LLM reasoning performance. However, due to the autoregressive nature of LLM decoding, longer CoT outputs lead to a linear increase in inference latency, adversely affecting user experience, particularly when the CoT exceeds 10,000 tokens. To address this limitation, we analyze the semantic importance of tokens within CoT outputs and reveal that their contributions to reasoning vary. Building on this insight, we propose \method, a simple yet effective approach that enables LLMs to selectively skip less important tokens, allowing for controllable CoT compression. Extensive experiments across various models and tasks demonstrate the effectiveness of \method in reducing CoT token usage while preserving strong reasoning performance. Notably, when applied to Qwen2.5-14B-Instruct, \method reduces reasoning tokens by $40\%$ (from 313 to 181) on GSM8K, with less than a $0.4\%$ performance drop. We release our code and checkpoints in \url{https://github.com/hemingkx/TokenSkip}.
\end{abstract}

\section{Introduction}

Chain-of-Thought (CoT) prompting~\cite{Nye:2021, cot, Kojima:2022cotzero} has emerged as a cornerstone strategy for enhancing Large Language Models (LLMs) in complex reasoning tasks. By eliciting step-by-step inference, CoT enables LLMs to decompose intricate problems into manageable subtasks, thereby improving their problem-solving performance~\cite{Yao:2023tot, Wang:2023self-consistency, Zhou:2023least, Shinn:2023Reflexion}. Recent advancements, such as OpenAI's o1~\cite{o1} and DeepSeek-R1~\cite{deepseekr1}, further demonstrate that scaling up CoT lengths from hundreds to thousands of reasoning steps could continuously improve LLM reasoning. These breakthroughs have underscored CoT’s potential to advance LLM capabilities, expanding the boundaries of AI-driven problem-solving.

\begin{figure}[t]
\centering
    \includegraphics[width=0.95\columnwidth]{fig/intro.pdf}
    \caption{In contrast to vanilla CoT that generates all reasoning tokens sequentially, \method enables LLMs to \textit{skip} tokens with less semantic importance (\textit{e.g.,} \includegraphics[width=7pt]{fig/token.pdf}~) and learn shortcuts between critical reasoning tokens, facilitating controllable CoT compression.}
    \label{fig:intro}
\end{figure}

Despite its effectiveness, the increased length of CoT sequences introduces substantial computational overhead. Due to the autoregressive nature of LLM decoding, longer CoT outputs lead to proportional increases in both inference latency and memory footprints of key-value cache. Additionally, the quadratic computational cost of attention layers further exacerbates this burden. These issues become particularly pronounced when CoT sequences extend into thousands of reasoning steps, resulting in significant computational costs and prolonged response times. While prior research has explored methods for selectively skipping reasoning steps~\cite{Ding:2024cotshortcut, liu2024skipstep}, recent findings~\cite{jin:2024cotlength, Merrill:2024cotlength} suggest that such reductions may conflict with test-time scaling~\cite{o1-blog, snell2025scaling}, ultimately impairing LLM reasoning performance. Therefore, striking an optimal balance between CoT efficiency and reasoning accuracy remains a critical open challenge.

In this work, we delve into CoT efficiency and seek the answer to an important question: \textit{``Does every token in the CoT output contribute equally to deriving the answer?''} We empirically analyze the semantic importance of tokens within CoT outputs and reveal that their contributions to the reasoning performance vary, as depicted in Figure 2. Building on this insight, we introduce \method, a simple yet effective approach that enables LLMs to \textit{skip} less important tokens within CoT sequences and learn shortcuts between critical reasoning tokens, thereby allowing for controllable CoT compression with adjustable ratios. Specifically, as shown in Figure~\ref{fig:intro}, \method constructs compressed CoT training data with various compression ratios, by pruning unimportance tokens from original LLM CoT trajectories. Then, it conducts a general supervised fine-tuning process on target LLMs with this training data, facilitating LLMs to automatically trim redundant tokens during reasoning.

We conduct extensive experiments across various models, including LLaMA-3.1-8B-Instruct and the Qwen2.5-Instruct series, using two widely recognized math reasoning benchmarks: GSM8K and MATH-500. The results validate the effectiveness of \method in compressing CoT outputs while maintaining robust reasoning performance. Notably, Qwen2.5-14B-Instruct exhibits almost \textbf{NO} performance drop (less than $0.4\%$) with a $\bm{40\%}$ reduction in token usage on GSM8K. On the challenging MATH-500 dataset, LLaMA-3.1-8B-Instruct effectively reduces CoT token usage by $\bm{30}\%$ with a performance decline of less than $4\%$, resulting in a $\bm{1.4}\times$ inference speedup. Further analysis underscores the coherence of \method in specified compression ratios and its potential scalability with stronger compression techniques.

\method is distinguished by its low training cost. For Qwen2.5-14B-Instruct, \method fine-tunes only 0.2\% of the model's parameters using LoRA. The size of the compressed CoT training data is no larger than that of the original training set, with 7,473 examples in GSM8K and 7,500 in MATH. The training is completed in approximately 2 hours for the 7B model and 2.5 hours for the 14B model on two 3090 GPUs. These characteristics make \method an efficient and reproducible approach, suitable for use in efficient and cost-effective LLM deployment.

To sum up, our key contributions are:
\begin{enumerate}
    \item To the best of our knowledge, this work is the \textit{first} to investigate the potential of enhancing CoT efficiency through \textit{token skipping}, inspired by the varying semantic importance of tokens in CoT trajectories of LLMs.
    \item We introduce \method, a simple yet effective approach that enables LLMs to skip redundant tokens within CoTs and learn shortcuts between critical tokens, facilitating CoT compression with adjustable ratios.
    \item Our experiments validate the effectiveness of \method. When applied to Qwen2.5-14B-Instruct, \method reduces reasoning tokens by $40\%$ (from 313 to 181) on GSM8K, with less than a $0.4\%$ performance drop.
\end{enumerate}

\section{Background and Preliminaries}
\label{sec:preliminaries}

In this section, we discuss the relevant research background and present preliminary studies on token efficiency in CoT sequences, exploring its impact on the reasoning performance of LLMs.

\subsection{Token Importance}
\label{sec:token-importance}

We first investigate a critical research question to CoT efficiency: \textit{``Does every token in the CoT output contribute equally to deriving the answer?''} In other words, we would like to know if there is any token redundancy in CoT sequences that could be eliminated to improve CoT efficiency.

Token redundancy has been recognized as a longstanding and fundamental issue in LLM efficiency~\cite{hou:2022tokendropbert, zhang2023h2o, lin2024criticaltokenpretrain, Chen:2024FastV}. Recently, it has garnered intensive research attention in prompt compression~\cite{li:2023selective, jiang2023:llmlingua, pan:2024llmlingua2}, which focuses on removing redundant tokens from input prompt to reduce API token usage. To address this issue, Selective Context~\cite{li:2023selective} proposed to measure the importance of tokens in a piece of text based on the semantic confidence of LLMs:
\begin{equation}
I_1\left(x_i\right)=-\log P\left(x_i \mid \bm{x}_{<{i}}; \bm{\theta}_{\M_L}\right),
\label{eq:selectivecontext}
\end{equation}
where $\boldsymbol{x}=\left\{x_i\right\}_{i=1}^{n}$ is the given text, $x_i$ denotes a token, and $\M_L$ denotes the LLM used to compute the confidence of each token. Intuitively, such measurement could be seamlessly applied to CoT tokens generated by LLMs. We show an example of this measurement in Figure~\ref{fig:token-importance}.

\begin{figure}[t]
    \centering
    \resizebox{\columnwidth}{!}{
    \fbox{\parbox[c]{1.1\columnwidth}{
        \textbf{Problem: } Marcus is half of Leo’s age and five years younger than Deanna. Deanna is 26. How old is Leo?

        \vskip 0.1in

        \textbf{Chain-of-Thought: } {\setlength{\fboxsep}{-1pt}
         \colorize{100}{Let}
\colorize{10.6}{'s}
\colorize{58.2}{break}
\colorize{20.8}{it}
\colorize{2}{down}
\colorize{37.9}{step}
\colorize{16}{by}
\colorize{0}{step}
\colorize{41.4}{:}
\colorize{4.6}{1.}
\colorize{100}{Deanna}
\colorize{36.8}{is}
\colorize{48.8}{26}
\colorize{8.6}{years}
\colorize{0}{old}
\colorize{14.6}{.}
\colorize{100}{2.}
\colorize{100}{Marcus}
\colorize{48.0}{is}
\colorize{92.8}{five}
\colorize{3.15}{years}
\colorize{28.7}{younger}
\colorize{0}{than}
\colorize{91.9}{Deanna}
\colorize{15.4}{,}
\colorize{38.9}{so}
\colorize{29.9}{Marcus}
\colorize{4.30}{is}
\colorize{65.5}{26}
\colorize{22.9}{-}
\colorize{1.32}{5}
\colorize{0}{=}
\colorize{0}{21}
\colorize{2.4}{years}
\colorize{2.5}{old}
\colorize{9.3}{.}
\colorize{100}{3.}
\colorize{87.8}{Marcus}
\colorize{41.9}{is}
\colorize{87.1}{half}
\colorize{9.8}{of}
\colorize{100}{Leo}
\colorize{31.8}{'s}
\colorize{0.1}{age}
\colorize{15.1}{,}
\colorize{30.1}{so}
\colorize{12.4}{Leo}
\colorize{14.8}{'s}
\colorize{0}{age}
\colorize{3}{is}
\colorize{12.0}{twice}
\colorize{4.8}{Marcus}
\colorize{3.8}{'s}
\colorize{1.2}{age}
\colorize{3.2}{.}
\colorize{100}{4.}
\colorize{76.6}{Since}
\colorize{100}{Marcus}
\colorize{27.8}{is}
\colorize{38.4}{21,}
\colorize{74.2}{Leo}
\colorize{23.6}{'s}
\colorize{3.1}{age}
\colorize{8.0}{is}
\colorize{22.0}{2}
\colorize{39.0}{x}
\colorize{6.7}{21}
\colorize{6.0}{=}
\colorize{0}{42}
\colorize{9.0}{.}
(Selective Context)

        }

        \vskip 0.1in

        \textbf{Chain-of-Thought: } {\setlength{\fboxsep}{-1pt}
         \colorize{0.7}{Let}
\colorize{2.4}{'s}
\colorize{98.9}{break}
\colorize{11.0}{it}
\colorize{90.3}{down}
\colorize{50.4}{step}
\colorize{39.7}{by}
\colorize{31.9}{step}
\colorize{20.7}{:}
\colorize{47.8}{1.}
\colorize{100}{Deanna}
\colorize{1.6}{is}
\colorize{100}{26}
\colorize{71.0}{years}
\colorize{83.5}{old}
\colorize{25.3}{.}
\colorize{24.7}{2.}
\colorize{100}{Marcus}
\colorize{7.8}{is}
\colorize{96.7}{five}
\colorize{86.6}{years}
\colorize{98.8}{younger}
\colorize{4.4}{than}
\colorize{42.2}{Deanna}
\colorize{6.4}{,}
\colorize{1.3}{so}
\colorize{57.5}{Marcus}
\colorize{1.9}{is}
\colorize{98.2}{26}
\colorize{98.1}{-}
\colorize{97.0}{5}
\colorize{84.9}{=}
\colorize{99.8}{21}
\colorize{74.0}{years}
\colorize{77.5}{old}
\colorize{27.3}{.}
\colorize{21.2}{3.}
\colorize{96.4}{Marcus}
\colorize{7.9}{is}
\colorize{98.0}{half}
\colorize{19.1}{of}
\colorize{99.6}{Leo}
\colorize{94.6}{'s}
\colorize{97.9}{age}
\colorize{3.2}{,}
\colorize{5.6}{so}
\colorize{88.2}{Leo}
\colorize{78.7}{'s}
\colorize{81.2}{age}
\colorize{1.5}{is}
\colorize{98.3}{twice}
\colorize{98.4}{Marcus}
\colorize{87.9}{'s}
\colorize{88.1}{age}
\colorize{73.3}{.}
\colorize{31.4}{4.}
\colorize{2.8}{Since}
\colorize{98.1}{Marcus}
\colorize{4.0}{is}
\colorize{98.2}{21,}
\colorize{98.5}{Leo}
\colorize{91.1}{'s}
\colorize{95.1}{age}
\colorize{3.4}{is}
\colorize{98.8}{2}
\colorize{98.5}{x}
\colorize{99.0}{21}
\colorize{94.6}{=}
\colorize{99.8}{42}
\colorize{98.3}{.}
(LLMLingua-2)

        }

        \vskip 0.05in

        \textbf{Final Answer: } 42.
    }}}
    \caption{Visualization of token importance within a CoT sequence, with darker colors indicating higher values. This figure compares two token importance measurements: Selective Context and LLMLingua-2.}
    \label{fig:token-importance}
\end{figure}

Despite its simplicity, LLMLingua-2~\cite{pan:2024llmlingua2} argued that there exist two major limitations in the aforementioned measurement that hinder the compression performance. Firstly, as shown in Figure~\ref{fig:token-importance}, the intrinsic nature of LLM perplexity leads to lower importance measures (i.e., higher confidence) for tokens at the end of the sentence. Such position dependency impacts the factual importance measurement of each token. Furthermore, the unidirectional attention mechanism in causal LMs may fail to capture all essential information needed for token importance within the text. 

To tackle these limitations, LLMLingua-2 introduced utilizing a bidirectional BERT-like LM~\cite{bert} for token importance measurement. It utilizes GPT-4~\cite{gpt-4} to label each token as ``\textit{important}'' or not and trains the bidirectional LM with a token
classification objective. The token importance is measured by the predicted probability of each token:
\begin{equation}
I_2\left(x_i\right)= P\left(x_i \mid \bm{x}_{\le n}; \bm{\theta}_{\M_B}\right),
\label{eq:llmlingua2}
\end{equation}
where $\M_B$ denotes the bidirectional LM. 

In this study, we apply LLMLingua-2 as the token importance measurement to LLM CoT outputs. Similar to plain text, we observe that the semantic importance of tokens within CoT outputs varies, as shown in Figure~\ref{fig:token-importance}. For instance, mathematical equations tend to have a greater contribution to the final answer, consistent with recent research~\cite{Ma:2024mathmatters}. In contrast, semantic connectors such as ``\textit{so}'' and ``\textit{since}'' generally contribute less. These findings highlight the token redundancy in LLM CoT outputs and the substantial potential to enhance CoT efficiency by trimming this redundancy.

\begin{figure}[t]
\begin{tcolorbox}[colback=blue!5!white,colframe=blue!75!black,title=Revovering the Compressed Chain-of-Thought,fontupper=\footnotesize,fonttitle=\scriptsize]
\textbf{Compressed CoT}: break down Deanna 26 Marcus five younger 26 - 5 21 Marcus half Leo's age twice Marcus Marcus 21, Leo's age 2 x 21 = 42.

\vskip 0.1in
        
\textbf{Recovered Compressed CoT}: Let's break it down step by step. Deanna is 26 years old. Marcus is five years younger than Deanna: M = D - 5. Marcus's age: M = 26 - 5 = 21. Marcus is half of Leo's age: M = L / 2. Leo is twice Marcus's age: L = 2M. Leo's age: L = 2 x 21 = 42.

\end{tcolorbox}
\caption{Recovering the compressed CoT for GSM8K math word problem using LLaMA-3.1-8B-Instruct.}
\label{fig:recovery}
\end{figure}

\subsection{CoT Recovery}
\label{sec:cot-recovery}
We further explore the following research question: \textit{``Are LLMs capable of restoring the CoT process from compressed outputs?''} The answer is yes. As shown in Figure~\ref{fig:recovery} and detailed in Appendix~\ref{appendix:recovery}, examples restored from compressed CoTs using LLaMA-3.1-8B-Instruct demonstrate that LLMs could effectively comprehend the semantic information encoded in the compressed CoT and restore the CoT process. This capability ensures that the interpretability of compressed CoTs is maintained. Additionally, when required by users, the complete CoT process can be recovered and presented.

In summary, the empirical analysis above underscores the potential of trimming redundant tokens to enhance CoT efficiency, as well as the ability of LLMs to restore CoT from compressed outputs. However, enabling LLMs to autonomously skip redundant CoT tokens and identify shortcuts between critical reasoning tokens presents a non-trivial challenge. To the best of our knowledge, this work is the \textit{first} to explore CoT compression through \textit{token skipping}. In the following sections, we present our proposed methodology in detail.


\begin{figure*}[t!]
    \centering
    \includegraphics[width=\linewidth]{Figure/TokenSwift.pdf}
    % \vskip -0.1 in
    \caption{\textbf{Illustration of \ours Framework.} First, target model (LLM) with partial KV cache and three linear layers outputs 4 logits in a single forward pass. Tree-based attention is then applied to construct candidate tokens. Secondly, top-$k$ candidate $4$-grams are retrieved accordingly. These candidates compose draft tokens, which are fed into the LLM with full KV cache to generate target tokens. The verification is performed by checking if draft tokens match exactly with target tokens (\cref{alg:algorithm}). Finally, we randomly select one of the longest valid draft tokens, and update $n$-gram table and KV cache accordingly.}
    \label{fig:frame}
    % \vskip -0.15 in
\end{figure*}

\section{\ours}
\label{sec:method}
To achieve \textbf{lossless acceleration in generating ultra-long sequences}, we propose tailored solutions for each challenge inherent to this process. These solutions are seamlessly integrated into a unified framework, \ie \ours.

\subsection{Overview}
\label{sec:overall}
The overall framework is depicted in \cref{fig:frame}. \ours 
% is highly lightweight and conceptually similar to \ac{sd}. It 
generate a sequence of draft tokens with self-drafting, which are then passed to the target (full) model for validation using a tree-based attention mechanism (See \cref{app:tree_attn} for more tree-based attention details). This process ensures that the final generated output aligns with the target model’s predictions, effectively achieving lossless acceleration.

\ours is lightweight because the draft model is the target model itself with a partial KV cache. This eliminates the need to train a separate draft \ac{llm}; instead, only $\gamma$ linear layers need to be trained, where $\gamma + 1$\footnote{The target model itself can also predict one logit, making the total number of logits $\gamma+1$. We take $\gamma=3$.} represents the number of logits predicted in a single forward pass. In addition, during the verification process, once we obtain the target tokens from the target model with full KV cache, we directly compare draft tokens with target tokens sequentially to ensure that the process is lossless~\citep{rest}.

\subsection{Multi-token Generation and Token Reutilization}
\label{sec:multi_token}
\paragraph{Multi-token Self-Drafting} 
% Inspired by Medusa~\citep{medusa}, we propose a modification where the final output of \ac{llm} is used as input to train $3$ linear layers, enabling the model to generate multiple draft tokens in a single forward pass. However, we argue that the generated draft tokens should not be independent of each other. Unlike Medusa, where the linear layers operate entirely independently, we introduce a simple adjustment to this structure. 
Inspired by Medusa~\citep{medusa}, we enable the \ac{llm} to generate multiple draft tokens in a single forward pass by incorporating $\gamma$ additional linear layers. However, we empirically note that \textbf{the additional linear layers should not be independent of each other}. Specifically, we propose the following structure:
\begin{equation}
\label{equ:ours}
% \small
% \resizebox{.9\hsize}{!}{
% $
    \begin{aligned}
    h_1=f_1(h_0) + h_0,\quad{}h_2=&f_2(h_1) + h_1,\quad{}h_3=f_3(h_2) + h_2,\\
l_0,~l_{1},~l_{2},~l_{3}=&~g(h_0),~g(h_1),~g(h_2),~g(h_3),
    \end{aligned}
% $
% }
\end{equation}
where $h_0$ denotes the last hidden state of \ac{llm}, $f_i(\cdot)$ represents the $i$-th linear layer, $h_i$ refers to the $i$-th hidden representation, $g(\cdot)$ represents the LM Head of target model, and $l_i$ denotes output logits.
% By comparing \cref{equ:medsua} (Medusa) and \cref{equ:ours} (\ours), it is evident that in \ours, the generation of each token depends on the previously generated token, which aligns more closely with the \ac{ar} nature of the model. Moreover, this adjustment incurs no additional computational cost.
This structure aligns more closely with the \ac{ar} nature of the model. Moreover, this adjustment incurs no additional computational cost.
\vspace{-0.05 in}
\paragraph{Token Reutilization} 
% Given the relatively low acceptance rate of using linear to approximate the entire \ac{llm} for generating draft tokens, we propose a method named \textbf{token reutilization}  to further reduce the frequency of model reloads. 
Given the relatively low acceptance rate of using linear layers to generate draft tokens, we propose a method named \textbf{token reutilization} to further reduce the frequency of model reloads. The idea behind token reutilization is that some phrases could appear frequently, and they are likely to reappear in subsequent generations.

% Specifically, we define $(\mathcal{G}, \mathcal{F})$, where $\mathcal{G}=\{x_{i+1}, ..., x_{i+n}\}$ represents an $n$-gram and $\mathcal{F}$ denotes its corresponding frequency $\mathcal{F}$ within the generated token sequence $S=\{x_0, x_1, ..., x_{t-1}\}$ by time step $t$ ($t \geq n$). At subsequent time steps, we use the token generated by target model as the first token to select top-$k$ most frequent $n$-grams $\{\mathcal{G}_1, \mathcal{G}_2,...,\mathcal{G}_k\}$ and incorporate them as additional draft tokens. These selected draft tokens, along with the newly generated ones, are then fed to the \ac{llm} for parallel validation. 
Specifically, we maintain a set of tuples $\{(\mathcal{G}, \mathcal{F})\}$, where $\mathcal{G}=\{x_{i+1}, ..., x_{i+n}\}$ represents an $n$-gram and $\mathcal{F}$ denotes its corresponding frequency $\mathcal{F}$ within the generated token sequence $S=\{x_0, x_1, ..., x_{t-1}\}$ by time step $t$ ($t \geq n$). After obtaining $\{p_0,\ldots, p_3\}$ as described in \S \ref{sec:penalty}, we retrieve the top-$k$ most frequent $n$-grams beginning with token $\arg\max p_0$ to serve as additional draft tokens.

Although this method can be applied to tasks with long prefixes, its efficacy is constrained by the limited decoding steps, which reduces the opportunities for accepting $n$-gram candidates. Additionally, since the long prefix text is not generated by the \ac{llm} itself, a distributional discrepancy exists between the generated text and the authentic text~\citep{detectgpt}. As a result, this method is particularly suitable for generating ultra-long sequences.
 
% \subsection{Dynamic and Memory-Saving KV Pruning}
\subsection{Dynamic KV Cache Management}
\label{sec:kv_update}
\paragraph{Dynamic KV Cache Updates}
Building upon the findings of~\citet{stram_llm}, we preserve the initial $|S|$ KV pairs within the cache during the drafting process, while progressively evicting less important KV pairs. Specifically, we enforce a fixed budget size $|B|$, ensuring that the KV cache at any given time can be represented as:
\begin{equation}
    \nonumber
    % \resizebox{\hsize}{!}{$
    \mathbf{KV}=\{(\mathbf{K}_0,\mathbf{V}_0), ..., (\mathbf{K}_{|S|},\mathbf{V}_{|S|}), (\mathbf{K}_{|S|+1},\mathbf{V}_{|S|+1}),..., (\mathbf{K}_{|B|-1},\mathbf{V}_{|B|-1})\},
   % $},
\end{equation}
where the first $|S|$ pairs remain fixed, and the pairs from position $|S|$ to $|B|-1$ are ordered by decreasing importance. 
As new tokens are generated, less important KV pairs are gradually replaced, starting from the least important ones at position $|B|-1$ and moving towards position $|S|$. Once replacements extend beyond the $|S|$ position, we recalculate the \textit{importance scores} of all preceding tokens and select the most relevant $|B|-|S|$ pairs to reconstruct the cache. 
This process consistently preserves the critical information required for ultra-long sequence generation. 
\vspace{-0.05 in}
% \paragraph{Memory-Saving Top-K Pruning} 
% To implement dynamic updates efficiently, we employ a simple yet effective Top-K pruning strategy. Specifically, we rank the KV pairs based on the importance scores derived from the dot product between the query ($\mathbf{Q}$) and key ($\mathbf{K}$), \ie $\mathbf{Q}\mathbf{K}^T$. 
\paragraph{Importance Score of KV pairs} 
We rank the KV pairs based on the \textit{importance scores} derived from the dot product between the query ($\mathbf{Q}$) and key ($\mathbf{K}$), \ie $\mathbf{Q}\mathbf{K}^T$. 

In the case of Group Query Attention (GQA), since each $\mathbf{K}$ corresponds to a group of $\mathcal{Q}=\{\mathbf{Q}_0, ..., \mathbf{Q}_{g-1}\}$, direct dot-product computation is not feasible. Unlike methods such as SnapKV~\citep{snapkv}, we do not replicate the $\mathbf{K}$. Instead, we partition the $\mathcal{Q}$, as shown in \cref{equ:gqa}:
\begin{equation}
    \label{equ:gqa}
    \vspace{-2mm}
    \text{importance score}_i = \sum_{j=i\cdot g}^{((i+1)\cdot g)-1}\mathbf{Q}_j \cdot \mathbf{K}_i,
        % \vspace{-2mm}
\end{equation}
where for position $i$, $\mathbf{Q}_j$ in the group $\mathcal{Q}_i$ are dot-product with the same $\mathbf{K}_i$, and their results are aggregated to obtain the final \textit{importance score}. This approach enhances memory saving while preserving the quality of the attention mechanism, ensuring that each query is effectively utilized without introducing unnecessary redundancy.

\subsection{Contextual Penalty and Random N-gram Selection}
\label{sec:penalty}
% \paragraph{Contextual Length Penalty} 
\paragraph{Contextual Penalty} 
% To mitigate repetition in generated text, we have explored various sampling strategies. However, with the significantly larger sequence length, the likelihood of repetition increases compared to generating shorter texts (\cref{sec:repeat}). As a result, we decided to apply an additional penalty to the generated tokens to further mitigate repetition.
To mitigate repetition in generated text, we have explored various sampling strategies. However, with the significantly larger sequence length, the likelihood of repetition increases significantly (\S \ref{sec:repeat}). As a result, we decided to apply an additional penalty to the generated tokens to further mitigate repetition.

The penalized sampling approach proposed in \citep{penalty} suggests applying a penalty to all generated tokens. However, when generating ultra-long sequences, the set of generated tokens may cover nearly all common words, which limits the ability to sample appropriate tokens. Therefore, we propose an improvement to this method. 

Specifically, we introduce a fixed \emph{penalty window} $W$ and apply \emph{penalty value} $\theta$ to the most recent $W$ tokens, denoted as $\mathbb{W}$, generated up to the current position, as illustrated in \cref{equ:repeat}: 
\begin{equation}
% \small
    \label{equ:repeat}
% \vspace{-3mm}
    \begin{aligned}
        p_i &= \frac{\exp \big(l_i/(t\cdot I(l_i))\big)}{\sum_j \exp \big(l_j/(t\cdot I(l_j))\big)},\\
    I(l)=\theta\,\,&\text{if}\,\,l \in \mathbb{W}\,\text{else}\,\,1.0,\quad \theta \in (1, \infty),
    \end{aligned}
    % \vspace{-1mm}
\end{equation}
where $t$ denotes temperature, $l_i$ and $p_i$ represent the logit and probability of $i$-th token. This adjustment aims to maintain diversity while still mitigating repetitive generation.

\section{\thename}
\subsection{End-to-End Driving Policy}
The overall framework of \thename{} is depicted in Fig.~\ref{fig:framework}. 
\thename{} takes multi-view image sequences as input, transforms the sensor data into scene token embeddings, outputs the probabilistic distribution of actions, and samples an action to control the vehicle. 

\boldparagraph{BEV Encoder.} 
We first employ a BEV encoder~\cite{li2022bevformer} to transform multi-view image features from the perspective view to the Bird's Eye View (BEV), obtaining a feature map in the BEV space. This feature map is then used to learn instance-level map features and agent features.

\boldparagraph{Map Head.} 
Then we utilize a group of map tokens~\cite{maptrv2, liao2022maptr, lanegap} to learn the vectorized map elements of the driving scene from the BEV feature map, including lane centerlines, lane dividers, road boundaries, arrows, traffic signals, \etc.

\boldparagraph{Agent Head.} 
Besides, a group of agent tokens~\cite{jiang2022pip} is adopted to predict the motion information of other traffic participants, including location, orientation, size, speed, and multi-mode future trajectories.

\boldparagraph{Image Encoder.} 
Apart from the above instance-level map and agent tokens, we also use an individual image encoder~\cite{vit,he2016resnet} to transform the original images into image tokens. These image tokens provide dense and rich scene information for planning, complementary to the instance-level tokens.

\begin{figure}[t]
\centering
\includegraphics[width=0.98\linewidth]{fig/post-training-2.pdf} 
\caption{\textbf{Post-training.}  $N$  workers parallelly run. The generated rollout data $(s_t,a_t, r_{t+1},s_{t+1},...)$ are recorded in a rollout buffer. Rollout data and human driving demonstrations are used in RL- and IL-training steps to fine-tune the AD policy synergistically.
}
\label{fig:post-training}
\end{figure}

\boldparagraph{Action Space.} 
To accelerate the convergence of RL training, we design a decoupled discrete action representation. 
We divide the action into two independent components: lateral action and longitudinal action. 
The action space is constructed over a short $0.5$-second time horizon, during which the vehicle's motion is approximated by assuming constant linear and angular velocities. 
Under this assumption, the lateral action $a^x$ and longitudinal action $a^y$ can be directly computed based on the current linear and angular velocities.
By combining decoupling with a limited temporal scope and simplified motion model, our approach effectively reduces the dimensionality of the action space, accelerating training convergence.


\boldparagraph{Planning Head.} 
We use $E_\text{scene}$ to denote the scene representation, which consists of map tokens, agent tokens, and image tokens. We initialize a planning embedding denoted as $E_\text{plan}$. A cascaded Transformer decoder $\phi$ takes the planning embedding $E_\text{plan}$ as the query and the scene representation $E_\text{scene}$ as both key and value.

The output of the decoder $\phi$ is then combined with navigation information $E_\text{navi}$ and ego state $E_\text{state}$ to output the probabilistic distributions of the lateral action $a^x$ and the longitudinal action $a^y$:
\begin{equation}
\begin{aligned}
     \pi(a^x\mid s) = & \text{softmax}(\text{MLP}(\phi(E_\text{plan}, E_\text{scene}) \\
    & + E_\text{navi} + E_\text{state})), \\
     \pi(a^y\mid s) = & \text{softmax}(\text{MLP}(\phi(E_\text{plan}, E_\text{scene}) \\
     & + E_\text{navi} + E_\text{state})),
\label{eq:action distribution}
\end{aligned}
\end{equation}
where $E_\text{plan}$, $E_\text{navi}$, $E_\text{state}$, and the output of $\text{MLP}$ are all of the same dimension ($1 \times D$).

The planning head also outputs the value functions $V_x(s)$ and $V_y(s)$, which estimate the expected cumulative rewards for the lateral and longitudinal actions, respectively: 
\begin{equation}
\begin{aligned}
    & V_x(s) = \text{MLP}(\phi(E_\text{plan}, E_\text{scene}) + E_\text{navi} + E_\text{state}), \\
    & V_y(s) = \text{MLP}(\phi(E_\text{plan}, E_\text{scene}) + E_\text{navi} + E_\text{state}).
\end{aligned}
\end{equation}
The value functions are used in RL training (Sec.~\ref{sec:optimization}).

\subsection{Training Paradigm}
We adopt a three-stage training paradigm: perception pre-training, planning pre-training, and reinforced post-training, as shown in Fig.~\ref{fig:framework}.

\boldparagraph{Perception Pre-Training.} 
Information in the image is sparse and low-level. In the first stage,  
the map head and the agent head explicitly output map elements and agent motion information, which are supervised with ground-truth labels. Consequently,  
map tokens and agent tokens implicitly encode the corresponding high-level information.  
In this stage, we only update the parameters of the BEV encoder, the map head, and the agent head.



\boldparagraph{Planning Pre-Training.} 
In the second stage, to prevent the unstable cold start of RL training, IL is first performed to initialize the probabilistic distribution of actions based on large-scale real-world driving demonstrations from expert drivers. In this stage, we only update the parameters of the image encoder and the planning head, while the parameters of the BEV encoder, map head, and agent head are frozen. The optimization objectives of perception tasks and planning tasks may conflict with each other. However, with the training stage and parameters decoupled, such conflicts are mostly avoided.

\boldparagraph{Reinforced Post-Training.} 
In the reinforced post-training, RL and IL synergistically fine-tune the distribution. RL aims to guide the policy to be sensitive to critical risky events and adaptive to out-of-distribution situations. IL serves as the regularization term to keep the policy's behavior similar to that of humans.

We select a large amount of risky dense-traffic clips from collected driving demonstrations. For each clip, we train an independent 3DGS model that reconstructs the clip and serves as a digital driving environment.  
As shown in Fig.~\ref{fig:post-training}, we set $N$ parallel workers.  
Each worker randomly samples a 3DGS environment and begins rollout, i.e., the AD policy controls the ego vehicle to move and iteratively interacts with the 3DGS environment. After the rollout process of this 3DGS environment ends, the generated rollout data $(s_t,a_t, r_{t+1},s_{t+1},...)$ are recorded in a rollout buffer, and the worker will sample a new 3DGS environment for another round of rollout.

As for policy optimization, we iteratively perform RL-training steps and IL-training steps. For RL-training steps, we sample data from the rollout buffer and follow the Proximal Policy Optimization (PPO) framework~\cite{PPO} to update the AD policy. For IL-training steps, we use real-world driving demonstrations to update the policy. After a fixed number of training steps, the updated AD policy is sent to every worker to replace the old one, to avoid a distribution shift between data collection and optimization.
We only update the parameters of the image encoder and the planning head. The parameters of the BEV encoder, the map head, and the agent head are frozen.  
The detailed RL design is presented below.

\subsection{Interaction Mechanism between AD Policy and 3DGS Environment}
In the 3DGS environment, the ego vehicle acts according to the AD policy. Other traffic participants act according to real-world data in a log-replay manner.  
A simplified kinematic bicycle model is employed to iteratively update the ego vehicle's pose at every $\Delta t$ seconds as follows:  
\begin{equation}
\begin{aligned}
x_{t+1}^{w} & = x_{t}^w + v_t \cos \left(\psi_{t}^w\right) \Delta t, \\
y_{t+1}^{w} & = y_{t}^w + v_t \sin \left(\psi_{t}^w\right) \Delta t, \\
\psi_{t+1}^{w} & = \psi_{t}^w + \frac{v_t}{L} \tan \left(\delta_t\right) \Delta t,
\label{equation:kinematic_model}
\end{aligned}
\end{equation}  
where $x_t^{w}$ and $y_t^{w}$ denote the position of the ego vehicle relative to the world coordinate; $\psi_t^w$ is the heading angle that defines the vehicle's orientation with respect to the world $x$-coordinate; $v_t$ is the linear velocity of the ego vehicle; $\delta_t$ is the steering angle of the front wheels; and $L$ is the wheelbase, i.e., the distance between the front and rear axles.

During the rollout process, the AD policy outputs actions $(a_t^x, a_t^y)$ for a $0.5$-second time horizon at time step $t$. We derive the linear velocity $v_t$ and steering angle $\delta_t$ based on $(a_t^x, a_t^y)$.  
Based on the kinematic model in Eq.~\ref{equation:kinematic_model},  
the pose of the ego vehicle in the world coordinate system is updated from ${p}_t = (x_{t}^w, y_{t}^w, \psi_{t}^w)$ to ${p}_{t+1} = (x_{t+1}^{w}, y_{t+1}^{w}, \psi_{t+1}^{w})$.  

Based on the updated ${p}_{t+1}$, the 3DGS environment computes the new ego vehicle's state $s_{t+1}$. The updated pose ${p}_{t+1}$ and state $s_{t+1}$ serve as the input for the next iteration of the inference process.

The 3DGS environment also generates rewards $\mathcal{R}$ (Sec.~\ref{sec:reward}) according to multi-source information (including trajectories of other agents, map information, the expert trajectory of the ego vehicle, and the parameters of Gaussians), which are used to optimize the AD policy (Sec.~\ref{sec:optimization}).

\begin{figure}[t]
\centering
\includegraphics[width=1.0\linewidth]{fig/reward.pdf} 
\caption{\textbf{Example diagram of four types of reward sources.}  (1): Collision with a dynamic obstacle ahead triggers a reward $r_{\text{dc}}$. (2): Hitting a static roadside obstacle incurs a reward $r_{\text{sc}}$. (3): Moving onto the curb exceeds the positional deviation threshold $d_{\text{max}}$, triggering a reward $r_{\text{pd}}$. (4): Drifting toward the adjacent lane exceeds the heading deviation threshold $\psi_{\text{max}}$, triggering a reward $r_{\text{hd}}$.
}
\label{fig: reward source}
\end{figure}
\subsection{Reward Modeling}
\label{sec:reward}
The reward is the source of the training signal, which determines the optimization direction of RL. The reward function is designed to guide the ego vehicle's behavior by penalizing unsafe actions and encouraging alignment with the expert trajectory. It is composed of four reward components: (1) collision with dynamic obstacles, (2) collision with static obstacles, (3) positional deviation from the expert trajectory, and (4) heading deviation from the expert trajectory:
\begin{equation}
\begin{aligned}
\mathcal{R} = \{r_{\text{dc}}, r_{\text{sc}}, r_{\text{pd}}, r_{\text{hd}}  \}. 
\end{aligned}
\end{equation}

As illustrated in Fig.~\ref{fig: reward source}, these reward components are triggered under specific conditions.  
In the 3DGS environment, dynamic collision is detected if the ego vehicle's bounding box overlaps with the annotated bounding boxes of dynamic obstacles, triggering a negative reward $r_{\text{dc}}$. Similarly, static collision is identified when the ego vehicle's bounding box overlaps with the Gaussians of static obstacles, resulting in a negative reward $r_{\text{sc}}$.  
Positional deviation is measured as the Euclidean distance between the ego vehicle's current position and the closest point on the expert trajectory. A deviation beyond a predefined threshold $d_{\text{max}}$ incurs a negative reward $r_{\text{pd}}$.  
Heading deviation is calculated as the angular difference between the ego vehicle's current heading angle $ \psi_t $ and the expert trajectory's matched heading angle $\psi_{\text{expert}}$. A deviation beyond a threshold $ \psi_{\text{max}}$ results in a negative reward $r_{\text{hd}}$.

Any of these events, including dynamic collision, static collision, excessive positional deviation, or excessive heading deviation, triggers immediate episode termination. Because after such events occur, the 3DGS environment typically generates noisy sensor data, which is detrimental to RL training.

\subsection{Policy Optimization}
\label{sec:optimization}
In the closed-loop environment, the error in each single step accumulates over time. The aforementioned rewards are not only caused by the current action but also by the actions of the preceding steps.  
The rewards are propagated forward with Generalized Advantage Estimation (GAE)~\cite{gae} to optimize the action distribution of the preceding steps.

Specifically, for each time step $t$, we store the current state $s_t$, action $a_t$, reward $r_t$, and the estimate of the value $V(s_t)$.  
Based on the decoupled action space, and considering that different rewards have different correlations to lateral and longitudinal actions, the reward $r_t$ is divided into lateral reward $r_t^x$ and longitudinal reward $r_t^y$:
\begin{equation}
\begin{aligned}
r_t^x &= r_t^{\text{sc}} + r_t^{\text{pd}} + r_t^{\text{hd}}, \\
r_t^y &= r_t^{\text{dc}}.
\label{eq:reward-decouple}
\end{aligned}
\end{equation}
Similarly, the value function $V(s_t)$ is decoupled into two components: $V_x(s_t)$ for the lateral dimension and $V_y(s_t)$ for the longitudinal dimension. These value functions estimate the expected cumulative rewards for the lateral and longitudinal actions, respectively. The advantage estimates $\hat{A}_t^x$ and $\hat{A}_t^y$ are then computed as follows:
\begin{equation}
\begin{aligned}
\delta_t^x &= r_t^x + \gamma V_x(s_{t+1}) - V_x(s_t), \\
\delta_t^y &= r_t^y + \gamma V_y(s_{t+1}) - V_y(s_t), \\
\hat{A}_t^x &= \sum_{l=0}^{\infty}(\gamma \lambda)^l \delta_{t+l}^x, \\
\hat{A}_t^y &= \sum_{l=0}^{\infty}(\gamma \lambda)^l \delta_{t+l}^y,
\label{eq:advantage}
\end{aligned}
\end{equation}
where $\delta_t^x$ and $\delta_t^y$ are the temporal difference errors for the lateral and longitudinal dimensions, $\gamma$ is the discount factor, and $\lambda$ is the GAE parameter that controls the trade-off between bias and variance.

To further clarify the relationship between the advantage estimates and the reward components, we decompose $\hat{A}_t^x$ and $\hat{A}_t^y$ based on the reward decomposition in Eq.~\ref{eq:reward-decouple} and the advantage estimation in Eq.~\ref{eq:advantage}. Specifically, we derive the following decomposition:
\begin{equation}
\begin{aligned}
\hat{A}_t^x &= \hat{A}_t^{\text{sc}} + \hat{A}_t^{\text{pd}} + \hat{A}_t^{\text{hd}}, \\
\hat{A}_t^y &= \hat{A}_t^{\text{dc}},
\end{aligned}
\end{equation}
where $\hat{A}_t^{\text{sc}}$ is the advantage estimate for avoiding static collisions, $\hat{A}_t^{\text{pd}}$ is the advantage estimate for minimizing positional deviations, $\hat{A}_t^{\text{hd}}$ is the advantage estimate for minimizing heading deviations, and $\hat{A}_t^{\text{dc}}$ is the advantage estimate for avoiding dynamic collisions.

These advantage estimates are used to guide the update of the AD policy $\pi_{\theta}$, following the PPO framework~\cite{PPO}. By leveraging the decomposed advantage estimates $\hat{A}_t^x$ and $\hat{A}_t^y$, we can independently optimize the lateral and longitudinal dimensions of the policy. This is achieved by defining separate objective functions $\mathcal{L}_x^{\text{CLIP}}(\theta)$ and $\mathcal{L}_y^{\text{CLIP}}(\theta)$ for each dimension,  as follows:
\begin{equation}
\begin{aligned}
\mathcal{L}_x^{\text{PPO}}(\theta) &= \mathbb{E}_t \left[ \min \left( \rho_t^x \hat{A}_t^x, \ \text{clip}(\rho_t^x, 1-\epsilon_x, 1+\epsilon_x) \hat{A}_t^x \right) \right], \\
\mathcal{L}_y^{\text{PPO}}(\theta) &= \mathbb{E}_t \left[ \min \left( \rho_t^y \hat{A}_t^y, \ \text{clip}(\rho_t^y, 1-\epsilon_y, 1+\epsilon_y) \hat{A}_t^y \right) \right], \\
\mathcal{L}^{\text{PPO}}(\theta) &= \mathcal{L}_x^{\text{PPO}}(\theta) + \mathcal{L}_y^{\text{PPO}}(\theta),
\end{aligned}
\end{equation}
where $\rho_t^x = \frac{\pi_{\theta}(a_t^x \mid s_t)}{\pi_{\theta_{\text{old}}}(a_t^x \mid s_t)}$ is the importance sampling ratio for the lateral dimension, $\rho_t^y = \frac{\pi_{\theta}(a_t^y \mid s_t)}{\pi_{\theta_{\text{old}}}(a_t^y \mid s_t)}$ is the importance sampling ratio for the longitudinal dimension, $\epsilon_x$ and $\epsilon_y$ are small constants that control the clipping range for the lateral and longitudinal dimensions, ensuring stable policy updates.

The clipped objective function $\mathcal{L}^{\text{PPO}}(\theta)$ prevents excessively large updates to the policy parameters $\theta$, thereby maintaining training stability.

\begin{table*}[ht]
    \centering
{
\begin{tabular}{lccccccccc}
    \toprule
    RL:IL & CR$\downarrow$ & DCR$\downarrow$ & SCR$\downarrow$ & DR$\downarrow$ & PDR$\downarrow$ & HDR$\downarrow$ &ADD$\downarrow$ & Long. Jerk$\downarrow$ & Lat. Jerk$\downarrow$ \\
    \midrule
     0:1  & 0.229 & 0.211 & 0.018 & 0.066 & 0.039 & 0.027  & 0.238 & 3.928 & 0.103\\
     1:0  & 0.143 & 0.128 & 0.015 &0.080 &0.065 &0.015 &0.345 &4.204 &0.085\\
     2:1 & 0.137 & 0.125 & 0.012 & 0.059 & 0.050 & 0.009  & 0.274 & 4.538 & 0.092\\
     4:1 & 0.089 & 0.080 & 0.009 & 0.063 & 0.042 & 0.021  & 0.257 & 4.495 & 0.082 \\
     8:1 & 0.125 & 0.116 & 0.009 & 0.084 & 0.045 & 0.039  & 0.323 & 5.285 & 0.115\\
    \bottomrule
\end{tabular}
}
    \caption{\textbf{Ablation on RL-to-IL step mixing ratios in the reinforced post-training stage.}}
    \label{tab:ratio}
\end{table*}

\subsection{Auxiliary Objective}
RL usually faces the challenge of sparse rewards, which makes the convergence process unstable and slow. To speed up convergence, we introduce auxiliary objectives that provide dense guidance to the entire action distribution.

The auxiliary objectives are designed to penalize undesirable behaviors by incorporating specific reward sources, including dynamic collisions, static collisions, positional deviations, and heading deviations. These objectives are computed based on the actions \( a_t^{x, \text{old}} \) and \( a_t^{y, \text{old}} \) selected by the old AD policy \( \pi_{\theta_{\text{old}}} \) at time step \( t \). To facilitate the evaluation of these actions, we separate the probability distribution of the action into four parts:
\begin{equation}
\begin{aligned}
\Delta \pi_y^{\text{dec}} &= \sum_{a_t^y < a_t^{y, \text{old}}} \pi_\theta(a_t^y \mid s_t), \\
\Delta \pi_y^{\text{acc}} &= \sum_{a_t^y > a_t^{y, \text{old}}} \pi_\theta(a_t^y \mid s_t), \\
\Delta \pi_x^{\text{left}} &= \sum_{a_t^x < a_t^{x, \text{old}}} \pi_\theta(a_t^x \mid s_t), \\
\Delta \pi_x^{\text{right}} &= \sum_{a_t^x > a_t^{x, \text{old}}} \pi_\theta(a_t^x \mid s_t).
\end{aligned}
\end{equation}
Here, \( \Delta \pi_y^{\text{dec}} \) represents the total probability of deceleration actions, \( \Delta \pi_y^{\text{acc}} \) represents the total probability of acceleration actions, \( \Delta \pi_x^{\text{left}} \) represents the total probability of leftward steering actions, and \( \Delta \pi_x^{\text{right}} \) represents the total probability of rightward steering actions.

\boldparagraph{Dynamic Collision Auxiliary Objective.}  
The dynamic collision auxiliary objective adjusts the longitudinal control action \(a_t^y\) based on the location of potential collisions relative to the ego vehicle. If a collision is detected ahead, the policy prioritizes deceleration actions (\(a_t^y < a_t^{y, \text{old}}\)); if a collision is detected behind, it encourages acceleration actions (\(a_t^y > a_t^{y, \text{old}}\)). To formalize this behavior, we define a directional factor \(f_\text{dc}\):
\begin{equation}
\begin{aligned}
f_\text{dc} = \begin{cases} 
1 & \text{if the collision is ahead}, \\
-1 & \text{if the collision is behind}.
\end{cases} 
\end{aligned}
\end{equation}

The auxiliary objective for dynamic collision avoidance is defined as:
\begin{equation}
\begin{aligned}
\mathcal{L}_\text{dc}(\theta_y) = \mathbb{E}_t \left[ 
    \hat{A}_t^\text{dc} \cdot f_\text{dc} \cdot (\Delta \pi_y^{\text{dec}} - \Delta \pi_y^{\text{acc}})
\right],
\end{aligned}
\end{equation}
where \(\hat{A}_t^\text{dc}\) is the advantage estimate for dynamic collision avoidance.

\boldparagraph{Static Collision Auxiliary Objective.}  
The static collision auxiliary objective adjusts the steering control action $a_t^x$ based on the proximity to static obstacles. If the static obstacle is detected on the left side, the policy promotes rightward steering actions ($a_t^x > a_t^{x,\text{old}}$); if the static obstacle is detected on the right side, it promotes leftward steering actions ($a_t^x < a_t^{x,\text{old}}$). To formalize this behavior, we define a directional factor $f_\text{sc}$:  
\begin{equation}
\begin{aligned}
f_\text{sc} = \begin{cases} 
1 & \text{if static obstacle is on the left}, \\
-1 & \text{if static obstacle is on the right}.
\end{cases} 
\end{aligned}
\end{equation}

The auxiliary objective for static collision avoidance is defined as:  
\begin{equation}
\begin{aligned}
\mathcal{L}_\text{sc}(\theta_x) = \mathbb{E}_t \left[ 
    \hat{A}_t^\text{sc} \cdot f_\text{sc} \cdot (\Delta \pi_x^{\text{right}} - \Delta \pi_x^{\text{left}})
\right],
\end{aligned}
\end{equation}  
where $\hat{A}_t^\text{sc}$ is the advantage estimate for static collision avoidance.  

\boldparagraph{Positional Deviation Auxiliary Objective.}  
The positional deviation auxiliary objective adjusts the steering control action $a_t^x$ based on the ego vehicle's lateral deviation from the expert trajectory. If the ego vehicle deviates leftward, the policy promotes rightward corrections ($a_t^x > a_t^{x,\text{old}}$); if it deviates rightward, it promotes leftward corrections ($a_t^x < a_t^{x,\text{old}}$). We formalize this with a directional factor $f_\text{pd}$:  
\begin{equation}
\begin{aligned}
f_\text{pd} = \begin{cases} 
1 & \text{if ego vehicle deviates leftward}, \\
-1 & \text{if ego vehicle deviates rightward}.
\end{cases} 
\end{aligned}
\end{equation}

The auxiliary objective for positional deviation correction is:
\begin{equation}
\begin{aligned}
\mathcal{L}_\text{pd}(\theta_x) = \mathbb{E}_t \left[ 
    \hat{A}_t^\text{pd} \cdot f_\text{pd} \cdot (\Delta \pi_x^{\text{right}} - \Delta \pi_x^{\text{left}})
\right],
\end{aligned}
\end{equation}  
where $\hat{A}_t^\text{pd}$ estimates the advantage of trajectory alignment.

\boldparagraph{Heading Deviation Auxiliary Objective.}  
The heading deviation auxiliary objective adjusts the steering control action $a_t^x$ based on the angular difference between the ego vehicle’s current heading and the expert’s reference heading. If the ego vehicle deviates counterclockwise, the policy promotes clockwise corrections ($a_t^x > a_t^{x,\text{old}}$); if it deviates clockwise, it promotes counterclockwise corrections ($a_t^x < a_t^{x,\text{old}}$). To formalize this behavior, we define a directional factor $f_\text{hd}$:  
\begin{equation}
\begin{aligned}
f_\text{hd} = \begin{cases} 
1 & \text{if ego vehicle deviates clockwise}, \\
-1 & \text{if ego vehicle deviates counterclockwise}.
\end{cases} 
\end{aligned}
\end{equation}

The auxiliary objective for heading deviation correction is then defined as:  
\begin{equation}
\begin{aligned}
\mathcal{L}_\text{hd}(\theta_x) = \mathbb{E}_t \left[ 
    \hat{A}_t^\text{hd} \cdot f_\text{hd} \cdot (\Delta \pi_x^{\text{right}} - \Delta \pi_x^{\text{left}})
\right],
\end{aligned}
\end{equation}  
where $\hat{A}_t^\text{hd}$ is the advantage estimate for heading alignment.  

\begin{table*}[ht]
\begin{center}
\centering
\resizebox{0.98\textwidth}{!}{
\begin{tabular}{cccccccccccccc}
\toprule
\multirow{2}{*}{ID} & Dynamic & Static & Position & Heading & \multirow{2}{*}{CR$\downarrow$} &\multirow{2}{*}{DCR$\downarrow$} &\multirow{2}{*}{SCR$\downarrow$} &\multirow{2}{*}{DR$\downarrow$} &\multirow{2}{*}{PDR$\downarrow$} &\multirow{2}{*}{HDR$\downarrow$} &\multirow{2}{*}{ADD$\downarrow$} &\multirow{2}{*}{Long. Jerk$\downarrow$} &\multirow{2}{*}{Lat. Jerk$\downarrow$}\\
& Collision & Collision & Deviation & Deviation & & & & & & & & & \\
\midrule
1 & \cmark  &  &  &  & 0.172 & 0.154 & 0.018 & 0.092 & 0.033 & 0.059  & 0.259 & 4.211 & 0.095 \\
2 &  & \cmark & \cmark & \cmark & 0.238 & 0.217 & 0.021 & 0.090 & 0.045 & 0.045  & 0.241 & 3.937 & 0.098 \\
3 & \cmark &  & \cmark & \cmark & 0.146 & 0.128 & 0.018 & 0.060 & 0.030 & 0.030  & 0.263 & 3.729 & 0.083\\
4 & \cmark & \cmark &  & \cmark & 0.151 & 0.142 & 0.009 & 0.069 & 0.042 & 0.027 & 0.303 & 3.938 & 0.079\\
5 & \cmark & \cmark & \cmark &  & 0.166 & 0.157 & 0.009 & 0.048 & 0.036 & 0.012 & 0.243 & 3.334 & 0.067\\
6 & \cmark & \cmark & \cmark & \cmark & 0.089 & 0.080 & 0.009 & 0.063 & 0.042 & 0.021 & 0.257 & 4.495 & 0.082 \\
\bottomrule
\end{tabular}
}
\end{center}
\vspace{-2mm}
\caption{\textbf{Ablation on reward sources.} The table shows the impact of different reward components on performance.}
\label{tab:reward_ablation}
\end{table*}

\begin{table*}[ht]
\begin{center}
\centering
\resizebox{0.98\textwidth}{!}{
\begin{tabular}{ccccccccccccccc}
\toprule
\multirow{2}{*}{ID} & \multirow{2}{*}{PPO Obj.}  & Dynamic Col. & Static Col. & Position Dev. & Heading Dev. & \multirow{2}{*}{CR$\downarrow$} & \multirow{2}{*}{DCR$\downarrow$}  & \multirow{2}{*}{SCR$\downarrow$} & \multirow{2}{*}{DR$\downarrow$} & \multirow{2}{*}{PDR$\downarrow$} & \multirow{2}{*}{HDR$\downarrow$} & \multirow{2}{*}{ADD$\downarrow$} & \multirow{2}{*}{Long. Jerk$\downarrow$} & \multirow{2}{*}{Lat. Jerk$\downarrow$} \\
& & Auxiliary Obj. & Auxiliary Obj. & Auxiliary Obj. & Auxiliary Obj. & & & & & & & & & \\
\midrule
1 &\cmark&  &  &  &  & 0.249 & 0.223 & 0.026 & 0.077 & 0.047 & 0.030  & 0.266 & 4.209 & 0.104 \\
2 &\cmark& \cmark &  &  &  & 0.178 & 0.163 & 0.015 & 0.151 & 0.101 & 0.050 & 0.301 & 3.906 & 0.085 \\
3 &\cmark&  & \cmark & \cmark & \cmark & 0.137 & 0.125 & 0.012 & 0.157 & 0.145 & 0.012 & 0.296 & 3.419 & 0.071 \\
4 &\cmark& \cmark &  & \cmark & \cmark & 0.169 & 0.151 & 0.018 & 0.075 & 0.042 & 0.033 & 0.254 & 4.450 & 0.098 \\
5 &\cmark& \cmark & \cmark &  & \cmark & 0.149 & 0.134 & 0.015 & 0.063 & 0.057 & 0.006 & 0.324 & 3.980 & 0.086 \\
6 &\cmark& \cmark & \cmark & \cmark & & 0.128 & 0.119  & 0.009 & 0.066 & 0.030 & 0.036  & 0.254 & 4.102 & 0.092 \\
7 &&\cmark  &\cmark  &\cmark  &\cmark  & 0.187 &0.175  &0.012 &0.077 &0.056  &0.021  &0.309  &5.014  &0.112  \\
8 &\cmark& \cmark & \cmark & \cmark & \cmark & 0.089 & 0.080 & 0.009 & 0.063 & 0.042 & 0.021  & 0.257 & 4.495 & 0.082 \\
\bottomrule
\end{tabular}
}
\end{center}
\vspace{-2mm}
\caption{\textbf{Ablation on auxiliary objectives.} The table shows the impact of different auxiliary objectives on performance.}
\label{tab:auxiliary_ablation}
\end{table*}

\boldparagraph{Overall Auxiliary Objectives.}  
The overall auxiliary objectives are a weighted sum of the individual objectives:
\begin{equation}
\begin{aligned}
\mathcal{L}_\text{aux}(\theta) = &\lambda_1 \mathcal{L}_\text{dc}(\theta_y) + \lambda_2 \mathcal{L}_\text{sc}(\theta_x)  + \\ 
&\lambda_3 \mathcal{L}_\text{pd}(\theta_x) +\lambda_4 \mathcal{L}_\text{hd}(\theta_x),
\end{aligned}
\end{equation}
where $\lambda_1$, $\lambda_2$, $\lambda_3$, and $\lambda_4$ are weighting coefficients that balance the contributions of each auxiliary objective.

\boldparagraph{Optimization Objective.}  
The final optimization objective combines the clipped PPO objective with the auxiliary objective:
\begin{equation}
\mathcal{L}(\theta) = \mathcal{L}^{\text{PPO}}(\theta) + \mathcal{L}_\text{aux}(\theta).
\end{equation}

\vspace{-0.05 in}
\paragraph{Random $n$-gram Selection}
% In the process of reutilizing generated $n$-grams as draft tokens and applying repetition penalty, there exists an inherent trade-off. Meanwhile, 

In our experiments, we observe that the draft tokens provided to the target model for parallel validation often yield multiple valid groups. Building on this observation, we randomly select one valid $n$-gram to serve as the final output. By leveraging the fact that multiple valid $n$-grams emerge during verification, we ensure that the final output is both diverse and accurate.

% we observe that the draft tokens provided to the target model for parallel validation can yield multiple valid groups.

In summary, the overall flow of our framework is presented in \cref{alg:algorithm}. 

\section{Experiment}\label{sec: exp}
In this section, we assess the efficacy of our algorithm by addressing the following key questions. 
(1) Can offline RL algorithms achieve stronger performance on the reduced datasets selected by~\name?
(2) How does \name~perform compare to other offline data selection methods? 
(3) What are the factors that contribute to \name's effectiveness?

\begin{figure}[t]
    \centering
    \subfigure{\includegraphics[scale=0.24]{d4rl-hard/walker2d-medium-v0-hard.pdf}}
    \hspace{0.2cm}
    \subfigure{\includegraphics[scale=0.24]{d4rl-hard/walker2d-expert-v0-hard.pdf}}
    \hspace{0.2cm}
    \subfigure{\includegraphics[scale=0.24]{d4rl-hard/walker2d-medium-replay-v0-hard.pdf}}
    % \subfigure{\includegraphics[scale=0.20]{d4rl-hard/walker2d-medium-expert-v0-hard.pdf}}
    \subfigure{\includegraphics[scale=0.24]{d4rl-hard/hopper-medium-v0-hard.pdf}}
    \hspace{0.2cm}
    \subfigure{\includegraphics[scale=0.24]{d4rl-hard/hopper-expert-v0-hard.pdf}}
    \hspace{0.2cm}
    \subfigure{\includegraphics[scale=0.24]{d4rl-hard/hopper-medium-replay-v0-hard.pdf}}
    % \subfigure{\includegraphics[scale=0.20]{d4rl-hard/hopper-medium-expert-v0-hard.pdf}}
    \subfigure{\includegraphics[scale=0.24]{d4rl-hard/halfcheetah-medium-expert-v0-hard.pdf}}
    \hspace{0.2cm}
    \subfigure{\includegraphics[scale=0.24]{d4rl-hard/halfcheetah-expert-v0-hard.pdf}}
    \hspace{0.2cm}
    \subfigure{\includegraphics[scale=0.24]{d4rl-hard/halfcheetah-medium-replay-v0-hard.pdf}}
    % \subfigure{\includegraphics[scale=0.20]{d4rl-hard/halfcheetah-medium-v0-hard.pdf}}
    \caption{Experimental results on the D4RL (Hard) offline datasets. All experiment results were averaged over five random seeds. Our method achieves better or
    comparable results than the baselines with lower computational complexity.}
    \label{fig: d4rl hard}
    \vspace{-0.5cm}
\end{figure}

% \begin{figure*}[t]
%     \centering
%     \includegraphics[width=\linewidth]{mujoco/fig1.pdf}
%     \vspace{-2em}
%     \caption{Sample-based selection performance of several baselines and \name~with different selected subset sizes~($x\%$).
%     The horizontal line is the performance of TD3+BC trained with the original dataset.}
%     \label{fig: d4rl minimal ratio}
%     \vspace{-1em}
% \end{figure*}

% \begin{figure}[t]
%     \centering
%     \includegraphics[width=\linewidth]{mujoco/traj.pdf}
%     \caption{In trajectory-based selection, \name~outperforms behavior cloning (\nameh) using trajectories with the highest accumulative returns, presenting a robust method for selecting the most useful data from training sets of compromised quality.}
%     \label{fig: d4rl topbc}
%     \vspace{-1em}
% \end{figure}

\subsection{Setup}
We evaluate algorithms on the offline RL benchmark D4RL~\citep{fu2020d4rl} to answer the aforementioned questions.
In addition, we consider a more challenging scenario where we add additional low-quality data to the dataset to simulate noise in real-world tasks, named D4RL~(hard).
The evaluation process commences with the selection of offline data, followed by the training of a widely recognized offline RL algorithm, TD3+BC~\citep{fujimoto2021minimalist}, on this reduced dataset for 1 million time steps.
To ensure a fair comparison, we apply the same offline RL algorithm to data subsets obtained by different algorithms. 
Evaluation points are set at every 5,000 training time steps and involve calculating the return of 10 episodes per point.
The results, comprising averages and standard deviations, are computed with five independent random seeds.
On the other hand, we can also incorporate our method into offline model-based approaches, such as MOPO~\citep{yu2020mopo} and MoERL~\citep{kidambi2020morel}.
Similarly, we only need to replace the current offline loss with the corresponding policy and model loss.

\textbf{Baselines}. 
We compare \name~with data selection methods in RL.
Specifically, previous work on prioritized experience replay for online RL~\citep{schaul2015prioritized} aligns closely with our objective. 
We make this a baseline \namep~where samples with the highest TD losses form the reduced dataset. 
Baseline \nameo~presents the performance by training TD3+BC with the original, complete dataset. 
Baseline \namer~randomly selects subsets from the D4RL dataset that are of the same size as \name.
We also compare our method with general dataset reduction techniques from supervised learning.
Specifically, we adopt the coherence criterion from Kernel recursive least squares~($\mathtt{KRLS}$)~\citep{engel2004kernel}, the log det criterion by forward selection in informative vector machines~($\mathtt{LogDet}$)~\citep{seeger2004greedy} and the adapting kernel representation~($\mathtt{BlockGreedy}$)~\citep{schlegel2017adapting} as our baselines.

%Specifically, we consider randomly selecting offline coreset as our baseline algorithms.
% In addition, we consider separately selecting high-reward offline datasets and low-reward offline datasets as our baseline algorithms.

\subsection{Experimental Results}
\label{sec:exp_perf}
% To compare the performance of different algorithms, we adopt two data selection schemes: sample-based selection and trajectory-based selection. They differ in the smallest unit of selection: the first selects samples in each iteration, while the second selects trajectories.

% As for the trajectory-based selection, prioritized sampling is no loner applicable. As an alternative, we compare with \nameh, which selects trajectories with the highest accumulative reward from the complete dataset. We again compare with the \nameo~as the reference to an upper limit of performance.

\begin{table*}[t]
    \centering
    \begin{tabular}{c|cccc}
    \toprule
    & KRLS & Log-Det & BlockGreedy & \name \\
    \midrule
    Hopper-medium-v0 & 69.4$\pm$2.5 & 58.4$\pm$3.6 & 83.7$\pm$2.2 & \textbf{94.3$\pm$4.6}\\
    Hopper-expert-v0 & 91.0$\pm$1.1 & 90.7$\pm$1.3 & 98.7$\pm$0.5 & \textbf{110.0$\pm$0.5}\\
    Hopper-medium-replay-v0 & 28.5$\pm$3.2 & 29.4$\pm$1.2 & 30.5$\pm$2.4 & \textbf{35.3$\pm$3.2}\\
    Walker2d-medium-v0 & 49.1$\pm$2.8 & 47.5$\pm$3.4 & 53.3$\pm$3.6 & \textbf{80.5$\pm$2.9}\\
    Walker2d-expert-v0 & 68.4$\pm$3.2 & 67.5$\pm$5.6 & 74.8$\pm$3.4 & \textbf{104.6$\pm$2.5}\\
    Walker2d-medium-replay-v0 & 14.3$\pm$1.2 & 15.2$\pm$2.2 & 16.7$\pm$1.3 & \textbf{21.1$\pm$1.8}\\
    Halfcheetah-medium-v0 & 23.4$\pm$0.5 & 21.9$\pm$0.9 & 27.5$\pm$0.7 & \textbf{41.0$\pm$0.2}\\
    Halfcheetah-expert-v0 & 73.9$\pm$1.4 & 72.1$\pm$2.2 & 79.2$\pm$1.8 & \textbf{88.5$\pm$2.4}\\
    Halfcheetah-medium-replay-v0 & 39.5$\pm$0.3 &39.9$\pm$0.5 & 40.5$\pm$1.0 & \textbf{41.1$\pm$0.4}\\
    \bottomrule
    \end{tabular}
    \caption{Experimental results on the D4RL~(Hard) offline datasets. All experiment results were averaged over five random seeds. Our method performs better than the dataset reduction baselines.}
    \label{tab: varied performance}
\end{table*}

\begin{figure}[t]
    \centering
    \subfigure{\includegraphics[scale=0.20]{d4rl/halfcheetah-medium-expert-v0.pdf}}
    \subfigure{\includegraphics[scale=0.20]{d4rl/hopper-medium-v0.pdf}}
    \subfigure{\includegraphics[scale=0.20]{d4rl/hopper-medium-expert-v0.pdf}}
    \subfigure{\includegraphics[scale=0.20]{d4rl/walker2d-medium-expert-v0.pdf}}
    \caption{Experimental results on the D4RL offline datasets. All experiment results were averaged over five random seeds. Our method achieves better or comparable results than the baselines consistently.}
    \label{fig: d4rl original}
\end{figure}

\paragraph{Answer of Question 1:}
To show that \name~can improve offline RL algorithms, we compare \name~with Complete Dataset, Prioritized, and Random in the Mujoco domain.
The experimental results in Figure~\ref{fig: d4rl hard} show that our method achieves superior performance than baselines.
By leveraging the reduced dataset generated from \name, the agent can learn much faster than learning from the complete dataset.
Further, the results in Figure~\ref{fig: d4rl original} show that \name~also performs better than the complete dataset and data selection RL baselines in the standard D4RL datasets. 
This is because prior methods select data in a random or loss-priority manner, which lacks guidance for subset selection and leads to degraded performance for downstream tasks.

In addition, to test \name's generality across various offline RL algorithms on various domains, we also conduct experiments on Antmaze tasks.
We use IQL~\citep{kostrikov2021offline} as the backbone of offline RL algorithms.
The experimental results in Table~\ref{tab: other domain2} show that our method achieves stronger performance than baselines.
In the antmaze tasks, the agent is required to stitch together various trajectories to reach the target location.
In this scenario, randomly removing data could result in the loss of critical data, thereby preventing complete the task.
Differently, \name~extracts valuable subset by balancing data quantity with performance, achieving a stronger performance than the complete dataset.

% In Figure~\ref{fig: d4rl minimal ratio}, we show the performance of different algorithms with the sample-based selection scheme. The experimental results show that \name~can achieve performance close to \nameo~with a small amount of data. For example, we use only $3\%$ of the original dataset in the Hopper tasks. \namer~and \namep, on the other hand, present a stark contrast, even not showing a stable learning trend with the same amount of training data. 
% In addition, we also evaluate the performance on the trajectory-based selection setting. Please refer to Appendix~\ref{appendix: trajectory} for the detailed experimental results.
% For the trajectory-based selection, experimental results in Figure~\ref{fig: d4rl topbc} show that \name~maintains its superiority in this setting with suboptimal (e.g., \texttt{medium}) datasets. This evidence suggests that \name~provides a valuable strategy for selecting data conducive to effective training under conditions of compromised data quality.

\paragraph{Answer of Question 2:}
To test whether \name~can select more valuable data than the data selection algorithms in supervised learning, we compare our method with KRLS~\citep{engel2004kernel}, Log-Det~\citep{seeger2004greedy} and BlockGreedy~\citep{schlegel2017adapting} in the D4RL~(Hard) datasets.
The experimental results in Table~\ref{tab: varied performance} show that our method generally outperforms baselines.
We hypothesize that supervised learning is static with fixed learning objectives, while offline RL's dynamic nature makes the target values evolve with policy updates, complicating the data selection process.
Therefore, the data selection methods in supervised learning cannot be directly applied to offline RL scenarios.

% Additionally, we observe that  $\texttt{Random}$ performs better than $\texttt{Q-diff}$.
% We attribute this phenomenon to the broader data coverage of $\texttt{Random}$, while the data coverage of $\texttt{Q-diff}$ is narrow.
% However, we also note that in some tasks, such as $\texttt{Hopper-medium-expert-v0}$, $\texttt{Hopper-expert-v0}$ and $\texttt{Walker2d-expert-v0}$, $\texttt{Random}$ initially performs well, but as training progresses, its performance starts to decline.
% We find that this coincides with unstable Q-values, which can be attributed to the increased extrapolation error caused by the reduced training dataset.
% In contrast, \name~performs better since it closely approximates the original gradients, thus preventing Q-values from diverging.


% For this reason, when the dataset quality is high~(e.g., \texttt{medium-expert} dataset), TopBC performs comparably to \name.

% \begin{table*}[t]
%     \centering
%     \caption{\name~with varying dataset sizes~($x\%$). Highlighted is the performance comparable to training TD3+BC with the complete dataset. \name~typically achieves good results with 5\%-15\% data, indicating that existing offline RL datasets contain a high degree of redundancy.
%     We adopt the normalized score metric proposed by the D4RL benchmark. Scores roughly range from 0 to 100, where 0 corresponds to the performance of a random policy and 100 indicates the performance of an expert.} 
%     \label{tab: varied performance}
%     \begin{tabular}{c|cccc}
%     \toprule
%         & 5\% & 10\% & 15\% & 20\% \\
%         \midrule
%         Hopper-medium-v0 & 91.8$\pm$3.6 & 92.6$\pm$3.0 & 94.0$\pm$4.8 & 95.2$\pm$1.6\\
%         Walker2d-medium-v0 & 14.8$\pm$7.3 & 57.9$\pm$3.6 & 69.3$\pm$4.0 & 71.7$\pm$1.9 \\
%         Halfcheetah-medium-v0 & 40.5$\pm$0.0 & 40.9$\pm$0.1 & 41.3$\pm$0.1 & 41.2$\pm$0.5 \\
%         Hopper-expert-v0 & 111.6$\pm$0.9 & 110.6$\pm$1.9 & 112.7$\pm$0.1 & 112.4$\pm$0.1 \\
%         Walker2d-expert-v0 & 74.5$\pm$6.4 & 84.4$\pm$5.0 & 97.6$\pm$3.1 & 100.2$\pm$1.0 \\
%         Halfcheetah-expert-v0 & 57.5$\pm$6.4 & 84.3$\pm$2.7 & 97.8$\pm$0.8 & 100.1$\pm$3.0 \\
%         Hopper-medium-expert-v0 & 108.1$\pm$1.1 & 112.4$\pm$0.3 & 112.3$\pm$0.05 & 112.8$\pm$0.1\\
%         Walker2d-medium-expert-v0 & 79.3$\pm$2.1 & 85.4$\pm$5.3 & 96.2$\pm$6.7 & 101.4$\pm$3.6 \\
%         Halfcheetah-medium-expert-v0 & 67.5$\pm$0.5 & 86.2$\pm$5.0 & 85.8$\pm$1.5 & 92.4$\pm$1.3\\
%     \bottomrule
%     \end{tabular}
% \end{table*}


% \subsection{Ablation Study}\label{sec:exp_ab}
% \textbf{Varying dataset size}.\ \ In Table~\ref{tab: varied performance}, we show the performance of \name~with varying dataset sizes ranging from $5\%$ to $20\%$.
% The results demonstrate that \name~requires only $5\%$ or $10\%$ of the original dataset to obtain good performance.
% Further, \name~can achieve similar performance with \nameo~with $20\%$ data of the original dataset.
% This indicates that existing offline RL datasets are characterized by a high degree of redundancy.

\begin{figure}[t]
    \centering
    \includegraphics[width=0.97\linewidth]{visual.jpg}
    \caption{Visualization of the \textcolor{blue}{complete dataset} and the \textcolor{orange}{reduced dataset} in \texttt{halfcheetah} task. The higher opacity of a point represents a large time step towards the end of an episode. The dataset embedding is characterized by its division into different components. 
    % In \texttt{walker2d} (upper), components vary with time steps.
     Samples selected by \name~connect different components by focusing on the data related to the task.}
    \label{fig: t-sne}
\end{figure}

\begin{table}[t]
    \centering 
    \begin{tabular}{c|cccc}
    \toprule
        Env & Random & Prioritized & Complete Dataset & \name\\
        \midrule
        Antmaze-umaze-v0 & 75.1$\pm$2.5 & 70.2$\pm$3.6 & 87.5$\pm$1.3 & \textbf{90.7$\pm$3.3}\\
        Antmaze-umaze-diverse-v0 & 46.3$\pm$1.9 & 44.7$\pm$2.7 & 62.2$\pm$2.0 & \textbf{76.7$\pm$2.2} \\
        Antmaze-medium-play-v0 & 59.3$\pm$1.6 & 60.3$\pm$2.9 & 71.2$\pm$2.2 & \textbf{80.3$\pm$2.9}\\
        Antmaze-medium-diverse-v0 & 43.6$\pm$2.7 & 46.9$\pm$3.8 & 70.0$\pm$1.6 & \textbf{84.9$\pm$3.8}\\
        Antmaze-large-play-v0 &	3.7$\pm$0.7 & 15.0$\pm$3.5 & 39.6$\pm$3.6 & \textbf{46.0$\pm$3.5}\\
        Antmaze-large-diverse-v0 & 16.0$\pm$3.6 & 20.5$\pm$3.7 & 47.5$\pm$1.1 & \textbf{52.0$\pm$3.7}\\
    \bottomrule
    \end{tabular}
    \caption{Experimental results on the Antmaze offline datasets. All experiment results were averaged over five random seeds. Our method performs better than baselines. }
    \label{tab: other domain2}
\end{table}

% \begin{figure*}[t]
%     \centering
%     \subfigure{\includegraphics[scale=0.27]{ablation_moduler1.pdf}}
%     \hspace{0.3cm}\subfigure{\includegraphics[scale=0.27]{ablation_moduler2.pdf}}
%     \caption{Ablation results on D4RL~(Hard) tasks with the normalized score metric.}
%     \label{fig: modular ablation}
% \end{figure*}

% In this subsection, we conduct ablation studies to study the effect of different modules and import hyper-parameters.


\paragraph{Answer of Question 3:}
To study the contribution of each component in our learning framework, we conduct the following ablation study. 
\nameq: We replace the empirical returns used to update Q functions with the standard target Q function in the TD loss function. 
\namei: We set the number of data selection rounds to 1 and study the function of multi-round data selection.
The experimental results in Figure~\ref{fig: modular ablation} in Appendix~\ref{sec: ablation} show that removing any of these two modules will worsen the performance of \name. In case like $\texttt{walker2d-medium}$, ablation \namei~even decrease the performance by over 80\%, and ablation \nameq~results in a 95\% performance drop in $\texttt{walker2d-expert}$. Furthermore, we also find that in the $\texttt{halfcheetah}$ tasks, the impact of removing the two modules is relatively small. This result can be attributable to the fact that this task has a limited state space, and we can directly apply OMP to the entire dataset and identify important and diverse data.

We visualize the selected data by \name~to better understand how it works. 
Figure~\ref{fig: t-sne} displays the t-SNE low-dimensional embeddings, with the complete dataset in blue and the selected data in orange. 
The higher opacity of a point indicates a larger time step. The dataset's structure is revealed by its segmentation into diverse components: 
In \texttt{halfcheetah}, each component reflects a distinct skill of the agent.
For example, from 1 to 7, they represent falling, leg lifting, jumping, landing, leg swapping, stepping, and starting, respectively.
We can observe that the selected samples by \name~ not only cover each component of the dataset but also effectively bridge the gaps between them, enhancing the dataset's versatility and coherence. 
Moreover, we find that \name~is less concerned with the falling data and instead focuses on the data related to the task.
This observation can explain the improved performance of \name. For additional visualizations, please refer to Appendix~\ref{appendix: visual}.

% \textbf{Generalizability of \name}. \ \
% We evaluate the generalizability of \name~from two perspectives.
% First, we add IQL~\cite{kostrikov2021offline} as a baseline and apply \name~to IQL by using the gradient of the training loss of the V-function in IQL as the criterion.
% On the other hand, we evaluate \name~on the other domains, such as robotic manipulation (Adroit) and sparse reward (Antmaze) tasks.
% The experiments in Appendix~\ref{appendix: other domain} and Appendix~\ref{appendix: other algorithm} show that \name~is not only applicable to other algorithms, such as IQL~\cite{kostrikov2021offline}, but also to other domains.

% \textbf{Generalizability of subset}. \ \
% To test the generalizability of the dataset selected by~\name, we select subset by applying~\name~to TD3+BC.
% Then we evaluate the performance of IQL on the selected subset. 
% The experimental results in Table~\ref{tab: td3bc2iql} in Appendix~\ref{appendix: tb3bc2iql} demonstrate that the selected subset based on TD3+BC is effectively applicable to IQL.

% \textbf{Sensitivity for hyperparameter}. \ \
% We evaluate the performance of \name~with various cluster numbers~(from 1 to 50) and approximation bounds~(from 0.0001 to 0.05).
% The experimental results in Appendix~\ref{appendix: cluster number} and Appendix~\ref{appendix: approx bound} show that the suitable cluster number is between 25 and 50.
% Too few clusters (e.g., less than 5) are detrimental to the algorithm.
% In addition, a smaller approximation bound represents a larger reduced dataset.
% Similar to the ablation of the size of the reduced dataset in Table~\ref{tab: varied performance}, \name~requires only a 0.01 approximation bound to obtain good performance.

\subsection{Computational complexity}
We report the computational overhead of \name~on various datasets. 
All experiments are conducted on the same computational device (GeForce RTX 3090 GPU). 
The results in Appendix~\ref{appendix: computation complexity} indicate that even on datasets containing millions of data points, the computational overhead of our method remains low~(e.g., several minutes).
This low computational complexity can be attributed to the trajectory-based selection technique in Sec.~\ref{sec: offline omp}~(II) and the regularized constraint technique in Sec.~\ref{sec:method:outer}, making our method easily scalable to large-scale datasets. 

% This low computational complexity can be attributed to the batch mechanism designed in section 3.2 (IV), which reduces the computational complexity from $O(MN)$ to $O(|\mathcal{B}|N)$, making our method easily scalable to large-scale datasets. $M, N, |\mathcal{B}|$ are the size of the full dataset, reduced dataset, and batch respectively.

% We conduct t-SNE based dimensionality reduction to the cluster centroids and these five trajectories.
% The experimental results are shown in the , where darker colors indicate moving towards the end of the trajectory.

% From the experimental results, we find that in the walker2d task, \name~ tends to select more low-reward but more diverse data points ~(upper right) while selecting a few high-reward data points~(left and bottom).
% We attribute this phenomenon to the narrow distribution of the high-reward points, allowing us to approximate the original gradients with only a few points. 
% In the halfcheetah task, \name~ connects useful information while ignoring low-quality data~(e.g., data point \texttt{1}).

\subsection{Analysis}
\label{sec:analysis}
\begin{figure}[t]
\centering
    \includegraphics[width=0.95\columnwidth]{fig/allratio.pdf}
    \caption{Comparison of ratio adherence across different compression ratio settings. The experimental results are obtained with LLaMA-3.1-8B-Instruct on GSM8K.}
    \label{fig:allratio}
\end{figure}

\begin{figure}[t]
\centering
    \includegraphics[width=0.95\columnwidth]{fig/compressor.pdf}
    \caption{Performance comparison of \method using different token importance metrics, evaluated with LLaMA-3.1-8B-Instruct on GSM8K.}
    \label{fig:compressor}
\end{figure}

\paragraph{Compression Ratio} 
In our main results, we focus on compression ratios greater than 0.5. To further investigate the performance of \method at lower compression ratios, we train an additional variant, denoted as \texttt{More Ratio}, with extra compression ratios of 0.3 and 0.4. As shown in Figure~\ref{fig:allratio}, the ratio adherence of models largely degrades at these lower ratios. We attribute this decline to the excessive trimming of reasoning tokens, which likely causes a loss of critical information in the completions, hindering the effective training of LLMs to learn CoT compression. Furthermore, we observe that the overall adherence of \texttt{More Ratio} is not as good as \method with the default settings, which further supports our hypothesis.

\paragraph{Importance Metric} 
Figure~\ref{fig:compressor} presents a performance comparison of \method across different token importance metrics. In addition to the metrics discussed in Section \ref{sec:token-importance}, we include \texttt{GPT-4o}\footnote{We use the \texttt{gpt-4o-2024-08-06} version for experiments.} as a strong token importance metric for comparison. Specifically, for a given CoT trajectory, we prompt \texttt{GPT-4o} to trim redundant tokens according to a specified compression ratio, without adding any additional tokens. Additionally, we ask \texttt{GPT-4o} to suggest the \textit{optimal} compression format of the CoT trajectory, referred to as \texttt{GPT-4o-Optimal} in Figure~\ref{fig:compressor}. We incorporate all training data generated by \texttt{GPT-4o} to train a variant of \method. We use the ``\texttt{[optimal]}'' token to prompt the model, obtaining the results of \texttt{GPT-4o-Optimal}.

As illustrated in Figure~\ref{fig:compressor}, \method utilizing LLMLingua-2~\cite{pan:2024llmlingua2} outperforms the variant with Selective Context~\cite{li:2023selective}, which aligns with our demonstrations in Section \ref{sec:token-importance}. Additionally, incorporating \texttt{GPT-4o} for token importance measurement further enhances compression performance, suggesting that a more robust CoT compressor could improve \method even further. However, the API costs associated with \texttt{GPT-4o} make it impractical for processing large datasets. In contrast, LLMLingua-2, which includes a BERT-size model, offers a cost-effective and efficient alternative for training \method. Furthermore, \texttt{GPT-4o-Optimal} achieves a better balance between reasoning accuracy and CoT token reduction, emphasizing the potential of flexible compression ratios in CoT generation --- an avenue we plan to explore in future work.

\paragraph{Length Budget} 
As outlined in Section~\ref{sec:exp_setup}, we adjust the maximum length budget to \texttt{max\_len}$\times\gamma$ when evaluating \method on MATH-500, ensuring a fair comparison of compression ratios. However, this brute-force length truncation inevitably impacts the reasoning performance of LLMs, as LLMs are unable to complete the full generation. In this analysis, we explore whether LLMs can ``\textit{think}'' more effectively using a compressed CoT format. Specifically, we evaluate \method under the same length budget as the original LLM (e.g., 1024 for MATH-500). The experimental results, shown in Figure~\ref{fig:budget}, demonstrate a significant performance improvement of \method under this length budget, compared to those adjusted by compression ratios. Notably, with compression ratios of 0.7, 0.8, and 0.9, \method outperforms the original LLM, yielding an absolute performance increase of 1.3 to 2.6 points. These findings highlight \method's potential to enhance the reasoning capabilities of LLMs within the same length budget.

\begin{figure}[t]
\centering
    \includegraphics[width=0.95\columnwidth]{fig/budget.pdf}
    \caption{Performance comparison of \method with varying maximum length constraints, evaluated with LLaMA-3.1-8B-Instruct on the MATH-500 dataset.}
    \label{fig:budget}
\end{figure}

\begin{figure*}[t]
\centering
\includegraphics[width=1.0\textwidth]{fig/cases.pdf}
\caption{Three CoT compression examples from \method. For each sample, we list the question, original CoT outputs from corresponding LLMs, and the compressed CoT by \method. The tokens that appear in both the original CoT and the compressed CoT are highlighted in \sethlcolor{pink}\hl{red}.}
\label{fig:cases}
\end{figure*}

\paragraph{Case Study} 
Figure~\ref{fig:cases} presents several examples of \method, derived from the test sets of GSM8K and MATH-500. These examples clearly illustrate that \method allows LLMs to learn shortcuts between critical reasoning tokens, rather than generating shorter CoTs from scratch. For instance, in the first case, \method facilitates LLaMA-3.1-8B-Instruct to skip semantic connectors such as ``\textit{of}'' and ``\textit{the}'', as well as expressions that contribute minimally to the reasoning, such as the first sentence. Notably, we observe that numeric values and mathematical equations are prioritized for retention in most cases. This finding aligns with recent research~\cite{Ma:2024mathmatters}, which suggests that mathematical expressions may contribute more significantly to reasoning than CoT in natural language. Furthermore, we find that \method does not reduce the number of reasoning steps but instead trims redundant tokens within those steps.
\section{Related Work}

\paragraph{Efficient CoT} While Chain-of-Thought (CoT) enhances task performance by simulating human-like reasoning patterns, its reasoning steps introduce significant computational overhead. As a result, researchers have sought methods to reduce this overhead while retaining the benefits of CoT. One intuitive approach is to simplify, skip~\cite{marconato2024not, Ding:2024cotshortcut, liu2024skipstep}, or generate thinking steps in parallel~\cite{ning2023skeleton} to improve efficiency. Another strategy involves compressing reasoning steps into continuous latent representations~\cite{goyal2024think, deng2024explicit, hao2024training, cheng2024compressed}, allowing LLMs to reason without explicitly generating discrete word tokens. To minimize the generation of redundant natural language information that has minimal impact on reasoning, \citet{hu2023chain} implements structured syntax and symbols, while \citet{han2024token} guides token consumption through dynamic token budget estimation. Similarly, \citet{kang2024c3ot} prompts larger LLMs (i.e., \texttt{GPT-4}) to directly compress CoT, then fine-tunes LLMs to reason using these compressed CoTs. In contrast, this work focuses on pruning CoT tokens based on their semantic importance. Additionally, \method leverages a small LM for token pruning, significantly reducing computational overhead.

\paragraph{Prompt Compression} As LLMs advance in their zero-shot capabilities, the growing demand for complex instructions and long-context prompts has led to substantial computational and memory challenges in processing lengthy inputs. To address this bottleneck, researchers have explored various prompt compression techniques. One intuitive approach involves using a lightweight LM to generate more concise, semantically similar prompts~\cite{chuang-etal-2024-learning}. However, given that explicit natural language representations often contain redundant information, some researchers have turned to implicit continuous tokens to represent task prompts~\cite{wingate-etal-2022-prompt, mu2024learning} and long-context inputs~\cite{chevalier-etal-2023-adapting, ge2024incontext, mohtashami2023randomaccess}. Other approaches focus on directly compressing input prompts by filtering and retaining high-informative tokens~\cite{li:2023selective, jiang2023:llmlingua, pan:2024llmlingua2}. For instance, Selective Context uses the perplexity of LLMs to measure token importance and removes tokens deemed less important. LLMLingua-2~\cite{pan:2024llmlingua2} introduces a small bidirectional language model for token importance measurement and trains this LM with \texttt{GPT-4} compression data, which serves as the token importance metric in this work.
\section{Conclusion}
We operationalized the theory of instrumental interaction for generative AI, with an in-depth unpacking of the principles of reification of user intent, reflection, and grounding. We argue that leveraging this re-appropriated and refined theory can drive the creation of a \textit{new generation of expressive AI-Instruments} that afford better expression of intent, make it easier to discover what is possible, and provide powerful degrees of freedom for steering the generation towards the best possible results. Those new tools and instruments can truly leverage the polymorphic and non-deterministic behavior of generative AI models, unleashing new and empowering forms of expressive HCI+AI experiences. 

Beyond our focus on AI-Instruments, theories play an important role in the advancement of our wider research field~\cite{rogers_hci_2012, halverson_activity_2002}. Rogers argues that there is a need for theories as lenses bringing critical design characteristics into focus, and which can function as a generative source: providing "\textit{design dimensions and constructs to inform the design and selection of interactive representations}"~\cite{rogers_new_2004}. We hope that our work on operationalizing the theory of instrumental interaction for AI can inspire other new -- and re-appropriated -- theories to advance HCI+AI. 









\section*{Limitations}
\label{subsec:limitation} 

Due to computational constraints, experiments with larger LLMs, such as Qwen2.5-32B-Instruct and Qwen2.5-72B-Instruct, were not conducted. We believe that \method could achieve a more favorable trade-off between reasoning performance and CoT token usage on these models. Additionally, the token importance measurement used in our study, derived from the LLMLingua-2 compressor~\cite{pan:2024llmlingua2}, was not specifically trained on mathematical data. This limitation may affect the compression effectiveness, as the model is not optimized for handling numerical tokens and mathematical expressions. Furthermore, experiments with long-CoT LLMs, such as QwQ-32B-Preview, were also excluded due to computational constraints. We plan to explore these aspects in future work, as we anticipate that \method’s potential can be further realized in these contexts.

\section*{Ethics Statement}
\label{subsec:ethics} 
The datasets used in our experiment are publicly released and labeled through interaction with humans in English. In this process, user privacy is protected, and no personal information is contained in the dataset. The scientific artifacts that we used are available for research with permissive licenses. And the use of these artifacts in this paper is consistent with their intended use. Therefore, we believe that our research work meets the ethics of ACL. 


% Bibliography entries for the entire Anthology, followed by custom entries
%\bibliography{anthology,custom}
% Custom bibliography entries only
\bibliography{custom}

\clearpage

\appendix

\section*{Appendix}
\section{CoT Recovery}
\label{appendix:recovery}

In this section, we provide the detailed prompt for our recovery experiments, which is illustrated in Figure~\ref{fig:recovery-prompt}. Besides, we present the CoT recovery result from \texttt{GPT-4o}~\cite{gpt-4} in Figure~\ref{fig:recovery-gpt}. The recovered results demonstrate that \texttt{GPT-4o} could understand the compressed CoT content and correctly restore the original CoT process.

\begin{figure}[htbp]
\begin{tcolorbox}[colback=blue!5!white,colframe=blue!75!black,title=Input Prompt for CoT Recovery,fontupper=\footnotesize,fonttitle=\scriptsize]
Could you please recover the following compressed Chain-of-Thought output of a mathematical question to its original full content?

\vskip 0.1in

The question is: 

Marcus is half of Leo’s age and five years younger than Deanna. Deanna is 26. How old is Leo?

\vskip 0.1in

The compressed Chain-of-Thought: 

break down Deanna 26 Marcus five years younger 26 - 5 21 Marcus half Leo's age Leo twice Marcus's age Marcus 21, Leo's age 2 x 21 = 42.

\vskip 0.1in

Original Full Chain-of-Thought:
\end{tcolorbox}
\vskip -0.1in
\caption{Input prompt for LLaMA-3.1-8B-Instruct designed to recover the compressed CoT from a GSM8K math problem.}
\vskip -0.1in
\label{fig:recovery-prompt}
\end{figure}

\begin{figure}[htbp]
\begin{tcolorbox}[colback=blue!5!white,colframe=blue!75!black,title=Revovering the Compressed Chain-of-Thought,fontupper=\footnotesize,fonttitle=\scriptsize]
\textbf{Compressed CoT}: break down Deanna 26 Marcus five younger 26 - 5 21 Marcus half Leo's age twice Marcus Marcus 21, Leo's age 2 x 21 = 42.

\vskip 0.1in
        
\textbf{Recovered CoT}: 1. We know that Deanna is 26 years old. 2. Marcus is five years younger than Deanna. So, Marcus’s age is \( 26 - 5 = 21 \). 3. Marcus is also half of Leo’s age, which means Leo’s age is twice Marcus’s age. 4. Since Marcus is 21 years old, Leo’s age is \( 2 \times 21 = 42 \). So, Leo is 42 years old.
\end{tcolorbox}
\vskip -0.1in
\caption{Recovering the compressed CoT for GSM8K math word problem using GPT-4o.}
\vskip -0.1in
\label{fig:recovery-gpt}
\end{figure}

\section{Experimental Details}
\label{appendix:exp_details}

\subsection{Implementation Details}
\label{appendix:training_details}
We utilize LLMLingua-2~\cite{pan:2024llmlingua2} as the token importance metric to generate our compressed CoT training data. The compression ratio $\gamma$ is randomly selected from $\{0.5, 0.6, 0.7, 0.8, 0.9, 1.0\}$ for each training sample. We adopt LoRA~\cite{lora} to train our models. The rank $r$ is set to 8, and the scaling parameter $\alpha$ is set to 16. We train the models for 3 epochs on both datasets. The peak learning rate is set to 5e-5, following a cosine decay schedule. We use AdamW~\cite{AdamW} for optimization, with a warmup ratio of 0.1. We implement our training process using the \texttt{LLaMA-Factory}~\cite{llamafactory} library. Inference for both our method and all baselines is performed using the Huggingface transformers package. During inference, the maximum number of tokens \texttt{max\_len} is set to 512 for GSM8K and 1024 for MATH. All experiments are conducted using Pytorch 2.1.0 on 2$\times$NVIDIA GeForce RTX 3090 GPU (24GB) with CUDA 12.1, and an Intel(R) Xeon(R) Platinum 8370C CPU with 32 cores.

\subsection{Detailed Results with Qwen}
\label{appendix:qwen_detail}
We provide detailed experimental results of the Qwen2.5-Instruct series evaluated on GSM8K in Table~\ref{tab:qwen}. As the model scale increases, there is less performance degradation at higher compression ratios, indicating that larger LLMs are better at identifying shortcuts between critical reasoning tokens, enabling more efficient CoT generation.

\begin{table}[htbp]
\centering
\small
\setlength{\tabcolsep}{1.4mm}
%\resizebox{\linewidth}{!}{
\begin{tabular}{@{}llcrrc@{}}
\toprule
\textbf{Scale} &\textbf{Methods} &\textbf{Ratio} &Accuracy &Tokens &\textit{Act}Ratio \\ \midrule
\multirow{7}{*}{\texttt{3B}} & \texttt{Original}  & - & 83.7\cred{(0.0 \downarrow)} &314.87 &- \\ \cmidrule{2-6}
&\multirow{6}{*}{\method}  &1.0 & 83.4\cred{(0.3 \downarrow)} &318.79 & 1.00  \\
& &0.9 & 83.2\cred{(0.5 \downarrow)} &262.99 & 0.83\\
& &0.8 & 81.6\cred{(2.1 \downarrow)} &250.71 & 0.79\\
& &0.7 & 80.1\cred{(3.6 \downarrow)} &233.03 & 0.73\\
& &0.6 & 77.3\cred{(6.4 \downarrow)} &199.55 & 0.63\\
& &0.5 & 74.4\cred{(9.3 \downarrow)} &170.55 & 0.54 \\\midrule
\multirow{7}{*}{\texttt{7B}} & \texttt{Original}  & - & 91.4\cred{(0.0 \downarrow)} &297.83 &- \\ \cmidrule{2-6}
&\multirow{6}{*}{\method}  &1.0 & 91.7\cred{(0.3 \uparrow)} &295.78 & 1.00  \\
& &0.9 & 91.1\cred{(0.3 \downarrow)} &254.77 & 0.86\\
& &0.8 & 90.1\cred{(1.3 \downarrow)} &237.27 & 0.80\\
& &0.7 & 89.9\cred{(1.5 \downarrow)} &216.73 & 0.73\\
& &0.6 & 87.9\cred{(3.5 \downarrow)} &178.07 & 0.60\\
& &0.5 & 86.0\cred{(5.4 \downarrow)} &151.44 & 0.51 \\\midrule
\multirow{7}{*}{\texttt{14B}} & \texttt{Original}  & - & 93.1\cred{(0.0 \downarrow)} &313.11 &- \\ \cmidrule{2-6}
&\multirow{6}{*}{\method}  &1.0 & 93.0\cred{(0.1 \downarrow)} &314.55 & 1.00  \\
& &0.9 & 93.3\cred{(0.2 \uparrow)} &269.22 & 0.86\\
& &0.8 & 93.2\cred{(0.1 \uparrow)} &247.24 & 0.79\\
& &0.7 & 93.4\cred{(0.3 \uparrow)} &218.62 & 0.70\\
& &0.6 & 92.7\cred{(0.4 \downarrow)} &180.68 & 0.57\\
& &0.5 & 91.4\cred{(1.7 \downarrow)} &156.85 & 0.50 \\
\bottomrule
\end{tabular}%}
\caption{Experimental results on the Qwen2.5-Instruct series. We report accuracy, average CoT token count, and actual compression ratio (\textit{Act}Ratio) for comparison.}
\label{tab:qwen}
\end{table}

\end{document}
