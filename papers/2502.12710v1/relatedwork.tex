\section{Related Work}
\label{sec:related-work}
	Over time, different information hiding methods and implementations for watermarking and steganography have been published. Due to the increasing number of diverse method classes, cover types, and areas of application, different literature reviews and surveys exist to organize the cluttered research and application landscape. A more detailed discussion on selected methods is provided in section \ref{sec:related-methods}.
	
	\citet{Bender.1996} present one of the first comprehensive overviews of data hiding methods for image, audio, and text files. Later on, \citet{Petitcolas.1999} made a survey on information hiding techniques, focusing on steganography, watermarking, and fingerprinting techniques, including information about possible attacks and a basic theory for overall principles. A more specialized overview with solutions focusing on text steganography is provided by \citet{Ahvanooey.2019}, \citet{Krishnan.2017}, and \citet{Majeed.2021}. Current challenges are discussed by \citet{Ahvanooey.2022} and \citet{Tyagi.2023}, while the latter provide concrete application possibilities.
	
	Besides the reviews, researchers started analyzing and comparing existing methods to identify their strengths and weaknesses. \citet{Ahvanooey.2018b} compare watermarking and steganography methods by differentiating their embedding technique and analyzing them based on the evaluation criteria of imperceptibility, embedding capacity, robustness, security, and computational cost~\citep{Ahvanooey.2018b}. One of the latest evaluations on text steganography is published by \citet{Knochel.2024} where the authors compared the capacity, imperceptibility, robustness, and complexity with a specialized focus on Malware.
	
	\subsection{Related Methods}
	\label{sec:related-methods}
	In order to integrate our proposed TREND method into the research landscape, we present the most relevant text watermarking and steganography methods in the following. In connection with the \ac{LLM} problem domain initially set out, \citet{Kirchenbauer.2023} and \citet{Christ.2024b} present a token-based watermarking scheme, while the latter focus on undetectability, completeness, and soundness. Such ideas are integrated into SynthID, the watermarking engine for Google's Gemini \ac{LLM} \citep{GoogleDeepMind.2024}. \citet{Steinebach.2024} generates a cover text based on sets of letters. Those methods are classified in the domain of lingustic methods since they make use of the \ac{LLM} text generation. Nevertheless, those linguistic methods are problematic for cover texts where semantics are important. Thus, we focus on format-based methods that use insertion or substitution-based embedding techniques~\citep{Knochel.2024}. Other types of format-based methods, as well as linguistic or random and statistical generation methods~\citep{Knochel.2024,Majeed.2021} are not considered further since they either do not work on pure text documents due to missing format options like color or fonts or because they change the semantic or structure of the cover text. We implemented all presented methods for our benchmark evaluation in Section~\ref{sec:evaluation}. A summarized overview is depicted in Table~\ref{tab:related-methods}.
	
	\paragraph{SNOW.} One of the first whitespace steganography methods for ASCII texts is the \ac{SNOW} by \citet{Kwan.2013}. While the first release goes back to the 20th century, the last update with a change to the open source license Apache 2.0 was made in 2013~\citep{Kwan.2013}. It includes a Java and Windows DOS version and a C-implementation last updated in 2016~\citep{Kwan.2016}. The embedding process encodes the secret message into tab and space characters and appends it at the end of the cover text, starting with a tab character under the consideration of a pre-defined line length~\citep{Kwan.2013}. Upstream compression and encryption can be optionally enabled before the encoding process.
	
	\paragraph{UniSpaCh.} A very well-known algorithm in the field of information hiding for text documents is UniSpaCh, proposed by \citet{Por.2012}. It is an extended version of WhiteSteg, which replaces one single space character between words and paragraphs with either one or two whitespaces to encode a 0 or 1~\citep{Por.2008}. UniSpaCh uses two different types to embed the secret message in the text. For spaces between words and sentences, regular whitespaces either remain as they are or are extended by adding an additional Thin, Six-Per-Em, or Hair space to encode two bits per embedding location~\citep{Por.2012}. For end-of-line and inter-paragraph spacings, the remaining space is filled with a combination of Hair, Six-Per-Em, Punctuation, and Thin spaces to encode two bits per character~\citep{Por.2012}.
	
	\paragraph{AITSteg.} \citet{Ahvanooey.2018} propose a text steganography technique for SMS or social media communication. The embedding method transforms the secret message into zero-width characters with the help of a Gödel function and by using the sending/receiving time and the length of the secret message to add it before the cover text~\citep{Ahvanooey.2018}.
	
	\paragraph{Shiu et al.} The data hiding method proposed by \citet{Shiu.2018} focus on communication over messengers of social media networks. Due to the small width of a messaging window, the method is based on a fixed line length and can hide three bits per line of a cover text~\citep{Shiu.2018}. After encoding a secret message into a bit stream based on the ASCII mapping, it embeds the first bit by adding a whitespace at the end of a line, changing the length of the line to embed the second bit, and adding a whitespace between two words to embed the third bit~\citep{Shiu.2018}.
	
	\paragraph{Rizzo et al.} A text watermarking technique based on specific replacement of Unicode characters with their confusables, also known as homoglyphs, was initially proposed in \citet{Rizzo.2016} and extended to a fine-grain watermarking approach in \citet{Rizzo.2019}. Based on their latest approach, it generates a watermark by using a keyed hash function with a secret message as a watermark and a secret password~\citep{Rizzo.2019}. Aftward, the watermark is embedded in the cover text by replacing specific characters with their confusables or leaving them as they are to embed one bit each and replacing whitespaces with a set of specific whitespaces to embed three bits per space~\citep{Rizzo.2019}.
	
	\paragraph{StegCloak.} The open-source implementation StegCloak published by \citet{KuroLabs.2020} as described in \citet{Mohanasundar.2020} is a JavaScript steganography tool that is able to hide a secret message inside a cover text with optional password encryption and \ac{HMAC}. In the embedding process, the secret message is compressed, optionally encrypted, and encoded in a set of zero-width characters to be inserted in one location after a classical whitespace of the cover text~\citep{Mohanasundar.2020}.
	
	\paragraph{Lookalikes.} Another implementation is the Unicode Lookalikes algorithm by \citet{Thompson.2021} as part of the Python package pyUnicodeSteganography. Similar to \citet{Rizzo.2019}, the method replaces specific characters with their confusables to encode a secret message inside the cover text~\citep{Thompson.2021}.
	
	\paragraph{CovertSYS.} \citet{Ahvanooey.2022b} presents a multilingual steganography method focusing on short messages on social networks. Like the previous approach in \citet{Ahvanooey.2018}, four zero-width characters and a timestamp are used to encode the secret message. Further, a password-based approach with an \ac{OTP} and XOR operation are used to transform the secret message into an encrypted bit stream that is appended at the end of the cover text~\citep{Ahvanooey.2022b}.
	
	\paragraph{Shazzad-Ur-Rahmen et al.} The data hiding approach of \citet{ShazzadUrRahman.2021} is able to embed five bits per embeddable location, while their updated version of \citet{ShazzadUrRahman.2023} can embed six bits per embeddable location. The main idea of the latest embedding procedure starts by encrypting the secret message using AES and converting the resulting binary stream into blocks of 6 bits~\citep{ShazzadUrRahman.2023}. With the help of two lists, specific Unicode characters are replaced with their confusables, and whitespaces are replaced with a particular combination of multiple smaller whitespaces to embed the secret message in the cover text~\citep{ShazzadUrRahman.2023}.
	
	\begin{table}[!htb]
		\centering
		\caption{Overview of Related Methods}
		\label{tab:related-methods}
		\begin{tabular}{p{3.5cm} p{1.8cm} p{2.5cm} p{7cm}}
			\toprule
			Name & Release & Type & Techniques \\
			\midrule
			SNOW \citep{Kwan.2013,Kwan.2016} & Before 1998 & Docu \& Im\-ple\-ment\-ation & Hides data at the end of the text by appending additional tabs and whitespaces.\\
			UniSpaCh \citep{Por.2012} & 2012 & Paper & Hides data by adding small whitespace characters between words and sentences and by filling up lines and inter-paragraph spacings.\\
			AITSteg \citep{Ahvanooey.2018} & 2018 & Paper & Hides data at the beginning of the cover text by using symmetric key encoding and a transformation into zero-width characters.\\
			\citet{Shiu.2018} & 2018 & Paper & Hides data line-wise by either adding a whitespace between words or at the end of line or by changing the line length.\\
			\citet{Rizzo.2016,Rizzo.2019} & 2016/2019 & Paper & Hides data by replacing Unicode characters and whitespaces with their confusables.\\
			StegCloak \citep{KuroLabs.2020,Mohanasundar.2020} & 2020 & Blog post \& Im\-ple\-ment\-ation & Hides data using zero-width characters in one spacing location in the cover text.\\
			Lookalikes \citep{Thompson.2021} & 2021 & Im\-ple\-ment\-ation & Hides data by replacing specific Unicode characters with their confusables.\\
			CovertSYS \citep{Ahvanooey.2022b} & 2022 & Paper & Hides data using zero-width characters at the end of the cover text by using the current date and time and a \ac{OTP}.\\
			\citet{ShazzadUrRahman.2021,ShazzadUrRahman.2023} & 2021/2023 & Paper & Hides data by replacing conufsables and whitespaces with a specific combination of small whitespaces.\\
			\ac{TREND} \citep{FraunhoferISST.2025} & 2025 & Paper \& Im\-ple\-ment\-ation & Hides data by replacing whitespaces with a specific set of robust similar-looking whitespaces.\\
			\bottomrule
		\end{tabular}
	\end{table}
	
	\FloatBarrier