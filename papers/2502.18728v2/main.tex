\pdfoutput=1
%% For double-blind review submission, w/o CCS and ACM Reference (max submission space)
% \documentclass[acmsmall,review, anonymous]{acmart}\settopmatter{printfolios=true,printccs=false,printacmref=false}
%% For double-blind review submission, w/ CCS and ACM Reference
%\documentclass[acmsmall,review,anonymous]{acmart}\settopmatter{printfolios=true}
%% For single-blind review submission, w/o CCS and ACM Reference (max submission space)
\documentclass[acmsmall]{acmart}
%% For single-blind review submission, w/ CCS and ACM Reference
%\documentclass[acmsmall,review]{acmart}\settopmatter{printfolios=true}
%% For final camera-ready submission, w/ required CCS and ACM Reference
%\documentclass[acmsmall]{acmart}\settopmatter{}


%%% The following is specific to OOPSLA1 '25 and the paper
%%% 'Scaling Optimization over Uncertainty via Compilation'
%%% by Minsung Cho, John Gouwar, and Steven Holtzen.
%%%
% \setcopyright{cc}
% \setcctype{by}
\acmDOI{10.1145/3720500}
\acmYear{2025}
\acmJournal{PACMPL}
\acmVolume{9}
\acmNumber{OOPSLA1}
\acmArticle{135}
\acmMonth{4}
\received{2024-10-15}
\received[accepted]{2025-02-18}
\setcopyright{cc}
\setcctype{by}

%% Copyright information
%% Supplied to authors (based on authors' rights management selection;
%% see authors.acm.org) by publisher for camera-ready submission;
%% use 'none' for review submission.
% \setcopyright{none}
% \setcopyright{acmcopyright}
%\setcopyright{acmlicensed}
% \setcopyright{rightsretained}
%\copyrightyear{2018}           %% If different from \acmYear

%% Bibliography style
\bibliographystyle{ACM-Reference-Format}
%% Citation style
%% Note: author/year citations are required for papers published as an
%% issue of PACMPL.
\citestyle{acmauthoryear}   %% For author/year citations


%%%%%%%%%%%%%%%%%%%%%%%%%%%%%%%%%%%%%%%%%%%%%%%%%%%%%%%%%%%%%%%%%%%%%%
%% Note: Authors migrating a paper from PACMPL format to traditional
%% SIGPLAN proceedings format must update the '\documentclass' and
%% topmatter commands above; see 'acmart-sigplanproc-template.tex'.
%%%%%%%%%%%%%%%%%%%%%%%%%%%%%%%%%%%%%%%%%%%%%%%%%%%%%%%%%%%%%%%%%%%%%%


%% Some recommended packages.
\usepackage{booktabs}   %% For formal tables:
                        %% http://ctan.org/pkg/booktabs
\usepackage{subcaption} %% For complex figures with subfigures/subcaptions
                        %% http://ctan.org/pkg/subcaption
\usepackage{meu, proof}

\newcommand{\diff}[1]{{\color{red} #1}}
% \newcommand{\appcite}[1]{#1}
\newcommand{\appcite}[1]{the appendix}
% \newcommand{\diff}[1]{#1}

\begin{document}

\pgfplotstableread[col sep = comma]{data/hmm-data.csv}\hmmdata
\pgfplotstableread[col sep = comma]{data/ladder-long.csv}\ladderlongdata
\pgfplotstableread[col sep = comma]{data/ladder4.csv}\ladderfour
\pgfplotstableread[col sep = comma]{nested-pineappl/pineappl-nested.csv}\Nested
\pgfplotstableread[col sep = comma]{nested-pineappl/pineappl-nested-fit.csv}\Nestedfit

%% Title information
\title{Scaling Optimization over Uncertainty via Compilation}
                                        %% when present, will be used in
                                        %% header instead of Full Title.
% \titlenote{with title note}             %% \titlenote is optional;
%                                         %% can be repeated if necessary;
%                                         %% contents suppressed with 'anonymous'
% \subtitle{Subtitle}                     %% \subtitle is optional
% \subtitlenote{with subtitle note}       %% \subtitlenote is optional;
%                                         %% can be repeated if necessary;
%                                         %% contents suppressed with 'anonymous'


%% Author information
%% Contents and number of authors suppressed with 'anonymous'.
%% Each author should be introduced by \author, followed by
%% \authornote (optional), \orcid (optional), \affiliation, and
%% \email.
%% An author may have multiple affiliations and/or emails; repeat the
%% appropriate command.
%% Many elements are not rendered, but should be provided for metadata
%% extraction tools.

%% Author with single affiliation.
% \author{First1 Last1}
% \authornote{with author1 note}          %% \authornote is optional;
%                                         %% can be repeated if necessary
% \orcid{nnnn-nnnn-nnnn-nnnn}             %% \orcid is optional
% \affiliation{
%   \position{Position1}
%   \department{Department1}              %% \department is recommended
%   \institution{Institution1}            %% \institution is required
%   \streetaddress{Street1 Address1}
%   \city{City1}
%   \state{State1}
%   \postcode{Post-Code1}
%   \country{Country1}                    %% \country is recommended
% }
% \email{first1.last1@inst1.edu}          %% \email is recommended

% %% Author with two affiliations and emails.
% \author{First2 Last2}
% \authornote{with author2 note}          %% \authornote is optional;
%                                         %% can be repeated if necessary
% \orcid{nnnn-nnnn-nnnn-nnnn}             %% \orcid is optional
% \affiliation{
%   \position{Position2a}
%   \department{Department2a}             %% \department is recommended
%   \institution{Institution2a}           %% \institution is required
%   \streetaddress{Street2a Address2a}
%   \city{City2a}
%   \state{State2a}
%   \postcode{Post-Code2a}
%   \country{Country2a}                   %% \country is recommended
% }
% \email{first2.last2@inst2a.com}         %% \email is recommended
% \affiliation{
%   \position{Position2b}
%   \department{Department2b}             %% \department is recommended
%   \institution{Institution2b}           %% \institution is required
%   \streetaddress{Street3b Address2b}
%   \city{City2b}
%   \state{State2b}
%   \postcode{Post-Code2b}
%   \country{Country2b}                   %% \country is recommended
% }
% \email{first2.last2@inst2b.org}         %% \email is recommended
\title{Scaling Optimization over Uncertainty via Compilation}

\author{Minsung Cho}
\orcid{0009-0006-6170-6033}
\affiliation{%
  \institution{Northeastern University}
  \city{Boston}
  \country{USA}
}
\email{minsung@ccs.neu.edu}

\author{John Gouwar}
\orcid{0000-0003-0494-7245}
\affiliation{%
  \institution{Northeastern University}
  \city{Boston}
  \country{USA}
}
\email{gouwar.j@northeastern.edu}

\author{Steven Holtzen}
\orcid{0000-0002-8190-5412}
\affiliation{%
  \institution{Northeastern University}
  \city{Boston}
  \country{USA}
}
\email{s.holtzen@northeastern.edu}
%% Abstract
%% Note: \begin{abstract}...\end{abstract} environment must come
%% before \maketitle command

%% 2012 ACM Computing Classification System (CSS) concepts
%% Generate at 'http://dl.acm.org/ccs/ccs.cfm'.
\begin{CCSXML}
<ccs2012>
   <concept>
       <concept_id>10002950.10003648.10003649.10003653</concept_id>
       <concept_desc>Mathematics of computing~Decision diagrams</concept_desc>
       <concept_significance>500</concept_significance>
       </concept>
   <concept>
       <concept_id>10002950.10003648.10003662</concept_id>
       <concept_desc>Mathematics of computing~Probabilistic inference problems</concept_desc>
       <concept_significance>500</concept_significance>
       </concept>
 </ccs2012>
\end{CCSXML}

\ccsdesc[500]{Mathematics of computing~Probabilistic inference problems}
\ccsdesc[500]{Mathematics of computing~Decision diagrams}
%% End of generated code


%% Keywords
%% comma separated list
\keywords{probabilistic programming languages, maximum expected utility, maximum marginal a posteriori.}  %% \keywords are mandatory in final camera-ready submission


%% \maketitle
%% Note: \maketitle command must come after title commands, author
%% commands, abstract environment, Computing Classification System
%% environment and commands, ana keywords command.

\begin{abstract}
Probabilistic inference is fundamentally hard, yet many tasks require
optimization on top of inference, which is even harder.  We present a new
\textit{optimization-via-compilation} strategy to scalably solve a certain
class of such problems.  In particular, we introduce a new intermediate
representation (IR), binary decision diagrams weighted by a novel notion of
\textit{branch-and-bound semiring}, that enables a scalable branch-and-bound
based optimization procedure. This IR automatically \textit{factorizes}
problems through program structure and \textit{prunes} suboptimal values via a
straightforward branch-and-bound style algorithm to find optima.
Additionally, the IR is naturally amenable to \textit{staged compilation},
allowing the programmer to query for optima mid-compilation to inform further
executions of the program.  We showcase the effectiveness and flexibility of
the IR by implementing two performant languages that both compile to it:
$\dappl$ and $\pineappl$.  $\dappl$ is a functional language that solves
maximum expected utility problems with first-class support for rewards,
decision making, and conditioning.  $\textsc{pineappl}$ is an imperative
language that performs exact probabilistic inference with support for nested
marginal maximum a posteriori (MMAP) optimization via staging.
\end{abstract}

\maketitle

\section{Introduction}\label{sec:introduction}

\section{Introduction}


\begin{figure}[t]
\centering
\includegraphics[width=0.6\columnwidth]{figures/evaluation_desiderata_V5.pdf}
\vspace{-0.5cm}
\caption{\systemName is a platform for conducting realistic evaluations of code LLMs, collecting human preferences of coding models with real users, real tasks, and in realistic environments, aimed at addressing the limitations of existing evaluations.
}
\label{fig:motivation}
\end{figure}

\begin{figure*}[t]
\centering
\includegraphics[width=\textwidth]{figures/system_design_v2.png}
\caption{We introduce \systemName, a VSCode extension to collect human preferences of code directly in a developer's IDE. \systemName enables developers to use code completions from various models. The system comprises a) the interface in the user's IDE which presents paired completions to users (left), b) a sampling strategy that picks model pairs to reduce latency (right, top), and c) a prompting scheme that allows diverse LLMs to perform code completions with high fidelity.
Users can select between the top completion (green box) using \texttt{tab} or the bottom completion (blue box) using \texttt{shift+tab}.}
\label{fig:overview}
\end{figure*}

As model capabilities improve, large language models (LLMs) are increasingly integrated into user environments and workflows.
For example, software developers code with AI in integrated developer environments (IDEs)~\citep{peng2023impact}, doctors rely on notes generated through ambient listening~\citep{oberst2024science}, and lawyers consider case evidence identified by electronic discovery systems~\citep{yang2024beyond}.
Increasing deployment of models in productivity tools demands evaluation that more closely reflects real-world circumstances~\citep{hutchinson2022evaluation, saxon2024benchmarks, kapoor2024ai}.
While newer benchmarks and live platforms incorporate human feedback to capture real-world usage, they almost exclusively focus on evaluating LLMs in chat conversations~\citep{zheng2023judging,dubois2023alpacafarm,chiang2024chatbot, kirk2024the}.
Model evaluation must move beyond chat-based interactions and into specialized user environments.



 

In this work, we focus on evaluating LLM-based coding assistants. 
Despite the popularity of these tools---millions of developers use Github Copilot~\citep{Copilot}---existing
evaluations of the coding capabilities of new models exhibit multiple limitations (Figure~\ref{fig:motivation}, bottom).
Traditional ML benchmarks evaluate LLM capabilities by measuring how well a model can complete static, interview-style coding tasks~\citep{chen2021evaluating,austin2021program,jain2024livecodebench, white2024livebench} and lack \emph{real users}. 
User studies recruit real users to evaluate the effectiveness of LLMs as coding assistants, but are often limited to simple programming tasks as opposed to \emph{real tasks}~\citep{vaithilingam2022expectation,ross2023programmer, mozannar2024realhumaneval}.
Recent efforts to collect human feedback such as Chatbot Arena~\citep{chiang2024chatbot} are still removed from a \emph{realistic environment}, resulting in users and data that deviate from typical software development processes.
We introduce \systemName to address these limitations (Figure~\ref{fig:motivation}, top), and we describe our three main contributions below.


\textbf{We deploy \systemName in-the-wild to collect human preferences on code.} 
\systemName is a Visual Studio Code extension, collecting preferences directly in a developer's IDE within their actual workflow (Figure~\ref{fig:overview}).
\systemName provides developers with code completions, akin to the type of support provided by Github Copilot~\citep{Copilot}. 
Over the past 3 months, \systemName has served over~\completions suggestions from 10 state-of-the-art LLMs, 
gathering \sampleCount~votes from \userCount~users.
To collect user preferences,
\systemName presents a novel interface that shows users paired code completions from two different LLMs, which are determined based on a sampling strategy that aims to 
mitigate latency while preserving coverage across model comparisons.
Additionally, we devise a prompting scheme that allows a diverse set of models to perform code completions with high fidelity.
See Section~\ref{sec:system} and Section~\ref{sec:deployment} for details about system design and deployment respectively.



\textbf{We construct a leaderboard of user preferences and find notable differences from existing static benchmarks and human preference leaderboards.}
In general, we observe that smaller models seem to overperform in static benchmarks compared to our leaderboard, while performance among larger models is mixed (Section~\ref{sec:leaderboard_calculation}).
We attribute these differences to the fact that \systemName is exposed to users and tasks that differ drastically from code evaluations in the past. 
Our data spans 103 programming languages and 24 natural languages as well as a variety of real-world applications and code structures, while static benchmarks tend to focus on a specific programming and natural language and task (e.g. coding competition problems).
Additionally, while all of \systemName interactions contain code contexts and the majority involve infilling tasks, a much smaller fraction of Chatbot Arena's coding tasks contain code context, with infilling tasks appearing even more rarely. 
We analyze our data in depth in Section~\ref{subsec:comparison}.



\textbf{We derive new insights into user preferences of code by analyzing \systemName's diverse and distinct data distribution.}
We compare user preferences across different stratifications of input data (e.g., common versus rare languages) and observe which affect observed preferences most (Section~\ref{sec:analysis}).
For example, while user preferences stay relatively consistent across various programming languages, they differ drastically between different task categories (e.g. frontend/backend versus algorithm design).
We also observe variations in user preference due to different features related to code structure 
(e.g., context length and completion patterns).
We open-source \systemName and release a curated subset of code contexts.
Altogether, our results highlight the necessity of model evaluation in realistic and domain-specific settings.






\section{Overview}\label{sec:overview}

\section{Overview}

\revision{In this section, we first explain the foundational concept of Hausdorff distance-based penetration depth algorithms, which are essential for understanding our method (Sec.~\ref{sec:preliminary}).
We then provide a brief overview of our proposed RT-based penetration depth algorithm (Sec.~\ref{subsec:algo_overview}).}



\section{Preliminaries }
\label{sec:Preliminaries}

% Before we introduce our method, we first overview the important basics of 3D dynamic human modeling with Gaussian splatting. Then, we discuss the diffusion-based 3d generation techniques, and how they can be applied to human modeling.
% \ZY{I stopp here. TBC.}
% \subsection{Dynamic human modeling with Gaussian splatting}
\subsection{3D Gaussian Splatting}
3D Gaussian splatting~\cite{kerbl3Dgaussians} is an explicit scene representation that allows high-quality real-time rendering. The given scene is represented by a set of static 3D Gaussians, which are parameterized as follows: Gaussian center $x\in {\mathbb{R}^3}$, color $c\in {\mathbb{R}^3}$, opacity $\alpha\in {\mathbb{R}}$, spatial rotation in the form of quaternion $q\in {\mathbb{R}^4}$, and scaling factor $s\in {\mathbb{R}^3}$. Given these properties, the rendering process is represented as:
\begin{equation}
  I = Splatting(x, c, s, \alpha, q, r),
  \label{eq:splattingGA}
\end{equation}
where $I$ is the rendered image, $r$ is a set of query rays crossing the scene, and $Splatting(\cdot)$ is a differentiable rendering process. We refer readers to Kerbl et al.'s paper~\cite{kerbl3Dgaussians} for the details of Gaussian splatting. 



% \ZY{I would suggest move this part to the method part.}
% GaissianAvatar is a dynamic human generation model based on Gaussian splitting. Given a sequence of RGB images, this method utilizes fitted SMPLs and sampled points on its surface to obtain a pose-dependent feature map by a pose encoder. The pose-dependent features and a geometry feature are fed in a Gaussian decoder, which is employed to establish a functional mapping from the underlying geometry of the human form to diverse attributes of 3D Gaussians on the canonical surfaces. The parameter prediction process is articulated as follows:
% \begin{equation}
%   (\Delta x,c,s)=G_{\theta}(S+P),
%   \label{eq:gaussiandecoder}
% \end{equation}
%  where $G_{\theta}$ represents the Gaussian decoder, and $(S+P)$ is the multiplication of geometry feature S and pose feature P. Instead of optimizing all attributes of Gaussian, this decoder predicts 3D positional offset $\Delta{x} \in {\mathbb{R}^3}$, color $c\in\mathbb{R}^3$, and 3D scaling factor $ s\in\mathbb{R}^3$. To enhance geometry reconstruction accuracy, the opacity $\alpha$ and 3D rotation $q$ are set to fixed values of $1$ and $(1,0,0,0)$ respectively.
 
%  To render the canonical avatar in observation space, we seamlessly combine the Linear Blend Skinning function with the Gaussian Splatting~\cite{kerbl3Dgaussians} rendering process: 
% \begin{equation}
%   I_{\theta}=Splatting(x_o,Q,d),
%   \label{eq:splatting}
% \end{equation}
% \begin{equation}
%   x_o = T_{lbs}(x_c,p,w),
%   \label{eq:LBS}
% \end{equation}
% where $I_{\theta}$ represents the final rendered image, and the canonical Gaussian position $x_c$ is the sum of the initial position $x$ and the predicted offset $\Delta x$. The LBS function $T_{lbs}$ applies the SMPL skeleton pose $p$ and blending weights $w$ to deform $x_c$ into observation space as $x_o$. $Q$ denotes the remaining attributes of the Gaussians. With the rendering process, they can now reposition these canonical 3D Gaussians into the observation space.



\subsection{Score Distillation Sampling}
Score Distillation Sampling (SDS)~\cite{poole2022dreamfusion} builds a bridge between diffusion models and 3D representations. In SDS, the noised input is denoised in one time-step, and the difference between added noise and predicted noise is considered SDS loss, expressed as:

% \begin{equation}
%   \mathcal{L}_{SDS}(I_{\Phi}) \triangleq E_{t,\epsilon}[w(t)(\epsilon_{\phi}(z_t,y,t)-\epsilon)\frac{\partial I_{\Phi}}{\partial\Phi}],
%   \label{eq:SDSObserv}
% \end{equation}
\begin{equation}
    \mathcal{L}_{\text{SDS}}(I_{\Phi}) \triangleq \mathbb{E}_{t,\epsilon} \left[ w(t) \left( \epsilon_{\phi}(z_t, y, t) - \epsilon \right) \frac{\partial I_{\Phi}}{\partial \Phi} \right],
  \label{eq:SDSObservGA}
\end{equation}
where the input $I_{\Phi}$ represents a rendered image from a 3D representation, such as 3D Gaussians, with optimizable parameters $\Phi$. $\epsilon_{\phi}$ corresponds to the predicted noise of diffusion networks, which is produced by incorporating the noise image $z_t$ as input and conditioning it with a text or image $y$ at timestep $t$. The noise image $z_t$ is derived by introducing noise $\epsilon$ into $I_{\Phi}$ at timestep $t$. The loss is weighted by the diffusion scheduler $w(t)$. 
% \vspace{-3mm}

\subsection{Overview of the RTPD Algorithm}\label{subsec:algo_overview}
Fig.~\ref{fig:Overview} presents an overview of our RTPD algorithm.
It is grounded in the Hausdorff distance-based penetration depth calculation method (Sec.~\ref{sec:preliminary}).
%, similar to that of Tang et al.~\shortcite{SIG09HIST}.
The process consists of two primary phases: penetration surface extraction and Hausdorff distance calculation.
We leverage the RTX platform's capabilities to accelerate both of these steps.

\begin{figure*}[t]
    \centering
    \includegraphics[width=0.8\textwidth]{Image/overview.pdf}
    \caption{The overview of RT-based penetration depth calculation algorithm overview}
    \label{fig:Overview}
\end{figure*}

The penetration surface extraction phase focuses on identifying the overlapped region between two objects.
\revision{The penetration surface is defined as a set of polygons from one object, where at least one of its vertices lies within the other object. 
Note that in our work, we focus on triangles rather than general polygons, as they are processed most efficiently on the RTX platform.}
To facilitate this extraction, we introduce a ray-tracing-based \revision{Point-in-Polyhedron} test (RT-PIP), significantly accelerated through the use of RT cores (Sec.~\ref{sec:RT-PIP}).
This test capitalizes on the ray-surface intersection capabilities of the RTX platform.
%
Initially, a Geometry Acceleration Structure (GAS) is generated for each object, as required by the RTX platform.
The RT-PIP module takes the GAS of one object (e.g., $GAS_{A}$) and the point set of the other object (e.g., $P_{B}$).
It outputs a set of points (e.g., $P_{\partial B}$) representing the penetration region, indicating their location inside the opposing object.
Subsequently, a penetration surface (e.g., $\partial B$) is constructed using this point set (e.g., $P_{\partial B}$) (Sec.~\ref{subsec:surfaceGen}).
%
The generated penetration surfaces (e.g., $\partial A$ and $\partial B$) are then forwarded to the next step. 

The Hausdorff distance calculation phase utilizes the ray-surface intersection test of the RTX platform (Sec.~\ref{sec:RT-Hausdorff}) to compute the Hausdorff distance between two objects.
We introduce a novel Ray-Tracing-based Hausdorff DISTance algorithm, RT-HDIST.
It begins by generating GAS for the two penetration surfaces, $P_{\partial A}$ and $P_{\partial B}$, derived from the preceding step.
RT-HDIST processes the GAS of a penetration surface (e.g., $GAS_{\partial A}$) alongside the point set of the other penetration surface (e.g., $P_{\partial B}$) to compute the penetration depth between them.
The algorithm operates bidirectionally, considering both directions ($\partial A \to \partial B$ and $\partial B \to \partial A$).
The final penetration depth between the two objects, A and B, is determined by selecting the larger value from these two directional computations.

%In the Hausdorff distance calculation step, we compute the Hausdorff distance between given two objects using a ray-surface-intersection test. (Sec.~\ref{sec:RT-Hausdorff}) Initially, we construct the GAS for both $\partial A$ and $\partial B$ to utilize the RT-core effectively. The RT-based Hausdorff distance algorithms then determine the Hausdorff distance by processing the GAS of one object (e.g. $GAS_{\partial A}$) and set of the vertices of the other (e.g. $P_{\partial B}$). Following the Hausdorff distance definition (Eq.~\ref{equation:hausdorff_definition}), we compute the Hausdorff distance to both directions ($\partial A \to \partial B$) and ($\partial B \to \partial A$). As a result, the bigger one is the final Hausdorff distance, and also it is the penetration depth between input object $A$ and $B$.


%the proposed RT-based penetration depth calculation pipeline.
%Our proposed methods adopt Tang's Hausdorff-based penetration depth methods~\cite{SIG09HIST}. The pipeline is divided into the penetration surface extraction step and the Hausdorff distance calculation between the penetration surface steps. However, since Tang's approach is not suitable for the RT platform in detail, we modified and applied it with appropriate methods.

%The penetration surface extraction step is extracting overlapped surfaces on other objects. To utilize the RT core, we use the ray-intersection-based PIP(Point-In-Polygon) algorithms instead of collision detection between two objects which Tang et al.~\cite{SIG09HIST} used. (Sec.~\ref{sec:RT-PIP})
%RT core-based PIP test uses a ray-surface intersection test. For purpose this, we generate the GAS(Geometry Acceleration Structure) for each object. RT core-based PIP test takes the GAS of one object (e.g. $GAS_{A}$) and a set of vertex of another one (e.g. $P_{B}$). Then this computes the penetrated vertex set of another one (e.g. $P_{\partial B}$). To calculate the Hausdorff distance, these vertex sets change to objects constructed by penetrated surface (e.g. $\partial B$). Finally, the two generated overlapped surface objects $\partial A$ and $\partial B$ are used in the Hausdorff distance calculation step.

\section{Optimization-via-Compilation}\label{sec:bbir}

In this section, we give a formal account of the intuitions
reflected in Section~\ref{subsubsec:optimization-via-compilation}
and~\cref{subsubsec:staging}.
We will describe the
branch-and-bound semiring (Section~\ref{subsec:bb semiring}),
a class of lattice semirings (recall Definition~\ref{def:latticed semiring})
equipped with an additional total order that is \textit{compatible} with the existing lattice.
Afterwards, we introduce the BBIR
how it represents MEU and MMAP
(Section~\ref{subsec:msp jsp}),
and how it admits a polynomial time upper- and lower-bound
algorithm (Section~\ref{subsec:bounds}),
lending itself well to a branch-and-bound
approach (Section~\ref{subsec:meu with evidence}).

\subsection{The Branch-and-Bound Semiring}
\label{subsec:bb semiring}

In Section~\ref{subsubsec:optimization-via-compilation},
we introduced the definition of a lattice semiring
(Definition~\ref{def:latticed semiring}) and how it generalizes the
interchange law between max and sum
($\max_{x} \sum_y f(x,y) \leq \sum_y \max_x f(x,y)$) in the reals.
However, in a lattice semiring,
$\sqsubseteq$ is a partial order, so in general elements may not be able to be compared:
for example, in the expectation semiring, we cannot compare the values $(0.5, 1)$ and $(1,0)$
as $0.5 \leq 1$ but $1 \not\leq 0$.
However, if we were to compare the values $(0.5, 1)$ and $(1,0)$ as
values of
$\AMC{}$ corresponding to total (as opposed to partial) policies,
then the comparison is
obvious: we select $(0.5, 1)$ as it has the higher utility.
To reflect this intuition, we enrich lattice semirings with a total order,
which gives the definition of a branch-and-bound semiring:

\begin{definition}[Branch-and-Bound Semiring]\label{def:branch-and-bound semiring}
  A branch-and-bound semiring is a lattice semiring
  $(\mathcal R, \oplus, \otimes, \mathbf 0, \mathbf 1, \sqsubseteq)$
  equipped with an additional total order
  $\leq$ such that for all $a,b \in \mathcal R$,
  $a \sqsubseteq b$ implies $a \leq b$,
  which we henceforth call \emph{compatibility}.
\end{definition}

The real semiring $\R$ is a branch-and-bound semiring
in which the two orders are identical: the usual total order on the reals.
However, the intuition above is reflected most prominently in the expectation semiring:

\begin{proposition}\label{prop:S is bb semiring}
  The expectation semiring $\mathcal S$,
  as seen in Definition~\ref{def:expectation semiring},
  forms a branch-and-bound semiring with:
  \begin{enumerate*}
    \item $(p,u) \sqsubseteq (q,v)$ iff $p \leq q$ and $u \leq v$,
    with join $\bigsqcup$ being a coordinatewise max and
    meet $\bigsqcap$ being a coordinatewise min, and
    \item $(p,u) \leq (q,v)$ iff $u < v$ or $u=v$ and $p \leq q$.
  \end{enumerate*}
\end{proposition}

\begin{proof}
  Let $(p,u), (q,v) \in \mathcal S$ such that
  $(p,u) \sqsubseteq (q,v)$. Then $u \leq v$;. if $u <v$ we are done.
  If $u=v$ then $p \leq q$ and we are done.
\end{proof}

The distinction between $\sqsubseteq$ and $\leq$ is required when
comparing partial and total policies in~\cref{subsec:bounds}.
Compatibility will be required when we know, for $p \sqsubseteq q$,
that $p$ and $q$ are associated with total policies as opposed to partial.

% Continuing the intuition outlined
% for Definition~\ref{def:branch-and-bound semiring},
% recall that the AMC over $\mathcal S$ computes the
% expected utility, particularly that the expected
% utility of a policy.
% If we know two policies, then we can simply evaluate
% the expected utility of both and pick the greater; this
% is reflected in $\leq$
% in Proposition~\ref{prop:S is bb semiring}.
% On the other hand, $\sqsubseteq$ allows us to compare
% \textit{partial policies}--that is, a deterministic assignment
% to only \textit{some} components of the action space.
% How we reduce a partially evaluated expected utility calculation
% of a partial policy into a value in $\mathcal S$ is done through
% the join; we will see this in Section~\ref{subsec:bounds}.
% \sh{After reading
% this again, I do feel like either we should move a bunch of this to section 2,
% or try to move some of section 2 here; I'm not sure which is best, and
% we may need to wait until after the deadline to make these adjustments unless you
% see a quick way to do it. }

\subsection{The Branch-and-Bound IR}\label{subsec:msp jsp}
Now that we have defined the branch-and-bound semiring, we are ready
to reconstruct the branch-and-bound circuits in the
motivating examples in Section~\ref{sec:overview}.
% We had a motivating example (Figure~\ref{fig:bb-overview})
% that was then compiled to a binary decision diagram
% (Figure~\ref{fig:motiv-a-bdd}).
What additional information should the BDD in Figure~\ref{fig:motiv-a-bdd} have to fully represent
a decision scenario?
Of course we should specify which variables
to optimize over and which to not, and weights for all variables
present. But additionally we need to incorporate potential \textit{evidence}
showing the events to condition on as we evaluate the program.
We represent exactly this set of information in the BBIR.

\begin{definition}[Branch-and-bound IR]\label{def:bbir}
  A branch-and-bound intermediate representation (BBIR)
  over a branch-and-bound semiring $\mathcal B$
  is a
  tuple $\BBIR$ in which:
  \begin{itemize}
    \item $\{\varphi_i\}$ are propositional formulae
    in the factorized representation of a
    multi-rooted BDD~\citep{darwiche2002knowledge,clarke2018handbook},
    \item $X \subseteq \bigcup_i vars(\varphi_i)$ a selection of variables
    on which to branch over,
    \item $w : \bigcup_i lits(\varphi_i) \to \mathcal B$ a weight function.
  \end{itemize}
\end{definition}

% Over a BBIR we can define optimization problems over Boolean formulae.


% \begin{definition}[Max-Sum Problem]\label{def:msp over bbir}
%   Let $\BBIR$ be a BBIR over a branch-and-bound semiring
%   $\mathcal B$ and $\psi \in \{\varphi\}$.
%   Let $inst(X)$ be the set of all assignments of variables in $X$.
%   Then the max-sum problem (MSP) of $\psi$ over the BBIR is
%   \begin{equation}
%     MSP(\psi) = \max_{\pi \in inst(X)} \bigoplus_{m \models \psi|_{\pi}} f(m, \pi).
%   \end{equation}
% \end{definition}

% We demonstrate how this optimization problem aptly generalizes
% the MEU problem, along with several others
% present in the literature to showcase its generality.

We demonstrate below the definition of MEU and MMAP over BBIR below.

\subsubsection{The MEU Problem with Evidence}\label{subsubsec:meu}

Here, we give a formulation of the
MEU problem with evidence, a generalization of the MEU problem
addressed in~\cref{subsec:dappl-overview}
which allows us to eventually handle
\dapplcode{observe} statements in \dappl{}.
In particular we introduce an additional $\AMC$
in the denominator of the optimization function.
This additional model count can be handled by efficient
computation of bounds;
see Section~\ref{subsec:meu with evidence}
for full detail.

We represent this problem as a BBIR
$(\{\varphi \land \gamma_{\pi} : \pi \in \mathcal A\}, A, w)$, in which:

\begin{enumerate}[leftmargin=*]
  \item $\varphi$ is the Boolean formula detailing the control and data flow
  of the decision making model,
  \item $\gamma_{\pi}$ represent witnessed evidence for
  each policy $\pi \in \mathcal A$, where $\mathcal A$ is the collection of
  all possible policies (i.e., complete instantiations of choices)
  \item $A$ is the collection of variables representing choices.
  % on which we define instantiations (assignments) of $A$ as $inst(A) = \mathcal A$, and
  \item $w$ a weight function to denote rewards.
\end{enumerate}
On which the MEU problem reduces to the following optimization problem:
{\footnotesize\begin{equation}\label{eq:bbir-meu}
  \MEUfn{(\{\varphi \land \gamma_{\pi} : \pi \in \mathcal A\}, A, w)}
    \triangleq \max_{\pi \in \mathcal A} \frac{\AMC(\varphi|_{\pi} \land \gamma_{\pi},w)_{\EU}}{\AMC(\gamma_{\pi}, w)_{\Pr}},
\end{equation}}
where division is the normal division in $\R$  with the additional property that
division by $0$ is defined as $-\infty$.
The subscript $\EU$ and $\Pr$ denote the first and second projections over the expectation semiring,
referring to the $\AMC$ invariant proven in~\cref{appendix:amc invariant}.

To give a concrete example of this optimization problem, consider the example of~\cref{fig:motivation-dappl},
with the \dapplcode{observe} statement uncommented.
We can define
\begin{equation}
  \varphi = (u \land \varphi_u) \lor (\overline u \land \varphi_{\overline u}),
  \qquad
  \gamma_u = \gamma_{\overline u} = r,
  \qquad
  A = \{u\},
\end{equation}
where $\varphi_u$ and $\varphi_{\overline u}$ are
defined in~\cref{eq:formula-umbrella,eq:formula-no-umbrella} and $w$ are the weights
as defined in~\cref{sub@fig:bb circuit example}. Then we observe that
{\footnotesize
\begin{equation}
  \MEUfn{\{\varphi \land \gamma_i \mid i \in \{u, \overline u\}\}, A, w}
  = \max \left\{
  \frac{\AMC(\varphi_u \land r)_{\EU}}{\AMC(r)_{\Pr}},
  \frac{\AMC(\varphi_{\overline u} \land r)_{\EU}}{\AMC(r)_{\Pr}}
  \right\}
  = 10,
\end{equation}
}
validating the computations in~\cref{eqn:example-with-observe}.
\subsubsection{The Marginal Maximum A Posteriori (MMAP) Problem}\label{subsubsec:mmap}

We conclude with a formulation of the MMAP problem in full generality over a BBIR.
$\pineappl$ supports a limited form of conditioning, where observations can only occur
with a call to MMAP or a query (see~\cref{subsec:pineappl-sem} for details),
but we present a formulation of the MMAP problem which supports global conditioning.
We do so by defining the
BBIR $(\{\varphi, \gamma\}, M, w)$ where:

\begin{enumerate}[leftmargin=*]
  \item $M$ are our \emph{MAP variables} to compute the most likely state of,
  a subset of the variables of $\varphi$,
  \item $\varphi$ is our probabilistic model and $\gamma$ is our evidence to condition on,
  with $vars(\varphi) = M \cup V \cup E$ disjoint sets of variables where $E$
  is some set of variables representing priors and $V$ are probabilistic variables, and
  \item $w$ is a weight function with codomain in the real branch-and-bound semiring
  $\R$ where $\sqsubseteq, \leq$ are the usual total order.
\end{enumerate}

Then we can solve the following optimization problem for some priors $e \in inst(E)$,
where $inst$ denotes the set of all instantiations to a set of variables and $\varphi|_m$
denotes the formula derived by applying the literals of $m$ to $\varphi$:
{\footnotesize
\begin{align}\label{eq:bbir-mmap}
  \mathrm{MMAP}{(\{\varphi, \gamma\}, M, w, e)}
  &= \argmax_{m \in inst(M)} \sum_{\substack{v \in inst(V), \\ m \cup v \cup e \models \varphi}}
  \Pr[m \cup v \cup e \mid \gamma|_e]
  = \argmax_{m \in inst(M)} \frac{\AMC_\R(\varphi|_{m,e} \land \gamma|_e)}{\AMC(\gamma|_e)}.
\end{align}
}

When there are no priors, we elide $e$ in the arguments.
To give a concrete example of this problem, consider the example
given in the first four lines of~\cref{fig:motiv-pineappl}.
We can define:

{\footnotesize
\begin{equation*}
  \varphi =  \texttt{disease} \leftrightarrow f_{0.5} \land
  \texttt{headache} \leftrightarrow (\texttt{disease} \land f_{0.7} \lor \overline{\texttt{disease}} \land f_{0.1}).
  \qquad
  \gamma = \texttt{headache},
  \qquad
  M = \texttt{disease},
\end{equation*}
}
where $w(f_{n}) = 1 - w(\overline{f_{n}})$,
and the weight is 1 for all other literals.
\noindent Then we observe that with $V = \{f_{0.5}, f_{0.7}, f_{0.1}\}$,

{\footnotesize
\begin{align*}
  \mathrm{MMAP}{(\{\varphi, \gamma\}, \{\texttt{disease}\},w)}
  &=\max \left\{
    \sum_{\substack{v \in inst(V), \\
    v \cup \texttt{disease} \models \varphi}} \Pr\left[\texttt{disease} \cup v \mid \gamma\right],
    \sum_{\substack{v \in inst(V), \\
    v \cup \overline{\texttt{disease}} \models \varphi}} \Pr\left[\overline{\texttt{disease}} \cup v \mid \gamma\right],
  \right\} \\
  &=\max\{0.92, 0.08\} = 0.92,
\end{align*}
}
validating the computations in~\cref{eq:mmap-motiv-pineappl}.

Prior work, such as that of~\citet{huang2006solving} and~\citet{lee2016exact},
have leveraged
techniques in knowledge compilation to solve the MMAP problem via
a branch-and-bound algorithm.
Our method, to the best of our knowledge,
is the first method to generalize this approach beyond MMAP.


\subsection{Efficiently Upper-Bounding Algebraic Model Counts on BBIR}
\label{subsec:bounds}

We have demonstrated how the BBIR can represent important optimization problems
over probabilistic inference, as promised in~\cref{fig:bb-overview}.
However, a new problem representation
is moot without
gains in efficiency. Where does that happen?

Recall from Definition~\ref{def:bbir}
that the BBIR is over a branch-and-bound semiring, on which the
partial order $\sqsubseteq$ allowed the comparison of partially computed
algebraic model counts.
This is where the BBIR comes into play: it allows us to give an
upper- or lower-bound of partially computed algebraic model counts
on \textit{any} formula defined within the BBIR.
This is efficient--in particular, polynomial in the size of BBIR, more specifically
the BDD within. Thus, we can fully take advantage of the factorization of the
BDD while maintaining a way to compare partially computed values of AMC:

% To make this precise, we first need a definition.

\begin{definition}[Partial policies and completions]\label{def:partial policy}
  Let $\BBIR$ be a BBIR.
  Then, we can define $P$ a \emph{partial policy} of
  $X$ as instantiation of a subset of variables in $X$.
  A \textit{completion} $T$ of $P$ is an instantiation of variables of $X$
  such that $P \subseteq T$.
\end{definition}

\begin{figure}
  \begin{subfigure}[t]{0.37\linewidth}
    \begin{mdframed}{\footnotesize\begin{algorithmic}[1]
      \Procedure{$ub$}{$\BBIR, \varphi, P$}
      \State $pm \gets \bigotimes_{p \in P} w(p)$
      \State $acc \gets h(\varphi|_P, X,w)$
      \State \textbf{return} $pm \otimes acc$
      \EndProcedure
    \end{algorithmic}}
    \end{mdframed}
    \caption{The upper bound algorithm $ub$ takes in a BBIR, $\varphi \in \{\varphi\}$,
    and a $P$ a partial policy of $X$
    to find an upper bound of $\AMC(\varphi|_{T},w)$ for any completion $T$ of $P$.}
    \label{algorithm:ub}
  \end{subfigure}
  \hfill
  \begin{subfigure}[t]{0.6\textwidth}
    \begin{mdframed}{\footnotesize\begin{algorithmic}[1]
      \Procedure{$h$}{$\varphi, X, w$}
      \If {$\varphi = \top$} \textbf{return 1}
      \ElsIf {$\varphi = \bot$} \textbf{return 0}
      \Else{ \textbf{let }$v \gets \mathrm{root}(\varphi)$}
        \If{$v \in X$}  \textbf{return}
        $w(v) \otimes h(\varphi|_v) \sqcup w(\overline v) \otimes h(\varphi|_{\overline v}$)
        \Else{ \textbf{return}
        $w(v)\otimes h(\varphi|_v) \oplus w(\overline v)\otimes h(\varphi|_{\overline v}$)}
        \EndIf
      \EndIf
      \EndProcedure
    \end{algorithmic}}
    \end{mdframed}
    \caption{The helper function $h$ as seen on Line 3 in Fig.~\ref{algorithm:ub}. }
    \label{algorithm:h}
  \end{subfigure}
  \caption{A single top-down pass upper-bound function. The function $\mathrm{root}$
  returns the topmost variable in the BDD. In order to scale efficiently, these
  procedures must be memoized; we omit these details.}
  \label{fig:h}
\end{figure}

With this definition in mind, we can give the pseudocode for our
upper bound algorithm in \Cref{fig:h}.
Algorithm~\ref{algorithm:h} runs in polynomial-time
in the size of the BBIR, as it is known
conditioning takes polynomial time in a
binary decision diagram~\citep{darwiche2002knowledge}.
However, it is not clear what \Cref{algorithm:ub} is
upper-bounding.
The key is observing that, at any choice variable, taking the join $\sqcup$
greedily chooses the best possible value,
without caring about whether it is associated to a policy or not.
This allows us to upper-bound all completions $T$ of $P$,
as we demonstrate in the following theorem, proven in~\cref{appendix:proof ub correctness}.

\begin{theorem}\label{thm:ub correctness}
  Let $\BBIR$ be a BBIR and let $\varphi \in \{\varphi_i\}$. Let $P$ be a partial policy of $X$.
  Then for all completions $T$ of $P$ we have
  \begin{equation}\label{eq:ub correctness}
    ub(\BBIR, \varphi, P)
      \sqsupseteq \bigoplus_{m \models \varphi|_T} \bigotimes_{\ell \in m \cup T} w(\ell)
      = \AMC(\varphi|_T) \bigotimes_{\ell \in T} w(\ell).
  \end{equation}
\end{theorem}

Importantly, we can define a dual \textit{lower bound} algorithm $lb$
by taking Algorithm~\ref{algorithm:h}
and changing the join $\sqcup$ in line 5 to a meet $\sqcap$.
This proves vital when achieving full generality of the branch-and-bound,
as a simultaneous
lower and upper bound is required to maintain sound pruning in the presence of evidence.
We also state an important Lemma
that holds for both upper-and lower-bounds,
whose proof amounts to observing that for
total policies, no join is ever done when bounding,
leading to an exact value.

\begin{lemma}\label{lemma:ub on total policy is amc}
  For any BBIR $\BBIR$ and $\varphi \in \{\varphi\}$, for any total policy $T$
  of $X$, we have
  \begin{equation}
    ub(\BBIR, \varphi, T) =lb(\BBIR, \varphi, T)= \AMC(\varphi|_T, w).
  \end{equation}
\end{lemma}

\subsection{Upper Bounds in Action: a General Branch-and-Bound Algorithm}
\label{subsec:meu with evidence}

We have, so far, demonstrated some of the theory and intuition that leads into
the BBIR, and the efficient upper- and lower-bound operation it supports.
Now,
we can use it to our full advantage to implement a general branch-and-bound style
algorithm to solve optimization problems expressed over BBIR.
This subsumes a previous algorithm for MMAP by~\citet{huang2006solving}
and generalizes it to MEU and to any other branch and bound semiring.

The algorithm is given in Algorithm~\ref{algorithm:bb}. It finds the maximum of an
objective function $f$
(for example, the problems of~\cref{eq:bbir-meu,eq:bbir-mmap})
given an upper-bound $\mathsf{UB}_f$ for $f$ over partial policies,
which we describe for MEU and MMAP in~\cref{appendix:ub_f}.
$\mathsf{UB}_f$ for MEU and MMAP take full advantage of
Algorithm~\ref{algorithm:ub}, and are completed in polynomial time.

% For notational simplicity, instead of using the BBIR
% $(\{\varphi \land \gamma_{\pi} : \pi \in \mathcal A\}, A, w)$,
% we will use
% the tuple $(\{\varphi, \gamma\}, A, w)$
% in which $\gamma$ is the formula in which
% for all $\pi \in \mathcal A$, $\gamma|_{\pi} = \gamma_{\pi}$.
% For a given BBIR for MEU, and a policy $\pi \in \mathcal A$,
% we write $
%   \mathrm{EU}(\pi) =
%   \frac{\AMC((\varphi \land \gamma)|_{\pi}, w)_{\EU}}{\AMC(\gamma|_{\pi})_{\Pr}}
% $, with $w$ implicit.
% Note that this can be solved via two calls to $ub$,
% on $(\varphi \land \gamma)|_{\pi}$
% and $\gamma|_{\pi}$ respectively;
% this is an application of Lemma~\ref{lemma:ub on total policy is amc}.
% More generally, for any value $(a,b) \in \mathcal S$ and $r \in \R$, we define
% scalar division for $\mathcal S$:
% \begin{equation}
%   \frac{(a,b)}{r} = \begin{cases}
%     \paren{\frac a r , \frac b r} & r \neq 0, \\
%     (0, -\infty) & r = 0.
%   \end{cases}
% \end{equation}

\begin{figure}
  \begin{mdframed}{\footnotesize\begin{algorithmic}[1]
    \Procedure{$bb$}{$\BBIR,
    R,
    b,
    P_{curr}$}
    \If {$R = \eset$}
      \State $n = f(P_{curr})$
      \Comment{$P_{curr}$ will be a total policy of $X$}
      \State \textbf{return } $\max(n, b)$\Comment{max uses the total order.}
    \Else
       \State $r = pop(R)$
       \For {$\ell \in \{r, \overline r\}$}
        \State $m = \mathsf{UB}_f((\{\varphi|_{\ell}\}, X, w), P_{curr} \cup \{\ell\})$\label{line:join}
        \If {$m \not\sqsubseteq b$} \label{line:prune}
          \State $n = bb(\{\varphi|_{\ell}, \gamma|_{\ell}\}, R, b, P_{curr} \cup \{\ell\})$
          \Comment{$n$ will always be from a policy}
          \State $b = \max(n,b)$
        \EndIf
       \EndFor
       \State \textbf{return } $b$
    \EndIf
  \EndProcedure
  \end{algorithmic}}\end{mdframed}
  \caption{The branch-and-bound style algorithm
  calculating the optimum of a function $f$ admitting an upper-bound function
  $\mathsf{UB}_f$ for every partial policy.
  The tuple $\BBIR$ is a BBIR,
  $R$ is the remaining search space (initialized to $X$),
  $b$ is a lower-bound,
  and $P_{curr}$ is the current partial policy (initialized to $\emptyset$).}
  \label{algorithm:bb}
\end{figure}

We give a quick walkthrough of \Cref{algorithm:bb}. If $R = \eset$,
we hit a base case, in which our accumulated policy, $P_{curr}$ is
a total policy. We evaluate the expected utility and update our upper bound
accordingly. If $R \neq \eset$, then we let $r$ be some variable in $R$ and
$\ell \in \{r, \overline r\}$ a literal.
Then we consider the extension of partial policy $P_{curr}$ with $\{\ell\}$,
which is still a partial policy. We compute an upper bound for the BBIR
conditioned on this partial policy to form $m$.

The pruning is at Line~\ref{line:prune}; if $m \not\sqsubseteq b$, then
there is no recursion, pruning any policies containing
$P_{curr} \cup \{\ell\}$. This pruning is sound, as shown by the
following theorem, proven in~\cref{appendix:soundness of bb proof}.

\begin{theorem}\label{thm:soundness of bb}
  Algorithm~\ref{algorithm:bb}
  solves the MEU and MMAP problems of~\cref{subsubsec:mmap,subsubsec:meu}.
  % More generally, given that $\mathsf{UB}_f$ is sound, Algorithm~\ref{algorithm:bb}
  % solves the optimization problem $\max_{\pi \in inst(X)} f(\pi)$.
\end{theorem}

\textit{Remark.}
It should be noted that, although we have put in hard work to take advantage of
the factorized representation of the BBIR as much as possible,
\Cref{algorithm:bb} can run in possibly exponential time with respect
to the size of $A$ in the worst case. This is because in the worst case we still
face the \textit{search-space explosion} discussed in
Section~\ref{sec:overview}.  The worst case will happen when there is no
pruning: if the guard of Line~\ref{line:prune} is always satisfied, we will
iterate through all possible partial models, which is of size $2^{|A|}$.

However,
we ensured that the inner-loop of partial and total policy evaluation (\cref{line:join}
of \Cref{algorithm:bb}) runs in polynomial time \textit{with respect
to the size of the already factorized representation of the BBIR}.
So, even though we have a search-space explosion,
we can much more efficiently search through that policy space than an approach
that does not leverage compilation.


%%% Local Variables:
%%% mode: LaTeX
%%% TeX-master: "../oopsla-appendix.tex"
%%% End:


\section{$\dappl$: A Language for Maximum Expected Utility}\label{sec:dappl}

In the next two sections we will showcase the flexibility of our new
branch-and-bound IR by using it to implement two languages that support
very different kinds of reasoning over optimization.  By design we keep these
languages small so that they can be feasibly compiled into BBIR: in particular,
they will both support only bounded-domain discrete random variables and
statically bounded loops. These two restrictions are common in existing
compiled PPLs such as \texttt{Dice}~\citep{holtzen2020scaling}.

In this section we describe the syntax and semantics of \dappl{}.
In order to do this we describe first a small sublanguage of \dappl{},
named \util{}, in Section~\ref{subsec:util}.
Our goal for the semantics of \util{} is to yield the expected utility
of a policy, akin to the computations
via expectations done
in~\cref{eqn:example}.
Then, in Section~\ref{subsec:dappl},
we give \dappl{}'s syntax as an extension of that of \util{},
and its semantics as an evaluation function $\mathsf{MEU}$, a maximization
over \util{} programs derived from
applying a policy to a \dappl{} program,
The compilation rules to BBIR are given in~\cref{subsec:compiling-dappl},
concluding with an example compilation of~\cref{fig:motivation-dappl} to BBIR.


\subsection{The Syntax and Semantics of~\util}\label{subsec:util}

\util{} is a small functional first-order probabilistic programming language
with support for Bayesian conditioning, if-then-else,
and \texttt{flip}s of a biased coin with bias in the interval $[0,1]$.
We augment the syntax with the additional syntactic form, $\reward k e$,
to specify a utility of $k$ awarded before evaluating expression $e$.

\begin{wrapfigure}{r}{0.6\linewidth}
  \begin{mdframed}
  {\footnotesize\begin{align*}
    \text{Atomic expressions } \texttt{aexp} ::= \ & x \mid \tt \mid \ff &\\
    \text{Logical expressions } P ::= \ & \texttt{aexp} \mid P \land P \mid P \lor P \mid \neg P &\\
    \text{Expressions } \texttt{e} ::=\ & \return P  \mid \flip{\theta} \mid  \reward{k}{e} &\\
    &\mid  \ite x e e &\\ &\mid \bind{x}{e}{e} \mid \observe{x}{e} &
  \end{align*}}
  \end{mdframed}
  \caption{Syntax of \util{}, our core calculus for computing expected utility without decision-making.}
  \label{fig: util syntax}
\end{wrapfigure}

The syntax of $\util$ is given in Figure \ref{fig: util syntax}.
Programs are expressions without free variables.
We distinguish between pure computations $P$, which
take the form of logical operations as the only values
are Booleans, and impure computations $e$, which
represent probablistic \texttt{flip}s, \texttt{reward} accumulation,
and their control flow.
Observed events take the form
of exclusively pure computations.
We enforce such restrictions via the
more general \dappl{} type system given in~\cref{appendix:typesystem}.
There are only two types in
\util{}: the Boolean type \Bool{} and distributions over $\Bool$, $\Giry \Bool$,
constructed via the Giry monad~\citep{giry2006categorical}.

The semantics follows the denotational approach
of~\citet{staton2020probabilistic} or~\citet{li2023lilac}.
Expressions $\Gamma \proves e : \Giry \Bool$\footnote{
all $\util$ expressions are of type $\Giry \Bool$, proven in~\citet{cho2025scaling}.}
are interpreted as
a function $\denote e$ from assignments of free variables to Booleans ($\denote \Gamma$)
to a distribution over either pairs of Booleans and reals or $\bot$:
$\mathcal D ((\Bool \times \R) \cup \{ \bot \})$.
The intuition is that utilities are attached to successful program executions--that is,
programs that do not encounter a falsifying \texttt{observe}.
A successful $\util$ program execution will either end in $\tt$ or $\ff$; the
\texttt{reward}s encountered along the way are summed up and weighted by the
probability of the successful trace.
For details see~\cref{appendix:util semantics}.


Using this definition, we can define the expected utility of a $\util$ program.

\begin{definition}\label{def:eu util}
  Let $\cdot \proves e : \Giry \Bool$ be a $\util$ program.
  Let $\mathcal D = \denote{e}\bullet$, where $\denote{e}$ is the map taking
  the empty assignment $\bullet \in \denote{\cdot}$
  to a distribution $\mathcal D$ over either pairs of Booleans and reals or $\bot$.
  The \emph{expected utility} of $e$ is defined to be the
  conditional expected value of the real values in $\mathcal D$
  attached to a successful program execution returning $\tt$
  conditional on not achieving $\bot$:
  \begin{equation}
    \EU(e) = \sum_{r \in \R} r \times \mathcal D ((\tt,r) \mid \text{not }\bot).
    \footnote{The sum is computable because there can only be a finite number of program traces evaluating to true.}
  \end{equation}
\end{definition}

\subsection{The Syntax and Semantics of \dappl}\label{subsec:dappl}

\dappl~augments the syntax of \util{}
(as shown in Figure \ref{fig: util syntax}) with two new expressions:
\begin{itemize}
  \item $[\alpha_1, \cdots, \alpha_n]$,
  where $\alpha_1,\cdots, \alpha_n$ are a nonzero number of fresh names,\footnote{
  We style
  the capitalization of names of $\alpha_1, \cdots \alpha_n$,
  in a manner consistent with how variant names are capitalized in ML.}
  to construct a \textit{choice} between
  binary \textit{alternatives} $\alpha_1, \cdots, \alpha_n$, and
  \item $\choose e {\alpha_i \implies e_i}$ to destruct a choice
  in a syntax akin to ML-style pattern matching.
\end{itemize}

However, writing a semantics for $\dappl{}$ in the same fashion as $\util{}$
is not as simple as it looks.
The problem lies in the type of optimization problem being solved:
recall that MEU takes the maximum over expected utilities
(see \cref{subsubsec:meu-example}).
In particular, we are not nesting maxima and expected utility calculation,
of the form $\max\sum\max\sum \cdots \sum f(x)$, which is not equal to,
in general, to the general form of an MEU computation $\max \sum f(x)$,
a phenomenon we noticed in~\cref{sec:overview}.

To avoid this,
we use \util's already established semantics to our advantage.
For a \dappl~program $e$ with $m$ many choices,
let $C_k$ denote the $k$-th choice in some arbitrary ordering.
Then we say $\mathcal A = C_1 \times C_2 \times \cdots \times C_m$
is the \textit{policy space} for the expression
in which elements $\pi \in \mathcal A$ are \textit{policies}.
In essence, each $\pi$ denotes a sequence of
valid alternatives that can are chosen in a \dappl~program.

Given a \dappl~program $e$ and a policy $\pi$ for the program,
we can reduce $e$ into a \util~program
by (1) removing any syntax constructing choices $[\alpha_1, \cdots, \alpha_n]$, and
(2) reducing each choice destructor $\choose e {\alpha_i \implies e_i}$ to the $e_i$
corresponding to the name $\alpha_i$ present in $\pi$.
We make formal this transformation in~\citet{cho2025scaling},
as well as prove it sound for well-typed \dappl~programs.
We refer as $e|_{\pi}$ the \util~program derived by applying policy $\pi$ to \dappl~program $e$.

With this in mind, we can introduce an \emph{evaluation function} $\mathrm{MEU}:
\dappl{} \to \R$ which computes the maximum expected utility, completing our
semantics. This evaluation function is proved total for all well-typed \dappl{}
programs in~\cref{appendix:util semantics}.

\begin{definition}\label{def:meu for dappl}
  For a well-typed \dappl~program $e$, define
  \begin{equation}\label{eq:meu for dappl}
    \MEUfn{e} \triangleq \max_{\pi \in \mathcal A} \EU(e|_{\pi}),
  \end{equation}
  in which $\mathcal A$ is the policy space defined by all of the decisions in $e$.
\end{definition}

We endow
\dappl{} with significant syntactic sugar, including
discrete random variables and statically-bounded loops.
\subsection{Compiling \dappl}
\label{subsec:compiling-dappl}

In Section~\ref{subsec:dappl} we described
the syntax and semantics of \dappl.
In Section~\ref{sec:bbir} we described the BBIR
and how it admits an algorithm to solve MEU with evidence.
Now we discuss \dappl{}'s compilation to BBIR, formalizing
our intuition from
computing the example in
Figure~\ref{fig:motivation-dappl} into
the BDD in Figure~\ref{fig:motiv-a-bdd}.

We compile \dappl~expressions into a tuple $(\varphi, \gamma, w, R)$, where:
\begin{itemize}[leftmargin=*]
  \item $\varphi$ is an \textit{unnormalized formula},
  representing the control and data flow without observations,
  \item $\gamma$ is an \textit{accepting formula},
  representing observations,
  \item $w: vars(\varphi) \to \mathcal S$ is a weight function, and
  \item $R$ is a set of reward variables.
\end{itemize}

We write $e \leadsto \target$
to denote that a \dappl~program compiles to the tuple $\target$.
Then we apply the map $\target \mapsto (\{\varphi \land R,\gamma\},  D(\varphi), w)$,
where $D(\varphi)$ is the set of Boolean variables representing choices in $\varphi$,
to transform it into a BBIR for Algorithm~\ref{algorithm:bb}.

Selected compilation rules are given in Figure~\ref{fig:dappl bc},
and full compilation rules are given in~\citet{cho2025scaling}.
Many rules are
influenced by similar compilation schemes found in the
literature~\citep{holtzen2020scaling,de2007problog,saad2021sppl}.
We use $T,F$ to denote true and false in propositional logic,
distinguishing it from the $\tt,\ff$ Boolean values in \dappl.
We write $\conjneg{R}$ to denote the conjunction of all
negations of variables in $R$. To remark on the intution behind
several rules:

\begin{figure}
\begin{mdframed}
{\footnotesize
\begin{align*}
  \infer[\texttt{bc/reward}]
  {
    \reward k  e \leadsto
    (\varphi, \gamma, R \cup \{r_k\},
      w \cup \{r_k \mapsto (1,k), \overline{r_k} \mapsto (1,0)\})}
  {
    \text{fresh } r_k
    & e \leadsto (\varphi, \gamma, R,w)
  }
\end{align*}
\begin{align*}
  \infer[\texttt{bc/[]}]
  {[a_1, \cdots, a_n] \leadsto
  (\exactlyone{(v_1,\cdots, v_n)}, \top, \{v_i \mapsto (1,0), \overline{v_i} \mapsto (1,0)\}_{i \leq n}, \eset )}
  {\text{fresh }v_1, \cdots, v_n}
\end{align*}
\begin{align*}
  \infer[\texttt{bc/ite}]
  {
    \ite{x}{e_T}{e_E} \leadsto
    \begin{gathered}
      \big(
        (x \land \varphi_T \land R_T \land \conjneg{R_E})
          \lor (\overline{x} \land \varphi_E \land R_E \land \conjneg{R_T}), \\
        (x \land \gamma_T) \lor (\overline{x} \land \gamma_E),
        w_T \cup w_E,
        \eset \big)
    \end{gathered}
  } {
    x \leadsto
    (x, \top, \eset, \eset, \eset)
    &
    e_T \leadsto
    (\varphi_T, \gamma_T, w_T, R_T)
    &
    e_E \leadsto
    (\varphi_E, \gamma_E, w_E, R_E)
  }
\end{align*}
\begin{align*}
  \infer[\texttt{bc/choose}]
  {\choose x {a_i \implies e_i}
  \leadsto
  \begin{gathered}
    \Big(x \land \bigvee_i (a_i \land e_i
    \land \bigwedge_{j \neq i} \conjneg{R_j}),
    x \land \bigvee_i (a_i \land \gamma_i), \\
    \bigcup_i w_i, \bigcup_i R_i \Big)
  \end{gathered}
}
  {x \leadsto
  (x, \top, \eset, \eset, \eset)
  &
  \forall \ i.  \ e_i \leadsto (\varphi_i, \gamma_i, w_i, R_i)}
\end{align*}
}
\end{mdframed}
\caption{Selected Boolean compilation rules of \texttt{dappl}.
For complete rules see~\cref{appendix:dappl bc}.
}
\label{fig:dappl bc}
\end{figure}

\begin{enumerate}[leftmargin=*]
  \item The union of weight functions $w \cup w'$ is
  non-aliased -- there will never be
  $x \in \mathrm{dom}(w) \cap \mathrm{dom}(w')$ such that
  $w(x) \neq w'(x)$ or $w(\overline x) \neq w' (\overline x)$.
  \item The \texttt{bc/[]} enforces an ExactlyOne constraint
  on the introduced fresh Boolean variables $v_1,\cdots, v_n$.
  This is to disallow the behavior of evaluating multiple
  patterns in a \texttt{choose} statement.
  \item \texttt{bc/ite}
  enforces the condition that one cannot
  incorporate the rewards of one branch while branching into
  another by conjoining $\conjneg{R_E}$ and $\conjneg{R_T}$
  onto the disjuncts.
  We did this implicitly in the examples of
  Section~\ref{sec:overview} --
  without this constraint, we would get the incorrect
  expected utility for the policy \texttt{Umbrella},
  as the model $\{u, r, R_{10}, R_{-5}, R_{100}\}$
  would be a valid model.
  The $\bigwedge_{j \neq i} \conjneg{R_j}$ in \texttt{bc/choose}
  imposes the same restriction for choice pattern matching.
  \item We reset the accumulated rewards in \texttt{bc/ite}, as the rewards
  need to be scaled by the probability distribution defined by the value to be
  substituted into $x$. Thus, we start discharge our accumulated rewards to scale them
  appropriately and start anew.
\end{enumerate}

\begin{figure}
{\footnotesize
\begin{align*}
  \infer
  {
    \begin{array}{l}
      \dapplcode{s <- flip 0.5;} \\
      \dapplcode{choose [u,n]} \\
      \dapplcode{| u -> if r then reward 10 else reward -5} \\
      \dapplcode{| n -> if r then reward 100 else ()}
    \end{array}
    \leadsto
    \begin{array}{l}
      \mathrm{ExactlyOne}(u,n) \\
      \land (u \land ((f_{0.5} \land r_{10} \land \overline{r_5})
      \lor (\overline{f_{0.5}} \land r_{10} \land \overline{r_5})) \land \overline{r_{-100}})\\
      \land (n \land (f_{0.5} \land r_{-100}) \land \overline{r_{10}} \land \overline{r_5})\\
    \end{array}
  }
  {
    \infer
    {
      \dapplcode{flip 0.5} \leadsto f_{0.5}
    }
    {
      \text{fresh }f_{0.5}
    }
    &
    \infer
    {
      \begin{array}{l}
        \dapplcode{choose [u,n]} \\
        \dapplcode{| u ->} \\
        \dapplcode{  if s then reward 10 else reward -5} \\
        \dapplcode{| n ->} \\
        \dapplcode{  if s then reward 100 else ()}
      \end{array}
      \leadsto
      \begin{array}{l}
        \mathrm{ExactlyOne}(u,n) \\
        \land (u \land ((s \land r_{10} \land \overline{r_5})
        \lor (\overline{s} \land r_{10} \land \overline{r_5})) \land \overline{r_{-100}})\\
        \land (n \land (s \land r_{-100}) \land \overline{r_{10}} \land \overline{r_5})\\
      \end{array}
    }
    {
      \infer
      {
        \dapplcode{[u,n]} \leadsto \mathrm{ExactlyOne}(u,n)
      }
      {
        \text{fresh } u,n
      }
      &
      \infer
      {
        \begin{array}{l}
        \dapplcode{if s} \\
        \dapplcode{then reward 10} \\
        \dapplcode{else reward 5}
        \end{array}
        \leadsto
        \begin{array}{l}
          (s \land r_{10} \land \overline{r_5}) \\
          \lor (\overline{s} \land r_{5} \land \overline{r_{10}})
        \end{array}
      }
      {
        \vdots
      }
      &&&&
      \vdots
    }
  }
\end{align*}
}
\caption{Partial compilation tree of the code in Figure~\ref{fig:motivation-dappl},
showing the compiled unnormalized formula. We omit the accepting formula
as it evalutes to $\top$ as there is no evidence. We give only $\varphi$
for visual clarity.}
\label{fig:compilation tree}
\end{figure}

The following theorem
connects the \dappl{} semantics of Section~\ref{sec:dappl}
to the branch-and-bound algorithm discussed in
Section~\ref{subsec:meu with evidence}.
For proofs see~\cref{appendix:dappl correctness}:

\begin{theorem}\label{thm:compiler correctness}
  Let $e$ be a well-typed \dappl{} program.
  Let $e \leadsto \target$. Then we have
  \begin{equation}
    \MEUfn{e} = \mathrm{bb}(\{\varphi \land R, \gamma\}, w, D(\varphi)).
  \end{equation}
\end{theorem}

To see this theorem in action, we return to our original example code in
Figure~\ref{fig:motivation-dappl}. It compiles to the Boolean formula seen in
Figure~\ref{fig:compilation tree}. Let the compiled formula be $\varphi$.
Then we see that
\begin{align}
  \varphi|_u &= (f_{0.5} \land r_{10} \land \overline{r_5})
  \lor (\overline{f_{0.5}} \land r_{10} \land \overline{r_5}) \land \overline{r_{-100}} \\
  \varphi|_n &= (f_{0.5} \land r_{-100}) \land \overline{r_{10}} \land \overline{r_5}.
\end{align}
The $\AMC$ of $\varphi|_u$ and $\varphi|_n$ exactly match that of
$\varphi_u$ and $\varphi_{\overline u}$ in Equation~\ref{eq:formula-umbrella},
which completes the picture.

\section{$\pineappl$: A Language for MMAP}\label{sec:pineappl}


In this section, we describe the syntax, semantics, and boolean compilation of
$\pineappl$. $\pineappl$ is
different from $\dappl$ in the fact that it is
a first-order, imperative probabilistic programming
language with support for first-class MMAP computation, along with
marginal probability computations.
Much like the organization of~\cref{sec:dappl},
we will first introduce the syntax and semantics (\cref{subsec:pineappl-sem}),
then outline the Boolean compilation (\cref{subsec:pineappl-compl}).

\subsection{Syntax and Semantics of $\pineappl$}
\label{subsec:pineappl-sem}


\begin{wrapfigure}{r}{0.61\linewidth}
  \begin{mdframed}
    {\footnotesize\begin{align*}
    \text{Expressions } \texttt{e} ::= \ & \texttt{x} \mid \tt \mid \ff \mid
    \texttt{e} \land \texttt{e} \mid \texttt{e} \lor \texttt{e} \mid
    \neg \texttt{e} &\\
    \text{Statements } \texttt{s} ::= \ & \texttt{x} = e \mid x = \flip{\theta} \mid \texttt{s} ; \texttt{s}&\\
    &\mid
    \texttt{if e \{s\} else \{s\}} &\\
    &\mid (\texttt{m}_1, \hdots, \texttt{m}_n) = \texttt{mmap}(\texttt{x}_1, \hdots, \texttt{x}_n) &\\
    &\mid (\texttt{m}_1, \hdots, \texttt{m}_n) = \texttt{mmap}(\texttt{x}_1, \hdots, \texttt{x}_n) \; \texttt{with} \; \{ \texttt{e} \} &\\
    \text{Programs } P ::= \ & \texttt{s} ; \Pr(\texttt{e}) \mid \texttt{s} ; \Pr(\texttt{e}) \; \texttt{with} \; \{ \texttt{e}  \}
    \end{align*}}
  \end{mdframed}
    \caption{Full $\pineappl$ syntax.}
    \label{fig:pineappl-syntax}
\end{wrapfigure}

The full syntax of pineappl is in~\cref{fig:pineappl-syntax}.  A $\pineappl$
program is made of two parts: statements and a query.  Statements consist of
(1) variables bound to either \pineapplcode{flip}s or expressions over them,
(2) a \pineapplcode{mmap} statement for binding a set of variables
$\texttt{x}_1, \hdots, \texttt{x}_n$
to the MMAP state of variables $\texttt{m}_1, \hdots, \texttt{m}_n$, or
(3) a sequence of the above.
A query asks for the marginal probabilty of an
expression.
We assume all variables have unique names.
Note that
\pineapplcode{mmap} and \pineapplcode{Pr} can be followed by \pineapplcode{ with \{e\}},
denoting the \textit{observation} of expression \pineapplcode{e}.
We impose the additional restriction that no variables referenced in the observed
expression have been bound by \pineapplcode{mmap}.
We endow more sugar in the full language, including support for
multiple queries, categorical discrete random variables, and bounded loops
in~\citet{cho2025scaling}.


\begin{figure}
\begin{mdframed}
  {\footnotesize
  \begin{align*}
    \infer[\texttt{s/flip}]
    {
      (\texttt{x = flip} \ \theta, \mathcal{D}) \Downarrow
      \{\sigma \cup [x \mapsto \top] \mapsto \theta \times \mathcal{D}(\sigma)\} \cup
      \{\sigma \cup [x \mapsto \bot] \mapsto (1 - \theta) \times \mathcal{D}(\sigma)\}
    }
    {
      \mathrm{fresh} \; x
    }
  \end{align*}
  \begin{align*}
    \infer[\texttt{s/assn}]
    {
      (\texttt{x = e}, \mathcal{D}) \Downarrow
      \begin{gathered}
      \{\sigma \cup [x \mapsto \top] \mapsto p \times\mathcal{D}(\sigma) \mid e[\sigma] = \top, \sigma \in \dom(\mathcal{D})\}
      \\
      \cup
      \{\sigma \cup [x \mapsto \bot] \mapsto (1-p) \times \mathcal{D}(\sigma) \mid e[\sigma] = \bot, \sigma \in \dom(\mathcal{D})\}
      \end{gathered}
    }
    {
      \mathrm{fresh} \; x
      &
      p = \Pr_{\mathcal D}[\texttt e]
    }
  \end{align*}
  \begin{align*}
    \infer[\texttt{s/seq}]
    {
        (s_1; s_2, \mathcal{D}) \Downarrow \mathcal{D}''
    }
    {
        (s_1, \mathcal{D}) \Downarrow \mathcal{D}'
        & (s_2, \mathcal{D}') \Downarrow \mathcal{D}''
    }
  \end{align*}
  \begin{align*}
    \infer[\texttt{s/if}]
    {
      (\texttt{if e \{s$_1$\} else \{s$_2$\}}, \mathcal{D}) \Downarrow
      \{\sigma \mapsto p \times \mathcal{D}_1(\sigma) +
      (1-p) \times \mathcal{D}_2(\sigma)| \sigma \in \dom(\mathcal{D}_1)\}
    }
    {
      (\texttt s_1, \mathcal{D}) \Downarrow \mathcal{D}_1
      & (\texttt s_2, \mathcal{D}) \Downarrow \mathcal{D}_2
      & \Pr_{\mathcal D}[\texttt e] = p
    }
  \end{align*}
  \begin{align*}
    \infer[\texttt{s/mmap}]
    {
      (\vec{\texttt{m}}\texttt{ = mmap }\vec{\texttt{x}}, \mathcal{D}) \Downarrow
      \{\sigma \cup \sigma_m \mapsto \mathcal{D}(\sigma) \mid \sigma \in \dom(\mathcal{D}) \}
    }
    {
      \vec{A} = MMAP_{\mathcal D}(\vec{\texttt{x}})
      & \sigma_m = \{m_i \mapsto A_i \mid i \in [1,n]\}
    }
  \end{align*}
  \begin{align*}
    \infer[\texttt{s/mmap/with}]
    {
      (\vec{\texttt{m}}\texttt{ = mmap }\vec{\texttt{x}} \ \pineapplcode{ with \{e\}}, \mathcal{D})
      \Downarrow
      \{\sigma \cup \sigma_m \mapsto \mathcal{D}(\sigma) \mid \sigma \in \dom(\mathcal{D}) \}
    }
    {
      \vec{A} = MMAP_{\mathcal D}(\vec{\texttt{x}} \mid e)
      & \sigma_m = \{m_i \mapsto A_i \mid i \in [1,n]\}
    }
  \end{align*}
  \begin{align*}
    \infer[\texttt{p/pr}]
    {
      \texttt{s; Pr(e)}  \Downarrow_P \Pr_{\mathcal D} [\texttt{e}]
    }
    {
      (\texttt s, \eset) \Downarrow \mathcal{D}
    } \;\;
    \qquad
    \infer[\texttt{p/pr/with}]
    {
      \texttt{s; Pr(e$_1$) with \{e$_2$\}} \Downarrow_P \frac{\Pr_{\mathcal{D}}[\texttt e_1 \land \texttt e_2]}{\Pr_{\mathcal{D}}[\texttt e_2]}
    }
    {
    (\texttt s, \eset) \Downarrow \mathcal{D}
    }
  \end{align*}
  }
\end{mdframed}
\caption{Operational semantics for $\pineappl$.
  All variable names in a $\pineappl$ program are assumed unique.
  $x_i$ denotes the $i$-th component of a vector $\vec{x} = (x_1, \cdots, x_n)$.}
\label{fig:pineappl-os}
\end{figure}


$\pineappl{}$'s semantics are given by two relations: $\Downarrow$ and $\Downarrow_P$,
described in~\cref{fig:pineappl-os}.
The
$\Downarrow$ relation is a big-step operational semantics
relating pairs of statements and distributions $(s, \mathcal D)$
to a new distribution $\mathcal D'$.
These distributions are over assignments of variables.
The
$\Downarrow_P$ relation relates a $\pineappl{}$ program $P = s; q$
to a real number correponding to the probability of query $q$.


To remark on the notation behind several rules:

\begin{enumerate}[leftmargin=*]
  \item The $\Pr_{\mathcal D}[e]$ notation used in
  \texttt{s/assn}, \texttt{s/if}, \texttt{p/pr}, \texttt{p/pr/with}
  denotes the probability of the event in $\mathcal D$ that the Boolean expression $e$ is satisfied.
  \item $MMAP_{\mathcal D}$, as used in \texttt{s/mmap} and \texttt{s/mmap/with},
  is the marginal MAP operator of some vector of variables $\vec{\mathtt{x}}$
  over a distribution $\mathcal D$, potentially conditioned on an expression $e$. More precisely we can define $MMAP_{\mathcal D}$ as follows:
  \begin{equation}
    MMAP_{\mathcal D}(\vec{\mathtt x} \mid e) = \max_{\sigma \in inst(\vec{\mathtt x})} \mathcal D(\sigma \mid e),
  \end{equation}
  where $\mathcal D(\sigma \mid e)$ is the probability of the instantiation $\sigma$ in $\mathcal D$ conditional on $e$.
\end{enumerate}


% For each binding
% and existing trace, two additional traces are created, one where the bound
% variable is $\top$ and the other where it is $\bot$. Each trace is then
% re-weighted by the probability of the particular instantiation of variables in
% that trace. To accomplish this re-weighting, we rely on two evaluation functions,
% $\Pr_{\mathcal{T}}(\varphi)$ in \cref{def:pr-trace} and
% $\mathrm{MMAP}_{\mathcal{T}}(v)$ in \cref{def:mmap-trace}, which compute the
% weighted trace probability of $\varphi$ or the MMAP state of variables $v$,
% given a set of weighted traces $\mathcal{T}$, respectively.

% \begin{definition}[$\Pr$ over weighted traces]
%   \label{def:pr-trace}
%     For some set of weighted traces $\mathcal{T}$ and boolean formula $\varphi$,
%     \[
%       Pr_{\mathcal{T}}(\varphi) =
%         \sum_{\{(\sigma, w) \in \mathcal{T} \mid (\sigma \models \varphi)\}} w
%     \]
% \end{definition}
% \begin{definition}[MMAP over weighted traces]
%   \label{def:mmap-trace}
%   For some set of weighted traces $\mathcal{T}$, let $M$ be a set of boolean
%   variables such that $M \subseteq \mathcal{T}^{-1}$ and let $V = \mathcal{T}^{-1}
%   \setminus M$. Then,
%   \[
%     MMAP_{\mathcal{T}}(M) = \argmax_{\sigma \in \mathcal{M}} \sum_{v \in V}
%       \Pr_{\sigma \cap \mathcal{T}}(v)
%   \]
%   where $\mathcal{M}$ is the set of all possible assignments to the variables in
%   $M$.
% \end{definition}
% Thus, assignment re-weights traces by the probability of satisfying the assigned
% expression and bindings introduced by \texttt{if-else} statements re-weight
% traces by the probability of satisfying the guard.

% The only statement which breaks from the mold is the first-class \texttt{mmap}
% statement. The statement relies on the already computed traces to query for the
% MMAP state of a set of variables, these traces are then augmented with new
% bindings to this state. Notably, this expression does not double the number of
% traces, because we do not need to consider both assignments to the MMAP, since
% there is exactly one correct answer to the optimization problem posed by MMAP.
% The evaluation functions also give rise to the MMAP and marginal probability
% queries for the whole program. The full semantics are presented in
% \Cref{fig:pineappl-os}.\footnote{These semantics rely on a function $e \leadsto
% \varphi$, which compiles $\pineappl$ expressions to boolean formulae, this
% compilation is excedingly straightforward and thus we elide its presentation. We
% will reuse this compilation function in \cref{subsec:pineappl-compl}.}

% To handle observation, we must consider each query separately. For programs
% making a marginal probability query, \texttt{Pr(e) with o}, we first compute the
% unnormalized probability of \texttt{e \&\& o} and then compute the normalizing
% constant as the marginal probability of the observation \texttt{o}, this follows
% directly from Bayes' rule. As an example consider the $\pineappl$ program:
% \begin{pineapplcodeblock}
%  x $\sim$ flip(0.5); y $\sim$ flip(0.5); $\Pr$ (x $\land$ y $\land$ x) with x
% \end{pineapplcodeblock}
% After running the operational semantics on the statement of the program, the
% result is $\mathcal{T} = \{ ([x \mapsto \top; y \mapsto \top], 0.25),
%   ([x \mapsto \top; y \mapsto \bot], 0.25)
%   ([x \mapsto \bot; y \mapsto \top], 0.25)
%   ([x \mapsto \bot; y \mapsto \bot], 0.25)
% \}$. Therefore, the result of the query is
% \[
%   \frac{\Pr_{\mathcal{T}}(x \land y \land x)}{\Pr_{\mathcal{T}}(x)} = \frac{0.25}{0.5} = 0.5.
% \]

Finally, we can query the probability of an expression \texttt{e} over the
compiled distribution via $\Downarrow_P$. To handle observation, \pineapplcode{Pr(e) with \{o\}},
as with the rule \texttt{p/pr/with},
we first compute the unormalized probability of the observation being
true jointly with the query, $\Pr_{\mathcal{D}}[e \land o = \tt]$, and then divide by the
normalizing constant, $\Pr_{\mathcal{D}}[o = \tt]$; this is Bayes' rule.

\subsection{Boolean Compilation of $\pineappl{}$}
\label{subsec:pineappl-compl}
\begin{figure}
\begin{mdframed}
  {\footnotesize
  \begin{align*}
    \infer[\texttt{bc/mmap}]
    {
      (\vec{\texttt{m}}\texttt{ = mmap }\vec{\texttt{x}}, \mathcal{F}, w) \leadsto
      (\mathcal{F} \cup \{(m_i, k_i) \}, w \cup w_{M})
    }
    {
      \mathrm{fresh} \; k_i
      & \vec A = MMAP(\{\bigwedge_{(x, \varphi) \in \mathcal{F}} x \leftrightarrow \varphi, \eset\}, \vec{\texttt{x}}, w)
      &
      w_{M} = \{m_i \mapsto (1,1), k_i \mapsto A_i \}
    }
  \end{align*}
  \begin{align*}
    \infer[\texttt{bc/mmap/with}]
    {
      (\vec{\texttt{m}}\texttt{ = mmap }\vec{\texttt{x}} \ \pineapplcode{ with \{e\}}, \mathcal{F}, w) \leadsto
      (\mathcal{F} \cup \{(m_i, k_i) \}, w \cup w_{M})
    }
    {
      \mathrm{fresh} \; k_i
      & \texttt e \leadsto_E \psi
      & \vec A = MMAP(\{\bigwedge_{(x, \varphi) \in \mathcal{F}} x \leftrightarrow \varphi, \psi\}, \vec{\texttt{x}}, w)
      &
      w_{M} = \{m_i \mapsto (1,1), k_i \mapsto A_i \}
    }
  \end{align*}
  \begin{align*}
    \infer[\texttt{bc/pr}]
    {
      \texttt{s; Pr(e)} \leadsto_P (\chi \land \left( \bigwedge_{(x, \varphi) \in \mathcal{F}} x \leftrightarrow \varphi \right), \top,w)
    }
    {
      (\texttt{s}, \eset, \eset) \leadsto (\mathcal F, w)
      &
      \texttt{e} \leadsto_E \chi
    }
  \end{align*}
  \begin{align*}
    \infer[\texttt{bc/pr/with}]
    {
      \pineapplcode{s; Pr(e$_1$) with \{e$_2$\}} \leadsto_P (\chi \land \left(\bigwedge_{(x, \varphi) \in \mathcal{F}} x \leftrightarrow \varphi \right), \psi, w)
    }
    {
      (s, \eset, \eset) \leadsto (\mathcal F, w)
      &
      \texttt{e$_1$} \leadsto_E \chi
      &
      \texttt{e$_2$} \leadsto_E \psi
    }
  \end{align*}
  }
\end{mdframed}
\caption{Selected Boolean compilation rules for $\pineappl$. As shorthand, we write
$w \cup \{x \mapsto (a, b)\}$ instead of $w \cup \{(x \mapsto \top) \mapsto a, (x \mapsto \bot) \mapsto b\}$.
The $\leadsto_E$ relation translates expressions into Boolean formulae; explicit rules are given in~\cref{appendix:pineappl bc}. The symbol $\leftrightarrow$ denotes logical if-and-only-if.}
\label{fig:pineappl-compl}
\end{figure}

%%% Local Variables:
%%% mode: LaTeX
%%% TeX-master: "../oopsla-appendix"
%%% End:


Like $\dappl{}$, we compile $\pineappl$ programs to Boolean formulae as a tractable
representation. Key rules are in~\cref{fig:pineappl-compl} and full rules are in~\cref{appendix:pineappl-compl-complete}.
The BBIR is used in the \texttt{bc/mmap} and \texttt{bc/mmap/with} rule, where
the premise $MMAP$ is identical to that defined in~\cref{subsubsec:mmap}, and is solved via
Algorithm~\ref{algorithm:bb}. We define three relations:
\begin{itemize}[leftmargin=*]
  \item $e \leadsto_E \varphi$ compiles a $\pineappl$ expression to a Boolean formula,
  \item $(s, \mathcal{F}, w) \leadsto (\mathcal{F}', w)$ compiles a $\pineappl$ statement $s$,
  a set of pairs of identifers and formulae $\mathcal F$, and a weight map of literals $w$
  into a set $\mathcal{F'}$ and weight map $w'$, and
  \item $s ; q \leadsto_P (\varphi, \psi, w)$ with an unnormalized formula $\varphi$, an accepting formula $\psi$, and a weight map $w$.
\end{itemize}
% The restriction of $\mathcal F$ to formulas which take the form $x \leftrightarrow \varphi$
% is important as it preserves the binding structure of new variables.

To conclude the section, we give a correctness theorem, akin to~\cref{thm:compiler correctness}, proven in~\cref{appendix:proof-pineappl-correctness}.
% ~\citet{cho2025scaling}.

\begin{theorem}\label{thm:pineappl correctness}
  For a $\pineappl{}$ program $s;q$, let $s ; q\Downarrow_P p$ and
  $s ; q \leadsto_P (\chi \land \paren{\bigwedge_{(x, \varphi) \in \mathcal{F}} x \leftrightarrow \varphi},\psi, w)$.
  Then
  {\footnotesize\begin{equation}
    p = \frac{\AMC_{\R}\left(\chi \land \paren{\bigwedge_{(x, \varphi) \in \mathcal{F}} x \leftrightarrow \varphi} \land \psi,w\right)}{\AMC_{\R}(\psi \land \paren{\bigwedge_{(x, \varphi) \in \mathcal{F}} x \leftrightarrow \varphi}, w)}.
  \end{equation}}
\end{theorem}

%%% Local Variables:
%%% mode: LaTeX
%%% TeX-master: "../oopsla-appendix"
%%% End:


\section{Empirical Evaluation of $\dappl$ and $\pineappl$}\label{sec:eval}



\section{Implementation and Evaluation}
\label{sec:evaluation}

We prototype our proposal into a tool \toolName, using approximately 5K lines of OCaml (for the program analysis) and 5K lines of Python code (for the repair). 
In particular, we employ Z3~\cite{DBLP:conf/tacas/MouraB08} as the SMT solver, clingo~\cite{DBLP:books/sp/Lifschitz19} as the ASP solver, and Souffle~\cite{scholz2016fast} as the Datalog engine. %, respectively.
To show the effectiveness, 
we design the experimental evaluation to answer the 
following research questions (RQ):
(Experiments ran on a server with an Intel® Xeon® Platinum 8468V, 504GB RAM, and 192 cores. All the dataset are publicly available from \cite{zenodo_benchmark})

\begin{itemize}[align=left, leftmargin=*,labelindent=0pt]
\item \textbf{RQ1:} How effective is \toolName in verifying CTL properties for relatively small but complex programs, compared to the state-of-the-art tool  \function~\cite{DBLP:conf/sas/UrbanU018}?


\item \textbf{RQ2:} What is the effectiveness of \toolName in detecting real-world bugs, which can be encoded using both CTL and linear temporal logic (LTL), such as non-termination gathered from GitHub \cite{DBLP:conf/sigsoft/ShiXLZCL22} and unresponsive behaviours in protocols  \cite{DBLP:conf/icse/MengDLBR22}, compared with \ultimate~\cite{DBLP:conf/cav/DietschHLP15}?

\item \textbf{RQ3:} How effective is \toolName in repairing CTL violations identified in RQ1 and RQ2? which has not been achieved by any existing tools. 


 

\end{itemize}



% \begin{itemize}[align=left, leftmargin=*,labelindent=0pt]
% \item \textbf{RQ1:} What is the effectiveness of \toolName in verifying CTL properties in a set of relatively small yet challenging programs, compared to the state-of-the-art tools, T2~\cite{DBLP:conf/fmcad/CookKP14},  \function~\cite{DBLP:conf/sas/UrbanU018}, and \ultimate~\cite{DBLP:conf/cav/DietschHLP15}?


% \item \textbf{RQ2:} What is the effectiveness of \toolName in finding  real-world bugs, which can be encoded using CTL properties, such as non-termination 
% gathered from GitHub \cite{DBLP:conf/sigsoft/ShiXLZCL22} and unresponsive behaviours in protocol implementations \cite{DBLP:conf/icse/MengDLBR22}?

% \item \textbf{RQ3:} What is the effectiveness of \toolName in repairing CTL bugs from RQ1--2?

% \end{itemize}

%The benchmark programs are from various sources. More specifically, termination bugs from real-world projects \cite{DBLP:conf/sigsoft/ShiXLZCL22} and CTL analysis \cite{DBLP:conf/fmcad/CookKP14} \cite{DBLP:conf/sas/UrbanU018}, and temporal bugs in real-world protocol implementations \cite{DBLP:conf/icse/MengDLBR22}. 



% \ly{are termination bugs ok? Do we need to add new CTL bugs?}
\subsection{RQ1: Verifying CTL Properties}

% Please add the following required packages to your document preamble:
%  \Xhline{1.5\arrayrulewidth}

\hide{\begin{figure}[!h]
\vspace{-8mm}
\begin{lstlisting}[xleftmargin=0.2em,numbersep=6pt,basicstyle=\footnotesize\ttfamily]
(*@\textcolor{mGray}{//$EF(\m{resp}{\geq}5)$}@*)
int c = *; int resp = 0;
int curr_serv = 5; 
while (curr_serv > 0){ 
 if (*) {  
   c--; 
   curr_serv--;
   resp++;} 
 else if (c<curr_serv){
   curr_serv--; }}
\end{lstlisting} 
\vspace{-2mm}
\caption{A possibly terminating loop} 
\label{fig:terminating_loop}
\vspace{-2mm}
\end{figure}}


%loses precision due to a \emph{dual widening} \cite{DBLP:conf/tacas/CourantU17}, and 

The programs listed in \tabref{tab:comparewithFuntionT2} were obtained from the evaluation benchmark of \function, which includes a total of 83 test cases across over 2,000 lines of code. We categorize these test cases into six groups, labeled according to the types of CTL properties. 
These programs are short but challenging, as they often involve complex loops or require a more precise analysis of the target properties. The \function tends to be conservative, often leading it to return ``unknown" results, resulting in an accuracy rate of 27.7\%. In contrast, \toolName demonstrates advantages with improved accuracy, particularly in \ourToolSmallBenchmark. 
%achieved by the novel loop summaries. 
The failure cases faced by \toolName highlight our limitations when loop guards are not explicitly defined or when LRFs are inadequate to prove termination. 
Although both \function and \toolName struggle to obtain meaningful invariances for infinite loops, the benefits of our loop summaries become more apparent when proving properties related to termination, such as reachability and responsiveness.  




\begin{table}[!t]
\vspace{1.5mm}
\caption{Detecting real-world CTL bugs.}
\normalsize
\label{tab:comparewithCook}
\renewcommand{\arraystretch}{0.95}
\setlength{\tabcolsep}{4pt}  
\begin{tabular}{c|l|c|cc|cc}
\Xhline{1.5\arrayrulewidth}
\multicolumn{1}{l|}{\multirow{2}{*}{\textbf{}}} & \multirow{2}{*}{\textbf{Program}}        & \multirow{2}{*}{\textbf{LoC}} & \multicolumn{2}{c|}{\textbf{\ultimateshort}}   & \multicolumn{2}{c}{\textbf{\toolName}}             \\ \cline{4-7} 
  \multicolumn{1}{l|}{}                           &                                          &                               & \multicolumn{1}{c|}{\textbf{Res.}} & \textbf{Time} & \multicolumn{1}{c|}{\textbf{Res.}} & \textbf{Time} \\ \hline
  1 \xmark                                      & \multirow{2}{*}{\makecell[l]{libvncserver\\(c311535)}}   & 25                            & \multicolumn{1}{c|}{\xmark}      & 2.845         & \multicolumn{1}{c|}{\xmark}      & 0.855         \\  
  1 \cmark                                      &                                          & 27                            & \multicolumn{1}{c|}{\cmark}      & 3.743         & \multicolumn{1}{c|}{\cmark}      & 0.476         \\ \hline
  2 \xmark                                      & \multirow{2}{*}{\makecell[l]{Ffmpeg\\(a6cba06)}}         & 40                            & \multicolumn{1}{c|}{\xmark}      & 15.254        & \multicolumn{1}{c|}{\xmark}      & 0.606         \\  
  2 \cmark                                      &                                          & 44                            & \multicolumn{1}{c|}{\cmark}      & 40.176        & \multicolumn{1}{c|}{\cmark}      & 0.397         \\ \hline
  3 \xmark                                      & \multirow{2}{*}{\makecell[l]{cmus\\(d5396e4)}}           & 87                            & \multicolumn{1}{c|}{\xmark}      & 6.904         & \multicolumn{1}{c|}{\xmark}      & 0.579         \\  
  3 \cmark                                      &                                          & 86                            & \multicolumn{1}{c|}{\cmark}      & 33.572        & \multicolumn{1}{c|}{\cmark}      & 0.986         \\ \hline
  4 \xmark                                      & \multirow{2}{*}{\makecell[l]{e2fsprogs\\(caa6003)}}      & 58                            & \multicolumn{1}{c|}{\xmark}      & 5.952         & \multicolumn{1}{c|}{\xmark}      & 0.923         \\  
  4 \cmark                                      &                                          & 63                            & \multicolumn{1}{c|}{\cmark}      & 4.533         & \multicolumn{1}{c|}{\cmark}      & 0.842         \\ \hline
  5 \xmark                                      & \multirow{2}{*}{\makecell[l]{csound-an...\\(7a611ab)}} & 43                            & \multicolumn{1}{c|}{\xmark}      & 3.654         & \multicolumn{1}{c|}{\xmark}      & 0.782         \\  
  5 \cmark                                      &                                          & 45                            & \multicolumn{1}{c|}{TO}          & -             & \multicolumn{1}{c|}{\cmark}      & 0.648         \\ \hline
  6 \xmark                                      & \multirow{2}{*}{\makecell[l]{fontconfig\\(fa741cd)}}     & 25                            & \multicolumn{1}{c|}{\xmark}      & 3.856         & \multicolumn{1}{c|}{\xmark}      & 0.769         \\  
  6 \cmark                                      &                                          & 25                            & \multicolumn{1}{c|}{Error}       & -             & \multicolumn{1}{c|}{\cmark}      & 0.651         \\ \hline
  7 \xmark                                      & \multirow{2}{*}{\makecell[l]{asterisk\\(3322180)}}       & 22                            & \multicolumn{1}{c|}{\unk}        & 12.687        & \multicolumn{1}{c|}{\unk}        & 0.196         \\  
  7 \cmark                                      &                                          & 25                            & \multicolumn{1}{c|}{\unk}        & 11.325        & \multicolumn{1}{c|}{\unk}        & 0.34          \\ \hline
  8 \xmark                                      & \multirow{2}{*}{\makecell[l]{dpdk\\(cd64eeac)}}          & 45                            & \multicolumn{1}{c|}{\xmark}      & 3.712         & \multicolumn{1}{c|}{\xmark}      & 0.447         \\  
  8 \cmark                                      &                                          & 45                            & \multicolumn{1}{c|}{\cmark}      & 2.97          & \multicolumn{1}{c|}{\unk}        & 0.481         \\ \hline
  9 \xmark                                      & \multirow{2}{*}{\makecell[l]{xorg-server\\(930b9a06)}}   & 19                            & \multicolumn{1}{c|}{\xmark}      & 3.111         & \multicolumn{1}{c|}{\xmark}      & 0.581         \\  
  9 \cmark                                      &                                          & 20                            & \multicolumn{1}{c|}{\cmark}      & 3.101         & \multicolumn{1}{c|}{\cmark}      & 0.409         \\ \hline
  10 \xmark                                      & \multirow{2}{*}{\makecell[l]{pure-ftpd\\(37ad222)}}      & 42                            & \multicolumn{1}{c|}{\cmark}      & 2.555         & \multicolumn{1}{c|}{\xmark}      & 0.933         \\  
  10 \cmark                                      &                                          & 49                            & \multicolumn{1}{c|}{\cmark}        & 2.286         & \multicolumn{1}{c|}{\cmark}      & 0.383         \\ \hline
  11 \xmark  & \multirow{2}{*}{\makecell[l]{live555$_a$\\(181126)}} & 34  & \multicolumn{1}{c|}{\cmark} &  2.715         & \multicolumn{1}{c|}{\xmark}    & 0.513   \\  
  11 \cmark  &     &   37    & \multicolumn{1}{c|}{\cmark} &  2.837         & \multicolumn{1}{c|}{\cmark}      & 0.341 \\ \hline
  12 \xmark  & \multirow{2}{*}{\makecell[l]{openssl\\(b8d2439)}} & 88  & \multicolumn{1}{c|}{\xmark} &  4.15          & \multicolumn{1}{c|}{\xmark}    & 0.78   \\
  12 \cmark  &     &  88     & \multicolumn{1}{c|}{\cmark} &  3.809         & \multicolumn{1}{c|}{\cmark}      & 0.99 \\ \hline
  13 \xmark  & \multirow{2}{*}{\makecell[l]{live555$_b$\\(131205)}} & 83  & \multicolumn{1}{c|}{\xmark} & 2.838         & \multicolumn{1}{c|}{\xmark}    & 0.602     \\  
  13 \cmark  &    &   84     & \multicolumn{1}{c|}{\cmark} &  2.393         & \multicolumn{1}{c|}{\cmark}      & 0.565 \\ \Xhline{1.5\arrayrulewidth}
                                                   & {\bf{Total}}                                  & 1249  & \multicolumn{1}{c|}{\bestBaseLineReal}          & $>$180       & \multicolumn{1}{c|}{\ourToolRealBenchmark}              & 16.01        \\ \Xhline{1.5\arrayrulewidth}
  \end{tabular}
  \end{table}

\subsection{RQ2: CTL Analysis on  Real-world Projects}




Programs in \tabref{tab:comparewithCook} are from real-world repositories, each associated with a Git commit number where developers identify and fix the bug manually. 
In particular, the property used for programs 1-9 (drawn from \cite{DBLP:conf/sigsoft/ShiXLZCL22}) is  \code{AF(Exit())}, capturing non-termination bugs. The properties used for programs 10-13 (drawn from \cite{DBLP:conf/icse/MengDLBR22}) are of the form \code{AG(\phi_1{\rightarrow}AF(\phi_2))}, capturing unresponsive behaviours from the protocol implementation. 
We extracted the main segments of these real-world bugs into smaller programs (under 100 LoC each), preserving features like data structures and pointer arithmetic. Our evaluation includes both buggy (\eg 1\,\xmark) and developer-fixed (\eg 1\,\cmark) versions.
After converting the CTL properties to LTL formulas, we compared our tool with the latest release of UltimateLTL (v0.2.4), a regular participant in SV-COMP \cite{svcomp} with competitive performance. 
Both tools demonstrate high accuracy in bug detection, while \ultimateshort often requires longer processing time. 
This experiment indicates that LRFs can effectively handle commonly seen real-world loops, and \toolName performs a more lightweight summary computation without compromising accuracy. 



%Following the convention in \cite{DBLP:conf/sigsoft/ShiXLZCL22}, t
%Prior works \cite{DBLP:conf/sigsoft/ShiXLZCL22} gathered such examples by extracting 
%\toolName successfully identifies the majority of buggy and correct programs, with the exception of programs 7 and 8. 







{
\begin{table*}[!h]
  \centering
\caption{\label{tab:repair_benchmark}
{Experimental results for repairing CTL bugs. Time spent by the ASP solver is separately recorded. 
}
}
\small
\renewcommand{\arraystretch}{0.95}
  \setlength{\tabcolsep}{9pt}
\begin{tabular}{l|c|c|c|c|c|c|c|c}
  \Xhline{1.5\arrayrulewidth}
  \multicolumn{1}{c|}{\multirow{2}{*}{\textbf{Program}}} & \multicolumn{1}{c|}{\multirow{2}{*}{\shortstack{\textbf{LoC}\\\textbf{(Datalog)}}}} & \multicolumn{3}{c|}{\textbf{Configuration}}                                 & \multicolumn{1}{c|}{\multirow{2}{*}{\textbf{Fixed}}} & \multicolumn{1}{c|}{\multirow{2}{*}{\textbf{\#Patch}}} & \multicolumn{1}{c|}{\multirow{2}{*}{\textbf{ASP(s)}}} & \multirow{2}{*}{\textbf{Total(s)}} \\ \cline{3-5}

  \multicolumn{1}{c|}{}                                  & \multicolumn{1}{c|}{}                              & \multicolumn{1}{c|}{\textbf{Symbols}} & \multicolumn{1}{c|}{\textbf{Facts}} & \multicolumn{1}{c|}{\textbf{Template}} & \multicolumn{1}{c|}{} & \multicolumn{1}{c|}{} & \multicolumn{1}{c|}{}  &                                      \\ \hline

AF\_yEQ5 (\figref{fig:first_Example})                                           & 115                           & 3+0                   & 0+1                & Add                & \cmark     & 1                   & 0.979                              & 1.593                                \\
test\_until.c                                         & 101                            & 0+3                   & 1+0                & Delete                & \cmark     & 1                   & 0.023                              & 0.498                                \\
next.c                                                & 87                            & 0+4                   & 1+0                & Delete                & \cmark     & 1                   & 0.023                              & 0.472                                \\
libvncserver                                          & 118                            & 0+6                   & 1+0                & Delete                & \cmark     & 3                   & 0.049                              & 1.081                                \\
Ffmpeg                                                & 227                           & 0+12                  & 1+0                & Delete                & \cmark     & 4                   & 13.113                              & 13.335                                \\
cmus                                                  & 145                           & 0+12                  & 1+0                & Delete                & \cmark     & 4                   & 0.098                              & 2.052                                \\
e2fsprogs                                             & 109                           & 0+8                   & 1+0                & Delete                & \cmark     & 2                   & 0.075                              & 1.515                                \\
csound-android                                        & 183                           & 0+8                   & 1+0                & Delete                & \cmark     & 4                   & 0.076                              & 1.613                                \\
fontconfig                                            & 190                           & 0+11                  & 1+0                & Delete                & \cmark     & 6                   & 0.098                              & 2.507                                \\
dpdk                                                  & 196                           & 0+12                  & 1+0                & Delete                & \cmark     & 1                   & 0.091                              & 2.006                                \\
xorg-server                                           & 118                            & 0+2                   & 1+0                & Delete                & \cmark     & 2                   & 0.026                              & 0.605                                \\
pure-ftpd                                             & 258                           & 0+21                  & 1+0                & Delete                & \cmark     & 2                   & 0.069                              & 3.590                               \\
live$_a$                                              & 112                            & 3+4                   & 1+1                & Update                & \cmark     & 1                   & 0.552                              & 0.816                                \\
openssl                                               & 315                           & 1+0                   & 0+1                & Add.                & \cmark     & 1                   & 1.188                              & 2.277                                \\
live$_b$                                              & 217                           & 1+0                   & 0+1                & Add                & \cmark     & 1                   & 0.977                              & 1.494                                 \\
  \Xhline{1.5\arrayrulewidth}
\textbf{Total}                                                 & 2491                          &                       &                    &                   &           &                     & 17.437                              & 35.454                               \\ 
  \Xhline{1.5\arrayrulewidth}           
\end{tabular}

\vspace{-2mm}
\end{table*}
}


\subsection{RQ3: Repairing CTL Property Violations} 


\tabref{tab:repair_benchmark} gathers all the program instances (from \tabref{tab:comparewithFuntionT2} and \tabref{tab:comparewithCook}) that violate their specified CTL properties and are sent to \toolName for repair.   
The \textbf{Symbols} column records the number of symbolic constants + symbolic signs, while the \textbf{Facts} column records the number of facts allowed to be removed + added. 
We gradually increase the number of symbols and the maximum number of facts that can be added or deleted. 
The \textbf{Configuration} column shows the first successful configuration that led to finding patches, and we record the total searching time till reaching such configurations. 
We configure \toolName to apply three atomic templates in a breadth-first manner with a depth limit of 1, \ie, \tabref{tab:repair_benchmark} records the patch result after one iteration of the repair. 
The templates are applied sequentially in the order: delete, update, and add. The repair process stops when a correct patch is found or when all three templates have been attempted. 
%without success. 
% Because of this configuration, \toolName only finds one patch for Program 1 (AF\_yEQ5). 
% The patch inserting \plaincode{if (i>10||x==y) \{y=5; return;\}} mentioned in \figref{fig:Patched-program} cannot be found in current configuration, as it requires deleting facts then adding new facts on the updated program.
% The `Configuration' column in \tabref{tab:repair_benchmark} shows the number of symbolic constants and signs, the number of facts allowed to be removed and added, and the template used when a patch is found.

Due to the current configuration, \toolName only finds patch (b) for Program 1 (AF\_yEQ5), while the patch (a) shown in \figref{fig:Patched-program} can be obtained by allowing two iterations of the repair: the first iteration adds the conditional than a second iteration to add a new assignment on the updated program. 
Non-termination bugs are resolved within a single iteration by adding a conditional statement that provides an earlier exit. 
For instance, \figref{fig:term-Patched-program} illustrates the main logic of 1\,\xmark, which enters an infinite loop when \code{\m{linesToRead}{\leq}0}. 
\toolName successfully 
provides a fix that prevents \code{\m{linesToRead}{\leq}0} from occurring before entering the loop. Note that such patches are more desirable which fix the non-termination bug without dropping the loops completely. 
%much like the example shown in  \figref{fig:term-Patched-program}. At the same time, 
Unresponsive bugs involve adding more function calls or assignment modifications. 
%Most repairs were completed within seconds. 

On average, the time taken to solve ASP accounts for 49.2\% (17.437/35.454) of the total repair time. We also keep track of the number of patches that successfully eliminate the CTL violations. More than one patch is available for non-termination bugs, as some patches exit the entire program without entering the loop. 
While all the patches listed are valid, those that intend to cut off the main program logic can be excluded based on the minimum change criteria. 
After a manual inspection of each buggy program shown in \tabref{tab:repair_benchmark}, we confirmed that at least one generated patch is semantically equivalent to the fix provided by the developer. 
As the first tool to achieve automated repair of CTL violations, \toolName successfully resolves all reported bugs. 



\begin{figure}[!t]
\begin{lstlisting}[xleftmargin=6em,numbersep=6pt,basicstyle=\footnotesize\ttfamily]
void main(){ //AF(Exit())
  int lines ToRead = *;
  int h = *;
  (*@\repaircode{if ( linesToRead <= 0 )  return;}@*)
  while(h>0){
    if(linesToRead>h)  
        linesToRead=h; 
    h-=linesToRead;} 
  return;}
\end{lstlisting}
\caption{Fixing a Possible Hang Found in libvncserver \cite{LibVNCClient}}
\label{fig:term-Patched-program}
\end{figure}



\section{Related Work}\label{sec:related-work}

\section{RELATED WORK}
\label{sec:relatedwork}
In this section, we describe the previous works related to our proposal, which are divided into two parts. In Section~\ref{sec:relatedwork_exoplanet}, we present a review of approaches based on machine learning techniques for the detection of planetary transit signals. Section~\ref{sec:relatedwork_attention} provides an account of the approaches based on attention mechanisms applied in Astronomy.\par

\subsection{Exoplanet detection}
\label{sec:relatedwork_exoplanet}
Machine learning methods have achieved great performance for the automatic selection of exoplanet transit signals. One of the earliest applications of machine learning is a model named Autovetter \citep{MCcauliff}, which is a random forest (RF) model based on characteristics derived from Kepler pipeline statistics to classify exoplanet and false positive signals. Then, other studies emerged that also used supervised learning. \cite{mislis2016sidra} also used a RF, but unlike the work by \citet{MCcauliff}, they used simulated light curves and a box least square \citep[BLS;][]{kovacs2002box}-based periodogram to search for transiting exoplanets. \citet{thompson2015machine} proposed a k-nearest neighbors model for Kepler data to determine if a given signal has similarity to known transits. Unsupervised learning techniques were also applied, such as self-organizing maps (SOM), proposed \citet{armstrong2016transit}; which implements an architecture to segment similar light curves. In the same way, \citet{armstrong2018automatic} developed a combination of supervised and unsupervised learning, including RF and SOM models. In general, these approaches require a previous phase of feature engineering for each light curve. \par

%DL is a modern data-driven technology that automatically extracts characteristics, and that has been successful in classification problems from a variety of application domains. The architecture relies on several layers of NNs of simple interconnected units and uses layers to build increasingly complex and useful features by means of linear and non-linear transformation. This family of models is capable of generating increasingly high-level representations \citep{lecun2015deep}.

The application of DL for exoplanetary signal detection has evolved rapidly in recent years and has become very popular in planetary science.  \citet{pearson2018} and \citet{zucker2018shallow} developed CNN-based algorithms that learn from synthetic data to search for exoplanets. Perhaps one of the most successful applications of the DL models in transit detection was that of \citet{Shallue_2018}; who, in collaboration with Google, proposed a CNN named AstroNet that recognizes exoplanet signals in real data from Kepler. AstroNet uses the training set of labelled TCEs from the Autovetter planet candidate catalog of Q1–Q17 data release 24 (DR24) of the Kepler mission \citep{catanzarite2015autovetter}. AstroNet analyses the data in two views: a ``global view'', and ``local view'' \citep{Shallue_2018}. \par


% The global view shows the characteristics of the light curve over an orbital period, and a local view shows the moment at occurring the transit in detail

%different = space-based

Based on AstroNet, researchers have modified the original AstroNet model to rank candidates from different surveys, specifically for Kepler and TESS missions. \citet{ansdell2018scientific} developed a CNN trained on Kepler data, and included for the first time the information on the centroids, showing that the model improves performance considerably. Then, \citet{osborn2020rapid} and \citet{yu2019identifying} also included the centroids information, but in addition, \citet{osborn2020rapid} included information of the stellar and transit parameters. Finally, \citet{rao2021nigraha} proposed a pipeline that includes a new ``half-phase'' view of the transit signal. This half-phase view represents a transit view with a different time and phase. The purpose of this view is to recover any possible secondary eclipse (the object hiding behind the disk of the primary star).


%last pipeline applies a procedure after the prediction of the model to obtain new candidates, this process is carried out through a series of steps that include the evaluation with Discovery and Validation of Exoplanets (DAVE) \citet{kostov2019discovery} that was adapted for the TESS telescope.\par
%



\subsection{Attention mechanisms in astronomy}
\label{sec:relatedwork_attention}
Despite the remarkable success of attention mechanisms in sequential data, few papers have exploited their advantages in astronomy. In particular, there are no models based on attention mechanisms for detecting planets. Below we present a summary of the main applications of this modeling approach to astronomy, based on two points of view; performance and interpretability of the model.\par
%Attention mechanisms have not yet been explored in all sub-areas of astronomy. However, recent works show a successful application of the mechanism.
%performance

The application of attention mechanisms has shown improvements in the performance of some regression and classification tasks compared to previous approaches. One of the first implementations of the attention mechanism was to find gravitational lenses proposed by \citet{thuruthipilly2021finding}. They designed 21 self-attention-based encoder models, where each model was trained separately with 18,000 simulated images, demonstrating that the model based on the Transformer has a better performance and uses fewer trainable parameters compared to CNN. A novel application was proposed by \citet{lin2021galaxy} for the morphological classification of galaxies, who used an architecture derived from the Transformer, named Vision Transformer (VIT) \citep{dosovitskiy2020image}. \citet{lin2021galaxy} demonstrated competitive results compared to CNNs. Another application with successful results was proposed by \citet{zerveas2021transformer}; which first proposed a transformer-based framework for learning unsupervised representations of multivariate time series. Their methodology takes advantage of unlabeled data to train an encoder and extract dense vector representations of time series. Subsequently, they evaluate the model for regression and classification tasks, demonstrating better performance than other state-of-the-art supervised methods, even with data sets with limited samples.

%interpretation
Regarding the interpretability of the model, a recent contribution that analyses the attention maps was presented by \citet{bowles20212}, which explored the use of group-equivariant self-attention for radio astronomy classification. Compared to other approaches, this model analysed the attention maps of the predictions and showed that the mechanism extracts the brightest spots and jets of the radio source more clearly. This indicates that attention maps for prediction interpretation could help experts see patterns that the human eye often misses. \par

In the field of variable stars, \citet{allam2021paying} employed the mechanism for classifying multivariate time series in variable stars. And additionally, \citet{allam2021paying} showed that the activation weights are accommodated according to the variation in brightness of the star, achieving a more interpretable model. And finally, related to the TESS telescope, \citet{morvan2022don} proposed a model that removes the noise from the light curves through the distribution of attention weights. \citet{morvan2022don} showed that the use of the attention mechanism is excellent for removing noise and outliers in time series datasets compared with other approaches. In addition, the use of attention maps allowed them to show the representations learned from the model. \par

Recent attention mechanism approaches in astronomy demonstrate comparable results with earlier approaches, such as CNNs. At the same time, they offer interpretability of their results, which allows a post-prediction analysis. \par



\section{Conclusion and Future Work}

\section{Conclusion}
In this work, we propose a simple yet effective approach, called SMILE, for graph few-shot learning with fewer tasks. Specifically, we introduce a novel dual-level mixup strategy, including within-task and across-task mixup, for enriching the diversity of nodes within each task and the diversity of tasks. Also, we incorporate the degree-based prior information to learn expressive node embeddings. Theoretically, we prove that SMILE effectively enhances the model's generalization performance. Empirically, we conduct extensive experiments on multiple benchmarks and the results suggest that SMILE significantly outperforms other baselines, including both in-domain and cross-domain few-shot settings.

\section*{Data Availability Statement}
The software that supports~\cref{sec:eval} is available on Zenodo~\citep{artifact}.

\section*{Acknowledgments}
We thank the anonymous reviewers for their helpful guidance.
This project was supported by the National Science Foundation under grant \#2220408.

\bibliography{meu.bib}


% Appendix
\appendix
\subsection{Lloyd-Max Algorithm}
\label{subsec:Lloyd-Max}
For a given quantization bitwidth $B$ and an operand $\bm{X}$, the Lloyd-Max algorithm finds $2^B$ quantization levels $\{\hat{x}_i\}_{i=1}^{2^B}$ such that quantizing $\bm{X}$ by rounding each scalar in $\bm{X}$ to the nearest quantization level minimizes the quantization MSE. 

The algorithm starts with an initial guess of quantization levels and then iteratively computes quantization thresholds $\{\tau_i\}_{i=1}^{2^B-1}$ and updates quantization levels $\{\hat{x}_i\}_{i=1}^{2^B}$. Specifically, at iteration $n$, thresholds are set to the midpoints of the previous iteration's levels:
\begin{align*}
    \tau_i^{(n)}=\frac{\hat{x}_i^{(n-1)}+\hat{x}_{i+1}^{(n-1)}}2 \text{ for } i=1\ldots 2^B-1
\end{align*}
Subsequently, the quantization levels are re-computed as conditional means of the data regions defined by the new thresholds:
\begin{align*}
    \hat{x}_i^{(n)}=\mathbb{E}\left[ \bm{X} \big| \bm{X}\in [\tau_{i-1}^{(n)},\tau_i^{(n)}] \right] \text{ for } i=1\ldots 2^B
\end{align*}
where to satisfy boundary conditions we have $\tau_0=-\infty$ and $\tau_{2^B}=\infty$. The algorithm iterates the above steps until convergence.

Figure \ref{fig:lm_quant} compares the quantization levels of a $7$-bit floating point (E3M3) quantizer (left) to a $7$-bit Lloyd-Max quantizer (right) when quantizing a layer of weights from the GPT3-126M model at a per-tensor granularity. As shown, the Lloyd-Max quantizer achieves substantially lower quantization MSE. Further, Table \ref{tab:FP7_vs_LM7} shows the superior perplexity achieved by Lloyd-Max quantizers for bitwidths of $7$, $6$ and $5$. The difference between the quantizers is clear at 5 bits, where per-tensor FP quantization incurs a drastic and unacceptable increase in perplexity, while Lloyd-Max quantization incurs a much smaller increase. Nevertheless, we note that even the optimal Lloyd-Max quantizer incurs a notable ($\sim 1.5$) increase in perplexity due to the coarse granularity of quantization. 

\begin{figure}[h]
  \centering
  \includegraphics[width=0.7\linewidth]{sections/figures/LM7_FP7.pdf}
  \caption{\small Quantization levels and the corresponding quantization MSE of Floating Point (left) vs Lloyd-Max (right) Quantizers for a layer of weights in the GPT3-126M model.}
  \label{fig:lm_quant}
\end{figure}

\begin{table}[h]\scriptsize
\begin{center}
\caption{\label{tab:FP7_vs_LM7} \small Comparing perplexity (lower is better) achieved by floating point quantizers and Lloyd-Max quantizers on a GPT3-126M model for the Wikitext-103 dataset.}
\begin{tabular}{c|cc|c}
\hline
 \multirow{2}{*}{\textbf{Bitwidth}} & \multicolumn{2}{|c|}{\textbf{Floating-Point Quantizer}} & \textbf{Lloyd-Max Quantizer} \\
 & Best Format & Wikitext-103 Perplexity & Wikitext-103 Perplexity \\
\hline
7 & E3M3 & 18.32 & 18.27 \\
6 & E3M2 & 19.07 & 18.51 \\
5 & E4M0 & 43.89 & 19.71 \\
\hline
\end{tabular}
\end{center}
\end{table}

\subsection{Proof of Local Optimality of LO-BCQ}
\label{subsec:lobcq_opt_proof}
For a given block $\bm{b}_j$, the quantization MSE during LO-BCQ can be empirically evaluated as $\frac{1}{L_b}\lVert \bm{b}_j- \bm{\hat{b}}_j\rVert^2_2$ where $\bm{\hat{b}}_j$ is computed from equation (\ref{eq:clustered_quantization_definition}) as $C_{f(\bm{b}_j)}(\bm{b}_j)$. Further, for a given block cluster $\mathcal{B}_i$, we compute the quantization MSE as $\frac{1}{|\mathcal{B}_{i}|}\sum_{\bm{b} \in \mathcal{B}_{i}} \frac{1}{L_b}\lVert \bm{b}- C_i^{(n)}(\bm{b})\rVert^2_2$. Therefore, at the end of iteration $n$, we evaluate the overall quantization MSE $J^{(n)}$ for a given operand $\bm{X}$ composed of $N_c$ block clusters as:
\begin{align*}
    \label{eq:mse_iter_n}
    J^{(n)} = \frac{1}{N_c} \sum_{i=1}^{N_c} \frac{1}{|\mathcal{B}_{i}^{(n)}|}\sum_{\bm{v} \in \mathcal{B}_{i}^{(n)}} \frac{1}{L_b}\lVert \bm{b}- B_i^{(n)}(\bm{b})\rVert^2_2
\end{align*}

At the end of iteration $n$, the codebooks are updated from $\mathcal{C}^{(n-1)}$ to $\mathcal{C}^{(n)}$. However, the mapping of a given vector $\bm{b}_j$ to quantizers $\mathcal{C}^{(n)}$ remains as  $f^{(n)}(\bm{b}_j)$. At the next iteration, during the vector clustering step, $f^{(n+1)}(\bm{b}_j)$ finds new mapping of $\bm{b}_j$ to updated codebooks $\mathcal{C}^{(n)}$ such that the quantization MSE over the candidate codebooks is minimized. Therefore, we obtain the following result for $\bm{b}_j$:
\begin{align*}
\frac{1}{L_b}\lVert \bm{b}_j - C_{f^{(n+1)}(\bm{b}_j)}^{(n)}(\bm{b}_j)\rVert^2_2 \le \frac{1}{L_b}\lVert \bm{b}_j - C_{f^{(n)}(\bm{b}_j)}^{(n)}(\bm{b}_j)\rVert^2_2
\end{align*}

That is, quantizing $\bm{b}_j$ at the end of the block clustering step of iteration $n+1$ results in lower quantization MSE compared to quantizing at the end of iteration $n$. Since this is true for all $\bm{b} \in \bm{X}$, we assert the following:
\begin{equation}
\begin{split}
\label{eq:mse_ineq_1}
    \tilde{J}^{(n+1)} &= \frac{1}{N_c} \sum_{i=1}^{N_c} \frac{1}{|\mathcal{B}_{i}^{(n+1)}|}\sum_{\bm{b} \in \mathcal{B}_{i}^{(n+1)}} \frac{1}{L_b}\lVert \bm{b} - C_i^{(n)}(b)\rVert^2_2 \le J^{(n)}
\end{split}
\end{equation}
where $\tilde{J}^{(n+1)}$ is the the quantization MSE after the vector clustering step at iteration $n+1$.

Next, during the codebook update step (\ref{eq:quantizers_update}) at iteration $n+1$, the per-cluster codebooks $\mathcal{C}^{(n)}$ are updated to $\mathcal{C}^{(n+1)}$ by invoking the Lloyd-Max algorithm \citep{Lloyd}. We know that for any given value distribution, the Lloyd-Max algorithm minimizes the quantization MSE. Therefore, for a given vector cluster $\mathcal{B}_i$ we obtain the following result:

\begin{equation}
    \frac{1}{|\mathcal{B}_{i}^{(n+1)}|}\sum_{\bm{b} \in \mathcal{B}_{i}^{(n+1)}} \frac{1}{L_b}\lVert \bm{b}- C_i^{(n+1)}(\bm{b})\rVert^2_2 \le \frac{1}{|\mathcal{B}_{i}^{(n+1)}|}\sum_{\bm{b} \in \mathcal{B}_{i}^{(n+1)}} \frac{1}{L_b}\lVert \bm{b}- C_i^{(n)}(\bm{b})\rVert^2_2
\end{equation}

The above equation states that quantizing the given block cluster $\mathcal{B}_i$ after updating the associated codebook from $C_i^{(n)}$ to $C_i^{(n+1)}$ results in lower quantization MSE. Since this is true for all the block clusters, we derive the following result: 
\begin{equation}
\begin{split}
\label{eq:mse_ineq_2}
     J^{(n+1)} &= \frac{1}{N_c} \sum_{i=1}^{N_c} \frac{1}{|\mathcal{B}_{i}^{(n+1)}|}\sum_{\bm{b} \in \mathcal{B}_{i}^{(n+1)}} \frac{1}{L_b}\lVert \bm{b}- C_i^{(n+1)}(\bm{b})\rVert^2_2  \le \tilde{J}^{(n+1)}   
\end{split}
\end{equation}

Following (\ref{eq:mse_ineq_1}) and (\ref{eq:mse_ineq_2}), we find that the quantization MSE is non-increasing for each iteration, that is, $J^{(1)} \ge J^{(2)} \ge J^{(3)} \ge \ldots \ge J^{(M)}$ where $M$ is the maximum number of iterations. 
%Therefore, we can say that if the algorithm converges, then it must be that it has converged to a local minimum. 
\hfill $\blacksquare$


\begin{figure}
    \begin{center}
    \includegraphics[width=0.5\textwidth]{sections//figures/mse_vs_iter.pdf}
    \end{center}
    \caption{\small NMSE vs iterations during LO-BCQ compared to other block quantization proposals}
    \label{fig:nmse_vs_iter}
\end{figure}

Figure \ref{fig:nmse_vs_iter} shows the empirical convergence of LO-BCQ across several block lengths and number of codebooks. Also, the MSE achieved by LO-BCQ is compared to baselines such as MXFP and VSQ. As shown, LO-BCQ converges to a lower MSE than the baselines. Further, we achieve better convergence for larger number of codebooks ($N_c$) and for a smaller block length ($L_b$), both of which increase the bitwidth of BCQ (see Eq \ref{eq:bitwidth_bcq}).


\subsection{Additional Accuracy Results}
%Table \ref{tab:lobcq_config} lists the various LOBCQ configurations and their corresponding bitwidths.
\begin{table}
\setlength{\tabcolsep}{4.75pt}
\begin{center}
\caption{\label{tab:lobcq_config} Various LO-BCQ configurations and their bitwidths.}
\begin{tabular}{|c||c|c|c|c||c|c||c|} 
\hline
 & \multicolumn{4}{|c||}{$L_b=8$} & \multicolumn{2}{|c||}{$L_b=4$} & $L_b=2$ \\
 \hline
 \backslashbox{$L_A$\kern-1em}{\kern-1em$N_c$} & 2 & 4 & 8 & 16 & 2 & 4 & 2 \\
 \hline
 64 & 4.25 & 4.375 & 4.5 & 4.625 & 4.375 & 4.625 & 4.625\\
 \hline
 32 & 4.375 & 4.5 & 4.625& 4.75 & 4.5 & 4.75 & 4.75 \\
 \hline
 16 & 4.625 & 4.75& 4.875 & 5 & 4.75 & 5 & 5 \\
 \hline
\end{tabular}
\end{center}
\end{table}

%\subsection{Perplexity achieved by various LO-BCQ configurations on Wikitext-103 dataset}

\begin{table} \centering
\begin{tabular}{|c||c|c|c|c||c|c||c|} 
\hline
 $L_b \rightarrow$& \multicolumn{4}{c||}{8} & \multicolumn{2}{c||}{4} & 2\\
 \hline
 \backslashbox{$L_A$\kern-1em}{\kern-1em$N_c$} & 2 & 4 & 8 & 16 & 2 & 4 & 2  \\
 %$N_c \rightarrow$ & 2 & 4 & 8 & 16 & 2 & 4 & 2 \\
 \hline
 \hline
 \multicolumn{8}{c}{GPT3-1.3B (FP32 PPL = 9.98)} \\ 
 \hline
 \hline
 64 & 10.40 & 10.23 & 10.17 & 10.15 &  10.28 & 10.18 & 10.19 \\
 \hline
 32 & 10.25 & 10.20 & 10.15 & 10.12 &  10.23 & 10.17 & 10.17 \\
 \hline
 16 & 10.22 & 10.16 & 10.10 & 10.09 &  10.21 & 10.14 & 10.16 \\
 \hline
  \hline
 \multicolumn{8}{c}{GPT3-8B (FP32 PPL = 7.38)} \\ 
 \hline
 \hline
 64 & 7.61 & 7.52 & 7.48 &  7.47 &  7.55 &  7.49 & 7.50 \\
 \hline
 32 & 7.52 & 7.50 & 7.46 &  7.45 &  7.52 &  7.48 & 7.48  \\
 \hline
 16 & 7.51 & 7.48 & 7.44 &  7.44 &  7.51 &  7.49 & 7.47  \\
 \hline
\end{tabular}
\caption{\label{tab:ppl_gpt3_abalation} Wikitext-103 perplexity across GPT3-1.3B and 8B models.}
\end{table}

\begin{table} \centering
\begin{tabular}{|c||c|c|c|c||} 
\hline
 $L_b \rightarrow$& \multicolumn{4}{c||}{8}\\
 \hline
 \backslashbox{$L_A$\kern-1em}{\kern-1em$N_c$} & 2 & 4 & 8 & 16 \\
 %$N_c \rightarrow$ & 2 & 4 & 8 & 16 & 2 & 4 & 2 \\
 \hline
 \hline
 \multicolumn{5}{|c|}{Llama2-7B (FP32 PPL = 5.06)} \\ 
 \hline
 \hline
 64 & 5.31 & 5.26 & 5.19 & 5.18  \\
 \hline
 32 & 5.23 & 5.25 & 5.18 & 5.15  \\
 \hline
 16 & 5.23 & 5.19 & 5.16 & 5.14  \\
 \hline
 \multicolumn{5}{|c|}{Nemotron4-15B (FP32 PPL = 5.87)} \\ 
 \hline
 \hline
 64  & 6.3 & 6.20 & 6.13 & 6.08  \\
 \hline
 32  & 6.24 & 6.12 & 6.07 & 6.03  \\
 \hline
 16  & 6.12 & 6.14 & 6.04 & 6.02  \\
 \hline
 \multicolumn{5}{|c|}{Nemotron4-340B (FP32 PPL = 3.48)} \\ 
 \hline
 \hline
 64 & 3.67 & 3.62 & 3.60 & 3.59 \\
 \hline
 32 & 3.63 & 3.61 & 3.59 & 3.56 \\
 \hline
 16 & 3.61 & 3.58 & 3.57 & 3.55 \\
 \hline
\end{tabular}
\caption{\label{tab:ppl_llama7B_nemo15B} Wikitext-103 perplexity compared to FP32 baseline in Llama2-7B and Nemotron4-15B, 340B models}
\end{table}

%\subsection{Perplexity achieved by various LO-BCQ configurations on MMLU dataset}


\begin{table} \centering
\begin{tabular}{|c||c|c|c|c||c|c|c|c|} 
\hline
 $L_b \rightarrow$& \multicolumn{4}{c||}{8} & \multicolumn{4}{c||}{8}\\
 \hline
 \backslashbox{$L_A$\kern-1em}{\kern-1em$N_c$} & 2 & 4 & 8 & 16 & 2 & 4 & 8 & 16  \\
 %$N_c \rightarrow$ & 2 & 4 & 8 & 16 & 2 & 4 & 2 \\
 \hline
 \hline
 \multicolumn{5}{|c|}{Llama2-7B (FP32 Accuracy = 45.8\%)} & \multicolumn{4}{|c|}{Llama2-70B (FP32 Accuracy = 69.12\%)} \\ 
 \hline
 \hline
 64 & 43.9 & 43.4 & 43.9 & 44.9 & 68.07 & 68.27 & 68.17 & 68.75 \\
 \hline
 32 & 44.5 & 43.8 & 44.9 & 44.5 & 68.37 & 68.51 & 68.35 & 68.27  \\
 \hline
 16 & 43.9 & 42.7 & 44.9 & 45 & 68.12 & 68.77 & 68.31 & 68.59  \\
 \hline
 \hline
 \multicolumn{5}{|c|}{GPT3-22B (FP32 Accuracy = 38.75\%)} & \multicolumn{4}{|c|}{Nemotron4-15B (FP32 Accuracy = 64.3\%)} \\ 
 \hline
 \hline
 64 & 36.71 & 38.85 & 38.13 & 38.92 & 63.17 & 62.36 & 63.72 & 64.09 \\
 \hline
 32 & 37.95 & 38.69 & 39.45 & 38.34 & 64.05 & 62.30 & 63.8 & 64.33  \\
 \hline
 16 & 38.88 & 38.80 & 38.31 & 38.92 & 63.22 & 63.51 & 63.93 & 64.43  \\
 \hline
\end{tabular}
\caption{\label{tab:mmlu_abalation} Accuracy on MMLU dataset across GPT3-22B, Llama2-7B, 70B and Nemotron4-15B models.}
\end{table}


%\subsection{Perplexity achieved by various LO-BCQ configurations on LM evaluation harness}

\begin{table} \centering
\begin{tabular}{|c||c|c|c|c||c|c|c|c|} 
\hline
 $L_b \rightarrow$& \multicolumn{4}{c||}{8} & \multicolumn{4}{c||}{8}\\
 \hline
 \backslashbox{$L_A$\kern-1em}{\kern-1em$N_c$} & 2 & 4 & 8 & 16 & 2 & 4 & 8 & 16  \\
 %$N_c \rightarrow$ & 2 & 4 & 8 & 16 & 2 & 4 & 2 \\
 \hline
 \hline
 \multicolumn{5}{|c|}{Race (FP32 Accuracy = 37.51\%)} & \multicolumn{4}{|c|}{Boolq (FP32 Accuracy = 64.62\%)} \\ 
 \hline
 \hline
 64 & 36.94 & 37.13 & 36.27 & 37.13 & 63.73 & 62.26 & 63.49 & 63.36 \\
 \hline
 32 & 37.03 & 36.36 & 36.08 & 37.03 & 62.54 & 63.51 & 63.49 & 63.55  \\
 \hline
 16 & 37.03 & 37.03 & 36.46 & 37.03 & 61.1 & 63.79 & 63.58 & 63.33  \\
 \hline
 \hline
 \multicolumn{5}{|c|}{Winogrande (FP32 Accuracy = 58.01\%)} & \multicolumn{4}{|c|}{Piqa (FP32 Accuracy = 74.21\%)} \\ 
 \hline
 \hline
 64 & 58.17 & 57.22 & 57.85 & 58.33 & 73.01 & 73.07 & 73.07 & 72.80 \\
 \hline
 32 & 59.12 & 58.09 & 57.85 & 58.41 & 73.01 & 73.94 & 72.74 & 73.18  \\
 \hline
 16 & 57.93 & 58.88 & 57.93 & 58.56 & 73.94 & 72.80 & 73.01 & 73.94  \\
 \hline
\end{tabular}
\caption{\label{tab:mmlu_abalation} Accuracy on LM evaluation harness tasks on GPT3-1.3B model.}
\end{table}

\begin{table} \centering
\begin{tabular}{|c||c|c|c|c||c|c|c|c|} 
\hline
 $L_b \rightarrow$& \multicolumn{4}{c||}{8} & \multicolumn{4}{c||}{8}\\
 \hline
 \backslashbox{$L_A$\kern-1em}{\kern-1em$N_c$} & 2 & 4 & 8 & 16 & 2 & 4 & 8 & 16  \\
 %$N_c \rightarrow$ & 2 & 4 & 8 & 16 & 2 & 4 & 2 \\
 \hline
 \hline
 \multicolumn{5}{|c|}{Race (FP32 Accuracy = 41.34\%)} & \multicolumn{4}{|c|}{Boolq (FP32 Accuracy = 68.32\%)} \\ 
 \hline
 \hline
 64 & 40.48 & 40.10 & 39.43 & 39.90 & 69.20 & 68.41 & 69.45 & 68.56 \\
 \hline
 32 & 39.52 & 39.52 & 40.77 & 39.62 & 68.32 & 67.43 & 68.17 & 69.30  \\
 \hline
 16 & 39.81 & 39.71 & 39.90 & 40.38 & 68.10 & 66.33 & 69.51 & 69.42  \\
 \hline
 \hline
 \multicolumn{5}{|c|}{Winogrande (FP32 Accuracy = 67.88\%)} & \multicolumn{4}{|c|}{Piqa (FP32 Accuracy = 78.78\%)} \\ 
 \hline
 \hline
 64 & 66.85 & 66.61 & 67.72 & 67.88 & 77.31 & 77.42 & 77.75 & 77.64 \\
 \hline
 32 & 67.25 & 67.72 & 67.72 & 67.00 & 77.31 & 77.04 & 77.80 & 77.37  \\
 \hline
 16 & 68.11 & 68.90 & 67.88 & 67.48 & 77.37 & 78.13 & 78.13 & 77.69  \\
 \hline
\end{tabular}
\caption{\label{tab:mmlu_abalation} Accuracy on LM evaluation harness tasks on GPT3-8B model.}
\end{table}

\begin{table} \centering
\begin{tabular}{|c||c|c|c|c||c|c|c|c|} 
\hline
 $L_b \rightarrow$& \multicolumn{4}{c||}{8} & \multicolumn{4}{c||}{8}\\
 \hline
 \backslashbox{$L_A$\kern-1em}{\kern-1em$N_c$} & 2 & 4 & 8 & 16 & 2 & 4 & 8 & 16  \\
 %$N_c \rightarrow$ & 2 & 4 & 8 & 16 & 2 & 4 & 2 \\
 \hline
 \hline
 \multicolumn{5}{|c|}{Race (FP32 Accuracy = 40.67\%)} & \multicolumn{4}{|c|}{Boolq (FP32 Accuracy = 76.54\%)} \\ 
 \hline
 \hline
 64 & 40.48 & 40.10 & 39.43 & 39.90 & 75.41 & 75.11 & 77.09 & 75.66 \\
 \hline
 32 & 39.52 & 39.52 & 40.77 & 39.62 & 76.02 & 76.02 & 75.96 & 75.35  \\
 \hline
 16 & 39.81 & 39.71 & 39.90 & 40.38 & 75.05 & 73.82 & 75.72 & 76.09  \\
 \hline
 \hline
 \multicolumn{5}{|c|}{Winogrande (FP32 Accuracy = 70.64\%)} & \multicolumn{4}{|c|}{Piqa (FP32 Accuracy = 79.16\%)} \\ 
 \hline
 \hline
 64 & 69.14 & 70.17 & 70.17 & 70.56 & 78.24 & 79.00 & 78.62 & 78.73 \\
 \hline
 32 & 70.96 & 69.69 & 71.27 & 69.30 & 78.56 & 79.49 & 79.16 & 78.89  \\
 \hline
 16 & 71.03 & 69.53 & 69.69 & 70.40 & 78.13 & 79.16 & 79.00 & 79.00  \\
 \hline
\end{tabular}
\caption{\label{tab:mmlu_abalation} Accuracy on LM evaluation harness tasks on GPT3-22B model.}
\end{table}

\begin{table} \centering
\begin{tabular}{|c||c|c|c|c||c|c|c|c|} 
\hline
 $L_b \rightarrow$& \multicolumn{4}{c||}{8} & \multicolumn{4}{c||}{8}\\
 \hline
 \backslashbox{$L_A$\kern-1em}{\kern-1em$N_c$} & 2 & 4 & 8 & 16 & 2 & 4 & 8 & 16  \\
 %$N_c \rightarrow$ & 2 & 4 & 8 & 16 & 2 & 4 & 2 \\
 \hline
 \hline
 \multicolumn{5}{|c|}{Race (FP32 Accuracy = 44.4\%)} & \multicolumn{4}{|c|}{Boolq (FP32 Accuracy = 79.29\%)} \\ 
 \hline
 \hline
 64 & 42.49 & 42.51 & 42.58 & 43.45 & 77.58 & 77.37 & 77.43 & 78.1 \\
 \hline
 32 & 43.35 & 42.49 & 43.64 & 43.73 & 77.86 & 75.32 & 77.28 & 77.86  \\
 \hline
 16 & 44.21 & 44.21 & 43.64 & 42.97 & 78.65 & 77 & 76.94 & 77.98  \\
 \hline
 \hline
 \multicolumn{5}{|c|}{Winogrande (FP32 Accuracy = 69.38\%)} & \multicolumn{4}{|c|}{Piqa (FP32 Accuracy = 78.07\%)} \\ 
 \hline
 \hline
 64 & 68.9 & 68.43 & 69.77 & 68.19 & 77.09 & 76.82 & 77.09 & 77.86 \\
 \hline
 32 & 69.38 & 68.51 & 68.82 & 68.90 & 78.07 & 76.71 & 78.07 & 77.86  \\
 \hline
 16 & 69.53 & 67.09 & 69.38 & 68.90 & 77.37 & 77.8 & 77.91 & 77.69  \\
 \hline
\end{tabular}
\caption{\label{tab:mmlu_abalation} Accuracy on LM evaluation harness tasks on Llama2-7B model.}
\end{table}

\begin{table} \centering
\begin{tabular}{|c||c|c|c|c||c|c|c|c|} 
\hline
 $L_b \rightarrow$& \multicolumn{4}{c||}{8} & \multicolumn{4}{c||}{8}\\
 \hline
 \backslashbox{$L_A$\kern-1em}{\kern-1em$N_c$} & 2 & 4 & 8 & 16 & 2 & 4 & 8 & 16  \\
 %$N_c \rightarrow$ & 2 & 4 & 8 & 16 & 2 & 4 & 2 \\
 \hline
 \hline
 \multicolumn{5}{|c|}{Race (FP32 Accuracy = 48.8\%)} & \multicolumn{4}{|c|}{Boolq (FP32 Accuracy = 85.23\%)} \\ 
 \hline
 \hline
 64 & 49.00 & 49.00 & 49.28 & 48.71 & 82.82 & 84.28 & 84.03 & 84.25 \\
 \hline
 32 & 49.57 & 48.52 & 48.33 & 49.28 & 83.85 & 84.46 & 84.31 & 84.93  \\
 \hline
 16 & 49.85 & 49.09 & 49.28 & 48.99 & 85.11 & 84.46 & 84.61 & 83.94  \\
 \hline
 \hline
 \multicolumn{5}{|c|}{Winogrande (FP32 Accuracy = 79.95\%)} & \multicolumn{4}{|c|}{Piqa (FP32 Accuracy = 81.56\%)} \\ 
 \hline
 \hline
 64 & 78.77 & 78.45 & 78.37 & 79.16 & 81.45 & 80.69 & 81.45 & 81.5 \\
 \hline
 32 & 78.45 & 79.01 & 78.69 & 80.66 & 81.56 & 80.58 & 81.18 & 81.34  \\
 \hline
 16 & 79.95 & 79.56 & 79.79 & 79.72 & 81.28 & 81.66 & 81.28 & 80.96  \\
 \hline
\end{tabular}
\caption{\label{tab:mmlu_abalation} Accuracy on LM evaluation harness tasks on Llama2-70B model.}
\end{table}

%\section{MSE Studies}
%\textcolor{red}{TODO}


\subsection{Number Formats and Quantization Method}
\label{subsec:numFormats_quantMethod}
\subsubsection{Integer Format}
An $n$-bit signed integer (INT) is typically represented with a 2s-complement format \citep{yao2022zeroquant,xiao2023smoothquant,dai2021vsq}, where the most significant bit denotes the sign.

\subsubsection{Floating Point Format}
An $n$-bit signed floating point (FP) number $x$ comprises of a 1-bit sign ($x_{\mathrm{sign}}$), $B_m$-bit mantissa ($x_{\mathrm{mant}}$) and $B_e$-bit exponent ($x_{\mathrm{exp}}$) such that $B_m+B_e=n-1$. The associated constant exponent bias ($E_{\mathrm{bias}}$) is computed as $(2^{{B_e}-1}-1)$. We denote this format as $E_{B_e}M_{B_m}$.  

\subsubsection{Quantization Scheme}
\label{subsec:quant_method}
A quantization scheme dictates how a given unquantized tensor is converted to its quantized representation. We consider FP formats for the purpose of illustration. Given an unquantized tensor $\bm{X}$ and an FP format $E_{B_e}M_{B_m}$, we first, we compute the quantization scale factor $s_X$ that maps the maximum absolute value of $\bm{X}$ to the maximum quantization level of the $E_{B_e}M_{B_m}$ format as follows:
\begin{align}
\label{eq:sf}
    s_X = \frac{\mathrm{max}(|\bm{X}|)}{\mathrm{max}(E_{B_e}M_{B_m})}
\end{align}
In the above equation, $|\cdot|$ denotes the absolute value function.

Next, we scale $\bm{X}$ by $s_X$ and quantize it to $\hat{\bm{X}}$ by rounding it to the nearest quantization level of $E_{B_e}M_{B_m}$ as:

\begin{align}
\label{eq:tensor_quant}
    \hat{\bm{X}} = \text{round-to-nearest}\left(\frac{\bm{X}}{s_X}, E_{B_e}M_{B_m}\right)
\end{align}

We perform dynamic max-scaled quantization \citep{wu2020integer}, where the scale factor $s$ for activations is dynamically computed during runtime.

\subsection{Vector Scaled Quantization}
\begin{wrapfigure}{r}{0.35\linewidth}
  \centering
  \includegraphics[width=\linewidth]{sections/figures/vsquant.jpg}
  \caption{\small Vectorwise decomposition for per-vector scaled quantization (VSQ \citep{dai2021vsq}).}
  \label{fig:vsquant}
\end{wrapfigure}
During VSQ \citep{dai2021vsq}, the operand tensors are decomposed into 1D vectors in a hardware friendly manner as shown in Figure \ref{fig:vsquant}. Since the decomposed tensors are used as operands in matrix multiplications during inference, it is beneficial to perform this decomposition along the reduction dimension of the multiplication. The vectorwise quantization is performed similar to tensorwise quantization described in Equations \ref{eq:sf} and \ref{eq:tensor_quant}, where a scale factor $s_v$ is required for each vector $\bm{v}$ that maps the maximum absolute value of that vector to the maximum quantization level. While smaller vector lengths can lead to larger accuracy gains, the associated memory and computational overheads due to the per-vector scale factors increases. To alleviate these overheads, VSQ \citep{dai2021vsq} proposed a second level quantization of the per-vector scale factors to unsigned integers, while MX \citep{rouhani2023shared} quantizes them to integer powers of 2 (denoted as $2^{INT}$).

\subsubsection{MX Format}
The MX format proposed in \citep{rouhani2023microscaling} introduces the concept of sub-block shifting. For every two scalar elements of $b$-bits each, there is a shared exponent bit. The value of this exponent bit is determined through an empirical analysis that targets minimizing quantization MSE. We note that the FP format $E_{1}M_{b}$ is strictly better than MX from an accuracy perspective since it allocates a dedicated exponent bit to each scalar as opposed to sharing it across two scalars. Therefore, we conservatively bound the accuracy of a $b+2$-bit signed MX format with that of a $E_{1}M_{b}$ format in our comparisons. For instance, we use E1M2 format as a proxy for MX4.

\begin{figure}
    \centering
    \includegraphics[width=1\linewidth]{sections//figures/BlockFormats.pdf}
    \caption{\small Comparing LO-BCQ to MX format.}
    \label{fig:block_formats}
\end{figure}

Figure \ref{fig:block_formats} compares our $4$-bit LO-BCQ block format to MX \citep{rouhani2023microscaling}. As shown, both LO-BCQ and MX decompose a given operand tensor into block arrays and each block array into blocks. Similar to MX, we find that per-block quantization ($L_b < L_A$) leads to better accuracy due to increased flexibility. While MX achieves this through per-block $1$-bit micro-scales, we associate a dedicated codebook to each block through a per-block codebook selector. Further, MX quantizes the per-block array scale-factor to E8M0 format without per-tensor scaling. In contrast during LO-BCQ, we find that per-tensor scaling combined with quantization of per-block array scale-factor to E4M3 format results in superior inference accuracy across models. 



\end{document}
%%% Local Variables:
%%% mode: LaTeX
%%% TeX-master: t
%%% End:
