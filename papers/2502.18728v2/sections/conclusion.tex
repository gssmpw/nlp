We presented the BBIR, a new intermediate representation for optimization problems
over discrete probabilistic inference. The BBIR can represent important optimization
problems such as MEU and MMAP with additional features such as staged compilation,
conditioning, and reasoning beyond probabilities. The flexibility of the BBIR
was showcased through two very different programming languages: $\dappl$,
a function decision-theoretic PPL with support for Bayesian conditioning,
and $\pineappl$, an imperative PPL with first-class meta-optimization support
via MMAP.

Our efforts in this paper focused on designing a new scalable intermediate
representation to support a broad class of optimization problem; hence, we
simplified the design of our surface-level languages to simplify this
compilation.  In the future we aim to provide more expressive surface-level
languages that compile to BBIR. The most tractable would be to add support
for language features like top-level functions and dynamically-bounded
surely-terminating loops; languages like \dice{} and \texttt{ProbLog} support
these features.
Next, it would also be interesting to explore adding features to support applications in game theory,
such as multiple decision-making agents or stochastic policies,
as our current framework is limited to one decision-making agent and deterministic policies.
Finally, it would be interesting to explore
the extent to which we can provide more ergonomic and unified surface languages
for efficiently programming with decision-making, for instance by developing
efficient implementations of selection monad~\citep{abadi2021smart,lago2022reinforcement}.



% The first is to extend $\dappl$ and $\pineappl$ to not complement each other,
% but have the same features: that is, allow meta-optimization in $\dappl$
% and allow conditioning via $\dapplcode{observe}$s in $\pineappl$.
% Another direction along this vein is to combine
% multiple optimization problems
% into one language that compiles into one BBIR.



% We presented \dappl{}, a functional decision-theoretic probabilistic programming language
% with support for Bayesian reasoning. The core contribution of \dappl{} is its new
% approach to scalable expected utility computation based on knowledge compilation.
% As future work, we see several future directions. One direction, as hinted in Section~\ref{sec:eval},
% is a hybrid MEU solving strategy via a synthesis of branch-and-bound and
% and order-constrained approaches over BBIR. Another direction would be to adapt the language
% to support MEU with multiple decision makers, which would allow us to
% model game-theoretic scenarios~\citep{gan2022bayesian,osborne2004introduction},
% or to support addition problems such as MMAP along with MEU.
% Lastly, we would like to see a compilation scheme to BBIR that is language-agnostic,
% allowing for easy development of probabilistic programming languages that support
% the optimization-via-compilation scheme.