\subsection{$\pineappl$ Ergonomics and Syntactic Sugar}
\label{appendix:pineappl-sugar}

We endow the core of $\pineappl$ with a few pieces of syntactic sugar to aid in
modeling.

\subsubsection{Support for discrete distributions} 
Similar to $\dappl$, we extend $\pineappl$ with support for discrete
distributions, rather than a simple flip, a program can sample from a discrete
set of events, with either a uniform prior, or custom priors on each event
(provided that those priors sum to 1). This is accomplished via a one-hot
encoding, similar to \citep{holtzen2020scaling}. This also introduces a
predicate $is$ which tests for the presence of a categorical-variable.  

\subsubsection{Support for multiple queries} 
As a result of $\pineappl$ compiling the statements of a program to a boolean
formula, it is trivial to extend the program with the ability to make multiple
queries over the same set of statements. Full $\pineappl$ programs can end with 
any number of query expressions, and the results are returned as a list. 

\subsubsection{Support for MMAP as a terminal query}
In addition to using MMAP as a first-class primitive, it can also be useful to 
obtain the map state of some variables at the end of the program. This can easily
done using BBIR directly at the end of the program.

\subsubsection{Support for bounded loops}
We implement bounded loops in $\pineappl$ as a hygenic macro expansion of the
code inside the loop. Loops relax $\pineappl$'s demand for entirely fresh names
at the source-level; after expansion, the compiler will enforce the freshness
constraint with hygiene, potential introduction of join points for loops that
occurr within the branches of an if-statement, and a global renaming pass to
ensure that all names bound in the loop are referenced appropriately in later
code. \cref{fig:loop-pineappl-sugar} is a simple $\pineappl$ program that uses
a loop and \cref{fig:loop-pineappl-desugar} shows its expansion. Note, that
\pineapplcode{pr(a)} is rewritten to \pineapplcode{pr(a2)} in the renaming pass
to ensure that the query refers to the ``latest'' value of \pineapplcode{a}.
\cref{fig:loop-pineappl-ite-sugar} is a $\pineappl$ program containing an
if-statement where each branch has a loop. Since the loop expansion binds fresh
names, variables bound in the loop must be explicitly joined at the end of the
if-statement, and then global renaming pass utilizes the joined name for
subsequent uses of the variable.  Clearly, loops expand to syntactically valid
$\pineappl$ programs. Determining the last binding introduced by a loop for
join-points and rewriting can be done as a lightweight analysis at expansion time.

\begin{figure}
  \begin{subfigure}[t]{0.45\textwidth}
    \begin{pineapplcodeblock}
      a = flip 0.5;
      loop 3 {
        tmp = flip 0.1;
        a = a || tmp; 
      }
      pr(a)
    \end{pineapplcodeblock}
    \caption{A simple $\pineappl$ program with a loop}
    \label{fig:loop-pineappl-sugar}
  \end{subfigure}
  \begin{subfigure}[t]{0.45\textwidth}
    \begin{pineapplcodeblock}
      a = flip 0.5;
      tmp0 = flip 0.1;
      a0 = a || tmp0; 
      tmp1 = flip 0.1;
      a1 = a0 || tmp1;
      tmp2 = flip 0.1;
      a2 = a1 || tmp2;
      pr(a2)
    \end{pineapplcodeblock}
    \caption{An expansion and renaming of the program from (a).}
    \label{fig:loop-pineappl-desugar}
  \end{subfigure}
  \begin{subfigure}[t]{0.45\textwidth}
    \begin{pineapplcodeblock}
      x = flip 0.5;
      y = flip 0.5;
      if x {
        loop 2 {
          tmp = flip 0.3;
          y = y && tmp;
        }
      }
      else {
        loop 3 {
          tmp = flip 0.7;
          y = y || tmp; 
        }
      }
      pr(y)
    \end{pineapplcodeblock}
    \caption{A $\pineappl$ program with loops in both branches of an if statement.}
    \label{fig:loop-pineappl-ite-sugar}
  \end{subfigure}
  \begin{subfigure}[t]{0.45\textwidth}
    \begin{pineapplcodeblock}
      x = flip 0.5;
      y = flip 0.5;
      if x {
        tmp0 = flip 0.2;
        y0 = y && tmp0;
        tmp1 = flip 0.2;
        y1 = y0 && tmp1;
      }
      else {
        tmp2 = flip 0.7;
        y2 = y || tmp2;
        tmp3 = flip 0.7;
        y3 = y2 || tmp3;
        tmp4 = flip 0.7;
        y4 = y3 || tmp4;
      }
      tmp_j = (x && tmp1) || (!x && tmp4);
      y_j = (x && y1) || (!x && y4);
      pr(y_j)
    \end{pineapplcodeblock}
    \caption{An expansion, introduction of join-points, and renaming of the program
    from (c).}
    \label{fig:loop-pineappl-ite-desugar}
\end{subfigure}
\caption{Examples of loop expansion in $\pineappl$}
\label{fig:loop-pineappl}
\end{figure}


%%% Local Variables:
%%% mode: LaTeX
%%% TeX-master: "../../oopsla-appendix.tex"
%%% End:
