\subsection{The $\leadsto_E$ relation for $\pineappl{}$}
\label{appendix:pineappl bc}

See~\cref{fig:pineappl-e}.

\begin{figure}
\begin{mdframed}
  {\footnotesize
  \begin{align*}
    \infer[\texttt{e/x}]
    {
      \texttt{x} \leadsto_E x
    }
    {\mathrm{fresh} \; x}
    \qquad
    \infer[\texttt{e/tt}]
    {
      \texttt{tt} \leadsto_E \top
    }
    {}
    \qquad
    \infer[\texttt{e/ff}]
    {
      \texttt{ff} \leadsto_E \bot
    }
    {}
  \end{align*}
  \begin{align*}
    \infer[\texttt{e/}\land]
    {
      \texttt{e1} \land \texttt{e2}
    \leadsto_E e_1 \land e_2
    }
    { \texttt{e1} \leadsto_E e_1 & \texttt{e2} \leadsto_E e_2}
    \qquad
    \infer[\texttt{e/}\lor]
    {
      \texttt{e1} \lor \texttt{e2} \leadsto_E e_1 \lor e_2
    }
    { \texttt{e1} \leadsto_E e_1 & \texttt{e2} \leadsto_E e_2}
    \qquad
    \infer[\texttt{e/}\neg]
    {
      \neg \texttt{e} \leadsto_E \neg e
    }
    { \texttt{e} \leadsto_E e}
  \end{align*}
  }
\end{mdframed}
\caption{The $\leadsto_E$ relation for $\pineappl{}$ expressions.}
\label{fig:pineappl-e}
\end{figure}

\subsection{Full Boolean compilation rules for $\pineappl{}$}
\label{appendix:pineappl-compl-complete}

See~\cref{fig:pineappl-compl-complete}.

\begin{figure}
\begin{mdframed}
  {\footnotesize
  \begin{align*}
    \infer[\texttt{bc/flip}]
    {
        ((\texttt{x = flip} \ \theta, \mathcal{F}, w) \leadsto
        (\{(x, f)\} \cup \mathcal{F},
        w \cup \{x \mapsto (1, 1), f \mapsto (\theta, 1-\theta)\}))
    }
    {\mathrm{fresh} \; f}
  \end{align*}
  \begin{align*}
    \infer[\texttt{bc/assn}]
    {
      (\texttt{x = e}, \mathcal{F}, w) \leadsto
      (\{(x,  \varphi)\} \cup \mathcal{F}, w \cup \{x \mapsto (1, 1)\})
    }
    {
      \texttt e \leadsto_E \varphi
    }
  \end{align*}
  \begin{align*}
    \infer[\texttt{bc/seq}]
    {
        (\texttt{$s_1$; $s_2$}, w) \leadsto (\mathcal{F}'', w'')
    }
    {
        (s_1, \mathcal{F}, w) \leadsto (\mathcal{F}', w')
        & (s_2, \mathcal{F}', w') \leadsto (\mathcal{F}'', w'')
    }
  \end{align*}
  \begin{align*}
    \infer[\texttt{bc/if}]
    {
      (\texttt{if e \{$s_1$\} else \{$s_2$\}}, \mathcal{F}, w) \leadsto
    (\{(x_i, (\chi \land \varphi_i \lor \neg\chi \land \psi_i))\} \cup \mathcal{F}, w_1 \cup w_2)
    % could include extra parens around and clauses, but and binds tighter than or
    }
    {
        \texttt e \leadsto_E \chi
        & (s_1, \mathcal{F}, w) \leadsto (\{(x_i, \varphi_i)\} \cup \mathcal{F}, w_1)
        & (s_2, \mathcal{F}, w) \leadsto (\{(x_i, \psi_i)\} \cup \mathcal{F}, w_2)
    }
  \end{align*}
  \begin{align*}
    \infer[\texttt{bc/mmap}]
    {
      (\vec{\texttt{m}}\texttt{ = mmap }\vec{\texttt{x}}, \mathcal{F}, w) \leadsto
      (\mathcal{F} \cup \{(m_i, k_i) \}, w \cup w_{M})
    }
    {
      \mathrm{fresh} \; k_i
      & \vec A = MMAP(\{\bigwedge_{(x, \varphi) \in \mathcal{F}} x \leftrightarrow \varphi, \eset\}, \vec{\texttt{x}}, w)
      &
      w_{M} = \{m_i \mapsto (1,1), k_i \mapsto A_i \}
    }
  \end{align*}
  \begin{align*}
    \infer[\texttt{bc/mmap/with}]
    {
      (\vec{\texttt{m}}\texttt{ = mmap }\vec{\texttt{x}} \ \pineapplcode{ with \{e\}}, \mathcal{F}, w) \leadsto
      (\mathcal{F} \cup \{(m_i, k_i) \}, w \cup w_{M})
    }
    {
      \mathrm{fresh} \; k_i
      & \texttt e \leadsto_E \psi
      & \vec A = MMAP(\{\bigwedge_{(x, \varphi) \in \mathcal{F}} x \leftrightarrow \varphi, \psi\}, \vec{\texttt{x}}, w)
      &
      w_{M} = \{m_i \mapsto (1,1), k_i \mapsto A_i \}
    }
  \end{align*}
  \begin{align*}
    \infer[\texttt{bc/pr}]
    {
      \texttt{s; Pr(e)} \leadsto_P (\varphi \land \left( \bigwedge_{(x, \varphi) \in \mathcal{F}} x \leftrightarrow \varphi \right), \top,w)
    }
    {
      (\texttt{s}, \eset, \eset) \leadsto (\mathcal F, w)
      &
      \texttt{e} \leadsto_E \varphi
    }
  \end{align*}
  \begin{align*}
    \infer[\texttt{bc/pr/with}]
    {
      \pineapplcode{s; Pr(e1) with \{e2\}} \leadsto_P (\varphi \land \left(\bigwedge_{(x, \varphi) \in \mathcal{F}} x \leftrightarrow \varphi \right), \psi, w)
    }
    {
      (s, \eset, \eset) \leadsto (\mathcal F, w)
      &
      e \leadsto_E \varphi
    }
  \end{align*}
  }
\end{mdframed}
\caption{Full Boolean compilation rules for $\pineappl{}$ statements and programs.}
\label{fig:pineappl-compl-complete}
\end{figure}

\subsection{Proof of Theorem~\ref{thm:pineappl correctness}}
\label{appendix:proof-pineappl-correctness}

The proof is by way of simulation.

\begin{definition}[$\sim$]
Let $\mathcal D$ be a distribution over assignments to variables,
$\mathcal F$ a set of formulae of shape $x \leftrightarrow \varphi$,
and $w$ a weight function
of literals in $\mathcal F$ to the reals.
Let the variables in $\mathcal F$ be a superset of those in $\mathcal D$.
Then we define $D \sim (F,w)$ if and only if
for all $\sigma \in \dom(D)$,
\begin{equation}
  D(\sigma) = \left[ \prod_{\ell \in \sigma} w(\ell) \right]
    \AMC_{\R} \paren{
      \bigwedge_{(x, \varphi) \in \mathcal F} (x \leftrightarrow \varphi)|_{\sigma} , w
    }
  .
\end{equation}
\end{definition}

On this relation we can define a helpful Lemma:

\begin{lemma}\label{lemma:pineappl simulate expressions}
  Let $D \sim (F,w)$.
  Let $e$ be a Boolean expression in $\pineappl$ on which $\Pr_{\mathcal D} (e)$ is well-defined.
  Let $e \leadsto_E \varphi$. Then the following holds:
  \begin{equation}
    \Pr_{\mathcal D} (e) = \AMC\left(\varphi \land \paren{\bigwedge_{(x, \varphi) \in \mathcal F} x \leftrightarrow \varphi}, w\right).
  \end{equation}
\end{lemma}

\begin{proof}
  The proof is a straightforward induction on the syntax of $e$.
\end{proof}

Now, we have the necessary machinery to prove the theorem.
Let $s;q$ be a $\pineappl{}$ program. We first prove the following.

\begin{theorem}\label{thm:pineappl statements invariant}
  Let $D, \mathcal F, w$ such that $D \sim (\mathcal F, w)$.
  Let $(s, D) \Downarrow D'$ and $(s, \mathcal F, w) \leadsto (\mathcal F', w')$.
  Then $D' \sim (\mathcal F', w')$.
\end{theorem}

\begin{proof}
  We induct on syntax.
  \begin{itemize}[leftmargin=*]
    \item If $s = \pineapplcode{x = flip }\ \theta$, then observe that $\mathcal F' =\mathcal F \cup \{x \leftrightarrow f_\theta\}$.
    For any trace where $x \mapsto \top$, we get
    \begin{align}
      D'(\sigma \cup \{x \mapsto \top\})
        &= \theta \times D(\sigma)\\
        &= w(f_{\theta}) \times
        \left[ \prod_{\ell \in \sigma} w(\ell) \right]
        \AMC_{\R} \paren{
          \bigwedge_{(x, \varphi) \in \mathcal F} (x \leftrightarrow \varphi)|_{\sigma} , w
        }
        \\
        &= \AMC_{\R}((x \leftrightarrow f_{\theta})\mid_{x = \top}, w) \times
        \left[ \prod_{\ell \in \sigma \cup \{x \mapsto \top\} } w(\ell) \right]
        \AMC_{\R} \paren{
          \bigwedge_{(x, \varphi) \in \mathcal F} (x \leftrightarrow \varphi)|_{\sigma} , w
        }
        \\
        &=
        \left[ \prod_{\ell \in \sigma \cup \{x \mapsto \top\} } w(\ell) \right]
        \AMC_{\R} \paren{
          (x \leftrightarrow f_{\theta}) \land \bigwedge_{(x, \varphi) \in \mathcal F} (x \leftrightarrow \varphi)|_{\sigma \cup \{x \mapsto \top\}} , w
        }
    \end{align}
    where (49) is the inductive hypothesis and (51) used~\cref{lemma:ind conj prob}.
    The case when $x \mapsto \bot$ is symmetrical.

    \item If $s = \pineapplcode{x = e}$, let $e \leadsto_E \chi$.
    Then observe that $\mathcal F' =\mathcal F \cup \{x \leftrightarrow \varphi\}$.
    Then for any trace where $e[\sigma] = \top$, we get
    \begin{align}
      D'(\sigma \cup \{x \mapsto \top\})
        &= \Pr_{\mathcal D}[e] \times D(\sigma)\\
        &= \AMC(\chi \land \bigwedge_{(x, \varphi) \in \mathcal F} (x \leftrightarrow \varphi), w) \nonumber\\
        &\quad \times
        \left[ \prod_{\ell \in \sigma} w(\ell) \right]
        \AMC_{\R} \paren{
          \bigwedge_{(x, \varphi) \in \mathcal F} (x \leftrightarrow \varphi)|_{\sigma} , w
        }
        \\
        &= \AMC_{\R}((y \leftrightarrow \chi)\mid_{y = \top} \land \bigwedge_{(x, \varphi) \in \mathcal F} (x \leftrightarrow \varphi), w) \nonumber\\
        &\quad \times
        \left[ \prod_{\ell \in \sigma \cup \{x \mapsto \top\} } w(\ell) \right]
        \AMC_{\R} \paren{
          \bigwedge_{(x, \varphi) \in \mathcal F} (x \leftrightarrow \varphi)|_{\sigma} , w
        }
        \\
        &=
        \left[ \prod_{\ell \in \sigma \cup \{x \mapsto \top\} } w(\ell) \right]
        \AMC_{\R} \paren{
         \left[(x \leftrightarrow \varphi)| \land \bigwedge_{(x, \varphi) \in \mathcal F} (x \leftrightarrow \varphi)\right]\mid_{\sigma \cup \{x \mapsto \top\}} , w
        }
    \end{align}
    where (53) follows from~\cref{lemma:pineappl simulate expressions}.
    (55) is valid because $\varphi$ consists exclusively of variables occuring in $\sigma$,
    so $\bigwedge_{(x, \varphi) \in \mathcal F} (x \leftrightarrow \varphi)|_{\sigma}$ has no variables in common.
    Furthermore the restriction of $\bigwedge_{(x, \varphi) \in \mathcal F} (x \leftrightarrow \varphi)$ to only those
    that satisfy $\sigma \cup \{x \cup \top\}$ eliminates the larger.x
    The case when $x\mapsto \bot$ is symmetrical.
    \item If $s = \pineapplcode{s1 ; s2}$, the proof is straightforward and is omitted.
    \item If $s = \pineapplcode{if e \{s1\} else  \{s2\}}$, let $e \leadsto_E \varphi$, then let
    \begin{itemize}
      \item $e \leadsto_E \chi$,
      \item $(s_1, \mathcal D) \Downarrow D_1$,
      \item $(s_2, \mathcal D) \Downarrow D_2$,
      \item $(s_1, \mathcal F, w) \leadsto (\mathcal F \cup \{x_i \leftrightarrow \varphi_i\}, w_1)$, and
      \item $(s_2, \mathcal F, w) \leadsto (\mathcal F \cup \{x_i \leftrightarrow \psi_i\}, w_2)$.
    \end{itemize}
    Without loss of generality assume that $D_1$ and $D_2$ are over the same domain.
    This is possible because if there exists some variable $v$ such that $v \in \sigma$ in which $D_1(\sigma)$ is defined but $D_2(\sigma)$ is not,
    then we can extend all $\tau \in \dom(D_2)$ with $v$, and vice versa. Similarly without loss of generality assume that the $w_1$ and $w_2$ have the same domain.

    Let $I = \bigwedge_{(x, \varphi) \in \mathcal F} (x \leftrightarrow \varphi)$.
    Then we can deduce, for some $\sigma$,
    \begin{align}
      \mathcal D' (\sigma)
        &= p \times \mathcal D_1(\sigma) + (1-p) \times \mathcal D_2(\sigma)
        \\
        &= \AMC(\chi \land I, w) \times \mathcal D_1(\sigma) + \AMC(\overline \chi \land I, w) \times \mathcal D_2(\sigma)
        \\
        &= \AMC(\chi \land I, w) \times
          \left[ \prod_{\ell \in \sigma } w(\ell) \right]
            \AMC_{\R} \paren{
              \{x_i \leftrightarrow \varphi_i\} \mid_{\sigma} \land I|_{\sigma} , w
          }
        \\
        &\quad+ \AMC(\overline \chi \land I, w) \times
          \left[ \prod_{\ell \in \sigma} w(\ell) \right]
            \AMC_{\R} \paren{
              \{x_i \leftrightarrow \varphi_i\} \mid_{\sigma} \land I|_{\sigma} , w
          }
        \\
        &= \left[ \prod_{\ell \in \sigma} w(\ell) \right] \nonumber \\
        &\quad \times
            \AMC_{\R} \paren{
              \chi \land I \land \paren{\{x_i \leftrightarrow \varphi_i\} \mid_{\sigma} \land I|_{\sigma}}
              \lor
              \overline \chi \land I \land \paren{\{x_i \leftrightarrow \varphi_i\} \mid_{\sigma} \land I|_{\sigma}} , w
            }.
    \end{align}
    Consider the formula within the AMC in (60). We observe that
    \begin{align}
      &\chi \land I \land \paren{\{x_i \leftrightarrow \varphi_i\} \mid_{\sigma} \land I|_{\sigma}}
        \lor
        \overline \chi \land I \land \paren{\{x_i \leftrightarrow \varphi_i\} \mid_{\sigma} \land I|_{\sigma}}\\
      &\quad =
      \bigwedge_{(x, \varphi) \in \mathcal F} (x \leftrightarrow \varphi)|_{\sigma} \land \paren{\chi \land \paren{\{x_i \leftrightarrow \varphi_i\} \mid_{\sigma}}
        \lor
        \overline \chi \land \paren{\{x_i \leftrightarrow \varphi_i\} \mid_{\sigma}}}\\
      &\quad =
      \bigwedge_{(x, \varphi) \in \mathcal F} (x \leftrightarrow \varphi)|_{\sigma} \land
        \paren{\paren{\{x_i \leftrightarrow \chi \land \varphi_i\} \mid_{\sigma}}
        \lor
        \paren{\{x_i \leftrightarrow \overline \chi \land  \varphi_i\} \mid_{\sigma}}}\\
      &\quad =
      \bigwedge_{(x, \varphi) \in \mathcal F} (x \leftrightarrow \varphi)|_{\sigma} \land
        \paren{\paren{\{x_i \leftrightarrow \chi \land \varphi_i\lor \overline \chi \land  \varphi_i \} \mid_{\sigma}}}
    \end{align}
    as desired by repeated usage of~\cref{lemma:inc exc prob,lemma:ind conj prob}.
    \item If $s = \vec{\texttt{m}}\texttt{ = mmap }\vec{\texttt{x}}$,
    we defer the proof to the next case, with the specialization that $e = \tt$.
    \item If $s = \vec{\texttt{m}}\texttt{ = mmap }\vec{\texttt{x}} \ \pineapplcode{ with \{e\}}$,
    then it suffices to show that, for $e \leadsto_E \psi$,
    \begin{equation*}
      MMAP_{\mathcal D} (\vec x \mid e) = MMAP(\{\bigwedge_{\mathcal F} x_i \leftrightarrow \varphi_i, \psi\}, \vec x, w).
    \end{equation*}
    We observe that
    \begin{align}
      MMAP_{\mathcal D} (\vec x \mid e)
      & = \argmax_{\sigma \in inst(\vec x)} \mathcal D (\sigma  \mid e)\\
      & = \argmax_{\sigma \in inst(\vec x)} \frac{\mathcal D (\sigma \land e)}{\Pr_{\mathcal D}[e]}
      \\
      & = \argmax_{\sigma \in inst(\vec x)} \frac{\AMC(\bigwedge \mathcal F |_\sigma \land \psi, w)}{\AMC_{\mathbb R}(\psi, w)}\\
      & = MMAP(\{\bigwedge_{\mathcal F} x_i \leftrightarrow \varphi_i, \psi\},\vec x, w)
    \end{align}
    as desired.
  \end{itemize}
\end{proof}

Now, we can finally prove~\cref{thm:pineappl correctness}.

\begin{proof}[Proof of~\cref{thm:pineappl correctness}]
  Let $s;q$ be a $\pineappl{}$ program.
  Let $(s, \eset) \Downarrow \mathcal D$ and $(s, \eset,\eset) \leadsto (\mathcal F, w)$.
  By~\cref{thm:pineappl statements invariant} we know that $D \sim (\mathcal F, w)$.
  It suffices to prove correctness for $q = \pineapplcode{Pr(e1) with \{e2\}}$ as the other case is identical with $\texttt{e2}  = \tt$.
  We observe that, as an application of~\cref{lemma:pineappl simulate expressions}
  \begin{align}
    \frac{\Pr_{\mathcal D}[e_1 \land e_2]}{\Pr_{\mathcal D}[e_2]}
    &=\frac{\AMC_{\mathbb R}(\varphi \land (\land_{(x, \varphi) \in \mathcal F} x \leftrightarrow \varphi) \land \psi,w)}
      {\AMC_{\mathbb R}(\psi \land (\land_{(x, \varphi) \in \mathcal F} x \leftrightarrow \varphi),w)}
  \end{align}
  which completes the proof.
\end{proof}