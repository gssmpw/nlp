\subsection{The $\AMC{}$ invariant}\label{appendix:amc invariant}

We can define the expected utility of a Boolean formula as an expectation:

\begin{definition}[Expected Utility of a Boolean Formula]\label{def:eu-of-formula}
  Let $\varphi$ be a Boolean formula that consists of reward variables
  $\mathcal{R} = \{R_i\}$ and probabilistic variables $\mathcal{P} = \{P_j\}$; 
  we will denote literals -- assignments to Boolean variables -- using lower-case letters.
  Assume we are given a \emph{utility map} $U$ that maps
  reward literals to a real-valued reward, and \emph{probability map} $\Pr$ that maps
  probabilistic literals to probabilities.
  Then, the \emph{probability of model} $\{r_i,p_j\}$
  is the product of probabilities of each probabilistic variable:
  % Let $\Omega$ be the set of all assignments to variables in $\varphi$. The 
  % distribution on $\Omega$ 
  % induced by $\Pr$ is defined by, for each $m \in \Omega$:
  % \begin{align}
    $\Pr(\{r_i, p_j\}) \triangleq \prod_{j} \Pr(p_j).$
  % \end{align}
  The expected utility of $\varphi$ is the probability-weighted sum of
  utilities that satisfy the formula:
  \begin{align}
    \EU[\varphi] \triangleq \sum_{\{r_i, p_j\} \models \varphi} \Pr(\{r_i, p_j\}) \paren{\sum_{i} U(r_i)}.
  \end{align}
\end{definition}

The relation is as follows:

\begin{theorem}\label{thm:amc invariant}
  Let $\varphi$ be a Boolean formula consisting of probabilistic variables $P_i$
  and reward variables $R_i$, with utility map $U$ and probability map $\Pr$.
  Let $w: \lits(\varphi) \rightarrow \mathcal{S}$ be a weight function that maps
  literals to elements of the expectation semiring, where for probabilistic 
  variables we assign $w(P_i) = (\Pr(P_i), 0), w(\overline{P_i})$ and
  $w(\overline{P_i}) = (\Pr(\overline{P_i}), 0)$, and for reward variables we
  assign $w(R_i) = (1, U(R_i))$ and $w(\overline{R_i}) = (1, 0)$. Then
  $\AMC(\varphi, w)_{\EU} =  \EU[\varphi]$.
\end{theorem}

The proof relies on the following lemmata, 
whose proofs are straightforward inductions:

\begin{lemma}\label{lemma:distrib.eu}
  Let $\{(p_i, v_i)\}\subseteq \mathcal S$. Then
  \begin{equation*}
    \left[\bigotimes_i (p_i, v_i)\right]_{\EU} = \sum_{i} v_i\paren{\prod_{j \neq i} p_j}.
  \end{equation*} 
\end{lemma}

\begin{lemma}\label{lemma:pull.eu}
  Let $\{(p_i, u_i)\}\subseteq \mathcal S$. Then
  \begin{equation*}
    \left[\bigoplus_i (p_i, u_i)\right]_{\EU} = \sum_{i} (p_i, u_i)_{\EU}.
  \end{equation*} 
\end{lemma}

Unfolding, we see that
  \begin{align*}
    \EU[(\varphi, w)]
      &=\sum_{m \models \varphi} \prod_{\ell \in m} w(\ell)_{\Pr} \paren{\sum_{\ell \in m} \frac{w(\ell)_{\EU}}{w(\ell)_{\Pr}}} = \sum_{m \models \varphi}  \sum_{\ell \in m}\paren{ \frac{w(\ell)_{\EU}}{w(\ell)_{\Pr}}\prod_{j \in m} w(j)_{\Pr}}\\
      &=\sum_{m \models \varphi} \sum_{\ell \in m} \paren{ w(\ell)_{\EU}\paren{\prod_{j \neq \ell} w(j)_{\Pr}}}\\
      &=\sum_{m \models \varphi} \left[\bigotimes_{\ell \in m} w(\ell)\right]_{\EU}&(\star)\\&=\left[\bigoplus_{m \models \varphi} \bigotimes_{\ell \in m} w(\ell)\right]_{\EU} = [\AMC_{\mathcal S}(\varphi, w)]_{\EU}&(\dagger)
  \end{align*}
  where $(\star)$ is the usage of Lemma \ref{lemma:distrib.eu} and $(\dagger)$ denotes usage of Lemma \ref{lemma:pull.eu}.