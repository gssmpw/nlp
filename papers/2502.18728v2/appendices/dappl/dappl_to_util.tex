\subsection{Soundness of reduction from \dappl{} to \util{}}
\label{appendix:reduction soundness}

The transformation of \dappl{} to \util{} programs are given as
equational rules. To set up, let $\Gamma \proves e : \tau$ a \dappl{} 
expression. Let $\mathcal A$ be the policy space of $e$. Additionally we can 
consider policies on the context $\Gamma$, which we precisely define below.

\begin{definition}\label{def:context policy space}
  Let $\Gamma \proves e : \tau$ a \dappl{} expression. 
  For any $x_i : \Choice{\alpha_i} \in \Gamma$, 
  call $\{\alpha_i\}$ the \emph{choice} of $x_i$. 
  The product of all choices of $x \in \Gamma$ is called 
  the \emph{context policy space}, written $\mathcal{A}_{\Gamma}$.
\end{definition}

This will prove useful when we are attempting to reduce expression of form
$\choose x {\alpha_i \implies e_i}$ to \util{}.
Let $\pi \in \mathcal A$ and let $\pi_{\Gamma} \in \mathcal{A}_{\Gamma}$. 
Then we can consider the joint policy $\pi \cup \pi_{\Gamma}$ on which to
reduce $e$ with.
Selected rules are given in Figure~\ref{fig:dappl to util}.
Omitted rules follow the standard recursive application of $|_{\pi \cup \pi_{\Gamma}}$
to subexpressions.

\begin{figure}[H]
  \begin{mdframed}
    \begin{align*}
      [\alpha_1, \cdots, \alpha_n]|_{\pi \cup \pi_{\Gamma}} = \return \tt
    \end{align*}
    \begin{align*}
      {(\choose x {\alpha_i \implies e_i})|_{\pi \cup \pi_{\Gamma}} = e_i|_{\pi \cup \pi_{\Gamma}}}
      \text{ for $i$ s.t. $\alpha_i = \mathrm{proj}_k \pi \cup \pi_{\Gamma}$ for some $k$}
    \end{align*}
  \end{mdframed}
  \caption{Selected reduction rules from \dappl~to \util.}
  \label{fig:dappl to util}
\end{figure}

At this point it is important to prove 
our reduction sound, which we will do so.

\begin{lemma}\label{lemma:dappl to util}
  Let $\Gamma \proves e : \tau$ be a well-typed \dappl~expression, 
  and let $\pi \in \mathcal A$ and $\pi_{\Gamma} \in \mathcal{A}_{\Gamma}.$ 
  Then $e|_{\pi \cup \pi_{\Gamma}}$ is a well-typed \util~expression, and 
  in particular it is well-typed with respect to the context $\Gamma$ with 
  all instances of variables with type $\Choice S$ for some $S$ removed.
\end{lemma}

\begin{proof}
  We do this by induction on the structure of the expression $e$. 
  Most cases are omitted as they are straightforward; we show the cases for the 
  rules of~\cref{fig:dappl to util}.
  \begin{itemize}
    \item If $e = [\alpha_1, \cdots, \alpha_n]$, clearly $\return \tt$ is a valid
    \util{} expression. 
    Furthermore it is well-typed in any context, concluding the case.
    \item If $e = \choose x {\alpha_i \implies e_i}$, then we first need to prove 
    that there exists $i$ such that 
    $\alpha_i = \mathrm{proj}_k \pi$ for some $k$.
    We show that the set of names $\{\alpha_i, \cdots, \alpha_n\}$ 
    that we are matching on is a factor of the policy space 
    $\mathcal A$, as the result follows since $\mathcal A$ is a 
    finite product and $\{\alpha_i, \cdots, \alpha_n\}$ is a finite set.
    Indeed, this is enforced by the type of the subexpression $x$
    as seen in Figure~\ref{fig:dappl typing}. At this point, the 
    IH shows that $e_i|_{\pi \cup \pi_{\Gamma}}$ can be well-typed 
    with respect to the context $\Gamma$ with 
    all instances of variables with type $\Choice S$ for some $S$ removed,
    concluding the proof.
  \end{itemize}
  
  
\end{proof}
