\subsection{Denotational semantics of $\util$}\label{appendix:util semantics}

We specify a Lemma:

\begin{lemma}\label{lemma:util context}
  Let $\Gamma \proves e : \tau$ be a $\util$ expression via the typing rules 
  of~\cref{fig:dappl typing}. Then $\Gamma$ can only be a list of variables of
  type $\Bool$.
\end{lemma}

\begin{proof}
  Proof is by induction on the typing rules of util.
\end{proof}

We define the a distribution $\mathcal D((\Bool \times \R) \cup \{\bot\})$
as a function $(\Bool \times \R) + \{\bot\} \to \R$, although we use the notation
$\{v_1 \mapsto p_1, \cdots, v_n \mapsto p_n\}$ 
for explicit values $v_i \in (\Bool \times \R) + \{\bot\}$ mapping to probabilities $p_i$
when it is more convenient, with 
the implicit assumption that any value not present has probability zero.


We use the shorthands $\mathbf{TT} = \{(\tt,0) \mapsto 1\}$, $\mathbf{FF} = \{(\ff,0) \mapsto 1\}$, and
$\pmb{\bot} = \{\bot \mapsto 1\}$.

By~\cref{lemma:util context}, we can say that the denotation for $\Gamma$, $\denote{\Gamma}$, 
are maps from free variables of $e$ to either $\mathbf{TT}$ or $\mathbf{FF}$.
Thus expressions $\Gamma \proves e : \Giry \Bool$ can be denoted as
functions $\denote{e} : \denote{\Gamma} \to \mathcal D((\Bool \times \R) \cup \{\bot\})$

The symbol \fishbone is the monadic bind operation for probability distributions with
finite support. The interpretation of logical operations over pure expressions are 
defined to be the operation lifted to probability distributions.

\begin{align*}
  \denote{x} &= \lambda g . \ g(x)  \\
  \denote{\tt} &= \lambda g . \ \mathbf{TT} \\
  \denote{\ff} &= \lambda g . \ \mathbf{FF} \\
  \denote{\return e} &= \lambda g . \denote{e} g\ \\
  \denote{\flip \theta} &= \lambda g . \ \{(\tt,0) \mapsto \theta , (\ff,0) \mapsto (1 - \theta)\} \\
  \denote{\reward k e} &= \lambda g . \ 
    \lambda v. \ 
    \begin{cases}
      \denote{e}(g)(b, r-k) & v = (b, r) \\
      \denote{e}(g)(v)  & \text{else}
    \end{cases}  \\
  \denote{\ite x {e_1} {e_2}} 
    &= \lambda g . \ \begin{cases}
      \denote{e_1}g & g(x) = \mathbf{TT} \\
      \denote{e_1}g & g(x) = \mathbf{FF} \\
      {\color{red}\texttt{abort}} & \text{else}
    \end{cases} \\
  \denote{\observe x e} 
    &= \lambda g . \ \begin{cases}
      \denote{e}g & g(x) = \mathbf{TT} \\
      \pmb{\bot} & g(x) = \mathbf{FF} \\
      {\color{red}\texttt{abort}} & \text{else}
    \end{cases} \\
  \denote{\bind x e {e'}}
    &= \lambda g . \ \denote{e} g  \ \fishbone \
      \lambda x. \ \begin{cases}
        \lambda y . \ 
          \begin{cases}
            \denote{e'} (g \cup \{x \mapsto \mathbf{TT}\}) (b,(s-r)) & y = (b,s)\\
            \denote{e'} (g \cup \{x \mapsto \mathbf{TT}\}) y     & \text{else}
          \end{cases} & x = (\tt, r) \\
          \lambda y . \ 
          \begin{cases}
            \denote{e'} (g \cup \{x \mapsto \mathbf{FF}\}) (b,(s-r)) & y = (b,s)\\
            \denote{e'} (g \cup \{x \mapsto \mathbf{FFS}\}) y     & \text{else}
          \end{cases} & x = (\ff, r) \\
        \pmb{\bot} & x = \bot 
      \end{cases}
\end{align*}