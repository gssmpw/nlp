\section{Related Work}
Recent advances in LLMs have spurred a variety of approaches to procedural content generation and narrative design in games. These methods can broadly be categorized into level generation, narrative scene creation, and RPG assistance, each with its own strengths and limitations.

\subsection{Level Generation With LLMs}
MarioGPT \cite{MarioGPT} is an early example that employs a GPT-2 \cite{GPT2} model to generate Super Mario Bros levels. By combining text-to-level generation with a novelty search algorithm, MarioGPT can produce diverse levels in an open-ended manner. Although 88.4\% of the generated levels were deemed playable by a powerful Super Mario algorithm, the system struggles to strictly adhere to design prompts. Its generated levels often feature inaccuracies in element counts (such as pipes, enemies, blocks, and elevation), and the model tends to memorize and regurgitate data rather than innovate unique experiences. Similar challenges are discussed in work on level generation through LLMs, where issues of spatial token representation and data scarcity are prominent \cite{Sokoban}. A related study \cite{Metavoidal} addresses level generation for a more complex and recent game (Metavoidal) by engineering constraint-based prompts for GPT-3 \cite{GPT3}. Despite these efforts, the approach achieved only a 37\% playable level generation rate and required significant human-in-the-loop augmentation, highlighting the difficulties of scaling LLM-driven level design to larger, more complex environments.

\subsection{Narrative Scene Generation}
SceneCraft \cite{SceneCraft} represents another line of work that targets interactive narrative scene generation. This framework transforms an author's written story summary into a choose-your-own-adventure–style interaction by using an LLM to generate narrative scenes. SceneCraft also extracts emotions and gestures from the generated text to enrich non-player character (NPC) interactions. However, while SceneCraft is innovative in extracting emotions and gestures, it is limited by its focus on creating short, isolated narrative scenes rather than an overarching narrative structure. Its focus on single, self-contained scenes means that it lacks a higher-level narrative planner to bind these scenes into a cohesive larger narrative. A study of the game 1001 Nights expressed a similar struggle with keeping the entirely generated responses of the King character on task \cite{1001Nights}. With simple prompting to the LLM, a player could force the King to break, ruining player immersion and the narrative of Persian folklore that the game was set in \cite{1001Nights}.   

\subsection{RPG Assistance and Dungeon Master Tools}
In the domain of role-playing games, Calypso \cite{Calypso} offers an LLM-based Dungeon Master (DM) assistant with interfaces tailored to support Dungeons \& Dragons (D\&D) gameplay. Its modules focus mainly on encounters, assisting with enemy information and battle narratives. Although Calypso excels at setting up and refining encounters, its general conversational interface is standard ChatGPT \cite{chatgpt} and does not extend to real-time narrative creation. While other approaches focus on generating RPG quests and dialogues using GPT models \cite{SceneCraft, 1001Nights}, they all similarly address individual components of gameplay in small digestible chunks that are easy for an LLM to comprehend. Without a framework to manage the context, the LLM has trouble generating expansive narratives.

% \subsection{VBTA in Context}
% In contrast to these prior efforts, the VBTA framework adopts a hybrid approach that integrates structured task descriptions and capability profiles with both semantic reasoning from LLMs and the dynamic game state planning of CBS for agent action decisions. This integration enables VBTA to generate comprehensive game content that spans narrative dialogue, combat scenarios, and environmental interaction. By automating decisions that have previously required external intervention—whether in guiding NPC behavior, resolving ambiguous prompts, or planning in-game actions—VBTA addresses several limitations observed in earlier approaches. In doing so, VBTA supports the creation of rich, continuously evolving game worlds that offer both high replayability and deep player immersion.