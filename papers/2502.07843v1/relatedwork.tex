\section{Related work}
\subsection{Emotional EEG analysis}

Electroencephalography (EEG) has played a crucial role in emotion analysis research as a tool for real-time measurement of brain electrical activity \cite{moon2016implicit}. EEG signals can be interpreted through both time-domain and frequency-domain analyses to understand emotional states.

Frequency domain analysis, in particular, has been widely adopted in EEG-based emotion research. Liu et al. \cite{liu2017real} conducted a study distinguishing emotions such as joy, anger, fear, and sadness in response to visual stimuli by combining power spectral density (PSD) and asymmetry indices . Furthermore, Zhang et al. \cite{zhang2016approach} demonstrated that emotional states could be more effectively estimated by introducing time-frequency distribution analysis, which combines temporal and frequency characteristics.

Recent advances in deep learning technology have revolutionized EEG signal analysis, offering superior feature extraction capabilities compared to conventional methods. Among various deep neural network models, CNNs have been a major architecture for EEG analysis. Yanagimoto and Sugimoto \cite{yanagimoto2016convolutional} demonstrated the effectiveness of automated feature learning in emotion recognition by utilizing CNN to classify emotional valence and arousal states from EEG signals. 
Li et al. \cite{li2018hierarchical} proposed a method of transforming EEG scalp maps into 2D images for CNN analysis, showcasing the potential to effectively handle EEG signal complexity while accounting for individual differences in emotional responses.


\subsection{Brain connectivity in emotion analysis}
Brain connectivity analysis has emerged as an effective approach in EEG-based emotion analysis. Previous studies have emphasized that analyzing interactions between brain regions, beyond the activation of specific brain regions, provides advantages. Costa et al. \cite{costa2006eeg} discovered significant correlations between EEG phase synchronization patterns and emotional states, while Lee and Hsieh \cite{lee2014classifying} developed and applied a connectivity measurement method combining correlation, coherence, and phase synchronization for emotion classification.
Recent research has actively explored the integration of connectivity matrices with deep learning approaches. Moon et al. \cite{moon2020emotional} demonstrated that connectivity matrices could be effectively utilized as CNN inputs to learn patterns related to emotional valence.

\begin{figure}[t]
    \centering
    \includegraphics[width=0.48\textwidth]{conmatonecol.pdf}
    \caption{Activation (right) after the first convolutional layer for a connectivity matrix (left)}
    \label{fig:actconmat}
\end{figure}

\begin{figure}[t]
    \includegraphics[width=0.48\textwidth]{cifaronecol.pdf}
    \caption{Activation (right) after the first convolutional layer for a natural image (left) from the CIFAR-10 dataset }
    \label{fig:actcifar}
\end{figure}

\begin{figure*}[t]
    \centering
    \includegraphics[width=2\columnwidth]{overview.png}  % 
    \caption{Overview of the proposed classification system}
    \label{fig:overview}
\end{figure*}