Facial parsing—the segmentation of fine-grained facial components such as \textit{eyes, nose, mouth, and hair}—is a fundamental task in computer vision, supporting applications in \textit{face recognition}~\cite{wang2021deep}, \textit{augmented reality}~\cite{lee2012implementing}, and \textit{facial expression analysis}~\cite{corneanu2016survey}. Recent advances in deep learning have significantly improved segmentation accuracy~\cite{chen2017deeplab, lin2020towards}, yet existing models primarily optimize for benchmark performance while often neglecting key concerns such as: (1) \textit{fairness} across demographic groups, (2) \textit{robustness} to noise, occlusions, and domain shifts, and (3) the \textit{impact} of segmentation on downstream generative models. While face parsers may perform well on clean, well-represented data, they often degrade sharply for underrepresented demographics~\cite{buolamwini2018gender, grother2019face, park2022fair} or in challenging real-world conditions~\cite{ghiasi2014parsing, geirhos2018imagenet, hendrycks2019benchmarking}. Such biases and fragility not only reduce trust and usability in applications like \textit{identity verification} and \textit{facial editing} but also propagate into \textit{generative synthesis}, amplifying disparities in downstream tasks.

Recent efforts have explored \textit{multi-objective optimization} for general segmentation~\cite{kendall2018multi, standley2020tasks, sener2018multi} and fairness-aware approaches in facial analysis~\cite{merler2019diversity, navon2020auxiliary}. However, a unified strategy that jointly optimizes \textit{accuracy, fairness, and robustness} in face parsing remains underexplored. Furthermore, integrating fair and robust segmentation with \textit{generative models} introduces additional complexities: state-of-the-art \textbf{GAN-based}~\cite{goodfellow2014generative} and \textbf{diffusion-based} models~\cite{zhang2023adding} rely on semantically structured segmentation maps~\cite{park2019semantic, wang2018high} to generate realistic and controllable faces. If the segmentation model introduces bias or lacks robustness, these deficiencies are propagated—and often amplified—by generative models, leading to unnatural or demographically skewed outputs~\cite{menon2020pulse, tan2020improving, friedrich2023fair}. This issue is particularly pronounced in \textbf{GAN-based synthesis}, where segmentation errors cause unnatural facial reconstructions, and in \textbf{diffusion-based models} such as ControlNet, where inaccurate parsing reduces \textit{semantic alignment} and \textit{editability}.

To address these challenges, we propose a \textbf{homotopy-based multi-objective learning framework} for face parsing that explicitly balances \textit{accuracy, fairness, and robustness}. Our method dynamically adjusts training objectives over time, shifting from an accuracy-first paradigm in early training to a balanced trade-off incorporating fairness and robustness. This approach enables stronger segmentation performance across diverse demographic groups while improving resilience to \textit{occlusions, noise, and domain shifts}. Unlike prior works that optimize for fairness or robustness in isolation, our framework \textit{unifies these perspectives} within a single pipeline and systematically evaluates their impact on generative face synthesis.

To validate our approach, we integrate \textbf{multi-objective and single-objective U-Net models} into a \textbf{GAN-based face synthesis pipeline (Pix2PixHD)} and assess their impact on generative quality. We further conduct \textbf{preliminary experiments} with \textbf{ControlNet}, a structured diffusion model, to examine how segmentation quality affects \textit{guided image generation}. Our evaluations span real-world perturbations—including Gaussian noise, occlusions, blur, and lighting shifts—as well as multiple demographic groups, measuring both segmentation performance (\textit{mIoU, fairness variance}) and generative quality (\textit{Fréchet Inception Distance (FID), LPIPS similarity}~\cite{yu2021frechet, zhang2018perceptual}). Our key contributions are as follows:

\textbf{(1) Fairness-Aware Face Parsing:} We introduce a \textit{multi-objective learning framework} that explicitly optimizes \textit{accuracy, fairness, and robustness}. 

\textbf{(2) Systematic Fairness \& Robustness Evaluation:} We quantify \textit{segmentation fairness} via \textit{mIoU variance} and assess robustness under \textit{occlusions, noise, and domain shifts}. 

\textbf{(3) Impact on GAN-Based Face Synthesis:} We show that fairness-aware segmentation improves GAN-generated face quality, reducing \textit{FID scores} and enhancing perceptual realism (\textit{lower LPIPS scores}). 

\textbf{(4) Preliminary Diffusion-Based Analysis:} We explore how segmentation quality influences \textit{diffusion-based synthesis (ControlNet)}.


Our findings highlight the importance of \textbf{fair and robust face parsing} in developing \textit{bias-aware generative models} with applications in \textit{face editing, identity verification, and ethical AI deployment}. 
The remainder of this paper is structured as follows: we discuss \textit{related work} in Section~\ref{sec:relatedwork}, present our \textit{proposed method} and \textit{experiments} in Section~\ref{sec:method}, describe \textit{results} in Section~\ref{sec:results}, and conclude with \textit{implications and future research directions} in Section~\ref{sec:future}.
