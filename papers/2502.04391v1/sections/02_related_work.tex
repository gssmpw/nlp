\subsection{Multi-Objective Optimization in Computer Vision}
Multi-objective optimization is widely used in computer vision to balance competing objectives such as accuracy, efficiency, and robustness~\cite{sharma2022comprehensive}. Traditional methods rely on fixed weighting schemes for loss functions, limiting adaptability across different tasks. More recent techniques, such as \textit{homotopy-based optimization}, introduce dynamic weighting mechanisms that shift priorities during training~\cite{chien2022gpu}. These methods have shown promise in solving complex optimization problems, particularly in high-dimensional polynomial systems~\cite{morgan1987computing}. However, their application to specialized areas such as \textit{face parsing and generative modeling} remains largely unexplored. Our work extends homotopy-based optimization to structured face parsing, explicitly integrating \textit{accuracy, fairness, and robustness} into a single multi-objective framework.

\subsection{Generative Adversarial Networks and Multi-Objective Training}
Generative Adversarial Networks (GANs) are widely used for tasks such as \textit{face generation, editing, and domain adaptation}~\cite{goodfellow2014generative}. Unlike diffusion models, which rely on iterative denoising, GANs synthesize high-quality images in a single forward pass, making them efficient for applications such as \textit{interactive facial editing}~\cite{karras2020analyzing}. GAN-based architectures provide structured control over facial attributes through techniques such as \textit{semantic segmentation-guided generation}~\cite{park2019semantic} and \textit{latent space manipulation}~\cite{wu2021stylespace}. 

Multi-objective training of GANs has been explored through \textit{multi-discriminator architectures} to improve stability and diversity~\cite{albuquerque2019multi} and \textit{evolutionary optimization approaches} for adversarial training~\cite{wang2019evolutionary}. However, GANs remain susceptible to \textit{mode collapse}, demographic biases, and robustness issues, particularly when trained on imbalanced datasets~\cite{tan2020improving}. Our work introduces a homotopy-based optimization framework that explicitly balances \textit{perceptual realism, semantic alignment, and demographic fairness} by leveraging segmentation maps as conditioning inputs. This allows us to systematically evaluate how fairness-aware parsing influences structured image synthesis. Additionally, we extend our analysis to \textit{diffusion-based synthesis (ControlNet)}, enabling a direct comparison between GANs and diffusion models in terms of \textit{controllability, fairness, and robustness}.

\subsection{Diffusion Models and Structured Conditioning}
Diffusion models have emerged as powerful alternatives to GANs for high-resolution image generation, achieving state-of-the-art performance in photorealistic synthesis~\cite{ho2020denoising}. Structured conditioning mechanisms, such as \textit{ControlNet}~\cite{zhang2023adding}, improve controllability by integrating external control signals, including segmentation or edge maps. While prior work has demonstrated the effectiveness of diffusion models for face synthesis~\cite{rombach2022high}, their dependence on structured inputs has not been systematically examined in the context of fairness-aware segmentation pipelines. Our study investigates the role of multi-objective face parsing in guiding diffusion-based synthesis and compares its impact against traditional GAN-based conditioning.

\subsection{Fairness in Face Parsing and Generative Models}
Fairness has been extensively studied in face recognition and classification, where demographic biases in deep learning models have been well documented~\cite{buolamwini2018gender}. However, fairness-aware segmentation remains underexplored~\cite{grother2019face}, despite evidence that segmentation models exhibit higher error rates for underrepresented demographic groups~\cite{dhar2021pass}. These disparities can propagate into downstream applications, such as attribute editing and face synthesis, amplifying biases in generative outputs.

While existing work has introduced fairness-aware regularization for generative models~\cite{tan2020improving}, few studies explicitly examine how segmentation biases affect generative synthesis pipelines. Our approach addresses this gap by incorporating fairness as an explicit training objective in face parsing and evaluating its effect on both GAN- and diffusion-based synthesis. By demonstrating how fairness-aware segmentation improves photorealism and demographic consistency, we establish a framework for more equitable face generation.

\subsection{Face Parsing for Generative Synthesis}
Face parsing, which involves segmenting facial components such as eyes, lips, and hair, plays a critical role in tasks like face editing, synthesis, and attribute manipulation~\cite{luo2012hierarchical}. Previous work has explored using segmentation maps to enhance GAN-based face editing~\cite{park2019semantic}. For instance, regional GAN inversion techniques leverage parsing maps to enable fine-grained control over facial feature editing~\cite{xu2021facecontroller}. However, existing methods prioritize accuracy without explicitly addressing fairness or robustness.

Our framework extends face parsing for generative synthesis by integrating segmentation and GAN training into a unified multi-objective pipeline. Using homotopy-based optimization, we balance \textit{realism, semantic alignment, and fairness}, ensuring that segmentation maps remain robust to variations in demographic attributes and imaging conditions.

\subsection{Robustness and Cross-Domain Generalization}
Robustness to noise, occlusion, and domain shifts remains a key challenge in vision models~\cite{geirhos2018imagenet, hendrycks2019benchmarking}. While segmentation models are often evaluated under controlled conditions, their deployment in real-world applications requires generalization across diverse datasets and imaging environments~\cite{minaee2021image}. Domain adaptation methods, such as \textit{CycleGAN}, have been explored for cross-domain transfer~\cite{zhu2017unpaired}, but they do not explicitly enforce robustness constraints in segmentation.

Our work incorporates robustness as an explicit optimization objective in face parsing, ensuring consistent performance under occlusions, noise, and cross-domain shifts. We validate our models across multiple datasets, including CelebAMask-HQ~\cite{CelebAMask-HQ}, demonstrating that multi-objective optimization enhances generalization and resilience.

\subsection{Contributions of Our Work}
While prior work has explored elements of \textit{multi-objective optimization, fairness-aware segmentation, and GAN-conditioned synthesis}, our approach is the first to integrate these into a \textit{unified homotopy-based framework}. By dynamically balancing \textit{accuracy, fairness, robustness, and semantic fidelity}, we improve face parsing performance and demonstrate its downstream impact on generative tasks.

Unlike prior methods that address fairness at the synthesis stage, our approach ensures \textit{fairness and robustness at the segmentation level}, reducing biases before they propagate into generative models. We systematically evaluate how segmentation quality affects both \textbf{GAN-based synthesis (Pix2PixHD)} and \textbf{diffusion-based synthesis (ControlNet)}, showing improvements in photorealism, demographic consistency, and structured conditioning. Our findings highlight the importance of fairness-aware segmentation for bias-aware generative modeling in face editing and synthesis.
