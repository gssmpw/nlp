\vspace{-5mm}
\section{Limitations and Future Directions}
\label{sec:future}

Despite notable improvements in fairness, robustness, and segmentation quality, several challenges remain, presenting opportunities for further research.  First, the CelebAMask-HQ dataset, while diverse, remains imbalanced across demographic groups, which may limit generalization. Addressing this requires more strategic data augmentation, active reweighting, or leveraging larger, demographically-balanced datasets to further mitigate bias and enhance equitable performance. Second, our current framework treats GANs as passive consumers of segmentation maps. Incorporating \textit{bi-directional optimization}, where segmentation feedback influences GAN training, could improve both parsing fidelity and generative realism. Such an approach could be extended to diffusion models, where structured conditioning remains underexplored in fairness-aware synthesis.  

Additionally, while our method is broadly applicable beyond facial segmentation, extending it to domains such as medical imaging, autonomous perception, or video-based synthesis may require task-specific adaptations. Future research should explore domain-aware multi-objective formulations that account for context-specific biases and robustness challenges. Finally, while homotopy scheduling improves optimization efficiency, fairness-aware training introduces additional computational overhead due to subgroup evaluations. Exploring adaptive sampling strategies or efficient approximations could make large-scale deployments more feasible, especially for real-time applications.  

 Our findings underscore that multi-objective training does not impose rigid trade-offs—adaptive optimization can integrate fairness and robustness without sacrificing accuracy. By extending these ideas to broader datasets, generative frameworks, and real-world applications, future research can drive the development of more equitable and resilient vision models for AI-driven image synthesis and recognition.
