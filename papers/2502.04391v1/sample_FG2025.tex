%%%%%%%%%%%%%%%%%%%%%%%%%%%%%%%%%%%%%%%%%%%%%%%%%%%%%%%%%%%%%%%%%%%%%%%%%%%%%%%%
%2345678901234567890123456789012345678901234567890123456789012345678901234567890
%        1         2         3         4         5         6         7         8
%
% Slightly modified by Shaun Canavan for FG2025
%

%\documentclass[letterpaper, 10 pt, conference]{ieeeconf}  % Comment this line out
                                                          % if you need a4paper
\documentclass[a4paper, 10pt, conference]{ieeeconf}      % Use this line for a4
                                                          % paper
\usepackage{FG2025}
\usepackage{amsmath}
\usepackage{algorithm}
\usepackage{algpseudocode}
\usepackage{graphicx}
\usepackage{booktabs}
\usepackage{adjustbox}
\usepackage{multirow}
\usepackage{subfigure}
\usepackage{amssymb}
\usepackage{hyperref}

\FGfinalcopy % *** Uncomment this line for the final submission



\IEEEoverridecommandlockouts                              % This command is only
                                                          % needed if you want to
                                                          % use the \thanks command
\overrideIEEEmargins
% See the \addtolength command later in the file to balance the column lengths
% on the last page of the document

% The following packages can be found on http:\\www.ctan.org
%\usepackage{graphics} % for pdf, bitmapped graphics files
%\usepackage{epsfig} % for postscript graphics files
%\usepackage{mathptmx} % assumes new font selection scheme installed
%\usepackage{times} % assumes new font selection scheme installed
%\usepackage{amsmath} % assumes amsmath package installed
%\usepackage{amssymb}  % assumes amsmath package installed

\def\FGPaperID{0217} % *** Enter the FG2025 Paper ID here

\title{\LARGE \bf
Towards Fair and Robust Face Parsing for Generative AI: A Multi-Objective Approach
}

%use this in case of a single affiliation
%\author{\parbox{16cm}{\centering
%    {\large Huibert Kwakernaak}\\
%    {\normalsize
%    Faculty of Electrical Engineering, Mathematics and Computer Science, University of Twente, Enschede, The Netherlands\\}}
%    \thanks{This work was not supported by any organization.}% <-this % stops a space
%}

%use this in case of several affiliations
% \author{\parbox{16cm}{\centering
%     {\large Huibert Kwakernaak$^1$ and Pradeep Misra$^2$}\\
%     {\normalsize
%     $^1$ Faculty of Electrical Engineering, Mathematics and Computer Science, University of Twente, Enschede, The Netherlands\\
%     $^2$ Department of Electrical Engineering, Wright State University, Dayton, USA}}
%     \thanks{This work was not supported by any organization}% <-this % stops a space
% }

\author{%
  \parbox{\textwidth}{\centering
    {\large Sophia J.~Abraham$^{1}$, Jonathan D.~Hauenstein$^{2}$, Walter J.~Scheirer$^{1}$}\\
    {\normalsize
      $^{1}$Department of Computer Science and Engineering, University of Notre Dame, Notre Dame, IN 46556\\
      $^{2}$Department of Applied and Computational Mathematics and Statistics, University of Notre Dame, Notre Dame, IN 46556
    }\\
  }
}

\begin{document}

\ifFGfinal
\thispagestyle{empty}
\pagestyle{empty}
\else
\author{Anonymous FG2025 submission\\ Paper ID \FGPaperID \\}
\pagestyle{plain}
\fi
\maketitle



%%%%%%%%%%%%%%%%%%%%%%%%%%%%%%%%%%%%%%%%%%%%%%%%%%%%%%%%%%%%%%%%%%%%%%%%%%%%%%%%
%%%%%%%%%%%%%%%%%%%%%%%%%%%%%%%%%%%%%%%%%%%%%%%%%%%%%%%%%%%%%%%%%%%%%%%%%%%%%%%%
\begin{abstract}
End-to-end imitation learning offers a promising approach for training robot policies. However, generalizing to new settings—such as unseen scenes, tasks, and object instances—remains a significant challenge. Although large-scale robot demonstration datasets have shown potential for inducing generalization, they are resource-intensive to scale. In contrast, human video data is abundant and diverse, presenting an attractive alternative. Yet, these human-video datasets lack action labels, complicating their use in imitation learning. Existing methods attempt to extract grounded action representations (e.g., hand poses), but resulting policies struggle to bridge the embodiment gap between human and robot actions.
% our approach
We propose an alternative approach: leveraging language-based reasoning from human videos - essential for guiding robot actions - to train generalizable robot policies. Building on recent advances in reasoning-based policy architectures, we introduce Reasoning through Action-free Data (RAD). RAD learns from both robot demonstration data (with reasoning and action labels) and action-free human video data (with only reasoning labels). The robot data teaches the model to map reasoning to low-level actions, while the action-free data enhances reasoning capabilities. Additionally, we will release a new dataset of 3,377 human-hand demonstrations compatible with the Bridge V2 benchmark. This dataset includes chain-of-thought reasoning annotations and hand-tracking data to help facilitate future work on reasoning-driven robot learning.
% experiments
Our experiments demonstrate that RAD enables effective transfer across the embodiment gap, allowing robots to perform tasks seen only in action-free data. Furthermore, scaling up action-free reasoning data significantly improves policy performance and generalization to novel tasks. These results highlight the promise of reasoning-driven learning from action-free datasets for advancing generalizable robot control. 
% releasing dataset
Website: \href{https://rad-generalization.github.io}{here}.
 
% Keep your abstract concise (~150-250 words).
% Summarize the key problem, approach, results, and significance.
\end{abstract}


%%%%%%%%%%%%%%%%%%%%%%%%%%%%%%%%%%%%%%%%%%%%%%%%%%%%%%%%%%%%%%%%%%%%%%%%%%%%%%%%
\section{Introduction}
\label{sec:intro}
\begin{figure}[ht]
    \centering
    \includegraphics[width=0.8\linewidth]{graphs/greater_than_naive.pdf}
    \vspace{0.5cm}
    \includegraphics[width=0.8\linewidth]{graphs/p1_bottom.png}
    \vspace{-5pt}
    \caption{\textcolor{positional}{Positional} vs.\ \textcolor{nonpositional}{non-positional} circuits. In a \textcolor{nonpositional}{non-positional} circuit, the same edges must be included at all positions. A \textcolor{positional}{positional} circuit can distinguish between the same edge at different positions. This specificity yields better trade-offs between circuit size and faithfulness. It can also increase both precision and recall.}
    \label{fig:p1}
    \vspace{-5pt}
\end{figure}

\section{Introduction}

\looseness=-1
A primary goal of interpretability research is to characterize the internal mechanisms in language models (LMs) and other NLP models. 
A core approach in this area is \textbf{circuit discovery}---identifying the minimal subgraph within the model's computation graph that performs a specific task \citep{olah2021framework,olah-mech}.
Typically, the nodes of a circuit represent model components (e.g., attention heads, neurons, or layers).
While manual circuit discovery methods can yield position-specific insights \citep{wanginterpretability,goldowskydill2023localizingmodelbehaviorpath}, \emph{automatic methods often overlook positional information}, treating components as uniformly relevant across all input token positions \citep{conmytowards,syed2023attribution}. 
For instance, if an attention head is included in a circuit, it is assumed to contribute equally to the computation for every position in the input sequence.
The assumption that circuits are position-invariant ignores the fact that different positions often require distinct computations.
By ignoring positions, current methods limit their ability to capture mechanisms that operate across positions, such as interactions between attention heads across positions.

In this study, we start by demonstrating that positional agnosticism is a significant limitation (\S\ref{sec:motivating}). Then, to address these limitations, we introduce a new approach: position-aware edge attribution patching (PEAP; \S\ref{sec:full_circ_discovery}; Figure~\ref{fig:p1}). Current approaches  assume that if an edge is in a circuit, then the same edge will be in the circuit at all positions, thus leading to low precision. It is also assumed that an edge's importance should be aggregated across positions before deciding whether it should be included in the circuit; this can lead to cancellation effects, and thus low recall. PEAP instead allows us to compute the importance of cross-positional edges, and separately evaluates edge importance at each position. We show that this leads to smaller and more accurate circuits; see Figure~\ref{fig:p1}.

Incorporating positional information into circuit discovery is straightforward when inputs have the same length and structure across examples.

However, realistic datasets are not nearly this templatic.
How, then, can we incorporate positional information into automatic circuit discovery?
To address this challenge, we propose \textbf{schemas} (\S\ref{sec:schema}). 
Schemas assign semantic labels to spans of tokens, enabling information aggregation across examples even when the spans differ in length.

For example, in the input ``The \textcolor{positional}{war} lasted from 1453 to 14\underline{\hspace{1em}},'' the span ``\textcolor{positional}{war}'' could be labeled as ``\emph{Subject}''.
This enables handling spans with varying lengths: the phrase ``\textcolor{positional}{Black Plague}'' in another example can be treated as a single positional span with the same role as ``\textcolor{positional}{war}''.
In experiments with two LMs and three tasks, we find that circuits discovered using schemas achieve a better trade-off between circuit size and faithfulness to the model's behavior than position-agnostic circuits.
Importantly, position-aware circuits offer a more precise representation of the underlying mechanisms, providing a more concise foundation for mechanistic explanations.

We also present a fully automated pipeline for schema generation and application (\S\ref{sec:schema-generation}) using large language models (LLMs). 
We evaluate the quality of the generated schemas and their utility in discovering position-aware circuits (\S\ref{sec:schema-eval}).
Notably, circuits derived using automatically generated and applied schemas achieve comparable faithfulness scores to circuits discovered with human-designed and manually applied schemas.

We summarize our contributions as follows:
\begin{itemize}[noitemsep,leftmargin=*,topsep=1pt,parsep=1pt]
    \item Introduce a position-aware circuit discovery method, which obtains better faithfulness than position-agnostic discovery.  
    \item Introduce dataset schemas,  facilitating positional circuit discovery in more naturalistic settings. 
    \item Develop an automated schema generation and application pipeline with LLMs, yielding schemas that are comparable to manually-annotated ones.
\end{itemize}

% 1. Problem Statement: Introduce the broad area of face parsing and/or face and gesture recognition.
% 2. Motivation: Why is fairness + robustness critical for these tasks?
% 3. Limitations of prior work: Summarize the gap your paper addresses.
% 4. Proposed Approach: Quick overview of your multi-objective homotopy-based method or main contribution.
% 5. Contributions: List bullet points of main contributions.


%%%%%%%%%%%%%%%%%%%%%%%%%%%%%%%%%%%%%%%%%%%%%%%%%%%%%%%%%%%%%%%%%%%%%%%%%%%%%%%%
\section{Related Work}
\label{sec:relatedwork}

\section{Related work}


Recent advances in single-image animatable head avatar generation can be categorized into mainly 2D-based and 3D-based approaches. 

\paragraph{\bf Image to 2D Animatable Avatar.}
2D-based methods, leveraging the power of convolutional neural networks (CNNs)~\cite{DBLP:conf/cvpr/KarrasLAHLA20,DBLP:conf/cvpr/IsolaZZE17,DBLP:conf/nips/GoodfellowPMXWOCB14}, often employ generative adversarial networks (GANs)~\cite{DBLP:conf/cvpr/StyleGAN} for direct image synthesis. Early approaches~\cite{DBLP:conf/cvpr/WangDYSW23,DBLP:conf/cvpr/BurkovPGL20,DBLP:conf/iccv/ZakharovSBL19} focus on injecting expression and pose features into the generator network, often utilizing architectures like U-Net or StyleGAN~\cite{DBLP:conf/cvpr/StyleGAN}.
Some other 2D methods~\cite{DBLP:journals/corr/abs-2407-03168,DBLP:conf/cvpr/ZhangQZZW0CW023,DBLP:conf/cvpr/HongZS022,DBLP:conf/mm/DrobyshevCKILZ22,DBLP:conf/cvpr/BurkovPGL20,DBLP:conf/nips/SiarohinLT0S19} represent expressions and poses as warping fields applied to the source image. 
Benefiting from advances in image and video diffusion networks, more recent 2D-based works~\cite{DBLP:journals/corr/abs-2410-07718,DBLP:journals/corr/abs-2406-08801,DBLP:conf/eccv/TianWZB24} get improved results with diffusion techniques. 
However, these methods still face challenges related to long generation times and significant computational resource demands. Audio-driven 2D control methods~\cite{DBLP:conf/cvpr/ZhangCWZSGSW23,DBLP:journals/corr/abs-2211-12368,DBLP:conf/iccv/GuoCLLBZ21} are easy to use but cannot explicitly control facial expressions and poses. 2D-based techniques often struggle with large pose or expression variations due to the lack of an explicit 3D structure, sometimes producing unrealistic distortions or identity changes. While some 2D methods~\cite{SadTalker,StyleHEAT,Pirenderer,DBLP:conf/cvpr/WangM021,MegaPortraits} incorporate 3D Morphable Models (3DMMs)~\cite{DBLP:conf/fgr/GerigMBELSV18,DBLP:journals/tog/LiBBL017,DBLP:conf/avss/PaysanKARV09,DBLP:conf/siggraph/BlanzV99} to mitigate these issues, they typically cannot achieve free-viewpoint rendering. 

\vspace{-0.1in}

\begin{figure*}[h]
    \centering
    \includegraphics[width=0.9\linewidth]{images/framework.pdf}
    \caption{\textbf{Overall Framework.} Our framework utilizes learnable query features attached to FLAME vertices to perform cross-attention with the extracted multi-level image features. The extracted features are then decoded to reconstruct the Gaussian avatar in the canonical space, which can be animated utilizing standard linear blend skinning (LBS) and corrective blendshapes as the FLAME model did and rendered in real-time on various platforms.}
    \label{fig:framework}
\end{figure*}

\paragraph{\bf Image to 3D Animatable Avatar.}
3D-aware methods offer improved geometric consistency and free-viewpoint rendering capabilities. Early 3D approaches~\cite{DBLP:conf/eccv/KhakhulinSLZ22,DBLP:conf/cvpr/XuYCWDJT20} utilize 3DMMs for head avatar reconstruction. With the advent of Neural Radiance Fields (NeRFs)~\cite{DBLP:conf/eccv/MildenhallSTBRN20}, many recent methods~\cite{DBLP:conf/siggraph/YuFZWYBCSWSW23,DBLP:conf/cvpr/MaZQLZ23,DBLP:conf/cvpr/LiZWZ0CZWB023,GPAvatar,ye2024real3d,deng2024portrait4d,deng2024portrait4d2,DBLP:conf/eccv/KiMC24,DBLP:conf/cvpr/BaiFWZSYS23,PointAvatar,Nerfies,INSTA} have adopted this representation for higher fidelity, particularly in modeling fine details like hair. However, NeRF-based~\cite{DBLP:conf/cvpr/ZhangZLHLWGCL024,HAvatar,DBLP:conf/cvpr/BaiTHSTQMDDOPTB23,AD-NeRF,DBLP:journals/tog/GaoZXHGZ22,DBLP:journals/tog/ParkSHBBGMS21,DBLP:conf/cvpr/AtharXSSS22,DBLP:journals/corr/abs-2112-05637,DBLP:conf/iccv/TretschkTGZLT21,DBLP:conf/cvpr/GafniTZN21,DBLP:conf/eccv/KiMC24,DBLP:conf/cvpr/BaiFWZSYS23,PointAvatar,Nerfies,DBLP:conf/siggraph/YuFZWYBCSWSW23,DBLP:conf/cvpr/MaZQLZ23,DBLP:conf/cvpr/LiZWZ0CZWB023} approaches often require extensive training data, including multi-view or single-view videos, raising privacy concerns and limiting generalization to unseen identities. Some methods~\cite{DBLP:conf/cvpr/SunWWLZZL23,DBLP:conf/3dim/ZhuangMKS22,DBLP:journals/pami/SunWZHWL24,DBLP:journals/tvcg/TangZYZCMW24,DBLP:conf/iclr/XuZLZBFS23} bypass this data requirement by training generators with random noise and then inverting them for identity-specific reconstruction, but inversion accuracy remains a challenge. Test-time optimization offers another alternative, but its computational cost limits practical applications. Several recent works~\cite{goha2023,hidenerf2023,gpavatar2024,ye2024real3d,ma2024cvthead,deng2024portrait4d,deng2024portrait4d2,GGHead} have explored one-shot 3D head reconstruction to address the limitations of data requirements and computational cost. These methods employ various techniques, such as tri-plane features, deformation fields, point-based expression fields, and vertex-feature transformers. Despite these advancements, NeRF-based methods often struggle with real-time rendering. 
Recently, 3D Gaussian Splatting~\cite{GaussianSplatting} has emerged as a promising alternative, offering both high-quality results and fast rendering speeds. However, existing Gaussian Splatting methods~\cite{GaussianAvatar,DBLP:conf/cvpr/XuCL00ZL24} typically rely on video data for training for each person, limiting their ability to generalize to new identities. Instead, the most recent work, GAGAvatar~\cite{GAGAvatar}, proposes a one-shot 3D Gaussian-based head avatar generation method. However, it still relies heavily on complex 2D neural post-processing to achieve optimal animation outcomes, thus it is not a pure 3D solution and the extra neural network hinders its application on various platforms. In contrast, our work generates Gaussian heads that are immediately animatable and renderable without additional networks or post-processing steps, enabling seamless integration into existing rendering pipelines for real-time animation and rendering across a wide range of platforms, including mobile phones. 
% 1. Face Parsing and Synthesis: Overview of traditional or recent SOTA methods (U-Net, DeepLabV3+, SPADE, etc.).
% 2. Fairness in Vision Systems: Summarize attempts to mitigate bias or ensure fairness, especially in face tasks.
% 3. Robustness in Face/Gesture Recognition: Past methods for handling occlusions, noise, domain shifts.
% 4. Multi-Objective Optimization: Dynamic weighting or homotopy-based approaches in the literature.


%%%%%%%%%%%%%%%%%%%%%%%%%%%%%%%%%%%%%%%%%%%%%%%%%%%%%%%%%%%%%%%%%%%%%%%%%%%%%%%%
%%% PROPOSED METHOD
\section{Proposed Method}
\label{sec:method}

In this section, we introduce our homotopy-based multi-objective framework for face parsing and its integration with both \textbf{GAN-based} and \textbf{diffusion-based} face editing models. We outline the problem formulation, dataset preparation, model architecture, training strategy, and evaluation pipeline, emphasizing \textbf{fairness}, \textbf{robustness}, and \textbf{semantic alignment}.

\subsection{Problem Formulation}
\label{subsec:problem_formulation}

We define the dataset \(\mathbf{X} = \{x_i\}\), where each face image is paired with a segmentation mask \( y_i \in \mathbf{Y} \), mapping to 19 facial components (e.g., hair, eyes, mouth). Demographic attributes are denoted as \(\mathbf{a}\) (e.g., \texttt{Male}, \texttt{Young}, \texttt{Wearing Hat}). Our objective is to train a segmentation function \( f_\theta(\cdot) \) that predicts \(\hat{y}_i\) while optimizing for accuracy, fairness, and robustness. Accuracy is maximized by aligning \(\hat{y}_i\) with \(y_i\) using Dice loss~\cite{sudre2017generalised}. Fairness is enforced by minimizing variance \(\mathrm{Var}(\mathrm{mIoU}_g)\) across demographic groups, ensuring equitable segmentation quality. Robustness is maintained by penalizing performance degradation (\(\mathrm{mIoU}\) drop) under input perturbations such as noise and occlusion.

\begin{algorithm}[h][t]
\caption{Multi-Objective Face Parsing (Pseudo-code)}
\label{alg:multi_objective_pseudocode}
\begin{algorithmic}[1]
\Require Homotopy function \(h(t)\) providing \((\alpha, \beta, \gamma)\) for epoch \(t\)
\For{epoch \(t = 1 \dots T\)}
    \State \((\alpha, \beta, \gamma) = h(t)\)
    \For{each batch in DataLoader}
        \State \textbf{Load} images \(\{x\}\), masks \(\{m\}\), attributes \(\{a\}\)
        \State outputs \(= f_{\theta}(x)\) \Comment{U-Net forward pass}
        \State \(\mathcal{L}_{\mathrm{acc}} = \mathrm{DiceLoss}(outputs, m)\)
        \State outputs\(_{\mathrm{noisy}} = outputs + \text{random\_noise}()\)
        \State \(\mathcal{L}_{\mathrm{rob}} = -\mathrm{mIoU}(\mathrm{softmax}(outputs_{\mathrm{noisy}}), m)\)
        \State \(\mathcal{L}_{\mathrm{fair}} = \mathrm{Var}\left[\mathrm{mIoU}_g\right]\)
        \State \(\mathcal{L}_{\text{total}} = \alpha\,\mathcal{L}_{\mathrm{acc}} + \beta\,\mathcal{L}_{\mathrm{rob}} + \gamma\,\mathcal{L}_{\mathrm{fair}}\)
        \State \textbf{Backward} and \textbf{update} \(\theta\)
    \EndFor
\EndFor
\end{algorithmic}
\end{algorithm}

\subsection{Dataset Preparation}
\label{subsec:dataset_preparation}

We employ the CelebAMask-HQ dataset \cite{CelebAMask-HQ}, divided into training, validation, and test sets. Each image and mask are resized to \(256 \times 256\) for compatibility with our U-Net architecture. Demographic attributes are extracted from annotations to compute fairness metrics.

\subsection{Model Architecture}
\label{subsec:model_architecture}

Our segmentation model utilizes a U-Net architecture with a ResNet-34 encoder pre-trained on ImageNet. It outputs 19 channels corresponding to distinct facial regions, balancing computational efficiency with high segmentation accuracy.

\subsection{Multi-Objective Training}
\label{subsec:multi_objective}

We train the U-Net segmentation models by optimizing a weighted sum of accuracy, fairness, and robustness losses, dynamically adjusted using homotopy-based scheduling. The training process is outlined in Algorithm~\ref{alg:multi_objective_pseudocode}.

\paragraph{Loss Components}
\begin{itemize}
    \item \textbf{Accuracy Loss (\(\mathcal{L}_{\mathrm{acc}}\)):} Dice loss measures the overlap between predicted and ground truth masks.
    \item \textbf{Robustness Loss (\(\mathcal{L}_{\mathrm{rob}}\)):} Negative \(\mathrm{mIoU}\) under perturbed predictions to ensure stability.
    \item \textbf{Fairness Loss (\(\mathcal{L}_{\mathrm{fair}}\)):} Variance of \(\mathrm{mIoU}\) across demographic groups to promote equitable performance.
\end{itemize}

\textbf{Alternative Fairness Computation:} We also compute per-group \(\mathrm{mIoU}\) for each demographic attribute, enabling detailed analysis of performance disparities (see Section~\ref{subsec:fairness_comparison}).

\subsection{Homotopy-Based Loss Scheduling}
\label{subsec:homotopy}

We dynamically balance the three loss components using epoch-dependent weights \(\alpha(t)\), \(\beta(t)\), and \(\gamma(t)\), ensuring \(\alpha(t) + \beta(t) + \gamma(t) = 1\). Initially, accuracy is prioritized, with weights shifting towards robustness and fairness over time. We explore three scheduling strategies:

\begin{itemize}
    \item \textbf{Linear:} \(\alpha(t)\) decreases linearly, while \(\beta(t)\) and \(\gamma(t)\) increase proportionally.
    \item \textbf{Sigmoid:} Smooth logistic transitions for gradual emphasis shifts.
    \item \textbf{Piecewise:} Abrupt changes in weight distribution at predefined training stages.
\end{itemize}

\begin{figure}[t]
    \centering
    \includegraphics[width=\columnwidth]{figures/parameters_by_homotopy-crop.pdf}
    \caption{Comparison of \(\alpha\), \(\beta\), and \(\gamma\) schedules across three homotopy methods (Linear, Sigmoid, and Piecewise) over 30 epochs. Each subplot illustrates the evolution of a parameter (\(\alpha\), \(\beta\), or \(\gamma\)) as it adapts during training, highlighting the differences in transition dynamics across homotopy strategies. The legend below the figure identifies the homotopy method for each curve.}
    \label{fig:homotopy-schedules}
\end{figure}


Figure~\ref{fig:homotopy-schedules} illustrates the evolution of these weights across training epochs for each homotopy method.

\subsection{Integration with Generative Models}

\subsubsection{GAN-Based Face Editing}
\label{subsec:gan_integration}

We utilize the trained U-Nets to generate segmentation maps for the training and validation sets, which are then used to train a Pix2PixHD GAN. The GAN architecture comprises:

\begin{itemize}
    \item \textbf{Generator} \(G\): Transforms segmentation maps into RGB images.
    \item \textbf{Discriminator} \(D\): Distinguishes real images from generated ones.
\end{itemize}

The GAN training involves a combination of adversarial loss and pixel-level \(L_1\) reconstruction loss:
\[
\mathcal{L}_{\mathrm{GAN}} = \mathcal{L}_{\mathrm{adv}}(G, D) + \lambda \, \|\hat{x} - x\|_1,
\]
where \(\hat{x} = G(\text{segmentation\_map})\) and \(x\) is the real image.

During testing, the GAN generates images using segmentation maps from the test set produced by both single-objective and multi-objective U-Nets, enabling evaluation of how segmentation quality impacts generative performance.

\subsubsection{ControlNet-Based Face Editing}
\label{subsec:controlnet_integration}

In addition to GANs, we integrate \textbf{ControlNet} \cite{zhang2023adding} for diffusion-based face editing. ControlNet leverages segmentation maps to guide the diffusion process, enhancing image fidelity and semantic alignment. Our setup includes:

\begin{itemize}
    \item \textbf{ControlNet Model:} Pre-trained on Stable Diffusion, fine-tuned on our segmentation maps.
    \item \textbf{Diffusion Pipeline:} Combines ControlNet with a text encoder and U-Net backbone to generate photorealistic faces conditioned on segmentation maps.
\end{itemize}

\textbf{Training Procedure:} ControlNet is fine-tuned for a single epoch using segmentation maps from the training set. In diffusion-based experiments, we compare only the single-objective model with the multi-objective linear homotopy model to manage computational resources effectively. The training minimizes the standard denoising loss:
\[
\mathcal{L}_{\mathrm{ControlNet}} = \mathcal{L}_{\mathrm{denoise}},
\]
where \(\mathcal{L}_{\mathrm{denoise}}\) is the Mean Squared Error between predicted and actual noise. During testing, ControlNet generates images using test set segmentation maps from both U-Net models, allowing assessment of segmentation quality's effect on diffusion-based generation.

\subsection{Evaluation Metrics and Setup}
\label{subsec:evaluation}

\paragraph{Segmentation Metrics}  
We evaluate segmentation performance using the mean Intersection-over-Union (\(\mathrm{mIoU}\)) across 19 facial classes. Fairness is quantified by the variance \(\mathrm{Var}(\mathrm{mIoU}_g)\) across demographic groups, and robustness is assessed through performance under Gaussian noise, occlusions, and blur.

\paragraph{Generative Metrics}  
For GAN outputs, we evaluate image quality using \textbf{Fréchet Inception Distance (FID)}, which quantifies realism by comparing feature distributions between generated and real images. Additionally, \textbf{Learned Perceptual Image Patch Similarity (LPIPS)} measures perceptual similarity, where lower scores indicate greater visual resemblance to real images.

\paragraph{Implementation Details}  
All experiments are implemented in PyTorch and trained on four NVIDIA A10 GPUs using the Adam optimizer with a learning rate of \(10^{-4}\). For ControlNet, we fine-tune the pre-trained \texttt{control\_v11p\_sd15\_seg} model based on Stable Diffusion v1.5. Our pipeline supports gradient accumulation and mixed precision (FP16) for computational efficiency. Homotopy-based loss scheduling is configurable (\texttt{linear}, \texttt{sigmoid}, \texttt{piecewise}). Detailed training configurations will be released alongside our code and models to ensure reproducibility.

\paragraph{Workflow Summary}  
\begin{enumerate}
    \item \textbf{Train U-Nets:} Train single-objective and multi-objective U-Nets on the training set, validate on the validation set.
    \item \textbf{Generate Segmentation Maps:} Use trained U-Nets to produce segmentation maps for training, validation, and test sets.
    \item \textbf{Train GAN:} Train the Pix2PixHD GAN using segmentation maps from the training and validation sets.
    \item \textbf{Fine-Tune ControlNet:} Fine-tune ControlNet on training set segmentation maps for one epoch.
    \item \textbf{Generate and Evaluate Images:} Generate images using GAN and ControlNet with test set segmentation maps from both U-Net models; evaluate using FID and LPIPS.
\end{enumerate}

In the following section, we present quantitative and qualitative results demonstrating the effectiveness of our approach across various conditions and demographic groups.

% Formally define the tasks of face parsing, generation, or both. 
% Introduce notations (X for images, Y for segmentation masks, etc.).

%%%%%%%%%%%%%%%%%%%%%%%%%%%%%%%%%%%%%%%%%%%%%%%%%%%%%%%%%%%%%%%%%%%%%%%%%%%%%%%%
%%% RESULTS
\section{Results}\label{sect:results}

\subsection{Established Benchmarks}\label{sect:results_west}
We begin by evaluating all vision-language models on established benchmarks, based on ImageNet and COCO Captions, among other datasets. As revealed in Table~\ref{tab:west_standard_setup}, increasing the dataset size from 10 billion to 100 billion examples does not improve performance substantially. This is statistically supported by Wilcoxon's signed rank test~\cite{wilcoxon1992individual}, which gives a $p$-value of 0.9, indicating that differences are not significant.


In addition, we also fit data scaling laws for every combination of model and dataset following the recipe proposed in~\citet{alabdulmohsin2022revisiting}. This allows us to evaluate whether or not the performance gap is expected to increase or decrease in the infinite-compute regime. We report the resulting scaling exponents and asymptotic performance limits in the tables. Again, we do not observe  significant differences at the 95\% confidence level ($p$-value of 0.09).


\subsection{Cultural Diversity}
Unlike the Western-oriented metrics reported in Section~\ref{sect:results_west}, cultural diversity metrics present an entirely different picture. We observe \emph{notable} gains when scaling the size of the dataset from 10 billion to 100 billion examples in Table~\ref{tab:culture_standard_setup}. 
For example, scaling training data from 10 billion to 100 billion examples yields substantial gains on Dollar Street 10-shot classification task, where ViT-L and ViT-H see absolute improvements of 5.8\% and 5.4\%, respectively. These gains outperform the typical improvements (less than 1\%) observed on Western-oriented 10-shot metrics by a large margin.
Using Wilcoxon's signed rank test, we obtain a $p$-value of 0.002, indicating a statistically significant evidence at the 99\% confidence level.


\subsection{Multilinguality}

Our multilingual benchmark, Crossmodal-3600 zero-shot retrieval~\cite{thapliyal2022crossmodal}, shows a disparity in performance gains: low-resource languages benefit more from the 100 billion scale than the high-resource ones. The disparity, illustrated in Figure~\ref{fig:multilinguality}, which not only exists in all model sizes but also widens as the models become larger. Detailed results for each language can be found in Appendix~\ref{appendix:data_scale}.

% source: https://colab.corp.google.com/drive/1AKgGDITZqTC2hQjVc-Iv8xuysh5giP0i#scrollTo=2EtEXMbly8dB&line=1&uniqifier=1
\begin{figure}[h!]
    % \includegraphics[width=\linewidth]{figures/multilang-Average_Multilingual__Low-Resource_Lang.pdf}
    % \includegraphics[width=0.86\linewidth]{figures/multilang-Average_Multilingual__High-Resource_Lang.pdf}
    \includegraphics[width=\linewidth]{figures/multilang-Average_XM3600_Retrieval.pdf}
    \caption{Scaling up to 100B examples leads to more notable improvements in low-resource languages. $\Delta$ denotes the improved accuracy when scaling from 10B examples to 100B.}
    \label{fig:multilinguality}
\end{figure}


\subsection{Fairness}
For fairness, we report on 3 metrics discussed in Section~\ref{sect:evals}. 

\paragraph{Representation Bias.} The first metric is representation bias (RB), with results detailed in Table~\ref{tab:rb}. We observe that models trained on unbalanced web data have a significantly higher preference to associate a randomly chosen image from ImageNet~\cite{deng2009imagenet} with the label ``Male'' over the label ``Female.'' 

In fact, this occurs nearly 85\% of the time. Training on 100B examples does not mitigate this effect. This finding aligns with previous research highlighting the necessity of bias mitigation strategies, such as data balancing~\cite{alabdulmohsin2024clip}, to address inherent biases in web-scale datasets.

% \begin{table}[h]
%     \centering\scriptsize
%     \caption{representation bias with respect to gender using imagenet. Here, a value of 0.8, for example, indicates that the model would prefer to associate a randomly chosen image from ImageNet with the label ``Male'' over the label ``Female''.}
%     \label{tab:rb}
%     \begin{tabularx}{\columnwidth}{@{}c|YYY@{}}
%     \toprule
%     \bf Model&\bf1B &\bf10B &\bf100B\\
%     \midrule
% B & 83.2&84.5&85.2
% \\
% L & 88.2&86.4&85.5\\
% H & 86.8&85.0&86.6\\
% \bottomrule
%     \end{tabularx}
% \end{table}

\begin{table}[h]
\begin{tabularx}{\columnwidth}{c|YYY@{}}
    \toprule
    \bf Model&\bf1B &\bf10B &\bf100B\\
    \midrule
B & 83.2&84.5&85.2\\
L & 88.2&86.4&85.5\\
H & 86.8&85.0&86.6\\
\bottomrule
\end{tabularx}
\captionof{table}{Representation bias w.r.t. gender (see Section~\ref{sect:results}). Here,  values [\%] indicate how often the model prefers to associate a random  image with the label ``Male'' over ``Female''.} \label{tab:rb}
\end{table}



\paragraph{Association Bias.} Second, Figure~\ref{fig:ab} shows the association bias in SigLIP-H/14 between gender and occupation as we scale the data from 10 to 100 billion examples. Specifically, we plot the probability that the model would prefer a particular occupation label, such as ``{\fontfamily{lmodern}\selectfont secretary}'' over another label, such as ``{\fontfamily{lmodern}\selectfont manager}'' when images correspond to males or females. In this evaluation, we use the Fairface~\cite{karkkainen2021fairface} dataset. The labels we compare are: ``{\fontfamily{lmodern}\selectfont librarian}'' vs. ``{\fontfamily{lmodern}\selectfont scientist}'', ``{\fontfamily{lmodern}\selectfont nurse}'' vs. ``{\fontfamily{lmodern}\selectfont doctor}'', ``{\fontfamily{lmodern}\selectfont housekeeper}'' vs. ``{\fontfamily{lmodern}\selectfont homeowner}'', ``{\fontfamily{lmodern}\selectfont receptionist}'' vs. ``{\fontfamily{lmodern}\selectfont executive}'' and ``{\fontfamily{lmodern}\selectfont secretary}'' vs. ``{\fontfamily{lmodern}\selectfont manager}''. Again, we do not see a reduction in association bias by simply increasing the size of the training data. %Full results are in Appendix~\ref{appendix:ab}.

%Additionally, we are unable to evaluate cultural diversity and fairness in PaliGemma's transfer tasks due to the lack of appropriate benchmarks. This is an open question that we hope to address in the future.

\paragraph{Performance Disparity.} Finally, one common definition of fairness in machine learning is maintaining similar performance across different groups. See, for instance,~\citet{dehghani2023scaling} and the related notions of ``Equality of Opportunity'' and ``Equalized Odds''~\cite{hardt2016equalityopportunitysupervisedlearning}. Table~\ref{tab:perf_disparity} show that scaling the data to 100 billion examples improves performance disparity, which is consistent with the improvement in cultural diversity.

%  to show on top of page
% \begin{table}[h]
    \centering\scriptsize
    \begin{tabularx}{\columnwidth}{l|YYY@{}}
    \toprule
    \bf Model & \bf 1B & \bf 10B & \bf 100B\\ \midrule
    &\multicolumn{3}{c}{\em 0-shot Dollar Street}\\[2pt]
B & 32.5 & 29.9 & \bf29.0\\
L & \bf29.7 & 29.8 & 30.4 \\
H & 32.2 & 33.0 & \bf32.1\\
\midrule 
    &\multicolumn{3}{c}{\em 0-shot GeoDE}\\[2pt]
B & 4.7 & 5.5 & \bf4.4\\
L & 3.2 & 4.0 & \bf2.8 \\
H & 3.6 & 3.0 & \bf2.7\\
\bottomrule
 
    \end{tabularx}
    \caption{Performance disparity (lower is better) for models pretrained on 100B seen examples of different data scales. Pretraining on 100B examples tends to lower disparity.}
    \label{tab:per_disp_mini}
\end{table}
% \FloatBarrier

\begin{table*}[h]
    \centering\scriptsize
    \caption{Performance disparity results for various SigLIP models pretrained on 100 billion seen examples of 1B, 10B, and 100B datasets. Here, disparity corresponds to the maximum gap across subgroups in Dollar Street (by income level) and GeoDE (by geographic region). Pretraining on 100B examples tends to improve disparity overall.}
    \label{tab:perf_disparity}
    \begin{tabularx}{2\columnwidth}{ll|YYYYYY|Y}
    \toprule
    \bf Model & \bf Data Scale &\multicolumn{6}{c}{\bf Performance per Subgroup} & \bf Disparity\\ \midrule
    \multicolumn{8}{c}{\em 0-shot Dollar Street}\\[2pt]
    & & \bf 0-200	& \bf 200-685	& \bf 685-1998	& \bf $>$1998
    & & & \\ \midrule
B&1B&29.4&43.9&56.5&62.0&&&32.5\\
B&10B&31.6&44.0&55.4&61.5&&&29.9\\
B&100B&32.0&44.3&56.3&61.0&&&\bf29.0\\[3pt]
L&1B&33.7&44.7&57.3&63.4&&&\bf29.7\\
L&10B&35.7&47.8&58.7&65.5&&&29.8\\
L&100B&33.7&46.6&59.5&64.1&&&30.4\\[3pt]
H&1B&32.3&44.9&58.4&64.5&&&32.2\\
H&10B&33.9&46.3&58.6&66.9&&&33.0\\
H&100B&34.1&48.2&62.2&66.1&&&\bf32.1\\ \midrule

    \multicolumn{8}{c}{\em 0-shot GeoDE}\\[2pt]
    & & \bf Africa	& \bf Americas	& \bf East-Asia	& \bf Europe & \bf South-East Asia & \bf West Asia
    & \\ \midrule
B&1B&89.4&92.1&91.8&94.1&92.5&93.4&4.7\\
B&10B&88.4&91.8&91.4&94.0&92.2&93.0&5.5\\
B&100B&88.8&91.4&91.0&93.3&91.7&92.2&\bf4.4\\[3pt]
L&1B&92.0&94.0&94.0&95.2&94.2&94.9&3.2\\
L&10B&91.8&94.4&94.0&95.8&94.2&94.7&4.0\\
L&100B&93.5&95.1&95.4&96.2&95.0&95.8&\bf2.8\\[3pt]
H&1B&91.5&94.4&94.7&95.2&94.1&94.5&3.6\\
H&10B&93.4&95.4&95.0&96.5&95.1&95.6&3.0\\
H&100B&93.6&95.1&95.3&96.3&95.2&95.8&\bf2.7\\

 \bottomrule
    \end{tabularx}
\end{table*}


\subsection{Transfer To Generative Models}
\label{sec:transfer}

\begin{table}[h!]
\centering
\footnotesize

% Note: We removed 1b result to avoid confusion to readers. See https://docs.google.com/document/d/1YxRpUO7elSaviOQ5XIXtnUWajgYAF7FfRO0vmXQCUtU/edit?resourcekey=0-pPjeeIrYEXRuvnuBFZn5Uw&tab=t.0#heading=h.gaczi2wqv0go.
\begin{tabular}{p{0pt}l|rrrrr}
\toprule
& Data & Semantics & OCR & Multiling & RS & Avg \\
\midrule
% % source: https://docs.google.com/spreadsheets/d/1W5_VNitkO6k-HSKBV6sGX81EGXgma877m0gGsg8zPzw/edit?resourcekey=0-2zH_U5z5kL9I5Nnhg7SChQ&gid=1972601917#gid=1972601917
\includegraphics[width=8pt]{images/snowflake_2744-fe0f.png} & 1B & 76.0 & 66.8 & 67.0 & 92.3 & 73.6 \\
\includegraphics[width=8pt]{images/snowflake_2744-fe0f.png} & 10B & 75.4 & 65.2 & 66.3 & 91.9 & 72.7 \\
\includegraphics[width=8pt]{images/snowflake_2744-fe0f.png} & 100B & 76.4 & 67.0 & 66.9 & 92.1 & 73.9 \\
\includegraphics[width=8pt]{images/fire_1f525.png} & 1B & 77.1 & 69.5 & 66.9 & 92.0 & 75.1 \\
\includegraphics[width=8pt]{images/fire_1f525.png} & 10B & 76.4 & 66.9 & 66.0 & 91.8 & 73.7 \\
\includegraphics[width=8pt]{images/fire_1f525.png} & 100B & {77.2} & {70.0} & {67.0} & {91.8} & {75.3} \\
\bottomrule
\end{tabular}

\caption{
The PaliGemma transfer results of ViT-L/16 models pretrained on 10B and 100B examples, with both frozen (top) %({\includegraphics[width=8pt]{images/snowflake_2744-fe0f.png}}) 
and unfrozen (bottom) %({\includegraphics[width=8pt]{images/fire_1f525.png}}) 
vision components. Results are aggregated.
}
\label{tab:transfer_avg}
\end{table}


We use PaliGemma~\citep{beyer2024paligemma} with both frozen and unfrozen vision component to assess the transferability of our vision models, which were contrastively pre-trained on datasets of different scales. In Table~\ref{tab:transfer_avg}, when taking the noise level into consideration, we do not observe consistent performance gains across downstream tasks as we scale the pre-training dataset. More details can be found in Appendix~\ref{appendix:transfer}.


%\paragraph{Recognizing Sensitive Attributes.}
%Finally, we also report the performance of the models in recognizing sensitive attributes, following a similar evaluation in~\citet{radford2021learning}. We report the accuracy in predicting perceived gender in Fairface~\cite{karkkainen2021fairface} and predicting perceived race in UTK~\cite{utkface_url}. Overall, we observe that scaling the data to 100 billion examples improves this aspect of fairness as well. Table~\ref{tab:fairness_pred} provides the full results. We do not observe a particular pattern in this type of evaluation.
%\begin{table}[t]
    \centering\scriptsize
    \begin{tabularx}{\columnwidth}{@{}ll|YY@{}}
    \toprule
    \bf Model & \bf Data Scale & \bf Gender & \bf Race\\ \midrule
B	&	1B	& 91.0	& 58.6\\
B	&	10B	& 91.7 &	\bf59.5\\
B	&	100B &	\bf91.9	& 53.1\\[3pt]
L	&	1B	& 0.94.8	& \bf55.4\\
L	&	10B	& 93.9	& 53.1\\
L	&	100B &	\bf95.0	& 54.0\\[3pt]
H	&	1B	& 94.5	& \bf54.5\\
H	&	10B	& \bf95.4	& 54.3\\
H	&	100B	& \bf95.4	& 50.2 \\
 \bottomrule
    \end{tabularx}
    \caption{Accuracy in recognizing sensitive attributes using Fairface and UTK datasets. See Section~\ref{sect:results} for details.}
    \label{tab:fairness_pred}
\end{table}


%%%%%%%%%%%%%%%%%%%%%%%%%%%%%%%%%%%%%%%%%%%%%%%%%%%%%%%%%%%%%%%%%%%%%%%%%%%%%%%%
%%% DISCUSSION
\section{Discussion}
\subsection{Case Study}


Fig. \ref{fig:casestudy} shows 2-D UMAP \cite{mcinnes2020umapuniformmanifoldapproximation} projections of embedding vectors for PetClinic Microservices \cite{microapps2024petclinic} using VoyageAI, ME-unixcoder-340K, and ME-llm2vec-340K. ME-unixcoder and ME-llm2vec show clearer microservice clusters compared to VoyageAI and Fig. \ref{fig:mexample}. For instance, \textit{API-Gateway} service classes are split in VoyageAI's representation but closer in the other models. ME-llm2vec demonstrates the closest grouping within microservices and clearest separation between them. In fact, ME-llm2vec's figure shows only 6 clear outliers which we review in detail and display their names and neighbors.



The two \textit{MetricConfig} classes, \textit{ResourceNotFoundException} and \textit{CacheConfig} lack domain-specific terms since they are utility classes, which highlights the importance of separating them from domain-related ones during the decomposition. However, ME-llm2vec was able correctly represent classes with even slight domain hints. For instance, most models struggle to differentiate between the nearly identical entry-point classes (e.g. \textit{ConfigServerApplication}), as seen in Fig. \ref{fig:mexample} and \ref{fig:casestudy} while ME-llm2vec managed to correctly place them within their services. On the other hand, the class \textit{PetRequest}, which was closer to \textit{API-Gateway} instead of \textit{Customers}, shows an intriguing outlier. Despite ME-llm2vec correctly matching the "Pet" related classes, it failed with \textit{PetRequest}. its function as a Request object, which is typically associated with the Gateway pattern, is a potential reason. Notably, ME-llm2vec successfully identified \textit{API-Gateway} classes, differentiating them from \textit{Customers}. We find this interesting because \textit{API-Gateway} includes classes representing various bounded contexts, often causing confusion in other models. ME-llm2vec recognized these classes' distinct purpose, grouping them together despite their diverse domains.

% Both \textit{API-Gateway} and \textit{Customers} services contain a "PetType" class. But in \textit{Customers}'s case, this class was closer to the "Specialty" class from \textit{Vets}, which is likely due to nearly identical source code they have. 

\subsection{Discussion}


We designed the analysis component to be as abstract as possible to accommodate the rapidly evolving representation learning landscape. As new and improved embedding models are published, they can be integrated with minimal effort. While our evaluation results show that with ME-LLM2Vec, we can generate highly cohesive and consistent decompositions, one of our objectives is to highlight the potential of Language Models in generating more efficient representations than traditional approaches for the decomposition problem. In fact, MonoEmbed is both a decomposition approach (when considering the full approach) and an embedding model (when using models such as ME-LLM2Vec). These models can be used to enrich existing decomposition approaches. For example, MicroMiner's CodeBERT \cite{trabelsi2023microminer} can be replaced with ME-LLM2Vec and the GNN based methods \cite{desai2021cogcn,yedida2023deeply,mathai2022chgnn,qian2023gdcdvf} can be extended by using ME-LLM2Vec as the encoder. In fact, it can be used as an additional representation type in approaches such as \cite{khaled2022hydecomp,qian2023gdcdvf}. These models can be even extended further by incorporating unstructured inputs (e.g. resources, configurations, documentation) and different PLs.




\subsection{Threats to Validity}
\subsubsection{Internal Validity}
Clustering algorithms and decomposition approaches have hyper-parameters that can affect performance on evaluation benchmarks. To mitigate this threat, we compared their performance with different hyper-parameter inputs across a varied set of evaluation applications.

\subsubsection{External Validity}
To address the threat of our approach to generalize on monolithic applications and PLs, we used a large set of monolithic and microservices applications from related work \cite{kalia2021mono2micro,khaled2022hydecomp,yedida2023deeply,jin2021fosci} to benchmark decomposition approaches. 

\subsubsection{Construct Validity}
This threat can potentially be in the form of the evaluation metrics used in our experiments. In order to mitigate this threat, we employ established metrics in supervised learning tasks (RQ1-3) and different metrics from decomposition research \cite{khaled2022hydecomp,kalia2021mono2micro,jin2021fosci,yedida2023deeply,mathai2022chgnn} (RQ4). 

%%%%%%%%%%%%%%%%%%%%%%%%%%%%%%%%%%%%%%%%%%%%%%%%%%%%%%%%%%%%%%%%%%%%%%%%%%%%%%%%


\addtolength{\textheight}{-3cm}   % This command serves to balance the column lengths
                                  % on the last page of the document manually. It shortens
                                  % the textheight of the last page by a suitable amount.
                                  % This command does not take effect until the next page
                                  % so it should come on the page before the last. Make
                                  % sure that you do not shorten the textheight too much.

%%%%%%%%%%%%%%%%%%%%%%%%%%%%%%%%%%%%%%%%%%%%%%%%%%%%%%%%%%%%%%%%%%%%%%%%%%%%%%%%


%%%%%%%%%%%%%%%%%%%%%%%%%%%%%%%%%%%%%%%%%%%%%%%%%%%%%%%%%%%%%%%%%%%%%%%%%%%%%%%%
% \section{ACKNOWLEDGMENTS}

% The authors gratefully acknowledge the contribution of reviewers' comments, etc. (if desired). Put sponsor acknowledgments in the unnumbered footnote on the first page.


%%%%%%%%%%%%%%%%%%%%%%%%%%%%%%%%%%%%%%%%%%%%%%%%%%%%%%%%%%%%%%%%%%%%%%%%%%%%%%%%

%%%%%%%%%%%%%%%%%%%%%%%%%%%%%%%%%%%%%%%%%%%%%%%%%%%%%%%%%%%%%%%%%%%%%%%%%%%%%%%%
\section*{ETHICAL IMPACT STATEMENT}
Our research focuses on fairness-aware and robust face parsing for generative AI, addressing biases in segmentation models and their downstream impact on generative synthesis. While our work aims to mitigate demographic disparities and improve model resilience, we acknowledge potential ethical concerns related to dataset biases, misuse, and unintended societal impact.

\textbf{Potential Risks and Negative Impacts:}
Face parsing and generative models can be misused for unethical applications, such as surveillance, deepfake generation, or reinforcing demographic stereotypes. Despite our efforts to improve fairness, residual biases in datasets (e.g., CelebAMask-HQ) may persist, potentially leading to unequal model performance across demographic groups. Additionally, robustness improvements could inadvertently be leveraged to enhance adversarial facial synthesis, raising concerns about identity fraud.

\textbf{Risk-Mitigation Strategies:}
To mitigate these risks, we employ fairness-aware multi-objective training to reduce demographic disparities and systematically evaluate robustness against real-world perturbations. Our methodology prioritizes transparency and reproducibility—our dataset choices, fairness metrics, and evaluation protocols will be made publicly available to facilitate scrutiny and improvement. Furthermore, we emphasize ethical use cases, discouraging applications in deceptive or harmful generative AI practices.

\textbf{Human Subject and Data Ethics:}
Our study does not involve human subjects or personally identifiable information (PII). The datasets used (CelebAMask-HQ) are publicly available, and we adhere to all ethical guidelines concerning their use. While we acknowledge that publicly available datasets can contain biases, our methodology explicitly addresses this issue through fairness-aware training and demographic evaluation.

\textbf{Future Ethical Considerations:}
Future research should extend fairness-aware segmentation to more diverse and representative datasets, ensuring broader applicability and minimizing demographic bias. Additionally, interdisciplinary collaborations with ethicists, policymakers, and domain experts will be crucial to guiding responsible deployment and regulation of AI-generated content.

By integrating fairness and robustness into face parsing, we aim to contribute to the development of ethical, bias-aware AI models that enhance inclusivity and reliability in computer vision applications.
%%%%%%%%%%%%%%%%%%%%%%%%%%%%%%%%%%%%%%%%%%%%%%%%%%%%%%%%%%%%%%%%%%%%%%%%%%%%%%%%

% References are important to the reader; therefore, each citation must be complete and correct. If at all possible, references should be commonly available publications.

{\small
\bibliographystyle{ieee}
\bibliography{egbib}
}

\end{document}
