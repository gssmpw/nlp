\section{Results: Qualitative Analysis}



\subsection{Top Rated Images for Instagram, 4o, and HumorSkills for Target Images}

% HumorSkills excels at creating captions involving relatable narratives. 
All three generation techniques were able to produce captions that some people found funny. Figure \ref{top_score} shows the top-rated captions for each of the three techniques. 
The top-rated IG comment is “American instruments,” which pokes fun at US gun culture. This caption is good because it's short but also an excellent social critique. The GPT-4o caption for this image is also apt: “Which one shreds harder?” It understands the image and associates the two objects through word play ("shred"). The HumorSkills caption for this image draws an analogy to video games. It's not as sharp a critique as the IG captions, but it's very relatable and critiques the silly weapons available in many video games. Although this association might not be clear to everyone, it will be to Gen-Z.

% , however, stands out by framing the situation in the context of gaming with “me picking weapons in a video game.” This shifts the focus from simply observing the objects like IG and GPT-4o captions to engaging the audience in a narrative, making it more relatable to gamers, particularly Gen Z, by immersing them in a scenario where unconventional items are chosen as weapons in a game.

The top-rated GPT-4o caption says "running on fumes, literally", for an image with a runner smoking a cigarette. This is apt and connects the odd parts of the image (running and smoking). It again uses wordplay. 
The IG caption was the highest rating for this image, with an unexpected video game reference of ``bro playing on max difficulty.''
The HumorSkills caption did least well for this image, it uses dark humor to critique the runner for smoking and point out the absurdity: “They really said ‘let’s race to the hospital.’” 

% using dark humor to emphasize the absurdity of smoking during a race. By tapping into the visual contradiction and the deadly consequences of smoking, HumorSkills excels at offering a more pointed and edgy interpretation by utilizing its visual extraction and ideation skills to notice the counterintuitive nature of running to be healthy and that smoking kills.

The top-rated HumorSkills caption was the highest-rated caption in the entire study. It reads ``minecraft character looking ahh.'' (The word "ahh'" is Gen-Z slang roughly meaning "ass".) Although this might not be broadly funny, it is funny to Gen-Z because of this insider slang, because it's short, and because it is a critique of his looks related to Minecraft. In the online Minecraft videogame, characters are made out of blocks, like virtual legos, and their heads are often square, which is reminiscent of how the face in this photo to cropped.
The IG caption (``bro ready for his job interview at mojang'') also relates the image to Minecraft (Mojang the developer of Minecraft). Gen-Z grew up on Minecraft, and jokes about it resonate. The GPT-4o also makes fun of his hairstyle but relates it to cropping an image, rather than Minecraft, which is not as resonant. 

% Lastly, HumorSkills' fine-tune allows for the style of captions to be Gen Z. In the top-rated HumorSkills comment image, which features a selfie of a man with a square-shaped beard, Instagram’s caption, “Bro ready for his job interview at Mojang,” ties the man’s appearance to the creators of Minecraft, making a direct connection that implies that this man wants a job at Mojang badly enough to style his beard in the style of Minecraft. GPT-4’s caption, “When you accidentally click ‘Crop’ in real life,” references how his sharp-edged beard looks like it was cropped using a digital tool. HumorSkills, however, excels past IG and GPT-4o by embracing Gen Z slang with “Minecraft character looking ahh,” which uses Gen Z slang through the phrase "ahh" (ahh is Gen Z slang replacing the word "ass") to point out the beard’s similarity to Minecraft characters. This caption is more in tune with Gen Z culture and language, tapping directly into the visual extraction and playful tone that resonates with younger audiences.This caption leans into Gen Z style and topics utilized in the fine-tune and incorporates the visual extraction skills to observe that the beard in the image is a square, calling out the resemblance to Minecraft characters. 


\begin{figure*}[h]
    \centering
    \includegraphics[width=.95\textwidth]{fig/Lingo.png}
    \caption{Images across all 3 datasets with Gen Z slang. Instagram (left), Flickr (center), Museum Art (right)}
    % \Description{DynEx's Implementation User Interface with important components called out.}
    \label{fig:visual_images}
\end{figure*}





\subsection{Gen Z Slang and Fine-Tuning}
Figure \ref{fig:visual_images} shows three images from across all three datasets. Although the HumorSkills captions are rated slightly better than the GPT-4o jokes, they are fairly good.

Continuing with examples from the Instagram dataset shows that HumorSkills can easily replicate Gen Z slang through its fine-tuning. The first image, featuring a man with a long tongue, has the generated HumorSkills caption “bro looks like a living snapchat filter.” This caption plays on the exaggerated feature of his tongue — and ties it directly to Snapchat. Snapchat filters are an embedded part of Gen Z’s social media use, making the caption incorporate phrases familiar to Gen Z, such as Snapchat. This is ignored in the GPT-4o caption, "Trying to lick the last bit of ice cream from the cone." While eating ice cream is commonplace for many, Snapchat is a phrase more unique to Gen Z.

Even in a non-target dataset, such as the Flickr dataset, HumorSkills’ ability to produce Gen Z slang can be seen in the second image, displaying a dog standing defiantly in the middle of an urban street. The caption, “why does this look like a GTA loading screen?” cleverly references Grand Theft Auto (GTA), a highly popular action-adventure game among Gen Z. In GTA, loading screens often feature characters and animals in striking, dramatic poses against the backdrop of the game’s open-world urban environment. These screens typically display gritty street scenes filled with elements of crime, defiance, and tension, mirroring the tone of the game itself.

The image of the street dog captures this same defiant, street-wise aesthetic that players recognize from the game, evoking the vibe of the cityscapes and characters GTA is known for. HumorSkills recognizes this visual parallel and ties it to a cultural touchstone that is deeply ingrained in Gen Z’s gaming experiences. The phrase “GTA loading screen” instantly brings to mind the iconic visuals and rebellious energy of the game, enhancing the humor through a shared cultural reference and common phrase popular among Gen Z. On the other hand, the GPT-4o caption, "This dog is clearly in charge of the neighborhood," while relevant to the dog and urban environment, does not contain a cultural or popular phrase among Gen Z like GTA.

In the museum art dataset, for more abstract images, such as the third image with bold, chaotic strokes and splashes of color scattered across the canvas, the caption, “adhd be like,” uses the chaotic and unfocused nature of the painting as a metaphor for adhd. This humor also resonates with Gen Z, who are more comfortable discussing mental health and will throw around the term "adhd" more. The abstract painting lends itself to this caption, as it reflects the disordered, busy, and scattered hardship associated with the condition that Gen Z mentions a lot. The GPT-4o caption, "Me after two cups of coffee and no plan for the day," also addresses a similar condition of feeling chaotic after drinking coffee, but does not invoke a relatable condition relating to mental health like "adhd" that is popular among Gen Z.

\begin{figure*}[h]
    \centering
    \includegraphics[width=.95\textwidth]{fig/Narrative.png}
    \caption{Images across all 3 datasets with narrative generation. Instagram (left), Flickr (center), Museum Art (right)}
    % \Description{DynEx's Implementation User Interface with important components called out.}
    \label{fig:narratives}
\end{figure*}

\subsection{Using Narrative Extrapolation for Relatability}
While humor generation with Gen Z slang seems natural on images HumorSkills was built for and trained on, HumorSkills can generate narratives for images where it is not obvious there is Gen Z slang available to produce into a humorous caption. 

The first image from Instagram shows a comically large telescope on top of a rifle as if it were a sniper scope. With no apparent narrative for the image, HumorSkills generates one by assuming the event in which the US Space Force has faced budget cuts with the caption, "That's one way to deal with the Space Force budget cuts," implying that the telescope, used in space, and the rifle, used by the military, has been jury-rigged together for use by the US Space Force due to budget constraints. Without the narrative generation, GPT-4o produces a relevant caption about the telescope, "he really said 'scope out the stars'," but it does not contain a narrative or core conflict to apply to the image. 

The second image shows a group of medical professionals gathered around a laptop, seemingly discussing something serious. The caption, “Medical staff having a discussion on whether it’s worth it to save the uninsured man,” comments on the state of healthcare, particularly in countries like the United States, where lack of insurance can impact the quality of care. This caption utilizes narratives to insert a relatable critique of the healthcare system, framing the scene as if the doctors are debating the value of a human life based on their insurance status. The humor here comes from how HumorSkills can create an extreme suggestion using a narrative play on a real-world issue many can relate to the cost of healthcare. The GPT-4o caption, "When the surgery team gets a crash course in the break room," while addressing the humorous nature of expert surgeons receiving lessons while on break - suggesting the incompetency of the surgeons, does not establish a coherent narrative with a clear conflict like saving an uninsured man like in the HumorSkills example. Rather, the issue is that the images imply the surgeons are incompetent (who are all in the image) rather than another character (the man) with a conflict (he is uninsured). 

The third image shows an abstract painting with a web of intersecting lines. The caption, "How your charger looks when you leave for 5 seconds," utilizes a narrative of getting up after using an electronic device such as a mobile phone or computer with a charger attached, only to come back to find it twisted and tangled. While there are no physical wires in the painting, HumorSkills generates a hypothetical narrative as if the lines are wires attached to a charger. The GPT-4o caption, "When you attempt abstract art and accidentally create a web of confusion," while also a narrative, pokes fun at the attempt of the actual painter trying to create abstract art only for it to be confusing. The HumorSkills caption is unique in that the narrative is separate from the actual image by generating a scenario of the painting representing someone's charger wires for their devices. The GPT-4o narrative focuses on a narrative for the actual image itself and the identity of its painter.

\begin{figure*}[h]
    \centering
    \includegraphics[width=.95\textwidth]{fig/Visuals.png}
    \caption{Images across all 3 datasets with notable visual extractions. Instagram (left), Flickr (center), Museum Art (right)}
    % \Description{DynEx's Implementation User Interface with important components called out.}
    \label{fig:visual_system}
\end{figure*}

\subsection{Visual Detail Extraction and Visual Humor Ideation}

In the first image, we see a man eating food with a large forehead. The caption, "Foreheard so high, it's getting its own zip code," refers to the abnormally large forehead of the man, that the height of the forehead is so far that it warrants its own area code. While GPT-4o most definitely can pick up on this visual abnormality, it often drifts towards non-significant visual details and fixates on them, as seen in the caption, "That moment when you're caught off guard mid-snack, and now it's awkward," fixing on the man-eating food while being photographed while chewing. GPT-4o definitely will pick up on the man's forehead, but because it is not looking for visual oddities that may be considered funny (visual humor ideation) or prioritizing visual elements to the scene, it often reverts to a less prominent visual element. 

In the second image, we see a man with a mustache sitting alone with a beer. The caption, “dude looks like he’s about to write a manifesto,” plays on the stereotype of a brooding, isolated man, on the verge of drafting some kind of extreme ideological statement like many extremist historical figures. This is an example of visual extraction, where the system uses the man’s appearance of the mustache, brooding imagery, longing face, and demeanor to label an assumption about the man's political beliefs. The GPT-4o image caption, "That 'I came to the bar for solitude' vibe" directly addresses the solitude nature of the man in the photo without addressing the brooding emotion of the man, the mustache, and other elements like in the HumorSkills caption that helps assume a political ideology or belief system.

In the third image, it is a painting of a Japanese woman in a kimono lying flat reading a book. HumorSkills generated, "bruh looks like they're waiting for a text back from the shogun," highlighting that HumorSkills picks up the cultural visuals of the image from Japan by referencing the term shogun. While GPT-4o's caption is relatable with, "That 'I'll start my homework in 5 minutes' energy," it lacks the visual cue of a Japanese artwork and misses mentioning it in the image.

