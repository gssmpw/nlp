\section{Introduction}
% Motivation
Producing humor is a difficult yet quintessential endeavor that underpins the everyday human experience. Whether it is cracking a joke with coworkers, sending memes to friends, or playfully flirting with a romantic interest on a date, most people interface with humor every day in some capacity. We use humor to connect with people, to impress people, to point out absurdities, to make light of a bad situation, and to recognize universal struggles - big and small. 

% Problem
Humor has long been thought to be “too difficult" or "too human” for AI to generate. Humor is complex, and there is no single formula for it. Different people find different things funny at different times. 
Humor seems to require social skills to know one's audience's point of view, understand human relationships, see things from multiple points of view, and judge what is socially appropriate to make fun of. 
Humor requires cognitive and reasoning skills because most jokes are clever in some way - they don't just say the obvious, they construct facts in ways that bring out absurdities, contradictions, and surprises.
Humor is a creative act, and it benefits from making keen observations, trying multiple angles, and multiple wordings.
Getting a person to do all these things is hard enough. Asking AI to do it is even harder.

% % Appraoch
However, generative AI has created new possibilities for computational humor. From training on large corpora of text, large language models have a broad base of knowledge and an implicit understanding of what humor is. However, even state-of-the-art models are known for being disappointing in their humor abilities~\cite{gptnotfunny}. Theories of humor indicate that there are patterns to humor~\cite{humor_Raskin2009}, which makes it theoretically possible for generative AI to implicitly learn these patterns, and its attempts at humor show that it generally understands the structure. And yet humor is complex enough that knowing the structure is not enough - it also requires human-like skills such as understanding people, culture, and the human experience. 
% \color{red}
% Other skills like cognitive, social, and creative skills have been used to allow generative AI to make progress on hard tasks.
\color{black}
% We aim to test whether a combination of these skills can be put together to tackle AI humor generation.

% % Solution
We test whether AI-generated humor that uses creative, social, and cognitive skills can come close to human performance. As a research context, we study the specific problem of generating humorous captions for images posted to Instagram. This is a popular humor type within Gen Z that is widely made and widely appreciated by a well-defined audience. There is ground truth (in the form of upvotes) for what captions the audience finds funny. It is a fairly complex form of humor that uses both visual and language skills to generate, it applies to a wide variety of input images and requires implicit Gen Z cultural knowledge to generate. 

Our method of generating humor uses three steps, following the divergent and convergent stages often used in cognitive models of creativity. 
First is an observation phase where AI takes in the image and makes careful observations of things in the photo.
Next is a divergent phase where AI generates multiple possible humorous angles. We have two strategies for this: one that poses humor strictly related to the image content and one that departs from the image content by 
% % elaborating on it and 
bringing in relatable social conflicts that are analogous to it.
Next, a generation phase where based on the angles, we generate 30 possible captions. As with most ideation, we favor quantity and diversity over quality. 
Lastly, a ranking phrase where a ``Gen-Z humor expert'' agent rates the generated captions and selects the 5 best to return. 

We compare the HumorSkills generation method to two other sources of captions: 1) the five most upvoted comments on Instagram for the same image and 2) a state-of-the-art VLM (GPT-4o) with only prompt engineering. Through a humor rating survey of 20 images with 15 captions each, we found that HumorSkills is significantly funnier than the VLM baseline, and as funny as the top Instagram comments. HumorSkills captions were rated only 0.08 points lower on a 5-point scale with p=0.053, thus making HumorSkills not statistically less funny than the best human captions. 
% which not statistically different at the 0.05 level, thus making HumorSkills not

We end with a discussion of how imbuing generative AI with multiple human-like skills can enhance its ability to do complex human communication. 
% % Based on a study of target audience rating 