
% \section{Methods}

\section{Humor Rating Study}
\subsection{Methods}
We ask people to rate humorous captions humor for images taken from 8 popular Instagram (IG) humor captioning accounts. For each image, users were shown 15 captions
1) The top 5 most upvotes captions from IG
2) 5 captions from GPT-4o
3) 5 captions from HumorSkills. 

Data was collected from an online survey. Users would see an image and then rate 15 captions for it on a scale of 1 to 5 where 1 was "not funny", 3 was "somewhat funny" and 5 was "very funny". For each user, the images were presented in a random order and the 15 captions for each image were also shown in a random order to avoid any possible ordering effects. Users were not told the condition of each caption. 

The survey was distributed through email announcements to clubs and classes at a local university. Users were not told that AI was involved in the humor writing. The announcements advertised as being for people who identify as "Gen-Z" and who "like Instagram caption humor". These qualifications were added to ensure the raters were part of the target audience, and most university students fit this demographic.  The survey took about 30 minutes and users were paid \$10.

Our Hypotheses were that:

H1) Our system would be rated as funnier than GPT-4o, but that 

H2) Our system would be rated as funny as the top-rated Instagram captions.

\subsection{Results}
In total, 32 people responded to the survey, 5 male, 9 female, and 18 declined to say their gender. 14 of 32 respondents were aged 18-25. 1 were aged "25+", and 17 declined to say their age group.

To test the relative humor score of the three conditions, we ran a Generalized Linear Mixed Model (GLMM) to estimate the funniness rating of each condition. The response types were ordinal numbers (ratings), the fixed effects were the caption type (IG, GPT-4o, System), and the random effects were individual rater IDs and image IDs.

\subsubsection{H1: HumorSkills vs. GPT-4o}
On the whole, captions from our system were rated as 2.27 (out of 5) for funniness. Captions written by GPT-4o were rated less funny (by 0.213 points), which is statistically significant at a p<0.0001 level. This indicates our system writes more humorous captions than a state-of-the-art VLM with prompt engineering. 
Results are shown in figure \ref{glmm_1}. 
Thus \textbf{H1 is supported. Our system is funnier than GPT-4o.}

\subsubsection{H2: HumorSkills vs. top IG captions}
On the whole, the top IG captions were rated only slightly better (0.08 points on a 5-point scale).
But the difference was not significantly different at the 5\% level. However, it was very close (p=0.053)
% The p-value is 0.053, technically not significant at the 0.05 level. 
% The result is statistically significant at the 10\% level (p = 0.053) but does not meet the conventional 5\% threshold
This indicates that the system is competitive with the top IG captions. This effect could disappear with more data, but this dataset had over 6000 observations, and the 0.08 difference in average score is only 2\% of the 1-5 scale.
Thus \textbf{H2 is supported. Based on our sample, our system was rated as funny as the top 5 Instagram comments.} Or more precisely, our system was not statistically less funny than the top IG captions. 

\begin{table}
\caption{Mixed Linear Model Regression Results}
\label{glmm_1}
\begin{center}
\begin{tabular}{llll}
\hline
Model:            & MixedLM & Dependent Variable: & rating       \\
No. Observations: & 6015    & Method:             & REML         \\
No. Groups:       & 402     & Scale:              & 1.6212       \\
Min. group size:  & 2       & Log-Likelihood:     & -10271.0088  \\
Max. group size:  & 15      & Converged:          & Yes          \\
Mean group size:  & 15.0    &                     &              \\
\hline
\end{tabular}
\end{center}

\begin{center}
\begin{tabular}{lrrrrrr}
\hline
                        &  Coef. & Std.Err. &      z & P$> |$z$|$ & [0.025 & 0.975]  \\
\hline
Intercept (HumorSkill rating)             &  2.273 &    0.040 & 56.601 &       0.000 &  2.194 &  2.351  \\
GPT-4o rating & -0.213 &    0.040 & -5.285 &       0.000 & -0.291 & -0.134  \\
Top5 IG rating &  0.078 &    0.040 &  1.934 &       0.053 & -0.001 &  0.157  \\
Group Var               &  0.323 &    0.025 &        &             &        &         \\
\hline
\end{tabular}
\end{center}
\end{table}


\subsection{HumorSkills performance on non-target images}
The HumorSkills approach was designed and fine-tuned on a particular type of humor - Instagram captioning. It is possible that this approach is not generalizable, and too well tailored to one domain. The Instagram Caption Challenge is selected by the posters for particular qualities, which aren't stated, but a consistent theme is that the images are already interesting and unusual.  They almost always include people interacting. However, other types of images could be harder to caption humorously. We test two other sources of images to see how well HumorSkills could caption those images compared to GPT-4o.

1. \textbf{Camera roll images} -- the everyday photos people take -- typically contain people, but the images are not necessarily interesting in their subject or composition. 

2. \textbf{Museum Art}. Art is very different than photos. Although some of it depicts people doing things. Most of those things are historical, and not particularly relatable. Some paintings are abstract, like Jackson Pollock's, do not concretely depict any discernible subject. Additionally, many of the art pieces are not paintings but objects like vases or chairs. Any image without people are challenging to make relatable and create a narrative for.

We randomly selected 30 images from the Flickr Image Dataset\cite{young2014image} and 30 museum art images to test on our target audience. We take 15 images from the Museum of Modern Art (The MoMA) and 15 images of art from from the Metropolitan Art Museum (The Met). For every image, we generate 5 HumorSkills captions and 5 GPT-4o captions. We create two separate surveys for rating the captions - one for art and only for camera roll images. As before, we randomize the order of the images and the captions to mitigate ordering effects. We recruited raters the same as for the previous study.

We hypothesize that HumorSkills' captions will be rated more funny than GPT-4o captions. We did not compare to top-ranked human-written captions because these images were not taken from Instagram.

\subsection{Results: Camera Roll Humor Ratings}
As before, we ran a GLMM regression with image ID and user ID as random effects. There were a total of 4765 ratings. There were a total of 21 respondents, there were a total of 10 female, 4 male, and 7 declined to say their gender. 12 respondents were aged 18-25, 2 were aged "25+", and 7 declined to say their age group.

The results show that HumorSkills is 0.291 points higher than GPT-4o (2.19 vs. 1.9 out of 5). This difference is statistically significantly higher at the 0.05 level (p=0.022). Results are shown in Table \ref{glmm_flickr}.


\begin{table}
\caption{FlickrImage GLMM Regression Results}
\label{glmm_flickr}
\begin{center}
\begin{tabular}{llll}
\hline
Model:            & MixedLM & Dependent Variable: & Rating       \\
No. Observations: & 4765    & Method:             & REML         \\
No. Groups:       & 30      & Scale:              & 0.5792       \\
Min. group size:  & 149     & Log-Likelihood:     & -6209.2910   \\
Max. group size:  & 170     & Converged:          & Yes          \\
Mean group size:  & 158.8   &                     &              \\
\hline
\end{tabular}

\vspace{0.5cm}

\begin{tabular}{lcccccc}
\hline
\textbf{Variable} & \textbf{Coef.} & \textbf{Std.Err.} & \textbf{z} & \textbf{P>|z|} & \textbf{[0.025} & \textbf{0.975]} \\
\hline
Intercept (GPT-4o rating)      & 1.901  & 0.054  & 35.357 & 0.000 & 1.796 & 2.006 \\
HumorSkill rating   & 0.291  & 0.022  & 13.207 & 0.000 & 0.248 & 0.334 \\
Group Var      & 0.000  &        &        &       &       &       \\
User\_ID Var   & 1.260  & 0.116  &        &       &       &       \\
\hline
\end{tabular}
\end{center}
\end{table}

\subsection{Results: Museum Art Humor Ratings}
There were a total of 4044 ratings. There were a total of 17 respondents, 7 female, 3 male, and 7 declined to say their gender. 8 respondents were aged 18-25, 2 were "25+", and 7 declined to say their age group.

The results show that HumorSkills is 0.18 points higher than GPT-4o (2.18 vs. 2.00 out of 5). This difference is statistically significantly higher at the 0.05 level (p=0.023). Results are shown in Table \ref{glmm_art}. This shows that HumorSkills can perform out-of-domain images as these images do not have obvious humorous qualities to them. Nonetheless, HumorSkills still apply.

\begin{table}
\caption{Art Image GLMM Regression Results}
\label{glmm_art}
% \begin{center}
\begin{tabular}{llll}
\hline
Model:            & MixedLM & Dependent Variable: & Rating       \\
No. Observations: & 4044    & Method:             & REML         \\
No. Groups:       & 30      & Scale:              & 0.5241       \\
Min. group size:  & 129     & Log-Likelihood:     & -5106.3412   \\
Max. group size:  & 150     & Converged:          & Yes          \\
Mean group size:  & 134.8   &                     &              \\
\hline
\end{tabular}

\vspace{0.5cm}


\begin{tabular}{lcccccc}
\hline
\textbf{Variable} & \textbf{Coef.} & \textbf{Std.Err.} & \textbf{z} & \textbf{P>|z|} & \textbf{[0.025} & \textbf{0.975]} \\
\hline
Intercept (GPT-4o rating)      & 2.000  & 0.061  & 32.774 & 0.000 & 1.881 & 2.120 \\
HumorSkill rating   & 0.175  & 0.023  & 7.677  & 0.000 & 0.130 & 0.219 \\
Group Var      & 0.001  &        &        &       &       &       \\
User\_ID Var   & 1.387  & 0.147  &        &       &       &       \\
\hline
\end{tabular}
% \end{center}
\end{table}

\begin{figure}
    \centering
    \includegraphics[width=.8\textwidth]{fig/TOP_SCORE.png}
    \caption{Top Rated Image Captions (marked in green) for Instagram, GPT-4o, and HumorSkills and corresponding top scorers for each image. From left to right, the images contain 1) a guitar next to a machine gun hanging on a wall, 2) a man running a race while smoking a cigarette and, 3) a man with a long beard and his head cropped to a trapezoid shape.}
    % \Description{}
    \label{top_score}
\end{figure}



