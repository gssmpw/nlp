\section{Related Work}
\subsection{Humor Theory}
% Humor is a highly valued human skill. It is a sign of creativity and intelligence that enriches
% many domains: entertainment, advertising, social bonding, education, and journalism.

Humor is an intellectual challenge, humor has been studied by many great Western thinkers – Plato, Kant, Freud~\cite{sep_humor}.
Many theories have been proposed about why we have a sense of humor and what we find funny, and many cite social phenomena as the root cause.
% . Many of these theories use social phenomena as their root cause. 
Philosopher Daniel Dennett ~\cite{hurley2011inside} theorizes that the reason humans evolved humor was so that we had a positive incentive (laughing) to learn from the mistakes of others, and even our own. This fits with the benign violation theory of humor~\cite{Raskin2009} that says we find things funny when our expectations are violated in a way that is surprising, but not too threatening, disturbing, or wrong. This is why many jokes are insults that violate expectations about people's physical and mental well-being (insults), breaking social or cultural norms (rude behavior), or linguistic norms (word play). These jokes also help us sense what is unexpected, and learn from it.

Humor also has structure. Jokes typically have a setup that establishes expectations and a punchline that violates them. The semantic script theory of humor~\cite{Raskin2009} also describes how the setup leads listeners to infer one interpretation of the information, but the punchline leads them to infer a different interpretation that is also consistent with the facts (and the second interpretation of ``script'' often that violates a norm). Take this classic insult joke, \textit{"There are three types of people in the world. People who can count and people who can't."} The setup leads you to infer the person will list three types of people, but the punchline only lists two, which is unexpected. The second interpretation is that the person is bad at counting (insulting their own mental acuity). Constructing this joke required many cognitive skills beyond just knowing the setup-punchline structure - it required writing from someone's (mistaken) point of view, constructing a logical error someone could make, and finding a bland set-up line to misdirect listeners' attention to heighten the surprise. Beyond structure, it is unclear what is required of human or machine cognition to generate humor.

Although there is no definitive formula for humor, many humorists have described a set of techniques and skills for writing jokes. Key themes includes:

\textbf{Jokes should be relatable} ~\cite{Dean2000,Holloway2010, Vorhaus1994, Kaplan2013,Carter2001}, listeners have to related and empathize with them in some way.  
Jokes typically engage human emotions - fear, hope, curiosity, cringing, and other heightened states that raise the stakes that get us to listen and relate to the material  ~\cite{Carter2001}.  
Often jokes are for an in-group ~\cite{hurley2011inside} - using the shared knowledge and experiences of a group to create exclusive material which only that group could relate to.

% using observations and details or phrases from that ingroup helps it be relatable to that group.

\textbf{Jokes have details and observations}. Observing the difficulties and absurdities of life is a good way to find relatable material~\cite{Kaplan2013}.  But obvious absurdities tend not to be surprising, so jokes often come from observing details that others likely missed. ~\cite{Carter2001, Vorhaus1994}.

\textbf{Jokes contain narratives that include a point of view}.~\cite{Carter2001} 
Like stories and narratives, jokes often use multiple points of view to see a situation in a new and surprising way. 
% Like in stories, 
% considering multiple points of view is helpful - information is more immediate when it is told from the point of view of the person directly affected.  
Understanding peoples' thoughts, actions, and behaviors helps fully ``act out''~\cite{Carter2001} the story. 
For example, \textit{"Question: How do you get an Amish person to change a lightbulb? Answer: What's a lightbulb?"} this takes the Amish person's point of view in the answer, which is unexpected, the information is more immediate when it is told from the point of view of the person directly affected.
Sometimes that story can be based on a metaphor that helps the listener see something in a new way: \textit{"Playing an unamplified guitar is like strumming on a picnic table" (Dave Barry)} ~\cite{sep_humor}. Finding a good narrative, or point of view for a situation can help show the humor in it.

\textbf{Jokes are creative and can be intentionally constructed}
Almost all humorists who write books about humor
echo the design literature that divergent and convergent thinking are helpful processes in writing humor. 
Exploration is necessary to expand topics~\cite{Holloway2010} and many alternatives for punchlines should be explored to find the one that resonates the most~\cite{humor_Dean2000}.

Overall, jokes require complex human and social understanding. This is probably why humor difficult for machines to produce.

% Overall, jokes contain a lot of human-proper



\subsection{Computational Creativity and Humor}
% Computational Creativity~\cite{tuhin_artifice}\todo{WORK HERE}


Computational humor is an outstanding challenge in AI. Even
classifying whether or not something is funny is a computational challenge. This has been tried with some limited success by training deep learning classifiers on New Yorker Captions ~\cite{shahafjokes}, and by "unfunning" jokes to create more training data~\cite{unfun}.
% Before generative AI, techniques were limited. 
Wordplay is a common target for computational humor generation, and it relies on linguistic violations rather than social or cultural ones. Thus many successful systems have produced computational word play ~\cite{jape, twss, witscript1, he2019pungenerationsurprise, Taylor04computationallyrecognizing}, but these techniques don't extend to broader types of humor. 
% With generative AI there are more possibilities.
Generating humor without wordplay is harder. 

LLMs have opened this as a possibility but humor continues to be a challenge for LLMs like ChatGPT~\cite{gptnotfunny, deepmind_humor}, even with fine tuning~\cite{anonymous2024funlms}. 
Recently, ChatGPT has been shown to be funnier than jokes written by Turkers (as rated by other Turkers), but not nearly as funny as professionally published humor ~\cite{gptvsturk}.

A common technique for non-wordplay humor is finding and relating unexpected associations ~\cite{witscript2}, using multiple humor strategies for associations~\cite{witscript3}.
Adding multi-step reasoning can also be applied~\cite{tikhonov2024humormechanicsadvancinghumor}, but is not sufficient. 
These attempts are largely in the right direction. Using associations, reasoning, and creative processes like divergent and convergent thinking made sense. But there are many more dimensions to humor. Considering more human-centric ideas like point of view, observation of details, social understanding, and understanding the audience is also essential for humor generation. 

% humor to an in-crowd that will appreciate the tone, the insights, and the details. 




\subsection{AI Generation Techniques}

Prompt engineering is the most general AI generation technique~\cite{gpt2}. Crafting prompts that clearly define the problem and the expected output is the first technique to try to generate better outputs. However, when prompt engineering is insufficient there are many other generation paradigms that have been shown to achieve better results.
Fine-tuning allows the model to learn the specific patterns of a target output type. This can help learn writing style, topics, and other implicit aspects of language.
Results from a model can be chained together to solve a bigger problem~\cite{cai_ai_chains}, allowing the model to focus on solving one problem at a time. 
Chain of thought~\cite{cot} is a technique that helps models get more accurate answers to math questions by talking through the steps. It is somewhat equivalent to the human technique of "showing one's work". 
Similarly, thought experiments can better solve moral questions by thinking through the consequences of possible actions before making judgments ~\cite{ma2023letsthoughtexperimentusing}.
Reflection ~\cite{reflection} is a technique for AIs to evaluate their own outputs, and try to improve based on their own evaluation. 
Using AI to evaluate its own outputs, does not always produce good results, but it is commonly used~\cite{ai_self_eval}. 
Agents~\cite{joon_agents} can be useful to have AIs talk to one another to reason through a problem. Agents of different skills (or roles) can collaborate to do hard tasks that require multiple perspectives or roles like software engineering, and solving physics problems. A mixture of experts or perspectives can also be used to improve results ~\cite{cai2024surveymixtureexperts}.

% Generate-and-Ranking is a common technique that uses both chaining and agents to create better outcomes by having a generation agent generate many possible outputs, and then chain to a ranking agent to select the best ones. This technique mimics the divergent and convergent stages of creativity. 

In our system, we use fine-tuning to give the LLM a better sense of the target artifact. Implicit in this is the tone, the style, and the vocabulary expected in the humor. 
We use chains to separate stages of the humor generation process. We have an observation stage that makes implication information in images explicit, similar to the spirit of chain-of-thought and thought experiments. We also have an LLM as an evaluation of outputs - using a fine-tuned LLM trained to evaluate humor specifically for the target audience, somewhat like a mixture of experts model.
There are several other AI paradigms that we did not explore here, but could be relevant to humor including: reflection planning, question asking, retrieval augments generation (RAG), and human reinforcement learning. 



