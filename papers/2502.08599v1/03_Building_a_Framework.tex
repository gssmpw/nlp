\section{SPeCtrum: Framework for Multidimensional Identity Representation in LLM-based Agents}

The SPeCtrum framework is grounded in the concept of \textbf{self-concept}, which is an individual's set of beliefs and perceptions about themselves. This encompasses their beliefs, values, abilities, and attributes \citep{bracken1996handbook,markus1987dynamic,oyserman_self-concept_2001}. In particular, researchers have posited that self-concept primarily comprises two core components: Social Identity and Personal Identity \citep{Jones2000ACM, nario-redmond_social_2004}.

\paragraph{Social Identity (S)}
Social Identity refers to the shared characteristics of an individual as a member of various groups, including innate categories (e.g., gender, race) and acquired traits (e.g., education, occupation) \cite{oyserman_self-concept_2001}. These factors shape one’s self-concept and social interactions. To model this in LLM agents, we compiled a \textbf{set of 19 questions} focused on demographics and socio-economic status. This data is incorporated as a key element in constructing the social aspect of an individual's self-concept (see \ref{tab:demo}).

\paragraph{Personal Identity (P)}
Personal Identity encompasses deeper psychological traits and values, representing qualities individuals often consider core to their ``inner self'' \citep{Jones2000ACM, fearon1999identity}. It includes personal attributes and characteristics such as personality and value systems \citep{fearon1999identity}.

\textbf{\textit{Personality}} is defined as the dynamic organization of psychophysical systems within an individual that determines their adaptation to the environment \citep{allport1937personality}. By incorporating the personality factor, we aim to capture the primary personality traits that shape an individual's unique thought patterns, emotions, and behavior \citep{corr&matthews2009, weinberg_foundations_2019}. To model this in LLM-based agents, we used the 30-item Big Five Inventory-2-Short Form (\textbf{BFI-2-S}) \citep{bfi_short_2017}, a widely used and well-validated measure of the personality traits (e.g., extraversion).

\textbf{\textit{Values}} play a crucial role in shaping one's identity, influencing not only moment-to-moment behaviors but also guiding overarching life orientations \citep{schwartz1994there}. To integrate an individual’s value system into LLM agents, we employed the 21-item Portrait Values Questionnaire (\textbf{PVQ}) \citep{schwartzpvq2009basic}, which evaluates values across ten dimensions (e.g., hedonism, achievement, power). Incorporating the PVQ results provides a comprehensive perspective on an individual's values and their role in shaping personal identity.

\paragraph{Personal Life Context (C)}
Lastly, we incorporated Personal Life Context (C) to provide a more nuanced and dynamic understanding of how social and personal identity are expressed and enacted in an individual's daily life \cite{chen_meta-analysis_2024, frederickx2014role, schwartz1994there}.
To integrate Personal Life Context into our framework, we elicited \textbf{two short open-ended questions}: 1) one's \textbf{preferences} (listing five things one loves and hates) and 2) a short essay detailing \textbf{typical daily routines}, using the Behavioral Essay format \citep{boyd_values_2021}, which asks respondents to describe their typical activities on weekdays and weekends (see \ref{tab:openended}).
We expected that personal preferences would provide insight into one's tastes and interests while daily routines would reveal time management and priorities, thereby enhancing the realism and authenticity of LLM agents.

\subsection{Knowledge Injection Process}
\label{sec:knowledge_injection_pipieline}

We integrated the aforementioned information sources to elicit S, P, and C aspects. In injecting this data into LLMs, S component data, such as gender and race, were formatted into a list based on structured demographic questionnaires (see \ref{sec:appendix_profile_example_S_A1}). For C, which was collected in an open-ended format, we integrated the data directly into prompts without further processing (see \ref{sec:appendix_profile_example_C_A3}). This approach leverages evidence that contextual details, such as character utterances and writing styles, help produce a more nuanced, authentic representation \citep{han_meet_2022, ahn_mpchat_2023, shao_character-llm_2023}.

In processing the scores from the BFI-2-S and PVQ on a 1-7-point Likert scale into the P component, we followed the methodology of \citet{serapio-garcia_personality_2023} to avoid summaries that merely echo the verbatim of the scale items (see Figure \ref{fig:1}). 

To achieve this, we first averaged facet scores for each scale and converted them into natural language descriptions. For example, an Extraversion score of 3 was phrased as ``\textit{Extraversion is slightly below average.}'' Next, we applied the Chain of Density (CoD) technique \citep{adams_sparse_2023} with GPT-4o, the state-of-art model (SOTA) at the time, for progressive summarization. This involved first generating a technical summary of the facet descriptions and then iteratively condensing it into denser and more insightful summaries. Through this process, the final personality and value description produced two overviews of an individual's personality and value system: one in expert terminology (psychotherapist's language) and the other in everyday language (see \ref{sec:appendix_profile_example_P_A2}), providing a comprehensive understanding of one's personal identity.

Through this process, we constructed an individual’s holistic profile encompassing S, P, and C. We then prompted LLMs to ``embody the character,'' focusing on how a character’s traits manifest in both personal and social contexts without directly referencing the provided data (see \ref{sec:appendix_profile_example_base_prompt}). 


