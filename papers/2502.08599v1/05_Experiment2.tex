\section{Human Evaluation with Real-world individuals}
To assess whether the findings from the automated evaluation would be applied to real-world individuals, we conducted a human evaluation via Prolific\footnote{https://www.prolific.com/} with 80 U.S. participants (aged 18+), compensating each with \$6 USD.

Participants accessed a dedicated research website and first completed a survey to provide their S, P, and C components. Specifically, for C, participants were asked to write two short essays of approximately 450 characters each—one describing their typical weekday routines and the other their typical weekend routines—totaling around 900 characters. This length was chosen to align with the 732–1856 character range observed in the C data during the automated evaluation, ensuring consistency while accommodating variability in participant responses.

Next, following the knowledge injection process outlined in \ref{sec:knowledge_injection_pipieline}, the inputs for S and C were used as provided, while P data was processed to generate both an expert-level and an everyday-language overview of participants' personality and value systems. 
This data was then used to create four variations of agent personas (S, P, C, SPC) using GPT-4o, the SOTA model at the time.

Each participant agent (S, P, C, SPC) was tasked with writing short essays on four topics: self-introduction, a vision of their life, strategies for managing stress, and how they define happiness (see \ref{sec:appendix_human_eval_essay_prompt} and \ref{sec:appendix_human_eval_essay_topic}). Participants then evaluated four essays on each topic, each generated by a different agent variant, rating the perceived similarity (overlap) between the essay content and their self-concepts on a scale from 0 to 100\%  while blinded to the agent condition (see \ref{sec:appendix_human_eval_protocol}). They were also asked to provide brief, open-ended feedback on the essay they found most aligned and least aligned with their self-perception.


\subsection{Results: SPC as Holistic Self-Concept Representation for Real-world Individuals}

We conducted a linear mixed model analysis using R version 4.3.1, with perceived similarity as the dependent variable. The model included fixed effects for experimental conditions (S, P, C, SPC) while treating each participant as a random effect. AI perception and self-awareness, measured via well-established questionnaires \citep{naeimi2019validating, wang2023measuring, sindermann2021assessing}, were included as covariates, given their potential influence on participants' perceptions of AI agents \citep{jiang-etal-2024-personallm, kross2017self}.

Results showed significant effects based on the experimental conditions (see Figure \ref{fig:5}). The intercept for the baseline condition C was significantly positive $(b = 71.50, \mathit{SE} = 9.06, t = 7.89, p < .001)$. Both S $(b = -4.53, \mathit{SE} = 1.71, t = -2.65, p = .008)$ and P $(b = -6.91, \mathit{SE} = 1.71, t = -4.047, p < .001)$ conditions were associated with decreased perceived similarity, while SPC resulted in a significant increase $(b = 5.13, \mathit{SE} = 1.71, t = 3.00, p = .003)$.

\begin{figure}
    \centering
    \includegraphics[width=\linewidth]{Figure5_human_result.png}
    \caption{Perceived Similarity Ratings across Conditions and Essay Topics}
    \label{fig:5}
\end{figure}

Subsequent pairwise comparisons revealed that, consistent with the automated evaluation, S and P did not differ significantly $(p = 0.50)$ and C received higher similarity ratings than S $(t = 2.65, p = 0.04)$ and P $(t = 4.04, p < .001)$. Interestingly, unlike the automated evaluation, the integrated SPC condition exhibited significantly higher perceived similarity than the C-only condition $(t = -3.00, p = 0.01)$. There was no significant effect of essay topics $(p > 0.05$).

These results suggest that for real-world individuals who are underrepresented in LLM training data, more comprehensive data may be necessary to reflect an individual’s self-concept more authentically. To exemplify this point, one participant (P27) remarked:
``\textit{Essay (S) felt very generic, which can apply to anyone. However, Essay (SPC) really knew me well. It articulated my thoughts perfectly}.''
These comments highlight the advantages of the holistic approach in the SPeCtrum framework.


\begin{figure}
    \centering
    \includegraphics[width=\linewidth]{Figure6_reinfer_s_bar.png}
    \caption{Accuracy of Automated and Human Evaluation in Inferring Social Identity Attributes (S) from Personal Life Context (C)}
    \label{fig:6}
\end{figure}

\subsection{Importance of Broader Data Integration in Identity Representation for Real-World Individuals}
To explore the reasons behind SPC outperformed C in human evaluations, we inferred S and P attributes from C using the same setup in \ref{sec:inference_test}.

Our analysis revealed notable differences in the inference of S from C between automated and human evaluations. Fictional characters exhibited high accuracy across most categorical variables, whereas human samples showed greater variability and generally lower accuracy for categorical items in S (See Figure \ref{fig:6}). This pattern was also observed in continuous variables, such as age, social class, and household income (e.g., $\rho$ = 0.59 in the automated sample vs. 0.37 in the human sample; see the full correlation differences in \ref{sec:appendix_human_eval_inference_comparison}).

Next, inferring P elements from C showed moderate correlations with the golden answers from participants, with BFI-2-S yielding a mean $r$  of 0.621 (SD = 0.43), comparable to automated samples ($r$ = 0.686). However, PVQ correlations were significantly lower (mean $r$ = 0.362, SD = 0.38) compared to automated samples ($r$  = 0.71).

These discrepancies between drama characters and human samples highlighted potential shortcomings in using Personal Life Context (C) alone to represent real-world human identities in LLMs. Although C appears to be highly informative, it could have limitations in fully capturing the complex nature of real-world identities. In particular, the lower accuracy and weaker correlations across S and P elements in human samples underscore the necessity of structured and broader data integration to more model human complexities in LLMs, as demonstrated in the SPeCtrum framework.

