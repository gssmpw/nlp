\begin{abstract}
Existing methods for simulating individual identities often oversimplify human complexity, which may lead to incomplete or flattened representations. To address this, we introduce \textbf{SPeCtrum}\footnote{Code and data available at: \url{https://github.com/keyeun/spectrum-framework-llm}}, a grounded framework for constructing authentic LLM agent personas by incorporating an individual's multidimensional self-concept. SPeCtrum integrates three core components: \textbf{Social Identity (S), Personal Identity (P), and Personal Life Context (C)}, each contributing distinct yet interconnected aspects of identity. To evaluate SPeCtrum’s effectiveness in identity representation, we conducted automated and human evaluations. Automated evaluations using popular drama characters showed that Personal Life Context (C)—derived from short essays on preferences and daily routines—modeled characters' identities more effectively than Social Identity (S) and Personal Identity (P) alone and performed comparably to the full SPC combination. In contrast, human evaluations involving real-world individuals found that the full SPC combination provided a more comprehensive self-concept representation than C alone. Our findings suggest that while C alone may suffice for basic identity simulation, integrating S, P, and C enhances the authenticity and accuracy of real-world identity representation. Overall, SPeCtrum offers a structured approach for simulating individuals in LLM agents, enabling more personalized human-AI interactions and improving the realism of simulation-based behavioral studies.




\end{abstract}