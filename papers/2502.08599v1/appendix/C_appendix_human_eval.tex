\section{Human Evaluation}
\label{sec:appendix_human_eval}
\subsection{Prompt for Essay Generation}
\label{sec:appendix_human_eval_essay_prompt}
Below are the prompts that an agent created with a profile built according to the framework will respond to each essay.

\noindent\rule{\columnwidth}{0.3mm}



\begin{tcolorbox}[colback=white, colframe=white, boxrule=0mm, breakable, parskip=5pt]

\vspace{0.5em}
You're a doppelgänger of this real person. Embody this person. Using the provided profile, replicate the person's attitudes, thoughts, and mannerisms as accurately as possible. Dive deep into this person's psyche to respond to questions authentically.

\vspace{1em}

Question: \{Question\}

\vspace{1em}

TASK:

Provide an answer that this person, based on their profile, would likely give.

\vspace{1em}

RULES:

- Avoid generic responses; offer insights that resonate with this person's personal experiences and worldview.

- Use the profile to infer tone, preferences, and personality, showing a deep understanding of them.

- DO NOT directly cite profile phrases. Describe how these traits manifest in daily life and interactions.

- Ensure responses are natural, reflecting personal introspection, not just profile summaries.

- Use simple, everyday language typical of casual conversations. Think of how this person would speak in a casual, real-life conversation.

- Respond negatively if the person has a negative or cynical attitude.

- Base responses on reasonable inferences from the profile, avoiding leaps of logic.

- EXTREMELY IMPORTANT. Strictly follow these rules to create a compelling and believable doppelgänger portrayal.

\end{tcolorbox}

\noindent\rule{\columnwidth}{0.3mm}
\newline

\subsection{Essay Topics Detail}
\label{sec:appendix_human_eval_essay_topic}


In Human evaluation, we generated essays on the following four topics.

\begin{table}[htbp]
\centering
\begin{tabularx}{\textwidth}{lX}
\toprule
\textbf{Topic} & \textbf{Prompt} \\
\midrule
Self-introduction & How would you define yourself in one sentence? \\
\midrule
Future Life Vision (ten years hence) & In one sentence, define where you want to be in 10 years. \\
\midrule
Stress Causes and Relief Strategies & Complete all of the following sentences. 
\newline
I tend to feel stressed when \underline{\hspace{3cm}}. \newline
When I feel stressed, I try to relieve it by \underline{\hspace{3cm}}. \\
\midrule
Definition of Happiness & Complete the following sentences. 
\newline
To me, happiness is \underline{\hspace{3cm}}. \\
\bottomrule
\end{tabularx}
\caption{Essay topics used in our experiment}
\label{tab:essay_topics}
\end{table}

\noindent\rule{\columnwidth}{0.3mm}


\subsection{Human Evaluation Protocol}
\label{sec:appendix_human_eval_protocol}


\noindent\rule{\columnwidth}{0.3mm}


\begin{figure*}[h]
    \centering
    \includegraphics[width=\linewidth]{human_eval_consent.png}
    \caption{A screenshot of getting consent from human evaluation participants}
    \label{fig:screenshot_consent}
\end{figure*}


\begin{figure*}
    \centering
    \includegraphics[width=\linewidth, height=0.8\textheight, keepaspectratio]{human_eval_essay.png}
    \caption{A screenshot of the human evaluation experiment. Participants read essays, rate the degree of perceived similarity on a scale, and rank the essays in order of similarity.}
    \label{fig:screenshot_experiment}
\end{figure*}


\subsection{Correlation between Actual Social Identity and C-Inferred Social Identity}
\label{sec:appendix_human_eval_inference_comparison}
The table presents Spearman's $\rho$ values comparing the associations between actual social identity elements and inferred social identity attributes from C. All correlation test results were statistically significant (\textit{p} < 0.001)). The evaluations are split into two categories: Auto Evaluation and Human Evaluation. The results demonstrate relatively weak associations in Human Evaluation compared to Auto Evaluation. In the automated evaluation, the degrees of freedom are 223, while in the human evaluation, they are 398.

\begin{table}[ht]
\centering
\label{tab:evaluation_comparison}
\begin{adjustbox}{width=\textwidth}
\begin{tabular}{lcc}
\toprule
\textbf{Element} & \textbf{Automated Sample} & \textbf{Human Sample} \\
\midrule
Age & 0.59 & 0.37 \\
Education & 0.41 & 0.19 \\
Household Income & 0.68 & 0.43 \\
Income Satisfaction & 0.60 & 0.40 \\
Perceived Income Position & 0.62 & 0.34 \\
Political Stance & 0.45 & 0.15 \\
Social Class & 0.67 & 0.33 \\
\bottomrule
\end{tabular}
\end{adjustbox}
\caption{Comparison of Correlations Between Auto Evaluation and Human Evaluation}
\end{table}