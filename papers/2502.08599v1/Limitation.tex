\section{Limitations and Future Directions}
While the SPeCtrum framework aims to integrate key components of self-concept, not all attributes would hold equal significance for everyone. Additionally, our study was conducted exclusively with U.S. participants and applied the SPeCtrum framework only in English. These limitations highlight the need for continuing refinement to enhance the framework’s applicability. Future work will focus on incorporating weighted attributes and expanding to diverse linguistic and cultural contexts to improve its generalizability.

Regarding our automated evaluation methods, the ``Guess Who'' and Twenty Statements Test (TST) were designed to comprehensively assess the SPeCtrum framework. However, both exhibit certain limitations. The ``Guess Who'' test assumed uniform knowledge across all LLMs regarding TV series and characters, which may not accurately reflect variations in model knowledge bases. Meanwhile, the TST relied on a binary rating system to assess response accuracy, but a more nuanced approach could better evaluate how well each statement encapsulates both the explicit and latent aspects of an individual's self-concept.

For human evaluations, the effectiveness of the SPC condition may have been significantly influenced by the choice of questionnaire and individual differences in participation, such as writing quality and fidelity. This suggests the need for further research into diverse question sets that could better capture the complexities of self-concept and explore ways to enhance procedural stability by mitigating individual differences, such as incorporating non-self-reported data.
