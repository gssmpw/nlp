\section{Introduction}
\begin{quote}
\textit{Every man is more than just himself; he also represents the unique [..] point at which the world's phenomena intersect. - Herman Hesse (1919)}
\end{quote}
Identity is a complex, multifaceted concept that encompasses an individual's social, personal, and contextual attributes \citep{hall2015cultural,hesse2013demian, mead1934mind}. Recent advances in large language models (LLMs) have inspired their use in simulating individual behavior and thought processes \citep{park_social_2022, park_generative_2023, kang_values_2023}. However, some methods for simulating identities result in LLM agents that portray stereotypical characteristics of certain demographic groups \citep{argyle_out_2023,gupta_bias_2023, demollm2024}, often oversimplifying the complexity and diversity of individual human beings \citep{cheng_compost_2023, santurkar_whose_2023, petrov_limited_2024, bommasani2022}. 


\begin{table}[t]
\centering
\fontsize{8.5pt}{10pt}\selectfont
\begin{tabular}{>{\raggedright\arraybackslash}m{1.6cm} >{\raggedright\arraybackslash}m{2.8cm} >{\raggedright\arraybackslash}m{2cm}}
\toprule
Component & Description & Source \\ \midrule
Social Identity (S) & One's innate and acquired qualities linked to a social group & 19 Demographic questionnaire items \\ \midrule
Personal Identity (P) & One's psychological traits and values & BFI-2-S and PVQ scale \\ \midrule
Personal Life Context (C) & One's unique realization of identity & Short essays (preference, daily routines) \\ \bottomrule
\end{tabular}
\caption{Components of the SPeCtrum Framework.}
\label{tab:components}
\end{table}

\begin{figure*}[h]
    \centering
    \includegraphics[width=\textwidth]{Figure1_framework_overview.png}
    \caption{Overview of the SPeCtrum Framework for Multidimensional Identity Representation}
    \label{fig:1}
\end{figure*}


To address this limitation, we introduce the \textbf{SPeCtrum} framework (\textbf{S}ocial Identity, \textbf{P}ersonal Identity, and Personal Life \textbf{C}ontext), a grounded approach for developing LLM agents that effectively reflect the multidimensional nature of real-world individuals (see Table \ref{tab:components}). Grounded in social science approaches to one's self-concept \citep{Jones2000ACM, mead1934mind}, the SPeCtrum framework constructs identity through three key components: \textbf{Social Identity} (S), which refers to one's innate and acquired qualities linked to a social group, captured through demographic questionnaires; \textbf{Personal Identity} (P), which encompasses one's psychological traits and values,  assessed using established scales; and \textbf{Personal Life Context (C)}, representing one’s unique realization of identity, gathered through short open-ended essays on daily routines and personal preferences. 

To validate the SPeCtrum framework in representing one's self-concepts, we conducted both automated and human evaluations across different combinations of S, P, and C. For the automated evaluation, we created a dataset of popular drama characters and used the ``Guess Who test'' for character identification, along with the ``Twenty Statement Test'' (TST) using four LLMs: GPT-4o, GPT-3.5 Turbo, Claude-3.5-Sonnet, and Claude-3-Sonnet. In the human evaluations, 80 participants compared four types of agents—built on S, P, C, and SPC—based on how well each reflected their own self-perceptions.

The automated evaluation showed that Personal Life Context (C) was the most effective in capturing character identity, outperforming S and P, and performing comparably to the SPC combination. A follow-up experiment further tested whether C alone could infer S and P aspects. Results indicated that C content alone could reasonably infer demographic information (S) and personality traits and values (P) of drama characters.

Building on these findings, we conducted a human evaluation to assess whether this pattern held for real-world individuals who would be less prominently represented in LLMs' training data. The results indicated that while C continued to outperform S and P, the SPC combination provided a more comprehensive representation of self-concept for real-world individuals than C alone. Inference tests that derived S and P from C showed lower overall accuracy compared to inferences made from drama characters.

Such divergence between automated and human evaluations highlights the complexity of representing human identity, emphasizing the need for the full combination of S, P, and C as a comprehensive approach to capture self-concepts in real-world individuals accurately. Taken together, our evaluations validate the SPeCtrum framework as an effective foundation for representing multidimensional self-concepts and highlight its potential to enhance human-AI interactions and social simulations through more personalized and authentic identity representation.


Our primary contributions are as follows:
\begin{itemize}
    \item Introduce the SPeCtrum framework to facilitate the authentic and structured representation of real-world individuals in LLM agents.
    \item Demonstrate the effectiveness and potential of SPeCtrum through systematic evaluations, including automated evaluations using popular drama characters and human evaluations involving real-world individuals.
\end{itemize}

