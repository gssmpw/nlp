\section{Conclusion}
In this paper, we introduced the SPeCtrum framework, a grounded approach for generating authentic, multidimensional personas using LLMs. This framework integrates Social Identity (S), Personal Identity (P), and Personal Life Context (C), drawing upon the concept of self-concept.


We validated the SPeCtrum framework using both automated and human evaluations. The automated evaluation demonstrated that C alone performed comparably to SPC in characterizing the identities of popular drama characters. However, human evaluation involving real-world individuals revealed that the SPC combination was superior to C alone in modeling real-world individuals. Reverse inference from C to S and P elements further highlighted the limitations of relying solely on C, particularly when applied to real-world individuals. 

Overall, these results suggest that incorporating Personal Life Context (C)—encompassing daily routines and preferences—is essential for modeling individuals, serving as a rich foundation for identity representation. However, due to the complexity of human identity and the limitations of LLM training data, a more accurate and authentic simulation of real-world individuals requires the broader integration of all identity components as in the SPeCtrum framework.

In conclusion, the SPeCtrum framework presents a promising approach for generating authentic, multidimensional personas in LLMs by integrating comprehensive identity components. However, incorporating multi-sourced data beyond self-reported inputs could further enhance its effectiveness. We hope researchers and developers could build upon this framework as a foundation for creating LLM-based personas for both academic and practical applications across various domains.


