\section{Related Work}
Recent advancements in LLMs have significantly expanded their application across diverse academic fields such as social science ____, behavioral economics ____, and human-computer interaction ____, primarily through the creation and deployment of diverse agent personas. These LLM-created personas aim to simulate complex human and social behaviors ____, enabling the development of increasingly personalized applications such as recommendation systems ____.

Existing frameworks for persona creation primarily emphasize isolated human traits, focusing mainly on socio-demographic characteristics ____ for modeling specific human subpopulations ____. Although researchers have expanded these frameworks to incorporate other dimensions such as personality traits ____ and value systems ____, their approaches often yield biased or incomplete representations, as evidenced by homogeneous depictions of socially underrepresented groups ____.

Research in social psychology demonstrates that an individual's self-concept emerges from the dynamic interplay of multiple identity dimensions, including personal traits, social interactions, and lived experiences ____. This fundamental understanding highlights the critical importance of incorporating such multidimensional aspects in developing LLM agents that authentically reflect real-world individuals and their behavioral and thought patterns ____.

To address these limitations, we introduce the SPeCtrum framework (see Figure \ref{fig:1}), which enables structured and authentic persona representations through (a) identifying essential elements of multidimensional self-concept based on social science theories and research methodologies and (b) developing systematic pipelines for integrating diverse identity sources. Moving beyond the dominant focus on isolated traits, our framework emphasizes the dynamic interactions between identity components to create LLM-based agents that capture the rich complexity of real-world individuals.