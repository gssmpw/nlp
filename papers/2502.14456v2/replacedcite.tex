\section{Related Work}
\subsection{Long-Form Script Generation}

Long-form narrative generation is a key research area in natural language processing, aiming to produce coherent and creative stories. ____ introduced LeakGAN, which combines generative adversarial networks (GANs) with policy gradients to guide long-text generation. ____ introduced the "Plan-and-Write" framework, which divides the story generation into two stages: planning and writing. ____ proposed the "EIPE-text" method, which refines plans iteratively using an evaluation mechanism to produce more coherent narratives.  In the domain of scriptwriting, ____ developed the Dramatron system, which leverages large language models (LLMs) to co-write movie and theatre scripts. Dramatron generates coherent scripts by hierarchically creating titles, characters, story beats, location descriptions, and dialogues.

\subsection{Automatic Itinerary Planning}

Numerous studies have addressed automated travel itinerary planning, employing various methods to tackle the problem. Some studies use exact algorithms, such as ____, which applies a branch-and-cut approach to solve self-guided tour planning. Since travel itinerary planning is NP-hard ____, approximation methods are often employed to enhance solution efficiency. Consequently, metaheuristic algorithms are commonly used. For instance, ____ employed a genetic algorithm to address itinerary planning, focusing on time and multimodal transport constraints. ____ use a cooperative co-evolutionary algorithm for cross-city itinerary planning, while ____ apply an improved ant colony algorithm, considering restaurant and hotel selections. More recently, some researchers have explored the use of LLMs for itinerary planning. For example, ____ leverage LLMs for personalized travel itinerary planning, and ____ utilize the ChatGPT model to enable users to generate travel plans and suggestions based on keywords.