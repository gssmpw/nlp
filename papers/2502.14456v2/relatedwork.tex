\section{Related Work}
\subsection{Long-Form Script Generation}

Long-form narrative generation is a key research area in natural language processing, aiming to produce coherent and creative stories. \citet{guo2018long} introduced LeakGAN, which combines generative adversarial networks (GANs) with policy gradients to guide long-text generation. \citet{yao2019plan} introduced the "Plan-and-Write" framework, which divides the story generation into two stages: planning and writing. \citet{you2023eipe} proposed the "EIPE-text" method, which refines plans iteratively using an evaluation mechanism to produce more coherent narratives.  In the domain of scriptwriting, \citet{mirowski2023co} developed the Dramatron system, which leverages large language models (LLMs) to co-write movie and theatre scripts. Dramatron generates coherent scripts by hierarchically creating titles, characters, story beats, location descriptions, and dialogues.

\subsection{Automatic Itinerary Planning}

Numerous studies have addressed automated travel itinerary planning, employing various methods to tackle the problem. Some studies use exact algorithms, such as \citet{verbeeck2014extension}, which applies a branch-and-cut approach to solve self-guided tour planning. Since travel itinerary planning is NP-hard \cite{liao2018using, castro2015fast, gavalas2013cluster}, approximation methods are often employed to enhance solution efficiency. Consequently, metaheuristic algorithms are commonly used. For instance, \citet{abbaspour2009itinerary} employed a genetic algorithm to address itinerary planning, focusing on time and multimodal transport constraints. \citet{zhang2024cooperative} use a cooperative co-evolutionary algorithm for cross-city itinerary planning, while \citet{chen2023application} apply an improved ant colony algorithm, considering restaurant and hotel selections. More recently, some researchers have explored the use of LLMs for itinerary planning. For example, \citet{singh2024personal} leverage LLMs for personalized travel itinerary planning, and \citet{li2023everywheregpt} utilize the ChatGPT model to enable users to generate travel plans and suggestions based on keywords.