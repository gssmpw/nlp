\begin{abstract}
    Message Passing Neural Networks (\inline{MPNNs}) are a class of Graph Neural Networks (\inline{GNNs}) that leverage the graph topology to propagate messages across increasingly larger neighborhoods. The message-passing scheme leads to two distinct challenges: \emph{over-smoothing} and \emph{over-squashing}. While several algorithms, \eg \inline{DropEdge} and its variants -- \inline{DropNode}, \inline{DropAgg} and \inline{DropGNN} -- have successfully addressed the over-smoothing problem, their impact on over-squashing remains largely unexplored. This represents a critical gap in the literature as failure to mitigate over-squashing would make these methods unsuitable for long-range tasks. In this work, we take the first step towards closing this gap by studying the aforementioned algorithms in the context of over-squashing. We present novel theoretical results that characterize the negative effects of \inline{DropEdge} on sensitivity between distant nodes, suggesting its unsuitability for long-range tasks. Our findings are easily extended to its variants, allowing us to build a comprehensive understanding of how they affect over-squashing. We evaluate these methods using real-world datasets, demonstrating their detrimental effects. Specifically, we show that while \inline{DropEdge}-variants improve test-time performance in short-range tasks, they deteriorate performance in long-range ones. Our theory explains these results as follows: random edge-dropping lowers the effective receptive field of \inline{GNNs}, which although beneficial for short-range tasks, misaligns the models on long-range ones. This forces the models to overfit to short-range artefacts in the training set, resulting in poor generalization. Our conclusions highlight the need to re-evaluate various methods designed for training deep \inline{GNNs}, with a renewed focus on modelling long-range interactions. % The codebase associated to this work is available at \href{https://github.com/ignasa007/DropEdge-on-OverSquashing}{https://github.com/ignasa007/DropEdge-on-OverSquashing}.
\end{abstract}