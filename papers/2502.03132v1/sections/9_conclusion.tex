\section{Limitations}
Aiming to be a general and user-friendly benchmark, \spark has several potential limitations slated for future improvements.

In the current version, the safe control library supports first-order safe control, which implicitly assumes the robot can track an arbitrary velocity command immediately. However, as the motors have limited torques, if the velocity goal is generated to be too different from the current velocity, there may be delays in tracking the goal and impacting safety. To mitigate this issue, we will need to support higher-order safe controls in the future. 

Another limitation is that the current implementation does not distinguish between inevitable collisions from method failures (i.e., there are feasible collision-free trajectories but the method could not find one). Method failures happen when multiple safety constraints are in conflict, which does not necessarily imply a collision is inevitable. To mitigate this issue, more research is needed to either improve the safe control methods in handling multiple constraints or introduce advanced methods in detecting inevitable collisions. 

Finally, the sim to real gap also exists in model-based control systems, although it is called differently as ``model mismatch". For the real deployment, the robot trajectory might be different from the simulation due to ``model mismatch". To mitigate this problem, the system model needs to be robustified and the system control needs to be aware of potential gaps, which will be left for future work.

%For future work, we plan to expand the safe control library by incorporating additional safety indices and dynamics, including second-order controllers. Additionally, we aim to advance theoretical research on multi-constraint safe control problems, further enhancing \spark's capabilities and broadening its applications.

%Another key objective is to integrate \spark with other reinforcement learning platforms. This would enable \spark to function as both a safe RL task environment and a model-based safe RL library.

%Finally, we intend to stay aligned with advancements in humanoid hardware and research, ensuring \spark is updated to support emerging safe control strategies and new hardware systems. 

\section{Conclusion and Future Work}
In this paper, we presented \spark, a comprehensive benchmark designed to enhance the safety of humanoid autonomy and teleoperation. We introduced a safe humanoid control framework and detailed the core safe control algorithms upon which \spark is built.

\spark offers configurable trade-offs between safety and performance, allowing it to meet diverse user requirements. Its modular design, coupled with accessible APIs, ensures compatibility with a wide range of tasks, hardware systems, and customization levels. Additionally, \spark includes a simulation environment featuring a variety of humanoid safe control tasks that serve as benchmark baselines. By utilizing \spark, researchers and practitioners can accelerate humanoid robotics development while ensuring robust hardware and environmental safety.


Beyond what was mentioned in the previous section, there are several future directions for \spark that could benefit from community collaboration. 
First of all, it is important to further lower the barrier for users to adopt state-of-the-art safety measures \cite{ji2023safety}, e.g., safe reinforcement learning approaches \cite{zhao2023absolute}\cite{zhao2023learn}\cite{zhao2023state} \cite{yao2024constraint}, by integrating them into \spark algorithm modules. 
Secondly, to enhance the robustness and reliability of the deployment pipeline of \spark, future works are needed to extend \spark compatibility with standardized safety assessment pipelines, such as stress testing tools and formal verification tools \cite{xie2024framework}. 
Thirdly, simplifying task specification is critical for usability. Enhancements to \spark's interface, such as intuitive configuration, natural language-based task definitions \cite{lin2023text2motion}, and interactive visualizations \cite{park2024dexhub}, will enable more efficient safety policy design and debugging. 
Finally, automating the selection and tuning \cite{victoria2021automatic} of safety strategies will make \spark more adaptive. Future efforts may explore automatic model-based synthesis approaches such as meta-control \cite{wei2024metacontrol} techniques to dynamically optimize safety measures based on task requirements and environmental conditions. 