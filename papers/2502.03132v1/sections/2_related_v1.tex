\section{Related Work}  

In this section, we briefly survey previous work closely tied to our objective of enabling efficient, modular safe controller synthesis and benchmarking for robotic systems. We mainly focus on safe control for humanoids as humanoids are one of the most complex platforms in robotics.

\paragraph{Modular Safe Control Toolboxes}  
Safe control toolboxes aim to provide modular and general safe control algorithms that can be integrated into various robotic systems. The Benchmark of Interactive Safety (BIS)~\citep{wei2019safe} implements a set of energy-function-based safe control algorithms, such as the Potential Field Method (PFM)~\citep{khatib1986real}. Similarly, \citep{schoer2024control} introduces a toolbox focused specifically on Control Barrier Functions (CBF). However, these toolboxes do not offer the modularity required for seamless compatibility with other frameworks and lack tasks tailored for complex robotic systems.  

When it comes to complex robotic systems such as humanoids, specialized safe control methods have been proposed to address their high-dimensional dynamics.
In particular, CBFs are widely used to ensure locomotion safety of bipedal robots~\citep{nguyen2015footstepCBF,peng2024real}, safe humanoid navigation~\citep{agrawal2017dcbfnavigation}, self-collision avoidance for humanoids~\citep{khazoom2022humanoid, huang2024whole}, and humanoid whole-body task space safety~\citep{paredes2024wholebodysafe}.
% For the locomotion of bipedal robots, ~\citep{nguyen2015footstepCBF} ensures precise footstep placements by combining CBFs and control Lyapunov functions (CLFs), achieving safe and stable dynamic walking over randomly generated discrete footholds and overhead obstacles.
% \citep{agrawal2017dcbfnavigation} develops a discrete-time CBF to provide safety for bipedal robots navigating through moving obstacles.
Moreover, ARMOR~\citep{kim2024armor} employs a novel egocentric perception system to learn data-driven safe motion planning policies for humanoid robots.
% \citep{khazoom2022humanoid} and~\citep{ding2022wbccbf} achieve self-collision avoidance of humanoids using CBFs. 
% \citep{paredes2024wholebodysafe} enhances the safety of humanoid robots in the whole-body task space by developing acceleration-based exponential CBFs and demonstrates its effectiveness in both arms and legs.
Despite these advancements, no existing benchmark provides standardized tasks for comparing current methods or supports the development of new safe control approaches in a consistent and modular way.  

\paragraph{Benchmark Platforms for Safety-critical Tasks}  

As safe reinforcement learning (RL) has emerged as a powerful framework for learning safe control policies from data, such as Constrained Policy Optimization (CPO)~\citep{achiam2017cpo} and State-wise CPO (SCPO)~\citep{zhao2023state}, many of the existing benchmark platforms for safety-critical tasks are tailored for safe RL. For instance, GUARD~\citep{zhao2023guard} offers a comprehensive set of environments featuring safe RL tasks alongside state-of-the-art safe RL algorithms. Similarly, Safety-Gymnasium~\citep{ji2023safety} integrates a wide range of safe RL environments into a unified benchmark. Safe Control Gym~\citep{yuan2022safe} provides a collection of safe control environments to facilitate model-free safe control policy learning.  
However, these platforms do not cater to tasks that invovle high-dimensional robotic systems, such as humanoid robots. In addition, they place limited emphasis on model-based methods, which remain essential for addressing safe control challenges due to their analytical rigor, composability, and interpretability.  

\paragraph{Benchmark Platforms for Humanoid Robots}  
Humanoid robots are among the most complex robotic systems, posing significant challenges~\cite{gu2025humanoid} in control. Learning-based methods have shown impressive results in humanoid control, and various open-source humanoid benchmarks have been developed. H2O~ \cite{he2024learning} and OmniH2O~\cite{he2024omnih2o} develop RL-based whole-body humanoid teleoperation, with H2O using an RGB camera and OmniH2O extending to multimodal control and autonomy.~\cite{fu2024humanplus} enables humanoids to learn motion and autonomous skills from human data via RL and behavior cloning. ExBody~\cite{cheng2024expressive} and ExBody2~\cite{ji2024exbody2} develop RL-based whole-body control for humanoid robots, with ExBody enabling expressive motion and ExBody2 enhancing generalization and fidelity via a privileged teacher policy. Both~\cite{zhuang2024humanoid} and ~\cite{zhang2024wococo} employed pure reinforcement learning methods to achieve humanoid robot locomotion. HOVER~\cite{he2024hover} unifies diverse humanoid control tasks via multi-mode policy distillation, enabling seamless transitions without retraining. HumanoidBench~\citep{sferrazza2024humanoidbench} provides a benchmark for humanoid RL on whole-body tasks, while Humanoid-Gym~\cite{gu2024humanoid} enables zero-shot sim-to-real humanoid locomotion training.  Mimicking-Bench~\cite{liu2024mimicking} establishes a benchmark for humanoid-scene interaction learning using large-scale human animation data, while MS-HAB~\cite{shukla2024maniskill} accelerates in-home manipulation research with a GPU-optimized benchmark and scalable demonstration filtering. BiGym~\cite{chernyadev2024bigym} is a benchmark for bi-manual robotic manipulation with diverse tasks, human demonstrations, and multi-modal observations.
Nevertheless, none of these benchmarks specifically focus on testing the safety of humanoid robots. Achieving safety for humanoids requires not only model-free~\cite{zhao2021model,yang2023model} approaches but also model-based methods~\cite{berkenkamp2017safe,yun2024safe}, which offer valuable analytical insights and interpretability in addition to their control capabilities.  
