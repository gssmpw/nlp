\section{Use Case 3: Safe Autonomy on Real Robot}\label{sec: usecase_safe_auto_real}






% To illustrate the real-world applicability of the \spark framework, we deploy its safe control pipelines on a physical robot. We present two use cases: Section \ref{sec: usc} demonstrates safe autonomous operation and Section \ref{sec:usecase3} showcases safe teleopration. These use cases of hardware deployment validate the framework's capability to handle real-world tasks effectively while ensuring safety.
% Furthermore, the transition between use cases is conveniently simple

% For both real robot tasks, we use the Real Robot Agent to interface with the Unitree G1 humanoid robot.

% We used a Unitree G1 humanoid robot, for which \spark provides an interface in the agent module. G1 humanoid physically features 29 DOFs, but we modeled it as a 20 DOF mobile dual-manipulator system by simplifying the locomotion.
% For the upper body dynamics, each arm was treated as a general manipulator with seven DOFs, while the waist was equipped with three rotational joints (roll, pitch, yaw).
% Regarding locomotion, three DOFs were considered: longitudinal, lateral, and rotational in the humanoid's body frame. 
% Table~\ref{tb: tasks} in the Appendix shows the detailed configuration for the G1 humanoid.
To showcase the practical applicability of the \spark framework, we deployed its safe control pipelines on a real robot.
For our experiments, we utilized the Unitree G1 humanoid robot, which features 29 degrees of freedom (DOFs).
The perception module incorporated an Apple Vision Pro to capture human gestures and determine the robot’s position.
The robot’s dynamic system was modeled as a mobile dual-manipulator system.

Building on the benchmark use case introduced in \Cref{sec: usecase_benchmark}, we only need to replace the agent module with the real G1 SDK and integrate the Apple Vision Pro interface into the task module.
With these modifications, the control algorithms validated in the benchmark can be directly deployed on the real robot.


We begin with the simplest task, in which the robot remains in a fixed position while avoiding potential collisions with the human user. 
As shown in Figure~\ref{Static}, when the user attempts to approach the robot arm from various angles, the robot reacts by moving away from the human hand if the minimum distance between them becomes smaller than $d_\mathrm{min}$. Once the robot detects that the human hand has moved away and the surrounding environment is safe, it resumes following the nominal controller's inputs, which commands it to remain in the target static position.

We further evaluate the performance of the \spark safe controller in tracking a dynamic target position by designing a dynamic limb-level collision avoidance test case.
Unlike the static test, where the nominal controller simply tracks a fixed target, the nominal controller in the dynamic test is tasked with following a dynamic target $\xb^R_\mathrm{target}$.
Specifically, the nominal controller tracks a circular trajectory for the right hand.
In this case, the robot must track the target trajectory while simultaneously avoiding collisions with the human user using the same safety index $\phi$ as previously defined.



From Figure~\ref{Dynamic}, we observe that when the human hand remains outside the $d_\mathrm{min}$ region, the robot follows the reference trajectory and moves along the circular path. 
If a human hand gets too close to the robot's hands, the humanoid will use both its waist and arm movements to avoid a collision, regardless of the target's motion. In other words, the safety constraint takes precedence over the normal control input, ensuring the humanoid remains safe.
Once the humans hand moves away, the robot returns to the reference trajectory while remaining prepared for potential collision avoidance.

\begin{figure*}[htbp!]
    \centering
    \vspace{1.5cm}  % Optional vertical space
    \begin{tikzpicture}[transform canvas={xshift=0cm}]
        % Define image size and spacing
        \def\imgwidth{15cm}  % Image width
        \def\ygap{-3} % Y-axis spacing (vertical)

        % First subfigure (Dynamic 1)
        \node at (0, 0) {\includegraphics[width=\imgwidth]{figure/real_robot/dynamic_1_masked.jpg}};

        % Second subfigure (Dynamic 2)
        \node at (0, \ygap) {\includegraphics[width=\imgwidth]{figure/real_robot/dynamic_2_masked.jpg}};


    \end{tikzpicture}
    \vspace{4.5cm}
    \caption{\ref{sec: usecase_safe_auto_real} Limb-level collision avoidance with dynamic humanoid reference poses: when the human hand stays outside \( d_{\min} \), the robot follows the reference trajectory. If it gets too close, the humanoid adjusts its waist and arms to avoid collision, prioritizing safety. Once the hand moves away, the robot resumes its trajectory while remaining prepared.  
 }
    \label{Dynamic}
\end{figure*}

