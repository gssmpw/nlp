\section{Use Case 4: \\ Safe Teleoperation With Real Robot}\label{sec: usecase_safe_teleop_real}

As teleoperation becomes more widely used to control humanoid robots~\cite{goodrich2013teleop,kourosh2023teleop}, it is paramount to ensure safety in this operating mode.
Thus, this section assesses the safe controller's performance in more general scenarios, in which the humanoid robot is tasked with following a human user's teleoperation commands while ensuring collision avoidance. This setup introduces the concept of ``Safe Teleoperation."

In this test, the nominal humanoid controller's target, $\xb^R_\mathrm{target}$, is not pre-designed but is instead generated in real-time by the human teleoperator. This adds complexity, as the robot must generate safe motions in an unpredictable and dynamic environment.

To implement this, we only needed to make a simple modification to the task module of \spark, adding the operator's hand positions as the goal positions for the robot's arms. Meanwhile, other human participants were treated as obstacles to ensure safe human-robot interaction.

In our experiment, we created a realistic scenario where the robot attempts to retrieve objects from a table. 
\Cref{Teleop} shows if the human user reaches for the same object as the robot, the safe controller is triggered. 
The robot prioritizes collision avoidance over executing the teleoperation commands, ensuring safe interaction and protecting both the humanoid robot and the human from potential hazards caused by the limited perception of a remote teleoperator.