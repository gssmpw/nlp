\section{Introduction}
\label{sec: introduction}

Safety is an essential element for robotic systems not only in industrial settings but also in everyday life.
From an algorithmic perspective, safe control ensures a system operates within user-specified safety constraints to minimize risks and harm while achieving objectives. 
Recognizing the vital importance of maintaining these constraints, numerous safe control approaches have been developed and successfully applied to a wide range of robotic systems.

Among these approaches, the general pathway is to formulate the safe control problem as a constrained optimization problem, where the objective is to accomplish a given task while satisfying the safety constraints imposed on the robot.
Analytical methods can transform this optimization problem into a quadratic programming (QP) formulation, which can then be solved efficiently by existing solvers. Alternatively, data-driven methods leverage black-box policies to generate control solutions directly.
However, regardless of the success of the approaches, the fundamental challenge remains: \textit{how to allow non-experts to efficiently synthesize and deploy safe controllers in diverse applications?} The efficient synthesis and deployment require easy specification of both the objective and constraints that ensure safety while fulfilling task requirements, informed selection of the best safe control approaches, and tuning-free deployment on the real hardware. 
% One effective approach to safe control is energy-function-based algorithms, such as the Safe Set Algorithm~\cite{liu2014control}, Control Barrier Function~\cite{ames2019control}, Potential Field Algorithm~\cite{khatib1986real}, and Sliding Mode Algorithm~\cite{gracia2013reactive}. 
% These methods utilize energy-like functions that quantify a system’s safety level and dissipate energy to keep the system safe.
% In most cases, synthesizing these energy-like functions require accurate dynamic models.
% On the other hand, learning-based safe control methods, such as Constrained Policy Optimization (CPO)~\cite{achiam2017cpo} and State-wise CPO (SCPO)~\cite{zhao2023state}, attempt to achieve safe control via data-driven approaches without relying on explicit system models. 
% Nevertheless, a carefully crafted cost function is critical to effectively encode safety requirements, ensuring that these methods learn appropriate and reliable safe control policies.


As model-free data-driven approaches have not been successful in safety assurance in high-dimensional systems, and the formal verification of these methods still requires the system model~\cite{fulton2018safe}, 
this paper focuses on efficient synthesis and deployment of model-based approaches. 

Synthesizing a model-based safe controller for simple scenarios is relatively straightforward.
Consider a differential-drive 2D autonomous vehicle tracking a trajectory with an upperbound velocity constraint for safety. 
It is not difficult to design an appropriate energy-like function that ensures the vehicle does not go over this speed limit.
However, as the scenario grows more complex (e.g., more complex constraints and objectives,  more complex robot dynamics, or more complex environments), synthesizing model-based safe controllers for each case becomes increasingly challenging and requires significantly more time and resources.

Now, consider replacing the differential-drive vehicle with a humanoid robot delivering a package with the same upperbound velocity constraint.
Due to the different dynamics of a humanoid robot, such as its joint limits, the safe controller previously synthesized for the vehicle may no longer be valid.
Furthermore, once the humanoid robot successfully arrives at the package dropoff destination, e.g., a loading truck, the task transitions from navigation to upper body manipulation, requiring the robot to load the box onto the truck while avoiding collisions.
The humanoid dynamic can no longer be treated as a mobile robot and the upper body needs to be incorporated into the safety constraint.
Even for the same loading task, changing the control mode from autonomous operation to teleoperation or language-vision-based commands introduces new challenges for safe control.
% For example, a human teleoperator or a Large Language Model (LLM)-based planner may have limited information about the robot’s surroundings such as the obstacle positions.

Finally, even for similar safety constraints, the safe controller may need re-synthesis when the environment changes. 
For instance, consider a warehouse where `a humanoid robot operates alongside human workers who are loading boxes. In this setting, the robot must avoid both static obstacles, e.g., a parked truck, and dynamic human participants.
Safe controller synthesis in this scenario must also ensure robust human safety, potentially achieved by integrating a human motion prediction model into the control stack.

Hence, synthesizing a safe controller from scratch on a case-by-case basis can quickly become time-consuming and inefficient, as each change in \textit{robot}, \textit{task}, or \textit{environment} requires re-synthesis by experts. 
At the same time, designing a single, universal safe controller capable of handling every possible scenario - varying tasks, diverse robot dynamics, ever-changing environments, and arbitrary safety constraints - is highly challenging, if not impossible. 
This inherent inefficiency hinders the practical development and deployment of safe controllers across real-world robotic applications.

To lower the barrier and increase the efficiency of safe controller synthesis and deployment, a plug-and-play modular framework is essential. 
To this end, we present the \textbf{Safe Protective and Assistive Robot Kit (\spark)}, as a modular safe control framework. 
With \spark, users can efficiently design, verify, and benchmark safe controllers for complex robot systems without starting from scratch. 
This approach reduces the effort of synthesis while promoting scalability and generalization of safe control methods across diverse robot configurations and task scenarios.
Guided by these principles, we designed \spark to be
\\
\noindent\textbf{Composable:}
\begin{itemize}
    \item Provides a set of modular components that users can assemble to create safe robotic control scenarios with custom task goals and safety requirements as shown in \Cref{fig: title_figure}.
    \item Offers predefined module options to facilitate large-scale benchmark scenario generation for evaluating safe control approaches.
    \item Enables users to switch between similar scenarios by simply replacing the corresponding module.
\end{itemize}
\noindent\textbf{Extensible:}
\begin{itemize}
    \item Allows users to customize each module, such as developing their own safe controllers or modifying robot dynamics, using provided templates.
    \item Supports the integration of additional user-defined modules, including external sensors and task planners.
    \item Provides a general wrapper to seamlessly bridge \spark with other existing benchmarks.
\end{itemize}
\noindent\textbf{Deployable:}
\begin{itemize}
    \item Enables users to deploy synthesized safe control algorithms on real robotic hardware by wrapping the hardware SDK (Software Development Kit) with the \spark interface.
    \item Ensures compatibility with real-time middleware such as ROS (Robot Operating System) and DDS (Data Distribution Service).
    \item Maintains robustness across diverse real-world task settings, including human-robot interaction and teleoperation through Apple Vision Pro.
\end{itemize}

% \begin{itemize}
%     \item \textbf{Decompose complex scenarios into independent modules:}
%     \begin{itemize}
%         \item \textbf{Robot Module:} Extract general information about the robot, such as degrees of freedom, controllable motors, and sensors, independent of the specific task.
%         \item \textbf{Task Module:} Define tasks without considering safety, e.g., navigating to a goal position, following teleoperation targets, or maintaining stable velocity.
%         \item \textbf{Environment Module:} Convert raw inputs from sensors like LiDAR, radar, or RGBD cameras into a structured format.
%     \end{itemize}
%     \item \textbf{Enable generalized synthesis of safe controllers:} Using the modular structure, safe controllers can be synthesized under a unified control system, integrating safety laws efficiently.
%     \item \textbf{Support efficient benchmarking:} A flexible benchmarking tool allows users to generate diverse safety scenarios by combining modular templates and provide a baseline for scenario customization.
% \end{itemize}

Ultimately, we hope \spark alleviates many of the challenges in safe controller synthesis and benchmarking, empowering robots to operate safely and reliably in diverse real-world scenarios.
