%%%%%%%% ICML 2025 EXAMPLE LATEX SUBMISSION FILE %%%%%%%%%%%%%%%%%

\documentclass{article}

% Recommended, but optional, packages for figures and better typesetting:
\usepackage{microtype}
\usepackage{graphicx}
\usepackage{booktabs} % for professional tables
\usepackage{subfigure}
\usepackage{multirow}
\usepackage{tabularx}
\usepackage{comment}
\newcommand{\weihua}[1]{\textcolor{blue}{(Weihua) #1}}
%\newcommand{\weihua}[1]{\textcolor{blue}{}}
\newcommand{\yy}[1]{\textcolor{red}{(YY:) #1}}

% hyperref makes hyperlinks in the resulting PDF.
% If your build breaks (sometimes temporarily if a hyperlink spans a page)
% please comment out the following usepackage line and replace
% \usepackage{icml2025} with \usepackage[nohyperref]{icml2025} above.
\usepackage{hyperref}

% Attempt to make hyperref and algorithmic work together better:
\newcommand{\theHalgorithm}{\arabic{algorithm}}

% Use the following line for the initial blind version submitted for review:
%\usepackage{icml2025}

% If accepted, instead use the following line for the camera-ready submission:
\usepackage[accepted]{icml2025}

% For theorems and such
\usepackage{amsmath}
\usepackage{amssymb}
\usepackage{mathtools}
\usepackage{amsthm}

% if you use cleveref..
\usepackage[capitalize,noabbrev]{cleveref}

%%%%%%%%%%%%%%%%%%%%%%%%%%%%%%%%
% THEOREMS
%%%%%%%%%%%%%%%%%%%%%%%%%%%%%%%%
\theoremstyle{plain}
\newtheorem{theorem}{Theorem}[section]
\newtheorem{proposition}[theorem]{Proposition}
\newtheorem{lemma}[theorem]{Lemma}
\newtheorem{corollary}[theorem]{Corollary}
\theoremstyle{definition}
\newtheorem{definition}[theorem]{Definition}
\newtheorem{assumption}[theorem]{Assumption}
\theoremstyle{remark}
\newtheorem{remark}[theorem]{Remark}

% Todonotes is useful during development; simply uncomment the next line
%    and comment out the line below the next line to turn off comments
%\usepackage[disable,textsize=tiny]{todonotes}
\usepackage[textsize=tiny]{todonotes}

% The \icmltitle you define below is probably too long as a header.
% Therefore, a short form for the running title is supplied here:
\icmltitlerunning{Optimizing Temperature for Language Models with Multi-Sample Inference}

\begin{document}

\twocolumn[
%\icmltitle{Uncovering Optimal Temperature in Language Model Inference Strategies}
\icmltitle{Optimizing Temperature for Language Models with Multi-Sample Inference}
%Yiming suggested title:
%\icmltitle{Optimizing Temperature for Sampling-Based Language Model Inference}

% It is OKAY to include author information, even for blind
% submissions: the style file will automatically remove it for you
% unless you've provided the [accepted] option to the icml2025
% package.

% List of affiliations: The first argument should be a (short)
% identifier you will use later to specify author affiliations
% Academic affiliations should list Department, University, City, Region, Country
% Industry affiliations should list Company, City, Region, Country

% You can specify symbols, otherwise they are numbered in order.
% Ideally, you should not use this facility. Affiliations will be numbered
% in order of appearance and this is the preferred way.
\icmlsetsymbol{equal}{*}

\begin{icmlauthorlist}
\icmlauthor{Weihua Du}{yyy}
%\icmlauthor{Firstname2 Lastname2}{equal,yyy,comp}
\icmlauthor{Yiming Yang}{yyy}
\icmlauthor{Sean Welleck}{yyy}
%\icmlauthor{Firstname5 Lastname5}{yyy}
%\icmlauthor{Firstname6 Lastname6}{sch,yyy,comp}
%\icmlauthor{Firstname7 Lastname7}{comp}
%\icmlauthor{}{sch}
%\icmlauthor{Firstname8 Lastname8}{sch}
%\icmlauthor{Firstname8 Lastname8}{yyy,comp}
%\icmlauthor{}{sch}
%\icmlauthor{}{sch}
\end{icmlauthorlist}

\icmlaffiliation{yyy}{Language Technologies Institute, Carnegie Mellon University}
%\icmlaffiliation{comp}{Company Name, Location, Country}
%\icmlaffiliation{sch}{School of ZZZ, Institute of WWW, Location, Country}

\icmlcorrespondingauthor{Weihua Du}{weihuad@cs.cmu.edu}
\icmlcorrespondingauthor{Yiming Yang}{yiming@cs.cmu.edu}
\icmlcorrespondingauthor{Sean Welleck}{wellecks@cmu.edu}

% You may provide any keywords that you
% find helpful for describing your paper; these are used to populate
% the "keywords" metadata in the PDF but will not be shown in the document
\icmlkeywords{Large Language Models, Inference Time Compute, ICML}

\vskip 0.3in
]

% this must go after the closing bracket ] following \twocolumn[ ...

% This command actually creates the footnote in the first column
% listing the affiliations and the copyright notice.
% The command takes one argument, which is text to display at the start of the footnote.
% The \icmlEqualContribution command is standard text for equal contribution.
% Remove it (just {}) if you do not need this facility.

\printAffiliationsAndNotice{}  % leave blank if no need to mention equal contribution
%\printAffiliationsAndNotice{\icmlEqualContribution} % otherwise use the standard text.

\begin{abstract} 
%Sampling-based inference strategies, such as majority voting and best-of-N sampling, are commonly employed in language models to improve accuracy on problem-solving tasks. While sampling temperature significantly influences the performance of these strategies, it is often set to a default value or tuned using an additional validation set without enough care. In this work, we demonstrate that the suitable temperature range can differ across various model architectures and tasks, and we observe a correlation between the optimal temperature and the similarity between models and tasks. Furthermore, we propose an entropy-based approach to automatically identify a suitable temperature without the need for any labeled data. Our method achieves a hit rate of xxx in recovering the true optimal temperature and beating down all fixed temperature settings. Finally, we employ a stochastic process model to provide interpretability for our approach.
Multi-sample aggregation strategies, such as majority voting and best-of-N sampling, are widely used in contemporary large language models (LLMs) to enhance predictive accuracy across various tasks. A key challenge in this process is temperature selection, which significantly impacts model performance. Existing approaches either rely on a fixed default temperature or require labeled validation data for tuning, which are often scarce and difficult to obtain. This paper addresses the challenge of automatically identifying the (near)-optimal temperature for different LLMs using multi-sample aggregation strategies, without relying on task-specific validation data. We provide a comprehensive analysis of temperature’s role in performance optimization, considering variations in model architectures, datasets, task types, model sizes, and predictive accuracy. Furthermore, we propose a novel entropy-based metric for automated temperature optimization, which consistently outperforms fixed-temperature baselines. Additionally, we incorporate a stochastic process model to enhance interpretability, offering deeper insights into the relationship between temperature and model performance.
\end{abstract}

% tl,dr: By analyzing the behavior of various LLMs across different temperatures and tasks, we propose an entropy-based metric for automated temperature optimization in multi-sample aggregation strategies, eliminating the need for labeled validation data.
\section{Introduction}

\begin{figure}[h]
    \centering
    \begin{overpic}[trim=0cm 0cm 0cm 0cm,clip,angle=0,origin=c,width=.4\linewidth]{images/teaser_absolute.png}
        %  trim={<left> <lower> <right> <upper>}
        %  \put(horiz, vert)
        %  \put(horiz, vert){\rotatebox{90}{Text}}
        %
        \put(107, 32){$\mathbf{\to}$}
    \end{overpic}\hspace{1cm}
    \begin{overpic}[trim=0cm 0cm 0cm 0cm,clip,angle=0,origin=c,width=.4\linewidth]{images/teaser_translated_yellow.png}
        %  trim={<left> <lower> <right> <upper>}
        %  \put(horiz, vert)
        %  \put(horiz, vert){\rotatebox{90}{Text}}
        %
    \end{overpic}
    \caption{Using translation methods, a controller trained on an environment with a given visual variation \textit{(left)} can be reused without any training or fine-tuning on a different environment (\textit{right}) with comparable performance. In red we see the trajectory of a car driven by the same controller when connected to two different encoders, one for each visual variation.
    }
    \label{fig:teaser}
\end{figure}

Deep Reinforcement Learning (RL) has enabled agents to achieve remarkable performance in complex decision-making tasks, from robotic manipulation to high-dimensional games (Mnih et al., 2015; Silver et al., 2017). 
Although recent RL techniques achieved strong improvements over sample efficiency \citep{yarats2021drqv2, kostrikov2020image}, training new agents remains a costly process, both in computational and temporal terms.
Despite these advances, most methods still require at least partial retraining when dealing with domain shifts such as visual appearance, reward functions, or action spaces \citep{pmlr-v97-cobbe19a, zhang2020learning}. These domain changes typically require expensive retraining, which can be prohibitive for real-world settings that require millions of interactions.

A variety of approaches have been proposed to address these shifting conditions. Domain randomization \citep{tobin2017domain, sadeghi2016cad2rl} trains agents across diverse visual styles or physics settings, promoting invariant features but demanding broader coverage of possible variations. Multi-task RL \citep{parisotto2015actor, teh2017distral} attempts to learn shared representations across multiple tasks.

In the supervised setting, recent representation learning techniques \citep{Moschella2022-yf,maiorca2023latent, norelli2022b, cannistraci2023bricks}, show that it is possible to zero-shot recombine encoders and decoders to perform new tasks across different modalities (images, text..) and tasks (classification, reconstruction) and even architectures.
In RL, methods adopting the relative representation framework \citep{Moschella2022-yf} have shown promising results in adapting encoders to different controllers with zero or few-shots adaptation, for robotic control from proprioceptive states \citep{jian2021adversarial} or for playing games in the Gymnasium suite \citep{towers2024gymnasium} from pixels \citep{ricciardi2025r3lrelativerepresentationsreinforcement}.
These methods, however, still require training models to use the new relative representations.

By contrast, \cite{maiorca2023latent} suggest that modules from independently trained neural networks can be connected via a simple linear or affine transformation, with no training constraint or fine-tuning required, if such transformations can be reliably estimated from a small set of “anchor” samples, pairs of states or observations deemed semantically equivalent.

Our main contribution is the implementation of a RL method based on semantic alignment to map between latent spaces of different neural models, so that their encoders and controllers can be stitched with the goal of creating new agents that can act on visual-task combinations never seen together in training. This includes the use of the transformations to map modules from different networks, and the collection of anchor samples used to estimate these transformations. We call our method Semantic Alignment for Policy Stitching (\textbf{SAPS}).
We perform analyses and empirical tests on the CarRacing and LunarLander environments to show the performance of new agents created via zero-shot stitching of encoders and controllers trained on different visual-task variations, demonstrating significant gains compared to existing zero-shot methods.
\section{Related Work}
\label{sec:related}


\noindentbold{2D visual foundation models}
In recent years, we have witnessed the emergence of large pretrained models—so-called foundation models that are trained on large-scale datasets and serve as a \textit{foundation} for many downstream tasks.
These models demonstrate remarkable versatility across multiple modalities, including language~\cite{team2023gemini,touvron2023llama,touvron2023llama2,dubey2024llama3,vicuna2023,radford2019language,brown2020language,chung2024scaling,achiam2023gpt,bai2023qwen,yang2024qwen2,jiang2023mistral,jiang2024mixtral}, vision~\cite{sam,ravi2024sam,dino_v1,oquab2023dinov2,zou2024segment,rombach2022high,ho2020denoising,nichol2021improved,songdenoising,songscore}, audio~\cite{deshmukh2023pengi,zhang2023speechgpt,rubenstein2023audiopalm,borsos2023audiolm}. 
Furthermore, they enable multi-modal reasoning capabilities that bridge across different modalities~\cite{girdharImageBindOneEmbedding2023,Qwen-VL,llava,radfordLearningTransferableVisual2021,jia2021scaling,team2024gemini}.
Among these models, those that operate on visual modalities are known as visual foundation models (VFM).
VFMs excel in various computer vision tasks such as image segmentation~\cite{sam,ravi2024sam,zou2024segment,zou2023generalized,cheng2021per,cheng2022masked,jain2023oneformer,li2024semantic}, object detection~\cite{liu2023grounding,carion2020end}, representation learning~\cite{dino_v1,oquab2023dinov2}, and open-vocabulary understanding~\cite{radfordLearningTransferableVisual2021,li2022language,ghiasi2022scaling,ram,ram_pp,yu2023convolutions,kang2024defense,naeem2024silc,cho2024cat}.
When integrated with large language models, they enable sophisticated visual reasoning and natural language interactions~\cite{llava,Qwen-VL,girdharImageBindOneEmbedding2023,team2024gemini,guo2024regiongpt,yuan2024osprey,you2023ferret}.
We use such vision language models to construct open vocabulary segmentation and captions for point clouds based on multiview images.







\noindentbold{Open-vocabulary 3D segmentation}
Building on the success of 2D VFMs, recent work have extended open-vocabulary capabilities to 3D scene understanding.
OpenScene~\cite{Peng2023OpenScene} first introduced zero-shot 3D semantic segmentation by distilling knowledge from language-aligned image encoders~\cite{li2022language,ghiasi2022scaling}.
Subsequent methods~\cite{ding2022pla,yang2024regionplc,jiang2024open} leverage multiview images to generate textual captions, which then serve as training supervision.
However, these methods face challenges in generating high-quality 3D mask-text pairs at scale.
For open-vocabulary 3D instance segmentation, existing methods~\cite{takmaz2023openmask3d,nguyen2024open3dis,huang2024openins3d} typically rely on closed-vocabulary proposal networks such as Mask3D~\cite{schult2023mask3d}, which inherently constrains their ability to detect novel object categories. 
Moreover, these methods leverage 2D VFMs like CLIP~\cite{radfordLearningTransferableVisual2021} for region classification by projecting 3D regions onto multiple 2D views.
This approach requires both 2D images and 3D point clouds during inference. Additionally, it necessitates multiple inferences of large 2D models on projected masks, resulting in high computational costs. 
We address these limitations by developing the first single-stage open-vocabulary 3D instance segmentation model that operates directly in 3D without ground truth labels, using our \dataname dataset and Segment3D~\cite{huang2024segment3d} proposals.

\noindentbold{3D vision-language datasets}
Several datasets align 3D scenes with textual annotations to facilitate language-driven 3D understanding. 
ScanRefer~\cite{chen2020scanrefer}, ReferIt3D~\cite{achlioptas2020referit_3d} and EmbodiedScan~\cite{wangEmbodiedScanHolisticMultiModal2023} provide fine-grained object-level localization through detailed referential phrases, while ScanQA~\cite{azuma2022scanqa} targets spatially grounded question-answering. 
In contrast, SceneVerse~\cite{jiaSceneVerseScaling3D2024} and MMScan~\cite{lyu2024mmscan} employ large-language models or vision-language models to partially automate annotation.
Despite leveraging advanced models, these datasets depend significantly on costly human annotations derived from closed-vocabulary sources, limiting their support for open-vocabulary and scalability for large-scale 3D segmentation tasks.

\begin{figure}[ht]
    \centering
    % Top sub-figure
    \begin{subfigure}
        \centering
        \includegraphics[width=0.48\textwidth]{figs/correlation_math_v2.png}
    \end{subfigure}
    \vskip -1em  % vertical space
    % Bottom sub-figure
    \begin{subfigure}
        \centering
        \includegraphics[width=0.48\textwidth]{figs/correlation_code_v2.png}
    \end{subfigure}
    \vskip -2em
    \caption{Plot of midpoints of optimal temperature ranges (x-axis, sample size 128) vs. distances between models and tasks (y-axis). A strong negative correlation is observed on the MATH and MBPP datasets, with correlation coefficients of -0.895 and -0.777.}
    \label{fig: correlation code generation}
    \vspace{-8mm}
\end{figure}
\section{Correlation Between Model Training and Optimal Temperature}
\label{sec: 3}

Multi-sample aggregation strategies—commonly used in problem-solving, code generation, and related domains—leverage information from multiple samples, which helps escape local minima and improve robustness. In these settings, \emph{sample diversity} becomes crucial: a diverse set of candidate samples increases the likelihood that the correct solution appears in the pool, rather than repeating the same mistake. The \emph{temperature} parameter is a primary lever for controlling this diversity.

We hypothesize that how a model is trained impacts the optimal temperature for multi-sample inference strategies. In particular, a more specialized or fine-tuned model can safely explore higher temperatures without drifting into low-quality outputs. In contrast, a general-purpose model typically benefits from a lower temperature to remain focused on relevant content.

We investigate this in two steps: In Section~\ref{sec: temperature varies}, we show that the optimal temperature varies for a base, instruction-tuned, and fine-tuned model. Then in Section~\ref{sec: temperature correlation}, we establish a general relationship between a model’s proximity to the target task and its corresponding optimal temperature. Our key insight is that token-level entropy is a proxy of distance from a task, which motivates our entropy-based method for automatic temperature selection in Section~\ref{sec: 4}.

%Our empirical studies reveal that different language models under various training stages respond differently to performance changes in temperature. In particular, a strong relationship exists between how closely a model’s training data aligns with the target task and the temperature that yields optimal performance. Intuitively, a more specialized or fine-tuned model can safely explore higher temperatures without drifting into low-quality outputs. In contrast, a general-purpose model typically benefits from a lower temperature to remain focused on relevant content. This observation underpins our hypothesis that the more closely a model’s training data aligns with a target task, the higher the temperature at which sampling-based strategies will excel.

\subsection{Optimal Temperature Range Varies}
\label{sec: temperature varies}

%First, we show that the best temperature varies based on whether a model is a base model, an instruction-tuned model, or a model fine-tuned specifically for the target task.

%To do so, we evaluate the accuracy of each model using multi-sample aggregation with a variety of temperatures. As we see in the curve in Figure .. the best temperature varies by model. For example, the best temperature for the base model is 0.5 when 128 samples are used, compared to 1.0 for the task-specific model.

%We also make two observations that we rely on in the subsequent experiments. First, as shown in the heatmap, several temperatures may be “optimal”, in that there is a very small performance gap. Hence we consider an optimal temperature window (define).

%Second, the optimal temperature varies based on the number of samples. However, we noticed that the optimal temperature stabilizes after a sufficient number of samples (here 32). Hence we will focus on the sufficient sample setting (namely 128) in the rest of the paper.

We first demonstrate that the optimal sampling temperature varies by model type. We test three \emph{Mistral-7B} variants: the \emph{pretrained base model}, the \emph{instruction-finetuned version (Mistral-7B-Instruct)}, and a \emph{task-finetuned model for MATH}\footnote{Model link: \href{https://huggingface.co/peiyi9979/mistral-7b-sft}{https://huggingface.co/peiyi9979/mistral-7b-sft}}~\cite{wang2024math}. Each model is evaluated using multi-sample aggregation across different temperatures.
Figure~\ref{fig: teaser}(a) presents the accuracy heatmap for the Mistral-7B-Instruct model on the MATH dataset. At smaller sample sizes, lower temperatures tend to produce better accuracy. However, higher temperatures can yield better results as the sample size increases. For a fixed sample size, the accuracy curve follows a single-peak pattern: it rises as temperature increases and peaks, and then gradually declines, staying relatively steady near the peak.

Since the single-peak behavior, we define the \textbf{$\epsilon$-optimal temperature range}. This range encompasses temperatures $T$ where the accuracy $A(T)$ is no less than $A(T^*) - \epsilon$, with $A(T^*)$ representing the peak accuracy. Given the curve's single-peak nature, this range forms an interval around $T^*$. For our analysis, we set $\epsilon = 0.02$, effectively capturing the temperatures close to the peak where the accuracy remains relatively high.

We then plot the midpoint of this optimal temperature range for each model variant and various sample sizes (Figure~\ref{fig: teaser}(b)). We observe that the pretrained model has the lowest midpoint, the instruction-finetuned model has a higher midpoint, and the task-finetuned model has the highest. Another observation is that optimal temperature ranges change slowly once beyond a sample size of 32. Therefore, we choose a sample size of 128 in our following experiments to ensure stable performance in the rich-sample setting.

From these observations, we hypothesize a general relationship between how closely a model is tuned to a particular task and the temperature that yields the best accuracy. We discuss this hypothesis further in the next section.

\subsection{Correlation Between Training-Task Similarity and Optimal Temperature}
\label{sec: temperature correlation}
Our goal is to establish a general relationship between a model’s learned distribution and its optimal temperature for a task. Our key intuition is that token-level entropy can serve as a surrogate for a model's `distance' from a target task and that this distance helps identify the optimal temperature.

Specifically, we define a distance metric that measures how similar a model’s training data is to a given task. Let \(\mathcal{T} = \{X_1, ..., X_k\}\) be the task with $k$ problem instances. We define this distance \(\mathcal{D}(\mathcal{M}, \mathcal{T})\) as the average of token-level entropy \(\mathcal{H}(.)\) of the language model \(\mathcal{M}\) when generating the answers \(\mathcal{A} = \{Y_1, ..., Y_k\}\) for the problems in \(\mathcal{T}\):
\begin{align}
\mathcal{D}(\mathcal{M}, \mathcal{T})
&=
\frac{1}{k}
\sum_{i=1}^{k}
\left[
  \frac{1}{|Y_i|}
  \sum_{j=1}^{|Y_i|}
  \mathcal{H}\bigl(p_\mathcal{M}\bigl(\cdot \mid X_i,\,Y_{i,<j}\bigr)\bigr)
\right],
\end{align}
\vspace{-5mm}
where
\begin{align}
\label{eq: entropy}
\mathcal{H}(p)&=
  -\sum_{v \in p} 
     p\bigl(v) 
     \log p\bigl(v\bigr).
\end{align}
To avoid bias toward ground-truth references, we use model-generated sequences \(\{Y_i\}\) instead of official gold solutions. Meanwhile, the distance is measured at a low temperature $T=0.5$ to ensure the generation stability.

We evaluated several language models on the MATH and MBPP datasets, including pretrained, instruction-finetuned, and task-finetuned models. Figure~\ref{fig: correlation code generation} plots the midpoint of the optimal temperature range against our distance metric, demonstrating a strong negative correlation. Specifically, across our model set, the correlation on MATH is \(-0.895\), while on MBPP it is \(-0.777\).

In practice, this suggests using a higher temperature (e.g., \(T = 0.9 \sim 1.1\)) for task-finetuned models and a lower temperature (e.g., \(T = 0.5 \sim 0.7\)) for more general-purpose models (pretrained or instruction-finetuned).

\iffalse
\subsection{Link to Token Probability Distribution}

Given the numerous training methods available for language models, the training data and target tasks can vary significantly. During inference, the distribution of token probabilities also differs. As shown in Figure~\ref{fig: teaser}~(a), we applied three types of language models to the MATH dataset and measured the average probability of the top 20 tokens, each of the models has a different purpose (a pretrained model, an instruction-finetuned model, and a task-finetuned model). For the pretrained model, the token probability distribution is relatively flat compared with the instruction-finetuned model, which has been more specialized for question answering. As expected, the token probability distribution becomes even more concentrated when a model is specifically fine-tuned on the task’s training set.

Differences in token probability distributions also affect performance in sample-based inference strategies. Intuitively, a flat token probability distribution model may need a lower temperature to focus on the most relevant tokens. In contrast, a model with a concentrated token probability distribution may require a higher temperature to explore more diverse tokens. As shown in Figure~\ref{fig: teaser}~(b), we evaluated the three models under various sampling sizes and temperatures and then visualized the results in a heatmap. These results indicate that the optimal temperature for the pretrained model is lower than for the instruction-finetuned model, while the task-finetuned model benefits from the highest temperature.
\fi
\begin{figure}[ht]
\centering
\includegraphics[width=0.48\textwidth]{figs/entropy_curve_v2.png}
\vspace{-6mm}
\caption{\textbf{Entropy Curve Characteristics.} 
\textbf{(a)} The token-level entropy \(\mathcal{H}\) (solid blue line) increases slowly at lower temperatures and then jumps sharply at a critical turning point. In contrast, the entropy for a fixed (greedy) generation stays low (dotted blue line). \(\log(\mathcal{H})\) (red line) reveals a transition from concavity to convexity that aligns with the sharp increase in \(\mathcal{H}\), marking the entropy turning point (EntP). \textbf{(b)} EntP hits the best temperature, and it varies between different models.}
\label{fig: entropy_curve}
\vspace{-5mm}
\end{figure}
\section{Entropy-Based Automatic Temperature Selection}
\label{sec: 4}
Determining an optimal sampling temperature is crucial in multi-sample aggregation strategies, yet existing approaches often rely on labeled data or tuning on a validation set. This reliance becomes problematic when no such data are available. 
In this section, we show how to leverage token-level entropy as an intrinsic property to pinpoint a suitable temperature without labeled data. We first demonstrate a spike on \emph{token-level entropy} as a signal of quality collapse in Section~\ref{sec: spike}. Then develop a method that automatically selects temperature using an \emph{entropy turning point (EntP)} derived from the spike in Section~\ref{sec: turn}. Finally, we applied a stochastic process model to explain the mechanism of our algorithm in Section~\ref{sec: toy model}.
\subsection{Entropy Spike as an Indicator of Quality Collapse}
\label{sec: spike}
First, we discover a surprising phenomenon that we call the entropy spike. Specifically, increasing the temperature smoothly increases the model’s entropy, until a dramatic spike where the entropy rapidly increases. We believe the spike is a good indicator of sample quality collapse.

As illustrated in Figure~\ref{fig: entropy_curve}(a), we calculate the token-level entropy at different temperature levels (solid blue line). To reduce computational overhead, we compute the entropy only over the top-$K$ tokens (with the highest probabilities) at each step, setting $K=1000$ in all subsequent experiments. The entropy curve remains stable for lower temperatures but then shows a sudden rise. One might attribute this behavior to temperature’s role in flattening the distribution (Equation~\ref{Formula 1}). However, the following analysis indicates that this spike reflects a substantial change in the model’s next-token distribution.

Specifically, we constrain the model to generate the same outputs produced by greedy decoding while evaluating entropy under a higher temperature (dotted blue
line). If temperature alone were responsible for the entropy spike, these fixed outputs would yield a similarly high entropy. However, as shown in Figure~\ref{fig: entropy_curve}(a), we observe a significant gap between these two entropy curves, indicating that the actual sampling distribution undergoes a large shift.

Thus, we infer that the sudden rise in the entropy curve implies a substantial drop in sample quality. Setting the temperature around this sudden rise can balance sufficient diversity without a large quality drop, which is suitable for multi-sample aggregation strategies.

\subsection{Turning Point Temperature Selection (\textsc{TURN})}
\label{sec: turn}
Given the token-level entropy curve of a language model on a specific task, how can we identify a suitable temperature for multi-sample aggregation strategies? Inspired by the difference in the shapes of the entropy curve: When the temperature remains low, the entropy increases \emph{flatly}. However, when the sampling temperature is near the spike, the entropy increases \emph{(super)-exponentially}, implying a quality drop in samples. Therefore, after taking the logarithm of the entropy curve (shown in Figure~\ref{fig: entropy_curve}(a), red line), the flat part becomes concave while the exponentially-increase part becomes convex. We define the \emph{entropy turning point (EntP)} as the temperature where the log entropy curve becomes convex. Figure~\ref{fig: entropy_curve}(b) tests the llemma-7b base model and its task-finetuned variant\footnote{Model link: \href{https://huggingface.co/ScalableMath/llemma-7b-sft-metamath-level-1to3-hf}{https://huggingface.co/ScalableMath/llemma-7b-sft-metamath-level-1to3-hf}}~\cite{sun2024easy}, and EntP matches the position with the highest accuracy and varies between different models. Based on EntP, we develop a new method for automatic temperature prediction in multi-sample aggregation strategies, called Turning Point Temperature Selection (\textsc{TURN}).

The optimal temperature should be around EntP to achieve both sample quality and diversity. At the same time, we found that some aggregation methods may be more tolerant of quality drops (e.g., for best-of-N, only one sample is enough to be correct). So we added a small adaptation factor $\beta$ based on the aggregation function, and it is set to $0$ and $+0.1$ for majority voting and best-of-N, respectively. The aggregation adaptation for best-of-N is calculated in the MATH dataset but can be directly applied to other tasks. Refer to Appendix~\ref{app: bias} for details.

Specifically, given a language model $\mathcal{M}$, a task $\mathcal{T}=\{X_1,\ldots,X_k\}$ with $k$ input instances, and an aggregation method $\mathcal{A}$.
To estimate the token-level entropy, we random sample $N$ times. In each time, we randomly choose an input instance $X_i$, and generate one sample by $\mathcal{M}$ under each candidate temperature $t_j = j \cdot t$ (with $t$ being the temperature interval and $j = 0,1,\ldots, J$, where $J=\lfloor t_{\max}/t\rfloor$). These entropies are then aggregated to calculate the average entropy $\mathcal{H}(t_j)$ at each temperature $t_j$. By taking the logarithm, we obtain $\ell(t_j) = \log \mathcal{H}(t_j)$.

Next, we identify EntP index $j^*$, where the second derivative of $\ell$ changes from negative to positive and select its corresponding temperature $j^*\cdot t$. Then we add the aggregation adaptation factor $\beta$ to form the final prediction.
%The corresponding temperature $j^* \cdot t$ is then adjusted with a bias $\beta_{\mathcal{A}}$ related to the aggregation method $\mathcal{A}$, resulting in the final prediction $t^*_{\text{biased}} = t^* + \beta_{\mathcal{A}}$. 
The pseudocode for our algorithm is listed in Algo. \ref{alg:auto find}.
\begin{algorithm}
\caption{Turning Point Temperature Selection \textsc(TURN)}
\label{alg:auto find}
\begin{algorithmic}[1]
\STATE {\bfseries Input:} Language Model $\mathcal{M}$, task $\mathcal{T}=\left(X_1, ..., X_k\right)$, temperature interval $t$, maximum temperature $t_{\max}$, sample size $N$, aggregation method $\mathcal{A}$.
\STATE {\bfseries Output:} Predicted Temperature $T_{\text{pred}}$.
\STATE Compute $J = \lfloor t_{\max}/t \rfloor$ \COMMENT{Number of choices}
\STATE Initialize entropy list $\mathcal{E} = []$
\FOR{$n = 1$ to $N$}
    \STATE Randomly select $X_i$ from $\mathcal{T}$
    \FOR{$j = 0$ to $J$}
        \STATE Generate a sample $Y$ using $\mathcal{M}$ with $T = j\cdot t$
        \STATE Compute token-level entropy of $Y$, add to $\mathcal{E}[j]$
    \ENDFOR
\ENDFOR
\STATE Compute $\mathcal{H}(j)=\text{Mean}\left(\mathcal{E}(j)\right)$ for all $j$
\STATE Compute $\ell(j) = \log \mathcal{H}(j)$ for all $j$
\STATE Find $j^* = \arg\min_j \left( \frac{d^2\ell}{dt^2}>0 \right)$
\STATE Compute $t^* = j^* \cdot t$
\STATE Add adaptation factor $\beta_{\mathcal{A}}$: $T_{\text{pred}} = t^* + \beta_{\mathcal{A}}$
\STATE {\bfseries Return} $T_{\text{pred}}$
%\STATE {\bfseries Return} $t^*$
\end{algorithmic}
\end{algorithm}
\vspace{-5mm}

%\weihua{The turning point of the entropy curve means the quality of samples begins to quickly drop ..., due to the initial difference in sampling strategies, best-of-N is better ..., we give an $+\epsilon$ bias, and we simply set it to 0.1 in our experiment.}
\begin{figure}[ht]
\includegraphics[width=0.48\textwidth]{figs/toy_model.png}
\vspace{-5mm}
\caption{\textbf{Stochastic Process Model.} We run our process model in the setting: $N_0=10$, $N_1=30000$, $L_0=0$, $\sigma_0=1$, $L_1=-10$, and $\alpha=2$. \textbf{(a)} The entropy curve is similar to that of the real language model: flat at first, and then sharply increases. \textbf{(b)} We calculate the relation between temperature and the percentage of improper tokens in simulation, and the percentage of improper tokens quickly increases after EntP.}
\vspace{-5mm}
\label{fig: toy_model}
\end{figure}

\subsection{A Stochastic Process Model}
\label{sec: toy model}
We applied a stochastic process model to explain why the entropy curve exhibits a sudden spike and what that spike signifies.

Because inference is sequential, when the language model makes an error (for example, by sampling an improper token), it increases the likelihood of further mistakes. Meanwhile, the model may occasionally recover and return to a correct trajectory.

To simulate this process, we adopt a stochastic process model with \(K\) steps in sequential, generating a token in each step. At the start, the model has an initial error rate \(p = p_{\text{init}}\), representing the probability of selecting an improper token. At each step, if the model selects an improper token, the likelihood of further errors increases to \(1 - (1 - p)^\alpha\), where \(\alpha > 1\) is called the noise tolerance rate. Conversely, if the model selects a proper token, the error probability decreases to \(p^\alpha\) (but cannot be smaller than \(p_\text{init}\)).

To build a bridge between the temperature \(T\) and the initial error rate \(p_{\text{init}}\), we propose an estimation. All tokens are labeled proper or improper irrelevant to contexts, and the number of improper tokens (\(N_1\)) is much larger than that of proper tokens (\(N_0\)). In the beginning, proper tokens have high logits \(L_0\) with a variance \(\mathcal{N}(0, \sigma_0^2)\) to reflect the nature that there may be several proper next tokens with similar logits. Improper tokens have uniformly low logits \(L_1\). Then, the initial error rate \(p_{\text{init}}\) is determined as the probability of selecting an improper token based on the logits and temperature. 
During inference, all improper tokens equally share the error rate \(p\), while proper tokens account for the remaining probability based on their logits.

Using this setup, we can estimate the token-level entropy. As shown in Figure~\ref{fig: toy_model}(a), the simulated entropy curve (blue line) aligns well with the observed entropy curves of a real language model (Figure~\ref{fig: entropy_curve}(1) solid blue line). Meanwhile, Figure~\ref{fig: toy_model}(b) shows the relationship between the temperature and the percentage of improper tokens, which rises quickly after EntP. This observation suggests that, before EntP, increasing the temperature can help explore the proper tokens. However, after EntP, the increase in the percentage of improper tokens makes the model uncertain and creates errors, implying a quick drop in sample quality. The behavior of the stochastic process model is consistent with our observations of language models, proving that token-level entropy is a good indicator of sample quality. Detailed formulas and experiments can be found in Appendix~\ref{app: toy model}.

\iffalse
\subsection{Balancing Diversity and Quality}
\begin{figure}[ht] 
\centering 
\includegraphics[width=0.48\textwidth]{figs/quality_diversity.png} 
\caption{\textbf{Diversity and Quality Dynamics}: Diversity and quality trends for four language models across varying temperatures, and the suitable temperature ranges are marked in brackets after LM names. The Answer Diversity demonstrates a (sub-)linear increase, whereas sample quality remains stable initially before experiencing a sharp decline. We also noticed that models specifically tuned on the dataset (i.e., (d)) may be tolerant to temperature, resulting in a large suitable temperature range.}
\label{fig: diversity and quality}
\end{figure}
Intuitively, increasing the sampling temperature enhances the diversity of generated answers but tends to reduce their quality. To quantify these two aspects, we define:

\begin{itemize}
    \item \textbf{Quality:} The average accuracy of a single sample.
    \item \textbf{Diversity:} The Shannon entropy of a set of answers. We notice that, at high temperatures, some generated answers become incomplete or invalid. Therefore, we ignore these samples and apply the Miller–Meadow correction~\cite{miller1955note} to mitigate the bias introduced by ignorance.
\end{itemize}
Figure~\ref{fig: diversity and quality} illustrates how sample quality and diversity change with temperature for four language models. Generally, quality remains relatively stable as the temperature increases, then experiences a sharp decline. In contrast, diversity typically grows approximately linearly or sub-linearly but may drop at very high temperatures, where the model seldom produces coherent outputs for answer parsing.

From these observations, one might choose a temperature that maintains a reasonable level of quality while maximizing diversity. However, such a selection relies on accessing quality metrics from a validation set, which is impossible when labeled data are unavailable.
\fi
\section{Experiments}
\subsection{Dataset}
We employ the AudioCaps \cite{kim2019audiocaps}, and Clotho \cite{drossos2020clotho} for our experiments. AudioCaps includes around 49,000 audio samples, each lasting about 10 seconds. Each audio is paired with a single sentence in the training set, while in both the validation and test sets, each audio has five associated sentences. The Clotho dataset consists of 6,974 audio samples, each ranging from 15 to 30 seconds long and annotated with five sentences. It is split into 3,839 training samples, 1,045 validation samples, and 1,045 test samples. 

Additionally, to assess our scheme's performance in the ML-ATR task, we use the Deepseek \cite{bi2024deepseek} API to translate the text from AudioCaps and Clotho into seven widely spoken languages, including French (fra), German (deu), Spanish (spa), Dutch (nld), Catalan (cat), Japanese (jpn), and Chinese (zho).

\subsection{Models}
\textbf{Audio Encoder}: 
We utilize the recently proposed CED-Base model \cite{dinkel2024ced}, a vision transformer with 86 million parameters for the Audio Encoder. Trained on Audioset through knowledge distillation from a large teacher ensemble, the model processes 64-dimensional Mel-spectrograms derived from a 16 kHz signal. It then extracts non-overlapping 16 × 16 patches from the spectrogram, resulting in 248 patches over a 10-second input (4 × 62).
\\
\textbf{Text Encoder}:
The key to multilingual audio-text retrieval is the text encoder's ability to handle texts in multiple languages. In this work, we focus solely on the SONAR-TE model \cite{duquenne2023sonar}. SONAR-TE generates a single vector bottleneck to encapsulate the entire text, avoiding the token-level cross-attention typically employed in conventional sequence-to-sequence machine translation models. The fixed-size text representation is derived by pooling the token-level outputs from the encoder. In the following sections, SONAR refers specifically to the text encoder.

\subsection{Setup}
We use ML-CLAP \cite{yan2024bridging} as the baseline, which is the state-of-the-art for ML-ATR tasks. To have a fair comparison, the model is initialized using the pre-trained weights of ML-CLAP and is further fine-tuned on our multilingual Audiocaps and Clotho datasets using three training methods: ML-CLAP, proposed CACL, and proposed KCL.

All models were fine-tuned for 10 epochs on a single A100 80GB PCIe GPU with a batch size of 24, a learning rate of $5 \times 10^{-6}$, using the Adam optimizer. The temperature hyperparameter $\tau$ was set to 0.07 for all configurations. The audio was sampled at $1.6\times 10^{4}$. We selected the model with the best recall performance during the fine-tuning period for each scheme to perform the experiments.

\subsection{Evaluation Metric}
We use the recall of rank k (R@k) and the average precision of rank 10 (mAP10) as the metrics for the retrieval performance of the model to show that reducing data distribution errors improves the retrieval performance in each language. R@k refers to the fact that for a query, R@k is 1 if the target-value item occurs in the first k retrieved items, and 0 otherwise. mAP10 calculates the average precision of all the queries among the first 10 retrieved results. With these two metrics, we can comprehensively evaluate the retrieval performance of the model on multilingual datasets. 

To assess the consistency of the embedding space across languages, we use three metrics: embedding space gap $\vec{\triangle}_{gap,k}$ \cite{liang2022mind}, average embedding distance $\vec{\triangle}_{dis,k}$, mean rank variance (MRV). The computation of $\vec{\triangle}_{gap,k}$, $\vec{\triangle}_{dis,k}$ and MRV is shown below:


\begin{equation}
\small
    \begin{aligned}
        \vec{\triangle}_{gap,k}=\frac 1 N \sum^N_{i=1}g_\phi(t_{i1})-\frac 1 N \sum^N_{i=1}g_\phi(t_{ik}),
    \end{aligned}
\end{equation}

\begin{equation}
\small
    \begin{aligned}
        \vec{\triangle}_{dis,k}=\frac 1 N \sum^N_{i=1}||g_\phi(t_{i1})-g_\phi(t_{ik})||,
    \end{aligned}
\end{equation}

\begin{equation}
\small
    \begin{aligned}
        MRV=\frac 1 {NK} \sum^N_{i=1}\sum^K_{k=1}|Rank_{ik}-\overline{Rank_j}|^2.
    \end{aligned}
\end{equation}

$\vec{\triangle}_{gap,k}$ and $\vec{\triangle}_{dis,k}$ denotes the embedding space gap and average embedding distance between English and $k$-th language respectively. $Rank_{ik}$ denotes the similarity ranking of the $k$-th language under the $i$-th data, and $\overline{Rank_i}$ denotes the average similarity ranking under the $i$-th data.

\begin{table*}[ht]
\caption{Recall and precision results for baseline and our method under multilingual AudioCaps and Clotho dataset}
\small
\centering
\begin{tabular}{c|c|ccc|ccc|ccc|ccc}
\hline
\multirow{3}{*}{\rotatebox{90}{\textbf{Scheme}}} & \multirow{3}{*}{\textbf{Lang}} & \multicolumn{6}{c|}{\textbf{AudioCaps}} & \multicolumn{6}{c}{\textbf{Clotho}}\\ 
\cline{3-14} & & \multicolumn{3}{c|}{T2A} & \multicolumn{3}{c|}{A2T} & \multicolumn{3}{c|}{T2A} & \multicolumn{3}{c}{A2T}\\
\cline{3-14}
 & & R@1 & R@5 & mAP10 & R@1 & R@5 & mAP10 & R@1 & R@5 & mAP10 & R@1 & R@5 & mAP10 \\ \cline{1-14}
\multirow{9}{*}{\rotatebox{90}{ML-CLAP}} & eng & 47.31 & 80.65 & 61.44 & 64.91 &	90.54 &	38.62 &	25.98 &	54.5 & 38.15 & 34.03 &	61.05 &	21.19 \\ 
& fra & 45.88 &	78.92 &	60.01 &	61.65 &	89.39 &	37.90 &	24.42 &	52.51 &	36.24 &	30.95 &	57.59 &	19.66 \\ 
& deu & 45.60 &	79.49 &	59.93 &	62.65 &	88.76 &	37.88 &	24.08 &	52.61 &	36.40 &	31.62 &	57.40 &	19.39 \\ 
& spa & 45.00 &	79.32 &	59.62 &	63.04 &	88.86 &	37.38 &	24.05 &	52.75 &	36.22 &	31.43 &	57.98 &	19.65 \\ 
& nld & 45.88 &	79.64 &	59.92 &	62.50 &	90.33 &	37.72 &	23.88 &	51.53 &	35.73 &	31.40 &	57.98 &	19.58 \\ 
& cat & 44.36 &	77.89 &	58.58 &	61.65 &	87.60 &	36.43 &	22.83 &	50.84 &	34.80 &	30.91 &	56.43 &	18.26 \\ 
& jpn & 43.04 &	76.86 &	57.54 &	59.45 &	87.81 &	35.20 &	23.04 &	50.34 &	34.89 &	31.43 &	56.55 &	18.77 \\ 
& zho & 41.70 &	74.72 &	55.74 &	53.67 &	84.76 &	33.38 &	21.65 &	48.84 &	33.53 &	28.41 &	56.14 &	17.26 \\ \cline{2-14}
& avg & 44.84 &	78.43 &	59.09 &	61.19 &	88.50 &	36.81 &	23.84 &	51.74 &	35.74 &	31.27 &	57.64 &	19.22 \\ \hline

\multirow{9}{*}{\rotatebox{90}{our CACL}} & eng & 49.05 &	82.14 &	63.07 &	66.31 &	\textbf{91.49} &	39.41 &	26.36 &	55.19 &	38.68 &	34.71 &	61.34 &	\textbf{21.57} \\ 
& fra & 46.86 &	79.97 &	60.83 &	63.23 &	89.48 &	37.92 &	\textbf{24.90} &	\textbf{53.09} &	36.67 &	\textbf{32.40} &	58.55 &	19.85 \\ 
& deu & 46.21 &	80.08 &	60.62 &	63.13 &	\textbf{89.91} &	38.14 &	24.51 &	52.86 &	36.52 &	\textbf{33.36} &	58.07 &	19.49 \\ 
& spa & 46.68 &	80.52 &	60.90 &	63.23 &	\textbf{90.12} &	37.45 &	\textbf{24.59} &	52.71 &	\textbf{36.72} &	32.40 &	58.17 &	19.75 \\ 
& nld & 47.41 &	80.23 &	61.22 &	63.23 &	\textbf{90.86} &	37.95 &	24.15 &	51.75 &	36.05 &	32.21 &	58.65 &	19.5 \\ 
& cat & 45.27 &	78.61 &	59.43 &	61.23 &	88.44 &	36.49 &	23.28 &	51.42 &	35.17 &	30.67 &	56.05 &	18.67 \\ 
& jpn & 44.76 &	78.50 &	58.97 &	61.55 &	88.67 &	34.91 &	23.36 &	51.53 &	35.28 &	\textbf{31.82} &	\textbf{58.26} &	18.99 \\ 
& zho & 42.01 &	76.02 &	56.23 &	56.40 &	86.65 &	33.93 &	22.50 &	49.42 &	34.01 &	27.69 &	\textbf{57.59} &	17.48 \\ \cline{2-14}
& avg & 46.03 &	79.50 &	60.15 &	62.28 &	89.45 &	37.02 &	24.20 &	52.24 &	36.27 &	31.90 &	58.33 &	19.41
 \\ \hline


\multirow{9}{*}{\rotatebox{90}{our KCL}} & eng & \textbf{49.68} &	\textbf{82.44} &	\textbf{63.34} &	\textbf{66.59} &	91.34 &	\textbf{40.52} &	\textbf{26.67} &	\textbf{55.46} &	\textbf{38.97} &	\textbf{36.34} &	\textbf{64.13} &	21.36 \\
& fra & \textbf{47.79} &	\textbf{80.52} &	\textbf{61.53} &	\textbf{63.41} &	\textbf{}\textbf{89.57} &	\textbf{39.21} &	24.61 &	52.73 &	\textbf{36.79} &	31.82 &	\textbf{60.76} &	\textbf{20.02} \\ 
& deu & \textbf{47.81} &	\textbf{80.81} &	\textbf{61.78} &	\textbf{63.34} &	89.28 &	\textbf{39.02} &	\textbf{24.90} &	\textbf{53.25} &	\textbf{37.02} &	33.17 &	\textbf{59.61} &	\textbf{19.90} \\ 
& spa & \textbf{47.33} &	\textbf{80.67} &	\textbf{61.49} &	\textbf{63.76} &	89.39 &	\textbf{38.73} &	24.31 &	\textbf{52.96} &	36.55 &	\textbf{33.36} &	\textbf{61.25} &	\textbf{20.27} \\ 
& nld & \textbf{47.92} &	\textbf{}\textbf{80.76} &	\textbf{61.70} &	\textbf{63.55} &	90.52 &	\textbf{39.14} &	\textbf{24.53} &	\textbf{52.51} &	\textbf{36.61} &	\textbf{33.55} &	\textbf{62.30} &	\textbf{19.98} \\ 
& cat & \textbf{46.44} &	\textbf{79.62} &	\textbf{60.42} &	\textbf{62.71} &	\textbf{89.49} &	\textbf{37.65} &	\textbf{23.67} &	\textbf{51.86} &	\textbf{35.70} &	\textbf{31.53} &	\textbf{57.98} &	\textbf{1}\textbf{8.90} \\ 
& jpn & \textbf{45.27} &	\textbf{78.86} &	\textbf{59.49} &	\textbf{62.28} &	\textbf{89.16} &	\textbf{36.81} &	\textbf{23.65} &	\textbf{52.17} &	\textbf{35.68} &	31.25 &	57.50 &	\textbf{19.49} \\ 
& zho & \textbf{42.25} &	\textbf{76.38} &	\textbf{56.75} &	\textbf{57.66} &	\textbf{87.28} &	\textbf{34.79} &	\textbf{23.09} &	\textbf{49.90} &	\textbf{34.60} &	\textbf{30.48} &	56.34 &	\textbf{17.85} \\ \cline{2-14}
& avg & \textbf{46.81} &	\textbf{80.00} &	\textbf{60.81} &	\textbf{62.91} &	\textbf{89.50} &	\textbf{38.23} &	\textbf{24.42} &	\textbf{52.60} &	\textbf{36.49} &	\textbf{32.68} &	\textbf{59.98} &	\textbf{19.72} \\ \hline
\end{tabular}
\label{Tab:recall and meanr}
\end{table*}

\subsection{Evaluation Result of Recall and Precision}
We present a detailed numerical comparison analysis of the experiment results in Tab \ref{Tab:recall and meanr}, focusing on the performance improvements of our proposed methods, CACL and KCL, over the baseline ML-CLAP across various languages and datasets.

\subsubsection{Analysis of Evaluation Results}
Overall, the proposed CACL and KCL consistently outperform ML-CLAP across the majority of languages and datasets in terms of recall at 1 (R@1), recall at 5 (R@5), and mean average precision at the top 10 results (mAP10) for both Text-to-Audio (T2A) and Audio-to-Text (A2T) tasks. Notably, our proposed KCL achieves state-of-the-art performance, delivering a 5\% improvement in R@1 for the English-oriented monolingual ATR task and a 4.3\% improvement in R@1 for the multilingual ATR task compared to ML-CLAP. This experimental result corroborates our theoretical analysis of the weighting error in Sect. \ref{Sect:Mathematical Demonstration about Inconsistency}. Here is the detailed analysis:

CACL's average metrics across languages are higher than ML-CLAP, while KCL's average metrics across languages have further improvement compared to CACL. Our theoretical analyses in Sect can explain this phenomenon. \ref{Sect:Mathematical Demonstration about Inconsistency}:
\begin{itemize}
    \item CACL uses audio and text together as the anchor point for modality alignment in other languages, which can effectively reduce the data distribution error and modality alignment error, thus achieving better modality alignment results and improved metrics compared to ML-CLAP.
    \item Compared to CACL, which mitigates data distribution errors, KCL theoretically eliminates these errors. As a result, KCL achieves superior modality alignment compared to CACL, leading to further improvements in both recall and precision.
\end{itemize}


\subsubsection{Analysis of Special Situations}

\textbf{Occasional Metric Anomalies}: We observed occasional anomalies where a small proportion of KCL metrics were lower than CACL metrics, and some CACL metrics were lower than ML-CLAP metrics. We attribute these discrepancies to noise in the dataset. Specifically, the weight error in Eq. \eqref{Eq:weight error} represents the difference between the current and optimal model weights for fitting the training data. If the dataset is too noisy, the optimal weights may not improve the test set's performance. As a result, KCL and CACL, which have lower weight errors, may still underperform ML-CLAP on certain metrics. The higher frequency of such anomalies in the noisier Clotho dataset, compared to Audiocaps, supports this explanation. Given that these anomalies are rare among the 108 evaluated metrics, we consider them acceptable and conclude that they do not impact the overall performance advantage of CACL and KCL in the ML-ATR task.

\textbf{Performance Gaps Across Languages}: The lower metrics for Japanese and Chinese in Tab. \ref{Tab:recall and meanr} are mainly due to their significant syntactic differences from other languages, making them harder for the model to learn. Expanding the dataset for these languages could improve the model's performance by providing more representative data.

\textbf{Better Replicated Performance}: Compared to the original paper, our replicated ML-CLAP model achieves significant improvements across all metrics, mainly due to differences in data quality. Compared to the SONAR-translated text used by baseline, the multilingual text we translated with LLM is of higher quality, which in turn can improve the retrieval performance of the model.

\begin{table}[ht]
\caption{Results of spatial differences in the embedding of other languages and English}
\small
\centering
\begin{tabular}{c|c|cc|cc}
\hline
\multirow{3}{*}{\rotatebox{90}{\textbf{Scheme}}} & \multirow{3}{*}{\textbf{Lang}} & \multicolumn{2}{c|}{\textbf{AudioCaps}} & \multicolumn{2}{c}{\textbf{Clotho}}\\ 
\cline{3-6} & & \multicolumn{2}{c|}{E2T} & \multicolumn{2}{c}{E2T}\\
\cline{3-6}
 & & Gap & Dis & Gap & Dis \\ \cline{1-6}
\multirow{8}{*}{\rotatebox{90}{ML-CLAP}} & fra & 0.199 & 0.094 & 0.120 & 0.301\\ 
& deu & 0.210 & 0.370 & 0.124 & 0.289\\ 
& spa & 0.147 & 0.290 & 0.117 & 0.284\\ 
& nld & 0.204 & 0.346 & 0.121 & 0.274\\ 
& cat & 0.151 & 0.357 & 0.121 & 0.307\\ 
& jpn & 0.237 & 0.445 & 0.123 & 0.353\\ 
& zho & 0.181 & 0.414 & 0.177 & 0.323\\ \cline{2-6}
& avg & 0.189 & 0.330 & 0.129 & 0.304\\ \hline

\multirow{8}{*}{\rotatebox{90}{our CACL}} & fra & 0.160 &  0.281 & 0.112 & 0.288\\ 
& deu & 0.194 & 0.334 & 0.103 & 0.261\\ 
& spa & 0.090 & 0.210 & 0.099 & 0.265\\ 
& nld & 0.172 & 0.325 & 0.106 & 0.255\\ 
& cat & 0.104 & 0.252 & 0.108 & 0.280\\ 
& jpn & 0.217 & 0.402 & 0.122 & 0.359\\ 
& zho & 0.192 & 0.381 & 0.159 & 0.352\\ \cline{2-6}
& avg & 0.161 & 0.312 & 0.115 & 0.294\\ \hline

\multirow{8}{*}{\rotatebox{90}{our KCL}} & fra & \textbf{0.145} & \textbf{0.274} & \textbf{0.094} & \textbf{0.261}\\ 
& deu & \textbf{0.155} & \textbf{0.290} & \textbf{0.084} & \textbf{0.231}\\ 
& spa & \textbf{0.081} & \textbf{0.192} & \textbf{0.084} & \textbf{0.230}\\ 
& nld & \textbf{0.148} & \textbf{0.285} & \textbf{0.072} & \textbf{0.204}\\ 
& cat & \textbf{0.092} & \textbf{0.245} & \textbf{0.087} & \textbf{0.243}\\ 
& jpn & \textbf{0.188} & \textbf{0.356} & \textbf{0.106} & \textbf{0.310}\\ 
& zho & \textbf{0.181} & \textbf{0.379} & \textbf{0.123} & \textbf{0.312}\\ \cline{2-6}
& avg & \textbf{0.141} & \textbf{0.288} & \textbf{0.092} & \textbf{0.255}\\ \hline
\end{tabular}
\label{Tab:embeddings gap}
\end{table}


\subsection{Evaluation Result of Consistency}
\subsubsection{Analysis of Embedding Space Consistency}
The results of the consistency metrics embedding space gap $\vec{\triangle}_{gap,k}$ and average embedding distance $\vec{\triangle}_{dis,k}$ are shown in Tab. \ref{Tab:embeddings gap}. In addition, we give a visualization of the embedding space in Appendix \ref{Appe:Embedding Space} and case analysis in Appendix \ref{Appe:Case Analysis} to further illustrate the effectiveness of ATRI in solving the consistency problem.

Smaller values of $\vec{\triangle}_{gap,k}$ and $\vec{\triangle}_{dis,k}$ indicate better alignment of a language's embedding space with English, leading to more consistent retrieval in the ML-ATR task. Compared to the baseline ML-CLAP, CACL achieves an average reduction of 12.9\% in Gap and 4.4\% in Dis, while KCL reduces Gap by 19.1\% and Dis by 14.3\%, demonstrating improved cross-language retrieval consistency.

\begin{table}[ht]
\caption{Results of Mean Rank Variance}
\small
\centering
\begin{tabular}{c|c|c}
\hline
\multirow{2}{*}{\textbf{Scheme}} & \textbf{AudioCaps} & \textbf{Clotho}\\ \cline{2-3}
 & MRV & MRV \\ \hline
ML-CLAP & 10.38 & 347.34 \\ \hline
CACL & 8.71 & 274.87 \\ \hline
KCL & \textbf{7.52} & \textbf{263.15} \\ \hline
\end{tabular}
\label{Tab:MRV}
\end{table}

\subsubsection{Analysis of Rank Consistency}
MRV quantifies the consistency of search rankings across languages, with lower values indicating more consistent results across languages. Unlike metrics based on embedding space, MRV offers a more direct assessment of model consistency in the ML-ATR task. As shown in Tab. \ref{Tab:MRV}, KCL achieves the lowest MRV, representing a 25.9\% reduction compared to ML-CLAP, while CACL achieves a 22.3\% reduction. This effectively shows that the inconsistency issue can be effectively mitigated by reducing the data distribution error.

We note that the MRV metrics under the Audiocaps dataset are significantly lower than Clotho's. This is due to the fact that the Clotho dataset is much noisier and more difficult to get consistent retrieval results across languages.

\begin{table}[ht]
\caption{Evaluation results in GPU memory overheads and time overheads}
\small
\centering
\begin{tabular}{c|cc|cc}
\hline
\multirow{2}{*}{\textbf{Scheme}} & \multicolumn{2}{c|}{\textbf{AudioCaps}} & \multicolumn{2}{c}{\textbf{Clotho}} \\ \cline{2-5}
 & GMO(MB) & TO(s) & GMO(MB) & TO(s)\\ \hline
ML-CLAP & 22172 & 3349 & 30912 & 1592\\ \hline
our CACL & 26788 & 3745 & 31528 & 1714\\ \hline
our KCL & 68256 & 4209 & 79480 & 1884\\ \hline
\end{tabular}
\label{Tab:overhead}
\end{table}

\subsection{Evaluation Results about Training Overhead}
Tab.\ref{Tab:overhead} summarises the GPU memory overhead (GMO) and time overhead (TO) during training for three scenarios: ML-CLAP, CACL, and KCL. KCL training requires simultaneous input of text in eight languages, which significantly increases overhead, resulting in a higher GMO of about 2.8 times and a 27\% increase in TO compared to ML-CLAP. In contrast, CACL inputs just twice as much text as ML-CLAP, resulting in a modest increase of about 10\% in both overheads. This makes CACL more suitable for scenarios that prioritize lower training overheads, while KCL is more suitable for applications that emphasize retrieval performance.
\section{Conclusion}
In this paper, we introduced Atom of Thoughts (\our), a novel framework that transforms complex reasoning processes into a Markov process of atomic questions. By implementing a two-phase transition mechanism of decomposition and contraction, \our eliminates the need to maintain historical dependencies during reasoning, allowing models to focus computational resources on the current question state. Our extensive evaluation across diverse benchmarks demonstrates that \our serves effectively both as a standalone framework and as a plug-in enhancement for existing test-time scaling methods. These results validate \our's ability to enhance LLMs' reasoning capabilities while optimizing computational efficiency through its Markov-style approach to question decomposition and atomic state transitions.

\bibliography{ref}
\bibliographystyle{icml2025}

%%%%%%%%%%%%%%%%%%%%%%%%%%%%%%%%%%%%%%%%%%%%%%%%%%%%%%%%%%%%%%%%%%%%%%%%%%%%%%%
%%%%%%%%%%%%%%%%%%%%%%%%%%%%%%%%%%%%%%%%%%%%%%%%%%%%%%%%%%%%%%%%%%%%%%%%%%%%%%%
% APPENDIX
%%%%%%%%%%%%%%%%%%%%%%%%%%%%%%%%%%%%%%%%%%%%%%%%%%%%%%%%%%%%%%%%%%%%%%%%%%%%%%%
%%%%%%%%%%%%%%%%%%%%%%%%%%%%%%%%%%%%%%%%%%%%%%%%%%%%%%%%%%%%%%%%%%%%%%%%%%%%%%%
\newpage
\appendix
\onecolumn
\section{Inference Configuration} \label{app:inference_config}

\subsection{Software} 
Our experiments build upon two open-source projects: \emph{Easy-to-Hard Generalization}~\cite{sun2024easy} for the MATH dataset and \emph{bigcode-evaluation-harness}~\cite{bigcode-evaluation-harness} for the MBPP dataset. We employ \emph{vLLM}~\cite{kwon2023efficient} to accelerate inference. All experiments can be reproduced on a single L40S or A6000 GPU.

\subsection{Sampling Parameters}
We use zero-shot inference for models fine-tuned specifically for each dataset. For general-purpose models, we use four in-context examples (few-shot inference) to ensure correct output formatting. The maximum output length is set to 1024 tokens for all tasks. For the MATH dataset, we use top-k sampling with $k = 20$. No additional sampling constraints are imposed for the MBPP dataset.

\subsection{Metric Calculation}
To compute majority-vote results for the MATH dataset, we consider two samples to have the same answer if they match after normalization. For the pass@K metric, we follow the definition in~\citet{chen2021evaluating}. Let $N$ be the total number of samples and $C$ be the number of correct samples. Then \(\mathrm{pass}@K\) is defined as:
\begin{align}
\mathrm{pass}@K = 1 - \frac{\binom{N - C}{K}}{\binom{N}{K}}.
\end{align}
\section{Details of the Stochastic Process Model}
\label{app: toy model}
We introduce a stochastic process model to explain that (1) the token-level entropy increases steadily at the beginning but rises rapidly when the sampling temperature reaches a certain threshold. (2) The optimal temperature is near the turning point when using multi-sample aggregation strategies.

The stochastic process model has two underlying assumptions: (1) Every token can be labeled as `proper' or `improper' at each decoding step. Generally, proper tokens have relatively higher logits than improper tokens, while the number of improper tokens is much higher than that of proper tokens. (2) When an improper token is generated, improper tokens have a higher generation probability in the next step, and vice versa.

Under these two assumptions, we can calculate the token-level entropy under different sampling temperatures, and the temperature-entropy curve fits that of real language models. Meanwhile, the percentage of improper tokens quickly increases after the turning point, implying a quick drop in sample quality in real language models.

\subsection{Model Setup}
\subsubsection{Initial Conditions}
We consider a discrete-time process \(\{x_t\}_{t=0}^{K}\) where each \(x_t \in [0,1]\) represents the model’s probability of producing an improper token at time step \(t\). We start with an initial error rate:
\[
x_0 = x_{\text{init}} \in [0,1].
\]
Conceptually, \(x_{\text{init}}\) corresponds to the model’s baseline `error propensity' at the start. This value is related to the sampling temperature \(T\) of the language model: higher \(T\) typically yields a flatter probability distribution over tokens, increasing the chance of selecting an improper token and thus increasing \(x_{\text{init}}\). (See Section \ref{sec: temp_to_init} for a heuristic link between temperature and initial error rate.)

\subsubsection{Interpreting the Error Rate}
At each step \(t\), the language model chooses a single token, and each token is classified as proper or improper. Although in practice, the correctness of a token depends on the context and is not truly binary, we approximate this by treating correctness as a Bernoulli trial:
\begin{itemize}
   \item Probability of producing an improper token: \(x_t\);
   \item Probability of producing a proper token: \(1 - x_t\).
\end{itemize}
Define a random variable \(E_t\) that indicates whether an error occurred at time \(t\):
   \[
   E_t = \begin{cases} 
   1 & \text{with probability } x_t, \text{ (improper token)},\\
   0 & \text{with probability } 1 - x_t, \text{ (proper token)}.
   \end{cases}
   \]

\subsubsection{Error Rate Update Rules}
   After each step, the error rate \(x_{t+1}\) is updated based on whether the token at time \(t\) was improper or proper:

   \paragraph{If an error occurs (\(E_t = 1\)):}
   The error rate is increased. Intuitively, making a mistake can make the model more likely to continue making errors. Formally, we update:
   \[
   x_{t+1} = 1 - (1 - x_t)^\alpha.
   \]
   For \(x_t \in [0,1]\) and $\alpha > 1$ ($\alpha$ is a hyperparameter), it can be seen that \(x_{t+1} \ge x_t\). Here $\alpha$ can be regarded as the noise tolerance rate, measuring how stable the model is when suffering from unexpected noises, and we will try different $\alpha$ in experiments.

   \paragraph{If a proper token is produced (\(E_t = 0\)):}
   The error rate is reduced, reflecting a `reinforcement' of correct behavior. We do this by:
   \[
   x_{t+1} = \max(x_t^\alpha,\ x_{\text{init}}).
   \]
   It generally makes it smaller, so this step lowers the error rate. However, we do not allow the error rate to drop below the initial baseline \(x_{\text{init}}\).

\subsubsection{Linking Initial Error Rate and Temperature}
\label{sec: temp_to_init}
   At step $t$, the model’s token probability mass is divided into:
   \begin{itemize}
       \item \textbf{Improper tokens} with total probability \(P_{1, \text{improper}} = x_t\).
       \item \textbf{Proper tokens} with total probability \(P_{0, \text{proper}} = 1 - x_t\).
   \end{itemize}
Therefore, by definition:
   \[
   x_{\text{init}} = P_{1, \text{improper}}.
   \]
Under higher temperatures \(T\), the softmax distribution flattens, increasing \(P_{1,\text{improper}}\) because the number of improper tokens is large but their logits are low. Thus, $x_\text{init}$ increases as \(T\) increases.

\subsubsection{Type of Tokens during decoding}

The probability of tokens when decoding is usually multi-peak (i.e., except for the token with the highest logit, some other tokens have reasonably high logits and are also acceptable during decoding), so it is natural to consider a scenario with three categories of tokens:
   \begin{itemize}
       \item \textbf{Proper tokens:} A small number of tokens with high logits. Let $N_0$ be the number of proper tokens. To capture the multi-peak behavior, the logits of proper tokens $l_{0,1}, ..., l_{0, N_0}$are sampled from the Gaussian distribution $\mathcal{N}(L_0, \sigma_0)$.
       \item \textbf{Low Probability Improper tokens:} Many low-logit tokens where language models seldom choose them. Let $N_1$ be the number of tokens in this type and their logits are set to $L_1$ for simplicity.
       \item \textbf{High Probability Improper tokens:} Due to insufficient training or errors in training data, some tokens may have exceptionally high logits but are logically improper in (\textit{e.g., } the token $3$ in $1 + 1 = 3$). Since different language models behave differently and this type of token will be selected regardless of temperature, we only consider decoding without high-probability improper tokens in our discussion.
   \end{itemize}
For the first two types of tokens, we have:
   \[
    L_{0} > L_{1}, \ \ N_0 \ll N_1.
   \]
\subsubsection{Token Probability During Sampling}

At each step \(t\), the probability of producing improper tokens is \(x_t\). Let $p_{t, \text{proper/improper}, j}$ be the probability of the $j$-th proper/improper tokens at step $t$. For the improper tokens, we have:
\[
p_{t,\text{improper}, j} = \frac{x_t}{N_1},\ \ \forall j \in [1, N_1].
\]
For the proper tokens, we allocate the remaining probability \(1 - x_{t}\) according to their relative logits:
\[
p_{t,\text{proper}, j} = (1 - x_{t}) \, \text{softmax}(l_{0,1}, ..., l_{0, N_0})_j, \ \ \forall j \in [1, N_0].
\]
This ensures that the relative ordering of probabilities for the proper tokens remains determined by their logits, while the total mass allocated to improper tokens is \(x_t\).

\subsubsection{Entropy Calculation}

We define the token-level entropy \(\mathcal{H}\) over a sequence of decoding steps:
\[
\mathcal{H} = \frac{1}{K} \sum_{t=1}^{K} \left( -\sum_{\text{j}}^{N_0} p_{t,\text{proper}, j} \log p_{t,\text{proper}, j} -\sum_{\text{j}}^{N_1} p_{t,\text{improper}, j} \log p_{t,\text{improper}, j} \right)\]
Here, $K$ is the total number of decoding steps considered.

\subsection{Experiment}

\subsubsection{Model Hyperparameter}

The toy model has the following hyperparameters:
\begin{itemize}
    \item The numbers of proper and improper tokens: $N_0, N_1$;
    \item The logits of proper and improper tokens: $l_{0,\{1, ..., N_0\}}, l_{1,\{1, ..., N_1\}}$, where:
\[l_{0, i} \sim \mathcal{N}(L_0, \sigma_0), \ \ l_{1, i} = L_1.\]
    \item The number of total steps $K$.
    \item The noise tolerance rate $\alpha$.
\end{itemize}
The input is temperature $(T)$ and the output is the average token-level entropy $(E)$ over 500 samples.
In our experiment, we set the hyperparameters to be:
\begin{align*}
N_0=10, \ \ N_1 = 30000, \ \ L_0 = 0,\ \  L_1 = -10,\ \ \sigma_0 = 1,\ \ K=512.
\end{align*}
We test different noise tolerance rates $\alpha \in [1.5, 2.0, 2.5, 3.0]$ to show behaviors under different noise tolerance rates.

\subsubsection{Result}

Here is the temperature-entropy curve derived from the toy model under different noise tolerance rates $\alpha$ (Figure~\ref{fig:toy_curves}):
\begin{figure}[ht]
    \centering
    \begin{subfigure}
        \centering
        \includegraphics[width=0.6\textwidth]{figs/simple_model_all.png}
        \caption{The temperature-entropy curves.}
        \label{fig:first_image}
    \end{subfigure}
    \begin{subfigure}
        \centering
        \includegraphics[width=0.6\textwidth]{figs/simple_model_all_log.png}
        \caption{The temperature-log entropy curves.}
        \label{fig:second_image}
    \end{subfigure}
    \caption{The curves derived from the stochastic process model under different $\alpha$.}
    \label{fig:toy_curves}
\end{figure}
\begin{figure}[ht]
    \vspace{-3mm}
    \centering
    \includegraphics[width=0.6\textwidth]{figs/simple_model_all_correct_rate.png}
    \caption{The Temperature-Improper Token (\%) curves.}
    \label{fig:correct rate}
\end{figure}
\begin{figure}[ht]
\vspace{-3mm}
\centering
\includegraphics[width=0.6\textwidth]{figs/entropy.png}
\vspace{-3mm}
\caption{The temperature-entropy curves.}
\label{fig:curves}
\end{figure}
\begin{figure}[h!]
\vspace{-3mm}
\centering
\includegraphics[width=0.6\textwidth]{figs/log_entropy.png}
\vspace{-3mm}
\caption{The temperature-log entropy curves.}
\label{fig:real_curves}
\vspace{-3mm}
\end{figure}
The curves under different noise tolerances have a similar shape. Generally, the curve can be divided into two parts: \textbf{(sub)-linear} increase and \textbf{(super)-exponential} increase. In the first part, the increase is due to the various choices among the proper tokens, while the sharp rise in the second part is due to the loss of control (i.e., the model frequently chooses nonsense tokens).

The curve is very similar to the behavior of real language models, and some reference entropy curves and log-entropy curves are shown in Figure~\ref{fig:curves} and~\ref{fig:real_curves}.

\paragraph{Relation to Improper Token Rate} It is natural to consider proper tokens can lead to correct answers and the improper tokens will result in incorrect answers, so we measure the percentage of improper tokens. As shown in Figure~\ref{fig:correct rate}, when the temperature exceeds the turning point, the percentage of improper tokens increases quickly, implying a quality drop in samples. Interestingly, the difference in noise tolerance rates has little inference on the turning point but controls the improper token increasing speed after the turning point. Nevertheless, the percentage of improper tokens increases quickly under all tested $\alpha$.
%Fig. \ref{fig:correct rate} is the temperature-correct rate curve, and the big drop in the correct rate happens at the same point as the entropy begins to increase.

\subsection{Conclusion}
Our toy model provides a simplified yet insightful framework for understanding how temperature-dependent sampling dynamics may give rise to characteristic shifts in the model’s output distribution. The model predominantly chooses from the proper tokens in the low-temperature (or sub-linear growth) regime, resulting in relatively stable and controlled outputs. The distribution flattens and nonsense tokens gain significant probability mass as temperature increases beyond a certain threshold. This transition leads to a sudden and steep increase in entropy—mirroring observations in actual large language models—and a corresponding drop in the correct rate. Therefore, increasing the temperature can initially increase generation diversity (sampling among proper tokens) with a small correctness drop. However, its performance suffers a quick drop after reaching a certain threshold (i.e., the turning point).

\section{Aggregation Adaptation Calculation}
\label{app: bias}
The choice of aggregation function affects the optimal generation temperature. For example, in majority voting, the final answer must be selected by the majority, whereas in best-of-N, only a single correct instance out of the N samples is required.

In the case of majority voting, the turning point on the entropy curve aligns with the optimal temperature, so we set its adaptation to 0. For best-of-N, we computed an adaptation on MATH and then tested it on MBPP to confirm generality. Specifically, we averaged the difference between the midpoints of the optimal temperature ranges for best-of-N and majority voting across 13 models on MATH. This difference was $0.092$ on average. Therefore, for simplicity, we set the aggregation adaptation for best-of-N to $0.1$.

\begin{table}[ht]
\centering
\caption{\textbf{Aggregation Adaptation for Best-of-N}: we calculate midpoints of optimal temperature ranges on Majority Voting and Best-of-N for MATH. The difference between the average of midpoints is $0.092$, so we set the adaptation factor to $0.1$.}
\begin{tabular}{c|ccccccccccccc|c}
\toprule
Aggregation & \multicolumn{13}{c|}{Individual Models (Models are listed in the same order as Table~\ref{table: hit rate})} & Average \\
\hline
Best-of-N & 0.6 & 0.8 & 0.6 & 0.6 & 0.7 & 0.5 & 0.6 & 1.1 & 1.2 & 0.5 & 0.6 & 1.3 & 1.0 & 0.7769\\
Majority Voting & 0.6 & 0.9 & 0.6 & 0.5 & 0.3 & 0.6 & 0.5 & 1.1 & 0.9 & 0.5 & 0.6 & 1.0 & 0.8 & 0.6846\\
\bottomrule
\end{tabular}
\label{tab:averages}
\end{table}

\section{Results of All Tested Models}

We present the accuracy heatmaps and entropy estimations for all tested models. Figure~\ref{fig: heatmap MATH} shows the heatmaps of model accuracy for the MATH dataset, while Figure~\ref{fig: heatmap MBPP} displays the heatmaps for the MBPP dataset. Additionally, Figure~\ref{fig: curve_math} illustrates the entropy curve estimations for the MATH dataset, and Figure~\ref{fig: curve_mbpp} provides the entropy curve estimations for the MBPP dataset.

\label{app: results}
\begin{figure*}[ht]
    \center
\includegraphics[width=0.95\textwidth]{figs/merged_heatmap_MATH.png}
    \caption{The accuracy heatmap for all tested models on the MATH dataset. The green line is our predicted temperature.}
    \label{fig: heatmap MATH}
\end{figure*}
\begin{figure*}[ht]
    \center
    \includegraphics[width=0.95\textwidth]{figs/merged_heatmap_MBPP.png}
    \caption{The accuracy heatmap for all tested models on the MBPP dataset. The green line is our predicted temperature.}
    \label{fig: heatmap MBPP}
\end{figure*}
\begin{figure*}[ht]
    \center
\includegraphics[width=0.95\textwidth]{figs/curve_math.png}
    \caption{The entropy curves and turning points of language models when testing on the MATH dataset.}
    \label{fig: curve_math}
\end{figure*}
\begin{figure*}[ht]
    \center
    \includegraphics[width=0.95\textwidth]{figs/curve_code.png}
    \caption{The entropy curves and turning points of language models when testing on the MBPP dataset.}
    \label{fig: curve_mbpp}
\end{figure*}
%%%%%%%%%%%%%%%%%%%%%%%%%%%%%%%%%%%%%%%%%%%%%%%%%%%%%%%%%%%%%%%%%%%%%%%%%%%%%%%
%%%%%%%%%%%%%%%%%%%%%%%%%%%%%%%%%%%%%%%%%%%%%%%%%%%%%%%%%%%%%%%%%%%%%%%%%%%%%%%

\end{document}
