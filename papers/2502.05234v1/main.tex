%%%%%%%% ICML 2025 EXAMPLE LATEX SUBMISSION FILE %%%%%%%%%%%%%%%%%

\documentclass{article}

% Recommended, but optional, packages for figures and better typesetting:
\usepackage{microtype}
\usepackage{graphicx}
\usepackage{booktabs} % for professional tables
\usepackage{subfigure}
\usepackage{multirow}
\usepackage{tabularx}
\usepackage{comment}
\newcommand{\weihua}[1]{\textcolor{blue}{(Weihua) #1}}
%\newcommand{\weihua}[1]{\textcolor{blue}{}}
\newcommand{\yy}[1]{\textcolor{red}{(YY:) #1}}

% hyperref makes hyperlinks in the resulting PDF.
% If your build breaks (sometimes temporarily if a hyperlink spans a page)
% please comment out the following usepackage line and replace
% \usepackage{icml2025} with \usepackage[nohyperref]{icml2025} above.
\usepackage{hyperref}

% Attempt to make hyperref and algorithmic work together better:
\newcommand{\theHalgorithm}{\arabic{algorithm}}

% Use the following line for the initial blind version submitted for review:
%\usepackage{icml2025}

% If accepted, instead use the following line for the camera-ready submission:
\usepackage[accepted]{icml2025}

% For theorems and such
\usepackage{amsmath}
\usepackage{amssymb}
\usepackage{mathtools}
\usepackage{amsthm}

% if you use cleveref..
\usepackage[capitalize,noabbrev]{cleveref}

%%%%%%%%%%%%%%%%%%%%%%%%%%%%%%%%
% THEOREMS
%%%%%%%%%%%%%%%%%%%%%%%%%%%%%%%%
\theoremstyle{plain}
\newtheorem{theorem}{Theorem}[section]
\newtheorem{proposition}[theorem]{Proposition}
\newtheorem{lemma}[theorem]{Lemma}
\newtheorem{corollary}[theorem]{Corollary}
\theoremstyle{definition}
\newtheorem{definition}[theorem]{Definition}
\newtheorem{assumption}[theorem]{Assumption}
\theoremstyle{remark}
\newtheorem{remark}[theorem]{Remark}

% Todonotes is useful during development; simply uncomment the next line
%    and comment out the line below the next line to turn off comments
%\usepackage[disable,textsize=tiny]{todonotes}
\usepackage[textsize=tiny]{todonotes}

% The \icmltitle you define below is probably too long as a header.
% Therefore, a short form for the running title is supplied here:
\icmltitlerunning{Optimizing Temperature for Language Models with Multi-Sample Inference}

\begin{document}

\twocolumn[
%\icmltitle{Uncovering Optimal Temperature in Language Model Inference Strategies}
\icmltitle{Optimizing Temperature for Language Models with Multi-Sample Inference}
%Yiming suggested title:
%\icmltitle{Optimizing Temperature for Sampling-Based Language Model Inference}

% It is OKAY to include author information, even for blind
% submissions: the style file will automatically remove it for you
% unless you've provided the [accepted] option to the icml2025
% package.

% List of affiliations: The first argument should be a (short)
% identifier you will use later to specify author affiliations
% Academic affiliations should list Department, University, City, Region, Country
% Industry affiliations should list Company, City, Region, Country

% You can specify symbols, otherwise they are numbered in order.
% Ideally, you should not use this facility. Affiliations will be numbered
% in order of appearance and this is the preferred way.
\icmlsetsymbol{equal}{*}

\begin{icmlauthorlist}
\icmlauthor{Weihua Du}{yyy}
%\icmlauthor{Firstname2 Lastname2}{equal,yyy,comp}
\icmlauthor{Yiming Yang}{yyy}
\icmlauthor{Sean Welleck}{yyy}
%\icmlauthor{Firstname5 Lastname5}{yyy}
%\icmlauthor{Firstname6 Lastname6}{sch,yyy,comp}
%\icmlauthor{Firstname7 Lastname7}{comp}
%\icmlauthor{}{sch}
%\icmlauthor{Firstname8 Lastname8}{sch}
%\icmlauthor{Firstname8 Lastname8}{yyy,comp}
%\icmlauthor{}{sch}
%\icmlauthor{}{sch}
\end{icmlauthorlist}

\icmlaffiliation{yyy}{Language Technologies Institute, Carnegie Mellon University}
%\icmlaffiliation{comp}{Company Name, Location, Country}
%\icmlaffiliation{sch}{School of ZZZ, Institute of WWW, Location, Country}

\icmlcorrespondingauthor{Weihua Du}{weihuad@cs.cmu.edu}
\icmlcorrespondingauthor{Yiming Yang}{yiming@cs.cmu.edu}
\icmlcorrespondingauthor{Sean Welleck}{wellecks@cmu.edu}

% You may provide any keywords that you
% find helpful for describing your paper; these are used to populate
% the "keywords" metadata in the PDF but will not be shown in the document
\icmlkeywords{Large Language Models, Inference Time Compute, ICML}

\vskip 0.3in
]

% this must go after the closing bracket ] following \twocolumn[ ...

% This command actually creates the footnote in the first column
% listing the affiliations and the copyright notice.
% The command takes one argument, which is text to display at the start of the footnote.
% The \icmlEqualContribution command is standard text for equal contribution.
% Remove it (just {}) if you do not need this facility.

\printAffiliationsAndNotice{}  % leave blank if no need to mention equal contribution
%\printAffiliationsAndNotice{\icmlEqualContribution} % otherwise use the standard text.

\begin{abstract} 
%Sampling-based inference strategies, such as majority voting and best-of-N sampling, are commonly employed in language models to improve accuracy on problem-solving tasks. While sampling temperature significantly influences the performance of these strategies, it is often set to a default value or tuned using an additional validation set without enough care. In this work, we demonstrate that the suitable temperature range can differ across various model architectures and tasks, and we observe a correlation between the optimal temperature and the similarity between models and tasks. Furthermore, we propose an entropy-based approach to automatically identify a suitable temperature without the need for any labeled data. Our method achieves a hit rate of xxx in recovering the true optimal temperature and beating down all fixed temperature settings. Finally, we employ a stochastic process model to provide interpretability for our approach.
Multi-sample aggregation strategies, such as majority voting and best-of-N sampling, are widely used in contemporary large language models (LLMs) to enhance predictive accuracy across various tasks. A key challenge in this process is temperature selection, which significantly impacts model performance. Existing approaches either rely on a fixed default temperature or require labeled validation data for tuning, which are often scarce and difficult to obtain. This paper addresses the challenge of automatically identifying the (near)-optimal temperature for different LLMs using multi-sample aggregation strategies, without relying on task-specific validation data. We provide a comprehensive analysis of temperature’s role in performance optimization, considering variations in model architectures, datasets, task types, model sizes, and predictive accuracy. Furthermore, we propose a novel entropy-based metric for automated temperature optimization, which consistently outperforms fixed-temperature baselines. Additionally, we incorporate a stochastic process model to enhance interpretability, offering deeper insights into the relationship between temperature and model performance.
\end{abstract}

% tl,dr: By analyzing the behavior of various LLMs across different temperatures and tasks, we propose an entropy-based metric for automated temperature optimization in multi-sample aggregation strategies, eliminating the need for labeled validation data.


\section{Introduction}
\IEEEPARstart{I}{n} recent years, flourishing of Artificial Intelligence Generated Content (AIGC) has sparked significant advancements in modalities such as text, image, audio, and even video. 
Among these, AI-Generated Image (AGI) has garnered considerable interest from both researchers and the public.
Plenty of remarkable AGI models and online services, such as StableDiffusion\footnote{\url{https://stability.ai/}}, Midjourney\footnote{\url{https://www.midjourney.com/}}, and FLUX\footnote{\url{https://blackforestlabs.ai/}}, offer users an excellent creative experience.
However, users often remain critical of the quality of the AGI due to image distortions or mismatches with user intentions.
Consequently, methods for assessing the quality of AGI are becoming increasingly crucial to help improve the generative capabilities of these models.

Unlike Natural Scene Image (NSI) quality assessment, which focuses primarily on perception aspects such as sharpness, color, and brightness, AI-Generated Image Quality Assessment (AGIQA) encompasses additional aspects like correspondence and authenticity. 
Since AGI is generated on the basis of user text prompts, it may fail to capture key user intentions, resulting in misalignment with the prompt.
Furthermore, authenticity refers to how closely the generated image resembles real-world artworks, as AGI can sometimes exhibit logical inconsistencies.
While traditional IQA models may effectively evaluate perceptual quality, they are often less capable of adequately assessing aspects such as correspondence and authenticity.

\begin{figure}\label{fig:radar}
    \centering
    \includegraphics[width=1.0\linewidth]{figures/radar_plot.pdf}
    \caption{A comparison on quality, correspondence, and authenticity aspects of AIGCIQA2023~\cite{wang2023aigciqa2023} dataset illustrates the superior performance of our method.}
\end{figure}

Several methods have been proposed specifically for the AGIQA task, including metrics designed to evaluate the authenticity and diversity of generated images~\cite{gulrajani2017improved,heusel2017gans}. 
Nevertheless, these methods tend to compare and evaluate grouped images rather than single instances, which limits their utility for single image assessment.
Beginning with AGIQA-1k~\cite{zhang2023perceptual}, a series of AGIQA databases have been introduced, including AGIQA-3k~\cite{li2023agiqa}, AIGCIQA-20k~\cite{li2024aigiqa}, etc.
Concurrently, there has been a surge in research utilizing deep learning methods~\cite{zhou2024adaptive,peng2024aigc,yu2024sf}, which have significantly benefited from pre-trained models such as CLIP~\cite{radford2021learning}. 
These approaches enhance the analysis by leveraging the correlations between images and their descriptive texts.
While these models are effective in capturing general text-image alignments, they may not effectively detect subtle inconsistencies or mismatches between the generated image content and the detailed nuances of the textual description.
Moreover, as these models are pre-trained on large-scale datasets for broad tasks, they might not fully exploit the textual information pertinent to the specific context of AGIQA without task-specific fine-tuning.
To overcome these limitations, methods that leverage Multimodal Large Language Models (MLLMs)~\cite{wang2024large,wang2024understanding} have been proposed.
These methods aim to fully exploit the synergies of image captioning and textual analysis for AGIQA.
Although they benefit from advanced prompt understanding, instruction following, and generation capabilities, they often do not utilize MLLMs as encoders capable of producing a sequence of logits that integrate both image and text context.

In conclusion, the field of AI-Generated Image Quality Assessment (AGIQA) continues to face significant challenges: 
(1) Developing comprehensive methods to assess AGIs from multiple dimensions, including quality, correspondence, and authenticity; 
(2) Enhancing assessment techniques to more accurately reflect human perception and the nuanced intentions embedded within prompts; 
(3) Optimizing the use of Multimodal Large Language Models (MLLMs) to fully exploit their multimodal encoding capabilities.

To address these challenges, we propose a novel method M3-AGIQA (\textbf{M}ultimodal, \textbf{M}ulti-Round, \textbf{M}ulti-Aspect AI-Generated Image Quality Assessment) which leverages MLLMs as both image and text encoders. 
This approach incorporates an additional network to align human perception and intentions, aiming to enhance assessment accuracy. 
Specially, we distill the rich image captioning capability from online MLLMs into a local MLLM through Low-Rank Adaption (LoRA) fine-tuning, and train this model with human-labeled data. The key contributions of this paper are as follows:
\begin{itemize}
    \item We propose a novel AGIQA method that distills multi-aspect image captioning capabilities to enable comprehensive evaluation. Specifically, we use an online MLLM service to generate aspect-specific image descriptions and fine-tune a local MLLM with these descriptions in a structured two-round conversational format.
    \item We investigate the encoding potential of MLLMs to better align with human perceptual judgments and intentions, uncovering previously underestimated capabilities of MLLMs in the AGIQA domain. To leverage sequential information, we append an xLSTM feature extractor and a regression head to the encoding output.
    \item Extensive experiments across multiple datasets demonstrate that our method achieves superior performance, setting a new state-of-the-art (SOTA) benchmark in AGIQA.
\end{itemize}

In this work, we present related works in Sec.~\ref{sec:related}, followed by the details of our M3-AGIQA method in Sec.~\ref{sec:method}. Sec.~\ref{sec:exp} outlines our experimental design and presents the results. Sec.~\ref{sec:limit},~\ref{sec:ethics} and~\ref{sec:conclusion} discuss the limitations, ethical concerns, future directions and conclusions of our study.
\section{Related Work}
\label{sec:related}


\noindentbold{2D visual foundation models}
In recent years, we have witnessed the emergence of large pretrained models—so-called foundation models that are trained on large-scale datasets and serve as a \textit{foundation} for many downstream tasks.
These models demonstrate remarkable versatility across multiple modalities, including language~\cite{team2023gemini,touvron2023llama,touvron2023llama2,dubey2024llama3,vicuna2023,radford2019language,brown2020language,chung2024scaling,achiam2023gpt,bai2023qwen,yang2024qwen2,jiang2023mistral,jiang2024mixtral}, vision~\cite{sam,ravi2024sam,dino_v1,oquab2023dinov2,zou2024segment,rombach2022high,ho2020denoising,nichol2021improved,songdenoising,songscore}, audio~\cite{deshmukh2023pengi,zhang2023speechgpt,rubenstein2023audiopalm,borsos2023audiolm}. 
Furthermore, they enable multi-modal reasoning capabilities that bridge across different modalities~\cite{girdharImageBindOneEmbedding2023,Qwen-VL,llava,radfordLearningTransferableVisual2021,jia2021scaling,team2024gemini}.
Among these models, those that operate on visual modalities are known as visual foundation models (VFM).
VFMs excel in various computer vision tasks such as image segmentation~\cite{sam,ravi2024sam,zou2024segment,zou2023generalized,cheng2021per,cheng2022masked,jain2023oneformer,li2024semantic}, object detection~\cite{liu2023grounding,carion2020end}, representation learning~\cite{dino_v1,oquab2023dinov2}, and open-vocabulary understanding~\cite{radfordLearningTransferableVisual2021,li2022language,ghiasi2022scaling,ram,ram_pp,yu2023convolutions,kang2024defense,naeem2024silc,cho2024cat}.
When integrated with large language models, they enable sophisticated visual reasoning and natural language interactions~\cite{llava,Qwen-VL,girdharImageBindOneEmbedding2023,team2024gemini,guo2024regiongpt,yuan2024osprey,you2023ferret}.
We use such vision language models to construct open vocabulary segmentation and captions for point clouds based on multiview images.







\noindentbold{Open-vocabulary 3D segmentation}
Building on the success of 2D VFMs, recent work have extended open-vocabulary capabilities to 3D scene understanding.
OpenScene~\cite{Peng2023OpenScene} first introduced zero-shot 3D semantic segmentation by distilling knowledge from language-aligned image encoders~\cite{li2022language,ghiasi2022scaling}.
Subsequent methods~\cite{ding2022pla,yang2024regionplc,jiang2024open} leverage multiview images to generate textual captions, which then serve as training supervision.
However, these methods face challenges in generating high-quality 3D mask-text pairs at scale.
For open-vocabulary 3D instance segmentation, existing methods~\cite{takmaz2023openmask3d,nguyen2024open3dis,huang2024openins3d} typically rely on closed-vocabulary proposal networks such as Mask3D~\cite{schult2023mask3d}, which inherently constrains their ability to detect novel object categories. 
Moreover, these methods leverage 2D VFMs like CLIP~\cite{radfordLearningTransferableVisual2021} for region classification by projecting 3D regions onto multiple 2D views.
This approach requires both 2D images and 3D point clouds during inference. Additionally, it necessitates multiple inferences of large 2D models on projected masks, resulting in high computational costs. 
We address these limitations by developing the first single-stage open-vocabulary 3D instance segmentation model that operates directly in 3D without ground truth labels, using our \dataname dataset and Segment3D~\cite{huang2024segment3d} proposals.

\noindentbold{3D vision-language datasets}
Several datasets align 3D scenes with textual annotations to facilitate language-driven 3D understanding. 
ScanRefer~\cite{chen2020scanrefer}, ReferIt3D~\cite{achlioptas2020referit_3d} and EmbodiedScan~\cite{wangEmbodiedScanHolisticMultiModal2023} provide fine-grained object-level localization through detailed referential phrases, while ScanQA~\cite{azuma2022scanqa} targets spatially grounded question-answering. 
In contrast, SceneVerse~\cite{jiaSceneVerseScaling3D2024} and MMScan~\cite{lyu2024mmscan} employ large-language models or vision-language models to partially automate annotation.
Despite leveraging advanced models, these datasets depend significantly on costly human annotations derived from closed-vocabulary sources, limiting their support for open-vocabulary and scalability for large-scale 3D segmentation tasks.

\begin{figure}[ht]
    \centering
    % Top sub-figure
    \begin{subfigure}
        \centering
        \includegraphics[width=0.48\textwidth]{figs/correlation_math_v2.png}
    \end{subfigure}
    \vskip -1em  % vertical space
    % Bottom sub-figure
    \begin{subfigure}
        \centering
        \includegraphics[width=0.48\textwidth]{figs/correlation_code_v2.png}
    \end{subfigure}
    \vskip -2em
    \caption{Plot of midpoints of optimal temperature ranges (x-axis, sample size 128) vs. distances between models and tasks (y-axis). A strong negative correlation is observed on the MATH and MBPP datasets, with correlation coefficients of -0.895 and -0.777.}
    \label{fig: correlation code generation}
    \vspace{-8mm}
\end{figure}
\section{Correlation Between Model Training and Optimal Temperature}
\label{sec: 3}

Multi-sample aggregation strategies—commonly used in problem-solving, code generation, and related domains—leverage information from multiple samples, which helps escape local minima and improve robustness. In these settings, \emph{sample diversity} becomes crucial: a diverse set of candidate samples increases the likelihood that the correct solution appears in the pool, rather than repeating the same mistake. The \emph{temperature} parameter is a primary lever for controlling this diversity.

We hypothesize that how a model is trained impacts the optimal temperature for multi-sample inference strategies. In particular, a more specialized or fine-tuned model can safely explore higher temperatures without drifting into low-quality outputs. In contrast, a general-purpose model typically benefits from a lower temperature to remain focused on relevant content.

We investigate this in two steps: In Section~\ref{sec: temperature varies}, we show that the optimal temperature varies for a base, instruction-tuned, and fine-tuned model. Then in Section~\ref{sec: temperature correlation}, we establish a general relationship between a model’s proximity to the target task and its corresponding optimal temperature. Our key insight is that token-level entropy is a proxy of distance from a task, which motivates our entropy-based method for automatic temperature selection in Section~\ref{sec: 4}.

%Our empirical studies reveal that different language models under various training stages respond differently to performance changes in temperature. In particular, a strong relationship exists between how closely a model’s training data aligns with the target task and the temperature that yields optimal performance. Intuitively, a more specialized or fine-tuned model can safely explore higher temperatures without drifting into low-quality outputs. In contrast, a general-purpose model typically benefits from a lower temperature to remain focused on relevant content. This observation underpins our hypothesis that the more closely a model’s training data aligns with a target task, the higher the temperature at which sampling-based strategies will excel.

\subsection{Optimal Temperature Range Varies}
\label{sec: temperature varies}

%First, we show that the best temperature varies based on whether a model is a base model, an instruction-tuned model, or a model fine-tuned specifically for the target task.

%To do so, we evaluate the accuracy of each model using multi-sample aggregation with a variety of temperatures. As we see in the curve in Figure .. the best temperature varies by model. For example, the best temperature for the base model is 0.5 when 128 samples are used, compared to 1.0 for the task-specific model.

%We also make two observations that we rely on in the subsequent experiments. First, as shown in the heatmap, several temperatures may be “optimal”, in that there is a very small performance gap. Hence we consider an optimal temperature window (define).

%Second, the optimal temperature varies based on the number of samples. However, we noticed that the optimal temperature stabilizes after a sufficient number of samples (here 32). Hence we will focus on the sufficient sample setting (namely 128) in the rest of the paper.

We first demonstrate that the optimal sampling temperature varies by model type. We test three \emph{Mistral-7B} variants: the \emph{pretrained base model}, the \emph{instruction-finetuned version (Mistral-7B-Instruct)}, and a \emph{task-finetuned model for MATH}\footnote{Model link: \href{https://huggingface.co/peiyi9979/mistral-7b-sft}{https://huggingface.co/peiyi9979/mistral-7b-sft}}~\cite{wang2024math}. Each model is evaluated using multi-sample aggregation across different temperatures.
Figure~\ref{fig: teaser}(a) presents the accuracy heatmap for the Mistral-7B-Instruct model on the MATH dataset. At smaller sample sizes, lower temperatures tend to produce better accuracy. However, higher temperatures can yield better results as the sample size increases. For a fixed sample size, the accuracy curve follows a single-peak pattern: it rises as temperature increases and peaks, and then gradually declines, staying relatively steady near the peak.

Since the single-peak behavior, we define the \textbf{$\epsilon$-optimal temperature range}. This range encompasses temperatures $T$ where the accuracy $A(T)$ is no less than $A(T^*) - \epsilon$, with $A(T^*)$ representing the peak accuracy. Given the curve's single-peak nature, this range forms an interval around $T^*$. For our analysis, we set $\epsilon = 0.02$, effectively capturing the temperatures close to the peak where the accuracy remains relatively high.

We then plot the midpoint of this optimal temperature range for each model variant and various sample sizes (Figure~\ref{fig: teaser}(b)). We observe that the pretrained model has the lowest midpoint, the instruction-finetuned model has a higher midpoint, and the task-finetuned model has the highest. Another observation is that optimal temperature ranges change slowly once beyond a sample size of 32. Therefore, we choose a sample size of 128 in our following experiments to ensure stable performance in the rich-sample setting.

From these observations, we hypothesize a general relationship between how closely a model is tuned to a particular task and the temperature that yields the best accuracy. We discuss this hypothesis further in the next section.

\subsection{Correlation Between Training-Task Similarity and Optimal Temperature}
\label{sec: temperature correlation}
Our goal is to establish a general relationship between a model’s learned distribution and its optimal temperature for a task. Our key intuition is that token-level entropy can serve as a surrogate for a model's `distance' from a target task and that this distance helps identify the optimal temperature.

Specifically, we define a distance metric that measures how similar a model’s training data is to a given task. Let \(\mathcal{T} = \{X_1, ..., X_k\}\) be the task with $k$ problem instances. We define this distance \(\mathcal{D}(\mathcal{M}, \mathcal{T})\) as the average of token-level entropy \(\mathcal{H}(.)\) of the language model \(\mathcal{M}\) when generating the answers \(\mathcal{A} = \{Y_1, ..., Y_k\}\) for the problems in \(\mathcal{T}\):
\begin{align}
\mathcal{D}(\mathcal{M}, \mathcal{T})
&=
\frac{1}{k}
\sum_{i=1}^{k}
\left[
  \frac{1}{|Y_i|}
  \sum_{j=1}^{|Y_i|}
  \mathcal{H}\bigl(p_\mathcal{M}\bigl(\cdot \mid X_i,\,Y_{i,<j}\bigr)\bigr)
\right],
\end{align}
\vspace{-5mm}
where
\begin{align}
\label{eq: entropy}
\mathcal{H}(p)&=
  -\sum_{v \in p} 
     p\bigl(v) 
     \log p\bigl(v\bigr).
\end{align}
To avoid bias toward ground-truth references, we use model-generated sequences \(\{Y_i\}\) instead of official gold solutions. Meanwhile, the distance is measured at a low temperature $T=0.5$ to ensure the generation stability.

We evaluated several language models on the MATH and MBPP datasets, including pretrained, instruction-finetuned, and task-finetuned models. Figure~\ref{fig: correlation code generation} plots the midpoint of the optimal temperature range against our distance metric, demonstrating a strong negative correlation. Specifically, across our model set, the correlation on MATH is \(-0.895\), while on MBPP it is \(-0.777\).

In practice, this suggests using a higher temperature (e.g., \(T = 0.9 \sim 1.1\)) for task-finetuned models and a lower temperature (e.g., \(T = 0.5 \sim 0.7\)) for more general-purpose models (pretrained or instruction-finetuned).

\iffalse
\subsection{Link to Token Probability Distribution}

Given the numerous training methods available for language models, the training data and target tasks can vary significantly. During inference, the distribution of token probabilities also differs. As shown in Figure~\ref{fig: teaser}~(a), we applied three types of language models to the MATH dataset and measured the average probability of the top 20 tokens, each of the models has a different purpose (a pretrained model, an instruction-finetuned model, and a task-finetuned model). For the pretrained model, the token probability distribution is relatively flat compared with the instruction-finetuned model, which has been more specialized for question answering. As expected, the token probability distribution becomes even more concentrated when a model is specifically fine-tuned on the task’s training set.

Differences in token probability distributions also affect performance in sample-based inference strategies. Intuitively, a flat token probability distribution model may need a lower temperature to focus on the most relevant tokens. In contrast, a model with a concentrated token probability distribution may require a higher temperature to explore more diverse tokens. As shown in Figure~\ref{fig: teaser}~(b), we evaluated the three models under various sampling sizes and temperatures and then visualized the results in a heatmap. These results indicate that the optimal temperature for the pretrained model is lower than for the instruction-finetuned model, while the task-finetuned model benefits from the highest temperature.
\fi
\section{SAPS: Semantic Alignment for Policy Stitching}\label{sec:method-alignment}
Relative representations \citep{Moschella2022-yf}, used as a base for zero-shot stitching in R3L, involve computing a distance function between a set of samples, called \say{anchors}, to project the output of each encoder to a shared latent space, enabling the subsequent training of a universal policy. Semantic alignment, instead, estimates a direct mapping between latent spaces.

Consider the environment $\mathcal{M}_u^j$ for which no dedicated policy exists. However, we do have an encoder $\phi_u^i$ and a controller $\psi_v^j$, extracted from policies $\pi_u^i$ and $\pi_v^j$, respectively. 
We estimate an affine transformation $\tau_u^v$: $\mathcal{X}_{u}^{i} \mapsto \mathcal{X}_{v}^{j}$, mapping embeddings produced by $\phi_u^i$ into the space of $\pi_v^j$. This yields a new latent space:

\begin{align}
    & \tau_u^v(\enc_u^i(\mathbf{o}_u)) \approx \enc_v^j(\mathbf{o}_v)\\
    & \tau_u^v(\mathbf{x}_{u}^i) \approx \mathbf{x}_{v}^j
\end{align}

that is compatible with the existing $\psi_v^j$.
This enables the stitching of encoders and controllers from $\pi_u^i$ and $\pi_v^j$, respectively, to obtain a new policy $\tilde{\pi}_u^j$ that can act in $\mathcal{M}_u^j$, without additional training:
\begin{equation}\label{eq:2}
    \tilde{\pi}_u^j(o_u) = \con_v^j[\tau_u^v(\enc_u^i(\mathbf{o}_u))]
\end{equation}

\paragraph{Estimating $\tau$}
As in \cite{maiorca2023latent}, assume to be given latent spaces $\mathbf{X}_u$ and $\mathbf{X}_v$ which here correspond to the embedding of two visual variations in the space of observations.
We use SVD to obtain an affine transformation $\tau_u^v(\mathbf{x}_u) = \mathbf{R} \mathbf{X}_u + \mathbf{b}$.


\paragraph{Collecting the Dataset.}
The anchor embeddings $\mathbf{X}_u$ and $\mathbf{X}_v$ derive from sets of anchor points $\mathbf{A}_u$ and $\mathbf{A}_v$. Following previous works \citep{maiorca2023latent, Moschella2022-yf, ricciardi2025r3lrelativerepresentationsreinforcement} anchor pair ($\mathbf{a}_u$, $\mathbf{a}_v$) must share a semantic correspondence, meaning both samples represent the same underlying concept (e.g., the same spatial position in a racing track, viewed under two different visual styles).
In supervised learning contexts, anchor pairs can come from paired datasets (e.g., bilingual corpora). In the context of online RL, however, such datasets do not naturally exist. Hence, we collect datasets sharing a correspondence.
This correspondence can be obtained by either rolling out a policy and replaying the same set of actions with different visual variations, as already done in \cite{jian2023policy, ricciardi2025r3lrelativerepresentationsreinforcement}, or by simply applying visual transformations to the image in pixel space. This yields corresponding observation sets $\mathbf{A}_u$ and $\mathbf{A}_v$ that can be embedded by each domain’s encoder to create $\mathbf{X}_u$ and $\mathbf{X}_v$. Finally, we solve for $\tau_u^v$ using the SVD-based procedure above.


%\AR{da inserire forse: Specifically, we estimate $\tau_u^v$, following the technique used in \cite{maiorca2023latent}. which suggests that, given two latent spaces $\mathbf{X} \in \mathbb{R}^{n \times d1}$ and $\mathbf{Y} \in \mathbb{R}^{m \times d2}$ from independently trained deep neural networks, the transformation $\tau$ that directly maps $\mathbf{X}$ to $\mathbf{Y}$: (i) is mostly orthogonal and (ii) can be estimated from a few corresponding elements between the two spaces. In our work, $\mathbf{X}$ and $\mathbf{Y}$ are produced by $\enc_u$ and $\enc_v$, respectively. As in \cite{maiorca2023latent}, we use \textit{Singular Value Decomposition} (SVD) to estimate the optimal orthogonal transformation.}

In our context, we assume that an agent trained end-to-end to solve a specific task in a specific environment will generate a comprehensive set of observations, providing a reasonable approximation of the entire latent space. Nevertheless, forcing the agent to explore more could be beneficial in this context.
In our experiments, we gather parallel samples either by directly translating the observation in pixel space, when there is a well-defined known visual variation between the environments, or by replaying the same sequence of actions in both environments, that in this case must be deterministic and initialized with the same random seed. We leave to future research other possible approximation techniques for translating observations between different environments.

\section{Experiment}
\textbf{Datasets.} 
We assess the robustness of \ModelName\footnote{Our code is available at \url{https://github.com/RingBDStack/DiffSP}} in graph and node classification tasks. 
For graph classification, we use MUTAG~\cite{ivanov2019understanding}, IMDB-BINARY~\cite{ivanov2019understanding}, IMDB-MULTI~\cite{ivanov2019understanding}, REDDIT-BINARY~\cite{ivanov2019understanding}, and COLLAB~\cite{ivanov2019understanding}. For node classification, we test on Cora~\cite{yang2016revisiting}, CiteSeer~\cite{yang2016revisiting}, Polblogs~\cite{adamic2005political}, and Photo~\cite{shchur2018pitfalls}. Details are in Appendix~\ref{appendix:datasets}.

\noindent\textbf{Baselines.} 
Due to the limited research on robust GNNs targeting graph classification, we compare \ModelName\ with robust representation learning and structure learning methods designed for graph classification, including IDGL~\cite{chen2020iterative}, GraphCL~\cite{you2020graph}, VIB-GSL~\cite{sun2022graph}, G-Mixup~\cite{han2022g}, SEP~\cite{wu2022structural}, MGRL~\cite{ma2023multi}, SCGCN~\cite{zhao2024graph}, HSP-SL~\cite{zhang2019hierarchical}, SubGattPool~\cite{bandyopadhyay2020hierarchically} DIR~\cite{wu2022discovering}, and VGIB~\cite{yu2022improving}.
For node classification, we choose baselines from: 1) \textit{Structure Learning Based} methods, including GSR~\cite{zhao2023self}, GARNET~\cite{deng2022garnet}, and GUARD~\cite{li2023guard}; 2) \textit{Preprocessing Based} methods, including SVDGCN~\cite{entezari2020all} and JaccardGCN~\cite{wu2019adversarial}; 3) \textit{Robust Aggregation Based} methods, including RGCN~\cite{zhu2019robust}, Median-GCN~\cite{chen2021understanding}, GNNGuard~\cite{zhang2020gnnguard}, SoftMedian~\cite{geisler2021robustness}, and ElasticGCN~\cite{liu2021elastic}; and 4) \textit{Adversarial Training Based} methods, represented by the GraphADV~\cite{xu2019topology}.
Details of baselines can be found in Appendix~\ref{appendix:baselines}.
% #######

\noindent\textbf{Adversarial Attack Settings.}
For graph classification, we evaluate the performance against three strong evasion attacks: PR-BCD~\cite{geisler2021robustness}, GradArgmax~\cite{dai2018adversarial}, and CAMA-subgraph~\cite{wang2023revisiting}. 
For node classification, we evaluate six evasion attacks: 1) \textit{Targeted Attacks}: PR-BCD~\cite{geisler2021robustness}, Nettack~\cite{zugner2018adversarial}, and GR-BCD~\cite{geisler2021robustness}; 2) \textit{Non-targeted Attacks}: MinMax~\cite{li2020deeprobust}, DICE~\cite{zugner2018metalearningu}, and Random~\cite{li2020deeprobust}.
Further details on the attack methods and budget settings are provided in Appendix~\ref{appendix:attacks}.

\noindent\textbf{Hyperparameter Settings.} Details are provided in Appendix~\ref{appendix:implements}.
% \vspace{-0.2\baselineskip}

\begin{table*}[!tp]
  \captionsetup{skip=5pt}
  \centering
  \caption{Accuracy score (\% ± standard deviation) of \textit{node classification} task on real-world datasets against \textit{targeted attack}. 
  % The best results are shown in bold type and the runner-ups are \underline{underlined}.
  }
  \label{table:node_classification_targeted}
  \tabcolsep=0.1cm
  \resizebox{\linewidth}{!}{ 
\begin{tabular}{c|c|cccccccccccc|c}
    \toprule
    \textbf{Dataset} & \textbf{Attack} & GCN   & GSR   & GARNET & GUARD & SVD   & Jaccard & RGCN  & MedianGCN & GNNGuard & SoftMedian & ElasticGCN & GraphAT & \textbf{DiffSP} \\
    \midrule
    \multirow{4}[0]{*}{\textbf{Cora}} & PR-BCD & 55.59\scalebox{0.8}{±1.47} & \underline{74.75\scalebox{0.8}{±0.53}} & 66.80\scalebox{0.8}{±0.46} & 65.71\scalebox{0.8}{±0.79} & 64.66\scalebox{0.8}{±0.35} & 60.49\scalebox{0.8}{±1.00} & 55.91\scalebox{0.8}{±0.65} & 61.77\scalebox{0.8}{±0.68} & 65.14\scalebox{0.8}{±1.07} & 59.36\scalebox{0.8}{±0.63} & 63.86\scalebox{0.8}{±1.38} & 63.74\scalebox{0.8}{±0.99} & \textbf{75.13\scalebox{0.8}{±1.27}} \\
          & Nettack & 49.25\scalebox{0.8}{±5.28} & 67.25\scalebox{0.8}{±5.20} & 62.95\scalebox{0.8}{±4.75} & 52.50\scalebox{0.8}{±4.08} & 70.25\scalebox{0.8}{±0.79} & 56.75\scalebox{0.8}{±2.65} & 47.50\scalebox{0.8}{±1.67} & \underline{76.25\scalebox{0.8}{±5.17}} & 76.00\scalebox{0.8}{±5.03} & 67.50\scalebox{0.8}{±4.25} & 65.25\scalebox{0.8}{±3.22} & 73.50\scalebox{0.8}{±9.14} & \textbf{77.75\scalebox{0.8}{±3.62}} \\
          & GR-BCD & 66.34\scalebox{0.8}{±1.45} & \textbf{78.86\scalebox{0.8}{±0.53}} & 72.35\scalebox{0.8}{±0.91} & 72.08\scalebox{0.8}{±1.23} & 65.34\scalebox{0.8}{±0.72} & 71.88\scalebox{0.8}{±0.76} & 69.74\scalebox{0.8}{±2.08} & 72.90\scalebox{0.8}{±1.06} & 70.45\scalebox{0.8}{±1.20} & 75.52\scalebox{0.8}{±0.86} & 78.44\scalebox{0.8}{±1.42} & 77.06\scalebox{0.8}{±1.24} & \underline{76.83\scalebox{0.8}{±0.65}} \\
          & \cellcolor{gray!20}Average & \cellcolor{gray!20}{57.06} & \cellcolor{gray!20}\underline{73.62} & \cellcolor{gray!20}{67.37} & \cellcolor{gray!20}{63.43} & \cellcolor{gray!20}{66.75} & \cellcolor{gray!20}{63.04} & \cellcolor{gray!20}{57.72} & \cellcolor{gray!20}{70.31} & \cellcolor{gray!20}{70.53} & \cellcolor{gray!20}{67.46} & \cellcolor{gray!20}{69.18} & \cellcolor{gray!20}{71.43} & \cellcolor{gray!20}\textbf{76.57} \\
    \midrule
    \multirow{4}[0]{*}{\textbf{CiteSeer}} & PR-BCD & 45.06\scalebox{0.8}{±1.83} & \underline{63.33\scalebox{0.8}{±0.60}} & 55.75\scalebox{0.8}{±1.71} & 54.48\scalebox{0.8}{±0.96} & 59.61\scalebox{0.8}{±0.51} & 48.72\scalebox{0.8}{±1.20} & 41.08\scalebox{0.8}{±1.55} & 49.72\scalebox{0.8}{±0.71} & 49.78\scalebox{0.8}{±2.33} & 49.20\scalebox{0.8}{±0.89} & 48.79\scalebox{0.8}{±1.41} & 61.54\scalebox{0.8}{±1.01} & \textbf{64.35\scalebox{0.8}{±0.89}} \\
          & Nettack & 60.75\scalebox{0.8}{±8.34} & 75.25\scalebox{0.8}{±2.65} & 72.00\scalebox{0.8}{±2.84} & 59.25\scalebox{0.8}{±3.92} & \underline{77.25\scalebox{0.8}{±1.84}} & 71.50\scalebox{0.8}{±3.16} & 42.25\scalebox{0.8}{±4.78} & 74.00\scalebox{0.8}{±2.93} & 77.00\scalebox{0.8}{±3.50} & 59.00\scalebox{0.8}{±2.11} & 63.50\scalebox{0.8}{±3.76} & 73.25\scalebox{0.8}{±5.14} & \textbf{78.80\scalebox{0.8}{±4.53}} \\
          & GR-BCD & 50.56\scalebox{0.8}{±2.17} & \underline{65.50\scalebox{0.8}{±0.57}} & 57.04\scalebox{0.8}{±2.57} & 54.74\scalebox{0.8}{±1.82} & 60.40\scalebox{0.8}{±0.59} & 59.83\scalebox{0.8}{±1.17} & 44.82\scalebox{0.8}{±1.60} & 55.17\scalebox{0.8}{±1.31} & 58.88\scalebox{0.8}{±3.38} & 55.65\scalebox{0.8}{±0.93} & 60.37\scalebox{0.8}{±2.91} & 62.25\scalebox{0.8}{±1.25} & \textbf{65.63\scalebox{0.8}{±1.30}} \\
          & \cellcolor{gray!20}Average & \cellcolor{gray!20}{52.12} & \cellcolor{gray!20}\underline{68.02} & \cellcolor{gray!20}{61.60} & \cellcolor{gray!20}{56.16} & \cellcolor{gray!20}{65.75} & \cellcolor{gray!20}{60.02} & \cellcolor{gray!20}{42.72} & \cellcolor{gray!20}{59.63} & \cellcolor{gray!20}{61.89} & \cellcolor{gray!20}{54.62} & \cellcolor{gray!20}{57.55} & \cellcolor{gray!20}{65.68} & \cellcolor{gray!20}\textbf{69.59} \\
    \midrule
    \multirow{4}[0]{*}{\textbf{PolBlogs}} & PR-BCD & 73.73\scalebox{0.8}{±1.19} & 86.50\scalebox{0.8}{±0.52} & 75.52\scalebox{0.8}{±0.50} & 81.82\scalebox{0.8}{±1.06} & 78.02\scalebox{0.8}{±0.16} & 51.45\scalebox{0.8}{±1.23} & 74.01\scalebox{0.8}{±0.32} & 65.07\scalebox{0.8}{±4.21} & 51.93\scalebox{0.8}{±2.54} & \underline{87.88\scalebox{0.8}{±1.29}} & 74.71\scalebox{0.8}{±2.89} & 80.67\scalebox{0.8}{±0.85} & \textbf{90.24\scalebox{0.8}{±0.92}} \\
          & Nettack & 74.75\scalebox{0.8}{±4.92} & 75.75\scalebox{0.8}{±1.69} & 83.75\scalebox{0.8}{±3.77} & 76.75\scalebox{0.8}{±3.13} & 80.75\scalebox{0.8}{±1.69} & 47.75\scalebox{0.8}{±6.06} & 76.50\scalebox{0.8}{±1.75} & 46.00\scalebox{0.8}{±2.11} & 50.24\scalebox{0.8}{±6.52} & 83.50\scalebox{0.8}{±3.37} & \textbf{86.00\scalebox{0.8}{±4.12}} & 83.95\scalebox{0.8}{±2.72} & \underline{84.55\scalebox{0.8}{±5.90}} \\
          & GR-BCD & 71.31\scalebox{0.8}{±3.41} & 84.75\scalebox{0.8}{±0.66} & 75.49\scalebox{0.8}{±0.77} & 87.13\scalebox{0.8}{±3.63} & 90.27\scalebox{0.8}{±0.36} & 50.71\scalebox{0.8}{±1.98} & 79.13\scalebox{0.8}{±0.54} & 56.95\scalebox{0.8}{±5.15} & 51.26\scalebox{0.8}{±1.78} & 87.50\scalebox{0.8}{±0.81} & 91.12\scalebox{0.8}{±2.71} & \underline{92.70\scalebox{0.8}{±0.18}} & \textbf{92.75\scalebox{0.8}{±0.38}} \\
          & \cellcolor{gray!20}Average & \cellcolor{gray!20}{73.26} & \cellcolor{gray!20}{82.33} & \cellcolor{gray!20}{78.25} & \cellcolor{gray!20}{81.90} & \cellcolor{gray!20}{83.01} & \cellcolor{gray!20}{49.97} & \cellcolor{gray!20}{76.55} & \cellcolor{gray!20}{56.01} & \cellcolor{gray!20}{51.14} & \cellcolor{gray!20}\underline{86.29} & \cellcolor{gray!20}{83.94} & \cellcolor{gray!20}{85.77} & \cellcolor{gray!20}\textbf{89.18} \\
    \midrule
    \multirow{4}[0]{*}{\textbf{Photo}} & PR-BCD & 65.35\scalebox{0.8}{±2.48} & 73.81\scalebox{0.8}{±1.90} & 77.58\scalebox{0.8}{±1.93} & \underline{84.14\scalebox{0.8}{±3.75}} & 80.04\scalebox{0.8}{±1.13} & 66.13\scalebox{0.8}{±2.82} & 63.79\scalebox{0.8}{±11.99} & 79.75\scalebox{0.8}{±0.96} & 65.62\scalebox{0.8}{±2.63} & 76.84\scalebox{0.8}{±1.46} & 76.21\scalebox{0.8}{±1.89} & 78.72\scalebox{0.8}{±2.13} & \textbf{84.78\scalebox{0.8}{±1.82}} \\
          & Nettack & 83.70\scalebox{0.8}{±5.16} & 83.75\scalebox{0.8}{±4.12} & 88.00\scalebox{0.8}{±3.07} & 84.25\scalebox{0.8}{±2.65} & 82.75\scalebox{0.8}{±5.45} & 84.00\scalebox{0.8}{±5.43} & 75.50\scalebox{0.8}{±3.07} & 86.50\scalebox{0.8}{±3.16} & 87.50\scalebox{0.8}{±5.77} & \textbf{88.75\scalebox{0.8}{±1.32}} & 83.00\scalebox{0.8}{±3.29} & 87.25\scalebox{0.8}{±12.30} & \underline{87.75\scalebox{0.8}{±4.32}} \\
          & GR-BCD & 69.11\scalebox{0.8}{±7.85} & \underline{84.84\scalebox{0.8}{±2.29}} & 85.27\scalebox{0.8}{±1.57} & 82.15\scalebox{0.8}{±2.24} & 83.74\scalebox{0.8}{±1.11} & 76.24\scalebox{0.8}{±2.98} & 68.60\scalebox{0.8}{±7.28} & 84.23\scalebox{0.8}{±1.49} & 79.20\scalebox{0.8}{±1.80} & 79.69\scalebox{0.8}{±1.19} & 83.94\scalebox{0.8}{±0.95} & 87.49\scalebox{0.8}{±1.26} & \textbf{87.58\scalebox{0.8}{±0.58}} \\
          & \cellcolor{gray!20}Average & \cellcolor{gray!20}{72.72} & \cellcolor{gray!20}{80.80} & \cellcolor{gray!20}{83.62} & \cellcolor{gray!20}{83.51} & \cellcolor{gray!20}{82.18} & \cellcolor{gray!20}{75.46} & \cellcolor{gray!20}{69.30} & \cellcolor{gray!20}{83.49} & \cellcolor{gray!20}{77.44} & \cellcolor{gray!20}{81.76} & \cellcolor{gray!20}{81.05} & \cellcolor{gray!20}\underline{84.48} & \cellcolor{gray!20}\textbf{86.70} \\
    \bottomrule
\end{tabular}
}

\vspace{-0.5em}
\end{table*}


\begin{table*}[htbp]
  \centering
  \captionsetup{skip=5pt}
  \caption{Accuracy score (\% ± standard deviation) of \textit{node classification} task on real-world datasets against \textit{non-targeted attack}. 
  % The best results are shown in bold type and the runner-ups are \underline{underlined}.
  }
  \label{table:node_classification_non_targeted}
  \tabcolsep=0.1cm
  \resizebox{\linewidth}{!}{ 
   \begin{tabular}{c|c|cccccccccccc|c}
    \toprule
    \textbf{Dataset} & \textbf{Attack} & GCN   & GSR   & GARNET & GUARD & SVD   & Jaccard & RGCN  & MedianGCN & GNNGuard & SoftMedian & ElasticGCN & GraphAT & \textbf{DiffSP} \\
    \midrule
    \multirow{4}[0]{*}{\textbf{Cora}} & MinMax & 59.91\scalebox{0.8}{±2.60} & 67.80\scalebox{0.8}{±2.18} & 65.68\scalebox{0.8}{±0.58} & 61.62\scalebox{0.8}{±2.85} & 64.75\scalebox{0.8}{±0.96} & 64.43\scalebox{0.8}{±2.48} & 62.49\scalebox{0.8}{±2.19} & 56.35\scalebox{0.8}{±3.34} & 63.63\scalebox{0.8}{±2.40} & \underline{74.53\scalebox{0.8}{±0.70}} & 17.05\scalebox{0.8}{±5.33} & 63.35\scalebox{0.8}{±2.60} & \textbf{75.00\scalebox{0.8}{±1.12}} \\ 
          & DICE  & 69.58\scalebox{0.8}{±2.17} & 74.55\scalebox{0.8}{±0.74} & 68.88\scalebox{0.8}{±1.08} & 71.50\scalebox{0.8}{±2.68} & 59.52\scalebox{0.8}{±0.39} & 71.89\scalebox{0.8}{±0.56} & 69.92\scalebox{0.8}{±0.97} & 71.61\scalebox{0.8}{±0.72} & 68.82\scalebox{0.8}{±0.95} & 73.38\scalebox{0.8}{±0.68} & 74.11\scalebox{0.8}{±1.28} & \underline{75.84\scalebox{0.8}{±0.57}} & \textbf{75.96\scalebox{0.8}{±0.87}} \\ 
          & Random & 70.43\scalebox{0.8}{±2.22} & 77.37\scalebox{0.8}{±0.88} & 75.63\scalebox{0.8}{±0.93} & 74.96\scalebox{0.8}{±0.51} & 62.54\scalebox{0.8}{±0.65} & 73.74\scalebox{0.8}{±0.60} & 72.74\scalebox{0.8}{±1.00} & 74.31\scalebox{0.8}{±0.95} & 68.33\scalebox{0.8}{±1.72} & 77.52\scalebox{0.8}{±0.65} & 74.06\scalebox{0.8}{±3.87} & \underline{77.39\scalebox{0.8}{±0.91}} & \textbf{77.63\scalebox{0.8}{±0.80}} \\ 
          & \cellcolor{gray!20}Average & \cellcolor{gray!20}{66.64}  & \cellcolor{gray!20}\underline{73.24}  & \cellcolor{gray!20}{70.06}  & \cellcolor{gray!20}{69.36}  & \cellcolor{gray!20}{62.27}  & \cellcolor{gray!20}{70.02}  & \cellcolor{gray!20}{68.38}  & \cellcolor{gray!20}{67.42}  & \cellcolor{gray!20}{66.93}  & \cellcolor{gray!20}{75.14}  & \cellcolor{gray!20}{55.07}  & \cellcolor{gray!20}{72.19}  & \cellcolor{gray!20}\textbf{76.20} \\
    \midrule
    \multirow{4}[0]{*}{\textbf{CiteSeer}} & MinMax & 52.07\scalebox{0.8}{±6.63} & 54.74\scalebox{0.8}{±4.92} & 59.00\scalebox{0.8}{±2.35} & 58.02\scalebox{0.8}{±1.44} & 35.83\scalebox{0.8}{±1.89} & 56.65\scalebox{0.8}{±3.81} & 42.85\scalebox{0.8}{±7.72} & 53.39\scalebox{0.8}{±3.44} & 57.98\scalebox{0.8}{±2.97} & 60.84\scalebox{0.8}{±1.40} & 17.05\scalebox{0.8}{±5.33} & \underline{61.54\scalebox{0.8}{±3.70}} & \textbf{61.59\scalebox{0.8}{±1.10}} \\
          & DICE  & 57.46\scalebox{0.8}{±1.63} & 62.48\scalebox{0.8}{±1.08} & 55.59\scalebox{0.8}{±3.01} & 62.19\scalebox{0.8}{±0.99} & 57.33\scalebox{0.8}{±0.49} & 63.00\scalebox{0.8}{±0.87} & 50.88\scalebox{0.8}{±1.59} & 59.95\scalebox{0.8}{±0.97} & 58.85\scalebox{0.8}{±3.22} & 59.85\scalebox{0.8}{±0.81} & 60.30\scalebox{0.8}{±1.46} & \underline{65.28\scalebox{0.8}{±0.81}} & \textbf{65.43\scalebox{0.8}{±0.70}} \\
          & Random & 56.19\scalebox{0.8}{±3.08} & 64.01\scalebox{0.8}{±1.08} & 56.34\scalebox{0.8}{±3.70} & 62.47\scalebox{0.8}{±0.88} & 54.54\scalebox{0.8}{±0.62} & 64.20\scalebox{0.8}{±0.46} & 50.13\scalebox{0.8}{±1.95} & 60.60\scalebox{0.8}{±0.81} & 61.51\scalebox{0.8}{±3.32} & 58.66\scalebox{0.8}{±1.49} & 58.00\scalebox{0.8}{±3.61} & \underline{64.94\scalebox{0.8}{±1.12}} & \textbf{66.78\scalebox{0.8}{±0.54}} \\
          & \cellcolor{gray!20}Average & \cellcolor{gray!20}{55.24}  & \cellcolor{gray!20}{60.41}  & \cellcolor{gray!20}{56.98}  & \cellcolor{gray!20}{60.89}  & \cellcolor{gray!20}{49.23}  & \cellcolor{gray!20}{61.28}  & \cellcolor{gray!20}{47.95}  & \cellcolor{gray!20}{57.98}  & \cellcolor{gray!20}{59.45}  & \cellcolor{gray!20}{59.78}  & \cellcolor{gray!20}{45.12}  & \cellcolor{gray!20}\underline{63.92}  & \cellcolor{gray!20}\textbf{64.60} \\
    \midrule
    \multirow{4}[0]{*}{\textbf{PolBlogs}} & MinMax & 86.96\scalebox{0.8}{±0.43} & 88.56\scalebox{0.8}{±0.82} & 87.85\scalebox{0.8}{±0.19} & \underline{89.51\scalebox{0.8}{±0.85}} & 87.11\scalebox{0.8}{±0.32} & 51.01\scalebox{0.8}{±1.75} & 87.04\scalebox{0.8}{±0.19} & 87.95\scalebox{0.8}{±4.81} & 50.32\scalebox{0.8}{±1.19} & 88.76\scalebox{0.8}{±0.37} & 87.33\scalebox{0.8}{±0.62} & 88.32\scalebox{0.8}{±0.35} & \textbf{89.52\scalebox{0.8}{±3.08}} \\
          & DICE  & 76.52\scalebox{0.8}{±2.76} & 80.75\scalebox{0.8}{±4.72} & 85.05\scalebox{0.8}{±1.01} & 83.76\scalebox{0.8}{±0.78} & 82.84\scalebox{0.8}{±0.20} & 50.27\scalebox{0.8}{±1.91} & 81.50\scalebox{0.8}{±0.44} & 74.19\scalebox{0.8}{±3.02} & 50.79\scalebox{0.8}{±1.59} & 86.47\scalebox{0.8}{±0.45} & 82.40\scalebox{0.8}{±2.24} & \underline{87.39\scalebox{0.8}{±0.44}} & \textbf{88.85\scalebox{0.8}{±1.32}} \\
          & Random & 83.24\scalebox{0.8}{±5.81} & 87.81\scalebox{0.8}{±1.03} & 83.42\scalebox{0.8}{±1.59} & 87.48\scalebox{0.8}{±1.51} & 85.59\scalebox{0.8}{±0.31} & 51.02\scalebox{0.8}{±1.75} & 85.46\scalebox{0.8}{±0.40} & 83.57\scalebox{0.8}{±2.71} & 50.28\scalebox{0.8}{±1.13} & 90.35\scalebox{0.8}{±0.56} & 49.50\scalebox{0.8}{±2.20} & \underline{90.50\scalebox{0.8}{±0.56}} & \textbf{92.61\scalebox{0.8}{±0.93}} \\
          & \cellcolor{gray!20}Average & \cellcolor{gray!20}{82.24}  & \cellcolor{gray!20}{85.71}  & \cellcolor{gray!20}{85.44}  & \cellcolor{gray!20}{86.92}  & \cellcolor{gray!20}{85.18}  & \cellcolor{gray!20}{50.77}  & \cellcolor{gray!20}{84.67}  & \cellcolor{gray!20}{81.90}  & \cellcolor{gray!20}{50.46}  & \cellcolor{gray!20}{88.53}  & \cellcolor{gray!20}{73.08}  & \cellcolor{gray!20}\underline{88.74}  & \cellcolor{gray!20}\textbf{90.33} \\
    \midrule 
    \multirow{4}[0]{*}{\textbf{Photo}} & MinMax & 73.12\scalebox{0.8}{±3.17} & 76.36\scalebox{0.8}{±3.09} & 81.75\scalebox{0.8}{±1.91} & 75.89\scalebox{0.8}{±3.28} & 69.92\scalebox{0.8}{±5.50} & 74.20\scalebox{0.8}{±3.94} & \underline{87.04\scalebox{0.8}{±0.19}} & 67.43\scalebox{0.8}{±4.31} & 71.44\scalebox{0.8}{±6.66} & 85.23\scalebox{0.8}{±2.12} & 8.56\scalebox{0.8}{±3.24} & 81.70\scalebox{0.8}{±2.48} & \textbf{88.51\scalebox{0.8}{±0.61}} \\ 
          & DICE  & 84.60\scalebox{0.8}{±1.17} & 82.52\scalebox{0.8}{±1.66} & \underline{85.43\scalebox{0.8}{±0.92}} & 82.92\scalebox{0.8}{±1.27} & 76.42\scalebox{0.8}{±1.39} & 83.20\scalebox{0.8}{±1.44} & 81.57\scalebox{0.8}{±0.44} & 82.83\scalebox{0.8}{±2.45} & 83.87\scalebox{0.8}{±1.19} & 84.72\scalebox{0.8}{±0.90} & 81.86\scalebox{0.8}{±3.61} & \textbf{87.22\scalebox{0.8}{±1.13}} & 83.52\scalebox{0.8}{±1.19} \\ 
          & Random & 85.38\scalebox{0.8}{±1.76} & 83.62\scalebox{0.8}{±2.91} & 84.12\scalebox{0.8}{±3.95} & 85.49\scalebox{0.8}{±1.55} & 79.13\scalebox{0.8}{±2.84} & 83.37\scalebox{0.8}{±1.93} & \textbf{86.87\scalebox{0.8}{±2.89}} & 84.07\scalebox{0.8}{±2.52} & 83.24\scalebox{0.8}{±4.83} & 85.95\scalebox{0.8}{±1.06} & 75.32\scalebox{0.8}{±2.38} & \underline{86.23\scalebox{0.8}{±2.26}} & 84.60\scalebox{0.8}{±0.46} \\ 
          & \cellcolor{gray!20}Average & \cellcolor{gray!20}{81.03}  & \cellcolor{gray!20}{80.83}  & \cellcolor{gray!20}{83.77}  & \cellcolor{gray!20}{81.43}  & \cellcolor{gray!20}{75.16}  & \cellcolor{gray!20}{80.26}  & \cellcolor{gray!20}{85.16}  & \cellcolor{gray!20}{78.11}  & \cellcolor{gray!20}{79.52}  & \cellcolor{gray!20}\underline{85.30}  & \cellcolor{gray!20}{55.25}  & \cellcolor{gray!20}{85.05}  & \cellcolor{gray!20}\textbf{85.54} \\
    \bottomrule 
    \end{tabular}
}

% \vspace{-0.3em}
\end{table*}



\subsection{Adversarial Robustness}
\textbf{Graph Classification Robustness.}
We evaluated the robustness of the graph classification task under three adversarial attacks across five datasets. Since the choice of classifier affects attack effectiveness, especially in graph classification due to pooling operations, it is crucial to standardize the model architecture. Simple changes like adding a linear layer can reduce the impact of attacks. To ensure a fair comparison, we used a two-layer GCN with a linear layer and mean pooling for both the baselines and our proposed \ModelName. Each experiment was repeated 10 times, with results shown in Table~\ref{table:graph_classification}.

\textit{Result.} 1) \ModelName~ consistently outperforms all baselines under the PR-BCD attack and achieves the highest average robustness across all attacks on five datasets, with a notable 4.80\% average improvement on the IMDB-BINARY dataset.
2) It's important to note that while baselines may excel against specific attacks, they often struggle with others. In contrast, \ModelName\ maintains consistent robustness across both datasets and attacks, thanks to its ability to learn clean distributions and purify adversarial graphs without relying on specific priors about the dataset or attack strategies.


\begin{figure*}[!t]
    \begin{minipage}{0.33\textwidth}
        \centering
        \includegraphics[width=\textwidth]{figure/ablation.pdf}
    \vspace{-2.5em}
        
        \caption{Ablation Study}
        \label{fig_ablation}
    \end{minipage} 
    \begin{minipage}{0.33\textwidth}
    \centering
    \includegraphics[width=\textwidth]{figure/time_influence.pdf}
    \vspace{-2.5em}
    
    \caption{Purification Steps Study}
    \label{fig_diffusion_steps}
\end{minipage}
\begin{minipage}{0.33\textwidth}
    \centering
    \includegraphics[width=\textwidth]{figure/tg_influence.pdf}
    \vspace{-2.5em}
    
    \caption{Guide Scale Study}
    \label{fig_graph_transfer_entropy}
\end{minipage}\\
\begin{minipage}{1.0\textwidth}
    \centering
    \includegraphics[width=\textwidth]{figure/visualize.pdf}
    \vspace{-2.5em}
    \caption{Visualization Study}
    \label{fig_visualization}
    \vspace{-1em}
\end{minipage}
\end{figure*}

\noindent\textbf{Node Classification Robustness.}
We evaluate the robustness of \ModelName\ on the node classification task against six attacks across four datasets, using the same other settings as in the graph classification experiments. The results are presented in Table \ref{table:node_classification_targeted} and Table \ref{table:node_classification_non_targeted}.

\textit{Result.} We have two key observations: 
1) \ModelName\ achieves the best average performance across both targeted and non-targeted attacks on all datasets, demonstrating its robust adaptability across diverse scenarios. 
2) \ModelName\ performs particularly well under stronger attacks but is less effective against weaker ones like Random and DICE. This is because these attacks introduce numerous noisy edges, many of which do not exhibit distinctly adversarial characteristics. Instead, these edges are often plausible within the graph. Consequently, these additional perturbations can mislead \ModelName, making it harder to discern the correct information within the graph, leading the generated graph to deviate from the target clean graph.

% \subsection{Model Analyses}

\vspace{-0.5em}

\subsection{Ablation Study}
In this subsection, we analyze the effectiveness of \ModelName's two core components: 1) \ModelName~ (w/o LN), which excludes the LID-Driven Non-Isotropic Diffusion Mechanism; and 2) \ModelName\ (w/o TG), which excludes the Graph Transfer Entropy Guided Denoising Mechanism. We evaluate variants on the IMDB-BINARY and COLLAB datasets under PR-BCD and GradArgmax attacks for graph classification and on the Cora and CiteSeer dataset under PR-BCD and MinMax attacks for node classification. Results are shown in Figure~\ref{fig_ablation}.

\textit{Result.} \ModelName~ consistently outperforms the other variants. \ModelName\ (w/o LN) over-perturbs the valuable parts of the graph leading to degraded performance. Similarly, \ModelName~ (w/o TG) increases the uncertainty of generation, causing deviations from the target clean graph. These reduce the robustness against evasion attacks.

\vspace{-1em}
\subsection{Study on Cross-Dataset Generalization}
We assess \ModelName's generalization ability. The goal is to determine whether \ModelName~ effectively learns the predictive patterns of clean graphs. We train \ModelName\ on IMDB-BINARY and use the trained model to purify graphs on IMDB-MULTI, and vice versa.

\textit{Result.} As shown in Table~\ref{table:transfer}, \ModelName~ trained on different datasets, still demonstrates strong robustness compared to GCN trained and tested on the same dataset. Furthermore, \ModelName~ exhibits only a small performance gap compared to when it is trained and tested on the same dataset directly.
These results highlight \ModelName's ability to learn the underlying clean distribution of a category of data and capture predictive patterns that generalize across diverse datasets.

\begin{figure*}[!ht]
    \centering
    \includegraphics[width=.95\linewidth]{figures-src/transfer.pdf}
    \caption{SliderSpace directions for the ``person'' concept successfully generalize to related ``police'' and ``athlete'' concepts. They also transfer to out-of-domain concepts like ``dog''}
    \label{fig:transfer}
\end{figure*}



\subsection{Study on Purification Steps}
We evaluate the performance as the number of diffusion steps varies. For graph classification on the IMDB-BINARY dataset, we adjust the diffusion steps from 1 to 9 under GradArgMax, PR-BCD, and CAMA-Subgraph attacks. For node classification on the Cora dataset, we vary the diffusion steps from 1 to 12 under the GR-BCD, PR-BCD, and MinMax attacks.
The results are shown in Figure~\ref{fig_diffusion_steps}.


\textit{Result.}
We observe that all-time step settings demonstrate the ability to effectively purify adversarial graphs. At smaller time steps, the overall trend shows increasing accuracy as the number of diffusion steps increases. This is likely because fewer time steps do not introduce enough noise to sufficiently suppress the adversarial information in the graph. As the diffusion steps increase, we do not see a significant decline in performance. This stability can be attributed to our LID-Driven Non-Isotropic Diffusion Mechanism, which minimizes over-perturbation of the clean graph parts. Additionally, we found that the time required for purifying increased linearly.




\subsection{Study on Scale of Graph Transfer Entropy}
To analyze the impact of the guidance scale $\lambda$, we vary $\lambda$ from $\text{1e}^{-1}$ to $\text{1e}^{5}$. The results are presented in Figure~\ref{fig_graph_transfer_entropy}. For graph classification, experiments are conducted on the IMDB-BINARY dataset under the GradArgmax, PR-BCD, and CAMA-Subgraph attacks. For node classification, experiments are performed on the Cora dataset under the PR-BCD, GR-BCD, and MinMax attacks.

\textit{Result.} The results show that smaller values of $\lambda$ have minimal effect on accuracy. However, they reduce the stability of the purification during the reverse denoising process, leading to a higher standard deviation. This instability arises because the model is less effective at reducing uncertainty and guiding the generation process when $\lambda$ is too small.  On the other hand, large $\lambda$ values decrease accuracy by overemphasizing guidance, causing the model to reintroduce adversarial information into the generated graph.



\subsection{Graph Purification Visualization}
We visualize snapshots of different purification time steps on the IMDB-BINARY dataset using NetworkX~\cite{hagberg2008exploring}, as shown in Figure~\ref{fig_visualization}. The visualization process demonstrates that \ModelName\ has mastered the ability to generate clean graphs, achieving graph purification.

% More experiments and analyses are provided in Appendix~\ref{appendix:additional_analysis}.

\vspace{-5pt}
\section{Discussion}
\textbf{Conclusion.}
In this work, we propose the \textit{\methodname{}} metric, $M_{AP}$, to evaluate preference data quality in alignment.
By measuring the gap from the model's current implicit reward margin to the target explicit reward margin, $M_{AP}$ quantifies the discrepancy between the current model and the aligned optimum, thereby indicating the potential for alignment enhancement.
Extensive experiments validate the efficacy of $M_{AP}$ across various training settings under offline and self-play preference learning scenarios.

\textbf{Limitations and future work}. 
Despite the performance improvements, $M_{AP}$ requires tuning a parameter $\beta$ to combine the explicit and implicit margins; future work could explore how to set this ratio automatically.
Additionally, while our experiments focus on the widely applied DPO and SimPO objectives, a broader investigation with alternative preference learning methods is crucial in future works.

% \section{Conclusion}
% In this paper, we introduce the \methodname{} metric to evaluate preference data quality in LLM alignment.
% By measuring the discrepancy between the model's current implicit reward margin to the target explicit reward margin, this metric quantifies the gap between the current model and the aligned optimum, thereby indicating the potential for alignment enhancement.
% Empirical results demonstrate that training on data selected by our metric consistently improves alignment performance, outperforming existing metrics across different base models and training objectives.
% Moreover, this metric extends to data generation scenarios (\ie self-play alignment): by identifying high-quality data from the intrinsic self-generated context, our metric yields superior results across various training settings, providing a comprehensive solution for enhancing LLM alignment through optimized
% preference data generation, selection, and utilization.


\section*{Impact Statement}
This paper presents work whose goal is to advance the field of Machine Learning. There are many potential societal consequences of our work, none which we feel must be specifically highlighted here.

\bibliography{ref}
\bibliographystyle{icml2025}

%%%%%%%%%%%%%%%%%%%%%%%%%%%%%%%%%%%%%%%%%%%%%%%%%%%%%%%%%%%%%%%%%%%%%%%%%%%%%%%
%%%%%%%%%%%%%%%%%%%%%%%%%%%%%%%%%%%%%%%%%%%%%%%%%%%%%%%%%%%%%%%%%%%%%%%%%%%%%%%
% APPENDIX
%%%%%%%%%%%%%%%%%%%%%%%%%%%%%%%%%%%%%%%%%%%%%%%%%%%%%%%%%%%%%%%%%%%%%%%%%%%%%%%
%%%%%%%%%%%%%%%%%%%%%%%%%%%%%%%%%%%%%%%%%%%%%%%%%%%%%%%%%%%%%%%%%%%%%%%%%%%%%%%
\newpage
\appendix
\onecolumn
\section{Inference Configuration} \label{app:inference_config}

\subsection{Software} 
Our experiments build upon two open-source projects: \emph{Easy-to-Hard Generalization}~\cite{sun2024easy} for the MATH dataset and \emph{bigcode-evaluation-harness}~\cite{bigcode-evaluation-harness} for the MBPP dataset. We employ \emph{vLLM}~\cite{kwon2023efficient} to accelerate inference. All experiments can be reproduced on a single L40S or A6000 GPU.

\subsection{Sampling Parameters}
We use zero-shot inference for models fine-tuned specifically for each dataset. For general-purpose models, we use four in-context examples (few-shot inference) to ensure correct output formatting. The maximum output length is set to 1024 tokens for all tasks. For the MATH dataset, we use top-k sampling with $k = 20$. No additional sampling constraints are imposed for the MBPP dataset.

\subsection{Metric Calculation}
To compute majority-vote results for the MATH dataset, we consider two samples to have the same answer if they match after normalization. For the pass@K metric, we follow the definition in~\citet{chen2021evaluating}. Let $N$ be the total number of samples and $C$ be the number of correct samples. Then \(\mathrm{pass}@K\) is defined as:
\begin{align}
\mathrm{pass}@K = 1 - \frac{\binom{N - C}{K}}{\binom{N}{K}}.
\end{align}
\section{Details of the Stochastic Process Model}
\label{app: toy model}
We introduce a stochastic process model to explain that (1) the token-level entropy increases steadily at the beginning but rises rapidly when the sampling temperature reaches a certain threshold. (2) The optimal temperature is near the turning point when using multi-sample aggregation strategies.

The stochastic process model has two underlying assumptions: (1) Every token can be labeled as `proper' or `improper' at each decoding step. Generally, proper tokens have relatively higher logits than improper tokens, while the number of improper tokens is much higher than that of proper tokens. (2) When an improper token is generated, improper tokens have a higher generation probability in the next step, and vice versa.

Under these two assumptions, we can calculate the token-level entropy under different sampling temperatures, and the temperature-entropy curve fits that of real language models. Meanwhile, the percentage of improper tokens quickly increases after the turning point, implying a quick drop in sample quality in real language models.

\subsection{Model Setup}
\subsubsection{Initial Conditions}
We consider a discrete-time process \(\{x_t\}_{t=0}^{K}\) where each \(x_t \in [0,1]\) represents the model’s probability of producing an improper token at time step \(t\). We start with an initial error rate:
\[
x_0 = x_{\text{init}} \in [0,1].
\]
Conceptually, \(x_{\text{init}}\) corresponds to the model’s baseline `error propensity' at the start. This value is related to the sampling temperature \(T\) of the language model: higher \(T\) typically yields a flatter probability distribution over tokens, increasing the chance of selecting an improper token and thus increasing \(x_{\text{init}}\). (See Section \ref{sec: temp_to_init} for a heuristic link between temperature and initial error rate.)

\subsubsection{Interpreting the Error Rate}
At each step \(t\), the language model chooses a single token, and each token is classified as proper or improper. Although in practice, the correctness of a token depends on the context and is not truly binary, we approximate this by treating correctness as a Bernoulli trial:
\begin{itemize}
   \item Probability of producing an improper token: \(x_t\);
   \item Probability of producing a proper token: \(1 - x_t\).
\end{itemize}
Define a random variable \(E_t\) that indicates whether an error occurred at time \(t\):
   \[
   E_t = \begin{cases} 
   1 & \text{with probability } x_t, \text{ (improper token)},\\
   0 & \text{with probability } 1 - x_t, \text{ (proper token)}.
   \end{cases}
   \]

\subsubsection{Error Rate Update Rules}
   After each step, the error rate \(x_{t+1}\) is updated based on whether the token at time \(t\) was improper or proper:

   \paragraph{If an error occurs (\(E_t = 1\)):}
   The error rate is increased. Intuitively, making a mistake can make the model more likely to continue making errors. Formally, we update:
   \[
   x_{t+1} = 1 - (1 - x_t)^\alpha.
   \]
   For \(x_t \in [0,1]\) and $\alpha > 1$ ($\alpha$ is a hyperparameter), it can be seen that \(x_{t+1} \ge x_t\). Here $\alpha$ can be regarded as the noise tolerance rate, measuring how stable the model is when suffering from unexpected noises, and we will try different $\alpha$ in experiments.

   \paragraph{If a proper token is produced (\(E_t = 0\)):}
   The error rate is reduced, reflecting a `reinforcement' of correct behavior. We do this by:
   \[
   x_{t+1} = \max(x_t^\alpha,\ x_{\text{init}}).
   \]
   It generally makes it smaller, so this step lowers the error rate. However, we do not allow the error rate to drop below the initial baseline \(x_{\text{init}}\).

\subsubsection{Linking Initial Error Rate and Temperature}
\label{sec: temp_to_init}
   At step $t$, the model’s token probability mass is divided into:
   \begin{itemize}
       \item \textbf{Improper tokens} with total probability \(P_{1, \text{improper}} = x_t\).
       \item \textbf{Proper tokens} with total probability \(P_{0, \text{proper}} = 1 - x_t\).
   \end{itemize}
Therefore, by definition:
   \[
   x_{\text{init}} = P_{1, \text{improper}}.
   \]
Under higher temperatures \(T\), the softmax distribution flattens, increasing \(P_{1,\text{improper}}\) because the number of improper tokens is large but their logits are low. Thus, $x_\text{init}$ increases as \(T\) increases.

\subsubsection{Type of Tokens during decoding}

The probability of tokens when decoding is usually multi-peak (i.e., except for the token with the highest logit, some other tokens have reasonably high logits and are also acceptable during decoding), so it is natural to consider a scenario with three categories of tokens:
   \begin{itemize}
       \item \textbf{Proper tokens:} A small number of tokens with high logits. Let $N_0$ be the number of proper tokens. To capture the multi-peak behavior, the logits of proper tokens $l_{0,1}, ..., l_{0, N_0}$are sampled from the Gaussian distribution $\mathcal{N}(L_0, \sigma_0)$.
       \item \textbf{Low Probability Improper tokens:} Many low-logit tokens where language models seldom choose them. Let $N_1$ be the number of tokens in this type and their logits are set to $L_1$ for simplicity.
       \item \textbf{High Probability Improper tokens:} Due to insufficient training or errors in training data, some tokens may have exceptionally high logits but are logically improper in (\textit{e.g., } the token $3$ in $1 + 1 = 3$). Since different language models behave differently and this type of token will be selected regardless of temperature, we only consider decoding without high-probability improper tokens in our discussion.
   \end{itemize}
For the first two types of tokens, we have:
   \[
    L_{0} > L_{1}, \ \ N_0 \ll N_1.
   \]
\subsubsection{Token Probability During Sampling}

At each step \(t\), the probability of producing improper tokens is \(x_t\). Let $p_{t, \text{proper/improper}, j}$ be the probability of the $j$-th proper/improper tokens at step $t$. For the improper tokens, we have:
\[
p_{t,\text{improper}, j} = \frac{x_t}{N_1},\ \ \forall j \in [1, N_1].
\]
For the proper tokens, we allocate the remaining probability \(1 - x_{t}\) according to their relative logits:
\[
p_{t,\text{proper}, j} = (1 - x_{t}) \, \text{softmax}(l_{0,1}, ..., l_{0, N_0})_j, \ \ \forall j \in [1, N_0].
\]
This ensures that the relative ordering of probabilities for the proper tokens remains determined by their logits, while the total mass allocated to improper tokens is \(x_t\).

\subsubsection{Entropy Calculation}

We define the token-level entropy \(\mathcal{H}\) over a sequence of decoding steps:
\[
\mathcal{H} = \frac{1}{K} \sum_{t=1}^{K} \left( -\sum_{\text{j}}^{N_0} p_{t,\text{proper}, j} \log p_{t,\text{proper}, j} -\sum_{\text{j}}^{N_1} p_{t,\text{improper}, j} \log p_{t,\text{improper}, j} \right)\]
Here, $K$ is the total number of decoding steps considered.

\subsection{Experiment}

\subsubsection{Model Hyperparameter}

The toy model has the following hyperparameters:
\begin{itemize}
    \item The numbers of proper and improper tokens: $N_0, N_1$;
    \item The logits of proper and improper tokens: $l_{0,\{1, ..., N_0\}}, l_{1,\{1, ..., N_1\}}$, where:
\[l_{0, i} \sim \mathcal{N}(L_0, \sigma_0), \ \ l_{1, i} = L_1.\]
    \item The number of total steps $K$.
    \item The noise tolerance rate $\alpha$.
\end{itemize}
The input is temperature $(T)$ and the output is the average token-level entropy $(E)$ over 500 samples.
In our experiment, we set the hyperparameters to be:
\begin{align*}
N_0=10, \ \ N_1 = 30000, \ \ L_0 = 0,\ \  L_1 = -10,\ \ \sigma_0 = 1,\ \ K=512.
\end{align*}
We test different noise tolerance rates $\alpha \in [1.5, 2.0, 2.5, 3.0]$ to show behaviors under different noise tolerance rates.

\subsubsection{Result}

Here is the temperature-entropy curve derived from the toy model under different noise tolerance rates $\alpha$ (Figure~\ref{fig:toy_curves}):
\begin{figure}[ht]
    \centering
    \begin{subfigure}
        \centering
        \includegraphics[width=0.6\textwidth]{figs/simple_model_all.png}
        \caption{The temperature-entropy curves.}
        \label{fig:first_image}
    \end{subfigure}
    \begin{subfigure}
        \centering
        \includegraphics[width=0.6\textwidth]{figs/simple_model_all_log.png}
        \caption{The temperature-log entropy curves.}
        \label{fig:second_image}
    \end{subfigure}
    \caption{The curves derived from the stochastic process model under different $\alpha$.}
    \label{fig:toy_curves}
\end{figure}
\begin{figure}[ht]
    \vspace{-3mm}
    \centering
    \includegraphics[width=0.6\textwidth]{figs/simple_model_all_correct_rate.png}
    \caption{The Temperature-Improper Token (\%) curves.}
    \label{fig:correct rate}
\end{figure}
\begin{figure}[ht]
\vspace{-3mm}
\centering
\includegraphics[width=0.6\textwidth]{figs/entropy.png}
\vspace{-3mm}
\caption{The temperature-entropy curves.}
\label{fig:curves}
\end{figure}
\begin{figure}[h!]
\vspace{-3mm}
\centering
\includegraphics[width=0.6\textwidth]{figs/log_entropy.png}
\vspace{-3mm}
\caption{The temperature-log entropy curves.}
\label{fig:real_curves}
\vspace{-3mm}
\end{figure}
The curves under different noise tolerances have a similar shape. Generally, the curve can be divided into two parts: \textbf{(sub)-linear} increase and \textbf{(super)-exponential} increase. In the first part, the increase is due to the various choices among the proper tokens, while the sharp rise in the second part is due to the loss of control (i.e., the model frequently chooses nonsense tokens).

The curve is very similar to the behavior of real language models, and some reference entropy curves and log-entropy curves are shown in Figure~\ref{fig:curves} and~\ref{fig:real_curves}.

\paragraph{Relation to Improper Token Rate} It is natural to consider proper tokens can lead to correct answers and the improper tokens will result in incorrect answers, so we measure the percentage of improper tokens. As shown in Figure~\ref{fig:correct rate}, when the temperature exceeds the turning point, the percentage of improper tokens increases quickly, implying a quality drop in samples. Interestingly, the difference in noise tolerance rates has little inference on the turning point but controls the improper token increasing speed after the turning point. Nevertheless, the percentage of improper tokens increases quickly under all tested $\alpha$.
%Fig. \ref{fig:correct rate} is the temperature-correct rate curve, and the big drop in the correct rate happens at the same point as the entropy begins to increase.

\subsection{Conclusion}
Our toy model provides a simplified yet insightful framework for understanding how temperature-dependent sampling dynamics may give rise to characteristic shifts in the model’s output distribution. The model predominantly chooses from the proper tokens in the low-temperature (or sub-linear growth) regime, resulting in relatively stable and controlled outputs. The distribution flattens and nonsense tokens gain significant probability mass as temperature increases beyond a certain threshold. This transition leads to a sudden and steep increase in entropy—mirroring observations in actual large language models—and a corresponding drop in the correct rate. Therefore, increasing the temperature can initially increase generation diversity (sampling among proper tokens) with a small correctness drop. However, its performance suffers a quick drop after reaching a certain threshold (i.e., the turning point).

\section{Aggregation Adaptation Calculation}
\label{app: bias}
The choice of aggregation function affects the optimal generation temperature. For example, in majority voting, the final answer must be selected by the majority, whereas in best-of-N, only a single correct instance out of the N samples is required.

In the case of majority voting, the turning point on the entropy curve aligns with the optimal temperature, so we set its adaptation to 0. For best-of-N, we computed an adaptation on MATH and then tested it on MBPP to confirm generality. Specifically, we averaged the difference between the midpoints of the optimal temperature ranges for best-of-N and majority voting across 13 models on MATH. This difference was $0.092$ on average. Therefore, for simplicity, we set the aggregation adaptation for best-of-N to $0.1$.

\begin{table}[ht]
\centering
\caption{\textbf{Aggregation Adaptation for Best-of-N}: we calculate midpoints of optimal temperature ranges on Majority Voting and Best-of-N for MATH. The difference between the average of midpoints is $0.092$, so we set the adaptation factor to $0.1$.}
\begin{tabular}{c|ccccccccccccc|c}
\toprule
Aggregation & \multicolumn{13}{c|}{Individual Models (Models are listed in the same order as Table~\ref{table: hit rate})} & Average \\
\hline
Best-of-N & 0.6 & 0.8 & 0.6 & 0.6 & 0.7 & 0.5 & 0.6 & 1.1 & 1.2 & 0.5 & 0.6 & 1.3 & 1.0 & 0.7769\\
Majority Voting & 0.6 & 0.9 & 0.6 & 0.5 & 0.3 & 0.6 & 0.5 & 1.1 & 0.9 & 0.5 & 0.6 & 1.0 & 0.8 & 0.6846\\
\bottomrule
\end{tabular}
\label{tab:averages}
\end{table}

\section{Results of All Tested Models}

We present the accuracy heatmaps and entropy estimations for all tested models. Figure~\ref{fig: heatmap MATH} shows the heatmaps of model accuracy for the MATH dataset, while Figure~\ref{fig: heatmap MBPP} displays the heatmaps for the MBPP dataset. Additionally, Figure~\ref{fig: curve_math} illustrates the entropy curve estimations for the MATH dataset, and Figure~\ref{fig: curve_mbpp} provides the entropy curve estimations for the MBPP dataset.

\label{app: results}
\begin{figure*}[ht]
    \center
\includegraphics[width=0.95\textwidth]{figs/merged_heatmap_MATH.png}
    \caption{The accuracy heatmap for all tested models on the MATH dataset. The green line is our predicted temperature.}
    \label{fig: heatmap MATH}
\end{figure*}
\begin{figure*}[ht]
    \center
    \includegraphics[width=0.95\textwidth]{figs/merged_heatmap_MBPP.png}
    \caption{The accuracy heatmap for all tested models on the MBPP dataset. The green line is our predicted temperature.}
    \label{fig: heatmap MBPP}
\end{figure*}
\begin{figure*}[ht]
    \center
\includegraphics[width=0.95\textwidth]{figs/curve_math.png}
    \caption{The entropy curves and turning points of language models when testing on the MATH dataset.}
    \label{fig: curve_math}
\end{figure*}
\begin{figure*}[ht]
    \center
    \includegraphics[width=0.95\textwidth]{figs/curve_code.png}
    \caption{The entropy curves and turning points of language models when testing on the MBPP dataset.}
    \label{fig: curve_mbpp}
\end{figure*}
%%%%%%%%%%%%%%%%%%%%%%%%%%%%%%%%%%%%%%%%%%%%%%%%%%%%%%%%%%%%%%%%%%%%%%%%%%%%%%%
%%%%%%%%%%%%%%%%%%%%%%%%%%%%%%%%%%%%%%%%%%%%%%%%%%%%%%%%%%%%%%%%%%%%%%%%%%%%%%%

\end{document}
