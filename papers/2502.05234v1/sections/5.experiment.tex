\section{Evaluating \textsc{TURN}}
%\begin{table}[ht]
\centering
\caption{Accuracy of the models when applied \textsc{TURN}, compared with the best temperature from grid search.\weihua{Maybe is a better metric compared with hit rate}}
\label{table: accuracy}
\small
\centering
\begin{tabularx}{0.48\textwidth}{@{} l c c c @{}}
\toprule
\multicolumn{4}{c}{\textbf{MATH Dataset with Majority Voting}} \\
\midrule
\textbf{Model} & \textbf{Predict} & \textbf{Best} & $\Delta$\\
\midrule
mistral-7b-sft & 0.4625 & 0.47 & \\
math-shepherd-mistral-7b-rl & 0.5275 & 0.5275 \\
Mistral-7B-v0.3 & 0.2663 & 0.27 \\
Mistral-7B-Instruct-v0.3 & 0.3325 & 0.3325 \\
deepseek-math-7b-base & 0.57 & 0.575 \\
deepseek-math-7b-instruct & 0.6525 & 0.66 \\
llemma-7b & 0.3913 & 0.3913 \\
llemma-7b-sft-metamath-hf & 0.5075 & 0.5075 \\
Llama-3.1-8B-Instruct & 0.655 & 0.6525 \\
Llama-3.1-8B & 0.4025 & 0.4075 \\
Llama-3.2-3B-Instruct & 0.645 & 0.645 \\
Llama-3.2-3B & 0.1975 & 0.2075 \\
OpenMath2-Llama3.1-8B & 0.755 & 0.7575 \\
\midrule
\multicolumn{4}{c}{\textbf{MBPP Dataset with Best-of-N}} \\
\midrule
\textbf{Model} & \textbf{Predict} & \textbf{Best} & $\Delta$ \\
\midrule
deepseek-coder-7b-base-v1.5 & 0.891 & 0.918 \\
deepseek-coder-7b-instruct-v1.5 & 0.8952 & 0.9129 \\
CodeLlama-7b-hf & 0.8297 & 0.8265 \\
CodeLlama-7b-Python-hf & 0.8471 & 0.8542 \\
CodeLlama-7b-instruct-hf & 0.8154 & 0.8154 \\
Qwen2.5-Coder-7B & 0.9610 & 0.9610 \\
Qwen2.5-Coder-7B-Instruct & 0.9778 & 0.9778 \\
Yi-coder-9B & 0.9579 & 0.9676 \\
Yi-Coder-9B-chat & 0.9501 & 0.9501 \\
Llama-3.1-8B & 0.8486 & 0.8594 \\
Llama-3.1-8B-Instruct & 0.8638 & 0.8677 \\
Mistral-7B-v0.3 & 0.8129 & 0.8109 \\
Mistral-7B-Instruct-v0.3 & 0.7990 & 0.7990 \\
\bottomrule
\end{tabularx}
\end{table}
\begin{table}[ht]
\vspace{-3mm}
\centering
\caption{The prediction from our algorithm \textsc{TURN}, the \emph{optimal temperature ranges} from grid search, and the \emph{performance drop (PD)} for various models tested in the MATH and MBPP datasets. \textsc{TURN} achieved hit rates of $12/13$ and $11/13$, average temperature gaps of \(0.023\) and \(0.015\), and average performance drop of $0.32\%$ and $0.59\%$.}
\centering
\label{table: hit rate}
\scriptsize
\begin{tabularx}{0.48\textwidth}{@{} l c c c @{}}
\toprule
\multicolumn{4}{c}{\textbf{MATH Dataset with Majority Voting}} \\
\midrule
\textbf{Model} & \textbf{Predict} & \textbf{Optimal Range} & \textbf{PD($\downarrow$)} \\
\midrule
mistral-7b-sft & \textbf{0.9} & 0.5--1.5 & 0.75\% \\
math-shepherd-mistral-7b-rl & \textbf{0.9} & 0.5--1.3 & 0\% \\
Mistral-7B-v0.3 & \textbf{0.7} & 0.3--0.7 & 0.37\% \\
Mistral-7B-Instruct-v0.3 & \textbf{0.7} & 0.5--0.7 & 0\% \\
deepseek-math-7b-base & \textbf{0.6} & 0.5--0.7 & 0.5\%\\
deepseek-math-7b-instruct & \textbf{0.8} & 0.5--1.3 & 0.75\% \\
llemma-7b & \textbf{0.7} & 0.3--0.7 & 0\%\\
llemma-7b-sft-metamath-hf & \textbf{1.1} & 0.7--1.5 & 0\%\\
Llama-3.1-8B-Instruct & \textbf{0.6} & 0.3--0.7 & 0\%\\
Llama-3.1-8B & \textbf{0.6} & 0.5--0.7 & 0.5\%\\
Llama-3.2-3B-Instruct & \textbf{0.7} & 0.5--0.7 & 0\%\\
Llama-3.2-3B & 0.6 & 0.3 & 1\%\\
OpenMath2-Llama3.1-8B & \textbf{0.8} & 0.5--1.1 & 0.25\%\\
\midrule
\multicolumn{4}{c}{\textbf{MBPP Dataset with Best-of-N}} \\
\midrule
\textbf{Model} & \textbf{Predict} & \textbf{Optimal Range} & \textbf{PD($\downarrow$)}\\
\midrule
deepseek-coder-7b-base-v1.5 & 1.0 & 0.7--0.9 & 2.70\%\\
deepseek-coder-7b-instruct-v1.5 & 1.0 & 1.1--1.3 & 1.77\% \\
CodeLlama-7b-hf & \textbf{0.8} & 0.7--0.9 & 0\%\\
CodeLlama-7b-Python-hf & \textbf{0.9} & 0.7--0.9 & 0.71\%\\
CodeLlama-7b-instruct-hf & \textbf{0.9} & 0.9--1.1 &0\%\\
Qwen2.5-Coder-7B & \textbf{0.9} & 0.7--1.1 &0\%\\
Qwen2.5-Coder-7B-Instruct & \textbf{0.9} & 0.7--1.10 & 0\%\\
Yi-coder-9B & \textbf{0.7} & 0.7--1.3 & 0.97\%\\
Yi-Coder-9B-chat & \textbf{0.9} & 0.9--1.5 & 0\%\\
Llama-3.1-8B & \textbf{0.9} & 0.5--0.9 & 1.08\%\\
Llama-3.1-8B-Instruct & \textbf{0.8} & 0.7--1.1 & 0.39\%\\
Mistral-7B-v0.3 & \textbf{0.8} & 0.5--0.9 & 0\%\\
Mistral-7B-Instruct-v0.3 & \textbf{0.7} & 0.7--0.9 & 0\%\\
\bottomrule
\end{tabularx}
\end{table}
\begin{table*}[t]
\vspace{-3mm}
\centering
\caption{\textbf{Comparison Between TURN and Fixed Temperatures.} We compared TURN to various fixed temperatures under two metrics: The sum of \emph{Temperature Gap} and the average \emph{Performance Drop}. `-Ada.' means removing the aggregation adaptation factor $\beta$. Although some temperatures are generally suitable for multi-sample aggregation strategies (i.e., $T=0.7$ or $T=0.9$), \textsc{TURN} can outperform any single fixed temperature across any dataset, highlighting the strong performance of TURN in automatic temperature selection. The underline means not inferior to the best fixed temperature, and the bold is the best result.}
\label{tab:temperature_comparison}
\begin{tabular}{l|l|cccccc|c|c}
\toprule
& & \multicolumn{6}{c|}{\textbf{Fixed Temperature}} & \textbf{TURN} & \textbf{TURN}  \\
& & \textbf{0.1} & \textbf{0.3} & \textbf{0.5} & \textbf{0.7} & \textbf{0.9} & \textbf{1.1} &  & -Ada.  \\
\hline
\multirow{2}{*}{\textbf{MATH}} 
  & Sum(Gap) ($\downarrow$) & 4.6 & 2.0 & \underline{0.4} & \underline{0.4} & 2.0 & 3.6 & \underline{\textbf{0.3}} & \underline{\textbf{0.3}} \\
  & Mean(Drop) ($\downarrow$) & 8.6\% & 3.5\% & 1.0\% & \underline{0.8\%} & 3.1\% & 7.2\% & \underline{\textbf{0.3\%}} & \underline{\textbf{0.3\%}}  \\
\hline
\multirow{2}{*}{\textbf{MBPP}}
  & Sum(Gap) ($\downarrow$) & 8.2 & 5.6 & 3.0 & 0.8 & \textbf{\underline{0.2}} & 1.2 & \underline{\textbf{0.2}} & 0.6  \\
  & Mean(Drop) ($\downarrow$) & 22.5\% & 10.7\% & 5.1\% & 1.5\% & \underline{0.9\%} & 4.2\% & \underline{\textbf{0.5\%}} & 1.1\%  \\
\hline
\multirow{2}{*}{\textbf{Average}} 
  & Sum(Gap) ($\downarrow$) & 6.4 & 3.8 & 1.7 & \underline{0.6} & 1.1 & 2.4 & \underline{\textbf{0.25}} & \underline{0.45} \\
  & Mean(Drop) ($\downarrow$) & 15.55\% & 7.1\% & 3.05\% & \underline{1.15\%} & 2.0\% & 5.7\% & \underline{\textbf{0.4\%}} & \underline{0.7\%}\\
\bottomrule
\end{tabular}
\vspace{-4mm}
\end{table*}

We want to answer the following research questions about our approach \textsc{TURN} for selecting the optimal temperature:
\begin{itemize}
    \item \textbf{RQ1}: How is the accuracy of TURN in automatic temperature prediction?
    \item \textbf{RQ2}: How efficient is TURN regarding the number of samples (the parameter $N$ in Algo.~\ref{alg:auto find})?
\end{itemize}
Through experiments, \textsc{TURN} proves effective across models, aggregation strategies, and tasks while remaining efficient, requiring only a few samples for temperature prediction.
\subsection{Experiment Setup}
We evaluate our methods in two scenarios where sampling-based inference is widely used: \emph{Math Problem Solving with Majority Voting} and \emph{Code Generation with Best-of-N}. The datasets and models are as follows:

\textbf{Math Problem Solving:}\ We assess language models’ reasoning abilities using the MATH dataset~\cite{hendrycks2021measuring}, which consists of competition-level math problems. To accommodate multiple models, we randomly select 200 test problems (40 per difficulty level). Accuracy is measured based on majority voting. We test general-purpose models (Llama~\cite{dubey2024llama}, Mistral~\cite{jiang2023mistral}), domain-specific models (Llemma~\cite{azerbayev2023llemma}, OpenMath2~\cite{toshniwal2024openmathinstruct}, Deepseek-Math~\cite{shao2024deepseekmath}), and fine-tuned models (Math-Shepherd~\cite{wang2024math}, Easy-to-Hard~\cite{sun2024easy}).

\textbf{Code Generation:}\ For code generation, we use the MBPP dataset~\cite{austin2021program}, selecting the first 100 programming problems. Accuracy is measured using pass@K, where correctness is determined by passing provided unit tests. We regard the unit tests as the best-of-N strategy with a perfect reward model to rank answers. Besides general-purpose models, we evaluate code-specific models, including Deepseek-Coder~\cite{guo2024deepseek}, CodeLlama~\cite{roziere2023code}, Qwen2.5-Coder~\cite{hui2024qwen2}, and Yi-coder~\cite{yicoder}.


\textbf{Implement Details:}\ For both tasks, we sample 256 times per question at each temperature level and compute accuracy across different sampling sizes. For temperature prediction in \textsc{TURN}, we use an interval of \( t=0.1 \) and set \( N = 8 \times \text{dataset size} \) (an excessive sample size, see Section~\ref{sec: sample efficiency} for discussion). Additional inference configurations are detailed in Appendix~\ref{app:inference_config}.

\begin{table}[ht]
\centering
\caption{\textbf{Variance of Entropy Estimation.} We report the average variance of the entropy curve and the variance of estimated temperature \(T_{\text{pred}}\) under different sample sizes with $50$ trails on Llama3.1-8B-Instruct on MATH. A small sample size (e.g., $40$) is sufficient for entropy estimation in \textsc{TURN} for its low prediction variance and small performance drop.}
\label{tab: sample variance}
\small
\centering
\begin{tabularx}{0.48\textwidth}{c|XXXX}
\toprule
                      & \multicolumn{4}{c}{\textbf{Sample Size (\(N\))}} \\
\multicolumn{1}{r|}{} & 10      & 40     & 100  & 800  \\ \hline
\(\text{Mean}\left(\text{Var}\left({\mathcal{H(.)}}\right)\right)\)             &    0.084    &    0.022    &    0.010  & 0.001 \\
\(\text{Var}\left(T_{\text{pred}}\right)\) & 0.020 & 0.005 & 0.003 & 0.001 \\
\hline
Performance Drop($\downarrow$) & 0.9\% & 0.2\% & 0.1\% & 0.0\% \\
\bottomrule
\end{tabularx}
\vspace{-4mm}
\end{table}
\subsection{Evaluation Metrics}

To assess the performance of our algorithm for automatically selecting the optimal sampling temperature, we define the following key metrics (all the metrics are calculated under a large sample size of 128, refer to Section~\ref{sec: temperature varies} for discussion):

\textbf{Metrics:} We use the following metrics to evaluate the accuracy and reliability of our temperature prediction algorithm:
\begin{itemize}
\item{\textbf{Hit Rate (HR):}} The frequency with which \textsc{TURN} selects a temperature within the \emph{\(\epsilon\)-optimal range}\footnote{Defined in Section~\ref{sec: temperature varies}.}, indicating practical reliability.  
\item{\textbf{Temperature Gap (TG):}} The absolute difference between the predicted temperature and the nearest boundary of the \emph{\(\epsilon\)-optimal temperature range}.
\item{\textbf{Performance Drop (PD):}} The accuracy loss compared to the best temperature found via grid search.
\end{itemize}

\subsection{Baseline}
As no existing method automatically adjusts temperatures in multi-sample aggregation strategies, we compare against a \textbf{fixed temperature} baseline. We search over \(\{0.1, 0.3, 0.5, 0.7, 0.9, 1.1\}\) and select the temperature that maximizes overall accuracy. This mimics a common yet suboptimal practice where developers apply a single temperature across all models, disregarding variations in model behavior and task requirements.  

\subsection{Results}  
We evaluated 13 models on two tasks—MATH (with majority voting) and MBPP (with Best-of-N)—and present the results in Table~\ref{table: hit rate}. Recall Figure~\ref{fig: temp_in_intro}(b), the \emph{correlation coefficient} between the accuracy of the predicted temperature and the best accuracy from grid search is $0.9998$ for MATH (and $0.9913$ for MBPP). \textsc{TURN} achieves a Hit Rate of \(12/13\) on MATH and \(11/13\) on MBPP, indicating strong performance across most models. The Temperature Gap remains minimal even when the predicted temperature falls outside the $\epsilon$-optimal range (0.023 for MATH and 0.015 for MBPP). Compared to the best temperatures found via grid search, \textsc{TURN} incurs only a small average performance drop \((0.32\%\) and \(0.59\%\), respectively). Full per-model results and predicted turning points are provided in Appendix~\ref{app: results}.

\textbf{Comparison with Fixed Temperatures:} 
We next compare \textsc{TURN} to a fixed temperature baseline. Specifically, we sample temperatures from 0.1 to 1.1 at intervals of 0.2 and report the \emph{Temperature Gap (TG)} and \emph{Performance Drop (PD)} in Table~\ref{tab:temperature_comparison}. Our method outperforms the best of fixed temperatures by 0.5\% on MATH and 0.4\% on MBPP in average accuracy. When both tasks are combined, the margin increases to 0.75\%, highlighting the benefit of adaptive temperature selection over a uniform fixed temperature.

\textbf{Number of Samples for Temperature Estimation:}
\label{sec: sample efficiency}
Finally, we assess the efficiency of \textsc{TURN} by examining the prediction variance under different sample sizes for Llama-3.1-8B-Instruct on MATH. As shown in Table~\ref{tab: sample variance}, we report the average variance of the entropy curve across all choices of \(T\), the variance of predicted temperature, and the average performance drop. We find out that even with a moderate sample size (e.g. $40$ samples), the variance remains low and the performance drop is tiny (0.2\%), suggesting that a small sampling budget is sufficient for accurate temperature estimation and thus proves the efficiency of our algorithm.

%\weihua{Why sample size = 128, calculating the optimal temperature range overlap}

%\paragraph{Instruction Following}: Instruction following tasks measure how well language models can follow and execute given instructions. We use the InstructGPT dataset~\cite{ouyang2022training}, which includes a variety of tasks such as summarization, translation, and question-answering. The evaluation metric is the average score given by human annotators based on the correctness and relevance of the generated outputs.