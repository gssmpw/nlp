\section{limitations}
% Despite the impressive performance of SRNLD in certain tasks, its effectiveness remains inconsistent and lacks generalizability. The current approach exhibits significant variability across different LLMs, with some high-performance models failing to achieve the anticipated results. Moreover, the rule-based SRNLD conversion method may constrain the model's flexibility and generalization.

% To address these limitations of SR-LLM, we propose several avenues for future research. Our primary objective is to develop a more robust and adaptive structured language representation that demonstrates consistent performance across a diverse range of NLP tasks. This endeavor may involve synthesizing the strengths of AMR, PST, and FOL to more effectively capture semantic and logical relationships. Additionally, we intend to explore task-specific optimization strategies, acknowledging that different NLP tasks may benefit from distinct types of structured information. We will also investigate more sophisticated SRNLD conversion techniques and experiment with novel model architecture capable of directly precessing structured representations. Lastly, we plan to broaden our evaluation scope to encompass a wider array of language models and datasets, thereby ensuring the method's generalizability and applicability across various domains and model types.

Despite SR-NLD's promising performance in certain tasks, its effectiveness remains inconsistent across different LLMs. The rule-based conversion method may constrain flexibility. Future research should focus on developing a more robust and adaptive structured representation, exploring task-specific optimizations, and investigating advanced conversion techniques and novel model architectures. Expanding evaluation to diverse language models and datasets will be crucial to enhance the method's consistency, flexibility, and applicability in various NLP domains.