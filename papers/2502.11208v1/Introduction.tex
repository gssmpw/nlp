\section{Introduction} \label{Sec: Intro}
Ensuring compliance with the General Data Protection Regulation (GDPR)~\cite{EU2016GDPR} is essential for platforms handling personal data, as it establishes legal requirements for data processing, security, and user rights.
One of these rights is the \textit{right of access by data subjects}, which enlists provisions whereby data subjects (i.e., end users) can request access to their data being processed by platforms (Article 15(3) GDPR).
Platforms have to comply with these requests within a stipulated period of one month and provide the user with a copy of their data.
Usually, platforms provide this data in the form of data download packages (DDPs)~\cite{valkenburg2024time}.
These DDPs often include information about the users’ activities on and off the platform, device and app information, inferred preferences, and other information depending on the business of the platform.

\noindent
\textbf{Implementation across platforms}: In fact, the GDPR stipulates that information shall be provided to users in commonly used electronic forms (Art. 15(3)(2) GDPR).  
Such an implementation is crucial for enabling users to exercise their \textit{right to data portability} (Article 20), which allows end users to transmit data from one platform to the other without hindrance~\cite{EU2016GDPR}.
However, the GDPR does not provide explicit standards for the structure or content of DDPs, leading to varying implementations of Article 15 across platforms~\cite{borem2024data}.

Understandably, platforms that offer different types of services interpret and implement the GDPR provisions differently.
For instance, the DDP of a digital marketplace like Amazon is naturally distinct from that of a digital public space like TikTok. 
However, platforms offering similar services should be able to provide comparable information to users. 
For example, platforms providing short-format video streaming services should, in principle, offer similar details through their DDPs. 
To the best of our knowledge, such a comparative study of the implementation of the right of access under the GDPR is rare in the literature. 
This observation leads to a key unexplored research question as follows: \textit{RQ1: How similarly do platforms offering similar services implement the right of access under the GDPR} (\textit{Article} 15(3))?

\noindent
\textbf{Desideratum 1: Comprehensibility}: Irrespective of the answer to RQ1, these implementations (DDPs) should adhere to a set of requirements outlined in the GDPR to ensure user comprehensibility.
According to Article 12(1), these DDPs should be (a)~concise, (b)~transparent, (c)~intelligible, (d)~easily accessible, and (e)~using clear and plain language~\cite{EU2016GDPR}.
However, neither the GDPR nor accompanying guidelines~\cite{EDPB2018Transparency} provide a clear technical definition of these criteria.
Further, adherence to these requirements in the current implementations also remains unexplored in the research community.
The authors in~\cite{borem2024data} came closest to understanding the question of adherence, but they restricted their analyses to conciseness.
This gap leads us to the second research question in this paper: \textit{RQ2: How comprehensible are the DDPs provided by platforms under Articles} 12 \textit{and} 15 \textit{of the GDPR?}

\noindent
\textbf{Desideratum 2: Reliability}: Finally, another important expectation for the effective implementation of the provision is that the information in the DDPs is reliable.
With the increased restriction on access to platform data for research~\cite{Stokel-Walker2023Twitter, Bellan2024Meta}, many researchers are relying on data elicited through exercising the right of access for understanding the dynamic interplay between population and platforms. 
In these so-called ``data donations'' based studies, consenting participants first request their own data from the platforms, and then provide this data to researchers, typically for financial compensation. 
Several recent scientific studies use this setup to explore fundamental questions about digital platforms— such as user engagement~\cite{zannettou2024analyzing}, content recommendation~\cite{vombatkere2024tiktok}, and ad targeting processes~\cite{wei2020twitter} — and even questions beyond online activities, e.g., offline risks~\cite{alsoubai2024profiling} or the likelihood of addiction~\cite{yang2024coupling} using the resultant DDPs. 
However, if the reliability of the information provided by the platforms is compromised, then the insights drawn from such data sources might be invalid.

To this end, a recently conducted survey among researchers indicates that 86\% of researchers report concerns regarding the reliability of the data collected using DDPs~\cite{valkenburg2024time}.
The most important dimensions of reliability that researchers are concerned about include: (a)~completeness, (b)~correctness, and (c)~consistency of the shared information.
This brings us to our final research question: \textit{ RQ3: How reliable is the information within the DDPs provided by platforms under Article 15}(\textit{3}) \textit{of the GDPR}?

\noindent
\textbf{The current work}:
Although the current work could have been conducted on any digital platform, due to the growing popularity in consumption of short--format videos~\cite{zannettou2024analyzing, Mousavi_Gummadi_Zannettou_2024} and the impact they have on the population~\cite{DSA2023VLOP}, we investigate the aforementioned research questions on Instagram, TikTok, and YouTube.
%The pipeline of the conducted research is summarized in \Cref{Fig: Pipeline}.
~\Cref{Fig: Pipeline} describes the pipeline of our conducted research.
After analyzing the DDPs shared by these platforms, we conducted an extensive survey among 400 participants from Germany, France, Italy, and Spain to evaluate their assessment of the comprehensibility of the respective DDPs. % Ingmar: maybe "to evaluate their assessment of ..." -- kept unchanged for now
Further, by utilizing (a)~participants' and (b)~European Data Protection Board's interpretations of the requirements, (c)~current advancements in large language models (LLMs), and (d)~our technical expertise, we propose data representations for the different categories of information.
These recommended representations are then evaluated against the best representations of respective categories in the current implementation of the platforms by 200 participants.

To answer the questions regarding the correctness and completeness of the DDPs, we browse the three platforms using sock-puppet accounts to (a)~automatically log the browsing behavior and (b)~request the DDPs from each of the platforms. 
We repeat this process for a period of one month to understand the consistency in the shared DDPs for the accounts. 
In addition, to better understand the consistency across various accounts, we also collect DDPs requested by real-world participants from the three platforms.
\begin{figure}[t]
    \centering
    \includegraphics[width=\columnwidth]{Figures/Pipeline-GDPR-Compliance.pdf}
    \vspace{-2 mm}
    \caption{ Our pipeline to evaluate comprehensibility and reliability of the current evaluation of Article 15(3) of the GDPR.}%\sz{if we need space we can drop this figure}}
    \vspace{-3 mm}
    \label{Fig: Pipeline}
\end{figure}
We summarize our major findings with respect to each research question below.

\noindent
\textbf{RQ1: The studied platforms implement Article 15 of the GDPR differently}: Instagram, TikTok, and Youtube share different categories of data in their DDPs. While Instagram and TikTok share more data with their users, YouTube's shared data is rather limited. Irrespective of the amount of data, the only underlying similarity is that none of the platforms share the purpose of data collection and processing in their DDPs.

\noindent
\textbf{RQ2: Instagram's current data representation is the most comprehensible}: Based on the preference votes of 400 participants, we find Instagram's DDP adheres to GDPR's requirements more than the other two platforms.  

\noindent
\textbf{RQ2: Considering interpretation from different stakeholders significantly improves the data representation}: 200 survey participants prefer our proposed data representation to current representations along all requirements except conciseness. In fact, our proposal is preferred in as many as 44 out of the 52 evaluations (13 categories $\times$ 4 requirements).  

\noindent
\textbf{RQ3: TikTok's DDP is the most reliable as per our evaluation}: Our analyses using sock-puppet accounts reveal TikTok's DDP presents a complete, correct, and consistent representation of our browsing behavior.
Instagram and YouTube exhibit varying degrees of missing data across DDPs collected over the period of our study.
However, the analyses among real users show Instagram and TikTok share data for different periods in their DDPs for different categories, whereas YouTube shares all the data types for similar duration. 


To the best of our knowledge, this study is the first detailed evaluation 
of adherence of different requirements mentioned in the GDPR. 
The insights drawn from our study further underline the need for standardization and follow-up on the implementations to ensure that the primary goals of data protection laws are achieved.
