\section{Concluding discussion}
\label{Sec: Discussion}
\textbf{Summary of the insights}: To the best of our knowledge, ours is the first systematic attempt to evaluate the implementation of the \textit{right of access} (Article 15) across platforms offering similar services.
Our detailed evaluation of Instagram, TikTok, and YouTube, focusing on their DDPs, reveals that Instagram and TikTok share more data categories, while YouTube’s shared data is limited.
The only underlying similarity that cuts across platforms is that of non-disclosing the purpose of data collection and processing. 
While Instagram’s data representation is deemed most comprehensible, TikTok’s is found to be the most reliable. 
Finally, a proposed data representation, leveraging interpretations of different important stakeholders, is found to be preferred by participants across multiple countries along the different requirements.\\
\noindent
\textbf{Limitations of the current work}: Like any other work it has its own share of limitations. 
The survey population could have been more diverse and representative of common users on the street. 
However, note that our work is comparative by nature and we do not expect that the choice of participants will impact the relative rankings and comparisons presented across the paper. 
Further, our study solely focuses on data shared by platforms upon request and \textit{not} the data that is actually collected from the users. 
Including this information may further enrich the quality of the work.

\noindent
\textbf{Recommendations for stakeholders}:  
Platforms should improve the consistency among different categories of data that they share with end-users. 
The way different related categories of data are being shared differently (e.g., like vs. browsing history on Instagram) raises concerns about the lack of coordination within the platform while implementing the regulations. 
The other significant room for improvement for platforms is increasing the comprehensibility of the shared data. 
As we show in our work if we consider the interpretation of different stakeholders, the resulting representations can already be a significant improvement on the current representations. \\
The current differential and ineffective implementation not only affects end-users' rights of access but also affects their rights to port their data (Article 20). 
Such rights of users will only be useful if data protection authorities provide more detailed technical specifications and/or standardization for implementation. 
As \Cref{Sec: Recommendation} shows involving different stakeholders may further improve the standard of implementations. 
Moreover, authorities need to enforce these requirements--otherwise platforms may not have any ramifications to worry about if they do not comply with them.
To this end, the current study also indicates how effective enforcement of the GDPR rights may benefit from computer science research by relying on audit strategies %-- including user data donation and/or bot accounts -- 
to observe compliance at scale. 

