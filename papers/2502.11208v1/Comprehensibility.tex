\section{RQ2(a): How comprehensible are DDPs?}
\label{Sec: Comprehensibility}

Here we investigate whether the current implementations of platforms adhere to the requirements
Chapter 3 of the GDPR outlines a number of rights of data subjects (i.e., end users), including the right of access (Article 15)~\cite{EU2016GDPR}.
Before outlining these rights, the GDPR provides a list of requirements for how information should be communicated to end users. 
Specifically, the first paragraph of Article 12 of GDPR reads, 
\textit{``The controller shall take appropriate measures to provide any information referred to in Articles 13 and 14 and any communication under Article 15 to 22 and 34 relating to processing to the data subject in a \underline{concise}, \underline{transparent}, \underline{intelligible} and easily \underline{accessible} form, using clear and plain \underline{language} $\dots$''}~\cite{EU2016GDPR}.\\ 
As the GDPR does not provide more succinct definitions of these requirements, we adopt two approaches to interpret these terms. First, following trends in empirical contract research to employ user surveys for interpretation of legal terms~\cite{BenShaharContracts2017}, we conduct a large-scale user survey among participants from different European countries to understand their interpretations of some of the requirements. Second, we apply the interpretations of the requirements by the European Data Protection Board (EDPB)~\cite{EDPB2018Transparency}. Finally, we evaluate the current implementations, i.e., the DDPs obtained from Instagram, TikTok, and YouTube, against the interpretations by users and the EDPB.\\%%%%%%%%%%%%%%%%%%%%%%%%%%%%%%%%%%%%%%%%%%%%%%%%%%%%%%%%%%%%%%%%%%%%%%%%%%%
%\subsection{Survey setting}\label{Sec: ComprSurveySetting}
\noindent
\textbf{Participant recruitment}:
We recruited participants to conduct our survey from the crowd-sourcing platform Prolific\cite{prolific2025prolific}. % \footnote{\url{https://prolific.co/}}
On Prolific, we hired 100 participants each from Germany, France, Spain, and Italy (400 in total) who had a high approval rate of at least 98\% to take part in our survey.
Our motivation for a general set of respondents stems from the fact that the GDPR entitles general residents in the European Union(EU) with the right to data access, and requirements are intended to benefit these residents. 
Though, eventually, our plan is to extend the study to \emph{all} countries within the EU, we chose to start with the countries that have the highest GDP~\cite{statista_gdp}.
We also selected the standard gender breakdown while setting up the surveys.
In total, we recruited 228 males (57.0\%), 161 females (40.3\%), 9 self-reported as ``other'' (2.3\%), and 2 participants who preferred not to disclose their gender (0.5\%) (See~\Cref{Tab: Demographics} in \Cref{appendix:comprehensible} for details).

Prior to entering our survey, we presented an online consent form for expressing explicit consent from the participants. 
Upon completion of the survey, we compensated each participant at a rate of \pounds 9 per hour, which is recommended by Prolific to be a good and ethical rate of remuneration~\cite{prolific_remuneration}.
Our survey comprised three components: (a)~Awareness of participants about the rights, (b)~Interpretation of the requirements by the participants and (c)~adherence evaluation.
Next, we elaborate abut the survey setup and important observations for each of the components.

%%%%%%%%%%%%%%%%%%%%%%%%%%%%%%%%%%%%%%%%%%%%%%%%%%%%%%%%%%%%%%%%%%%


%%%%%%%%%%%%%%%%%%%%%%%%%%%%%%%%%%%%%%%%%%%%%%%%%%%%%%%%%%%%%%%%%%%
\subsection{Awareness about the GDPR right of access}
\label{appendix: Awareness}
%\ad{Another candidate to get into appendix.}
\textbf{Survey setup}: In the first component, we first tried to educate our participants about Article 15 of the GDPR and how they can request their data from online platforms. 
We also provided a link to a Google Drive folder where they could see an example of a DDP. 
At this stage, we asked our participants three questions regarding (1) whether they were aware of such right before taking our survey, (2) whether they had exercised this right by requesting their data, and (3) if they had exercised their rights, then why.
We designed this part of the survey to understand the general awareness of the participants and make them aware of the said right. 

\noindent
\textbf{Observations}: Out of the 400 participants, nearly $72\%$ participants responded that they were aware of the right of access to data before participating in our survey.
At the same time, only $29.2\%$ of them answered in affirmation when it comes to exercising their rights by requesting the data on some platform. 
Although there is a massive gap between awareness and exercise of the data requests, the numbers are surprisingly high.
We also asked our participants to provide reasons for exercising their data requests in free-form tex.
Through manual annotations the responses were characterized into four groups. 
Almost half of the participants (51\%) mentioned \textit{curiosity}, knowing what information platforms collect about them, to be the primary reason for their data request.
Further, 18\% mentioned \textit{seeking specific information}, e.g., identifying a song they had listened to earlier, cross-checking specific details, or determining the amount of time spent on the platform, etc., as the primary reason for their requests.
Finally, 11\% of our participants made their requests to \textit{keep a backup}.% of their personal data.

Interestingly, one-fifth of the participants (20\%) mentioned that they requested data from several platforms for participating in some research study. 
Note that several recent studies--for understanding social media-- use data donations from users on Prolific (\cite{yang2024coupling}, \cite{vombatkere2024tiktok}) and other crowd-sourcing platforms as their primary data source.
% \sz{put refs?}
This observation likely explains the surprisingly high percentage of awareness and exercise statistics that we reported above.

%%%%%%%%%%%%%%%%%%%%%%%%%%%%%%%%%%%%%%%%%%%%%%%%%%%%%%%%%%%%%%%%%%%

%%%%%%%%%%%%%%%%%%%%%%%%%%%%%%%%%%%%%%%%%%%%%%%%%%%%%%%%%%%%%%%%%%%%
\subsection{Interpretation of the requirements}
\label{Sec: ComprehensibilityDesiderata}

\noindent
\textbf{Survey setup}: To understand the interpretation of requirements by common end users, in the second part of our survey, we showed the participants the paragraph of Article 12 (see above), which mentions the different requirements of DDPs. We asked our participants about their interpretation or expectation of the shared personal data in the DDPs to follow any of these requirements.
Next, we elaborate on the different interpretations of these requirements from the responses of our participants while contextualizing them further with the guidelines of the EDPB~\cite{EDPB2018Transparency}, which represents the data protection authorities of all EU member states.

\noindent
\textbf{Conciseness}
\if 0 
Conciseness in communication refers to communicating the complete information about an idea without any unnecessary details. 
Conciseness is often considered as a virtue in expository writing wherein a sentence should contain no unnecessary words, and a paragraph should contain no unnecessary sentences~\cite{strunk2007elements}.
However, defining conciseness in the context of personal data shared with data subjects is tricky.
\fi 
%\noindent
\textit{Interpretations by the survey participants}: We perform thematic analysis of the responses provided by the 400 participants using BERTopic~\cite{grootendorst2022bertopic}. % to conduct topic modeling~\cite{grootendorst2022bertopic}.
The important themes that resonated across the responses spanning all countries are -- \colorbox{yellow}{prioritize important \& relevant data}, \colorbox{yellow}{summarize important observations}, and present data in a \colorbox{yellow}{structured way} that is \colorbox{yellow}{easier to navigate}. 

However, a fourth theme emerged in the interpretations shared by participants from Italy, France, and Spain, but not by those from Germany.
Specifically, these participants touched upon the concept of \textit{data minimization}.
To this end, some participants mentioned the documents should be shorter, while others mentioned avoiding an overwhelming amount of text. In short, while participants from Germany only prioritize the informativeness of the content, other participants prioritize the length of the content as well.

\noindent
\textit{Interpretations by the EDPB}: In this context, the European Data Protection Board interprets concise presentation to be done efficiently and succinctly to avoid information fatigue~\cite{EDPB2018Transparency}.
While the above interpretation falls within the interpretations mentioned by our survey participants, the EDPB also asserts that the presentation should differentiate privacy-sensitive information from non-privacy-sensitive information.
\if 0 
\noindent
\textit{Our interpretation: }
While we, in principle, agree with all the prior interpretations we posit that conciseness should not be considered as a global property of the personal data to be shared. 
Rather it should be a local property evaluated in a case by case basis depending on the type of the underlying information. 
Note that, one of the popular interpretation among participants also mentions relevance of information which may hint at understanding the nuance in a per case basis.
As such, what is a concise representation of the underlying information would be much easier to establish. 
For example, if the same question is posed for one of the data categories (e.g., Watch history)-- it is much easier to come up with a succinct representation which is concise. 
\todo{Maybe some reference to Data Minimization is needed as well?}
\fi 

\noindent
\textbf{Transparency }
\if 0 
Although the GDPR never succinctly defines transparency, its recitals provide some indications on how to interpret this term~\cite{EU2016GDPR}.
For example, Recital 39 of the GDPR describes transparency to be easily accessible and understandable to a `natural person'.
Further, in Recital 58, the GDPR goes one step further in elaborating the relevance of transparency in contexts where technological complexity can make it difficult for data subjects to understand whether, by whom, and why data about them is being collected.
\fi 
%\noindent
\textit{Interpretation by the survey participants}: Unlike the case of conciseness, there are no major differences in how participants from all countries interpret transparency. 
The interpretation of transparency can be divided into two major aspects. 
First, participants call for unambiguous, full disclosure of all the data that has been collected about them in the way they are collected. 
Second, and most importantly, some participants also expect the DDPs to mention how the data is processed and for what purpose the data is collected. 
To this extent, the response from one of the participants captures all the notions beautifully: ``It (Transparency) means providing \colorbox{yellow}{clear} \colorbox{yellow}{honest} and \colorbox{yellow}{complete} information about how the data is being used. There should be no \colorbox{yellow}{hidden} details or \colorbox{yellow}{surprises} for the individual. People should easily understand \colorbox{yellow}{what} data is being collected \colorbox{yellow}{why} its being collected \colorbox{yellow}{who} will access it and how long it will be kept.''
At this point, we would also like to note that as per the participants' interpretation, conciseness and transparency are two competing requirements, i.e., to satisfy one, the other possibly has to be violated.

\noindent
\textit{Interpretation by the EDPB}: While the EDPB draws most of its interpretations from Recital 39 of the GDPR, it interprets transparency as allowing data subjects to determine the scope and consequences of the concerned data processing~\cite{EDPB2018Transparency}.
To this end, the EDPB calls for data controllers (i.e., platforms) to separately spell out the most important consequences and risks that the data processing may entail.
\if 0 
\noindent
\textit{Our interpretation:}
All online platforms have their data privacy policies where the explicitly mention what are the data that they collect from different users. 
Therefore, another interesting perspective in Transparency is to couple the DDPs of the platforms with their data privacy policies and to check what fraction of the details mentioned in the privacy policies are present in the DDPs as well. 
\fi 

\noindent
\textbf{Intelligibility}
\textit{Interpretation by the survey participants}: Among all the requirements provided by the GDPR, the most subjective one is that of \textit{intelligibility}. However, almost all the participants explicitly interpret it to be \colorbox{yellow}{easy to understand} and link it closely with the \colorbox{yellow}{language being straightforward} and \colorbox{yellow}{without any technical and/or legal jargon}. 
Participants from all countries also mentioned that intelligible data should have a \colorbox{yellow}{clear explanation} of how and why a certain piece of information is collected.
The participants also noted some interesting expectations in this regard. For instance, participants from Germany mentioned that platforms shall share their DDPs in \colorbox{yellow}{one's language of comfort} (native language or other language of choice). Moreover, participants from Italy stated that platforms should share their DDPs in the form of \colorbox{yellow}{elegant visualization}.\\ 
% \iw{What is a "language of comfort"? Native language?}
\noindent
\textit{Interpretation by the EDPB}: The EDPB describes intelligibility to be a representation that should be understood by an average member of the intended audience. In addition, they also mention that it is closely linked to the requirement to use clear and plain language.
\if 0 
\noindent
\textit{Our reservation about the interpretation}:
While in principle one cannot disagree that intelligibility means ``being easy to understand'', we believe that some distinctions should be made from merely presenting the information in simple language.
This narrative stems from the age-old conundrum of whether thought and language are similar concepts or not. 
While common people believe that thought and language are similar concepts, linguists disagree with this belief~\cite{vygotsky2012thought}. 
Let us try to understand this distinction in the context of the ``watch history''.
While TikTok and YouTube provide a URL to the content that was viewed by the user, Instagram does not do such a thing. 
To this end, if someone wants to truly understand their watch history, it may not just be sufficient to show them the authors whose posts they watched in the past couple of weeks.
One cannot understand what specific video from a creator they watched just by looking at the author's details; whereas if the link is provided, at least the user can make an additional post-hoc effort to visit the link and understand what the content was that was consumed earlier. 
However, along with the above information (author and link) if the video title information is shown upfront, even the said post-hoc effort is also not needed, making the representation even more intelligible.
\fi 

\noindent
\textbf{Clear and plain language}
%\noindent
\textit{Interpretation by the survey participants}: Understandably, most of the participants interpret plain and simple language to be words that are \colorbox{yellow}{easily understandable} and \colorbox{yellow}{unambiguous}. Some participants elaborate on the nuance by stating that \colorbox{yellow}{legal and technical terms should be avoided} or \colorbox{yellow}{at least explained in a simpler way}. To this end, one clear observation from our survey responses is that participants do not distinguish between intelligibility and language.\\ 
\noindent
\textit{Interpretation by the EDPB}: Along the lines of the interpretations mentioned above, the EDPB interprets this requirement as being definitive and should not be phrased in ambivalent terms or leave room for different interpretations~\cite{EDPB2018Transparency}.

\noindent
\textbf{Accessibility}
%\noindent
\textit{Interpretation by the survey participants}: Most of our participants interpret accessibility as being \colorbox{yellow}{easily available}, i.e., it is \colorbox{yellow}{easy to find} the place to request the data and to access the data.
However, there are certain interesting and valid interpretations which are worth mentioning. 
Some participants mentioned that the data needs to be available in a format so that it can be \colorbox{yellow}{opened without any sophisticated software or tools}. 
At the same time, the timeliness at which the data is provided upon request is another interesting aspect touched upon by some of our participants. 
Another interesting interpretation was the data should be available to everyone, including people with \colorbox{yellow}{visual impairments} by using accessible design features like screen readers or alternative text making sure the users can access data without any \colorbox{yellow}{barriers}.\\ 
\noindent
\textit{Interpretation by the EDPB}: The EDPB's interpretation of accessibility also follows from the above popular responses from the survey participants, wherein they say that it should be immediately apparent to end-users as to where and how the DDP can be accessed~\cite{EDPB2018Transparency}. \\
\noindent
\textit{Our observations on accessibility}: Despite agreeing with all of the above interpretations, we observe a few differences across the platforms because \textit{accessibility} is the most objective property for evaluation.
In terms of \textit{ease of request}, from the content page of a user's account, one needs to click \textbf{6}, \textbf{6}, and \textbf{10} times on TikTok, YouTube, and Instagram, respectively, to be able to request the data in the human-readable format. Notice that, while on Instagram, a user needs to click more, the platform provides many options to the end-user, such as the time duration that their DDP should contain.
Moreover, based on our anecdotal observations, the turnaround time for TikTok upon request was instantaneous during this research study.
For Instagram and YouTube, it often takes 10 to 15 minutes for the DDP to become available.
However, all the turnaround times are within the GDPR-prescribed maximum duration of \textit{one} month.
Further, one can download one's data on TikTok and Instagram from both their applications and web versions, but YouTube shares the DDP with its end users by sending an e-mail with a URL to the DDP.
All these platforms provide DDPs in a compressed format, which makes it harder for the end-users to open them on a smartphone.
At the same time, an end user needs to have a browser or text reader to open the HTML (YouTube and Instagram) and TXT (TikTok) DDPs.
In summary, while the data request and download are relatively accessible in the current setting, reading the data is still complicated for the average end user.
At this point, we would also like to note that we are not aware of any special provisions that platforms may have for people with visual impairments to access the data.
\if 0 
\noindent
\textit{Our interpretation: }Again on this property, our interpretations are well covered by the interpretations mentioned by the above stakeholders. 
\fi 
%%%%%%%%%%%%%%%%%%%%%%%%%%%%%%%%%%%%%%%%%%%%%%%%%%%%%%%%%%%%%%%%%%%%%%%%%%%%%



%%%%%%%%%%%%%%%%%%%%%%%%%%%%%%%%%%%%%%%%%%%%%%%%%%%%%%%%%%%%%%%%%%%%%%%%%%%%%
\subsection{Adherence evaluation}
\label{Sec: AdherenceEval}

\textbf{Survey setup}: In the final part of the survey, we presented our participants with the information shared in the DDPs of Instagram, TikTok, and YouTube in an anonymized and randomized format.
We showed different categories of content and the whole DDP to our participants.
The different categories of content that we shared include data from all of the categories of information that we elaborated in~\Cref{Sec: CurrentImp}.
~\Cref{Tab: FieldsEval} (refer \Cref{appendix:comparitive_study}) summarizes the specific thirteen categories that we evaluated.\footnote{We should point out that YouTube does not provide like history, autofill, location, and off-platform activity. Hence, for these categories, we asked our participants to evaluate between Instagram and TikTok.}
Note that, for this comparative study, we used DDPs that are shared in a human-readable format by all three platforms.
Instagram and Tiktok share the human-readable version in HTML and TXT formats, respectively. 
However, YouTube DDP only contains watch history, search history, personal information, and login history in HTML format. 
All other categories of data are found in CSV or JSON formats.
Since the primary goal of this study is to compare the contents, we showed them consistently in a $<category, value>$ pair in our survey. 
We acknowledge that the way data is shown to users may affect some of the results. %\stefan{aspects? or better results?}
However, in this study, we restrict ourselves to the evaluation of what content is shared rather than how it is shown.
We plan to explore the visualization aspects in future studies.\\
For each category, we showed participants the content shared by the three platforms side by side and asked them which platform's data representation they found to be more concise/intelligible/transparent/in clear and plain language.\footnote{We exclude \textit{accessibility} from the evaluation because it pertains to getting access to request and download DDPs more than to their representations.
Also, we stated our observations about this requirement in~\Cref{Sec: ComprehensibilityDesiderata}.}
~\Cref{Fig: comparitive_study} (see \Cref{appendix:comparitive_study}) depicts a representative example of one of the questions that we used in our survey.
The first preference in~\Cref{Fig: comparitive_study} means the most preferred choice of data representation of a participant, whereas the third preference means their least preferred choice of data representation.
We posed this question as a multiple-choice grid, so each participant had to provide an ordered preference for the three representations. % Ingmar: this was "could provide". I changed this to "had to provide" as we're forcing a ranked list
For example, if a participant rates Instagram's representation as their first preference, then they cannot rate any of the other platforms as their first preference.\\
Note that the final part of the survey consisted of 14 (13 categories + overall DDP) $\times$4 evaluations where participants gave us an ordered preference. 
% \sz{explain where this 14 comes from? above you mentioned 13 categories. i am a bit confused}
Hence, to avoid participant fatigue, we asked 50\% of our participants to evaluate intelligibility and transparency and asked the other half to evaluate conciseness and language properties.
Hence, 200 participants answered each of the evaluation questions.
\paragraph{Observations}

Next, we report our findings on the adherence evaluation of four requirements.\\
\noindent
\textbf{Language}: \Cref{Fig: ComprFirstPref} shows the percentage of first preference votes elicited by the representation of each of the platforms for the four requirements that we surveyed.
In the evaluation of clarity and simplicity of language, Instagram won the first preference votes in eleven out of the thirteen data categories (\Cref{fig:language_first_preference}). 
In contrast, for the remaining two categories, participants preferred the language in which YouTube presents its search and watch histories.\\ 
To understand the overall preference of participants across the board, we weight every first preference vote with a score of 3, second preference vote with a score of 2, and third preference vote with a score of 1.
For categories, where we are comparing only two platforms, we assigned scores 3, 1 for first, second preference respectively.
Then, we evaluate an \textit{average preference score} for each of the platforms based on participants' preference orderings.
\Cref{Fig: ComprHeat} shows the average preference scores of each of the platforms in each category of data. Based on the average preference scores, apart from watch, search histories and location, the gap between preference toward the language of Instagram's data representation is way more than that of TikTok and YouTube (\Cref{fig:language_heatmap}).
Across the different categories of data, the participants' average preference score for Instagram's language is 2.48, whereas TikTok and YouTube fail to reach an average preference score of 2.\\
Apart from the thirteen categories of data, we also asked our participants to evaluate the overall DDP from the three platforms. When participants were asked to evaluate the overall DDP, 55\% of them preferred the language of Instagram's DDP over the other two. TikTok and YouTube DDPs are preferred by 25\% and 20\% of the participants, respectively.\\
\noindent
\textbf{Intelligibility}: Similar to Language, Instagram won the first preference vote for eleven out of the thirteen categories evaluated (\Cref{fig:intelligible_first_preference}).
Among these, except for devices and location categories, nearly more than 75\% of the participants preferred Instagram's data representation to be the most intelligible.\\
The average preference scores almost follow the same trend as that of language, where the Instagram DDP's intelligibility (2.53) is substantially higher than that of YouTube (1.75) and TikTok's (1.64) (\Cref{fig:intelligible_heatmap}). For the evaluation of the overall DDP, 63\% of the participants identified the Instagram DDP to be the most intelligible.
TikTok and YouTube were chosen as the first preference by a considerably smaller proportion of our participants, nearly 18\% each.\\
\noindent
\textbf{Transparency}:
YouTube won the first preference for five categories, where over 70\% of the participants chose it for watch history, search history, and login, followed by devices and content with 55\% and 46\%, respectively (\Cref{fig:tranparency_first_preference}).
For the remaining eight out of the thirteen categories, Instagram won the first preference of the participants in a large percentages.\\
We find that the average preference scores of Instagram and YouTube are comparable (Instagram: 2.23, YouTube: 2.15), whereas TikTok (1.66) has lower score (\Cref{fig:tranparency_heatmap}).
In the evaluation of the overall DDP, Instagram is preferred more compared to the other two platforms, with nearly 46\% of the participants choosing it to be the most transparent.
On the other hand, YouTube is preferred by 35\% of the participants, and the remaining 19\% selected TikTok.\\
\noindent
\textbf{Conciseness}: Based on~\Cref{fig:concise_first_preference}, we found that in nine out of thirteen categories of data that we surveyed, respondents preferred TikTok's data representation as the most concise, whereas Instagram's data representation is found to be the most preferred for the remaining four categories. While TikTok's representation is preferred substantially in the location, autofill, and off-platform categories ($\ge$60\% first preference votes), Instagram's representation is preferred more for likes history and contents ($\ge$65\% first preference votes).
Based on the average preference scores for conciseness, we see that the representations are found to be very comparable in most of the data categories (\Cref{fig:concise_heatmap}).
The average preference scores of Instagram and TikTok are very comparable (Instagram: 2.12, TikTok: 2.15), whereas that of YouTube is considerably lower (1.60). While evaluating the overall DDP, 43\% of the participants found Instagram's DDP to be more concise, followed by Tiktok and YouTube with 35\% and 22\%, respectively.\\
\noindent
\textbf{Variations across countries}: We do not observe qualitative variation across the four countries for three requirements -- language, intelligibility, and transparency. 
However, for conciseness, participants from Germany predominantly preferred Instagram, with over 50\% of the participants selecting it as their first choice across most categories. 
In contrast, participants from other countries primarily selected TikTok as having the most concise representation for all the categories, except likes, connections, and content.
To understand this, we manually examined how participants across different demographics interpret conciseness. 
We found that, except for participants from Germany, most demographics have an interpretation of conciseness as: (1) minimizing document length, (2) avoiding information overload etc. 
This difference in interpretation is reflected in how our participants evaluated the data representations: while participants from Germany gave more priority to content, other participants gave more priority to data categories having less content.\\
%\noindent
\begin{figure} % Use figure* to span both columns
    \centering
    % First row
    
    \hfill
    \begin{subfigure}[t]{0.49\columnwidth}
        \centering
        \includegraphics[width=\textwidth, height=3.5cm]{Figures/stacked_Language_1.pdf}
        \caption{Clear and Plain Language}
        \label{fig:language_first_preference}
    \end{subfigure}
    % Second row
    \begin{subfigure}[t]{0.49\columnwidth}
        \centering
        \includegraphics[width=\textwidth, height=3.5cm]{Figures/stacked_intelligible.pdf}
        \caption{Intelligible}
        \label{fig:intelligible_first_preference}
    \end{subfigure}
    \begin{subfigure}[t]{0.49\columnwidth}
        \centering
        \includegraphics[width=\textwidth, height=3.5cm]{Figures/stacked_transparent.pdf}
        \caption{Transparency}
        \label{fig:tranparency_first_preference}
    \end{subfigure}
    \hfill
    \begin{subfigure}[t]{0.49\columnwidth}
        \centering
        \includegraphics[width=\textwidth, height=3.5cm]{Figures/stacked_Conciseness.pdf}
        \caption{Conciseness}
        \label{fig:concise_first_preference}
    \end{subfigure}

    \caption{Percentage of first preferences across all categories and for the entire DDP. Instagram is the top choice for most categories across all requirements, except for conciseness. For watch and search histories, YouTube is the most preferred platform in transparency, intelligibility, and language.}

    \label{Fig: ComprFirstPref}
\end{figure}

\begin{figure*}% Use figure* to span both columns
    \centering
    % First row
    \begin{subfigure}[t]{0.24\textwidth}
        \centering
        \includegraphics[width=\textwidth, height=5cm]{Figures/heatmap_language.pdf}
        \caption{Clear and Plain Language}
        \label{fig:language_heatmap}
    \end{subfigure}
    %\vspace{0.5cm} % Space between rows
    % Second row
    \begin{subfigure}[t]{0.24\textwidth}
        \centering
        \includegraphics[width=\textwidth, height=5cm]{Figures/heatmap_intelligible.pdf}
        \caption{Intelligible}
        \label{fig:intelligible_heatmap}
    \end{subfigure}
    \begin{subfigure}[t]{0.25\textwidth}
        \centering
        \includegraphics[width=\textwidth, height=5cm]{Figures/heatmap_transparent.pdf}
        \caption{Transparency}
        \label{fig:tranparency_heatmap}
    \end{subfigure}
      \hfill
    \begin{subfigure}[t]{0.25\textwidth}
        \centering
        \includegraphics[width=\textwidth,height=5cm]{Figures/heatmap_conciseness.pdf}
        \caption{Conciseness}
        \label{fig:concise_heatmap}
    \end{subfigure}
    \caption{Comparison of 13 different data categories, including an evaluation of overall DDPs for the four requirements. Instagram achieves the highest overall DDP score for all the four requirements.} 
    \label{Fig: ComprHeat}
\end{figure*}
\noindent
\textbf{Important takeaways}\\ %Next, we summarize the key takeaways of this section.
%\vspace{0.5 mm}
\noindent
\faHandPointRight~ We found participants' interpretation of the requirements are in line with that of European Data Protection Board.
However, there seems to be conflicting interpretations of conciseness and transparency among both stakeholders.\\
%\vspace{0.5 mm}
\noindent
\faHandPointRight~ We found Instagram's current data representation to be the most comprehensible and participants' responses suggest that it adheres to the GDPR requirements more than the other two platforms. However, participants' preferences vary across different data categories. For instance, participants believe that YouTube's representations for search and watch history adhere to the GDPR requirements more than Instagram's.%%%%%%%%%%%%%%%%%%%%%%%%%%%%%%%%%%%%%%%%%%%%%%%%%%%%%%%%%%%%%%%%%%%%%%%%%%%%%