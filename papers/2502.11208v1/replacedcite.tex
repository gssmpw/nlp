\section{Background and Related Work}
\label{Sec: Related}

To position the current work in the perspective of the literature, we present a brief overview of the following: (a)~GDPR and the rights of access, (b)~leveraging data download packages for platform audits, (c)~audits of implementations of provisions of the GDPR.\\%%%%%%%%%%%%%%%%%%%%%%%%%%%%%%%%%%%%%%%%%%%%%%%%%%%%%%%%%%%%%%%%%%%%%%%%%%%%%%%%%%%%
\noindent
\textbf{GDPR and the right of access}:
The GDPR is a landmark privacy law designed to give individuals greater control over their personal data and to standardize data protection rules across the European Union (EU)____.
While the GDPR has many provisions and rights enlisted, our work focuses on Article 15(3): the \textit{right of access}, which allows users to request a copy of their data collected by online platforms. Most platforms implement this requirement by providing a Data Download Package (DDP) upon request.
For such implementations, GDPR provides a set of desiderata for the DDPs.
Specifically, Article 12(1) requires a DDP to be concise, intelligible, transparent, accessible, and in clear and plain language____.
Moreover, the GDPR gives the right to end-users to be able to port their data from one platform to another under Article 20. 
However, if platforms offering similar services implement these rights in significantly different ways, platforms will essentially deprive end users of their right to portability.\\
%%%%%%%%%%%%%%%%%%%%%%%%%%%%%%%%%%%%%%%%%%%%%%%%%%%%%%%%%%%%%%%%%%%%%%%%%%%%%%%%%%%%
%%%%%%%%%%%%%%%%%%%%%%%%%%%%%%%%%%%%%%%%%%%%%%%%%%%%%%%%%%%%%%%%%%%%%%%%%%%%%%%%%%%%
\noindent
\textbf{Leveraging data download packages}:
The richness of DDPs has enabled new research directions previously deemed infeasible.
Indeed, many recent studies are utilizing them through donations made from participants with their explicit consent.
To this extent, researchers have used DDPs to conduct research on topics related to personal health and safety____, news and politics____, auditing recommendation algorithms____, analyzing user behavior on social media platforms____, and understanding ad targeting types and user perceptions____.\\
%%%%%%%%%%%%%%%%%%%%%%%%%%%%%%%%%%%%%%%%%%%
%%%%%%%%%%%%%%%%%%%%%%%%%%%%%%%%%%%%%%%%%%%%%%%%%%%%%%%%%%%%%%%%%%%%%%%%%%%%%%%%%%%%
\noindent
\textbf{Audits of implementations of GDPR}:
Many studies have investigated (a)~whether consent forms are designed properly by platforms____, (b)~whether platforms are undermining user privacy under the pretense of `legitimate interest'____, (c)~inconsistencies in platform's data withdrawal behavior____.
%
However, little attention has been given to platforms' implementation of the right to access. 
A recent study shows that end users often view DDPs as overwhelming and unclear, leaving key questions unanswered while triggering privacy concerns____.
Moreover, researchers have expressed concerns about the reliability of the DDPs and have faced challenges in using them____.\\
\noindent
\textbf{The current work:} To the best of our knowledge, our work is one of the first studies that tries to unravel the current implementations of Article 15(3) of the GDPR from the lens of comprehensibility and reliability.
Furthermore, our proposed recommendations for the potential implementation of DDPs may not only improve the current implementations, but also help in the implementation of other rights provided to users under the GDPR, such as \textit{the right of data portability}.

%%%%%%%%%%%%%%%%%%%%%%%%%%%%%%%%%%%%%%%%%%%%%%%%%%%%%%%%%%%%%%%%%%%%%%%%%%%%%%%%%%%%