\section{RQ1: How similarly do platforms offering similar services implement GDPR Article 15?}
\label{Sec: CurrentImp}

Before analyzing the comprehensibility and reliability of DDPs, we first look into the content of the DDPs provided by each of the platforms (i.e., we compare how the platforms are currently implementing GDPR Article 15(3)).
\subsection{Platforms under consideration}
\label{Sec: Platforms}

Since the primary focus of the study is to understand how similar platforms implement the same provision in the GDPR, we decided to focus on multiple platforms providing similar services. 
To this end, there are many options for digital streaming platforms (e.g., Netflix, Disney, etc.), digital marketplaces (e.g., Amazon, Zalando, etc.), and social networking platforms (e.g., Facebook, X, LinkedIn, etc.).
In this study, we focus on three platforms that provide \textit{short-format video} streaming services: TikTok, Instagram, and YouTube.
Our motivation to study these platforms stems from the recent increase in popularity of short-format video platforms~\cite{zannettou2024analyzing, Mousavi_Gummadi_Zannettou_2024}.
Furthermore, all three platforms have been designated as \textit{Very Large Online Platforms}  (\textit{VLOPs}) by the EU under the recently enacted Digital Services Act (DSA), indicating the far-reaching impact that these platforms have on the European society~\cite{DSA2023VLOP}. 

TikTok, Instagram, and YouTube implement Article 15(3) of the GDPR, through which they share DDPs with their end-users upon request.
In the current implementations, platforms share data primarily in two formats (a)~machine-readable -- which is usually a single JSON or a collection of JSON files; (b)~human-readable -- where Instagram and YouTube opt for HTML version, whereas TikTok shares it in TXT files segregated across many directories.
If one takes a closer look at the DDPs received from any of the platforms, their contents can be broadly divided into the following categories: (a)~user's usage, (b)~user's content, (c)~personal details, (d)~advertisements, (e)~miscellaneous.  
% Next, we look into the content of the DDPs from Instagram, TikTok, and YouTube in each of the categories and how they are the presented.
Next, we examine the DDPs that Instagram, TikTok, and YouTube provide and observe their content and presentation. Most of the observations discussed below are summarized in Table~\ref{Tab: DDPInformation} (see \Cref{appendix:DDP}).
%%%%%%%%%%%%%%%%%%%%%%%%%%%%%%%%%%%%%%%%%%%%%%%%%%%%%%%%%%%%%%%%%%%%%%%%%%%%%%%%%%%%%%%%%%%


%%%%%%%%%%%%%%%%%%%%%%%%%%%%%%%%%%%%%%%%%%%%%%%%%%%%%%%%%%%%%%%%%%%%%%%%%%%%%%%%%%%%%%%%%%%
\subsection{Information regarding users' usage}
\label{Sec: Usage}

One of the most important facets of the information contained within the DDPs is how a user uses the social media platform. 
Such usage information may include the content that a user watches, searches for, or engages with implicitly or explicitly.

\vspace{1 mm}
\noindent
\textit{Watch history}: Watch history is an ordered list of all the content that a user has watched before requesting the data.
DDPs of all three platforms contain watch history data of the data subject. 
However, as~\Cref{tab:watch_history_comparison} shows the data provided across the three platforms are widely varying. Among these, two points are strikingly different -- (a) while YouTube provides this data for the entire \underline{lifetime} of a user on the platform, Instagram provides data of at most the last \underline{two weeks} and TikTok provides data for around \underline{six months} before the data request (without any clear explanation for the reduced duration); (b) while TikTok and YouTube provide the watch history all at once, Instagram segregates it into three files based on the type of content, i.e., ads, posts, and videos watched by the user.


\begin{table}[t]
\centering
\resizebox{\columnwidth}{!}{
\begin{tabular}{@{}llll@{}}
\toprule
\textbf{Feature}           & \textbf{YouTube}                              & \textbf{Instagram}                                                                          & \textbf{TikTok}    \\ \midrule
\textbf{Duration}          & Entire user's lifetime                   & 2 weeks                                                                                & 6 months      \\
\textbf{Content URL}       & \textbf{\fgcheck}                             & \textbf{\redtimes}                                                                          & \textbf{\fgcheck}  \\
\textbf{Timestamp}         & \textbf{\fgcheck}                             & \textbf{\fgcheck}                                                                           & \textbf{\fgcheck}  \\
\textbf{Content title}     & \textbf{\fgcheck}                             & \textbf{\redtimes}                                                                          & \textbf{\redtimes} \\
\textbf{Author name}       & \textbf{\fgcheck} (with author's URL) & \textbf{\fgcheck}                                                                           & \textbf{\redtimes} \\
\textbf{Ad identification} & \textbf{\fgcheck} (whether ad or not)         &  \textbf{\fgcheck}                                                                          & \textbf{\redtimes} \\
\textbf{File segmentation} & Single file                                   & \begin{tabular}[c]{@{}l@{}}Three files\end{tabular} & Single file        \\ \bottomrule
\end{tabular}
}
\caption{Comparison of watch history data across platforms.}
\label{tab:watch_history_comparison}
\end{table}


%\vspace{1 mm}
\noindent
\textit{Like history}: Similar to watch history, like history is an ordered list of all the contents that a user has liked before requesting the data. The details across the three platforms are noted in~\Cref{table:like_history}. The table demonstrates that there is a clear lack of consistency in how platforms currently implement Article 15(3) GDPR.
While one platform (YouTube) records the like history on its application, it does not share this information in the YouTube DDP~\footnote{Although Like history is not there in YouTube ddp, it is included in the Google DDP.}. 
Another platform (Instagram) provides the watch history for just a couple of weeks without any link to the watched content, but it provides the like history of a lifetime, including references to the liked content. 


\begin{table}[t]
\centering
\small
\begin{tabular}{@{}lcl@{}}
\toprule
\textbf{Platform} & \textbf{\begin{tabular}[c]{@{}l@{}}Provides \\ like \\ history?\end{tabular}} & \textbf{Details provided}                                                                                           \\ \midrule
TikTok            & \textbf{\fgcheck}                                                             & \begin{tabular}[c]{@{}l@{}}Link to the liked content,\\ timestamp of when it was liked\end{tabular}                \\
Instagram         & \textbf{\fgcheck}                                                             & \begin{tabular}[c]{@{}l@{}}Link to the liked content, author details,\\ timestamp of when it was liked\end{tabular} \\
YouTube           & \textbf{\redtimes}                                                             & \begin{tabular}[c]{@{}l@{}}Shows like history on mobile app and web\\version but does not share it in the DDP\end{tabular}   \\ \bottomrule
\end{tabular}
\caption{Comparison of like history details across platforms.}
\label{table:like_history}
\end{table}

%\vspace{1 mm}
\noindent
\textit{Time spent on platform}: While all three platforms have a detailed dashboard that mentions the average time spent and the exact amount of time users spend on the application daily, the same information is not provided in the DDP in response to Article 15 GDPR. 
On a positive note, TikTok recently added a new field in the data, namely ``Activity Summary,'' which enlists how many videos a user has commented on, shared, or watched until the end since their registration.


%\vspace{1 mm}
\noindent
\textit{Other usage activities on platform}: 
~\Cref{Tab: DDPInformation} (in \Cref{appendix:DDP}) shows some other important usage activities.
Most of the activities, e.g., comment, search, save, share, and writing a message are recorded by all platforms where the features are applicable.
However, sharing across applications, i.e., when a user shares the video on other social media platforms or copies the link for posting elsewhere, is only shared in TikTok DDPs.
Also, while both TikTok and Instagram maintain a list of inferred interests of the data subject, YouTube DDPs do not have them. %details in their DDP.
%%%%%%%%%%%%%%%%%%%%%%%%%%%%%%%%%%%%%%%%%%%%%%%%%%%%%%%%%%%%%%%%%%%%%%%%%%%%%%%%%%%%%%%%%%%



%%%%%%%%%%%%%%%%%%%%%%%%%%%%%%%%%%%%%%%%%%%%%%%%%%%%%%%%%%%%%%%%%%%%%%%%%%%%%%%%%%%%%%%%%%%
\subsection{Information regarding user's content}
\label{Sec: Content}

While the usage data is about how users consume or behave on video streaming platforms, information regarding their content refers to the content that a user creates and uploads to the platform for others to consume. This information encompasses the media (image/audio/video) a user uploads on their profile, along with any textual captions, locations, date and time information, etc.
\begin{table}[t]
\centering
\resizebox{\columnwidth}{!}{
\begin{tabular}{@{}lll@{}}
\toprule
\textbf{Aspect}           & \textbf{Details provided}                                                                                                                    & \textbf{Platforms}                                                       \\ \midrule
\textbf{Content created}  & \begin{tabular}[c]{@{}l@{}}Media (image/audio/video), \\ textual captions,\\ date, and time information\end{tabular}              & \begin{tabular}[c]{@{}l@{}}TikTok, \\ Instagram,\\  YouTube\end{tabular} \\ \midrule
\textbf{Addnl. details}   & \begin{tabular}[c]{@{}l@{}}Metadata such as software used\\ for uploading (e.g., Android gallery),\\ device ID, camera metadata\end{tabular} & Instagram                                                                \\ \midrule
\textbf{Location details} & \begin{tabular}[c]{@{}l@{}}Longitude and latitude of upload site\\ if the author tags the location\end{tabular}                              & \begin{tabular}[c]{@{}l@{}}TikTok,\\ Instagram,\\  YouTube\end{tabular}  \\ \bottomrule
\end{tabular}
}
\caption{Details of user created content shared by platforms.}
\label{table:content_details}
\end{table}
While all three platforms provide a copy of the media and text details, Instagram shares additional information including software used to upload the content, device ID, metadata about the camera, etc. (\Cref{table:content_details}).


\if{0}\vspace{1 mm}
\noindent
\textit{An important omission}: While all these data are present in the DDPs currently shared by platforms, how other users on the platforms engage with a data subject's content is not shared with the data subject. \sz{why will the platform provide this though? its not about the user. i have a bit of trouble of getting the point of this paragraph and in particular failing to understand why this is important.}
From the perspective of those other users and considering their privacy concerns, the omission of these engagement data may be a justifiable action.
However, such data might be valuable if users are subjected to hate speech in response to their posts, especially when they go unnoticed by the platforms' hate speech detectors and other content moderation policies.
\stefan{I agree with Savvas. I think this is an interesting point, but it's not clear that the right of access entitles end users to access this information. Who watches my content on Tiktok does not involve my personal data, and so Art. 15(3) does not apply. Perhaps one could mention this as an observation, but without talking about the hate speech issues, as these are not issues regulated by the GDPR.}\fi
%%%%%%%%%%%%%%%%%%%%%%%%%%%%%%%%%%%%%%%%%%%%%%%%%%%%%%%%%%%%%%%%%%%%%%%%%%%%%%%%%%%%%%%%%%%



%%%%%%%%%%%%%%%%%%%%%%%%%%%%%%%%%%%%%%%%%%%%%%%%%%%%%%%%%%%%%%%%%%%%%%%%%%%%%%%%%%%%%%%%%%%
\subsection{Information regarding personal details}
\label{Sec: Personal}

While the two facets discussed above focus more on the content that users consume, engage with, or create, the information regarding personal details is more privacy-sensitive.
These details contain basic personally identifiable information (PII), e.g., account details (which may include one's name, phone number, e-mail address, date of birth, profile picture, etc.). 
However, apart from these basic personal details, most of the DDPs also include information about the different devices through which people visit the platforms.
TikTok and Instagram directly store these details as part of the login information, whereas YouTube does not offer this data in its DDP. 
For YouTube, this data can be accessed only if someone requests their entire Google DDP.\footnote{One justification may be that one can not sign in to their YouTube account unless they sign in to their Google account on the same device.}

%\vspace{1 mm}
\noindent
\textit{Login history}: In light of the prior descriptions, login history contains the list of login activities made by the data subject.
The login history details are significantly different in TikTok and Instagram DDPs.
TikTok provides the timestamp, IP, device model, operating system, network type, and carrier provider in its DDP. 
On the other hand, Instagram's login history data is very detailed. 
Along with providing the above details, Instagram provides (and therefore collects) cookie information, language code, Instagram app version, display properties of the device, hardware identifier, and some internal identifiers.
%
Such differences in data collection and sharing practices point to two important possibilities: (a)~TikTok might be collecting some of these details and not sharing them with its end-users and, therefore, is not being transparent; (b)~Instagram might be collecting more information than necessary, thus becoming more intrusive to privacy.
Diving deep into these possibilities is beyond the scope of this study, and we would like to explore these directions in future work.
%%%%%%%%%%%%%%%%%%%%%%%%%%%%%%%%%%%%%%%%%%%%%%%%%%%%%%%%%%%%%%%%%%%%%%%%%%%%%%%%%%%%%%%%%%



%%%%%%%%%%%%%%%%%%%%%%%%%%%%%%%%%%%%%%%%%%%%%%%%%%%%%%%%%%%%%%%%%%%%%%%%%%%%%%%%%%%%%%%%%%

\subsection{Information regarding advertisements}\label{Sec: Ads}
Advertisement on social media platforms is a prevalent phenomenon, as users are constantly served with ads when they scroll through Instagram, TikTok, or YouTube.
However, based on our observations, there is no standardized way of including ads that users have seen in their requested DDP. 

%\vspace{1 mm}
\noindent
\textit{Ads viewed}: As mentioned in~\Cref{Sec: Usage}, Instagram maintains a separate list of ads viewed by users. 
However, the data provided by Instagram only mentions the author/advertiser name and timestamp.
TikTok, on the other hand, does not demarcate any advertised content that a user has seen in their timeline. 
In contrast, YouTube distinguishes all the ads that a user has encountered from the organic content. 
Also, YouTube provides a link to the ad video along with its title and when it was viewed.
However, neither of the three platforms provides information on ad targeting parameters.
From the ads viewed details, a user can understand these are the ads they got to watch on their timeline, but why those particular ads were served to them remains an enigma.
While this information does appear when an advertisement or paid content appears on the three platforms, it is not included in the shared DDPs.

%\vspace{1 mm}
\noindent
\textit{Ad targeting data sources}: The Instagram DDP contains a list of advertisers who have used data about one's online activity or profile for targeted advertising purposes.
This list also conveys information about how these advertisers obtained users' data.
In specific, as per this information, there are three ways in which the advertiser might target a person: (a)~\textit{custom audience} -- where the advertiser targets the data subject through a custom audience with a list of customer data (e.g., email addresses, phone numbers, or other identifiers) which the advertiser might have obtained from other third party sources; (b)~\textit{remarketing custom audience} -- where the advertiser targets the data subject because (s)he might have interacted with some of the advertiser's content, visited their website, etc.; (c)~ \textit{in-person store visit} -- where advertiser can use location data or check in data because the data subject might have visited one of their physical stores. 
The DDPs of YouTube and TikTok do not provide any such information on advertisers and how they gather data about a particular user.

%\vspace{1 mm}
\noindent
\textit{Off-platform activities}: Another important source of data through which these platforms collect enormous amounts of digital behavioral data is off-platform activities.
This is a list of all the activities in external platforms (outside the source platform) that these platforms have tracked or linked to a user.
Note that this might be an artifact of the said external platforms' installed Meta/Google/TikTok pixels.
A user might have used a Meta or Google login to sign up, other tools embedded in the software development kit of the external platform, or any data-sharing agreement between the external company and any of the platforms we investigate.\\
For instance, if a user visits the website or application that also installs one of the tracking techniques of any of the short-format platforms, then the said website can share the user's activities, such as page view, purchase, search, logins, etc., along with other activities, such as adding a product to the cart, view contents, install app, launch app, etc.
Instagram and TikTok DDPs contain off-platform activities of users, whereas YouTube does not have such data.
Both Instagram and TikTok share the external platform name, event or activity type, and the timestamp in their DDPs.\\
The DDPs from the three platforms also provide a wide variety of other details.
For example, they may contain information on link history, survey, application setting, shopping, etc.
For more details on this, please refer to the summary of all the discussions above, listed in~\Cref{Tab: DDPInformation} (see \Cref{appendix:DDP}).
All three platforms miss one of the important goals set by GDPR Article 15(1)(a), which is to include the \textbf{purposes of the data processing} and collection.
None of the DDPs include a purpose statement explaining why certain data is collected, nor do they provide a README or manual to help users understand the significance of the data being collected.\\%%%%%%%%%%%%%%%%%%%%%%%%%%%%%%%%%%%%%%%%%%%%%%%%%%%%%%%%%%%%%%%%%%%%%%%%%%%%%%%%%%%%%%%%%
\noindent
\textbf{Important takeaways}\\ %Next, we summarize the key takeaways of this section.
%\vspace{0.5 mm}
\noindent
\faHandPointRight~ The three platforms implement Article 15 very differently. The Instagram and TikTok DDPs contain a significantly larger number of data types than the YouTube DDP. Instagram shares (and thus collects) more data.\\
%\vspace{0.5 mm}
\noindent
\faHandPointRight~ There are inconsistencies in the reporting of data within platforms, e.g., while some information is available on apps or websites, the same is not included in the DDPs.\\
%\vspace{0.5 mm}
\noindent
\faHandPointRight~ None of the platforms we studied disclose the purpose of data collection or processing in their DDPs, as required by Article 15(1)(a) GDPR.