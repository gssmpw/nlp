\section{RQ2(b): Improving DDPs comprehensibility}
\label{Sec: Recommendation}

\begin{figure}
    \centering
    \includegraphics[width=0.90\columnwidth, height=3.75cm]{Figures/Recommendation.pdf}
    \caption{ Comparison of watch history representation in our proposal and that of YouTube's current DDP implementation.
    }
    \label{fig:recommendation_representation}
\end{figure}


\begin{figure}[t] % Use figure* to span both columns
    \centering
    % First row
     \begin{subfigure}[t]{0.49\columnwidth}
        \centering
        \includegraphics[width=\textwidth, height=3.5cm]{Figures/proposal_stacked_Language.pdf}
        \caption{Clear and Plain Language}
        \label{fig:language_first_preference_our_prop}
    \end{subfigure}
    \begin{subfigure}[t]{0.49\columnwidth}
        \centering
        \includegraphics[width=\textwidth, height=3.5cm]{Figures/proposal_stacked_Intelligible.pdf}
        \caption{Intelligible}
        \label{fig:intelligible_first_preference_our_prop}
    \end{subfigure}
    \hfill
    % Second row
    \begin{subfigure}[t]{0.49\columnwidth}
        \centering
        \includegraphics[width=\textwidth, height=3.5cm]{Figures/proposal_stacked_transparency.pdf}
        \caption{Transparency}
        \label{fig:tranparency_first_preference_our_prop}
    \end{subfigure}
    \hfill
    \begin{subfigure}[t]{0.49\columnwidth}
        \centering
        \includegraphics[width=\textwidth, height=3.5cm]{Figures/proposal_stacked_concise.pdf}
        \caption{Conciseness}
        \label{fig:concise_first_preference_our_prop}
    \end{subfigure}
    
    \caption{ Percentage of votes for the proposal and the earlier winning platform across different categories for each requirement. Our proposal emerged as the top choice in all fields for the \textit{language} and \textit{intelligible} requirements.}
    \label{Fig: RecoEval}
\end{figure}


In~\Cref{Sec: Comprehensibility}, we observed that none of the platforms adhere to the prescribed requirements of the GDPR properly across all data categories, highlighting the need for better representations.
Motivated by this, here we utilize the different interpretations we elicit from survey participants to come up with novel recommendations that lead to better representations.% of different data categories.

\subsection{Methodology}
\label{Sec: RecMethodology}

To come up with an actionable recommendation for data representation, 
we take the following steps.


\noindent
\textbf{Step 1: Summarize the interpretations}: As a first step, for each of the requirements, we collect the interpretations from our surveyed participants and that of the EDPB~\cite{EDPB2018Transparency}.
Next, we feed this entire set of 401 interpretations to a large language model (LLM) -- Gemini 2.0 Flash \cite{google2024gemini}-- using the Google AI Studio platform. 
We ask the LLM to generate a set of recommendations that encapsulate the most prevalent interpretations by summarizing the above responses. For reproducibility purposes, we make the prompts we used and the responses in ~\Cref{appendix:survey}.
Note that in this approach, we include the interpretations of not only the data subjects, but also of data protection authorities (through the interpretation of EDPB). 

\noindent
\textbf{Step 2: Generate recommended data field for each data category}: In the second step, by providing the top recommendations for each of the requirement, we ask the LLM to make some assumptions.
Specifically, we state that we are data engineers working in a short-format video platform operating in the European Union.
Moreover, an end user requests us to provide them with their personal data for a given data category.
Then, we ask the LLM to let us know of the specific data fields that we should provide to the end user.
Subsequently, the LLM provides us with a set of recommendations for data fields of information in the queried category in the form of a JSON object (available in~\Cref{appendix:survey}). 

\noindent
\textbf{Step 3: Manual inspection by researchers}: In the final stage, three researchers who are part of this study and are conversant with DDPs of a whole array of platforms assess these recommendations.
The primary goal of their assessment is to remove any instance of hallucination in the form of potentially irrelevant data fields that may have been recommended by the LLM. 
Next, given their expertise in understanding data donations in other platforms like Netflix, Prime video etc., we ask them to suggest any additions or removals to the recommended fields.
By taking these steps, we ultimately come up with the final proposal of recommendations.
Note that while Step 1 introduces the end users' and data protection authorities' interpretations, this step also instils the common interpretations of tech experts on other platforms and researchers who want to unravel the dynamics of platforms using these DDPs.

An example of the recommended representation for watch history and that of the most adhering representation among the existing one's (YouTube's) is shown in \Cref{fig:recommendation_representation}.
Note that apart from the details about the video watched, the proposal also shares the duration of the video and what fraction of it was watched by the user and on which device and network.


\subsection{Evaluation of the proposed representation}\label{Sec: EvalProposal}
To evaluate our proposed representation, we conducted another survey among 200 participants (who have not participated in the earlier survey mentioned in \Cref{Sec: Comprehensibility}) from Germany, France, Spain, and Italy. We follow the identical participant recruitment strategy as elaborated in~\Cref{Sec: Comprehensibility}. \Cref{Tab: Recommendation-Demographics} (See \Cref{appendix:improved_ddps}) shows the user demographics.
During the survey, we showed the participants two representations for the evaluated categories: (a)~proposed representation: which we generate from the steps mentioned in~\Cref{Sec: RecMethodology}, (b)~winner representation: for each data category, the representation among Instagram/TikTok/YouTube which won the first preference along more requirements (i.e., the representation which participants suggested to be more adhering to the GDPR).
We anonymize and randomly present each pair of representations to the survey participants.
Then, we ask the participants to select one of the two representations that they find to be more intelligible/transparent/concise/in clear and plain language.

\noindent
\textbf{Observations}
We compare the representations of the data category fields of our proposal and that of the best among the existing short-format video platforms for 13 different categories across four GDPR requirements.\footnote{Notice that the data categories are slightly different than the evaluations done in \Cref{Sec: AdherenceEval}. We are not considering personal information, as there were no suggestions for adding or removing any fields. Instead, we include activity summary, which is provided only by TikTok, for comparison.}
~\Cref{Fig: RecoEval} shows the percentage of votes for the two representations obtained in the 52 evaluations (13 data categories $\times$ 4 requirements), that we surveyed.
We observe that the participants preferred our proposed representation in as many as 44 categories.
However, their preferences vary for different requirements.


\noindent
\textbf{Language and intelligibility}: For clarity and simplicity of language and intelligibility of the representation, our proposal gets upwards of 67\% of the votes in most of the categories (except language in comments and intelligibility in autofill categories) suggesting our proposed representation is both intelligible and in clear and plain language.

\noindent
\textbf{Transparency}: Our proposal is voted to be more transparent than the best of the existing short-format video platforms in 11 out of the 13 categories. 
Apart from search history, in all the other winning 10 categories, our representation gets upwards of 75\% of the votes.
In watch history, the split between our proposal and that of YouTube's DDP is 47\% and 53\%, respectively, indicating that both representations are of comparable quality.
However, in autofill information category, Instagram's data representation gets nearly 93\% of the votes from our survey participants. However, we argue that having less autofill information is indicative of less privacy-intrusive data collection.
While our proposed recommendation includes only the email address, name, address, and phone number, that of Instagram contains much more sensitive information.

\noindent
\textbf{Conciseness}:
% We intentionally discuss the conciseness results in the end. 
Notice that in~\Cref{Sec: ComprehensibilityDesiderata}, we mentioned conciseness and transparency, and to some extent intelligibility, are mutually competing requirements.
We observe the same conflict in the results of this survey.
Unsurprisingly, our proposal was preferred to be the most concise representation in merely 7 out of the 13 categories.
Even then, in watch history, search history, login history, and device details -- our proposed representation was found to be more concise by upwards of 75\% of the participants. 
It is also worth mentioning that, much like the survey presented in~\Cref{Sec: AdherenceEval}, the participants from Germany have different preferences compared to those from other countries. 
Except for autofill information, over 66\% of German participants preferred our proposed representation in all other categories. \\
Despite our proposed recommendation coming out as a significant improvement over any of the existing data representations, achieving this was not the main objective of this section.
Our primary intention of this section has been to convey how, relatively easily, one can come up with more effective data representations by taking into account the interpretations of different stakeholders.
This potentially indicates that currently, platforms are probably operating in silos without much deliberation and sincere effort into implementing Article 15 of the GDPR.
At the same time, because of the mutually conflicting nature of the requirements, it is harder than anticipated to satisfy all the requirements with a single representation.
\\
\noindent
\textbf{Important takeaways}\\ %Next, we summarize the key takeaways of this section.\\
%\vspace{0.5 mm}
\noindent
\faHandPointRight~ Taking into account the interpretations of different stakeholders (e.g., end users, legal scholars, technical experts, and researchers) can offer a significant improvement to the data representations across different categories.\\
%\vspace{0.5 mm}
\noindent
\faHandPointRight~ Our proposed recommendation was favored by the surveyed participants in 44 out of the 52 evaluations conducted across various pairs of data categories and requirements.\\
%\vspace{0.5 mm}
\noindent
\faHandPointRight~ It is not difficult to come up with a proposal that adheres with the requirements of  of Article 12 GDPR. With the right intentions and an intensive dialogue among all the stakeholders, one can build far better content composition of the DDPs.