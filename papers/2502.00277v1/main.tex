%%%%%%%% ICML 2025 EXAMPLE LATEX SUBMISSION FILE %%%%%%%%%%%%%%%%%

\documentclass{article}

% Recommended, but optional, packages for figures and better typesetting:
\usepackage{microtype}
\usepackage{graphicx}
\usepackage{subfigure}
\usepackage{booktabs} % for professional tables
\usepackage{bbm}
% hyperref makes hyperlinks in the resulting PDF.
% If your build breaks (sometimes temporarily if a hyperlink spans a page)
% please comment out the following usepackage line and replace
% \usepackage{icml2025} with \usepackage[nohyperref]{icml2025} above.
\usepackage{hyperref}
\usepackage{multirow}
\usepackage[dvipsnames]{xcolor}
\definecolor{mygreen}{HTML}{3cb44b}
%\usepackage[table]{xcolor}
% Attempt to make hyperref and algorithmic work together better:
\newcommand{\theHalgorithm}{\arabic{algorithm}}

% Use the following line for the initial blind version submitted for review:
%\usepackage{icml2025}

% If accepted, instead use the following line for the camera-ready submission:
\usepackage[accepted]{icml2025}

% For theorems and such
\usepackage{amsmath}
\usepackage{amssymb}
\usepackage{mathtools}
\usepackage{amsthm}

% if you use cleveref..
\usepackage[capitalize,noabbrev]{cleveref}

%%%%%%%%%%%%%%%%%%%%%%%%%%%%%%%%
% THEOREMS
%%%%%%%%%%%%%%%%%%%%%%%%%%%%%%%%
\theoremstyle{plain}
\newtheorem{theorem}{Theorem}[section]
\newtheorem{proposition}[theorem]{Proposition}
\newtheorem{lemma}[theorem]{Lemma}
\newtheorem{corollary}[theorem]{Corollary}
\theoremstyle{definition}
\newtheorem{definition}[theorem]{Definition}
\newtheorem{assumption}[theorem]{Assumption}
\theoremstyle{remark}
\newtheorem{remark}[theorem]{Remark}
\usepackage[many]{tcolorbox}
\usepackage{listings}

\lstset{
  language=Python, % Change this to your desired language
  basicstyle=\ttfamily\small,
  keywordstyle=\color{blue},
  commentstyle=\color{gray},
  stringstyle=\color{red},
  numbers=left,
  numberstyle=\tiny,
  breaklines=true,
}


% Todonotes is useful during development; simply uncomment the next line
%    and comment out the line below the next line to turn off comments
%\usepackage[disable,textsize=tiny]{todonotes}
\usepackage[textsize=tiny]{todonotes}
\newcommand{\yy}[1]{\textcolor{red}{\bf\small [#1 --YY]}}
\newcommand{\sy}[1]{\textcolor{blue}{\bf\small [#1 --SF]}}

% The \icmltitle you define below is probably too long as a header.
% Therefore, a short form for the running title is supplied here:
\icmltitlerunning{Regularized Langevin Dynamics for Combinatorial Optimization}

\begin{document}

\twocolumn[
\icmltitle{Regularized Langevin Dynamics for Combinatorial Optimization }

% It is OKAY to include author information, even for blind
% submissions: the style file will automatically remove it for you
% unless you've provided the [accepted] option to the icml2025
% package.

% List of affiliations: The first argument should be a (short)
% identifier you will use later to specify author affiliations
% Academic affiliations should list Department, University, City, Region, Country
% Industry affiliations should list Company, City, Region, Country

% You can specify symbols, otherwise they are numbered in order.
% Ideally, you should not use this facility. Affiliations will be numbered
% in order of appearance and this is the preferred way.
\icmlsetsymbol{equal}{*}

\begin{icmlauthorlist}
\icmlauthor{Shengyu  Feng}{1}
\icmlauthor{Yiming Yang}{1}
\end{icmlauthorlist}

\icmlaffiliation{1}{Language Technologies Institute, Carnegie Mellon University}
%\icmlaffiliation{comp}{Company Name, Location, Country}
%\icmlaffiliation{sch}{School of ZZZ, Institute of WWW, Location, Country}

\icmlcorrespondingauthor{Shengyu Feng}{shengyuf@cs.cmu.edu}
%\icmlcorrespondingauthor{Firstname2 Lastname2}{first2.last2@www.uk}

% You may provide any keywords that you
% find helpful for describing your paper; these are used to populate
% the "keywords" metadata in the PDF but will not be shown in the document
\icmlkeywords{Machine Learning, ICML}

\vskip 0.3in
]

% this must go after the closing bracket ] following \twocolumn[ ...

% This command actually creates the footnote in the first column
% listing the affiliations and the copyright notice.
% The command takes one argument, which is text to display at the start of the footnote.
% The \icmlEqualContribution command is standard text for equal contribution.
% Remove it (just {}) if you do not need this facility.

\printAffiliationsAndNotice{}  % leave blank if no need to mention equal contribution
%\printAffiliationsAndNotice{\icmlEqualContribution} % otherwise use the standard text.

\begin{abstract}

Hypotheses are central to information acquisition, decision-making, and discovery. However, many real-world hypotheses are abstract, high-level statements that are difficult to validate directly. 
This challenge is further intensified by the rise of hypothesis generation from Large Language Models (LLMs), which are prone to hallucination and produce hypotheses in volumes that make manual validation impractical. Here we propose \mname, an agentic framework for rigorous automated validation of free-form hypotheses. 
Guided by Karl Popper's principle of falsification, \mname validates a hypothesis using LLM agents that design and execute falsification experiments targeting its measurable implications. A novel sequential testing framework ensures strict Type-I error control while actively gathering evidence from diverse observations, whether drawn from existing data or newly conducted procedures.
We demonstrate \mname on six domains including biology, economics, and sociology. \mname delivers robust error control, high power, and scalability. Furthermore, compared to human scientists, \mname achieved comparable performance in validating complex biological hypotheses while reducing time by 10 folds, providing a scalable, rigorous solution for hypothesis validation. \mname is freely available at \url{https://github.com/snap-stanford/POPPER}.




\end{abstract}




\section{Introduction}
Combinatorial Optimization (CO) problems are central challenges in computer science and operations research \citep{papadimitriou1998combinatorial}, with diverse real-world applications such as supply chain management, logistics optimization \citep{chopra2001strategy}, workforce scheduling \citep{ernst2004staff}, financial portfolio management \citep{rubinstein2002markowitz,lobo2007portfolio}, compiler optimization \citep{trofin2021mlgo,zheng2022alpa}, and bioinformatics \citep{gusfield1997algorithms}. Despite their wide-ranging utility, CO problems are inherently difficult due to their non-convex nature and often NP-hard complexity, making them intractable in polynomial time by exact solvers.  Traditional CO algorithms often rely on hand-crafted, domain-specific heuristics, which are costly and difficult to design, posing significant challenges in solving novel or complex CO problems.

Recent advancements in neural network (NN)-based learning \citep{bengio2020machine} and simulated annealing (SA) \citep{kirkpatrick1983SA} algorithms  have redefined approaches to combinatorial optimization by minimizing dependence on manual heuristics:
\begin{itemize}
    \item \textbf{Neural Network Models}: NN-based methods leverage reinforcement learning \citep{Khalil2017DQNCO, qiu2022dimes}, unsupervised learning \citep{Karalias2020ErdosGN, wang2022unsupervised, wang2023unsupervised, SanokowskiHL24} or generative models \citep{kool2018attention, zhang2023let, sun2023difusco, li2023from, li2024fast} to learn optimization strategies directly from data. By automating the process, these models replace handcrafted heuristics with learned representations and decision-making processes, enabling tailored solutions refined through training rather than manual adjustment.
    \item \textbf{Simulated Annealing}: SA is a general-purpose optimization algorithm that explores the solution space probabilistically, avoiding dependence on problem-specific heuristics. Although its cooling schedule and acceptance criteria require some design decisions, SA is highly adaptable across diverse problems free from detailed domain knowledge \citep{Johnson1991opsa}.
\end{itemize}

%Recent examples of NN-based CO solvers highlight the remarkable progress in the field. DIMES \citep{qiu2022dimes} scales reinforcement learning from graphs with up to 100 nodes to 10,000 nodes without sacrificing prediction accuracy.
Discrete Langevin dynamics (LD) \citep{zhang2022langevinlike, sun2022path} and the corresponding diffusion models \citep{chen2023analog, austin2021structured} have greatly advanced the recent development of both NN and SA solvers. The key idea of LD is to guide the iterative sampling via the gradient, for a more efficient searching/generation process. For instance, DIFUSCO \citep{sun2023difusco} adopts continuous diffusion models from computer vision to address the discrete nature of CO problems, outperforming previous end-to-end neural models in both accuracy and computational efficiency. Additionally, DiffUCO \citep{SanokowskiHL24} generalizes DIFUSCO by eliminating the need for labeled training data, using unsupervised learning for CO problems.
Meanwhile, advanced SA-based CO solvers  have demonstrated performance on par with state-of-the-art NN-based approaches.  \citet{sun2023revisiting} underscore the advantages of LD-based SA method, including their simplicity, superior speed-quality trade-offs, and generalizability to new CO problems, as they require no training or problem-specific customization.
However, existing discrete LD/diffusion methods are all adapted from the methods \citep{Welling2011LD, dickstein2015nonequ, song2019generative, Ho2020denoising, song2020improved} in the continuous domain, this raises important questions:  Is there any difference between CO and continuous optimization? Do these adapted methods sufficiently consider the nature of discrete data? Exploring these questions is the central focus of this paper.



Our key observation is that the optimization process is more prone to local optima in a discrete domain than in a continuous one.  That is, local optima in a continuous domain typically has a zero gradient (under the smoothness condition)
but this is often not true in a discrete domain, where the gradients may be very large in magnitude but pointing to an infeasible region. Such a difference makes the escaping of local optima more difficult in a discrete domain than in a continuous one, with the common strategy of  adding a random noise as in LD.  We propose  
to address this issue by enforcing a constant norm of the expected distance between the sampled solution and the current solution during the searching process. In other words, we control the magnitude of the update in LD, encouraging the search to explore more promising areas. We name this sampling method \textit{Regularized Langevin Dynamics (RLD)}. We apply RLD on both SA and NN-based CO solvers, leading to Regularized Langevin Simulated Annealing (RLSA) and Regularized Langevin Neural Network (RLNN).  Our empirical evaluation on three CO problems demonstrate thes significant improvement of RLSA and RLNN over both SA and NN baselines. Notably, RLSA only needs 20\% running time to outperform the previous  SOTA SA baselines. It shows a clear efficiency advantage with either less or more searching steps.

To summarize, we propose a new variant of discrete Langevin dynamics for CO by regularizing the expected update magnitude on the current solution at each step. Our method is featured by its simplicity, effectiveness, and wide applicability to both SA and NN-based solvers, indicating its strong potential in addressing CO problems.








\section{Preliminary}
\subsection{Combinatorial Optimization Problems}
Following \citet{Papadimitriou1982CO}, we formulate the combinatorial optimization (CO) problem as a constrained optimization problem, i.e., 
\begin{equation}
\label{eq:standard}
    \min_{\mathbf{x}\in\{0,1\}^N} a(\mathbf{x}) \quad \text{s.t.} \quad b(\mathbf{x})=0,
\end{equation}
where $a(\mathbf{x})$ stands for the target to optimize and $b(\mathbf{x})\geq 0$ corresponds to the amount of constraint violation ($0$ means no violation). In particular, we focus on the penalty form that can be written as
\begin{equation}
\label{eq:penalty}
    \min_{\mathbf{x}\in\{0,1\}^N} H(\mathbf{x})=a(\mathbf{x})+\beta b(\mathbf{x}),
\end{equation}
where $\beta>0$ is the penalty coefficient that should be sufficiently large, such that the minima of Equation \ref{eq:penalty} corresponds to the feasible solutions in Equation \ref{eq:standard}. \( H(\mathbf{x}) \) is also generally named as the energy function, and its associated energy-based model (EBM) is defined as
\begin{equation}
\label{eq:ebm}
    p_{\tau}(\mathbf{x}) = \frac{\exp(-H(\mathbf{x})/\tau)}{Z},
\end{equation}
where $\tau>0$ is the temperature  controlling the smoothness of the distribution, and \( Z = \sum_{\mathbf{x} \in \{0,1\}^N} \exp(-H(\mathbf{x})/\tau) \) is the normalization factor, typically intractable. When $\tau$ is small, the probability mass of \( p_{\tau} \) tends to concentrate around low-energy samples, making the task of solving Equation \ref{eq:standard} equivalent to sampling from \( p_{\tau}(\mathbf{x}) \). Markov Chain Monte Carlo (MCMC) \citep{Lecun2006ebm} is the most widely used method for sampling from the EBM defined above. However, directly applying MCMC may lead to inefficiencies due to the non-smoothness introduced by the small \( \tau \). To mitigate this issue, the simulated annealing (SA) technique is commonly employed to gradually decrease \( \tau \) towards zero during the MCMC process.

\subsection{Langevin Dynamics}
Langevin dynamics (LD) \citep{Welling2011LD} is an efficient MCMC algorithm initially developed in the continuous domain. It takes a noisy gradient ascent update at each step to gradually increase the log-likelihood of the sample:
\begin{equation}
\label{eq:ld}
    \mathbf{x}' = \mathbf{x} + \frac{\alpha}{2}  s(\mathbf{x}) + \sqrt{\alpha} \zeta, \quad \zeta\in\mathcal{N}(0, \mathbf{I}_{N\times N}),
\end{equation}
where $s(\mathbf{x})=\nabla \log p(\mathbf{x})$ is known as the score function (gradient of the log likelihood), and $\alpha>0$ represents the step size. By iteratively performing the above update, the sample $\mathbf{x}$ would eventually end up at a stationary distribution approximately equal to $p(\mathbf{x})$.

 Recently, \citet{zhang2022langevinlike} have extended LD  to discrete space  by rewriting Equation \ref{eq:ld} as
 \begin{equation}
    \label{eq:ld_exp}
    q(\mathbf{x}'|\mathbf{x})=\frac{\exp(-{\frac{1}{2\alpha}\|\mathbf{x}'-\mathbf{x}-\frac{\alpha}{2}s(\mathbf{x})}\|_2^2)}{Z(\mathbf{x})},
 \end{equation}
 where $Z(\mathbf{x})$ is the normalization factor. For discrete data, the above distribution could be factorized coordinatewisely, i.e., $q(\mathbf{x}'|\mathbf{x}) = \prod_{i=1}^Nq(\mathbf{x}'_i|\mathbf{x})$, into a set of categorical distributions:
\begin{equation}
    q(\mathbf{x}'_i|\mathbf{x})\propto \exp({\frac{1}{2}s(\mathbf{x})_i(\mathbf{x}_i'-\mathbf{x}_i)-\frac{(\mathbf{x}_i'-\mathbf{x}_i)^2}{2\alpha})}.
\end{equation}
When $\mathbf{x}$ is a binary vector, we can obtain the flipping (changing the value of $\mathbf{x}_i$ from $0$ to $1$, or $1$ to $0$)  probability $q(\mathbf{x}'_i=1-\mathbf{x}_i|\mathbf{x})$ as 
\begin{equation}
\label{eq:flip}
\texttt{Sigmoid}(\frac{1}{2}s(\mathbf{x})_i(1-2\mathbf{x}_i)-\frac{1}{2\alpha}).
\end{equation}
In particular, it can be shown that the discrete Langevin sampler is a first-order approximation to the locally-informed proposal \citep{zanella2017informed} in the following form.
\begin{equation}
\label{eq:local}
q(\mathbf{x}'_i|\mathbf{x})\propto \exp({\frac{1}{2}p(\mathbf{x}')-\frac{1}{2}p(\mathbf{x})-\frac{(\mathbf{x}_i'-\mathbf{x}_i)^2}{2\alpha})}.
\end{equation}


\section{Method}
\section{CaseGen}

\begin{figure}[t]
\centering
\includegraphics[width=\linewidth]{example.pdf}
\caption{An task example in CaseGen (translated from Chinese).}
\label{figure:example}
\end{figure}





\begin{figure*}[t]
\centering
\vspace{-5mm}
\includegraphics[width=0.9\linewidth]{pipeline.pdf}
% \vspace{-3mm}
\caption{The overview of CaseGen. CaseGen includes four key generation tasks and uses LLMs-as-a-judge as the primary evaluation method.}
\label{figure:task}
\vspace{-5mm}
\end{figure*}

Developed from high-quality, real-world legal cases, CaseGen comprises 500 instances, each structured into seven distinct sections: \textbf{Prosecution}, \textbf{Defense}, \textbf{Evidence}, \textbf{Events}, \textbf{Trial Fact}, \textbf{Reasoning}, and \textbf{Judgment}.
The Prosecution is a formal document filed by the plaintiff to initiate litigation, detailing the claims and supporting facts.
The Defense is the responds to the Prosecution, in which the defendant challenges the plaintiff's claims and presents their own arguments.
The Evidence includes all expert-annotated case-related evidence details, with each piece corresponding to an event in the trial facts.
The Facts, Reasoning, and Judgment sections form the core components of a legal case document. A more detailed description can be found in Section \ref{sec:pre}. We provide a task sample of CaseGen in Figure \ref{figure:example}.




\subsection{Task Definition}
CaseGen includes four key tasks: (1) drafting defense statements, (2) writing trial facts, (3) composing legal reasoning, and (4) generating judgment results. 
These tasks reflect different stages in the document creation process, each with its own writing logic and evaluation criteria, enabling a more precise and comprehensive assessment of LLMs.


\subsubsection{Drafting Defense Statements}
The task of drafting defense statements involves systematically responding to the claims in the prosecution based on the provided evidence list. An effective defense should be clear and logically organized.  Furthermore, it should directly address each claim while integrating relevant legal knowledge and supporting evidence.


\subsubsection{Writing Trial Facts}
The task of writing trial facts can be defined as verifying the true course of events and identifying the key facts based on the provided evidence list, prosecution, and defense statement. 
Since the factual statements in the prosecution and defense may be incomplete or even contradictory, the court must evaluate the evidence to establish the trial facts.
High-quality trial facts should be presented in a clear narrative structure, with a complete timeline and evidentiary chain. Furthermore, all information should be directly relevant to the legal proceedings, with unnecessary details kept to a minimum.

\subsubsection{Composing Legal Reasoning}
Legal reasoning refers to the process by which judges analyze case facts and apply legal principles to justify their rulings. High-quality legal reasoning should clearly identify all key issues in dispute and present the corresponding judicial perspectives. Since legal reasoning requires balancing multiple legal arguments and precisely applying legal provisions, it is one of the most challenging task of legal case documents generation.


\subsubsection{Generating Judgment Results}
Generating a judgment results involves formulating the final ruling based on established trial facts and legal reasoning. This section typically cites relevant legal articles and specifies the corresponding penalties. A well-crafted judgment must be legally sound, enforceable, and logically reasoned, ensuring judicial integrity and fairness.


Figure \ref{figure:task} illustrates the relationship between different tasks in CaseGen.
To effectively prevent error accumulation, each subtask uses authentic documents as input rather than model-generated content.
For example, the input for writing trial facts is the authentic defense statement, not the model-generated defense from the previous task.
This multi-stage generation approach allows for a more precise evaluation of the strengths and weaknesses of the current LLM in legal document drafting tasks.
Due to space constraints, additional task examples and the prompts used are provided in Appendix \ref{sec:more task}.











\subsection{Data Construction}


\subsubsection{Data Source and Processing}


CaseGen is built on high-quality legal documents. We collected hundreds of thousands of legal case documents from China Judgments Online~\footnote{\url{https://wenshu.court.gov.cn/}} and implemented rigorous data filtering and processing techniques to ensure data integrity and quality.

Specifically, we first filter out cases where the fact section contains fewer than 50 chinese characters or involves simplified procedures, as these cases are considered too simplistic. Additionally, we exclude cases with incomplete structures or formatting errors to maintain data consistency.
During filtering, we found that not all legal documents fully record both the plaintiff's claims and the defendant's defenses. Therefore, we carefully selected 50,000 cases that explicitly include both.
To further enhance the diversity and representativeness of the dataset, we then use BGE-base-zh~\cite{bge_embedding} to generate case embeddings and apply K-Means~\cite{ahmed2020k} clustering to group similar documents. From these clusters, we select 500 representative cases evenly as the core dataset for CaseGen.


Then, we utilized regular expressions and text parsing techniques to extract key structural information from legal case documents. The extracted data is structured in JSON format. For cases that are difficult to parse automatically, we manually extract the various sections and conduct thorough verification.



\subsubsection{Data Annotation}
Although high-quality legal case documents generally contain well-structured information, the full details of evidence are often not publicly disclosed. These case documents usually list the names of the submitted evidence without providing their content. 
To ensure data completeness and usability, we recruit legal experts to annotate the content of the evidence.


The annotation follows three core principles: (1) \textbf{Authenticity.} Since LLMs cannot independently verify the authenticity of evidence, all annotated evidence is authentic, excluding any uncertain or potentially falsified information.
(2) \textbf{Completeness.}  The annotated evidence must accurately align with the content of the legal case document, ensuring that the entire trail fact can be reconstructed from the provided evidence.
(3) \textbf{Textual Representation.} All evidence is presented in textual form. For non-text evidence, such as audio recordings or images, experts provide descriptive textual summaries to ensure clarity and usability.
Additionally, annotation experts need to convert litigation claims and defense arguments from the Procedure section into structured prosecution and defense statements. For more detailed annotation guidelines, please refer to Appendix~\ref{sec:guid}.


Our annotation team comprises five legal experts, all of whom have passed the National Unified Legal Professional Qualification Examination and possess a strong legal background. The team includes two male and three female experts, all based in China. To protect the rights and interests of annotators, we established legally binding agreements with all team members before the annotation work began.
These agreements ensure compliance with legal standards and protect the experts' rights throughout the annotation process.


To ensure data quality, all annotators must complete comprehensive training. We first provide a detailed explanation of the task objectives, data formatting requirements, and applicable legal standards. Subsequently, some example cases are provided to help annotators understand the required format and standards.
We provided several hours of in-depth training to ensure annotators fully understood the annotation standards. Following this, each annotator was required to complete five pilot annotation tasks.  Our gold annotator, who hold a Ph.D. in law, conducted cross-check evaluations to review and verify the accuracy of the pilot annotations. Only annotators with an approval rate of 90\% or higher were permitted to proceed with formal annotations.



For each annotated dataset, we employ a dual verification process using both LLMs and human experts.
We first employ an LLM for automated review to verify evidence completeness. Then, legal experts conduct cross-checks to ensure legal compliance and accuracy. The detailed review process can be found in Appendix ~\ref{sec:guid}.
For each successfully reviewed example, we paid \$10.95 to the legal annotators. A total of 500 examples were annotated, amounting to a total payment of \$5,475. 



\begin{table}[t]
\centering
\begin{tabular}{cc}
\hline
\textbf{Statistic}        & \textbf{\#Number} \\ \hline
Total Legal Case Document & 500               \\
Avg. Full Case Length     & 5,223             \\
Avg. Complaint Length     & 1,187             \\
Avg. Defense Length       & 1,100             \\
Avg. Fact Length          & 1,057             \\
Avg. Reasoning Length     & 1,241             \\
Avg. Judgement Length     & 450               \\
Avg. Evidence per Query   & 7.92              \\
Avg. Evidence Length      & 706               \\ \hline
\end{tabular}
\caption{Basic statistic of CaseGen.}
\label{table:sta}
\vspace{-5mm}
\end{table}



\subsection{Data Statistics}
After careful manual verification, CaseGen consists of four types of tasks, with each task containing 500 test samples. Table \ref{table:sta} presents the basic statistical information.
Compared to general-domain texts, legal case documents are significantly longer. On average, each case contains 7.92 pieces of evidence, with each piece averaging 706 characters in length. Additionally, the generated texts can reach lengths of up to 1,000 characters. This poses a significant challenge for LLMs in handling long-text processing effectively.



\subsection{Evaluation Pipeline}
Evaluating legal case documents is a challenging task. Traditional evaluation metrics, such as BLEU~\cite{papineni2002bleu}, ROUGE~\cite{lin2004rouge}, and BERTScore~\cite{zhang2019bertscore}, fail to capture key aspects like fluency, logical coherence, and factuality. 
While human evaluation is reliable, it is time-consuming and labor-intensive, making it difficult to scale for large-scale assessments.
Therefore, we adopt LLM-as-a-judge as the core evaluation method in CaseGen. Recently, LLMs have gained widespread recognition for their effectiveness as evaluators, achieving a high level of consistency with human annotations~\cite{li2024llms}. Compared to traditional automated evaluation metrics, LLM-as-a-judge enables a more fine-grained, multi-dimensional assessment~\cite{10.1145/3627673.3679677,li2024calibraeval}.


However, evaluating legal case documents poses even greater challenges for LLM judges, requiring not only domain-specific expertise but also strict logical reasoning. Moreover, each section follows distinct evaluation criteria, further complicating the evaluation process.
Following Wang et al.,~\cite{wang2024user}, we developed a multi-dimensional automated evaluation framework for legal case documents generation, ensuring both professionalism and reliability. 
As shown in Figure~\ref{figure:task}, the evaluation framework includes the following four key features:

\textbf{Pointwise Scoring.} 
We employ a pointwise scoring method, which offers greater flexibility compared to pairwise comparisons. Specifically, LLM judges perform a multi-dimensional analysis of the generated documents and assign a final score from 1 to 10, with higher scores indicating better quality.


\textbf{Task-Oriented Criteria.}
Different sections of legal case documents require distinct evaluation criteria.
To address these variations, we establish fine-grained evaluation criteria based on expert-defined standards, covering multiple dimensions such as accuracy, logical consistency, completeness, and legal applicability.
For each task, we provide specific evaluation dimensions with detailed explanations to ensure LLM judges accurately reflects the quality of the generated documents.
Additionally, we establish scoring standards for the LLM judges, with each 2-point increment representing a different rating level. 



\textbf{Chain-of-Thought Reasoning.}
To enhance the reliability of LLM judges, we incorporate Chain of Thought (CoT) reasoning, allowing the LLMs to assess the generated content step by step rather than assigning a score directly. Specifically, the LLM judge first compares the generated output with the reference answer, then assigns scores for each evaluation dimension, and finally consider all dimensions to determine the overall score.

\textbf{Reference-Based Evaluation.}
Evaluating legal case documents requires extensive legal expertise. To address this, we adopt the reference-based evaluation approach, where the ground truth is provided as part of the input to the LLM judges. This allows the LLM to contextually compare the generated text with authoritative references, ensuring a more informed and precise evaluation.


More detailed explanations and examples are provided in Appendix \ref{sec:more evaluation}. We further validate the effectiveness of our evaluation framework through human annotations in Section \ref{sec:human}.


\subsection{Legal and Ethical Considerations}
Due to the sensitivity of the legal domain, we have conducted a thorough review of this benchmark. All the open-source datasets we use are licensed. We have also carefully screened and filtered the datasets to avoid any content containing personal identifiable information, discriminatory material, explicit, violent, or offensive content. A more detailed discussion can be found in Appendix \ref{sec:dis}.

\section{Experiments}

% 13650
\section{Experiment}

\begin{table*}[t]
\small
\begin{tabular}{lcccccccccccc}
\hline
\multirow{2}{*}{\textbf{Model}} & \multicolumn{3}{c}{\textbf{Defense}}                     & \multicolumn{3}{c}{\textbf{Fact}}                       & \multicolumn{3}{c}{\textbf{Reasoning}}                   & \multicolumn{3}{c}{\textbf{Judgement}}                   \\
                       & ROU.           & BS.            & LLM           & ROU.           & BS.            & LLM           & ROU.           & BS.            & LLM           & ROU.           & BS.            & LLM           \\ \hline
LexiLaw                & 6.18           & 62.38          & 1.17          & 8.16           & 59.70          & 1.18          & 8.27           & 67.65          & 2.36          & 13.20          & 66.17          & 2.22          \\
ChatLaw                & 6.44           & 64.03          & 2.09          & 25.62          & 70.19          & 2.43          & 9.53           & 69.35          & 3.41          & 21.80          & 67.54          & 2.27          \\
GLM-4-flash            & \textbf{25.28} & 74.07          & 4.26          & 39.59          & 75.05          & 3.82          & 19.71          & 71.57          & 5.01          & 26.28          & 72.71          & 3.42          \\
GLM-4                  & 23.05          & 73.31          & 4.47          & 39.55          & 74.82          & 4.32          & 18.68          & 71.58          & 5.39          & \textbf{26.69} & \textbf{75.85} & 3.59          \\
Qwen2.5-72b-instruct   & 22.39          & \textbf{75.45} & \textbf{4.97} & 46.32          & 76.47          & 4.58          & 23.50          & 71.33          & \textbf{6.19} & 19.45          & 74.23          & \textbf{4.46} \\
Llama-3.3-70b-instruct & 21.11          & 70.68          & 4.07          & 37.43          & 74.57          & 3.58          & 23.54          & 72.72          & 4.87          & 21.65          & 70.69          & 4.05          \\
GPT-3.5-turbo          & 19.89          & 71.67          & 4.90          & 38.22          & 73.98          & 4.31          & 22.59          & 71.18          & 5.90          & 17.78          & 70.71          & 3.99          \\
GPT-4o-mini            & 20.84          & 71.35          & 4.83          & 36.00          & 73.69          & 3.99          & 22.46          & 71.50          & 5.66          & 18.52          & 71.03          & 3.88          \\
Claude-sonnet          & 23.60          & 73.31          & 4.91          & \textbf{53.03} & \textbf{77.92} & \textbf{4.75} & \textbf{25.16} & \textbf{72.74} & 5.77          & 25.62          & 77.00          & 4.00          \\ \hline
\end{tabular}
\caption{The main results of the four tasks in CaseGen. ``ROU.'' represents the ROUGE-L score (\%), ``BS.'' stands for BERTScore (\%), and ``LLM'' refers to the scores assigned by the LLM Judge. The best results are highlighted in bold.}
\label{table:main}
\end{table*}



\subsection{Experimental Settings}
We evaluated several popular commercial and open-source models, including GLM-4-flash~\cite{glm2024chatglm}, GLM-4~\cite{glm2024chatglm}, Claude-3.5-sonnet, GPT-3.5-turbo~\cite{achiam2023gpt}, GPT-4o-mini~\cite{achiam2023gpt}, Qwen2.5-72B-Instruct~\cite{yang2024qwen2}, and LLaMA-3.3-70B-Instruct~\cite{touvron2023llama}. Additionally, we assessed legal-specific LLMs, including ChatLaw~\cite{cui2023chatlaw} and LexiLaw~\cite{LexiLaw}.


To ensure reproducibility, we set the temperature of all LLMs to 0. 
All LLMs are evaluated with the same prompt to ensure a fair comparison.
When the input text exceeds the LLM's maximum context window, we truncate the input sequence from the middle since the front and end of the
input may contain crucial information.
We use GPT-4o as the LLM judge to evaluate the performance of other LLMs. In addition to LLM Judge scores, we also provide ROUGE-L~\cite{lin2004rouge} and BERTScore~\cite{zhang2019bertscore} as reference metrics. Due to space limitations, more implementation details are provided in Appendix \ref{sec:exper}.




\subsection{Main Result}
The performance comparison of different LLMs is presented in Table \ref{table:main}. We derive the following observations from the experiment results.

\begin{itemize}[leftmargin=*]
\item \textbf{Legal-specific LLMs exhibit suboptimal performance.}
Despite additional training on legal datasets, legal-specific LLMs such as Lexilaw and ChatLaw perform worse than general LLMs in legal case document generation tasks. 
This may be attributed to two key reasons.
First, the performance of legal-specific LLMs may be limited by the constraints of their base models, which often lack the advanced comprehension and long-text processing capabilities of state-of-the-art general LLMs such as GPT-3.5 and Qwen2.5. For example, Lexilaw has a maximum input length of only 2048 tokens, which may lead to information loss when processing lengthy legal case documents, significantly impacting the quality of the generated context.
Another possible reason is that continuous training on legal corpora may reduce the reasoning abilities inherited from the original base model, limiting its overall effectiveness in generating complex legal cases.
This suggests that legal-specific LLMs need further optimization of training strategies to improve legal reasoning capabilities.




\item \textbf{Open-Source LLMs Demonstrate Competitive Performance in Legal Case Documents Generation.}
Compared to closed-source models like GPT-3.5-turbo and Claude-sonnet, open-source LLMs have achieved competitive performance in legal case documents generation tasks.
Qwen2.5-72B-Instruct achieved the highest LLM judge scores in drafting defense statements, writing trial facts, and generating judgment results.
These results highlight the potential of open-source LLMs as a viable alternative to commercial LLMs.
With continued improvements, open-source LLMs are expected to play an increasingly important role in legal AI applications, making further exploration and development essential.




\item \textbf{Existing LLMs Still Struggle with Legal Case Documents Generation.}
Across multiple tasks evaluated by CaseGen,  most LLMs achieve unsatisfactory scores (below 6 points), indicating that they fail to meet the basic quality standards required for legal case documents. 
This highlights the significant challenges that existing LLMs still face in handling complex legal reasoning.
These LLMs often struggle to generate text that is not only legally precise but also logically coherent. 
Furthermore, it emphasizes the value and challenges of CaseGen as a benchmark for legal document generation, providing clear guidance for the future development of legal AI and specialized LLMs.

\end{itemize}





% 在不同 LLM 的实验中,GPT-3.5 和 LLaMA3.1- 8B 表现不佳,在 LegalAgentBench 上的成功率低于 30%。这可能是因为他们有效使用工具的能力有限,限制了他们在复杂法律任务中解决问题的能力。GLM4、GLM4-Plus、Qwen-Max 和 GPT-4o-mini 的代币消耗量相近。不过,GLM4-Plus 和 Qwen-Max 的性能更胜一筹。Claude-Sonnet 也取得了有竞争力的结果,在 P-S 和 P-E 两种方法下性能最佳。不过,与其他 LLM 相比,它往往需要更多的令牌。在 ReAct 方法下,GPT-4o 以相对较少的令牌取得了最佳性能,成功率达到 78.75%。总的来说,LegalA- gentBench 有效地区分了法律知识管理的能力,并为法律知识管理提供了一种新的方法。


% 通过比较不同的方法,我们发现 ReAct 对于多跳问题通常能产生更好的结果。然而,这种优势伴随着更高的令牌消耗,这表明允许更多的推理时间可以提高性能。此外,我们发现当 LLM 能力有限时,P-E 方法并不总是优于 P-S。GLM-4、LLaMa3.1-8B 和 GPT-3.5 等 LLM 也呈现出类似的趋势。这可能是由于计划更新增加了上下文长度,从而降低了注意力机制的有效性。对于相同的 LLM,不同推理方法之间的性能差距可达 65%,这说明有效的方法能更好地利用 LLM 的潜力。此外,在设计有效的推理方法时,平衡模型能力、推理时间和令牌消耗也是至关重要的。



\subsubsection{Human Evaluation on CaseGen}
\label{sec:human}

In this section, we recruit legal experts to evaluate LLM-generated texts and assess the consistency between LLM judges and human annotations.
Due to cost limitations, we randomly select 50 cases from CaseGen. For each question, we obtain the response from three LLMs: Qwen2.5-72b-instruct, GPT-4o-mini, and Claude-sonnet, as these LLMs demonstrate competitive performance on CaseGen.
Each LLM completes four tasks from CaseGen, generating a total of 600 samples to be evaluated.

We recruit three legal experts, all of whom have passed the National Unified Legal Professional Qualification Examination, to carry out the annotation tasks.
We convert each sample into a input-response-reference triple and present it to human annotators.
To prevent potential bias, annotators were unaware of which LLM generated the response, and the responses were provided in random order.
The annotation criteria provided to the experts align with those given to the LLM judges, ensuring a fair comparison.
We use the Kappa statistic~\cite{warrens2015five} to measure the consistency and quality of the human annotations. The Kappa values~\cite{warrens2015five} for the three annotators across the four tasks are 0.428/0.488/0.539/0.494, respectively, indicating the reliability of our annotations. We paid \$0.21 for each annotation example, totaling \$385.20.


\begin{table}[t]
\small
\centering
\begin{tabular}{lcccc}
\hline
Model                & Defense       & Fact          & Reasoning     & Judgement     \\ \hline
Qwen2.5-72b & \textbf{4.76} & 4.98          & \textbf{5.56} & \textbf{6.02} \\
GPT-4o-mini          & 4.16          & 5.5           & 5.32          & 5.78          \\
Claude-sonnet        & 4.66          & \textbf{5.98} & 5.46          & 5.64          \\ \hline
\end{tabular}
\caption{Results of Human Annotation. The best results are highlighted in bold.}
\label{human}
\end{table}


Table \ref{human} presents the results of the human annotations.
For legal experts, the legal case documents generated by LLMs are still unsatisfactory (below 6 points). 
Even the most advanced LLMs still cannot generate legal case documents that are truly suitable for practical use.
Moreover, we observe that legal experts gave slightly higher average ratings for the tasks of writing trial facts and generating judgment results compared to the LLM judges.
This may be because legal experts can better understand the context and nuances of legal provisions, allowing them to make more accurate judgments based on real-world cases.
On the other hand, LLM judges face limitations in accuracy and logical rigor when dealing with complex legal relationships and dynamic statutes, as they can only compare responses to reference answers.
Further improvements are still needed in the performance of LLM judges within the legal field.


Then, we calculated Spearman's rank correlation coefficient~\cite{zar2005spearman}, Kendall rank correlation coefficient~\cite{abdi2007kendall}, and Pearson correlation coefficient~\cite{sedgwick2012pearson} between the automated metrics and human evaluation results. 
Since the evaluation dimensions vary across tasks, we calculated the consistency for each task separately and then averaged the results.
Table \ref{coefficient} presents the consistency result.
We observe that the LLM judge score demonstrates the highest level of consistency with human annotations, with a Spearman's rank correlation coefficient reaching 75\%. 
In contrast, Rough-L, which relies on lexical matching, demonstrated lower consistency.
BERTScore, which compresses context into vectors to calculate similarity, results in the loss of important details and thus demonstrates the lowest consistency with human annotations.
In conclusion, our evaluation pipeline shows high consistency with human assessments, making it a reliable alternative for large-scale evaluations.


\begin{table}[t]
\centering
\begin{tabular}{lccc}
\hline
Metrics  & \multicolumn{1}{c}{LLM Score} & \multicolumn{1}{c}{Rouge-L} & \multicolumn{1}{c}{BERTScore} \\ \hline
Kendall  & \textbf{0.667}                & 0.333                       & 0.166                         \\
Pearson  & \textbf{0.726}                & 0.264                       & 0.239                         \\
Spearman & \textbf{0.750}                & 0.375                       & 0.250                         \\ \hline
\end{tabular}
\caption{The consistency between different automated metrics and human annotations. he best results are highlighted in bold}
\label{coefficient}
\end{table}










\section{Related Work}
\subsection{Neural Solvers for Combinatorial Optimization}
The neural network (NN) models have recently garnered vast attention in solving CO problem \citep{bengio2020machine}. The NN-based solvers could be roughly categorized into three classes according to the training methods, including  the supervised learning-based  \citep{Li2018CombinatorialOW, GasseCFCL19, sun2023difusco, li2023from, li2024fast}, unsupervised learning-based \citep{Karalias2020ErdosGN, wang2022unsupervised, wang2023unsupervised, zhang2023let, SanokowskiHL24}, and reinforcement learning-based \citep{Khalil2017DQNCO, qiu2022dimes} methods. Our proposed RLNN method is partially based on reinforcement learning, but could be efficiently trained with a local objective. Such a feature has greatly improved its training efficiency by eliminating the need to estimate the future return.


\subsection{Sampling for Combinatorial Optimization}
Sampling-based methods \citep{Metropolis1953EquationOS, Hastings1970MonteCS, Neal1996SamplingFM, IBA_2001} have been commonly applied in CO problems \citep{tsp_sample, Bhattacharya2014SimulatedAA, TAVAKKOLIMOGHADDAM2007406, SECKINER200731, Chen2004MultiobjectiveVP}. However, earlier methods often encountered slow convergence compared to learning-based approaches due to an inefficient proposal. Recent advancements in discrete Monte Carlo Markov Chain \citep{Grathwohl2021OopsIT, zhang2022langevinlike, sun2022path} have revitalized sampling-based methods and  \citet{sun2023revisiting} demonstrated that simulated annealing (SA) can surpass neural CO solvers. In this work, we have advanced the current study on discrete Langevin dynamics, and proposed a novel SA algorithm. Our conclusion supports the previous study the further advances the development of the field.




\section{Conclusion \& Limitation}
\section{Conclusion}


In this paper, we present CaseGen, the first comprehensive benchmark designed to evaluate LLMs in legal case documents generation task. 
CaseGen fills a critical gap by providing a robust framework for evaluating LLMs in multi-stage legal document generation.
By covering all key stages of legal document creation—from prosecution to judgment—it enables a more nuanced evaluation of LLM performance in tasks that capture the complexities of real-world legal work.
Additionally, CaseGen supports four key tasks: drafting defense statements, writing trial facts, composing legal reasoning, and generating judgment results. It offers both researchers and practitioners a means to identify strengths and weaknesses in current LLMs, laying the foundation for future improvements in automated legal case documents generation.
In the future, we will further refine the automated evaluation framework for legal documents generation to achieve more accurate and comprehensive assessment results.

\section*{Impact Statement}
This paper presents work whose goal is to advance the field of Machine Learning. There are many potential societal consequences of our work, none which we feel must be specifically highlighted here.
% In the unusual situation where you want a paper to appear in the
% references without citing it in the main text, use \nocite
%\nocite{langley00}

\bibliography{main}
\bibliographystyle{icml2025}


%%%%%%%%%%%%%%%%%%%%%%%%%%%%%%%%%%%%%%%%%%%%%%%%%%%%%%%%%%%%%%%%%%%%%%%%%%%%%%%
%%%%%%%%%%%%%%%%%%%%%%%%%%%%%%%%%%%%%%%%%%%%%%%%%%%%%%%%%%%%%%%%%%%%%%%%%%%%%%%
% APPENDIX
%%%%%%%%%%%%%%%%%%%%%%%%%%%%%%%%%%%%%%%%%%%%%%%%%%%%%%%%%%%%%%%%%%%%%%%%%%%%%%%
%%%%%%%%%%%%%%%%%%%%%%%%%%%%%%%%%%%%%%%%%%%%%%%%%%%%%%%%%%%%%%%%%%%%%%%%%%%%%%%
\newpage
\appendix
\onecolumn
\subsection{Lloyd-Max Algorithm}
\label{subsec:Lloyd-Max}
For a given quantization bitwidth $B$ and an operand $\bm{X}$, the Lloyd-Max algorithm finds $2^B$ quantization levels $\{\hat{x}_i\}_{i=1}^{2^B}$ such that quantizing $\bm{X}$ by rounding each scalar in $\bm{X}$ to the nearest quantization level minimizes the quantization MSE. 

The algorithm starts with an initial guess of quantization levels and then iteratively computes quantization thresholds $\{\tau_i\}_{i=1}^{2^B-1}$ and updates quantization levels $\{\hat{x}_i\}_{i=1}^{2^B}$. Specifically, at iteration $n$, thresholds are set to the midpoints of the previous iteration's levels:
\begin{align*}
    \tau_i^{(n)}=\frac{\hat{x}_i^{(n-1)}+\hat{x}_{i+1}^{(n-1)}}2 \text{ for } i=1\ldots 2^B-1
\end{align*}
Subsequently, the quantization levels are re-computed as conditional means of the data regions defined by the new thresholds:
\begin{align*}
    \hat{x}_i^{(n)}=\mathbb{E}\left[ \bm{X} \big| \bm{X}\in [\tau_{i-1}^{(n)},\tau_i^{(n)}] \right] \text{ for } i=1\ldots 2^B
\end{align*}
where to satisfy boundary conditions we have $\tau_0=-\infty$ and $\tau_{2^B}=\infty$. The algorithm iterates the above steps until convergence.

Figure \ref{fig:lm_quant} compares the quantization levels of a $7$-bit floating point (E3M3) quantizer (left) to a $7$-bit Lloyd-Max quantizer (right) when quantizing a layer of weights from the GPT3-126M model at a per-tensor granularity. As shown, the Lloyd-Max quantizer achieves substantially lower quantization MSE. Further, Table \ref{tab:FP7_vs_LM7} shows the superior perplexity achieved by Lloyd-Max quantizers for bitwidths of $7$, $6$ and $5$. The difference between the quantizers is clear at 5 bits, where per-tensor FP quantization incurs a drastic and unacceptable increase in perplexity, while Lloyd-Max quantization incurs a much smaller increase. Nevertheless, we note that even the optimal Lloyd-Max quantizer incurs a notable ($\sim 1.5$) increase in perplexity due to the coarse granularity of quantization. 

\begin{figure}[h]
  \centering
  \includegraphics[width=0.7\linewidth]{sections/figures/LM7_FP7.pdf}
  \caption{\small Quantization levels and the corresponding quantization MSE of Floating Point (left) vs Lloyd-Max (right) Quantizers for a layer of weights in the GPT3-126M model.}
  \label{fig:lm_quant}
\end{figure}

\begin{table}[h]\scriptsize
\begin{center}
\caption{\label{tab:FP7_vs_LM7} \small Comparing perplexity (lower is better) achieved by floating point quantizers and Lloyd-Max quantizers on a GPT3-126M model for the Wikitext-103 dataset.}
\begin{tabular}{c|cc|c}
\hline
 \multirow{2}{*}{\textbf{Bitwidth}} & \multicolumn{2}{|c|}{\textbf{Floating-Point Quantizer}} & \textbf{Lloyd-Max Quantizer} \\
 & Best Format & Wikitext-103 Perplexity & Wikitext-103 Perplexity \\
\hline
7 & E3M3 & 18.32 & 18.27 \\
6 & E3M2 & 19.07 & 18.51 \\
5 & E4M0 & 43.89 & 19.71 \\
\hline
\end{tabular}
\end{center}
\end{table}

\subsection{Proof of Local Optimality of LO-BCQ}
\label{subsec:lobcq_opt_proof}
For a given block $\bm{b}_j$, the quantization MSE during LO-BCQ can be empirically evaluated as $\frac{1}{L_b}\lVert \bm{b}_j- \bm{\hat{b}}_j\rVert^2_2$ where $\bm{\hat{b}}_j$ is computed from equation (\ref{eq:clustered_quantization_definition}) as $C_{f(\bm{b}_j)}(\bm{b}_j)$. Further, for a given block cluster $\mathcal{B}_i$, we compute the quantization MSE as $\frac{1}{|\mathcal{B}_{i}|}\sum_{\bm{b} \in \mathcal{B}_{i}} \frac{1}{L_b}\lVert \bm{b}- C_i^{(n)}(\bm{b})\rVert^2_2$. Therefore, at the end of iteration $n$, we evaluate the overall quantization MSE $J^{(n)}$ for a given operand $\bm{X}$ composed of $N_c$ block clusters as:
\begin{align*}
    \label{eq:mse_iter_n}
    J^{(n)} = \frac{1}{N_c} \sum_{i=1}^{N_c} \frac{1}{|\mathcal{B}_{i}^{(n)}|}\sum_{\bm{v} \in \mathcal{B}_{i}^{(n)}} \frac{1}{L_b}\lVert \bm{b}- B_i^{(n)}(\bm{b})\rVert^2_2
\end{align*}

At the end of iteration $n$, the codebooks are updated from $\mathcal{C}^{(n-1)}$ to $\mathcal{C}^{(n)}$. However, the mapping of a given vector $\bm{b}_j$ to quantizers $\mathcal{C}^{(n)}$ remains as  $f^{(n)}(\bm{b}_j)$. At the next iteration, during the vector clustering step, $f^{(n+1)}(\bm{b}_j)$ finds new mapping of $\bm{b}_j$ to updated codebooks $\mathcal{C}^{(n)}$ such that the quantization MSE over the candidate codebooks is minimized. Therefore, we obtain the following result for $\bm{b}_j$:
\begin{align*}
\frac{1}{L_b}\lVert \bm{b}_j - C_{f^{(n+1)}(\bm{b}_j)}^{(n)}(\bm{b}_j)\rVert^2_2 \le \frac{1}{L_b}\lVert \bm{b}_j - C_{f^{(n)}(\bm{b}_j)}^{(n)}(\bm{b}_j)\rVert^2_2
\end{align*}

That is, quantizing $\bm{b}_j$ at the end of the block clustering step of iteration $n+1$ results in lower quantization MSE compared to quantizing at the end of iteration $n$. Since this is true for all $\bm{b} \in \bm{X}$, we assert the following:
\begin{equation}
\begin{split}
\label{eq:mse_ineq_1}
    \tilde{J}^{(n+1)} &= \frac{1}{N_c} \sum_{i=1}^{N_c} \frac{1}{|\mathcal{B}_{i}^{(n+1)}|}\sum_{\bm{b} \in \mathcal{B}_{i}^{(n+1)}} \frac{1}{L_b}\lVert \bm{b} - C_i^{(n)}(b)\rVert^2_2 \le J^{(n)}
\end{split}
\end{equation}
where $\tilde{J}^{(n+1)}$ is the the quantization MSE after the vector clustering step at iteration $n+1$.

Next, during the codebook update step (\ref{eq:quantizers_update}) at iteration $n+1$, the per-cluster codebooks $\mathcal{C}^{(n)}$ are updated to $\mathcal{C}^{(n+1)}$ by invoking the Lloyd-Max algorithm \citep{Lloyd}. We know that for any given value distribution, the Lloyd-Max algorithm minimizes the quantization MSE. Therefore, for a given vector cluster $\mathcal{B}_i$ we obtain the following result:

\begin{equation}
    \frac{1}{|\mathcal{B}_{i}^{(n+1)}|}\sum_{\bm{b} \in \mathcal{B}_{i}^{(n+1)}} \frac{1}{L_b}\lVert \bm{b}- C_i^{(n+1)}(\bm{b})\rVert^2_2 \le \frac{1}{|\mathcal{B}_{i}^{(n+1)}|}\sum_{\bm{b} \in \mathcal{B}_{i}^{(n+1)}} \frac{1}{L_b}\lVert \bm{b}- C_i^{(n)}(\bm{b})\rVert^2_2
\end{equation}

The above equation states that quantizing the given block cluster $\mathcal{B}_i$ after updating the associated codebook from $C_i^{(n)}$ to $C_i^{(n+1)}$ results in lower quantization MSE. Since this is true for all the block clusters, we derive the following result: 
\begin{equation}
\begin{split}
\label{eq:mse_ineq_2}
     J^{(n+1)} &= \frac{1}{N_c} \sum_{i=1}^{N_c} \frac{1}{|\mathcal{B}_{i}^{(n+1)}|}\sum_{\bm{b} \in \mathcal{B}_{i}^{(n+1)}} \frac{1}{L_b}\lVert \bm{b}- C_i^{(n+1)}(\bm{b})\rVert^2_2  \le \tilde{J}^{(n+1)}   
\end{split}
\end{equation}

Following (\ref{eq:mse_ineq_1}) and (\ref{eq:mse_ineq_2}), we find that the quantization MSE is non-increasing for each iteration, that is, $J^{(1)} \ge J^{(2)} \ge J^{(3)} \ge \ldots \ge J^{(M)}$ where $M$ is the maximum number of iterations. 
%Therefore, we can say that if the algorithm converges, then it must be that it has converged to a local minimum. 
\hfill $\blacksquare$


\begin{figure}
    \begin{center}
    \includegraphics[width=0.5\textwidth]{sections//figures/mse_vs_iter.pdf}
    \end{center}
    \caption{\small NMSE vs iterations during LO-BCQ compared to other block quantization proposals}
    \label{fig:nmse_vs_iter}
\end{figure}

Figure \ref{fig:nmse_vs_iter} shows the empirical convergence of LO-BCQ across several block lengths and number of codebooks. Also, the MSE achieved by LO-BCQ is compared to baselines such as MXFP and VSQ. As shown, LO-BCQ converges to a lower MSE than the baselines. Further, we achieve better convergence for larger number of codebooks ($N_c$) and for a smaller block length ($L_b$), both of which increase the bitwidth of BCQ (see Eq \ref{eq:bitwidth_bcq}).


\subsection{Additional Accuracy Results}
%Table \ref{tab:lobcq_config} lists the various LOBCQ configurations and their corresponding bitwidths.
\begin{table}
\setlength{\tabcolsep}{4.75pt}
\begin{center}
\caption{\label{tab:lobcq_config} Various LO-BCQ configurations and their bitwidths.}
\begin{tabular}{|c||c|c|c|c||c|c||c|} 
\hline
 & \multicolumn{4}{|c||}{$L_b=8$} & \multicolumn{2}{|c||}{$L_b=4$} & $L_b=2$ \\
 \hline
 \backslashbox{$L_A$\kern-1em}{\kern-1em$N_c$} & 2 & 4 & 8 & 16 & 2 & 4 & 2 \\
 \hline
 64 & 4.25 & 4.375 & 4.5 & 4.625 & 4.375 & 4.625 & 4.625\\
 \hline
 32 & 4.375 & 4.5 & 4.625& 4.75 & 4.5 & 4.75 & 4.75 \\
 \hline
 16 & 4.625 & 4.75& 4.875 & 5 & 4.75 & 5 & 5 \\
 \hline
\end{tabular}
\end{center}
\end{table}

%\subsection{Perplexity achieved by various LO-BCQ configurations on Wikitext-103 dataset}

\begin{table} \centering
\begin{tabular}{|c||c|c|c|c||c|c||c|} 
\hline
 $L_b \rightarrow$& \multicolumn{4}{c||}{8} & \multicolumn{2}{c||}{4} & 2\\
 \hline
 \backslashbox{$L_A$\kern-1em}{\kern-1em$N_c$} & 2 & 4 & 8 & 16 & 2 & 4 & 2  \\
 %$N_c \rightarrow$ & 2 & 4 & 8 & 16 & 2 & 4 & 2 \\
 \hline
 \hline
 \multicolumn{8}{c}{GPT3-1.3B (FP32 PPL = 9.98)} \\ 
 \hline
 \hline
 64 & 10.40 & 10.23 & 10.17 & 10.15 &  10.28 & 10.18 & 10.19 \\
 \hline
 32 & 10.25 & 10.20 & 10.15 & 10.12 &  10.23 & 10.17 & 10.17 \\
 \hline
 16 & 10.22 & 10.16 & 10.10 & 10.09 &  10.21 & 10.14 & 10.16 \\
 \hline
  \hline
 \multicolumn{8}{c}{GPT3-8B (FP32 PPL = 7.38)} \\ 
 \hline
 \hline
 64 & 7.61 & 7.52 & 7.48 &  7.47 &  7.55 &  7.49 & 7.50 \\
 \hline
 32 & 7.52 & 7.50 & 7.46 &  7.45 &  7.52 &  7.48 & 7.48  \\
 \hline
 16 & 7.51 & 7.48 & 7.44 &  7.44 &  7.51 &  7.49 & 7.47  \\
 \hline
\end{tabular}
\caption{\label{tab:ppl_gpt3_abalation} Wikitext-103 perplexity across GPT3-1.3B and 8B models.}
\end{table}

\begin{table} \centering
\begin{tabular}{|c||c|c|c|c||} 
\hline
 $L_b \rightarrow$& \multicolumn{4}{c||}{8}\\
 \hline
 \backslashbox{$L_A$\kern-1em}{\kern-1em$N_c$} & 2 & 4 & 8 & 16 \\
 %$N_c \rightarrow$ & 2 & 4 & 8 & 16 & 2 & 4 & 2 \\
 \hline
 \hline
 \multicolumn{5}{|c|}{Llama2-7B (FP32 PPL = 5.06)} \\ 
 \hline
 \hline
 64 & 5.31 & 5.26 & 5.19 & 5.18  \\
 \hline
 32 & 5.23 & 5.25 & 5.18 & 5.15  \\
 \hline
 16 & 5.23 & 5.19 & 5.16 & 5.14  \\
 \hline
 \multicolumn{5}{|c|}{Nemotron4-15B (FP32 PPL = 5.87)} \\ 
 \hline
 \hline
 64  & 6.3 & 6.20 & 6.13 & 6.08  \\
 \hline
 32  & 6.24 & 6.12 & 6.07 & 6.03  \\
 \hline
 16  & 6.12 & 6.14 & 6.04 & 6.02  \\
 \hline
 \multicolumn{5}{|c|}{Nemotron4-340B (FP32 PPL = 3.48)} \\ 
 \hline
 \hline
 64 & 3.67 & 3.62 & 3.60 & 3.59 \\
 \hline
 32 & 3.63 & 3.61 & 3.59 & 3.56 \\
 \hline
 16 & 3.61 & 3.58 & 3.57 & 3.55 \\
 \hline
\end{tabular}
\caption{\label{tab:ppl_llama7B_nemo15B} Wikitext-103 perplexity compared to FP32 baseline in Llama2-7B and Nemotron4-15B, 340B models}
\end{table}

%\subsection{Perplexity achieved by various LO-BCQ configurations on MMLU dataset}


\begin{table} \centering
\begin{tabular}{|c||c|c|c|c||c|c|c|c|} 
\hline
 $L_b \rightarrow$& \multicolumn{4}{c||}{8} & \multicolumn{4}{c||}{8}\\
 \hline
 \backslashbox{$L_A$\kern-1em}{\kern-1em$N_c$} & 2 & 4 & 8 & 16 & 2 & 4 & 8 & 16  \\
 %$N_c \rightarrow$ & 2 & 4 & 8 & 16 & 2 & 4 & 2 \\
 \hline
 \hline
 \multicolumn{5}{|c|}{Llama2-7B (FP32 Accuracy = 45.8\%)} & \multicolumn{4}{|c|}{Llama2-70B (FP32 Accuracy = 69.12\%)} \\ 
 \hline
 \hline
 64 & 43.9 & 43.4 & 43.9 & 44.9 & 68.07 & 68.27 & 68.17 & 68.75 \\
 \hline
 32 & 44.5 & 43.8 & 44.9 & 44.5 & 68.37 & 68.51 & 68.35 & 68.27  \\
 \hline
 16 & 43.9 & 42.7 & 44.9 & 45 & 68.12 & 68.77 & 68.31 & 68.59  \\
 \hline
 \hline
 \multicolumn{5}{|c|}{GPT3-22B (FP32 Accuracy = 38.75\%)} & \multicolumn{4}{|c|}{Nemotron4-15B (FP32 Accuracy = 64.3\%)} \\ 
 \hline
 \hline
 64 & 36.71 & 38.85 & 38.13 & 38.92 & 63.17 & 62.36 & 63.72 & 64.09 \\
 \hline
 32 & 37.95 & 38.69 & 39.45 & 38.34 & 64.05 & 62.30 & 63.8 & 64.33  \\
 \hline
 16 & 38.88 & 38.80 & 38.31 & 38.92 & 63.22 & 63.51 & 63.93 & 64.43  \\
 \hline
\end{tabular}
\caption{\label{tab:mmlu_abalation} Accuracy on MMLU dataset across GPT3-22B, Llama2-7B, 70B and Nemotron4-15B models.}
\end{table}


%\subsection{Perplexity achieved by various LO-BCQ configurations on LM evaluation harness}

\begin{table} \centering
\begin{tabular}{|c||c|c|c|c||c|c|c|c|} 
\hline
 $L_b \rightarrow$& \multicolumn{4}{c||}{8} & \multicolumn{4}{c||}{8}\\
 \hline
 \backslashbox{$L_A$\kern-1em}{\kern-1em$N_c$} & 2 & 4 & 8 & 16 & 2 & 4 & 8 & 16  \\
 %$N_c \rightarrow$ & 2 & 4 & 8 & 16 & 2 & 4 & 2 \\
 \hline
 \hline
 \multicolumn{5}{|c|}{Race (FP32 Accuracy = 37.51\%)} & \multicolumn{4}{|c|}{Boolq (FP32 Accuracy = 64.62\%)} \\ 
 \hline
 \hline
 64 & 36.94 & 37.13 & 36.27 & 37.13 & 63.73 & 62.26 & 63.49 & 63.36 \\
 \hline
 32 & 37.03 & 36.36 & 36.08 & 37.03 & 62.54 & 63.51 & 63.49 & 63.55  \\
 \hline
 16 & 37.03 & 37.03 & 36.46 & 37.03 & 61.1 & 63.79 & 63.58 & 63.33  \\
 \hline
 \hline
 \multicolumn{5}{|c|}{Winogrande (FP32 Accuracy = 58.01\%)} & \multicolumn{4}{|c|}{Piqa (FP32 Accuracy = 74.21\%)} \\ 
 \hline
 \hline
 64 & 58.17 & 57.22 & 57.85 & 58.33 & 73.01 & 73.07 & 73.07 & 72.80 \\
 \hline
 32 & 59.12 & 58.09 & 57.85 & 58.41 & 73.01 & 73.94 & 72.74 & 73.18  \\
 \hline
 16 & 57.93 & 58.88 & 57.93 & 58.56 & 73.94 & 72.80 & 73.01 & 73.94  \\
 \hline
\end{tabular}
\caption{\label{tab:mmlu_abalation} Accuracy on LM evaluation harness tasks on GPT3-1.3B model.}
\end{table}

\begin{table} \centering
\begin{tabular}{|c||c|c|c|c||c|c|c|c|} 
\hline
 $L_b \rightarrow$& \multicolumn{4}{c||}{8} & \multicolumn{4}{c||}{8}\\
 \hline
 \backslashbox{$L_A$\kern-1em}{\kern-1em$N_c$} & 2 & 4 & 8 & 16 & 2 & 4 & 8 & 16  \\
 %$N_c \rightarrow$ & 2 & 4 & 8 & 16 & 2 & 4 & 2 \\
 \hline
 \hline
 \multicolumn{5}{|c|}{Race (FP32 Accuracy = 41.34\%)} & \multicolumn{4}{|c|}{Boolq (FP32 Accuracy = 68.32\%)} \\ 
 \hline
 \hline
 64 & 40.48 & 40.10 & 39.43 & 39.90 & 69.20 & 68.41 & 69.45 & 68.56 \\
 \hline
 32 & 39.52 & 39.52 & 40.77 & 39.62 & 68.32 & 67.43 & 68.17 & 69.30  \\
 \hline
 16 & 39.81 & 39.71 & 39.90 & 40.38 & 68.10 & 66.33 & 69.51 & 69.42  \\
 \hline
 \hline
 \multicolumn{5}{|c|}{Winogrande (FP32 Accuracy = 67.88\%)} & \multicolumn{4}{|c|}{Piqa (FP32 Accuracy = 78.78\%)} \\ 
 \hline
 \hline
 64 & 66.85 & 66.61 & 67.72 & 67.88 & 77.31 & 77.42 & 77.75 & 77.64 \\
 \hline
 32 & 67.25 & 67.72 & 67.72 & 67.00 & 77.31 & 77.04 & 77.80 & 77.37  \\
 \hline
 16 & 68.11 & 68.90 & 67.88 & 67.48 & 77.37 & 78.13 & 78.13 & 77.69  \\
 \hline
\end{tabular}
\caption{\label{tab:mmlu_abalation} Accuracy on LM evaluation harness tasks on GPT3-8B model.}
\end{table}

\begin{table} \centering
\begin{tabular}{|c||c|c|c|c||c|c|c|c|} 
\hline
 $L_b \rightarrow$& \multicolumn{4}{c||}{8} & \multicolumn{4}{c||}{8}\\
 \hline
 \backslashbox{$L_A$\kern-1em}{\kern-1em$N_c$} & 2 & 4 & 8 & 16 & 2 & 4 & 8 & 16  \\
 %$N_c \rightarrow$ & 2 & 4 & 8 & 16 & 2 & 4 & 2 \\
 \hline
 \hline
 \multicolumn{5}{|c|}{Race (FP32 Accuracy = 40.67\%)} & \multicolumn{4}{|c|}{Boolq (FP32 Accuracy = 76.54\%)} \\ 
 \hline
 \hline
 64 & 40.48 & 40.10 & 39.43 & 39.90 & 75.41 & 75.11 & 77.09 & 75.66 \\
 \hline
 32 & 39.52 & 39.52 & 40.77 & 39.62 & 76.02 & 76.02 & 75.96 & 75.35  \\
 \hline
 16 & 39.81 & 39.71 & 39.90 & 40.38 & 75.05 & 73.82 & 75.72 & 76.09  \\
 \hline
 \hline
 \multicolumn{5}{|c|}{Winogrande (FP32 Accuracy = 70.64\%)} & \multicolumn{4}{|c|}{Piqa (FP32 Accuracy = 79.16\%)} \\ 
 \hline
 \hline
 64 & 69.14 & 70.17 & 70.17 & 70.56 & 78.24 & 79.00 & 78.62 & 78.73 \\
 \hline
 32 & 70.96 & 69.69 & 71.27 & 69.30 & 78.56 & 79.49 & 79.16 & 78.89  \\
 \hline
 16 & 71.03 & 69.53 & 69.69 & 70.40 & 78.13 & 79.16 & 79.00 & 79.00  \\
 \hline
\end{tabular}
\caption{\label{tab:mmlu_abalation} Accuracy on LM evaluation harness tasks on GPT3-22B model.}
\end{table}

\begin{table} \centering
\begin{tabular}{|c||c|c|c|c||c|c|c|c|} 
\hline
 $L_b \rightarrow$& \multicolumn{4}{c||}{8} & \multicolumn{4}{c||}{8}\\
 \hline
 \backslashbox{$L_A$\kern-1em}{\kern-1em$N_c$} & 2 & 4 & 8 & 16 & 2 & 4 & 8 & 16  \\
 %$N_c \rightarrow$ & 2 & 4 & 8 & 16 & 2 & 4 & 2 \\
 \hline
 \hline
 \multicolumn{5}{|c|}{Race (FP32 Accuracy = 44.4\%)} & \multicolumn{4}{|c|}{Boolq (FP32 Accuracy = 79.29\%)} \\ 
 \hline
 \hline
 64 & 42.49 & 42.51 & 42.58 & 43.45 & 77.58 & 77.37 & 77.43 & 78.1 \\
 \hline
 32 & 43.35 & 42.49 & 43.64 & 43.73 & 77.86 & 75.32 & 77.28 & 77.86  \\
 \hline
 16 & 44.21 & 44.21 & 43.64 & 42.97 & 78.65 & 77 & 76.94 & 77.98  \\
 \hline
 \hline
 \multicolumn{5}{|c|}{Winogrande (FP32 Accuracy = 69.38\%)} & \multicolumn{4}{|c|}{Piqa (FP32 Accuracy = 78.07\%)} \\ 
 \hline
 \hline
 64 & 68.9 & 68.43 & 69.77 & 68.19 & 77.09 & 76.82 & 77.09 & 77.86 \\
 \hline
 32 & 69.38 & 68.51 & 68.82 & 68.90 & 78.07 & 76.71 & 78.07 & 77.86  \\
 \hline
 16 & 69.53 & 67.09 & 69.38 & 68.90 & 77.37 & 77.8 & 77.91 & 77.69  \\
 \hline
\end{tabular}
\caption{\label{tab:mmlu_abalation} Accuracy on LM evaluation harness tasks on Llama2-7B model.}
\end{table}

\begin{table} \centering
\begin{tabular}{|c||c|c|c|c||c|c|c|c|} 
\hline
 $L_b \rightarrow$& \multicolumn{4}{c||}{8} & \multicolumn{4}{c||}{8}\\
 \hline
 \backslashbox{$L_A$\kern-1em}{\kern-1em$N_c$} & 2 & 4 & 8 & 16 & 2 & 4 & 8 & 16  \\
 %$N_c \rightarrow$ & 2 & 4 & 8 & 16 & 2 & 4 & 2 \\
 \hline
 \hline
 \multicolumn{5}{|c|}{Race (FP32 Accuracy = 48.8\%)} & \multicolumn{4}{|c|}{Boolq (FP32 Accuracy = 85.23\%)} \\ 
 \hline
 \hline
 64 & 49.00 & 49.00 & 49.28 & 48.71 & 82.82 & 84.28 & 84.03 & 84.25 \\
 \hline
 32 & 49.57 & 48.52 & 48.33 & 49.28 & 83.85 & 84.46 & 84.31 & 84.93  \\
 \hline
 16 & 49.85 & 49.09 & 49.28 & 48.99 & 85.11 & 84.46 & 84.61 & 83.94  \\
 \hline
 \hline
 \multicolumn{5}{|c|}{Winogrande (FP32 Accuracy = 79.95\%)} & \multicolumn{4}{|c|}{Piqa (FP32 Accuracy = 81.56\%)} \\ 
 \hline
 \hline
 64 & 78.77 & 78.45 & 78.37 & 79.16 & 81.45 & 80.69 & 81.45 & 81.5 \\
 \hline
 32 & 78.45 & 79.01 & 78.69 & 80.66 & 81.56 & 80.58 & 81.18 & 81.34  \\
 \hline
 16 & 79.95 & 79.56 & 79.79 & 79.72 & 81.28 & 81.66 & 81.28 & 80.96  \\
 \hline
\end{tabular}
\caption{\label{tab:mmlu_abalation} Accuracy on LM evaluation harness tasks on Llama2-70B model.}
\end{table}

%\section{MSE Studies}
%\textcolor{red}{TODO}


\subsection{Number Formats and Quantization Method}
\label{subsec:numFormats_quantMethod}
\subsubsection{Integer Format}
An $n$-bit signed integer (INT) is typically represented with a 2s-complement format \citep{yao2022zeroquant,xiao2023smoothquant,dai2021vsq}, where the most significant bit denotes the sign.

\subsubsection{Floating Point Format}
An $n$-bit signed floating point (FP) number $x$ comprises of a 1-bit sign ($x_{\mathrm{sign}}$), $B_m$-bit mantissa ($x_{\mathrm{mant}}$) and $B_e$-bit exponent ($x_{\mathrm{exp}}$) such that $B_m+B_e=n-1$. The associated constant exponent bias ($E_{\mathrm{bias}}$) is computed as $(2^{{B_e}-1}-1)$. We denote this format as $E_{B_e}M_{B_m}$.  

\subsubsection{Quantization Scheme}
\label{subsec:quant_method}
A quantization scheme dictates how a given unquantized tensor is converted to its quantized representation. We consider FP formats for the purpose of illustration. Given an unquantized tensor $\bm{X}$ and an FP format $E_{B_e}M_{B_m}$, we first, we compute the quantization scale factor $s_X$ that maps the maximum absolute value of $\bm{X}$ to the maximum quantization level of the $E_{B_e}M_{B_m}$ format as follows:
\begin{align}
\label{eq:sf}
    s_X = \frac{\mathrm{max}(|\bm{X}|)}{\mathrm{max}(E_{B_e}M_{B_m})}
\end{align}
In the above equation, $|\cdot|$ denotes the absolute value function.

Next, we scale $\bm{X}$ by $s_X$ and quantize it to $\hat{\bm{X}}$ by rounding it to the nearest quantization level of $E_{B_e}M_{B_m}$ as:

\begin{align}
\label{eq:tensor_quant}
    \hat{\bm{X}} = \text{round-to-nearest}\left(\frac{\bm{X}}{s_X}, E_{B_e}M_{B_m}\right)
\end{align}

We perform dynamic max-scaled quantization \citep{wu2020integer}, where the scale factor $s$ for activations is dynamically computed during runtime.

\subsection{Vector Scaled Quantization}
\begin{wrapfigure}{r}{0.35\linewidth}
  \centering
  \includegraphics[width=\linewidth]{sections/figures/vsquant.jpg}
  \caption{\small Vectorwise decomposition for per-vector scaled quantization (VSQ \citep{dai2021vsq}).}
  \label{fig:vsquant}
\end{wrapfigure}
During VSQ \citep{dai2021vsq}, the operand tensors are decomposed into 1D vectors in a hardware friendly manner as shown in Figure \ref{fig:vsquant}. Since the decomposed tensors are used as operands in matrix multiplications during inference, it is beneficial to perform this decomposition along the reduction dimension of the multiplication. The vectorwise quantization is performed similar to tensorwise quantization described in Equations \ref{eq:sf} and \ref{eq:tensor_quant}, where a scale factor $s_v$ is required for each vector $\bm{v}$ that maps the maximum absolute value of that vector to the maximum quantization level. While smaller vector lengths can lead to larger accuracy gains, the associated memory and computational overheads due to the per-vector scale factors increases. To alleviate these overheads, VSQ \citep{dai2021vsq} proposed a second level quantization of the per-vector scale factors to unsigned integers, while MX \citep{rouhani2023shared} quantizes them to integer powers of 2 (denoted as $2^{INT}$).

\subsubsection{MX Format}
The MX format proposed in \citep{rouhani2023microscaling} introduces the concept of sub-block shifting. For every two scalar elements of $b$-bits each, there is a shared exponent bit. The value of this exponent bit is determined through an empirical analysis that targets minimizing quantization MSE. We note that the FP format $E_{1}M_{b}$ is strictly better than MX from an accuracy perspective since it allocates a dedicated exponent bit to each scalar as opposed to sharing it across two scalars. Therefore, we conservatively bound the accuracy of a $b+2$-bit signed MX format with that of a $E_{1}M_{b}$ format in our comparisons. For instance, we use E1M2 format as a proxy for MX4.

\begin{figure}
    \centering
    \includegraphics[width=1\linewidth]{sections//figures/BlockFormats.pdf}
    \caption{\small Comparing LO-BCQ to MX format.}
    \label{fig:block_formats}
\end{figure}

Figure \ref{fig:block_formats} compares our $4$-bit LO-BCQ block format to MX \citep{rouhani2023microscaling}. As shown, both LO-BCQ and MX decompose a given operand tensor into block arrays and each block array into blocks. Similar to MX, we find that per-block quantization ($L_b < L_A$) leads to better accuracy due to increased flexibility. While MX achieves this through per-block $1$-bit micro-scales, we associate a dedicated codebook to each block through a per-block codebook selector. Further, MX quantizes the per-block array scale-factor to E8M0 format without per-tensor scaling. In contrast during LO-BCQ, we find that per-tensor scaling combined with quantization of per-block array scale-factor to E4M3 format results in superior inference accuracy across models. 


%%%%%%%%%%%%%%%%%%%%%%%%%%%%%%%%%%%%%%%%%%%%%%%%%%%%%%%%%%%%%%%%%%%%%%%%%%%%%%%
%%%%%%%%%%%%%%%%%%%%%%%%%%%%%%%%%%%%%%%%%%%%%%%%%%%%%%%%%%%%%%%%%%%%%%%%%%%%%%%

\end{document}


% This document was modified from the file originally made available by
% Pat Langley and Andrea Danyluk for ICML-2K. This version was created
% by Iain Murray in 2018, and modified by Alexandre Bouchard in
% 2019 and 2021 and by Csaba Szepesvari, Gang Niu and Sivan Sabato in 2022.
% Modified again in 2023 and 2024 by Sivan Sabato and Jonathan Scarlett.
% Previous contributors include Dan Roy, Lise Getoor and Tobias
% Scheffer, which was slightly modified from the 2010 version by
% Thorsten Joachims & Johannes Fuernkranz, slightly modified from the
% 2009 version by Kiri Wagstaff and Sam Roweis's 2008 version, which is
% slightly modified from Prasad Tadepalli's 2007 version which is a
% lightly changed version of the previous year's version by Andrew
% Moore, which was in turn edited from those of Kristian Kersting and
% Codrina Lauth. Alex Smola contributed to the algorithmic style files.
