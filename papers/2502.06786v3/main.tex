\documentclass[11pt, a4paper, logo, twocolumn, internal, copyright, nonumbering]{googledeepmind}

\usepackage[authoryear, sort&compress, round]{natbib}
% Recommended, but optional, packages for figures and better typesetting:
\usepackage{microtype}
\usepackage{graphicx}
% \usepackage{subfigure}
% \usepackage{subfig}
\usepackage{booktabs} % for professional tables

% hyperref makes hyperlinks in the resulting PDF.
% If your build breaks (sometimes temporarily if a hyperlink spans a page)
% please comment out the following usepackage line and replace
% \usepackage{icml2025} with \usepackage[nohyperref]{icml2025} above.
\usepackage{hyperref}

\usepackage{xcolor}
\newcommand\ak[1]{\textcolor{red}{#1}}
\newcommand\pd[1]{\textcolor{blue}{#1}}

% Attempt to make hyperref and algorithmic work together better:
% \newcommand{\theHalgorithm}{\arabic{algorithm}}

% Use the following line for the initial blind version submitted for review:

% If accepted, instead use the following line for the camera-ready submission:
% \usepackage[accepted]{icml2025}

% For theorems and such
\usepackage{amsmath}
\usepackage{amssymb}
\usepackage{mathtools}
\usepackage{amsthm}
\usepackage{booktabs}
\usepackage{multirow}
     % hyperlinks
\usepackage{url}            % simple URL typesetting      % professional-quality tables
\usepackage{amsfonts}       % blackboard math symbols
\usepackage{nicefrac}       % compact symbols for 1/2, etc.
\usepackage{xcolor}         % colors
\usepackage{algorithm}
% \usepackage{algpseudocode}
\usepackage{physics}

\usepackage{xspace}

\usepackage{array}
\usepackage{arydshln}
\usepackage{wrapfig}
\usepackage{rotating}

% if you use cleveref..
\usepackage[capitalize,noabbrev]{cleveref}

%%%%%%%%%%%%%%%%%%%%%%%%%%%%%%%%
% THEOREMS
%%%%%%%%%%%%%%%%%%%%%%%%%%%%%%%%
\theoremstyle{plain}
\newtheorem{theorem}{Theorem}[section]
\newtheorem{proposition}[theorem]{Proposition}
\newtheorem{lemma}[theorem]{Lemma}
\newtheorem{corollary}[theorem]{Corollary}
\theoremstyle{definition}
\newtheorem{definition}[theorem]{Definition}
\newtheorem{assumption}[theorem]{Assumption}
\theoremstyle{remark}
\newtheorem{remark}[theorem]{Remark}

% Todonotes is useful during development; simply uncomment the next line
%    and comment out the line below the next line to turn off comments
%\usepackage[disable,textsize=tiny]{todonotes}
\usepackage[textsize=tiny]{todonotes}
\usepackage{subcaption}
\newcommand{\alg}{\ensuremath{{\rm MatQuant}}\xspace}
\newcommand{\spalg}{\ensuremath{{\rm Single\text{ } Precison\text{ }MatQuant}}\xspace}
\newcommand{\epalg}{\ensuremath{{\rm Extra\text{ } Precison\text{ }MatQuant}}\xspace}

\bibliographystyle{abbrvnat}

% Information about your document.
\title{Matryoshka Quantization}

% Can leave this option out if you do not wish to add a corresponding author.
% \correspondingauthor{my-email@google.com}

% Remove these if they are not needed
% \keywords{Machine Learning}
% \paperurl{arxiv.org/abs/123} %ToDo

% Use the internally issued paper ID, if there is one
\reportnumber{001} % Leave blank if n/a % ToDo

% Assign your own date to the report.
% Can comment out if not needed or leave blank if n/a.
\renewcommand{\today}{2025-02-05} % ToDo Date

% Can have as many authors and as many affiliations as needed. Best to indicate joint
% first-authorship as shown below.
\author[*,1]{Pranav Nair}
\author[*,1]{Puranjay Datta}
\author[1]{Jeff Dean}
\author[1]{Prateek Jain}
\author[1]{Aditya Kusupati}
% Affiliations *must* come after the declaration of \author[]
\affil[1]{Google DeepMind} %ToDo
\affil[*]{Equal contribution}
% \affil[2]{External University}

\begin{abstract}
End-to-end imitation learning offers a promising approach for training robot policies. However, generalizing to new settings—such as unseen scenes, tasks, and object instances—remains a significant challenge. Although large-scale robot demonstration datasets have shown potential for inducing generalization, they are resource-intensive to scale. In contrast, human video data is abundant and diverse, presenting an attractive alternative. Yet, these human-video datasets lack action labels, complicating their use in imitation learning. Existing methods attempt to extract grounded action representations (e.g., hand poses), but resulting policies struggle to bridge the embodiment gap between human and robot actions.
% our approach
We propose an alternative approach: leveraging language-based reasoning from human videos - essential for guiding robot actions - to train generalizable robot policies. Building on recent advances in reasoning-based policy architectures, we introduce Reasoning through Action-free Data (RAD). RAD learns from both robot demonstration data (with reasoning and action labels) and action-free human video data (with only reasoning labels). The robot data teaches the model to map reasoning to low-level actions, while the action-free data enhances reasoning capabilities. Additionally, we will release a new dataset of 3,377 human-hand demonstrations compatible with the Bridge V2 benchmark. This dataset includes chain-of-thought reasoning annotations and hand-tracking data to help facilitate future work on reasoning-driven robot learning.
% experiments
Our experiments demonstrate that RAD enables effective transfer across the embodiment gap, allowing robots to perform tasks seen only in action-free data. Furthermore, scaling up action-free reasoning data significantly improves policy performance and generalization to novel tasks. These results highlight the promise of reasoning-driven learning from action-free datasets for advancing generalizable robot control. 
% releasing dataset
Website: \href{https://rad-generalization.github.io}{here}.

\end{abstract}

\setlength {\marginparwidth }{2cm} %ToDo
\setcounter{secnumdepth}{3}
\begin{document}

\maketitle

\section{Introduction}
\label{sec:intro}


\ps{Challenges of technology scaling}

The growing demand for computing performance has always been met by increasing the number of transistors per chip, which is only possible due to CMOS technology scaling.
However, as we keep pushing the boundaries of technology scaling, we encounter multiple challenges.
Firstly, whenever we transition to a more advanced technology node, the non-recurring cost due to physical design, verification, software, mask sets, and prototyping almost doubles \cite{cost-tech-node}.
As a result, designing a chip in an advanced technology node is only economically viable if the chip is manufactured in vast quantities.
Secondly, many chip components such as I/O drivers, analog circuits, or \gls{srams} have reached their scaling limit.
This means that we cannot shrink these components further, even if we use a more advanced technology with a smaller feature size.
Thirdly, advanced technology nodes suffer from high defect rates, diminishing the yield and inflating the recurring cost.
To tackle these challenges, new chip-design paradigms have been developed.

\ps{Why 2.5D integration?}

One of these new paradigms is 2.5D integration, where multiple silicon dies called chiplets are integrated into the same package.
Once designed, a single chiplet can be reused in multiple 2.5D stacked chips, which increases the ratio of production volume to non-recurring cost.
Another advantage is that multiple chiplets - fabricated in different technologies - can be integrated into the same package.
This means that only components that can take full advantage of technology scaling are built in bleeding-edge technologies.
Components that have reached their scaling limit are fabricated in more mature and hence less costly technology nodes.
Furthermore, chiplets are smaller than monolithic chips.
Therefore, manufacturing chiplets results in less silicon area loss due to fabrication defects and hence a higher yield.
Due to these economic advantages, chip vendors such as AMD \cite{amd-chiplet} and NVIDIA \cite{chiplet-book} have adopted the 2.5D integration paradigm.  

\ps{Challenges of 2.5D integration}

An important challenge when designing 2.5D stacked chips is the construction of a low-latency and high-throughput \gls{ici}. 
To build an \gls{ici}, we connect different chiplets using \gls{d2d} links.
These links are fabricated in an organic package substrate, silicon bridge, or silicon interposer, and they are connected to the chiplets using \gls{c4} bumps or microbumps.
The number of bumps per chiplet is limited, and so is the bandwidth of \gls{d2d} links.
In addition to having lower bandwidth than links in monolithic chips, \gls{d2d} links also have higher latency.
This latency is caused by wire delay and by \gls{phys} that are necessary in both the sending and the receiving chiplet.
\gls{phys} are needed to convert between protocols, voltage levels, and frequencies, which are usually different between on-chiplet links and \gls{d2d} links.
Due to these limitations, the \gls{ici} can quickly become a bottleneck.

\ps{How we solve these challenges differently than the related work does.}

Existing approaches to maximize the performance of the \gls{ici} either optimize the placement of chiplets (with potentially heterogeneous shapes) for a predetermined \gls{ici} topology 
\cite{ho,liu,seemuth,eris,osmolovskyi,tap25d,chiou}, select one topology out of a set of candidates \cite{coskun-1, coskun-2}, or they optimize the \gls{ici} topology for a 2D grid of homogeneously shaped chiplets on an active interposer \cite{butterdonut, cluscross, kite}.
To the best of our knowledge, there is no prior work on \gls{ici} topologies for chips with heterogeneously shaped chiplets or with passive silicon interposers or silicon bridges.
To fill this gap, we propose \name, a novel optimization methodology to jointly optimize the chiplet placement and \gls{ici} topology of such architectures.
\ifnb
\else
\newpage
\fi

\ps{Details on \name~and the key idea}

The key idea is as follows: 
We optimize the chiplet placement without a predetermined topology.
For each placement generated by an optimization algorithm, we infer a placement-based \gls{ici} topology by connecting chiplets that are in close proximity in that specific placement.
We then compute the latency and throughput of this combination of placement and topology for different traffic types.
These latencies and throughputs together with the total chip area are used to compute a user-defined quality-score of the placement, which is returned to the optimization algorithm.
Based on this quality score, the algorithm can further optimize the placement.
By following this iterative process, we jointly optimize the chiplet placement and the \gls{ici} topology.

\ps{Short evaluation-summary}

We provide our open-source framework implementing the proposed placement and topology co-optimization methodology, which we evaluate using both synthetic traffic and traffic traces.
A 2D grid of chiplets with a mesh topology is used as a baseline since many proposals for 2.5D stacked chips \cite{dataflow_accel_dnn, cifher, simba, hecaton, dojo} use such an architecture.
We reduce the latency of synthetic L1-to-L2 and L2-to-memory traffic, the two most important traffic types for cache coherency traffic, by up to 28\% and 62\% respectively.
For real traffic traces, we reduce the average packet latency for almost all traces and architectures considered (reduced by an 8\% or 18\% on average depending on the configuration of \gls{phys} within a chiplet).
 % 2 pages including teaser and abstract
% % \newpage

\section{Related work}


Recent advances in single-image animatable head avatar generation can be categorized into mainly 2D-based and 3D-based approaches. 

\paragraph{\bf Image to 2D Animatable Avatar.}
2D-based methods, leveraging the power of convolutional neural networks (CNNs)~\cite{DBLP:conf/cvpr/KarrasLAHLA20,DBLP:conf/cvpr/IsolaZZE17,DBLP:conf/nips/GoodfellowPMXWOCB14}, often employ generative adversarial networks (GANs)~\cite{DBLP:conf/cvpr/StyleGAN} for direct image synthesis. Early approaches~\cite{DBLP:conf/cvpr/WangDYSW23,DBLP:conf/cvpr/BurkovPGL20,DBLP:conf/iccv/ZakharovSBL19} focus on injecting expression and pose features into the generator network, often utilizing architectures like U-Net or StyleGAN~\cite{DBLP:conf/cvpr/StyleGAN}.
Some other 2D methods~\cite{DBLP:journals/corr/abs-2407-03168,DBLP:conf/cvpr/ZhangQZZW0CW023,DBLP:conf/cvpr/HongZS022,DBLP:conf/mm/DrobyshevCKILZ22,DBLP:conf/cvpr/BurkovPGL20,DBLP:conf/nips/SiarohinLT0S19} represent expressions and poses as warping fields applied to the source image. 
Benefiting from advances in image and video diffusion networks, more recent 2D-based works~\cite{DBLP:journals/corr/abs-2410-07718,DBLP:journals/corr/abs-2406-08801,DBLP:conf/eccv/TianWZB24} get improved results with diffusion techniques. 
However, these methods still face challenges related to long generation times and significant computational resource demands. Audio-driven 2D control methods~\cite{DBLP:conf/cvpr/ZhangCWZSGSW23,DBLP:journals/corr/abs-2211-12368,DBLP:conf/iccv/GuoCLLBZ21} are easy to use but cannot explicitly control facial expressions and poses. 2D-based techniques often struggle with large pose or expression variations due to the lack of an explicit 3D structure, sometimes producing unrealistic distortions or identity changes. While some 2D methods~\cite{SadTalker,StyleHEAT,Pirenderer,DBLP:conf/cvpr/WangM021,MegaPortraits} incorporate 3D Morphable Models (3DMMs)~\cite{DBLP:conf/fgr/GerigMBELSV18,DBLP:journals/tog/LiBBL017,DBLP:conf/avss/PaysanKARV09,DBLP:conf/siggraph/BlanzV99} to mitigate these issues, they typically cannot achieve free-viewpoint rendering. 

\vspace{-0.1in}

\begin{figure*}[h]
    \centering
    \includegraphics[width=0.9\linewidth]{images/framework.pdf}
    \caption{\textbf{Overall Framework.} Our framework utilizes learnable query features attached to FLAME vertices to perform cross-attention with the extracted multi-level image features. The extracted features are then decoded to reconstruct the Gaussian avatar in the canonical space, which can be animated utilizing standard linear blend skinning (LBS) and corrective blendshapes as the FLAME model did and rendered in real-time on various platforms.}
    \label{fig:framework}
\end{figure*}

\paragraph{\bf Image to 3D Animatable Avatar.}
3D-aware methods offer improved geometric consistency and free-viewpoint rendering capabilities. Early 3D approaches~\cite{DBLP:conf/eccv/KhakhulinSLZ22,DBLP:conf/cvpr/XuYCWDJT20} utilize 3DMMs for head avatar reconstruction. With the advent of Neural Radiance Fields (NeRFs)~\cite{DBLP:conf/eccv/MildenhallSTBRN20}, many recent methods~\cite{DBLP:conf/siggraph/YuFZWYBCSWSW23,DBLP:conf/cvpr/MaZQLZ23,DBLP:conf/cvpr/LiZWZ0CZWB023,GPAvatar,ye2024real3d,deng2024portrait4d,deng2024portrait4d2,DBLP:conf/eccv/KiMC24,DBLP:conf/cvpr/BaiFWZSYS23,PointAvatar,Nerfies,INSTA} have adopted this representation for higher fidelity, particularly in modeling fine details like hair. However, NeRF-based~\cite{DBLP:conf/cvpr/ZhangZLHLWGCL024,HAvatar,DBLP:conf/cvpr/BaiTHSTQMDDOPTB23,AD-NeRF,DBLP:journals/tog/GaoZXHGZ22,DBLP:journals/tog/ParkSHBBGMS21,DBLP:conf/cvpr/AtharXSSS22,DBLP:journals/corr/abs-2112-05637,DBLP:conf/iccv/TretschkTGZLT21,DBLP:conf/cvpr/GafniTZN21,DBLP:conf/eccv/KiMC24,DBLP:conf/cvpr/BaiFWZSYS23,PointAvatar,Nerfies,DBLP:conf/siggraph/YuFZWYBCSWSW23,DBLP:conf/cvpr/MaZQLZ23,DBLP:conf/cvpr/LiZWZ0CZWB023} approaches often require extensive training data, including multi-view or single-view videos, raising privacy concerns and limiting generalization to unseen identities. Some methods~\cite{DBLP:conf/cvpr/SunWWLZZL23,DBLP:conf/3dim/ZhuangMKS22,DBLP:journals/pami/SunWZHWL24,DBLP:journals/tvcg/TangZYZCMW24,DBLP:conf/iclr/XuZLZBFS23} bypass this data requirement by training generators with random noise and then inverting them for identity-specific reconstruction, but inversion accuracy remains a challenge. Test-time optimization offers another alternative, but its computational cost limits practical applications. Several recent works~\cite{goha2023,hidenerf2023,gpavatar2024,ye2024real3d,ma2024cvthead,deng2024portrait4d,deng2024portrait4d2,GGHead} have explored one-shot 3D head reconstruction to address the limitations of data requirements and computational cost. These methods employ various techniques, such as tri-plane features, deformation fields, point-based expression fields, and vertex-feature transformers. Despite these advancements, NeRF-based methods often struggle with real-time rendering. 
Recently, 3D Gaussian Splatting~\cite{GaussianSplatting} has emerged as a promising alternative, offering both high-quality results and fast rendering speeds. However, existing Gaussian Splatting methods~\cite{GaussianAvatar,DBLP:conf/cvpr/XuCL00ZL24} typically rely on video data for training for each person, limiting their ability to generalize to new identities. Instead, the most recent work, GAGAvatar~\cite{GAGAvatar}, proposes a one-shot 3D Gaussian-based head avatar generation method. However, it still relies heavily on complex 2D neural post-processing to achieve optimal animation outcomes, thus it is not a pure 3D solution and the extra neural network hinders its application on various platforms. In contrast, our work generates Gaussian heads that are immediately animatable and renderable without additional networks or post-processing steps, enabling seamless integration into existing rendering pipelines for real-time animation and rendering across a wide range of platforms, including mobile phones.  % max 1 page
% % \newpage
\label{sec:method}

In this section, we introduce the method used to conduct the investigation on a set of \pc papers that discuss relevant bias issues.
Specifically, to construct the initial set of relevant work, we search the keywords ``bias" or ``fair" in the title of papers from NeurIPS, ICML, ICLR and FAccT published before February 2025. 
We include papers that discuss bias issues whose manifestation aligns with either Type I Bias or Type II Bias (we will detail the unification in~\cref{sec:unifying}).
We exclude papers that address other bias issues such as inductive bias~\cite{baxter2000model,zietlow2021demystifying}, implicit bias~\cite{fitzgerald2017implicit,camuto2021asymmetric}, selection bias~\cite{hernan2004structural,akbari2021recursive}, sampling bias~\cite{winship1992models,xu2022alleviating}, spectral bias~\cite{fang2024addressing}, exposure bias~\cite{li2024alleviating} or bias-variance~\cite{ha2024fine, chen2024on}.
Furthermore, to ensure we do not overlook any relevant papers without these keywords or from other prominent conferences such as CVPR, ICCV, and ECCV, we manually traversal the citation graph of the paper in the initial set and append the relevant papers that are either cited by or cite the papers in the initial set.






Once we identify the scope of the investigated papers, we read these papers to determine which type of bias they address by examining two aspects: problem statement and evaluation protocol.
We will elaborate on the criterion for categorizing papers into our definitions in~\cref{sec:unifying}.
To accommodate the recent emerging direction of addressing unlabeled and unknown bias, we enrich the taxonomy with an additional dimension about the status of attribute $A$.
As shown in~\cref{tab:taxonomy}, we count the number of papers in each category. 
Note that the total number is not equal to \pc since some papers address both types of biases.
We present the categorization list of all \pc investigated papers in Appendix.


\begin{table}[htbp]
\caption{The taxonomy of bias issues based on \pc papers.}
\label{tab:taxonomy}
\centering
\resizebox{0.45\textwidth}{!}{%

\begin{tabular}{lcccc}
\toprule
\multirow{2}{*}{Type of Bias} & \multicolumn{2}{c}{Attribute $A$} & \multirow{2}{*}{Papers} & \multirow{2}{*}{Examples}                                                   \\
\cmidrule(lr){2-3} 
                              & Known           & Labeled         &                         &                                                                             \\
                              \midrule
\multirow{3}{*}{Type I Bias}  & \cmark          & \cmark          & 253                     & \cite{DebFace,GAC,RL_RBN}                                                   \\
                              & \cmark          & \xmark          & -                       & -                                                                           \\
                              & \xmark          & \xmark          & -                       & -                                                                           \\
                              \midrule
\multirow{3}{*}{Type II Bias} & \cmark          & \cmark          & 246                     & \cite{learn_not_to_learn_Colored_MNIST,CSAD,End}                            \\
                              & \cmark          & \xmark          & 8                       & \cite{HEX_texture_bias1, ReBias_texture_bias2,rubi} \\
                              & \xmark          & \xmark          & 30                      & \cite{LfF_CelebA_Bias_conflicting,ECS,UBNet}                               \\
                              \midrule
Survey                        & -               & -               & 25                       & \cite{MLbias_survey,prediciton_quality_disparity,discussion_on_DP_EO}      \\
\bottomrule
\end{tabular}
}

\end{table}

 % 1.5 page including mix'n'match
% % \newpage
\begin{table}[t]
\centering
\caption{Results over the benchmark datasets. The mIoU is reported. %
}
\label{tab:sota_results}
\resizebox{\columnwidth}{!}{
\begin{tabular}{ccccccc}
\toprule
\textbf{Method} & \begin{tabular}[c]{@{}c@{}}Inference\\ Vocab. \end{tabular} & A-847 & PC-459 & A-150 & PC-59 & VOC-20 \\ \midrule
SAN \cite{xu2023side} & \checkmark & 12.4 & 15.7 & 27.5 & 53.8 & 94.0 \\
AttrSeg \cite{ma2024open} & \checkmark & -- & -- & -- & 56.3 & 91.6 \\
SCAN \cite{liu2024open} & \checkmark & 14.0 & 16.7 & 30.8 & \textbf{58.4} & \textbf{97.0} \\
EBSeg \cite{shan2024open} & \checkmark & 13.7 & 21.0 & 30.0 & 56.7 & 94.6 \\
SED \cite{xie2024sed} & \checkmark & 11.4 & 18.6 & 31.6 & 57.3 & 94.4 \\
CAT-Seg \cite{cho2024cat} & \checkmark & \textbf{16.0} & \textbf{23.8} & \textbf{31.8} & 57.5 & 94.6 \\ \midrule
CaSED + SAM \cite{conti2024vocabulary} & \xmark & -- & -- & 6.1 & 7.5 & 13.7 \\
CaSED + SAN \cite{conti2024vocabulary} & \xmark & -- & -- & 7.2 & 15.5 & 26.9 \\
DenseCaSED \cite{conti2024vocabulary} & \xmark & -- & -- & 8.6 & 13.4 & 20.5 \\
\textbf{Chick.-and-egg} (CaSED) & \xmark & 3.2 & 4.4 & 9.7 & \textbf{23.1} & \textbf{47.6} \\
\textbf{Chick.-and-egg} (RAM) & \xmark & \textbf{3.7} & \textbf{7.1} & \textbf{15.6} & 23.0 & 47.5  \\
\bottomrule
\end{tabular}
}
\end{table}

\section{Experiments}
\label{ch:results}

We conduct a comprehensive experimental analysis to investigate how different components affect VSS performance.
First, we evaluate the proposed two-stage approach on standard benchmarks to establish the baseline (\Cref{sec:benchmark}). We then present an in-depth analysis of the text encoder's behaviour and its impact on segmentation quality (\Cref{sec:text}). To better understand the relationship between the two-stages, we examine the image tagging accuracy and its influence on the segmentation task (\Cref{sec:tagging}). Finally, we study how different assignment thresholds in the evaluation protocol affect the reported performance (\Cref{sec:thresholds}).

\textbf{Implementation Details:}
The model is trained on the COCO-Stuff dataset \cite{caesar2018coco}, which contains 118k annotated images across 171 categories, following \cite{cho2024cat}. All results are based on CLIP \cite{radford2021learning} with a ViT-B/16 backbone. The image encoder and cost aggregation module are trained with per-pixel binary cross-entropy loss. 
The training parameters follow \cite{cho2024cat}. The batch size is 4, and models are trained for 80k iterations.
We performed image tagging and instance description using a frozen VLM model not trained on the testing dataset. More in detail, we examined the robustness of two models RAM \cite{zhang2024recognize} and Llava-1.6  \cite{liu2024llavanext}.

\textbf{Test Datasets:} The evaluation covers several datasets to ensure comprehensive testing. We used ADE20K \cite{zhou2019semantic} with both 150 and 847 class configurations, Pascal Context \cite{mottaghi2014role} with 59 and 459 class setups, and Pascal VOC \cite{everingham2010pascal} with its 20 classes. 

\begin{figure*}[t]
    \centering
    \resizebox{\textwidth}{!}{%
    \begin{tabular}{@{}ccccc@{}}
        
        
        \includegraphics[width=0.25\textwidth]{fig/qualitative/ADE_val_00000049_img.png} &
        \includegraphics[width=0.25\textwidth]{fig/qualitative_new/ADE_val_00000049_zeroseg.png} &
        \includegraphics[width=0.25\textwidth]{fig/qualitative/ADE_val_00000049_cased_labels_bigger.png} &
        \includegraphics[width=0.25\textwidth]{fig/qualitative/ADE_val_00000049_ours_labels_bigger.png} &
        \includegraphics[width=0.25\textwidth]{fig/qualitative/ADE_val_00000049_labels_bigger.png} \\[0.2cm]
        



        \includegraphics[width=0.25\textwidth] {fig/qualitative_new/ADE_val_00000683_img.png} &
        \includegraphics[width=0.25\textwidth] {fig/qualitative_new/ADE_val_00000683_zero_seg.png} &
        \includegraphics[width=0.25\textwidth]{fig/qualitative_new/ADE_val_00000683_real_image_cased.png} &
        \includegraphics[width=0.25\textwidth]{fig/qualitative_new/ADE_val_00000683_real_image.png} &
        \includegraphics[width=0.25\textwidth]{fig/qualitative_new/ADE_val_00000683_ground_truth.png} \\[0.2cm]


        \includegraphics[width=0.25\textwidth] {fig/qualitative_new/2007_008415_img.png} &
        \includegraphics[width=0.25\textwidth] {fig/qualitative_new/2007_008415_zeroseg.png} &
        \includegraphics[width=0.25\textwidth]{fig/qualitative_new/2007_008415_real_image_cased.png} &
        \includegraphics[width=0.25\textwidth]{fig/qualitative_new/2007_008415_real_image.png} &
        \includegraphics[width=0.25\textwidth]{fig/qualitative_new/2007_008415_ground_truth.png} \\[0.2cm]
        
        
        \textbf{Image} &
        \textbf{ZeroSeg} & 
        \textbf{Chicken-and-egg} (CaSED) & 
        \textbf{Chicken-and-egg} (RAM) & 
        \textbf{GT}
    \end{tabular}
    }
    \caption{Comparison of segmentation results across ZeroSeg \cite{rewatbowornwong2023zero} and \textbf{Chicken-and-Egg} (CaSED \cite{conti2024vocabulary} and RAM \cite{zhang2024recognize}), and ground-truth labels.}
    \label{fig:sota_qualitative_comparison}
\end{figure*}





\begin{table*}[t]
\centering
\caption{Results over the benchmark datasets by using soft assignment. † Results come from their original work. * mapped with Llama-2 \cite{ulger2024autovocabularysemanticsegmentation} rather than Sentence-BERT \cite{reimers2019sentence}. The soft assignment has threshold zero (i.e., all the words are assigned to a class in the evaluation vocabulary).}
\label{tab:mapping_results}
\resizebox{.99\textwidth}{!}{%
    \begin{tabular}{l|cc|cc|cccccccccc}
    \toprule
    \multirow{2}{*}{\textbf{Method}} & \multirow{2}{*}{\begin{tabular}[c]{@{}c@{}}\textbf{Vision}\\ \textbf{Backbone}\end{tabular}} & \multirow{2}{*}{\textbf{Stages}} &\multicolumn{2}{c}{\textbf{Components}} & A-847 & PC-459 & A-150 & PC-59 & VOC-20 \\
     & & & Tagging & Segmentation & &&&&& \\ \midrule
    Zero-Seg† \cite{rewatbowornwong2023zero} & ViT-B/16 & Mask2Tag & CLIP+GPT-2 & DINO & -- & -- & -- & 11.2 & 8.1 \\
    Auto-Seg† \cite{ulger2024autovocabularysemanticsegmentation} & ViT-L/16 & Tag2Mask & BLIP-2 & X-Decoder & 5.9* & -- & -- & 11.7* & \textbf{87.1}* \\
    TAG† \cite{kawano2024tag} & ViT-L/14 & Mask2Tag &CLIP+DB & DINO & -- & -- & 6.6 & 20.2 & 56.9 \\
    \textbf{Chicken-and-egg} (CaSED) & ViT-B/16 & Tag2Mask & CLIP+DB & CAT-Seg & 4.3 & 3.1 & 7.8 & \textbf{27.9} & 82.3 \\
    \textbf{Chicken-and-egg} (RAM) & ViT-B/16 & Tag2Mask & CLIP+Swin & CAT-Seg & \textbf{6.7} & \textbf{8.0} & \textbf{18.8} & 27.8 & 81.8 \\
        \bottomrule
    \end{tabular}
}
\end{table*}

\subsection{Benchmark Evaluation}\label{sec:benchmark}
We first conducted a comprehensive benchmark evaluation comparing existing approaches to establish a strong foundation for VSS and identify the most promising direction. This analysis served two key purposes: (1) to understand the current state-of-the-art performance in VSS and (2) to determine which baseline architecture would be the most suitable.

\textbf{Quantitatives:} \Cref{tab:sota_results} compares the mIoU across the Open-Vocabulary benchmarks. The proposed pipeline outperforms the previous VSS methods by a constant margin in all the datasets. To better accommodate VSS methods, they adopt a class remapping strategy that reduces penalization in cases where an exact class match is not found. This approach is reflected in \Cref{tab:mapping_results}, where the soft evaluation assignment takes place as described in \Cref{sec:assignment}.

\textbf{Qualitatives:} As shown in \Cref{fig:sota_qualitative_comparison}, the current approach fills the gap between the predictions and original dataset labels without a predefined vocabulary, offering finer-grained details across diverse scenarios (indoor and outdoor). %
The maps obtained suggest that current evaluation metrics might be overly pessimistic about the qualitative performance of the results. This issue arises from dataset limitations, where many instances struggle to find appropriate matches (e.g., in the third image, "husky" instead of "dog"). Mask2Tag methods like ZeroSeg \cite{rewatbowornwong2023zero} tend to over-segment the instances, getting improper text matches. On the other hand, Chicken-and-egg with CaSED tends to limit the number of predicted tags, %
while coupled with RAM it reaches the best compromise.




\subsection{Segmentatation Analysis} \label{sec:text}
\textbf{Perfect Tagger:} Our empirical results on the OVSS task - presented in \Cref{tab:gt_labels} - revealed that providing only image-specific text labels, %
rather than the entire vocabulary, during training led to improved segmentation performance when applying the same adjustment at inference. Although having access to inference labels is unrealistic, this setup represents the best achievable performance if tagger predictions were 100\% accurate. 
More in detail, in \Cref{tab:gt_labels}, the set of class names is defined for each batch as \(\mathcal{C}_b \subset \mathcal{C}\) during training, where \(\mathcal{C}_b\) represents the batch-specific subset of the entire class vocabulary \(\mathcal{C}\), dynamically selected based on the batch's unique context or requirements. This subset approach allows the model to focus on relevant classes without being overwhelmed by the entire vocabulary. However, we observed no gain when the text labels in inference are predicted from an image tagger. Nevertheless, this represents the upper bound currently obtainable with the state-of-the-art open-vocabulary method \cite{cho2024cat}. Moreover, we show in \Cref{tab:attvsadj} that when performing inference on perfect predictions (100\% accuracy from the tagger) we can boost performance by providing additional textual information.
\begin{table}[t]
\centering
\caption{\textbf{Comparison using CAT-Seg \cite{cho2024cat}, using ground truth classes as text embeddings at different stages}, where $T$ represents training and $I$ represents inference. The mIoU is reported on ADE-20K (A)\cite{zhou2019semantic}, Pascal Context (PC)\cite{mottaghi2014role}, and Pascal VOC (VOC) \cite{everingham2010pascal}.
}
\label{tab:gt_labels}
\resizebox{\columnwidth}{!}{%
\begin{tabular}{ccccccccc}
\toprule
\textbf{Method} & \multicolumn{2}{c}{\textbf{\begin{tabular}[c]{@{}c@{}}Only GT\\ Text Labels\end{tabular}}} & COCO & A-847 & PC-459 & A-150 & PC-59 & VOC-20 \\ \cmidrule{2-3}
 & T & I &  &  &  &  &  &  \\ \midrule
Base &  &  & 47.11 & 11.95 & 18.95 & 31.78 & 57.20 & 95.30 \\
L.Bound & \checkmark &  & 43.73 & 10.89 & 16.63 & 30.29 & 55.99 & 94.20 \\
U.Bound (I) &  & \checkmark & 56.15 & 12.38 & 18.38 & 45.53 & 69.77 & \textbf{95.87} \\
U.Bound & \checkmark & \checkmark & \textbf{64.03} & \textbf{13.98} & \textbf{24.04} & \textbf{51.21} & \textbf{72.79} & 94.38 \\
\bottomrule
\end{tabular}
}
\end{table}

\begin{table}[t]
\centering
\caption{All methods are based on \cite{cho2024cat}, changing textual descriptors, while performing inference on GT classes. (a)-(c) are trained using the predicted VLM information on COCO dataset.
}
\label{tab:attvsadj}
\resizebox{\columnwidth}{!}{%
\begin{tabular}{ccccccccc}
\toprule
\textbf{Method} & \multicolumn{2}{c}{\textbf{VLM input}} & COCO & A-847 & PC-459 & A-150 & PC-59 \\ \cmidrule{2-3}
& $Image$& $Text$ & & & \\ \midrule
Baseline \cite{cho2024cat} & & & 56.15 & 12.38 & 18.38 %
& 45.53 & 69.77 &\\ %
(a) Caption & \checkmark & & 58.17 & 12.71 & 17.07 %
& 47.09 & 71.04 &\\ %
(b) Class Adjectives & & \checkmark & 62.33 & 14.96 & 19.13 & 48.77 & 60.47 \\ %
(c) Instance Adjectives & \checkmark & \checkmark & \textbf{65.13} & \textbf{15.40} & \textbf{23.20} & \textbf{54.43} & \textbf{72.04} \\ %
\bottomrule
\end{tabular}
}
\end{table}

\begin{table*}[t]
\centering
\caption{Prompts for different algorithms for \cref{tab:attvsadj} results.}
\label{tab:prompt}
\resizebox{.9\textwidth}{!}{%
\begin{tabular}{cc}
\toprule
\textbf{\begin{tabular}[c]{@{}c@{}}Description\\ Level\end{tabular}} & \textbf{Prompts} \\ \midrule
Class & \begin{tabular}[c]{@{}c@{}}1. "Please group the classes in this list $<$dataset-class-list$>$ into groups of classes that are similar to each other \\ meaning they could be confused in an image. Every class should be in one group and only in one group. \\ Make sure there are no classes from the original list missing in your grouping. \\ This is an example of how the output should look: dog, cat, kitten, bird -- couch, desk, sofa, lamp -- knife, fork, plate"\\ 2. " The classes in the group are: $<$group$>$. Please generate a short list of adjectives for each class \\ that describe how the object looks in an image. The adjectives should be distinctive within each group meaning that \\ the same attribute should not appear for two classes in the same group. Generate at least one adjective for each class. \\ This is an example how the output should look. {giraffe: [tall, brown, spotted, yellow], tree: [tall, green], armchair: [comfortable]}\end{tabular}\\ \midrule
Instance & \begin{tabular}[c]{@{}c@{}}"The objects in the image are: $<$dataset-class-list$>$. Please generate a short list of adjectives\\ for each object that describes how the object looks in the image. \\ This is an example of how the output should look. \{giraffe: {[}tall, brown, spotted, interacting{]}, tree: {[}tall, green, leafy{]}\}"\end{tabular} \\ \bottomrule
\end{tabular}
}
\end{table*}

\textbf{Aiding Text Encoder with Descriptions:} Previous works \cite{ma2024open} used adjectives with the assumption to find the common class features that better describe each class. For example, a "dalmata" could be described as "a white dog with black spots". However, in typical recognition tasks, the categories are much broader, such as simply "dog", and a "dog" could be described very differently in terms of color and size. Hence, AttrSeg \cite{ma2024open} have focused on training strategies to find the optimal set of descriptions that could enhance class distinguishability while still being able to represent each class. While this approach has merit, it can result in the loss of fine-grained details. For instance, a "table" or "hat" could be of any size or color, and even a "wall" that is typically "white" could be "bricked" or some other texture.
Zhao et al. \cite{zhao2024gradient} experimented on CLIP's ability to identify different types of object attributes, including shape, material, color, size, and position. 
For shape and material attributes, CLIP showed a certain but limited knowledge, with the heat maps highlighting partial correct attention on obvious objects, but also exhibiting false positive and false negative errors. For color attributes, the results further verified that CLIP has a good ability to distinguish different colors.
For comparative attributes like size and position, CLIP produced some erroneous results, demonstrating that it relies more on the primary object (e.g., "cube", "red") rather than the comparative attribute (e.g., "small", "left"). Overall, their analysis suggests that CLIP has advantages with common perceptual attributes.
Therefore, we adopted a pre-trained VLM to find the corresponding descriptions given each image and its specific set of class names - the text labels of each image-, and we tried to enforce general language descriptions.
The prompts used are shown in \Cref{tab:prompt}. For generating captions, we employed the BLIP-2 model \cite{li2023blip} without any query input, whereas for the multimodal model, we utilized Llava-1.6 \cite{liu2024llavanext}. These models were selected because they both incorporate CLIP as their text encoder.
The text embedding of the captions is employed as a query within an additional cross-attention module, linking it to the embeddings of the classes. In the case of the adjectives, they are sampled and used within the template "A photo of a \{adjective\} \{class name\}".
We report the results in table \Cref{tab:attvsadj}, where adding image-specific content results beneficial, specially for large numbers of classes.
It is important to notice that, when using predicted labels from the image tagger or applying the complete set of image labels during inference, we did not observe the same benefit. %
In the VSS scenario, ambiguities with other classes are largely resolved during the CLIP segmentation stage by directly predicting the image's content using the image tagger. However, misclassifications may still occur at this stage, a behaviour explored in the next paragraph.
\begin{table}[t]
\centering
\caption{Class recognition accuracy of different VLMs with $T_\text{SBERT}$=$0.0$. \\ * using vocabulary.
\# FN = average number of missed classes, \# FP = average number of classes predicted but not in the ground truth.}
\label{tab:accuracy}
\resizebox{.5\textwidth}{!}{%
\begin{tabular}{ccccccccccc}
\toprule
\multirow{2}{*}{\textbf{\begin{tabular}[c]{@{}c@{}}Predicted\\ Classes\end{tabular}}} & \multirow{2}{*}{\textbf{\begin{tabular}[c]{@{}c@{}}Mapping\\ Model\end{tabular}}} & \multicolumn{3}{c}{A-150} & \multicolumn{3}{c}{PC-59} & \multicolumn{3}{c}{VOC-20} \\ \cmidrule{3-11} 
 &  & Acc & \#FP & \#FN & Acc & \#FP & \#FN & Acc & \#FP & \#FN \\ \midrule
CaSED & - & 10 & 10.7 & 7.8 & 22 & 9.3 & 4.0 & 50 & 9.5 & 0.9 \\
CaSED & SBERT & 23 & 7.4 & 6.8 & 42 & 5.4 & 3.1 & 84 & 4.2 & 0.3 \\
Llava-1.6 & - & 26 & 4.9 & 6.3 & 29 & 3.7 & 3.5 & 53 & \textbf{3.5} & 0.8 \\
Llava-1.6 & SBERT & 39 & \textbf{2.6} & 5.2 & 47 & \textbf{1.8} & 2.7 & 91 & 1.9 & 0.2 \\
RAM & - & 34 & 10.4 & 5.9 & 41 & 11.8 & 3.1 & 68 & 12.2 & 0.5 \\
RAM & SBERT & \textbf{46} & 5.7 & \textbf{4.8} & \textbf{61} & 5.4 & \textbf{2.2} & \textbf{96} & 4.8 & \textbf{0.1} \\
\midrule
RAM* & - & 79 & 16.7 & 1.95 & 80 & 6.2 & 1.1 & 97 & 1.5 & 0.1  \\
\bottomrule
\end{tabular}
}
\end{table}

\subsection{Image Tagging Analysis}\label{sec:tagging}
In \Cref{tab:accuracy}, we investigated various image tagging methods to understand how different types of errors affect the sensitivity of the segmentation module, particularly the text encoder since we use the tags as input to CLIP. We evaluated the impact of three architectures: a training-free method, CaSED \cite{conti2024vocabulary}, a multi-step trained method, RAM \cite{zhang2024recognize}, and a general-purpose multimodal model, Llava \cite{liu2024llavanext}.
CaSED %
uses a pre-trained vision-language model and an external database to extract candidate categories and assign the image to the best match. 
On the other hand, RAM %
generates large-scale image tags through automatic semantic parsing, followed by training a model to annotate images using both captioning and tagging tasks. A data engine then refines these annotations, and the model is retrained on this enhanced data, with final fine-tuning on a higher-quality subset.
\Cref{tab:accuracy} shows that using SBERT for evaluation avoids discarding words merely due to the absence of an exact match with the chosen word by the annotators. RAM achieves the best overall results across the evaluated datasets. In the table, the performance of Llava \cite{liu2024llavanext} %
demonstrates the versatility of powerful vision-language architectures. Note that the current baseline, RAM, does not reach a perfect accuracy even when the whole list of desired classes (i.e., non-vocabulary free), hence this represents the current limitation of such an approach. 
Furthermore, compared to CaSED, RAM demonstrates higher class recognition accuracy, but with more false positives on average. To investigate this further, we examined in \Cref{fig:miss_vs_false_sim} how the model is influenced by simulating a drop rate and false positives on top of the ground truth text classes in each image. In the table, the false positives are randomly selected from the vocabulary set. The influence of false negatives deeply influences the performance, while introducing false positives only leads to marginal degradation. These results confirm why RAM outperforms current alternatives: it has the fewest misclassifications, despite having a higher rate of false positives.

\begin{figure}
    \centering
    \includegraphics[width=.9\columnwidth, trim=0cm 0.55cm 0cm 0.75cm, clip]{fig/fpfn_full_stefano.png}
    \caption{Simulating missing classes or adding wrong ones over the OVSS baseline by assuming the labels are known at inference time.}
    \label{fig:miss_vs_false_sim}
\end{figure}


\subsection{Evaluation Assignment Thresholds} \label{sec:thresholds}
In \Cref{fig:thresh} we show the effect of providing different values for $T_\text{SBERT}$. Unlike Zero-Seg \cite{rewatbowornwong2023zero}, we did not observe a consistent trend in the optimal threshold across datasets. Respectively, $0.6-0.7$ for A-847, PC-459 and A-150, $0.5$ for PC-59 and $0.1$ for VOC-20. 
Our findings suggest that as the number of classes increases, we need to be more confident in the assignment, hence a higher threshold leads to a better score.

\begin{figure}
    \centering
    \includegraphics[width=.9\columnwidth, trim=0cm 0.55cm 0cm 0.75cm, clip]{fig/thresholds_stefano.png}
    \caption{Ablation over different thresholds for the evaluation mapping.%
    }
    \label{fig:thresh}
\end{figure}



 % 2 pages including tables and plots. 
% % \newpage
\section{Discussion}
\subsection{Case Study}


Fig. \ref{fig:casestudy} shows 2-D UMAP \cite{mcinnes2020umapuniformmanifoldapproximation} projections of embedding vectors for PetClinic Microservices \cite{microapps2024petclinic} using VoyageAI, ME-unixcoder-340K, and ME-llm2vec-340K. ME-unixcoder and ME-llm2vec show clearer microservice clusters compared to VoyageAI and Fig. \ref{fig:mexample}. For instance, \textit{API-Gateway} service classes are split in VoyageAI's representation but closer in the other models. ME-llm2vec demonstrates the closest grouping within microservices and clearest separation between them. In fact, ME-llm2vec's figure shows only 6 clear outliers which we review in detail and display their names and neighbors.



The two \textit{MetricConfig} classes, \textit{ResourceNotFoundException} and \textit{CacheConfig} lack domain-specific terms since they are utility classes, which highlights the importance of separating them from domain-related ones during the decomposition. However, ME-llm2vec was able correctly represent classes with even slight domain hints. For instance, most models struggle to differentiate between the nearly identical entry-point classes (e.g. \textit{ConfigServerApplication}), as seen in Fig. \ref{fig:mexample} and \ref{fig:casestudy} while ME-llm2vec managed to correctly place them within their services. On the other hand, the class \textit{PetRequest}, which was closer to \textit{API-Gateway} instead of \textit{Customers}, shows an intriguing outlier. Despite ME-llm2vec correctly matching the "Pet" related classes, it failed with \textit{PetRequest}. its function as a Request object, which is typically associated with the Gateway pattern, is a potential reason. Notably, ME-llm2vec successfully identified \textit{API-Gateway} classes, differentiating them from \textit{Customers}. We find this interesting because \textit{API-Gateway} includes classes representing various bounded contexts, often causing confusion in other models. ME-llm2vec recognized these classes' distinct purpose, grouping them together despite their diverse domains.

% Both \textit{API-Gateway} and \textit{Customers} services contain a "PetType" class. But in \textit{Customers}'s case, this class was closer to the "Specialty" class from \textit{Vets}, which is likely due to nearly identical source code they have. 

\subsection{Discussion}


We designed the analysis component to be as abstract as possible to accommodate the rapidly evolving representation learning landscape. As new and improved embedding models are published, they can be integrated with minimal effort. While our evaluation results show that with ME-LLM2Vec, we can generate highly cohesive and consistent decompositions, one of our objectives is to highlight the potential of Language Models in generating more efficient representations than traditional approaches for the decomposition problem. In fact, MonoEmbed is both a decomposition approach (when considering the full approach) and an embedding model (when using models such as ME-LLM2Vec). These models can be used to enrich existing decomposition approaches. For example, MicroMiner's CodeBERT \cite{trabelsi2023microminer} can be replaced with ME-LLM2Vec and the GNN based methods \cite{desai2021cogcn,yedida2023deeply,mathai2022chgnn,qian2023gdcdvf} can be extended by using ME-LLM2Vec as the encoder. In fact, it can be used as an additional representation type in approaches such as \cite{khaled2022hydecomp,qian2023gdcdvf}. These models can be even extended further by incorporating unstructured inputs (e.g. resources, configurations, documentation) and different PLs.




\subsection{Threats to Validity}
\subsubsection{Internal Validity}
Clustering algorithms and decomposition approaches have hyper-parameters that can affect performance on evaluation benchmarks. To mitigate this threat, we compared their performance with different hyper-parameter inputs across a varied set of evaluation applications.

\subsubsection{External Validity}
To address the threat of our approach to generalize on monolithic applications and PLs, we used a large set of monolithic and microservices applications from related work \cite{kalia2021mono2micro,khaled2022hydecomp,yedida2023deeply,jin2021fosci} to benchmark decomposition approaches. 

\subsubsection{Construct Validity}
This threat can potentially be in the form of the evaluation metrics used in our experiments. In order to mitigate this threat, we employ established metrics in supervised learning tasks (RQ1-3) and different metrics from decomposition research \cite{khaled2022hydecomp,kalia2021mono2micro,jin2021fosci,yedida2023deeply,mathai2022chgnn} (RQ4).  % 1 page for discussion and ablations
% !TEX root = template.tex

\section{Conclusion}
\label{sec:conclusion}
This work focuses on MAS coordination and synchronization under recurring LTL. We extended the bottom-up scheme for distributed motion and task coordination of MAS in \cite{meng_paper}, reducing computational complexity to enhance scalability and enable deployment on robotic hardware. The package was developed in ROS2, with a synchronization mechanism to handle action delays in experiments. Future work will focus on developing additional actions and incorporating human-in-the-loop scenarios.
 % 1 page buffer to fill in tables etc., 

\begin{table*}[!th]
\centering
\resizebox{\textwidth}{!}{%
\begin{tabular}{@{}llcccccccccc@{}}
\toprule
& & \multicolumn{2}{c}{\textbf{Intent Detection}} & \multicolumn{2}{c}{\textbf{Topic Mining}} & \multicolumn{2}{c}{\textbf{Domain Discovery}} & \multicolumn{1}{c}{\textbf{Type}} & \multicolumn{1}{c}{\textbf{Emotion}} & \\
\cmidrule(lr){3-4} \cmidrule(lr){5-6} \cmidrule(lr){7-8} \cmidrule(lr){9-9} \cmidrule(lr){10-10}  %\cmidrule(lr){11-11}
\textbf{Model} & \textbf{Method} & \textbf{BANKING} & \textbf{CLINC} & \textbf{Reddit} & \textbf{StackEx} & \textbf{MTOP} & \textbf{CLINC(D)} & \textbf{FewEvent} & \textbf{GoEmotion} & \textbf{AVG} \\ \midrule \midrule
GPT-4o-mini & Standard Prompting & 0.652 & 0.792 & 0.534 & 0.482 & 0.896 & 0.536 & 0.630 & 0.378 & 0.613 \\
& Self-Consistency & 0.666 & 0.802 & 0.586 & 0.494 & 0.902 & 0.530 & 0.640 & 0.382 & 0.625 \\
& TestNUC & 0.712 & 0.858 & 0.614 & 0.528 & 0.936 & 0.544 & 0.674 & 0.410 & 0.660 \\
& \cellcolor{gray!18}TestNUC\textdagger & \cellcolor{gray!18}\textbf{0.764} & \cellcolor{gray!18}\textbf{0.864} & \cellcolor{gray!18}\textbf{0.646} & \cellcolor{gray!18}\textbf{0.540} & \cellcolor{gray!18}\textbf{0.948} & \cellcolor{gray!18}\textbf{0.554} & \cellcolor{gray!18}\textbf{0.680} & \cellcolor{gray!18}\textbf{0.414} & \cellcolor{gray!18}\textbf{0.676} \\ \midrule \midrule
Llama-3.1-8B & Standard Prompting & 0.572 & 0.726 & 0.502 & 0.492 & 0.892 & 0.528 & 0.530 & 0.332 & 0.572 \\
& Self-Consistency & 0.620 & 0.774 & 0.564 & 0.526 & 0.902 & 0.518 & 0.564 & 0.340 & 0.601 \\
& TestNUC & 0.694 & 0.806 & 0.618 & 0.558 & 0.934 & 0.528 & 0.596 & 0.356 & 0.636 \\
& \cellcolor{gray!18}TestNUC\textdagger & \cellcolor{gray!18}\textbf{0.724} & \cellcolor{gray!18}\textbf{0.812} & \cellcolor{gray!18}\textbf{0.646} & \cellcolor{gray!18}\textbf{0.576} & \cellcolor{gray!18}\textbf{0.940} & \cellcolor{gray!18}\textbf{0.542} & \cellcolor{gray!18}\textbf{0.614} & \cellcolor{gray!18}\textbf{0.360} & \cellcolor{gray!18}\textbf{0.652} \\ \midrule \midrule
Claude-3-Haiku & Standard Prompting & 0.680 & 0.848 & 0.486 & 0.564 & 0.892 & 0.552 & 0.594 & 0.336 & 0.619 \\
& Self-Consistency & 0.702 & 0.870 & 0.510 & 0.578 & 0.904 & 0.564 & 0.568 & 0.350 & 0.631 \\
& TestNUC & 0.762 & 0.894 & 0.596 & 0.588 & 0.940 & 0.590 & 0.620 & 0.348 & 0.667 \\
& \cellcolor{gray!18}TestNUC\textdagger & \cellcolor{gray!18}\textbf{0.804} & \cellcolor{gray!18}\textbf{0.902} & \cellcolor{gray!18}\textbf{0.612} & \cellcolor{gray!18}\textbf{0.600} & \cellcolor{gray!18}\textbf{0.946} & \cellcolor{gray!18}\textbf{0.622} & \cellcolor{gray!18}\textbf{0.660} & \cellcolor{gray!18}\textbf{0.368} & \cellcolor{gray!18}\textbf{0.689} \\ \midrule \midrule
GPT-4o & Standard Prompting & 0.746 & 0.924 & 0.712 & 0.674 & 0.962 & 0.614 & 0.682 & 0.406 & 0.715 \\
& Self-Consistency & 0.758 & 0.922 & 0.720 & 0.688 & 0.958 & 0.624 & 0.696 & 0.426 & 0.724 \\
&TestNUC & 0.804 & 0.934 & 0.744 & \textbf{0.710} & 0.974 & 0.644 & 0.692 & 0.446 & 0.744 \\
& \cellcolor{gray!18}TestNUC\textdagger & \cellcolor{gray!18}\textbf{0.824} & \cellcolor{gray!18}\textbf{0.940} & \cellcolor{gray!18}\textbf{0.750} & \cellcolor{gray!18}\textbf{0.710} & \cellcolor{gray!18}\textbf{0.978} & \cellcolor{gray!18}\textbf{0.654} & \cellcolor{gray!18}\textbf{0.708} & \cellcolor{gray!18}\textbf{0.464} & \cellcolor{gray!18}\textbf{0.754} \\
\bottomrule
\end{tabular}%
}
\caption{Accuracy comparison with Standard Prompting and Self-Consistency across four diverse LLMs. TestNUC consistently improves the inference performance on all benchmark datasets. $\dagger$ denotes that 50 neighbors are utilized.}
\label{tab:main_compare_sc}
\end{table*}
\section{Errata}
\label{sec:errata}
In the first draft of the paper, we had a bug and used the following equation to train and quantize our models:
\begin{equation}
\label{eqn:slicing_old}
   S(q^c, r) = \left(\left\lfloor \frac{q^c}{2^{c - r}} \right\rceil\right) * 2^{c - r}
%   \vspace{-3mm}
\end{equation}
Equation~\ref{eqn:slicing_old} clearly allows an extra bucket to be included into the quantization range, i.e, a $r$-bit model would have $2^r + 1$ possible values instead of $2^r$. For example, consider slicing the first two MSBs from an unsigned int8 value, $234$. As per Equation~\ref{eqn:slicing}, $234$ first gets rounded to $4$, following which it gets clipped to $3$, and finally is scaled up to $3 * 64 = 192$ (Note that \alg int2 allows for $0$, $64$, $128$, $192$). However, since the clipping operation is missing in Equation~\ref{eqn:slicing_old}, $4$ is never clipped down to $3$, and $S(q^c, r)$ is now $4 * 64 = 256$ Thus, for certain int2 values in our final quantized model, we will have to store an extra bit. This is the case with int3, int4 and int6 as well where an extra bit is required to represent certain values. In Table~\ref{tab:old_omniquant}, we can see that the fraction of parameters that fall into this extra bucket is very small. However, for our 2-bit models, this additional bucket gives significant improvements in performance, for example, in Table~\ref{tab:old_omniquant} int2 Gemma-2 9B's average downstream accuracy goes up by $5$\% when trained with an additional bucket (referred to as \epalg in Table~\ref{tab:old_omniquant}). This number is further boosted to $6$\% with co-distillation, as evidenced by Table~\ref{tab:codistill_old}. We hypothesize that this additional bucket helps with capturing the outliers and thus leads to a significant performance boost. As highlighted by recent work~\citep{dettmers2023spqr, squeezellm}, it is crucial to store certain outliers full precision. Interestingly, we show that even a single bit is enough to capture several of these outliers, especially for low bit quantization. Finally, note that this performance boost is not very evident in higher precisions where there are enough buckets to account for the outliers.

\begin{figure}[!t]
  \centering
  \vspace{-2mm}
        \includegraphics[width=0.8\columnwidth]{TabsNFigs/mixnmatch_pdf_EP.pdf}
        % \vspace*{-5pt}
        \vspace{-3mm}
        \caption{Mix'n'Match on Gemma-2 9B model trained using \epalg with OmniQuant as the base algorithm allows elastic pareto-optimal accuracy-vs-cost model extraction for free during deployment.}
\label{fig:omniquant-mnm-old}
\vspace*{-3mm}
\end{figure}

\paragraph{Mix'n'Match} As shown in Figure~\ref{fig:omniquant-mnm-old} with a strong int2 model (i.e., 2.050 bits on average), \epalg Mix’n’Match densely spans the Pareto-optimal accuracy-vs-bits-per-FFN-parameter (memory/cost) trade-off for Gemma-2 9B model trained using MatQuant with Omni-Quant – sometimes even improving on the bfloat16 model accuracy. Consequently, hardware supporting only int2 and int4 data types can still accommodate a model with a memory footprint similar to that of an int3 quantized model, and quality comparable or superior to int3; the additional bits required in the case of int2 can be packed into int2/int4. However, custom CUDA kernel would be required to enable sparse additions of these additional bits to the model weights. 


\section*{Impact Statement}
This paper introduces a novel technique designed to advance the field of machine learning, specifically in the domain of model compression and efficient deployment for large language models. By enabling the creation of versatile, multi-scale models that can operate across various bit-widths, our work has the potential to democratize access to these powerful technologies by making them more resource-efficient and deployable on a wider range of hardware. This could lead to positive impacts such as more sustainable AI systems and greater accessibility for users with limited computational resources. While there are potential risks associated with the broad deployment of powerful AI systems, these are not unique to our work, and we believe the benefits of efficient and accessible AI through innovations like \alg have significant potential for societal good. We encourage further investigation into how novel quantization techniques can play a role in future sustainable AI development.
\section*{Acknowledgements}

This work was supported by LG AI Research. This work was partly supported by an IITP grant funded by the Korean Government (MSIT) (No. RS-2020-II201361, Artificial Intelligence Graduate School Program (Yonsei University)). This work was supported by the National Research Foundation of Korea (NRF) grant funded by the Korea government (MSIT) (RS-2024-00354218). This work was supported by Institute of Information \& communications Technology Planning \& Evaluation (IITP) grant funded by the Korea government (MSIT) (No. RS-2024-00457882, AI Research Hub Project).

\bibliography{local}

%%%%%%%%%%%%%%%%%%%%%%%%%%%%%%%%%%%%%%%%%%%%%%%%%%%%%%%%%%%%%%%%%%%%%%%%%%%%%%%
%%%%%%%%%%%%%%%%%%%%%%%%%%%%%%%%%%%%%%%%%%%%%%%%%%%%%%%%%%%%%%%%%%%%%%%%%%%%%%%
% APPENDIX
%%%%%%%%%%%%%%%%%%%%%%%%%%%%%%%%%%%%%%%%%%%%%%%%%%%%%%%%%%%%%%%%%%%%%%%%%%%%%%%
%%%%%%%%%%%%%%%%%%%%%%%%%%%%%%%%%%%%%%%%%%%%%%%%%%%%%%%%%%%%%%%%%%%%%%%%%%%%%%%
\newpage
\appendix
\onecolumn
\newpage
\appendix
\onecolumn

\renewcommand{\thetable}{A\arabic{table}} % Prefix table numbers with 'A'
\renewcommand{\thefigure}{A\arabic{figure}} % Prefix figure numbers with 'A'
\renewcommand{\theequation}{A\arabic{equation}} % Prefix equation numbers with 'A'

\setcounter{table}{0} % Reset table counter
\setcounter{figure}{0} % Reset figure counter
\setcounter{equation}{0} % Reset equation counter

\section*{Appendix}

\section{Optimal Brain Surgeon Derivation}
\label{OBS_ALGORITHM}

In the original setup in OBS, we have a local quadratic model for the loss $L$ given by:
$$
    \delta L = L(w + \delta w) \approx L(w) + \nabla_w L^T \delta w + \frac{1}{2} \delta w^T H \delta w
$$
Since OBS is a pruning-after-training approach, they discarded the 1-st order component. Reducing the expression for saliency as:
$$
    \delta L = \frac{1}{2} \delta w^T H \delta w
$$
To remove a single parameter, the authors of OBS introduced the constraint $e_q^T \delta w + w_q = 0$, with $e_q$ being the $q^{\text{th}}$ canonical basis vector. The pruning is defined as a constrained optimization problem of the form:
$$
    \min_{\delta w \in \mathbb{R^d}} \left( \frac{1}{2} \delta w^T H \delta w\right),
    ~~\text{s.t}~~
    e_q^T \delta w + w_q = 0.
$$
And the choice of which parameter to remove becomes:
$$
    \min_{q \in \mathcal{Q}} \left\{
        \min_{\delta w \in \mathbb{R^d}} \left( \frac{1}{2} \delta w^T H \delta w\right),
        ~~\text{s.t}~~
        e_q^T \delta w + w_q = 0
    \right\}.
$$
To solve the internal problem, we use a Lagrange multiplier $\lambda$ to write the problem as an unconstrained optimization case as follows:
$$
    \mathcal{L}(\delta w, \lambda) =
    \frac{1}{2} \delta w^T H \delta w +
    \lambda(e_q^T \delta w + w_q).
$$
Then, to find the stationary conditions, we compute the partial derivatives with respect to $\delta w$ and $\lambda$, and equate them to 0, obtaining:
$$
    \nabla_{\delta w} \mathcal{L} = 
    H \delta w + \lambda e_q = 0 
    \rightarrow
    \delta w = - \lambda H^{-1} e_q
$$
$$
    \nabla_{\lambda} \mathcal{L} =
    e_q^T \delta w + w_q = 0
    \rightarrow
    e_q^T \delta w = -w_q
$$
With some replacements, we get:
$$
    e_q^T \delta w = -w_q
    \rightarrow
    e_q^T \left( 
        - \lambda H^{-1} e_q
    \right) = -w_q
    \rightarrow
    - \lambda e_q^T H^{-1} e_q = -w_q
    \rightarrow
    \lambda = \frac{w_q}{e_q^T H^{-1} e_q} = \frac{w_q}{[H^{-1}]_{qq}}
$$
$$
    \delta w = - \frac{w_q H^{-1} e_q}{[H^{-1}]_{qq}}
$$
Replacing the expression for $\delta w$ in the saliency expression, we have:
\begin{align*}
    \delta L = \frac{1}{2} \delta w^T H \delta w
    &= \frac{1}{2}\left(
        - \frac{w_q H^{-1} e_q}{[H^{-1}]_{qq}}
    \right)^T
    H
    \left(
        - \frac{w_q H^{-1} e_q}{[H^{-1}]_{qq}}
    \right)
    \nonumber \\
    &= 
    \frac{w_q^2}{2[H^{-1}]_{qq}^2}
    \left(
        H^{-1} e_q
    \right)^T
    H
    \left(
        H^{-1} e_q
    \right)
    \nonumber \\
    &= 
    \frac{w_q^2}{2[H^{-1}]_{qq}^2}
    e_q ^T
    H^{-1}
    e_q
    = 
    \frac{w_q^2}{2[H^{-1}]_{qq}^2}
    [H^{-1}]_{qq}
    = 
    \frac{w_q^2}{2[H^{-1}]_{qq}}
    \nonumber \\
\end{align*}
%------------------------------------------------------------------------------------------------
\newpage
\section{Fisher Brain Surgeon Sensitivity Derivation}
\label{FBSS_ALGORITHM}
As we considered a PBT setting, it is not possible to ignore the first-order term in the local quadratic approximation of the error as it could still be informative. In this case, our model for sensitivity is given by: 
$$
    \delta L = \nabla_w L^T \delta w + \frac{1}{2} \delta w^T H \delta w
$$
The process to remove a single parameter remains similar; the constraint $e_q^T \delta w + w_q = 0$, with $e_q$ is still valid, redefining the optimization problem as:
$$
    \min_{\delta w \in \mathbb{R^d}} \left(
        \nabla_w L^T \delta w +  \frac{1}{2} \delta w^T H \delta w
    \right),
    ~~\text{s.t}~~
    e_q^T \delta w + w_q = 0.
$$
And the choice of which parameter to remove becomes:
$$
    \min_{q \in \mathcal{Q}} \left\{
        \min_{\delta w \in \mathbb{R^d}} \left(
            \nabla_w L^T \delta w + \frac{1}{2} \delta w^T H \delta w
        \right),
        ~~\text{s.t}~~
        e_q^T \delta w + w_q = 0
    \right\}.
$$
Using a Lagrange multiplier $\lambda$ as in the reference case, we solve the following unconstrained optimization problem:
$$
    \mathcal{L}(\delta w, \lambda) =
    \nabla_w L^T \delta w + 
    \frac{1}{2} \delta w^T H \delta w +
    \lambda(e_q^T \delta w + w_q).
$$
With the following stationary conditions:
$$
    \nabla_{\delta w} \mathcal{L} = 
    \nabla_w L + H \delta w + \lambda e_q = 0 
    \rightarrow
    \delta w = - (\lambda H^{-1}e_q + H^{-1} \nabla_w L)
$$
$$
    \nabla_{\lambda} \mathcal{L} =
    e_q^T \delta w + w_q = 0
    \rightarrow
    e_q^T \delta w = -w_q
$$
The expression for $\lambda$ is redefined as follows:
\begin{align*}
    e_q^T \left(
        - (\lambda H^{-1}e_q + H^{-1} \nabla_w L)
    \right) 
    &= -w_q
    \nonumber \\
    \lambda e_q^T H^{-1} e_q + e_q^T H^{-1} \nabla_w L
    &= w_q
    \nonumber \\
    \lambda [H^{-1}]_{qq} 
    &= w_q - e_q^T H^{-1} \nabla_w L
    \nonumber \\
    \lambda
    &= \frac{w_q - e_q^T H^{-1} \nabla_w L}{[H^{-1}]_{qq}}
\end{align*}
Replacing the expression for $\delta w$ in our sensitivity expression, we have:
\begin{align*}
    \delta L = \nabla_w L^T \delta w + \frac{1}{2} \delta w^T H \delta w
    &= 
    \nabla_w L^T \left[
        - (\lambda H^{-1}e_q + H^{-1} \nabla_w L)
    \right]
    \nonumber \\
    &+
    \frac{1}{2}\left[
        - (\lambda H^{-1}e_q + H^{-1} \nabla_w L)
    \right]^T
    H
    \left[
        - (\lambda H^{-1}e_q + H^{-1} \nabla_w L)
    \right]
    \nonumber \\
    &= 
    - \lambda \nabla_w L^T H^{-1}e_q - \nabla_w L^T H^{-1} \nabla_w L
    \nonumber \\
    &+
    \frac{1}{2}\left[
        (\lambda H^{-1}e_q)^T + (H^{-1} \nabla_w L)^T
    \right]
    \left[
        \lambda H H^{-1}e_q + H H^{-1} \nabla_w L)
    \right]
    \nonumber \\
    &= 
    - \lambda \nabla_w L^T H^{-1}e_q - \nabla_w L^T H^{-1} \nabla_w L
    \nonumber \\
    &+
    \frac{1}{2}\left[
        (\lambda H^{-1}e_q)^T + (H^{-1} \nabla_w L)^T
    \right]
    \left[
        \lambda e_q + \nabla_w L
    \right]
    \nonumber \\
    &= 
    - \lambda \nabla_w L^T H^{-1}e_q - \nabla_w L^T H^{-1} \nabla_w L
    \nonumber \\
    &+
    \frac{1}{2}\left[
        (\lambda H^{-1}e_q)^T \lambda e_q
        + (H^{-1} \nabla_w L)^T \lambda e_q
        + (\lambda H^{-1}e_q)^T \nabla_w L
        + (H^{-1} \nabla_w L)^T \nabla_w L
    \right]
    \nonumber \\
    &= 
    - \lambda \nabla_w L^T H^{-1}e_q - \nabla_w L^T H^{-1} \nabla_w L
    \nonumber \\
    &+
    \frac{1}{2}\left[
        \lambda^2 e_q^T H^{-1} e_q
        + \lambda \nabla_w L^T H^{-1} e_q
        + \lambda e_q^T H^{-1} \nabla_w L
        + \nabla_w L^T H^{-1} \nabla_w L
    \right]
    \nonumber \\
    &= 
    \frac{1}{2}\left[
        \lambda^2 [H^{-1}]_{qq}
        - \lambda \nabla_w L^T H^{-1} e_q
        + \lambda e_q^T H^{-1} \nabla_w L
        - \nabla_w L^T H^{-1} \nabla_w L
    \right]
    \nonumber \\
\end{align*}
Finally, replacing the $\lambda$:
\begin{align*}
    \delta L 
    &= 
    \frac{1}{2}\left[
        \lambda^2 [H^{-1}]_{qq}
        - \lambda \nabla_w L^T H^{-1} e_q
        + \lambda e_q^T H^{-1} \nabla_w L
        - \nabla_w L^T H^{-1} \nabla_w L
    \right]
    \nonumber \\
    &= 
    \frac{1}{2[H^{-1}]_{qq}}\left[
        (w_q - e_q^T H^{-1} \nabla_w L)^2 
        + (w_q - e_q^T H^{-1} \nabla_w L)(e_q^T H^{-1} \nabla_w L - \nabla_w L^T H^{-1} e_q)
        - \nabla_w L^T H^{-1} \nabla_w L
    \right]
    \nonumber \\
    &= 
    \frac{1}{2[H^{-1}]_{qq}}[
        w_q^2
        - 2 w_q (e_q^T H^{-1} \nabla_w L)
        + (e_q^T H^{-1} \nabla_w L)^2
        + w_q (e_q^T H^{-1} \nabla_w L)
    \nonumber \\
        &- w_q (\nabla_w L^T H^{-1} e_q)
        - (e_q^T H^{-1} \nabla_w L)(e_q^T H^{-1} \nabla_w L)
        + (e_q^T H^{-1} \nabla_w L)(\nabla_w L^T H^{-1} e_q)
        - \nabla_w L^T H^{-1} \nabla_w L
    ]
    \nonumber \\
    &= 
    \frac{1}{2[H^{-1}]_{qq}}[
        w_q^2
        - w_q (e_q^T H^{-1} \nabla_w L)
        + (e_q^T H^{-1} \nabla_w L)^2
    \nonumber \\
        &- w_q (\nabla_w L^T H^{-1} e_q)
        - (e_q^T H^{-1} \nabla_w L)^2
        + (e_q^T H^{-1} \nabla_w L)(\nabla_w L^T H^{-1} e_q)
        - \nabla_w L^T H^{-1} \nabla_w L
    ]
    \nonumber \\
    &= 
    \frac{1}{2[H^{-1}]_{qq}}\left[
        w_q^2
        - 2 w_q (e_q^T H^{-1} \nabla_w L)
        + (e_q^T H^{-1} \nabla_w L)^2
        - \nabla_w L^T H^{-1} \nabla_w L
    \right]
    \nonumber \\
    &= 
    \frac{1}{2[\hat{F}^{-1}]_{qq}}
    \left[
        w_q - (e_q^T \hat{F}^{-1} \nabla \mathcal{L}(w_0))
    \right]^2
\end{align*}

%------------------------------------------------------------------------------------------------

\newpage
\section{Training and Testing Details}
\label{appendix:training_parameters}

We perform an 80:20 stratified split, with a constant seed, on the CIFAR10/100 training dataset to obtain a validation set with the same class distribution. For both datasets, we have a training set with 40,000 samples, a validation set with 10,000 samples, and a testing set of 10,000 samples. Validation is performed after each training step, and the weights of the best-performing validation step (based on top-1 accuracy) are utilized for the final evaluation on the testing set. Table \ref{tab:table_training_parameters} summarizes the training parameters.

\begin{table}[h]
\caption{Training parameters used for ResNet18 and VGG19 on the CIFAR-10/100 datasets.}
\label{tab:table_training_parameters}
\vskip 0.15in
\begin{center}
\begin{small}
\begin{sc}
\begin{tabular}{lcc}
\toprule
Parameter & ResNet18 & VGG19 \\
\midrule
Number of steps       & 160 & 160 \\
Criterion             & CE & CE \\
Optimizer             & SGD & SGD \\
Learning rate         & 0.01 & 0.1 \\
Momentum              & 0.9 & 0.9 \\
Weight decay          & $5 \times 10^{-4}$ & $1 \times 10^{-4}$ \\
Learning rate drops   & [60, 120] & [60, 120] \\
Learning rate drop factor & 0.2 & 0.1 \\
\bottomrule
\end{tabular}
\end{sc}
\end{small}
\end{center}
\vskip -0.1in
\end{table}

%------------------------------------------------------------------------------------------------

\newpage
\section{Results CIFAR10}
\subsection{ResNet18}
\label{appendix:CIFAR10_ResNet18}

\begin{table}[h]
\caption{Performance of different sensitivity methods for pruning evaluated using ResNet18 on the CIFAR-10 testset. The right side of the table presents our proposed criteria. The mean accuracy and standard deviation are reported across three initialization seeds for various sparsity levels. Baseline, no pruning: $91.78 \pm 0.09$.}
\label{tab:resnet18_cifar10_compressors}
\vskip 0.15in
\begin{center}
\begin{small}
\begin{sc}
\resizebox{\textwidth}{!}{%
\begin{tabular}{lccccc|cccc}
\toprule
Sparsity  & Random & Magnitude & GN & SNIP & GraSP & FD & FP & FTS & FBSS \\
\midrule
0.10  & 91.71 ± 0.21 & 91.72 ± 0.07 & 91.57 ± 0.15 & 91.72 ± 0.07 & 89.16 ± 0.05 & 91.87 ± 0.13 & 91.63 ± 0.21 & 91.53 ± 0.12 & 91.76 ± 0.08 \\
0.20  & 91.63 ± 0.11 & 91.42 ± 0.12 & 91.51 ± 0.09 & 91.64 ± 0.16 & 88.69 ± 0.34 & 91.50 ± 0.12 & 91.65 ± 0.14 & 91.53 ± 0.15 & 91.54 ± 0.13 \\
0.30  & 91.45 ± 0.18 & 91.61 ± 0.13 & 91.68 ± 0.20 & 91.65 ± 0.08 & 88.67 ± 0.26 & 91.65 ± 0.18 & 91.44 ± 0.27 & 91.49 ± 0.05 & 91.62 ± 0.07 \\
0.40  & 91.59 ± 0.18 & 91.06 ± 0.16 & 91.61 ± 0.09 & 91.55 ± 0.08 & 88.24 ± 0.33 & 91.51 ± 0.05 & 91.38 ± 0.13 & 91.56 ± 0.28 & 91.39 ± 0.05 \\
0.50  & 91.60 ± 0.06 & 91.32 ± 0.13 & 91.44 ± 0.13 & 91.22 ± 0.07 & 87.69 ± 0.15 & 91.30 ± 0.18 & 91.58 ± 0.16 & 91.46 ± 0.19 & 91.41 ± 0.05 \\
0.60  & 91.10 ± 0.16 & 91.18 ± 0.16 & 91.59 ± 0.13 & 91.24 ± 0.04 & 87.48 ± 0.55 & 91.34 ± 0.07 & 91.35 ± 0.16 & 91.40 ± 0.11 & 91.38 ± 0.18 \\
0.70  & 91.17 ± 0.04 & 91.07 ± 0.07 & 91.19 ± 0.17 & 91.33 ± 0.18 & 87.26 ± 0.34 & 91.34 ± 0.23 & 91.42 ± 0.23 & 91.18 ± 0.18 & 91.27 ± 0.14 \\
0.80  & 90.78 ± 0.08 & 91.10 ± 0.12 & 90.95 ± 0.35 & 90.74 ± 0.10 & 87.18 ± 0.51 & 90.95 ± 0.11 & 91.08 ± 0.06 & 90.94 ± 0.22 & 90.73 ± 0.33 \\
0.90  & 89.35 ± 0.13 & 89.88 ± 0.28 & 90.39 ± 0.23 & 90.36 ± 0.34 & 86.60 ± 0.51 & 90.04 ± 0.21 & 90.20 ± 0.08 & 90.55 ± 0.23 & 89.22 ± 0.30 \\
0.95  & 87.59 ± 0.11 & 89.23 ± 0.19 & 89.00 ± 0.05 & 89.31 ± 0.17 & 86.50 ± 0.05 & 88.61 ± 0.28 & 89.50 ± 0.18 & 89.47 ± 0.32 & 87.58 ± 0.25 \\
0.98  & 83.47 ± 0.20 & 85.70 ± 0.33 & 86.43 ± 0.05 & 87.26 ± 0.28 & 85.99 ± 0.08 & 85.61 ± 0.20 & 86.97 ± 0.22 & 87.24 ± 0.32 & 83.40 ± 0.74 \\
0.99  & 78.28 ± 0.45 & 71.99 ± 0.28 & 83.47 ± 0.15 & 84.54 ± 0.04 & 84.56 ± 0.46 & 82.13 ± 0.28 & 83.74 ± 0.48 & 84.85 ± 0.18 & 77.60 ± 1.02 \\
\bottomrule
\end{tabular}}
\end{sc}
\end{small}
\end{center}
\vskip -0.1in
\end{table}

%------------------------------------------------------------------------------------------------
\clearpage
\subsection{VGG19}
\label{appendix:CIFAR10_VGG19}

As discussed earlier, introducing a warm-up phase effectively mitigates layer collapse in data-dependent pruning methods. Here, we evaluate the impact of different warm-up durations by comparing no warm-up, a single warm-up epoch, and five warm-up epochs. Table \ref{tab:VGG19_cifar10_compressors} demonstrates how performance drastically degrades with increasing sparsity, ultimately leading to layer collapse at 0.90 sparsity. However, as shown in the results, a single warm-up epoch is sufficient to prevent collapse and stabilize pruning performance. Moreover, as seen in Table \ref{tab:VGG19_cifar10_compressors_warmup5}, increasing the warm-up period to five epochs provides no substantial additional improvement. This indicates that prolonged warm-up training is not necessary; a single training step is enough to achieve gradient stabilization and overcome layer collapse.

\begin{table}[h]
\caption{Performance of different sensitivity methods for pruning evaluated using VGG19 on the CIFAR-10 test set. The right side of the table presents our proposed criteria. The mean accuracy and standard deviation are reported across three initialization seeds for various sparsity levels. Baseline, no pruning: $89.21 \pm 0.22$.}
\label{tab:VGG19_cifar10_compressors}
\vskip 0.15in
\begin{center}
\begin{small}
\begin{sc}
\resizebox{\textwidth}{!}{%
\begin{tabular}{lccccc|cccc}
\toprule
Sparsity  & Random & Magnitude & GN & SNIP & GraSP & FD & FP & FTS & FBSS \\
\midrule
0.10  & 88.40 ± 0.95 & 89.12 ± 0.55 & 90.14 ± 0.10 & 90.16 ± 0.18 & 87.81 ± 1.66 & 90.20 ± 0.29 & 90.21 ± 0.37 & 90.25 ± 0.38 & 89.06 ± 0.75 \\
0.20  & 89.19 ± 0.22 & 89.65 ± 0.60 & 89.59 ± 0.69 & 90.06 ± 0.04 & 89.57 ± 0.34 & 89.91 ± 0.28 & 90.28 ± 0.55 & 89.80 ± 0.28 & 88.89 ± 0.76 \\
0.30  & 88.93 ± 0.83 & 88.77 ± 1.07 & 90.23 ± 0.09 & 89.88 ± 0.59 & 89.14 ± 0.19 & 90.25 ± 0.09 & 89.97 ± 0.26 & 90.46 ± 0.41 & 89.06 ± 0.36 \\
0.40  & 88.28 ± 1.08 & 89.38 ± 0.53 & 90.50 ± 0.23 & 89.79 ± 0.67 & 88.20 ± 0.31 & 90.51 ± 0.12 & 90.37 ± 0.24 & 90.23 ± 0.14 & 10.00 ± 0.00 \\
0.50  & 88.96 ± 0.82 & 89.03 ± 0.59 & 90.46 ± 0.60 & 90.38 ± 0.25 & 88.67 ± 0.23 & 89.54 ± 0.86 & 90.47 ± 0.52 & 90.19 ± 0.31 & 10.00 ± 0.00 \\
0.60  & 88.15 ± 0.68 & 89.47 ± 0.18 & 89.95 ± 0.30 & 90.32 ± 0.25 & 88.82 ± 0.32 & 90.02 ± 0.40 & 90.18 ± 0.33 & 90.14 ± 0.36 & 10.00 ± 0.00 \\
0.70  & 88.02 ± 0.53 & 89.63 ± 0.44 & 89.69 ± 0.42 & 89.23 ± 0.19 & 89.62 ± 0.81 & 89.85 ± 0.08 & 90.01 ± 0.34 & 10.00 ± 0.00 & 10.00 ± 0.00 \\
0.80  & 88.28 ± 0.34 & 89.62 ± 0.91 & 85.72 ± 0.63 & 89.39 ± 0.43 & 88.82 ± 0.14 & 10.00 ± 0.00 & 88.29 ± 0.11 & 10.00 ± 0.00 & 10.00 ± 0.00 \\
0.90  & 85.82 ± 0.19 & 89.29 ± 0.79 & 10.00 ± 0.00 & 80.85 ± 0.62 & 24.28 ± 20.2 & 10.00 ± 0.00 & 10.00 ± 0.00 & 10.00 ± 0.00 & 10.00 ± 0.00 \\
0.95  & 84.41 ± 0.05 & 10.00 ± 0.00 & 10.00 ± 0.00 & 10.00 ± 0.00 & 10.00 ± 0.00 & 10.00 ± 0.00 & 10.00 ± 0.00 & 10.00 ± 0.00 & 10.00 ± 0.00 \\
0.98  & 80.04 ± 0.90 & 10.00 ± 0.00 & 10.00 ± 0.00 & 10.00 ± 0.00 & 10.00 ± 0.00 & 10.00 ± 0.00 & 10.00 ± 0.00 & 10.00 ± 0.00 & 10.00 ± 0.00 \\
0.99  & 76.89 ± 0.26 & 10.00 ± 0.00 & 10.00 ± 0.00 & 10.00 ± 0.00 & 10.00 ± 0.00 & 10.00 ± 0.00 & 10.00 ± 0.00 & 10.00 ± 0.00 & 10.00 ± 0.00 \\
\bottomrule
\end{tabular}}
\end{sc}
\end{small}
\end{center}
\vskip -0.1in
\end{table}
\newpage
%------------------------------------------------------------------------------------------------
\begin{table*}[h]
\caption{Performance of different compression methods evaluated after 1 warmup epoch using VGG19 on the CIFAR-10 dataset. We report the mean accuracy between three initialization seeds across various sparsity levels. Baseline, no pruning: $89.21 \pm 0.22$.}
\label{tab:VGG19_cifar10_compressors_warmup1}
\vskip 0.15in
\begin{center}
\begin{small}
\begin{sc}
\resizebox{\textwidth}{!}{%
\begin{tabular}{lccccc|cccc}
\toprule
Sparsity  & Random & Magnitude & GN & SNIP & GraSP & FD & FP & FTS & FBSS \\
\midrule
0.80  & 88.73 ± 0.38 & 88.35 ± 0.54 & 86.76 ± 0.27 & 87.39 ± 0.66 & 87.24 ± 0.25 & 87.14 ± 0.45 & 87.00 ± 0.87 & 87.68 ± 0.33 & 64.33 ± 15.91 \\
0.90  & 87.26 ± 0.42 & 88.62 ± 0.49 & 85.96 ± 0.75 & 86.75 ± 0.76 & 87.47 ± 0.33 & 86.69 ± 0.72 & 87.09 ± 0.31 & 87.42 ± 0.21 & 46.16 ± 7.62 \\
0.95  & 85.47 ± 0.64 & 87.68 ± 0.49 & 86.66 ± 0.27 & 86.00 ± 1.10 & 86.71 ± 1.24 & 85.71 ± 1.35 & 86.73 ± 0.36 & 87.56 ± 0.62 & 46.30 ± 5.32 \\
0.98  & 80.44 ± 0.30 & 86.61 ± 0.62 & 84.72 ± 1.69 & 87.22 ± 0.23 & 86.45 ± 0.64 & 80.34 ± 6.43 & 86.07 ± 0.39 & 86.36 ± 0.29 & 49.05 ± 4.31 \\
0.99  & 77.24 ± 0.73 & 83.69 ± 1.36 & 80.28 ± 2.04 & 83.49 ± 1.77 & 85.39 ± 0.43 & 75.11 ± 7.80 & 84.40 ± 1.27 & 85.35 ± 1.05 & 47.10 ± 4.41 \\
\bottomrule
\end{tabular}}
\end{sc}
\end{small}
\end{center}
\vskip -0.1in
\end{table*} 
%------------------------------------------------------------------------------------------------

\begin{table}[h]
\caption{Performance of different sensitivity methods for pruning evaluated after 5 warmup epochs using VGG19 on the CIFAR-10 testset. The right side of the table presents our proposed criteria. The mean accuracy and standard deviation are reported across three initialization seeds for various sparsity levels. Baseline, no pruning: $89.21 \pm 0.22$.}
\label{tab:VGG19_cifar10_compressors_warmup5}
\vskip 0.15in
\begin{center}
\begin{small}
\begin{sc}
\resizebox{\textwidth}{!}{%
\begin{tabular}{lccccc|cccc}
\toprule
Sparsity  & Random & Magnitude & GN & SNIP & GraSP & FD & FP & FTS & FBSS \\
\midrule
0.80  & 88.84 ± 0.43 & 88.41 ± 0.47 & 87.58 ± 0.52 & 88.15 ± 1.09 & 86.77 ± 1.14 & 87.28 ± 0.90 & 88.22 ± 0.82 & 86.68 ± 0.61 & 70.52 ± 9.25 \\
0.90  & 87.56 ± 0.62 & 88.60 ± 0.93 & 86.73 ± 0.37 & 87.89 ± 0.25 & 87.10 ± 0.47 & 87.50 ± 1.42 & 88.18 ± 0.47 & 86.98 ± 0.14 & 47.78 ± 1.26 \\
0.95 & 85.51 ± 0.69 & 87.66 ± 1.19 & 87.44 ± 0.46 & 87.71 ± 0.82 & 87.05 ± 0.16 & 86.83 ± 1.47 & 87.36 ± 0.52 & 87.00 ± 0.74 & 48.83 ± 2.52 \\
0.98 & 82.09 ± 0.17 & 86.24 ± 0.52 & 84.66 ± 1.33 & 86.55 ± 0.84 & 86.04 ± 0.66 & 85.44 ± 0.64 & 86.64 ± 0.13 & 84.89 ± 0.51 & 49.48 ± 0.85 \\
0.99 & 77.22 ± 1.03 & 83.93 ± 1.80 & 81.62 ± 2.17 & 84.53 ± 0.70 & 81.33 ± 5.77 & 81.71 ± 1.41 & 85.02 ± 0.69 & 83.78 ± 0.80 & 41.24 ± 1.55 \\
\bottomrule
\end{tabular}}
\end{sc}
\end{small}
\end{center}
\vskip -0.1in
\end{table}

%------------------------------------------------------------------------------------------------

\newpage
\section{Results CIFAR100}
\subsection{ResNet18}
\label{sec:resnet_cifar-100}

CIFAR-100 results exhibit a similar trend to those observed on CIFAR-10, further reinforcing the robustness of our proposed Fisher-Taylor Sensitivity (FTS) criterion. Across all evaluated sparsity levels, FTS consistently maintains strong performance, frequently ranking among the top-performing methods. This trend is particularly evident at extreme sparsities, where many pruning approaches suffer significant performance degradation. The stability of FTS across both datasets highlights its effectiveness in preserving network expressivity despite aggressive pruning.

\begin{table}[h]
\caption{Performance of different compression methods evaluated using ResNet18 on the CIFAR-100 dataset. We report the mean accuracy between three initialization seeds across various sparsity levels. Baseline, no pruning: $69.57 \pm 0.19$.}
\label{tab:resnet18_cifar100_compressors}
\vskip 0.15in
\begin{center}
\begin{small}
\begin{sc}
\resizebox{\textwidth}{!}{%
\begin{tabular}{lccccc|cccc}
\toprule
Sparsity  & Random & Magnitude & GN & SNIP & GraSP & FD & FP & FTS & FBSS \\
\midrule
0.10  & 69.16 ± 0.11 & 69.37 ± 0.14 & 69.63 ± 0.34 & 69.42 ± 0.07 & 64.26 ± 0.27 & 69.66 ± 0.30 & 69.08 ± 0.21 & 69.16 ± 0.11 & 69.07 ± 0.10 \\
0.20  & 69.16 ± 0.30 & 69.06 ± 0.24 & 69.19 ± 0.11 & 69.30 ± 0.08 & 63.28 ± 0.58 & 69.60 ± 0.30 & 69.35 ± 0.35 & 69.41 ± 0.43 & 69.07 ± 0.20 \\
0.30  & 69.36 ± 0.18 & 68.58 ± 0.36 & 69.37 ± 0.13 & 68.82 ± 0.17 & 62.02 ± 0.43 & 69.24 ± 0.40 & 68.84 ± 0.13 & 68.80 ± 0.55 & 68.96 ± 0.11 \\
0.40  & 69.41 ± 0.20 & 68.50 ± 0.29 & 69.16 ± 0.26 & 68.95 ± 0.19 & 61.18 ± 0.19 & 69.17 ± 0.16 & 68.88 ± 0.25 & 69.02 ± 0.21 & 68.92 ± 0.25 \\
0.50  & 69.12 ± 0.46 & 68.17 ± 0.20 & 68.94 ± 0.20 & 68.63 ± 0.11 & 61.11 ± 0.40 & 69.13 ± 0.13 & 68.68 ± 0.12 & 68.71 ± 0.12 & 68.71 ± 0.57 \\
0.60  & 68.66 ± 0.27 & 67.78 ± 0.35 & 68.77 ± 0.17 & 68.63 ± 0.42 & 61.40 ± 0.78 & 68.34 ± 0.43 & 67.98 ± 0.23 & 68.41 ± 0.14 & 68.60 ± 0.15 \\
0.70  & 67.95 ± 0.43 & 67.51 ± 0.24 & 68.29 ± 0.39 & 68.08 ± 0.18 & 59.43 ± 0.76 & 68.03 ± 0.46 & 67.96 ± 0.15 & 68.29 ± 0.06 & 68.16 ± 0.07 \\
0.80  & 67.26 ± 0.48 & 66.55 ± 0.19 & 67.20 ± 0.37 & 67.21 ± 0.38 & 59.08 ± 0.22 & 66.70 ± 0.05 & 67.05 ± 0.06 & 66.77 ± 0.65 & 66.62 ± 0.43 \\
0.90  & 64.75 ± 0.16 & 64.48 ± 0.18 & 64.87 ± 0.27 & 65.70 ± 0.08 & 59.16 ± 0.91 & 64.74 ± 0.44 & 65.46 ± 0.30 & 65.41 ± 0.13 & 63.90 ± 0.31 \\
0.95  & 61.01 ± 0.32 & 62.20 ± 0.06 & 62.20 ± 0.23 & 63.20 ± 0.20 & 57.91 ± 0.09 & 62.14 ± 0.42 & 63.22 ± 0.25 & 63.21 ± 0.47 & 61.25 ± 0.44 \\
0.98  & 54.72 ± 0.22 & 55.44 ± 0.18 & 57.34 ± 0.31 & 58.83 ± 0.35 & 54.85 ± 0.35 & 55.57 ± 0.17 & 58.05 ± 0.18 & 58.59 ± 0.12 & 55.02 ± 0.34 \\
0.99  & 45.62 ± 0.55 & 40.39 ± 0.36 & 50.46 ± 0.61 & 52.96 ± 0.10 & 49.13 ± 0.19 & 48.02 ± 0.32 & 49.98 ± 0.60 & 52.85 ± 0.24 & 44.91 ± 0.52 \\
\bottomrule
\end{tabular}}
\end{sc}
\end{small}
\end{center}
\vskip -0.1in
\end{table}

%------------------------------------------------------------------------------------------------
\clearpage
\subsection{VGG19}
The results on VGG19 with CIFAR-100 exhibit a similar trend to those observed on CIFAR-10, reinforcing the effectiveness of our proposed approach. Once again, we identify the occurrence of layer collapse at extreme sparsities when no warm-up is applied, leading to a significant drop in accuracy. Introducing a single warm-up epoch effectively resolves this issue, restoring pruning performance across all evaluated criteria. However, increasing the warm-up phase to five epochs does not yield any additional advantage, indicating that a brief warm-up period is sufficient to stabilize gradient-based importance scores and prevent collapse.

\label{sec:vgg_cifar-100}

\begin{table}[h]
\caption{Performance of different compression methods evaluated using VGG19 on the CIFAR-100 dataset. We report the mean accuracy between three initialization seeds across various sparsity levels. Baseline, no pruning: $58.96 \pm 2.30$.}
\label{tab:VGG19_cifar100_compressors}
\vskip 0.15in
\begin{center}
\begin{small}
\begin{sc}
\resizebox{\textwidth}{!}{%
\begin{tabular}{lccccc|cccc}
\toprule
Sparsity & Random & Magnitude & GN & SNIP & GraSP & FD & FP & FTS & FBSS \\
\midrule
0.10  & 60.31 ± 0.40 & 59.13 ± 1.29 & 61.93 ± 0.48 & 61.98 ± 0.29 & 59.32 ± 0.63 & 62.13 ± 0.61 & 60.45 ± 3.47 & 61.56 ± 1.04 & 58.79 ± 0.98 \\
0.20  & 60.43 ± 1.14 & 59.27 ± 0.34 & 62.64 ± 0.21 & 62.68 ± 0.24 & 61.21 ± 0.41 & 63.04 ± 0.43 & 62.71 ± 1.02 & 62.24 ± 0.44 & 60.48 ± 0.48 \\
0.30  & 58.32 ± 0.60 & 59.35 ± 1.43 & 62.61 ± 0.23 & 63.11 ± 0.35 & 59.30 ± 0.43 & 62.85 ± 0.42 & 61.43 ± 0.61 & 62.65 ± 0.54 & 58.77 ± 1.02 \\
0.40  & 56.50 ± 3.20 & 60.04 ± 1.02 & 62.36 ± 0.02 & 62.39 ± 0.55 & 56.34 ± 1.49 & 62.38 ± 0.75 & 61.56 ± 1.25 & 62.67 ± 0.06 & 1.00 ± 0.00 \\
0.50  & 58.47 ± 1.49 & 61.49 ± 1.22 & 62.02 ± 0.64 & 62.76 ± 0.50 & 54.43 ± 0.84 & 62.84 ± 0.33 & 62.25 ± 0.33 & 62.47 ± 0.42 & 1.00 ± 0.00 \\
0.60  & 57.54 ± 0.74 & 61.50 ± 0.30 & 62.55 ± 0.13 & 63.08 ± 0.55 & 56.76 ± 0.69 & 62.40 ± 0.57 & 62.70 ± 0.63 & 62.17 ± 0.23 & 1.00 ± 0.00 \\
0.70  & 57.63 ± 0.80 & 61.71 ± 0.25 & 60.85 ± 0.79 & 60.58 ± 0.39 & 57.76 ± 0.84 & 60.44 ± 0.34 & 60.92 ± 0.41 & 60.51 ± 1.67 & 1.00 ± 0.00 \\
0.80  & 57.84 ± 0.57 & 61.89 ± 1.02 & 55.09 ± 0.49 & 59.84 ± 0.29 & 58.39 ± 0.74 & 1.00 ± 0.00 & 43.16 ± 1.02 & 58.66 ± 2.28 & 1.00 ± 0.00 \\
0.90  & 58.41 ± 0.41 & 62.60 ± 0.91 & 1.00 ± 0.00 & 8.35 ± 10.39 & 42.88 ± 1.64 & 1.00 ± 0.00 & 1.00 ± 0.00 & 8.87 ± 11.13 & 1.00 ± 0.00 \\
0.95  & 54.84 ± 1.08 & 1.00 ± 0.00 & 1.00 ± 0.00 & 1.00 ± 0.00 & 1.00 ± 0.00 & 1.00 ± 0.00 & 1.00 ± 0.00 & 1.00 ± 0.00 & 1.00 ± 0.00 \\
0.98  & 50.21 ± 0.72 & 1.00 ± 0.00 & 1.00 ± 0.00 & 1.00 ± 0.00 & 1.00 ± 0.00 & 1.00 ± 0.00 & 1.00 ± 0.00 & 1.00 ± 0.00 & 1.00 ± 0.00 \\
0.99  & 46.69 ± 0.45 & 1.00 ± 0.00 & 1.00 ± 0.00 & 1.00 ± 0.00 & 1.00 ± 0.00 & 1.00 ± 0.00 & 1.00 ± 0.00 & 1.00 ± 0.00 & 1.00 ± 0.00 \\
\bottomrule
\end{tabular}}
\end{sc}
\end{small}
\end{center}
\vskip -0.1in
\end{table}

%------------------------------------------------------------------------------------------------

\begin{table}[h]
\caption{Performance of different compression methods evaluated after 1 warmup epoch using VGG19 on the CIFAR-100 dataset. We report the mean accuracy between three initialization seeds across various sparsity levels. Baseline, no pruning: $58.96 \pm 2.30$.}
\label{tab:VGG19_cifar100_compressors_warmup1}
\vskip 0.15in
\begin{center}
\begin{small}
\begin{sc}
\resizebox{\textwidth}{!}{%
\begin{tabular}{lccccc|cccc}
\toprule
Sparsity & Random & Magnitude & GN & SNIP & GraSP & FD & FP & FTS & FBSS \\
\midrule
0.80  & 60.39 ± 1.16 & 58.91 ± 0.41 & 52.81 ± 1.32 & 55.62 ± 2.27 & 55.15 ± 2.25 & 56.71 ± 0.31 & 58.03 ± 0.93 & 52.41 ± 3.07 & 52.74 ± 5.16 \\
0.90  & 58.90 ± 0.98 & 60.95 ± 0.81 & 50.56 ± 4.59 & 55.89 ± 2.05 & 56.01 ± 1.58 & 52.07 ± 3.24 & 53.65 ± 0.57 & 52.45 ± 3.75 & 19.65 ± 1.68 \\
0.95  & 56.10 ± 0.85 & 57.64 ± 2.63 & 50.34 ± 1.00 & 53.70 ± 3.60 & 56.16 ± 0.41 & 54.44 ± 1.38 & 53.24 ± 3.54 & 53.56 ± 1.26 & 17.24 ± 0.44 \\
0.98  & 50.97 ± 0.40 & 54.66 ± 2.56 & 43.43 ± 5.32 & 50.19 ± 1.59 & 54.64 ± 1.50 & 42.75 ± 1.91 & 50.59 ± 3.39 & 48.56 ± 5.25 & 16.42 ± 0.64 \\
0.99  & 46.52 ± 0.45 & 43.33 ± 5.83 & 33.90 ± 5.35 & 42.65 ± 5.32 & 45.98 ± 4.48 & 29.67 ± 8.49 & 49.11 ± 3.46 & 48.70 ± 2.59 & 13.25 ± 0.84 \\
\bottomrule
\end{tabular}}
\end{sc}
\end{small}
\end{center}
\vskip -0.1in
\end{table}


%------------------------------------------------------------------------------------------------

\begin{table}[h]
\caption{Performance of different compression methods evaluated after 5 warmup epochs using VGG19 on the CIFAR-100 dataset. We report the mean accuracy between three initialization seeds across various sparsity levels. Baseline, no pruning: $58.96 \pm 2.30$.}
\label{tab:VGG19_cifar100_compressors_warmup5}
\vskip 0.15in
\begin{center}
\begin{small}
\begin{sc}
\resizebox{\textwidth}{!}{%
\begin{tabular}{lccccc|cccc}
\toprule
Sparsity & Random & Magnitude & GN & SNIP & GraSP & FD & FP & FTS & FBSS \\
\midrule
0.80  & 60.41 ± 1.39 & 58.38 ± 0.85 & 60.86 ± 0.79 & 61.63 ± 0.45 & 56.25 ± 0.49 & 59.59 ± 0.76 & 59.37 ± 3.50 & 60.86 ± 0.53 & 46.93 ± 9.04 \\
0.90  & 60.32 ± 0.09 & 57.74 ± 1.64 & 57.77 ± 2.41 & 58.23 ± 4.07 & 56.27 ± 1.02 & 60.19 ± 0.63 & 61.23 ± 0.50 & 60.52 ± 0.37 & 21.66 ± 1.95 \\
0.95 & 57.86 ± 0.53 & 59.55 ± 1.15 & 56.09 ± 0.97 & 58.83 ± 0.65 & 55.26 ± 1.25 & 55.80 ± 2.77 & 59.83 ± 0.94 & 58.52 ± 1.32 & 19.98 ± 2.62 \\
0.98 & 51.75 ± 0.43 & 47.75 ± 7.63 & 52.26 ± 4.06 & 55.27 ± 1.69 & 54.59 ± 0.96 & 49.46 ± 4.98 & 57.40 ± 1.26 & 56.00 ± 1.08 & 17.59 ± 1.36 \\
0.99 & 47.59 ± 0.80 & 42.46 ± 7.95 & 46.58 ± 2.00 & 53.13 ± 0.84 & 53.91 ± 1.53 & 42.87 ± 4.63 & 53.17 ± 1.18 & 53.05 ± 2.14 & 13.92 ± 0.14 \\
\bottomrule
\end{tabular}}
\end{sc}
\end{small}
\end{center}
\vskip -0.1in
\end{table}


%------------------------------------------------------------------------------------------------
\clearpage

\section{Mask Batch Size for Other Sparsities}
The Effect of batch size on pruning performance across different sparsities. 
As sparsity increases, the effect of batch size on pruning performance becomes more pronounced. 
At lower sparsities (0.90, 0.95), the differences across batch sizes are less evident, suggesting that even smaller batches provide a reasonable estimation of parameter importance. However, at extreme sparsities (0.98, 0.99), we observe a clear trend where larger batch sizes consistently lead to better parameter selection, ultimately improving accuracy. This aligns with our hypothesis that larger batches help reduce variance in gradient estimation, leading to more stable and effective pruning decisions. 
\label{batch_size_heatmaps}

\begin{figure}[h]
    \centering
    \includegraphics[width=0.8\linewidth]{imgs/cifar10_resnet18_heatmap_warmup_0.png}
    \caption{Effect of batch size on pruning performance at increasing sparsities.}
    \label{fig:enter-label}
\end{figure}

%------------------------------------------------------------------------------------------------

\clearpage
\section{Comparison of our criteria with magnitude-based pruning}

Figure \ref{fig:our_criterion_vs_magnitude} illustrates the relationship between parameter magnitude and different sensitivity-based pruning metrics. Each point represents a model parameter, with red points indicating the top-ranked parameters selected for retention by each criterion. The green dashed line marks the 99th percentile of parameter magnitudes.

A key observation is that the most effective pruning criteria, such as Fisher-Taylor Sensitivity, tend to retain parameters with a broad range of magnitudes, including many that are relatively small (left of the green line). This shows that the estimated importance does not always prioritize parameters based on their magnitude. 


\begin{figure}[htp]
    \centering
    \includegraphics[width=0.9\linewidth]{imgs/cifar_10_mag_vs_criteria_s_99.png}
    \caption{Our criteria vs. Magnitude parameter selection for 99\% sparsity (ResNet18, CIFAR-10, Seed 0)} 
    \label{fig:our_criterion_vs_magnitude}
\end{figure}


\end{document}