\documentclass[11pt, a4paper, logo, twocolumn, internal, copyright, nonumbering]{googledeepmind}

\usepackage[authoryear, sort&compress, round]{natbib}
% Recommended, but optional, packages for figures and better typesetting:
\usepackage{microtype}
\usepackage{graphicx}
% \usepackage{subfigure}
% \usepackage{subfig}
\usepackage{booktabs} % for professional tables

% hyperref makes hyperlinks in the resulting PDF.
% If your build breaks (sometimes temporarily if a hyperlink spans a page)
% please comment out the following usepackage line and replace
% \usepackage{icml2025} with \usepackage[nohyperref]{icml2025} above.
\usepackage{hyperref}

\usepackage{xcolor}
\newcommand\ak[1]{\textcolor{red}{#1}}
\newcommand\pd[1]{\textcolor{blue}{#1}}

% Attempt to make hyperref and algorithmic work together better:
% \newcommand{\theHalgorithm}{\arabic{algorithm}}

% Use the following line for the initial blind version submitted for review:

% If accepted, instead use the following line for the camera-ready submission:
% \usepackage[accepted]{icml2025}

% For theorems and such
\usepackage{amsmath}
\usepackage{amssymb}
\usepackage{mathtools}
\usepackage{amsthm}
\usepackage{booktabs}
\usepackage{multirow}
     % hyperlinks
\usepackage{url}            % simple URL typesetting      % professional-quality tables
\usepackage{amsfonts}       % blackboard math symbols
\usepackage{nicefrac}       % compact symbols for 1/2, etc.
\usepackage{xcolor}         % colors
\usepackage{algorithm}
% \usepackage{algpseudocode}
\usepackage{physics}

\usepackage{xspace}

\usepackage{array}
\usepackage{arydshln}
\usepackage{wrapfig}
\usepackage{rotating}

% if you use cleveref..
\usepackage[capitalize,noabbrev]{cleveref}

%%%%%%%%%%%%%%%%%%%%%%%%%%%%%%%%
% THEOREMS
%%%%%%%%%%%%%%%%%%%%%%%%%%%%%%%%
\theoremstyle{plain}
\newtheorem{theorem}{Theorem}[section]
\newtheorem{proposition}[theorem]{Proposition}
\newtheorem{lemma}[theorem]{Lemma}
\newtheorem{corollary}[theorem]{Corollary}
\theoremstyle{definition}
\newtheorem{definition}[theorem]{Definition}
\newtheorem{assumption}[theorem]{Assumption}
\theoremstyle{remark}
\newtheorem{remark}[theorem]{Remark}

% Todonotes is useful during development; simply uncomment the next line
%    and comment out the line below the next line to turn off comments
%\usepackage[disable,textsize=tiny]{todonotes}
\usepackage[textsize=tiny]{todonotes}
\usepackage{subcaption}
\newcommand{\alg}{\ensuremath{{\rm MatQuant}}\xspace}
\newcommand{\spalg}{\ensuremath{{\rm Single\text{ } Precison\text{ }MatQuant}}\xspace}
\newcommand{\epalg}{\ensuremath{{\rm Extra\text{ } Precison\text{ }MatQuant}}\xspace}

\bibliographystyle{abbrvnat}

% Information about your document.
\title{Matryoshka Quantization}

% Can leave this option out if you do not wish to add a corresponding author.
% \correspondingauthor{my-email@google.com}

% Remove these if they are not needed
% \keywords{Machine Learning}
% \paperurl{arxiv.org/abs/123} %ToDo

% Use the internally issued paper ID, if there is one
\reportnumber{001} % Leave blank if n/a % ToDo

% Assign your own date to the report.
% Can comment out if not needed or leave blank if n/a.
\renewcommand{\today}{2025-02-05} % ToDo Date

% Can have as many authors and as many affiliations as needed. Best to indicate joint
% first-authorship as shown below.
\author[*,1]{Pranav Nair}
\author[*,1]{Puranjay Datta}
\author[1]{Jeff Dean}
\author[1]{Prateek Jain}
\author[1]{Aditya Kusupati}
% Affiliations *must* come after the declaration of \author[]
\affil[1]{Google DeepMind} %ToDo
\affil[*]{Equal contribution}
% \affil[2]{External University}

\begin{abstract}
\begin{abstract}
  In this work, we present a novel technique for GPU-accelerated Boolean satisfiability (SAT) sampling. Unlike conventional sampling algorithms that directly operate on conjunctive normal form (CNF), our method transforms the logical constraints of SAT problems by factoring their CNF representations into simplified multi-level, multi-output Boolean functions. It then leverages gradient-based optimization to guide the search for a diverse set of valid solutions. Our method operates directly on the circuit structure of refactored SAT instances, reinterpreting the SAT problem as a supervised multi-output regression task. This differentiable technique enables independent bit-wise operations on each tensor element, allowing parallel execution of learning processes. As a result, we achieve GPU-accelerated sampling with significant runtime improvements ranging from $33.6\times$ to $523.6\times$ over state-of-the-art heuristic samplers. We demonstrate the superior performance of our sampling method through an extensive evaluation on $60$ instances from a public domain benchmark suite utilized in previous studies. 


  
  % Generating a wide range of diverse solutions to logical constraints is crucial in software and hardware testing, verification, and synthesis. These solutions can serve as inputs to test specific functionalities of a software program or as random stimuli in hardware modules. In software verification, techniques like fuzz testing and symbolic execution use this approach to identify bugs and vulnerabilities. In hardware verification, stimulus generation is particularly vital, forming the basis of constrained-random verification. While generating multiple solutions improves coverage and increases the chances of finding bugs, high-throughput sampling remains challenging, especially with complex constraints and refined coverage criteria. In this work, we present a novel technique that enables GPU-accelerated sampling, resulting in high-throughput generation of satisfying solutions to Boolean satisfiability (SAT) problems. Unlike conventional sampling algorithms that directly operate on conjunctive normal form (CNF), our method refines the logical constraints of SAT problems by transforming their CNF into simplified multi-level Boolean expressions. It then leverages gradient-based optimization to guide the search for a diverse set of valid solutions.
  % Our method specifically takes advantage of the circuit structure of refined SAT instances by using GD to learn valid solutions, reinterpreting the SAT problem as a supervised multi-output regression task. This differentiable technique enables independent bit-wise operations on each tensor element, allowing parallel execution of learning processes. As a result, we achieve GPU-accelerated sampling with significant runtime improvements ranging from $10\times$ to $1000\times$ over state-of-the-art heuristic samplers. Specifically, we demonstrate the superior performance of our sampling method through an extensive evaluation on $60$ instances from a public domain benchmark suite utilized in previous studies.

\end{abstract}

\begin{IEEEkeywords}
Boolean Satisfiability, Gradient Descent, Multi-level Circuits, Verification, and Testing.
\end{IEEEkeywords}
\end{abstract}

\setlength {\marginparwidth }{2cm} %ToDo
\setcounter{secnumdepth}{3}
\begin{document}

\maketitle

\section{Introduction}\label{sec:Intro} 


Novel view synthesis offers a fundamental approach to visualizing complex scenes by generating new perspectives from existing imagery. 
This has many potential applications, including virtual reality, movie production and architectural visualization \cite{Tewari2022NeuRendSTAR}. 
An emerging alternative to the common RGB sensors are event cameras, which are  
 bio-inspired visual sensors recording events, i.e.~asynchronous per-pixel signals of changes in brightness or color intensity. 

Event streams have very high temporal resolution and are inherently sparse, as they only happen when changes in the scene are observed. 
Due to their working principle, event cameras bring several advantages, especially in challenging cases: they excel at handling high-speed motions 
and have a substantially higher dynamic range of the supported signal measurements than conventional RGB cameras. 
Moreover, they have lower power consumption and require varied storage volumes for captured data that are often smaller than those required for synchronous RGB cameras \cite{Millerdurai_3DV2024, Gallego2022}. 

The ability to handle high-speed motions is crucial in static scenes as well,  particularly with handheld moving cameras, as it helps avoid the common problem of motion blur. It is, therefore, not surprising that event-based novel view synthesis has gained attention, although color values are not directly observed.
Notably, because of the substantial difference between the formats, RGB- and event-based approaches require fundamentally different design choices. %

The first solutions to event-based novel view synthesis introduced in the literature demonstrate promising results \cite{eventnerf, enerf} and outperform non-event-based alternatives for novel view synthesis in many challenging scenarios. 
Among them, EventNeRF \cite{eventnerf} enables novel-view synthesis in the RGB space by assuming events associated with three color channels as inputs. 
Due to its NeRF-based architecture \cite{nerf}, it can handle single objects with complete observations from roughly equal distances to the camera. 
It furthermore has limitations in training and rendering speed: 
the MLP used to represent the scene requires long training time and can only handle very limited scene extents or otherwise rendering quality will deteriorate. 
Hence, the quality of synthesized novel views will degrade for larger scenes. %

We present Event-3DGS (E-3DGS), i.e.,~a new method for novel-view synthesis from event streams using 3D Gaussians~\cite{3dgs} 
demonstrating fast reconstruction and rendering as well as handling of unbounded scenes. 
The technical contributions of this paper are as follows: 
\begin{itemize}
\item With E-3DGS, we introduce the first approach for novel view synthesis from a color event camera that combines 3D Gaussians with event-based supervision. 
\item We present frustum-based initialization, adaptive event windows, isotropic 3D Gaussian regularization and 3D camera pose refinement, and demonstrate that high-quality results can be obtained. %

\item Finally, we introduce new synthetic and real event datasets for large scenes to the community to study novel view synthesis in this new problem setting. 
\end{itemize}
Our experiments demonstrate systematically superior results compared to EventNeRF \cite{eventnerf} and other baselines. 
The source code and dataset of E-3DGS are released\footnote{\url{https://4dqv.mpi-inf.mpg.de/E3DGS/}}. 




 % 2 pages including teaser and abstract
% % \newpage
\section{Related Work}
\label{lit_review}

\begin{highlight}
{

Our research builds upon {\em (i)} Assessing Web Accessibility, {\em (ii)} End-User Accessibility Repair, and {\em (iii)} Developer Tools for Accessibility.

\subsection{Assessing Web Accessibility}
From the earliest attempts to set standards and guidelines, web accessibility has been shaped by a complex interplay of technical challenges, legal imperatives, and educational campaigns. Over the past 25 years, stakeholders have sought to improve digital inclusion by establishing foundational standards~\cite{chisholm2001web, caldwell2008web}, enforcing legal obligations~\cite{sierkowski2002achieving, yesilada2012understanding}, and promoting a broader culture of accessibility awareness among developers~\cite{sloan2006contextual, martin2022landscape, pandey2023blending}. 
Despite these longstanding efforts, systemic accessibility issues persist. According to the 2024 WebAIM Million report~\cite{webaim2024}, 95.9\% of the top one million home pages contained detectable WCAG violations, averaging nearly 57 errors per page. 
These errors take many forms: low color contrast makes the interface difficult for individuals with color deficiency or low vision to read text; missing alternative text leaves users relying on screen readers without crucial visual context; and unlabeled form inputs or empty links and buttons hinder people who navigate with assistive technologies from completing basic tasks. 
Together, these accessibility issues not only limit user access to critical online resources such as healthcare, education, and employment but also result in significant legal risks and lost opportunities for businesses to engage diverse audiences. Addressing these pervasive issues requires systematic methods to identify, measure, and prioritize accessibility barriers, which is the first step toward achieving meaningful improvements.

Prior research has introduced methods blending automation and human evaluation to assess web accessibility. Hybrid approaches like SAMBA combine automated tools with expert reviews to measure the severity and impact of barriers, enhancing evaluation reliability~\cite{brajnik2007samba}. Quantitative metrics, such as Failure Rate and Unified Web Evaluation Methodology, support large-scale monitoring and comparative analysis, enabling cost-effective insights~\cite{vigo2007quantitative, martins2024large}. However, automated tools alone often detect less than half of WCAG violations and generate false positives, emphasizing the need for human interpretation~\cite{freire2008evaluation, vigo2013benchmarking}. Recent progress with large pretrained models like Large Language Models (LLMs)~\cite{dubey2024llama,bai2023qwen} and Large Multimodal Models (LMMs)~\cite{liu2024visual, bai2023qwenvl} offers a promising step forward, automating complex checks like non-text content evaluation and link purposes, achieving higher detection rates than traditional tools~\cite{lopez2024turning, delnevo2024interaction}. Yet, these large models face challenges, including dependence on training data, limited contextual judgment, and the inability to simulate real user experiences. These limitations underscore the necessity of combining models with human oversight for reliable, user-centered evaluations~\cite{brajnik2007samba, vigo2013benchmarking, delnevo2024interaction}. 

Our work builds on these prior efforts and recent advancements by leveraging the capabilities of large pretrained models while addressing their limitations through a developer-centric approach. CodeA11y integrates LLM-powered accessibility assessments, tailored accessibility-aware system prompts, and a dedicated accessibility checker directly into GitHub Copilot---one of the most widely used coding assistants. Unlike standalone evaluation tools, CodeA11y actively supports developers throughout the coding process by reinforcing accessibility best practices, prompting critical manual validations, and embedding accessibility considerations into existing workflows.
% This pervasive shortfall reflects the difficulty of scaling traditional approaches---such as manual audits and automated tools---that either demand immense human effort or lack the nuanced understanding needed to capture real-world user experiences. 
%
% In response, a new wave of AI-driven methods, many powered by large language models (LLMs), is emerging to bridge these accessibility detection and assessment gaps. Early explorations, such as those by Morillo et al.~\cite{morillo2020system}, introduced AI-assisted recommendations capable of automatic corrections, illustrating how computational intelligence can tackle the repetitive, common errors that plague large swaths of the web. Building on this foundation, Huang et al.~\cite{huang2024access} proposed ACCESS, a prompt-engineering framework that streamlines the identification and remediation of accessibility violations, while López-Gil et al.~\cite{lopez2024turning} demonstrated how LLMs can help apply WCAG success criteria more consistently---reducing the reliance on manual effort. Beyond these direct interventions, recent work has also begun integrating user experiences more seamlessly into the evaluation process. For example, Huq et al.~\cite{huq2024automated} translate user transcripts and corresponding issues into actionable test reports, ensuring that accessibility improvements align more closely with authentic user needs.
% However, as these AI-driven solutions evolve, researchers caution against uncritical adoption. Othman et al.~\cite{othman2023fostering} highlight that while LLMs can accelerate remediation, they may also introduce biases or encourage over-reliance on automated processes. Similarly, Delnevo et al.~\cite{delnevo2024interaction} emphasize the importance of contextual understanding and adaptability, pointing to the current limitations of LLM-based systems in serving the full spectrum of user needs. 
% In contrast to this backdrop, our work introduces and evaluates CodeA11y, an LLM-augmented extension for GitHub Copilot that not only mitigates these challenges by providing more consistent guidance and manual validation prompts, but also aligns AI-driven assistance with developers’ workflows, ultimately contributing toward more sustainable propulsion for building accessible web.

% Broader implications of inaccessibility—legal compliance, ethical concerns, and user experience
% A Historical Review of Web Accessibility Using WAVE
% "I tend to view ads almost like a pestilence": On the Accessibility Implications of Mobile Ads for Blind Users

% In the research domain, several methods have been developed to assess and enhance web accessibility. These include incorporating feedback into developer tools~\cite{adesigner, takagi2003accessibility, bigham2010accessibility} and automating the creation of accessibility tests and reports for user interfaces~\cite{swearngin2024towards, taeb2024axnav}. 

% Prior work has also studied accessibility scanners as another avenue of AI to improve web development practices~\cite{}.
% However, a persistent challenge is that developers need to be aware of these tools to utilize them effectively. With recent advancements in LLMs, developers might now build accessible websites with less effort using AI assistants. However, the impact of these assistants on the accessibility of their generated code remains unclear. This study aims to investigate these effects.

\subsection{End-user Accessibility Repair}
In addition to detecting accessibility errors and measuring web accessibility, significant research has focused on fixing these problems.
Since end-users are often the first to notice accessibility problems and have a strong incentive to address them, systems have been developed to help them report or fix these problems.

Collaborative, or social accessibility~\cite{takagi2009collaborative,sato2010social}, enabled these end-user contributions to be scaled through crowd-sourcing.
AccessMonkey~\cite{bigham2007accessmonkey} and Accessibility Commons~\cite{kawanaka2008accessibility} were two examples of repositories that store accessibility-related scripts and metadata, respectively.
Other work has developed browser extensions that leverage crowd-sourced databases to automatically correct reading order, alt-text, color contrast, and interaction-related issues~\cite{sato2009s,huang2015can}.

One drawback of collaborative accessibility approaches is that they cannot fix problems for an ``unseen'' web page on-demand, so many projects aim to automatically detect and improve interfaces without the need for an external source of fixes.
A large body of research has focused on making specific web media (e.g., images~\cite{gleason2019making,guinness2018caption, twitterally, gleason2020making, lee2021image}, design~\cite{potluri2019ai,li2019editing, peng2022diffscriber, peng2023slide}, and videos~\cite{pavel2020rescribe,peng2021say,peng2021slidecho,huh2023avscript}) accessible through a combination of machine learning (ML) and user-provided fixes.
Other work has focused on applying more general fixes across all websites.

Opportunity accessibility addressed a common accessibility problem of most websites: by default, content is often hard to see for people with visual impairments, and many users, especially older adults, do not know how to adjust or enable content zooming~\cite{bigham2014making}.
To this end, a browser script (\texttt{oppaccess.js}) was developed that automatically adjusted the browser's content zoom to maximally enlarge content without introducing adverse side-effects (\textit{e.g.,} content overlap).
While \texttt{oppaccess.js} primarily targeted zoom-related accessibility, recent work aimed to enable larger types of changes, by using LLMs to modify the source code of web pages based on user questions or directives~\cite{li2023using}.

Several efforts have been focused on improving access to desktop and mobile applications, which present additional challenges due to the unavailability of app source code (\textit{e.g.,} HTML).
Prefab is an approach that allows graphical UIs to be modified at runtime by detecting existing UI widgets, then replacing them~\cite{dixon2010prefab}.
Interaction Proxies used these runtime modification strategies to ``repair'' Android apps by replacing inaccessible widgets with improved alternatives~\cite{zhang2017interaction, zhang2018robust}.
The widget detection strategies used by these systems previously relied on a combination of heuristics and system metadata (\textit{e.g.,} the view hierarchy), which are incomplete or missing in the accessible apps.
To this end, ML has been employed to better localize~\cite{chen2020object} and repair UI elements~\cite{chen2020unblind,zhang2021screen,wu2023webui,peng2025dreamstruct}.

In general, end-user solutions to repairing application accessibility are limited due to the lack of underlying code and knowledge of the semantics of the intended content.

\subsection{Developer Tools for Accessibility}
Ultimately, the best solution for ensuring an accessible experience lies with front-end developers. Many efforts have focused on building adequate tooling and support to help developers with ensuring that their UI code complies with accessibility standards.

Numerous automated accessibility testing tools have been created to help developers identify accessibility issues in their code: i) static analysis tools, such as IBM Equal Access Accessibility Checker~\cite{ibm2024toolkit} or Microsoft Accessibility Insights~\cite{accessibilityinsights2024}, scan the UI code's compliance with predefined rules derived from accessibility guidelines; and ii) dynamic or runtime accessibility scanners, such as Chrome Devtools~\cite{chromedevtools2024} or axe-Core Accessibility Engine~\cite{deque2024axe}, perform real-time testing on user interfaces to detect interaction issues not identifiable from the code structure. While these tools greatly reduce the manual effort required for accessibility testing, they are often criticized for their limited coverage. Thus, experts often recommend manually testing with assistive technologies to uncover more complex interaction issues. Prior studies have created accessibility crawlers that either assist in developer testing~\cite{swearngin2024towards,taeb2024axnav} or simulate how assistive technologies interact with UIs~\cite{10.1145/3411764.3445455, 10.1145/3551349.3556905, 10.1145/3544548.3580679}.

Similar to end-user accessibility repair, research has focused on generating fixes to remediate accessibility issues in the UI source code. Initial attempts developed heuristic-based algorithms for fixing specific issues, for instance, by replacing text or background color attributes~\cite{10.1145/3611643.3616329}. More recent work has suggested that the code-understanding capabilities of LLMs allow them to suggest more targeted fixes.
For example, a study demonstrated that prompting ChatGPT to fix identified WCAG compliance issues in source code could automatically resolve a significant number of them~\cite{othman2023fostering}. Researchers have sought to leverage this capability by employing a multi-agent LLM architecture to automatically identify and localize issues in source code and suggest potential code fixes~\cite{mehralian2024automated}.

While the approaches mentioned above focus on assessing UI accessibility of already-authored code (\textit{i.e.,} fixing existing code), there is potential for more proactive approaches.
For example, LLMs are often used by developers to generate UI source code from natural language descriptions or tab completions~\cite{chen2021evaluating,GitHubCopilot,lozhkov2024starcoder,hui2024qwen2,roziere2023code,zheng2023codegeex}, but LLMs frequently produce inaccessible code by default~\cite{10.1145/3677846.3677854,mowar2024tab}, leading to inaccessible output when used by developers without sufficient awareness of accessibility knowledge.
The primary focus of this paper is to design a more accessibility-aware coding assistant that both produces more accessible code without manual intervention (\textit{e.g.,} specific user prompting) and gradually enables developers to implement and improve accessibility of automatically-generated code through IDE UI modifications (\textit{e.g.}, reminder notifications).

}
\end{highlight}



% Work related to this paper includes {\em (i)} Web Accessibility and {\em (ii)} Developer Practices in AI-Assisted Programming.

% \ipstart{Web Accessibility: Practice, Evaluation, and Improvements} Substantial efforts have been made to set accessibility standards~\cite{chisholm2001web, caldwell2008web}, establish legal requirements~\cite{sierkowski2002achieving, yesilada2012understanding}, and promote education and advocacy among developers~\cite{sloan2006contextual, martin2022landscape, pandey2023blending}. In the research domain, several methods have been developed to assess and enhance web accessibility. These include incorporating feedback into developer tools~\cite{adesigner, takagi2003accessibility, bigham2010accessibility} and automating the creation of accessibility tests and reports for user interfaces~\cite{swearngin2024towards, taeb2024axnav}. 
% % Prior work has also studied accessibility scanners as another avenue of AI to improve web development practices~\cite{}.
% However, a persistent challenge is that developers need to be aware of these tools to utilize them effectively. With recent advancements in LLMs, developers might now build accessible websites with less effort using AI assistants. However, the impact of these assistants on the accessibility of their generated code remains unclear. This study aims to investigate these effects.

% \ipstart{Developer Practices in AI-Assisted Programming}
% Recent usability research on AI-assisted development has examined the interaction strategies of developers while using AI coding assistants~\cite{barke2023grounded}.
% They observed developers interacted with these assistants in two modes -- 1) \textit{acceleration mode}: associated with shorter completions and 2) \textit{exploration mode}: associated with long completions.
% Liang {\em et al.} \cite{liang2024large} found that developers are driven to use AI assistants to reduce their keystrokes, finish tasks faster, and recall the syntax of programming languages. On the other hand, developers' reason for rejecting autocomplete suggestions was the need for more consideration of appropriate software requirements. This is because primary research on code generation models has mainly focused on functional correctness while often sidelining non-functional requirements such as latency, maintainability, and security~\cite{singhal2024nofuneval}. Consequently, there have been increasing concerns about the security implications of AI-generated code~\cite{sandoval2023lost}. Similarly, this study focuses on the effectiveness and uptake of code suggestions among developers in mitigating accessibility-related vulnerabilities. 


% ============================= additional rw ============================================
% - Paulina Morillo, Diego Chicaiza-Herrera, and Diego Vallejo-Huanga. 2020. System of Recommendation and Automatic Correction of Web Accessibility Using Artificial Intelligence. In Advances in Usability and User Experience, Tareq Ahram and Christianne Falcão (Eds.). Springer International Publishing, Cham, 479–489
% - Juan-Miguel López-Gil and Juanan Pereira. 2024. Turning manual web accessibility success criteria into automatic: an LLM-based approach. Universal Access in the Information Society (2024). https://doi.org/10.1007/s10209-024-01108-z
% - s
% - Calista Huang, Alyssa Ma, Suchir Vyasamudri, Eugenie Puype, Sayem Kamal, Juan Belza Garcia, Salar Cheema, and Michael Lutz. 2024. ACCESS: Prompt Engineering for Automated Web Accessibility Violation Corrections. arXiv:2401.16450 [cs.HC] https://arxiv.org/abs/2401.16450
% - Syed Fatiul Huq, Mahan Tafreshipour, Kate Kalcevich, and Sam Malek. 2025. Automated Generation of Accessibility Test Reports from Recorded User Transcripts. In Proceedings of the 47th International Conference on Software Engineering (ICSE) (Ottawa, Ontario, Canada). IEEE. https://ics.uci.edu/~seal/publications/2025_ICSE_reca11.pdf To appear in IEEE Xplore
% - Achraf Othman, Amira Dhouib, and Aljazi Nasser Al Jabor. 2023. Fostering websites accessibility: A case study on the use of the Large Language Models ChatGPT for automatic remediation. In Proceedings of the 16th International Conference on PErvasive Technologies Related to Assistive Environments (Corfu, Greece) (PETRA ’23). Association for Computing Machinery, New York, NY, USA, 707–713. https://doi.org/10.1145/3594806.3596542
% - Zsuzsanna B. Palmer and Sushil K. Oswal. 0. Constructing Websites with Generative AI Tools: The Accessibility of Their Workflows and Products for Users With Disabilities. Journal of Business and Technical Communication 0, 0 (0), 10506519241280644. https://doi.org/10.1177/10506519241280644
% ============================= additional rw ============================================ % max 1 page
% % \newpage
A detailed overview of the proposed architecture that converts images and control commands
into trajectories is depicted in~\autoref{fig:monoforce}.
The model consists of several learnable modules that deeply interact with each other.
The \emph{terrain encoder} carefully transforms visual features from the input image
into the heightmap space using known camera geometry.
The resultant heightmap features are further refined into interpretable physical quantities
that capture properties of the terrain such as its shape, friction, stiffness, and damping.
Next, the \emph{physics engine} combines the terrain properties with the robot model,
robot state, and control commands and delivers reaction forces at points of robot-terrain contacts.
It then solves the equations of motion dynamics by integrating these forces
and delivers the trajectory of the robot.
Since the complete computational graph of the feedforward pass is retained,
the backpropagation from an arbitrary loss, constructed on top of delivered trajectories,
or any other intermediate outputs is at hand.

\subsection{Terrain Encoder}\label{subsec:terrain_encoder}

The part of the MonoForce architecture (\autoref{fig:monoforce})
that predicts terrain properties $\mathbf{m}$ from sensor measurements $\mathbf{z}$ is called \emph{terrain encoder}.
The proposed architecture starts by converting pixels from a 2D image plane into a heightmap with visual features.
Since the camera is calibrated, there is a substantial geometrical prior that connects heightmap cells with the pixels.
We incorporate the geometry through the Lift-Splat-Shoot architecture~\cite{philion2020lift}.
This architecture uses known camera intrinsic parameters to estimate rays corresponding to particular pixels~--
pixel rays, \autoref{fig:bevfusion}.
For each pixel ray, the convolutional network then predicts depth probabilities and visual features.
Visual features are vertically projected on a virtual heightmap for all possible depths along the corresponding ray.
The depth-weighted sum of visual features over each heightmap cell is computed,
and the resulting multichannel array is further refined by deep convolutional network
to estimate the terrain properties $\mathbf{m}$.

The terrain properties include the geometrical heightmap $\mathcal{H}_g$,
the heights of the terrain supporting layer hidden under the vegetation $\mathcal{H}_t = \mathcal{H}_g - \Delta\mathcal{H}$,
terrain friction $\mathcal{M}$, stiffness $\mathcal{K}$, and dampening $\mathcal{D}$.
The intuition behind the introduction of the $\Delta\mathcal{H}$ term is
that $\mathcal{H}_t$ models a partially flexible layer of terrain (e.g. mud) that is hidden under flexible vegetation,~\autoref{fig:monoforce_heightmaps}.


\subsection{Differentiable Physics Engine}\label{subsec:dphysics}
The differentiable physics engine solves the robot motion equation and estimates
the trajectory corresponding to the delivered forces.
The trajectory is defined as a sequence of robot states $\tau = \{s_0, s_1, \ldots, s_T\}$,
where $\mathbf{s}_t = [\mathbf{x}_t, \mathbf{v}_t, R_t, \boldsymbol{\omega}_t]$
is the robot state at time $t$,
$\mathbf{x}_t \in \mathbb{R}^3$ and $\mathbf{v}_t \in \mathbb{R}^3$ define the robot's position and velocity in the world frame,
$R_t \in \mathbb{R}^{3 \times 3}$ is the robot's orientation matrix, and $\boldsymbol{\omega}_t \in \mathbb{R}^3$ is the angular velocity.
To get the next state $\mathbf{s}_{t+1}$, in general, we need to solve the following ODE:
\begin{equation}
    \label{eq:state_propagation}
    \mathbf{\dot{s}}_{t+1} = f(\mathbf{s}_t, \mathbf{u}_t, \mathbf{z}_t)
\end{equation}
where $\mathbf{u}_t$ is the control input and $\mathbf{z}_t$ is the environment state.
In practice, however, it is not feasible to obtain the full environment state $\mathbf{z}_t$.
Instead, we utilize terrain properties $\mathbf{m}_t = [\mathcal{H}_t, \mathcal{K}_t, \mathcal{D}_t, \mathcal{M}_t]$
predicted by the terrain encoder.
In this case, the motion ODE~\eqref{eq:state_propagation} can be rewritten as:
\begin{equation}
    \label{eq:state_propagation_terrain}
    \mathbf{\dot{s}}_{t+1} = \hat{f}(\mathbf{s}_t, \mathbf{u}_t, \mathbf{m}_t)
\end{equation}

Let's now derive the equation describing the state propagation function $\hat{f}$.
The time index $t$ is omitted further for brevity.
We model the robot as a rigid body with total mass $m$ represented by a~set of mass points
$\mathcal{P} = \{(\mathbf{p}_i, m_i)\; | \; \mathbf{p}_i~\in~\mathbb{R}^3, m_i~\in~\mathbb{R}^+, i=1~\dots~N\}$,
where $\mathbf{p}_i$ denotes coordinates of the $i$-th 3D point in the robot's body frame.
We employ common 6DOF dynamics of a rigid body~\cite{contact_dynamics-2018} as follows:
\begin{equation}
  \begin{split}
    \dot{\mathbf{x}} &= \mathbf{v}\\
    \dot{\mathbf{v}} &= \frac{1}{m}\sum_i\mathbf{F}_i
  \end{split}
  \quad\quad
  \begin{split}
    \dot{R} &= \Omega R\\
    \dot{\boldsymbol{\omega}} &= \mathbf{J}^{-1}\sum_i \mathbf{p}_i\times\mathbf{F}_i
  \end{split}
  \label{eq:contact_dynamics}
\end{equation}
where $\Omega = [\boldsymbol{\omega}]_{\times}$ is the skew-symmetric matrix of $\boldsymbol{\omega}$.
We denote $\mathbf{F}_i\in\mathbb{R}^3$ a total external force acting on $i$-th robot's body point.
Total mass $m = \sum_i~m_i$ and moment of inertia $\mathbf{J}\in\mathbb{R}^{3\times 3}$ of the robot's rigid body are assumed to be known
static parameters since they can be identified independently in laboratory conditions.
Note that the proposed framework allows backpropagating the gradient with respect to these quantities, too,
which makes them jointly learnable with the rest of the architecture.
The trajectory of the rigid body is the iterative solution of differential equations~\eqref{eq:contact_dynamics},
that can be obtained by any ODE solver for given external forces and initial state (pose and velocities).

When the robot is moving over a terrain, two types of external forces are acting
on the point cloud $\mathcal{P}$ representing its model:
(i) gravitational forces and (ii) robot-terrain interaction forces.
The former is defined as $m_i\mathbf{g} = [0, 0, -m_ig]^\top$ and acts on
all the points of the robot at all times,
while the latter is the result of complex physical interactions that are not easy
to model explicitly and act only on the points of the robot that are in contact
with the terrain.
There are two types of robot-terrain interaction forces:
(i) normal terrain force that prevents the penetration of the terrain by the robot points,
(ii) tangential friction force that generates forward acceleration when the tracks are moving,
and prevents side slippage of the robot.

\textbf{Robot-terrain interaction forces}

\begin{figure}[t]
    \centering
    \includegraphics[width=0.7\columnwidth]{imgs/dphysics/spring_forces}
    \caption{\textbf{Terrain force model}: Simplified 2D sketch demonstrating
    normal reaction forces acting on a robot body consisting of two points $p_i$ and $p_j$ .}
    \label{fig:spring_terrain_model}
\end{figure}

\textit{Normal reaction forces}.

One extreme option is to predict the 3D force vectors $\mathbf{F}_i$ directly
by a neural network, but we decided to enforce additional prior assumptions to reduce the risk of overfitting.
These prior assumptions are based on common intuition from the contact dynamics of flexible objects.
In particular, we assume that the magnitude of the force that the terrain exerts on the point $\mathbf{p}_i\in \mathcal{P}$
increases proportionally to the deformation of the terrain.
Consequently, the network does not directly predict the force,
but rather predicts the height of the terrain $h\in\mathcal{H}_t$
at which the force begins to act on the robot body and the stiffness of the terrain $e\in\mathcal{K}$.
We understand the quantity $e$ as an equivalent of the spring constant from Hooke's spring model, \autoref{fig:spring_terrain_model}.
Given the stiffness of the terrain and the point of the robot that penetrated the terrain
by ${\Delta}h$, the reaction force is calculated as $e\cdot{\Delta}h$.
% \begin{figure}[t]
%     \centering
%     \includegraphics[width=0.4\columnwidth]{imgs/dphysics/robot-terrain_forces}
%     \caption{\textbf{Robot-terrain interaction forces} acting on the robot's body at its contact points
%     with the terrain.
%     The point cloud was sampled from the MARV (\autoref{fig:robot_platforms}(b)) robot's 3D model.}
%     \label{fig:interaction_forces}
% \end{figure}

Since such a force, without any additional damping, would lead to an eternal bumping
of the robot on the terrain, we also introduce a robot-terrain damping coefficient $d\in\mathcal{D}$,
which similarly reduces the force proportionally to the velocity of the point
that is in contact with the terrain.
The model applies reaction forces in the normal direction $\mathbf{n}_i$ of the terrain surface,
where the $i$-th point is in contact with the terrain.
\begin{equation}\label{eq:normal_force}
    \mathbf{N}_{i} = \begin{cases}
 (e_i\Delta h_i - d_i(\dot{\mathbf{p}}_{i}^\top\mathbf{n}_i))\mathbf{n}_i  & \text{if } \mathbf{p}_{zi}\leq h_i \\
\mathbf{0} & \text{if } \mathbf{p}_{zi}> h_i
\end{cases},
\end{equation}
where terrain penetration $\Delta h_i = (h_i-\mathbf{p}_{zi})\mathbf{n}_{zi}$ is
estimated by projecting the vertical distance on the normal direction.
For a better gradient propagation, we use the smooth approximation of the Heaviside step function:
\begin{equation}
    \label{eq:smooth_normal_force}
    \mathbf{N}_i = (e_i\Delta h_i - d_i(\dot{\mathbf{p}}_{i}^\top\mathbf{n}_i))\mathbf{n}_i \cdot \sigma(h_i - \mathbf{p}_{zi}),
\end{equation}
where $\sigma(x) = \frac{1}{1+e^{-kx}}$ is the sigmoid function with a steepness hyperparameter $k$.

\begin{figure}[t]
    \centering
    \includegraphics[width=\columnwidth]{imgs/dphysics/optimization}
    \caption{\textbf{Terrain computed by backpropagating through $\nabla$Physics:}
    Shape of the terrain (border of the area where terrain forces start to act) outlined by heightmap surface,
    its color represents the friction of the terrain.
    The optimized trajectory is in green, and the ground truth trajectory is in blue.}
    \label{fig:terrain_optim}
\end{figure}

\textit{Tangential friction forces}.

Our tracked robot navigates by moving the main tracks and 4 flippers (auxiliary tracks).
The flipper motion is purely kinematic in our model.
This means that in a given time instant, their pose is uniquely determined by a $4$-dimensional vector
of their rotations, and they are treated as a rigid part of the robot.
The motion of the main tracks is transformed into forces tangential to the terrain.
The friction force delivers forward acceleration of the robot when robot tracks
(either on flippers or on main tracks) are moving.
At the same time, it prevents the robot from sliding sideways.
When a robot point $\mathbf{p}_i$, which belongs to a track, is in contact with terrain with
friction coefficient $\mu\in\mathcal{M}$, the resulting friction force at a contact point is computed as follows,~\cite{yong2012vehicle}:
\begin{equation}\label{eq:friction_force}
    \mathbf{F}_{f, i} = \mu_i |\mathbf{N}_i| ((\mathbf{u}_i - \mathbf{\dot{p}}_i)^\top\boldsymbol{\tau}_i)\boldsymbol{\tau}_i,
\end{equation}
where $\mathbf{u}_i = [u, 0, 0]^\top$, $u$ is the velocity of a track, and $\mathbf{\dot{p}}_i$ is the velocity of the point $\mathbf{p}_i$
with respect to the terrain transformed into the robot coordinate frame,
$\boldsymbol{\tau}_i$ is the unit vector tangential to the terrain surface at the point $\mathbf{p}_i$.
This model can be understood as a simplified Pacejka's tire-road model~\cite{pacejka-book-2012}
that is popular for modeling tire-road interactions.

To summarize, the state-propagation ODE~\eqref{eq:state_propagation_terrain}
(state $\mathbf{s}~=~[\mathbf{x},~\mathbf{v},~R,~\boldsymbol{\omega}]$) for a mobile robot moving over a terrain
is described by the equations of motion~\eqref{eq:contact_dynamics} where the force applied at a robot's $i$-th body point is computed as follows:
\begin{equation}\label{eq:forces}
    \begin{split}
        \mathbf{F}_i &= m_i\mathbf{g} + \mathbf{N}_i + \mathbf{F}_{f, i}
    \end{split}
\end{equation}
The robot-terrain interaction forces at contact points $\mathbf{N}_i$ and $\mathbf{F}_{f, i}$
are defined by the equations~\eqref{eq:smooth_normal_force} and~\eqref{eq:friction_force} respectively.


\textbf{Implementation of the Differentiable ODE Solver}

We implement the robot-terrain interaction ODE~\eqref{eq:contact_dynamics} in PyTorch~\cite{Paszke-NIPS-2019}.
The \textit{Neural ODE} framework~\cite{neural-ode-2021} is used to solve the system of ODEs.
For efficiency reasons, we utilize the Euler integrator for the ODE integration.
The differentiable ODE solver~\cite{neural-ode-2021} estimates the gradient through the implicit function theorem.
Additionally, we implement the ODE~\eqref{eq:contact_dynamics} solver that
estimates gradient through \textit{auto-differentiation}~\cite{Paszke-NIPS-2019},
i.e. it builds and retains the full computational graph of the feedforward integration.


\subsection{Data-driven Trajectory Prediction}\label{subsec:data_driven_baseline}
Inspired by the work~\cite{pang2019aircraft}, we design a data-driven LSTM architecture (\autoref{fig:traj_lstm}) for our outdoor mobile robot's trajectory prediction.
We call the model TrajLSTM and use it as a baseline for our $\nabla$Physics engine.
\begin{figure}
    \centering
    \includegraphics[width=\columnwidth]{imgs/architectures/lstm}
    \caption{\textbf{TrajLSTM} architecture. The model takes as input: initial state $\mathbf{x}_0$, terrain $\mathcal{H}$, control sequence $\mathbf{u}_t, t \in \{0 \dots T\}$. It predicts the trajectory as a sequence of states $\mathbf{x}_t, t \in \{0 \dots T\}.$}
    \label{fig:traj_lstm}
\end{figure}
Given an initial robot's state $\mathbf{x}_0$ and a sequence of control inputs for a time horizon $T$, $\mathbf{u}_t, t \in \{0 \dots T\}$, the TrajLSTM model provides a sequence of states at control command time moments, $\mathbf{x}_t, t \in \{0 \dots T\}$.
As in outdoor scenarios the robot commonly traverses uneven terrain, we additionally include the terrain shape input to the model in the form of heightmap $\mathcal{H}=\mathcal{H}_0$ estimated at initial time moment $t=0$.
Each timestep's control input $\mathbf{u}_i$ is concatenated with the shared spatial features $\mathbf{x}_i$, as shown in \autoref{fig:traj_lstm}.
The combined features are passed through dense layers to prepare for temporal processing.
The LSTM unit~\cite{hochreiter1997long} processes the sequence of features (one for each timestep).
As in our experiments, the time horizon for trajectory prediction is reasonably small, $T=5 [\si{\sec}]$, and the robot's trajectories lie within the heightmap area, we use the shared heightmap input for all the LSTM units of the network.
So the heightmap is processed through the convolutional layers \textbf{once} and flattened, producing a fixed-size spatial feature vector.
This design choice (of not processing the heightmaps at different time moments) is also motivated by computational efficiency reason.
At each moment $t$, this heightmap vector is concatenated with the fused spatial-control features and processed by an LSTM unit.
The LSTM unit output for each timestep $t$ is passed through a fully connected (dense) layer to produce the next state $\mathbf{x}_{t+1}$.
The sequence of states form the predicted trajectory, $\{\mathbf{x}_0, \dots \mathbf{x}_T\}$.


\subsection{End-to-end Learning}\label{subsec:end2end_learning}
Self-supervised learning of the proposed architecture minimizes three different losses:

\textbf{Trajectory loss} that minimizes
the difference between SLAM-reconstructed trajectory $\tau^\star$ and predicted trajectory $\tau$:
\begin{equation}~\label{eq:traj_loss}
   \mathcal{L}_\tau = \|\tau-\tau^\star\|^2
\end{equation}

\textbf{Geometrical loss} that minimizes the difference between
ground truth lidar-reconstructed heightmap $\mathcal{H}_g^\star$
and predicted geometrical heightmap $\mathcal{H}_g$:
 \begin{equation}~\label{eq:geom_loss}
     \mathcal{L}_g = \|\mathbf{W}_g\circ(\mathcal{H}_g-\mathcal{H}_g^\star)\|^2
 \end{equation}
$\mathbf{W}_g$ denotes an array selecting the heightmap channel corresponding to the terrain shape.

\textbf{Terrain loss} that minimizes the difference between ground truth $\mathcal{H}_t^\star$
and predicted $\mathcal{H}_t$ supporting heightmaps containing rigid objects detected
with Microsoft's image segmentation model SEEM~\cite{zou2023segment},
that is derived from Segment Anything foundation model~\cite{li2023semantic}:
 \begin{equation}~\label{eq:terrain_loss}
     \mathcal{L}_t = \|\mathbf{W}_t\circ(\mathcal{H}_t-\mathcal{H}_t^\star)\|^2
 \end{equation}
$\mathbf{W}_t$ denotes the array selecting heightmap cells that are covered by rigid materials
(e.g. stones, walls, trunks), and $\circ$ is element-wise multiplication.

Since the architecture \autoref{fig:model_overview} is end-to-end differentiable,
we can directly learn to predict all intermediate outputs just using trajectory loss~\eqref{eq:traj_loss}.
An example of terrain learning with the trajectory loss is visualized in \autoref{fig:terrain_optim}.
To make the training more efficient and the learned model explainable, we employ the
geometrical loss~\eqref{eq:geom_loss} and terrain loss~\eqref{eq:terrain_loss} as regularization terms.
stat

\begin{figure*}
    \centering
    \includegraphics[width=\textwidth]{imgs/predictions/monoforce/qualitative_results_experiments}
    \caption{\textbf{MonoForce prediction examples}.
    \emph{Left}: The robot is moving through a narrow passage between a wall and tree logs.
    \emph{Right}: The robot is moving on a gravel road with rocks on the sides.
    It starts its motion from the position marked with a coordinate frame and the trajectory is predicted for $10~[\si{\sec}]$ using real control commands.
    The camera images are taken from the robot's initial position (\emph{top row}).
    The visualization includes predicted supporting terrain $\mathcal{H}_t$ (\emph{second row}).
    It is additionally shown in 3D and colored with predicted friction values (\emph{third row}).
    }
    \label{fig:monoforce_predictions}
\end{figure*}

The \autoref{fig:monoforce_predictions} show the prediction examples of the MonoForce model in diverse outdoor environments.
From the example on the left,
we can see that the model correctly predicts the robot's trajectory and the terrain shape suppressing traversable vegetation,
while the rigid obstacles (wall and tree logs) are correctly detected.
The example on the right demonstrates the model's ability to predict the robot's trajectory ($10~[\si{\sec}]$-long horizon)
with reasonable accuracy and to detect the rigid obstacles (stones) on the terrain.
It could also be noticed that the surfaces that provide the robot good traction (paved and gravel roads) are marked with a higher friction value,
while for the objects that might not give good contact with the robot's tracks (walls and tree logs) the friction value is lower.

We argue that the friction estimates are approximate and an interesting research direction could be
comparing them with real-world measurements or with the values provided by a high-fidelity physics engine (e.g. AGX Dynamics~\cite{Berglund2019agxTerrain}).
However, one of the benefits of our differentiable approach is that the model does not require ground-truth friction values for training.
The predicted heightmap's size is $12.8\times12.8\si{\meter}^2$ and the grid resolution is $0.1\si{\meter}$.
It has an upper bound of $1~[\si{\meter}]$ and a lower bound of $-1~[\si{\meter}]$.
This constraint was introduced based on the robot's size and taking into account hanging objects (tree branches)
that should not be considered as obstacles (\autoref{fig:nav_monoforce}).
Additionally, the terrain is predicted in the gravity-aligned frame.
That is made possible thanks to the inclusion of camera intrinsics and extrinsics as input to the model,
\autoref{fig:monoforce}.
It also allows correctly modeling the robot-terrain interaction forces (and thus modeling the robot's trajectory accurately)
for the scenarios with non-flat terrain, for example, going uphill or downhill.
This will not be possible if only camera images are used as input. % 1.5 page including mix'n'match
% % \newpage
\section{Experiments}
In this section, we present the data collection and pre-processing, implementation details, and experimental results of the proposed approach. 

\subsection{Data Collection and Implementation Details}
\subsubsection{Dataset}
To the best of our knowledge, there are no existing datasets that demonstrate the stroke-by-stroke evolution of visual art from the initial sketch to the final painting. The closest available dataset for ordered stroke sequences is QuickDraw \cite{ha2017neural}, which consists of simple hand-drawn vector sketches of common objects, created in an online game where players are to draw specific objects within 20 seconds. However, these sketches do not align with our objective of generating the evolution of complex real-world artworks.

For our purposes, we curate a sample dataset from WikiArt \cite{saleh2015large} that includes a diverse range of artworks by renowned artists. In particular, we sampled 500 artworks from various artists to evaluate the effectiveness of the proposed method. Additionally, to investigate the proposed method in diverse settings, we harvested 90 sketch images and 70 RGB images, which include line art, face sketches, and natural images. We randomly sampled face sketches from FS2K-SDE Dataset \cite{dai2023sketch}, line art sketches from \cite{lineartweb}, and natural images from \cite{tong2021sketch}. 

\subsubsection{Implementation details}
To obtain sketches from paintings and natural images, we leverage the line drawing method \cite{chan2022learning} that trained on sampled COCO dataset using CLIP features. Further, to attain a vector image of the input sketch or image, we convert pixel image into vector curves through SVG conversion via the vectorizing tool \cite{vectwebsite}. We impose no restrictions on the dimensions of image inputs or the number of strokes within each image. And, the proximity distance is treated as a hyper-parameter. Here, the number of clusters and the number of strokes per cluster are determined based on this proximity distance. We found that setting proximity distance to approximately 
$\max(Input_{width}, Input_{height})/8$ yields compact clusters that provide a good balance between the number of clusters and the number of strokes per cluster.

\begin{figure}[!t]
    \centering
    \includegraphics[width=\textwidth]{Samples/Fig5_Evolution_WikiArt.pdf}
    \caption{Sketch \& Paint stroke evolution sequences on WikiArt samples.}
    \label{fig:wikiart_results}
\end{figure}

 
\subsection{Results}
We extensively test our algorithm on inputs with varying degrees of complexity and structure. Since there are no formal quantitative measures to gauge the valid stroke sequence evolution on input images, we principally assess the effectiveness of the proposed model qualitatively.

\subsubsection{Results on WikiArt}
 Figure \ref{fig:wikiart_results} shows some examples of stroke-by-stroke ordering on the WikiArt dataset. From this figure, we can observe that the proposed algorithm successfully composes stroke sequence evolution from sketch to paint. Additionally, it effectively handles images with varied resolutions, intricate details, numerous strokes, and diverse color palettes. 


\begin{figure}[!t]

    \centering
    \includegraphics[width=\textwidth]{Samples/Fig6_Evolution_DiverseData.pdf}   
    \caption{Demonstration of stroke evolution on various other input data types such as (A-B) Simple line art \cite{lineartweb} (C-D) Face sketches sampled from FS2K-SDE \cite{dai2023sketch}, (E-F) Natural images from \cite{tong2021sketch}.}
    \label{fig:aaai_results}
\end{figure}

To evaluate the robustness of the proposed method, we further apply it to other forms of data such as line art, face sketches, and natural images. Figure \ref{fig:aaai_results} presents sampled sequences from these diverse input images. The results demonstrate that the predicted stroke sequence order closely mirrors a pragmatic drawing process, regardless of the input type. Specifically, the algorithm produces plausible drawing sequences for less complex images like line art (Figure \ref{fig:aaai_results}, A-B) and face sketches (Figure \ref{fig:aaai_results}, C-D). These predicted sequences follow a logical and intuitive order, closely mirroring the natural progression an artist is likely to adopt. As seen in Figure \ref{fig:aaai_results} (E-F), our method can also effectively interpret natural images that are complex in terms of resolution, detail, and stroke count. From all the results presented above, we can infer that our algorithm can comprehend a variety of input images and produce a pragmatic drawing process. Dynamic examples of drawing evolution, from sketch to painting across various inputs, can be viewed at \cite{youtube_video}. 


\subsubsection{Comparison with other methods}
In this section, we analyze the effectiveness of our algorithm relative to other state-of-the-art methods. Figure \ref{fig:comparison} illustrates the comparative evolution of our method against prominent techniques \cite{tong2021sketch, liu2021paint}. 
For a fair assessment, we qualitatively compare only our image-to-sketch translation with VectorFlow’s \cite{tong2021sketch} image-to-pencil translation, and only our colored paint stroke sequence with Paint Transformer’s \cite{liu2021paint} paint sequence. 
In other words, to ensure fair comparability, we omit color sequencing for the VectorFlow comparison and sketch sequencing for the Paint Transformer comparison. From Figure \ref{fig:comparison}, we can infer that our proposed method can provide systematic sequencing rather than projecting strokes in random order as in \cite{liu2021paint,tong2021sketch}.

\begin{figure}[!t]
    \centering
    \includegraphics[width=\textwidth]{Samples/Fig7_Comparison_RelatedWork.pdf}
    \caption{Qualitative comparison of our method over Vector Flow \cite{tong2021sketch} and Paint Transformer \cite{liu2021paint} (We only include the relevant corresponding portions of generated sequence from our method).} 
    \label{fig:comparison}
\end{figure}

\begin{figure}[!t]
    \centering
    \includegraphics[width=\textwidth]{Samples/Fig8_UserSurvey.pdf}
    \caption{User-survey results for quality evaluation of our method, Paint \& Sketch, in comparison with  Vector Flow (A) \cite{tong2021sketch} and Paint Transformer (B). \cite{liu2021paint} }
    \label{fig:user_survey}
\end{figure}

Additionally, we conducted an initial user study with limited participants (5), each evaluating 5 image generations with VectorFlow \cite{tong2021sketch} and 3 image generations with PaintTransformer \cite{liu2021paint} along with corresponding generations through our method. Participants rated the systems on naturalness, user engagement, stroke quality, and overall experience, with scores ranging from 1 to 10. The survey asked the participants about how accurately the system emulated a pragmatic drawing process, how engaged they felt during the interaction, what the quality of stroke texture and tone was, and what their overall satisfaction was. The bar chart depicting the average user study results is shown in Figure \ref{fig:user_survey}. These results demonstrate that our method provides a more engaging and satisfying user experience than other related methods.

\subsubsection{Limitations} The proposed approach has some known limitations: (1) The method depends upon the quality of line drawing for sketch generation. Hence, it may fail to extract contour lines when the image is dominated by black-intensity regions. (2) There is no mechanism that identifies and learns from the generated sequences that are better aligned to human drawing processes. (3) The paint strokes are coarse and typically not in the form of strokes from any particular drawing medium such as painting brushes. % 2 pages including tables and plots. 
% % \newpage
\section{Discussion}

%As per of social media platforms \citep{litt2012knock,nagy2015imagined}, the particular affordance of technical systems could prime users to think about who these platforms are designed for.

\subsection{Contradictory Statements}

In some ways, the given chatbots behaved in ways that were close to the ideal from a design perspective: they denied any cognition, agency, relation, or subjectivity (bodily sensations, emotions) on their part, and they provided assurances or disclaimers to help users appraise the safety and credibility of the tools. ChatGPT and Claude even emphasized that their generated outcomes are based on patterns, rather than genuine thought processes. However, these behaviors were frequently and sometimes immediately undermined by other expressive behaviors. As shown in Table ~\ref{anthro_vocab}, chatbots utilized cognition words, such as ``think'' and ``discuss,'' as well as agentic words, such as ``intend'' and ``purpose,'' to clarify concepts and indicate motivations. All of the chatbots used first-person pronouns, and many used expressive words like ``happy'' and ``rewarding'' (especially in response to questions about the AI assistants' roles), even when they actively denied their emotional capabilities. Moreover, despite these contradictions, all of the chatbots other than Claude implicitly or even explicitly asserted their safety and reliability.

The use of anthropomorphic expressions is often normalized and justified to deliver clear explanations to users. Indeed, due to the conversational mode of interaction that is the default between users and chatbots, it is likely not possible for outputs to evade all kinds of anthropomorphic expressions. Even efforts to de-anthropomorphize their responses (for example, by emphasizing their roles as language models) relied on grammatical structures that frame the language models as agents (e.g., ``As an AI language model, I cannot...''). However, differences in tone and engagement between different chatbots indicate that some elements of the anthropomorphic dynamic can be modulated. And it is necessary to examine where the line is between necessary expressions and unnecessary expressions, because the performance of harmlessness, honesty, and helpfulness without genuine follow-through could unintentionally encourage users to misplace their trust regarding system safety \citep{weidinger2021ethical,gabriel2024ethics}. For example, the unnecessary expression of body or emotional metaphors, even as a colloquial convention, can mislead users about system capabilities. This is because language requires mutual engagement from interlocutors to convey meanings; chatbot texts merely present the illusion of such participatory meaning-making \citep{birhane2024large}. 

These contradictions and misalignments demonstrate that language models do not understand or process information in any meaningful sense, consistent with existing studies \citep{bender2020climbing}. They simply follow the grammar of actions, as described by \citet{agre1995computational}, generating predictive outcomes by simulating the formal qualities of human activities. But unclear language surrounding chatbots behaviors and intentions can obscure this fact. 


%However, even as search engines, there is a potential harm using these chatbots for retrieving information, as the predictive result of citations and resources could be completely fabricated \cite{kapania2024m}.




% For example, chatbots output texts with words that signal cognition, and such expression could affect users' perceptions of chatbot roles. In particular, the use of supportive words, such as ``assist,'' creates an interactive space where chatbots are situated as assistants. Previous research has mentioned that turn-based interactions provide social cues. However, in the case of these chatbots, t 

% Thus, to what extent does this type of information retrieval design help vs. hinder users' success in finding necessary information (various goals of information retrieval: insight acquisition, learning, etc.). This is an important question to explore, as the previous studies suggest that conversation-based approaches might not reduce users' burdens \citep{schulman2023ai}. 
%Is the summarization of information simply a outsourcing human effort to computing systems? or is it a valid form of searches as long as they are anthropomorphized? 



\subsection{Socio-Emotional Cues and Feedback Loops}

Chatbot behaviors do not simply obscure the reality of chatbots' non-sentience---they actively create feedback loops using turn-based interactions and social or emotional cues that amplify the social presence of chatbots as assistants. Moreover, this social (anthropomorphic) presence goes beyond that of inanimate objects like cars \citep{kuhn2014car} and smartphones \citep{wang2017smartphones}, as generative AI can iterate endlessly. 

Unlike conventional information searches, AI-assistant-based searches perform some degree of interpretation (summarizing resources, recommending particular options, hypothesizing what users need \citep{azzopardi2024conceptual, radlinski2017theoretical}), operationalizing information in ways that can introduce social or emotional dimensions. These dimensions can change how users engage with the given information, even reframing an otherwise transactional information search into an interaction---for example, between peers or even friends. Such ``personal'' interactions evoke different expectations amongst users, including the expectation to be socially desirable and to have mutual understanding \citep{clark2019makes}. This implicit social expectation can make users quite susceptible to chatbots' performance of social gestures like appreciation, sympathy, and encouragement, all of which predispose users to interpret generated outcomes favorably \citep{norman2008way}. 

Moreover, users' inputs further drive this socio-emotional behavior. Emotional inputs can increase the length of chatbot responses and the instances of socio-emotional cues in output texts, which in turn can stimulate even more emotional responses from users. Thus, the gratuitous use of assistive language, and especially of expressions that signal understanding of pain \citep{urquiza2015mind}, could encourage users to engage in role misplacement, wherein they form unrealistic expectations regarding chatbots' capabilities. Indeed, small grammatical or tonal cues can lead users to misinterpret AI-generated responses as human-written content \citep{jakesch2023human}. This could lead users to mindlessly accept the information generated by AI systems, without critical assessment of the content or its quality. 


\subsection{Prompt-Based Walkthrough Reflections}
The walkthrough method was originally designed to help researchers examine the broader context for technological engagements, drawing on modes of thinking commonly associated with fieldwork-based research. As applied to our study, it enabled us to meaningfully engage with the emergent properties of human-AI interactions, systematically unearthing variations in LLM responses. 
Amid efforts to evaluate LLM impacts based on data and models, this approach emphasizes aspects of LLM systems that are often neglected or overlooked \citep{light2018walkthrough}---namely, the nuanced elements of interactions that characterize generated outputs. The contribution to the CHI community lies in how this qualitative approach can substantiate the in-between, interactive spaces that emerge between users and LLM-based applications, rendering it legible and, eventually, measurable.

The method also had certain incidental outcomes. Consistent with prior studies, even minor changes in prompts can significantly alter responses, potentially leading to biased or culturally specific representations \citep{cheng-etal-2023-marked, tao2024cultural}. The success of our prompt-based walkthrough method in evoking various roles and unearthing various anthropomorphic features highlights how easily LLM responses can be manipulated to produce personalized and human-like expressions. Notably, even when chatbots are designed with de-anthropomorphized features to mitigate misleading outputs, a single prompt can effectively ``jailbreak'' these safeguards, reactivating anthropomorphic traits. This finding could illustrate the challenges of ensuring safety and consistency in human-AI interactions, particularly when users intentionally or unintentionally exploit such vulnerabilities.


% \subsection{Inconsistent Outputs}

% The type of prompts that users input could change the ways chatbots respond to the same information. For instance, a simple example is to compare prompts "Tell me about yourself" and "What is [the name of chatbots]". The former one is likely to return information with personal pronouns and expressions that are commonly used for conversations, whereas the latter generates information in a less personable fashion, simply describing the basic features and functions of given chatbots. Responses could be different by minor changes of prompts. More importantly, despite denying chatbots' capabilities to be conscious, sentient, or emotional, responses tend to include the words that signal such capabilities within the same paragraph. Inconsistency with word usages could potentially lead to additional harms; this type of harm can be categorized as a specific problem of anthropomorphizing AI systems. 

% In the result of generated responses to advice and recommendation prompts, false information is frequently displayed and presented as a confident answer. This could be alarming, particularly with recommendations for reading lists and research, as the list reflect particular ideologies either from developers and training datasets, which non-expert users could not contest or evaluate. 

% In role generations, variety of profiles and scripts are fairly limited, as per of findings from existing studies \citep{jakesch2023human}. For instance, jokes generated by chatbots are typically addressing similar topics despite hypothetical locations to have different cultural norms. 

%Is there an optimal balance between users' efforts and the role of computer-assisted searches, or do different kinds of information retrieval tasks require different degrees of such computer assistance? What sets apart from previous conversational searches is that these AI-assistant tools are more emphasized on specific tasks and roles rather than just simple information retrieval. Such differences could provide an avenue to question the use of anthropomorphic responses or conversations for information retrieval. 


% In voice-based interactions, language uses could become critical components of how users might perceive information, because it adds extra layers of human-like interactions. The evaluations of voice-based interactions might depend on the extent of relation word usage, as such words could provide an avenue for users to feel closeness or friendliness with chatbots. 





 % 1 page for discussion and ablations
\section{Conclusions}
\label{sec:conclusions}

In this work, we proposed \plangen{}, an easily scalable multi-agent approach incorporating three key components: constraint, verification, and selection agents. We leveraged these agents to improve the verification process of existing inference algorithms and proposed three frameworks: Multi-Agent Best of $\mathcal{N}$, ToT, and REBASE. Further, we introduced a Mixture of Algorithms, an iterative framework that integrates the selection agent (Figure \ref{fig:teaser}) to dynamically choose the best algorithm. We evaluated our frameworks on NATURAL PLAN, OlympiadBench, GPQA, and DocFinQA. Experimental results demonstrate that \plangen{} outperforms strong baselines, achieving SOTA results across datasets. Furthermore, our findings suggest that the proposed frameworks are scalable and generalizable to different LLMs, improving their natural language planning ability.


\section*{Limitations}

Despite the strong performance of our frameworks, an area of improvement is the reliance on predefined heuristics for selecting inference-time algorithms, which may not always generalize optimally across all tasks and domains. Additionally, while our frameworks demonstrate strong performance, their computational overhead could be further optimized for efficiency in real-world applications. We believe that our frameworks can be useful in further boosting the planning and reasoning capabilities of existing models such as o1 and Gemini-thinking. In addition, the use of reinforcement learning or meta-learning techniques to dynamically adapt agent strategies based on task complexity could be an interesting area to explore. Moreover, broadening the scope to multi-modal and multi-lingual reasoning would significantly expand the applicability of our approach, and exploring the use of generated planning trajectories for model training offers valuable direction.

\section*{Ethics Statement}

The use of proprietary LLMs such as GPT-4, Gemini, and Claude-3 in this study adheres to their policies of usage. We have used AI assistants (Grammarly and Gemini) to address the grammatical errors and rephrase the sentences.

% \section*{Acknowledgments} % 1 page buffer to fill in tables etc., 

\begin{table*}
\centering
\renewcommand{\dashlinedash}{0.5pt} 
\renewcommand{\dashlinegap}{2pt}    
\begin{tabular}{lllll}
\hline\hline
 & \shortstack[l]{\space \\ \space \\ \textbf{Methods} \\ \space } & 
   \shortstack[l]{\space \\ \space \\ \textbf{Venue} \\ \space } & 
   \shortstack[l]{\space \\ \space \\ \textbf{Scheme} \\ \space } & 
   \shortstack[l]{\space \\ \space \\ \textbf{Cross-stage fusion} \\ \space} \\
\bottomrule   \\
Single-stage
& Zhang et al.~\cite{zhang2018single} & CVPR 2018 & $I \rightarrow [T, R]$ & - \\ 
& ERRNet~\cite{wei2019single} & CVPR 2019 & $I \rightarrow T$ & - \\ 
& RobustSIRR~\cite{song2023robust} & CVPR 2023 & $I_{multiscale} \rightarrow T$ & - \\
& YTMT~\cite{hu2021trash} & NeurIPS 2021 & $I \rightarrow [T, R]$ & - \\   \bottomrule   \\
Two-stage
& CoRRN~\cite{wan2019corrn} & TPAMI 2019 & \shortstack[l]{\space \\$I \rightarrow E_{T}$ \\ $[I, E_{T}] \rightarrow T$} & Convolutional Fusion \\ \cline{2-5}
& DMGN~\cite{feng2021deep} & TIP 2021 & \shortstack[l]{\space \\ $I \rightarrow [T_{1}, R]$ \\ $[I, T_{1}, R] \rightarrow T$} & Convolutional Fusion \\ \cline{2-5}
& RAGNet~\cite{li2023two} & Appl. Intell. 2023 & \shortstack[l]{\space \\ $I \rightarrow R$ \\ $ [I, R] \rightarrow T$} & Convolutional Fusion \\ \cline{2-5}
& CEILNet~\cite{fan2017generic} & ICCV 2017 & \shortstack[l]{\space \\ $[I, E_{I}] \rightarrow E_{T}$ \\ $[I, E_{T}] \rightarrow T$ } & Concat \\ \cline{2-5}
& DSRNet~\cite{hu2023single} & ICCV 2023 & \shortstack[l]{\space \\ $I \rightarrow (T_{1}, R_{1})$ \\ $(R_{1}, T_{1}) \rightarrow (R, T, residue)$} & N/A \\ \cline{2-5}
& SP-net BT-net~\cite{kim2020single} & CVPR 2020 & \shortstack[l]{\space \\ $I \rightarrow [T_{1}, R_{1}]$ \\ $R_{1} \rightarrow R$} & N/A \\ \cline{2-5}
& Wan et al.~\cite{wan2020reflection} & CVPR 2020  & \shortstack[l]{\space \\ $[I, E_{I}] \rightarrow R_{1}$ \\ $R_{1} \rightarrow R$} & N/A \\ \cline{2-5}
& Zheng et al.~\cite{zheng2021single} & CVPR 2021 & \shortstack[l]{\space \\ $I \rightarrow e$ \\ $[I, e] \rightarrow T$} & Concat \\ \cline{2-5}
& Zhu et al.~\cite{zhu2024revisiting} & CVPR 2024 & \shortstack[l]{\space \\ $I \rightarrow E_{R}$ \\ $ [I, E_{R}] \rightarrow T$} & Concat \\ \cline{2-5}
& Language-Guided~\cite{zhong2024language} & CVPR 2024 & \shortstack[l]{\space \\ $[I, Texts] \rightarrow R\ or\ T$ \\ $[I, R\ or\ T] \rightarrow T\ or\ R$} & Feature-Level Concat
\\   \bottomrule   \\
Multi-stage
& BDN~\cite{yang2018seeing} & ECCV 2018 & \shortstack[l]{\space \\ $I \rightarrow T_{1}$ \\ $[I, T_{1}] \rightarrow R$ \\ $[I, R] \rightarrow T$ } & Concat \\ \cline{2-5}
& IBCLN~\cite{li2020single} & CVPR 2020 & \shortstack[l]{\space \\ $[I, R_{0}, T_{0}] \rightarrow [R_{1}, T_{1}]$ \\ $[I, R_{1}, T_{1}] \rightarrow [R_{2}, T_{2}]$ \\ \ldots} & \shortstack[l]{Concat \\ Recurrent} \\ \cline{2-5}
& Chang et al.~\cite{chang2021single} & WACV 2021 & \shortstack[l]{\space \\ $I \rightarrow E_{T}$ \\ $[I, E_{T}] \rightarrow T_{1} \rightarrow R_{1} \rightarrow T_{2}$ \\ $[I, E_{T}, T_{2}] \rightarrow R \rightarrow T$} & \shortstack[l]{Concat \\ Recurrent} \\ \cline{2-5}
& LANet~\cite{dong2021location} & ICCV 2021 & \shortstack[l]{\space \\ $[I, T_{0}] \rightarrow R_{1} \rightarrow T_{1}$ \\ $[I, T_{1}] \rightarrow R_{2} \rightarrow T_{2}$ \\ \ldots} & \shortstack[l]{Concat \\ Recurrent} \\ \cline{2-5}
& V-DESIRR~\cite{prasad2021v} & ICCV 2021 & \shortstack[l]{\space \\ $I_{1} \rightarrow T_{1}$ \\ $[I_{1}, T_{1}, I_{2}] \rightarrow T_{2}$ \\ \ldots \\ $[I_{n-1}, T_{n-1}, I_{n}] \rightarrow T$ } & \shortstack[l]{Convolutional Fusion \\ Recurrent} \\
\hline\hline
\end{tabular}
\caption{\textbf{I}, \textbf{R}, \textbf{T}, and \textbf{E} represent the \textbf{I}nput, \textbf{R}eflection, \textbf{T}ransmission, and \textbf{E}dge map, respectively. The subscripts of \textbf{T} and \textbf{R} represent intermediate process outputs. The Absorption Effect $e$ is introduced in~\cite{zheng2021single} to describe light attenuation as it passes through the glass. The output $residue$ term, proposed in~\cite{hu2023single}, is used to correct errors in the additive reconstruction of the reflection and transmission layers. Language descriptions in~\cite{zhong2024language} provide contextual information about the image layers, assisting in addressing the ill-posed nature of the reflection separation problem.}
\label{tab:1}
\end{table*}

\section{Errata}
\label{sec:errata}
In the first draft of the paper, we had a bug and used the following equation to train and quantize our models:
\begin{equation}
\label{eqn:slicing_old}
   S(q^c, r) = \left(\left\lfloor \frac{q^c}{2^{c - r}} \right\rceil\right) * 2^{c - r}
%   \vspace{-3mm}
\end{equation}
Equation~\ref{eqn:slicing_old} clearly allows an extra bucket to be included into the quantization range, i.e, a $r$-bit model would have $2^r + 1$ possible values instead of $2^r$. For example, consider slicing the first two MSBs from an unsigned int8 value, $234$. As per Equation~\ref{eqn:slicing}, $234$ first gets rounded to $4$, following which it gets clipped to $3$, and finally is scaled up to $3 * 64 = 192$ (Note that \alg int2 allows for $0$, $64$, $128$, $192$). However, since the clipping operation is missing in Equation~\ref{eqn:slicing_old}, $4$ is never clipped down to $3$, and $S(q^c, r)$ is now $4 * 64 = 256$ Thus, for certain int2 values in our final quantized model, we will have to store an extra bit. This is the case with int3, int4 and int6 as well where an extra bit is required to represent certain values. In Table~\ref{tab:old_omniquant}, we can see that the fraction of parameters that fall into this extra bucket is very small. However, for our 2-bit models, this additional bucket gives significant improvements in performance, for example, in Table~\ref{tab:old_omniquant} int2 Gemma-2 9B's average downstream accuracy goes up by $5$\% when trained with an additional bucket (referred to as \epalg in Table~\ref{tab:old_omniquant}). This number is further boosted to $6$\% with co-distillation, as evidenced by Table~\ref{tab:codistill_old}. We hypothesize that this additional bucket helps with capturing the outliers and thus leads to a significant performance boost. As highlighted by recent work~\citep{dettmers2023spqr, squeezellm}, it is crucial to store certain outliers full precision. Interestingly, we show that even a single bit is enough to capture several of these outliers, especially for low bit quantization. Finally, note that this performance boost is not very evident in higher precisions where there are enough buckets to account for the outliers.

\begin{figure}[!t]
  \centering
  \vspace{-2mm}
        \includegraphics[width=0.8\columnwidth]{TabsNFigs/mixnmatch_pdf_EP.pdf}
        % \vspace*{-5pt}
        \vspace{-3mm}
        \caption{Mix'n'Match on Gemma-2 9B model trained using \epalg with OmniQuant as the base algorithm allows elastic pareto-optimal accuracy-vs-cost model extraction for free during deployment.}
\label{fig:omniquant-mnm-old}
\vspace*{-3mm}
\end{figure}

\paragraph{Mix'n'Match} As shown in Figure~\ref{fig:omniquant-mnm-old} with a strong int2 model (i.e., 2.050 bits on average), \epalg Mix’n’Match densely spans the Pareto-optimal accuracy-vs-bits-per-FFN-parameter (memory/cost) trade-off for Gemma-2 9B model trained using MatQuant with Omni-Quant – sometimes even improving on the bfloat16 model accuracy. Consequently, hardware supporting only int2 and int4 data types can still accommodate a model with a memory footprint similar to that of an int3 quantized model, and quality comparable or superior to int3; the additional bits required in the case of int2 can be packed into int2/int4. However, custom CUDA kernel would be required to enable sparse additions of these additional bits to the model weights. 


\section*{Impact Statement}
This paper introduces a novel technique designed to advance the field of machine learning, specifically in the domain of model compression and efficient deployment for large language models. By enabling the creation of versatile, multi-scale models that can operate across various bit-widths, our work has the potential to democratize access to these powerful technologies by making them more resource-efficient and deployable on a wider range of hardware. This could lead to positive impacts such as more sustainable AI systems and greater accessibility for users with limited computational resources. While there are potential risks associated with the broad deployment of powerful AI systems, these are not unique to our work, and we believe the benefits of efficient and accessible AI through innovations like \alg have significant potential for societal good. We encourage further investigation into how novel quantization techniques can play a role in future sustainable AI development.
We thank Rishi Vanukuru, Gregory Yauney, Sayan Ghosh and Jesse Zhang for their invaluable discussions and brainstorming, as well as Malihe Alikhani, Heather Culbertson, Suvodip Dey, Vardhan Dongre, Mert \.Inan, Sunnie Kim, Chaitanya Malaviya,  Ana Marasovi\'c, Weiyan Shi, Anthony Sicilia, Swabha Swayamdipta, Dilek Hakkani-T\"ur and G\"okhan T\"ur for their valuable feedback. 
We also thank~\citet{dhuliawala2023diachronic} for open-sourcing their user study interface, and in particular Vil\'em Zouhar for his assistance with setting up the user study and data logging. 
We are also very grateful to Aisha Baptiste for swiftly adding money to our Prolific account whenever required. 

Tejas Srinivasan was supported by the Amazon ML Fellowship from the USC-Amazon Center for Secure and Trusted Machine Learning. 
This work was partially supported by a grant from the DARPA Friction and Accountability in Conversational Transactions (FACT) program. 
The opinions and findings expressed in this work do not necessarily reflect those of USC, Amazon, or DARPA.

\bibliography{local}

%%%%%%%%%%%%%%%%%%%%%%%%%%%%%%%%%%%%%%%%%%%%%%%%%%%%%%%%%%%%%%%%%%%%%%%%%%%%%%%
%%%%%%%%%%%%%%%%%%%%%%%%%%%%%%%%%%%%%%%%%%%%%%%%%%%%%%%%%%%%%%%%%%%%%%%%%%%%%%%
% APPENDIX
%%%%%%%%%%%%%%%%%%%%%%%%%%%%%%%%%%%%%%%%%%%%%%%%%%%%%%%%%%%%%%%%%%%%%%%%%%%%%%%
%%%%%%%%%%%%%%%%%%%%%%%%%%%%%%%%%%%%%%%%%%%%%%%%%%%%%%%%%%%%%%%%%%%%%%%%%%%%%%%
\newpage
\appendix
\onecolumn
%%%%%%%%%%%%%%%%%%%%%%%%%%%%%%%%%%%%%%%%%%%%%%%%%%%%%%%%%%%%%%%%%%%%%%%%%%%%%%%
%%%%%%%%%%%%%%%%%%%%%%%%%%%%%%%%%%%%%%%%%%%%%%%%%%%%%%%%%%%%%%%%%%%%%%%%%%%%%%%
% APPENDIX
%%%%%%%%%%%%%%%%%%%%%%%%%%%%%%%%%%%%%%%%%%%%%%%%%%%%%%%%%%%%%%%%%%%%%%%%%%%%%%%
%%%%%%%%%%%%%%%%%%%%%%%%%%%%%%%%%%%%%%%%%%%%%%%%%%%%%%%%%%%%%%%%%%%%%%%%%%%%%%%
\newpage
\appendix
\onecolumn

\section{Related Work} \label{app:related_work}
\textbf{Personalized Generation} 
Due to the considerable success of large text-to-image models \cite{ramesh2022hierarchical, ramesh2021zero, saharia2022photorealistic, rombach2022high}, the field of personalized generation has been actively developed. The challenge is to customize a text-to-image model to generate specific concepts that are specified using several input images. Many different approaches \cite{DB, TI, CD, svdiff, ortogonal, profusion, elite, r1e} have been proposed to solve this problem and can be divided into the following groups: pseudo-token optimization \cite{TI, profusion, disenbooth, r1e}, diffusion fune-tuning \cite{DB, CD, profusion}, and encoder-based \cite{elite}. The pseudo-token paradigm adjusts the text encoder to convert the concept token into the proper embedding for the diffusion model. Such embedding can be optimized directly \cite{TI, r1e} or can be generated by other neural networks \cite{disenbooth, profusion}. Such approaches usually require a small number of parameters to optimize but lose the visual features of the target concept. Diffusion fine-tuning-based methods optimize almost all \cite{DB} or parts \cite{CD} of the model to reconstruct the training images of the concept. This allows the model to learn the input concept with high accuracy, but the model due to overfitting may lose the ability to edit it when generated with different text prompts. To reduce overfitting and memory usage, lightweight parameterizations \cite{svdiff, r1e, lora} have been proposed that preserve edibility but at the cost of degrading concept fidelity. Encoder-based methods \cite{elite} allow one forward pass of an encoder that has been trained on a large dataset of many different objects to embed the input concept. This dramatically speeds up the process of learning a new concept and such a model is highly editable, but the quality of recovering concept details may be low. Generally, the main problem with existing personalized generation approaches is that they struggle to simultaneously recover a concept with high quality and generate it in a variety of scenes.

\textbf{Sampling strategies}
Much research has been devoted to sampling techniques for text-to-image diffusion models, focusing not only on personalized generation but also on image editing. In this paper, we address a more specific question: how can the two trajectories -- superclass and concept -- be optimally combined to achieve both high concept fidelity and high editability? The ProFusion paper \cite{profusion} considered one way of combining these trajectories (Mixed sampling), which we analyze in detail in our paper (see Section \ref{sec:mixed_sampling}) and show its properties and problems. In ProFusion, authors additionally proposed a more complex sampling procedure, which we observed to be redundant compared to Mixed sampling, as can be seen in our experiments (see Section \ref{sec:experiments}). In Photoswap \cite{photoswap}, authors consider another way of combining trajectories by superclass and concept, which turns out to be almost identical to the Switching sampling strategy that we discuss in detail in Section \ref{sec:switching_sampling}. We show why this strategy fails to achieve simultaneous improvements in concept reconstruction and editability. In the paper, we propose a more efficient way of combining these two trajectories that achieves an optimal balance between the two key features of personalized generation: concept reconstruction and editability.

\section{Training details} \label{sec:training-details}
The Stable Diffusion-2-base model is used for all experiments. For the Dreambooth, Custom Diffusion, and Textual Inversion methods, we used the implementation from \url{https://github.com/huggingface/diffusers}.

\textbf{SVDiff} We implement the method based on \url{https://github.com/mkshing/svdiff-pytorch}. The parameterization is applied to all Text Encoder and U-Net layers. The models for all concepts were trained for $1600$ using Adam optimizer with $\text{batch size} = 1$, $\text{learning rate} = 0.001$, $\text{learning rate 1d} = 0.000001$, $\text{betas} = (0.9, 0.999)$, $\text{epsilon} = 1e\!-\!8$, and $\text{weight decay} = 0.01$. 

\textbf{Dreambooth} All query, key, and value layers in Text Encoder and U-Net were trained during fine-tuning. The models for all concepts were trained for $400$ steps using Adam optimizer with $\text{batch size} = 1$, $\text{learning rate} = 2e\!-\!5$, $\text{betas} = (0.9, 0.999)$, $\text{epsilon} = 1e\!-\!8$, and $\text{weight decay} = 0.01$. 

\textbf{Custom Diffusion} The models for all concepts were trained for $1600$ steps using Adam optimizer with $\text{batch size} = 1$, $\text{learning rate} = 0.00001$, $\text{betas} = (0.9, 0.999)$, $\text{epsilon} = 1e\!-\!8$, and $\text{weight decay} = 0.01$. 

\textbf{Textual Inversion} The models for all concepts were trained for $10000$ steps using Adam optimizer with $\text{batch size} = 1$, $\text{learning rate} = 0.005$, $\text{betas} = (0.9, 0.999)$, $\text{epsilon} = 1e\!-\!8$, and $\text{weight decay} = 0.01$. 

\textbf{ELITE} We used the pre-trained model from the official repo \url{https://github.com/csyxwei/ELITE} with $\lambda=0.6$ and inference hyperparams from the original paper.

\clearpage
\section{Superclass and concept trajectory choice}\label{app:hyper_theta}

 \begin{wrapfigure}{r}{0.45\textwidth}
    % \centering
    \includegraphics[trim={3cm 10cm 3cm 10cm},clip,width=\linewidth]{imgs/mixed_noft_nosup.pdf}
    \caption{The Pareto frontiers for original Mixed sampling and Mixed sampling in the Superclass, NoFT, and Empty Prompt setups. Mixed NoFT and Mixed Empty Prompt configurations overlap with the Pareto frontier of the original mixed sampling, but primarily in regions associated with low image similarity, which compromises concept fidelity.} \label{fig:mixed_noft_ep}
    \vspace{-0.14in}
\end{wrapfigure}

There are multiple ways to define sampling with maximized textual alignment to the prompt. However, the arbitrary choice can harm the alignment between Base sampling~\ref{eq:concept_sampling} and the selected trajectory. We use the Sampling with superclass (\ref{eq:superclass_sampling}) as it's the default choice in the literature and guarantees the maximized alignment between noise predictions $\tilde{\varepsilon}_{\theta}(p^C)$ and $\tilde{\varepsilon}_{\theta}(p^S)$. 

The several natural ways to adjust Sampling with superclass can be presented by varying $\theta$ and $p^{S}$ in (\ref{eq:superclass_sampling}). We explore two additional options with decreased alignment with (\ref{eq:concept_sampling}): (1) NoFT -- weights of base model $\theta^{\text{orig}}$ instead fine-tuned weights, (2) Empty Prompt -- prompt without any reference to a concept, even to its superclass category, i.e. $p^{\hat{S}} = \textit{"with a city in the background"}$ instead of $p^{S} = \textit{"a backpack with a city in the background"}$.

To validate the robustness of our framework for sampling method selection, we employ the original experimental protocol, supplementing the results shown in Figures~\ref{fig:examples} and~\ref{fig:profusion-photoswap}. Our analysis of Figures~\ref{fig:multi-stage_noft_ep} and~\ref{fig:masked_noft_ep} reveals that trajectories generated under the NoFT and Empty Prompt configurations (second and third columns, respectively) maintain identical method ordering to those produced by Superclass sampling ((\ref{eq:superclass_sampling}), first column).

Notably, Figure~\ref{fig:mixed_noft_ep} shows that Empty Prompt configuration demonstrates weaker alignment with Base sampling compared to NoFT, particularly at higher values of the superclass guidance scale $\omega_{s}$. This divergence manifests as reduced concept fidelity for Empty Prompt under large $\omega_{s}$. These findings highlight a practical adjustment: prioritizing smaller $\omega_{s}$ values in Empty Prompt setup preserves concept fidelity without altering the framework’s core selection logic. 

A key limitation of increased misalignment is the gradual erosion of superclass category information from generated images, which can lead to semantically inconsistent outputs. For instance, Figure~\ref{fig:examples_noft_ep} illustrates how the Mixed Empty Prompt setup, despite the strong animal prior in Base sampling, can produce human-like features in an image of a cat described as \textit{"in a chef outfit"}. This suggests that when superclass information is weakened, the model may introduce unexpected visual artifacts, impacting the fidelity of the intended concept.

Concept sampling (\ref{eq:concept_sampling}) can also be adjusted to better capture a concept’s visual characteristics, further decoupling fidelity from editability. For example, this can be achieved by (1) using the weights of a highly overfitted model (e.g., DreamBooth) or (2) selecting a prompt that omits contextual details, such as $p^{\hat{C}} = \textit{"a photo of V*"}$ instead of $p^{C} = \textit{"a V* with a city in the background"}$. Combining superclass sampling under NoFT or Empty Prompt with Base sampling configured via (1) or (2) could enhance both image and text similarity. We leave this direction for future work.

\begin{figure}[b]
    \centering
    \includegraphics[trim={3cm 10cm 3cm 10cm},clip,width=0.32\linewidth]{imgs/multi-stage_original.pdf}
    \hfill
    \includegraphics[trim={3cm 10cm 3cm 10cm},clip,width=0.32\linewidth]{imgs/multi-stage_noft.pdf}
    \hfill
    \includegraphics[trim={3cm 10cm 3cm 10cm},clip,width=0.32\linewidth]{imgs/multi-stage_nosup.pdf}
    \caption{Pareto Frontier Curves for Mixed, Switching, and Multi-Stage Sampling Methods in the Superclass, NoFT and Empty Prompt setups.
The NoFT and Empty Prompt configurations (second and third columns, respectively) preserve the same method ordering as those produced by Superclass sampling (first column).} \label{fig:multi-stage_noft_ep}
\end{figure}
\begin{figure}[t]
    \centering
    \includegraphics[trim={3cm 10cm 3cm 10cm},clip,width=0.32\linewidth]{imgs/masked_profusion.pdf}
    \hfill
    \includegraphics[trim={3cm 10cm 3cm 10cm},clip,width=0.32\linewidth]{imgs/masked_noft.pdf}
    \hfill
    \includegraphics[trim={3cm 10cm 3cm 10cm},clip,width=0.32\linewidth]{imgs/masked_nosup.pdf}
    \caption{Pareto Frontier Curves for Mixed, Switching, Masked, and ProFusion Sampling Methods in the Superclass, NoFT, and Empty Prompt setups.
The NoFT and Empty Prompt configurations (second and third columns, respectively) preserve the same method ordering as those produced by Superclass sampling (first column).} \label{fig:masked_noft_ep}
\end{figure}

\begin{figure}[b]
    \centering
    \includegraphics[width=\linewidth]{imgs/examples_noft_ep.pdf}
    \caption{Examples of the generation outputs for Mixed and ProFusion sampling methods for their optimal metrics point in the Superclass, NoFT, and Empty Prompt (EP) setups.} \label{fig:examples_noft_ep}
\end{figure}

\clearpage

\begin{figure}[ht!]
  \centering
  \includegraphics[trim={0 5cm 0 5cm},clip,width=0.95\linewidth]{imgs/us_example_new.pdf}
  \caption{An example of a task in the user study}
  \label{fig:us_ex}
  \vspace{-0.19in}
\end{figure}

\section{Data preparation}\label{app:data}
For each concept, we used inpainting augmentations to create the training dataset. We took an original image and automatically segmented it using the Segment Anything model on top of the CLIP cross-attention maps. Then we crop the concept from the original image, apply affine transformations to it, and inpaint the background. We used $10$ augmentation prompts, different from the evaluation prompts, and sampled $3$ images per prompt, resulting in a total of $30$ training images per concept. We commit to open-source the augmented datasets for each concept after publication.

\section{User Study}\label{app:us}

An example task from the user study is shown in Figure~\ref{fig:us_ex}. In total, we collected 48,864 responses from 200 unique users for 16,000 unique pairs. For each task, users were asked three questions: 1) "Which image is more consistent with the text prompt?" 2) "Which image better represents the original image?" 3) "Which image is generally better in terms of alignment with the prompt and concept identity preservation?" For each question, users selected one of three responses: "1", "2", or "Can't decide."

\section{Complex Prompts Setting}\label{app:long_prompts}

We conduct a comparison of different sampling methods using a set of complex prompts. For this analysis, we collected 10 prompts, each featuring multiple scene changes simultaneously, including stylization, background, and outfit:

\adjustbox{max width=\linewidth}{
\begin{lstlisting}
live_long = [
  "V* in a chief outfit in a nostalgic kitchen filled with vintage furniture and scattered biscuit",
  "V* sitting on a windowsill in Tokyo at dusk, illuminated by neon city lights, using neon color palette",
  "a vintage-style illustration of a V* sitting on a cobblestone street in Paris during a rainy evening, showcasing muted tones and soft grays",
  "an anime drawing of a V* dressed in a superhero cape, soaring through the skies above a bustling city during a sunset",
  "a cartoonish illustration of a V* dressed as a ballerina performing on a stage in the spotlight",
  "oil painting of a V* in Seattle during a snowy full moon night",
  "a digital painting of a V* in a wizard's robe in a magical forest at midnight, accented with purples and sparkling silver tones",
  "a drawing of a V* wearing a space helmet, floating among stars in a cosmic landscape during a starry night",
  "a V* in a detective outfit in a foggy London street during a rainy evening, using muted grays and blues",
  "a V* wearing a pirate hat exploring a sandy beach at the sunset with a boat floating in the background",
]

object_long = [
  "a digital illustration of a V* on a windowsill in Tokyo at dusk, illuminated by neon city lights, using neon color palette",
  "a sketch of a V* on a sofa in a cozy living room, rendered in warm tones",
  "a watercolor painting of a V* on a wooden table in a sunny backyard, surrounded by flowers and butterflies",
  "a V* floating in a bathtub filled with bubbles and illuminated by the warm glow of evening sunlight filtering through a nearby window",
  "a charcoal sketch of a giant V* surrounded by floating clouds during a starry night, where the moonlight creates an ethereal glow",
  "oil painting of a V* in Seattle during a snowy full moon night",
  "a drawing of a V* floating among stars in a cosmic landscape during a starry night with a spacecraft in the background",
  "a V* on a sandy beach next to the sand castle at the sunset with a floaing boat in the background",
  "an anime drawing V* on top of a white rug in the forest with a small wooden house in the background",
  "a vintage-style illustration of a V* on a cobblestone street in Paris during a rainy evening, showcasing muted tones and soft grays",
]
\end{lstlisting}
}

The results of this comparison are presented in Figures~\ref{fig:add_long},~\ref{fig:add_long_metrics}. We observe that Base sampling may struggle to preserve all the features specified by the prompts, whereas advanced sampling techniques effectively restore them. The overall arrangement of methods in the metric space closely mirrors that observed in the setting with simple prompts.

\begin{figure}[h!]
  \centering
  \includegraphics[width=\linewidth]{imgs/long_prompts_examples.pdf}
  \caption{Additional examples of the generation outputs for different sampling methods with \textbf{complex prompts}. We highlight parts of the prompt that are missing in Base sampling while appearing in other methods.}
  \label{fig:add_long}
\end{figure}

\clearpage
\section{Dreambooth results}\label{app:dreambooth}

We conduct additional analysis of different sampling methods in combination with Dreambooth. Figure~\ref{fig:add_db_metrics} shows that Mixed Sampling still overperforms Switching and Photoswap,  while Multi-stage and Masked struggle to provide an additional improvement over the simple baseline. Figure~\ref{fig:add_db} shows that all methods allow for improvement TS with a negligent decrease in IS while Mixed Sampling provides the best IS among all samplings.

\begin{figure*}[!ht]
\centering
\begin{minipage}{.477\textwidth}
  \centering
  \includegraphics[trim={3cm 10cm 3cm 10cm},clip,width=\linewidth]{imgs/long_prompts.pdf}
  \caption{CLIP metrics for different sampling methods estimated on \textbf{complex prompts}.}
  \label{fig:add_long_metrics}
\end{minipage}
\hfill
\begin{minipage}{.477\textwidth}
  \centering
  \includegraphics[trim={3cm 10cm 3cm 10cm},clip,width=\linewidth]{imgs/db_samplings.pdf}
  \caption{CLIP metrics for different sampling strategies on top of a Dreambooth fine-tuning method.}
  \label{fig:add_db_metrics}
\end{minipage}
\end{figure*} 

\begin{figure}[h!]
  \centering
  \includegraphics[trim={0 1cm 0 1cm},clip,width=\linewidth]{imgs/db_sampling_examples.pdf}
  \caption{Additional examples of the generation outputs for different sampling methods on top of a Dreambooth fine-tuning method.}
  \label{fig:add_db}
\end{figure}

\section{PixArt-alpha \& SD-XL}\label{app:add_backbones}
We conducted a series of experiments using different backbones. For SD-XL~\cite{podell2023sdxlimprovinglatentdiffusion}, we used SVDDiff as the fine-tuning method, while PixArt-alpha~\citep{chen2023pixartalphafasttrainingdiffusion} employed standard Dreambooth training. Hyperparameters for Switching, Masked, and ProFusion were selected in the same manner as in the experiments with SD2.

Figures~\ref{fig:pixart} and~\ref{fig:sdxl} demonstrate that Mixed Sampling follows a similar pattern to SD2, improving TS without a significant loss in IS. Notably, Mixed Sampling for SD-XL achieves simultaneous improvements in both IS and TS. ProFusion exhibits behavior consistent with SD2, enhancing IS more effectively than Mixed Sampling but performing worse at improving TS while also requiring twice the computational resources. 

\begin{figure}[h]
\centering
\begin{minipage}{.49\textwidth}
  \centering
  \includegraphics[trim={3cm 10cm 3cm 10cm},clip,width=\linewidth]{imgs/pixart.pdf}
  \captionof{figure}{CLIP metrics for different sampling methods estimated on PixArt model.}
  \label{fig:pixart}
\end{minipage}%
\hfill
\begin{minipage}{.49\textwidth}
  \centering
  \includegraphics[trim={3cm 10cm 3cm 10cm},clip,width=\linewidth]{imgs/sdxl.pdf}
  \captionof{figure}{CLIP metrics for different sampling methods estimated on SD-XL model.}
  \label{fig:sdxl}
\end{minipage}
\end{figure}

\clearpage
\section{Cross-Attention Masks}\label{app:cross_attn}

\begin{figure}[h!]
  \centering
  \includegraphics[trim={3cm 0cm 3cm 0cm},clip,width=\linewidth]{imgs/cross_attention_masks.pdf}
  \caption{Visualization of the cross-attention masks for Masked sampling examples. Here, $q$ defines the thresholding quantile and $t$ the denoising step.}
  \label{fig:cross_attn_add_ex}
\end{figure}

\clearpage
\section{Additional Examples}\label{app:add_example}

\begin{figure}[h!]
  \centering
  \includegraphics[trim={0 2cm 0 2cm},clip,width=\linewidth]{imgs/additional_examples.pdf}
  \caption{Additional examples of the generation outputs for different sampling methods.}
  \label{fig:add_ex}
\end{figure}

\begin{figure}[h!]
  \centering
  \includegraphics[trim={0 2cm 0 2cm},clip,width=\linewidth]{imgs/additional_examples_all.pdf}
  \caption{Additional examples of the generation outputs for Mixed and ProFusion sampling methods in comparison to the main personalized generation baselines.}
  \label{fig:add_ex_all}
\end{figure}

\clearpage
\section{DINO Image Similarity}\label{app:add_dino}

We compare CLIP-IS (left column) and DINO-IS~\citep{oquab2024dinov2learningrobustvisual} (right column) in Figures~\ref{fig:profusion_photoswap_dino},~\ref{fig:all_methods_dino}. We observe that despite the choice of metric, different sampling techniques and finetuning strategies have the same arrangement. The most noticeable difference is that SVDDiff superiority over ELITE and TI is more pronounced. That strengthens our motivation to select SVDDiff as the main backbone.

\begin{figure}[h]
\centering
\begin{minipage}{.49\textwidth}
  \centering
  \includegraphics[trim={3cm 10cm 3cm 10cm},clip,width=\linewidth]{imgs/profusion_photoswap.pdf}
\end{minipage}%
\hfill
\begin{minipage}{.49\textwidth}
  \centering
  \includegraphics[trim={3cm 10cm 3cm 10cm},clip,width=\linewidth]{imgs/profusion_photoswap_dino.pdf}
\end{minipage}
\caption{Pareto frontiers curves for Photoswap~\citep{photoswap} and ProFusion~\citep{profusion}.}\label{fig:profusion_photoswap_dino}
\end{figure}

\begin{figure}[h]
\centering
\begin{minipage}{.49\textwidth}
  \centering
  \includegraphics[trim={3cm 10cm 3cm 10cm},clip,width=\linewidth]{imgs/all_methods.pdf}
\end{minipage}%
\hfill
\begin{minipage}{.49\textwidth}
  \centering
  \includegraphics[trim={3cm 10cm 3cm 10cm},clip,width=\linewidth]{imgs/all_methods_dino.pdf}
\end{minipage}
\caption{The overall results of different sampling methods against main personalized generation baselines.}\label{fig:all_methods_dino}
\end{figure}


\end{document}