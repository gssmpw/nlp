% !TEX root = main.tex

\section{Problem setup}
\label{sec:background}
We formally define the problem setting and provide necessary background. We use the following notation: scalars are denoted by such italic lower case letters as $x,y$, Euclidean vectors by bold $\bx, \by$, and matrices by upper case $X, Y$, etc. $\| \cdot \|_p $ denotes the $\ell_p$-norm for $p \in [1, \infty]$. $\bm{1}$ refers to the all-one vector. For $T \in \naturals$, $[T] \define \{1, \ldots, T \}$, and for a sequence $(x_t)_{t \geq 1}$, define $x_k^t \triangleq (x_k, \ldots, x_t)$ if $k \geq 1$ and $(x_1, \ldots, x_t)$ otherwise. For $z \in \reals$,  $\sigmoid(z) \triangleq 1/(1+e^{-z}), \relu(z) \triangleq \max(0,z)$, and $\softplus(z) \define \log(1+e^z)$. $\mathrm{Unif} (S)$ denotes the uniform distribution over a set $S$ and $\mathrm{Dir}(\bbeta)$ denotes the Dirichlet distribution with parameter $\bbeta >0  $. $\KL{P}{Q}$ denotes the KL-divergence between distributions $P$ and $Q$. \looseness=-1
\subsection{Input data: Random Markov chains}
\label{sec:markov_background}
To investigate the ICL capabilities of Mamba, we let the input tokens to be stochastic and drawn from a random Markov chain of order $k$ \cite{edelman2024evolution}. That is, the token sequence $x = (x_t)_{t=1}^T \in \calX^T$ on the state space (vocabulary) $\calX$ follows the transition dynamics: 
\begin{align}
\prob{x_{t+1} = \cdot \mid x_1^{t}} = \prob{x_{t+1} = \cdot \mid x_{t-k+1}^{t}}, 
\end{align}
almost surely for all $t \in [T]$, and the \kth Markov kernels, $ \prob{x_{t+1} = \cdot \mid x_{t-k+1}^{t}=i_{t-k+1}^{t}}$, are sampled independently for each tuple $(i_{t-k+1},\cdots,i_t)$ from the Dirichlet prior $\dir{\beta}$, with $\beta >0$. When $\beta=1$, this corresponds to the uniform distribution on the $S$-dimensional simplex $\Delta_1^S$, where size $S=|\calX|$. 

% (\bP \pth{\cdot \mid i_1^k} )
The transition matrix $P =(P_{i_1^k})_{i_1^k \in \calX^k}, P_{i_1^k} \in [0,1]^S$, encapsulates the set of all $S^k$ conditional probabilities of the chain, each row corresponding to one of them. While this transition matrix governs the generation of each token $x_t$ for $t >k$, the first $k$-tokens $x_1,\ldots,x_k$ are drawn \iid from $\unif{\calX}$. This constitutes the joint law of the random variables $(P,x)$, termed random Markov distribution henceforth. More succinctly, 


\noindent {\bf Data generation} (Random Markov sequences).
\vspace{-0.5em}
    \begin{enumerate}
        \item Draw $P$ with each row sampled \iid from $\dir{\beta}$. \vspace{-0.5em}
        \item For $t=1,\ldots,k$, sample $x_t \sim \unif{\vocab}$.
        \vspace{-0.5em}
        \item For $t=k, \ldots, T$, sample $x_{t+1} \sim P_{x_{t-k+1}^t} $.
        \vspace{-0.5em}
        \item Return the input $x=(x_t)_{t=1}^T$.
              \vspace{-0.5em}
        \item Repeat the above steps to generate a batch $\{x^{(b)}\}_{b \in [B]}$.
    \end{enumerate}

\noindent {\bf Relation to ICL.} As a consequence of the generation process, no two sequences $x$ share the same transition matrix $P$ and hence every sequence follows a different Markov distribution. Owing to this fact, even when a model is trained on this random Markovian data, at inference, for every test sequence it has to estimate the next token in-context. Hence this data class serves as a good sandbox to gauge the ICL capabilities of Mamba, which was also used in a similar context for transformers \cite{nic2024trans}. In this paper, we let the state space $\vocab = \binary$ for the ease of exposition and the order $k$ to be any $k \geq 1$. Our results and insights can be readily extended to any finite vocabulary, as demonstrated for the natural language in \prettyref{sec:english}.


% Since a \kth Markov chain on $\vocab$ is equivalent to a $(k \cdot \log |\vocab|)^{\text{th}}$-order process on the binary state space $\binary$, in this paper, without loss of generality, we study binary Markov chains of order $k \geq 1$. \nb{Michael}.

% \begin{figure}[t]
%     \centering
%     \section{Data Generation}\label{sec:data_gen}

% The dataset contains synthetically generated problem instances. 
To generate a problem instance, we specify (i) the number of agents $n$, (ii) the interaction rate between agents $p_{\operatorname{int}}$, (iii) the maximum number of paths each agent can generate $n_d$, (iv) the minimum number of solutions $n_{sol}$ we want the problem instance to have, and (v) a random seed $\sigma$ since the generation process is not deterministic.

The parameters $n$ and $p_{\operatorname{int}}$ determine the stochastic generation of the interaction graph $\mathcal{G}_I$: $n$ corresponds to the number of nodes of $\mathcal{V}_I$ while $p_{\operatorname{int}}$ corresponds to the probability of an edge between two nodes, and therefore it varies between 0 and 1. $\mathcal{G}_I$ is generated as a connected graph because, when it is not connected, our algorithm can be applied to each connected component independently. To generate $\mathcal{G}_I$ as random connected graph, we operate the following steps: we first create a tree with nodes $\mathcal{V}_I$ and then we add random links between nodes with probability $p_{\operatorname{int}}$.
%is the same as the one for generating a Erd\H{o}s-R\'enyi graph \cite{Erdos:1959:pmd}, except that we enforce the resulting graph to be connected by adding one random edge for each node before adding any other edge.

The parameters $n_d$ and $n_{sol}$ are instead used to generate the constraint graph $\mathcal{G}_C$. Each agent in $\mathcal{G}_I$ generates a random number of paths between $1$ and $n_d$. Each path $d$ is associated with a utility value $u(d)$. We assume that each agent has one preferred path and other less desirable ones. The former has a utility value equal to 1, while the latter have a small utility value equal to 0.1. This choice follows a standard practice for experimentation in multi-alternative decisions in which the less-preferred options are considered distractors with an equally low value~\cite{Reina:2017jl}.
%Possible values for $u_h$ are either $0.1$ or $1$, with the constraint that exactly one path among the generated ones has a value of $1$. The rationale behind this design choice...  } 
The set of paths of all the agents correspond to the nodes $\mathcal{V}_C$ of $\mathcal{G}_C$.
The links in $\mathcal{G}_C$ are inserted according to the following procedure: we first construct $n_{sol}$ different solutions (see below), and the links constituting such solutions are then added to $\mathcal{G}_C$. At this point not all the nodes in $\mathcal{G}_C$ have a link and thus we randomly add one link for each node with degree $0$ to avoid having paths that are not compatible with any other path. When adding random links, we must take into account the interactions in $\mathcal{G}_I$. Indeed, if two agents do not interact, there is no reason to check the compatibility of their paths and thus we only randomly add links between paths of neighbouring agents.

The construction of a solution to the problem is straightforward: it is enough to randomly select one path per agent and then add all the possible links between paths of neighbouring agents in $\mathcal{G}_C$. Note that even if we add $n_{sol}$ solutions to a problem instance, the total number of solutions can be greater than $n_{sol}$ because the union of the links constituting two solutions can generate several additional solutions, especially if the number of nodes in $\mathcal{V}_C$ is small. This means that the total number of solutions in a problem instance cannot be controlled at generation time but must be computed after the generation, in a centralised fashion, by means of the CPLEX solver, as described in Section~\ref{sec:centralised_appr}. Figure 1 in the main text shows the distribution of the number of solutions in the problem instances of our dataset.  Note that 
%$n_{sol}$ determines the minimum number of solutions, but not the total number, as shown in Figure~\ref{fig:sol_distr}. Indeed, 
the highest number of solutions is found in instances in which $n$ is not much larger than $n_{sol}$, with instances in the group determined by $n=20$ and $n_{sol}=10$ having up to $10^4$ possible solutions. Instead, when $n$ is much larger than $n_{sol}$ (e.g., $n=100$ and $n_{sol}=10$), only the minimum number of solutions $n_{sol}$ is present.

We generated problem instances by fixing the interaction rate $p_{\operatorname{int}}=0.3$ and the maximum number of paths per train $n_d=8$. These values have been selected empirically to obtain a sufficiently rich topology for $\mathcal{G}_I$ and $\mathcal{G}_C$. Then, we generate problem instances by varying $n \in \{10, 20, 50, 100\}$ and $n_{sol}\in\{3, 5, 10\}$. For each combination, we generate 100 instances by varying the random seed $\sigma\in\{0,\ldots,99\}$. The seed $\sigma$ is  used to make the generation process reproducible. Overall, the total number of problem instances in our dataset is $1200$.\footnote{The dataset will be publicly available with the aim of stimulating research on dec-rtRTMP.}

%     \caption{Generation of random Markov chains.}
%     \label{fig:markov_gen}
% \end{figure}

\subsection{Mamba architecture}
\label{sec:mamba_background}
Selective SSMs such as Mamba and Mamba-2 are a class of sequence-to-sequence models that are closely related to RNNs and classical state space models \cite{mamba2023gu}. A key feature underpinning these models is the \emph{selectivity mechanism}, enabling them to selectively choose inputs at every timestep, as opposed to linear time-invariant (LTI) systems. While we believe our work captures the behavior of all selective SSMs, we will specifically focus on the state-of-the-art Mamba-2 model to simplify exposition. By slight abuse of terminology, henceforth we will also refer to this model simply as Mamba. Mathematically speaking, Mamba implements the sequence-to-sequence mapping $\mamba: \reals^{d \times T} \mapsto \reals^{d \times T} $, where given a sequence of input embeddings $\bx = (\bx_t)_{t=1}^T \inr{d\times T}$ of dimension $d$, it outputs the corresponding output embeddings $\bo = (\bo_t)_{t=1}^T \inr{d\times T}$ of the same dimension with $\bo = \mamba(\bx) $. More precisely, fix $t \in \set T$. Then the output $\bo_t$ at time $t$ is computed as $ \bo_t = \mamba(\bx_1^t)$ using the following recurrence equations \cite{dao2024transformers}:
\begin{equation}
\begin{split}
H_t &= a_t \, H_{t-1} + \tilde{\bx}_t \, \bb_t^\top \inr{ed \times N}, \\
\by_t &= H_t\, \bc_t \inr{ed}, \\
\bz_t &= \by_t \odot \relu(W_z \, \bx_t) \inr{ed}, \\
\bo_t &= W_o\, \bz_t \inr{d},
\end{split}
\label{eq:mamba_block}
\tag{Mamba}     
\end{equation}
where the initial state $H_0 = 0$, $W_z \inr{ed \times d}$ and $W_o \inr{d\times ed}$ are learnable parameters, and the input-selective
\begin{equation}
\begin{split}
a_t &\define \exp(-a \cdot \Delta_t) \in (0,1), \, \text{with} \\
\Delta_t &\define \softplus(\inner{\bw_{\Delta}}{\bx_t}+ \delta) \inr{}, \\
\tilde{\bx}_t  &\define \relu(\conv_X(W_X \, \bx_{t-w+1}^t)) \cdot \Delta_t \inr{ed}, \\
\bb_t &\define \relu(\conv_B(W_B \, \bx_{t-w+1}^t)) \inr{N}, \\
\bc_t &\define \relu(\conv_C(W_C \, \bx_{t-w+1}^t)) \inr{N},
\end{split}
\tag{Input selectivity} 
\label{eq:mamba-params}
\end{equation}
where $a \geq  0, \bw_{\Delta}\inr{d}, \delta\inr{}, W_X\inr{ed \times d}, W_B\inr{N \times d}$ and $W_C\inr{N\times d}$ are all learnable parameters and $\conv(\bz_{t-w+1}^t)$ is a (learnable) time-wise convolution of 
window $w \in \naturals$ with distinct kernels per dimension. Here $e\in\mathbb{N}$ denotes the expansion factor for the features, typically $2$. Let $\thetamamba$ denote the set of all these parameters. \looseness=-1

\noindent {\bf Intuition behind Mamba.} While the update equations in \ref{eq:mamba_block} might seem complicated at a first glance, the underlying intuition is simple: given a sequence of input embeddings $(\bx_t)$, we first capture their local temporal information using three separate convolutions to compute $\tilde{\bx}_t, \bb_t$, and $\bc_t$ respectively (\ref{eq:mamba-params}). Equipped with this local memory, we perform the classical linear state update to compute the current state $H_t$ from the past $H_{t-1}$, weighed by an input-dependent decay factor $a_t \in (0,1)$, and $(\tilde{\bx}_t, \bb_t)$. Subsequently, we compute the state projection $\by_t$, modulate it with an input-selective ReLU term to yield $\bz_t$, and finally project it down to get the $d$-dimensional output embedding $\bo_t$. Thus the output $\bo_t$ at time $t$ is a function of the entire input sequence till then, $\bx_1^t$, yielding $ \bo_t = \mamba(\bx_1^t)$. \looseness=-1

\begin{figure}[t]
\centering
   \scalebox{0.8}{\definecolor{embed}{HTML}{AEF78E}
\definecolor{attn}{HTML}{f9e3ae}
\definecolor{ff}{HTML}{90F3FF}
\definecolor{lin}{HTML}{E15CEC}
\definecolor{mask1}{HTML}{4D4847}
\definecolor{mask2}{HTML}{30343F}

\begin{tikzpicture}[>={Latex[length=1mm,width=1mm]}, node distance=1.5cm]
        % x_i
        \node (x1) {$x_1$};
        \node[right=0.5cm of x1]    (dotsl_x) {$\dots$};
        \node[right=0.5cm of dotsl_x] (xt)    {$x_t$};
        \node[right=0.5cm of xt]    (dotsr_x) {$\dots$};
        \node[right=0.5cm of dotsr_x] (xT)    {$x_T$};
        \node[right=-0.15cm of xT]       (set)  {$\in \{0,1\}$};
        % embed
        \foreach \i in {1,t,T}{
            \node[above=0.3cm of x\i, minimum height = 0.6cm] (embed\i) [draw, rectangle, fill=embed!70] {Embedding};
        }
        \node[above=1.6cm of x1] (x1-corner) {};
        \node[above=1.6cm of xt] (xt-corner) {};
        \node[above=1.6cm of xT] (xT-corner) {};
        % small mambda blocks connected by arrows
        \foreach \i in {1,t,T}{
            \node[above=2cm of x\i, minimum height = 0.8cm] (attn\i) [draw, rectangle, fill=attn!50] {Mamba};
        }
        \draw[->] (attn1.east) -- (attnt.west); 
        \draw[->] (attnt.east) -- (attnT.west); 
        % \node[above=2cm of xt, minimum width = 6cm, minimum height = 0.8cm] (attn) [draw, rectangle, fill=attn!50] {Mamba};   %% old `Attention' block
        %% shaded attention from xmfr, removed for mambda
        % \fill[draw=attn!70!orange, pattern={Lines[angle=12, line width=2pt, distance=0.5mm]},pattern color=attn!80!orange] ([yshift=0cm]x1-corner.south) -- ([yshift=0cm]xt-corner.south) -- (attn.south -| embedt.north) -- ([yshift=0cm]x1-corner.south) -- cycle;
        % \fill[draw=attn!70!orange, pattern={Lines[angle=6, line width=2pt, distance=0.5mm]},pattern color=attn!80!orange] ([yshift=0cm]x1-corner.south) -- ([yshift=0cm]xT-corner.south) -- (attn.south -| embedT.north) -- ([yshift=0cm]x1-corner.south) -- cycle;
        % big box containing everything
        \node[above=1.6cm of xt, minimum width = 7cm, minimum height = 3.2cm] (xfmr) [draw, rectangle] {};
        % arrows for \bx
        \foreach \i in {1,t,T}{
            \draw[->] (x\i) -- (x\i |- embed\i.south);
            \draw[->] (embed\i.north) -- (attn\i.south -| embed\i.north) node[pos=0.3,fill=white] (bx\i) {$\bx_{\i}$}; 
        }
        % dots for \bx
        \path (bx1) -- (bxt) node[midway] (midpoint) {$\dots$};
        \path (bxt) -- (bxT) node[midway] (midpoint) {$\dots$};
        % FF and arrows for \by
        \foreach \i in {1,t,T}{
            \node[above=3.6cm of x\i, minimum height = 0.6cm] (FF\i) [draw, rectangle, fill=ff] {MLP};
            \draw[->] (attn\i.north -| FF\i.south) -- (FF\i.south) node[pos=0.45,fill=white] (by\i) {$\bu_{\i}$}; 
        }
        % dots for \by
        \path (by1) -- (byt) node[midway] (midpoint) {$\dots$};
        \path (byt) -- (byT) node[midway] (midpoint) {$\dots$};
        % linear and arrows for \bz
        \foreach \i in {1,t,T}{
            \node[above=0.85cm of FF\i, minimum height = 0.6cm] (lin\i) [draw, rectangle, fill=lin!60] {Linear};
            \draw[->] (FF\i.north) -- (lin\i.south) node[pos=0.35,fill=white] (bz\i) {$\bv_{\i}$}; 
        }
        % dots for \bz
        \path (bz1) -- (bzt) node[midway] (midpoint) {$\dots$};
        \path (bzt) -- (bzT) node[midway] (midpoint) {$\dots$};
        % softmax and arrows for \logit
        \foreach \i in {1,t,T}{
            \node[above=0.83cm of lin\i, minimum height = 0.6cm] (soft\i) [draw, rectangle] {$\sigma(\cdot)$};
            \draw[->] (lin\i.north) -- (soft\i.south) node[pos=0.43,fill=white] (logit\i) {$\logit_{\i}$}; 
        }
        % dots for \logit
        \path (logit1) -- (logitt) node[midway] (midpoint) {$\dots$};
        \path (logitt) -- (logitT) node[midway] (midpoint) {$\dots$};
        % \bbP and arrows and dots
        % \node[above=0.4cm of soft1, minimum height = 0.6cm] (prob1) {\scalebox{0.8}{$\bbP_{\btheta}(x_{2} = 1 \mid x_1)$}};
        % \node[above=0.4cm of softn, minimum height = 0.6cm] (probn) {\scalebox{0.8}{$\bbP_{\btheta}(x_{n+1} = 1 \mid x_1^n)$}};
        % \node[above=0.4cm of softN, minimum height = 0.6cm] (probN) {\scalebox{0.8}{$\bbP_{\btheta}(x_{N+1} = 1 \mid x_1^N)$}};
        \foreach \i in {1,t,T}{
            \node[above=0.25cm of soft\i, minimum height = 0.6cm] (prob\i) {$f_{\btheta}(x_{1}^{\i})$};
            \draw[->] (soft\i.north) -- (prob\i);
        }
        \path (prob1) -- (probt) node[midway] (midpoint) {$\dots$};
        \path (probt) -- (probT) node[midway] (midpoint) {$\dots$};
    \end{tikzpicture}} 
    \caption{Mamba-based language model with binary input data: for each $t\in [T]$, the next-token prediction probability is $\predprob= \mathbb{P}_{\btheta}\pth{x_{t+1}=\cdot \mid x_1^t}$.}
    \label{fig:mamba}
\end{figure}

\noindent  {\bf Mamba-based language model.}  \ref{eq:mamba_block} block is then incorporated into a full-fledged language model as follows: let $ x= (x_1, x_2, \cdots, x_T) \in \vocab^T $ be an input token-sequence over the alphabet $\calX$; here $\calX= \binary$ as explained in \prettyref{sec:markov_background}. Then, at every $t\in [T]$, the output of the language model $\btheta$ is given by the following sequence of equations \cite{dao2024transformers}:
\begin{align}
\bx_t  &= \be
_{x_t} = x_t \, \be_1 + (1-x_t) \, \be_0 \inrd, \tag{{Embedding}}     \label{eq:embedding}  \\
\bu_t &= \bx_t + \mamba(\bx_1^t) \inrd, \tag{{Mamba}}\label{eq:mamba} \\
\bv_t &= \bu_t + W_2[ \relu(W_1 \bu_t) + \odot W_3 \bu_t ] \inrd, \tag{{\small MLP}} \label{eq:mlp} \\
\logit_t &= W_{\ell}\, \bv_t \inr{2}, \tag{{Linear}} \label{eq:logit} \\
f_{\btheta}(x_1^t) & \define \mathbb{P}_{\btheta}\pth{x_{t+1}= \cdot \mid x_1^t} = \softmax(\logit_t) \in [0,1]^2, \tag{{Prediction}} \label{eq:prediction}
\end{align}
where the parameters $\be_0, \be_1 \inrd, W_1 \inr{4d\times d}, W_2 \inr{d\times 4d}$ and $W_{\ell} \inr{2\times d}$ are learnable, and $\predprob$ is the probability law for the next symbol $x_{t+1}$ conditioned on the past $x_1^t$. We omit the layer norm here for simplicity. We compactly denote the set of all model parameters as $\btheta$, \ie $\btheta= (\be_0, \be_1, \thetamamba, W_{1,2,3},  W_\ell) \inr{D} $. \looseness=-1


\subsection{Learning task: next-token prediction}
With the objective of auto-regressively estimating the next token,  we train the model parameters $\btheta$ to minimize the cross-entropy loss between the next-token predicted probability $\predprob$ and the corresponding ground-truth symbol $x_{t+1}$ across all the positions $t \in \set T$:
\begin{equation}
L(\btheta) \define -\frac{1}{T} \sum_{t \in \set T} \Expect_P \Expect_{x_1^{t+1} \sim P}\big[x_{t+1} \cdot \log f_{\btheta}^{(1)} (x_1^t) + (1- x_{t+1}) \cdot \log f_{\btheta}^{(0)}(x_1^t) \big],
\label{eq:loss}
\end{equation}
where $f_{\btheta}^{(j)} (x_1^t) \define \mathbb{P}_{\btheta}\pth{x_{t+1}=j \mid x_1^t}$ for $j\in\binary$, and the expectation is both over the transition kernels $P$ and the Markov sequences $x= (x_t)_{t=1}^T$ sampled from $P$. In practice, it is replaced
by empirical average across a finite set of batches, sampled according to the random Markov distribution in \prettyref{sec:markov_background}. For our experiments we use the AdamW optimizer \cite{Kingma2017}. 
% At inference, the trained model is evaluated on a fresh Markov sequence of length $T$, fixed beforehand.

\subsection{Optimal estimator: Laplacian smoothing}
\label{sec:laplace}
Given the Bayesian prediction loss in \prettyref{eq:loss}, it is natural to ask: \emph{what's the optimal $\btheta$ minimizing it?} It follows from a classical result in statistics (\cite{rissanen1984}, \S~\ref{app:laplace}) that this minimum is achieved when the corresponding model prediction matches the (averaged) ground-truth predictive distribution, \ie $\mathbb{P}_{\btheta}\pth{x_{t+1}= 1 \mid x_1^t} = \Expect_{P|x_1^t} \qth{ \prob{x_{t+1} =1 \mid x_1^t}} $, for all $t$. Given the joint distribution of the pair $(P, x_1^{t+1})$ in \prettyref{sec:markov_background}, where the kernel $P \sim \dir{\beta}$, it can be shown (\S~\ref{app:laplace}) that the conditional expectation above simplifies to the well-known \emph{Laplacian smoothing}, also known as the \emph{add-$\beta$ estimator} (see e.g. \cite{merhav1998}):
\begin{align}
\mathbb{P}_\beta^{(k)}\pth{x_{t+1} = 1 \mid x_1^{t}}   \define \Expect_{P|x_1^t} \qth{ \prob{x_{t+1} =1 \mid x_1^t}}  =  \frac{n_1 + \beta}{n + 2 \beta},   
\tag{Laplacian smoothing}
\label{eq:laplace_smooth}
\end{align}
where $n_1$ is the number of times token $1$ follows the current \kth context $x_{t-k+1}^t$ in the sequence $x_1^t$, \ie $n_1 = | \{ i: (x_{i-k}^{i-1}, x_i) = (x_{t-k+1}^t, 1) \}  |  $ and $n$ is the frequency of this context, \ie $ n = | \{ i: x_{i-k}^{i-1} = x_{t-k+1}^t \}  | $. Adjusting these counts by $\beta$ plays the role of additive smoothing, which avoids assigning zero probabilities to unseen events, an idea dating back to Laplace \cite{laplace1814essaiprob}. It is also known that the add-$\beta$ estimator is asymptotically minimax optimal, as $T\to\infty$ \cite{xie-barron-1997, orlitsky2018mc}. If Mamba realizes this smoothing estimator, \ie $\mathbb{P}_{\btheta} = \mathbb{P}_{\beta}^{(k)}$, it automatically implies its ICL abilities: given a fresh test sequence at inference, in order to optimally predict the next token, it processes the input tokens in-context to compute the relevant counts, as in the \ref{eq:laplace_smooth}. \emph{But does it realize in practice this optimal smoothing-based counting estimator?}