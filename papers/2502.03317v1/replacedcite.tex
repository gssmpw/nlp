\section{Related Work}
In this section, we review the relevant research from two perspectives: the contact-aware motion planning approaches for robots and the essential feasible trajectory optimization methods. In Section
\ref{sec-related_contact_traj}, we primarily discuss the significance of contact in various applications and the challenges that arise in the planning field due to the lack of contact consideration. Regarding feasible trajectory optimization (see Section \ref{sec-related_feasible_traj}), we highlight how overly conservative feasibility criteria can limit the motion capabilities of mobile robots.
%feasible trajectory optimization and contact-aware motion planning. In Section \ref{sec-related_feasible_traj}, we highlight how overly conservative feasibility criteria can limit the motion capabilities of mobile robots. Regarding contact-aware motion planning (see Section \ref{sec-related_contact_traj}), we primarily discuss the significance of contact in various applications and the challenges that arise in the planning field due to the lack of contact consideration.
\subsection{Contact-Aware Trajectory Planning for Robots}
\label{sec-related_contact_traj}
As figured in Fig.\ref{fig-introduction}, contact-aware motion planning methods can develop the locomotion of mobile robots and accomplish tasks that conservative planning methods cannot. Two categories of tasks are particularly essential: navigation among movable objects (NAMO) and rearrangement of movable objects (RAMO). 
On the one hand, compared to basic navigation tasks, NAMO tasks require mobile robots to fully utilize the interactable information of objects in the scene to complete the navigation efficiently. %[CITE]. 
On the other hand, the RAMO skill specifies a desired target state for the movable objects. The main difference between RAMO and NAMO is that the target state of the movable objects can be arbitrary for the latter but not for the former. NAMO and RAMO are essential because they are fundamental sub-tasks for constructing more complex contact-rich tasks, such as exploring an unknown environment, %[CITE], 
completing daily chores in an open space, %[CITE], 
and so on. Contact-aware motion planning approaches must balance the cost of contact with movable objects and the cost of avoiding them to generate reasonable trajectories.
%Compared to basic navigation tasks, NAMO provides additional information on movable objects to mobile robots. Motion planning must balance the cost of contact with movable objects and the cost of avoiding them to generate reasonable trajectories. The main difference between RAMO and NAMO is that the former specifies a desired target state for the movable objects. In contrast, in the latter, the target state of the movable objects can be arbitrary. NAMO and RAMO are essential because they are fundamental sub-tasks for constructing more complex contact-rich tasks.

For these two types of tasks, much of the relevant research has focused on allocating reasonable discrete action sequences and determining appropriate target states ____. Planning strategies based on searching or sampling methods often face the challenge called the curse of dimensionality, as each state node generates two different motion modes due to whether contact occurrence ____. Additionally, sparse action sequences can amplify the potential collision risks between the robot and objects. Feasible continuous reference trajectories can improve the success rate of contact-rich tasks. %[CITE].

In the research on motion planning of robotic systems with multiple motion modes and control strategies for bipedal robots, the employment of contact-aware methods is no longer a secret. Posa et al. ____ proposed a direct method that can produce continuous trajectories with contact properties for robots without predefined motion modes. In a recent study focused on wheeled balancing robots ____, contact switches and impacts are incorporated into the trajectory optimization to adapt to discontinuous terrains. 

Furthermore, there are data-driven approaches ____ that attempt to actively consider contact in the motion model to drive robot arms to accomplish complex tasks. These methods, independently and coincidentally, employ complementarity constraints as a concise and elegant form to describe contact behaviors. Although complementarity constraints introduce additional complexity to the trajectory planning problem, they eliminate a prior specification of a mode schedule ____. Following this formalization, we introduce complementarity constraints into the motion planning of mobile robots, enabling desired contact between the robot and movable objects.

\subsection{Feasible Trajectory Optimization of Robots}
\label{sec-related_feasible_traj}
In the contact-aware motion planning problem for robots, the feasibility of trajectories is reflected in two aspects: the consistency of states with system dynamics and the compliance of robot-environment interaction with the physical laws. 
The system dynamics is often represented as equality constraints in nonlinear programming, approximating the relationship between adjacent discrete system states ____. For instance, in direct collocation methods, system dynamics are imposed with derivative constraints at collocation points ____. Alternatively, the system dynamics equation can be utilized to simulate the evolution of the robot starting from a given initial state ____. This approach requires high-performance numerical integration algorithms and often employs error control methods to ensure the accuracy of the simulation ____.

Most trajectory planning methods assume that contact between the robot and objects is prohibited. However, it is a highly conservative assumption regarding the feasibility of trajectories, as it implies that there is no need to consider the physical laws of contact ____. Under this assumption, the mobile robot and obstacles are abstractly represented as combinations of convex ____ or non-convex ____ geometric shapes. By detecting collisions between these geometric entities, the robot can be constrained to avoid collisions with obstacles. Additionally, we can enforce collision-free motion by constraining the robot's trajectory within a virtual safety corridor ____. This assumption is reasonable for some applications, such as cruising or racing ____. However, contact between the robot and objects is inevitable or even necessary for various tasks, such as coverage planning ____, or heavy boxes pushing ____. Therefore, a more general contact-aware motion planning method is necessary to explore the potential locomotion of mobile robots further. 

The main challenges for the compliance of contact between robots and movable objects lie in the fidelity of the complex friction model and the accurate determination of the contact state. In the study of kinematics with the involvement of friction, the quasi-static approximation assumes that frictional contact force dominates at low velocities and that inertial forces do not play a decisive role in the motion. %[CITE]. 
The limit surface, proposed by S. Goyal et al ____. is a geometric representation for mapping the friction force acting on an object to the resulting velocity. Hogan et al. ____ systematically analyze the kinematics of the pusher-slider system based on the concepts of quasi-static assumptions, the limit surface, and the motion cone, and propose an efficient linearization strategy to improve the computational efficiency. In addition, for complex robotic systems with multiple motion modes, Screenath et al. ____ defined transition maps to determine which mode a given system state is in. Characterizing the state of a system with multiple motion modes using a form of complementary constraints is another elegant way. %[CITE]. 
In this work, we follow the concepts of quasi-static assumptions, limit surface, etc. We combine them with nonlinear complementarity constraints to comprehensively model the contact behavior and propose a road map to solve such a complex optimization problem.

\begin{figure*}[th]
	\centering
	\includegraphics[width=1.0\textwidth]{figs/fig-camp_overview.pdf}
	\caption{The overview of the CAMP framework. The figure illustrates the workflow of CAMP. The perception module of the robot provides scene understanding of the CAMP framework. In the front-end of CAMP, the appropriate searching method is selected based on specific tasks. The method calculates a sequence of states and appropriate time allocations. The back-end optimization, using the ALM as the backbone, first solves an unconstrained optimization problem and then computes energy-optimal continuous polynomials based on the local optimal decision variables. The dual variables and penalty parameters are updated based on violating feasibility constraints. These steps are repeated until feasible and optimal trajectories are obtained.}
	\label{fig-system_overview}
\end{figure*}