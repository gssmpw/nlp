\textbf{**********Coding phase beginning **********}\\

TEXT: 1. ``I started chemotherapy on February 10, 2020\ldots After that
I will have 25 days of radiation. Reconstruction will begin six months
after that. So, 2020 has not been the year I hoped it would be. My
ordeal combined with the COVID-19 pandemic has been surreal. But through
it all, I have had great support from my family and friends.''- Darlene
Langley, breast cancer survivor. <url> We love your positive attitude! Show Darlene some love in comments\\

\textbf{Agent 1:} To analyze the provided text using the guidelines in the
CODEBOOK, I will carefully examine the narrative elements and the
narrator perspective.

\textbf{Narrative Event(s) related to breast cancer
(plot)}

\begin{enumerate}
\def\labelenumi{\arabic{enumi}.}
\tightlist
\item
  \textbf{Treatment}: The text mentions starting chemotherapy and
  planning for radiation and reconstruction. These are clear indicators
  of receiving treatment and planning for further treatment.

  \begin{itemize}
  \tightlist
  \item
    ``I started chemotherapy on February 10, 2020\ldots{}''
  \item
    ``After that I will have 25 days of radiation.''
  \item
    ``Reconstruction will begin six months after that.''
  \end{itemize}
\end{enumerate}

\textbf{Narrator perspective
(narrator\_perspective)}

The narrator is Darlene Langley, who is explicitly identified as a
breast cancer survivor. The perspective is clearly from the survivor's
point of view. - ``I started chemotherapy\ldots{}'' - ``My ordeal
combined with the COVID-19 pandemic has been surreal.'' - ``I have had
great support from my family and friends.''

Based on the analysis, the narrative events sequence (NES) includes
receiving treatment and planning for further treatment, and the narrator
perspective (NP) is that of a breast cancer survivor.

\textbf{JSON Output}

\begin{verbatim}
{
  "NES": "3",
  "NP": "1"
}
\end{verbatim}

\textbf{Agent 2:} In analyzing this Facebook post by a breast cancer
organization, we need to identify the narrative events and the
narrator's perspective based on the provided codebook.

\textbf{Narrative Event(s) Sequence
(NES)}

The post describes several key events in Darlene Langley's breast cancer
journey: 1. \textbf{Treatment}: The post mentions the start of
chemotherapy on February 10, 2020, and the upcoming 25 days of
radiation. This falls under ``Receiving treatment'' and ``Treatment
milestone or completion.'' 2. \textbf{Survivorship}: The post also
touches on the emotional and social aspects of her journey, including
the support from family and friends, which is part of the survivorship
experience.

Thus, the narrative events sequence includes: - Treatment (Receiving
treatment, Treatment milestone or completion) - Survivorship

\textbf{Narrator Perspective (NP)}

The narrator in this post is Darlene Langley, a breast cancer survivor,
as indicated by the direct quote and the attribution at the end of the
post.

\textbf{JSON Output}

Based on the analysis, the JSON output would be:

\begin{verbatim}
{
  "NES": "3,4",
  "NP": "1"
}
\end{verbatim}

This captures the narrative events related to treatment and
survivorship, and identifies the narrator as the breast cancer survivor
herself.

Judge: Different Agreement: False\\

TEXT: 2. ... (Text 2 - Text 20)\\

\textbf{**********Discussion phase beginning **********}\\

TEXT: 1. ``I started chemotherapy on February 10, 2020\ldots After that
I will have 25 days of radiation. Reconstruction will begin six months
after that. So, 2020 has not been the year I hoped it would be. My
ordeal combined with the COVID-19 pandemic has been surreal. But through
it all, I have had great support from my family and friends.''- Darlene
Langley, breast cancer survivor.
http://sgk.mn/2zSTQh5:=:https://www.komen.org/share-your-story/darlene-langley/ We love your positive attitude! Show Darlene some love in comments\\

<Round 1>\\

\textbf{Agent 1:} \#\#\# Analysis of the Provided Text

\textbf{Narrative Event(s) related to breast cancer
(plot)}

\begin{enumerate}
\def\labelenumi{\arabic{enumi}.}
\tightlist
\item
  \textbf{Treatment}: The text mentions starting chemotherapy, planning
  for radiation, and future reconstruction. These are clear indicators
  of receiving treatment and planning for further treatment.

  \begin{itemize}
  \tightlist
  \item
    ``I started chemotherapy on February 10, 2020\ldots{}''
  \item
    ``After that I will have 25 days of radiation.''
  \item
    ``Reconstruction will begin six months after that.''
  \end{itemize}
\item
  \textbf{Survivorship}: The text also touches on the emotional and
  social aspects of her journey, including the support from family and
  friends, which is part of the survivorship experience.

  \begin{itemize}
  \tightlist
  \item
    ``My ordeal combined with the COVID-19 pandemic has been surreal.''
  \item
    ``But through it all, I have had great support from my family and
    friends.''
  \end{itemize}
\end{enumerate}

\textbf{Narrator perspective
(narrator\_perspective)}

The narrator is Darlene Langley, who is explicitly identified as a
breast cancer survivor. The perspective is clearly from the survivor's
point of view. - ``I started chemotherapy\ldots{}'' - ``My ordeal
combined with the COVID-19 pandemic has been surreal.'' - ``I have had
great support from my family and friends.''

\textbf{Conclusion}

Based on the analysis, the narrative events sequence (NES) includes
receiving treatment and survivorship, and the narrator perspective (NP)
is that of a breast cancer survivor.

\textbf{JSON Output}

\begin{verbatim}
{
  "NES": "3,4",
  "NP": "1"
}
\end{verbatim}

This output captures the narrative events related to treatment and
survivorship, and identifies the narrator as the breast cancer survivor
herself.\\

\textbf{Agent 2:} \#\#\# Analysis of the Provided Text

\textbf{Narrative Event(s) Sequence (NES)}

The text describes several key events in Darlene Langley's breast cancer
journey: 1. \textbf{Treatment}: The text mentions the start of
chemotherapy on February 10, 2020, and the upcoming 25 days of
radiation. This falls under ``Receiving treatment'' and ``Treatment
milestone or completion.'' - ``I started chemotherapy on February 10,
2020\ldots{}'' - ``After that I will have 25 days of radiation.'' -
``Reconstruction will begin six months after that.''

\begin{enumerate}
\def\labelenumi{\arabic{enumi}.}
\setcounter{enumi}{1}
\tightlist
\item
  \textbf{Survivorship}: The text also touches on the emotional and
  social aspects of her journey, including the support from family and
  friends, which is part of the survivorship experience.

  \begin{itemize}
  \tightlist
  \item
    ``My ordeal combined with the COVID-19 pandemic has been surreal.''
  \item
    ``But through it all, I have had great support from my family and
    friends.''
  \end{itemize}
\end{enumerate}

Thus, the narrative events sequence includes: - Treatment (Receiving
treatment, Treatment milestone or completion) - Survivorship

\textbf{Narrator Perspective (NP)}

The narrator in this post is Darlene Langley, a breast cancer survivor,
as indicated by the direct quote and the attribution at the end of the
post. - ``I started chemotherapy\ldots{}'' - ``My ordeal combined with
the COVID-19 pandemic has been surreal.'' - ``I have had great support
from my family and friends.''

\textbf{Discussion of Other Social Scientists'
Responses}

One social scientist suggested that the narrative events sequence (NES)
should only include ``Treatment'' and not ``Survivorship.'' However, the
emotional and social aspects of Darlene's journey, including the support
from family and friends, are significant elements of survivorship.
Therefore, I maintain that both ``Treatment'' and ``Survivorship''
should be included in the NES.

\textbf{Conclusion}

Based on the analysis and the discussion, the JSON output would be:

\begin{verbatim}
{
  "NES": "3,4",
  "NP": "1"
}
\end{verbatim}

This captures the narrative events related to treatment and survivorship
and identifies the narrator as the breast cancer survivor herself.

Judge: Same Agreement: True\\

TEXT: 4. ... (other disagreement text)\\

\textbf{**********Update codebook phase beginning **********}\\

\textbf{Agent 1:} Based on the coding and discussion results, the original
CODEBOOK appears to adequately fit the current examples. Each rule in
the CODEBOOK has been applied at least once, and the rules are unique
with minimal overlap. Therefore, there is no need to revise the
CODEBOOK. Here is the original CODEBOOK for reference:

\begin{center}\rule{0.5\linewidth}{0.5pt}\end{center}

\textbf{CODEBOOK:}

\textbf{Codebook for Coders (July 1, 2021)}

\textbf{Narrative Elements in Facebook Posts by Breast Cancer
Organizations}

\textbf{Narrative Event(s) related to breast cancer (plot)} {[}Code the
occurrences of all events in the post, such as ``3'', ``2,4'', or
``2,4,5''{]} 1. \textbf{Prevention} 2. \textbf{Detection and/or
diagnosis} 3. \textbf{Treatment} - Receiving treatment (e.g., getting
the IV chemo, lying in the hospital bed) - Treatment effects (e.g., bald
head, flat chest, wearing a head wrap) - Treatment milestone or
completion (e.g., ringing the chemo bell, showing radiation therapy
completion certificate). 4. \textbf{Survivorship} - includes complete
remission/cancer free; recurrence; a second cancer; and death. 5.
\textbf{Fundraising, any prosocial or philanthropic activities}

\textbf{Narrator perspective (narrator\_perspective)} (evaluated based
on the original post; choose one) {[}Narrator is the person telling the
story. When coding, prioritize a perspective that is NOT the breast
cancer organization.{]} 1. \textbf{Breast cancer survivor} 2.
\textbf{Breast cancer survivor's family or friends} 3. \textbf{Mixed
(i.e., survivor + family or friends)} 4. \textbf{Journalists/news media}
5. \textbf{Breast cancer organization}

\begin{center}\rule{0.5\linewidth}{0.5pt}\end{center}

This CODEBOOK has been applied consistently across the examples, and it
covers all the necessary narrative elements and perspectives without
redundancy. Therefore, no changes are needed.\\

\textbf{Agent 2:} Based on the coding and discussion results, the original
CODEBOOK appears to adequately fit the current examples. Each rule in
the CODEBOOK has been applied at least once, and the rules are unique
with minimal overlap. Therefore, there is no need to revise the
CODEBOOK. Here is the original CODEBOOK:

\textbf{CODEBOOK:}

\textbf{Codebook for Coders (July 1, 2021)}

\textbf{Narrative Elements in Facebook Posts by Breast Cancer
Organizations}

\begin{itemize}
\tightlist
\item
  \textbf{Narrative Event(s) related to breast cancer (plot)}

  \begin{itemize}
  \tightlist
  \item
    {[}Code the occurrences of all events in the post, such as ``3'',
    ``2,4'', or ``2,4,5''{]}
  \end{itemize}

  \begin{enumerate}
  \def\labelenumi{\arabic{enumi}.}
  \tightlist
  \item
    Prevention
  \item
    Detection and/or diagnosis
  \item
    Treatment

    \begin{itemize}
    \tightlist
    \item
      Receiving treatment (e.g., getting the IV chemo, lying in the
      hospital bed)
    \item
      Treatment effects (e.g., bald head, flat chest, wearing a head
      wrap)
    \item
      Treatment milestone or completion (e.g., ringing the chemo bell,
      showing radiation therapy completion certificate).
    \end{itemize}
  \item
    Survivorship - includes complete remission/cancer free; recurrence;
    a second cancer; and death.
  \item
    Fundraising, any prosocial or philanthropic activities
  \end{enumerate}
\item
  \textbf{Narrator perspective (narrator\_perspective) (evaluated based
  on the original post; choose one)}

  \begin{itemize}
  \tightlist
  \item
    {[}Narrator is the person telling the story. When coding, prioritize
    a perspective that is NOT the breast cancer organization.{]}
  \end{itemize}

  \begin{enumerate}
  \def\labelenumi{\arabic{enumi}.}
  \tightlist
  \item
    Breast cancer survivor
  \item
    Breast cancer survivor's family or friends
  \item
    Mixed (i.e., survivor + family or friends)
  \item
    Journalists/news media
  \item
    Breast cancer organization
  \end{enumerate}
\end{itemize}

This version of the CODEBOOK covers all cases and patterns in the
examples provided, and each rule has been applied appropriately.
Therefore, no changes are necessary.

Mediator: \#\#\# Summary of Opinions

Both social scientists agree that the original CODEBOOK is adequate and
does not require any revisions. They both emphasize that:

\begin{enumerate}
\def\labelenumi{\arabic{enumi}.}
\tightlist
\item
  The original CODEBOOK has been applied consistently across the
  examples.
\item
  Each rule in the CODEBOOK has been applied at least once.
\item
  The rules are unique with minimal overlap.
\item
  The CODEBOOK covers all necessary narrative elements and perspectives
  without redundancy.
\end{enumerate}

\textbf{Updated CODEBOOK}

Based on the consensus that no changes are needed, here is the original
CODEBOOK for reference:

\begin{center}\rule{0.5\linewidth}{0.5pt}\end{center}

\textbf{CODEBOOK:}

\textbf{Codebook for Coders (July 1, 2021)}

\textbf{Narrative Elements in Facebook Posts by Breast Cancer
Organizations}

\textbf{Narrative Event(s) related to breast cancer (plot)} {[}Code the
occurrences of all events in the post, such as ``3'', ``2,4'', or
``2,4,5''{]} 1. \textbf{Prevention} 2. \textbf{Detection and/or
diagnosis} 3. \textbf{Treatment} - Receiving treatment (e.g., getting
the IV chemo, lying in the hospital bed) - Treatment effects (e.g., bald
head, flat chest, wearing a head wrap) - Treatment milestone or
completion (e.g., ringing the chemo bell, showing radiation therapy
completion certificate). 4. \textbf{Survivorship} - includes complete
remission/cancer free; recurrence; a second cancer; and death. 5.
\textbf{Fundraising, any prosocial or philanthropic activities}

\textbf{Narrator perspective (narrator\_perspective)} (evaluated based
on the original post; choose one) {[}Narrator is the person telling the
story. When coding, prioritize a perspective that is NOT the breast
cancer organization.{]} 1. \textbf{Breast cancer survivor} 2.
\textbf{Breast cancer survivor's family or friends} 3. \textbf{Mixed
(i.e., survivor + family or friends)} 4. \textbf{Journalists/news media}
5. \textbf{Breast cancer organization}

\begin{center}\rule{0.5\linewidth}{0.5pt}\end{center}

\textbf{Confirmation}

Dear Social Scientists,

Based on your feedback, the consensus is that the original CODEBOOK is
adequate and does not require any revisions. Here is the original
CODEBOOK for your reference.

Please confirm if you agree with this conclusion and the provided
CODEBOOK. If there are any further suggestions or concerns, kindly let
me know.

Thank you for your collaboration.

Best regards, {[}Your Name{]}\\

\textbf{Agent 1:} \#\#\# Confirmation

Dear Social Scientists,

Based on your feedback, the consensus is that the original CODEBOOK is
adequate and does not require any revisions. Here is the original
CODEBOOK for your reference:

\begin{center}\rule{0.5\linewidth}{0.5pt}\end{center}

\textbf{CODEBOOK:}

\textbf{Codebook for Coders (July 1, 2021)}

\textbf{Narrative Elements in Facebook Posts by Breast Cancer
Organizations}

\textbf{Narrative Event(s) related to breast cancer (plot)} {[}Code the
occurrences of all events in the post, such as ``3'', ``2,4'', or
``2,4,5''{]} 1. \textbf{Prevention} 2. \textbf{Detection and/or
diagnosis} 3. \textbf{Treatment} - Receiving treatment (e.g., getting
the IV chemo, lying in the hospital bed) - Treatment effects (e.g., bald
head, flat chest, wearing a head wrap) - Treatment milestone or
completion (e.g., ringing the chemo bell, showing radiation therapy
completion certificate). 4. \textbf{Survivorship} - includes complete
remission/cancer free; recurrence; a second cancer; and death. 5.
\textbf{Fundraising, any prosocial or philanthropic activities}

\textbf{Narrator perspective (narrator\_perspective)} (evaluated based
on the original post; choose one) {[}Narrator is the person telling the
story. When coding, prioritize a perspective that is NOT the breast
cancer organization.{]} 1. \textbf{Breast cancer survivor} 2.
\textbf{Breast cancer survivor's family or friends} 3. \textbf{Mixed
(i.e., survivor + family or friends)} 4. \textbf{Journalists/news media}
5. \textbf{Breast cancer organization}

\begin{center}\rule{0.5\linewidth}{0.5pt}\end{center}

Please confirm if you agree with this conclusion and the provided
CODEBOOK. If there are any further suggestions or concerns, kindly let
me know.

Thank you for your collaboration.

Best regards, {[}Your Name{]}\\

\textbf{Agent 2:} \#\#\# Confirmation

Dear Social Scientists,

Thank you for your thorough analysis and feedback. Based on our
discussions and the consistent application of the original CODEBOOK
across various examples, we have concluded that the original CODEBOOK is
adequate and does not require any revisions. Here is the original
CODEBOOK for your reference:

\begin{center}\rule{0.5\linewidth}{0.5pt}\end{center}

\textbf{CODEBOOK:}

\textbf{Codebook for Coders (July 1, 2021)}

\textbf{Narrative Elements in Facebook Posts by Breast Cancer
Organizations}

\textbf{Narrative Event(s) related to breast cancer (plot)} {[}Code the
occurrences of all events in the post, such as ``3'', ``2,4'', or
``2,4,5''{]} 1. \textbf{Prevention} 2. \textbf{Detection and/or
diagnosis} 3. \textbf{Treatment} - Receiving treatment (e.g., getting
the IV chemo, lying in the hospital bed) - Treatment effects (e.g., bald
head, flat chest, wearing a head wrap) - Treatment milestone or
completion (e.g., ringing the chemo bell, showing radiation therapy
completion certificate). 4. \textbf{Survivorship} - includes complete
remission/cancer free; recurrence; a second cancer; and death. 5.
\textbf{Fundraising, any prosocial or philanthropic activities}

\textbf{Narrator perspective (narrator\_perspective)} (evaluated based
on the original post; choose one) {[}Narrator is the person telling the
story. When coding, prioritize a perspective that is NOT the breast
cancer organization.{]} 1. \textbf{Breast cancer survivor} 2.
\textbf{Breast cancer survivor's family or friends} 3. \textbf{Mixed
(i.e., survivor + family or friends)} 4. \textbf{Journalists/news media}
5. \textbf{Breast cancer organization}

\begin{center}\rule{0.5\linewidth}{0.5pt}\end{center}

Please confirm if you agree with this conclusion and the provided
CODEBOOK. If there are any further suggestions or concerns, kindly let
me know.

Thank you for your collaboration.

Best regards, {[}Your Name{]}

Judge: Same Agreement: True