\textbf{Agent 1:} Given the task of sentiment annotation and the criteria for a
good CODEBOOK, I will review the original CODEBOOK and determine if any
modifications are necessary to ensure it covers all cases and patterns,
is unique, and is clear and precise.

\textbf{Original CODEBOOK:} - Positive: Positive sentiment of users
toward the issue/company. - Neutral: Neutral sentiment of users toward
the issue/company. - Negative: Negative sentiment of users toward the
issue/company.

\textbf{Criteria for a good CODEBOOK:} 1. The CODEBOOK should cover all
cases and patterns in the examples. 2. Each rule in the CODEBOOK should
be applied at least once. 3. Each rule in the CODEBOOK should be unique,
with minimal or no overlap with other rules. 4. This version simplifies
the language while maintaining clarity and precision.

\textbf{Review and Analysis:} 1. \textbf{Coverage of Cases and
Patterns:} - The original CODEBOOK is broad and general, which allows it
to cover a wide range of sentiments expressed in texts. However, it may
benefit from additional examples or clarifications to ensure consistent
application.

\begin{enumerate}
\def\labelenumi{\arabic{enumi}.}
\setcounter{enumi}{1}
\tightlist
\item
  \textbf{Application of Rules:}

  \begin{itemize}
  \tightlist
  \item
    The rules are straightforward and can be applied to any text
    expressing sentiment. However, without examples, it may be
    challenging to ensure consistent interpretation across different
    coders.
  \end{itemize}
\item
  \textbf{Uniqueness and Minimal Overlap:}

  \begin{itemize}
  \tightlist
  \item
    The rules are distinct and do not overlap, as they categorize
    sentiment into three clear categories: positive, neutral, and
    negative.
  \end{itemize}
\item
  \textbf{Clarity and Precision:}

  \begin{itemize}
  \tightlist
  \item
    The language is clear, but adding examples would enhance precision
    and help coders apply the rules more consistently.
  \end{itemize}
\end{enumerate}

\textbf{Proposed Modifications:} To improve the CODEBOOK, I will add
examples for each sentiment category to provide clarity and ensure
consistent application.

\textbf{Updated CODEBOOK:} - \textbf{Positive:} Positive sentiment of
users toward the issue/company. - Example: ``I love this company's new
policy! It's so beneficial for everyone.'' - Example: ``Great job on the
recent project! Keep up the good work.''

\begin{itemize}
\tightlist
\item
  \textbf{Neutral:} Neutral sentiment of users toward the issue/company.

  \begin{itemize}
  \tightlist
  \item
    Example: ``The company announced a new policy today.''
  \item
    Example: ``I heard about the recent changes, but I don't have an
    opinion yet.''
  \end{itemize}
\item
  \textbf{Negative:} Negative sentiment of users toward the
  issue/company.

  \begin{itemize}
  \tightlist
  \item
    Example: ``I'm really disappointed with the company's new policy.''
  \item
    Example: ``This is the worst service I've ever experienced.''
  \end{itemize}
\end{itemize}

By adding these examples, the CODEBOOK now provides clear guidance on
how to categorize different sentiments, ensuring that each rule is
applied consistently and accurately.

\textbf{Final Version:} - \textbf{Positive:} Positive sentiment of users
toward the issue/company. - Example: ``I love this company's new policy!
It's so beneficial for everyone.'' - Example: ``Great job on the recent
project! Keep up the good work.''

\begin{itemize}
\tightlist
\item
  \textbf{Neutral:} Neutral sentiment of users toward the issue/company.

  \begin{itemize}
  \tightlist
  \item
    Example: ``The company announced a new policy today.''
  \item
    Example: ``I heard about the recent changes, but I don't have an
    opinion yet.''
  \end{itemize}
\item
  \textbf{Negative:} Negative sentiment of users toward the
  issue/company.

  \begin{itemize}
  \tightlist
  \item
    Example: ``I'm really disappointed with the company's new policy.''
  \item
    Example: ``This is the worst service I've ever experienced.''
  \end{itemize}
\end{itemize}

This updated CODEBOOK maintains the original structure while providing
additional clarity and examples to ensure precise and consistent
application.\\

\textbf{Agent 2:} Given the task at hand and the criteria for a good CODEBOOK, I
will review the original CODEBOOK and determine if any modifications are
necessary to better fit the examples and ensure clarity and precision.

\textbf{Original CODEBOOK:} 1. Positive: Positive sentiment of users
toward the issue/company. 2. Neutral: Neutral sentiment of users toward
the issue/company. 3. Negative: Negative sentiment of users toward the
issue/company.

\textbf{Criteria for a good CODEBOOK:} 1. The CODEBOOK should cover all
cases and patterns in the examples. 2. Each rule in the CODEBOOK should
be applied at least once. 3. Each rule in the CODEBOOK should be unique,
with minimal or no overlap with other rules. 4. This version simplifies
the language while maintaining clarity and precision.

\textbf{Review and Analysis:} - The original CODEBOOK is straightforward
and covers the basic sentiment categories: positive, neutral, and
negative. - Each rule is distinct and does not overlap with others. -
The language is simple and clear, making it easy to apply to various
texts.

\textbf{Decision:} The original CODEBOOK appears to be adequate for the
task. It is simple, clear, and covers the necessary sentiment categories
without overlap. Therefore, I will keep the CODEBOOK unchanged.

\textbf{Original CODEBOOK:} 1. Positive: Positive sentiment of users
toward the issue/company. 2. Neutral: Neutral sentiment of users toward
the issue/company. 3. Negative: Negative sentiment of users toward the
issue/company.

This version of the CODEBOOK meets the criteria for a good CODEBOOK and
does not require any modifications.\\

\textbf{Mediator:} \#\#\# Summary of Opinions\\

\textbf{Agent 1:} - \textbf{Review and Analysis:} - The original
CODEBOOK is broad and general, covering a wide range of sentiments. -
Rules are straightforward but may benefit from examples for consistent
interpretation. - The rules are distinct and do not overlap. - Language
is clear, but examples would enhance precision.

\begin{itemize}
\tightlist
\item
  \textbf{Proposed Modifications:}

  \begin{itemize}
  \tightlist
  \item
    Add examples for each sentiment category to provide clarity and
    ensure consistent application.
  \end{itemize}
\item
  \textbf{Updated CODEBOOK:}

  \begin{itemize}
  \tightlist
  \item
    \textbf{Positive:} Positive sentiment of users toward the
    issue/company.

    \begin{itemize}
    \tightlist
    \item
      Example: ``I love this company's new policy! It's so beneficial
      for everyone.''
    \item
      Example: ``Great job on the recent project! Keep up the good
      work.''
    \end{itemize}
  \item
    \textbf{Neutral:} Neutral sentiment of users toward the
    issue/company.

    \begin{itemize}
    \tightlist
    \item
      Example: ``The company announced a new policy today.''
    \item
      Example: ``I heard about the recent changes, but I don't have an
      opinion yet.''
    \end{itemize}
  \item
    \textbf{Negative:} Negative sentiment of users toward the
    issue/company.

    \begin{itemize}
    \tightlist
    \item
      Example: ``I'm really disappointed with the company's new
      policy.''
    \item
      Example: ``This is the worst service I've ever experienced.''
    \end{itemize}
  \end{itemize}
\end{itemize}

\textbf{Agent 2:} - \textbf{Review and Analysis:} - The original
CODEBOOK is straightforward and covers the basic sentiment categories:
positive, neutral, and negative. - Each rule is distinct and does not
overlap with others. - The language is simple and clear, making it easy
to apply to various texts.

\begin{itemize}
\tightlist
\item
  \textbf{Decision:}

  \begin{itemize}
  \tightlist
  \item
    The original CODEBOOK is adequate for the task and does not require
    any modifications.
  \end{itemize}
\end{itemize}

\textbf{Proposed Updated CODEBOOK}

Based on the feedback from both agents, the following updated CODEBOOK
incorporates the clarity and simplicity emphasized by Agent 2 while
adding the examples suggested by Agent 1 to ensure consistent
application:

\textbf{Updated CODEBOOK:} - \textbf{Positive:} Positive sentiment of
users toward the issue/company. - Example: ``I love this company's new
policy! It's so beneficial for everyone.'' - Example: ``Great job on the
recent project! Keep up the good work.''

\begin{itemize}
\tightlist
\item
  \textbf{Neutral:} Neutral sentiment of users toward the issue/company.

  \begin{itemize}
  \tightlist
  \item
    Example: ``The company announced a new policy today.''
  \item
    Example: ``I heard about the recent changes, but I don't have an
    opinion yet.''
  \end{itemize}
\item
  \textbf{Negative:} Negative sentiment of users toward the
  issue/company.

  \begin{itemize}
  \tightlist
  \item
    Example: ``I'm really disappointed with the company's new policy.''
  \item
    Example: ``This is the worst service I've ever experienced.''
  \end{itemize}
\end{itemize}

\textbf{Request for Agreement}

Dear Social Scientists,

Please review the proposed updated CODEBOOK below and let me know if you
agree with the modifications. If there are any further suggestions or
concerns, please feel free to share them.

\textbf{Updated CODEBOOK:} - \textbf{Positive:} Positive sentiment of
users toward the issue/company. - Example: ``I love this company's new
policy! It's so beneficial for everyone.'' - Example: ``Great job on the
recent project! Keep up the good work.''

\begin{itemize}
\tightlist
\item
  \textbf{Neutral:} Neutral sentiment of users toward the issue/company.

  \begin{itemize}
  \tightlist
  \item
    Example: ``The company announced a new policy today.''
  \item
    Example: ``I heard about the recent changes, but I don't have an
    opinion yet.''
  \end{itemize}
\item
  \textbf{Negative:} Negative sentiment of users toward the
  issue/company.

  \begin{itemize}
  \tightlist
  \item
    Example: ``I'm really disappointed with the company's new policy.''
  \item
    Example: ``This is the worst service I've ever experienced.''
  \end{itemize}
\end{itemize}

Looking forward to your feedback.

Best regards, {[}Your Name{]}\\

\textbf{Agent 1:} Dear Social Scientists,

Thank you for your thoughtful feedback and analysis. Based on the
summary of opinions and the proposed modifications, I have reviewed the
updated CODEBOOK and agree that it effectively balances clarity,
simplicity, and the need for examples to ensure consistent application.

\textbf{Final Updated CODEBOOK}

\textbf{Positive:} Positive sentiment of users toward the issue/company.
- Example: ``I love this company's new policy! It's so beneficial for
everyone.'' - Example: ``Great job on the recent project! Keep up the
good work.''

\textbf{Neutral:} Neutral sentiment of users toward the issue/company. -
Example: ``The company announced a new policy today.'' - Example: ``I
heard about the recent changes, but I don't have an opinion yet.''

\textbf{Negative:} Negative sentiment of users toward the issue/company.
- Example: ``I'm really disappointed with the company's new policy.'' -
Example: ``This is the worst service I've ever experienced.''

This updated CODEBOOK maintains the original structure while providing
additional clarity and examples to ensure precise and consistent
application. It covers all cases and patterns, each rule is applied at
least once, and the rules are unique with minimal overlap.

I believe this version will serve us well in our sentiment annotation
task. If there are no further suggestions or concerns, we can proceed
with this updated CODEBOOK.

Best regards, Dr.~Emily Carter\\

\textbf{Agent 2:} The feedback from both agents provides valuable insights into
refining the CODEBOOK. Agent 1's suggestion to include examples for each
sentiment category is particularly useful for ensuring consistent
application, while Agent 2's affirmation of the original CODEBOOK's
clarity and simplicity is also important. Combining these perspectives,
the updated CODEBOOK will maintain the original categories but include
examples to enhance clarity and precision.

\textbf{Updated CODEBOOK:} - \textbf{Positive:} Positive sentiment of
users toward the issue/company. - Example: ``I love this company's new
policy! It's so beneficial for everyone.'' - Example: ``Great job on the
recent project! Keep up the good work.''

\begin{itemize}
\tightlist
\item
  \textbf{Neutral:} Neutral sentiment of users toward the issue/company.

  \begin{itemize}
  \tightlist
  \item
    Example: ``The company announced a new policy today.''
  \item
    Example: ``I heard about the recent changes, but I don't have an
    opinion yet.''
  \end{itemize}
\item
  \textbf{Negative:} Negative sentiment of users toward the
  issue/company.

  \begin{itemize}
  \tightlist
  \item
    Example: ``I'm really disappointed with the company's new policy.''
  \item
    Example: ``This is the worst service I've ever experienced.''
  \end{itemize}
\end{itemize}

This updated CODEBOOK should cover all cases and patterns in the
examples, ensure each rule is applied at least once, and maintain
unique, non-overlapping rules. The added examples will help in
consistently interpreting and applying the sentiment categories.

I agree with the proposed updated CODEBOOK and believe it meets the
criteria for a good CODEBOOK. If there are no further suggestions or
concerns, we can proceed with this version.

Judge: Same Agreement: True