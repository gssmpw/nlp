\documentclass{l4dc2025}
% The following packages will be automatically loaded:
% amsmath, amssymb, natbib, graphicx, url, algorithm2e

\usepackage[english]{babel}
%\usepackage{graphicx}
\usepackage{float}
\LinesNumbered
\usepackage{bm}
\usepackage{braket}

\usepackage{hhline} 
\usepackage{multirow}

\usepackage{booktabs}

\usepackage{filecontents,pgfplots}
\usepackage{tikz}

%\usepackage{tikz-3dplot}
\usetikzlibrary{positioning}
\usetikzlibrary{hobby}
\usetikzlibrary{shapes.misc}

\usepackage{pgfplots}
\pgfplotsset{compat=1.16}

\usepgfplotslibrary{patchplots}
\DeclareUnicodeCharacter{2212}{−}
\usepgfplotslibrary{groupplots,dateplot}
\usetikzlibrary{patterns,shapes.arrows}
\pgfplotsset{compat=newest}


\tikzset{cross/.style={cross out, draw=black, minimum size=2*(#1-\pgflinewidth), inner sep=0pt, outer sep=0pt}, cross/.default={2pt}}

\usepackage{todonotes}
\usepackage{wrapfig}
\usepackage{mathtools}

\DeclarePairedDelimiter\floor{\lfloor}{\rfloor}

\newcommand{\ours}{$\text{Q}$LASS}
\def\geom{\textbf{g}}
\def\NrVertsExtra{N_{\V+}}
\def\NrNeighConnect{N_{\text{nn}}}

\def\Vrw{\tilde{\V}}
\def\Erw{\tilde{\E}}
\def\GraphRw{\tilde{\Graph}}


\def\geomClass{\mathcal{G}}
\def\funcNSD{\tilde{\funcSD}}

\graphicspath{{images/}}
\makeatletter
\def\input@path{{images/}}
\makeatother

\title[Probabilistic Bubble Roadmap]{Probabilistic Bubble Roadmap}
\usepackage{times}
\coltauthor{%
  \Name{Bernhard Wullt} \Email{bernhard.wullt@se.abb.com}\\
  \Name{Mikael Norrlöf} \Email{mikael.norrlof@se.abb.com}\\
  \addr ABB Robotics, Sweden\\
  \Name{Per Mattsson} \Email{per.mattsson@it.uu.se}\\
  \Name{Thomas B. Schön} \Email{thomas.schon@it.uu.se}\\
  \addr Department of Information Technology, Uppsala University
  \vspace{-30pt}
}
\definecolor{darkmagenta12630156}{RGB}{126,30,156}
\definecolor{dodgerblue}{RGB}{30,144,255}
\definecolor{green}{RGB}{0,128,0}
\definecolor{midnightblue53662}{RGB}{53,6,62}
\definecolor{yellow25525520}{RGB}{255,255,20}
\usepackage{subfiles}
\usepackage{wrapfig,booktabs}


\begin{document}
\maketitle
\noindent
\begin{abstract}%
Finding a collision-free path is a fundamental problem in robotics, where the sampling based planners have a long line of success. However, this approach is computationally expensive, due to the frequent use of collision-detection. Furthermore, the produced paths are usually jagged and require further post-processing before they can be tracked. Due to their high computational cost, these planners are usually restricted to static settings, since they are not able to cope with rapid changes in the environment. In our work, we remove this restriction by introducing a learned signed distance function expressed in the configuration space of the robot. The signed distance allows us to form collision-free spherical regions in the configuration space, which we use to suggest a new multi-query path planner that also works in dynamic settings. We propose the probabilistic bubble roadmap planner, which enhances the probabilistic roadmap planner (PRM) by using spheres as vertices and compute the edges by checking for neighboring spheres which intersect. We benchmark our approach in a static setting where we show that we can produce paths that are shorter than the paths produced by the PRM, while having a smaller sized roadmap and finding the paths faster. Finally, we show that we can rapidly rewire the graph in the case of new obstacles introduced at run time and therefore produce paths in the case of moving obstacles.
\end{abstract}
\begin{keywords}%
  Motion planning, Signed distance function, Manipulators, Robotics %
\end{keywords}

\subfile{sections/introduction.tex}
\subfile{sections/scdf.tex}
\subfile{sections/pbrm.tex}
\subfile{sections/experiments.tex}

\section{Conclusions}
We have presented a novel approach to learn a signed distance function expressed in the configuration space for robots with 3 DOF. With the distance function we presented a new enhanced PRM, where each vertex covers a collision-free region. We benchmarked our planner with the PRM in a static setting, where our planner was the fastest and produced shorter paths, while having a smaller sized graph. We further evaluated the rewiring capabilities of our planner by introducing new test obstacles at runtime. We observed low computation times to produce paths, therefore allowing the planner to be used in dynamic settings.
\newpage
\acks{This research was supported by the \emph{Wallenberg AI, Autonomous Systems and Software Program (WASP)} funded by Knut and Alice Wallenberg Foundation.}
\bibliography{main}
\end{document}
