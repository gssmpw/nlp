\section{RELATED WORK}
% See \cite{lamsey2023stretch} for more! Also, some other people did things.

% \subsection{Exercise as Treatment for PD}

% Many types of exercise have been evaluated as treatments for PD. A review by Mak et. al. discusses several types of exercise used as PD treatments, including gait training, walking exercise, balance exercise, tai-chi, dance, and "exergaming"  \cite{mak2019exercise}. These target specific PD symptoms including hypometria, freezing of gait, coordination, balance, and muscle strength. Diverse exercise interventions spanning dance, aquatic, and cueing training have been shown to improve freezing of gait in people with PD \cite{gilat2021systematic}. Other symptoms require more specific interventions; gait training has been established as an effective method to reduce fall risk in PWP \cite{protas2005gait}, and PWP's performance of activities of daily living (ADLs) showed the greatest improvement when training was specifically tailored to ADLs \cite{foster2021occupational}.

% Whole-body, coordinated exercises, such as tai-chi and dance, have been shown to be effective treatments for PD. A meta-analysis of dance-based interventions asserted that dance practice yields greater improvements in motor symptoms and functional mobility in PWP compared to other physical activities (e.g., physiotherapy or self-directed exercises) \cite{dos2018effects}. Further, evaluations of partnered tango \cite{hackney2007effects} and tai-chi \cite{hackney2008tai} as treatments for PD yielded improvements in participants' balance and freezing of gait.

% \subsection{Robot-assisted Physical Therapy}

% Because PWP often struggle with intrinsic motivation \cite{pickering2013self}, it is essential to provide moderate to high doses of supervised exercise to support adherence and maximize therapeutic benefits \cite{clarke2016physiotherapy}. Thus, supplementing human-supervised exercise with technology-based interventions may enhance motivation and adherence.

% Robots have been used for assistance with physical therapy in diverse outlets. A 2024 review of robotics in physical rehabilitation \cite{banyai2024robotics} highlights the promise of robotic technologies such as exoskeletons, assistive training devices, and brain-computer interfaces for rehabilitation. Yet, this review cites consistent areas for improvement related to robots' ease of use and high system cost. Physically interactive robotic therapy systems have been tested in interventions for specific impairments, such as post-stroke rehabilitation of gross motor skills \cite{johnson2003design, krebs2004rehabilitation} and fine motor skills \cite{vakili2023impact, urrutia2023spasticity}, walking rehabilitation \cite{wuversatile, regmi2020design}, and motor-cognitive rehabilitation \cite{aprile2020robotic}. These systems combine physical and visual feedback to guide therapeutic exercises, which has proven to be beneficial in alleviating users' symptoms. However, they frequently incorporate purpose-built hardware that is too large, expensive, or complex for use outside of a clinical environment. Wearable rehabilitation robots, such as neck rehabilitation robots \cite{doss2023comprehensive} and hand exoskeletons \cite{tran2021hand}, are more portable, yet remain expensive and purpose-built for a single task.

% Socially Assistive Robots (SARs) have also been studied as an often lower cost and more accessible form of robotic therapy, including seated exercise \cite{fasola2012using} and tabletop rehabilitation games \cite{feingold2021robot}. Social-physical human-robot interaction (HRI) aims to combine the benefits of physically and socially interactive systems through exercises such as hand-clapping exercise games \cite{fitter2020exercising} and emotional support via hugging \cite{onishi2024moffuly}. While socially and physically interactive systems are rated as highly engaging, they may present safety concerns due to the robots' large sizes and heavy masses.

% Further, Augmented Reality (AR) and Virtual Reality (VR) have been used to augment therapeutic exercise for PD. A study involving AR for showing motivational exercise videos using a head-mounted display suggested that portable AR technology may improve adherence to exercise in the home\cite{tunur2020augmented}. However, another study which used AR to present movement cues to prevent freezing of gait found that AR has limited utility in treating freezing of gait, and at times can worsen these symptoms \cite{janssen2020effects}.

% \subsection{Technology Acceptance Models in Rehabilitation}

% % maybe some stuff from M. Johnson?

% Technology Acceptance Models (TAMs), such as those proposed by Venkatesh et. al. \cite{venkatesh2000theoretical, venkatesh2008technology}, serve as a basis for evaluating users' attitudes, including perceived usefulness and ease of use towards a technology. Similarly, the NASA Task Load Index (TLX) \cite{hart1988development, hart2006nasa} queries attributes like mental and physical strain while completing a task, which can be applied in the context of interacting with technology and performing rehabilitative tasks. Specific technology acceptance models have been applied to assistive devices, such as the Perceived Impact of Assistive Devices Scale (PIADS) \cite{jutai2002psychosocial}, which measures a technology's impact on a user's competence, adaptability, and self-esteem.

% Characterizing the perspective of clinicians who may administer robotic treatments is a critical component of user-centered design. Klaic et. al. applied an extended TAM alongside focus groups in a mixed-methods study with 34 rehabilitation clinicians, which highlighted healthcare providers' positive perceived usefulness and ease of use of an upper arm post-stroke rehabilitation robot \cite{klaic2024application}. Areas for improvement identified by clinicians in this study included increased access to training to use the robotic systems as well as improving embedded support for robotic systems. Through a survey administered to 379 clinicians, Sobrepara et. al. identified key themes and features for rehabilitation robots, including rich telepresence, ease of use, reliability, and cost \cite{sobrepera2022therapists}. 

% When designing a therapeutic robot system, it is also important to consider the perspective of potential patient users. Chen et. al. evaluated specific aspects of older adults' acceptance of a robot dance partner for rehabilitation, such as perceptions of successful task completion by the robot \cite{chen2015evaluation}. In the context of human-robot interaction for rehabilitative dance, Chen also presented a robot-specific TAM questionnaire \cite{chen2017older} which queried users' perceived usefulness, perceived ease of use, perceived enjoyment, positive attitudes, and intent to use the robot. Similarly, a study of 230 French adults using a telepresence robot to supervise at-home exercise used a TAM questionnaire to assess participant's perceived usefulness, enjoyment, ease of use, and intent to use \cite{mascret2023acceptance}, reporting positive outcomes across these measures.

% \subsection{Exercise Specialists in Technology-driven Interventions}

% ???