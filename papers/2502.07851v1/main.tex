\documentclass{DAC}
\usepackage{graphicx} % Required for inserting images
% \usepackage{ulem}
\usepackage{tikz}
\title{Fast and Safe Scheduling of Robots}
\author{}
\bibliography{references}
%\bibliographystyle{numeric}
\begin{document}

\maketitle

\section{Introduction}

Recent years have seen an increase in the use of moving autonomous robots within a range of environments, from manufacturing \cite{Liu2023} to Unmanned Aerial Vehicles \cite{qamar2023trmaxalloc}. We are particularly interested in the scheduling of robots within chemistry labs. This is motivated by a significant and expanding body of work concerning robotic chemists. Initial work on these systems focused on building robots performing reactions within fixed environments \cite{granda2018controlling,king2011rise,langner2019ternary,li2015synthesis,macleod2020selfdriving}, however recently Burger et al. \cite{burger2020mobile} have presented a robot capable of moving within a laboratory and completing tasks throughout the space. The works of Burger et al. \cite{burger2020mobile} and Liu et al. \cite{Liu2023} provide the main motivation for this work, namely the problem of moving robots within a laboratory environment (as presented by Burger et al. \cite{burger2020mobile}) while avoiding collisions (as investigated in the manufacturing context by Liu et al. \cite{Liu2023}).\looseness=-1

Throughout all such applications there are two clear objectives, (1) the robots need to complete their tasks quickly and (2) the robots must complete their tasks safely. Neither of these problems are trivial, indeed, the seemingly simple challenge of finding the best path for a robot to take in order to complete all its task is an example of the well known travelling salesman problem. In the other direction, scheduling the robots in such a way that they avoid collision is known to be NP-hard \cite{ourPreviousPaper}, even when the space is simplified to simple discrete structure such as a tree.\looseness=-1

In this paper, we present an analysis of a fast heuristic algorithm allowing us to find a fast, collision free, schedule for a set of robots on a line graph. In general, we show a direct comparison for the algorithm for finding an optimal schedule on these graphs when the tasks that each robot must complete all have the same length with the optimal solution. Additionally, we provide an integer program that solves this problem optimally, at the expense of far more computational resources being required. By comparing the solutions of these two algorithms, including the time required by the schedule itself, and the run time of each algorithm, we show that our heuristic is optimal or near optimal in nearly all cases, with a far faster run time than the integer program model.\looseness

% In addition to physical science motivation, our model and algorithmic results are strongly based on graph theory, in particular, graph exploration. Informally, we model our problem as a graph problem, where robots are represented as agents in the graph, with the goal of finding a set of walks for each robot, allowing every task to be completed without any collisions. Our model of movement for robots within the graph matches the exploration model given by Czyzowicz et al. \cite{CZYZOWICZ201770}, where agents (robots) start at fixed points within the graph, then can move provided that no pair of agents occupy the same vertex in the same timesteps. The primary difference between our model and that of \cite{CZYZOWICZ201770} is that in our setting, the agents are given a schedule from some central system rather than each having to determine the best route separately.\looseness=-1

% More general exploration problems have been considered in a variety of settings. Of particular interest to us are the works regarding the efficient exploration of temporal graphs. As in our setting, exploration is, in most cases, centrally controlled, with the primary goal of minimising the number of timesteps required to complete the exploration, corresponding to the length of the longest walk taken by any agent in the graph. Further, having the edge set of the graph change over time is similar to, and indeed can be closely mimicked by, the collision-avoiding condition in our problem, in the sense that the available moves for a given agent change throughout the lifetime of the graph.

{\color{red} TODO: add examples of experimental graph algorithms.}
There is a large number of results across many settings and variations of the temporal graph exploration problem, including when the number of vertices an agent can visit in one timestep is unbounded \cite{arrighi2023kernelizing,erlebach2022parameterized}, bounded \cite{erlebach_et_al:LIPIcs.ICALP.2019.141,erlebach2021temporal,michail2016traveling}, and for specific graph classes \cite{adamson2022faster,akrida2021temporal,bodlaender2019exploring,bumpus2023edge,deligkas2022optimizing,erlebach2022exploration,erlebach2018faster,taghian2020exploring}.
Particularly relevant to us is the work of Michail and Spirakis \cite{michail2016traveling}, who showed that the problem of determining the fastest exploration of a temporal graph is NP-hard, and, furthermore, no constant factor approximation algorithm exists of the shortest exploration (in terms of the length of the path found by the algorithm, compared to the shortest path exploring the graph) unless $P = NP$. As noted, the change in the structure of temporal graphs is close to the challenges implemented in our graph by agents blocking potential moves from each other.
In terms of positive results, the work of Erlebach et al. \cite{erlebach2021temporal} provided a substantial set of results that have formed the basis for much of the subsequent work on algorithmic results for temporal graph exploration. Of particular interest to us are the results that show that, for temporal graphs that are connected in every timestep, an agent can visit any subset of $m$ vertices in at most $O(n m)$ time, and provide constructions for faster explorations of graphs with $b$ agents and an $(r, b)$-division ($O(n^2 b / r + n r b^2)$ time), and $2 \times n$ grids with $4 \log n$ agents ($O(n \log n)$ time).

\subsection{Previous Work}%(partition on Line/Cycle/Trees)}
The work presented in this paper relies on previous work done by Adamson et al. in \cite{adamson2024collisionfreerobotscheduling}, particularly the partition algorithm which we shall now present here. 

The partition algorithm takes an instance of \textsc{Robot Scheduling} on a path and splits the graph into $k$ contiguous subpaths in order to assign one robot to each piece.
In order to know the schedules from partitioning the path it is required to know how long the shortest schedule for one robot to complete all tasks on a path would take. 
This is done by taking the minimum time of the robot going to the leftmost task first and then going right or going to the rightmost task first and then going left. 
Let $C_1(P,T,sv)$ be shortest schedule for one robot, starting at vertex $sv$, to complete all tasks in $T = t_1,...,t_m$ located on vertices $i_1,...,i_m$ on path $P$.  Here we assume $i_j < i_{j+1}$ for all  $j \in [m]$.
\[C_1(P,T,sv) =\min(\vert sv - i_1 \vert, \vert sv - i_m \vert) + i_m - i_1 + \sum\limits_{t \in T} t.\]

This can now be used in the dynamic programming algorithm shown below, here we have robots $R_1,...,R_k$ starting on initial vertices $sv_1,...,sv_k$ where $sv_i  < sv_{i+1}$ for all $i\in[k]$. 
\paragraph{$k$-Partition Algorithm}
Let $S[c,\ell ]$ be the length of the schedule involving the leftmost $c$ robots (i.e. robots $R_1,...,R_c$) and the leftmost $\ell$ tasks.

For all $\ell \in [m]$ we initialise $S[1,\ell] := C_1 (P,\{t_1,...,t_\ell \},sv_1 )$.

Then, for each $c \in [k] $ the partition algorithm iterates through each task-containing vertex and considers splitting the path such that the first $c-1$ robots complete all tasks to the left of that point and robot $R_c$ completes the rest. The algorithm chooses the splitting point with minimum timespan to store in the table. Formally:
$$S[c, \ell] = \min_{r \in [1, \ell]} \max(\vert C_1(P, (t_{r + 1}, t_{r + 2}, \dots, t_\ell), sv_c) \vert , S[c-1, r]).$$


This algorithm was shown in \cite{adamson2024collisionfreerobotscheduling} to be optimal for equal length tasks but a $k$-approximation otherwise. 



 
%We assume that the tasks $t_1,...,t_m$ are ordered such that $t_i$ is left of $t_j$  for all $i < j $, similarly the robot $R_1,...,R_k$ have initial positions $sv_1,...,sv_k$ such that $R_i$ is left of $R_j$ ( or $sv_i < sv_j$) for all $i < j$. 
%%The $k$-partition algorithm uses a dynamic programming approach .
%\paragraph{$k$-Partition Algorithm}
%Let $S[c,\ell ]$ be the length of the schedule involving the leftmost $c$ robots and the leftmost $\ell$ tasks.
%T, the algorithm then chooses the splitting point with the minimum timespan. 
%$$S[c, \ell] = \min_{r \in [1, \ell]} \max(\vert C_1(P, (t_{r + 1}, t_{r + 2}, \dots, t_\ell), sv_c) \vert , S[c-1, r]).$$

\section{Preliminaries}\label{sec:prelim}
Before we proceed we must first introduce some necessary definitions, firstly,  a graph $G=(V,E)$ is a pair consisting of a set of vertices $V$ and a set of edges $E \subseteq V \times V$. A \emph{walk} in a graph $G$ of length $\ell$ is a sequence of $\ell$ edges of the form $(v_1, v_2), (v_2, v_3), \dots, (v_{\ell - 1}, v_{\ell})$. Any walk $w$ can visit the same vertex multiple times and may use the same edge multiple times. 
A walk without any such repetitions is called a \emph{path}. 
Given a walk $w = (v_1, v_2), (v_2, v_3), \dots, (v_{\ell - 1}, v_{\ell})$, we denote by $\vert w \vert$ the total number of edges in $w$, and by $w[i]$ the $i^{th}$ edge in $w$. In this paper, we also allow walks to contain self-adjacent moves, i.e. moves of the form $(v_i, v_i)$ for every vertex in the graph. We do so to represent remaining at a fixed position for some length of time.

\section{Paths Connecting All Tasks} \label{sec:path_connecting_tasks_hardness}
As we have an algorithm for approximating a solution to the \textsc{Robot Scheduling} problem on a path, a natural subproblem is in general graphs - does there exist a path connecting all tasks, and if so, can we find one?

The problem PATH is : given a graph $G=(V,E)$ and a set $S \subseteq V$, is there a path connecting all vertices in $S$ and if yes , can we find one such path ?

\begin{theorem}\label{thm:path_NPC}
The problem PATH is NP-Complete.
\end{theorem}

\begin{proof}
Consider S=V. Then PATH has a yes answer if and only if $G$ has a Hamiltonian path.   
\end{proof}


\begin{theorem}\label{thm:path_polytime}
    The problem PATH can be solved in polynomial time for fixed $k = \vert S \vert$. %fixed parameter tractable with respect to the parameter $k = \vert S \vert$.
\end{theorem}

\begin{proof}
Any path (if it exists) connecting all vertices in $S$ will visit them in some particular order. We shall examine all possible orderings of the $k$ vertices. Let such an ordering be $v_1, v_2, ...v_k$ . 
We now construct a graph G' such that if there exists $k$ disjoint paths in $G'$ then we have a path connecting all vertices in $S$ in $G$.
 For each vertex $v_i \in S$ we add the pair of vertices $v_i , v'_i$ to $G'$. Where $v'_i$ is adjacent to both $v_i$ and the neighbourhood of $v_i$ in $G$. 
We define two vertex sets, $A$ and $B$, with $A = \{v_1,v'_1,v_3,v'_3,...\}$ being all odd-numbered vertices and $B = \{v_2,v'_2,v_4,v'_4,...\}$ being all the even-numbered vertices.
We also add to vertices $s$ and $t$ to $G'$, with $s$ being adjacent to all vertices in $A$ (other than $v'_1$) and $t$ being adjacent to all vertices in $B$.

The problem is now whether or not there exist $k$ internally disjoint paths between $s$ and $t$, this can be done in polynomial time for fixed $k$, we do this by using the algorithm in \cite{lochet2021} with the graph $G'$ and $k$ copies of the pair $(s,t)$.

In the yes instance the $k$ disjoint paths will be of the form $s,v'_1,...,v_2,t$ , $s,v_3,...,v'_2,t$, $s,v'_3,...,v_4,t$ ... etc   and so in the original graph $G$ we have the path between all vertices in $S$ in the order  $v_1,...,v_k$. 
Since we need to do this for all orderings of the $k$ vertices in $S$ this can be done in $O(k!\cdot n^{O(k^{5^k})})$ time.
\end{proof}

%\begin{theorem}
    %The problem PATH is W[1]-hard, parametrised by $k = \vert S \vert$
%\end{theorem}

\section{Shortest Paths between Tasks}
If the answer to problem PATH is yes then a subsequent challenge is to efficiently find a shortest path connecting all tasks in $S$.

\section{Integer Programming}
\label{sec:integer_program}


%In the  Integer Linear program we formulat
% {\color{blue} OLD: We {\color{red}{shall}} start by introducing the variables used in our integer linear program and what they represent. 
% $x_{r,v,t} \in \{0,1\}$ is equal to 1 if and only if the r\textsuperscript{th} robot is at vertex $v$ at timestep $t$.  Similarly $m_{r,v,t} \in \{0,1\}$ represents whether the r\textsuperscript{th} robot landed on vertex $v$ at timestep $t$ .  
% $TCR_{r,i,t} \in \{0,1\} $ is 1 if and only if the $i$\textsuperscript{th} task has been completed by the $r^\textrm{th}$ robot on or before timestep $t$. Similarly $TC_{i,t} \in \{0,1\}$ represents whether task $i$ has been complete by timestep $t$.
% $AC_t \in \{0,1\}$ is 1 if and only if all tasks have been complete on timestep $t$ (i.e. $AC_t =1 $ means the last task(s) to be done were completed on timestep $t$ and therefore $AC_t=0$ for all timesteps before or after $t$).We, of course wish to minimize the timespan of the schedule produced by the integer program, and so our objective function is to minimize the timestep $t$ such that $AC_t =1$.}

We now introduce our integer programming model. First, we outline the variables used in the program, before moving on to the constraints and objective function. We prove the correctness of the program as the corresponding constraints are presented.
The primary variables used in this program to represent the schedule of each robot are the set of binary variables $x_{r, v, t} \in \{0, 1\}$, where $r \in [k], v \in V$ and $t \in [\tau]$. Here $x_{r, v, t} = 1$ iff the $r^{th}$ robot is on vertex $v$ at timestep $t$. We use these variables to construct a schedule for each robot by setting robot $r$ to be at vertex $v$ at timestep $t$ iff $x_{r, v, t} = 1$. If the robot $r$ occupies some vertex $v$ for containing a task requiring $\ell$ timesteps for $\ell$ consecutive timesteps, we assume that $r$ completes the task while at that vertex.

Additionally, we introduce the binary variables $TC_{i, t} \in \{0, 1\}$, for every $i \in [T], t \in [\tau]$ with $TCR_{i, t} = 1$ iff the $i^{th}$ task has been completed by timestep $t$. Finally, we use the set of binary variables $AC_{t} \in \{0, 1\}$, for every $t \in [\tau]$ to denote if every task in the graph has been completed by timestep $t$.

% {\color{blue} Breaking up the constraints to make it clearer what property each handles.}

We now present our constraints. First, we introduce the \emph{movement constraints}, i.e. the constraints determining that the robot is at exactly one position at each timestep, that the robots can only stay in place, or move between adjacent vertices, between sequential timesteps, and that the schedule is collision free.

\begin{align}
    \sum\limits_{v \in V} x_{r, v, t} = 1 & & \forall r \in [k], t \in [\tau]\label{eq:only_one_robot_position_per_timestep}\\
    x_{r, v, t} \leq \sum\limits_{N(v) \cup \{ v \}} x_{r, v, t - 1} & & \forall r \in [k], t \in [2, \tau] \label{eq:adjancent_robots}\\
    \sum\limits_{r \in [k]} x_{r, v, t} \leq 1 & & \forall v \in V, t \in [\tau]\label{eq:only_one_robot_per_vertex_timestep}\\
    x_{r, v, t} + x_{r, v', t - 1} + x_{r', v, t - 1} + x_{r', v', t} \leq 3 & & \forall r \in [k], r' \in [k] \setminus \{ r \}, (v, v') \in E, t \in [\tau] \label{eq:no_crossover}
\end{align}

\begin{lemma}
    \label{lem:collsion_free}
    Constraints \ref{eq:only_one_robot_position_per_timestep}, \ref{eq:adjancent_robots}, \ref{eq:only_one_robot_per_vertex_timestep}, and \ref{eq:no_crossover} ensure that the schedule given by the integer program is collision free.
\end{lemma}

\begin{proof}
    From Constraint \ref{eq:only_one_robot_position_per_timestep} we have that each robot occupies exactly one position during each timestep, as otherwise $\sum\limits_{v \in V} x_{r, v, t} > 1$, if $r$ occupies two positions, or, $\sum\limits_{v \in V} x_{r, v, t} = 0$, if $r$ is not assigned any position. From Constraint \ref{eq:adjancent_robots} we have that robot $r$ can be at vertex $v$ at timestep $t$ iff $r$ was at $v$, or some vertex neighbouring $v$, in the previous timestep, for any timestep in $[2, \tau]$, i.e. any timestep other than the first one. From Constraint \ref{eq:only_one_robot_per_vertex_timestep} we have that no vertex can be occupied by more than one robot at any given timestep. Finally, Constraint \ref{eq:no_crossover} ensures that, given any pair of robots $r, r' \in [k]$ where $r \neq r'$, we ensure that $r$ and $r'$ do not attempt to cross the same edge in the same timestep. Explicitly, if $ x_{r, v, t} + x_{r, v', t - 1} + x_{r', v, t - 1} + x_{r', v', t} = 4$, then $r$ moves from $v'$ to $v$ in timestep $t$, and $r'$ moves from $v$ to $v'$ in timestep $t$. Otherwise, one or both of $r$ and $r'$ do not use $(v, v')$ in timestep $t$.
    Therefore, the schedule is collision free.
\end{proof}

We now present our constraints concerning the completion of tasks.

\begin{align}
    TC_{i, t} \leq TC_{i, t - 1} + \max_{r \in [k]} \left(\sum\limits_{j \in [t - d_i, t]} x_{r, v_i, j} / d_i\right) & & \forall i \in [T], t \in [\tau]\label{eq:note_when_tasks_are_complete}\\
    TC_{i, t} \geq TC_{i, t - 1} & & \forall i \in [T], t \in [\tau] \label{eq:tasks_stay_completed}
\end{align}

% {\color{blue} To late to really update, but we could use
% \begin{align*}
%         TC_{i, t} \geq \left(\sum\limits_{j \in [t - d_i, t]} x_{r, v_i, j} / (d_i - 1)\right) - 1 & & \forall r \in [k], i \in [T], t \in [\tau]
% \end{align*}
% As a simplification of constraint \ref{eq:note_when_tasks_are_complete} that also forces $TC_{i, t}$ to be one as soon as we complete the task.
% }

\begin{lemma}
    \label{lem:TC_constraints_work}
    The value of $TC_{i, t}$ is one only if some robot has completed task $i$ by timestep $t$.
\end{lemma}

\begin{proof}
    From Constraint \ref{eq:note_when_tasks_are_complete}, the value of $TC_{i, t}$ can be one only if either $TC_{i, t - 1} = 1$ or $\sum\limits_{j \in [t - d_i, t]} x_{r, v_i, j} / d_i = 1$, for some robot $r$. Note that, by our construction, we assume that if robot $r$ remains on the vertex $v$ for $d_i$ timesteps, then it will complete task $i$, therefore, if $\sum\limits_{j \in [t - d_i, t]} x_{r, v_i, j} / d_i = 1$ for any robot $r \in [k]$, the task $i$ must be completed. Further, by Constraint \ref{eq:tasks_stay_completed}, if $TC_{i, t - 1} = 1$, indicating that the task has been completed, then $TC_{i, t} = 1$.
    Therefore, the correctness follows.
\end{proof}

\begin{lemma}
    \label{lem:TC_constaints_allow_the_right_answer}
    Given any schedule where task $i$ is completed by timestep $t$, the value of $TC_{i,t}$ may be set to one.
\end{lemma}

\begin{proof}
    Observe that task $i$ can be completed at timestep $t$ if there exists some robot $r$ and timestep $t' \in [t]$ such that $x_{r, i j} = 1$, for every $j \in [t' - d_i, t']$. Therefore, by Constraint \ref{eq:note_when_tasks_are_complete}, $TC_{i, t'}$ may be equal to $1$. By extension, if $TC_{i, t'} = 1$ then, for any $t'' \in [t' + 1, \tau]$, we can set $TC_{i, t''}$ to $1$. Thus we get the statement.
\end{proof}

Finally, we provide our constraint for the all complete $AC_t$ variable.

\begin{align}
    AC_t \leq \sum\limits_{i \in [T]} TC_{i, t} / T & & \forall t \in [\tau]\label{eq:all_complete}
\end{align}

\begin{lemma}
    \label{lem:AC}
    The value of $AC_t$ is $1$ only if every task has been completed by timestep $t$.
\end{lemma}

\begin{proof}
    From Constraint \ref{eq:all_complete}, $AC_t$ can be $1$ if and only if $TC_{i, t} = 1$, $\forall i \in [T]$. From Lemma \ref{lem:TC_constaints_allow_the_right_answer}, the value of $TC_{i, t}$ is one only if the task $i$ has been complete by timestep $t$.
    Therefore, $AC_t$ is one only if all tasks have been complete.
    % From Lemma \ref{lem:C_constaints_allow_the_right_answer}, given any timestep $t' \in [T]$ where task $i$ has been completed, there is an assignment of variables such that $TC_{i, t} = 1$. The
\end{proof}

\begin{theorem}
    The integer program formulated with Constraints \ref{eq:only_one_robot_position_per_timestep}, \ref{eq:adjancent_robots}, \ref{eq:only_one_robot_per_vertex_timestep},
\ref{eq:no_crossover},
\ref{eq:note_when_tasks_are_complete},
\ref{eq:tasks_stay_completed},
\ref{eq:all_complete} and the optimisation goal $$ \textrm{Maximise } \sum_{t\in [\tau]} AC_t$$
correctly finds the fastest, collision free, task completing schedule for a given instance of $k$
\end{theorem}

\begin{proof}
    From Lemmas \ref{lem:collsion_free}, \ref{lem:TC_constraints_work}, 
\ref{lem:TC_constaints_allow_the_right_answer}, and \ref{lem:AC} we have that the program will compute a collision free schedule, with $AC_t = 1$ only if every task has been completed by timestep $t$. Now, observe that the optimal solution to this program is the solution such that $AC_t = 1$ for the maximum number of values of $t \in [T]$. Following Lemma \ref{lem:TC_constaints_allow_the_right_answer}, for any given schedule such that task $i$ is completed by timestep $t$, there is a valid assignment of variables such that $TC_{t, i} = 1$. Therefore, given any schedule such that every task is complete by timestep $t$, there is an assignment of variables such that $AC_t = 1$. Thus, by finding the assignment such that $\sum_{t \in [\tau]} AC_t$ is maximised, we find the fastest collision free schedule that completes all the tasks in the minimum number of timesteps, giving the proof.
\end{proof}

%Minimize TS \\
%Subject to:\\
%%$x_{r,v,t} \leq x_{r,v-1,t-1} + x_{r,v,t-1} + x_{r,v+1,t-1}$ for every timestep t, robot r, vertex v \\
%%$x_{r,v,t} \leq x_{r,v,t-1} + \sum_{v' \in N(v)} x_{r,v',t-1}$  for all $v \in V, r \in [k]$ and all timesteps $t$\\
%$x_{r,v,t} \leq \sum_{v' \in N(v) \cup \{v\}} x_{r,v',t-1}$  for all $v \in V, r \in [k]$ and all timesteps $t$\\
%$\sum_{r\in [k]} x_{r,v,t}  \leq 1$ for every $v\in V ,t \in \mathbb{N}$  \\ 
%$x_{r,v,t} + x_{r,v',t+1} + x_{r',v',t} + x_{r',v,t+1} \leq 3 $ for all $(r,r') \in [k] \times [k] $, $(v,v') \in E$, timestep t \\
%%x_{r,v,t} + x_{r,v+1,t+1} + x_{r',v+1,t} + x_{r',v,t+1} \leq 3 $ for every pair of robots $(r,r')$, $v \in [n]$, timestep t \\
%$m_{r,v,t} = \sum_{v' \in N(v)} x[r,v',t-1] \cdot x[r,v,t]$ for every  timestep t, $r \in [k]$ , and $v\in V$ \\
%$\sum_{v \in V} x_{r,v,t}  = 1$ for every $r \in [k] ,t \in \mathbb{N}$\\%(each robot can only be at one vertex at once)\\
%%$TCR_{r,i,t} \leq TCR_{r,i,t-1} + \frac{ \sum_{j = t-d_i}^{t} x_{r,v_i,j}}{d_i}$ for every robot r, $i \in [m]$, timestep t \\ %(TCR only 1 if r has spent $d_i$ timesteps at $v_i$)\\ 
%$TCR_{r,i,t} \leq TCR_{r,i,t-1} + \frac{ \sum_{j = t-d_i}^{t} (1-m_{r,v_i,j})\cdot x_{r,v_i,j}}{d_i}$ for every robot r, $i \in [m]$, timestep t \\ %(TCR only 1 if r has spent $d_i$ timesteps at $v_i$)\\ 
%$TCR_{r,i,t} \geq TCR_{r,i,t-1}$ for every robot r, $i \in [m]$, timestep t.\\
%$TC_{i,t} \leq \sum_{r \in [k]} TCR[r,i,t]$ for every timestep $t, i \in [m]$\\
%$TC_{i,t} \geq TC_{i,t-1}$ for all $i \in [m]$, timestep t.\\
%$AC_t \leq \frac{\sum_{i \in[m]} TC_{i,t}}{m}$ for all timesteps $t$ \\
%$\sum_t AC_{t} = 1$ \\
%TS = $\sum_t t\cdot AC_t$  \\

% {\color{blue} Old:

% \begin{align}
%    &\textrm{Minimise } \sum_{t\in [\tau]} t\cdot AC_t  \notag\\
%    %&\textrm{Minimise } TS \notag\\
%    &\text{Subject To:} \notag \\
% &x_{r,v,t} \leq \sum_{v' \in N(v) \cup \{v\}} x_{r,v',t-1}\ \forall v \in V, r \in [k], t \in [\tau]\\
% &\sum_{r\in [k]} x_{r,v,t}  \leq 1\ \forall v\in V ,t \in [\tau]  \\ 
% &x_{r,v,t} + x_{r,v',t+1} + x_{r',v',t} + x_{r',v,t+1} \leq 3\ \forall (r,r') \in [k] \times [k] , (v,v') \in E, t \in [\tau]\\
% &m_{r,v,t} = \sum_{v' \in N(v)} x[r,v',t-1] \cdot x[r,v,t]\ \forall t \in [\tau], r \in [k] , v\in V \\
% &\sum_{v \in V} x_{r,v,t}  = 1\ \forall r \in [k] ,t \in [\tau]\\%(each robot can only be at one vertex at once)\\
% &TCR_{r,i,t} \leq TCR_{r,i,t-1} + \frac{ \sum_{j = t-d_i}^{t} (1-m_{r,v_i,j})\cdot x_{r,v_i,j}}{d_i} \forall r\in[k], i \in [m], t \in [\tau] \\ %(TCR only 1 if r has spent d_i timesteps at v_i)\\ 
% &TCR_{r,i,t} \geq TCR_{r,i,t-1}\ \forall r \in [k], i \in [m], t \in [\tau].\\
% &TC_{i,t} \leq \sum_{r \in [k]} TCR[r,i,t]\ \forall t\in[\tau], i \in [m]\\
% &TC_{i,t} \geq TC_{i,t-1}\  \forall i \in [m], \in [\tau].\\
% &AC_t \leq \frac{\sum_{i \in[m]} TC_{i,t}}{m}\ \forall t\in[\tau] \\
% &\sum_{t\in [\tau]} AC_{t} = 1 
% %&TS = \sum_{t\in [\tau]} t\cdot AC_t  
% \end{align}
% %\begin{theorem}
% %The Integer Program produces an optimal solution.
% %\end{theorem}
% %
% %Constraint (1) ensures that each robot can only move to an adjacent vertex or stay where it is in one timestep.
% %Constraints (2) and (3) uphold the collision free properties.
% %Constraint (4) encodes movement and (5) ensures that each robot exists at every timestep. 
% %Constraint (6) ensures that TCR is only one if that robot has completed that task, it checks the value of $x_{r,v_j}$ for the past $d_i$ timesteps (the fraction can only be equal to 1 if the task is complete). 
% %Constraint (8) makes $TC_{i,t}$ true if any robot has completed task $i$ by time $t$.
% %Constraints (7) and (9) say that a task that is complete stays complete in future timesteps. 
% %Constraint (10) makes sure $AC_t$ is true only if all tasks have been complete and Constraint (11) makes sure $AC_t$ is only true for a single timestep. %and (12) ensures that timestep is stored in $TS$.
% %\begin{lemma}
% %    Constraints (2) and (3) ensure the collision-free property is upheld. i.e. There is no timestep such that any pair of robots occupy the same vertex or traverse the same edge.
% %\end{lemma}
% %\begin{proof}
% %    Constraint (2) gives that every vertex has at most one robot on it on every timestep, ensuring no collision on a vertex.
% %
% %    If we were to have a collision on an edge (i.e. two robots traverse the same edge) the only way for it to happen (given no collisions at vertices) would be for robots $r$ and $r'$ to swap places in subsequent timesteps and so for each pair of adjacent vertices $v,v'$ if $r$ is on $v$ at timestep $t$ and $r'$ on $v'$ respectively, then only only one of $r, r'$ are able to traverse the edge $(v,v')$, if neither of them traversed it the sum in constraint (3) would equal 2, if one of them did it would equal 3 and if they both did it would equal 4 (which is the condition we're disallowing as it causes a collision). 
% %\end{proof}
% %
% %\begin{lemma}
% %    Constraint (6) correctly models task completion, i.e. $TCR_{r,i,t}$ = 1 if and only if $TCR_{r,i,t-1} =1 $ or robot $r$ has finished task $i$ at timestep $t$. 
% %\end{lemma}
% %
% %\begin{proof}
% %If $TCR_{r,i,t-1} = 1$ then constraint (7) means that $TCR_{r,i,t}$ must be 1.
% %Otherwise, the only way for $TCR_{r,i,t}$ to be equal to 1 is if $ \sum_{j = t-d_i}^{t} (1-m_{r,v_i,j})\cdot x_{r,v_i,j} = d_i$. 
% %Which means that robot $r$ was on the vertex containing task $i$ for timesteps $t-d_i,...,t-1,t$ which completes task $i$.
% %
% %%The first claim, that $TCR_{r,i,t-1} =1 \Rightarrow TCR_{r,i,t} = 1$ is given by both constraint (7) and the fact that 
% %\end{proof}
% %
% %\begin{lemma}
% %    Constraint (10) ensures $AC_t$ is only 1 if all $m$ tasks are complete.
% %\end{lemma}
% %\begin{proof}
% %    If we have 
% %\end{proof}


% %VARIABLES: \\
% %$x_{r,v,t} \in \{0,1\}$ (1 if robot $r$ is at vertex $v$ at timestep $t$\\
% %$m_{r,v,t} \in \{0,1\}$ (1 if robot $r$ has just moved onto vertex $v$ at timestep $t$\\
% %$TCR_{r,i,t} \in \{0,1\}$ (1 if robot $r$ has completed task $i$ by timestep $t$)\\
% %$TC_{i,t} \in \{0,1\}$ (1 if task $i$ completed by timestep $t$)\\
% %$AC_t \in \{0,1\}$ (1 if all tasks completed at timestep $t$ - 0 if before or after)\\
% %TS $\in \mathbb{N}$ ( the timestep where all tasks have been completed on - i.e. the TimeSpan of the schedule) 
% }   

\section{Shortest Walk Connecting Tasks}
A natural extension  to Section \ref{sec:path_connecting_tasks_hardness} is if we have an instance where we cannot find a path connecting all the vertices in the set $S$ - what is the shortest walk we can find which visits all vertices in $S$.
\begin{lemma}
Given a graph $G=(V,E)$ and a set $S \subseteq V$ the shortest walk connecting all vertices in $S$ can be found in $O(k! + k \cdot( n\log n  +\vert E \vert ))$ time.
\end{lemma}
\begin{proof}
We start by using Dijkstra's Algorithm starting from every vertex in $S$, which can be done in $O(k \cdot( n\log n  +\vert E \vert ))$ time. We then consider every ordering of the $k$ vertices in $S$, say $v_1,...,v_k$ and for each of these orderings we sum up the distance from $v_i$ to $v_{i+1}$ for all $i \in [k]$ and choose the ordering with the minimum of all such sums.
\end{proof}
\section{Experiments}

\begin{enumerate}
    \item Compare line Partition \& optimal with random length tasks.
    \item Compare time for equal length tasks on the line of the IP solver wth the partition algorithm (to see if the IP solver is more/equally efficient in general.
    \item Compare grid approximation (find the shortest path connecting all tasks then do path partition) with IP solver.
    \begin{itemize}
        \item For both equal length \& any length tasks
    \end{itemize}
\end{enumerate}

\printbibliography
\end{document}
