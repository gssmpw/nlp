\section{Related Works}
There is a large number of results across many settings and variations of the temporal graph exploration problem, including when the number of vertices an agent can visit in one timestep is unbounded ____, bounded ____, and for specific graph classes ____.
Particularly relevant to us is the work of Michail and Spirakis ____, who showed that the problem of determining the fastest exploration of a temporal graph is NP-hard, and, furthermore, no constant factor approximation algorithm exists of the shortest exploration (in terms of the length of the path found by the algorithm, compared to the shortest path exploring the graph) unless $P = NP$. As noted, the change in the structure of temporal graphs is close to the challenges implemented in our graph by agents blocking potential moves from each other.
In terms of positive results, the work of Erlebach et al. ____ provided a substantial set of results that have formed the basis for much of the subsequent work on algorithmic results for temporal graph exploration. Of particular interest to us are the results that show that, for temporal graphs that are connected in every timestep, an agent can visit any subset of $m$ vertices in at most $O(n m)$ time, and provide constructions for faster explorations of graphs with $b$ agents and an $(r, b)$-division ($O(n^2 b / r + n r b^2)$ time), and $2 \times n$ grids with $4 \log n$ agents ($O(n \log n)$ time).