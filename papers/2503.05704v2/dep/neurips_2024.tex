\documentclass{article}


% if you need to pass options to natbib, use, e.g.:
%     \PassOptionsToPackage{numbers, compress}{natbib}
% before loading neurips_2024


% ready for submission
\usepackage{neurips_2024}
\usepackage{graphicx} % Required for inserting images
\usepackage{graphicx, centernot, amsmath, bbm, booktabs, multirow, array, capt-of, varwidth, natbib, amstext, tabularx}
%\DeclareMathOperator{\E}{\mathbb{E}}

\usepackage[utf8]{inputenc}
\usepackage{dirtytalk}
\usepackage{csquotes,enumitem}
\usepackage{graphicx}
\usepackage{amsmath,amsthm,amsfonts}
\usepackage{lipsum}
\usepackage[utf8]{inputenc}
\usepackage{hyperref}
\usepackage{float}
\hypersetup{
    colorlinks=true,
    linkcolor=blue,
    filecolor=magenta,      
    urlcolor=cyan,
    pdftitle={Overleaf Example},
    pdfpagemode=FullScreen,
    }
\usepackage{tikz}
\usetikzlibrary{bayesnet}
%
%
\setlength\unitlength{1mm}
\newcommand{\twodots}{\mathinner {\ldotp \ldotp}}
% bb font symbols
\newcommand{\Rho}{\mathrm{P}}
\newcommand{\Tau}{\mathrm{T}}

\newfont{\bbb}{msbm10 scaled 700}
\newcommand{\CCC}{\mbox{\bbb C}}

\newfont{\bb}{msbm10 scaled 1100}
\newcommand{\CC}{\mbox{\bb C}}
\newcommand{\PP}{\mbox{\bb P}}
\newcommand{\RR}{\mbox{\bb R}}
\newcommand{\QQ}{\mbox{\bb Q}}
\newcommand{\ZZ}{\mbox{\bb Z}}
\newcommand{\FF}{\mbox{\bb F}}
\newcommand{\GG}{\mbox{\bb G}}
\newcommand{\EE}{\mbox{\bb E}}
\newcommand{\NN}{\mbox{\bb N}}
\newcommand{\KK}{\mbox{\bb K}}
\newcommand{\HH}{\mbox{\bb H}}
\newcommand{\SSS}{\mbox{\bb S}}
\newcommand{\UU}{\mbox{\bb U}}
\newcommand{\VV}{\mbox{\bb V}}


\newcommand{\yy}{\mathbbm{y}}
\newcommand{\xx}{\mathbbm{x}}
\newcommand{\zz}{\mathbbm{z}}
\newcommand{\sss}{\mathbbm{s}}
\newcommand{\rr}{\mathbbm{r}}
\newcommand{\pp}{\mathbbm{p}}
\newcommand{\qq}{\mathbbm{q}}
\newcommand{\ww}{\mathbbm{w}}
\newcommand{\hh}{\mathbbm{h}}
\newcommand{\vvv}{\mathbbm{v}}

% Vectors

\newcommand{\av}{{\bf a}}
\newcommand{\bv}{{\bf b}}
\newcommand{\cv}{{\bf c}}
\newcommand{\dv}{{\bf d}}
\newcommand{\ev}{{\bf e}}
\newcommand{\fv}{{\bf f}}
\newcommand{\gv}{{\bf g}}
\newcommand{\hv}{{\bf h}}
\newcommand{\iv}{{\bf i}}
\newcommand{\jv}{{\bf j}}
\newcommand{\kv}{{\bf k}}
\newcommand{\lv}{{\bf l}}
\newcommand{\mv}{{\bf m}}
\newcommand{\nv}{{\bf n}}
\newcommand{\ov}{{\bf o}}
\newcommand{\pv}{{\bf p}}
\newcommand{\qv}{{\bf q}}
\newcommand{\rv}{{\bf r}}
\newcommand{\sv}{{\bf s}}
\newcommand{\tv}{{\bf t}}
\newcommand{\uv}{{\bf u}}
\newcommand{\wv}{{\bf w}}
\newcommand{\vv}{{\bf v}}
\newcommand{\xv}{{\bf x}}
\newcommand{\yv}{{\bf y}}
\newcommand{\zv}{{\bf z}}
\newcommand{\zerov}{{\bf 0}}
\newcommand{\onev}{{\bf 1}}

% Matrices

\newcommand{\Am}{{\bf A}}
\newcommand{\Bm}{{\bf B}}
\newcommand{\Cm}{{\bf C}}
\newcommand{\Dm}{{\bf D}}
\newcommand{\Em}{{\bf E}}
\newcommand{\Fm}{{\bf F}}
\newcommand{\Gm}{{\bf G}}
\newcommand{\Hm}{{\bf H}}
\newcommand{\Id}{{\bf I}}
\newcommand{\Jm}{{\bf J}}
\newcommand{\Km}{{\bf K}}
\newcommand{\Lm}{{\bf L}}
\newcommand{\Mm}{{\bf M}}
\newcommand{\Nm}{{\bf N}}
\newcommand{\Om}{{\bf O}}
\newcommand{\Pm}{{\bf P}}
\newcommand{\Qm}{{\bf Q}}
\newcommand{\Rm}{{\bf R}}
\newcommand{\Sm}{{\bf S}}
\newcommand{\Tm}{{\bf T}}
\newcommand{\Um}{{\bf U}}
\newcommand{\Wm}{{\bf W}}
\newcommand{\Vm}{{\bf V}}
\newcommand{\Xm}{{\bf X}}
\newcommand{\Ym}{{\bf Y}}
\newcommand{\Zm}{{\bf Z}}

% Calligraphic

\newcommand{\Ac}{{\cal A}}
\newcommand{\Bc}{{\cal B}}
\newcommand{\Cc}{{\cal C}}
\newcommand{\Dc}{{\cal D}}
\newcommand{\Ec}{{\cal E}}
\newcommand{\Fc}{{\cal F}}
\newcommand{\Gc}{{\cal G}}
\newcommand{\Hc}{{\cal H}}
\newcommand{\Ic}{{\cal I}}
\newcommand{\Jc}{{\cal J}}
\newcommand{\Kc}{{\cal K}}
\newcommand{\Lc}{{\cal L}}
\newcommand{\Mc}{{\cal M}}
\newcommand{\Nc}{{\cal N}}
\newcommand{\nc}{{\cal n}}
\newcommand{\Oc}{{\cal O}}
\newcommand{\Pc}{{\cal P}}
\newcommand{\Qc}{{\cal Q}}
\newcommand{\Rc}{{\cal R}}
\newcommand{\Sc}{{\cal S}}
\newcommand{\Tc}{{\cal T}}
\newcommand{\Uc}{{\cal U}}
\newcommand{\Wc}{{\cal W}}
\newcommand{\Vc}{{\cal V}}
\newcommand{\Xc}{{\cal X}}
\newcommand{\Yc}{{\cal Y}}
\newcommand{\Zc}{{\cal Z}}

% Bold greek letters

\newcommand{\alphav}{\hbox{\boldmath$\alpha$}}
\newcommand{\betav}{\hbox{\boldmath$\beta$}}
\newcommand{\gammav}{\hbox{\boldmath$\gamma$}}
\newcommand{\deltav}{\hbox{\boldmath$\delta$}}
\newcommand{\etav}{\hbox{\boldmath$\eta$}}
\newcommand{\lambdav}{\hbox{\boldmath$\lambda$}}
\newcommand{\epsilonv}{\hbox{\boldmath$\epsilon$}}
\newcommand{\nuv}{\hbox{\boldmath$\nu$}}
\newcommand{\muv}{\hbox{\boldmath$\mu$}}
\newcommand{\zetav}{\hbox{\boldmath$\zeta$}}
\newcommand{\phiv}{\hbox{\boldmath$\phi$}}
\newcommand{\psiv}{\hbox{\boldmath$\psi$}}
\newcommand{\thetav}{\hbox{\boldmath$\theta$}}
\newcommand{\tauv}{\hbox{\boldmath$\tau$}}
\newcommand{\omegav}{\hbox{\boldmath$\omega$}}
\newcommand{\xiv}{\hbox{\boldmath$\xi$}}
\newcommand{\sigmav}{\hbox{\boldmath$\sigma$}}
\newcommand{\piv}{\hbox{\boldmath$\pi$}}
\newcommand{\rhov}{\hbox{\boldmath$\rho$}}
\newcommand{\upsilonv}{\hbox{\boldmath$\upsilon$}}

\newcommand{\Gammam}{\hbox{\boldmath$\Gamma$}}
\newcommand{\Lambdam}{\hbox{\boldmath$\Lambda$}}
\newcommand{\Deltam}{\hbox{\boldmath$\Delta$}}
\newcommand{\Sigmam}{\hbox{\boldmath$\Sigma$}}
\newcommand{\Phim}{\hbox{\boldmath$\Phi$}}
\newcommand{\Pim}{\hbox{\boldmath$\Pi$}}
\newcommand{\Psim}{\hbox{\boldmath$\Psi$}}
\newcommand{\Thetam}{\hbox{\boldmath$\Theta$}}
\newcommand{\Omegam}{\hbox{\boldmath$\Omega$}}
\newcommand{\Xim}{\hbox{\boldmath$\Xi$}}


% Sans Serif small case

\newcommand{\Gsf}{{\sf G}}

\newcommand{\asf}{{\sf a}}
\newcommand{\bsf}{{\sf b}}
\newcommand{\csf}{{\sf c}}
\newcommand{\dsf}{{\sf d}}
\newcommand{\esf}{{\sf e}}
\newcommand{\fsf}{{\sf f}}
\newcommand{\gsf}{{\sf g}}
\newcommand{\hsf}{{\sf h}}
\newcommand{\isf}{{\sf i}}
\newcommand{\jsf}{{\sf j}}
\newcommand{\ksf}{{\sf k}}
\newcommand{\lsf}{{\sf l}}
\newcommand{\msf}{{\sf m}}
\newcommand{\nsf}{{\sf n}}
\newcommand{\osf}{{\sf o}}
\newcommand{\psf}{{\sf p}}
\newcommand{\qsf}{{\sf q}}
\newcommand{\rsf}{{\sf r}}
\newcommand{\ssf}{{\sf s}}
\newcommand{\tsf}{{\sf t}}
\newcommand{\usf}{{\sf u}}
\newcommand{\wsf}{{\sf w}}
\newcommand{\vsf}{{\sf v}}
\newcommand{\xsf}{{\sf x}}
\newcommand{\ysf}{{\sf y}}
\newcommand{\zsf}{{\sf z}}


% mixed symbols

\newcommand{\sinc}{{\hbox{sinc}}}
\newcommand{\diag}{{\hbox{diag}}}
\renewcommand{\det}{{\hbox{det}}}
\newcommand{\trace}{{\hbox{tr}}}
\newcommand{\sign}{{\hbox{sign}}}
\renewcommand{\arg}{{\hbox{arg}}}
\newcommand{\var}{{\hbox{var}}}
\newcommand{\cov}{{\hbox{cov}}}
\newcommand{\Ei}{{\rm E}_{\rm i}}
\renewcommand{\Re}{{\rm Re}}
\renewcommand{\Im}{{\rm Im}}
\newcommand{\eqdef}{\stackrel{\Delta}{=}}
\newcommand{\defines}{{\,\,\stackrel{\scriptscriptstyle \bigtriangleup}{=}\,\,}}
\newcommand{\<}{\left\langle}
\renewcommand{\>}{\right\rangle}
\newcommand{\herm}{{\sf H}}
\newcommand{\trasp}{{\sf T}}
\newcommand{\transp}{{\sf T}}
\renewcommand{\vec}{{\rm vec}}
\newcommand{\Psf}{{\sf P}}
\newcommand{\SINR}{{\sf SINR}}
\newcommand{\SNR}{{\sf SNR}}
\newcommand{\MMSE}{{\sf MMSE}}
\newcommand{\REF}{{\RED [REF]}}

% Markov chain
\usepackage{stmaryrd} % for \mkv 
\newcommand{\mkv}{-\!\!\!\!\minuso\!\!\!\!-}

% Colors

\newcommand{\RED}{\color[rgb]{1.00,0.10,0.10}}
\newcommand{\BLUE}{\color[rgb]{0,0,0.90}}
\newcommand{\GREEN}{\color[rgb]{0,0.80,0.20}}

%%%%%%%%%%%%%%%%%%%%%%%%%%%%%%%%%%%%%%%%%%
\usepackage{hyperref}
\hypersetup{
    bookmarks=true,         % show bookmarks bar?
    unicode=false,          % non-Latin characters in AcrobatÕs bookmarks
    pdftoolbar=true,        % show AcrobatÕs toolbar?
    pdfmenubar=true,        % show AcrobatÕs menu?
    pdffitwindow=false,     % window fit to page when opened
    pdfstartview={FitH},    % fits the width of the page to the window
%    pdftitle={My title},    % title
%    pdfauthor={Author},     % author
%    pdfsubject={Subject},   % subject of the document
%    pdfcreator={Creator},   % creator of the document
%    pdfproducer={Producer}, % producer of the document
%    pdfkeywords={keyword1} {key2} {key3}, % list of keywords
    pdfnewwindow=true,      % links in new window
    colorlinks=true,       % false: boxed links; true: colored links
    linkcolor=red,          % color of internal links (change box color with linkbordercolor)
    citecolor=green,        % color of links to bibliography
    filecolor=blue,      % color of file links
    urlcolor=blue           % color of external links
}
%%%%%%%%%%%%%%%%%%%%%%%%%%%%%%%%%%%%%%%%%%%



% !TeX root = main.tex 


\newcommand{\lnote}{\textcolor[rgb]{1,0,0}{Lydia: }\textcolor[rgb]{0,0,1}}
\newcommand{\todo}{\textcolor[rgb]{1,0,0.5}{To do: }\textcolor[rgb]{0.5,0,1}}


\newcommand{\state}{S}
\newcommand{\meas}{M}
\newcommand{\out}{\mathrm{out}}
\newcommand{\piv}{\mathrm{piv}}
\newcommand{\pivotal}{\mathrm{pivotal}}
\newcommand{\isnot}{\mathrm{not}}
\newcommand{\pred}{^\mathrm{predict}}
\newcommand{\act}{^\mathrm{act}}
\newcommand{\pre}{^\mathrm{pre}}
\newcommand{\post}{^\mathrm{post}}
\newcommand{\calM}{\mathcal{M}}

\newcommand{\game}{\mathbf{V}}
\newcommand{\strategyspace}{S}
\newcommand{\payoff}[1]{V^{#1}}
\newcommand{\eff}[1]{E^{#1}}
\newcommand{\p}{\vect{p}}
\newcommand{\simplex}[1]{\Delta^{#1}}

\newcommand{\recdec}[1]{\bar{D}(\hat{Y}_{#1})}





\newcommand{\sphereone}{\calS^1}
\newcommand{\samplen}{S^n}
\newcommand{\wA}{w}%{w_{\mathfrak{a}}}
\newcommand{\Awa}{A_{\wA}}
\newcommand{\Ytil}{\widetilde{Y}}
\newcommand{\Xtil}{\widetilde{X}}
\newcommand{\wst}{w_*}
\newcommand{\wls}{\widehat{w}_{\mathrm{LS}}}
\newcommand{\dec}{^\mathrm{dec}}
\newcommand{\sub}{^\mathrm{sub}}

\newcommand{\calP}{\mathcal{P}}
\newcommand{\totspace}{\calZ}
\newcommand{\clspace}{\calX}
\newcommand{\attspace}{\calA}

\newcommand{\Ftil}{\widetilde{\calF}}

\newcommand{\totx}{Z}
\newcommand{\classx}{X}
\newcommand{\attx}{A}
\newcommand{\calL}{\mathcal{L}}



\newcommand{\defeq}{\mathrel{\mathop:}=}
\newcommand{\vect}[1]{\ensuremath{\mathbf{#1}}}
\newcommand{\mat}[1]{\ensuremath{\mathbf{#1}}}
\newcommand{\dd}{\mathrm{d}}
\newcommand{\grad}{\nabla}
\newcommand{\hess}{\nabla^2}
\newcommand{\argmin}{\mathop{\rm argmin}}
\newcommand{\argmax}{\mathop{\rm argmax}}
\newcommand{\Ind}[1]{\mathbf{1}\{#1\}}

\newcommand{\norm}[1]{\left\|{#1}\right\|}
\newcommand{\fnorm}[1]{\|{#1}\|_{\text{F}}}
\newcommand{\spnorm}[2]{\left\| {#1} \right\|_{\text{S}({#2})}}
\newcommand{\sigmin}{\sigma_{\min}}
\newcommand{\tr}{\text{tr}}
\renewcommand{\det}{\text{det}}
\newcommand{\rank}{\text{rank}}
\newcommand{\logdet}{\text{logdet}}
\newcommand{\trans}{^{\top}}
\newcommand{\poly}{\text{poly}}
\newcommand{\polylog}{\text{polylog}}
\newcommand{\st}{\text{s.t.~}}
\newcommand{\proj}{\mathcal{P}}
\newcommand{\projII}{\mathcal{P}_{\parallel}}
\newcommand{\projT}{\mathcal{P}_{\perp}}
\newcommand{\projX}{\mathcal{P}_{\mathcal{X}^\star}}
\newcommand{\inner}[1]{\langle #1 \rangle}

\renewcommand{\Pr}{\mathbb{P}}
\newcommand{\Z}{\mathbb{Z}}
\newcommand{\N}{\mathbb{N}}
\newcommand{\R}{\mathbb{R}}
\newcommand{\E}{\mathbb{E}}
\newcommand{\F}{\mathcal{F}}
\newcommand{\var}{\mathrm{var}}
\newcommand{\cov}{\mathrm{cov}}


\newcommand{\calN}{\mathcal{N}}

\newcommand{\jccomment}{\textcolor[rgb]{1,0,0}{C: }\textcolor[rgb]{1,0,1}}
\newcommand{\fracpar}[2]{\frac{\partial #1}{\partial  #2}}

\newcommand{\A}{\mathcal{A}}
\newcommand{\B}{\mat{B}}
%\newcommand{\C}{\mat{C}}

\newcommand{\I}{\mat{I}}
\newcommand{\M}{\mat{M}}
\newcommand{\D}{\mat{D}}
%\newcommand{\U}{\mat{U}}
\newcommand{\V}{\mat{V}}
\newcommand{\W}{\mat{W}}
\newcommand{\X}{\mat{X}}
\newcommand{\Y}{\mat{Y}}
\newcommand{\mSigma}{\mat{\Sigma}}
\newcommand{\mLambda}{\mat{\Lambda}}
\newcommand{\e}{\vect{e}}
\newcommand{\g}{\vect{g}}
\renewcommand{\u}{\vect{u}}
\newcommand{\w}{\vect{w}}
\newcommand{\x}{\vect{x}}
\newcommand{\y}{\vect{y}}
\newcommand{\z}{\vect{z}}
\newcommand{\fI}{\mathfrak{I}}
\newcommand{\fS}{\mathfrak{S}}
\newcommand{\fE}{\mathfrak{E}}
\newcommand{\fF}{\mathfrak{F}}

\newcommand{\Risk}{\mathcal{R}}

\renewcommand{\L}{\mathcal{L}}
\renewcommand{\H}{\mathcal{H}}

\newcommand{\cn}{\kappa}
\newcommand{\nn}{\nonumber}


\newcommand{\Hess}{\nabla^2}
\newcommand{\tlO}{\tilde{O}}
\newcommand{\tlOmega}{\tilde{\Omega}}

\newcommand{\calF}{\mathcal{F}}
\newcommand{\fhat}{\widehat{f}}
\newcommand{\calS}{\mathcal{S}}

\newcommand{\calX}{\mathcal{X}}
\newcommand{\calY}{\mathcal{Y}}
\newcommand{\calD}{\mathcal{D}}
\newcommand{\calZ}{\mathcal{Z}}
\newcommand{\calA}{\mathcal{A}}
\newcommand{\fbayes}{f^B}
\newcommand{\func}{f^U}


\newcommand{\bayscore}{\text{calibrated Bayes score}}
\newcommand{\bayrisk}{\text{calibrated Bayes risk}}

\newtheorem{example}{Example}[section]
\newtheorem{exc}{Exercise}[section]
%\newtheorem{rem}{Remark}[section]

\newtheorem{theorem}{Theorem}[section]
\newtheorem{definition}{Definition}
\newtheorem{proposition}[theorem]{Proposition}
\newtheorem{corollary}[theorem]{Corollary}

\newtheorem{remark}{Remark}[section]
\newtheorem{lemma}[theorem]{Lemma}
\newtheorem{claim}[theorem]{Claim}
\newtheorem{fact}[theorem]{Fact}
\newtheorem{assumption}{Assumption}

\newcommand{\iidsim}{\overset{\mathrm{i.i.d.}}{\sim}}
\newcommand{\unifsim}{\overset{\mathrm{unif}}{\sim}}
\newcommand{\sign}{\mathrm{sign}}
\newcommand{\wbar}{\overline{w}}
\newcommand{\what}{\widehat{w}}
\newcommand{\KL}{\mathrm{KL}}
\newcommand{\Bern}{\mathrm{Bernoulli}}
\newcommand{\ihat}{\widehat{i}}
\newcommand{\Dwst}{\calD^{w_*}}
\newcommand{\fls}{\widehat{f}_{n}}


\newcommand{\brpi}{\pi^{br}}
\newcommand{\brtheta}{\theta^{br}}

% \newcommand{\M}{\mat{M}}
% \newcommand\Mmh{\mat{M}^{-1/2}}
% \newcommand{\A}{\mat{A}}
% \newcommand{\B}{\mat{B}}
% \newcommand{\C}{\mat{C}}
% \newcommand{\Et}[1][t]{\mat{E_{#1}}}
% \newcommand{\Etp}{\Et[t+1]}
% \newcommand{\Errt}[1][t]{\mat{\bigtriangleup_{#1}}}
% \newcommand\cnM{\kappa}
% \newcommand{\cn}[1]{\kappa\left(#1\right)}
% \newcommand\X{\mat{X}}
% \newcommand\fstar{f_*}
% \newcommand\Xt[1][t]{\mat{X_{#1}}}
% \newcommand\ut[1][t]{{u_{#1}}}
% \newcommand\Xtinv{\inv{\Xt}}
% \newcommand\Xtp{\mat{X_{t+1}}}
% \newcommand\Xtpinv{\inv{\left(\mat{X_{t+1}}\right)}}
% \newcommand\U{\mat{U}}
% \newcommand\UTr{\trans{\mat{U}}}
% \newcommand{\Ut}[1][t]{\mat{U_{#1}}}
% \newcommand{\Utinv}{\inv{\Ut}}
% \newcommand{\UtTr}[1][t]{\trans{\mat{U_{#1}}}}
% \newcommand\Utp{\mat{U_{t+1}}}
% \newcommand\UtpTr{\trans{\mat{U}_{t+1}}}
% \newcommand\Utptild{\mat{\widetilde{U}_{t+1}}}
% \newcommand\Us{\mat{U^*}}
% \newcommand\UsTr{\trans{\mat{U^*}}}
% \newcommand{\Sigs}{\mat{\Sigma}}
% \newcommand{\Sigsmh}{\Sigs^{-1/2}}
% \newcommand{\eye}{\mat{I}}
% \newcommand{\twonormbound}{\left(4+\DPhi{\M}{\Xt[0]}\right)\twonorm{\M}}
% \newcommand{\lamj}{\lambda_j}

% \renewcommand\u{\vect{u}}
% \newcommand\uTr{\trans{\vect{u}}}
% \renewcommand\v{\vect{v}}
% \newcommand\vTr{\trans{\vect{v}}}
% \newcommand\w{\vect{w}}
% \newcommand\wTr{\trans{\vect{w}}}
% \newcommand\wperp{\vect{w}_{\perp}}
% \newcommand\wperpTr{\trans{\vect{w}_{\perp}}}
% \newcommand\wj{\vect{w_j}}
% \newcommand\vj{\vect{v_j}}
% \newcommand\wjTr{\trans{\vect{w_j}}}
% \newcommand\vjTr{\trans{\vect{v_j}}}

% \newcommand{\DPhi}[2]{\ensuremath{D_{\Phi}\left(#1,#2\right)}}
% \newcommand\matmult{{\omega}}

%\documentclass{article}
%\usepackage[demo]{graphicx}
\usepackage{caption}
\usepackage{subcaption}

% to compile a preprint version, e.g., for submission to arXiv, add add the
% [preprint] option:
%     \usepackage[preprint]{neurips_2024}


% to compile a camera-ready version, add the [final] option, e.g.:
%     \usepackage[final]{neurips_2024}


% to avoid loading the natbib package, add option nonatbib:
%    \usepackage[nonatbib]{neurips_2024}


\usepackage[utf8]{inputenc} % allow utf-8 input
\usepackage[T1]{fontenc}    % use 8-bit T1 fonts
\usepackage{hyperref}       % hyperlinks
\usepackage{url}            % simple URL typesetting
\usepackage{booktabs}       % professional-quality tables
\usepackage{amsfonts}       % blackboard math symbols
\usepackage{nicefrac}       % compact symbols for 1/2, etc.
\usepackage{microtype}      % microtypography
\usepackage{xcolor}         % colors


%\title{Designing Experimental Evaluation of Algorithmic Interventions with Human Decision Makers In Mind}

\title{Evaluating Algorithmic Interventions \\ with Human Decision Makers In Mind}


% The \author macro works with any number of authors. There are two commands
% used to separate the names and addresses of multiple authors: \And and \AND.
%
% Using \And between authors leaves it to LaTeX to determine where to break the
% lines. Using \AND forces a line break at that point. So, if LaTeX puts 3 of 4
% authors names on the first line, and the last on the second line, try using
% \AND instead of \And before the third author name.


\author{%
  David S.~Hippocampus\thanks{Use footnote for providing further information
    about author (webpage, alternative address)---\emph{not} for acknowledging
    funding agencies.} \\
  Department of Computer Science\\
  Cranberry-Lemon University\\
  Pittsburgh, PA 15213 \\
  \texttt{hippo@cs.cranberry-lemon.edu} \\
  % examples of more authors
  % \And
  % Coauthor \\
  % Affiliation \\
  % Address \\
  % \texttt{email} \\
  % \AND
  % Coauthor \\
  % Affiliation \\
  % Address \\
  % \texttt{email} \\
  % \And
  % Coauthor \\
  % Affiliation \\
  % Address \\
  % \texttt{email} \\
  % \And
  % Coauthor \\
  % Affiliation \\
  % Address \\
  % \texttt{email} \\
}


\begin{document}


\maketitle


\begin{abstract}
Automated decision systems (ADS) are broadly deployed to inform or support human decision-making across a wide range of consequential contexts. An emerging approach to the assessment of such systems is through experimental evaluation, which aims to measure the causal impacts of the ADS deployment on decision making and outcomes. However, various context-specific details complicate the goal of establishing meaningful experimental evaluations for algorithmic interventions. Notably, current experimental designs rely on simplifying assumptions about human decision making in order to derive causal estimates. In reality, cognitive biases of human decision makers \emph{induced} by experimental design choices may significantly alter the observed effect sizes of the algorithmic intervention.
In this paper, we formalize and investigate various models of human decision-making in the presence of a predictive algorithmic aid. We show  that each of these behavioral models produces dependencies across decision subjects and results in the violation of existing assumptions, with consequences for treatment effect estimation. 
This work aims to further advance the scientific validity of intervention-based evaluation schemes for the assessment of algorithmic deployments.
\end{abstract}

\section{Introduction}

A growing body of research reveals a significant gap between developing predictive algorithms and ensuring they improve outcomes through better decision-making \citep{kleinberg2018human, liu2023reimagining, liu2024actionability}. Human discretion often influences the final decisions; hence there is "not necessarily a one-to-one mapping from the predictive tool to the final outcome.'' \citep{albright2019if} 
While the field of machine learning is focused on the evaluation of predictive models using accuracy metrics, often on benchmark datasets \citep{raji2021ai}, simply assessing the predictive models themselves do not necessarily reveal the full pattern of how humans \emph{interact} with the algorithm~\cite{green2019disparate}, and thus which decisions they will make to influence the outcomes for the impacted population. ~\cite{kleinberg2018human} goes further to illustrate how, unlike prediction evaluation which focuses on minimizing model generalization error, decisions involve a more complex calculation of cost-benefit trade-offs, requiring considerations beyond just model performance. 

The difference between \emph{predictions} and \emph{decisions} - also sometimes referred to as \emph{judgements} ~\cite{agrawal2018prediction} - thus necessitates a critical expansion of scope in terms of what is being evaluated.
Rather than scoping down to the output of a particular algorithm, the focus is instead on the outcomes of overall policy changes to the entire system. Once deployed, algorithmic models operate much more like typical policy interventions, not isolated prediction problems. These \emph{algorithmic interventions} are thus ideally evaluated around the causal question of how much the introduction of the algorithm impacts some important downstream outcome. This paradigm is about hypothesis testing a treatment - effectively thinking through the counterfactual measurement of what happens in the presence and in absence of the algorithmic intervention for a given decision-maker on a specific case. What we actually need to evaluate is not just the accuracy of the models' predictions, but the impact of such predictions on the actions of decision-makers and the appropriateness of those decisions.

One obvious approach to the estimation of these causal effects is through the implementation of \emph{experimental evaluations}. Several efforts in domains including criminal justice, healthcare, and education have begun conducting these evaluations to estimate the impact of introducing an algorithmic intervention to modify a given default policy. 

However, it is clear that those conducting these experiments follow the protocol for any other type of intervention context, and are not necessarily factoring in the particularities of the algorithmic context into their experiment design. We argue that these experiment design choices impact  the \emph{responsiveness} (i.e., their tendency to be swayed by an algorithmic decision aid) of the judges relying on algorithmic recommendations, which in turn distorts our understanding of the average treatment effect of algorithmic interventions. 

%In the case of our study, the causal question involved is examining how the introduction of risk assessments impacts judge sentencing, overall and for specific subgroups in a way that could exacerbate disparties.

%What we actually need to evaluate is not just the accuracy of the models' predictions, but the impact of such predictions on the actions of decision-makers and the appropriateness of those decisions. In other words, we need to take a more systems level view of what's happening with the algorithm. This paradigm is about hypothesis testing a treatment - effectively thinking through the counterfactual measurement of what happens in the presence and in absence of the algorithmic intervention for a given judge on a specific case. 

%In particular, 



%TODO:

Our contributions are as follows:

\begin{enumerate}[leftmargin=*]
    \item We examine the role of specific \emph{experiment design choices} in judge responsiveness -- specifically, choices that experiment designers make on (1) the treatment assignment model ($Z$), (2) the positive prediction rate  ($P (\hat{Y} = 1)$) (by setting the threshold of the model with respect to recommended action) and (3) the model correctness ($P (\hat{Y} = Y)$) (through model selection). We conclude that beyond intrinsic judge biases, experimentation design choice can also impact judge responsiveness. 
    \item We mathematically formalize how such experimental design choices impact judge responsiveness via novel models of judge decision making and prove that this leads to violations of SUTVA. We furthermore discuss how these design choices and their impact on judge responsiveness can lead to a mis-estimation of the average treatment effect in experimental settings. 
    \item Using existing experiment data of an algorithmic intervention \citep{imai2020experimental}, we simulate a scenario of modifying experimental design choices and observing differences in judge decisions and estimated average treatment effect. 
\end{enumerate}

\section{Related Work}


\paragraph{Experimental Evaluations}
The involved case study in this paper is an experimental randomized control trial study of the use of Public Safety Assessment (PSA) risk scores by judges in Dane county, Wisconsin~\cite{imai2020experimental}, and is actually one of the first known ~\emph{experimental} evaluations of an algorithmic intervention in the criminal justice context.

It is noteworthy that another area where the experimental evaluation of algorithmic systems has gained much traction recently has been in healthcare. Several standards have already been developed for reporting expectations on clinical trials for AI/ML-based healthcare interventions such as CONSORT-AI and SPIRIT-AI ~\cite{liu2020reporting}. However, a recent survey found that of the 41 RCTs of machine learning interventions done so far, no trials adhered to all CONSORT-AI reporting standards~\cite{plana2022randomized}. Furthermore, the RCTs included in that survey demonstrate many of the same design flaws and challenges we observe in ~\cite{imai2020experimental}.

Randomized control trials for risk assessments have been explored in other contexts. For example, \citet{wilder2021clinical} run a ``clinical trial'' on an algorithm used to select peer leaders in a public health program from homelessness youth. Similarly, several studies in education execute RCTs to assess the effectiveness of interventions where a model is used to predict which students (in real-world and MOOC settings) are most likely to drop out~\citep{dawson2017from, borrella2019predict}.  




\paragraph{Quasi-experimental investigations} In quasi-experimental investigations, most work that have considered judge responsiveness assume that the tendency to follow or not follow the algorithmic recommendation is inherent to each judge---an inherent attribute of the judges' internal state or their access to privileged decision and factors outside of the experimenter's control \citep{mullainathan2022diagnosing,albright2019if}. \citet{hoffman2018discretion} consider possible access to privileged information that impacted the hiring manager's judgement (and considers each hiring manager to have a particular \emph{exception rate} for making a different decision from the algorithm). \citet{angwin2016machine} discussed the cost $\eta$ of deviation, since the judge has to actively log deviations from the algorithmic recommendation. 


\paragraph{Spillover effects}
It is well-known, in the experimental economics literature \citep[][]{banerjee2009experimental,deaton2018understanding} for example, that insights from randomized controlled trials (RCT) often fail to scale up when the intervention is applied broadly across the entire target population. This failure is often attributed to \emph{interaction} or \emph{interference} between units, and this phenomenon has also been referred to, in various contexts, as \emph{general equilibrium effects} or \emph{spillover effects}. In the potential outcomes model for causal inference, this is referred to as a violation of the Stable Unit Treatment Value Assumption (SUTVA) \citep{rubin1980discussion,rubin1990formal}. Recognizing the presence of interference, experimenters typically employ a two-level randomized experiment design \citep{duflo2003role,crepon2013labor} in order to estimate average treatment effects under different treatment proportions (e.g. proportion of job seekers receiving an intervention); treatment proportions are randomly assigned to the treatment clusters, e.g., the labor market. Such experiment designs have not yet been applied to study human-algorithm decision making, where each treatment cluster is associated with a particular human decision-maker. We discuss further related work on lab studies and on spillover effects in Appendix~\ref{app:further_rel}.
%

%In these studies, again, we see a similar pattern in experiment design to our main case study. We thus focus on our main case study for this initial exploration and hope to explore these other experimental domains in future work.


%% spillover effects


\section{Existing experimental paradigm and assumptions}

We are interested in the problem of experimentally evaluating how algorithmic interventions in the form of predictive decision aids influence human decision-making. In this section, we describe the existing paradigm for experimental evaluations---we call this the \emph{case-independent} model---and the common assumptions required for causal identification. 

 In many existing experimental evaluations of algorithmic decision aids, the \emph{treatment unit} is presumed to be the decision subject and the \emph{treatment}, $Z_i$ is conceptualized as the provision of an predictive risk score to a decision maker who is responsible for making the final decision $D_i$. In the case of pre-trial detention decisions, the decision subject is the defendant in a particular court case, the decision maker is the judge presiding on the case, and the treatment $Z_i$ is a binary variable indicating if the PSA is shown to the judge for case $i$. $D_i$ represents the judge's binary detention decision, where $D_i = 1$ means detaining and $D_i = 0$ means releasing the arrestee before trial. $Y_i$ is the binary outcome variable, with $Y_i = 1$ indicating the arrestee committed an NCA, and $Y_i = 0$ indicating they did not. $X_i$ is a vector of observed pre-treatment covariates for case $i$, including age, gender, race, and prior criminal history. Independence is assumed between each case. We illustrate these variables and their dependencies (or lack thereof) in Figure~\ref{fig:indep_causal_model}.

 
 \begin{figure}[htbp]
  \centering
 \includegraphics[trim=1.5cm 2.5cm 3cm 10.5cm, clip, width=0.6\columnwidth]{alg_int_neurips24_bw_imai_simple.pdf}
  \caption{The case-independent model of human decision making with a predictive decision aid. This is the causal model assumed in prior work \citep[e.g.,][]{imai2020experimental}}
  \label{fig:indep_causal_model}
\end{figure}

The crucial assumption here is that of non-interference between treatment units (see Assumption~\ref{assmp:sutva}, SUTVA). In words, the treatment status of one unit should not affect the decision for another unit. As a result, $Z_i$ is typically randomized at a single-level, i.e., on individual cases; and there is no randomization at the level of decision makers. %This means that, for a random selection of cases, the algorithmic recommendation or result is shown to the corresponding decision-maker. 
In \citet{imai2020experimental} for example, where only one judge participated in the study, the algorithmic score is shown only for cases with an even case number (i.e., $Z_i =1$); for odd-numbered cases, no algorithmic recommendation is shown (i.e., $Z_i =0$). 

The assumption of non-interference across units is questionable, for multiple reasons. For one, in the psychology and behavioural economics literature, it is well-known that decision-makers have a bias towards consistency to their internal policies for decision-making and generally do not change on a case by case basis~\cite{tversky1985framing}. This means, in ~\cite{imai2020experimental} for example, where the judge only sees the risk score for a fraction of the cases, they might choose to ignore the PSA completely on cases where they do see it, or to infer patterns for cases where they don’t see the score, in order to be as consistent as possible in their decision-making (in this work, we mainly model effects such as the former). %Such case-based interventions thus allow for only a \emph{\textbf{partial treatment for the decision-makers under case randomization}}.
As such, experiment designs that assume the case-independent model does not allow us to draw clear conclusions on how a judge would modify their default decision-making in response to the algorithmic intervention implemented as a full-scale \emph{policy}, since we only observe the judge's decision-making under a \emph{partial} implementation of the would-be algorithmic intervention.

%In the next section, we describe our causal model of judge decision making and use it to demonstrate how design decisions about the experiment can impact judge responsiveness in ways that impact treatment effect estimates. 


\iffalse
\begin{figure}[H]
    %\label{imai_diagram}
	\centering	
	\begin{tikzpicture}[->,shorten >=1pt,auto,node distance=2cm,
	main node/.style={circle,draw,font=\Large}]
	
	\node[main node] (T1) {$X_{i}$};
	\node[main node] (X1) [above right =7mm of T1]{$D_{i}$};
	%\node[main node] (X1pre) [above left =1cm of X1]{$J_k$};
    %\node[main node] (Z1) [left =1cm of X1pre]{$Z_k$};
   \node[main node] (Z2) [above left = 7mm of T1]{$Z_{i}$};
	\node[main node] (Y1) [right=1.5 cm of T1] {$Y_{i}$};
  \node[main node] (P1) [left =7mm of T1]{$\hat{Y}_{i}$};
	
	%\plate [inner sep=.25cm,yshift=.2cm] {plate1} {(X1pre)(X1)(Y1)(T1)(Z1)} {$i$}; 
	
	\path[every node/.style={font=\small}]
	(X1) 	edge node [right] {} (Y1)
  %  (Z1) edge node [right] {} (X1pre)
	%(X1pre) edge node [right] {} (X1)
    (Z2) edge node [right] {} (X1)
    %(Z2) edge node [right] {} (X1pre)
    (T1) edge node [right] {} (P1)
    (P1)edge node [right] {} (X1)
	(T1) edge node [right] {} (Y1)
	edge node [right] {} (X1);
	\end{tikzpicture}
		\caption{Proposed causal model in case study ~\cite{imai2020experimental}}
\end{figure}
\fi

        

 




\section{Model}
In this section, we propose a novel causal model of algorithm-aided human decision making that is distinct from case-independent model, as depicted in Figure~\ref{fig:indep_causal_model}. Our model, depicted as a causal directed acyclic graph (Figure~\ref{fig:expanded_causal_model}), hypothesizes and accounts for dependence in the decisions across cases induced by the human decision maker's cognitive bias. We consider three types of cognitive bias that are particularly relevant to the current setting of human decision making with a predictive decision aid---all of which are directly influenced by experimental design choices (Table~\ref{table:cognitive_bias_models}). We do so by introducing two latent variables $J_{i,k}$ and $\epsilon_{i,k}$ that track the internal state of the decision maker $k$ and how it affects the decision for subject $i$.

\begin{figure}[htbp]
  \centering
  \includegraphics[trim=1.5cm 2cm 3cm 7cm, clip, width=0.6\columnwidth]{alg_int_neurips24_bw.pdf}
  \caption{Proposed causal model that accounts for human decision maker bias. We describe three versions of this model (see Table~\ref{table:cognitive_bias_models}). Under the \emph{treatment exposure} model, arrow (i) is activated, but not (ii) and (iii). Under the \emph{capacity constraint} model, arrows (i) and (ii) are activated, but not (iii). Under the \emph{low trust} model, all three arrows, (i-iii), are activated.}
  \label{fig:expanded_causal_model}
\end{figure}

As in the case-independent model, we assume that $X_i$'s are independent and identically distributed.
In contrast to the case-independent model that does not include the role of the human decision maker (i.e., the judge), we explicitly model the judge---we index each case decision $D_{i,k}$ by both the decision maker (index $k$ in Figure~\ref{fig:expanded_causal_model}) and the decision subject (index $i$ or $\ell$ in Figure~\ref{fig:expanded_causal_model}), as opposed to only indexing it by the decision subject. Similarly, the treatment assignment variable $Z_{i,k}$ denotes the treatment status of both the decision subject $i$ and the decision maker $k$. This provides a fuller account of the space of possible treatment assignment counterfactuals, e.g., a case could have been assigned to a different judge leading to a different decision and outcome, all else held constant. 
We further assume that the judge is sequentially exposed to cases, where the judges consider case $i$ after the case $i-1$, and one case at a time.

To be precise, we define
\begin{equation*}
    Z_{i,k}
 = 
\begin{cases}
1 \text{ if unit } i \text{ is assigned to Judge }k \text{ and unit } i \text{ received algorithmic treatment}\\
0 \text{ if unit } i \text{ is assigned to Judge }k \text{ and unit } i \text{ received no treatment} \\
-1 \text{ if unit } i \text{ is not assigned to Judge }k 
  \end{cases}
\end{equation*}  

As the decision outcome $Y_{i,k}$ is downstream of $D_{i,k}$, we also index it by both the judge and the decision subject.

\subsection{Judge's decision}\label{sec:judge_dec}


We model the judge's decision for each treatment unit in the experiment as a random event, whose probability is determined by the individual judge's decision parameters as well as aspects of the experimental design (e.g. the treatment assignments).

We assume that for each judge $k$, there exists a default decision function (in absence of any algorithmic decision aid) $\lambda_k$ that takes in observable characteristics $X_i \in \calX$. When no predictive decision aid is provided to the unit $i$ assigned to judge $k$ (i.e., $Z_{i,k} = 0$), the judge follows her default decision process and her decision $D_{i,k}$ takes value $\lambda_k(X_i)$.

Suppose, instead, that the unit $i$ is treated (i.e., $Z_{i,k} = 1$). In this case, the judge is provided with the algorithmic score $\hat{Y}_i$. We denote as $\bar{D}(y)$ the recommended decision function mapping a prediction $y$ to a decision (this models, for example, existing judge decision guidelines for PSA scores).

In the case that $\recdec{i}$ differs from the judge's default decision $\lambda_k(X_i)$, the judge may choose to `comply' with the $\recdec{i}$, or to disagree with the ADS recommendation and choose $\lambda_k(X_i)$. We model this as a random event as follows. Let $J_{i,k}$ denote the \emph{responsiveness} of Judge $k$ for treatment unit~$i$, i.e. the probability that the Judge follows the PSA recommendation for case $i$. In other words, $J_{i,k}$ models the automation bias of the judge at the point they are deciding on unit $i$. Let $\epsilon_{i,k} \sim Ber(J_{i,k})$ be the random variable denoting judge response, drawn independently for each $i$; when $\epsilon_{i,k} = 1$, the judge  `complies' with the ADS recommendation. 

To summarize, we have the following decision for judge $k$ on unit $i$:
\begin{equation*}
    D_{i,k} = \begin{cases}
        \lambda_k(X_i) \text{ if } Z_{i,k} = 0 \text{ (no PSA) } \\
        \begin{cases}
     \recdec{i} &\text{if } \recdec{i} = \lambda_k(X_i)  \text{ or } \epsilon_{i,k}=1 ~\text{(Judge complies)}\\
        \lambda_k(X_i) &\text{if }\recdec{i} \ne \lambda_k(X_i) \text{ and } \epsilon_{i,k}=0 ~\text{(Judge does not comply)}
      %  \recdec{i} & \text{o.w.} %&\text{ w.p. $J_k$}
        \end{cases} \text{o.w.}
    \end{cases}.
\end{equation*}
Equivalently:
\begin{equation}\label{eq:model_of_judge}
    D_{i,k} = \left( \bar{D}(\hat{Y}_i)\cdot \epsilon_{i,k} + \lambda_k(X_i) \cdot (1-\epsilon_{i,k})\right) \cdot Z_{i,k} + \lambda_k(X_i) \cdot (1- Z_{i,k})
\end{equation}

%Note : think of alternative to $\epsilon$

% \lnote{What happens when $\recdec{i} = \lambda_k(X_i)$?
% Alternative:
% \begin{equation*}
%     D_{i,k} = \begin{cases}
%         \lambda_k(X_i) \text{ if } Z_{i,k} = 0 \text{ (no PSA) } \\
%         \begin{cases}
%         \lambda_k(X_i) &\text{if } \epsilon_{i,k}=0 ~\text{(Judge does not comply) and } \lambda_k(X_i) \neq \recdec{i}\\
%         \recdec{i} & \text{o.w.} %&\text{ w.p. $J_k$}
%         \end{cases} \text{o.w.}
%     \end{cases}
% \end{equation*}
Note that in this case $J_{i,k}$ denotes the probability that a judge decides to follow $\recdec{i}$, in the case that $\lambda_k(X_i) \neq \recdec{i}$. Hence the judge responsiveness $J_{i,k}$ is a lower bound for actual frequency that the judge is makes the same decision as the algorithmic recommendation.\footnote{
Consider $\eta_k:=\Pr\{\lambda_k(X_i) = \recdec{i}\}$ the natural agreement rate between the judge's default and the ADS decision. 
The following expression is the probability that the judge makes a different decision from ADS:
\begin{align*}
    \Pr(D_{i,k} \neq \recdec{i} \mid Z_{i,k} = 1) = \Pr(\{\lambda_k(X_i) \neq \recdec{i}\} \cap \{ \epsilon_{i,k}=0 \})&= \Pr\{\lambda_k(X_i) \neq \recdec{i}\}\cdot \Pr  \{ \epsilon_{i,k}=0 \}\\
    &\leq (1-\eta_k)\cdot (1- J_{i,k})
\end{align*}
}


Drawing upon cognitive science and existing studies, we propose that the following three \emph{responsiveness factors} are likely to impact judge response: 1) Treatment exposure, 2) capacity constraint, and 3) low trust. We model each of these factors independently for ease of exposition but they may impact the judge's response at the same time. These factors are summarized in Table~\ref{table:cognitive_bias_models}.

\begin{table}[h!]
\centering
\begin{tabular}{| m{2cm} | m{6cm} | m{4cm} |}
  \hline
  \textbf{Model} & \textbf{Effect on Total Responsiveness $J_{i,k}$} & \textbf{Experiment design choices} \\
  \hline
  Treatment exposure & $J_{i,k}$ increases as average exposure to predictive decision aid increases & $Z_{i,k}$ \\
  \hline
  Capacity constraint & $J_{i,k}$ decreases as average exposure to `high risk' prediction rate increases. & $Z_{i,k}, \Pr(\hat{Y}_{i} > 0) $ \\
  \hline
  Low trust & $J_{i,k}$ decreases as average exposure to prediction error increases. & $Z_{i,k},  \Pr(\hat{Y_i} = Y_i)$ \\
  \hline
\end{tabular}
\caption{Description of Three Models of Cognitive Bias in Judge Responsiveness}
\label{table:cognitive_bias_models}
\end{table}

\paragraph{Treatment exposure model.} The judge becomes responsive to the algorithmic recommendation if she encounters the algorithmic recommendation for more units (i.e., greater fraction of units are treated).
\begin{equation}\label{eq:treatment_exposure}
    J_{i+1,k} = %b_k + P(Z_{i,k}) \approx 
b_k + f\left(\frac{1}{i} \sum_{m=1}^{i} Z_{m,k}\right)
\end{equation}
$b_k$ is the baseline responsiveness and $f\left(\frac{1}{n} \sum_{m=1}^{i} Z_{m,k}\right)$ is the adjustment to the responsiveness based on number of treated units seen so far. %\lnote{also: $b_k$ is default automation bias. $J_{i,k}$ is total automation bias.}

Judges may have a bias towards the consistency of their decision making process, making them more predisposed to following the algorithmic recommendation if they have a higher exposure to it.

To illustrate, a particular instantiation of this model is based on a simple thresholding of the average number of treated units:
\begin{equation}
J_{i,k}=
    \begin{cases}
        1 & \text{if } \sum_{m=1}^{i} Z_{m,k} > i\tau\\
        0 & \text{o.w.}
    \end{cases},
\end{equation}
where $\tau \in (0,1)$. This instance of the judge's decision making model suggests that the judge becomes always responsive to the algorithmic recommendation as long as the cumulative average treatment frequency is above $\tau$. We include simple instances of the following two models in Appendix~\ref{app:instance}.

\paragraph{Capacity constraint model} The judge has a limited capacity to respond to positive (or `high risk') predictions. They reduce their responsiveness as the rate of positive predictions they see from the algorithm increases. 
\begin{equation}\label{eq:capacity}
    J_{i+1,k} = b_k - f\left(\frac{1}{i}\sum_{m=1}^iZ_{m,k} \cdot \Ind{\hat{Y}_m > 0}\right)
\end{equation}


The model is not addressing the accuracy of the predictions in particular, but the capacity of the human decision maker to respond to the high number of `high risk' predictions; it leads to them becoming more likely to dismiss the predictions. Specifically, it doesn't make any assumption about whether the human decision maker is observing realized outcomes~$Y_{i,k}$. Hence this is a different model of cognitive bias than the following model, which addresses low accuracy directly.

\paragraph{Low trust model} The judge observes realized outcomes and knows when the algorithm has made a mistake (i.e., $\hat{Y} \ne Y$). They reduce their responsiveness as the error rate of the predictions they see from the algorithm increases. 
\begin{equation}\label{eq:low_trust}
    J_{i+1,k} = b_k - f\left(\frac{1}{i}\sum_{m=1}^iZ_{m,k} \cdot \Ind{\hat{Y}_m \neq Y_{m,k}}\right)
\end{equation}
This is related to the phenomenon of notification fatigue for ADS with high false positive rates \citep{wong2021external}. Note that this model cannot be applied to settings where the judge does not observe the realized outcome before the next decision, or if they only observe outcomes for a particular decision type, e.g., $Y_{i,k}(D_{i,k} = 1)$.

\iffalse
Additional assumptions:
\begin{itemize}
    \item $J_k = J_{k,t}$ is likely time varying, and learnt in some online manner. However, if random, it's possible that things will converge at some static behavior after a certain amount of time has passed. 

    \item $\hat{Y}_{i}$ does not depend on $J_k$. 
\end{itemize}
\fi



\section{Results: Characterization of causal experiments under Judge models}

In this section, we describe the implications that our model of human decision making has for causal experiments and the estimation of treatment effects. First, we give simple conditions under which the common assumption for treatment effect estimation, Stable Unit Treatment Value Assumption (SUTVA), is violated. Second, we show that the underestimation of treatment effect that occur due to the chosen randomization in the assignment of cases to judges, even when the treated cases are the same, focusing on the treatment exposure model.

\subsection{Violation of SUTVA}

Our first observation is that the judge's changing responsiveness induces interference between units. We will formally show that this results in the violation of SUTVA, stated as follows. 
\begin{assumption}[SUTVA \citep{rubin1990formal,angrist1996identification}]\label{assmp:sutva}
A set of treatment assignments, decisions and outcomes $(\vect{Z}, \vect{Y}, \vect{D})$ is said to satisfy SUTVA if both the following conditions hold.
\begin{enumerate}
    \item[a.] If $Z_{i,k} = Z_{i,k}'$, then $D_{i,k}(\vect{Z}) = D_{i,k}(\vect{Z}')$.
    \item[b.] If $Z_{i,k} = Z_{i,k}'$ and $D_{i,k} = D_{i,k}'$, then $Y_{i,k}(\vect{Z}, \vect{D}) = Y_{i,k}(\vect{Z}',\vect{D}')$.
\end{enumerate}
\end{assumption}

In the following result, we show that under simple conditions on $J_{i,k}$, Assumption~\ref{assmp:sutva} (part a) is violated. The proof hinges on the lack of conditional independence between $D_{i,k}$ and prior treatment assignments $(Z_{1,k}, \cdots, Z_{i-1,k})$, given current treatment $Z_{i,k}$. This may be intuitively apparent from the causal DAG in Figure~\ref{fig:expanded_causal_model}, where one can see that $D_{i,k}$ is not d-separated from $(Z_{1,k}, \cdots, Z_{i-1,k})$ by $Z_{i,k}$. We include the proof in Appendix~\ref{app:proofs}.

\begin{theorem}[Violation of SUTVA]\label{thm:sutva_violation}
    Fix $k>0$ and consider some $i > 1$. Assume the judge's decision model is as described in equation~\eqref{eq:model_of_judge}, where $J_{i,k}$ is a monotonically non-decreasing (or non-increasing) function of $Z_{1,k}, \cdots, Z_{i-1,k}$, and strictly increasing (resp. decreasing) in at least one of its arguments. Assume that the judge's default decision function $\lambda_k$ is such that $\Pr(\lambda_k(X_i) \neq \recdec{i}) > 0$.
    
    Then $D_{i,k}$ is not conditionally independent of $(Z_{1,k}, \cdots, Z_{i-1,k})$, given $Z_{i,k}$. In particular, there exists treatment assignments $\vect{Z}, \vect{Z}'$ such that $Z_{i,k} = Z_{i,k}'$ and 
    \begin{equation*}
        \Pr\left(D_{i,k}(\vect{Z}) = D_{i,k}(\vect{Z}')\right) < 1.
    \end{equation*}
\end{theorem}


Theorem~\ref{thm:sutva_violation} suggests that for the specific models of $J_{i,k}$ that we introduced in Section~\ref{sec:judge_dec}, the SUTVA is indeed violated. We state this formally in the following corollary.

\begin{corollary}
Suppose $J_{i,k}$ is as defined in \eqref{eq:treatment_exposure}, \eqref{eq:capacity}, or \eqref{eq:low_trust}.
For any $\lambda_k$, $X_{i}$, $Y_{i}$, there exists $\hat{Y}(\cdot)$, $f$ and $b_k$ such that Assumption~\ref{assmp:sutva} (part a) does not hold with some positive probability. 
% Consider the following three models of $J_{i,k}$.

%     Models 1, 2, 3 all satisfy the condition for SUTVA violation. (In all models,  $J_{i,k}$ is a monotonically non-decreasing function of $Z_{1,k}, \cdots, Z_{i-1,k}$, and strictly increasing in at least one of its arguments).
\end{corollary}
\begin{proof}
   Consider linear $f$ (i.e. $f(x) = ax$), $b_k = 0$ and $\hat{Y}$ such that $\hat{Y}_\ell > 0$ and $\hat{Y}_\ell \ne Y_{\ell,k}$ for some $\ell < i$. Then $J_{i,k}$ as defined in \eqref{eq:treatment_exposure}, \eqref{eq:capacity}, or \eqref{eq:low_trust} satisfies the assumptions of Theorem~\ref{thm:sutva_violation}.
\end{proof}

We note that under our causal model, the second part of the SUTVA (Assumption~\ref{assmp:sutva}, part b) is actually not violated. This is because $Y_{i,k}$ is conditionally independent of $(Z_{1,k}, \cdots, Z_{i-1,k})$ and  $(D_{1,k}, \cdots, D_{i-1,k})$ given $D_{i,k}$, as seen from Figure~\ref{fig:expanded_causal_model}.

\citet{imai2020experimental} argued that there was no statistically significant spillover effect in their experiment, hence supporting the assumption of SUTVA. However, the question of whether a conditional randmization test such as a permutation test can detect the violation SUTVA is intricate. Permutation tests typically reorder data to assess the null hypothesis, but in this case, the order does not affect the judge's average exposure to treatment, $\E[Z_{1,k}]$. Indeed, $J_{i,k}$ is a function of an average over i.i.d. variables as defined in \eqref{eq:treatment_exposure}, \eqref{eq:capacity}, or \eqref{eq:low_trust}, and it becomes effectively constant for all cases after a certain case count $i > T$, given the law of large numbers. This convergence in $J_{i,k}$ implies that reordering would not alter the joint distribution of $D_{i,k}$'s for $i$ large enough. The conditional randomness test used by \citet{imai2020experimental}, for example, maintains the average treatment proportion, randomizing only the spillover effects that are contingent on case order, and hence cannot effectively detect interference between units that is induced by $J_{i,k}$.

% Comment: Can we detect this SUTVA violation via a permutation test? 

% \begin{enumerate}
%     \item Can we detect this SUTVA violation via the permutation test?
%     \begin{itemize}
%         \item Changing the order of examples doesn't matter in this case, because it doesn't change the degree of exposure that defines $J_k$.
%         \item After a certain time $t > T$, $J_k$ is actually the same for every case $i$.
%         \item The determination of spillover effects does not mean that each case is independent of each other. 
%         \item The conditional random test does not modify the treatment proportion because the test keeps the same average treatment proportion, and only randomizes the spillover effect of changing the order of cases. This is in general true for the kind of spillover effect that depends on the proportion of treated units. We care about cumulative (all the cases prior) vs one-bit information (just the case before) on the impact on current case, and so actually not just on the previous case. 
%         \item If we want to something about the power of the test, look at distribution of p-values; ie. you  neighbor effecting you and so changing neighbors vs collective impact of unit treatment to that point. 
%         \item look at the test that Imai uses, where they have the same proportion of treated across null and alternative hypothesis. Cannot change the proportion of treated units. 
%     \end{itemize}
% \end{enumerate}


\subsection{Estimation of causal effects}

Having shown that SUTVA is violated under a model of judge bias, we now investigation the implications on the estimation of causal effects. Our goal in this section is to illustrate the underestimation of the treatment effect of predictive decision aids on the human decision that may occur, if interference due to judge bias is not taken into account in the experiment's randomization scheme. We discuss a worked example focusing on the \emph{treatment exposure} model, leaving detailed investigations of the \emph{capacity constraint} and \emph{low trust} models to future work.

Consider two different treatment assignments (to $n$ total units) that have the same treated and untreated units, but assign these treated units to different decision makers (and we assume each case is only seen by one judge in each treatment assignment). For simplicity, we consider two judges, $k=1,2$, who are assumed to be identical, apart from their assigned cases. Specifically, both of them have decision making model as described in equation~\eqref{eq:treatment_exposure}, where $f(z) = az$ is a linear function and $b_1 = b_2 = b$.

\begin{enumerate}
    \item (Uniform randomization) $\vect{Z} =(\vect{Z}_{\cdot, 1},\vect{Z}_{\cdot, 2})$ such that Judge 1 receives 50\% of total cases, 50\% of them are treated and 50\% are untreated. Judge 2 receives other 50\% of total cases, 50\% of them are treated and 50\% are untreated.
    \item (Two-level randomization) $\vect{Z}' = (\vect{Z'}_{\cdot, 1},\vect{Z'}_{\cdot, 2})$ such that Judge 1 receives 50\% of total cases, 100\% of them are treated. Judge 2 receives other 50\% of total cases, 100\% are untreated.
\end{enumerate}
The above two treatment assignments each represent a different randomization scheme: 1) single-level randomization by decision subjects (cases) only, or 2) two-level randomization by decision makers and decision subjects. This is illustrated in Figure~\ref{fig:comparison_randomization_level}.


\begin{figure}[ht]
	\centering	
 \begin{subfigure}{0.5\textwidth}
  \centering
  \captionsetup{justification=centering}
  \includegraphics[width=0.8\linewidth]{case_level_intervene.png}
  \caption{Uniform (case level) \\randomization.}
  \label{fig:sub1}
\end{subfigure}%
\begin{subfigure}{0.5\textwidth}
  \centering
  \captionsetup{justification=centering}
  \includegraphics[width=0.8\linewidth]{decision_maker_intervene.png}
  \caption{Two level (Decision-maker level) \\randomization.}
  \label{fig:sub2}
\end{subfigure}
\caption{All observed experimental designs randomize the treatment for the algorithmic intervention at (a) the case level, and not (b) the decision-maker level.}\label{fig:comparison_randomization_level}
\end{figure}

%We want to study the gap in the (conditional) treatment effects under uniform randomization vs. two-level randomization, e.g., by lower bounding the gap.


Recall the definition of the average treatment effect (ATE) of treatment $Z_{i,k}$ on decisions $D_{i,k}$:
\begin{equation}
    ATE := \E[D(Z = 1) - D(Z = 0)]
\end{equation}
In our two hypothetical experiments, suppose we use the following estimator of ATE:
\begin{equation}
    \widehat{ATE} := \frac{2}{n}\sum_{i=1}^n\sum_{k=1}^2D_{i,k}\Ind{ Z_{i,k} = 1} - D_{i,k} \Ind{Z_{i,k} = 0}
\end{equation}
Given that units are randomly assigned to treatment (with half of units being treated), $\widehat{ATE}$ appears to be an unbiased estimate of $ATE$. Yet, under our model of $D_{i,k}$ (e.g., \eqref{eq:treatment_exposure}), the $ATE$ is clearly under-specified, as $D_{i,k}$ in general depends not only on $Z_{i,k}$ but also $\{Z_{1, k}, \cdots, Z_{i-1, k}\}$. If the goal is to estimate the treatment effect of providing a predictive decision aid to a decision maker \emph{consistently}, we would really like, perhaps, to estimate the average treatment effect under total treatment.
\begin{equation}
    \overline{ATE} := \E[D(\vect{Z} = \vect{1}) - D(\vect{Z} = \vect{0})].
\end{equation} In that case, the standard estimator $\widehat{ATE}$ would suffer from large bias if used to estimate $\overline{ATE}$, resulting in underestimation of the treatment effect.

The following proposition shows that there can be a significant gap between the expectation of the $\widehat{ATE}$ estimator under different treatment assignments. Specifically, we compare $\widehat{ATE}_{\text{uniform}}:=\widehat{ATE}(\vect{Z})$ and that of $\widehat{ATE}_{\text{two-level}}:=\widehat{ATE}(\vect{Z}')$. We include the proof in Appendix~\ref{app:proofs}.

\lnote{add proposition for capacity constraint and low trust. Assumption is the rate of positive predictions $\E[\hat{Y}_m > 0] \geq r$ or rate of error rate $\E[\hat{Y}_m \ne Y_{m,k}] \geq r$ respectively.}

\begin{proposition}[Estimated treatment effects under treatment exposure model]\label{prop:treatment_effect}
Assume both judge 1 and 2's decision model is as described in equation~\eqref{eq:model_of_judge}, and $J_{i,k}$ follows \eqref{eq:treatment_exposure}, with $b_{k} = b \in(0,1) \forall k$ and $f(x) = ax, a \in (0,1-b)$. Suppose we have $\E[\bar{D}(\hat{Y}_1) - \lambda_k(X_1)] = \rho > 0$. Then
\begin{equation}
  \E[\widehat{ATE}_{\text{two-level}}]- \E[\widehat{ATE}_{\text{uniform}}] = \frac{a \cdot \rho }{2}
\end{equation}
\end{proposition}
Intuitively, the gap between the two estimates increases as 1) the judge responsiveness is more sensitive to past exposure ($a$ is larger) and 2) the expected difference between the judge's default decision and the algorithmically recommended decision increases ($\rho$ is larger).




In this section we illustrated the implications that different randomization schemes have on treatment effect estimation under our model of judge decision making bias. Although we have focused our investigation on estimation of the population treatment effect, our observation that treatment randomization schemes can introduce bias into the estimated treatment effects likely also extends to other types of treatment effects such as those based on principal stratification \citep[see e.g., Average Principal Causal Effects][]{imai2020experimental}. %In fact, estimands like the APCE are arguably even more relevant in the context of human decision making as they incorporate a notion of the `correctness' of the treatment effect on decisions, defined using the potential outcomes of decision outcomes, $Y_i(d)$. We empirically investigate how different judge models and treatment assignments affect the APCE estimates in Section~\ref{sec:expt}.

\section{Semi-synthetic Experiments}\label{sec:expt}

We test our findings empirically by setting up semi-synthetic simulations, using the data from the experiment conducted in ~\cite{imai2020experimental}. Further details of our experimental setting and the original experiment setting can be found in Appendix ~\ref{app:exps}, and all results are directly available in the uploaded Supplementary Materials. %In addition to reporting ATE effects, we include 

Using this experiment data, we simulate alternative scenarios of one of the discussed experiment design choices -- treatment assignment. Using our model of $J_{i,k}$, we can derive an alternate set of judge decisions to that in the original experiment. In the original study, all of the treated and untreated cases were assigned to a single judge. We simulate multiple judges, and a corresponding set of alternative decisions to observe the impact of these biases on estimations of the overall average treatment effect estimate on judge decisions. To do this, we change the simulated judge $J_k$ assigned to a particular case, and their subsequent decisions $D_{i,k}$, using our model in Equation ~\ref{eq:treatment_exposure}. We do not manipulate other case details directly for our simulation. We use a binary normalized form of judge decisions (i.e. $D_{i,k} = 0$ for a signature bond release, $D_{i,k} = 1$ for any kind of cash bond). We set $\lambda_k(X_i)$ to be the set of provided original judge decisions, and $\recdec{i}$ to be the decision recommended under official interpretation guidelines for the PSA. We include descriptive statistics for simulated individual judge decisions in the Appendix~\ref{app:exps}. Details on the implementation of data pre-processing semi-synthetic experiments can be found in the Python Notebook provided in the attached Supplementary Materials. The data from  ~\cite{imai2020experimental} is publicly available \href{https://dataverse.harvard.edu/dataverse/harvard?q=%20Replication%20data%20for:%20Experimental%20evaluation%20of%20algorithm-assisted%20human%20decision-making:%20Application%20to%20pretrial%20public%20safety%20assessment}{here}.

 \begin{figure}[htbp]
  \centering
 %\includegraphics[trim=1.5cm 2cm 3cm 10cm, clip, width=0.6\columnwidth]{alg_int_neurips24_bw_imai_simple.pdf}
 \includegraphics[clip, width=0.4\columnwidth]{aihuman_modifications_and_results/ATE_J3_333.png}
 \includegraphics[clip, width=0.4\columnwidth]{aihuman_modifications_and_results/ATE_J3_613.png}
  \caption{We empirically observe changes to the ATE estimate under different treatment assignments. In Figure (a), we see the ATE for a scenario where three judges are assigned $Z=1$ in about 33.3\% of their assigned cases. In Figure (b), $J_1$ is assigned $Z=1$ in about 60\% of their cases, while $J_2$ and $J_3$ are assigned $Z=1$ in about 10\%, and 30\% of their cases respectfully. Under our model, if a judge is likely to ignore the PSA if they see it less than 60\% of the time, then the ATE is higher for case (b), where at least one judge is guaranteed to see the PSA at that frequency.}
  \label{fig:indep_causal_model}
\end{figure}





\section{Conclusion and Future Work}

In this work, we describe how experiment design choices can impact human interactions with ADS systems, and importantly, bias our estimates of the treatment effects of algorithmic interventions. Our contribution is primarily theoretical and conceptual, and it remains to empirically validate the models of human decision making that we have proposed. We also focused on our numerical experiments on the treatment exposure model, and did not directly address capacity constraint and low trust in simulations.

Thus, future work in this area could explore several questions. First, how can we effectively modify the positive prediction rate ($P (\hat{Y} = 1)$) in an experimental evaluation? This could involve altering the prediction threshold, which may result in lowered recall but increased precision. Second, in terms of modifying model correctness ($P (\hat{Y} = Y)$), what does it mean to select a different model with higher observed accuracy over multiple time steps? Our case study setting shows a model accuracy of 53.7\% with respect to recorded downstream outcomes. Understanding the implications of choosing a model with improved (or worse) accuracy is crucial for the generalizability of these findings. 

We hope this work will spark a conversation about choices and challenges in experimental design in the context of algorithmic interventions, with the goal of better measuring and addressing the societal impact of algorithmic decision aids in consequential real-world settings.
\newpage

\bibliographystyle{plainnat} % We choose the "plain" reference style
\bibliography{ref} % Entries are in the refs.bib file



%%%%%%%%%%%%%%%%%%%%%%%%%%%%%%%%%%%%%%%%%%%%%%%%%%%%%%%%%%%%

\appendix

\section{Appendix / supplemental material}

\section{Further Related work}\label{app:further_rel}


\paragraph{Human Interaction Lab Studies}
In contrast to RCTs in the field, there has been work on controlled~\emph{lab} experiments in which experimenters assess the effect of providing algorithmic risk scores in simulated environments to instructed test subjects \citep{green2019disparate}. The test subjects are non-experts and are often role-playing real-world decision-makers on a contrived or real-world inspired task~\citep{lai2023towards}. In the current work, we focus on the setting of a field experiment as it more directly addresses the impact of ADS in deployment.

\paragraph{Spillover effects}
Recently, \citet{RCT_Service_Interventions} studied experiment design where the treatment is delivered by a human service provider who has limited resources. They show that treatment effect sizes are mediated by such capacity constraints, which are \emph{induced} by particular choices in the design of the experiment, such as the number of service providers recruited and the treatment assignment. Our work also views the human decision maker as a mediator of the treatment effect of using algorithmic decision aids in a decision process; however, we are motivated by modelling the human decision maker's cognitive biases induced by the experiment design, rather than their capacity constraints alone. Decision making is also a distinct setting from service provision.


\section{Examples of capacity constraint and low trust model}\label{app:instance}

A particular instantiation of the \emph{capacity constraint} model is based on a simple thresholding of the average number of `high risk' predictions seen:
\begin{equation}
J_{i,k}=
    \begin{cases}
        1 & \text{if } \sum_{m=1}^{i} Z_{m,k}\cdot \Ind{\hat{Y}_m > 0} < i\tau\\
        0 & \text{o.w.}
    \end{cases},
\end{equation}
where $\tau \in (0,1)$. This instance of the judge's decision making model suggests that the judge becomes always responsive to the algorithmic recommendation as long as the cumulative average `high risk' prediction is below~$\tau$. Otherwise, they always trust themselves over the model in moments of disagreement between their default decision and the ADS decision.

A particular instantiation of the \emph{low trust} model is based on a simple thresholding of the average number of predictive `errors' observed:
\begin{equation}
J_{i,k}=
    \begin{cases}
        1 & \text{if } \sum_{m=1}^{i}Z_{m,k} \cdot \Ind{\hat{Y}_m \neq Y_{m,k}} < i\tau\\
        0 & \text{o.w.}
    \end{cases},
\end{equation}
where $\tau \in (0,1)$. This instance of the judge's decision making model suggests that the judge becomes always responsive to the algorithmic recommendation as long as the cumulative average error rate of prediction $\hat{Y}$ is below $\tau$. Otherwise, they always trust themselves over the model in moments of disagreement between their default decision and the ADS decision. 

\section{Proofs}\label{app:proofs}


\begin{proof}[Proof of Theorem~\ref{thm:sutva_violation}]  Consider $\vect{Z}, \vect{Z}'$ such that $Z_{1,k}= Z_{2,k} = \cdots = Z_{i-1,k} = 0$ and  $Z'_{1,k}= Z'_{2,k} = \cdots = Z'_{i-1,k} = 1$. Also let $Z_{i,k} = Z_{i,k}' = 1$. First consider the case that $J_{i,k}$ is monotonically non-decreasing. Then we have
\begin{align*}
    J_{i,k}(\vect{Z}) < J_{i,k}(\vect{Z}'),
\end{align*}
by our assumption that $J_{i,k}(z_{1,k}, \cdots, z_{i-1,k})$ is strictly increasing in at least one of $z_{1,k}, \cdots, z_{i-1,k}$. In other words, we have $\Pr\left(\epsilon_{i,k}(\vect{Z}) = 1) < \Pr(\epsilon_{i,k}(\vect{Z}') = 1\right)$.

Applying the definition of $D_{i,k}$, we then lower bound the probability that $D_{i,k}(\vect{Z})$ differs from $D_{i,k}(\vect{Z}')$ as follows.
\begin{align*}
    \Pr\left(D_{i,k}(\vect{Z}) \ne D_{i,k}(\vect{Z}')\right) &\ge \Pr\{\lambda_k(X_i) \neq \recdec{i}\}\cdot \Pr\left(\epsilon_{i,k}(\vect{Z}) \ne \epsilon_{i,k}(\vect{Z}')\right) \\
    &> 0.
\end{align*}
  The proof proceeds analogously for the case where $J_{i,k}$ is monotonically non-increasing.
\end{proof}

\begin{proof}[Proof of Proposition~\ref{prop:treatment_effect}]
%     \todo{to review}
% Under equation~\eqref{eq:model_of_judge}
% \begin{align*}
%     &D_{i,k}(Z_{i,k} = 1) - D_{i,k}(Z_{i,k} = 0) \\= &\bar{D}(\hat{Y}_i)\cdot \epsilon_{i,k} + \lambda_k(X_i) \cdot (1-\epsilon_{i,k}) - \lambda_k(X_i) \\
%     =& \epsilon_{i,k} \cdot (\bar{D}(\hat{Y}_i) - \lambda_k(X_i))
% \end{align*}



Under uniform randomization, we have 
\begin{align*}
    \E[\widehat{ATE}_{\text{uniform}}] &= \frac{2}{n} \sum_{i=1}^n\sum_{k=1}^2 \E[D_{i,k}\Ind{ Z_{i,k} = 1} - D_{i,k} \Ind{Z_{i,k} = 0}]
    \\
   % &=\frac{2}{n}\sum_{i=1}^n\sum_{k=1}^2 \E[D_{i,k}]\Ind{ Z_{i,k} = 1} - \E[D_{i,k}]  \Ind{ Z_{i,k} = 1} \\
    &= \frac{1}{2n}\sum_{i=1}^n\E[D_{i,1}\mid Z_{i,1} = 1] +\E[D_{i,2}\mid Z_{i,2} = 1]\\
    &\quad\quad\quad -\E[D_{i,1}\mid Z_{i,1} = 0]-\E[D_{i,1}\mid Z_{i,1} = 0] \\
    &= \frac{1}{2n}\left(\sum_{i=1}^n\E[\bar{D}(\hat{Y}_i) - \lambda_k(X_i)]\cdot \E\left[b+ \frac{a}{i} \sum_{m=1}^{i} Z_{m,1} \mid Z_{i,1} = 1\right]  \right)\\
    &\quad +\frac{1}{2n}\left(\sum_{i=1}^n\E[\bar{D}(\hat{Y}_i) - \lambda_k(X_i)]\cdot \E\left[b+ \frac{a}{i} \sum_{m=1}^{i} Z_{m,2} \mid Z_{i,2} = 1\right]  \right)\\
    &= \frac{1}{2}\left(\E[\bar{D}(\hat{Y}_i) - \lambda_k(X_i)]\cdot (b+ \frac{a}{2}) \right)+\frac{n}{4}\left(\E[\bar{D}(\hat{Y}_i) - \lambda_k(X_i)]\cdot (b+ \frac{a}{2})  \right)\\
    &= \left(\E[\bar{D}(\hat{Y}_1) - \lambda_k(X_1)]\cdot (b+ \frac{a}{2}) \right)
%&\frac{2}{n} \sum_{i=1}^n \sum_{k\in\{1, 2\}}\E[D_{i,k} \Ind{Z_{i,k} = 1}]- \E[D_{i,k} \Ind{Z_{i,k} = 0}] \\
  %  &= \frac{2}{n} \sum_{i:Z_{i,1} = 1} \E[D_{i,1}] + \sum_{i:Z_{i,2} = 1} \E[D_{i,2}]- \sum_{i:Z_{i,1} = 0} \E[D_{i,1}]-\sum_{i:Z_{i,2} = 0} \E[D_{i,2}]
\end{align*}
Note that in the third equality we applied the definition of $D_{i,k}$ and $J_{i,k}$, as well as the independence between $(\hat{Y}_i, X_i)$ and $Z_{i,k}$. In the fourth equality, we use the fact that $\E[Z_{m,2}\mid Z_{i,2} = 1] = 1/2$ for this treatment assignment. In the last equality we used the fact that $\hat{Y}_i$'s abnd $X_i's$ are i.i.d.

Under two-level randomization, we have 

\begin{align*}
    \E[\widehat{ATE}_{\text{two-level}}] &=\frac{2}{n} \sum_{i=1}^n\sum_{k=1}^2 \E[D_{i,k}\Ind{ Z_{i,k} = 1} - D_{i,k} \Ind{Z_{i,k} = 0}] \\
    &= \frac{1}{n}\sum_{i=1}^n\E[D_{i,1}\mid Z_{i,1} = 1] -\E[D_{i,1}\mid Z_{i,2} = 0] \\
    &=\frac{1}{n}\sum_{i=1}^n\E[\bar{D}(\hat{Y}_i) - \lambda_k(X_i)]\cdot \E\left[b+ \frac{a}{i} \sum_{m=1}^{i} Z_{m,1} \mid Z_{i,1} = 1\right]  \\
    &= \left(\E[\bar{D}(\hat{Y}_1) - \lambda_k(X_1)]\cdot (b+ a) \right)
    %&= \frac{n}{4}\left(\E[\bar{D}(\hat{Y}_i) - \lambda_k(X_i)]\cdot \E\left[b+ \frac{a}{i} \sum_{m=1}^{i} Z_{m,1} \mid Z_{i,1} = 1\right]  \right)\\
    %&\quad +\frac{n}{4}\left(\E[\bar{D}(\hat{Y}_i) - \lambda_k(X_i)]\cdot \E\left[b+ \frac{a}{i} \sum_{m=1}^{i} Z_{m,2} \mid Z_{i,2} = 1\right]  \right)\\
    %&= \frac{n}{4}\left(\E[\bar{D}(\hat{Y}_i) - \lambda_k(X_i)]\cdot (b+ \frac{a}{2}) \right)+\frac{n}{4}\left(\E[\bar{D}(\hat{Y}_i) - \lambda_k(X_i)]\cdot (b+ \frac{a}{2})  \right)\\
   % &= \frac{n}{2}\left(\E[\bar{D}(\hat{Y}_1) - \lambda_k(X_1)]\cdot (b+ \frac{a}{2}) \right)
%&\frac{2}{n} \sum_{i=1}^n \sum_{k\in\{1, 2\}}\E[D_{i,k} \Ind{Z_{i,k} = 1}]- \E[D_{i,k} \Ind{Z_{i,k} = 0}] \\
  %  &= \frac{2}{n} \sum_{i:Z_{i,1} = 1} \E[D_{i,1}] + \sum_{i:Z_{i,2} = 1} \E[D_{i,2}]- \sum_{i:Z_{i,1} = 0} \E[D_{i,1}]-\sum_{i:Z_{i,2} = 0} \E[D_{i,2}]
\end{align*}

\end{proof}



\section{Experiment Case Study}\label{app:exps}

During a 30-month assignment period, spanning 2017 until 2019, judges in Dane County, Wisconsin were either given or not given a pretrial Public Safety Assessment (PSA) score for a given case. The randomization was done by an independent administrative court member - the PSA score was calculated for each case $i$, and, for every even-numbered case, the PSA was shown to the judge $k$ as part of the case files. Otherwise, the PSA was hidden from the judge. Given these scores, the judge needs to make the decision, $D_{i,k}$ to enforce a signature bond, a small case bail (defined as less than \$1000) or a large case bail (defined as more than \$1000). Information about judge decisions, and defendant outcomes, $Y_{i,k}$ (ie. failure to appear (FTA), new criminal activity (NCA) and new violent criminal activity (NVCA)) are tracked for a period of two years after randomization. 
The case study~\cite{imai2020experimental} only looks at one judge, allowing them to simplify the situation to a single decision-maker scenario.

Using this experiment data, we simulate alternative scenarios of one of the discussed experiment design choices -- treatment assignment. Using our model of $J_{i,k}$, we can derive an alternate set of judge decisions to that in the original experiment. In the original study, all of the treated and untreated cases were assigned to a single judge. We simulate multiple judges, and a corresponding set of alternative decisions to observe the impact of these biases on estimations of the overall average treatment effect estimate on judge decisions. To do this, we change the simulated judge $J_k$ assigned to a particular case, and their subsequent decisions $D_{i,k}$, using our model in Equation ~\ref{eq:treatment_exposure}. We do not manipulate other case details directly for our simulation. We use a binary normalized form of judge decisions (i.e. $D_{i,k} = 0$ for a signature bond release, $D_{i,k} = 1$ for any kind of cash bond). We set $\lambda_k(X_i)$ to be the set of provided original judge decisions, and $\recdec{i}$ to be the decision recommended under official interpretation guidelines for the PSA. Details on the implementation of data pre-processing semi-synthetic experiments can be found in the Python Notebook provided in the attached Supplementary Materials. The data from the ~\cite{imai2020experimental} study is publicly available \href{https://dataverse.harvard.edu/dataverse/harvard?q=%20Replication%20data%20for:%20Experimental%20evaluation%20of%20algorithm-assisted%20human%20decision-making:%20Application%20to%20pretrial%20public%20safety%20assessment}{here}.

We conduct an empirical study similar to that described in Section 5.2. Namely, for each study, we consider two different treatment assignments (to $n$ total units) that have the same treated and untreated units, but assign these treated units to different decision makers (and we assume each case is only seen by one judge in each treatment assignment). We consider $k$ judges, who are assumed to be identical, apart from their assigned cases. Specifically, both of them have decision making model as described in equation~\eqref{eq:treatment_exposure}, where $f(z) = az$ is a linear function and $b_1 = b_2 = b$.

Under these assumptions, we conduct the following set up for Experiment 1:
\begin{enumerate}
    \item (Uniform randomization) $\vect{Z} =(\vect{Z}_{\cdot, 1},\vect{Z}_{\cdot, 2})$ such that Judge 1 receives 50\% of total cases, 50\% of them are treated and 50\% are untreated. Judge 2 receives other 50\% of total cases, 50\% of them are treated and 50\% are untreated.
    \item (Two-level randomization) $\vect{Z}' = (\vect{Z'}_{\cdot, 1},\vect{Z'}_{\cdot, 2})$ such that Judge 1 receives 50\% of total cases, 100\% of them are treated. Judge 2 receives other 50\% of total cases, 100\% are untreated.
\end{enumerate}

Under these assumptions, we conduct the following set up for Experiment 2:
\begin{enumerate}
    \item (Uniform randomization) $\vect{Z} =(\vect{Z}_{\cdot, 1},\vect{Z}_{\cdot, 2},\vect{Z}_{\cdot, 3})$ such that each judge  receives 33.3\% of total cases, 33.3\% of them are treated .
    \item (Two-level randomization) $\vect{Z}' = (\vect{Z'}_{\cdot, 1},\vect{Z'}_{\cdot, 2},\vect{Z}_{\cdot, 3})$ such that Judge 1 receives 33.3\% of total cases, 60\% of them are treated. Judge 2 receives other 33.3\% of total cases, but only 30\% are untreated. Judge 3 receives other 33.3\% of total cases, but only 10\% are untreated.
\end{enumerate}

 \begin{figure}[htbp]
  \centering
 %\includegraphics[trim=1.5cm 2cm 3cm 10cm, clip, width=0.6\columnwidth]{alg_int_neurips24_bw_imai_simple.pdf}
 \includegraphics[clip, width=0.4\columnwidth]{aihuman_modifications_and_results/ATE_J2_0505.png}
 \includegraphics[clip, width=0.4\columnwidth]{aihuman_modifications_and_results/ATE_2J_01.png}
  \caption{\textbf{Results for Experiment 1:} We empirically observe changes to the ATE estimate under different treatment assignments. In Figure (a), we see the ATE for a scenario where two judges are assigned $Z=1$ in about 50\% of their assigned cases. In Figure (b), $J_1$ is assigned $Z=1$ in about 100\% of their cases, while $J_2$ is assigned $Z=1$ in about 50\% of their cases. Under our model, if a judge is likely to ignore the PSA if they see it less than 60\% of the time, then the ATE is higher for case (b), where at least one judge is guaranteed to see the PSA at that frequency.}
  \label{fig:exp1_res}
\end{figure}

 \begin{figure}[htbp]
  \centering
 %\includegraphics[trim=1.5cm 2cm 3cm 10cm, clip, width=0.6\columnwidth]{alg_int_neurips24_bw_imai_simple.pdf}
 \includegraphics[clip, width=0.4\columnwidth]{aihuman_modifications_and_results/ATE_J3_333.png}
 \includegraphics[clip, width=0.4\columnwidth]{aihuman_modifications_and_results/ATE_J3_613.png}
  \caption{\textbf{Results for Experiment 2:}We empirically observe changes to the ATE estimate under different treatment assignments. In Figure (a), we see the ATE for a scenario where three judges are assigned $Z=1$ in about 33.3\% of their assigned cases. In Figure (b), $J_1$ is assigned $Z=1$ in about 60\% of their cases, while $J_2$ and $J_3$ are assigned $Z=1$ in about 10\%, and 30\% of their cases respectfully. Under our model, if a judge is likely to ignore the PSA if they see it less than 60\% of the time, then the ATE is higher for case (b), where at least one judge is guaranteed to see the PSA at that frequency. Note that we could not}
  \label{fig:exp2_res}
\end{figure}

\begin{figure}[htbp]
  \centering
 %\includegraphics[trim=1.5cm 2cm 3cm 10cm, clip, width=0.6\columnwidth]{alg_int_neurips24_bw_imai_simple.pdf}
   \includegraphics[clip, width=0.23\columnwidth]{aihuman_modifications_and_results/FTAT_my_PSAdata_J1_norm_J2_0505.png}
   \includegraphics[clip, width=0.23\columnwidth]{aihuman_modifications_and_results/FTAT_my_PSAdata_J2_norm_J2_0505.png}
 \includegraphics[clip, width=0.23\columnwidth]{aihuman_modifications_and_results/FTAC_my_PSAdata_J1_norm_J2_0505.png}
 \includegraphics[clip, width=0.23\columnwidth]{aihuman_modifications_and_results/FTAC_my_PSAdata_J2_norm_J2_0505.png}
  \caption{\textbf{Results for Experiment 1a:} Changes in decision patterns for different risk levels - from left to right, decisions across ground truth FTA risk thresholds for treated cases with $J_1$ treated at 50\%, $J_2$ treated at 50\%, and $J_1$, $J_2$ for control cases. }
  \label{fig:indep_causal_model}
\end{figure}

 \begin{figure}[htbp]
  \centering
 %\includegraphics[trim=1.5cm 2cm 3cm 10cm, clip, width=0.6\columnwidth]{alg_int_neurips24_bw_imai_simple.pdf}
   \includegraphics[clip, width=0.23\columnwidth]{aihuman_modifications_and_results/FTAT_my_PSAdata_J1_norm_2J_01.png}
   \includegraphics[clip, width=0.23\columnwidth]{aihuman_modifications_and_results/FTAT_my_PSAdata_J1_norm_2J_01.png}
 \includegraphics[clip, width=0.23\columnwidth]{aihuman_modifications_and_results/FTAC_my_PSAdata_J2_norm_2J_01.png}
 \includegraphics[clip, width=0.23\columnwidth]{aihuman_modifications_and_results/FTAC_my_PSAdata_J2_norm_2J_01.png}
  \caption{\textbf{Results for Experiment 1b:} Changes in decision patterns for different risk levels - from left to right, decisions across ground truth FTA risk thresholds for treated cases with $J_1$ treated at 100\%, $J_2$ treated at 0\%, and $J_1$, $J_2$ for control cases.}
  \label{fig:indep_causal_model}
\end{figure}

\begin{figure}[htbp]
  \centering
 %\includegraphics[trim=1.5cm 2cm 3cm 10cm, clip, width=0.6\columnwidth]{alg_int_neurips24_bw_imai_simple.pdf}
   \includegraphics[clip, width=0.3\columnwidth]{aihuman_modifications_and_results/FTAT_my_PSAdata_J1_norm_J3_333.png}
   \includegraphics[clip, width=0.3\columnwidth]{aihuman_modifications_and_results/FTAT_my_PSAdata_J2_norm_J3_333.png}
 \includegraphics[clip, width=0.3\columnwidth]{aihuman_modifications_and_results/FTAT_my_PSAdata_J3_norm_J3_333.png}
 \includegraphics[clip, width=0.3\columnwidth]{aihuman_modifications_and_results/FTAC_my_PSAdata_J1_norm_J3_333.png}
 \includegraphics[clip, width=0.3\columnwidth]{aihuman_modifications_and_results/FTAC_my_PSAdata_J2_norm_J3_333.png}
  \includegraphics[clip, width=0.3\columnwidth]{aihuman_modifications_and_results/FTAC_my_PSAdata_J3_norm_J3_333.png}

  \caption{\textbf{Results for Experiment 2a:} Changes in decision patterns for different risk levels - from left to right, decisions across ground truth FTA risk thresholds for treated cases with $J_1$ treated at 33.3\%, $J_2$ treated at 33.3\%, $J_3$ treated at 33.3\%; and below $J_1$, $J_2$, $J_3$ for control cases.}
  \label{fig:indep_causal_model}
\end{figure}

\begin{figure}[htbp]
  \centering
 %\includegraphics[trim=1.5cm 2cm 3cm 10cm, clip, width=0.6\columnwidth]{alg_int_neurips24_bw_imai_simple.pdf}
   \includegraphics[clip, width=0.3\columnwidth]{aihuman_modifications_and_results/FTAT_my_PSAdata_J1_norm_J3_613.png}
   \includegraphics[clip, width=0.3\columnwidth]{aihuman_modifications_and_results/FTAT_my_PSAdata_J2_norm_J3_613.png}
 \includegraphics[clip, width=0.3\columnwidth]{aihuman_modifications_and_results/FTAT_my_PSAdata_J3_norm_J3_613.png}
 \includegraphics[clip, width=0.3\columnwidth]{aihuman_modifications_and_results/FTAC_my_PSAdata_J1_norm_J3_613.png}
 \includegraphics[clip, width=0.3\columnwidth]{aihuman_modifications_and_results/FTAC_my_PSAdata_J2_norm_J3_613.png}
  \includegraphics[clip, width=0.3\columnwidth]{aihuman_modifications_and_results/FTAC_my_PSAdata_J3_norm_J3_613.png}

  \caption{\textbf{Results for Experiment 2b:} Changes in decision patterns for different risk levels - from left to right, decisions across ground truth FTA risk thresholds for treated cases with $J_1$ treated at 60\%, $J_2$ treated at 10\%, $J_3$ treated at 30\%; and below $J_1$, $J_2$, $J_3$ for control cases.}
  \label{fig:indep_causal_model}
\end{figure}



Optionally include supplemental material (complete proofs, additional experiments and plots) in appendix.
All such materials \textbf{SHOULD be included in the main submission.}

%%%%%%%%%%%%%%%%%%%%%%%%%%%%%%%%%%%%%%%%%%%%%%%%%%%%%%%%%%%%

\newpage
\section*{NeurIPS Paper Checklist}

%%% BEGIN INSTRUCTIONS %%%
The checklist is designed to encourage best practices for responsible machine learning research, addressing issues of reproducibility, transparency, research ethics, and societal impact. Do not remove the checklist: {\bf The papers not including the checklist will be desk rejected.} The checklist should follow the references and follow the (optional) supplemental material.  The checklist does NOT count towards the page
limit. 

Please read the checklist guidelines carefully for information on how to answer these questions. For each question in the checklist:
\begin{itemize}
    \item You should answer \answerYes{}, \answerNo{}, or \answerNA{}.
    \item \answerNA{} means either that the question is Not Applicable for that particular paper or the relevant information is Not Available.
    \item Please provide a short (1–2 sentence) justification right after your answer (even for NA). 
   % \item {\bf The papers not including the checklist will be desk rejected.}
\end{itemize}

{\bf The checklist answers are an integral part of your paper submission.} They are visible to the reviewers, area chairs, senior area chairs, and ethics reviewers. You will be asked to also include it (after eventual revisions) with the final version of your paper, and its final version will be published with the paper.

The reviewers of your paper will be asked to use the checklist as one of the factors in their evaluation. While "\answerYes{}" is generally preferable to "\answerNo{}", it is perfectly acceptable to answer "\answerNo{}" provided a proper justification is given (e.g., "error bars are not reported because it would be too computationally expensive" or "we were unable to find the license for the dataset we used"). In general, answering "\answerNo{}" or "\answerNA{}" is not grounds for rejection. While the questions are phrased in a binary way, we acknowledge that the true answer is often more nuanced, so please just use your best judgment and write a justification to elaborate. All supporting evidence can appear either in the main paper or the supplemental material, provided in appendix. If you answer \answerYes{} to a question, in the justification please point to the section(s) where related material for the question can be found.

IMPORTANT, please:
\begin{itemize}
    \item {\bf Delete this instruction block, but keep the section heading ``NeurIPS paper checklist"},
    \item  {\bf Keep the checklist subsection headings, questions/answers and guidelines below.}
    \item {\bf Do not modify the questions and only use the provided macros for your answers}.
\end{itemize} 
 

%%% END INSTRUCTIONS %%%


\begin{enumerate}

\item {\bf Claims}
    \item[] Question: Do the main claims made in the abstract and introduction accurately reflect the paper's contributions and scope?
    \item[] Answer: \answerYes{} % Replace by \answerYes{}, \answerNo{}, or \answerNA{}.
    \item[] Justification: We indeed formalize and investigate various models of human decision-making (Section 4), and our results in Section 5 as well as simulations in Section 6 show that these behavioral models produce dependencies across units and affects treatment effect estimation.
    \item[] Guidelines:
    \begin{itemize}
        \item The answer NA means that the abstract and introduction do not include the claims made in the paper.
        \item The abstract and/or introduction should clearly state the claims made, including the contributions made in the paper and important assumptions and limitations. A No or NA answer to this question will not be perceived well by the reviewers. 
        \item The claims made should match theoretical and experimental results, and reflect how much the results can be expected to generalize to other settings. 
        \item It is fine to include aspirational goals as motivation as long as it is clear that these goals are not attained by the paper. 
    \end{itemize}

\item {\bf Limitations}
    \item[] Question: Does the paper discuss the limitations of the work performed by the authors?
    \item[] Answer: \answerYes{} % Replace by \answerYes{}, \answerNo{}, or \answerNA{}.
    \item[] Justification: We wrote: `Our contribution is primarily theoretical and conceptual, and it remains to empirically validate the models of human decision making that we have proposed. We also focused on our numerical experiments on the treatment exposure model, and did not directly address capacity constraint and low trust via simulations.'
    \item[] Guidelines:
    \begin{itemize}
        \item The answer NA means that the paper has no limitation while the answer No means that the paper has limitations, but those are not discussed in the paper. 
        \item The authors are encouraged to create a separate "Limitations" section in their paper.
        \item The paper should point out any strong assumptions and how robust the results are to violations of these assumptions (e.g., independence assumptions, noiseless settings, model well-specification, asymptotic approximations only holding locally). The authors should reflect on how these assumptions might be violated in practice and what the implications would be.
        \item The authors should reflect on the scope of the claims made, e.g., if the approach was only tested on a few datasets or with a few runs. In general, empirical results often depend on implicit assumptions, which should be articulated.
        \item The authors should reflect on the factors that influence the performance of the approach. For example, a facial recognition algorithm may perform poorly when image resolution is low or images are taken in low lighting. Or a speech-to-text system might not be used reliably to provide closed captions for online lectures because it fails to handle technical jargon.
        \item The authors should discuss the computational efficiency of the proposed algorithms and how they scale with dataset size.
        \item If applicable, the authors should discuss possible limitations of their approach to address problems of privacy and fairness.
        \item While the authors might fear that complete honesty about limitations might be used by reviewers as grounds for rejection, a worse outcome might be that reviewers discover limitations that aren't acknowledged in the paper. The authors should use their best judgment and recognize that individual actions in favor of transparency play an important role in developing norms that preserve the integrity of the community. Reviewers will be specifically instructed to not penalize honesty concerning limitations.
    \end{itemize}

\item {\bf Theory Assumptions and Proofs}
    \item[] Question: For each theoretical result, does the paper provide the full set of assumptions and a complete (and correct) proof?
    \item[] Answer: \answerYes{} % Replace by \answerYes{}, \answerNo{}, or \answerNA{}.
    \item[] Justification: We included proofs for every theorem, corollary and proposition in either the main text or the appendix.
    \item[] Guidelines:
    \begin{itemize}
        \item The answer NA means that the paper does not include theoretical results. 
        \item All the theorems, formulas, and proofs in the paper should be numbered and cross-referenced.
        \item All assumptions should be clearly stated or referenced in the statement of any theorems.
        \item The proofs can either appear in the main paper or the supplemental material, but if they appear in the supplemental material, the authors are encouraged to provide a short proof sketch to provide intuition. 
        \item Inversely, any informal proof provided in the core of the paper should be complemented by formal proofs provided in appendix or supplemental material.
        \item Theorems and Lemmas that the proof relies upon should be properly referenced. 
    \end{itemize}

    \item {\bf Experimental Result Reproducibility}
    \item[] Question: Does the paper fully disclose all the information needed to reproduce the main experimental results of the paper to the extent that it affects the main claims and/or conclusions of the paper (regardless of whether the code and data are provided or not)?
    \item[] Answer: \answerYes{} % Replace by \answerYes{}, \answerNo{}, or \answerNA{}.
    \item[] Justification: 
    \item[] Guidelines:
    \begin{itemize}
        \item The answer NA means that the paper does not include experiments.
        \item If the paper includes experiments, a No answer to this question will not be perceived well by the reviewers: Making the paper reproducible is important, regardless of whether the code and data are provided or not.
        \item If the contribution is a dataset and/or model, the authors should describe the steps taken to make their results reproducible or verifiable. 
        \item Depending on the contribution, reproducibility can be accomplished in various ways. For example, if the contribution is a novel architecture, describing the architecture fully might suffice, or if the contribution is a specific model and empirical evaluation, it may be necessary to either make it possible for others to replicate the model with the same dataset, or provide access to the model. In general. releasing code and data is often one good way to accomplish this, but reproducibility can also be provided via detailed instructions for how to replicate the results, access to a hosted model (e.g., in the case of a large language model), releasing of a model checkpoint, or other means that are appropriate to the research performed.
        \item While NeurIPS does not require releasing code, the conference does require all submissions to provide some reasonable avenue for reproducibility, which may depend on the nature of the contribution. For example
        \begin{enumerate}
            \item If the contribution is primarily a new algorithm, the paper should make it clear how to reproduce that algorithm.
            \item If the contribution is primarily a new model architecture, the paper should describe the architecture clearly and fully.
            \item If the contribution is a new model (e.g., a large language model), then there should either be a way to access this model for reproducing the results or a way to reproduce the model (e.g., with an open-source dataset or instructions for how to construct the dataset).
            \item We recognize that reproducibility may be tricky in some cases, in which case authors are welcome to describe the particular way they provide for reproducibility. In the case of closed-source models, it may be that access to the model is limited in some way (e.g., to registered users), but it should be possible for other researchers to have some path to reproducing or verifying the results.
        \end{enumerate}
    \end{itemize}


\item {\bf Open access to data and code}
    \item[] Question: Does the paper provide open access to the data and code, with sufficient instructions to faithfully reproduce the main experimental results, as described in supplemental material?
    \item[] Answer: \answerYes{} % Replace by \answerYes{}, \answerNo{}, or \answerNA{}.
    \item[] Justification: We submit all code.
    \item[] Guidelines:
    \begin{itemize}
        \item The answer NA means that paper does not include experiments requiring code.
        \item Please see the NeurIPS code and data submission guidelines (\url{https://nips.cc/public/guides/CodeSubmissionPolicy}) for more details.
        \item While we encourage the release of code and data, we understand that this might not be possible, so “No” is an acceptable answer. Papers cannot be rejected simply for not including code, unless this is central to the contribution (e.g., for a new open-source benchmark).
        \item The instructions should contain the exact command and environment needed to run to reproduce the results. See the NeurIPS code and data submission guidelines (\url{https://nips.cc/public/guides/CodeSubmissionPolicy}) for more details.
        \item The authors should provide instructions on data access and preparation, including how to access the raw data, preprocessed data, intermediate data, and generated data, etc.
        \item The authors should provide scripts to reproduce all experimental results for the new proposed method and baselines. If only a subset of experiments are reproducible, they should state which ones are omitted from the script and why.
        \item At submission time, to preserve anonymity, the authors should release anonymized versions (if applicable).
        \item Providing as much information as possible in supplemental material (appended to the paper) is recommended, but including URLs to data and code is permitted.
    \end{itemize}


\item {\bf Experimental Setting/Details}
    \item[] Question: Does the paper specify all the training and test details (e.g., data splits, hyperparameters, how they were chosen, type of optimizer, etc.) necessary to understand the results?
    \item[] Answer: \answerYes{} % Replace by \answerYes{}, \answerNo{}, or \answerNA{}.
    \item[] Justification: 
    \item[] Guidelines:
    \begin{itemize}
        \item The answer NA means that the paper does not include experiments.
        \item The experimental setting should be presented in the core of the paper to a level of detail that is necessary to appreciate the results and make sense of them.
        \item The full details can be provided either with the code, in appendix, or as supplemental material.
    \end{itemize}

\item {\bf Experiment Statistical Significance}
    \item[] Question: Does the paper report error bars suitably and correctly defined or other appropriate information about the statistical significance of the experiments?
    \item[] Answer:\answerYes{} % Replace by \answerYes{}, \answerNo{}, or \answerNA{}.
    \item[] Justification: Plots include error bars.
    \item[] Guidelines:
    \begin{itemize}
        \item The answer NA means that the paper does not include experiments.
        \item The authors should answer "Yes" if the results are accompanied by error bars, confidence intervals, or statistical significance tests, at least for the experiments that support the main claims of the paper.
        \item The factors of variability that the error bars are capturing should be clearly stated (for example, train/test split, initialization, random drawing of some parameter, or overall run with given experimental conditions).
        \item The method for calculating the error bars should be explained (closed form formula, call to a library function, bootstrap, etc.)
        \item The assumptions made should be given (e.g., Normally distributed errors).
        \item It should be clear whether the error bar is the standard deviation or the standard error of the mean.
        \item It is OK to report 1-sigma error bars, but one should state it. The authors should preferably report a 2-sigma error bar than state that they have a 96\% CI, if the hypothesis of Normality of errors is not verified.
        \item For asymmetric distributions, the authors should be careful not to show in tables or figures symmetric error bars that would yield results that are out of range (e.g. negative error rates).
        \item If error bars are reported in tables or plots, The authors should explain in the text how they were calculated and reference the corresponding figures or tables in the text.
    \end{itemize}

\item {\bf Experiments Compute Resources}
    \item[] Question: For each experiment, does the paper provide sufficient information on the computer resources (type of compute workers, memory, time of execution) needed to reproduce the experiments?
    \item[] Answer: \answerYes{} % Replace by \answerYes{}, \answerNo{}, or \answerNA{}.
    \item[] Justification: Our experiments are performed on a local machine.
    \item[] Guidelines:
    \begin{itemize}
        \item The answer NA means that the paper does not include experiments.
        \item The paper should indicate the type of compute workers CPU or GPU, internal cluster, or cloud provider, including relevant memory and storage.
        \item The paper should provide the amount of compute required for each of the individual experimental runs as well as estimate the total compute. 
        \item The paper should disclose whether the full research project required more compute than the experiments reported in the paper (e.g., preliminary or failed experiments that didn't make it into the paper). 
    \end{itemize}
    
\item {\bf Code Of Ethics}
    \item[] Question: Does the research conducted in the paper conform, in every respect, with the NeurIPS Code of Ethics \url{https://neurips.cc/public/EthicsGuidelines}?
    \item[] Answer:\answerYes{}% Replace by \answerYes{}, \answerNo{}, or \answerNA{}.
    \item[] Justification: We reviewed the NeurIPS Code of Ethics
    \item[] Guidelines: 
    \begin{itemize}
        \item The answer NA means that the authors have not reviewed the NeurIPS Code of Ethics.
        \item If the authors answer No, they should explain the special circumstances that require a deviation from the Code of Ethics.
        \item The authors should make sure to preserve anonymity (e.g., if there is a special consideration due to laws or regulations in their jurisdiction).
    \end{itemize}


\item {\bf Broader Impacts}
    \item[] Question: Does the paper discuss both potential positive societal impacts and negative societal impacts of the work performed?
    \item[] Answer: \answerYes{} % Replace by , \answerNo{}, or \answerNA{}.
    \item[] Justification: We discuss this in the introduction and conclusion.
    \item[] Guidelines:
    \begin{itemize}
        \item The answer NA means that there is no societal impact of the work performed.
        \item If the authors answer NA or No, they should explain why their work has no societal impact or why the paper does not address societal impact.
        \item Examples of negative societal impacts include potential malicious or unintended uses (e.g., disinformation, generating fake profiles, surveillance), fairness considerations (e.g., deployment of technologies that could make decisions that unfairly impact specific groups), privacy considerations, and security considerations.
        \item The conference expects that many papers will be foundational research and not tied to particular applications, let alone deployments. However, if there is a direct path to any negative applications, the authors should point it out. For example, it is legitimate to point out that an improvement in the quality of generative models could be used to generate deepfakes for disinformation. On the other hand, it is not needed to point out that a generic algorithm for optimizing neural networks could enable people to train models that generate Deepfakes faster.
        \item The authors should consider possible harms that could arise when the technology is being used as intended and functioning correctly, harms that could arise when the technology is being used as intended but gives incorrect results, and harms following from (intentional or unintentional) misuse of the technology.
        \item If there are negative societal impacts, the authors could also discuss possible mitigation strategies (e.g., gated release of models, providing defenses in addition to attacks, mechanisms for monitoring misuse, mechanisms to monitor how a system learns from feedback over time, improving the efficiency and accessibility of ML).
    \end{itemize}
    
\item {\bf Safeguards}
    \item[] Question: Does the paper describe safeguards that have been put in place for responsible release of data or models that have a high risk for misuse (e.g., pretrained language models, image generators, or scraped datasets)?
    \item[] Answer: \answerNA{}. % Replace by \answerYes{}, \answerNo{}, or \answerNA{}.
    \item[] Justification: \answerNA{}.
    \item[] Guidelines:
    \begin{itemize}
        \item The answer NA means that the paper poses no such risks.
        \item Released models that have a high risk for misuse or dual-use should be released with necessary safeguards to allow for controlled use of the model, for example by requiring that users adhere to usage guidelines or restrictions to access the model or implementing safety filters. 
        \item Datasets that have been scraped from the Internet could pose safety risks. The authors should describe how they avoided releasing unsafe images.
        \item We recognize that providing effective safeguards is challenging, and many papers do not require this, but we encourage authors to take this into account and make a best faith effort.
    \end{itemize}

\item {\bf Licenses for existing assets}
    \item[] Question: Are the creators or original owners of assets (e.g., code, data, models), used in the paper, properly credited and are the license and terms of use explicitly mentioned and properly respected?
    \item[] Answer: \answerNA{}. % Replace by \answerYes{}, \answerNo{}, or \answerNA{}.
    \item[] Justification: \answerNA{}.
    \item[] Guidelines:
    \begin{itemize}
        \item The answer NA means that the paper does not use existing assets.
        \item The authors should cite the original paper that produced the code package or dataset.
        \item The authors should state which version of the asset is used and, if possible, include a URL.
        \item The name of the license (e.g., CC-BY 4.0) should be included for each asset.
        \item For scraped data from a particular source (e.g., website), the copyright and terms of service of that source should be provided.
        \item If assets are released, the license, copyright information, and terms of use in the package should be provided. For popular datasets, \url{paperswithcode.com/datasets} has curated licenses for some datasets. Their licensing guide can help determine the license of a dataset.
        \item For existing datasets that are re-packaged, both the original license and the license of the derived asset (if it has changed) should be provided.
        \item If this information is not available online, the authors are encouraged to reach out to the asset's creators.
    \end{itemize}

\item {\bf New Assets}
    \item[] Question: Are new assets introduced in the paper well documented and is the documentation provided alongside the assets?
    \item[] Answer:\answerNA{}. % Replace by \answerYes{}, \answerNo{}, or \answerNA{}.
    \item[] Justification: \answerNA{}.
    \item[] Guidelines:
    \begin{itemize}
        \item The answer NA means that the paper does not release new assets.
        \item Researchers should communicate the details of the dataset/code/model as part of their submissions via structured templates. This includes details about training, license, limitations, etc. 
        \item The paper should discuss whether and how consent was obtained from people whose asset is used.
        \item At submission time, remember to anonymize your assets (if applicable). You can either create an anonymized URL or include an anonymized zip file.
    \end{itemize}

\item {\bf Crowdsourcing and Research with Human Subjects}
    \item[] Question: For crowdsourcing experiments and research with human subjects, does the paper include the full text of instructions given to participants and screenshots, if applicable, as well as details about compensation (if any)? 
    \item[] Answer: \answerNA{}. % Replace by \answerYes{}, \answerNo{}, or \answerNA{}.
    \item[] Justification: \answerNA{}.
    \item[] Guidelines:
    \begin{itemize}
        \item The answer NA means that the paper does not involve crowdsourcing nor research with human subjects.
        \item Including this information in the supplemental material is fine, but if the main contribution of the paper involves human subjects, then as much detail as possible should be included in the main paper. 
        \item According to the NeurIPS Code of Ethics, workers involved in data collection, curation, or other labor should be paid at least the minimum wage in the country of the data collector. 
    \end{itemize}

\item {\bf Institutional Review Board (IRB) Approvals or Equivalent for Research with Human Subjects}
    \item[] Question: Does the paper describe potential risks incurred by study participants, whether such risks were disclosed to the subjects, and whether Institutional Review Board (IRB) approvals (or an equivalent approval/review based on the requirements of your country or institution) were obtained?
    \item[] Answer: \answerNA{}. % Replace by \answerYes{}, \answerNo{}, or \answerNA{}.
    \item[] Justification: \answerNA{}.
    \item[] Guidelines:
    \begin{itemize}
        \item The answer NA means that the paper does not involve crowdsourcing nor research with human subjects.
        \item Depending on the country in which research is conducted, IRB approval (or equivalent) may be required for any human subjects research. If you obtained IRB approval, you should clearly state this in the paper. 
        \item We recognize that the procedures for this may vary significantly between institutions and locations, and we expect authors to adhere to the NeurIPS Code of Ethics and the guidelines for their institution. 
        \item For initial submissions, do not include any information that would break anonymity (if applicable), such as the institution conducting the review.
    \end{itemize}

\end{enumerate}


\end{document}