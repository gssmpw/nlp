Large language models (LLMs) often struggle with knowledge intensive NLP tasks,
such as answering ``Who won the latest World Cup?''
because the knowledge they learn during training may be insufficient or outdated.
Conditioning generation on retrieved documents---a technique known as retrieval augmented generation (RAG)---mitigates these shortcomings by allowing the model to leverage in-context information.
Practitioners can improve LLM RAG performance 
by fine-tuning on retrieval-augmented instructions, 
but must beware that this can cause undesirable model behaviors like hallucinations. 
We attribute this degradation to the fact that the training data is likely to be out-of-distribution for the model and may suffer from quality issues, 
such as misalignment between retrievals and target responses  
(since retrievals are frequently added post-hoc).
We propose a recipe for training RAG-enabled LLMs
using self-generated demonstrations,
thereby avoiding training on out-of-distribution text and integrating retrievals into the LLM responses.
We evaluate our method on knowledge intensive question answering (QA) tasks and show that
our method teaches LLMs to properly handle in-context retrievals and abstain from questions it will likely get wrong.
Compared to conventional RA-IT methods, our method prevents model degradation in non-RAG settings while exhibiting superior QA performance.
