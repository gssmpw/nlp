\section{Related Work}
The integration of Electronic Health Record (EHR) systems and Artificial Intelligence (AI) in the healthcare sector transformed the management of patient data. This section is structured around the application of EHR systems in healthcare organizations. In addition, it outlines recent developments and provides a critical analysis of existing systems to form the foundation of the proposed project.
\subsection{Paper-Based Medical Records}
Healthcare systems have long relied on paper-based medical records for the documentation of patient information. Handwritten notations and laboratory tests have been included in such records. However, paper records have significant disadvantages that make healthcare delivery less effective. A study titled "Paper-Based Medical Records"  presents insightful observations regarding such records.____
The study reveals that paper-based medical records have no efficiency in terms of information retrieval. There could be a delay in accessing important information in the case of an emergency as these records have to be stored in physical filing systems.
\subsection{EHR Systems}
Electronic Health Record (EHR) systems have become an integral part of modern-day healthcare organizations, allowing for enhanced management and access to patient information. According to a study in the International Journal of Health Sciences and Research, the use of EHR systems has significantly increased care delivery, particularly in coordination between care providers, reducing medical errors, and enhancing patient outcomes. By transforming patient information into a computerized format, EHR systems have optimized access for medical professionals to critical information, such as medical histories, laboratory tests, imaging tests, and medication lists, all delivered in a timely and efficient manner.____\\ 
The use of EHR systems has initiated a significant change in practice in terms of enhancing collaboration and information dissemination in a variety of care settings. EHR system interoperability ensures that critical patient information is accessible to approved care providers whenever and wherever it is needed, minimizing care delivery times and enhancing overall care service quality. According to IJHSR, EHR systems enable better decision-making by providing medical professionals with updated information about a patient's state.
The key EHR system traits include:

\begin{itemize}
    \item \textbf{Centralized Data Management} Storing patient data, such as medical history, laboratory tests, imaging, and medications, in a single digital repository.
    \item \textbf{Interoperability} Allowing for the exchange of information among healthcare providers.
    \item \textbf{Patient-Centered Care} Enabling patients through portals and mobile applications to view their health records and interact with providers.
\end{itemize}
\subsection{AI in EHR Systems}
The integration of AI in clinical workflows is very critical, specifically in Egypt, for effective utilization. Research conducted highlights the role of AI in automating labor-intensive tasks, including analysis and reports generation. It declares the importance of user centric AI tools that adapt to diverse healthcare environments.____ 
Public hospitals in Egypt often operate under significant pressure and this integration has the potential to optimize workflow and streamline data access through the organization, especially in emergency cases. Key applications of AI in healthcate include:
\begin{itemize}
    \item \textbf{Clinical Decision Support} Using AI to analyze patient data and provide evidence-based recommendations for diagnosis and treatment.
    \item \textbf{Predictive Analytics} Identifying high-risk patients and predicting disease outbreaks or complications.
\end{itemize}
In the editorial "Artificial Intelligence Algorithms for Healthcare," a thorough analysis is presented concerning the role of AI algorithms in modern-day healthcare. According to the authors, AI-powered approaches are revolutionizing diagnostics, personalized care planning, medical imaging, and administrative efficiency. In addition, they highlight an increased reliance on machine learning (ML), deep learning (DL), and natural language processing (NLP) methodologies, contributing to improvements in several aspects of delivering care, such as the analysis of patient information and predictive model development. ____.
Transitioning from a theoretical concept to a critical part of the healthcare industry. According to the authors, AI-powered tools have already proven significant improvements in early disease detection, medical imaging interpretation, and the automation of medical workflows. For instance, medical tools with inbuilt machine learning capabilities can detect cardiovascular disease and cancer with high accuracy. Likewise, algorithms in natural language processing are utilized to process unstructured medical documents, supporting effective information extraction and integration in medical care systems.
Also, AI-powered predictive analysis in electronic medical records enables high-risk patient identification. With such a feature, early interventions and smart distribution of resources can occur. In addition, AI-powered automation in medical billing and claims processing simplifies such processes.
\subsection{ Advances in AI-Based EHR Systems}
Recent advancements in AI have significantly enhanced the capabilities of EHR systems:
\subsubsection{Generative AI for Data Summarization}
Generative artificial intelligence, specifically in terms of data summarization through models based on Retrieval-Augumented Generation (RAG), is increasingly being utilized to generate and interpret complex medical information, and in the process, ease the cognitive burden for clinicians. For instance, in "Enhancing EHR Analysis: Leveraging RAG-Enabled Generative AI for Clinical Data Summarization," such technology is discussed in detail about how it can effectively present important patient information and actionable information.____ It highlights how this approach can help: 
\begin{itemize}
    \item \textbf{Summarize patient records} Extracting critical details like diagnosis history, medications, and lab results into concise reports.
    \item \textbf{Reduce physician workload} Automating the summarization of lengthy clinical notes, saving time during consultations.
    \item \textbf{Enhance decision-making} Providing real-time, context-aware insights based on patient data and clinical guidelines.
\end{itemize}
Although RAG maximizes medical information summarization, it can be affected by token restrictions present in the input information. Large language models (LLMs) rely on contextual information derived through documents that have been retrieved; yet, when dimensions in terms of input tokens become restricted, a variety of limitations become apparent:
\begin{itemize}
    \item Loss of Contextual Information
    \item Difficulty in Capturing Long-Term Patient History
    \item Inefficient Multi-Step Processing
\end{itemize}
A study, "Clinical Text Summarization: Adapting Large Language Models Can Outperform Human Experts," considers whether it is feasible for large language models (LLMs) to generate summaries of clinical documents not only at a level equivalent to but even exceeding that of expert clinicians. Inclusion of LLM-enabled summarization tools in EHRs can make it feasible to manage high volumes of clinical information in an efficient manner. By providing concise summaries of patient histories and pertinent information, such AI tools can ease clinicians' cognitive loads and can potentially contribute towards better patient outcomes through optimized decision-making.____
\subsubsection{Priority-Based Medical Segmentation}
Priority-Based Medical Segmentation is a new technique for enhancing both care delivery efficiency and efficacy, particularly in high-pressure environments such as in emergency departments. It involves prioritization and extraction of critical patient information from electronic health records (EHRs), and in doing so, enables quick access to most relevant information for immediate use in clinical decision-making. Research, such as in "Designing an AI Healthcare System: EHR and Priority-Based Medical Segmentation Approach," identifies the importance of such a technique in enhancing clinical workflows and patient care outcomes.____
\subsubsection{Real-Time Analytics in AI-Driven EHR Systems}
The integration of AI in EHRs has significantly enhanced real-time medical evaluation, particularly in radiology. One significant use of such technology is in deploying pretrained AI models for X-ray image analysis, allowing medical professionals to effectively evaluate medical visuals and make data-dependent decisions in an efficient manner.\\
"Leveraging Pretrained Models for Multimodal Medical Image Interpretation: An Exhaustive Experimental Analysis" describes how deep learning models can be utilized to promote multimodal medical image analysis, including X-ray diagnostics, through real-time AI evaluations in a clinical setting.____\\
Traditional X-ray interpretation is a labor-intensive, time-consuming process for radiologists. However, AI-powered real-time analytics facilitates:
\begin{itemize}
    \item \textbf{Real-time abnormality detection:} AI can signal potential problems (e.g., fractures, pneumonia, or lung nodules) in seconds.
    \item \textbf{Improved workflow productivity:} Real-time X-ray evaluation eliminates reporting delay time, allowing earlier diagnosis and treatment planning.
    \item \textbf{Integration with EHR systems:} AI-produced reports can be easily inserted into an electronic patient record.
\end{itemize}
\subsection{AI Integration in Clinical Workflows}
The integration of artificial intelligence (AI) in clinical workflows has potential for automation of workloads with high labor requirements, improving overall operational efficiency in general. AI technology can generate clinical documentation, freeing medical professionals from clerical workloads. In addition, AI-powered telemedicine platforms enable virtual consultation and continuous follow-up of long-term medical conditions.

\subsection{Electronic Health Record Systems and Scheduling Software in Egypt}

Health care professionals in Egypt increasingly use digital technology to improve operational efficiency and patient satisfaction. Scheduling software is a key tool for clinic workflows and service delivery in terms of service quality improvement. With increased demand for medical care, numerous scheduling programs have been designed specifically for use in countries with specific requirements, including Egypt. Not only can these tools schedule appointments, but they can also integrate with Electronic Health Record (EHR) platforms, providing an integrated platform to provide care to patients. The following table in figure \ref{comparison} shows software platforms in use in Egypt, prevalent in medical practice, and with important roles in the medical field.
\begin{figure}[h]
    \centering
    \includegraphics[width=1.0\linewidth]{images/market.png}
    \caption{Comparison of EHR applications in Egypt and Scheduling Software Solutions}
    \label{comparison}
\end{figure}

The table compares a selection of key Electronic Health Record (EHR) and scheduling software options in the Egyptian marketplace, with a focus on their key features, target segments, and capabilities. Miclinic EHR, HealthTag, and MedCloud stand out as key EHR platforms, each with a specific target in the healthcare sector. 

In particular, Miclinic EHR and HealthTag target small clinics and offer budget options with key capabilities such as scheduling, telemedicine, and basic EHR capabilities. However, artificial intelligence integration is not a key feature in them. 

On the other hand, MedCloud targets big hospitals, offering high-end capabilities such as strong AI integration, high-level telemedicine capabilities, and high scalability. Despite its high price, MedCloud's ability to manage complex networks in healthcare and enable interoperability makes it a preferred tool for larger healthcare providers.

Standalone scheduling software options, in contrast, have a sole focus on scheduling, an integral part of clinic and hospital operations. Scheduling software programs often integrate with EHR platforms to enable a harmonized workflow; however, EHR capabilities in them are limited. The table identifies a lack of interoperability in terms of data sharing between organizations, a critical consideration in modern-day healthcare networks. MedCloud, for instance, promotes high-level data sharing via HL7 and FHIR standards, but alternatives such as Miclinic EHR and HealthTag lack strong capabilities in such a feature.
To address such gaps in Egypt's healthcare technology environment, our project is focused on creating an upgraded EHR system with state-of-the-art artificial intelligence capabilities for enhancing information dissemination, interoperability, and clinical decision-support structures. In particular, the aim is to correct current systems' weaknesses, including the lack of integration of sophisticated AI in budget-friendly alternatives. Our proposed system incorporates AI models for efficient medical information summarization, and in doing so, enables medical professionals to extract actionable and concise information out of lengthy patient files. With such improvements, our project aims to introduce a cost-effective and EHR platform that not only addresses requirements for hospitals but can serve widespread networks, and in the long run, promote patient care and operational efficiency in Egypt's medical sector.