% Switch to full-page mode
\appendix
\section{User Simulator Performance}
\label{sec:User_Simulator_Performance}

\begin{table}[htbp]
\centering

\resizebox{1.0\linewidth}{!}{ 
\begin{tabular}{l|l|ccc|ccc}
\toprule

\textbf{Dataset} & \textbf{LLM Model} &  
\multicolumn{3}{c|}{\makecell{\textbf{BERTScore} \\ \textbf{User (F1)} }} & 
\multicolumn{3}{c}{\makecell{\textbf{BERTScore} \\ \textbf{Overall (F1)} }} \\  
\cmidrule(lr){3-5} \cmidrule(lr){6-8}

& & 
\makecell{\textbf{Single}} & \makecell{\textbf{Multi}} & \makecell{\textbf{Both}} & 
\makecell{\textbf{Single}} & \makecell{\textbf{Multi}} & \makecell{\textbf{Both}} \\ 
\midrule

\multirow{4}{*}{\textbf{SGD}}
 & {\ours}-{\small \GPTShortName}  & 0.6879 & 0.6712 & 0.6765 & 0.6821 & 0.6761 & 0.6780 \\ 
 & {\ours}-{\small \ClaudeShortName}   & 0.7053 & 0.6972 & 0.6997 & 0.6822 & 0.6852 & 0.6843 \\ 
 & {\ours}-{\small \DeepSeekShortName}   & 0.6844 & 0.6700 & 0.6746 & 0.6600 & 0.6554 & 0.6569 \\ 
 & {\ours}-{\small \LlamaShortName}    & 0.6853 & 0.6766 & 0.6794 & 0.6365 & 0.6256 & 0.6291 \\ 
\midrule

\multirow{4}{*}{\textbf{BiTOD}} 
 & {\ours}-{\small \GPTShortName} & 0.6794 &  0.6549 & 0.6627 & 0.6889 & 0.6671 & 0.6741 \\ 
 & {\ours}-{\small \ClaudeShortName} & 0.6963 & 0.6737 & 0.6809 & 0.6841 & 0.6625 & 0.6693 \\ 
 & {\ours}-{\small \DeepSeekShortName}   & 0.6979 & 0.6502 & 0.6653 & 0.6789 & 0.6530 & 0.6612 \\ 
 & {\ours}-{\small \LlamaShortName}   & 0.6534 & 0.6370 & 0.6423 & 0.6345 & 0.5990 & 0.6104 \\ 
\bottomrule

\end{tabular}%
}
\caption{Performance of the Dialog-Level User Simulator and Entire Conversation.}
\label{table_user_scores}
\end{table}

To evaluate the performance of our user simulator, we computed BERTScore on user turns (BERTScore User) and on the concatenation of both user and system turns (BERTScore Overall). As shown in Table \ref{table_user_scores}, most BERTScores fall within the range of approximately 0.6 to 0.7, showing that the simulated user responses maintain a reasonable level of similarity to the reference.
%The Instruction Prompt for the User Simulator is designed to generate realistic user responses based on the current dialogue state. The template dynamically incorporates relevant details, such as past and upcoming API calls, required slots, and dialogue history, ensuring a coherent and context-aware user simulation. The template is provided in Section \ref{sec:prompt3}.


\section{Example Dialog Responses}
\label{sec:Example_Dialog_Responses}

Tables \ref{conv_claude}, \ref{conv_gpt4o}, \ref{conv_deepseek}, and \ref{conv_llama} present an example of a multi-domain conversation where our User Simulator interacts with different variants of {\ours} (\ClaudeShortName, \GPTShortName, \DeepSeekShortName, and \LlamaShortName) to check the weather, schedule a property visit, and reserve a car. % We check if the user simulator asks for the required slots properly. Next, we examine the LLM responses to see whether they execute the API call {\cmark} as expected in the following turn. %This conversation highlights {\ours}'s ability to Multi-turn Task Completion within TOD Systems while understanding user intent, maintaining API calls, and tracking relevant information across domains. 
Additionally, Tables \ref{conv_soloist} and \ref{conv_zstod} showcase examples generated by the two baseline models SOLOIST and ZS-TOD. Both of these models struggle with task completion; SOLOIST, on the second turn, missed the API call. The same thing happened when ZS-TOD completely forgot to make the API call when handling a car reservation, ultimately failing to provide the requested information. For model errors, we use {\xmark}, and for correct API calls, we use {\cmark}.

%For the model's mistake, we put {\xmark} on the generated example conversation. 
% The key difference between {\ours} and these baseline models is that the user turns in baseline models are ground-truth dialogs from the dataset.




\onecolumn
% \section{User Simulator Performance}




%% Additional Table for API Metrics of SGD Test Set Across all the models
% work on this
%% Table 1
% \begin{table*}[htbp]
% \centering
% \resizebox{\textwidth}{!}{ 
% \begin{tabular}{l|l|ccc|ccc|ccc|ccc|ccc}
% \toprule

% \textbf{Dataset} & \textbf{LLM Model} & 
% \multicolumn{3}{c}{\makecell{\textbf{Method} \\ \textbf{Accuracy}}} & 
% \multicolumn{3}{c}{\makecell{\textbf{Key} \\ \textbf{Accuracy}}} & 
% \multicolumn{3}{c}{\makecell{\textbf{Value} \\ \textbf{Accuracy}}} & 
% \multicolumn{3}{c}{\makecell{\textbf{Full API} \\ \textbf{Accuracy}}} \\ 
% \cmidrule(lr){3-5} \cmidrule(lr){6-8} \cmidrule(lr){9-11} \cmidrule(lr){12-14} \cmidrule(lr){15-17}

% & & \makecell{\textbf{Single}} & \makecell{\textbf{Multi}} & \makecell{\textbf{Both}} & 
% \makecell{\textbf{Single}} & \makecell{\textbf{Multi}} & \makecell{\textbf{Both}} & 
% \makecell{\textbf{Single}} & \makecell{\textbf{Multi}} & \makecell{\textbf{Both}} & 
% \makecell{\textbf{Single}} & \makecell{\textbf{Multi}} & \makecell{\textbf{Both}} \\ 
% \midrule

% \multirow{8}{*}{\textbf{SGD}}
%  & SOLOIST                & 59.64 & 61.16 & 60.74 & 36.51 & 35.16 & 35.54 & 34.21 & 32.72 & 33.13 & 11.72 & 09.97 & 10.45 \\ 
%  & SimpleTOD              & 52.26 & 56.58 & 55.39 & 40.67 & 47.66 & 45.73 & 37.79 & 44.17 & 42.41 & 10.42 & 12.52 & 11.94 \\ 
%  & ZS-TOD                 & 67.51 & 59.65 & 61.82 & 39.33 & 40.49 & 40.17 & 37.90 & 38.23 & 38.13 &15.69 & 11.55 & 12.69 \\ 
%  & AutoTOD                & 57.84 & 60.93 & 60.08 & 59.67 & 65.73 & 64.06 & 55.60 & 61.40 & 59.80 & 42.28 & 42.17 & 42.20 \\ \cline{2-17}
%  & {\ours}-{\small \GPTShortName} & 80.60 & \underline{71.26} & \underline{72.81} & 84.72 & \underline{73.54} & \underline{75.40} & 81.44 & \underline{70.57} & \underline{72.38} & \underline{68.71} & \underline{57.89} & \underline{59.69} \\ 
%  & {\ours}-{\small \ClaudeShortName} & \textbf{88.29} & \textbf{81.74} & \textbf{82.83} & \textbf{88.91} & \textbf{80.59} & \textbf{81.97} & \textbf{85.66} & \textbf{77.08} & \textbf{78.51} & \textbf{72.58} & \textbf{63.04} & \textbf{64.63} \\ 
%  &  {\ours}-{\small \DeepSeekShortName} & \underline{81.11} & 69.25 & 71.22 & 83.78 & 70.80 & 72.96 & 79.42 & 66.42 & 68.58 &  63.27 & 51.67 & 53.59 \\ 
%  & {\ours}-{\small \LlamaShortName}  & 80.32 & 69.60 & 71.39 & \underline{86.65} & 71.46 & 73.99 & \underline{82.36} & 67.56 & 70.02 &  63.23 & 52.58 & 54.35 \\ 
% \midrule

% \end{tabular}%
% }
% \caption{Additional API Call Accuracy Breakdown Across all Models on the Unseen to the Baseline SGD dataset. Accuracy is reported across multiple metrics, including method, parameter name, parameter value, and overall full API accuracy.}
% \label{additioanl_api_scores}
% \end{table*}




% Additional BertScore on 
% \begin{table}[htbp]
% \centering
% \resizebox{0.5\textwidth}{!}{ 
% \begin{tabular}{l|l|ccc|ccc}
% \toprule

% \textbf{Dataset} & \textbf{LLM Model} & 
% \multicolumn{3}{c}{\makecell{\textbf{BERT Score} \\ \textbf{System (F1)}}} &  
% \multicolumn{3}{c}{\makecell{\textbf{Inform Accuracy}}} \\ 
% \cmidrule(lr){3-5} \cmidrule(lr){6-8} 

% & & 
% \makecell{\textbf{Single}} & \makecell{\textbf{Multi}} & \makecell{\textbf{Both}} & 
% \makecell{\textbf{Single}} & \makecell{\textbf{Multi}} & \makecell{\textbf{Both}} \\ 
% \midrule

% \multirow{8}{*}{\textbf{SGD}}
%  & SOLOIST  & 69.51   & 69.72   & 69.62     & 44.54  & 54.37   & 52.63  \\ 
%  & SimpleTOD  & 65.46  & 65.95   & 65.73     & 32.15  & 51.11   & 47.74   \\ 
%  & ZS-TOD  & 42.48    & 42.16   & 42.31    & 18.76  & 13.81  & 14.69    \\ 
%  & AutoTOD   & 57.34    & 59.68   & 58.60    & 76.34   & 75.84  & 75.93    \\  
%  & GPT-4o  & 65.47 & 65.44 &  65.45  & 82.93 & 76.89 & 78.03 \\ 
%  & Claude-3.5   & 65.52 & 66.94 & 66.49  &  79.44 & 71.37 & 72.85 \\ 
%  & DeepSeek-V3   & 63.45 &  63.84   &  63.72   & 82.81 &  75.45   &  76.88 \\ 
%  & Llama-3.3   & 60.19 & 59.79 & 59.92 & 73.49 &  70.26 & 70.88 \\ 
% \midrule

% \multirow{8}{*}{\textbf{BiTOD}} 
%  & SOLOIST   & 47.58   & 49.14   & 48.64   & 85.0  & 75.79  & 76.83   \\ 
%  & SimpleTOD     & 43.76  & 45.63   & 45.04    & 90.0   & 73.24  & 75.14   \\ 
%  & ZS-TOD   & 38.26    & 39.66   & 39.21  & 70.00  & 69.64  & 69.68   \\ 
%  & AutoTOD    & 55.96   & 59.39   & 58.29  & 97.5   & 84.39  & 85.87   \\  
%  & GPT-4o  & 65.43 &  64.47 & 64.77  & 85.18 &  79.78 & 80.26 \\ 
%  & Claude-3.5 & 65.23 & 63.92 &  64.34  & 64.70 & 63.51 & 63.61 \\ 
%  & DeepSeek-V3   & 64.54 & 62.78 & 63.34 & 86.66 & 79.25 & 79.93 \\ 
%  & Llama-3.3  & 59.27 & 56.94 & 57.69 & 64.28 & 65.06 & 65.00 \\ 
% \bottomrule

% \end{tabular}%
% }
% \caption{{\ours} Additioanl Model Metrics Comparison with the Baseline for Unseen SGD Data}
% \end{table}



\onecolumn
% \section{Instruction Prompt Templates}

\section*{Instruction Prompt Template \(P_1\)} \label{sec:prompt1}

% Instruction Prompt
\begin{tcolorbox}[colframe=black, colback=white, boxrule=1pt]
\textbf{\textcolor{cyan}{Task Description:}}

%Your task is to generate a dialog conversation between a User and a System. I will provide \textbf{\textcolor{orange}{Schema}} for \textbf{\textcolor{red!70}{\{domain\_X\}}}, one sample \underline{conversation} between a \underline{System} and a \underline{User}, optionally, search results from the database.  Understand the dialog relation to \textbf{\textcolor{orange}{Schema}}.

Your task is to generate a dialog conversation between a User and a System based on a given domain schema. I will provide a \textbf{{\textcolor{orange}{Schema}}} for \textbf{\textcolor{red!70}{\{domain\_X\}}}, which defines the structure and relevant entities, along with a corresponding dialog conversation for reference. Your goal is to analyze the relationship between the dialog and the schema and then generate a coherent and contextually appropriate dialog conversation for \textbf{\textcolor{teal}{\{domain\_Y\}}} while maintaining consistency with its schema.

\end{tcolorbox}


% Schema Section

\begin{tcolorbox}[colframe=black, colback=white, boxrule=1pt]

Here is a \textbf{{\textcolor{orange}{Schema}}} for the \textbf{\textcolor{red!70}{\{domain\_X\}}}

\textbf{service\_name:} \textbf{\textcolor{red!70}{\{domain\_X\}}}

\textbf{\textcolor{violet}{Intents}}

\quad \textbf{{intent\_no.}} 

\quad \textbf{name}: IntentName

\quad \textbf{is\_transactional}: True/False

\quad \textbf{\textcolor{blue!70}{required\_slots}}: required slot 1, required slot 2, required slot 3, ...

\quad \textbf{\textcolor{blue!70}{optional\_slots}}: optional slot 2, optional slot 2, optional slot 3, ...

\textbf{\textcolor{orange}{Slots}}

\quad \quad \textbf{slot\_name}: slot name 1, slot name 2, slot name 3, ...

\quad \quad \textbf{possible\_values}: value 1, value 2, value 3, ...

\textit{end of schema for \textcolor{teal}{\{domain\_X\}}}

A sample \texttt{\textbf{<< Dialog Conversation >>}} between a System and a User will be fetched.
\end{tcolorbox}


%% Re-Stating Instruction
\begin{tcolorbox}[colframe=black, colback=white, boxrule=1pt]
Now, understand the above conversation structure between a \underline{User} and a \underline{System}. You will be given a new \textbf{{\textcolor{orange}{Schema}}} for \textbf{\textcolor{teal}{\{domain\_Y\}}}. You have to generate a full-fledged conversation for the new domain that will be structured like the example above. 
\end{tcolorbox}

% Schema Domain Y Section
\begin{tcolorbox}[colframe=black, colback=white, boxrule=1pt]

Here is a \textbf{{\textcolor{orange}{Schema}}} for the \textbf{\textcolor{teal}{\{domain\_Y\}}}

\textbf{service\_name:} \textbf{\textcolor{teal}{\{domain\_Y\}}}

\textbf{\textcolor{violet}{Intents}}

\quad \textbf{{intent\_no.}} 

\quad \textbf{name}: IntentName

\quad \textbf{is\_transactional}: True/False

\quad \textbf{\textcolor{blue!70}{required\_slots}}: required slot 1, required slot 2, required slot 3, ...

\quad \textbf{\textcolor{blue!70}{optional\_slots}}: optional slot 2, optional slot 2, optional slot 3, ...

\textbf{\textcolor{orange}{Slots}}

\quad \quad \textbf{slot\_name}: slot name 1, slot name 2, slot name 3, ...

\quad \quad \textbf{possible\_values}: value 1, value 2, value 3, ...

\textit{end of schema for \textcolor{teal}{\{domain\_Y\}}}

\end{tcolorbox}

% restate the instruction
\begin{tcolorbox}[colframe=black, colback=white, boxrule=1pt]
Based on the above instructions and example conversation from the {domain\_X}, learn how to generate the full \underline{conversation} for the new {domain\_Y} domain.

\textit{End of Instructions.}
\end{tcolorbox}


\section*{Instruction Prompt Template \(P_2\)} \label{sec:prompt2}

% Instruction Prompt
\begin{tcolorbox}[colframe=black, colback=white, boxrule=1pt]
\textbf{\textcolor{cyan}{Task Description:}}

Think of yourself as an expert chat assistant specialized in the \textbf{\textcolor{red!70}{\{domain\_name\}}} domain. Your task is to generate the most natural and helpful responses for a given task-oriented dialog context. I will provide \textbf{\textcolor{orange}{Schema}} for \textbf{\textcolor{red!70}{\{domain\_name\}}}, one sample \underline{conversation} between a \underline{System} and a \underline{User}, optionally, search results from the database.  Understand the dialog relation to \textbf{\textcolor{orange}{Schema}}. You can request slot values from the \underline{User} to fulfill the \underline{User}’s current \textbf{\textcolor{violet}{intent}}. Remember that required \textbf{\textcolor{blue!70}{slots}} are more important than optional \textbf{\textcolor{blue!70}{slots}}. When making \textbf{\textbf{API}} calls, use column names from the \textbf{\textcolor{orange}{Schema}} as parameters. Match the required and optional \textbf{\textcolor{blue!70}{slots}} with the column names and use them in \textbf{\textbf{API}} calls. Before making the call, ensure you've gathered all required \textbf{\textcolor{blue!70}{slots}} from the \underline{User}. You can skip unnecessary parameters.


\end{tcolorbox}

% Schema Section

\begin{tcolorbox}[colframe=black, colback=white, boxrule=1pt]

Here is a \textbf{{\textcolor{orange}{Schema}}} for the \textbf{\textcolor{red!70}{{domain\_name}}}

\textbf{service\_name:} \textbf{\textcolor{red!70}{\{domain\_name\}}}

\textbf{\textcolor{violet}{Intents}}

\quad \textbf{{intent\_no.}} 

\quad \textbf{name}: IntentName

\quad \textbf{is\_transactional}: True/False

\quad \textbf{\textcolor{blue!70}{required\_slots}}: required slot 1, required slot 2, required slot 3, ...

\quad \textbf{\textcolor{blue!70}{optional\_slots}}: optional slot 2, optional slot 2, optional slot 3, ...

\textbf{\textcolor{orange}{Slots}}

\quad \quad \textbf{slot\_name}: slot name 1, slot name 2, slot name 3, ...

\quad \quad \textbf{possible\_values}: value 1, value 2, value 3, ...

\textit{end of schema}

A sample \texttt{\textbf{<< Dialog Conversation >>}} between a System and a User will be fetched.
\end{tcolorbox}

% Explanation of Conversation Structure
\begin{tcolorbox}[colframe=black, colback=white, boxrule=1pt]
Understand the above structure of \underline{conversation} between a \underline{User} and a \underline{System}. Learn how to interact with the \underline{User} and generate the most human-like conversational response to the \underline{User}'s \textbf{\textcolor{violet}{intent}}. You may need to make \textbf{\textbf{API}} Calls and use the \textbf{\textbf{API}} Call results. Based on the above instructions and examples from the \textbf{\textcolor{red!70}{\{domain\_name\}}} domain, learn how to interact with a \underline{User} to generate the most human-like conversational response to the \underline{User}'s current \textbf{\textcolor{violet}{intent}}.

\textit{End of Instructions.}
\end{tcolorbox}


% Guidelines Section
\begin{tcolorbox}[colframe=black, colback=white, boxrule=1pt]
Here are a few general \textbf{\textcolor{cyan}{Guidelines}} to follow:

- Please avoid asking for too many \textbf{\textcolor{blue!70}{slots}} in one turn; ideally, ask one slot at a time.

- Don't overwhelm the \underline{User} with too many questions or choices in one turn.

- Confirm the slot values with the \underline{User} before finalizing the \textbf{\textbf{API}} Call.

- Follow the structure of \textbf{\textbf{API}} Call from the above example whenever you are making an \textbf{\textbf{API}} Call.

- If you're unsure about something, it's always better to ask or confirm with the \underline{User}.

- Do not provide all the information in the search results to the \underline{User}. Provide details only if the \underline{User} requests them.

- If you feel the \underline{User} is confused, guide the \underline{User} with relevant suggestions and ensure it is relevant to their current \textbf{\textcolor{violet}{intent}}.

- You generate only one system response at a time and do not produce search results yourself; search results will be provided to you.

\vspace{10pt}

\textbf{Conversation history:} << conversation history >> up to turn $t$ will be fetched

\end{tcolorbox}

\iffalse
\section*{Instruction Prompt Template for User Simulator \(P_3\)} \label{sec:prompt3}
\begin{tcolorbox}[colframe=black, colback=white, boxrule=1pt]

\textbf{\textcolor{cyan}{User Simulator Prompt:}}

\texttt{prompt\_text = }
\texttt{You are a User Simulator for the domain: \textbf{\textcolor{red!70}{\{domain\_name\}}}},\\
\texttt{Generate a realistic user response based on the following information:}\\
\texttt{Next API Call:\textcolor{red}{APICall(method='MethodName', params=\{param\_name: \textcolor{teal}{param\_value} \})} or 'N/A'\}"},\\
\texttt{Next \textcolor{blue!70}{required slots}: {\{', '.join({next\_user\_req\_slots)}} if {next\_user\_req\_slots}} else 'N/A'\},\\
\texttt{Last API Call: \textcolor{red}{\{last\_APICall or 'N/A'\}}},\\
\texttt{Last \textcolor{blue!70}{required slots}: {\{', '.join({last\_user\_req\_slots)}} if {last\_user\_req\_slots}} else 'N/A'\},\\
\texttt{Dialog History:},\\
\texttt{{\textcolor{orange!80!black}{dialog\_history}}}\\
   
\texttt{return prompt\_text}
\end{tcolorbox}
\fi




%% Dialog Examples Section
% SGD Claude-3.5
% \section{Multi-domain SGD Comparison: {\ours}(Claude-3.5-Sonnet)}
\begin{table*}[t!]
\centering
\caption{Multi-Domain SGD Conversation - {\ours} (\ClaudeLongName)}
\label{tab:multi_sgd_results}
\begin{tabular}{p{15cm}}
\toprule
\multicolumn{1}{c}{\textbf{SGD Multi-Domain Conversation} (\ClaudeLongName)} \\
\midrule
\multicolumn{1}{c}{\textbf{Goal:} User is trying to check the weather in \textcolor{teal}{Vancouver}, then schedule a property} \\ \multicolumn{1}{c}{visit at the \textcolor{teal}{Golf Club Manor Apartments}, and reserve a \textcolor{teal}{Hatchback} car} \\
\midrule

\textbf{User:} What's the weather going to be like in Vancouver, BC on March 2nd?\\
\midrule
\textbf{RealTOD:} I'll help you check the weather in Vancouver, BC for March 2nd. \textcolor{red}{APICall(method='GetWeather', parameters=\{ city: \textcolor{teal}{Vancouver}, date: \textcolor{teal}{2024-03-02} \})}  {\cmark}\\
\midrule
\textbf{Search Results:} \textcolor{blue}{[\{'city': 'Vancouver', 'date': '2024-03-02', 'temperature': 68, 'precipitation': 25, 'humidity': 26, 'wind': 6\}]} \\
\midrule
\textbf{RealTOD:} On March 2nd in Vancouver, the temperature will be 68 degrees Fahrenheit with a 25 percent chance of precipitation. The humidity will be at 26\% with wind speeds of 6 mph. \\
\textbf{User:} Okay, sounds good. I'd like to schedule a visit to the Golf Club Manor Apartments.\\
\textbf{RealTOD:} I can help you schedule a property visit to Golf Club Manor Apartments. I notice you were checking the weather for March 2nd. Would you like to schedule the visit for that same date? \\
\textbf{User:} Yes, I'd like to schedule the visit. \\
\midrule
\textbf{RealTOD:} \textcolor{red}{APICall(method='ScheduleVisit', parameters=\{ property\_name: \textcolor{teal}{Golf Club Manor Apartments}, visit\_date: \textcolor{teal}{2024-03-02} \})}  {\cmark} \\
\midrule
\textbf{Search Results:} \textcolor{blue}{[\{'property\_name': 'Golf Club Manor Apartments', 'address': '375 Industrial Parkway \# 314', 'area': 'Hayward', 'price': 2500000, 'phone\_number': '510\-581\-0911'\}]} \\
\midrule
\textbf{RealTOD:} I've confirmed your visit to Golf Club Manor Apartments for March 2nd. The property is located at 375 Industrial Parkway \# 314 in Hayward. Would you like to know any specific details about the property? \\
\textbf{User:} Yes, I'd like to reserve a car there. I'd like to pick it up at Indira Gandhi International Airport at 3 o'clock in the afternoon. I'd like a Hatchback. \\
\textbf{RealTOD:} I'll help you reserve a Hatchback at Indira Gandhi International Airport. Would you like to pick up the car on the same date at 3:00 PM? Also, how long will you keep the car? \\
\textbf{User:} Yes, that sounds good. I'll return the car on the 3rd of this month. \\
\textbf{RealTOD:} I'll help you reserve a Hatchback from March 2nd to March 3rd. Would you like to add insurance to your rental? \\
\textbf{User:} Yes, I'd like to add insurance. \\
\midrule
\textbf{RealTOD:} \textcolor{red}{APICall(method='ReserveCar', parameters=\{ pickup\_location: \textcolor{teal}{Indira Gandhi International Airport}, car\_type: \textcolor{teal}{Hatchback}, start\_date: \textcolor{teal}{2019-03-02}, end\_date: \textcolor{teal}{2019-03-03}, pickup\_time: \textcolor{teal}{15:00}, add\_insurance: \textcolor{teal}{True} \})}  {\cmark}\\
\midrule
\textbf{Search Results:} \textcolor{blue}{[\{'car\_name': 'Fiat Panda', 'car\_type': 'Hatchback', 'pickup\_location': 'Indira Gandhi International Airport', 'pickup\_time': '15:00', 'price\_per\_day': 39.0, 'add\_insurance': True\}]} \\
\midrule
\textbf{RealTOD:} Your car reservation has been confirmed. You'll be getting a Fiat Panda Hatchback, which will cost \$39.00 per day with insurance included. You can pick it up at Indira Gandhi International Airport on March 2nd at 3:00 PM. Would you like to know any additional details about your car rental? \\
\textbf{User:} No, thank you for your help. \\
\textbf{RealTOD:} I'm happy to assist. Have a great day! \\
\bottomrule
\end{tabular}
\label{conv_claude}
\end{table*}



% SGD on GPT 4o
% \section{Multi-domain SGD Comparison: {\ours}(GPT-4o)}
\begin{table*}[ht]
\centering
\caption{Multi-SGD Conversation - {\ours} (\GPTLongName)}
\label{tab:multi_sgd_results}
\begin{tabular}{p{15cm}}
\toprule
\multicolumn{1}{c}{\textbf{SGD Multi-Domain Conversation } (\GPTLongName)} \\
\midrule
\multicolumn{1}{c}{\textbf{Goal:} User is trying to check the weather in \textcolor{teal}{Vancouver}, then schedule a property} \\ \multicolumn{1}{c}{visit at the \textcolor{teal}{Golf Club Manor Apartments}, and reserve a \textcolor{teal}{Hatchback} car} \\
\midrule

\textbf{User:} What's the weather going to be like in Vancouver, BC on March 2nd? \\
\midrule
\textbf{RealTOD:} \textcolor{red} {APICall(method='GetWeather', parameters=\{ city: \textcolor{teal}{Vancouver}, date: \textcolor{teal}{2019-03-02} \})} {\cmark}  \\
\midrule
\textbf{Search Results:} \textcolor{blue}{[\{'city': 'Vancouver', 'date': '2019-03-02', 'temperature': 67, 'precipitation': 49, 'humidity': 45, 'wind': 15\}]} \\
\midrule
\textbf{RealTOD:} The average temperature for the day will be about 67 degrees Fahrenheit, and there's a 49 percent chance of rain. \\
\textbf{User:} That sounds good. I'd like to schedule a visit to Golf Club Manor Apartments. \\
\textbf{RealTOD:} Could you confirm that you want to visit Golf Club Manor Apartments on March 2nd? \\
\textbf{User:} Sounds good. \\
\midrule
\textbf{RealTOD:} \textcolor{red}{APICall(method='ScheduleVisit', parameters=\{ property\_name: \textcolor{teal}{Golf Club Manor Apartments}, visit\_date: \textcolor{teal}{2019-03-02} \})} {\cmark}\\
\midrule
\textbf{Search Results:} \textcolor{blue}{[\{'property\_name': 'Golf Club Manor Apartments', 'address': '375 Industrial Parkway \# 314', 'number\_of\_baths': 3, 'number\_of\_beds': 3, 'phone\_number': '510-581-0911', 'price': 3950\}]} \\
\midrule
\textbf{RealTOD:} Your visit has been scheduled. \\
\textbf{User:} I would like to reserve a car there till the 3rd. \\
\textbf{RealTOD:} Do you like Hatchback, SUV, or another type of car? Would you like to pick up the car from March 2nd and from where? \\
\textbf{User:} Sounds good. Delhi Airport is a good place for me to pick it up. A hatchback sounds perfect. \\
\textbf{RealTOD:} Would you like insurance? What time will you pick it up? \\
\textbf{User:} I'd like insurance and will pick it up at 3:00 PM. \\
\textbf{RealTOD:} So you want to pick up the car from Indira Gandhi International Airport on March 2nd at 3 PM and drop it off on March 3rd? It's a Hatchback that comes with insurance. \\
\textbf{User:} Sounds perfect. \\
\midrule
\textbf{RealTOD:} \textcolor{red}{APICall(method='ReserveCar', parameters=\{ add\_insurance: \textcolor{teal}{True}, car\_type: \textcolor{teal}{Hatchback}, end\_date: \textcolor{teal}{2019-03-03}, pickup\_location: \textcolor{teal}{Indira Gandhi International Airport}, pickup\_time: \textcolor{teal}{15:00}, start\_date: \textcolor{teal}{2019-03-02} \})} {\cmark} \\
\midrule
\textbf{Search Results:} \textcolor{blue}{[\{'car\_name': 'Hyundai i20', 'car\_type': 'Hatchback', 'pickup\_location': 'Indira Gandhi International Airport', 'pickup\_time': '15:00', 'price\_per\_day': 35.00, 'add\_insurance': True\}]} \\
\midrule
\textbf{RealTOD:} Your reservation for a Hyundai i20 Hatchback has been confirmed. The cost is \$35.00 per day. If you have any more questions or need further assistance, feel free to ask! \\
\textbf{User:} Thank you. That's all I need for now. \\
\textbf{RealTOD:} Have a great day! \\
\bottomrule
\end{tabular}
\label{conv_gpt4o}
\end{table*}



%% SGD Deepseek Conv
% \section{Multi-domain SGD Comparison: {\ours}(DeepSeek)}
\begin{table*}[ht]
\centering
\caption{Multi-SGD Conversation - {\ours} (\DeepSeekLongName)}
\label{tab:multi_sgd_results}
\begin{tabular}{p{15cm}}
\toprule
\multicolumn{1}{c}{\textbf{SGD Multi-Domain Conversation } (\DeepSeekLongName)} \\
\midrule
\multicolumn{1}{c}{\textbf{Goal:} User is trying to check the weather in \textcolor{teal}{Vancouver}, then schedule a property} \\ \multicolumn{1}{c}{visit at the \textcolor{teal}{Golf Club Manor Apartments}, and reserve a \textcolor{teal}{Hatchback} car} \\
\midrule

\textbf{User:} What's the weather going to be like in Vancouver, BC on March 2nd? \\
\midrule
\textbf{RealTOD:} \textcolor{red}{APICall(method='GetWeather', parameters=\{ city: \textcolor{teal}{Vancouver}, date: \textcolor{teal}{2019-03-02} \})} {\cmark} \\
\midrule
\textbf{Search Results:} \textcolor{blue}{[\{'city': 'Vancouver', 'date': '2019-03-02', 'temperature': 45, 'precipitation': 10, 'humidity': 65, 'wind': 12\}]} \\
\midrule
\textbf{RealTOD:} On March 2nd in Vancouver, BC, the temperature will be 45 degrees Fahrenheit with a 10 percent chance of rain. The wind speed will be 12 mph. \\
\textbf{User:} I'd like to schedule a visit to the Golf Club Manor Apartments. \\
\textbf{RealTOD:} Got it! I can help you schedule a visit to the Golf Club Manor Apartments. Could you please confirm the date you'd like to schedule the visit for? \\
\textbf{User:} I'd like to visit the property on the 2nd of this month. \\
\midrule
\textbf{RealTOD:} \textcolor{red}{APICall(method='ScheduleVisit', parameters=\{ property\_name: \textcolor{teal}{Golf Club Manor Apartments}, visit\_date: \textcolor{teal}{2019-03-02} \})}  {\cmark} \\
\midrule
\textbf{Search Results:} \textcolor{blue}{[\{'property\_name': 'Golf Club Manor Apartments', 'address': '123 Fairway Drive', 'number\_of\_baths': 2, 'number\_of\_beds': 3, 'phone\_number': '604-555-1234', 'price': 2500\}]} \\
\midrule
\textbf{RealTOD:} Your visit to the Golf Club Manor Apartments on March 2nd has been confirmed. The property is located at 123 Fairway Drive and has 3 bedrooms and 2 bathrooms. Let me know if you’d like any additional details! \\
\textbf{User:} I'd like to book a car to get there. I'd like to pick it up at Indira Gandhi International Airport, and I'd like it to be a Hatchback. \\
\textbf{RealTOD:} I'll help you book a Hatchback for your trip. Could you please provide the date and time you'd like to pick up the car, the return date, and whether you'd like to add insurance? \\
\textbf{User:} I'd like to pick up the car at 3 o'clock in the afternoon, and I'd like to return it on the 3rd of this month. I'd like to add insurance to my rental. \\
\midrule
\textbf{RealTOD:} \textcolor{red}{APICall(method='ReserveCar', parameters=\{ car\_type: \textcolor{teal}{Hatchback}, pickup\_location: \textcolor{teal}{Indira Gandhi International Airport}, start\_date: \textcolor{teal}{2019-03-02}, pickup\_time: \textcolor{teal}{15:00}, end\_date: \textcolor{teal}{2019-03-03}, add\_insurance: \textcolor{teal}{True} \})} {\cmark} \\
\midrule
\textbf{Search Results:} \textcolor{blue}{[\{'car\_name': 'Fiat Panda', 'car\_type': 'Hatchback', 'pickup\_location': 'Indira Gandhi International Airport', 'pickup\_time': '15:00', 'price\_per\_day': 39.0, 'add\_insurance': True\}]} \\
\midrule
\textbf{RealTOD:} Your car reservation has been successfully made! You'll be getting a Fiat Panda Hatchback, which will cost \$39.00 per day with insurance included. You can pick it up at Indira Gandhi International Airport on March 2nd at 3:00 PM. Let me know if you’d like any additional details! \\
\textbf{User:} Thank you for your assistance. That will be all for now. \\
\textbf{RealTOD:} I'm happy to assist. Have a great day! \\
\bottomrule
\end{tabular}
\label{conv_deepseek}
\end{table*}



%% SGD on Llama3.3-71b-instruct

% \section{Multi-domain SGD Comparison: {\ours}(Llama-3.3-70B-Instruct)}
\begin{table*}[ht]
\centering
\caption{Multi-SGD Conversation - {\ours} (\LlamaLongName)}
\label{tab:multi_sgd_results}
\begin{tabular}{p{15cm}}
\toprule
\multicolumn{1}{c}{\textbf{SGD Multi-Domain Conversation} (\LlamaLongName)} \\
\midrule
\multicolumn{1}{c}{\textbf{Goal:} User is trying to check the weather in \textcolor{teal}{Vancouver}, then schedule a property} \\ \multicolumn{1}{c}{visit at the \textcolor{teal}{Golf Club Manor Apartments}, and reserve a \textcolor{teal}{Hatchback} car} \\
\midrule

\textbf{User:} What's the weather going to be like in Vancouver, BC on March 2nd? \\
\textbf{RealTOD:} To provide you with the weather information for Vancouver, BC on March 2nd, I'll need to make an API call. Before I do that, I just want to confirm the details with you. You're looking for the weather in Vancouver, BC, and the specific date you're interested in is March 2nd, correct? \\
\textbf{User:} Yes, that's correct. \\
\midrule
\textbf{RealTOD:} \textcolor{red}{APICall(method='GetWeather', parameters=\{ city: \textcolor{teal}{Vancouver}, date: \textcolor{teal}{2019-03-02} \})} {\cmark}\\
\midrule
\textbf{Search Results:} \textcolor{blue}{[\{'city': 'Vancouver', 'date': '2019-03-02', 'temperature': 68, 'precipitation': 25, 'humidity': 26, 'wind': 6\}]} \\
\midrule
\textbf{RealTOD:} It seems like the search results provided multiple weather forecasts for Vancouver, BC on March 2nd. To provide a clear answer, could you specify if you're more interested in temperature, precipitation, or humidity? \\
\textbf{User:} Yes, that's what I'm looking for. I'd like to schedule a visit to the Golf Club Manor Apartments on that day. \\
\textbf{RealTOD:} To schedule a visit to the Golf Club Manor Apartments, I need to confirm the details with you. You'd like to visit on March 2nd, correct? \\
\textbf{User:} Yes, that's correct. \\
\midrule
\textbf{RealTOD:} \textcolor{red}{APICall(method='ScheduleVisit', parameters=\{ property\_name: \textcolor{teal}{Golf Club Manor Apartments}, visit\_date: \textcolor{teal}{2019-03-02} \})} {\cmark} \\
\midrule
\textbf{Search Results:} \textcolor{blue}{[\{'property\_name': 'Golf Club Manor Apartments', 'address': '375 Industrial Parkway \# 314', 'number\_of\_baths': 1, 'number\_of\_beds': 2, 'phone\_number': '510\-581\-0911', 'price': 2500000\}]} \\
\midrule
\textbf{RealTOD:} Your visit to Golf Club Manor Apartments has been scheduled. The property is located at 375 Industrial Parkway \# 314. \\
\textbf{User:} I'd like to reserve a car there and pick it up at Indira Gandhi International Airport. \\
\textbf{RealTOD:} Would you like a Sedan, Hatchback, or SUV? Also, what time would you like to pick it up? \\
\textbf{User:} I'd like a Hatchback and will pick it up at 3:00 PM. \\
\midrule
\textbf{RealTOD:} \textcolor{red}{APICall(method='ReserveCar', parameters=\{ add\_insurance: \textcolor{teal}{True}, car\_type: \textcolor{teal}{Hatchback}, end\_date: \textcolor{teal}{2019-03-03}, pickup\_location: \textcolor{teal}{Indira Gandhi International Airport}, pickup\_time: \textcolor{teal}{15:00}, start\_date: \textcolor{teal}{2019-03-02} \})} {\cmark}\\
\midrule
\textbf{Search Results:} \textcolor{blue}{[\{'car\_name': 'Fiat Panda', 'car\_type': 'Hatchback', 'pickup\_location': 'Indira Gandhi International Airport', 'pickup\_time': '15:00', 'price\_per\_day': 39.00, 'add\_insurance': True\}]} \\
\midrule
\textbf{RealTOD:} Your reservation for a Fiat Panda Hatchback has been confirmed. The cost is \$39.00 per day. \\
\textbf{User:} Thank you very much. You've been very helpful. That will be all for now. \\
\textbf{RealTOD:} Have a great day! \\
\bottomrule
\end{tabular}
\label{conv_llama}
\end{table*}


%SOLOIST on SGD
% \section{Multi-domain SGD Comparison: SOLOIST}
\begin{table*}[ht]
\centering
\caption{Multi-SGD Conversation Baseline Model Dialog Example -(SOLOIST)}
\label{tab:multi_sgd_results}
\begin{tabular}{p{15cm}}
\toprule
\multicolumn{1}{c}{\textbf{SGD Multi-Domain Conversation} (SOLOIST)} \\
\midrule
\multicolumn{1}{c}{\textbf{Goal:} User is trying to check the weather in \textcolor{teal}{Vancouver}, then schedule a property} \\ \multicolumn{1}{c}{visit at the \textcolor{teal}{Golf Club Manor Apartments}, and reserve a \textcolor{teal}{Hatchback} car} \\
\midrule

\textbf{User:} I'd like to find out what the weather is going to be like on 2nd of March in Vancouver, BC. \\
\midrule
\textbf{SOLOIST:} \textcolor{red}{APICall: (Missed API Call Here)} {\xmark}\\
\midrule
\textbf{Search Results:} \textcolor{blue}{[Missed Search Results]} {\xmark}\\
\midrule

\textbf{SOLOIST:} The average temperature for the day should be 67 degrees Fahrenheit with a 49 percent chance of rain. \\
\textbf{User:} That sounds good. I'd like to schedule a visit to Golf Club Manor Apartments. \\
\textbf{SOLOIST:} What time do you want to pick it up? \\
\textbf{User:} Sounds good. \\
\midrule
\textbf{SOLOIST:} \textcolor{red}{APICall(method='ScheduleVisit', parameters=\{ property\_name: \textcolor{teal}{Golf Club Manor Apartments}, visit\_date: \textcolor{teal}{2019-03-02nd} \})} {\cmark}\\
\midrule
\textbf{Search Results:} \textcolor{blue}{[\{'address': '375 Industrial Parkway \# 314', 'has\_garage': False, 'in\_unit\_laundry': True, 'number\_of\_baths': 3, 'number\_of\_beds': 3, 'phone\_number': '510-581-0911', 'price': 3950, 'property\_name': 'Golf Club Manor Apartments', 'visit\_date': '2019-03-02'\}]} \\
\midrule
\textbf{SOLOIST:} Your visit has been scheduled. \\
\textbf{User:} I would like to reserve a car there till the 3rd. \\
\textbf{SOLOIST:} Could I help you with anything else? \\
\textbf{User:} Sounds good. Delhi Airport is a good place for me to pick it up. A hatchback sounds perfect. \\
\textbf{SOLOIST:} Please confirm: You want me to reserve a standard car from YVR International Airport on March 2nd at 3:30 pm until March 11th. \\
\textbf{User:} I'd like insurance and will pick it up at 3 PM. \\
\textbf{SOLOIST:} Do you want to pick it up on March 1st? You'll be picking it up at 3:30 pm. Is that right? \\
\textbf{User:} Sounds perfect. \\
\midrule
\textbf{SOLOIST:} \textcolor{red}{APICall(method='GetCarsAvailable', parameters=\{ add\_insurance: \textcolor{teal}{True}, car\_type: Standard, end\_date: \textcolor{teal}{2019-03-03}, pickup\_location: \textcolor{teal}{YVR International Airport}, pickup\_time: \textcolor{teal}{15:30}, start\_date: \textcolor{teal}{2019-03-01} \})} {\xmark} \\

\midrule
\textbf{Search Results:} \textcolor{blue}{[\{'car\_name': 'Hyundai Elantra', 'car\_type': 'Standard', 'pickup\_location': 'VVR International Airport', 'pickup\_time': '15:30', 'price\_per\_day': 34.0, 'add\_insurance': True\}]}  \\
\midrule
\textbf{SOLOIST:} Your car has been reserved. \\
\textbf{SOLOIST:} The car has been reserved. It is a Hyundai Elantra and it is \$34. \\
\textbf{User:} What is the name of the car and how much is it per day? \\
\textbf{SOLOIST:} It is \$34.00 per day and it's called a Hyundai Elantra. \\
\textbf{User:} Thanks for helping. That's it for now. \\
\textbf{SOLOIST:} Have a great day. \\
\bottomrule
\end{tabular}
\label{conv_soloist}
\end{table*}

% ZS TOD on SGD
% \section{Multi-domain SGD Comparison: ZS-TOD}
\begin{table*}[ht]
\centering
\caption{Multi-SGD Conversation Baseline Model Dialog Example - (ZS-TOD)}
\label{tab:multi_sgd_results}
\begin{tabular}{p{15cm}}
\toprule
\multicolumn{1}{c}{\textbf{SGD Multi-Domain Conversation} (ZS-TOD)} \\
\midrule
\multicolumn{1}{c}{\textbf{Goal:} User is trying to check the weather in \textcolor{teal}{Vancouver}, then schedule a property} \\ \multicolumn{1}{c}{visit at the \textcolor{teal}{Golf Club Manor Apartments}, and reserve a \textcolor{teal}{Hatchback} car} \\
\midrule

\textbf{User:} I'd like to find out what the weather is going to be like on 2nd of March in Vancouver, BC. \\
\midrule
\textbf{ZS-TOD:} \textcolor{red}{APICall(method='GetWeather', parameters=\{ city: \textcolor{cyan}{Vancouver}, date: \textcolor{cyan}{2019-03-02} \})} {\cmark} \\
\midrule
\textbf{Search Results:} \textcolor{blue}{[{ city: Vancouver, date: 2019-03-02, humidity: 45, precipitation: 49, temperature: 67, wind: 15 }]} \\
\midrule
\textbf{ZS-TOD:} The average temperature for the day will be about 67 degrees Fahrenheit and there's a 49 percent chance of rain. \\
\textbf{User:} That sounds good. I'd like to schedule a visit to Golf Club Manor Apartments. \\
\textbf{ZS-TOD:} Could you confirm that you want to visit Golf Club Manor Apartments on March 2nd? \\
\textbf{User:} Sounds good. \\
\midrule
\textbf{ZS-TOD:} \textcolor{red}{APICall(method='ScheduleVisit', parameters=\{ property\_name: \textcolor{cyan}{Golf Club Manor Apartments}, visit\_date: \textcolor{cyan}{2019-03-02} \})} {\cmark} \\
\midrule
\textbf{Search Results:} \textcolor{blue}{[{ address: 375 Industrial Parkway \# 314, has\_garage: False, in\_unit\_laundry: True, number\_of\_baths: 3, number\_of\_beds: 3, phone\_number: 510-581-0911, price: 3950, property\_name: Golf Club Manor Apartments, visit\_date: 2019-03-02 }]} \\
\midrule
\textbf{ZS-TOD:} Your visit has been scheduled. \\
\textbf{User:} I would like to reserve a car there till the 3rd. \\
\textbf{ZS-TOD:} Do you like Hatchback, SUV, or another type of car? Would you like to pick up the car from March 2nd and from where? \\
\textbf{User:} Sounds good. Delhi Airport is a good place for me to pick it up. A hatchback sounds perfect. \\
\textbf{ZS-TOD:} Would you like insurance? What time will you pick it up? \\
\textbf{User:} I'd like insurance and will pick it up at 3 PM. \\
\textbf{ZS-TOD:} So you want to pick up the car from Indira Gandhi International Airport on March 2nd at 3 PM and drop it off on March 3rd? It's a Hatchback that comes with insurance. \\

\midrule
\textbf{ZS-TOD:} \textcolor{red}{APICall: (Missed API Call Here)} {\xmark}\\
\midrule
\textbf{Search Results:} \textcolor{blue}{[Missed Search Results]} {\xmark}\\
\midrule

\textbf{User:} What is the name of the car and how much is it per day? \\
\textbf{ZS-TOD:} It is \$39.00 per day and it's called a Fiat Panda. \\
\textbf{User:} Thanks for helping. That's it for now. \\
\textbf{ZS-TOD:} Have a great day. \\
\bottomrule
\end{tabular}
\label{conv_zstod}
\end{table*}

\label{sec:appendix}


\onecolumn
\section{User Study Instructions}
\label{sec:templates}

\section*{Disclaimers of any risks to participants or annotators}

There are no significant risks associated with participating in this study. However, annotators may experience mild fatigue or cognitive strain due to prolonged reading and evaluation of multiple conversations. If you feel discomfort or fatigue, please take breaks as needed.

\section*{Instructions for Human Study Participants}

Your task is to evaluate model-generated responses in multi-turn, task-oriented conversations based on the following criteria:
\begin{enumerate}
    \item Fluency
    \item Informativeness
    \item Task Completion
\end{enumerate}

\section*{Task Overview}
\begin{itemize}
    \item You will be presented with multiple conversations, where a user interacts with a model to complete a task (e.g., booking a flight).
    \item Your job is to rate each model response independently using a 1-5 scale based on the provided criteria.
    \item This study is blind --- you will not know which model produced which response.
\end{itemize}

\section*{How to Rate Responses}
You will assign a score for each response based on the following:

\subsection*{1. Fluency}
Measures whether the response is grammatically correct, well-structured, and natural to read.

\begin{itemize}
    \item 1 = Unnatural, grammatically incorrect, or hard to read.
    \item 5 = Perfectly fluent, natural, and error-free.
\end{itemize}

\textbf{Example (Good Fluency)}
\begin{quote}
User: I need to book a flight to New York next Monday.\\
Model: Sure! What time would you like to depart?
\end{quote}

\textbf{Example (Poor Fluency)}
\begin{quote}
User: I need to book a flight to New York next Monday.\\
Model: Sure! You next Monday flight want time what?
\end{quote}

\subsection*{2. Informativeness}
Measures whether the response provides useful and relevant information to advance the task.

\begin{itemize}
    \item 1 = Vague, incorrect, or unhelpful.
    \item 5 = Precise, relevant, and useful.
\end{itemize}

\textbf{Example (Good Informativeness)}
\begin{quote}
User: Can you recommend a vegetarian restaurant nearby?\\
Model: Yes! Green Leaf Café is a highly rated vegetarian restaurant.
\end{quote}

\textbf{Example (Low Informativeness)}
\begin{quote}
User: Can you recommend a vegetarian restaurant nearby?\\
Model: There are many restaurants in your area.
\end{quote}

\subsection*{3. Task Completion}
Measures whether the response successfully progresses or completes the task in the conversation.

\begin{itemize}
    \item 1 = Fails to address the request.
    \item 5 = Fully completes the task.
\end{itemize}

\textbf{Example (Good Task Completion)}
\begin{quote}
User: I’d like to book a table for two at an Italian restaurant at 7 PM today.\\
Model: I found a reservation at Bella Italia for 7 PM. Would you like me to reserve it for 2 people at 7 PM today?
\end{quote}

\textbf{Example (Failed Task Completion)}
\begin{quote}
User: I’d like to book a table for two at an Italian restaurant at 7 PM.\\
Model: There are Italian restaurants in your area.
\end{quote}

\section*{Additional Guidelines}
\begin{itemize}
    \item Be objective --- rate based on quality, not personal preference.
    \item If a response is unclear or ambiguous, leave a comment.
    \item Do not try to guess which model produced the response.
\end{itemize}

\section*{Study Duration \& Submission}
\begin{itemize}
    \item The study will take approximately 10 minutes to complete.
    \item Once you have evaluated all responses, submit your ratings.
\end{itemize}