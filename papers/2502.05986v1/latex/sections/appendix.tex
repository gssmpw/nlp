
\section{Uncertainty estimation for monitors}
\label{sec:stat_measures}
In our monitoring, we use 3 statistical measures over the probability vector: Entropy, Varentropy and Kurtosis.
Given a probability vector $P$, Entropy, Varentropy and Kurtosis are defined as:
\begin{align*}
&\text{Entropy}(P) = -\sum_{p\in P} log(p)\cdot p \\
&\text{Varentropy}(P) = \\ 
&\hspace{1cm} -\sum_{p\in P} p \cdot (log(p) + \text{Entropy}(P))^2 \\
& \text{Kurtosis}(P) = \\
&\hspace{1cm} \frac{\sum_{p\in P} p (-log(p) - \text{Entropy}(P))^4}{\text{Varentropy}(P)^2}
\end{align*}


\section{\ourenv{}}
\label{app:attributes}
\paragraph{Attribute list} The suspects in \ourenv{} are defined by a collection of different attributes. These attributes are modular and can be swapped as need arises. In our experiments, we used 12 attributes for the asymmetric variant and 20 attributes for the symmetric variant. This was done to create a challenging yet feasible task for \ourenvsym{}, which is easier. The complete list of attributes and their possible values is given in Tab.~\ref{tab:attributes}: 




\begin{table}[t]
  \centering
  \footnotesize 
  \begin{tabular}{lccc}
    \toprule
    \textbf{Attribute} & \textbf{Values} \\
    \midrule
    hat & brown, black \\
    mood & happy, sad \\
    shirt color & pink, green \\
    hobby & basketball, dancing \\
    pants & long, short \\
    pants color & brown, black \\
    eye color & blue, brown, green \\
    eye glasses & circular, square \\
    shirt & button-up, tee \\
    shoe color & red, white \\
    hair & long, short \\
    watch & bronze, silver \\
    \midrule
    socks & dotted, white \\
    jacket & yellow, jean \\
    height & short, tall \\
    age & young, old \\
    build & medium, muscular \\
    personality & introverted, extroverted \\
    interests & sports, arts \\
    occupation & professional, student \\
    \bottomrule
  \end{tabular}

  \caption{\textbf{List of attributes used in the \ourenv{} variants experiments}. The top half appears in experiments of both variants, while the bottom one only appears in the symmetric variant experiments. This list is completely modular and can be changed to affect task complexity.}
  \label{tab:attributes}
\end{table}



\section{Experiments}
\subsection{Data Splits}
\label{app:datasplits}
For \ourenv{} we define games by suspect attributes. For \govsim{} we make use of the starting resource, $R_0$. For Train and validation we shuffle and split $\{105 + 5k | k\in [0, 19]\}$ into two exclusive parts. Test is simply $R_0\in \{[100], [210,300]\}$. We make sure $R_0=100$ stays in the test set to align with results from \cite{piatti2024cooperate}.

\subsection{Monitor Design}
\label{app:classifiers}
The monitor is made up of a classifier function together with a threshold.
To train the classifier we first split the data set into a train set and a validation set. We use the training set to min-max normalize the data to the [-1, 1] range, then create polynomial features over input features with degree D, and finally fit a Ridge function over the resulting polynomials. These are done using the scikit-learn library  \cite{scikit-learn}. We do this for $D\in[1,2,3,4,5]$ and for every combination of the 3 features. Turn count is always included in the model input.
To find the optimal degree, feature combination and final classifier threshold $\tau \in [0,1]$, we use the validation set. 

For every game, we use the monitor to test triggers against the golden task success labels from the run. We consider the monitor successful on a game if even one sample of a failed game is over the threshold, since that would trigger a reset. We consider the monitor unsuccessful on a game if either the classifier was never over the threshold for a failed game, or was over the threshold for any turn on a successful game. Finally, we pick the monitor which had the maximum \textbf{gain} over the validation set. This would be the monitor we use for testing. See Tab.~\ref{tab:classifiers_used} for monitors used in \ourenvasym{} and their predicted gain. \S\ref{app:triggers} shows a histogram of monitor turn triggers in a game without interventions.

For baselines, we used a random monitor with a set probability of triggering $p$. We applied the monitor to the validation set and chose $p\in[0,1]$ that maximized gains.

\begin{table}[t]
\centering
\setlength\tabcolsep{3.8pt}
\footnotesize
  \begin{tabular}{lccccc}
    \toprule
    \textbf{Model} & \textbf{Type} & \textbf{Features} & \textbf{Degree} & \textbf{Gain} & $\tau$\\
    \midrule
    \textsc{Qwen-2.5} & Accuser & Var & 1 & 5.8 & 0.51\\
    \hspace{0.5cm}\textsc{-72B} & Intel & Ent & 1 & 4.6 & 0.36 \\
    \midrule
    \textsc{Llama-3.1} & Accuser & Var Kur & 5 & 11.6 & 0.55\\
    \hspace{0.5cm}\textsc{-70B} & Intel & Ent Var & 4 & 11.3 & 0.5 \\  
    \midrule
    \multirow{2}{*}{\textsc{\gpt{}}} & Accuser & Ent & 3 & 4.9 & 0.37 \\
    & Intel & Ent Var & 5 & 5.3 & 0.43\\
    \bottomrule
  \end{tabular}
  \caption{\textbf{\ourenvasym{} best monitors}. Best monitors for different models had different features and degrees selected.}
  \label{tab:classifiers_used}
\end{table}



\begin{table}[t]
\centering
\setlength\tabcolsep{4pt}
\footnotesize
  \begin{tabular}{lp{4.5cm}}
    \toprule
    \textbf{Model} & \textbf{Full name} \\
    \midrule
    \textsc{\qwen{}} & Qwen/Qwen2.5-72B-Instruct \\
    \textsc{\llama{}} & meta-llama/Llama-3.1-70B-Instruct \\
    \textsc{\gpt{}} & gpt-4o-2024-08-06 \\
    \textsc{\qwenlarge{}} & Qwen/Qwen1.5-110B-Chat-GPTQ-Int4 \\
    \bottomrule
  \end{tabular}
  \caption{\textbf{Full model names used across experiments.}}
  \label{tab:models_used}
\end{table}



\subsection{Models}
\label{app:models}
For the open-source models, we required at most 320GB of GPU memory to load the model and run the experiments. A load of 180 games took on average 6 hours. Using \gpt{} took about one hour for 180 games, and cost about 30\$.
When using \govsim{}, \qwenlarge{} required 160GB of total GPU memory, and each game took about 3 hours of run time (depending on how long the agents last in the environment). \gpt{} took between one and two hours per game, and cost about ~8\$ per a game. See table \ref{tab:models_used} for specific model names (HuggingFace for open weights). We make use of the vLLM library for open source model loading \cite{kwon2023efficient} and query them with default parameters. 


\subsection{Full Experimental Results}
\label{app:fullwhodunit}
Due to the high variance we encountered during runs, we run a total of 180 test games for each variant of \ourenv{}. Due to cost, we only run \gpt{} three times on each game. The full results of each experiment, including \success{} and \precision{} can be seen at Tab.~\ref{tab:asymmresults} for the asymmetric variant and at Tab.~\ref{tab:symmresults} for the symmetric. 

\begin{table}
\centering
  \small
  \begin{tabular}{|c | c c c|}
    \toprule
    \multirow{2}{*}{\textbf{Model}} & \multirow{2}{*}{\textbf{Int. \#}} & \textbf{Success{}} & \multirow{2}{*}{\textbf{\precision{}}} \\
     &  & \textbf{-Rate{}} &  \\
    
\midrule
\multirow{16}{*}{\qwen{}} & \multirow{4}{*}{0} & 57.8 & 73.8 \\
 & & 62.8 & 79.0 \\
 & & 58.9 & 81.5 \\
 & & 60.0 & 77.1 \\
\cline{2-4}
 & \multirow{4}{*}{1} & 67.8 & 78.2 \\
 & & 73.9 & 87.5 \\
 & & 66.7 & 79.5 \\
 & & 72.2 & 81.2 \\
\cline{2-4}
 & \multirow{4}{*}{2}  & 67.2 & 72.9 \\
 & & 72.8 & 77.5 \\
 & & 76.7 & 80.7 \\
 & & 72.2 & 77.4 \\
\cline{2-4}
 & \multirow{4}{*}{Baseline} & 58.9 & 72.6 \\
 & & 63.9 & 76.7 \\
 & & 63.9 & 81.6 \\
 & & 63.3 & 78.1 \\
\midrule
\multirow{16}{*}{\llama{}} & \multirow{4}{*}{0} & 56.1 & 73.7 \\
 & & 51.1 & 69.2 \\
 & & 50.6 & 78.4 \\
 & & 56.7 & 78.5 \\
\cline{2-4}
 & \multirow{4}{*}{1} & 55.6 & 71.4 \\
 & & 66.1 & 75.8 \\
 & & 62.2 & 74.2 \\
 & & 73.3 & 82.0 \\
\cline{2-4}
 & \multirow{4}{*}{2} & 69.4 & 77.2 \\
 & & 67.8 & 75.8 \\
 & & 71.1 & 77.6 \\
 & & 67.8 & 77.2 \\
\cline{2-4}
 & \multirow{4}{*}{Baseline} & 53.9 & 75.2 \\
 & & 57.8 & 72.2 \\
 & & 54.4 & 74.2 \\
 & & 52.8 & 72.0 \\
\midrule
\multirow{12}{*}{\gpt{}} & \multirow{3}{*}{0} & 63.3 & 63.7 \\
 & & 61.7 & 61.7 \\
 & & 61.1 & 61.5 \\
\cline{2-4}
 & \multirow{3}{*}{1} & 70.0 & 70.4 \\
 & & 67.2 & 68.4 \\
 & & 67.8 & 67.8 \\
\cline{2-4}
 & \multirow{3}{*}{2} & 70.9 & 66.1 \\
 & & 69.8 & 65.0 \\
 & & 67.0 & 62.6 \\
\cline{2-4}
 & \multirow{3}{*}{Baseline} & 61.7 & 61.7 \\
 & & 63.3 & 63.3 \\
 & & 62.8 & 62.8 \\
    
    \bottomrule
  \end{tabular}

  \caption{\textbf{List of different experiments done with \ourenvasym{}} comparing normal running, using our method for one and two resets, and a baseline random classifier that can reset once. Each of these is an average over the exact same 180 games, which were defined with 10 suspects. For each run we bring the \success{} and \precision{}.}
  \label{tab:asymmresults}
\end{table}

\begin{table}
\centering
  \small
  \begin{tabular}{|c| c c c|}
    \toprule
    \multirow{2}{*}{\textbf{Model}} & \multirow{2}{*}{\textbf{Int. \#}} & \textbf{Success{}} & \multirow{2}{*}{\textbf{\precision{}}} \\
     &  & \textbf{-Rate{}} &  \\
    
    \midrule
    \multirow{16}{*}{\qwen{}} & \multirow{4}{*}{0} & 63.3 & 64.0 \\
     & & 58.3 & 58.3 \\
     & & 62.8 & 63.5 \\
     & & 61.7 & 62.0 \\
    \cline{2-4}
    
     & \multirow{4}{*}{1} & 62.2 & 62.9 \\
     & & 70.0 & 70.4 \\
     & & 65.6 & 65.9 \\
     & & 66.7 & 67.0 \\
    \cline{2-4}
    
     & \multirow{4}{*}{2} & 71.1 & 71.9 \\
     & & 69.4 & 69.4 \\
     & & 61.7 & 62.4 \\
     & & 61.7 & 61.7 \\
    \cline{2-4}
    
     & \multirow{4}{*}{Baseline} & 67.2 & 67.2 \\
     & & 66.7 & 67.0 \\
     & & 61.7 & 62.0 \\
     & & 61.1 & 61.5 \\
    
    \midrule
    \multirow{16}{*}{\llama{}} & \multirow{4}{*}{0} & 53.9 & 57.7 \\
     & & 55.6 & 59.9 \\
     & & 57.2 & 61.3 \\
     & & 58.9 & 63.5 \\
    \cline{2-4}
    
     & \multirow{4}{*}{1} & 61.7 & 64.5 \\
     & & 66.7 & 71.9 \\
     & & 62.8 & 64.9 \\
     & & 62.8 & 64.9 \\
    \cline{2-4}
    
     & \multirow{4}{*}{2} & 63.9 & 65.0 \\
     & & 63.3 & 65.5 \\
     & & 58.9 & 59.9 \\
     & & 61.1 & 63.6 \\
    \cline{2-4}
    
     & \multirow{4}{*}{Baseline} & 60.0 & 65.1 \\
     & & 62.2 & 66.7 \\
     & & 56.1 & 61.6 \\
     & & 60.6 & 62.3 \\
    
    \midrule
    
    \multirow{12}{*}{\gpt{}} & \multirow{3}{*}{0} & 65.0 & 65.0 \\
     & & 70.0 & 70.0 \\
     & & 66.7 & 66.7 \\
    \cline{2-4}
    
     & \multirow{3}{*}{1} & 68.3 & 68.3 \\
     & & 75.0 & 75.0 \\
     & & 70.6 & 70.6 \\
    \cline{2-4}
    
     & \multirow{3}{*}{2} & 68.9 & 68.9 \\
     & & 72.2 & 72.2 \\
     & & 68.9 & 68.9 \\
    \cline{2-4}
     & \multirow{3}{*}{Baseline} & 69.4 & 69.4 \\
     & & 67.2 & 67.2 \\
     & & 65.6 & 65.6 \\
    \bottomrule
  \end{tabular}

  \caption{\textbf{List of different experiments done with \ourenvsym{}}  using our method, including no resets and baseline random classifiers. Each of these is an average over the exact same 180 games, which were defined with 20 suspects. For each run we bring the \success{} and \precision{}. }
  \label{tab:symmresults}
\end{table}

\section{Analysis}
\subsection{Monitor trigger turns}
\label{app:triggers}
Fig.~\ref{fig:triggers} shows a histogram of the turn count for a monitor trigger on \ourenvasym{}, across all 3 models and all experiments. We see that triggers when the game ended up succeeding usually happen early. We hypothesize these are a result of a communication mistake that agents manged to recover from. On the other hand, communication mistakes later in the conversation often lead to task failure. 20\% of games failed without a monitor trigger, leaving us further room for monitor improvement and gain.

\begin{figure}[t]
\centering
  \includegraphics[width=\columnwidth]{latex/figures/triggers_asym.pdf}
  \caption{\textbf{Histogram of monitor triggers} for the \ourenvasym{} test set using \gpt{}, normalized by the 180 game count.}
  \label{fig:triggers}
\end{figure}


\section{\ourenv{} Prompts}
\label{app:prompts}
We make here available the full text of \ourenv{} prompts.
For \ourenvsym{} this includes System prompts for all agents (Fig.~\ref{fig:sym_sysprompts}), user prompts (Fig.~\ref{fig:sym_userprompts}), and message formats (Fig.~\ref{fig:sym_formats}).
For \ourenvasym{} this includes System prompts for Accuser (Fig.~\ref{fig:asym_sysprompts}) and Intel (Fig.~\ref{fig:asym_sysprompts_intel}), User prompts (Fig.~\ref{fig:asym_userprompts}) and message formats (Fig.~\ref{fig:asym_formats}).

\begin{figure*}[!ht]
    \centering
    \begin{tcolorbox}[colback=gray!10,arc=0pt,outer arc=0pt,boxrule=0pt,toprule=0.5mm,bottomrule=0.5mm,rightrule=0.5mm,leftrule=0.5mm] 
        \small
        \begin{minipage}{\linewidth} 
        
            \begin{tcolorbox}[colback=blue!5,
            colframe=blue!50,arc=0pt,outer arc=0pt,boxrule=0pt,toprule=0.5mm,bottomrule=0.5mm,rightrule=0.5mm,leftrule=0.5mm, title=System Prompt,coltitle=black]
                \textbf{Introduction:} \\
                - Your name is NAME, and you are a helpful AI assistant participating in a cooperative game with other players. There are \texttt{AGENT\_COUNT} players in total: \texttt{AGENT\_NAMES}. \\

                \textbf{Game Overview:} \\
                - The game involves characters, one of which is the Winner and will receive an award! Each player has information about all characters and partial knowledge about who the Winner is. \\
                - You will receive information in the following format: \\
                \begin{enumerate}
                    \item A list of characters with their descriptions.
                    \item A list of starting facts about the Winner, describing specific properties. These are only visible to you.
                    \item The current communication channel between you and other players, containing shared facts.
                \end{enumerate}

                \textbf{Game Mode:} \\
                - The game proceeds in rounds. In each round, you can either award the prize to a character you believe is the Winner, share one of your starting facts or skip your turn. \\
                - If there is only one character matching the criteria according to your information, you should award them the prize. Otherwise, if there's a fact that adds to your team's understanding, you should share a fact. If you believe you have nothing to add, you can skip your turn. \\
                - The goal is to find the correct Winner in the minimum number of rounds. Awarding an incorrect Winner will lose points. \\
                - \textbf{Output Format in Game Mode:} \\
                \begin{verbatim}
{
  "thoughts": "Thoughts describing your optimal action at this point. 
  Describe whether to share a fact, award a character or skip your turn,
  and why you chose this action.",
  "action": 1, 2 or 3,  // 1 to share a fact, 2 to award a character 
  or 3 to skip your turn
  "fact": If sharing, fact number to share. This must be a number.
  "character": If accusing, character number of accuse. This must be a number.
}
                \end{verbatim}

                \textbf{Instructions Recap:} \\
                - Focus on minimizing rounds to find the Winner. \\
                - Share informative facts to narrow down the possibilities. \\
                - Award a character only when sure of their identity as the Winner. \\
                - Follow the specified JSON format for output in Game mode. \\

                \texttt{SUSPECT\_INFORMATION}
            \end{tcolorbox} % End of the inner tcolorbox
        \end{minipage}
    \end{tcolorbox}
    \caption{\textbf{
    \ourenvsym{} system prompts.}}
    \label{fig:sym_sysprompts}
\end{figure*}

\begin{figure*}[!ht] % Changed to figure*
    \centering
    \begin{tcolorbox}[colback=gray!10,arc=0pt,outer arc=0pt,boxrule=0pt,toprule=0.5mm,bottomrule=0.5mm,rightrule=0.5mm,leftrule=0.5mm] % Adjust colors and rules as needed
        \small
        \begin{minipage}{\linewidth} % Removed minipage with specific width
            \begin{tcolorbox}[colback=blue!5,
            colframe=blue!50, arc=0pt,outer arc=0pt,boxrule=0pt,toprule=0.5mm,bottomrule=0.5mm,rightrule=0.5mm,leftrule=0.5mm, title=User prompt,coltitle=black]
                You received the following starting facts about the Winner (these are the facts you may share): \\
                \texttt{FACTS} \\

                The current communication channel between you and the other players is: \\
                \texttt{COMM\_CHANNEL} \\

                The current turn is \texttt{TURN\_COUNT} out of \texttt{MAX\_TURN\_COUNT}. \\
                It’s your turn. What would you do?
            \end{tcolorbox} % End of the inner tcolorbox
        \end{minipage}
    \end{tcolorbox}
    \caption{\textbf{
    \ourenvsym{} user prompts.}}
    \label{fig:sym_userprompts}
    
\end{figure*}

\begin{figure*}[!ht] % Changed to figure*
    \centering
    \begin{minipage}{\linewidth} % Spanning both columns
        \begin{tcolorbox}[colback=gray!10,arc=0pt,outer arc=0pt,boxrule=0pt,toprule=0.5mm,bottomrule=0.5mm,rightrule=0.5mm,leftrule=0.5mm] % Adjust colors and rules as needed
            \begin{minipage}{\linewidth}
                \begin{tcolorbox}[colback=blue!5,
            colframe=blue!50,arc=0pt,outer arc=0pt,boxrule=0pt,toprule=0.5mm,bottomrule=0.5mm,rightrule=0.5mm,leftrule=0.5mm, title=User Prompt and Messages,coltitle=black]

                     \begin{tcolorbox}[colback=white!5,
            arc=0pt,outer arc=0pt,boxrule=0pt,toprule=0.2mm,bottomrule=0.2mm,rightrule=0.2mm,leftrule=0.2mm]
                You received the following starting facts about the Winner (these are the facts you may share): \\
                \texttt{FACTS} \\

                The current communication channel between you and the other players is: \\
                \texttt{COMM\_CHANNEL} \\

                The current turn is \texttt{TURN\_COUNT} out of \texttt{MAX\_TURN\_COUNT}. \\
                It’s your turn. What would you do?
            \end{tcolorbox}
            
                    \begin{tcolorbox}[colback=white,arc=0pt,outer arc=0pt,boxrule=0.2mm,toprule=0.2mm,bottomrule=0.2mm,rightrule=0.2mm,leftrule=0.2mm]
                        Player \texttt{NAME} has decided to share a fact about the Winner: \texttt{FACT}.
                    \end{tcolorbox}

                    \begin{tcolorbox}[colback=white,arc=0pt,outer arc=0pt,boxrule=0.2mm,toprule=0.2mm,bottomrule=0.2mm,rightrule=0.2mm,leftrule=0.2mm]
                        The winner is \texttt{VALUE}.
                    \end{tcolorbox}
                

                \end{tcolorbox} % End of the outer tcolorbox
            \end{minipage}
        \end{tcolorbox}
            \caption{\textbf{
    \ourenvsym{} User prompt, message and fact formats.}}
    \label{fig:sym_formats}
    \end{minipage}
\end{figure*}

\begin{figure*}[!ht]
    \centering
    \begin{tcolorbox}[colback=gray!10,arc=0pt,outer arc=0pt,boxrule=0pt,toprule=0.5mm,bottomrule=0.5mm,rightrule=0.5mm,leftrule=0.5mm] % Adjust colors and rules as needed
    \small
        \begin{minipage}{\linewidth} % Removed minipage with specific width
            \begin{tcolorbox}[colback=blue!5,
            colframe=blue!50,arc=0pt,outer arc=0pt,boxrule=0pt,toprule=0.5mm,bottomrule=0.5mm,rightrule=0.5mm,leftrule=0.5mm, title=Accuser System Prompt,coltitle=black]
                \textbf{Introduction:} \\
                - Your name is NAME, and you are a helpful AI assistant participating in a cooperative game with another player: \texttt{PARTNER\_NAME}. \\

                \textbf{Game Overview:} \\
                - The game involves \texttt{SUSPECT\_NUM} characters, identified by numbers 1 to \texttt{SUSPECT\_NUM}. One of the characters has been selected as the Winner and will be awarded a prize. \\
                - You and your partner have access to different information. You know the Winner's description, while your partner knows the description of every character but doesn't know which one is the Winner. \\
                - There is only one Winner, and only that character will exactly fit the Winner's description. \\
                - You will receive information in the following format: \\
                \begin{enumerate}
                    \item A list of properties about the Winner.
                    \item The current communication channel between you and your partner, containing shared properties.
                \end{enumerate}

                \textbf{Game Modes:} \\

                \textbf{1. Game Mode:} \\
                - The game proceeds in rounds. In each round, you can either award the prize to a character, making them the Winner, or ask your partner for information. \\
                - If there is only one character matching the criteria according to your information, you should award them the prize. Otherwise, ask your partner for the most helpful information. \\
                - You can ask for information in two ways: \\
                \begin{enumerate}
                    \item Ask if a specific character has a specific value for a property, and get a boolean answer. You must include both a property and a value for this action.
                    \item Ask for general info, receiving a property and value combination, and a list of all characters that fit that combination. Your partner will decide what property to share.
                \end{enumerate}

                - The goal is to find the correct Winner in a minimum number of rounds. Once you award a character, the game will end. \\
                - \textbf{Output Format in Game Mode:} \\
                \begin{verbatim}
{
 "thoughts": "Thoughts describing your optimal action at this point. 
 Describe whether to request information or award a character, and why you chose this action.",
 "action": 1, 2 or 3, // 1 to request specific information, 2 to request general information,
 and 3 to award a character
 "character": Character id to request information about or award, depending on your action. 
 This must be a number.,
 "property": "Relevant if action is 1. The property to ask for information about.",
 "value": "Relevant if action is 1. Value of the property that is asked about"
}
                \end{verbatim}

                \textbf{2. Analysis Mode:} \\
                - In this mode, you will answer questions about the task. Rounds do not count towards the goal, and you will not request or award. \\
                - A different output format will be provided. \\

                \textbf{Instructions Recap:} \\
                - Focus on minimizing rounds to find the Winner. \\
                - Award as soon as you are sure of who the Winner is. \\
                - Request informative facts to narrow down the possibilities. \\
                - Follow the specified JSON format for output in Game mode. \\
                - Use your knowledge to eliminate characters who cannot be the Winner and to narrow down the possibilities. \\

                Possible attributes for Characters are: \texttt{CHARACTER\_ATTRIBUTES} \\

                The Winner is described as: \\
                \texttt{WINNER\_DESCRIPTION}
            \end{tcolorbox} % End of the inner tcolorbox
        \end{minipage}
    \end{tcolorbox}
    \caption{\textbf{
    \ourenvasym{} accuser system prompt.}}
    \label{fig:asym_sysprompts}
\end{figure*}

\begin{figure*}[!ht] % Changed to figure*
    \centering
    \begin{tcolorbox}[colback=gray!10,arc=0pt,outer arc=0pt,boxrule=0pt,toprule=0.5mm,bottomrule=0.5mm,rightrule=0.5mm,leftrule=0.5mm] % Adjust colors and rules as needed
    \small
        \begin{minipage}{\linewidth} % Removed minipage with specific width
            \begin{tcolorbox}[colback=blue!5,
            colframe=blue!50,arc=0pt,outer arc=0pt,boxrule=0pt,toprule=0.5mm,bottomrule=0.5mm,rightrule=0.5mm,leftrule=0.5mm, title=Intel System Prompt,coltitle=black]
                \textbf{Introduction:} \\
                Your name is \texttt{NAME}, and you are a helpful AI assistant participating in a cooperative game with another player, \texttt{PARTNER\_NAME}. \\

                \textbf{Game Overview:} \\
                The game involves characters, with one character identified as the Winner. \\
                You have descriptions for all characters but do not know which one is the Winner. \\
                Beth has the description of the Winner, but not the character descriptions. \\
                You must help your partner discover which of the characters is the winner. \\
                You will receive information in the following format: \\
                \begin{enumerate}
                    \item A list of characters with their descriptions.
                    \item The current communication channel between you and Beth, containing shared properties.
                \end{enumerate}

                \textbf{Game Modes:} \\

                \textbf{1. Game Mode:} \\
                The game proceeds in rounds. In each round, you will receive a request from your partner. \\
                The request would either be a Boolean Question about one of the characters or a request for a Broad Message. \\
                Boolean question: Given the property, value and one possible character, answer whether that combination is true or false. \\
                Broad Message: Share the value of a single property and a list of characters that have that value for the property. \\
                Where sharing a broad message, you choose what property to share. Try sharing the most efficient property and value combination. \\
                The goal is to identify the Winner in the minimum number of rounds. \\

                \textbf{Output Format in Game Mode:} \\
                \begin{verbatim}
{
"thoughts": "Describe your optimal action at this point. 
Explain what is your response and why.",
"action": 1 or 2 // 1 to answer a request, 2 for a broad message,
"value": If the action is 1, this should be a boolean True/False value. 
If the action is 2, this should be a combination of a property name and a value separated by a -:
property-value
"character": "Relevant only if action is 2. 
List of all characters who have the property and value combination shared."
}
                \end{verbatim}

                \textbf{2. Analysis Mode:} \\
                In this mode, you will answer questions about the task. \\
                Rounds in this mode do not count towards the goal, and you will not make requests or accusations. \\
                A different output format will be provided in this mode. \\

                \textbf{Instructions Recap:} \\
                Focus on minimizing rounds to identify the Winner. \\
                Fulfill Beth's requests or broadcast your own messages to narrow down possibilities. \\
                Follow the specified JSON format for output in Game Mode. \\
                Possible attributes for Characters are: \texttt{CHARACTER\_ATTRIBUTES} \\

                Following is information about the Characters: \\
                \texttt{SUSPECT\_INFORMATION}
            \end{tcolorbox} % End of the inner tcolorbox
        \end{minipage}
    \end{tcolorbox}
        \caption{\textbf{
    \ourenvasym{} Intel system prompts.}}
    \label{fig:asym_sysprompts_intel}
\end{figure*}

\begin{figure*}[!ht] % Changed to figure*
    \centering
    \begin{minipage}{\linewidth} % Spanning both columns
        \begin{tcolorbox}[colback=yellow!10,arc=0pt,outer arc=0pt,boxrule=0pt,toprule=0.5mm,bottomrule=0.5mm,rightrule=0.5mm,leftrule=0.5mm] % Adjust colors and rules as needed
            \begin{minipage}{\linewidth}
                \begin{tcolorbox}[colback=blue!5,
            colframe=blue!50,arc=0pt,outer arc=0pt,boxrule=0pt,toprule=0.5mm,bottomrule=0.5mm,rightrule=0.5mm,leftrule=0.5mm, title=User Prompt,coltitle=black]

                    The current communication channel between you and the other players is: \\
                    \texttt{COMMUNICATION\_CHANNEL} \\

                    The current turn is \texttt{TURN\_COUNT} out of \texttt{MAX\_TURN\_COUNT} \\
                    It’s your turn. What would you do?

                \end{tcolorbox} % End of the outer tcolorbox
            \end{minipage}
        \end{tcolorbox}
                    \caption{\textbf{
    \ourenvasym{} User Prompt, which is identical between Accuser and Intel.}}
    \label{fig:asym_userprompts}
    \end{minipage}
\end{figure*}

\begin{figure*}[!ht] % Changed to figure*
    \centering
    \begin{minipage}{\linewidth} % Spanning both columns
        \begin{tcolorbox}[colback=gray!10,arc=0pt,outer arc=0pt,boxrule=0pt,toprule=0.5mm,bottomrule=0.5mm,rightrule=0.5mm,leftrule=0.5mm] % Adjust colors and rules as needed
            \begin{minipage}{\linewidth}
                \begin{tcolorbox}[colback=blue!5,
            colframe=blue!50,arc=0pt,outer arc=0pt,boxrule=0pt,toprule=0.5mm,bottomrule=0.5mm,rightrule=0.5mm,leftrule=0.5mm, title=Agent Messages,coltitle=black]

                    \begin{tcolorbox}[colback=white,arc=0pt,outer arc=0pt,boxrule=0.2mm,toprule=0.2mm,bottomrule=0.2mm,rightrule=0.2mm,leftrule=0.2mm, title=Request Specific]
                        Agent \texttt{NAME} has requested information: \texttt{BOOLEAN\_QUESTION}
                    \end{tcolorbox}

                    \begin{tcolorbox}[colback=white,arc=0pt,outer arc=0pt,boxrule=0.2mm,toprule=0.2mm,bottomrule=0.2mm,rightrule=0.2mm,leftrule=0.2mm, title=Request Broad]
                        Agent \texttt{NAME} has asked for general information (a broad message)
                    \end{tcolorbox}

                    \begin{tcolorbox}[colback=white,arc=0pt,outer arc=0pt,boxrule=0.2mm,toprule=0.2mm,bottomrule=0.2mm,rightrule=0.2mm,leftrule=0.2mm, title=Specific Answer]
                        Agent \texttt{NAME} has responded that character \texttt{CHARACTER} [is | is not] \texttt{ANSWER}
                    \end{tcolorbox}

                    \begin{tcolorbox}[colback=white,arc=0pt,outer arc=0pt,boxrule=0.2mm,toprule=0.2mm,bottomrule=0.2mm,rightrule=0.2mm,leftrule=0.2mm, title=Broad message]
                        Agent \texttt{NAME} has decided to return a broad message: For characters \texttt{CHARACTER\_LIST}, the property \texttt{PROPERTY} is \texttt{ANSWER}
                    \end{tcolorbox}

                \end{tcolorbox} % End of the outer tcolorbox
            \end{minipage}
        \end{tcolorbox}
            \caption{\textbf{
    \ourenvasym{} message formats.}}
    \label{fig:asym_formats}
    \end{minipage}
\end{figure*}

\section{\ourenv{} Full Example - Hallucination}
\label{app:full_example}
For better understanding, we bring here the full text of the hallucination example, shown in \S\ref{sec:analysis}. Due to length, the example is split into 2 figures, the system prompt (Fig.~\ref{fig:ex_halluc_system}) and the user prompt and assistant answer (Fig.~\ref{fig:ex_halluc_user}).

\begin{figure*}[ht]
    \centering
    \begin{tcolorbox}[colback=gray!5,colframe=gray!50,title={Hallucination example - System prompt. Culprit is character 3}]
    \scriptsize
    \begin{verbatim}
System: 
**Introduction:**
- Your name is `Beth`, and you are a helpful AI assistant participating in a cooperative game with another player: Alex.

**Game Overview:**
- The game involves 10 characters, identified by numbers 1 to 10.
One of the characters has been selected as the Winner and will be awarded a prize.
- You and your partner have access to  different information.
   You know the Winner's description, while your partner knows the description of every character
   but doesn’t know which one is the Winner.
- There is only one Winner, and only that character will exactly fit the Winner’s description.
- You will receive information in the following format:
  1. A list of properties about the Winner.
  2. The current communication channel between you and your partner, containing shared properties.

**Game Modes:**

1. **Game Mode:**
- The game proceeds in rounds. In each round, you can either award the prize to a character,
making them the Winner, or ask your partner for information.
- If there is only one character matching the criteria according to your information, you should award them the prize.
  Otherwise, ask your partner for the most helpful information.
- You can ask for information in two ways:
  1. Ask if a specific character has a specific value for a property, and get a boolean answer.
    You must include both a property and a value for this action.
  2. Ask for general info, receiving a property and value combination, and a list of all characters that fit that combination.
    Your partner will decide what property to share.

- The goal is to find the correct Winner in a minimum number of rounds. Once you award a character, the game will end.
- **Output Format in Game Mode:**
```json
{
 "thoughts": "Thoughts describing your optimal action at this point.
   Describe whether to request information or award a character, and why you chose this action.",
 "action": 1, 2 or 3, // 1 to request specific information, 2 to request general information, and 3 to award a character
 "character": Character id to request information about or award, depending on your action. This must be a number.,
 "property": "Relevant if action is 1. The property to ask for information about.",
 "value": "Relevant if action is 1. Value of the property that is asked about"
}
```

**Instructions Recap:**
- Focus on minimizing rounds to find the Winner.
- Award as soon as you are sure of who the Winner is.
- Request informative facts to narrow down the possibilities.
- Follow the specified JSON format for output in Game mode.
- Use your knowledge to eliminate characters who cannot be the Winner and to narrow down the possibilities.
Possible attributes for Characters are: eye_color, eye_glasses, hair, hat, hobby, mood,
  pants, pants_color, shirt, shirt_color, shoe_color, watch.

The Winner is described as:
their eye color is blue, they have circular eye glasses, their hair is long, is wearing a brown hat, their hobby is basketball,
  is in a sad mood, they have red shoes, they have a bronze watch,
  they have short, black pants and a pink, button-up shirt.

The current communication channel between you and the other players is:
Agent Beth has requested information: is property eye_glasses of character 1 circular?
Agent Alex has responded that Character 1 is not wearing circular eye glasses
Agent Beth has asked for general information (a broad message)
Agent Alex has decided to return a broad message: For characters: [1, 3, 5, 8, 9], the property hobby is basketball
Agent Beth has requested information: is property eye_glasses of character 3 circular?
Agent Alex has responded that Character 3 is wearing circular eye glasses
Agent Beth has requested information: is property eye_color of character 3 blue?
Agent Alex has decided to return a broad message: For characters: [1, 2, 4, 5, 6, 7, 9, 10], the property eye_color is brown
Agent Beth has asked for general information (a broad message)
Agent Alex has decided to return a broad message: For characters: [1, 5, 8, 10], the property hat is black
Agent Beth has requested information: is property eye_color of character 3 blue?
Agent Alex has decided to return a broad message: For characters: [3, 8], the property eye_color is blue
Agent Beth has requested information: is property hat of character 3 brown?
Agent Alex has decided to return a broad message: For characters: [2, 3, 6, 10], the property eye_glasses is circular
Agent Beth has requested information: is property hat of character 3 brown?
Agent Alex has decided to return a broad message: For characters: [1, 2, 3, 4, 5, 6, 7, 9], the property shoe_color is red
Agent Beth has requested information: is property hat of character 3 brown?
Agent Alex has responded that Character 3 is wearing a brown hat
Agent Beth has requested information: is property hair of character 3 long?
Agent Alex has decided to return a broad message: For characters: [1, 4, 8, 9], the property mood is happy
Agent Beth has requested information: is property hair of character 3 long?
Agent Alex has responded that Character 3's hair is long
Agent Beth has requested information: is property mood of character 3 sad?
Agent Alex has responded that Character 3 is not in a sad mood

The current turn is 25 out of 31. It's your turn. What would you do?
    \end{verbatim}
    \end{tcolorbox}
    \caption{\textbf{Hallucination} example - System prompt.}
    \label{fig:ex_halluc_system}
\end{figure*}

\begin{figure*}[ht]
    \centering
    \begin{tcolorbox}[colback=gray!5,colframe=gray!50,title={Hallucination example - User prompt and answer. Culprit is character 3}]
    \scriptsize
The current communication channel between you and the other players is: \\
Agent Beth has requested information: is property eye\_glasses of character 1 circular? \\
Agent Alex has responded that Character 1 is not wearing circular eye glasses \\
Agent Beth has asked for general information (a broad message) \\
Agent Alex has decided to return a broad message: For characters: [1, 3, 5, 8, 9], the property hobby is basketball \\
Agent Beth has requested information: is property eye\_glasses of character 3 circular? \\
Agent Alex has responded that Character 3 is wearing circular eye glasses \\
Agent Beth has requested information: is property eye\_color of character 3 blue? \\
Agent Alex has decided to return a broad message: For characters: [1, 2, 4, 5, 6, 7, 9, 10], the property eye\_color is brown \\
Agent Beth has asked for general information (a broad message) \\
Agent Alex has decided to return a broad message: For characters: [1, 5, 8, 10], the property hat is black \\
Agent Beth has requested information: is property eye\_color of character 3 blue? \\
Agent Alex has decided to return a broad message: For characters: [3, 8], the property eye\_color is blue \\
Agent Beth has requested information: is property hat of character 3 brown? \\
Agent Alex has decided to return a broad message: For characters: [2, 3, 6, 10], the property eye\_glasses is circular \\
Agent Beth has requested information: is property hat of character 3 brown? \\
Agent Alex has decided to return a broad message: For characters: [1, 2, 3, 4, 5, 6, 7, 9], the property shoe\_color is red \\
Agent Beth has requested information: is property hat of character 3 brown? \\
Agent Alex has responded that Character 3 is wearing a brown hat \\
Agent Beth has requested information: is property hair of character 3 long? \\
Agent Alex has decided to return a broad message: For characters: [1, 4, 8, 9], the property mood is happy \\
Agent Beth has requested information: is property hair of character 3 long? \\
Agent Alex has responded that Character 3's hair is long \\
Agent Beth has requested information: is property mood of character 3 sad? \\
Agent Alex has responded that \textbf{Character 3 is not in a sad mood} \\

The current turn is 25 out of 31. \\
It's your turn. What would you do? \\


Answer:  '```json\{
"thoughts": "Character 3 matches most of the Winner\'s description, \textbf{but we found out they are not in a sad mood}.
We should now look for another character that matches the description of having blue eyes,
  circular glasses, long hair, a brown hat, and being in a sad mood.
Since we already know that character 3 has blue eyes, circular glasses, long hair,
  and a brown hat, we should look for another character that also has these properties \textbf{and is in a sad mood.}",
"action": 2,
"character": 0,
"property": "",
"value": ""
\}
```
I am asking for general information because we need to find another character that matches most of the Winner\'s description
  and is in a sad mood.
We can then ask more specific questions to narrow down the possibilities.'

    \end{tcolorbox}
    \caption{\textbf{Hallucination} example - User prompt and response.}
    \label{fig:ex_halluc_user}

\end{figure*}
