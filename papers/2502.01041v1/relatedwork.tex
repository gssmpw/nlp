\section{Related Work}
%

\revisedtext{
The problem of finding target objects has been addressed in the literature with a variety of  assumptions, falling in three categories, \emph{static targets}, \emph{pursuit evasion}, and \emph{search, acquisition, and tracking}. Here, we highlight how each category relate or differ with the problem and the approach we propose.}

%

\begin{figure*}[t!]
    \centering
    \includegraphics[width=\textwidth]{figs/example-large-modified-clear.pdf}
    \vspace{-3em}
    \caption{Simplified example over a small grid that shows: (1) \emph{Environment}, with two agents \revisedtext{($\ag_1$, $\ag_2$ in blue)} with their sensor ranges \revisedtext{(dashed circles)}, one object \revisedtext{($\ob$ in red)}, and one event of third-party reporting \revisedtext{($e$ in yellow circle)}; (2) \emph{Initial belief} without any prior, thus all cells unknown; (3) at time $t_0$ the shared belief after the initial sensor measurements by both agents with the third-party reporting and the frontiers \revisedtext{($f_0, f_1, f_2)$}; (4) at $t_1$ the agents move to the optimal frontier locations and get corresponding sensor measurements, where $\ag_1$ also detects an object $o$ with uncertainty \revisedtext{(red dotted circle)} and estimates its trajectory \revisedtext{(red dashed line)}; and (5) at $t_2$, the agents select their next action, resulting in $\ag_1$ achieving the tracking of the object\revisedtext{, i.e., reducing the uncertainty (green dotted circle) and clearing the target (green)}, and $\ag_2$ continuing independently on the search. \revisedtext{Note that the cells (in pink) represent the time-varying object occupancy proposed in this study where the lighter color, the more recently explored.}}
    \label{fig:example}
\end{figure*}

\textbf{Static targets.} Some work in the space of coverage~\cite{galceran2013survey}, exploration~\cite{quattrini2020exploration}, search and rescue~\cite{drew2021multi}, information gathering~\cite{bai2021information}, and source seeking~\cite{hajieghrary2017information,marjovi2010multi} seeks to discover information about features in the environment such as the location of or targets~\cite{rouvcek2020darpa}. In classical frontier-based exploration~\cite{yamauchi1998frontier}, the robot selects the closest location at the boundary between known and unknown parts of the environment---the \emph{frontier}---to proceed with exploration. Information-theoretic approaches include information gain together with the cost of visiting locations~\cite{burgard2005coordinated,basilico2011exploration}. A variation of the problem includes having competitive teams searching for a target~\cite{otte2018competitive}. Unlike our paper, these methods do not consider moving targets --- once an area is marked as explored, it is not re-visited. \revisedtext{We get inspiration from frontier-based exploration to identify areas to explore, including a time-varying component, which is not present in its vanilla version.}

\textbf{Pursuit/evasion.} Research that typically considers moving targets addresses the pursuit-evasion problem~\cite{chung2011search}.
The first body of work in pursuit-evasion studies the problem from a theoretical perspective, modeling the environment either as a graph~\cite{isler2006randomized,borie2011algorithms,kehagias2009graph} or as polygons~\cite{isler2005randomized,guibas1999visibility,quattrini2018search,stiffler2017complete}. These approaches try to find methods with worst-case theoretical guarantees  such as the number of pursuers needed to capture an evader, or the traveled distance before the evader is captured. Instead of studying the worst case, another body of work solves the problem of localizing and tracking other agents within a probabilistic optimization framework~\cite{li2021bayesian,chung2011analysis,lau2006probabilistic,matzliach2020cooperative}. Past work typically assumes that the searcher agent can rely on its own sensors, there is  a priori knowledge of where the targets might be, or there is a trail left behind, in a source-seeking problem formulation. Patrolling can be considered a dual problem to pursuit-evasion, in that an area is initially considered secure and should be protected by attacks~\cite{basilico2022recent}, where an adversarial model is assumed to be known. \revisedtext{Generally, methods from the pursuit/evasion literature might take longer trajectories to account for the adversarial behaviors. Instead,} in our paper, we assume that targets are not adversarial and that no prior knowledge is available. 
%



\textbf{Search, Acquisition, and Tracking.} Unlike  pursuit-evasion, other work studied non-adversarial but non-cooperative scenario---i.e., objects are not trying to actively evade, but are also not helping others to locate themselves. This problem is also called the  \emph{cooperative search, acquisition, and tracking problem} \cite{review-csat2018,taxonomy-review-2016,sat-swarm-review-2016} and is the focus of this paper.  
Some past efforts focus on a single task, e.g. search \cite{malika-search-2016} or tracking \cite{harvard-source-seek-2021}. 
%
Other efforts include both. 
Some resolve conflicting behaviors between search and tracking with \textit{mode switch} \cite{target-assign-csat-2008,increasing-autonomy-csat-2009,recursive-bayesian-csat-2006,search-track-cyprus-2021-journal,goldhoorn2018searching}. If agents detect a target, they switch to tracking mode to ensure the detected target's location is within an error bound. Others propose \textit{balanced} approaches, with a single objective function such as the \revisedtext{expected number of detections  \cite{vijay-sat-2017, gas-mapping-2019}}, \revisedtext{a multi-criteria optimization \cite{uav-route-csat-balance-2012,upenn-balance-2015, cell-mb-2023}}, \revisedtext{a multi-objective optimization \cite{pan-tilt-camera-2019, distributed-control-silvia-2018},} or swarm intelligence \revisedtext{\cite{blum2015swarm, optimal-swarm-2020, usv-pursuit-evasion-2023}}.  
%
These approaches consider \emph{homogeneous agents}, no available external information, and no time-varying uncertainty. Also, many approaches have been only tested in open-space environments, thereby overlooking realistic cluttered environments\revisedtext{, or considering cluttered environments but with a single agent (sensor platform) \cite{pan-tilt-camera-2019}}. \revisedtext{A time-varying 3-D coverage cone was introduced to consider a spatio-temporal space during tracking dynamic targets after detection. For the search phase, a} diffusion model was introduced to account for dynamic targets \cite{increasing-autonomy-csat-2009}. This model favors revisiting previously explored areas and maintains a map for each obstacle and updates the map using phantom obstacles. \revisedtext{To improve scalability with respect to the number of targets, we store and update only one time-varying map for exploration, without losing any information.} %
The trajectory estimation is based primarily on Kalman filter or iterative filtering approaches, which generally might fail in long-term predictions \revisedtext{\cite{intention-vehicle-prediction-2022}}, or PHD filter which might not capture complex trajectories\revisedtext{, due to its limit in distinguishing between objects \cite{phd-filter-limit-2022,vijay-sat-2017}}. 

%

 

%

%

\revisedtext{
Real-world scenarios for search and tracking include heterogeneous agents and third-party external information. Therefore, this paper investigates how to model and use effectively such information for improved search and tracking task (see \tab{literature-review} for an overview on the difference between our work and the literature).
}

 
%
%
%
%
    %
    %
    %
    %
    %
    %
    %
    %
    %
%




%