\begin{table*}[h]
\renewcommand{\arraystretch}{1.5}
    \resizebox{2\columnwidth}{!}{%


\begin{tabular}{@{}rlll@{}}
\toprule
\multicolumn{1}{l}{} & \multicolumn{1}{c}{\textbf{Close-ended}} & \multicolumn{1}{c}{\textbf{Open-ended}} & \textbf{Category} \\ \midrule
\textbf{Question} & \begin{tabular}[c]{@{}l@{}}The best way to estimate the relative strength of hydrogen\\ bonds between the molecules of halogen hydrides, H-X, is \\ by measuring:\end{tabular} & \begin{tabular}[c]{@{}l@{}}What is the best way to estimate the relative strength of\\ hydrogen bonds between the molecules of halogen\\ hydrides, H-X?\end{tabular} & \multirow{2}{*}{Chemistry} \\ \cmidrule(lr){2-3}
\textbf{Answer} & The enthalpies of vaporization & The enthalpies of vaporization. &  \\ \midrule
\textbf{Question} & \begin{tabular}[c]{@{}l@{}}Taking into account the general principles regarding the\\ minimum interval between the non-simultaneous\\ administration of vaccines, identify the minimum interval\\  between 2 attenuated vaccines:\end{tabular} & \begin{tabular}[c]{@{}l@{}}What is the minimum interval recommended between\\  the non-simultaneous administration of two attenuated \\ vaccines, according to general principles?\end{tabular} & \multirow{2}{*}{Nursing} \\ \cmidrule(lr){2-3}
\textbf{Answer} & Four weeks. & Four weeks. &  \\ \midrule
\textbf{Question} & \begin{tabular}[c]{@{}l@{}}We evaluated in the emergency room an adult person who\\ is irritable, yawning, complaining of muscle pain and\\  cramps. They are nauseous and have notable tearing. \\ The pupils are dilated. Which of the following is the \\ most probable diagnosis?\end{tabular} & \begin{tabular}[c]{@{}l@{}}An adult patient presents to the emergency room with\\ irritability, yawning, muscle pain and cramps, nausea, \\ notable tearing, and dilated pupils. What is the most \\ probable diagnosis based on these symptoms?\end{tabular} & \multirow{2}{*}{Medicine} \\ \cmidrule(r){1-3}
\textbf{Answer} & Opioid abstinence. & Opioid abstinence. &  \\ \bottomrule
\end{tabular}


}\caption{Examples of QA pairs: On the left, the close-ended version from \careqa{}, and on the right, the open-ended version.} \label{tab:careqa_examples}
\end{table*}