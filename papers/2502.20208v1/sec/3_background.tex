\section{From Implicit Surface Deformation ...}

\label{sec:background}
%\inparagraph{Time-varing Implicit Field.} 
\inparagraph{Implicit representation of the moving surface.}
% \RM{I would add a reference to \cite{anonymous2024implicit}, to clarify that these tools come mainly from there}
We represent the moving surface $\surf{t}$ in the volume $\dom\subset\mathbb{R}^3$ implicitly as the zero-crossing of a time-evolving signed distance function $\phi:\dom\times [0,T]\rightarrow\mathbb{R}$:
\eq{\label{eq:time_implicit}
\surf{t} = \{ \vt{x} \in \dom | ~ \phi(\vt{x},t) = 0\}\;.
}
This implies that
\eq{\label{eq:time_implicit}
\phi(\surf{t},t)=0 \quad\forall t.
}
It follows that the total time derivative is zero, i.e.
\eq{\label{eq:les}
\frac{\dd}{\dd t}\phi(\vt{x},t) = \partial _t \phi + \mathcal{V}^\top \nabla\phi   = 0,
}
where $\mathcal{V}=\frac{d}{dt}\surf{t}$ denotes the velocity of the moving surface.  Eq.~\eqref{eq:les} is known as the {\em level-set equation}~\cite{Dervieux-Thomasset-79,Dervieux-Thomasset-81}.  It tells us how to move the implicitly represented surface $\surf{t}$ along the vector field $\mathcal V$ by deforming the level-set function $\phi$. Since over time, the level-set function will generally lose its property of being a signed distance function, we add an Eikonal regularizer with a weight $\lambda_l$ to obtain the modified level-set equation~\cite{anonymous2024implicit}, i.e.
\eq{\label{eq:lse3}
   \partial _t \phi + \mathcal{V}^\top \nabla \phi = -\lambda_l \phi\mathcal{R}(\vt{x}, t) ~~ \text{in}~~\dom \times \I\;,
}
where $\mathcal{R}(\vt{x}, t) = - \dotp{(\nabla \mathcal{V})\frac{\nabla \phi}{\norm{\nabla \phi}}}{\frac{\nabla \phi}{\norm{\nabla \phi}}}$.
This equation combines the level-set equation and Eikonal constraint, freeing the level-set approach from the reinitialization process~\cite{anonymous2024implicit, fricke2023locally, bothe2023mathematical}.
% allows us to move the implicitly represented surface by specifying a suitable velocity field $\mathcal V$.

\comment{
The Implicit field represents the surface by its zero-crossing set.  As introduced in~\cite{anonymous2024implicit}, with additional time-space, the implicit field can express the surface deviation over different time $t$. For $\dom \subset \mathbb{R}^3$ is the point domain, the deformation surface is encoded in
\eq{\label{eq:time_implicit}
\surf{t} = \{ \vt{x} \in \dom | ~ f(\vt{x},t) = 0, t > 0.
}
% Each shape $\surf{t}$ in time $t$ is encoded by the zero-crossing of the implicit function $f$. 
function $f$, as a signed distance field, should satisfy the Eikonal equation, i.e., $\norm{\nabla f} = 1$ and it naturally equips the property that surface normal coincides with the function derivative on the zero-cross area $\vt{n}(\vt{x}) = \nabla f(\vt{x})$.

\inparagraph{Deform Implicit Field via External Velocity.} Consider a field function $\V{\vt{x},t}: \mathbb{R}^3 \to \mathbb{R}^3$ map arbitrary points to its current velocity field. 
% Given the starting and ending boundary conditions, we view the movement of each point consists a flow field.
Then we can recover the flow $\phi(\vt{x}, t)$ that describes the point cloud location via integrate the ODE
% In the Lagrangian description, the flow is described function $\phi(\vt{x}_0, t)$, given the starting point $\vt{x}_0$ and time variable $t$. To track the movement of each point, and also from the Eulerian point of view, the velocity field $\V{\vt{x},t} \in \mathbb{R}^3$ is often used. Here, the point $\vt{x}$ is the current position of the particle. One can easily link the two fields using
\eq{ \label{eq:flow_field}
\V{\vt{x},t},t) = \frac{\partial \phi(\vt{x}, t)}{\partial t} \;.
}
If the point $\vt{x}\in\dom$ is moved by the external velocity field $\mathcal{V}$ in Eq.~\eqref{eq:time_implicit}. That means at time $t$, the point is moved to $\phi(\vt{x},t)$. However, the point should still be on the surface, that is $f(\phi(\vt{x},t), t)=0$. Now we compute the partial derivatives w.r.t. $t$, then we get
\eq{
\partial _t f + \mathcal{V} \cdot \nabla f = 0\;. \label{eq:lse}
}
Eq.~\eqref{eq:lse} is called level-set equation (LES)~\cite{sussman1994level, sethian1996fast, sussman1999efficient} and is used to directly deform the implicit field via the external velocity field. Some previous methods adopt this equation when deforming on implicit field~\cite{Novello2023neural, mehta2022level}. However, the original level-set equation does not consider the natural property of the signed distance field. The work~\cite{anonymous2024implicit} proposed to use a modified level-set equation as
\begin{equation}\label{eq:lse3}
   \partial _t f + \mathcal{V} \cdot \nabla f = -\lambda_l f\mathcal{R}(\vt{x}, t) ~~ \text{in}~~\dom \times \I\;,
\end{equation}
where $\mathcal{R}(\vt{x}, t) = - \dotp{(\nabla \mathcal{V})\frac{\nabla f}{\norm{\nabla f}}}{\frac{\nabla f}{\norm{\nabla f}}}$.
Eq.~\eqref{eq:lse3} combined the implicit field, Eikonal loss, and velocity field in one equation. 
}

\inparagraph{Spatial Smoothness and Volume Preservation.} To make sure the velocity field models physical movement, we can impose respective regularizers on the velocity field. Two popular regularizers are spatial smoothness $\mathcal{L}_s$~\cite{dupuis1998, anonymous2024implicit} and volume preservation $\mathcal{L}_v$~\cite{eisenberger2018divergencefree, Cosmo2020}:
\eq{\label{eq:smoothness_volumn}
\mathcal{L}_{s} &= \int_{\dom}\norm{ (-\alpha \Delta + \gamma \vt{I}) \V{\vt{x}}, t}_{l^2} \dd \vt{x} , \\
\mathcal{L}_{v} &= \int_{\dom} | \nabla \cdot \V{\vt{x, t}}| \dd \vt{x} \;.
}


