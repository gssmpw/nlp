
% \vspace{-0.3cm}
% A number of Computer Vision problems rely on the capacity to infer continuous representations from a set of sparse and inaccurate observations. If 3D reconstruction is already challenging for single, rigid objects, an even more ambitious goal is to reconstruct a plausible motion between partial observations. In an ideal setting, this would consider unordered representations (e.g., point clouds), subject to topological changes and for which a precise dense correspondence is actually tricky to define. In this work, we address this challenging scenario by proposing a data-driven pipeline that, starting from a collection of shapes and noisy correspondences, learns both implicit and deformation fields. We do not only demonstrate superior performance w.r.t. state of the art, but also demonstrate, for the first time, that interpolation of noisy, partial, and topological changes shapes can be continuously modeled, opening to new applications like 4D Kinect sequences upsampling and complete-to-partial interpolation.
% \LS{TODO: 
% - currently we call our method 4Deform at some place. 1. do we want to keep the name? 2. Do we need a new one?
% - we need shorten some parts to fit into 8 pages.}\fb{I think 4Deform is fine}\fb{we should shorten the abstract a bit; no need for too many method details}
% Finding realistic intermediate shapes between a pair of non-rigidly deformed shapes is a compelling yet challenging task in computer vision, especially in real-world applications where acquired data is less structured (\eg, point clouds from 3D reconstructions). Non-rigid 3D reconstruction techniques are often applied frame by frame, resulting in a lack of temporal consistency between individual reconstructed shapes. Consequently, finding plausible intermediate shapes becomes difficult due to the lack of correspondence, or due to changing topology that occurs during shape deformation.
% Further, most existing interpolation methods are designed for finding realistic intermediate shapes for structured data (\ie, meshes), so that they are not applicable to real-world point clouds. In contrast, our approach utilizes neural implicit representation (NIR) to enable topology-free shape deformation. Unlike previous mesh-based methods, which only model vertex-based deformation fields on the (discrete) surface, our method learns a continuous velocity field in the Euclidean embedding space, making it applicable to less structured data, such as point clouds during inferenc. 
% Furthermore, our approach does not rely on intermediate-shape supervision during training. Instead, we incorporate physical and geometrical constraints to regularize the recovered velocity field. To reconstruct the intermediate surfaces, we employ a modified level-set equation to build a direct connection between our neural implicit representation and our velocity field. 
% The experiments demonstrate that our method significantly outperforms previous NIR-based approaches across various scenarios (noisy, partial, topology-changing, non-isometric shapes, etc.). Moreover, for the first time, our method enables many new applications such as 4D Kinect sequence upsampling and complete-to-partial shape interpolation. 
\begin{abstract}
Generating realistic intermediate shapes between non-rigidly deformed shapes is a challenging task in computer vision, especially with unstructured data (\eg, point clouds) where temporal consistency across frames is lacking, and topologies are changing. Most interpolation methods are designed for structured data (\ie, meshes) and do not apply to real-world point clouds. In contrast, our approach, 4Deform, leverages neural implicit representation (NIR) to enable free topology changing shape deformation. Unlike previous mesh-based methods that learn vertex-based deformation fields, our method learns a continuous velocity field in Euclidean space. Thus, it is suitable for less structured data such as point clouds.
Additionally, our method does not require intermediate-shape supervision during training; instead, we incorporate physical and geometrical constraints to regularize the velocity field. We reconstruct intermediate surfaces using a modified level-set equation, directly linking our NIR with the velocity field. Experiments show that our method significantly outperforms previous NIR approaches across various scenarios (\eg, noisy, partial, topology-changing, non-isometric shapes) and, for the first time, enables new applications like 4D Kinect sequence upsampling and real-world high-resolution mesh deformation.
\vspace*{-12pt}
\end{abstract}
\vspace*{-12pt}