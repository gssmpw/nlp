\section{Related Work}
\label{sec:related}
% \RM{I think the subsections "surface deformation", "shape interpolation", "Implicit field deformation" are enough.}
4D reconstruction from discrete observations involves recovering a continuous deformation space that not only aligns with the input data at specific time steps but also provides plausible intermediate reconstructions. The reconstructed sequences should preserve the input information while filling in any missing details absent in the original data, ultimately creating a complete and coherent representation of the observed sequence. This task involves shape deformation and interpolation.

\subsection{Surface Deformation}
% To model the 3D dynamic world, surface deformation is an important topic in many fields, such as gaming, simulation, and reconstruction. For a large portion of the tasks, one might need the deformation to be physically plausible, especially when humans, animals, or some rigid objects are involved. How to deform a given surface reasonably depends on the underlying representation of the source surface and target surface. 
Modeling the 3D dynamic world involves surface deformation, essential in fields like gaming, simulation, and reconstruction. Physically plausible deformations are often needed. However, the task relies heavily on the representation of the data.

\inparagraph{Mesh Deformation.} Mesh deformation typically involves directly adjusting vertex positions within a mesh pair, taking advantage of the inherent neighboring information. This allows mesh-based methods to incorporate physical constraints, such as As-Rigid-As-Possible (ARAP)~\cite{sorkine2007rigid, botsch2006primo}, area-preserving or elasticity loss~\cite{bastianxie2024hybrid} to constrain the deformation. While mesh deformation is well-studied~\cite {cao2023revisiting, cao2024motion2vecsets,eisenberger2018divergencefree, eisenberger2020hamiltonian, eisenberger2021neuromorph}, it requires predefined spatial discretization and fixed vertex connectivity to maintain topology. This leads to challenges with topology changes~\cite{cao2024spectral} and resolution limitations, as methods must process all vertices together. Consequently, techniques like LIMP~\cite{Cosmo2020} downsample meshes to 2,500 vertices, and others~\cite{cao2024spectral, roetzer2024discomatch, eisenberger2021neuromorph} re-mesh inputs to 5,000 vertices, with output resolution tied to these constraints.

\inparagraph{Implicit Field Deformation.} Unlike mesh representations, implicit surface representations offer several advantages. First, neural implicit fields allow flexible spatial resolution during inference, as discretization isn’t pre-fixed. Second, they can represent arbitrary topologies, making them well-suited for complex cases that mesh-based methods struggle with, such as noisy or partial observations. However, directly deforming an implicitly represented surface is challenging because surface properties aren’t explicitly stored, limiting direct manipulation of surface points. This area remains underexplored, with only a few approaches addressing it. For instance, NFGP~\cite{yang2021geometry} introduces a deformation field on the top of an implicit field, constraining it physically using user-defined handle points to match target shapes. This pioneering work directly deforms the implicit field; however, its practical applications are limited as it only provides the starting and ending shapes, with processing times for a single shape pair extending over several hours. Other methods, such as those in~\cite{Novello2023neural, mehta2022level}, focus on shape smoothing or morphing without targeting specific shapes. The work~\cite{anonymous2024implicit} introduced a fast, flexible framework for directly deforming the implicit field, capable of generating physically plausible intermediate shapes. However, as an optimization-based approach, it requires training separately for each shape pair and struggles with handling large deformations.

\subsection{Shape Interpolation}
% Shape interpolation is a specialized aspect of shape deformation, involving the generation of intermediate shapes between a given starting shape and a target shape. 
Shape interpolation is a specialized subset of shape deformation that involves generating intermediate shapes between a start and target shape. The interpolated sequence should reflect a physically meaningful progression,
% Typically, the sequence of shapes represents a progression with physical significance, 
making it essential that the interpolated shapes are not only geometrically accurate but also serve to complete the narrative implied by the initial and final shapes. Therefore, we emphasize the concept of \textbf{physically plausible} shape interpolation, which should be a guiding principle for tasks in this area. 
%There are two ways to solve the task. 
There are two main approaches: generative models and physics-based methods. Generative models~\cite{gan, diffusion} rely on extensive datasets to produce shapes, but their outputs can lack realism due to data dependency. Physics-based methods, like~\cite{Novello2023neural, liu2022learning, anonymous2024implicit}, optimize over a shape pair to generate intermediates but may have limited applications.
% The first involves generative models~\cite{gan, diffusion}, which learn from extensive pre-existing datasets to generate desired shapes. Such generative models often heavily rely on the quality and quantity of training data, and their outputs may not always be realistic. The second approach focuses on modeling physical movement based on provided inputs. A popular way is to use the optimization-based method, which is to optimize over a given pair to generate intermediate shapes. 
% Previous work such as~\cite{Novello2023neural, liu2022learning, anonymous2024implicit} are in this category. 
% That makes the method hard to have a border application. 
Another way is to learn a deformation space of shapes, such as using latent space~\cite{Goodfellow2016}, and hope that generating intermediate shapes is equivalent to interpolating the latent shape~\cite{Cosmo2020}. However, such methods suffer from the same problem as generative models in that the latent space interpolation may be far from physically plausible.\\
To address these limitations, we propose a lightweight solution that can be trained on datasets of any size. Our model adopts an AutoDecoder architecture to maintain generative capability and combines physical and geometric constraints to ensure the generated results are physically plausible. 

% \LS{should I mention diffusion or gan?}
% \RM{Not sure, I think you can try to have this subsection structured as follows:
% \begin{itemize}
% \item Common way to solve the problem: shape interpolation
% \item you can do that optimization based
% \item or you can learn a space of shapes, e.g., AutoEncoder, but these support only training data
% \item popular methods to obtain a better representation are generative methods, which learn a distribution support for the data manifold
% \item however, such methods are unstable, data greedy, \dots
% \end{itemize}
% }
% Shape interpolation is a more specific task under shape deformation. The task gives a starting shape and a target shape and asks for recovery of the intermediate shapes. Different than arbitrarily deforming a given surface, shape interpolation needs to recover the intermediate shapes after considering the characteristics of the starting shape and target shapes. Because the sequence of shapes normally describes a sequence of situations that has physical meaning. One would like the recovered intermediate shapes not only to be geometrically correct but also to fill the missing part of an event that is described by starting and ending shapes. Thus, we want to emphasize the \textbf{Pyshically plausible} concept that shape interpolation tasks should pay attention to. 

% \subsection{Shape Registration} 

