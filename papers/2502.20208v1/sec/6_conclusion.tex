
\section{Discussion and Future Work} While our method integrates physical plausibility, certain types of deformations, such as mechanical joints and fluid dynamics, may not yet be fully captured by our model. Future work could incorporate additional physical constraints to address these complexities. Additionally, some applications require separate deformation estimation, as in the case of a human with loose clothing, where the deformations of the body and clothing do not align. We plan to extend our work to handle these cases in future developments.



% \section{Conclusion}\label{sec:conclusion}
% In this paper, we proposed 4Deform that efficiently interpolates between pairs of inputs to generate physically plausible intermediate shapes. To this end, we propose two novel physical deformation losses that efficiently work on unordered data, such that even without connectivity information, our method can still recover intermediate shape for challenging cases. Moreover, our method can seamlessly work on data that is not aligned with training data, which broadens the real-world applications of our method. 


% we integrate some concepts of plausibility, but there are way more deformations that might still not be fully captured by our model, such as mechanical joints, fluids, ... where equations can be way more complicated... or also shapes with face multiple class of deformation (e.g., modeling the rigidity of a part, the non rigidiry of another, ...)
% Our method relies on a correspondence block to give reasonable correspondences between pairs of inputs, even though our method is robust to some level of the noise on the correspondences, it will fail if the correspondences block is failed.  