\begin{table}[t]
\footnotesize
\addtolength{\tabcolsep}{-2pt}
\begin{tabular}{lcccccc}
\specialrule{.1em}{.05em}{.05em} 
        \toprule
        Methods & Corr.Est. & Non-iso. & Part. &  Topo. & Seq. &  Real. \\
        \midrule
        SMS~\cite{cao2024spectral} & \gcmark & \gcmark & \rxmark & \rxmark & \gcmark & \rxmark \\
        Neuromorph~\cite{eisenberger2021neuromorph} & \gcmark & \gcmark & \rxmark & \rxmark & \gcmark & \rxmark \\
        LIMP~\cite{Cosmo2020}  & \rxmark & \gcmark & \rxmark & \rxmark & \gcmark & \rxmark \\
        NFGP~\cite{yang2021geometry}  & \rxmark & \rxmark & \rxmark & \gcmark & \rxmark & \rxmark \\
        NISE~\cite{Novello2023neural} & - & \rxmark & \rxmark & \gcmark & \rxmark & \rxmark \\
        LipMLP~\cite{liu2022learning} & - & \rxmark & \rxmark & \gcmark & \gcmark & \rxmark \\
        \cite{anonymous2024implicit} & \rxmark & \gcmark & \gcmark & \gcmark & \rxmark & \gcmark \\
        \textbf{Ours} & \gcmark & \gcmark & \gcmark & \gcmark & \gcmark &\gcmark \\
        \bottomrule
\end{tabular}
\caption{\textbf{Summary of Methods Capability.} We list the capabilities of previous mesh and NIR-based methods. The column Corr. Est. indicates if the method can estimate the correspondences (\gcmark) or needs ground-truth correspondences as input (\rxmark); 
% \fb{which methods, i.e. what is their task? Given that not all of them handle collections, the task seems not be temporal recosntruction...} 
If the methods can handle non-isometric deformation (Non-iso.), partial shape deformation (Part.), topological changes (Topo.), work for Sequences (Seq.), and for the real-world data (Real.).
% with training correspondences, \ie, offset data (Off.)\fb{what is offset data?}.
}
\vspace*{-2mm}
\label{tab:summary}
\end{table}
