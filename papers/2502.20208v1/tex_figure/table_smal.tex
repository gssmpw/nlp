\begin{table}[t]
\footnotesize
\addtolength{\tabcolsep}{-3pt}
\begin{tabular}{lcccc}
\specialrule{.1em}{.05em}{.05em} 
        \toprule
         & \multicolumn{4}{c}{Pairs S/S} \\
         \cmidrule{2-5} 
 & CD \tiny{$(\times 10^3)\downarrow$} & HD \tiny{$(\times 10^2)\downarrow$} & \SAstd \tiny{$(\times 10)\downarrow$} & P-RMSE \tiny{$(\times 10)\downarrow$} \\ 
\cmidrule{2-5}
% LIMP~\cite{Cosmo2020} & & & \\
NFGP~\cite{yang2021geometry} & 0.770 & 0.906 &  0.217 & \xmark\\
LipMLP~\cite{liu2022learning} & 68.452 & 43.327 & 1.192 & \xmark \\
NISE~\cite{Novello2023neural} & 7.223 & 1.237 & 0.771 & \xmark \\
\cite{anonymous2024implicit} & 0.173 & 0.626 & 0.064 & 0.081 \\
\midrule
\textbf{Ours} & \textbf{0.137} & \textbf{0.221} & \textbf{0.062} & \textbf{0.061} \\
% \textbf{Ours} (seq) & 0.163 & 0.289 & 0.066 & 0.084\\
\bottomrule
\end{tabular}
\caption{\textbf{Animal dataset isometric deformation metrics.} We show that our method achieves the best results on the SMAL dataset.}
% compared with previous methods. }
\vspace{-3mm}
\label{tab:comp_animal}
\end{table}
