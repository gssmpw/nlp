% CVPR 2025 Paper Template; see https://github.com/cvpr-org/author-kit

\documentclass[10pt,twocolumn,letterpaper]{article}

%%%%%%%%% PAPER TYPE  - PLEASE UPDATE FOR FINAL VERSION
% \usepackage{cvpr}              % To produce the CAMERA-READY version
% \usepackage[review]{cvpr}      % To produce the REVIEW version
\usepackage[pagenumbers]{cvpr} % To force page numbers, e.g. for an arXiv version

% Import additional packages in the preamble file, before hyperref


% It is strongly recommended to use hyperref, especially for the review version.
% hyperref with option pagebackref eases the reviewers' job.
% Please disable hyperref *only* if you encounter grave issues, 
% e.g. with the file validation for the camera-ready version.
%
% If you comment hyperref and then uncomment it, you should delete *.aux before re-running LaTeX.
% (Or just hit 'q' on the first LaTeX run, let it finish, and you should be clear).
\definecolor{cvprblue}{rgb}{0.21,0.49,0.74}
\usepackage[pagebackref,breaklinks,colorlinks,allcolors=cvprblue]{hyperref}

%
% --- inline annotations
%
\newcommand{\red}[1]{{\color{red}#1}}
\newcommand{\todo}[1]{{\color{red}#1}}
\newcommand{\TODO}[1]{\textbf{\color{red}[TODO: #1]}}
% --- disable by uncommenting  
% \renewcommand{\TODO}[1]{}
% \renewcommand{\todo}[1]{#1}



\newcommand{\VLM}{LVLM\xspace} 
\newcommand{\ours}{PeKit\xspace}
\newcommand{\yollava}{Yo’LLaVA\xspace}

\newcommand{\thisismy}{This-Is-My-Img\xspace}
\newcommand{\myparagraph}[1]{\noindent\textbf{#1}}
\newcommand{\vdoro}[1]{{\color[rgb]{0.4, 0.18, 0.78} {[V] #1}}}
% --- disable by uncommenting  
% \renewcommand{\TODO}[1]{}
% \renewcommand{\todo}[1]{#1}
\usepackage{slashbox}
% Vectors
\newcommand{\bB}{\mathcal{B}}
\newcommand{\bw}{\mathbf{w}}
\newcommand{\bs}{\mathbf{s}}
\newcommand{\bo}{\mathbf{o}}
\newcommand{\bn}{\mathbf{n}}
\newcommand{\bc}{\mathbf{c}}
\newcommand{\bp}{\mathbf{p}}
\newcommand{\bS}{\mathbf{S}}
\newcommand{\bk}{\mathbf{k}}
\newcommand{\bmu}{\boldsymbol{\mu}}
\newcommand{\bx}{\mathbf{x}}
\newcommand{\bg}{\mathbf{g}}
\newcommand{\be}{\mathbf{e}}
\newcommand{\bX}{\mathbf{X}}
\newcommand{\by}{\mathbf{y}}
\newcommand{\bv}{\mathbf{v}}
\newcommand{\bz}{\mathbf{z}}
\newcommand{\bq}{\mathbf{q}}
\newcommand{\bff}{\mathbf{f}}
\newcommand{\bu}{\mathbf{u}}
\newcommand{\bh}{\mathbf{h}}
\newcommand{\bb}{\mathbf{b}}

\newcommand{\rone}{\textcolor{green}{R1}}
\newcommand{\rtwo}{\textcolor{orange}{R2}}
\newcommand{\rthree}{\textcolor{red}{R3}}
\usepackage{amsmath}
%\usepackage{arydshln}
\DeclareMathOperator{\similarity}{sim}
\DeclareMathOperator{\AvgPool}{AvgPool}

\newcommand{\argmax}{\mathop{\mathrm{argmax}}}     



%%%%%%%%% PAPER ID  - PLEASE UPDATE
\def\paperID{112} % *** Enter the Paper ID here
\def\confName{CVPR}
\def\confYear{2025}

%%%%%%%%% TITLE - PLEASE UPDATE
\title{4Deform: Neural Surface Deformation for Robust Shape Interpolation}

%%%%%%%%% AUTHORS - PLEASE UPDATE
\author{Lu Sang$^{1,2}$, Zehranaz Canfes$^{1}$, Dongliang Cao$^{3}$,\\
Riccardo Marin$^{1,2}$, Florian Bernard$^{3}$, Daniel Cremers$^{1,2}$ \\
$^{1}$Technical University of Munich, $^{2}$Munich Center of Machine Learning\\
{\tt\small \{lu.sang, zehranaz.canfes, riccardo.marin, cremers\}@tum.de} \\
$^{3}$University of Bonn \\
{\tt\small \{dcao, fb\}@uni-bonn.de}
% For a paper whose authors are all at the same institution,
% omit the following lines up until the closing ``}''.
% Additional authors and addresses can be added with ``\and'',
% just like the second author.
% To save space, use either the email address or home page, not both
% \and
% Second Author\\
% Institution2\\
% First line of institution2 address\\
% {\tt\small secondauthor@i2.org}
}

\begin{document}
\twocolumn[{
    \renewcommand \twocolumn[1][]{#1}%%\tabcolsep{1pt} % Adjust this if needed for spacing
    \maketitle
    \centering
    \vspace{-0.8cm}
    \includegraphics[width=0.95\textwidth]{images/teaser.pdf}
    \captionof{figure}{
    % \textbf{4Deform} trains on a collection of shapes and learns realistic deformations. Starting from just a pair of frames and noisy correspondence estimation, we obtain realistic long interpolation where shapes change their topology (\eg, the human-object interaction from \textbf{BEHAVE}~\cite{bhatnagar22behave}, top two rows), even in the presence of partiality (\eg, the horse), or non-isometric deformation (\eg cougar to hippo), or with correspondence do not spatially align with the input data (\eg, the clothed human from \textbf{4DDRESS}~\cite{wang20244ddress}, lower right row).
    \textbf{4Deform} takes a sparse temporal sequence of point clouds as input
    % \fb{temporal sequence of 3D shapes? [we need to consistently use this throughout the paper]; just spoke to dongliang; it seems the main task is to produce realistic intermediate shapes between a soruce and target shape? if that is the case, this should be the focus of the teaser} 
    and generates realistic intermediate shapes. Starting from just pairs of point clouds and estimated sparse, noisy correspondences (indicated using colors in the point clouds),
    % \fb{where? add a label `noisy correspondence estimation' to the figure?})
    our method obtains realistic long-range interpolations, even for shapes with changing topology (\eg, the human-object interaction in the top row), and can generalize the interpolation results to real-world data (Kinect point clouds in the second row). Meanwhile, our method can handle non-isometrically deformed shapes (bottom left)  as well as partial shapes (bottom right).
    % \fb{typo in the top left: RegiAtration; increase font sizes a bit and make them equal}
    % \fb{fontsizes should be the same within the figure}
    % \fb{visually the teaser looks very nice. However, it is hard to understand what one should look at; there are 5 different shape types. maybe it'd be better to reduce this and introduce a clearer layour (eg using (a), (b), (c) to label subparts, and then in the caption refer to these)}
    }
    \label{fig:teaser}
    \vspace{1em} % Adjust vertical spacing if needed
}]

\begin{abstract}


The choice of representation for geographic location significantly impacts the accuracy of models for a broad range of geospatial tasks, including fine-grained species classification, population density estimation, and biome classification. Recent works like SatCLIP and GeoCLIP learn such representations by contrastively aligning geolocation with co-located images. While these methods work exceptionally well, in this paper, we posit that the current training strategies fail to fully capture the important visual features. We provide an information theoretic perspective on why the resulting embeddings from these methods discard crucial visual information that is important for many downstream tasks. To solve this problem, we propose a novel retrieval-augmented strategy called RANGE. We build our method on the intuition that the visual features of a location can be estimated by combining the visual features from multiple similar-looking locations. We evaluate our method across a wide variety of tasks. Our results show that RANGE outperforms the existing state-of-the-art models with significant margins in most tasks. We show gains of up to 13.1\% on classification tasks and 0.145 $R^2$ on regression tasks. All our code and models will be made available at: \href{https://github.com/mvrl/RANGE}{https://github.com/mvrl/RANGE}.

\end{abstract}

    
\section{Introduction}
Backdoor attacks pose a concealed yet profound security risk to machine learning (ML) models, for which the adversaries can inject a stealth backdoor into the model during training, enabling them to illicitly control the model's output upon encountering predefined inputs. These attacks can even occur without the knowledge of developers or end-users, thereby undermining the trust in ML systems. As ML becomes more deeply embedded in critical sectors like finance, healthcare, and autonomous driving \citep{he2016deep, liu2020computing, tournier2019mrtrix3, adjabi2020past}, the potential damage from backdoor attacks grows, underscoring the emergency for developing robust defense mechanisms against backdoor attacks.

To address the threat of backdoor attacks, researchers have developed a variety of strategies \cite{liu2018fine,wu2021adversarial,wang2019neural,zeng2022adversarial,zhu2023neural,Zhu_2023_ICCV, wei2024shared,wei2024d3}, aimed at purifying backdoors within victim models. These methods are designed to integrate with current deployment workflows seamlessly and have demonstrated significant success in mitigating the effects of backdoor triggers \cite{wubackdoorbench, wu2023defenses, wu2024backdoorbench,dunnett2024countering}.  However, most state-of-the-art (SOTA) backdoor purification methods operate under the assumption that a small clean dataset, often referred to as \textbf{auxiliary dataset}, is available for purification. Such an assumption poses practical challenges, especially in scenarios where data is scarce. To tackle this challenge, efforts have been made to reduce the size of the required auxiliary dataset~\cite{chai2022oneshot,li2023reconstructive, Zhu_2023_ICCV} and even explore dataset-free purification techniques~\cite{zheng2022data,hong2023revisiting,lin2024fusing}. Although these approaches offer some improvements, recent evaluations \cite{dunnett2024countering, wu2024backdoorbench} continue to highlight the importance of sufficient auxiliary data for achieving robust defenses against backdoor attacks.

While significant progress has been made in reducing the size of auxiliary datasets, an equally critical yet underexplored question remains: \emph{how does the nature of the auxiliary dataset affect purification effectiveness?} In  real-world  applications, auxiliary datasets can vary widely, encompassing in-distribution data, synthetic data, or external data from different sources. Understanding how each type of auxiliary dataset influences the purification effectiveness is vital for selecting or constructing the most suitable auxiliary dataset and the corresponding technique. For instance, when multiple datasets are available, understanding how different datasets contribute to purification can guide defenders in selecting or crafting the most appropriate dataset. Conversely, when only limited auxiliary data is accessible, knowing which purification technique works best under those constraints is critical. Therefore, there is an urgent need for a thorough investigation into the impact of auxiliary datasets on purification effectiveness to guide defenders in  enhancing the security of ML systems. 

In this paper, we systematically investigate the critical role of auxiliary datasets in backdoor purification, aiming to bridge the gap between idealized and practical purification scenarios.  Specifically, we first construct a diverse set of auxiliary datasets to emulate real-world conditions, as summarized in Table~\ref{overall}. These datasets include in-distribution data, synthetic data, and external data from other sources. Through an evaluation of SOTA backdoor purification methods across these datasets, we uncover several critical insights: \textbf{1)} In-distribution datasets, particularly those carefully filtered from the original training data of the victim model, effectively preserve the model’s utility for its intended tasks but may fall short in eliminating backdoors. \textbf{2)} Incorporating OOD datasets can help the model forget backdoors but also bring the risk of forgetting critical learned knowledge, significantly degrading its overall performance. Building on these findings, we propose Guided Input Calibration (GIC), a novel technique that enhances backdoor purification by adaptively transforming auxiliary data to better align with the victim model’s learned representations. By leveraging the victim model itself to guide this transformation, GIC optimizes the purification process, striking a balance between preserving model utility and mitigating backdoor threats. Extensive experiments demonstrate that GIC significantly improves the effectiveness of backdoor purification across diverse auxiliary datasets, providing a practical and robust defense solution.

Our main contributions are threefold:
\textbf{1) Impact analysis of auxiliary datasets:} We take the \textbf{first step}  in systematically investigating how different types of auxiliary datasets influence backdoor purification effectiveness. Our findings provide novel insights and serve as a foundation for future research on optimizing dataset selection and construction for enhanced backdoor defense.
%
\textbf{2) Compilation and evaluation of diverse auxiliary datasets:}  We have compiled and rigorously evaluated a diverse set of auxiliary datasets using SOTA purification methods, making our datasets and code publicly available to facilitate and support future research on practical backdoor defense strategies.
%
\textbf{3) Introduction of GIC:} We introduce GIC, the \textbf{first} dedicated solution designed to align auxiliary datasets with the model’s learned representations, significantly enhancing backdoor mitigation across various dataset types. Our approach sets a new benchmark for practical and effective backdoor defense.


 
\section{Related Work}
\label{sec:related-works}
\subsection{Novel View Synthesis}
Novel view synthesis is a foundational task in the computer vision and graphics, which aims to generate unseen views of a scene from a given set of images.
% Many methods have been designed to solve this problem by posing it as 3D geometry based rendering, where point clouds~\cite{point_differentiable,point_nfs}, mesh~\cite{worldsheet,FVS,SVS}, planes~\cite{automatci_photo_pop_up,tour_into_the_picture} and multi-plane images~\cite{MINE,single_view_mpi,stereo_magnification}, \etal
Numerous methods have been developed to address this problem by approaching it as 3D geometry-based rendering, such as using meshes~\cite{worldsheet,FVS,SVS}, MPI~\cite{MINE,single_view_mpi,stereo_magnification}, point clouds~\cite{point_differentiable,point_nfs}, etc.
% planes~\cite{automatci_photo_pop_up,tour_into_the_picture}, 


\begin{figure*}[!t]
    \centering
    \includegraphics[width=1.0\linewidth]{figures/overview-v7.png}
    %\caption{\textbf{Overview.} Given a set of images, our method obtains both camera intrinsics and extrinsics, as well as a 3DGS model. First, we obtain the initial camera parameters, global track points from image correspondences and monodepth with reprojection loss. Then we incorporate the global track information and select Gaussian kernels associated with track points. We jointly optimize the parameters $K$, $T_{cw}$, 3DGS through multi-view geometric consistency $L_{t2d}$, $L_{t3d}$, $L_{scale}$ and photometric consistency $L_1$, $L_{D-SSIM}$.}
    \caption{\textbf{Overview.} Given a set of images, our method obtains both camera intrinsics and extrinsics, as well as a 3DGS model. During the initialization, we extract the global tracks, and initialize camera parameters and Gaussians from image correspondences and monodepth with reprojection loss. We determine Gaussian kernels with recovered 3D track points, and then jointly optimize the parameters $K$, $T_{cw}$, 3DGS through the proposed global track constraints (i.e., $L_{t2d}$, $L_{t3d}$, and $L_{scale}$) and original photometric losses (i.e., $L_1$ and $L_{D-SSIM}$).}
    \label{fig:overview}
\end{figure*}

Recently, Neural Radiance Fields (NeRF)~\cite{2020NeRF} provide a novel solution to this problem by representing scenes as implicit radiance fields using neural networks, achieving photo-realistic rendering quality. Although having some works in improving efficiency~\cite{instant_nerf2022, lin2022enerf}, the time-consuming training and rendering still limit its practicality.
Alternatively, 3D Gaussian Splatting (3DGS)~\cite{3DGS2023} models the scene as explicit Gaussian kernels, with differentiable splatting for rendering. Its improved real-time rendering performance, lower storage and efficiency, quickly attract more attentions.
% Different from NeRF-based methods which need MLPs to model the scene and huge computational cost for rendering, 3DGS has stronger real-time performance, higher storage and computational efficiency, benefits from its explicit representation and gradient backpropagation.

\subsection{Optimizing Camera Poses in NeRFs and 3DGS}
Although NeRF and 3DGS can provide impressive scene representation, these methods all need accurate camera parameters (both intrinsic and extrinsic) as additional inputs, which are mostly obtained by COLMAP~\cite{colmap2016}.
% This strong reliance on COLMAP significantly limits their use in real-world applications, so optimizing the camera parameters during the scene training becomes crucial.
When the prior is inaccurate or unknown, accurately estimating camera parameters and scene representations becomes crucial.

% In early works, only photometric constraints are used for scene training and camera pose estimation. 
% iNeRF~\cite{iNerf2021} optimizes the camera poses based on a pre-trained NeRF model.
% NeRFmm~\cite{wang2021nerfmm} introduce a joint optimization process, which estimates the camera poses and trains NeRF model jointly.
% BARF~\cite{barf2021} and GARF~\cite{2022GARF} provide new positional encoding strategy to handle with the gradient inconsistency issue of positional embedding and yield promising results.
% However, they achieve satisfactory optimization results when only the pose initialization is quite closed to the ground-truth, as the photometric constrains can only improve the quality of camera estimation within a small range.
% Later, more prior information of geometry and correspondence, \ie monocular depth and feature matching, are introduced into joint optimisation to enhance the capability of camera poses estimation.
% SC-NeRF~\cite{SCNeRF2021} minimizes a projected ray distance loss based on correspondence of adjacent frames.
% NoPe-NeRF~\cite{bian2022nopenerf} chooses monocular depth maps as geometric priors, and defines undistorted depth loss and relative pose constraints for joint optimization.
In earlier studies, scene training and camera pose estimation relied solely on photometric constraints. iNeRF~\cite{iNerf2021} refines the camera poses using a pre-trained NeRF model. NeRFmm~\cite{wang2021nerfmm} introduces a joint optimization approach that simultaneously estimates camera poses and trains the NeRF model. BARF~\cite{barf2021} and GARF~\cite{2022GARF} propose a new positional encoding strategy to address the gradient inconsistency issues in positional embedding, achieving promising results. However, these methods only yield satisfactory optimization when the initial pose is very close to the ground truth, as photometric constraints alone can only enhance camera estimation quality within a limited range. Subsequently, 
% additional prior information on geometry and correspondence, such as monocular depth and feature matching, has been incorporated into joint optimization to improve the accuracy of camera pose estimation. 
SC-NeRF~\cite{SCNeRF2021} minimizes a projected ray distance loss based on correspondence between adjacent frames. NoPe-NeRF~\cite{bian2022nopenerf} utilizes monocular depth maps as geometric priors and defines undistorted depth loss and relative pose constraints.

% With regard to 3D Gaussian Splatting, CF-3DGS~\cite{CF-3DGS-2024} also leverages mono-depth information to constrain the optimization of local 3DGS for relative pose estimation and later learn a global 3DGS progressively in a sequential manner.
% InstantSplat~\cite{fan2024instantsplat} focus on sparse view scenes, first use DUSt3R~\cite{dust3r2024cvpr} to generate a set of densely covered and pixel-aligned points for 3D Gaussian initialization, then introduce a parallel grid partitioning strategy in joint optimization to speed up.
% % Jiang et al.~\cite{Jiang_2024sig} proposed to build the scene continuously and progressively, to next unregistered frame, they use registration and adjustment to adjust the previous registered camera poses and align unregistered monocular depths, later refine the joint model by matching detected correspondences in screen-space coordinates.
% \gjh{Jiang et al.~\cite{Jiang_2024sig} also implemented an incremental approach for reconstructing camera poses and scenes. Initially, they perform feature matching between the current image and the image rendered by a differentiable surface renderer. They then construct matching point errors, depth errors, and photometric errors to achieve the registration and adjustment of the current image. Finally, based on the depth map, the pixels of the current image are projected as new 3D Gaussians. However, this method still exhibits limitations when dealing with complex scenes and unordered images.}
% % CG-3DGS~\cite{sun2024correspondenceguidedsfmfree3dgaussian} follows CF-3DGS, first construct a coarse point cloud from mono-depth maps to train a 3DGS model, then progressively estimate camera poses based on this pre-trained model by constraining the correspondences between rendering view and ground-truth.
% \gjh{Similarly, CG-3DGS~\cite{sun2024correspondenceguidedsfmfree3dgaussian} first utilizes monocular depth estimation and the camera parameters from the first frame to initialize a set of 3D Gaussians. It then progressively estimates camera poses based on this pre-trained model by constraining the correspondences between the rendered views and the ground truth.}
% % Free-SurGS~\cite{freesurgs2024} matches the projection flow derived from 3D Gaussians with optical flow to estimate the poses, to compensate for the limitations of photometric loss.
% \gjh{Free-SurGS~\cite{freesurgs2024} introduces the first SfM-free 3DGS approach for surgical scene reconstruction. Due to the challenges posed by weak textures and photometric inconsistencies in surgical scenes, Free-SurGS achieves pose estimation by minimizing the flow loss between the projection flow and the optical flow. Subsequently, it keeps the camera pose fixed and optimizes the scene representation by minimizing the photometric loss, depth loss and flow loss.}
% \gjh{However, most current works assume camera intrinsics are known and primarily focus on optimizing camera poses. Additionally, these methods typically rely on sequentially ordered image inputs and incrementally optimize camera parameters and scene representation. This inevitably leads to drift errors, preventing the achievement of globally consistent results. Our work aims to address these issues.}

Regarding 3D Gaussian Splatting, CF-3DGS~\cite{CF-3DGS-2024} utilizes mono-depth information to refine the optimization of local 3DGS for relative pose estimation and subsequently learns a global 3DGS in a sequential manner. InstantSplat~\cite{fan2024instantsplat} targets sparse view scenes, initially employing DUSt3R~\cite{dust3r2024cvpr} to create a densely covered, pixel-aligned point set for initializing 3D Gaussian models, and then implements a parallel grid partitioning strategy to accelerate joint optimization. Jiang \etal~\cite{Jiang_2024sig} develops an incremental method for reconstructing camera poses and scenes, but it struggles with complex scenes and unordered images. 
% Similarly, CG-3DGS~\cite{sun2024correspondenceguidedsfmfree3dgaussian} progressively estimates camera poses using a pre-trained model by aligning the correspondences between rendered views and actual scenes. Free-SurGS~\cite{freesurgs2024} pioneers an SfM-free 3DGS method for reconstructing surgical scenes, overcoming challenges such as weak textures and photometric inconsistencies by minimizing the discrepancy between projection flow and optical flow.
%\pb{SF-3DGS-HT~\cite{ji2024sfmfree3dgaussiansplatting} introduced VFI into training as additional photometric constraints. They separated the whole scene into several local 3DGS models and then merged them hierarchically, which leads to a significant improvement on simple and dense view scenes.}
HT-3DGS~\cite{ji2024sfmfree3dgaussiansplatting} interpolates frames for training and splits the scene into local clips, using a hierarchical strategy to build 3DGS model. It works well for simple scenes, but fails with dramatic motions due to unstable interpolation and low efficiency.
% {While effective for simple scenes, it struggles with dramatic motion due to unstable view interpolation and suffers from low computational efficiency.}

However, most existing methods generally depend on sequentially ordered image inputs and incrementally optimize camera parameters and 3DGS, which often leads to drift errors and hinders achieving globally consistent results. Our work seeks to overcome these limitations.

%structure of paper:
%1 intro
%2 backgroud: systolic array: mix size regularity, replace flow: not good at PE array placement, without using regularity info( 1 page)
%3 preliminary and motivation: formulation of mixed size placement problem, (better quality) -> ILP, runtime-> regularity -> reduce searching space( 1 page)
%4 method (1.5 page)
%5 experiment(1 page) 


\begin{figure*}[ht]
    \centering
    \includegraphics[width=0.9\linewidth]{figure/workfolw15.pdf}
    \caption{Overview of the Lorecast methodology.}
    \label{fig:workfolw1}
\end{figure*}

% \thispagestyle{empty}
\section{Background}
\label{sec:background}

\subsection{Large Language Models }
\label{sec:graph_clustering}
Large Language Models (LLMs) are advanced AI systems designed to understand and generate human-like text by processing vast amounts of language data\cite{Radford2019LLM}. These models are typically based on Transformer architectures\cite{Vaswani2017Transformer} and employ billions of parameters to capture complex patterns and relationships within language, enabling them to perform tasks such as text generation, translation, summarization, and question answering\cite{Brown2020LLMLearners}. Prominent examples include OpenAI’s GPT-4 \cite{OpenAI2023GPT-4} and Google’s Gemini 1.5 Pro \cite{Google2024Gemini}, showcasing the immense capabilities of LLMs. With 175 billion parameters and a context capacity of 1 million tokens, these models achieve near-human performance across a variety of Natural Language Processing (NLP) tasks. The primary strength of LLMs lies in their pretraining on extensive text corpora, which gives them a general understanding of language that can be fine-tuned for specific tasks or domains \cite{Bommasani2022finite}\cite{Liu2024RTLCoder}\cite{Thakur2023Benchmarking}. As LLMs continue to evolve, their applications are expanding across various fields, including EDA, where they are increasingly employed for tasks such as code generation\cite{Wang2024ChatCPU}\cite{Wong2024VGV}\cite{Pei2024BetterV}.
\subsection{Verilog and Code Correctness}
Verilog is a hardware description language (HDL) widely used in digital design and circuit development for specifying and modeling electronic systems at various abstraction levels, from high-level functional descriptions to detailed structural representations \cite{Palnitkar2003Verilog}. Ensuring code correctness in Verilog is crucial because errors at the HDL level can lead to significant functional failures, performance inefficiencies, and increased costs when translated to physical hardware. Code correctness in Verilog typically includes both syntax correctness, ensuring code is free from syntax errors, and functional correctness, validating that the code’s behavior aligns with design specifications \cite{Harris2010CMOS}. Recent advances in machine learning and automated code generation, including the use of LLMs, have opened new avenues for improving Verilog code correctness through automated error detection and code synthesis\cite{Wang2024ChatCPU,Pei2024BetterV}.

\iffalse
\subsection{systolic array}
describe the pipeline and parallel computing function 
due to the repeating PE structure and its dataflow.
IO and control logic unmatch to the regular schematic of the PE array
\fi

Effective human-robot cooperation in CoNav-Maze hinges on efficient communication. Maximizing the human’s information gain enables more precise guidance, which in turn accelerates task completion. Yet for the robot, the challenge is not only \emph{what} to communicate but also \emph{when}, as it must balance gathering information for the human with pursuing immediate goals when confident in its navigation.

To achieve this, we introduce \emph{Information Gain Monte Carlo Tree Search} (IG-MCTS), which optimizes both task-relevant objectives and the transmission of the most informative communication. IG-MCTS comprises three key components:
\textbf{(1)} A data-driven human perception model that tracks how implicit (movement) and explicit (image) information updates the human’s understanding of the maze layout.
\textbf{(2)} Reward augmentation to integrate multiple objectives effectively leveraging on the learned perception model.
\textbf{(3)} An uncertainty-aware MCTS that accounts for unobserved maze regions and human perception stochasticity.
% \begin{enumerate}[leftmargin=*]
%     \item A data-driven human perception model that tracks how implicit (movement) and explicit (image transmission) information updates the human’s understanding of the maze layout.
%     \item Reward augmentation to integrate multiple objectives effectively leveraging on the learned perception model.
%     \item An uncertainty-aware MCTS that accounts for unobserved maze regions and human perception stochasticity.
% \end{enumerate}

\subsection{Human Perception Dynamics}
% IG-MCTS seeks to optimize the expected novel information gained by the human through the robot’s actions, including both movement and communication. Achieving this requires a model of how the human acquires task-relevant information from the robot.

% \subsubsection{Perception MDP}
\label{sec:perception_mdp}
As the robot navigates the maze and transmits images, humans update their understanding of the environment. Based on the robot's path, they may infer that previously assumed blocked locations are traversable or detect discrepancies between the transmitted image and their map.  

To formally capture this process, we model the evolution of human perception as another Markov Decision Process, referred to as the \emph{Perception MDP}. The state space $\mathcal{X}$ represents all possible maze maps. The action space $\mathcal{S}^+ \times \mathcal{O}$ consists of the robot's trajectory between two image transmissions $\tau \in \mathcal{S}^+$ and an image $o \in \mathcal{O}$. The unknown transition function $F: (x, (\tau, o)) \rightarrow x'$ defines the human perception dynamics, which we aim to learn.

\subsubsection{Crowd-Sourced Transition Dataset}
To collect data, we designed a mapping task in the CoNav-Maze environment. Participants were tasked to edit their maps to match the true environment. A button triggers the robot's autonomous movements, after which it captures an image from a random angle.
In this mapping task, the robot, aware of both the true environment and the human’s map, visits predefined target locations and prioritizes areas with mislabeled grid cells on the human’s map.
% We assume that the robot has full knowledge of both the actual environment and the human’s current map. Leveraging this knowledge, the robot autonomously navigates to all predefined target locations. It then randomly selects subsequent goals to reach, prioritizing grid locations that remain mislabeled on the human’s map. This ensures that the robot’s actions are strategically focused on providing useful information to improve map accuracy.

We then recruited over $50$ annotators through Prolific~\cite{palan2018prolific} for the mapping task. Each annotator labeled three randomly generated mazes. They were allowed to proceed to the next maze once the robot had reached all four goal locations. However, they could spend additional time refining their map before moving on. To incentivize accuracy, annotators receive a performance-based bonus based on the final accuracy of their annotated map.


\subsubsection{Fully-Convolutional Dynamics Model}
\label{sec:nhpm}

We propose a Neural Human Perception Model (NHPM), a fully convolutional neural network (FCNN), to predict the human perception transition probabilities modeled in \Cref{sec:perception_mdp}. We denote the model as $F_\theta$ where $\theta$ represents the trainable weights. Such design echoes recent studies of model-based reinforcement learning~\cite{hansen2022temporal}, where the agent first learns the environment dynamics, potentially from image observations~\cite{hafner2019learning,watter2015embed}.

\begin{figure}[t]
    \centering
    \includegraphics[width=0.9\linewidth]{figures/ICML_25_CNN.pdf}
    \caption{Neural Human Perception Model (NHPM). \textbf{Left:} The human's current perception, the robot's trajectory since the last transmission, and the captured environment grids are individually processed into 2D masks. \textbf{Right:} A fully convolutional neural network predicts two masks: one for the probability of the human adding a wall to their map and another for removing a wall.}
    \label{fig:nhpm}
    \vskip -0.1in
\end{figure}

As illustrated in \Cref{fig:nhpm}, our model takes as input the human’s current perception, the robot’s path, and the image captured by the robot, all of which are transformed into a unified 2D representation. These inputs are concatenated along the channel dimension and fed into the CNN, which outputs a two-channel image: one predicting the probability of human adding a new wall and the other predicting the probability of removing a wall.

% Our approach builds on world model learning, where neural networks predict state transitions or environmental updates based on agent actions and observations. By leveraging the local feature extraction capabilities of CNNs, our model effectively captures spatial relationships and interprets local changes within the grid maze environment. Similar to prior work in localization and mapping, the CNN architecture is well-suited for processing spatially structured data and aligning the robot’s observations with human map updates.

To enhance robustness and generalization, we apply data augmentation techniques, including random rotation and flipping of the 2D inputs during training. These transformations are particularly beneficial in the grid maze environment, which is invariant to orientation changes.

\subsection{Perception-Aware Reward Augmentation}
The robot optimizes its actions over a planning horizon \( H \) by solving the following optimization problem:
\begin{subequations}
    \begin{align}
        \max_{a_{0:H-1}} \;
        & \mathop{\mathbb{E}}_{T, F} \left[ \sum_{t=0}^{H-1} \gamma^t \left(\underbrace{R_{\mathrm{task}}(\tau_{t+1}, \zeta)}_{\text{(1) Task reward}} + \underbrace{\|x_{t+1}-x_t\|_1}_{\text{(2) Info reward}}\right)\right] \label{obj}\\ 
        \subjectto \quad
        &x_{t+1} = F(x_t, (\tau_t, a_t)), \quad a_t\in\Ocal \label{const:perception_update}\\ 
        &\tau_{t+1} = \tau_t \oplus T(s_t, a_t), \quad a_t\in \Ucal\label{const:history_update}
    \end{align}
\end{subequations} 

The objective in~\eqref{obj} maximizes the expected cumulative reward over \( T \) and \( F \), reflecting the uncertainty in both physical transitions and human perception dynamics. The reward function consists of two components: 
(1) The \emph{task reward} incentivizes efficient navigation. The specific formulation for the task in this work is outlined in \Cref{appendix:task_reward}.
(2) The \emph{information reward} quantifies the change in the human’s perception due to robot actions, computed as the \( L_1 \)-norm distance between consecutive perception states.  

The constraint in~\eqref{const:history_update} ensures that for movement actions, the trajectory history \( \tau_t \) expands with new states based on the robot’s chosen actions, where \( s_t \) is the most recent state in \( \tau_t \), and \( \oplus \) represents sequence concatenation. 
In constraint~\eqref{const:perception_update}, the robot leverages the learned human perception dynamics \( F \) to estimate the evolution of the human’s understanding of the environment from perception state $x_t$ to $x_{t+1}$ based on the observed trajectory \( \tau_t \) and transmitted image \( a_t\in\Ocal \). 
% justify from a cognitive science perspective
% Cognitive science research has shown that humans read in a way to maximize the information gained from each word, aligning with the efficient coding principle, which prioritizes minimizing perceptual errors and extracting relevant features under limited processing capacity~\cite{kangassalo2020information}. Drawing on this principle, we hypothesize that humans similarly prioritize task-relevant information in multimodal settings. To accommodate this cognitive pattern, our robot policy selects and communicates high information-gain observations to human operators, akin to summarizing key insights from a lengthy article.
% % While the brain naturally seeks to gain information, the brain employs various strategies to manage information overload, including filtering~\cite{quiroga2004reducing}, limiting/working memory, and prioritizing information~\cite{arnold2023dealing}.
% In this context of our setup, we optimize the selection of camera angles to maximize the human operator's information gain about the environment. 

\subsection{Information Gain Monte Carlo Tree Search (IG-MCTS)}
IG-MCTS follows the four stages of Monte Carlo tree search: \emph{selection}, \emph{expansion}, \emph{rollout}, and \emph{backpropagation}, but extends it by incorporating uncertainty in both environment dynamics and human perception. We introduce uncertainty-aware simulations in the \emph{expansion} and \emph{rollout} phases and adjust \emph{backpropagation} with a value update rule that accounts for transition feasibility.

\subsubsection{Uncertainty-Aware Simulation}
As detailed in \Cref{algo:IG_MCTS}, both the \emph{expansion} and \emph{rollout} phases involve forward simulation of robot actions. Each tree node $v$ contains the state $(\tau, x)$, representing the robot's state history and current human perception. We handle the two action types differently as follows:
\begin{itemize}
    \item A movement action $u$ follows the environment dynamics $T$ as defined in \Cref{sec:problem}. Notably, the maze layout is observable up to distance $r$ from the robot's visited grids, while unexplored areas assume a $50\%$ chance of walls. In \emph{expansion}, the resulting search node $v'$ of this uncertain transition is assigned a feasibility value $\delta = 0.5$. In \emph{rollout}, the transition could fail and the robot remains in the same grid.
    
    \item The state transition for a communication step $o$ is governed by the learned stochastic human perception model $F_\theta$ as defined in \Cref{sec:nhpm}. Since transition probabilities are known, we compute the expected information reward $\bar{R_\mathrm{info}}$ directly:
    \begin{align*}
        \bar{R_\mathrm{info}}(\tau_t, x_t, o_t) &= \mathbb{E}_{x_{t+1}}\|x_{t+1}-x_t\|_1 \\
        &= \|p_\mathrm{add}\|_1 + \|p_\mathrm{remove}\|_1,
    \end{align*}
    where $(p_\mathrm{add}, p_\mathrm{remove}) \gets F_\theta(\tau_t, x_t, o_t)$ are the estimated probabilities of adding or removing walls from the map. 
    Directly computing the expected return at a node avoids the high number of visitations required to obtain an accurate value estimate.
\end{itemize}

% We denote a node in the search tree as $v$, where $s(v)$, $r(v)$, and $\delta(v)$ represent the state, reward, and transition feasibility at $v$, respectively. The visit count of $v$ is denoted as $N(v)$, while $Q(v)$ represents its total accumulated return. The set of child nodes of $v$ is denoted by $\mathbb{C}(v)$.

% The goal of each search is to plan a sequence for the robot until it reaches a goal or transmits a new image to the human. We initialize the search tree with the current human guidance $\zeta$, and the robot's approximation of human perception $x_0$. Each search node consists consists of the state information required by our reward augmentation: $(\tau, x)$. A node is terminal if it is the resulting state of a communication step, or if the robot reaches a goal location. 

% A rollout from the expanded node simulates future transitions until reaching a terminal state or a predefined depth $H$. Actions are selected randomly from the available action set $\mathcal{A}(s)$. If an action's feasibility is uncertain due to the environment's unknown structure, the transition occurs with probability $\delta(s, a)$. When a random number draw deems the transition infeasible, the state remains unchanged. On the other hand, for communication steps, we don't resolve the uncertainty but instead compute the expected information gain reward: \philip{TODO: adjust notation}
% \begin{equation}
%     \mathbb{E}\left[R_\mathrm{info}(\tau, x')\right] = \sum \mathrm{NPM(\tau, o)}.
% \end{equation}

\subsubsection{Feasibility-Adjusted Backpropagation}
During backpropagation, the rewards obtained from the simulation phase are propagated back through the tree, updating the total value $Q(v)$ and the visitation count $N(v)$ for all nodes along the path to the root. Due to uncertainty in unexplored environment dynamics, the rollout return depends on the feasibility of the transition from the child node. Given a sample return \(q'_{\mathrm{sample}}\) at child node \(v'\), the parent node's return is:
\begin{equation}
    q_{\mathrm{sample}} = r + \gamma \left[ \delta' q'_{\mathrm{sample}} + (1 - \delta') \frac{Q(v)}{N(v)} \right],
\end{equation}
where $\delta'$ represents the probability of a successful transition. The term \((1 - \delta')\) accounts for failed transitions, relying instead on the current value estimate.

% By incorporating uncertainty-aware rollouts and backpropagation, our approach enables more robust decision-making in scenarios where the environment dynamics is unknown and avoids simulation of the stochastic human perception dynamics.

% \begin{table}[!t]
% \centering
% \scalebox{0.68}{
%     \begin{tabular}{ll cccc}
%       \toprule
%       & \multicolumn{4}{c}{\textbf{Intellipro Dataset}}\\
%       & \multicolumn{2}{c}{Rank Resume} & \multicolumn{2}{c}{Rank Job} \\
%       \cmidrule(lr){2-3} \cmidrule(lr){4-5} 
%       \textbf{Method}
%       &  Recall@100 & nDCG@100 & Recall@10 & nDCG@10 \\
%       \midrule
%       \confitold{}
%       & 71.28 &34.79 &76.50 &52.57 
%       \\
%       \cmidrule{2-5}
%       \confitsimple{}
%     & 82.53 &48.17
%        & 85.58 &64.91
     
%        \\
%        +\RunnerUpMiningShort{}
%     &85.43 &50.99 &91.38 &71.34 
%       \\
%       +\HyReShort
%         &- & -
%        &-&-\\
       
%       \bottomrule

%     \end{tabular}
%   }
% \caption{Ablation studies using Jina-v2-base as the encoder. ``\confitsimple{}'' refers using a simplified encoder architecture. \framework{} trains \confitsimple{} with \RunnerUpMiningShort{} and \HyReShort{}.}
% \label{tbl:ablation}
% \end{table}
\begin{table*}[!t]
\centering
\scalebox{0.75}{
    \begin{tabular}{l cccc cccc}
      \toprule
      & \multicolumn{4}{c}{\textbf{Recruiting Dataset}}
      & \multicolumn{4}{c}{\textbf{AliYun Dataset}}\\
      & \multicolumn{2}{c}{Rank Resume} & \multicolumn{2}{c}{Rank Job} 
      & \multicolumn{2}{c}{Rank Resume} & \multicolumn{2}{c}{Rank Job}\\
      \cmidrule(lr){2-3} \cmidrule(lr){4-5} 
      \cmidrule(lr){6-7} \cmidrule(lr){8-9} 
      \textbf{Method}
      & Recall@100 & nDCG@100 & Recall@10 & nDCG@10
      & Recall@100 & nDCG@100 & Recall@10 & nDCG@10\\
      \midrule
      \confitold{}
      & 71.28 & 34.79 & 76.50 & 52.57 
      & 87.81 & 65.06 & 72.39 & 56.12
      \\
      \cmidrule{2-9}
      \confitsimple{}
      & 82.53 & 48.17 & 85.58 & 64.91
      & 94.90&78.40 & 78.70& 65.45
       \\
      +\HyReShort{}
       &85.28 & 49.50
       &90.25 & 70.22
       & 96.62&81.99 & \textbf{81.16}& 67.63
       \\
      +\RunnerUpMiningShort{}
       % & 85.14& 49.82
       % &90.75&72.51
       & \textbf{86.13}&\textbf{51.90} & \textbf{94.25}&\textbf{73.32}
       & \textbf{97.07}&\textbf{83.11} & 80.49& \textbf{68.02}
       \\
   %     +\RunnerUpMiningShort{}
   %    & 85.43 & 50.99 & 91.38 & 71.34 
   %    & 96.24 & 82.95 & 80.12 & 66.96
   %    \\
   %    +\HyReShort{} old
   %     &85.28 & 49.50
   %     &90.25 & 70.22
   %     & 96.62&81.99 & 81.16& 67.63
   %     \\
   % +\HyReShort{} 
   %     % & 85.14& 49.82
   %     % &90.75&72.51
   %     & 86.83&51.77 &92.00 &72.04
   %     & 97.07&83.11 & 80.49& 68.02
   %     \\
      \bottomrule

    \end{tabular}
  }
\caption{\framework{} ablation studies. ``\confitsimple{}'' refers using a simplified encoder architecture. \framework{} trains \confitsimple{} with \RunnerUpMiningShort{} and \HyReShort{}. We use Jina-v2-base as the encoder due to its better performance.
}
\label{tbl:ablation}
\end{table*}

\section{Results}
\label{sec:results}

In this section, we present detailed results demonstrating \emph{CellFlow}'s state-of-the-art performance in cellular morphology prediction under perturbations, outperforming existing methods across multiple datasets and evaluation metrics.

\subsection{Datasets}

Our experiments were conducted using three cell imaging perturbation datasets: BBBC021 (chemical perturbation)~\cite{caie2010high}, RxRx1 (genetic perturbation)~\cite{sypetkowski2023rxrx1}, and the JUMP dataset (combined perturbation)~\cite{chandrasekaran2023jump}. We followed the preprocessing protocol from IMPA~\cite{palma2023predicting}, which involves correcting illumination, cropping images centered on nuclei to a resolution of 96×96, and filtering out low-quality images. The resulting datasets include 98K, 171K, and 424K images with 3, 5, and 6 channels, respectively, from 26, 1,042, and 747 perturbation types. Examples of these images are provided in Figure~\ref{fig:comparison}. Details of datasets are provided in \S\ref{sec:data}.

\subsection{Experimental Setup}

\textbf{Evaluation metrics.} We evaluate methods using two types of metrics: (1) FID and KID, which measure image distribution similarity via Fréchet and kernel-based distances, computed on 5K generated images for BBBC021 and 100 randomly selected perturbation classes for RxRx1 and JUMP; we report both overall scores across all samples and conditional scores per perturbation class. (2) Mode of Action (MoA) classification accuracy, which assesses biological fidelity by using a trained classifier to predict a drug’s effect from perturbed images and comparing it to its known MoA from the literature.

\textbf{Baselines.} We compare our approach against two baselines, PhenDiff~\cite{bourou2024phendiff} and IMPA~\cite{palma2023predicting}, the only two baselines that incorporate control images into their model design --- a crucial setup for distinguishing true perturbation effects from artifacts such as batch effects. PhenDiff uses diffusion models to first map control images to noise and then transform the noise into target images. In contrast, IMPA employs GANs with an AdaIN layer to transfer the style of control images to target images, specifically designed for paired image-to-image mappings. Our method uses flow matching, which is tailored for distribution-to-distribution mapping, providing a more suitable solution for our problem. We reproduce these baselines with official codes.

\textbf{Training details.} \emph{CellFlow} employs a UNet-based velocity field with a four-stage design. Perturbations are encoded following IMPA~\cite{palma2023predicting}. Training is conducted for 100 epochs on 4 A100 GPUs. Details are in \S\ref{sec:experimental}.


\begin{table*}[ht!]
\centering
\footnotesize
\renewcommand{\arraystretch}{1.2}
\begin{tabular}{p{2.15cm} p{14.05cm}}
\toprule
 & \includegraphics[height=2.6cm]{figure/qualitative_example.pdf}\\
\midrule
Original \phantom{00000} Description & Bromazepam binds to the GABA receptor GABA-A, causing a conformational change and \textbf{increasing inhibitory effects of GABA}. Other neurotransmitters are not influenced. One of the \textbf{benzodiazepines} that is \textbf{used in the treatment of anxiety disorders.} \\
\midrule
User & Explain the features of the given molecule that are most distinct from those of other molecules. \\
\midrule
Mol-LLaMA (Ours) & The given molecule is a member of the \textcolor{tabblue}{benzodiazepine class}, which is characterized by a fusion of a benzene ring and a seven-membered diazepine ring. This structure is crucial for its biological activity, particularly its role as \textcolor{tabblue}{a sedative and hypnotic agent}. ...$<$omitted$>$...\\

& The given molecule is also characterized by a ketone group within the diazepine ring, which is a common feature in benzodiazepines. \textcolor{tabblue}{This functional group is important for the interaction with the GABA-A receptor, enhancing the inhibitory effects of GABA}. ...$<$omitted$>$...\\

\midrule
GPT-4o & 1. **Quinazoline Core**: The molecule contains a \textcolor{tabred}{quinazoline core}, ...$<$omitted$>$... \\
 & Overall, the combination of a \textcolor{tabred}{quinazoline} structure with unique substitution patterns, including bromine halogenation and the presence of a pyridinyl group, ...$<$omitted$>$... . \\
\midrule
LLaMo & The molecule has a \textcolor{tabblue}{benzodiazepine} structure with a bromo substituent at the 7-position and \textcolor{tabred}{a methyl group at the 1-position}. It is distinct from other molecules in that it contains a bromine atom, a nitrogen atom, and a methyl group, which are not present in the other molecules. \\
\midrule
3D-MoLM & ...$<$omitted$>$... It belongs to the class of compounds called \textcolor{tabred}{quinazolines}, which are characterized by a bicyclic structure consisting of a benzene ring fused to a pyrimidine ring. ...$<$omitted$>$...\\
& The molecule's structure suggests potential applications in medicinal chemistry, as quinazolines have been found to possess various biological activities, including \textcolor{tabred}{antitumor, antimicrobial, and anti-inflammatory properties.} \\
\midrule
Mol-Instructions & The molecule is a \textcolor{tabred}{quinoxaline derivative}. \\
\bottomrule
\end{tabular}
\vspace{-0.1in}
\caption{\small Case study to compare molecular understanding and reasoning ability. Mol-LLaMA accurately understands the molecular features, answering a correct molecular taxonomy and providing its distinct properties that are relevant to the given molecule.}
\label{tab:qualitative}
\vspace{-0.1in}
\end{table*}

\subsection{Main Results}

\textbf{\emph{CellFlow} generates highly realistic cell images.}  
\emph{CellFlow} outperforms existing methods in capturing cellular morphology across all datasets (Table~\ref{tab:results}a), achieving overall FID scores of 18.7, 33.0, and 9.0 on BBBC021, RxRx1, and JUMP, respectively --- improving FID by 21\%–45\% compared to previous methods. These gains in both FID and KID metrics confirm that \emph{CellFlow} produces significantly more realistic cell images than prior approaches.

\textbf{\emph{CellFlow} accurately captures perturbation-specific morphological changes.}  
As shown in Table~\ref{tab:results}a, \emph{CellFlow} achieves conditional FID scores of 56.8 (a 26\% improvement), 163.5, and 84.4 (a 16\% improvement) on BBBC021, RxRx1, and JUMP, respectively. These scores are computed by measuring the distribution distance for each specific perturbation and averaging across all perturbations.   
Table~\ref{tab:results}b further highlights \emph{CellFlow}’s performance on six representative chemical and three genetic perturbations. For chemical perturbations, \emph{CellFlow} reduces FID scores by 14–55\% compared to prior methods.
The smaller improvement (5–12\% improvements) on RxRx1 is likely due to the limited number of images per perturbation type.

\textbf{\emph{CellFlow} preserves biological fidelity across perturbation conditions.} 
Table~\ref{tab:ablation}a presents mode of action (MoA) classification accuracy on the BBBC021 dataset using generated cell images. MoA describes how a drug affects cellular function and can be inferred from morphology. To assess this, we train an image classifier on real perturbed images and test it on generated ones. \emph{CellFlow} achieves 71.1\% MoA accuracy, closely matching real images (72.4\%) and significantly surpassing other methods (best: 63.7\%), demonstrating its ability to maintain biological fidelity across perturbations. Qualitative comparisons in Figure~\ref{fig:comparison} further highlight \emph{CellFlow}’s accuracy in capturing key biological effects. For example, demecolcine produces smaller, fragmented nuclei, which other methods fail to reproduce accurately.

\textbf{\emph{CellFlow} generalizes to out-of-distribution (OOD) perturbations.}  
On BBBC021, \emph{CellFlow} demonstrates strong generalization to novel chemical perturbations never seen during training (Table~\ref{tab:ablation}b). It achieves 6\% and 28\% improvements in overall and conditional FID over the best baseline. This OOD generalization is critical for biological research, enabling the exploration of previously untested interventions and the design of new drugs.

\textbf{Ablations highlight the importance of each component in \emph{CellFlow}.}  
Table~\ref{tab:ablation}c shows that removing conditional information, classifier-free guidance, or noise augmentation significantly degrades performance, leading to higher FID scores. These underscore the critical role of each component in enabling \emph{CellFlow}’s state-of-the-art performance.  

\begin{figure*}[!tb]
    \centering
     \includegraphics[width=\linewidth]{imgs/interpolation.pdf}
     \vspace{-2em}
    \caption{
    \textbf{\emph{CellFlow} enables new capabilities.} 
\textit{(a.1) Batch effect calibration.}  
\emph{CellFlow} initializes with control images, enabling batch-specific predictions. Comparing predictions from different batches highlights actual perturbation effects (smaller cell size) while filtering out spurious batch effects (cell density variations).  
\textit{(a.2) Interpolation trajectory.}  
\emph{CellFlow}'s learned velocity field supports interpolation between cell states, which might provide insights into the dynamic cell trajectory. 
\textit{(b) Diffusion model comparison.}  
Unlike flow matching, diffusion models that start from noise cannot calibrate batch effects or support interpolation.  
\textit{(c) Reverse trajectory.}  
\emph{CellFlow}'s reversible velocity field can predict prior cell states from perturbed images, offering potential applications such as restoring damaged cells.
    }
    \label{fig:interpolation}
    \vspace{-1em}
\end{figure*}

\subsection{New Capabilities}

\textbf{\emph{CellFlow} addresses batch effects and reveals true perturbation effects.}  
\emph{CellFlow}’s distribution-to-distribution approach effectively addresses batch effects, a significant challenge in biological experimental data collection. As shown in Figure~\ref{fig:interpolation}a, when conditioned on two distinct control images with varying cell densities from different batches, \emph{CellFlow} consistently generates the expected perturbation effect (cell shrinkage due to mevinolin) while recapitulating batch-specific artifacts, revealing the true perturbation effect. Table~\ref{tab:ablation}d further quantifies the importance of conditioning on the same batch. By comparing generated images conditioned on control images from the same or different batches against the target perturbation images, we find that same-batch conditioning reduces overall and conditional FID by 21\%. This highlights the importance of modeling control images to more accurately capture true perturbation effects—an aspect often overlooked by prior approaches, such as diffusion models that initialize from noise (Figure~\ref{fig:interpolation}b).

\textbf{\emph{CellFlow} has the potential to model cellular morphological change trajectories.}
Cell trajectories could offer valuable information about perturbation mechanisms, but capturing them with current imaging technologies remains challenging due to their destructive nature. Since \emph{CellFlow} continuously transforms the source distribution into the target distribution, it can generate smooth interpolation paths between initial and final predicted cell states, producing video-like sequences of cellular transformation based on given source images (Figure~\ref{fig:interpolation}a). This suggests a possible approach for simulating morphological trajectories during perturbation response, which diffusion methods cannot achieve (Figure~\ref{fig:interpolation}b). Additionally, the reversible distribution transformation learned through flow matching enables \emph{CellFlow} to model backward cell state reversion (Figure~\ref{fig:interpolation}c), which could be useful for studying recovery dynamics and predicting potential treatment outcomes.

\paragraph{Summary}
Our findings provide significant insights into the influence of correctness, explanations, and refinement on evaluation accuracy and user trust in AI-based planners. 
In particular, the findings are three-fold: 
(1) The \textbf{correctness} of the generated plans is the most significant factor that impacts the evaluation accuracy and user trust in the planners. As the PDDL solver is more capable of generating correct plans, it achieves the highest evaluation accuracy and trust. 
(2) The \textbf{explanation} component of the LLM planner improves evaluation accuracy, as LLM+Expl achieves higher accuracy than LLM alone. Despite this improvement, LLM+Expl minimally impacts user trust. However, alternative explanation methods may influence user trust differently from the manually generated explanations used in our approach.
% On the other hand, explanations may help refine the trust of the planner to a more appropriate level by indicating planner shortcomings.
(3) The \textbf{refinement} procedure in the LLM planner does not lead to a significant improvement in evaluation accuracy; however, it exhibits a positive influence on user trust that may indicate an overtrust in some situations.
% This finding is aligned with prior works showing that iterative refinements based on user feedback would increase user trust~\cite{kunkel2019let, sebo2019don}.
Finally, the propensity-to-trust analysis identifies correctness as the primary determinant of user trust, whereas explanations provided limited improvement in scenarios where the planner's accuracy is diminished.

% In conclusion, our results indicate that the planner's correctness is the dominant factor for both evaluation accuracy and user trust. Therefore, selecting high-quality training data and optimizing the training procedure of AI-based planners to improve planning correctness is the top priority. Once the AI planner achieves a similar correctness level to traditional graph-search planners, strengthening its capability to explain and refine plans will further improve user trust compared to traditional planners.

\paragraph{Future Research} Future steps in this research include expanding user studies with larger sample sizes to improve generalizability and including additional planning problems per session for a more comprehensive evaluation. Next, we will explore alternative methods for generating plan explanations beyond manual creation to identify approaches that more effectively enhance user trust. 
Additionally, we will examine user trust by employing multiple LLM-based planners with varying levels of planning accuracy to better understand the interplay between planning correctness and user trust. 
Furthermore, we aim to enable real-time user-planner interaction, allowing users to provide feedback and refine plans collaboratively, thereby fostering a more dynamic and user-centric planning process.




% \newpage

{
    \small
    \bibliographystyle{ieeenat_fullname}
    \bibliography{main}
}


% WARNING: do not forget to delete the supplementary pages from your submission 
% \clearpage
\pagenumbering{gobble}
\maketitlesupplementary

\section{Additional Results on Embodied Tasks}

To evaluate the broader applicability of our EgoAgent's learned representation beyond video-conditioned 3D human motion prediction, we test its ability to improve visual policy learning for embodiments other than the human skeleton.
Following the methodology in~\cite{majumdar2023we}, we conduct experiments on the TriFinger benchmark~\cite{wuthrich2020trifinger}, which involves a three-finger robot performing two tasks: reach cube and move cube. 
We freeze the pretrained representations and use a 3-layer MLP as the policy network, training each task with 100 demonstrations.

\begin{table}[h]
\centering
\caption{Success rate (\%) on the TriFinger benchmark, where each model's pretrained representation is fixed, and additional linear layers are trained as the policy network.}
\label{tab:trifinger}
\resizebox{\linewidth}{!}{%
\begin{tabular}{llcc}
\toprule
Methods       & Training Dataset & Reach Cube & Move Cube \\
\midrule
DINO~\cite{caron2021emerging}         & WT Venice        & 78.03     & 47.42     \\
DoRA~\cite{venkataramanan2023imagenet}          & WT Venice        & 81.62     & 53.76     \\
DoRA~\cite{venkataramanan2023imagenet}          & WT All           & 82.40     & 48.13     \\
\midrule
EgoAgent-300M & WT+Ego-Exo4D      & 82.61    & 54.21      \\
EgoAgent-1B   & WT+Ego-Exo4D      & \textbf{85.72}      & \textbf{57.66}   \\
\bottomrule
\end{tabular}%
}
\end{table}

As shown in Table~\ref{tab:trifinger}, EgoAgent achieves the highest success rates on both tasks, outperforming the best models from DoRA~\cite{venkataramanan2023imagenet} with increases of +3.32\% and +3.9\% respectively.
This result shows that by incorporating human action prediction into the learning process, EgoAgent demonstrates the ability to learn more effective representations that benefit both image classification and embodied manipulation tasks.
This highlights the potential of leveraging human-centric motion data to bridge the gap between visual understanding and actionable policy learning.



\section{Additional Results on Egocentric Future State Prediction}

In this section, we provide additional qualitative results on the egocentric future state prediction task. Additionally, we describe our approach to finetune video diffusion model on the Ego-Exo4D dataset~\cite{grauman2024ego} and generate future video frames conditioned on initial frames as shown in Figure~\ref{fig:opensora_finetune}.

\begin{figure}[b]
    \centering
    \includegraphics[width=\linewidth]{figures/opensora_finetune.pdf}
    \caption{Comparison of OpenSora V1.1 first-frame-conditioned video generation results before and after finetuning on Ego-Exo4D. Fine-tuning enhances temporal consistency, but the predicted pixel-space future states still exhibit errors, such as inaccuracies in the basketball's trajectory.}
    \label{fig:opensora_finetune}
\end{figure}

\subsection{Visualizations and Comparisons}

More visualizations of our method, DoRA, and OpenSora in different scenes (as shown in Figure~\ref{fig:supp pred}). For OpenSora, when predicting the states of $t_k$, we use all the ground truth frames from $t_{0}$ to $t_{k-1}$ as conditions. As OpenSora takes only past observations as input and neglects human motion, it performs well only when the human has relatively small motions (see top cases in Figure~\ref{fig:supp pred}), but can not adjust to large movements of the human body or quick viewpoint changes (see bottom cases in Figure~\ref{fig:supp pred}).

\begin{figure*}
    \centering
    \includegraphics[width=\linewidth]{figures/supp_pred.pdf}
    \caption{Retrieval and generation results for egocentric future state prediction. Correct and wrong retrieval images are marked with green and red boundaries, respectively.}
    \label{fig:supp pred}
\end{figure*}

\begin{figure*}[t]
    \centering
    \includegraphics[width=0.9\linewidth]{figures/motion_prediction.pdf}
    \vspace{-0.5mm}
    \caption{Motion prediction results in scenes with minor changes in observation.}
    \vspace{-1.5mm}
    \label{fig:motion_prediction}
\end{figure*}

\subsection{Finetuning OpenSora on Ego-Exo4D}

OpenSora V1.1~\cite{opensora}, initially trained on internet videos and images, produces severely inconsistent results when directly applied to infer future videos on the Ego-Exo4D dataset, as illustrated in Figure~\ref{fig:opensora_finetune}.
To address the gap between general internet content and egocentric video data, we fine-tune the official checkpoint on the Ego-Exo4D training set for 50 epochs.
OpenSora V1.1 proposed a random mask strategy during training to enable video generation by image and video conditioning. We adopted the default masking rate, which applies: 75\% with no masking, 2.5\% with random masking of 1 frame to 1/4 of the total frames, 2.5\% with masking at either the beginning or the end for 1 frame to 1/4 of the total frames, and 5\% with random masking spanning 1 frame to 1/4 of the total frames at both the beginning and the end.

As shown in Fig.~\ref{fig:opensora_finetune}, despite being trained on a large dataset, OpenSora struggles to generalize to the Ego-Exo4D dataset, producing future video frames with minimal consistency relative to the conditioning frame. While fine-tuning improves temporal consistency, the moving trajectories of objects like the basketball and soccer ball still deviate from realistic physical laws. Compared with our feature space prediction results, this suggests that training world models in a reconstructive latent space is more challenging than training them in a feature space.


\section{Additional Results on 3D Human Motion Prediction}

We present additional qualitative results for the 3D human motion prediction task, highlighting a particularly challenging scenario where egocentric observations exhibit minimal variation. This scenario poses significant difficulties for video-conditioned motion prediction, as the model must effectively capture and interpret subtle changes. As demonstrated in Fig.~\ref{fig:motion_prediction}, EgoAgent successfully generates accurate predictions that closely align with the ground truth motion, showcasing its ability to handle fine-grained temporal dynamics and nuanced contextual cues.

\section{OpenSora for Image Classification}

In this section, we detail the process of extracting features from OpenSora V1.1~\cite{opensora} (without fine-tuning) for an image classification task. Following the approach of~\cite{xiang2023denoising}, we leverage the insight that diffusion models can be interpreted as multi-level denoising autoencoders. These models inherently learn linearly separable representations within their intermediate layers, without relying on auxiliary encoders. The quality of the extracted features depends on both the layer depth and the noise level applied during extraction.


\begin{table}[h]
\centering
\caption{$k$-NN evaluation results of OpenSora V1.1 features from different layer depths and noising scales on ImageNet-100. Top1 and Top5 accuracy (\%) are reported.}
\label{tab:opensora-knn}
\resizebox{0.95\linewidth}{!}{%
\begin{tabular}{lcccccc}
\toprule
\multirow{2}{*}{Timesteps} & \multicolumn{2}{c}{First Layer} & \multicolumn{2}{c}{Middle Layer} & \multicolumn{2}{c}{Last Layer} \\
\cmidrule(r){2-3}   \cmidrule(r){4-5}  \cmidrule(r){6-7}  & Top1           & Top5           & Top1            & Top5           & Top1           & Top5          \\
\midrule
32        &  6.10           & 18.20             & 34.04               & 59.50             & 30.40             & 55.74             \\
64        & 6.12              & 18.48              & 36.04               & 61.84              & 31.80         & 57.06         \\
128       & 5.84             & 18.14             & 38.08               & 64.16              & 33.44       & 58.42 \\
256       & 5.60             & 16.58              & 30.34               & 56.38              &28.14          & 52.32        \\
512       & 3.66              & 11.70            & 6.24              & 17.62              & 7.24              & 19.44  \\ 
\bottomrule
\end{tabular}%
}
\end{table}

As shown in Table~\ref{tab:opensora-knn}, we first evaluate $k$-NN classification performance on the ImageNet-100 dataset using three intermediate layers and five different noise scales. We find that a noise timestep of 128 yields the best results, with the middle and last layers performing significantly better than the first layer.
We then test this optimal configuration on ImageNet-1K and find that the last layer with 128 noising timesteps achieves the best classification accuracy.

\section{Data Preprocess}
For egocentric video sequences, we utilize videos from the Ego-Exo4D~\cite{grauman2024ego} and WT~\cite{venkataramanan2023imagenet} datasets.
The original resolution of Ego-Exo4D videos is 1408×1408, captured at 30 fps. We sample one frame every five frames and use the original resolution to crop local views (224×224) for computing the self-supervised representation loss. For computing the prediction and action loss, the videos are downsampled to 224×224 resolution.
WT primarily consists of 4K videos (3840×2160) recorded at 60 or 30 fps. Similar to Ego-Exo4D, we use the original resolution and downsample the frame rate to 6 fps for representation loss computation.
As Ego-Exo4D employs fisheye cameras, we undistort the images to a pinhole camera model using the official Project Aria Tools to align them with the WT videos.

For motion sequences, the Ego-Exo4D dataset provides synchronized 3D motion annotations and camera extrinsic parameters for various tasks and scenes. While some annotations are manually labeled, others are automatically generated using 3D motion estimation algorithms from multiple exocentric views. To maximize data utility and maintain high-quality annotations, manual labels are prioritized wherever available, and automated annotations are used only when manual labels are absent.
Each pose is converted into the egocentric camera's coordinate system using transformation matrices derived from the camera extrinsics. These matrices also enable the computation of trajectory vectors for each frame in a sequence. Beyond the x, y, z coordinates, a visibility dimension is appended to account for keypoints invisible to all exocentric views. Finally, a sliding window approach segments sequences into fixed-size windows to serve as input for the model. Note that we do not downsample the frame rate of 3D motions.

\section{Training Details}
\subsection{Architecture Configurations}
In Table~\ref{tab:arch}, we provide detailed architecture configurations for EgoAgent following the scaling-up strategy of InternLM~\cite{team2023internlm}. To ensure the generalization, we do not modify the internal modules in InternML, \emph{i.e.}, we adopt the RMSNorm and 1D RoPE. We show that, without specific modules designed for vision tasks, EgoAgent can perform well on vision and action tasks.

\begin{table}[ht]
  \centering
  \caption{Architecture configurations of EgoAgent.}
  \resizebox{0.8\linewidth}{!}{%
    \begin{tabular}{lcc}
    \toprule
          & EgoAgent-300M & EgoAgent-1B \\
          \midrule
    Depth & 22    & 22 \\
    Embedding dim & 1024  & 2048 \\
    Number of heads & 8     & 16 \\
    MLP ratio &    8/3   & 8/3 \\
    $\#$param.  & 284M & 1.13B \\
    \bottomrule
    \end{tabular}%
    }
  \label{tab:arch}%
\end{table}%

Table~\ref{tab:io_structure} presents the detailed configuration of the embedding and prediction modules in EgoAgent, including the image projector ($\text{Proj}_i$), representation head/state prediction head ($\text{MLP}_i$), action projector ($\text{Proj}_a$) and action prediction head ($\text{MLP}_a$).
Note that the representation head and the state prediction head share the same architecture but have distinct weights.

\begin{table}[t]
\centering
\caption{Architecture of the embedding ($\text{Proj}_i$, $\text{Proj}_a$) and prediction ($\text{MLP}_i$, $\text{MLP}_a$) modules in EgoAgent. For details on module connections and functions, please refer to Fig.~2 in the main paper.}
\label{tab:io_structure}
\resizebox{\linewidth}{!}{%
\begin{tabular}{lcl}
\toprule
       & \multicolumn{1}{c}{Norm \& Activation} & \multicolumn{1}{c}{Output Shape}  \\
\midrule
\multicolumn{3}{l}{$\text{Proj}_i$ (\textit{Image projector})} \\
\midrule
Input image  & -          & 3$\times$224$\times$224 \\
Conv 2D (16$\times$16) & -       & Embedding dim$\times$14$\times$14    \\
\midrule
\multicolumn{3}{l}{$\text{MLP}_i$ (\textit{State prediction head} \& \textit{Representation head)}} \\
\midrule
Input embedding  & -          & Embedding dim \\
Linear & GELU       & 2048          \\
Linear & GELU       & 2048          \\
Linear & -          & 256           \\
Linear & -          & 65536     \\
\midrule
\multicolumn{3}{l}{$\text{Proj}_a$ (\textit{Action projector})} \\
\midrule
Input pose sequence  & -          & 4$\times$5$\times$17 \\
Conv 2D (5$\times$17) & LN, GELU   & Embedding dim$\times$1$\times$1    \\
\midrule
\multicolumn{3}{l}{$\text{MLP}_a$ (\textit{Action prediction head})} \\
\midrule
Input embedding  & -          & Embedding dim$\times$1$\times$1 \\
Linear & -          & 4$\times$5$\times$17     \\
\bottomrule
\end{tabular}%
}
\end{table}


\subsection{Training Configurations}
In Table~\ref{tab:training hyper}, we provide the detailed training hyper-parameters for experiments in the main manuscripts.

\begin{table}[ht]
  \centering
  \caption{Hyper-parameters for training EgoAgent.}
  \resizebox{0.86\linewidth}{!}{%
    \begin{tabular}{lc}
    \toprule
    Training Configuration & EgoAgent-300M/1B \\
    \midrule
    Training recipe: &  \\
    optimizer & AdamW~\cite{loshchilov2017decoupled} \\
    optimizer momentum & $\beta_1=0.9, \beta_2=0.999$ \\
    \midrule
    Learning hyper-parameters: &  \\
    base learning rate & 6.0E-04 \\
    learning rate schedule & cosine \\
    base weight decay & 0.04 \\
    end weight decay & 0.4 \\
    batch size & 1920 \\
    training iters & 72,000 \\
    lr warmup iters & 1,800 \\
    warmup schedule & linear \\
    gradient clip & 1.0 \\
    data type & float16 \\
    norm epsilon & 1.0E-06 \\
    \midrule
    EMA hyper-parameters: &  \\
    momentum & 0.996 \\
    \bottomrule
    \end{tabular}%
    }
  \label{tab:training hyper}%
\end{table}%

\clearpage

\clearpage
\pagenumbering{gobble}
\maketitlesupplementary

\section{Additional Results on Embodied Tasks}

To evaluate the broader applicability of our EgoAgent's learned representation beyond video-conditioned 3D human motion prediction, we test its ability to improve visual policy learning for embodiments other than the human skeleton.
Following the methodology in~\cite{majumdar2023we}, we conduct experiments on the TriFinger benchmark~\cite{wuthrich2020trifinger}, which involves a three-finger robot performing two tasks: reach cube and move cube. 
We freeze the pretrained representations and use a 3-layer MLP as the policy network, training each task with 100 demonstrations.

\begin{table}[h]
\centering
\caption{Success rate (\%) on the TriFinger benchmark, where each model's pretrained representation is fixed, and additional linear layers are trained as the policy network.}
\label{tab:trifinger}
\resizebox{\linewidth}{!}{%
\begin{tabular}{llcc}
\toprule
Methods       & Training Dataset & Reach Cube & Move Cube \\
\midrule
DINO~\cite{caron2021emerging}         & WT Venice        & 78.03     & 47.42     \\
DoRA~\cite{venkataramanan2023imagenet}          & WT Venice        & 81.62     & 53.76     \\
DoRA~\cite{venkataramanan2023imagenet}          & WT All           & 82.40     & 48.13     \\
\midrule
EgoAgent-300M & WT+Ego-Exo4D      & 82.61    & 54.21      \\
EgoAgent-1B   & WT+Ego-Exo4D      & \textbf{85.72}      & \textbf{57.66}   \\
\bottomrule
\end{tabular}%
}
\end{table}

As shown in Table~\ref{tab:trifinger}, EgoAgent achieves the highest success rates on both tasks, outperforming the best models from DoRA~\cite{venkataramanan2023imagenet} with increases of +3.32\% and +3.9\% respectively.
This result shows that by incorporating human action prediction into the learning process, EgoAgent demonstrates the ability to learn more effective representations that benefit both image classification and embodied manipulation tasks.
This highlights the potential of leveraging human-centric motion data to bridge the gap between visual understanding and actionable policy learning.



\section{Additional Results on Egocentric Future State Prediction}

In this section, we provide additional qualitative results on the egocentric future state prediction task. Additionally, we describe our approach to finetune video diffusion model on the Ego-Exo4D dataset~\cite{grauman2024ego} and generate future video frames conditioned on initial frames as shown in Figure~\ref{fig:opensora_finetune}.

\begin{figure}[b]
    \centering
    \includegraphics[width=\linewidth]{figures/opensora_finetune.pdf}
    \caption{Comparison of OpenSora V1.1 first-frame-conditioned video generation results before and after finetuning on Ego-Exo4D. Fine-tuning enhances temporal consistency, but the predicted pixel-space future states still exhibit errors, such as inaccuracies in the basketball's trajectory.}
    \label{fig:opensora_finetune}
\end{figure}

\subsection{Visualizations and Comparisons}

More visualizations of our method, DoRA, and OpenSora in different scenes (as shown in Figure~\ref{fig:supp pred}). For OpenSora, when predicting the states of $t_k$, we use all the ground truth frames from $t_{0}$ to $t_{k-1}$ as conditions. As OpenSora takes only past observations as input and neglects human motion, it performs well only when the human has relatively small motions (see top cases in Figure~\ref{fig:supp pred}), but can not adjust to large movements of the human body or quick viewpoint changes (see bottom cases in Figure~\ref{fig:supp pred}).

\begin{figure*}
    \centering
    \includegraphics[width=\linewidth]{figures/supp_pred.pdf}
    \caption{Retrieval and generation results for egocentric future state prediction. Correct and wrong retrieval images are marked with green and red boundaries, respectively.}
    \label{fig:supp pred}
\end{figure*}

\begin{figure*}[t]
    \centering
    \includegraphics[width=0.9\linewidth]{figures/motion_prediction.pdf}
    \vspace{-0.5mm}
    \caption{Motion prediction results in scenes with minor changes in observation.}
    \vspace{-1.5mm}
    \label{fig:motion_prediction}
\end{figure*}

\subsection{Finetuning OpenSora on Ego-Exo4D}

OpenSora V1.1~\cite{opensora}, initially trained on internet videos and images, produces severely inconsistent results when directly applied to infer future videos on the Ego-Exo4D dataset, as illustrated in Figure~\ref{fig:opensora_finetune}.
To address the gap between general internet content and egocentric video data, we fine-tune the official checkpoint on the Ego-Exo4D training set for 50 epochs.
OpenSora V1.1 proposed a random mask strategy during training to enable video generation by image and video conditioning. We adopted the default masking rate, which applies: 75\% with no masking, 2.5\% with random masking of 1 frame to 1/4 of the total frames, 2.5\% with masking at either the beginning or the end for 1 frame to 1/4 of the total frames, and 5\% with random masking spanning 1 frame to 1/4 of the total frames at both the beginning and the end.

As shown in Fig.~\ref{fig:opensora_finetune}, despite being trained on a large dataset, OpenSora struggles to generalize to the Ego-Exo4D dataset, producing future video frames with minimal consistency relative to the conditioning frame. While fine-tuning improves temporal consistency, the moving trajectories of objects like the basketball and soccer ball still deviate from realistic physical laws. Compared with our feature space prediction results, this suggests that training world models in a reconstructive latent space is more challenging than training them in a feature space.


\section{Additional Results on 3D Human Motion Prediction}

We present additional qualitative results for the 3D human motion prediction task, highlighting a particularly challenging scenario where egocentric observations exhibit minimal variation. This scenario poses significant difficulties for video-conditioned motion prediction, as the model must effectively capture and interpret subtle changes. As demonstrated in Fig.~\ref{fig:motion_prediction}, EgoAgent successfully generates accurate predictions that closely align with the ground truth motion, showcasing its ability to handle fine-grained temporal dynamics and nuanced contextual cues.

\section{OpenSora for Image Classification}

In this section, we detail the process of extracting features from OpenSora V1.1~\cite{opensora} (without fine-tuning) for an image classification task. Following the approach of~\cite{xiang2023denoising}, we leverage the insight that diffusion models can be interpreted as multi-level denoising autoencoders. These models inherently learn linearly separable representations within their intermediate layers, without relying on auxiliary encoders. The quality of the extracted features depends on both the layer depth and the noise level applied during extraction.


\begin{table}[h]
\centering
\caption{$k$-NN evaluation results of OpenSora V1.1 features from different layer depths and noising scales on ImageNet-100. Top1 and Top5 accuracy (\%) are reported.}
\label{tab:opensora-knn}
\resizebox{0.95\linewidth}{!}{%
\begin{tabular}{lcccccc}
\toprule
\multirow{2}{*}{Timesteps} & \multicolumn{2}{c}{First Layer} & \multicolumn{2}{c}{Middle Layer} & \multicolumn{2}{c}{Last Layer} \\
\cmidrule(r){2-3}   \cmidrule(r){4-5}  \cmidrule(r){6-7}  & Top1           & Top5           & Top1            & Top5           & Top1           & Top5          \\
\midrule
32        &  6.10           & 18.20             & 34.04               & 59.50             & 30.40             & 55.74             \\
64        & 6.12              & 18.48              & 36.04               & 61.84              & 31.80         & 57.06         \\
128       & 5.84             & 18.14             & 38.08               & 64.16              & 33.44       & 58.42 \\
256       & 5.60             & 16.58              & 30.34               & 56.38              &28.14          & 52.32        \\
512       & 3.66              & 11.70            & 6.24              & 17.62              & 7.24              & 19.44  \\ 
\bottomrule
\end{tabular}%
}
\end{table}

As shown in Table~\ref{tab:opensora-knn}, we first evaluate $k$-NN classification performance on the ImageNet-100 dataset using three intermediate layers and five different noise scales. We find that a noise timestep of 128 yields the best results, with the middle and last layers performing significantly better than the first layer.
We then test this optimal configuration on ImageNet-1K and find that the last layer with 128 noising timesteps achieves the best classification accuracy.

\section{Data Preprocess}
For egocentric video sequences, we utilize videos from the Ego-Exo4D~\cite{grauman2024ego} and WT~\cite{venkataramanan2023imagenet} datasets.
The original resolution of Ego-Exo4D videos is 1408×1408, captured at 30 fps. We sample one frame every five frames and use the original resolution to crop local views (224×224) for computing the self-supervised representation loss. For computing the prediction and action loss, the videos are downsampled to 224×224 resolution.
WT primarily consists of 4K videos (3840×2160) recorded at 60 or 30 fps. Similar to Ego-Exo4D, we use the original resolution and downsample the frame rate to 6 fps for representation loss computation.
As Ego-Exo4D employs fisheye cameras, we undistort the images to a pinhole camera model using the official Project Aria Tools to align them with the WT videos.

For motion sequences, the Ego-Exo4D dataset provides synchronized 3D motion annotations and camera extrinsic parameters for various tasks and scenes. While some annotations are manually labeled, others are automatically generated using 3D motion estimation algorithms from multiple exocentric views. To maximize data utility and maintain high-quality annotations, manual labels are prioritized wherever available, and automated annotations are used only when manual labels are absent.
Each pose is converted into the egocentric camera's coordinate system using transformation matrices derived from the camera extrinsics. These matrices also enable the computation of trajectory vectors for each frame in a sequence. Beyond the x, y, z coordinates, a visibility dimension is appended to account for keypoints invisible to all exocentric views. Finally, a sliding window approach segments sequences into fixed-size windows to serve as input for the model. Note that we do not downsample the frame rate of 3D motions.

\section{Training Details}
\subsection{Architecture Configurations}
In Table~\ref{tab:arch}, we provide detailed architecture configurations for EgoAgent following the scaling-up strategy of InternLM~\cite{team2023internlm}. To ensure the generalization, we do not modify the internal modules in InternML, \emph{i.e.}, we adopt the RMSNorm and 1D RoPE. We show that, without specific modules designed for vision tasks, EgoAgent can perform well on vision and action tasks.

\begin{table}[ht]
  \centering
  \caption{Architecture configurations of EgoAgent.}
  \resizebox{0.8\linewidth}{!}{%
    \begin{tabular}{lcc}
    \toprule
          & EgoAgent-300M & EgoAgent-1B \\
          \midrule
    Depth & 22    & 22 \\
    Embedding dim & 1024  & 2048 \\
    Number of heads & 8     & 16 \\
    MLP ratio &    8/3   & 8/3 \\
    $\#$param.  & 284M & 1.13B \\
    \bottomrule
    \end{tabular}%
    }
  \label{tab:arch}%
\end{table}%

Table~\ref{tab:io_structure} presents the detailed configuration of the embedding and prediction modules in EgoAgent, including the image projector ($\text{Proj}_i$), representation head/state prediction head ($\text{MLP}_i$), action projector ($\text{Proj}_a$) and action prediction head ($\text{MLP}_a$).
Note that the representation head and the state prediction head share the same architecture but have distinct weights.

\begin{table}[t]
\centering
\caption{Architecture of the embedding ($\text{Proj}_i$, $\text{Proj}_a$) and prediction ($\text{MLP}_i$, $\text{MLP}_a$) modules in EgoAgent. For details on module connections and functions, please refer to Fig.~2 in the main paper.}
\label{tab:io_structure}
\resizebox{\linewidth}{!}{%
\begin{tabular}{lcl}
\toprule
       & \multicolumn{1}{c}{Norm \& Activation} & \multicolumn{1}{c}{Output Shape}  \\
\midrule
\multicolumn{3}{l}{$\text{Proj}_i$ (\textit{Image projector})} \\
\midrule
Input image  & -          & 3$\times$224$\times$224 \\
Conv 2D (16$\times$16) & -       & Embedding dim$\times$14$\times$14    \\
\midrule
\multicolumn{3}{l}{$\text{MLP}_i$ (\textit{State prediction head} \& \textit{Representation head)}} \\
\midrule
Input embedding  & -          & Embedding dim \\
Linear & GELU       & 2048          \\
Linear & GELU       & 2048          \\
Linear & -          & 256           \\
Linear & -          & 65536     \\
\midrule
\multicolumn{3}{l}{$\text{Proj}_a$ (\textit{Action projector})} \\
\midrule
Input pose sequence  & -          & 4$\times$5$\times$17 \\
Conv 2D (5$\times$17) & LN, GELU   & Embedding dim$\times$1$\times$1    \\
\midrule
\multicolumn{3}{l}{$\text{MLP}_a$ (\textit{Action prediction head})} \\
\midrule
Input embedding  & -          & Embedding dim$\times$1$\times$1 \\
Linear & -          & 4$\times$5$\times$17     \\
\bottomrule
\end{tabular}%
}
\end{table}


\subsection{Training Configurations}
In Table~\ref{tab:training hyper}, we provide the detailed training hyper-parameters for experiments in the main manuscripts.

\begin{table}[ht]
  \centering
  \caption{Hyper-parameters for training EgoAgent.}
  \resizebox{0.86\linewidth}{!}{%
    \begin{tabular}{lc}
    \toprule
    Training Configuration & EgoAgent-300M/1B \\
    \midrule
    Training recipe: &  \\
    optimizer & AdamW~\cite{loshchilov2017decoupled} \\
    optimizer momentum & $\beta_1=0.9, \beta_2=0.999$ \\
    \midrule
    Learning hyper-parameters: &  \\
    base learning rate & 6.0E-04 \\
    learning rate schedule & cosine \\
    base weight decay & 0.04 \\
    end weight decay & 0.4 \\
    batch size & 1920 \\
    training iters & 72,000 \\
    lr warmup iters & 1,800 \\
    warmup schedule & linear \\
    gradient clip & 1.0 \\
    data type & float16 \\
    norm epsilon & 1.0E-06 \\
    \midrule
    EMA hyper-parameters: &  \\
    momentum & 0.996 \\
    \bottomrule
    \end{tabular}%
    }
  \label{tab:training hyper}%
\end{table}%

\clearpage



\end{document}
