\setbox\mybox=\hbox{%
  \begin{forest}
    for tree={circle,draw}
    [a[b][c]]
  \end{forest}%
}
\tikzset{
    my node/.style={
        align=center,
        draw=black,
        inner color=white,%gray!5,
        outer color=white,%gray!10,
        thick,
        minimum width=0.9cm,
        rounded corners=3,
        text width=2.35cm,
    %    text height=4ex,
    %    text depth=1ex,
        font=\sffamily,
%        rectangle,
%       drop shadow,
    }
}
\begin{figure*}[t]
\centering 
\resizebox{0.5\textwidth}{!}{
\begin{forest}
    for tree={%
        my node,
        % where level=0{my node=blue,text width=6em}{},
        % where level=1{my node=purple,text width=5em}{},
        where level=4{text width=2em}{},
        l sep+=5pt,
        grow'=east,
        edge={gray, thick},
        parent anchor=east,
        child anchor=west,
        if n children=0{tier=last}{},
        edge path={
            \noexpand\path [draw, \forestoption{edge}] (!u.parent anchor) -- +(10pt,0) |- (.child anchor)\forestoption{edge label};
        },
        if={isodd(n_children())}{
            for children={
                if={equal(n,(n_children("!u")+1)/2)}{calign with current}{}
            }
        }{}
    }
    [ML in physical sciences
    [Numerical ML
    
    ]
    [Symbolic ML
    [Theoretical Calculations
    ]
    [Scientific Discovery
    ]
    ]
    ]
  \end{forest}
  }
  \\[5pt]
  \caption{}%Taxonomy based on the type of symbolic regression methods. $\phi$ denotes a neural network function, $W$ denotes the set of learnable parameters in NN. $\mathbf{x}$ denotes the input data, $\mathbf{z}$ denotes a reduced representation of $\mathbf{x}$, and $\mathbf{x}^{\prime}$ denotes a new representation of $\mathbf{x}$, e.g., by defining new features based on the original ones. $\mathcal{T}$ represents the final population of selected expression trees in genetic programming.}
  \label{fig:srtaxonomy}
\end{figure*}


% \tikzset{
%     my node/.style={
%         align=center,
%         draw=black,
%         inner color=white,%gray!5,
%         outer color=white,%gray!10,
%         thick,
%         minimum width=0.9cm,
%         rounded corners=3,
%     %    text width=2.7cm,
%     %    text height=4ex,
%     %    text depth=1ex,
%         font=\sffamily,
% %        rectangle,
% %       drop shadow,
%     }
% }

% \begin{figure}[h]
%     \centering
%     \begin{forest}
%     for tree={%
%         my node,
%         l sep+=5pt,
%         grow'=east,
%         edge={gray, thick},
%         child anchor=north,
%         parent anchor=south,
%         grow=south,
%         align=center,
%         edge path={
%         \noexpand\path[line width=1pt, \forestoption{edge}]
%         (!u.parent anchor)  |- ([yshift=4pt].child anchor) -- (.child anchor) \forestoption{edge label};
%         },
%         % parent anchor=east,
%         % child anchor=west,
%         if n children=0{tier=last}{},
%         % edge path={
%         %     \noexpand\path [draw, \forestoption{edge}] (!u.parent anchor) -- +(10pt,0) |- (.child anchor)\forestoption{edge label};
%         % },
%         if={isodd(n_children())}{
%             for children={
%                 if={equal(n,(n_children("!u")+1)/2)}{calign with current}{}
%             }
%         }{}
%     }
%     [SR Benchmark
%     [Ground-truth problems[Physics equations][Mathematics equations]]
%     [Real-world problems[Observations][Measurements]]
%     ]
%     \end{forest}
%     %\includegraphics[width=.9\linewidth]{plots/Taxonomy_data.png}
%     \caption{Taxonomy based on the type of SR benchmark problems.}
%     \label{fig:taxonomy_data}
% \end{figure}