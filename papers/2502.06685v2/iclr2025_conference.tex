
\documentclass{article} %
\usepackage{iclr2025_conference,times}

\usepackage{hyperref}
\usepackage{refstyle}
\usepackage{amsmath}
\usepackage{cleveref}

\usepackage{booktabs}
\usepackage{multirow} %
\usepackage{soul}%
\usepackage{tabularx}
\usepackage{enumitem}

\usepackage{amssymb}


\DeclareMathOperator*{\argmax}{argmax} %
\usepackage{pifont}
\newcommand{\cmark}{\ding{51}}%
\newcommand{\xmark}{\ding{55}}%

\usepackage{graphicx}

\interfootnotelinepenalty=10000

\crefformat{section}{\S#2#1#3}
\crefformat{subsection}{\S#2#1#3}
\crefformat{subsubsection}{\S#2#1#3}
\crefrangeformat{section}{\S#3#1#4 to~\S#5#2#6}
\crefmultiformat{section}{\S#2#1#3}{ and~\S#2#1#3}{, #2#1#3}{ and~#2#1#3}
\Crefformat{figure}{#2Fig.~#1#3}
\Crefmultiformat{figure}{Figs.~#2#1#3}{ and~#2#1#3}{, #2#1#3}{ and~#2#1#3}
\Crefformat{table}{#2Tab.~#1#3}
\Crefmultiformat{table}{Tabs.~#2#1#3}{ and~#2#1#3}{, #2#1#3}{ and~#2#1#3}
\Crefformat{appendix}{#2Appx.~\S#1#3}
\crefformat{algorithm}{Alg.~#2#1#3}
\Crefformat{equation}{#2Eq.~#1#3}

\newcommand{\todo}[1]{{\color{red}[{TODO:} #1]}}

\newcommand{\task}{\textsc{1MKR}}

%%%%% NEW MATH DEFINITIONS %%%%%

\usepackage{amsmath,amsfonts,bm}
\usepackage{derivative}
% Mark sections of captions for referring to divisions of figures
\newcommand{\figleft}{{\em (Left)}}
\newcommand{\figcenter}{{\em (Center)}}
\newcommand{\figright}{{\em (Right)}}
\newcommand{\figtop}{{\em (Top)}}
\newcommand{\figbottom}{{\em (Bottom)}}
\newcommand{\captiona}{{\em (a)}}
\newcommand{\captionb}{{\em (b)}}
\newcommand{\captionc}{{\em (c)}}
\newcommand{\captiond}{{\em (d)}}

% Highlight a newly defined term
\newcommand{\newterm}[1]{{\bf #1}}

% Derivative d 
\newcommand{\deriv}{{\mathrm{d}}}

% Figure reference, lower-case.
\def\figref#1{figure~\ref{#1}}
% Figure reference, capital. For start of sentence
\def\Figref#1{Figure~\ref{#1}}
\def\twofigref#1#2{figures \ref{#1} and \ref{#2}}
\def\quadfigref#1#2#3#4{figures \ref{#1}, \ref{#2}, \ref{#3} and \ref{#4}}
% Section reference, lower-case.
\def\secref#1{section~\ref{#1}}
% Section reference, capital.
\def\Secref#1{Section~\ref{#1}}
% Reference to two sections.
\def\twosecrefs#1#2{sections \ref{#1} and \ref{#2}}
% Reference to three sections.
\def\secrefs#1#2#3{sections \ref{#1}, \ref{#2} and \ref{#3}}
% Reference to an equation, lower-case.
\def\eqref#1{equation~\ref{#1}}
% Reference to an equation, upper case
\def\Eqref#1{Equation~\ref{#1}}
% A raw reference to an equation---avoid using if possible
\def\plaineqref#1{\ref{#1}}
% Reference to a chapter, lower-case.
\def\chapref#1{chapter~\ref{#1}}
% Reference to an equation, upper case.
\def\Chapref#1{Chapter~\ref{#1}}
% Reference to a range of chapters
\def\rangechapref#1#2{chapters\ref{#1}--\ref{#2}}
% Reference to an algorithm, lower-case.
\def\algref#1{algorithm~\ref{#1}}
% Reference to an algorithm, upper case.
\def\Algref#1{Algorithm~\ref{#1}}
\def\twoalgref#1#2{algorithms \ref{#1} and \ref{#2}}
\def\Twoalgref#1#2{Algorithms \ref{#1} and \ref{#2}}
% Reference to a part, lower case
\def\partref#1{part~\ref{#1}}
% Reference to a part, upper case
\def\Partref#1{Part~\ref{#1}}
\def\twopartref#1#2{parts \ref{#1} and \ref{#2}}

\def\ceil#1{\lceil #1 \rceil}
\def\floor#1{\lfloor #1 \rfloor}
\def\1{\bm{1}}
\newcommand{\train}{\mathcal{D}}
\newcommand{\valid}{\mathcal{D_{\mathrm{valid}}}}
\newcommand{\test}{\mathcal{D_{\mathrm{test}}}}

\def\eps{{\epsilon}}


% Random variables
\def\reta{{\textnormal{$\eta$}}}
\def\ra{{\textnormal{a}}}
\def\rb{{\textnormal{b}}}
\def\rc{{\textnormal{c}}}
\def\rd{{\textnormal{d}}}
\def\re{{\textnormal{e}}}
\def\rf{{\textnormal{f}}}
\def\rg{{\textnormal{g}}}
\def\rh{{\textnormal{h}}}
\def\ri{{\textnormal{i}}}
\def\rj{{\textnormal{j}}}
\def\rk{{\textnormal{k}}}
\def\rl{{\textnormal{l}}}
% rm is already a command, just don't name any random variables m
\def\rn{{\textnormal{n}}}
\def\ro{{\textnormal{o}}}
\def\rp{{\textnormal{p}}}
\def\rq{{\textnormal{q}}}
\def\rr{{\textnormal{r}}}
\def\rs{{\textnormal{s}}}
\def\rt{{\textnormal{t}}}
\def\ru{{\textnormal{u}}}
\def\rv{{\textnormal{v}}}
\def\rw{{\textnormal{w}}}
\def\rx{{\textnormal{x}}}
\def\ry{{\textnormal{y}}}
\def\rz{{\textnormal{z}}}

% Random vectors
\def\rvepsilon{{\mathbf{\epsilon}}}
\def\rvphi{{\mathbf{\phi}}}
\def\rvtheta{{\mathbf{\theta}}}
\def\rva{{\mathbf{a}}}
\def\rvb{{\mathbf{b}}}
\def\rvc{{\mathbf{c}}}
\def\rvd{{\mathbf{d}}}
\def\rve{{\mathbf{e}}}
\def\rvf{{\mathbf{f}}}
\def\rvg{{\mathbf{g}}}
\def\rvh{{\mathbf{h}}}
\def\rvu{{\mathbf{i}}}
\def\rvj{{\mathbf{j}}}
\def\rvk{{\mathbf{k}}}
\def\rvl{{\mathbf{l}}}
\def\rvm{{\mathbf{m}}}
\def\rvn{{\mathbf{n}}}
\def\rvo{{\mathbf{o}}}
\def\rvp{{\mathbf{p}}}
\def\rvq{{\mathbf{q}}}
\def\rvr{{\mathbf{r}}}
\def\rvs{{\mathbf{s}}}
\def\rvt{{\mathbf{t}}}
\def\rvu{{\mathbf{u}}}
\def\rvv{{\mathbf{v}}}
\def\rvw{{\mathbf{w}}}
\def\rvx{{\mathbf{x}}}
\def\rvy{{\mathbf{y}}}
\def\rvz{{\mathbf{z}}}

% Elements of random vectors
\def\erva{{\textnormal{a}}}
\def\ervb{{\textnormal{b}}}
\def\ervc{{\textnormal{c}}}
\def\ervd{{\textnormal{d}}}
\def\erve{{\textnormal{e}}}
\def\ervf{{\textnormal{f}}}
\def\ervg{{\textnormal{g}}}
\def\ervh{{\textnormal{h}}}
\def\ervi{{\textnormal{i}}}
\def\ervj{{\textnormal{j}}}
\def\ervk{{\textnormal{k}}}
\def\ervl{{\textnormal{l}}}
\def\ervm{{\textnormal{m}}}
\def\ervn{{\textnormal{n}}}
\def\ervo{{\textnormal{o}}}
\def\ervp{{\textnormal{p}}}
\def\ervq{{\textnormal{q}}}
\def\ervr{{\textnormal{r}}}
\def\ervs{{\textnormal{s}}}
\def\ervt{{\textnormal{t}}}
\def\ervu{{\textnormal{u}}}
\def\ervv{{\textnormal{v}}}
\def\ervw{{\textnormal{w}}}
\def\ervx{{\textnormal{x}}}
\def\ervy{{\textnormal{y}}}
\def\ervz{{\textnormal{z}}}

% Random matrices
\def\rmA{{\mathbf{A}}}
\def\rmB{{\mathbf{B}}}
\def\rmC{{\mathbf{C}}}
\def\rmD{{\mathbf{D}}}
\def\rmE{{\mathbf{E}}}
\def\rmF{{\mathbf{F}}}
\def\rmG{{\mathbf{G}}}
\def\rmH{{\mathbf{H}}}
\def\rmI{{\mathbf{I}}}
\def\rmJ{{\mathbf{J}}}
\def\rmK{{\mathbf{K}}}
\def\rmL{{\mathbf{L}}}
\def\rmM{{\mathbf{M}}}
\def\rmN{{\mathbf{N}}}
\def\rmO{{\mathbf{O}}}
\def\rmP{{\mathbf{P}}}
\def\rmQ{{\mathbf{Q}}}
\def\rmR{{\mathbf{R}}}
\def\rmS{{\mathbf{S}}}
\def\rmT{{\mathbf{T}}}
\def\rmU{{\mathbf{U}}}
\def\rmV{{\mathbf{V}}}
\def\rmW{{\mathbf{W}}}
\def\rmX{{\mathbf{X}}}
\def\rmY{{\mathbf{Y}}}
\def\rmZ{{\mathbf{Z}}}

% Elements of random matrices
\def\ermA{{\textnormal{A}}}
\def\ermB{{\textnormal{B}}}
\def\ermC{{\textnormal{C}}}
\def\ermD{{\textnormal{D}}}
\def\ermE{{\textnormal{E}}}
\def\ermF{{\textnormal{F}}}
\def\ermG{{\textnormal{G}}}
\def\ermH{{\textnormal{H}}}
\def\ermI{{\textnormal{I}}}
\def\ermJ{{\textnormal{J}}}
\def\ermK{{\textnormal{K}}}
\def\ermL{{\textnormal{L}}}
\def\ermM{{\textnormal{M}}}
\def\ermN{{\textnormal{N}}}
\def\ermO{{\textnormal{O}}}
\def\ermP{{\textnormal{P}}}
\def\ermQ{{\textnormal{Q}}}
\def\ermR{{\textnormal{R}}}
\def\ermS{{\textnormal{S}}}
\def\ermT{{\textnormal{T}}}
\def\ermU{{\textnormal{U}}}
\def\ermV{{\textnormal{V}}}
\def\ermW{{\textnormal{W}}}
\def\ermX{{\textnormal{X}}}
\def\ermY{{\textnormal{Y}}}
\def\ermZ{{\textnormal{Z}}}

% Vectors
\def\vzero{{\bm{0}}}
\def\vone{{\bm{1}}}
\def\vmu{{\bm{\mu}}}
\def\vtheta{{\bm{\theta}}}
\def\vphi{{\bm{\phi}}}
\def\va{{\bm{a}}}
\def\vb{{\bm{b}}}
\def\vc{{\bm{c}}}
\def\vd{{\bm{d}}}
\def\ve{{\bm{e}}}
\def\vf{{\bm{f}}}
\def\vg{{\bm{g}}}
\def\vh{{\bm{h}}}
\def\vi{{\bm{i}}}
\def\vj{{\bm{j}}}
\def\vk{{\bm{k}}}
\def\vl{{\bm{l}}}
\def\vm{{\bm{m}}}
\def\vn{{\bm{n}}}
\def\vo{{\bm{o}}}
\def\vp{{\bm{p}}}
\def\vq{{\bm{q}}}
\def\vr{{\bm{r}}}
\def\vs{{\bm{s}}}
\def\vt{{\bm{t}}}
\def\vu{{\bm{u}}}
\def\vv{{\bm{v}}}
\def\vw{{\bm{w}}}
\def\vx{{\bm{x}}}
\def\vy{{\bm{y}}}
\def\vz{{\bm{z}}}

% Elements of vectors
\def\evalpha{{\alpha}}
\def\evbeta{{\beta}}
\def\evepsilon{{\epsilon}}
\def\evlambda{{\lambda}}
\def\evomega{{\omega}}
\def\evmu{{\mu}}
\def\evpsi{{\psi}}
\def\evsigma{{\sigma}}
\def\evtheta{{\theta}}
\def\eva{{a}}
\def\evb{{b}}
\def\evc{{c}}
\def\evd{{d}}
\def\eve{{e}}
\def\evf{{f}}
\def\evg{{g}}
\def\evh{{h}}
\def\evi{{i}}
\def\evj{{j}}
\def\evk{{k}}
\def\evl{{l}}
\def\evm{{m}}
\def\evn{{n}}
\def\evo{{o}}
\def\evp{{p}}
\def\evq{{q}}
\def\evr{{r}}
\def\evs{{s}}
\def\evt{{t}}
\def\evu{{u}}
\def\evv{{v}}
\def\evw{{w}}
\def\evx{{x}}
\def\evy{{y}}
\def\evz{{z}}

% Matrix
\def\mA{{\bm{A}}}
\def\mB{{\bm{B}}}
\def\mC{{\bm{C}}}
\def\mD{{\bm{D}}}
\def\mE{{\bm{E}}}
\def\mF{{\bm{F}}}
\def\mG{{\bm{G}}}
\def\mH{{\bm{H}}}
\def\mI{{\bm{I}}}
\def\mJ{{\bm{J}}}
\def\mK{{\bm{K}}}
\def\mL{{\bm{L}}}
\def\mM{{\bm{M}}}
\def\mN{{\bm{N}}}
\def\mO{{\bm{O}}}
\def\mP{{\bm{P}}}
\def\mQ{{\bm{Q}}}
\def\mR{{\bm{R}}}
\def\mS{{\bm{S}}}
\def\mT{{\bm{T}}}
\def\mU{{\bm{U}}}
\def\mV{{\bm{V}}}
\def\mW{{\bm{W}}}
\def\mX{{\bm{X}}}
\def\mY{{\bm{Y}}}
\def\mZ{{\bm{Z}}}
\def\mBeta{{\bm{\beta}}}
\def\mPhi{{\bm{\Phi}}}
\def\mLambda{{\bm{\Lambda}}}
\def\mSigma{{\bm{\Sigma}}}

% Tensor
\DeclareMathAlphabet{\mathsfit}{\encodingdefault}{\sfdefault}{m}{sl}
\SetMathAlphabet{\mathsfit}{bold}{\encodingdefault}{\sfdefault}{bx}{n}
\newcommand{\tens}[1]{\bm{\mathsfit{#1}}}
\def\tA{{\tens{A}}}
\def\tB{{\tens{B}}}
\def\tC{{\tens{C}}}
\def\tD{{\tens{D}}}
\def\tE{{\tens{E}}}
\def\tF{{\tens{F}}}
\def\tG{{\tens{G}}}
\def\tH{{\tens{H}}}
\def\tI{{\tens{I}}}
\def\tJ{{\tens{J}}}
\def\tK{{\tens{K}}}
\def\tL{{\tens{L}}}
\def\tM{{\tens{M}}}
\def\tN{{\tens{N}}}
\def\tO{{\tens{O}}}
\def\tP{{\tens{P}}}
\def\tQ{{\tens{Q}}}
\def\tR{{\tens{R}}}
\def\tS{{\tens{S}}}
\def\tT{{\tens{T}}}
\def\tU{{\tens{U}}}
\def\tV{{\tens{V}}}
\def\tW{{\tens{W}}}
\def\tX{{\tens{X}}}
\def\tY{{\tens{Y}}}
\def\tZ{{\tens{Z}}}


% Graph
\def\gA{{\mathcal{A}}}
\def\gB{{\mathcal{B}}}
\def\gC{{\mathcal{C}}}
\def\gD{{\mathcal{D}}}
\def\gE{{\mathcal{E}}}
\def\gF{{\mathcal{F}}}
\def\gG{{\mathcal{G}}}
\def\gH{{\mathcal{H}}}
\def\gI{{\mathcal{I}}}
\def\gJ{{\mathcal{J}}}
\def\gK{{\mathcal{K}}}
\def\gL{{\mathcal{L}}}
\def\gM{{\mathcal{M}}}
\def\gN{{\mathcal{N}}}
\def\gO{{\mathcal{O}}}
\def\gP{{\mathcal{P}}}
\def\gQ{{\mathcal{Q}}}
\def\gR{{\mathcal{R}}}
\def\gS{{\mathcal{S}}}
\def\gT{{\mathcal{T}}}
\def\gU{{\mathcal{U}}}
\def\gV{{\mathcal{V}}}
\def\gW{{\mathcal{W}}}
\def\gX{{\mathcal{X}}}
\def\gY{{\mathcal{Y}}}
\def\gZ{{\mathcal{Z}}}

% Sets
\def\sA{{\mathbb{A}}}
\def\sB{{\mathbb{B}}}
\def\sC{{\mathbb{C}}}
\def\sD{{\mathbb{D}}}
% Don't use a set called E, because this would be the same as our symbol
% for expectation.
\def\sF{{\mathbb{F}}}
\def\sG{{\mathbb{G}}}
\def\sH{{\mathbb{H}}}
\def\sI{{\mathbb{I}}}
\def\sJ{{\mathbb{J}}}
\def\sK{{\mathbb{K}}}
\def\sL{{\mathbb{L}}}
\def\sM{{\mathbb{M}}}
\def\sN{{\mathbb{N}}}
\def\sO{{\mathbb{O}}}
\def\sP{{\mathbb{P}}}
\def\sQ{{\mathbb{Q}}}
\def\sR{{\mathbb{R}}}
\def\sS{{\mathbb{S}}}
\def\sT{{\mathbb{T}}}
\def\sU{{\mathbb{U}}}
\def\sV{{\mathbb{V}}}
\def\sW{{\mathbb{W}}}
\def\sX{{\mathbb{X}}}
\def\sY{{\mathbb{Y}}}
\def\sZ{{\mathbb{Z}}}

% Entries of a matrix
\def\emLambda{{\Lambda}}
\def\emA{{A}}
\def\emB{{B}}
\def\emC{{C}}
\def\emD{{D}}
\def\emE{{E}}
\def\emF{{F}}
\def\emG{{G}}
\def\emH{{H}}
\def\emI{{I}}
\def\emJ{{J}}
\def\emK{{K}}
\def\emL{{L}}
\def\emM{{M}}
\def\emN{{N}}
\def\emO{{O}}
\def\emP{{P}}
\def\emQ{{Q}}
\def\emR{{R}}
\def\emS{{S}}
\def\emT{{T}}
\def\emU{{U}}
\def\emV{{V}}
\def\emW{{W}}
\def\emX{{X}}
\def\emY{{Y}}
\def\emZ{{Z}}
\def\emSigma{{\Sigma}}

% entries of a tensor
% Same font as tensor, without \bm wrapper
\newcommand{\etens}[1]{\mathsfit{#1}}
\def\etLambda{{\etens{\Lambda}}}
\def\etA{{\etens{A}}}
\def\etB{{\etens{B}}}
\def\etC{{\etens{C}}}
\def\etD{{\etens{D}}}
\def\etE{{\etens{E}}}
\def\etF{{\etens{F}}}
\def\etG{{\etens{G}}}
\def\etH{{\etens{H}}}
\def\etI{{\etens{I}}}
\def\etJ{{\etens{J}}}
\def\etK{{\etens{K}}}
\def\etL{{\etens{L}}}
\def\etM{{\etens{M}}}
\def\etN{{\etens{N}}}
\def\etO{{\etens{O}}}
\def\etP{{\etens{P}}}
\def\etQ{{\etens{Q}}}
\def\etR{{\etens{R}}}
\def\etS{{\etens{S}}}
\def\etT{{\etens{T}}}
\def\etU{{\etens{U}}}
\def\etV{{\etens{V}}}
\def\etW{{\etens{W}}}
\def\etX{{\etens{X}}}
\def\etY{{\etens{Y}}}
\def\etZ{{\etens{Z}}}

% The true underlying data generating distribution
\newcommand{\pdata}{p_{\rm{data}}}
\newcommand{\ptarget}{p_{\rm{target}}}
\newcommand{\pprior}{p_{\rm{prior}}}
\newcommand{\pbase}{p_{\rm{base}}}
\newcommand{\pref}{p_{\rm{ref}}}

% The empirical distribution defined by the training set
\newcommand{\ptrain}{\hat{p}_{\rm{data}}}
\newcommand{\Ptrain}{\hat{P}_{\rm{data}}}
% The model distribution
\newcommand{\pmodel}{p_{\rm{model}}}
\newcommand{\Pmodel}{P_{\rm{model}}}
\newcommand{\ptildemodel}{\tilde{p}_{\rm{model}}}
% Stochastic autoencoder distributions
\newcommand{\pencode}{p_{\rm{encoder}}}
\newcommand{\pdecode}{p_{\rm{decoder}}}
\newcommand{\precons}{p_{\rm{reconstruct}}}

\newcommand{\laplace}{\mathrm{Laplace}} % Laplace distribution

\newcommand{\E}{\mathbb{E}}
\newcommand{\Ls}{\mathcal{L}}
\newcommand{\R}{\mathbb{R}}
\newcommand{\emp}{\tilde{p}}
\newcommand{\lr}{\alpha}
\newcommand{\reg}{\lambda}
\newcommand{\rect}{\mathrm{rectifier}}
\newcommand{\softmax}{\mathrm{softmax}}
\newcommand{\sigmoid}{\sigma}
\newcommand{\softplus}{\zeta}
\newcommand{\KL}{D_{\mathrm{KL}}}
\newcommand{\Var}{\mathrm{Var}}
\newcommand{\standarderror}{\mathrm{SE}}
\newcommand{\Cov}{\mathrm{Cov}}
% Wolfram Mathworld says $L^2$ is for function spaces and $\ell^2$ is for vectors
% But then they seem to use $L^2$ for vectors throughout the site, and so does
% wikipedia.
\newcommand{\normlzero}{L^0}
\newcommand{\normlone}{L^1}
\newcommand{\normltwo}{L^2}
\newcommand{\normlp}{L^p}
\newcommand{\normmax}{L^\infty}

\newcommand{\parents}{Pa} % See usage in notation.tex. Chosen to match Daphne's book.

\DeclareMathOperator*{\argmax}{arg\,max}
\DeclareMathOperator*{\argmin}{arg\,min}

\DeclareMathOperator{\sign}{sign}
\DeclareMathOperator{\Tr}{Tr}
\let\ab\allowbreak


\usepackage{hyperref}
\usepackage{url}
\usepackage{graphicx}
\usepackage{makecell}
\usepackage{soul}
\usepackage{afterpage}


\hypersetup{
   breaklinks=true,   %
   colorlinks=true,   %
}
\definecolor{red}{HTML}{ca0020}
\definecolor{lightred}{HTML}{f4a582}
\definecolor{lightblue}{HTML}{92c5de}
\definecolor{green}{HTML}{008837}
\definecolor{blue}{HTML}{2c7bb6}


\def\ptarget{p_\text{target}}
\def\ptilde{\tilde{p}_\text{target}}

\title{No Trick, No Treat: \\Pursuits and Challenges Towards \\Simulation-free Training of Neural Samplers}


\author{Jiajun He$^{*, 1}$\thanks{Equal Contribution. Corresponding to \texttt{jh2383@cam.ac.uk}, \texttt{yuanqidu@cs.cornell.edu}},
Yuanqi Du$^{*, 2}$,
Francisco Vargas$^{1,3}$,
Dinghuai Zhang$^{4}$, \\
\AND
Shreyas Padhy$^{1}$,
RuiKang OuYang$^{1}$,
Carla Gomes$^{2}$,
José Miguel Hernández-Lobato$^{1}$
\Note $^1$University of Cambridge, $^2$Cornell University, $^3$Xaira Therapeutics, $^4$Microsoft Research
}



\newcommand{\fix}{\marginpar{FIX}}
\newcommand{\new}{\marginpar{NEW}}
\newcommand{\JJ}[1]{\textcolor{orange}{JH: #1}}
\newcommand{\SP}[1]{\textcolor{red}{(SP: #1)}}
 \iclrfinalcopy
\begin{document}


\maketitle

\begin{abstract}
We consider the \textit{sampling} problem, where the aim is to draw samples from a distribution whose density is known only up to a normalization constant. 
Recent breakthroughs in generative modeling to approximate a high-dimensional data distribution have sparked significant interest in developing neural network-based methods for this challenging problem. 
However, neural samplers typically incur heavy computational overhead due to simulating trajectories during training. 
This motivates the pursuit of \textit{simulation-free} training procedures of neural samplers. 
In this work, we propose an elegant modification to previous methods, which allows simulation-free training with the help of a time-dependent normalizing flow. 
However, it ultimately suffers from severe mode collapse. 
On closer inspection, we find that nearly all successful neural samplers rely on Langevin preconditioning to avoid mode collapsing. 
We systematically analyze several popular methods with various objective functions and demonstrate that, in the absence of  Langevin preconditioning, most of them fail to adequately cover even a simple target. 
Finally, we draw attention to a strong baseline by combining the state-of-the-art MCMC method, Parallel Tempering (PT), with an additional generative model to shed light on future explorations of neural samplers.
\end{abstract}

\section{Introduction}
\label{sec:introduction}
The business processes of organizations are experiencing ever-increasing complexity due to the large amount of data, high number of users, and high-tech devices involved \cite{martin2021pmopportunitieschallenges, beerepoot2023biggestbpmproblems}. This complexity may cause business processes to deviate from normal control flow due to unforeseen and disruptive anomalies \cite{adams2023proceddsriftdetection}. These control-flow anomalies manifest as unknown, skipped, and wrongly-ordered activities in the traces of event logs monitored from the execution of business processes \cite{ko2023adsystematicreview}. For the sake of clarity, let us consider an illustrative example of such anomalies. Figure \ref{FP_ANOMALIES} shows a so-called event log footprint, which captures the control flow relations of four activities of a hypothetical event log. In particular, this footprint captures the control-flow relations between activities \texttt{a}, \texttt{b}, \texttt{c} and \texttt{d}. These are the causal ($\rightarrow$) relation, concurrent ($\parallel$) relation, and other ($\#$) relations such as exclusivity or non-local dependency \cite{aalst2022pmhandbook}. In addition, on the right are six traces, of which five exhibit skipped, wrongly-ordered and unknown control-flow anomalies. For example, $\langle$\texttt{a b d}$\rangle$ has a skipped activity, which is \texttt{c}. Because of this skipped activity, the control-flow relation \texttt{b}$\,\#\,$\texttt{d} is violated, since \texttt{d} directly follows \texttt{b} in the anomalous trace.
\begin{figure}[!t]
\centering
\includegraphics[width=0.9\columnwidth]{images/FP_ANOMALIES.png}
\caption{An example event log footprint with six traces, of which five exhibit control-flow anomalies.}
\label{FP_ANOMALIES}
\end{figure}

\subsection{Control-flow anomaly detection}
Control-flow anomaly detection techniques aim to characterize the normal control flow from event logs and verify whether these deviations occur in new event logs \cite{ko2023adsystematicreview}. To develop control-flow anomaly detection techniques, \revision{process mining} has seen widespread adoption owing to process discovery and \revision{conformance checking}. On the one hand, process discovery is a set of algorithms that encode control-flow relations as a set of model elements and constraints according to a given modeling formalism \cite{aalst2022pmhandbook}; hereafter, we refer to the Petri net, a widespread modeling formalism. On the other hand, \revision{conformance checking} is an explainable set of algorithms that allows linking any deviations with the reference Petri net and providing the fitness measure, namely a measure of how much the Petri net fits the new event log \cite{aalst2022pmhandbook}. Many control-flow anomaly detection techniques based on \revision{conformance checking} (hereafter, \revision{conformance checking}-based techniques) use the fitness measure to determine whether an event log is anomalous \cite{bezerra2009pmad, bezerra2013adlogspais, myers2018icsadpm, pecchia2020applicationfailuresanalysispm}. 

The scientific literature also includes many \revision{conformance checking}-independent techniques for control-flow anomaly detection that combine specific types of trace encodings with machine/deep learning \cite{ko2023adsystematicreview, tavares2023pmtraceencoding}. Whereas these techniques are very effective, their explainability is challenging due to both the type of trace encoding employed and the machine/deep learning model used \cite{rawal2022trustworthyaiadvances,li2023explainablead}. Hence, in the following, we focus on the shortcomings of \revision{conformance checking}-based techniques to investigate whether it is possible to support the development of competitive control-flow anomaly detection techniques while maintaining the explainable nature of \revision{conformance checking}.
\begin{figure}[!t]
\centering
\includegraphics[width=\columnwidth]{images/HIGH_LEVEL_VIEW.png}
\caption{A high-level view of the proposed framework for combining \revision{process mining}-based feature extraction with dimensionality reduction for control-flow anomaly detection.}
\label{HIGH_LEVEL_VIEW}
\end{figure}

\subsection{Shortcomings of \revision{conformance checking}-based techniques}
Unfortunately, the detection effectiveness of \revision{conformance checking}-based techniques is affected by noisy data and low-quality Petri nets, which may be due to human errors in the modeling process or representational bias of process discovery algorithms \cite{bezerra2013adlogspais, pecchia2020applicationfailuresanalysispm, aalst2016pm}. Specifically, on the one hand, noisy data may introduce infrequent and deceptive control-flow relations that may result in inconsistent fitness measures, whereas, on the other hand, checking event logs against a low-quality Petri net could lead to an unreliable distribution of fitness measures. Nonetheless, such Petri nets can still be used as references to obtain insightful information for \revision{process mining}-based feature extraction, supporting the development of competitive and explainable \revision{conformance checking}-based techniques for control-flow anomaly detection despite the problems above. For example, a few works outline that token-based \revision{conformance checking} can be used for \revision{process mining}-based feature extraction to build tabular data and develop effective \revision{conformance checking}-based techniques for control-flow anomaly detection \cite{singh2022lapmsh, debenedictis2023dtadiiot}. However, to the best of our knowledge, the scientific literature lacks a structured proposal for \revision{process mining}-based feature extraction using the state-of-the-art \revision{conformance checking} variant, namely alignment-based \revision{conformance checking}.

\subsection{Contributions}
We propose a novel \revision{process mining}-based feature extraction approach with alignment-based \revision{conformance checking}. This variant aligns the deviating control flow with a reference Petri net; the resulting alignment can be inspected to extract additional statistics such as the number of times a given activity caused mismatches \cite{aalst2022pmhandbook}. We integrate this approach into a flexible and explainable framework for developing techniques for control-flow anomaly detection. The framework combines \revision{process mining}-based feature extraction and dimensionality reduction to handle high-dimensional feature sets, achieve detection effectiveness, and support explainability. Notably, in addition to our proposed \revision{process mining}-based feature extraction approach, the framework allows employing other approaches, enabling a fair comparison of multiple \revision{conformance checking}-based and \revision{conformance checking}-independent techniques for control-flow anomaly detection. Figure \ref{HIGH_LEVEL_VIEW} shows a high-level view of the framework. Business processes are monitored, and event logs obtained from the database of information systems. Subsequently, \revision{process mining}-based feature extraction is applied to these event logs and tabular data input to dimensionality reduction to identify control-flow anomalies. We apply several \revision{conformance checking}-based and \revision{conformance checking}-independent framework techniques to publicly available datasets, simulated data of a case study from railways, and real-world data of a case study from healthcare. We show that the framework techniques implementing our approach outperform the baseline \revision{conformance checking}-based techniques while maintaining the explainable nature of \revision{conformance checking}.

In summary, the contributions of this paper are as follows.
\begin{itemize}
    \item{
        A novel \revision{process mining}-based feature extraction approach to support the development of competitive and explainable \revision{conformance checking}-based techniques for control-flow anomaly detection.
    }
    \item{
        A flexible and explainable framework for developing techniques for control-flow anomaly detection using \revision{process mining}-based feature extraction and dimensionality reduction.
    }
    \item{
        Application to synthetic and real-world datasets of several \revision{conformance checking}-based and \revision{conformance checking}-independent framework techniques, evaluating their detection effectiveness and explainability.
    }
\end{itemize}

The rest of the paper is organized as follows.
\begin{itemize}
    \item Section \ref{sec:related_work} reviews the existing techniques for control-flow anomaly detection, categorizing them into \revision{conformance checking}-based and \revision{conformance checking}-independent techniques.
    \item Section \ref{sec:abccfe} provides the preliminaries of \revision{process mining} to establish the notation used throughout the paper, and delves into the details of the proposed \revision{process mining}-based feature extraction approach with alignment-based \revision{conformance checking}.
    \item Section \ref{sec:framework} describes the framework for developing \revision{conformance checking}-based and \revision{conformance checking}-independent techniques for control-flow anomaly detection that combine \revision{process mining}-based feature extraction and dimensionality reduction.
    \item Section \ref{sec:evaluation} presents the experiments conducted with multiple framework and baseline techniques using data from publicly available datasets and case studies.
    \item Section \ref{sec:conclusions} draws the conclusions and presents future work.
\end{itemize}

\section{Method}\label{sec:method}
\begin{figure}
    \centering
    \includegraphics[width=0.85\textwidth]{imgs/heatmap_acc.pdf}
    \caption{\textbf{Visualization of the proposed periodic Bayesian flow with mean parameter $\mu$ and accumulated accuracy parameter $c$ which corresponds to the entropy/uncertainty}. For $x = 0.3, \beta(1) = 1000$ and $\alpha_i$ defined in \cref{appd:bfn_cir}, this figure plots three colored stochastic parameter trajectories for receiver mean parameter $m$ and accumulated accuracy parameter $c$, superimposed on a log-scale heatmap of the Bayesian flow distribution $p_F(m|x,\senderacc)$ and $p_F(c|x,\senderacc)$. Note the \emph{non-monotonicity} and \emph{non-additive} property of $c$ which could inform the network the entropy of the mean parameter $m$ as a condition and the \emph{periodicity} of $m$. %\jj{Shrink the figures to save space}\hanlin{Do we need to make this figure one-column?}
    }
    \label{fig:vmbf_vis}
    \vskip -0.1in
\end{figure}
% \begin{wrapfigure}{r}{0.5\textwidth}
%     \centering
%     \includegraphics[width=0.49\textwidth]{imgs/heatmap_acc.pdf}
%     \caption{\textbf{Visualization of hyper-torus Bayesian flow based on von Mises Distribution}. For $x = 0.3, \beta(1) = 1000$ and $\alpha_i$ defined in \cref{appd:bfn_cir}, this figure plots three colored stochastic parameter trajectories for receiver mean parameter $m$ and accumulated accuracy parameter $c$, superimposed on a log-scale heatmap of the Bayesian flow distribution $p_F(m|x,\senderacc)$ and $p_F(c|x,\senderacc)$. Note the \emph{non-monotonicity} and \emph{non-additive} property of $c$. \jj{Shrink the figures to save space}}
%     \label{fig:vmbf_vis}
%     \vspace{-30pt}
% \end{wrapfigure}


In this section, we explain the detailed design of CrysBFN tackling theoretical and practical challenges. First, we describe how to derive our new formulation of Bayesian Flow Networks over hyper-torus $\mathbb{T}^{D}$ from scratch. Next, we illustrate the two key differences between \modelname and the original form of BFN: $1)$ a meticulously designed novel base distribution with different Bayesian update rules; and $2)$ different properties over the accuracy scheduling resulted from the periodicity and the new Bayesian update rules. Then, we present in detail the overall framework of \modelname over each manifold of the crystal space (\textit{i.e.} fractional coordinates, lattice vectors, atom types) respecting \textit{periodic E(3) invariance}. 

% In this section, we first demonstrate how to build Bayesian flow on hyper-torus $\mathbb{T}^{D}$ by overcoming theoretical and practical problems to provide a low-noise parameter-space approach to fractional atom coordinate generation. Next, we present how \modelname models each manifold of crystal space respecting \textit{periodic E(3) invariance}. 

\subsection{Periodic Bayesian Flow on Hyper-torus \texorpdfstring{$\mathbb{T}^{D}$}{}} 
For generative modeling of fractional coordinates in crystal, we first construct a periodic Bayesian flow on \texorpdfstring{$\mathbb{T}^{D}$}{} by designing every component of the totally new Bayesian update process which we demonstrate to be distinct from the original Bayesian flow (please see \cref{fig:non_add}). 
 %:) 
 
 The fractional atom coordinate system \citep{jiao2023crystal} inherently distributes over a hyper-torus support $\mathbb{T}^{3\times N}$. Hence, the normal distribution support on $\R$ used in the original \citep{bfn} is not suitable for this scenario. 
% The key problem of generative modeling for crystal is the periodicity of Cartesian atom coordinates $\vX$ requiring:
% \begin{equation}\label{eq:periodcity}
% p(\vA,\vL,\vX)=p(\vA,\vL,\vX+\vec{LK}),\text{where}~\vec{K}=\vec{k}\vec{1}_{1\times N},\forall\vec{k}\in\mathbb{Z}^{3\times1}
% \end{equation}
% However, there does not exist such a distribution supporting on $\R$ to model such property because the integration of such distribution over $\R$ will not be finite and equal to 1. Therefore, the normal distribution used in \citet{bfn} can not meet this condition.

To tackle this problem, the circular distribution~\citep{mardia2009directional} over the finite interval $[-\pi,\pi)$ is a natural choice as the base distribution for deriving the BFN on $\mathbb{T}^D$. 
% one natural choice is to 
% we would like to consider the circular distribution over the finite interval as the base 
% we find that circular distributions \citep{mardia2009directional} defined on a finite interval with lengths of $2\pi$ can be used as the instantiation of input distribution for the BFN on $\mathbb{T}^D$.
Specifically, circular distributions enjoy desirable periodic properties: $1)$ the integration over any interval length of $2\pi$ equals 1; $2)$ the probability distribution function is periodic with period $2\pi$.  Sharing the same intrinsic with fractional coordinates, such periodic property of circular distribution makes it suitable for the instantiation of BFN's input distribution, in parameterizing the belief towards ground truth $\x$ on $\mathbb{T}^D$. 
% \yuxuan{this is very complicated from my perspective.} \hanlin{But this property is exactly beautiful and perfectly fit into the BFN.}

\textbf{von Mises Distribution and its Bayesian Update} We choose von Mises distribution \citep{mardia2009directional} from various circular distributions as the form of input distribution, based on the appealing conjugacy property required in the derivation of the BFN framework.
% to leverage the Bayesian conjugacy property of von Mises distribution which is required by the BFN framework. 
That is, the posterior of a von Mises distribution parameterized likelihood is still in the family of von Mises distributions. The probability density function of von Mises distribution with mean direction parameter $m$ and concentration parameter $c$ (describing the entropy/uncertainty of $m$) is defined as: 
\begin{equation}
f(x|m,c)=vM(x|m,c)=\frac{\exp(c\cos(x-m))}{2\pi I_0(c)}
\end{equation}
where $I_0(c)$ is zeroth order modified Bessel function of the first kind as the normalizing constant. Given the last univariate belief parameterized by von Mises distribution with parameter $\theta_{i-1}=\{m_{i-1},\ c_{i-1}\}$ and the sample $y$ from sender distribution with unknown data sample $x$ and known accuracy $\alpha$ describing the entropy/uncertainty of $y$,  Bayesian update for the receiver is deducted as:
\begin{equation}
 h(\{m_{i-1},c_{i-1}\},y,\alpha)=\{m_i,c_i \}, \text{where}
\end{equation}
\begin{equation}\label{eq:h_m}
m_i=\text{atan2}(\alpha\sin y+c_{i-1}\sin m_{i-1}, {\alpha\cos y+c_{i-1}\cos m_{i-1}})
\end{equation}
\begin{equation}\label{eq:h_c}
c_i =\sqrt{\alpha^2+c_{i-1}^2+2\alpha c_{i-1}\cos(y-m_{i-1})}
\end{equation}
The proof of the above equations can be found in \cref{apdx:bayesian_update_function}. The atan2 function refers to  2-argument arctangent. Independently conducting  Bayesian update for each dimension, we can obtain the Bayesian update distribution by marginalizing $\y$:
\begin{equation}
p_U(\vtheta'|\vtheta,\bold{x};\alpha)=\mathbb{E}_{p_S(\bold{y}|\bold{x};\alpha)}\delta(\vtheta'-h(\vtheta,\bold{y},\alpha))=\mathbb{E}_{vM(\bold{y}|\bold{x},\alpha)}\delta(\vtheta'-h(\vtheta,\bold{y},\alpha))
\end{equation} 
\begin{figure}
    \centering
    \vskip -0.15in
    \includegraphics[width=0.95\linewidth]{imgs/non_add.pdf}
    \caption{An intuitive illustration of non-additive accuracy Bayesian update on the torus. The lengths of arrows represent the uncertainty/entropy of the belief (\emph{e.g.}~$1/\sigma^2$ for Gaussian and $c$ for von Mises). The directions of the arrows represent the believed location (\emph{e.g.}~ $\mu$ for Gaussian and $m$ for von Mises).}
    \label{fig:non_add}
    \vskip -0.15in
\end{figure}
\textbf{Non-additive Accuracy} 
The additive accuracy is a nice property held with the Gaussian-formed sender distribution of the original BFN expressed as:
\begin{align}
\label{eq:standard_id}
    \update(\parsn{}'' \mid \parsn{}, \x; \alpha_a+\alpha_b) = \E_{\update(\parsn{}' \mid \parsn{}, \x; \alpha_a)} \update(\parsn{}'' \mid \parsn{}', \x; \alpha_b)
\end{align}
Such property is mainly derived based on the standard identity of Gaussian variable:
\begin{equation}
X \sim \mathcal{N}\left(\mu_X, \sigma_X^2\right), Y \sim \mathcal{N}\left(\mu_Y, \sigma_Y^2\right) \Longrightarrow X+Y \sim \mathcal{N}\left(\mu_X+\mu_Y, \sigma_X^2+\sigma_Y^2\right)
\end{equation}
The additive accuracy property makes it feasible to derive the Bayesian flow distribution $
p_F(\boldsymbol{\theta} \mid \mathbf{x} ; i)=p_U\left(\boldsymbol{\theta} \mid \boldsymbol{\theta}_0, \mathbf{x}, \sum_{k=1}^{i} \alpha_i \right)
$ for the simulation-free training of \cref{eq:loss_n}.
It should be noted that the standard identity in \cref{eq:standard_id} does not hold in the von Mises distribution. Hence there exists an important difference between the original Bayesian flow defined on Euclidean space and the Bayesian flow of circular data on $\mathbb{T}^D$ based on von Mises distribution. With prior $\btheta = \{\bold{0},\bold{0}\}$, we could formally represent the non-additive accuracy issue as:
% The additive accuracy property implies the fact that the "confidence" for the data sample after observing a series of the noisy samples with accuracy ${\alpha_1, \cdots, \alpha_i}$ could be  as the accuracy sum  which could be  
% Here we 
% Here we emphasize the specific property of BFN based on von Mises distribution.
% Note that 
% \begin{equation}
% \update(\parsn'' \mid \parsn, \x; \alpha_a+\alpha_b) \ne \E_{\update(\parsn' \mid \parsn, \x; \alpha_a)} \update(\parsn'' \mid \parsn', \x; \alpha_b)
% \end{equation}
% \oyyw{please check whether the below equation is better}
% \yuxuan{I fill somehow confusing on what is the update distribution with $\alpha$. }
% \begin{equation}
% \update(\parsn{}'' \mid \parsn{}, \x; \alpha_a+\alpha_b) \ne \E_{\update(\parsn{}' \mid \parsn{}, \x; \alpha_a)} \update(\parsn{}'' \mid \parsn{}', \x; \alpha_b)
% \end{equation}
% We give an intuitive visualization of such difference in \cref{fig:non_add}. The untenability of this property can materialize by considering the following case: with prior $\btheta = \{\bold{0},\bold{0}\}$, check the two-step Bayesian update distribution with $\alpha_a,\alpha_b$ and one-step Bayesian update with $\alpha=\alpha_a+\alpha_b$:
\begin{align}
\label{eq:nonadd}
     &\update(c'' \mid \parsn, \x; \alpha_a+\alpha_b)  = \delta(c-\alpha_a-\alpha_b)
     \ne  \mathbb{E}_{p_U(\parsn' \mid \parsn, \x; \alpha_a)}\update(c'' \mid \parsn', \x; \alpha_b) \nonumber \\&= \mathbb{E}_{vM(\bold{y}_b|\bold{x},\alpha_a)}\mathbb{E}_{vM(\bold{y}_a|\bold{x},\alpha_b)}\delta(c-||[\alpha_a \cos\y_a+\alpha_b\cos \y_b,\alpha_a \sin\y_a+\alpha_b\sin \y_b]^T||_2)
\end{align}
A more intuitive visualization could be found in \cref{fig:non_add}. This fundamental difference between periodic Bayesian flow and that of \citet{bfn} presents both theoretical and practical challenges, which we will explain and address in the following contents.

% This makes constructing Bayesian flow based on von Mises distribution intrinsically different from previous Bayesian flows (\citet{bfn}).

% Thus, we must reformulate the framework of Bayesian flow networks  accordingly. % and do necessary reformulations of BFN. 

% \yuxuan{overall I feel this part is complicated by using the language of update distribution. I would like to suggest simply use bayesian update, to provide intuitive explantion.}\hanlin{See the illustration in \cref{fig:non_add}}

% That introduces a cascade of problems, and we investigate the following issues: $(1)$ Accuracies between sender and receiver are not synchronized and need to be differentiated. $(2)$ There is no tractable Bayesian flow distribution for a one-step sample conditioned on a given time step $i$, and naively simulating the Bayesian flow results in computational overhead. $(3)$ It is difficult to control the entropy of the Bayesian flow. $(4)$ Accuracy is no longer a function of $t$ and becomes a distribution conditioned on $t$, which can be different across dimensions.
%\jj{Edited till here}

\textbf{Entropy Conditioning} As a common practice in generative models~\citep{ddpm,flowmatching,bfn}, timestep $t$ is widely used to distinguish among generation states by feeding the timestep information into the networks. However, this paper shows that for periodic Bayesian flow, the accumulated accuracy $\vc_i$ is more effective than time-based conditioning by informing the network about the entropy and certainty of the states $\parsnt{i}$. This stems from the intrinsic non-additive accuracy which makes the receiver's accumulated accuracy $c$ not bijective function of $t$, but a distribution conditioned on accumulated accuracies $\vc_i$ instead. Therefore, the entropy parameter $\vc$ is taken logarithm and fed into the network to describe the entropy of the input corrupted structure. We verify this consideration in \cref{sec:exp_ablation}. 
% \yuxuan{implement variant. traditionally, the timestep is widely used to distinguish the different states by putting the timestep embedding into the networks. citation of FM, diffusion, BFN. However, we find that conditioned on time in periodic flow could not provide extra benefits. To further boost the performance, we introduce a simple yet effective modification term entropy conditional. This is based on that the accumulated accuracy which represents the current uncertainty or entropy could be a better indicator to distinguish different states. + Describe how you do this. }



\textbf{Reformulations of BFN}. Recall the original update function with Gaussian sender distribution, after receiving noisy samples $\y_1,\y_2,\dots,\y_i$ with accuracies $\senderacc$, the accumulated accuracies of the receiver side could be analytically obtained by the additive property and it is consistent with the sender side.
% Since observing sample $\y$ with $\alpha_i$ can not result in exact accuracy increment $\alpha_i$ for receiver, the accuracies between sender and receiver are not synchronized which need to be differentiated. 
However, as previously mentioned, this does not apply to periodic Bayesian flow, and some of the notations in original BFN~\citep{bfn} need to be adjusted accordingly. We maintain the notations of sender side's one-step accuracy $\alpha$ and added accuracy $\beta$, and alter the notation of receiver's accuracy parameter as $c$, which is needed to be simulated by cascade of Bayesian updates. We emphasize that the receiver's accumulated accuracy $c$ is no longer a function of $t$ (differently from the Gaussian case), and it becomes a distribution conditioned on received accuracies $\senderacc$ from the sender. Therefore, we represent the Bayesian flow distribution of von Mises distribution as $p_F(\btheta|\x;\alpha_1,\alpha_2,\dots,\alpha_i)$. And the original simulation-free training with Bayesian flow distribution is no longer applicable in this scenario.
% Different from previous BFNs where the accumulated accuracy $\rho$ is not explicitly modeled, the accumulated accuracy parameter $c$ (visualized in \cref{fig:vmbf_vis}) needs to be explicitly modeled by feeding it to the network to avoid information loss.
% the randomaccuracy parameter $c$ (visualized in \cref{fig:vmbf_vis}) implies that there exists information in $c$ from the sender just like $m$, meaning that $c$ also should be fed into the network to avoid information loss. 
% We ablate this consideration in  \cref{sec:exp_ablation}. 

\textbf{Fast Sampling from Equivalent Bayesian Flow Distribution} Based on the above reformulations, the Bayesian flow distribution of von Mises distribution is reframed as: 
\begin{equation}\label{eq:flow_frac}
p_F(\btheta_i|\x;\alpha_1,\alpha_2,\dots,\alpha_i)=\E_{\update(\parsnt{1} \mid \parsnt{0}, \x ; \alphat{1})}\dots\E_{\update(\parsn_{i-1} \mid \parsnt{i-2}, \x; \alphat{i-1})} \update(\parsnt{i} | \parsnt{i-1},\x;\alphat{i} )
\end{equation}
Naively sampling from \cref{eq:flow_frac} requires slow auto-regressive iterated simulation, making training unaffordable. Noticing the mathematical properties of \cref{eq:h_m,eq:h_c}, we  transform \cref{eq:flow_frac} to the equivalent form:
\begin{equation}\label{eq:cirflow_equiv}
p_F(\vec{m}_i|\x;\alpha_1,\alpha_2,\dots,\alpha_i)=\E_{vM(\y_1|\x,\alpha_1)\dots vM(\y_i|\x,\alpha_i)} \delta(\vec{m}_i-\text{atan2}(\sum_{j=1}^i \alpha_j \cos \y_j,\sum_{j=1}^i \alpha_j \sin \y_j))
\end{equation}
\begin{equation}\label{eq:cirflow_equiv2}
p_F(\vec{c}_i|\x;\alpha_1,\alpha_2,\dots,\alpha_i)=\E_{vM(\y_1|\x,\alpha_1)\dots vM(\y_i|\x,\alpha_i)}  \delta(\vec{c}_i-||[\sum_{j=1}^i \alpha_j \cos \y_j,\sum_{j=1}^i \alpha_j \sin \y_j]^T||_2)
\end{equation}
which bypasses the computation of intermediate variables and allows pure tensor operations, with negligible computational overhead.
\begin{restatable}{proposition}{cirflowequiv}
The probability density function of Bayesian flow distribution defined by \cref{eq:cirflow_equiv,eq:cirflow_equiv2} is equivalent to the original definition in \cref{eq:flow_frac}. 
\end{restatable}
\textbf{Numerical Determination of Linear Entropy Sender Accuracy Schedule} ~Original BFN designs the accuracy schedule $\beta(t)$ to make the entropy of input distribution linearly decrease. As for crystal generation task, to ensure information coherence between modalities, we choose a sender accuracy schedule $\senderacc$ that makes the receiver's belief entropy $H(t_i)=H(p_I(\cdot|\vtheta_i))=H(p_I(\cdot|\vc_i))$ linearly decrease \emph{w.r.t.} time $t_i$, given the initial and final accuracy parameter $c(0)$ and $c(1)$. Due to the intractability of \cref{eq:vm_entropy}, we first use numerical binary search in $[0,c(1)]$ to determine the receiver's $c(t_i)$ for $i=1,\dots, n$ by solving the equation $H(c(t_i))=(1-t_i)H(c(0))+tH(c(1))$. Next, with $c(t_i)$, we conduct numerical binary search for each $\alpha_i$ in $[0,c(1)]$ by solving the equations $\E_{y\sim vM(x,\alpha_i)}[\sqrt{\alpha_i^2+c_{i-1}^2+2\alpha_i c_{i-1}\cos(y-m_{i-1})}]=c(t_i)$ from $i=1$ to $i=n$ for arbitrarily selected $x\in[-\pi,\pi)$.

After tackling all those issues, we have now arrived at a new BFN architecture for effectively modeling crystals. Such BFN can also be adapted to other type of data located in hyper-torus $\mathbb{T}^{D}$.

\subsection{Equivariant Bayesian Flow for Crystal}
With the above Bayesian flow designed for generative modeling of fractional coordinate $\vF$, we are able to build equivariant Bayesian flow for each modality of crystal. In this section, we first give an overview of the general training and sampling algorithm of \modelname (visualized in \cref{fig:framework}). Then, we describe the details of the Bayesian flow of every modality. The training and sampling algorithm can be found in \cref{alg:train} and \cref{alg:sampling}.

\textbf{Overview} Operating in the parameter space $\bthetaM=\{\bthetaA,\bthetaL,\bthetaF\}$, \modelname generates high-fidelity crystals through a joint BFN sampling process on the parameter of  atom type $\bthetaA$, lattice parameter $\vec{\theta}^L=\{\bmuL,\brhoL\}$, and the parameter of fractional coordinate matrix $\bthetaF=\{\bmF,\bcF\}$. We index the $n$-steps of the generation process in a discrete manner $i$, and denote the corresponding continuous notation $t_i=i/n$ from prior parameter $\thetaM_0$ to a considerably low variance parameter $\thetaM_n$ (\emph{i.e.} large $\vrho^L,\bmF$, and centered $\bthetaA$).

At training time, \modelname samples time $i\sim U\{1,n\}$ and $\bthetaM_{i-1}$ from the Bayesian flow distribution of each modality, serving as the input to the network. The network $\net$ outputs $\net(\parsnt{i-1}^\mathcal{M},t_{i-1})=\net(\parsnt{i-1}^A,\parsnt{i-1}^F,\parsnt{i-1}^L,t_{i-1})$ and conducts gradient descents on loss function \cref{eq:loss_n} for each modality. After proper training, the sender distribution $p_S$ can be approximated by the receiver distribution $p_R$. 

At inference time, from predefined $\thetaM_0$, we conduct transitions from $\thetaM_{i-1}$ to $\thetaM_{i}$ by: $(1)$ sampling $\y_i\sim p_R(\bold{y}|\thetaM_{i-1};t_i,\alpha_i)$ according to network prediction $\predM{i-1}$; and $(2)$ performing Bayesian update $h(\thetaM_{i-1},\y^\calM_{i-1},\alpha_i)$ for each dimension. 

% Alternatively, we complete this transition using the flow-back technique by sampling 
% $\thetaM_{i}$ from Bayesian flow distribution $\flow(\btheta^M_{i}|\predM{i-1};t_{i-1})$. 

% The training objective of $\net$ is to minimize the KL divergence between sender distribution and receiver distribution for every modality as defined in \cref{eq:loss_n} which is equivalent to optimizing the negative variational lower bound $\calL^{VLB}$ as discussed in \cref{sec:preliminaries}. 

%In the following part, we will present the Bayesian flow of each modality in detail.

\textbf{Bayesian Flow of Fractional Coordinate $\vF$}~The distribution of the prior parameter $\bthetaF_0$ is defined as:
\begin{equation}\label{eq:prior_frac}
    p(\bthetaF_0) \defeq \{vM(\vm_0^F|\vec{0}_{3\times N},\vec{0}_{3\times N}),\delta(\vc_0^F-\vec{0}_{3\times N})\} = \{U(\vec{0},\vec{1}),\delta(\vc_0^F-\vec{0}_{3\times N})\}
\end{equation}
Note that this prior distribution of $\vm_0^F$ is uniform over $[\vec{0},\vec{1})$, ensuring the periodic translation invariance property in \cref{De:pi}. The training objective is minimizing the KL divergence between sender and receiver distribution (deduction can be found in \cref{appd:cir_loss}): 
%\oyyw{replace $\vF$ with $\x$?} \hanlin{notations follow Preliminary?}
\begin{align}\label{loss_frac}
\calL_F = n \E_{i \sim \ui{n}, \flow(\parsn{}^F \mid \vF ; \senderacc)} \alpha_i\frac{I_1(\alpha_i)}{I_0(\alpha_i)}(1-\cos(\vF-\predF{i-1}))
\end{align}
where $I_0(x)$ and $I_1(x)$ are the zeroth and the first order of modified Bessel functions. The transition from $\bthetaF_{i-1}$ to $\bthetaF_{i}$ is the Bayesian update distribution based on network prediction:
\begin{equation}\label{eq:transi_frac}
    p(\btheta^F_{i}|\parsnt{i-1}^\calM)=\mathbb{E}_{vM(\bold{y}|\predF{i-1},\alpha_i)}\delta(\btheta^F_{i}-h(\btheta^F_{i-1},\bold{y},\alpha_i))
\end{equation}
\begin{restatable}{proposition}{fracinv}
With $\net_{F}$ as a periodic translation equivariant function namely $\net_F(\parsnt{}^A,w(\parsnt{}^F+\vt),\parsnt{}^L,t)=w(\net_F(\parsnt{}^A,\parsnt{}^F,\parsnt{}^L,t)+\vt), \forall\vt\in\R^3$, the marginal distribution of $p(\vF_n)$ defined by \cref{eq:prior_frac,eq:transi_frac} is periodic translation invariant. 
\end{restatable}
\textbf{Bayesian Flow of Lattice Parameter \texorpdfstring{$\boldsymbol{L}$}{}}   
Noting the lattice parameter $\bm{L}$ located in Euclidean space, we set prior as the parameter of a isotropic multivariate normal distribution $\btheta^L_0\defeq\{\vmu_0^L,\vrho_0^L\}=\{\bm{0}_{3\times3},\bm{1}_{3\times3}\}$
% \begin{equation}\label{eq:lattice_prior}
% \btheta^L_0\defeq\{\vmu_0^L,\vrho_0^L\}=\{\bm{0}_{3\times3},\bm{1}_{3\times3}\}
% \end{equation}
such that the prior distribution of the Markov process on $\vmu^L$ is the Dirac distribution $\delta(\vec{\mu_0}-\vec{0})$ and $\delta(\vec{\rho_0}-\vec{1})$, 
% \begin{equation}
%     p_I^L(\boldsymbol{L}|\btheta_0^L)=\mathcal{N}(\bm{L}|\bm{0},\bm{I})
% \end{equation}
which ensures O(3)-invariance of prior distribution of $\vL$. By Eq. 77 from \citet{bfn}, the Bayesian flow distribution of the lattice parameter $\bm{L}$ is: 
\begin{align}% =p_U(\bmuL|\btheta_0^L,\bm{L},\beta(t))
p_F^L(\bmuL|\bm{L};t) &=\mathcal{N}(\bmuL|\gamma(t)\bm{L},\gamma(t)(1-\gamma(t))\bm{I}) 
\end{align}
where $\gamma(t) = 1 - \sigma_1^{2t}$ and $\sigma_1$ is the predefined hyper-parameter controlling the variance of input distribution at $t=1$ under linear entropy accuracy schedule. The variance parameter $\vrho$ does not need to be modeled and fed to the network, since it is deterministic given the accuracy schedule. After sampling $\bmuL_i$ from $p_F^L$, the training objective is defined as minimizing KL divergence between sender and receiver distribution (based on Eq. 96 in \citet{bfn}):
\begin{align}
\mathcal{L}_{L} = \frac{n}{2}\left(1-\sigma_1^{2/n}\right)\E_{i \sim \ui{n}}\E_{\flow(\bmuL_{i-1} |\vL ; t_{i-1})}  \frac{\left\|\vL -\predL{i-1}\right\|^2}{\sigma_1^{2i/n}},\label{eq:lattice_loss}
\end{align}
where the prediction term $\predL{i-1}$ is the lattice parameter part of network output. After training, the generation process is defined as the Bayesian update distribution given network prediction:
\begin{equation}\label{eq:lattice_sampling}
    p(\bmuL_{i}|\parsnt{i-1}^\calM)=\update^L(\bmuL_{i}|\predL{i-1},\bmuL_{i-1};t_{i-1})
\end{equation}
    

% The final prediction of the lattice parameter is given by $\bmuL_n = \predL{n-1}$.
% \begin{equation}\label{eq:final_lattice}
%     \bmuL_n = \predL{n-1}
% \end{equation}

\begin{restatable}{proposition}{latticeinv}\label{prop:latticeinv}
With $\net_{L}$ as  O(3)-equivariant function namely $\net_L(\parsnt{}^A,\parsnt{}^F,\vQ\parsnt{}^L,t)=\vQ\net_L(\parsnt{}^A,\parsnt{}^F,\parsnt{}^L,t),\forall\vQ^T\vQ=\vI$, the marginal distribution of $p(\bmuL_n)$ defined by \cref{eq:lattice_sampling} is O(3)-invariant. 
\end{restatable}


\textbf{Bayesian Flow of Atom Types \texorpdfstring{$\boldsymbol{A}$}{}} 
Given that atom types are discrete random variables located in a simplex $\calS^K$, the prior parameter of $\boldsymbol{A}$ is the discrete uniform distribution over the vocabulary $\parsnt{0}^A \defeq \frac{1}{K}\vec{1}_{1\times N}$. 
% \begin{align}\label{eq:disc_input_prior}
% \parsnt{0}^A \defeq \frac{1}{K}\vec{1}_{1\times N}
% \end{align}
% \begin{align}
%     (\oh{j}{K})_k \defeq \delta_{j k}, \text{where }\oh{j}{K}\in \R^{K},\oh{\vA}{KD} \defeq \left(\oh{a_1}{K},\dots,\oh{a_N}{K}\right) \in \R^{K\times N}
% \end{align}
With the notation of the projection from the class index $j$ to the length $K$ one-hot vector $ (\oh{j}{K})_k \defeq \delta_{j k}, \text{where }\oh{j}{K}\in \R^{K},\oh{\vA}{KD} \defeq \left(\oh{a_1}{K},\dots,\oh{a_N}{K}\right) \in \R^{K\times N}$, the Bayesian flow distribution of atom types $\vA$ is derived in \citet{bfn}:
\begin{align}
\flow^{A}(\parsn^A \mid \vA; t) &= \E_{\N{\y \mid \beta^A(t)\left(K \oh{\vA}{K\times N} - \vec{1}_{K\times N}\right)}{\beta^A(t) K \vec{I}_{K\times N \times N}}} \delta\left(\parsn^A - \frac{e^{\y}\parsnt{0}^A}{\sum_{k=1}^K e^{\y_k}(\parsnt{0})_{k}^A}\right).
\end{align}
where $\beta^A(t)$ is the predefined accuracy schedule for atom types. Sampling $\btheta_i^A$ from $p_F^A$ as the training signal, the training objective is the $n$-step discrete-time loss for discrete variable \citep{bfn}: 
% \oyyw{can we simplify the next equation? Such as remove $K \times N, K \times N \times N$}
% \begin{align}
% &\calL_A = n\E_{i \sim U\{1,n\},\flow^A(\parsn^A \mid \vA ; t_{i-1}),\N{\y \mid \alphat{i}\left(K \oh{\vA}{KD} - \vec{1}_{K\times N}\right)}{\alphat{i} K \vec{I}_{K\times N \times N}}} \ln \N{\y \mid \alphat{i}\left(K \oh{\vA}{K\times N} - \vec{1}_{K\times N}\right)}{\alphat{i} K \vec{I}_{K\times N \times N}}\nonumber\\
% &\qquad\qquad\qquad-\sum_{d=1}^N \ln \left(\sum_{k=1}^K \out^{(d)}(k \mid \parsn^A; t_{i-1}) \N{\ydd{d} \mid \alphat{i}\left(K\oh{k}{K}- \vec{1}_{K\times N}\right)}{\alphat{i} K \vec{I}_{K\times N \times N}}\right)\label{discdisc_t_loss_exp}
% \end{align}
\begin{align}
&\calL_A = n\E_{i \sim U\{1,n\},\flow^A(\parsn^A \mid \vA ; t_{i-1}),\N{\y \mid \alphat{i}\left(K \oh{\vA}{KD} - \vec{1}\right)}{\alphat{i} K \vec{I}}} \ln \N{\y \mid \alphat{i}\left(K \oh{\vA}{K\times N} - \vec{1}\right)}{\alphat{i} K \vec{I}}\nonumber\\
&\qquad\qquad\qquad-\sum_{d=1}^N \ln \left(\sum_{k=1}^K \out^{(d)}(k \mid \parsn^A; t_{i-1}) \N{\ydd{d} \mid \alphat{i}\left(K\oh{k}{K}- \vec{1}\right)}{\alphat{i} K \vec{I}}\right)\label{discdisc_t_loss_exp}
\end{align}
where $\vec{I}\in \R^{K\times N \times N}$ and $\vec{1}\in\R^{K\times D}$. When sampling, the transition from $\bthetaA_{i-1}$ to $\bthetaA_{i}$ is derived as:
\begin{equation}
    p(\btheta^A_{i}|\parsnt{i-1}^\calM)=\update^A(\btheta^A_{i}|\btheta^A_{i-1},\predA{i-1};t_{i-1})
\end{equation}

The detailed training and sampling algorithm could be found in \cref{alg:train} and \cref{alg:sampling}.






\begin{table*}[t]
\centering
\fontsize{11pt}{11pt}\selectfont
\begin{tabular}{lllllllllllll}
\toprule
\multicolumn{1}{c}{\textbf{task}} & \multicolumn{2}{c}{\textbf{Mir}} & \multicolumn{2}{c}{\textbf{Lai}} & \multicolumn{2}{c}{\textbf{Ziegen.}} & \multicolumn{2}{c}{\textbf{Cao}} & \multicolumn{2}{c}{\textbf{Alva-Man.}} & \multicolumn{1}{c}{\textbf{avg.}} & \textbf{\begin{tabular}[c]{@{}l@{}}avg.\\ rank\end{tabular}} \\
\multicolumn{1}{c}{\textbf{metrics}} & \multicolumn{1}{c}{\textbf{cor.}} & \multicolumn{1}{c}{\textbf{p-v.}} & \multicolumn{1}{c}{\textbf{cor.}} & \multicolumn{1}{c}{\textbf{p-v.}} & \multicolumn{1}{c}{\textbf{cor.}} & \multicolumn{1}{c}{\textbf{p-v.}} & \multicolumn{1}{c}{\textbf{cor.}} & \multicolumn{1}{c}{\textbf{p-v.}} & \multicolumn{1}{c}{\textbf{cor.}} & \multicolumn{1}{c}{\textbf{p-v.}} &  &  \\ \midrule
\textbf{S-Bleu} & 0.50 & 0.0 & 0.47 & 0.0 & 0.59 & 0.0 & 0.58 & 0.0 & 0.68 & 0.0 & 0.57 & 5.8 \\
\textbf{R-Bleu} & -- & -- & 0.27 & 0.0 & 0.30 & 0.0 & -- & -- & -- & -- & - &  \\
\textbf{S-Meteor} & 0.49 & 0.0 & 0.48 & 0.0 & 0.61 & 0.0 & 0.57 & 0.0 & 0.64 & 0.0 & 0.56 & 6.1 \\
\textbf{R-Meteor} & -- & -- & 0.34 & 0.0 & 0.26 & 0.0 & -- & -- & -- & -- & - &  \\
\textbf{S-Bertscore} & \textbf{0.53} & 0.0 & {\ul 0.80} & 0.0 & \textbf{0.70} & 0.0 & {\ul 0.66} & 0.0 & {\ul0.78} & 0.0 & \textbf{0.69} & \textbf{1.7} \\
\textbf{R-Bertscore} & -- & -- & 0.51 & 0.0 & 0.38 & 0.0 & -- & -- & -- & -- & - &  \\
\textbf{S-Bleurt} & {\ul 0.52} & 0.0 & {\ul 0.80} & 0.0 & 0.60 & 0.0 & \textbf{0.70} & 0.0 & \textbf{0.80} & 0.0 & {\ul 0.68} & {\ul 2.3} \\
\textbf{R-Bleurt} & -- & -- & 0.59 & 0.0 & -0.05 & 0.13 & -- & -- & -- & -- & - &  \\
\textbf{S-Cosine} & 0.51 & 0.0 & 0.69 & 0.0 & {\ul 0.62} & 0.0 & 0.61 & 0.0 & 0.65 & 0.0 & 0.62 & 4.4 \\
\textbf{R-Cosine} & -- & -- & 0.40 & 0.0 & 0.29 & 0.0 & -- & -- & -- & -- & - & \\ \midrule
\textbf{QuestEval} & 0.23 & 0.0 & 0.25 & 0.0 & 0.49 & 0.0 & 0.47 & 0.0 & 0.62 & 0.0 & 0.41 & 9.0 \\
\textbf{LLaMa3} & 0.36 & 0.0 & \textbf{0.84} & 0.0 & {\ul{0.62}} & 0.0 & 0.61 & 0.0 &  0.76 & 0.0 & 0.64 & 3.6 \\
\textbf{our (3b)} & 0.49 & 0.0 & 0.73 & 0.0 & 0.54 & 0.0 & 0.53 & 0.0 & 0.7 & 0.0 & 0.60 & 5.8 \\
\textbf{our (8b)} & 0.48 & 0.0 & 0.73 & 0.0 & 0.52 & 0.0 & 0.53 & 0.0 & 0.7 & 0.0 & 0.59 & 6.3 \\  \bottomrule
\end{tabular}
\caption{Pearson correlation on human evaluation on system output. `R-': reference-based. `S-': source-based.}
\label{tab:sys}
\end{table*}



\begin{table}%[]
\centering
\fontsize{11pt}{11pt}\selectfont
\begin{tabular}{llllll}
\toprule
\multicolumn{1}{c}{\textbf{task}} & \multicolumn{1}{c}{\textbf{Lai}} & \multicolumn{1}{c}{\textbf{Zei.}} & \multicolumn{1}{c}{\textbf{Scia.}} & \textbf{} & \textbf{} \\ 
\multicolumn{1}{c}{\textbf{metrics}} & \multicolumn{1}{c}{\textbf{cor.}} & \multicolumn{1}{c}{\textbf{cor.}} & \multicolumn{1}{c}{\textbf{cor.}} & \textbf{avg.} & \textbf{\begin{tabular}[c]{@{}l@{}}avg.\\ rank\end{tabular}} \\ \midrule
\textbf{S-Bleu} & 0.40 & 0.40 & 0.19* & 0.33 & 7.67 \\
\textbf{S-Meteor} & 0.41 & 0.42 & 0.16* & 0.33 & 7.33 \\
\textbf{S-BertS.} & {\ul0.58} & 0.47 & 0.31 & 0.45 & 3.67 \\
\textbf{S-Bleurt} & 0.45 & {\ul 0.54} & {\ul 0.37} & 0.45 & {\ul 3.33} \\
\textbf{S-Cosine} & 0.56 & 0.52 & 0.3 & {\ul 0.46} & {\ul 3.33} \\ \midrule
\textbf{QuestE.} & 0.27 & 0.35 & 0.06* & 0.23 & 9.00 \\
\textbf{LlaMA3} & \textbf{0.6} & \textbf{0.67} & \textbf{0.51} & \textbf{0.59} & \textbf{1.0} \\
\textbf{Our (3b)} & 0.51 & 0.49 & 0.23* & 0.39 & 4.83 \\
\textbf{Our (8b)} & 0.52 & 0.49 & 0.22* & 0.43 & 4.83 \\ \bottomrule
\end{tabular}
\caption{Pearson correlation on human ratings on reference output. *not significant; we cannot reject the null hypothesis of zero correlation}
\label{tab:ref}
\end{table}


\begin{table*}%[]
\centering
\fontsize{11pt}{11pt}\selectfont
\begin{tabular}{lllllllll}
\toprule
\textbf{task} & \multicolumn{1}{c}{\textbf{ALL}} & \multicolumn{1}{c}{\textbf{sentiment}} & \multicolumn{1}{c}{\textbf{detoxify}} & \multicolumn{1}{c}{\textbf{catchy}} & \multicolumn{1}{c}{\textbf{polite}} & \multicolumn{1}{c}{\textbf{persuasive}} & \multicolumn{1}{c}{\textbf{formal}} & \textbf{\begin{tabular}[c]{@{}l@{}}avg. \\ rank\end{tabular}} \\
\textbf{metrics} & \multicolumn{1}{c}{\textbf{cor.}} & \multicolumn{1}{c}{\textbf{cor.}} & \multicolumn{1}{c}{\textbf{cor.}} & \multicolumn{1}{c}{\textbf{cor.}} & \multicolumn{1}{c}{\textbf{cor.}} & \multicolumn{1}{c}{\textbf{cor.}} & \multicolumn{1}{c}{\textbf{cor.}} &  \\ \midrule
\textbf{S-Bleu} & -0.17 & -0.82 & -0.45 & -0.12* & -0.1* & -0.05 & -0.21 & 8.42 \\
\textbf{R-Bleu} & - & -0.5 & -0.45 &  &  &  &  &  \\
\textbf{S-Meteor} & -0.07* & -0.55 & -0.4 & -0.01* & 0.1* & -0.16 & -0.04* & 7.67 \\
\textbf{R-Meteor} & - & -0.17* & -0.39 & - & - & - & - & - \\
\textbf{S-BertScore} & 0.11 & -0.38 & -0.07* & -0.17* & 0.28 & 0.12 & 0.25 & 6.0 \\
\textbf{R-BertScore} & - & -0.02* & -0.21* & - & - & - & - & - \\
\textbf{S-Bleurt} & 0.29 & 0.05* & 0.45 & 0.06* & 0.29 & 0.23 & 0.46 & 4.2 \\
\textbf{R-Bleurt} & - &  0.21 & 0.38 & - & - & - & - & - \\
\textbf{S-Cosine} & 0.01* & -0.5 & -0.13* & -0.19* & 0.05* & -0.05* & 0.15* & 7.42 \\
\textbf{R-Cosine} & - & -0.11* & -0.16* & - & - & - & - & - \\ \midrule
\textbf{QuestEval} & 0.21 & {\ul{0.29}} & 0.23 & 0.37 & 0.19* & 0.35 & 0.14* & 4.67 \\
\textbf{LlaMA3} & \textbf{0.82} & \textbf{0.80} & \textbf{0.72} & \textbf{0.84} & \textbf{0.84} & \textbf{0.90} & \textbf{0.88} & \textbf{1.00} \\
\textbf{Our (3b)} & 0.47 & -0.11* & 0.37 & 0.61 & 0.53 & 0.54 & 0.66 & 3.5 \\
\textbf{Our (8b)} & {\ul{0.57}} & 0.09* & {\ul 0.49} & {\ul 0.72} & {\ul 0.64} & {\ul 0.62} & {\ul 0.67} & {\ul 2.17} \\ \bottomrule
\end{tabular}
\caption{Pearson correlation on human ratings on our constructed test set. 'R-': reference-based. 'S-': source-based. *not significant; we cannot reject the null hypothesis of zero correlation}
\label{tab:con}
\end{table*}

\section{Results}
We benchmark the different metrics on the different datasets using correlation to human judgement. For content preservation, we show results split on data with system output, reference output and our constructed test set: we show that the data source for evaluation leads to different conclusions on the metrics. In addition, we examine whether the metrics can rank style transfer systems similar to humans. On style strength, we likewise show correlations between human judgment and zero-shot evaluation approaches. When applicable, we summarize results by reporting the average correlation. And the average ranking of the metric per dataset (by ranking which metric obtains the highest correlation to human judgement per dataset). 

\subsection{Content preservation}
\paragraph{How do data sources affect the conclusion on best metric?}
The conclusions about the metrics' performance change radically depending on whether we use system output data, reference output, or our constructed test set. Ideally, a good metric correlates highly with humans on any data source. Ideally, for meta-evaluation, a metric should correlate consistently across all data sources, but the following shows that the correlations indicate different things, and the conclusion on the best metric should be drawn carefully.

Looking at the metrics correlations with humans on the data source with system output (Table~\ref{tab:sys}), we see a relatively high correlation for many of the metrics on many tasks. The overall best metrics are S-BertScore and S-BLEURT (avg+avg rank). We see no notable difference in our method of using the 3B or 8B model as the backbone.

Examining the average correlations based on data with reference output (Table~\ref{tab:ref}), now the zero-shoot prompting with LlaMA3 70B is the best-performing approach ($0.59$ avg). Tied for second place are source-based cosine embedding ($0.46$ avg), BLEURT ($0.45$ avg) and BertScore ($0.45$ avg). Our method follows on a 5. place: here, the 8b version (($0.43$ avg)) shows a bit stronger results than 3b ($0.39$ avg). The fact that the conclusions change, whether looking at reference or system output, confirms the observations made by \citet{scialom-etal-2021-questeval} on simplicity transfer.   

Now consider the results on our test set (Table~\ref{tab:con}): Several metrics show low or no correlation; we even see a significantly negative correlation for some metrics on ALL (BLEU) and for specific subparts of our test set for BLEU, Meteor, BertScore, Cosine. On the other end, LlaMA3 70B is again performing best, showing strong results ($0.82$ in ALL). The runner-up is now our 8B method, with a gap to the 3B version ($0.57$ vs $0.47$ in ALL). Note our method still shows zero correlation for the sentiment task. After, ranks BLEURT ($0.29$), QuestEval ($0.21$), BertScore ($0.11$), Cosine ($0.01$).  

On our test set, we find that some metrics that correlate relatively well on the other datasets, now exhibit low correlation. Hence, with our test set, we can now support the logical reasoning with data evidence: Evaluation of content preservation for style transfer needs to take the style shift into account. This conclusion could not be drawn using the existing data sources: We hypothesise that for the data with system-based output, successful output happens to be very similar to the source sentence and vice versa, and reference-based output might not contain server mistakes as they are gold references. Thus, none of the existing data sources tests the limits of the metrics.  


\paragraph{How do reference-based metrics compare to source-based ones?} Reference-based metrics show a lower correlation than the source-based counterpart for all metrics on both datasets with ratings on references (Table~\ref{tab:sys}). As discussed previously, reference-based metrics for style transfer have the drawback that many different good solutions on a rewrite might exist and not only one similar to a reference.


\paragraph{How well can the metrics rank the performance of style transfer methods?}
We compare the metrics' ability to judge the best style transfer methods w.r.t. the human annotations: Several of the data sources contain samples from different style transfer systems. In order to use metrics to assess the quality of the style transfer system, metrics should correctly find the best-performing system. Hence, we evaluate whether the metrics for content preservation provide the same system ranking as human evaluators. We take the mean of the score for every output on each system and the mean of the human annotations; we compare the systems using the Kendall's Tau correlation. 

We find only the evaluation using the dataset Mir, Lai, and Ziegen to result in significant correlations, probably because of sparsity in a number of system tests (App.~\ref{app:dataset}). Our method (8b) is the only metric providing a perfect ranking of the style transfer system on the Lai data, and Llama3 70B the only one on the Ziegen data. Results in App.~\ref{app:results}. 


\subsection{Style strength results}
%Evaluating style strengths is a challenging task. 
Llama3 70B shows better overall results than our method. However, our method scores higher than Llama3 70B on 2 out of 6 datasets, but it also exhibits zero correlation on one task (Table~\ref{tab:styleresults}).%More work i s needed on evaluating style strengths. 
 
\begin{table}%[]
\fontsize{11pt}{11pt}\selectfont
\begin{tabular}{lccc}
\toprule
\multicolumn{1}{c}{\textbf{}} & \textbf{LlaMA3} & \textbf{Our (3b)} & \textbf{Our (8b)} \\ \midrule
\textbf{Mir} & 0.46 & 0.54 & \textbf{0.57} \\
\textbf{Lai} & \textbf{0.57} & 0.18 & 0.19 \\
\textbf{Ziegen.} & 0.25 & 0.27 & \textbf{0.32} \\
\textbf{Alva-M.} & \textbf{0.59} & 0.03* & 0.02* \\
\textbf{Scialom} & \textbf{0.62} & 0.45 & 0.44 \\
\textbf{\begin{tabular}[c]{@{}l@{}}Our Test\end{tabular}} & \textbf{0.63} & 0.46 & 0.48 \\ \bottomrule
\end{tabular}
\caption{Style strength: Pearson correlation to human ratings. *not significant; we cannot reject the null hypothesis of zero corelation}
\label{tab:styleresults}
\end{table}

\subsection{Ablation}
We conduct several runs of the methods using LLMs with variations in instructions/prompts (App.~\ref{app:method}). We observe that the lower the correlation on a task, the higher the variation between the different runs. For our method, we only observe low variance between the runs.
None of the variations leads to different conclusions of the meta-evaluation. Results in App.~\ref{app:results}.

\section{Conclusion}
In this work, we propose a simple yet effective approach, called SMILE, for graph few-shot learning with fewer tasks. Specifically, we introduce a novel dual-level mixup strategy, including within-task and across-task mixup, for enriching the diversity of nodes within each task and the diversity of tasks. Also, we incorporate the degree-based prior information to learn expressive node embeddings. Theoretically, we prove that SMILE effectively enhances the model's generalization performance. Empirically, we conduct extensive experiments on multiple benchmarks and the results suggest that SMILE significantly outperforms other baselines, including both in-domain and cross-domain few-shot settings.

\newpage
\bibliography{iclr2025_conference}
\bibliographystyle{iclr2025_conference}

\newpage
\appendix



\section{Taxonomy of Objective Functions}

In this section, we briefly describe different objectives that we reviewed and used in the main text.

\subsection{Path measure alignment objectives}


The \textit{path measure alignment} framework aims to align the 
sampling process starting from $p_\text{prior}$ to a      ``target"  process starting from $p_\text{target}$. 
In the following, we denote  $\mathbb{Q}$ as the sampling process  and $\mathbb{P}$ as the ``target" process.
However, we should note that this notation does not necessarily imply that $\mathbb{Q}$ is the process parameterized by the model. 
In fact, this is only true for samplers like PIS or DDS. 
For escorted transport samplers like CMCD, both  $\mathbb{Q}$ and  $\mathbb{P}$ involve the model, and for annealed variance reduction sampler like MCD,  $\mathbb{Q}$ is fixed, and the model only appears in $\mathbb{P}$.
We now describe five commonly used objectives:

\textbf{Reverse KL divergence.} 
Reverse KL divergence is defined as
\begin{align}
    D_\mathrm{KL}[\mathbb{Q}||\mathbb{P}] = \E_{\mathbb{Q}} \left[
  \log   \frac{\mathrm{d} \mathbb{Q}}{\mathrm{d} \mathbb{P}}
    \right].
\end{align}
In practice, we approximate the expectation with Monte Carlo estimators, and calculate the log Radon–Nikodym derivative $\log   \frac{\mathrm{d} \mathbb{Q}}{\mathrm{d} \mathbb{P}}$, either through Gaussian approximation via Euler–Maruyama discretization or by applying Girsanov's theorem.

\textbf{Log-variance divergence.} Log-variance divergence optimizes the second moment of the log ratio
\begin{align}
    D_\mathrm{logvar}[\mathbb{Q}||\mathbb{P}] = \text{Var}_{\tilde{\mathbb{Q}}} \left(
  \log   \frac{\mathrm{d} \mathbb{Q}}{\mathrm{d} \mathbb{P}}
    \right).
\end{align}
Unlike KL divergence, which requires the expectation to be taken with respect to \(\mathbb{Q}\), log-variance allows the variance to be computed under a different measure \(\tilde{\mathbb{Q}}\). 
This flexibility suggests that we can detach the gradient of the trajectory or utilize a buffer to stabilize training.
On the other hand, when the variance is taken under  \(\mathbb{Q}\), the gradient of log-variance divergence w.r.t parameters in  \(\mathbb{Q}\) is the same as that of reverse KL divergence~\citep{richter2020vargrad}:
\begin{align}
    \frac{\mathrm{d}}{\mathrm{d} \mathbb{\theta}} \text{Var}_{\tilde{\mathbb{Q}}} \left(
  \log   \frac{\mathrm{d} \mathbb{Q}_{\theta}}{\mathrm{d} \mathbb{P}}
    \right)\Bigg|_{\tilde{\mathbb{Q}} = {\mathbb{Q}_{\theta}}} =    \frac{\mathrm{d}}{\mathrm{d} {\theta}}  D_\mathrm{KL}[\mathbb{Q}_{\theta}||\mathbb{P}].
\end{align}
However, we note that this conclusion holds only \emph{in expectation}.
In practice, when the objective is calculated with Monte Carlo estimators, they will exhibit different behavior.

\textbf{Trajectory balance.} Trajectory balance optimizes the squared log ratio
\begin{align}
    D_\mathrm{TB}[\mathbb{Q}||\mathbb{P}] = \E_{\tilde{\mathbb{Q}}} \left[
  \left(\log   \frac{\mathrm{d} \mathbb{Q}}{\mathrm{d} \mathbb{P}} - k\right)^2
    \right],
\end{align}
which is equivalent to the log-variance divergence with a learned baseline $k$.

\textbf{Sub-trajectory balance.} 
TB loss matches the entire $\mathbb{Q}$ and $\mathbb{P}$ as a whole. 
Alternatively, we can match segments of each trajectory individually to ensure consistency across the entire trajectory. This approach leads to the sub-trajectory balance objective. 
For simplicity, though it is possible to define sub-trajectory balance in continuous time, we define it with time discretization.
\par
With Euler–Maruyama discretization, we discretize $\mathbb{Q}$ and $\mathbb{P}$ into sequential produce of measure, with density given by:
\begin{align}
 p_0(X_0) \prod_{n=0}^{N-1} p_F(X_{n+1}|X_n) \quad\text{and}\quad   \ptilde(X_N) \prod_{t=0}^{N-1} p_B(X_n|X_{n+1}).
\end{align}
Note that the density for discretized $\mathbb{P}$ can be unnormalized.


Then, we introduce a sequence of intermediate densities \(\{\pi_n\}_{n=0}^{N}\), where the boundary conditions are given by \(\pi_0 = p_{\text{prior}}\) and \(\pi_N = \tilde{p}_{\text{target}}\). 
These intermediate distributions can either be prescribed as a fixed interpolation between the target and prior distributions or be learned adaptively through a parameterized neural network.

Finally, we define the sub-trajectory balance objective as
\begin{align}
      D_\mathrm{STB}[\mathbb{Q}||\mathbb{P}] = \E_{\tilde{\mathbb{Q}}} \left[ \sum_{0 \leq i<j\leq N}
  \left(\log   \frac{\pi_i(x_i) \prod_{n=i}^{j-1} p_F(x_{n+1}|x_n) }{\pi_j(x_j) \prod_{n'=i}^{j-1} p_B(x_{n'}|x_{n'+1})} + k_i - k_j\right)^2
    \right].
\end{align}

\textbf{Detailed balance.} 
Detailed balance can be viewed as an extreme case of sub-trajectory balance, where instead of summing over sub-trajectories of all lengths, we only calculate the sub-trajectory balance over each discretization step:

\begin{align}
      D_\mathrm{DB}[\mathbb{Q}||\mathbb{P}] = \E_{\tilde{\mathbb{Q}}} \left[ \sum_{0 \leq i\leq N-1}
  \left(\log   \frac{\pi_i(x_i) p_F(x_{i+1}|x_i) }{\pi_j(x_{i+1})  p_B(x_{i}|x_{i+1})} + k_i - k_{i+1}\right)^2
    \right].
\end{align}












\subsection{Marginal alignment objectives}
Unlike path measure alignment, \emph{marginal alignment} objectives directly enforce the sampling process at each time step $t$ to match with some marginal $\pi_t$.
$\pi_t$ can be either prescribed as an interpolation between the target and prior, with boundary conditions $\pi_0 = p_\text{prior}$ and $\pi_T = p_\text{target}$, or be learned through a network under the constraint of the boundary conditions. 
Commonly used objectives in this framework include PINN and action matching:

\textbf{PINN.} For the sampling process defined by $ dX_t =   \left(f_\theta(X_t, t) +\sigma_t^2  \nabla \log \pi_t (X_t)\right) dt + \sigma_t\sqrt{2} dW_t$, the PINN loss is given by
\begin{align}
    \mathcal{L}_{\text{PINN}} = \int_0^T \mathbb{E}_{\tilde{q}_t(X_t)}  ||
  \nabla \cdot f_\theta(X_t, t) + \nabla \log\pi_t(X_t) \cdot f_\theta(X_t, t) + (\partial_t \log\pi_t)(X_t) + \partial_t F(t) 
    ||^2 dt,
\end{align}
where $F(t) $ is parameterized by a neural network. 
Note that the expectation can be taken over an arbitrary $\tilde{q}_t$, as long as the marginal of $\mathbb{Q}$ at time $t$ is absolute continuous to $\tilde{q}_t$.
We also note PINN does not depend on the specific value of $\sigma_t$ in the sampling process.

\textbf{Action matching.} Similar to PINN, an action matching-based~\citep{neklyudov2023action} objective is derived by~\citep{albergo2024nets} for the PDE-constrained optimization problem
\begin{multline}
\mathcal{L}_{\text{AM}} =  \int_0^T  \mathbb{E}_{q_t(X_t)} \left[ \frac{1}{2} ||\nabla \phi_t(X_t)||^2 + \partial_t \phi_t(X_t)\right] dt \\+ \mathbb{E}_{p_{\text{prior}}(X_0)} \left[ \phi_0(X_0)\right] - \mathbb{E}_{p_{\text{target}}(X_T)} \left[ \phi_T(X_T) \right] ,
\end{multline}
where the vector field $b_t = \nabla \phi_t$, induced by a scalar potential, and $\phi_t$ is called the ``action''.


\section{Detailed Summary of Samplers}\label{appendix:review}
In this section, we provide a more detailed review of diffusion and control-based neural samplers.
We also discuss how these neural samplers rely on the Langevin preconditioning in the end.
\subsection{Sampling process and Objectives}
We write the sampling process as follows:
    \begin{align}\label{eq:x_appendix}
        d X_t = \left [\mu_t(X_t) + \sigma_t^2 b_t(X_t) \right] dt + \sigma_t \sqrt{2} dW_t, \quad X_0 \sim p_\text{prior},
    \end{align}

 \begin{enumerate}[label=({{\arabic*}}), leftmargin=*]
        \item Path Integral Sampler \citep[PIS,][]{zhangpath} and concurrently \citep[NSFS, ][]{vargas2021bayesian}: 
        PIS fixes $p_\text{prior} = \delta_0,  \sigma_t=1/\sqrt{2}$ and learns a network $f_\theta(\cdot) =\mu_t(\cdot) + \sigma_t^2 b_t(\cdot)$ so that \Cref{eq:x_appendix} approximate the time-reversal of the following SDE (Pinned Brownian Motion):
        \begin{align}\label{eq:pis_y_appendix}
             d Y_t = -\frac{Y_t}{T-t} dt + dW_t, \quad Y_0 \sim \ptarget.
        \end{align}
        We define \Cref{eq:pis_y_appendix} as the time-reversal of \Cref{eq:x_appendix} when $Y_t \sim X_{T-t}$.
        The network is learned by matching the reverse KL \citep{zhangpath,vargas2021bayesian} or log-variance divergence \citep{richterimproved} between the path measures of the sampling and the target process.
       
       \item Diffusion generative flow samplers~\citep[DFGS,][]{zhangdiffusion} learns to sample from  the same process as PIS,  but with a new introduction of local objectives including detailed balance and (sub-)trajectory balance.
        In fact, trajectory balance can been shown to be equivalent to the log-variance objective with a learned baseline rather than a Monte Carlo (MC) estimator for the first moment~\citep{nusken2021solving}.  
        \item Denoising Diffusion Sampler \citep[DDS,][]{vargasdenoising} and time-reversed Diffusion Sampler \citep[DIS,][]{berneroptimal}: 
        both DDS and DIS fix $\mu_t(X_t, t) = \beta_{T-t} X_t, \sigma_t = v \sqrt{\beta_{T-t}}, p_\text{prior} = \mathcal{N}(0, v^2I)$, and learn a network $f_\theta(\cdot, t) = b_t(\cdot, t)/2 $ so that \Cref{eq:x_appendix} approximates the time-reversal of the VP-SDE:
        \begin{align}\label{eq:dds_y_appendix}
            dY_t = -\beta_t Y_t dt + v \sqrt{2\beta_{t}} dW_t, \quad  Y_0 \sim \ptarget.
        \end{align}
        Similar to PIS, the network can be trained either with reverse KL divergence or log-variance divergence.
        In an optimal solution, $f_\theta$ will approximate the score $f_\theta(\cdot, t) \approx \nabla\log p_{T-t}(\cdot)$, where $ p_{t}(X) = \int  \mathcal{N}(X|\sqrt{1-\lambda_t}Y, v^2\lambda_t I) \ptarget(Y)dY$ and $\lambda_t = 1-\exp(-2\int_0^t  \beta_s ds)$.
        
       \item Iterated Denoising Energy Matching \citep[iDEM,][]{akhounditerated}: iDEM fixes $\mu_t(X_t, t) = 0, p_\text{prior} = \mathcal{N}(0, T^2 I)$, and learns a network $f_\theta(\cdot, t) = b_t(\cdot, t) / 2$ to approximate the score  $f_\theta(X_t, t) \approx \nabla \log p_{T-t} (X_t)$.
        This is achieved by writing the score with target score identity \citep[TSI, ][]{de2024target}, and estimating it with a self-normalized importance sampler:
        \begin{align}
            \nabla \log p_{T-t} (X_t) &\stackrel{\text{TSI}}{=} \int p_{T|T}(X_T | X_t)  \nabla \log \ptilde (X_T) dX_T\\
            &\stackrel{\text{Bayes' Rule}}{=}\int \frac{\ptilde (X_T)p_{t|T}(X_t|X_T)}{\int \ptilde (X_T)p_{t|T}(X_t|X_T) dX_T}  \nabla \log \ptilde (X_T) dX_T\\
            &=\int q_{T|t}(X_T | X_t)\frac{\ptilde (X_T)p_{t|T}(X_t|X_T) \nabla \log \ptilde (X_T) }{q_{T|t}(X_T | X_t) \int q_{T|t}(X_T | X_t) \frac{\ptilde (X_T)p_{t|T}(X_t|X_T) }{q_{T|t}(X_T | X_t)}dX_T} dX_T.
        \end{align}
        By choosing $q_{T|t}(X_T | X_t)\propto p_{t|T}(X_t|X_T)$, we obtain
        \begin{align}
            \nabla \log p_{T-t} (X_t) &=\int q(X_T | X_t)\frac{\ptilde (X_T)\nabla \log \ptilde (X_T)}{ \int q(X_T | X_t) \ptilde (X_T)dX_T}   dX_T\\
            &\approx\sum_{n} \frac{\ptilde (X_T^{(n)})}{ \sum_{n}  \ptilde (X_T^{(n)})}\nabla \log \ptilde (X_T^{(n)}), \quad X_T^{(n)}\sim  q_{T|t}(X_T | X_t)\\
            & =: \widehat{\nabla \log p_{T-t} (X_t)}.
        \end{align}
        Then, iDEM matches $f_\theta(X_t, t)$ with $\widehat{\nabla \log p_{T-t} (X_t)}$ by $L^2$ loss.
        In optimal, the sampling process will approximate the time-reversal of a VE-SDE:
         \begin{align}\label{eq:idem_y_appendix}
            dY_t =\sqrt{2t} dW_t, \quad  Y_0 \sim p_\text{target}.
        \end{align}
Several extensions have been developed based on iDEM:
Bootstrapped Noised Energy Matching \citep[BNEM, ][]{ouyang2024bnemboltzmannsamplerbased} generalizes the self-normalized importance sampling estimator of the score to the energy function, enabling the training of energy-parameterized diffusion models.
They also proposed a bootstrapping approach to reduce the training variance.
Diffusive KL \citep[DiKL, ][]{he2024training} integrates this estimator with variational score distillation techniques \citep{poole2022dreamfusion, luo2024diff} to train a one-step generator as the neural sampler.
Also, DiKL proposes using MCMC to draw samples from $p_{T|t}(X_T|X_t)$ to estimate the score,  instead of relying on the self-normalized importance sampling estimator with the proposal $q_{T|t}(X_T | X_t)\propto p_{t|T}(X_t|X_T)$, leading to lower variance during training.
\par
~ We also note that iDEM's score estimator is closely related to stochastic control problems.
   One can re-express the estimator regressed in iDEM in terms of the optimal drift of a stochastic control problem~\citep{huang2021schr}, the optimal control $f^*$ can be expressed in terms of the score (e.g. See Remark 3.5 in \cite{reusmooth}) :
        \begin{align}\label{eq:optimal_drift}
             f^*_{t}(X_t) = - \nabla \log \phi_{T-t}(X_t) = - \nabla \ln \nu_{T-t}^{\mathrm{ref}}(X_T) +\nabla \log {p_{T-t}(X_t)},
        \end{align} where $\phi_t(X_t) $ is the 
     value function, which can be expressed as a conditional expectation via the Feynman-Kac formula followed by the Hopf-Cole transform \citep{hopf1950partial,cole1951quasi,fleming1989logarithmic}:
\begin{align}\label{eq:opt_val_func_appendix}
\phi_t(x) = \mathbb{E}_{X_T \sim q_{T|t}(X_T|x)}\left [\frac{\ptilde}{\nu^{\mathrm{ref}}_T}(X_T)\right]. 
\end{align}
Where in the case for VE-SDE (i.e. iDEM) and 
$\nu^{\mathrm{ref}}_t(x) = \mathcal{N}(x| 0, t+ \sigma_{\mathrm{prior}}^2)$ and thus $\phi_{T-t}(X_t) = \frac{X_t}{T-t +\sigma_{\mathrm{init}}^2} +\nabla \log {p_{T-t}(X_t)}$.

Note the MC Estimator of $\nabla \log \phi_{T-t}(X_t)$ (e.g. Equation \ref{eq:opt_val_func}) was used in Schr\"odinger-F\"ollmer Sampler \citep[SFS, ][]{huang2021schr} to sample from time-reversal of pinned Brownian Motion, yielding an akin estimator to the one used in iDEM, in particular they carry out an MC estimator of the following quantity:

\begin{align}
   \nabla \phi_{T-t}(x) = \frac{\mathbb{E}_{Z \sim \mathcal{N}(0,I)}\left [\nabla\frac{\ptilde}{\nu^{\mathrm{ref}}_T}( \mu_{T|T-t}x + \sigma_{T|T-t}Z)\right]}{\mathbb{E}_{Z \sim \mathcal{N}(0,I)}\left [\frac{\ptilde}{\nu^{\mathrm{ref}}_T}( \mu_{T|T-t}x + \sigma_{T|T-t}Z)\right]}.
\end{align}

Where, we have assumed that $q_{T|t}(x_T|x_t) =\mathcal{N}(x_T| \mu_{T|t}x_t, \sigma_{T|t})$ as is the case with most time reversal based samplers and generative models.





        \item Monte Carlo Diffusion \citep[MCD,][]{doucet2022score}: unlike other neural samplers, MCD's sampling process is fixed as $\mu_t = 0, \sigma_t = 1, b_t(X_t, t) = \nabla \log \pi_t(X_t)$, where $\pi_t$ is the geometric interpolation between target and prior, i.e., $\pi_t(X_t) = p_\text{target}^{\beta_t}(X_t)p_\text{prior}^{1-\beta_t}(X_t)$.
    It can be viewed as sampling with AIS using ULA as the kernel.
    Note, that this transport is non-equilibrium, as the density of $X_t$ is not necessary $\pi_t(X_t)$.
    Therefore, MCD trains a network to approximate the time-reversal of the forward process and perform importance sampling (more precisely, AIS) to correct the bias of the non-equilibrium forward process.
        \item Controlled Monte Carlo Diffusion \citep[CMCD,][]{vargas2024transport} and Non-Equilibrium Transport Sampler \citep[NETS,][]{mate2023learning,albergo2024nets}: 
        Similar to MCD, CMCD and NETS also set $b_t(X_t, t) = \nabla \log \pi_t(X_t)$ and $\pi_t$ is the interpolation between target and prior.
Different from MCD where the sampling process is fixed, CMCD and NETS learn $f_\theta(\cdot, t) = \mu_t(\cdot, t)$ so that the marginal density of samples $X_t$ simulated by \Cref{eq:x} will approximate $\pi_t$.
As a special case, Liouville Flow Importance Sampler \citep[LFIS,][]{tian2024liouville} fixes $\sigma_t=0$ and learns an ODE to transport between $\pi_t$.

    
\end{enumerate}

\subsection{Lanvegin Preconditioning in Diffusion/Control-based Neural Samplers}
\label{appendix:langevin_precond}

\textbf{Explicit Langevin preconditioning. }
In samplers including PIS, DDS, DFGS, DIS, etc., 
The network is parameterized with a skip connection using Lanvegin preconditioning:
\begin{align}
  f_\theta(\cdot, t) = \text{NN}_{1,\theta}(\cdot, t) + \text{NN}_{2, \theta}(t) \circ  \nabla\log \ptarget(\cdot).
\end{align}
In samplers like MCD, the forward process is a sequence of Lanvegin dynamics with invariant density $\pi_t$ as the interpolation between prior and target.
In CMCD and NETS, the drift of forward process is given by the network output plus a score term: 
\begin{align}
    f_\theta(X_t, t) +\sigma_t^2  \nabla \log \pi_t (X_t).
\end{align}
\textbf{Implicit Langevin preconditioning. } In iDEM, we regress the network with 
\begin{align}
     \nabla \log p_{T-t} (X_t) 
            &\approx\sum_{n} \frac{\ptilde (X_T^{(n)})}{ \sum_{n}  \ptilde (X_T^{(n)})}\nabla \log \ptilde (X_T^{(n)}), \quad X_T^{(n)}\sim  q_{T|t}(X_T | X_t).
\end{align}
Note that while the network does not explicitly depend on the score of the target density, the objective compels it to learn gradient information.
This gradient information is utilized during simulation when collecting the buffer every few iterations, effectively inducing an implicit Langevin preconditioning.

\textbf{No Langevin preconditioning. } 
LFIS does not rely on Langevin preconditioning during simulation.
Like NETS, it employs the PINN loss, but its sampling process is governed by an ODE.
Thus, similar to our discussion in the main text on eliminating Langevin preconditioning for PINN-based CMCD, LFIS inherently removes this dependency in its design.

LFIS adopts several tricks to stabilize the training and ensure mode covering:
it learns the ODE drift progressively, starting from the prior and gradually transitioning to the target.
Additionally, it employs separate networks for different time steps to prevent forgetting.
But even without these tricks,  our results in \Cref{tab:pinn} confirm the robustness of the PINN loss to Langevin preconditioning when the interpolation and prior are carefully tuned.





\section{NF-DDS}\label{appendix:nf-dds}

Here we present a derivation of the NF-DDS objective.
\begin{align}
   \nonumber & D_\text{KL}[\mathbb{Q}||\mathbb{P}]\\  \nonumber = &\E_{\mathbb{Q}} \left[
    \int_0^T \frac{1}{4v^2\beta_t}\|
   \tilde{F}_\theta(Y_t, T-t) - v^2\beta_{t} \nabla\log q_\theta(Y_t, T-t) - \beta_t Y_t
    \|^2 dt
    \right] + D_\text{KL}[q_\theta(\cdot, T)|| p_\text{target}]\\ \nonumber 
   = & 
    \int_0^T \E_{\mathbb{Q}} \left[\frac{1}{4v^2\beta_t}\|
   \tilde{F}_\theta(Y_t, T-t) - v^2\beta_{t} \nabla\log q_\theta(Y_t, T-t) - \beta_t Y_t
    \|^2\right] dt+ D_\text{KL}[q_\theta(\cdot, T)|| p_\text{target}]\\ \nonumber 
   = & 
    \int_0^T \frac{1}{4v^2\beta_t}\E_{{q_\theta(Y, T-t)}}\|
   \tilde{F}_\theta(Y, T-t) - v^2\beta_{t} \nabla\log q_\theta(Y, T-t) - \beta_t Y
    \|^2 dt+ D_\text{KL}[q_\theta(\cdot, T)|| p_\text{target}]\\
  = &
    \int_0^T \frac{1}{4v^2\beta_{T-t}} \E_{{q_\theta(Y, t)}} \|
   \tilde{F}_\theta(Y, t) - v^2\beta_{T-t} \nabla\log q_\theta(Y, t) - \beta_{T-t} Y
    \|^2 dt+ D_\text{KL}[q_\theta(\cdot, T)|| p_\text{target}].\label{eq:flowDDS-obj_appendix}
\end{align}

\section{NF-CMCD}\label{appendix:nf-cmcd}
In this section, we proposed a CMCD variation with normalizing flow for simulation-free training.
\par
In CMCD, we match the forward sampling process:
\begin{align}\label{eq:nf-sde-cmcd}
        dX_t = \left(\tilde{F}_\theta(X_t, t) + \sigma_t^2 \nabla\log q_\theta(X_t, t) \right)dt + \sigma_t\sqrt{2} dW_t, X_0 \sim q_\theta(X_0, 0),
    \end{align}
 with a target backward process, calculated by Nelson's condition assuming the marginal of the SDE at each time step matches with a prescribed marginal density  e.g. $\pi_t(\cdot) = p_\text{target}^{\beta_t}(\cdot)p_\text{prior}^{1-\beta_t}(\cdot)$:
 \begin{multline}\label{eq:nf-sde-cmcd-target}
        dY_t = -\Big(\tilde{F}_\theta(Y_t, T-t) + \sigma_{T-t}^2 \nabla\log q_\theta(Y_t, T-t)\\- 2 \sigma_{T-t}^2 \nabla\log \pi_{T-t}(Y_t, T-t) \Big)dt + \sigma_{T-t}\sqrt{2} dW_t, Y_0 \sim p_\text{target}.
 \end{multline}
Again, similar to NF-DDS, the time-reversal of \Cref{eq:nf-sde-cmcd} can be calculated by Nelson's condition:
\begin{align}\label{eq:nf-sde-cmcd-backward}
     dY_t = -\left(\tilde{F}_\theta(Y_t, T-t) - \sigma_{T-t}^2 \nabla\log q_\theta(Y_t, T-t) \right)dt + \sigma_{T-t}\sqrt{2} dW_t, Y_0 \sim q_\theta(Y_0, T).
\end{align}
By Girsanov theorem, the KL divergence between the path measure by \Cref{eq:nf-sde-cmcd-backward} (denoted as $\mathbb{Q}$) and \Cref{eq:nf-sde-cmcd-target} (as $\mathbb{P}$) is:
\begin{align}
  D_\text{KL}[\mathbb{Q}||\mathbb{P}] = 
    \int_0^T \frac{1}{\sigma_{t}} \E_{{q_\theta(Y, t)}} \| \sigma_{t}^2 \nabla\log q_\theta(Y, t) - \sigma_{t}^2 \nabla\log \pi_{t}(Y, t) 
    \|^2 dt+ D_\text{KL}[q_\theta(\cdot, T)|| p_\text{target}].
\end{align}
This coincides with the Fisher divergence between each marginal.

\par
After training, we can sample from 
\begin{align}\label{eq:nf-sde-cmcd_forward}
        dX_t = \left(\tilde{F}_\theta(X_t, t) + \sigma_t^2 \nabla\log \pi_t(X_t) \right)dt + \sigma_t\sqrt{2} dW_t, X_0 \sim q_\theta(X_0, 0),
\end{align}
with approximate reversal
\begin{align}\label{eq:nf-sde-cmcd-backward_sample}
     dY_t = -\left(\tilde{F}_\theta(Y_t, T-t) - \sigma_{T-t}^2 \nabla\log \pi_{T-1}(Y_t) \right)dt + \sigma_{T-t}\sqrt{2} dW_t, Y_0 \sim \ptilde.
\end{align}


\section{Langevin Preconditioning as an Approximation to the Optimal Control}
\label{appendix:langevin_approx}

As we discussed for iDEM (similarly for PIS and DDS), the optimal drift can be expressed as the gradient of the value function in \Cref{appendix:langevin_precond}
$$f^*_{t}(X_t) = - \nabla \log \phi_{T-t}(X_t)$$ 
where the value function itself can be expressed as
\begin{align}
   \log \phi_{t}(x) = \log \mathbb{E}_{X_T \sim q_{T|t}(X_T|x)}\left [\frac{\ptilde}{\nu^{\mathrm{ref}}_T}(X_T)\right].
\end{align}
where we define the reference process to be the uncontrolled process starting from $p_\text{prior}$, $\nu^{\mathrm{ref}}_T$ is the marginal density for the last time step of the reference process, and $q_{T|t}(X_T|x)$ is the conditional density for the reference process.
Specifically, for PIS, the reference is 
\begin{align}
    d X_t = dW_t, X_0 = 0;
\end{align}
and for DDS, the reference is 
\begin{align}
     dX_t = -\beta_{T-t} X_t dt + v \sqrt{2\beta_{T-t}} dW_t, \quad  X_0 \sim p_\text{prior} = \mathcal{N}(0, v^2).
\end{align}
We hence have
\begin{align}
   & \nu^{\mathrm{ref}}_T = \mathcal{N}(0, TI) \text{ for PIS}, \quad  \nu^{\mathrm{ref}}_T = \mathcal{N}(0, v^2I)\text{ for DDS}\\
    &  \mathbb{E}[X_T |X_t=x]  = x \text{ for PIS}, \quad  \mathbb{E}[X_T |X_t=x] = \sqrt{1-\lambda_t} \text{ for DDS}
\end{align}
Now consider approximating the above expectation with a point estimate using the posterior mean (e.g. reversing Jensens Inequality):
\begin{align}
    \log \phi_{t}(x) &\approx   \log  \frac{\ptilde}{\nu^{\mathrm{ref}}_T}(\mathbb{E}[X_T |X_t=x]) \\
    &\approx  \log \ptilde (\mathbb{E}[X_T |X_t=x]) -\log \nu^{\mathrm{ref}}_T (\mathbb{E}[X_T |X_t=x])
\end{align}
Taking gradients on both sides:
\begin{align}
    \nabla_x     \log \phi_{t}(x) &\approx \nabla_x   \log \ptilde (\mathbb{E}[X_T |X_t=x]) - \nabla_x    \log\nu^{\mathrm{ref}}_T (\mathbb{E}[X_T |X_t=x])
\end{align}

where for PIS this reduces to :
\begin{align}
    \nabla_x  \log \phi_{t}(x) &\approx \nabla_x\log \ptilde (x) + \frac{x}{T}
\end{align}
and for DDS
\begin{align}
    \nabla_x  \log \phi_{t}(x) &\approx \nabla_x\log \ptilde (\sqrt{1-\lambda_t} x) + \frac{\sqrt{1-\lambda_t}x}{v^2}
\end{align}

In both cases we can see the general form  $ \nabla_x  \log  \phi_{t}(x) =\nabla_x \log  \ptilde (a_t x) + c_t x$ for some time varying coefficients $a_t,b_t$, and thus it seems like a reasonable inductive bias to employ the Langevin preconditioning,
\begin{align}
  f_\theta(\cdot, t) = \text{NN}_{1,\theta}(\cdot, t) + \text{NN}_{2, \theta}(t) \circ ( \nabla\log \ptarget(a_t\cdot) + b_t \cdot),
\end{align}
or more simply $f_\theta(\cdot, t) = \text{NN}_{1,\theta}(\cdot, t) + \text{NN}_{2, \theta}(t) \circ  \nabla\log \ptarget(\cdot)$ as typically done in many SDE based neural samplers.





\section{Additional Experiment Details}




\subsection{Evaluation Metrics}
In this paper, we evaluate the samples quality by ELBO, EUBO and MMD. 
The ELBO (Evidence Lower Bound) is a lower bound of the (log) normalization factor, reflecting how well the model is concentrated within each mode;
on the other hand, EUBO \citep[Evidence Upper Bound, ][]{blessingbeyond}
provides an upper bound, representing if the model successfully covers all modes.
\par
The MMD (Maximum Mean Discrepancy) measures the distributional discrepancy between the generated samples and the target distribution.
We base our MMD implementation on the code by \citet{chen2024diffusivegibbssampling} at \url{https://github.com/Wenlin-Chen/DiGS/blob/master/mmd.py}, using 10 kernels and fixing the \texttt{sigma = 100}.
\par
For all experiments, we evaluate ELBO, EUBO and MMD with 10000 samples.



\subsection{Hyperparameters}

\begin{table}[H]
\centering
   \captionof{table}{Hyperparameters used for experiments.\vspace{-6pt}}\label{tab:hyperparam}
    \begin{tabular}{@{}lccccc@{}}
    \toprule
    method & objective & prior & lr & precond & network size \\ \midrule
    \multirow{6}{*}{DDS} & rKL & $\mathcal{N}(0, 30^2I)$ &  5e-4 & LG & [64, 64] \\
    & LV & $\mathcal{N}(0, 30^2I)$ &  5e-4  & LG & [64, 64] \\
    & TB & $\mathcal{N}(0, 30^2I)$ &   5e-4 & LG & [64, 64]\\
   & rKL & $\mathcal{N}(0, 30^2I)$ &  5e-4 & - / $\log p_\text{target}$& [256, 256, 256, 256, 256] \\
    & LV & $\mathcal{N}(0, 30^2I)$ &  5e-4  & - / $\log p_\text{target}$& [256, 256, 256, 256, 256] \\
    & TB & $\mathcal{N}(0, 30^2I)$ &   5e-4 & - / $\log p_\text{target}$  & [256, 256, 256, 256, 256]\\\midrule
    \multirow{4}{*}{CMCD} & rKL & $\mathcal{N}(0, 30^2I)$ & 5e-4 & LG & [64, 64] \\
    & LV & $\mathcal{N}(0, 30^2I)$ & 5e-3 & LG & [64, 64]\\ 
    & TB & $\mathcal{N}(0, 30^2I)$ & 5e-4 & LG & [64, 64] \\ 
    & rKL & $\mathcal{N}(0, 30^2I)$ & 5e-4 & - & [256, 256, 256, 256, 256]  \\
    \bottomrule
    \end{tabular}
\end{table}

We summarize the hyperparameters for DDS and CMCD in \Cref{tab:hyperparam}.
These hyperparameters are chosen according to \citet{blessingbeyond}.
For PINN-based experiments shown in \Cref{tab:pinn}, we follow the hyperparameter used in NETS \citep{albergo2024nets}, including network size, learning rate and its schedule. etc.



\section{Additional Experimental Results}\label{appendix:visualize}
\newpage
\begin{figure}[H]
    \centering
    \includegraphics[width=0.9\linewidth, trim={100, 0, 150, 0}, clip]{figures/dds_vis.pdf}
    \caption{Sampled obtained by DDS with different settings.
    The first line shows the initialization. }
\end{figure}
\begin{figure}[H]
    \centering
    \includegraphics[width=0.9\linewidth, trim={100, 0, 150, 0}, clip]{figures/cmcd_vis.pdf}
    \caption{Sampled obtained by CMCD with different settings.
    The first line shows the initialization and N/A indicates diverging. 
    We can see when trained with Langevin preconditioning, we can see that CMCD already captures modes after initialization.}
\end{figure}


\end{document}
