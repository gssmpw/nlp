\title{\textbf{Brief analysis of DeepSeek R1 and its implications for Generative AI}}

\date{\vspace{-5ex}}
%%\renewcommand\Authfont{\fontsize{11}{14.4}\selectfont}
%%\renewcommand\Affilfont{\fontsize{11}{14.4}\selectfont}



\author[1]{Sarah Mercer\thanks{smercer@turing.ac.uk}}
\author[1]{Samuel Spillard}
\author[1]{Daniel P. Martin}
\affil[1]{The Alan Turing Institute}


\maketitle
\begin{abstract}
   In late January 2025, DeepSeek released their new reasoning model (DeepSeek R1); which was developed at a fraction of the cost yet remains competitive with OpenAI's models, despite the US's GPU export ban. This report discusses the model, and what its release means for the field of Generative AI more widely.  We briefly discuss other models released from China in recent weeks, their similarities; innovative use of Mixture of Experts (MoE), Reinforcement Learning (RL) and clever engineering appear to be key factors in the capabilities of these models.  This think piece has been written to a tight timescale, providing broad coverage of the topic, and serves as introductory material for those looking to understand the model's technical advancements, as well as its place in the ecosystem.  Several further areas of research are identified.
\end{abstract}
\vspace{10mm} 

%\thispagestyle{fancy} %maketitle automatically triggers \thispagestyle{plain} which changes the headers/footers