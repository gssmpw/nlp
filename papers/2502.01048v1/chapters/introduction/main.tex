\chapter{General Introduction}
\label{chap:intro}

\epigraph{``One sits down on a desert sand dune, sees nothing, hears nothing. Yet through the silence something throbs, and gleams.''}{\textit{Antoine de Saint-Exupéry}}

Some mysteries are meant to remain unsolved; Deep Learning is not one of them. At the entrance of this document, before exploring the subject of Deep Learning—where the \textit{silence} of our understanding contrasts sharply with the powerful \textit{noise} of its achievements—let us take a moment to reflect on the origins of AI.

The concept of creating thinking machines emerged during the Dartmouth Workshop in 1956~\cite{mccarthy2006proposal}\footnote{Interestingly, Lloyd Shapley, who we will encounter later in this manuscript, was invited and is already mentioned in the workshop proposal.}, a seminal event that marked the inception of Artificial Intelligence (AI) as a formal research discipline. This gathering laid the foundational stones for exploring the potential of developing machines endowed with intelligent capabilities. Since then, AI has traversed through various evolutionary stages, characterized by alternating waves of enthusiasm spurred by significant breakthroughs. Alan Turing's prediction regarding the ascent of artificial intelligence appears to have been prophetic, as evidenced by the successive emergence of Machine Learning and subsequently Deep Learning, two branches of statistical learning.

Approximately a decade ago, AI witnessed a transformation with the advent of Deep Learning (DL)~\cite{lecun2015deep,Serre2019DeepLT}. Deep Learning methodologies, rooted in deep neural networks, have catalyzed revolutionary advancements across diverse domains by showcasing exceptional aptitude in discerning complex patterns and behaviors from large datasets. The surge in Deep Learning's prominence can be attributed to several factors, including the exponential increase of data, advancements in hardware and software for machine learning, and pivotal breakthroughs in research methodologies.

A pivotal moment in the adoption of Deep Learning occurred in 2012, when the Computer Vision (CV) community witnessed a groundbreaking development. The winning solution of the ImageNet Large Scale Visual Recognition Challenge (ILSVRC), spearheaded by ~\cite{krizhevsky2012imagenet}, showcased the prowess of deep neural networks in image classification tasks. For the first time, a deep learning model outperformed traditional handcrafted methods by automatically learning rich and discriminative features directly from raw pixel data. This achievement marked a paradigm shift in computer vision, setting the stage for subsequent advancements in deep learning-based image analysis.
The success of deep learning methods extended beyond image classification, encompassing a broad spectrum of visual tasks, including object detection and segmentation. These techniques, empowered by the sheer complexity and expressiveness of deep neural networks, surpassed conventional approaches, exceeding expectations and inspiring further innovation.

Despite their remarkable achievements, deep learning models often operate as black boxes, with their decision-making processes obscured by their immense complexity. Moreover, they are susceptible to errors and can exhibit undesirable behaviors, such as learning shortcuts~\cite{geirhos2020shortcut} to achieve high accuracy on specific tasks. Recognizing these challenges, the need for eXplainable AI (XAI) methodologies has emerged~\cite{doshivelez2017rigorous}, aiming to elucidate the inner workings of deep learning models and enhance their transparency and trustworthiness.

This chapter aims to provide a succinct introduction and essential background in both deep learning and explainability, which will be useful for understanding the remainder of the manuscript. It is not intended as a comprehensive review of the state of the art but will offer key insights for grasping the thesis structure. The chapter is organized as follows: \autoref{sec:intro:deep_learning} revisits statistical learning fundamentals, deep learning, and its application in computer vision, while \autoref{sec:intro:xai_landscape} presents an overview of the explainability approaches in AI. We will conclude with \autoref{sec:intro:contrib}, where we describe the structure of the manuscript and the contributions it builds upon.

\newcommand{\risk}{\s{R}}
\newcommand{\emprisk}{{\risk_{\text{emp}}}}

\section{Deep Learning Background}
\label{sec:intro:deep_learning}

In this section, we will briefly revisit the framework of this study, namely deep neural networks. To do so, we will briefly review the statistical learning framework in which we operate, and then we will explore the different components of neural network architectures for vision tasks. For a more comprehensive background, we encourage the reader to refer to \cite{goodfellow2016deep}.

\subsection{Statistical Learning}

Deep learning methodologies are firmly rooted in the principles of statistical learning theory~\cite{vapnik1999overview}. Consider measurable spaces $\sx \subset \Real^d$ and $\sy \subset \Real$, representing the input and output spaces respectively\footnote{All topological spaces are equipped with their Borel $\sigma$-algebra}. In supervised learning, we are presented with a dataset of labeled instances:
\[
\s{D} = \{ (\vx_1, y_1), \ldots, (\vx_n, y_n) \} \in (\sx \times \sy)^n.
\]
Our objective in this context is to ascertain the best approximation $\f$, the stochastic relationship between input $\vx \in \sx$ and its corresponding label $y \in \sy$, which is formalized as the conditional probability measure $\E(y | \vx)$ under the probability measure $\P_{\rvx,\ry}$ on $(\sx \times \sy)$.

Achieving the best approximation entails specifying a hypothesis space $\fspace$ comprising potential predictors and defining an appropriate loss function $\ell : \sy \times \sy \to \Real$ that quantifies the discrepancy between a predictor and the true label. The concept of a loss function originated in statistical decision theory, pioneered by~\cite{wald1949statistical}, with roots tracing back to Laplace’s theory of errors.

For instance, in a dataset containing images of dogs and cats, $\sx$ denotes the space of images and $\sy$ represents the labels $\{+1, -1\}$, where $+1$ denotes the presence of a dog and $-1$ denotes the presence of a cat. A conceivable loss function could measure the Euclidean distance between the predicted label and the ground truth.

In essence, the learning problem can be formulated as:
\[
\risk(\f) \defas \underset{(\rvx, \ry) \sim \P_{\rvx,\ry}}{\E}\ell(\f(\rvx), \ry)
~~~~ \text{and} ~~~~
\f^\star = \argmin_{\f \in \fspace} \risk(\f).
\]

With a $\fspace$ the set of all possible functions from $\sx$ to $\sy$. In practical scenarios, access to the true distribution $\P_{\rvx,\ry}$ is typically unavailable\footnote{As described in the Notations section, we use $(\rvx, \ry)$ to denote random variables representing theoretical constructs of inputs and outputs from the probability distribution, and $(\vx, y)$ to denote specific samples or instances from our dataset. This notation clarifies the distinction between theoretical models and empirical data in our analysis.}. Hence, we resort to approximating the learning problem using the available training set $\s{D}$ to minimize the so-called \textit{empirical risk} $\emprisk(\f)$:

\begin{definition}[Empirical risk.]
For a training dataset $\s{D} = \{(\vx_1, y_1), \ldots,  (\vx_n, y_n) \}$ and a function $\f : \sx \to \sy$, the empirical risk with respect to the loss $\ell$ is defined as:
\[
\emprisk(\f) \defas \frac{1}{n}\sum_{i=1}^n \ell(\f(\vx_i), y_i).
\]
\end{definition}

The objective is to minimize the average loss over the training data. This fundamental learning approach is known as \textit{empirical risk minimization}(ERM):

\begin{definition}[ERM learning algorithm]
Given a hypothesis set $\fspace$, the ERM selects $\f^\star$, which minimizes the empirical risk within $\fspace$:
\[
\f^\star = \argmin_{\f \in \fspace} \emprisk(\f)
\]
\end{definition}

However, we face two challenges here. Firstly, the problem is not always convex (unless employing a simple model like linear regression), rendering empirical risk minimization computationally intractable in practice. Secondly, this approach does not guarantee minimization of errors on unseen data points, leading to overfitting issues that hinder generalization and necessitate regularization.

\paragraph{Regularization}

Minimizing $\emprisk(\cdot)$ alone does not suffice for achieving robust generalization. A common strategy involves augmenting the objective with a regularization term $\Omega(\f)$:
\[
\f^\star = \argmin_{\f \in \fspace} \emprisk(\f) + \Omega(\f).
\]

Here, $\Omega(\f)$ regulates the complexity of the function. Optimization of the adjusted loss function mitigates overfitting and facilitates better generalization. 



\paragraph{Stochastic Gradient Descent (SGD)}

In the pursuit of optimizing the empirical risk, our functions within the hypothesis space $\fspace$ are usually parameterized by a set of parameters $\parameters \in \Theta$. The goal of learning in this context becomes the optimization of these parameters to minimize the loss function, effectively finding the best approximation $\f^\star$ that represents our model. A cornerstone for this optimization of empirical risk, especially within the realm of deep learning, is \textit{Stochastic Gradient Descent} (SGD). This method stands in contrast to the classical Gradient Descent approach, which necessitates computing the gradient of the loss function $\ell$ across the entire dataset to execute a single parameter update. Instead, SGD opts for a more dynamic and computationally efficient route by iteratively adjusting the model parameters utilizing a randomly selected subset of the data at each iteration. This strategy markedly diminishes computational demands, thereby enabling the training of sophisticated models on voluminous datasets.

\begin{definition}[SGD]
Given a loss function $\ell$, a learning rate $\eta$, a training dataset $\s{D}$ and a mini-batch $\s{B} = \{ \vx_i, y_i \}_{i=0}^{|\s{B}|} \subset \s{D}$, SGD iteratively updates the model's parameters $\parameters$ by calculating the gradient of $\ell$ with respect to $\parameters$ on $\s{B}$:
\[
\parameters_{t+1} = \parameters_t - \eta \sum_i^{|\s{B}|} \nabla_{\parameters}  \ell(\f(\vx_i; \parameters_t), y_i),
\]
where $\parameters_{t}$ represents the successive parameters, $\f(\cdot; \parameters)$ ny prediction function parametrized by $\parameters$ and $t$ the current iteration step.
\end{definition}

This process of iterative parameter adjustment via SGD is a direct application of the empirical risk minimization principle, adapted for the practical challenges of training deep neural networks. It allows for efficient computation and robust search through the parameter space, even in the face of complex models and large datasets.

The element of randomness in SGD, by way of selecting data points, injects a beneficial noise into the optimization trajectory. This aspect can aid in circumventing local minima, thus potentially steering the optimization towards more optimal solutions in the complex, non-convex problem spaces typical of deep neural networks. Moreover, the ability of SGD to operate efficiently with mini-batches underscores its indispensability for deep learning models. This is particularly relevant in scenarios where the sheer scale of the dataset and the model's complexity render full-batch processing impractical.

For more detail on statistical learning theory, we refer the reader to \cite{hastie2009elements}. Having discuss the learning framework in which we operate, we proceed to introduce the focal point of this work: neural networks.

\subsection{Neural Networks}

In this section, we revisit the core components of deep learning: neural networks, emphasizing their parameterization and the pivotal role of convolution operations, particularly for image data. Neural networks, parameterized by weights and biases collectively denoted as $\parameters$, are foundational to deep learning's success in various domains.

\begin{definition}[Neuron]
A neuron is a function $\neuron: \Real^d \to \Real$, parameterized by $\parameters = \{\v{w}, b\}$, and defined as:
\[
\neuron(\vx; \parameters) \defas \sigma(\v{w}^\tr \vx + b),
\]
where $\v{w} \in \Real^d$ is the weight vector, $b \in \Real$ is the bias, and $\sigma: \Real \to \Real$ is a non-linear activation function.
\end{definition}

Neurons aggregate input signals linearly weighted by $\v{w}$, add a bias $b$, and apply a non-linear function $\sigma$ to produce an output. This process enables the model to learn complex relationships between inputs and outputs.

A neural network combines multiple neurons in layers, and multiple layers can be stacked to form a deep neural network. Let's denote $\parameters$ as the collection of all parameters across the network. Then, a fully connected feedforward neural network (FCNN) can be defined as follows:

\begin{definition}[Fully Connected Feedforward Neural Network (FCNN)]
A FCNN $\f(\vx; \parameters)$ of $L$ layers is defined as a composition of layers of neurons:
\[
\f(\vx; \parameters) \defas (\layer^{(L)} \circ \ldots \circ \layer^{(1)})(\vx),
\]
where $\layer^{(i)}(\v{a}; \parameters^{(i)}) = \sigma(\m{W}_{(i)}^\tr \v{a} + \v{b}_{(i)})$ denotes the $i$-th layer function, with $\parameters^{(i)} = \{\m{W}_{(i)}, \v{b}_{(i)}\}$ being the parameters of the $i$-th layer, and $\v{a}$ the activations from the previous layer.
\end{definition}

At the core, the ensemble of neural networks we will study in this work is characterized by this structured aggregation of distinct layers, or "blocks," each serving a unique computational purpose. For the rest of this work, we will refer to this architecture interchangeably as FCNN or MLP.
Among these, certain blocks hold particular relevance to our investigation. Consequently, we will dedicate the concluding segment of this section to a description of these components. Specifically, our focus will encompass convolutional blocks, residual connections, batch normalization and finally, we will delve into the attention block.

\paragraph{Convolution layer.} Those layers are particularly adept at handling grid-like data, such as images, through the use of convolution operations. Convolution leverages the spatial structure of data, allowing the network to learn filters that capture local patterns, and stacking convolution able the model to build more global features such as shape.

\begin{definition}[Convolution Operation]
The convolution of an input $\vx$ with a filter $\v{w}$, parameterized by $\parameters = \{\v{w}, b\}$, for a single channel, is defined as:
\[
(\vx \bigotimes \v{w})_{i,j} \defas \sum_{m}\sum_{n} \vx_{m,n} \cdot \v{w}_{i-m, j-n} + b,
\]
where $\bigotimes$ denotes the convolution operation. For multichannel inputs, this operation is performed independently for each channel and summed to produce a single output.
\end{definition}

Alternatively, the convolution operation can be understood in the frequency domain through the Fourier Transform~\cite{chi2020fast}, which translates the convolution into a point-wise product in the frequency space. Following a convolution operation in a CNN, the output is typically passed through a non-linear activation function, (e.g., ReLU), to introduce non-linearity into the model. The result of applying a convolution followed by an activation function is known as an activation or \textit{feature map}. Each feature map has a dimensionality of $W \times H \times C$, where $W$ and $H$ are the width and height of the map, respectively, and $C$ refers to the number of channels. These dimensions correspond to the spatial dimensions of the image being processed and the depth of the feature map, which represents the number of filters applied during the convolution. 

Modern neural networks often cascade multiple convolution layers, alternating them with pooling layers and activation functions. Pooling layers reduce the spatial dimensions ($W$ and $H$) of the feature maps, helping to decrease the computational load and increase the \textit{receptive field} of the features. The combination of convolution, activation, and pooling layers allows the network to learn hierarchical representations of the input data, where higher-level features are composed of lower-level ones.

\paragraph{Residual Connections.}
The introduction of residual connections marked a significant advancement in deep learning architectures. Residual connections was introduced in 
\cite{he2016deep} to address the \textit{vanishing gradient} problem~\cite{hochreiter1998vanishing} that arises in very deep networks by allowing gradients to flow through a shortcut path. It consists in re-applying activations of previous layer into the next layer: 

\begin{definition}[Residual Connection]
A residual connection in a neural network allows the input of a layer to be added to its output, facilitating the learning of an identity function. This is defined as:
\[
\f(\vx; \parameters) \defas \vx + \layer(\vx; \parameters),
\]
where $\f$ represents the function implemented by the layer with residual connection, $\layer$ is the layer's original transformation function, and $\vx$ is the input to the layer. The parameters $\parameters$ denote the weights and biases of $\layer$.
\end{definition}

It turns out that allowing information to bypass one or more layers facilitate the backpropagation, thus ensuring that deeper networks can still learn effectively. This innovation has been fundamental in the development of state-of-the-art architectures. 


\paragraph{Batch Normalization.} Still in the purpose of enhancing the training stability of deep neural networks, Batch Normalization~\cite{ioffe2015batch} emerges as a crucial innovation. This technique propose to adjust the internal covariate shift -- the distribution of each layer's inputs changes during training, as the parameters of the previous layers change. To do so, Batch Normalization standardizes the inputs to a layer for each mini-batch, thus stabilizing the learning process and allowing for higher learning rates and quicker convergence.

\begin{definition}[Batch Normalization]
Given a mini-batch of inputs $\s{B} = \{\vx_1, \ldots, \vx_n \}$, Batch Normalization normalizes the input of each feature to have zero mean and unit variance. Additionally, it introduces two trainable parameters, $\v{\gamma}$ and $\v{\beta}$, to scale and shift the normalized value. Mathematically, for an input feature $\vx$, the Batch Normalization transform is defined as:
\[
\text{BN}_{\v{\gamma}, \v{\beta}}(\vx) \defas \v{\gamma} \left(\frac{\vx - \v{\mu}_{\s{B}}}{\sqrt{\v{\sigma}_{\s{B}}^2 + \varepsilon}}\right) + \v{\beta},
\]
where $\v{\mu}_{\s{B}} = \frac{1}{n} \sum_i^n \v{x}_i$ is the empirical mean over the mini-batch $\s{B}$ ($\v{\sigma}_{\s{B}}^2$ the empirical variance), and $\varepsilon$ is a small constant\footnote{This constant may have a real impact on the training (see ~\cite{nado2020evaluating}) and are not usually well defined: $1e^{-3}$ for Tensorflow~\cite{tensorflow2015} and $1e^{-5}$ for Pytorch~\cite{paszke2019pytorch}.} added for numerical stability. The parameters $\v{\gamma}$ and $\v{\beta}$ are learned along with the original model parameters, allowing the network to undo the normalization if it is found to be counter-productive for the learning of certain layers.
\end{definition}

The $(\v{\gamma}, \v{\beta})$ parameters are usually of size $p$ with $p$ the number of features, which means that $\text{BN}_{\v{\gamma}, \v{\beta}}(\cdot)$ control the mean and variance on each channels for convolution neural net, or neurons for a MLP.

As previously stated, batch Normalization not only accelerates the training process by reducing the number of epochs required to train deep networks but also mitigates the problem of gradient vanishing/exploding, making it easier to train deep networks with saturating non-linearities. It has since become a standard component in the architecture of modern neural networks. However, lately new kind of normalization have emerged such as \textit{LayerNorm}~\cite{ba2016layer}.

\paragraph{Attention Mechanisms in Vision.} Attention mechanisms were introduced in~\cite{vaswani2017attention} and have had a profound impact across the deep learning community. Originating in NLP with the advent of Large Language Models (LLMs), these mechanisms have also significantly influenced the field of computer vision with the ViT architecture~\cite{dosovitskiy2020image,zhai2022scaling,steiner2021train}. The Attention, as initially described by~\cite{vaswani2017attention}, involves dynamically computing weights --- attention weights -- among multiple \textit{tokens} (e.g., words in a sentence, patches in an image) to facilitate their "mixing" to create a feature. This interaction among all input variables is often not possible with a single convolution (e.g., when we use filters smaller than the image size, the top-left pixel does not interact with the bottom-right pixel). From this perspective, convolution imposes an inductive bias of local interactions, whereas attention mechanisms enable all sorts of interactions, even between distant image patches. Formally, an image $\vx$ is divided into patches called tokens $\{\v{t}_1, \ldots, \v{t}_n\}, \v{t}_i \in \Real^{p}$ of dimension $p$. Each of these tokens is then processed through multiple MLPs to reduce their dimensions to $p' << p$, producing three matrices $\m{Q}, \m{K}, \m{V}$, termed key, query, and value, upon which the attention operation is then applied:



\begin{definition}
    
Given an input image $\vx$, segmented into a sequence of tokens $\m{T} = \{ \v{t}_1, \ldots, \v{t}_n \}$, where each $\v{t}_i \in \Real^p$ represents a patch of the image. These tokens are then processed through three separate MLPs, each one differently parametrized. The attention mechanism is then applied. Formally:

\begin{align*}
&\m{Q} \defas \text{MLP}_Q(\m{T}; \parameters_Q) ~~~
\m{K} \defas \text{MLP}_K(\m{T}; \parameters_K) ~~~
\m{V} \defas \text{MLP}_V(\m{T}; \parameters_V) \\
&\text{Attention}(\m{Q}, \m{K}, \m{V}) \defas \text{softmax}\left(\frac{\m{Q}\m{K}^\top}{\sqrt{p'}}\right)\m{V},
\end{align*}

where $\m{Q}$, $\m{K}$, and $\m{V}$ all lie in $\Real^{n \times p'}$. The softmax operation is applied to the rows of the resulting matrix, allowing the model to dynamically allocate attention across different regions of the input based on the relevance of each token to another token.

\end{definition}


This mechanism is especially advantageous in vision for its capacity to adaptively enhance the receptive field, enabling extensive interactions (as necessary for recognizing shapes, for instance). Leveraging attention allows models to process large volumes of visual data efficiently, focusing computational resources on the most informative parts of an image. This approach has led to the development of Transformer models, such as the Vision Transformer (ViT)~\cite{dosovitskiy2020image}, which is now considered as state-of-the-art across a wide array of computer vision tasks.

However, the attention mechanism's computational efficiency is hampered by the quadratic growth of the matrix-matrix multiplication $\m{Q}\m{K}^\top$ cost in relation to the number of tokens. This issue limits its scalability, particularly for high-resolution images or large datasets. In response, subsequent research has focused on devising strategies to mitigate this computational burden. Alternative approaches, such as sparse attention patterns, low-rank approximations, and locality-sensitive hashing, have been proposed to reduce the complexity from quadratic to sub-quadratic or even linear, with respect to the number of tokens. For a more in-depth discussion on these solutions, readers are encouraged to refer to \cite{zhang2023cab}.

\paragraph{Closing Note.} Recognizing the critical role of parameters ($\parameters$) in the various blocks we've discussed is essential. Deep neural networks are incredibly effective across numerous domains, largely due to their extensive parameterization -— for instance, ResNet50 with 25 million parameters and ViT-H boasting 632 million. This complex parametrization does not only boost their performance but also obscures their decision-making processes, making them \textit{black boxes}. This opacity underscores the necessity for Explainable AI (XAI). In the following section, we'll delve into the motivations behind XAI and explore how it can reveal the inner workings of these complex models, making their operations more transparent and understandable.


\section{Explainability Landscape}
\label{sec:intro:xai_landscape}


The objective of this section is twofold: firstly, to underscore the imperative of explainability within machine learning, and secondly, to delineate a concise overview of the diverse methodologies underpinning explainability -- to say it otherwise, to ``flag'' the existing sub-fields. We aim to acquaint the reader with pivotal terms and explainability methods discussed throughout this manuscript. To do so, we propose a taxonomy categorizing explainability methods into three dimensions: methods that explain individual predictions, those studying the model internal mechanics, and those interpreting the data's influence. This classification, albeit simplistic, facilitates a structured introduction to the landscape of explainability.

\begin{figure}[ht]
    \centering
    \includegraphics[width=0.8\textwidth]{assets/introduction/black_box_schema.png}
    \caption{\textbf{Illustration of the Black-Box Problem.} Neural networks undergo training on a \textit{Training Dataset} through a specific \textit{Learning Algorithm}. After training, the model performs inferences using the learned parameters to make \textit{Predictions}. However, the multitude of operations from \textit{Input} to prediction is excessively complex for human comprehension, thus the name Black-box.}
    \label{fig:intro:blackbox}
\end{figure}

\subsection{Motivation}
\label{sec:intro:motivation}

Tracing the origins of explainability in AI is akin to exploring the very essence of science, as the pursuit of explanations, particularly within the realm of scientific thought, has historically been a foundational pillar, as highlighted by~\cite{hospers1946explanation}, suggesting that the impetus for explanation is deeply rooted in the fabric of scientific discourse itself. 
A closer intellectual lineage to modern explainability could be found back over half a century, finding ground in the domain of mathematical logic~\cite{hempel1948studies}.
However, it was not until the advent of deep learning, that the modern conceptualization of explainability -- as it is addressed within this manuscript -- emerged. Unsurprisingly, it is deep learning that has been the catalyst for the establishment of this burgeoning research field, and it is important to delineate the goal of XAI as well as its expected outcomes prior to examining the existing body of work.

It would typically be prudent to start with a definition; however, the quest for a formal definition of explainability is unlikely to be straightforward. \cite{lombrozo2006structure} notes that explanations serve as the currency of our belief systems, a medium through which we exchange and interrogate our understanding of the world. This discourse raises fundamental questions about the nature of explanations and the criteria that distinguish more effective explanations from their less compelling counterparts. The academic community has variously characterized explanations as embodying a \textit{deductive-nomological} essence \cite{hempel1948studies}, akin to logical proofs, or as mechanisms that provide a deeper understanding of underlying processes, as proposed by \cite{bechtel2005explanation}. \cite{keil2006explanation} proposed a broader conceptualization, advocating for an understanding of explanations as embodying an implicit explanatory comprehension\footnote{Interestingly, one could interpret the essence of this article from an informational perspective on explainability as an addition of information \textit{given} a common body of knowledge.}.

Given the rich literature on this subject, attempting to distill a singular definition that encompasses the entire spectrum of use-cases and motivations within the field would be a \textit{Sisyphean} task and would take us too far. Therefore, we propose to adopts a pragmatic approach to defining explainable AI (XAI), not through abstract or absolute terms but by aligning with the specific objectives it seeks to achieve. This approach will thus have to settle for a localized and use-case specific definition of explainability, allowing us to focus on the technical aspects of the domain. Among the myriad objectives identified in the literature~\cite{jacovi2021formalizing,miller2017explanation,survey2019metrics,saeed2023explainable,weber2023beyond,antoniadi2021current,das2020opportunities}
, six primary goals could be noted, as central to the discourse on XAI:

\begin{itemize}
    \item \textbf{Building trust in model predictions.} For example, in healthcare, AI-assisted diagnostics can use explainability to highlight influential areas in medical images, helping clinicians trust and verify AI diagnoses by visually indicating regions of interest.
    
    \item \textbf{Elucidating important aspects of learned models.} The SNCF for example, could need explainability in autonomous railway systems to help engineers understand the decision-making process behind navigational actions, ensuring the AI correctly recognizes stop signs and detect obstacles.
    
    \item \textbf{Assisting in meeting regulatory requirements and facilitating the certification process.} Financial services leveraging AI for credit scoring can use explainability to detail how individual features influence credit scores, aiding in compliance with regulations like GDPR.
    
    \item \textbf{Uncovering and addressing biases or unintended effects learned by models.} Explainability can reveal if an AI recruitment tool unfairly weighs certain demographics, allowing developers to correct these biases.
    
    \item \textbf{Detecting and preempting potential failure cases.} In predictive maintenance for manufacturing, explainability reveals conditions leading to equipment failures, enabling preemptive actions to prevent or mitigate effects.
    
    \item \textbf{Debugging models to enhance training methodologies:} Still in the context of the SNCF's autonomous railway systems, explainability can help engineers and developers understand why a model might misinterpret sensor data or fail to correctly predict maintenance needs. By analyzing instances where the model's performance deviates from expectations, the teams can refine data inputs, adjust model parameters, and ultimately improve the reliability and safety of autonomous railway operations.
    
\end{itemize}

These objectives highlight the heterogeneity of aims within the field and underscore the magnitude of the challenges that confront us. Having established that XAI presents a real \textbf{conceptual challenge}, we will now see that it is also a real \textbf{technical challenge}.


\subsection{Explaining Predictions}

\begin{figure}[ht]
    \centering
    \includegraphics[width=1.\textwidth]{assets/introduction/attributions.jpeg}
    \caption{\textbf{Attribution Methods.} Attribution methods will be the subject of the \autoref{chap:attributions}. These methods aim to explain a specific prediction through heatmaps, where hotter areas indicate a greater significance of the pixel for the decision.}
    \label{fig:intro:attributions}
\end{figure}

The development of methods to explain model predictions has been a critical aspect of research, originating with the introduction of attribution methods~\cite{Zeiler2011}. These approaches aim to clarify the rationale behind a model's decision, whether it is the classification of an instance, the detection of an object within an image, or the prediction of a value in regression tasks. \textit{Attribution methods}, which produce a heatmap to represent the importance of each input variable (see \autoref{fig:intro:attributions}), are among the most widely used due to their straightforward implementation in automatic differentiation frameworks such as TensorFlow~\cite{tensorflow2015} and PyTorch~\cite{paszke2019pytorch}.

A broad range of attribution techniques exists, using gradients~\cite{zeiler2013visualizing,shrikumar2017learning,sundararajan2017axiomatic,smilkov2017smoothgrad}, perturbations~\cite{ancona2017better,petsiuk2018rise,Fong_2017,fel2021sobol,novello2022making}, or internal model activations~\cite{Selvaraju_2019,chattopadhay2018grad} to generate explanations. A general definition is given below:

\begin{definition}[Attribution Method.] 
\label{def:attributions}
For a model $\f : \sx \to \sy$ and an input $\vx \in \sx$, an attribution method is a functional:

\[
\explainer : \fspace \times \sx \to \Real^{|\sx|}
\]

where $\explanation = \explainer(\f, \vx)$ (with $\f \in \fspace$) represents an attribution map that explains the prediction of $\f$ for input $\vx$. The higher the scalar value in $\explanation$, the more important the variable is considered.
\end{definition}

Despite their utility, attribution methods face challenges related to reliability~\cite{adebayo2018sanity,sixt2020explanations,ghorbani2017interpretation,slack2021counterfactual,sturmfels2020visualizing,hsieh2020evaluations,hase2021out}, computational efficiency~\cite{novello2022making}, and the implicit assumptions about importance~\cite{eva2}. A dedicated chapter (\autoref{chap:attributions}) further explores these methods, addressing their complexities and constraints.

\subsection{Explaining the Model}

Explaining a model involves uncovering the internal mechanics that drive its predictions. This can be approached through various methodologies, each aiming to make the model's operations or internal states more transparent.

\begin{figure}[ht]
    \centering
    \includegraphics[width=1.\textwidth]{assets/introduction/cav.jpeg}
    \caption{\textbf{Concept Activation Vector (CAV).} An example of extracting the "striped" concept using images featuring this concept and random images. A classifier in the intermediate space is utilized to identify the CAV as the vector orthogonal to the decision boundary. Methods for analyzing concepts will be discussed in \autoref{chap:concepts}.}
    \label{fig:intro:concepts}
\end{figure}

\paragraph{Concept-based Explainability.} Recent developments in explainability have underscored the need to go beyond attribution methods~\cite{doshivelez2017rigorous}. A flagship of these methods is \textit{concept-based} explainability~\cite{kim2018interpretability}, which involves identifying human-understandable concepts within a model. Briefly, this method compares two datasets, one containing the concept of interest and a 'random' dataset used for one-class detection with a linear model. The orthogonal to the decision boundary is called a concept vector, see \autoref{fig:intro:concepts}. Further methods have been proposed to not just retrieve human-defined concepts, but to study concepts utilized by the model itself~\cite{ghorbani2019towards,fel2023craft,lrpconcepts,graziani2023concept,zhang2021invertible,fel2023holistic}. Unlike attribution methods that provide a heatmap of input importance, concept-based explainability seeks to discover "what" triggers a feature. A general approach to defining a concept within a model's operational framework is as follows:

\begin{definition}[Concept Vector.]
\label{def:intro:cav}
Given a Fully Connected Feedforward Neural Network (FCNN) $\f : \sx \to \actspace \subseteq \Real^d$, a concept vector $\v{v} \in \Real^d$ is identified as a vector representing a concept in the activation space $\actspace$ of the FCNN. Depending on the context, the alignment or dot product between an activation $\v{a} \in \actspace$ and the concept vector $\v{v}$ indicates the extent to which $\v{a}$ it embodies the concept.
\end{definition}

An entire chapter is dedicated to these methods, offering a more nuanced understanding of what the model has learned.

\begin{figure}[ht]
    \centering
    \includegraphics[width=1.\textwidth]{assets/introduction/feature_viz.jpeg}
    \caption{\textbf{Illustration of Feature Visualization (FV).} An example of visualization for neurons (ladybug and goldfish), channels of a convolutional network as well as for CAV using FV. Feature Visualizations will be discussed in \autoref{chap:concepts}.}
    \label{fig:intro:fviz}
\end{figure}

\paragraph{Feature Visualization.}
Feature visualization~\cite{olah2017feature} aims to generate images that maximally activate specific parts of the network, providing insights into the kinds of features to which the network is sensitive and identifying what each component of the network is looking for in its inputs.

\begin{definition}[Feature Visualization]
\label{def:intro:feature_viz}
Given a FCNN $\f : \sx \to \Real^d$ and a target structure (neurons, channels, concept) $\v{v} \in \Real^d$, the feature visualization $\vx^\star$ associated with $\v{v}$ is defined as:
\[
\vx^\star = \argmax_{\vx \in \sx} \langle \f(\vx), \v{v} \rangle - \Omega(\vx),
\]
where $\Omega(\vx)$ typically applies a penalty to ensure the resulting image remains within a natural image manifold, and $\v{v}$ can be a one-hot vector targeting a specific neuron or a more complex structure representing a concept.
\end{definition}

Feature visualization techniques have been instrumental in uncovering fascinating phenomena and features in convolutional models~\cite{nguyen2016multifaceted,nguyen2019understanding} and especially~\cite{cammarata2020thread}. \autoref{chap:concepts} has a section dedicated to this subject and proposes improvements to existing feature visualization methods.


\paragraph{Interpretability by Design.} Creating models with interpretability as a foundational goal entails architecting models to output not only predictions but also an explanation understandable to humans. 

Recently, a promising methodology~\cite{bohle2022b,bohle2023holistically} has been proposed, it involves training models that dynamically adjust their internal parameters in response to input data, reminiscent of synaptic plasticity~\cite{abraham1996metaplasticity}. Specifically, for a given input point $\vx$, these models generate a unique set of linear weights that directly map $\vx$ to its prediction $y = \f(\vx) \vx^\tr$ with $\f : \sx \to \Real^{|\sx|}$. This innovative approach promises not only enhanced interpretability, but also a direct mechanism for solving (at least locally) the problem of prediction specific explanations.

Historically, other methodologies have aimed at achieving interpretability by adhering to specific desiderata during model construction. For instance, the seminal work by~\cite{alvarezmelis2018robust} focused on developing models that are both robust and interpretable by design, ensuring that the model's behavior remains consistent and faithful to the data it was trained on, thereby enhancing trustworthiness and reliability.

Furthermore, the concept of prototypical networks, as discussed by~\cite{rudin2019stop}, introduces a framework where predictions are based on the similarity of input features to prototype examples. This methodology not only simplifies the interpretability of predictions by anchoring them to recognizable instances but also facilitates a more intuitive understanding by comparing new inputs to known, labeled examples.

In summary, focusing on making models interpretable has the literature to propose promising methods. This approach is a serious candidate to make deep neural networks work in a manner that humans can understand. We will  dive deeper into these methods in the chapter dedicated to Alignment (\autoref{chap:alignment}).

\subsection{Explaining through Data}

Understanding model behavior extends to examining the influence of training data on the model's learning and predictions. Influence functions~\cite{cook1980characterizations} are a key tool in this domain, enabling the estimation of how the model's parameters or predictions would change if a particular data point were \textit{removed}\footnote{The actual formulation is expressing the difference in the parameter space for an infinitesimal perturbation.} from the training set.

\begin{definition}[Influence Function]
Given a learning algorithm $\mathcal{A} : \sx^n \times \Real^n \to \bm{\Theta}$, where $\parameters \in \bm{\Theta}$ represents the parameter space of a predictor $\f(\cdot; \parameters)$. A dataset $\s{D} = \{(\vx_1, y_1), \ldots, (\vx_n, y_n)\} \subseteq \mathcal{X}^n$, and a vector of weights $\v{w} = (w_1, \ldots, w_n)$ for the data points in $\mathcal{D}$, the influence function approximates the effect on the parameter vector $\parameters$ when the weight of the data point indexed by $i$ is infinitesimally perturbed by an amount $\xi$. This is mathematically expressed as:
\[
\mathcal{I}(i ; \mathcal{A}, \s{D}, \v{w}) \defas \lim_{\xi \to 0} \frac{1}{\xi} \left( \mathcal{A}(\s{D}, \v{w} + \xi \mathbf{e}_i) - \mathcal{A}(\s{D}, \v{w}) \right),
\]
where $\mathbf{e}_i$ is the canonical vector with respect to the $i$-th data point's weight.
\end{definition}

Influence functions trace their roots to robust statistics, offering a lens to examine the sensitivity of parameter estimates to changes in the underlying data distribution. This concept has been instrumental in identifying leverage points and outliers in data analysis, where the influence of such points on statistical estimations can lead to biased or misleading conclusions.

Recent works~\cite{koh2017understanding} have significantly extended and refined the application of influence functions, offering more possibility to shape better insights into the data's role in shaping model behavior. 


In this section, we examine the application of attribution methods to models trained on the FRSign dataset~\cite{2020frsign}, and use our recently introduced Sobol method. This dataset, containing images of French railway signals, serves as a practical case for assessing our attribution technique's effectiveness in making models more transparent. Detailed setup is documented in~\autoref{sec:intro:frsign}. Our analysis primarily features results from ResNet50, yet findings are applicable to VGG and ViT models.

For this application, our focus will be twofold: firstly, we will analyze fidelity scores to determine which attribution methods is more faithful; secondly, we aim to understand the model's strategies for the classes under study. We will observe that for most classes, the model appears to use plausible features. However, for one class, the attributions are somewhat mysterious. To have deeper understanding, we will employ feature visualization to formulate a diagnosis and hypotheses.

\subsection{Fidelity Scores}

We begin with a fidelity measure to identify which attribution methods best transcribe the model's behavior. \autoref{tab:frsign:fidelity} displays the results computed from 100 randomly selected images from the test dataset\footnote{It should be noted that the question of whether it is relevant to apply explainability to the training set remains open. Up to my knowledge, I see no a priori issues with it, but out of an abundance of caution and to ensure that nothing is overlooked, we will exclusively conduct explainability analyses on the test set.}.

\begin{table}[h]
\centering
\begin{tabular}{l c c}
\textbf{Attribution Method}&\textbf{Deletion Score}&\textbf{Insertion Score}\\
\hline
Sobol&\textbf{0.329}&0.377\\
RISE&0.348&\textbf{0.396}\\
Saliency&0.402&0.325\\
Integrated Gradient&0.396&0.348\\
Grad-CAM&0.419&0.372\\
SmoothGrad&\underline{0.338}&0.363\\
Occlusion&0.345&\underline{0.380}\\
\\
\hline
\end{tabular}
\caption{\textbf{Insertion and Deletion Scores for Seven Attribution Methods on the FRSign Dataset.} This table presents the scores for each attribution method according to the fidelity metrics of insertion and deletion. It's important to remember that a lower deletion score is preferable, and a higher insertion score is considered better. The best method is highlighted in \textbf{bold}, while the second best is \underline{underlined}. The methods that appear to be the most effective are RISE, Sobol, and SmoothGrad.}
\label{tab:frsign:fidelity}
\end{table}

In \autoref{tab:frsign:fidelity}, we observe that the method we previously introduced also achieves favorable deletion scores. This finding is reassuring as it suggests that Sobol's performance generalizes beyond the datasets studied earlier. Additionally, RISE and SmoothGrad both exhibit strong performance across both metrics. Therefore, for the remainder of our study, we will primarily focus on these three methods to draw our conclusions.

\begin{figure}[ht!]
\centering
\includegraphics[width=0.9\textwidth]{assets/frsign/good_attributions1.jpg}
\includegraphics[width=0.9\textwidth]{assets/frsign/good_attributions2.jpg}
\caption{\textbf{Comparative Visual Analysis of Attribution Methods.} We applied seven attribution methods to our model trained on the FRSign dataset. According to~\autoref{tab:frsign:fidelity}, the most faithful methods are Sobol, RISE, and SmoothGrad. Remarkably, the model tends to focus on the areas it should, specifically the traffic lights (or light) for the target class, which is reassuring.}
\label{fig:frsign:good_attributions}
\end{figure}

\subsection{Comparative Visual Analysis}

After computing the fidelity scores, we have a clearer understanding of which attribution methods more accurately reflect the model's decisions. This allows us to place greater trust in certain methods over others based on these initial tests. Sobol, RISE, and SmoothGrad emerge as the top three methods. However, we will continue to consider all methods to comprehensively assess our results. An interesting observation is that when all methods achieve good fidelity scores but highlight different areas of importance, this could be interpreted as indicating multiple ways to explain the model's reasoning. This is an intriguing aspect to explore~\footnote{Some preliminary remarks have been done on this topic in~\cite{aggregating2020}, but the ``diversity'' of explanation and the capability to aggregate them is still an interesting open questions.}. Nonetheless, it is important to remember that methods with lower fidelity scores should be approached with caution.

Figure~\autoref{fig:frsign:good_attributions} displays examples of attributions for each class that appear to be accurate. For critical signals such as violet, red, and yellow lights, the model seems to focus on the specific light or lights it is supposed to, which could increase our confidence in the model's decision-making for these types of signals.

These examples focus solely on instances where the model's predictions are correct. Next, we will apply our attribution methods to investigate failure cases, that is, instances where the model has made incorrect predictions.

\subsection{Explaining Failure Cases}

Despite the ResNet-50 model being our most performant, with an accuracy above 90\%, it is not without its share of incorrect predictions. Figure \autoref{fig:frsign:bad_attributions} presents several examples of explanations for misclassified points.

\begin{figure}[ht!]
\centering
\includegraphics[width=0.95\textwidth]{assets/frsign/bad_labels.jpg}
\caption{\textbf{Attribution Methods on wrongly classified points.} We applied the same seven attribution methods to points where the model's predictions were incorrect. ``P'' denotes the model's prediction, $\f(\vx)$, and ``GT'' for the label $y$. Upon review, the human eye tends to agree with the model's predictions, which might lead us to suspect incorrect labeling. However, for some images, the labeling is indeed accurate, and it is light aberrations or capture problem that obscure the correct ground truth from view.}
\label{fig:frsign:bad_attributions}
\end{figure}

Upon further analysis, it appears that a portion of the data points were indeed incorrectly labeled, while a significant number are correctly labeled, although human observation alone may not always accurately identify the correct label due to noise, errors, or anomalies in the image capture process. This leads to an intriguing question that extends beyond the scope of this thesis: whether the model or the label is at fault. In other words, if in reality a signal was violet but appears red in our images, should we expect the model to perceive it as humans do, with all associated biases, or should it interpret the data optimally for the task at hand, potentially employing mechanisms or perceptions different from those of humans? These considerations open up a broader discourse, yet there is one final observation to be made before concluding this section.

The analysis of failure cases does not encompass the entirety of our observations. There remains one particularly perplexing scenario, observed post-analysis: the case of the white signals.

\subsection{White Signal}

The interpretability of white signals poses a challenge, as the focal points of the model remain unclear. This is illustrated in Figure \autoref{fig:frsign:white_signals}.

\begin{figure}[ht!]
\centering
\includegraphics[width=0.95\textwidth]{assets/frsign/white_fire_attributions.jpg}
\caption{\textbf{Attribution Methods on White Signals.} The set of explainability methods applied to images correctly predicted as white signals is concerning. The model appears to focus on areas other than the traffic lights; however, it is unclear what specifically garners the model's attention.}
\label{fig:frsign:white_signals}
\end{figure}

The areas of attention for white signals appear cryptic and are not consistently focused on the lights. Furthermore, the focus does not always seem to be located in the same manner, which prevents a clear understanding of what the model is observing or relying upon for its decisions. We will now employ feature visualizations to delve deeper into this issue.

\paragraph{Feature Visualization.} To gain a better understanding of the potential strategies employed by our model, we utilized feature visualization. The results are shown in \autoref{fig:frsign:fviz}.

\begin{figure}[ht]
\centering
\includegraphics[width=0.95\textwidth]{assets/frsign/fviz_fire_1.jpg}
\includegraphics[width=0.95\textwidth]{assets/frsign/fviz_fire_2.jpg}
\caption{\textbf{Feature Visualization for the Logits of $\f$.} The images represent the results of two settings of feature visualization (in Fourier space) for the image that maximizes the logits for the classes yellow, red, green, white, and violet.}
\label{fig:frsign:fviz}
\end{figure}

For crucial traffic lights such as red, yellow, and violet, the feature visualizations seem to make sense, which is reassuring. However, for white, the interpretations remain somewhat cryptic. Nonetheless, we can hypothesize that the model focuses on the frame (contour of the light) as indicated in the top feature visualization for the white light in \autoref{fig:frsign:fviz}.

\subsection{Conclusion}

Attribution methods serve as a valuable tool for understanding the model and verifying that it relies on plausible features. They provide reassurance in most cases by ensuring that the areas most important to the model are also those containing information meaningful to humans.

However, two main issues arise. Firstly, we wish to extend our methods to offer stronger guarantees; that is, to establish confidence bounds around our explanations to ensure the model's reliance on these interpretations. Secondly, in some instances, the features the model focuses on remain ambiguous, such as with the case of white signals. This observation suggests that further research is necessary to make attribution methods both safer and more informative.


\subsection{Technical Contributions}

\paragraph{Variational Inequalities}
Our first major contribution is introducing the class of mirror extragradient algorithms, a generalization of Korpelevich's extragradient method \cite{korpelevich1976extragradient} for solving VIs. We establish best-iterate convergence of the class of mirror extragradient algorithms to a $\varepsilon$-strong solution of VIs that satisfy the Minty condition and are Bregman-continuous in $O(\nicefrac{1}{\varepsilon^2})$ evaluations of the optimality operator of the VI (\Cref{thm:mirror_extragradient_global_convergence}). Our result generalizes the results and proof techniques of \citet{huang2023beyond} for the extragradient method, and extends the convergence results of \citet{zhang2023mirror} for the unconstrained mirror extragradient method to constrained domains. In addition, to provide further justification for the convergence of the mirror extrat\^atonnement process in balanced economies, we establish suitable conditions for the local convergence of the mirror extragradient algorithm to an $\varepsilon$-strong solution of any Bregman-continuous VI that does \emph{not\/} satisfy the Minty condition---to the best of our knowledge, the first result of its kind (\Cref{thm:vi_mirror_extragrad_local}).



\paragraph{Walrasian Economies}

While a characterization of the set of Walrasian equilibria of any Walrasian economy as the solution set of an associated complementarity problem (i.e., a VI where the constraint set is the positive orthant) seems to have already been known \cite{dafermos1990exchange}, for balanced economies, we provide the first computationally tractable characterization of Walrasian equilibria as the set of strong solutions of a VI that satisfies the Minty condition and whose constraint set is given by the unit box. We then apply the mirror extragradient method to obtain a novel natural price-adjustment process we call the mirror \emph{extrat\^atonnement\/} process (\Cref{alg:mirror_extratatonnement}), and prove its convergence in all balanced economies that satisfy pathwise Bregman-continuity (\Cref{thm:bregman_mirror_exta_tatonn_convergence}).

We then restrict our attention to a novel class of competitive economies, namely those which are variationally stable on the unit simplex, and establish the polynomial-time convergence of the mirror \emph{extrat\^atonnement} process in all such economies assuming bounded elasticity of excess demand (\Cref{thm:mirror_extratatonn_var_stable}). Our convergence result also provides the first polynomial-time convergence result for price-adjustment processes in the class of economies that satisfy WARP, and generalizes the well-known \emph{t\^atonnement\/} convergence result in competitive economies with bounded elasticity of excess demand that satisfy WGS \cite{codenotti2005market}.

We then apply the mirror \emph{extrat\^atonnement} process to the Scarf economy, and prove its polynomial-time convergence to the unique Walrasian equilibrium of the economy (\Cref{thm:scarf_convergence}). As such, the mirror \emph{extrat\^atonnement\/} process is the first discrete-time \emph{natural} price adjustment process to converge in the Scarf economy.

Finally, we run a series of experiments on a variety of competitive economies where we verify that the pathwise Bregman-continuity assumption holds, and demonstrate that our algorithm converges to a Walrasian equilibrium at the rate predicted by our theory. Importantly, our experiments include examples of randomly initialized very large competitive economies ($\sim 500$ consumers and $\sim 500$ commodities) which are known to be PPAD-complete (e.g., Leontief economies), for which we show that our algorithm computes a Walrasian equilibrium fast without failure in all cases. 
% \amy{AND FAST!!! your results are non-asymptotic, n'est-ce pas?} \deni{Maybe add something on Scarf's challenge}
