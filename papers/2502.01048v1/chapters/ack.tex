\section*{Remerciements}

Au seuil de cette aventure qu'a été ma thèse, je me trouve face à la tâche délicate de traduire en mots simples, mais chargés de sens, toute l'étendue de ma gratitude envers tous ceux qui m'ont accompagné durant ces trois ans.

En premier lieu, mes pensées se dirigent vers le Professeur Thomas Serre. Sans son soutien constant, les pages de cette thèse seraient restées désespérément blanches. Les nombreuses réflexions que nous avons partagées durant ces trois années ont été essentielles à cette aventure et m'ont permis de vivre une expérience incroyablement enrichissante, plongé avec passion dans le monde fascinant de l'explicabilité. Pour tout cela, Thomas, je te remercie du fond du cœur.

Je tiens à remercier les Professeurs George A. Alvarez et Céline Hudelot d'avoir accepté d'évaluer ma thèse de doctorat. Je remercie également Ruth Fong, Robert Geirhos et Ruffin Van Rullen  pour avoir accepté de faire partie de mon comité de thèse.

Je tiens ensuite à rendre un hommage chaleureux à mes compagnons de route, Agustin, Louis et Thibaut. Avec vous, chaque jour de travail était une aventure ; nos escapades, du désert de Salta aux nuits polaires d'Ushuaïa, resteront à jamais gravées dans ma mémoire. Merci pour ces instants de pure fraternité, pour nos interrogations naïves, mais surtout pour votre générosité qui a été un soutien inestimable.

Je ne saurai passer sous silence la gratitude que je porte à Rémi. Ta bonté et ton calme sont un phare pour ceux qui ont le privilège de te connaître. Cette thèse a été l'occasion de croiser ton chemin. Tes conseils, ton temps généreusement offert, et nos discussions, tantôt profondes, tantôt très absurdes, m'ont accompagné tout au long de ce voyage.

Un merci tout particulier à Gregory, Laurent et Claire qui ont été les artisans discrets de cette quête, me soutenant à chaque pas et veillant sur ma liberté intellectuelle. Leur encouragement a été un don précieux, permettant à cette recherche de s'épanouir. Laurent, nos conversations nourriront ma réflexion pour longtemps.
David Vigouroux mérite une mention spéciale. Tu as été l'étincelle initiale de mon épanouissement à l'IRT, un mentor dont l'intelligence et le soutien, souvent en coulisses, ont été déterminants. Franck, ta gentillesse, ta sagesse et ton calme ont été pour moi une source d'inspiration constante.
À toute l'équipe DEEL, Ana, David B., Adil et Paul, nos échanges, nos rires et nos soirées resteront parmi mes meilleurs souvenirs. À Justin et Mikaël, pour avoir gardé nos serveurs à flot durant ces trois années, ce qui n'a pas été facile, mais sans qui rien n'aurait été possible.
Mon parcours m'a ensuite mené à Brown, où j'ai eu l'immense chance de rencontrer un autre mentor exceptionnel, Drew Linsley. Ta guidance, empreinte de bienveillance, a été un cadeau. Ivan, nos sessions de codage, d'apprentissage et nos discussions transatlantiques resteront gravées dans ma mémoire.
Je suis profondément reconnaissant envers Katherine pour son influence bienveillante et intelligente, qui m’a permis d’envisager la suite de cette thèse avec confiance. Merci pour ta guidance précieuse durant ces moments de passage.
Victor, ta contribution à la dernière étape de ma thèse a été source d'inspiration sous bien des aspects. Tu m'as donné une vision claire du chercheur que j'aspire à devenir, merci. Mélanie, ta profonde expertise n'a d'égal que ton humilité. Tes formations et tes conseils ont été très précieux durant cette aventure.
Enfin, un merci du fond du cœur à tous ceux qui ont jalonné ce voyage, enrichissant chaque étape de leur présence. Merci à Mathieu pour son intelligence et son humour, à Léo pour m'avoir fait découvrir la prédiction conforme, à Lucas et Antonin pour les bons moments passé à développer Xplique, à Sabine pour les moments à Brown, et à Julien pour toutes les discussions enrichissantes et ces soirées à discuter d'explicabilité. Enfin, je voudrais remercier chaleureusement toute l'équipe de l'IA à la SNCF, qui m'a si bien accueilli.
Mes amis, piliers du quotidien, ont été d'un soutien constant. Un immense merci à JL pour avoir toujours su me remonter le moral. Un merci tout particulier également à Théo, Anthony, Rayane, Hamza, Bruno, Roxane, Bastien, Martin, Lucas, Damien, Théo et tous les autres, qui ont partagé avec moi les hauts et les bas de cette quête. Alessandra, ton accompagnement durant cette année charnière a été précieux, un véritable trésor.

Pour conclure cette section de remerciements, je tiens à dédier mes ultimes mots à mon frère Arthur ainsi qu'à mes parents. Ils ont été le socle solide sur lequel j'ai bâti mes rêves et ma persévérance. Sans votre amour et votre foi constante, bien des chemins auraient été plus ardus. Par-delà l'apport académique de cette thèse se cache une ambition, plus vulgaire, mais que je sais partager avec beaucoup : celle de voir, même brièvement, vos yeux s'illuminer d'un éclair de fierté.
