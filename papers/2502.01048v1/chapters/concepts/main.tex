\chapter{From Pixels to Features: Towards Deeper Explainability with Concepts}
\chaptermark{\protect\parbox{.5\textwidth}{From Pixels to Features}}
\label{chap:concepts}

\begin{chapterabstract}
\textit{
In this chapter, we address a challenge identified in \autoref{chap:attributions}: Is it possible to transcend attributions methods to forge methods that do more than just spotlight where a model directs its attention -- \where~the model is looking -- but also clarify \what~exactly it perceives? Essentially, existing methods primarily disclose the ``\where'' in terms of the model's focus, rather than elucidating the "\what" it discerns, in terms of feature. The question then becomes, how can we define and characterize this ``\what''? This is the subject of this chapter that aims to extend beyond attribution methods to lay a more robust foundation for a deeper and more precise Explainability. \\
Our exploration begins in \autoref{sec:concepts:craft}, which propose a significant advancement in concept-based explainability by introducing an automated method, \craft, for extracting a model's learned concepts. We demonstrate that it is feasible to easily assess the significance of these derived concepts using Sobol indices presented in \autoref{sec:attributions:sobol}. The findings from this work substantially improve upon the benchmarks established in \autoref{sec:attributions:metapred}, and offer new avenues for addressing complex scenarios requiring in-depth explainability.
Progressing to \autoref{sec:concepts:holistic}, the cornerstone of this chapter, we show \tbi{i} how \craft~and related research fit within a broader framework of dictionary learning. We propose a unified framework for concept extraction, paving the way for new methodologies. Further, \tbi{ii} we establish a link between concept importance estimation and traditional attribution methods, demonstrating that concept importance estimation methods can be viewed as attribution methods recontextualized within the concept space for evaluative purposes.
With this framework in place, we find it possible to derive insightful answers to literature questions such as ``where should concept decomposition be performed?'' or ``which importance method to choose''. Furthermore, we delve into the importance measure of concepts, revealing that this information can be utilized to address a significant open problem in Explainability: ``how to identify points classified for similar reasons'', by proposing the strategic clustering plot.
The final section of this chapter, \autoref{sec:concepts:maco}, is dedicated to scaling feature visualization through a reformulation of the optimization problem within the Fourier space, by constraining magnitude. This new module allows for the use of feature visualization to create prototypes of the concepts extracted with \craft.
In conclusion, we will showcase the powerful synergies this new framework offers with \Lens, a demo that enables the visualization of the concepts used by a ResNet50 model for the 1000 ImageNet classes.
In sum, this chapter not only tackles foundational questions within the domain of machine learning explainability, but also sets forth a comprehensive framework that integrates advanced methodologies for concept extraction and importance estimation.
}
\end{chapterabstract}

The work in this chapter has led to the publication of the following conference papers:
{\small{
\begin{itemize}

    \item \textbf{Thomas Fel}\equal, Agustin Picard\equal, Louis Bethune\equal, Thibaut Boissin\equal, David Vigouroux, Julien Colin, Rémi Cadène, Thomas Serre, (2023). \textit{``CRAFT: Concept Recursive Activation FacTorization for Explainability''.} In: \textit{IEEE/CVF Conference on Computer Vision and Pattern Recognition} (\textcolor{confcolor}{CVPR})
    
    \item \textbf{Thomas Fel}\equal, Victor Boutin\equal, Mazda Moayeri, Rémi Cadène, Louis Bethune, Mathieu Chalvidal, Thomas Serre (2023). \textit{``A Holistic Approach to Unifying Automatic Concept Extraction and Concept Importance Estimation''.} In: \textit{Advances in Neural Information Processing Systems}  (\textcolor{confcolor}{NeurIPS})
    
    \item \textbf{Thomas Fel}\equal, Thibaut Boissin\equal, Victor Boutin\equal, Agustin Picard\equal, Paul Novello\equal, Julien Colin, Drew Linsley, Tom Rousseau, Rémi Cadène, Lore Goetschalckx, Thomas Serre (2024). \textit{``Unlocking feature visualization for deep network with MAgnitude constrained optimization''.} In: \textit{Advances in Neural Information Processing Systems}  (\textcolor{confcolor}{NeurIPS})

\end{itemize}
}}

\minitoc
\clearpage

\section{Introduction}
\section{Introduction}
\label{sec:intro}

\begin{figure*}[tb]
    \centering
    \includegraphics[width=0.848\linewidth]{figs/circuitnn.pdf} 
    \caption{Illustration of differentiable CircuitNN. CircuitNN is designed based on differentiable NAND gates. After DAS is guided by PI and PO pairs of the truth table, CircuitNN can get the precise circuit architecture logic equivalent to the truth table.}
    \label{fig:circuitnn}
\end{figure*}

% 1. Describe the importance of logic synthesis
% 2. Existing Problems
% (a) Neural Architecture Search: Unstable, Predefined Setting, etc.
% (b) Circuit Generation: Probabilistic Model, Logic Equivalence

With the rapid advancement of technology, the scale of integrated circuits (ICs) has expanded exponentially. 
This expansion has introduced significant challenges in chip manufacturing, particularly concerning power and area metrics.
A primary objective in IC design is achieving the same circuit function with fewer transistors, thereby reducing power usage and area occupancy.

Logic synthesis~\cite{hachtel2005logicsynth}, a critical step in electronic design automation (EDA), transforms behavioral-level circuit designs into optimized gate-level circuits, ultimately yielding the final IC layout. 
The primary goal of logic synthesis is to identify the physical implementation with the fewest gates for a given circuit function. 
This task constitutes a challenging NP-hard combinatorial optimization problem. 
Current logic synthesis tools~\cite{brayton2010abc, wolf2013yosys} rely on human-designed heuristics, often leading to sub-optimal outcomes.

Differentiable architecture search (DAS) techniques~\cite{liu2018darts, chu2020darts} offer novel perspectives on addressing challenges in this problem.
Circuit functions can be represented through truth tables, which map binary inputs to their corresponding outputs. 
Truth tables provide a precise representation of input-output relationships, ensuring the design of functionally equivalent circuits.
Inspired by this, researchers~\cite{deepmind2024ai4sys, wang2024tnet} have begun exploring the application of DAS to synthesize circuits directly from truth tables.
Specifically, \citet{deepmind2024ai4sys} proposed CircuitNN, a framework that learns differentiable connection structures with logic gates, enabling the automatic generation of logic circuits from truth tables.
This approach significantly reduces the complexity of traditional circuit generation. 
Building on this, \citet{wang2024tnet} introduced T-Net, a triangle-shaped variant of CircuitNN, incorporating regularization techniques to enhance the efficiency of DAS.

Despite these advancements, several challenges remain. 
The computational complexity of DAS grows quadratically with the number of gates, posing scalability issues.
Although triangle-shaped architecture~\cite{wang2024tnet} partially mitigates this problem, redundancy persists. 
%Additionally, DAS is susceptible to converging to local optima, limiting the ability to search architectures that satisfy the given truth tables~\cite{liu2018darts}. 
%Furthermore, hyperparameters (network depth and layer width) require extensive searches, introducing complexity and prolonging the synthesis process. 
Additionally, DAS is susceptible to converging to local optima~\cite{liu2018darts} and hyperparameters (network depth and layer width) require extensive searches. 
The challenges arise from the vast search space in DAS. 
% Even with predefined settings for CircuitNN, finding a configuration that meets the truth table requires extensive trial and error during the DAS process. 
Intuitively, limiting the search space through predefined parameters (network depth, gates per layer, and connection probabilities) can significantly reduce the complexity.

Recent advances~\cite{openai2023gpt4, abramson2024alphafold3, esser2024sd3, li2024mar} in conditional generative models have demonstrated remarkable performance across language, vision, and graph generation tasks. 
Motivated by these developments, we propose a novel approach to circuit generation that generates preliminary circuit structures to guide DAS in generating refined circuits matching specified truth tables. 
Firstly, we introduce CircuitVQ, a tokenizer with a discrete codebook for circuit tokenization. 
Built upon our Circuit AutoEncoder framework~\cite{hou2022graphmae,li2023maskgae,wu2025mgvga}, CircuitVQ is trained through a circuit reconstruction task. 
Specifically, the CircuitVQ encoder encodes input circuits into discrete tokens using a learnable codebook, while the decoder reconstructs the circuit adjacency matrix based on these tokens.
Subsequently, the CircuitVQ encoder serves as a circuit tokenizer for CircuitAR pretraining, which employs a masked autoregressive modeling paradigm~\cite{chang2022maskgit, li2023mage}. 
In this process, the discrete codes function as supervision signals. 
After training, CircuitAR can generate discrete tokens progressively, which can be decoded into initial circuit structures by the decoder of the CircuitVQ. 
These prior insights can guide DAS in producing refined circuits that match the target truth tables precisely.

Our key contributions can be summarized as follows:
\begin{itemize}
\item We introduce CircuitVQ, a circuit tokenizer that facilitates graph autoregressive modeling for circuit generation, based on our Circuit AutoEncoder framework;
\item Develop CircuitAR, a model trained using masked autoregressive modeling, which generates initial circuit structures conditioned on given truth tables;
\item Propose a refinement framework that integrates differentiable architecture search to produce functionally equivalent circuits guided by target truth tables;
\item Comprehensive experiments demonstrating the scalability and capability emergence of our CircuitAR and the superior performance of the proposed circuit generation approach.
\end{itemize}

% Motivation
% (a) Diffusion (Vision, Graph), Autoregressive (Language, Vision)
% (b) Circuit Generation for Predefined Setting
% (c) Neural Architecture Search for Strict Logic Equivalence

% Contribution
% (a) Circuit Tokenizer (new transformer arch, training strategy)
% (b) CircuitAR (train and gen strategies, post-ar strategy)
% (c) Extensive Evaluation including BitD (Bit Distance) for Scalability

\clearpage

\section{CRAFT : Concept Recursive Activation FacTorization}
\label{sec:concepts:craft}
\subsection{Limitations}
\label{apx:craft:limitations}

Although we believe concept-based XAI to be a promising research direction, it isn't without pitfalls. It is capable of producing explanations that are ideally easy to understand by humans, but to what extent is a question that remains unanswered. The fact that there is no way to mathematically measure this prevents researchers from easily comparing the different techniques in the literature other than through time consuming and expensive experiments with human subjects. We think that developing a metric should be one of the field's priorities.%

With \craft, we address the question of \what~by showing a cluster of the images that better represent each concept. However, we recognize that it's not perfect: in some cases, concepts are difficult to clearly define -- put a label on what it represents --, and might induce some confirmation and selection bias. Feature visualization~\cite{olah2017feature} might help in better illustrating the specific concept (as done in appendix \ref{app:craft:feature-viz-val}), but we believe there's still space for improvement. For instance, an interesting idea could be to leverage image captioning methods to describe the clusters of image crops, as textual information could help humans in better understanding clusters.

Although we believe \craft~to be a considerable step in the good direction for the field of concept-based XAI, it also have some pitfalls. Namely, we chose the NMF as the activation factorization, which, while drastically improving the quality of extracted concepts, also comes with it's own caveats. For instance, it is known to be NP-hard to compute exactly, and in order to make it scalable, we had to use a tractable approximation by alternating the optimization of $\m{U}$ and $\m{W}$ through ADMM~\cite{boyd2011distributed}. This approach might indeed yield non-unique solutions. Our experiments (section \ref{sec:craft:expSobol}), have shown a low variance on between the runs, which comforts us about the stability of our results.%
However the absence of formal guarantee for uniqueness must be kept in mind: this subject is still an active topic of research and improvement could be expected in the near future. Namely, sparsity constraints and regularization seem to be promising paths.
Naturally, we also need enough samples of the class under study to be available for the factorization to construct a relevant concept bank, which might affect the quality of the explanations on frugal applications where data is very scarce. %

\subsection{Additional results}
\label{apx:craft:more-craft}

\subsubsection{Qualitative comparison with ACE}\label{apx:qualitative}

Figure~\ref{fig:app:craft:qualitative} compares the examples of concepts found by CRAFT against those found by ACE~\cite{ghorbani2019towards} for 3 classes of Imagenette. 
For each class the concepts are ordered by importance (the highest being the most important). 
ACE uses a clustering technique and TCAV to estimate importance, while CRAFT uses the method introduced in \ref{sec:craft:method} and Sobol to estimate importance. These examples illustrate one of the weaknesses of ACE: the segmentation used can introduce biases through the baseline value used~\cite{sturmfels2020visualizing,fong2017meaningful}. The concepts found by CRAFT seem distinct: (vault, cross, stained glass) for the Church class, (dumpster, truck door, two-wheeler) for the garbage truck, and (eyes, nose, fluffy ears) for the English Springer.

\begin{figure*}[ht]
  \includegraphics[width=0.99\textwidth]{assets/craft/craft_vs_ace.jpg}
  \caption{ \textbf{Qualitative comparison.} We compare concepts found by our method (top) to those extracted with ACE~\cite{ghorbani2019towards} (bottom) for the classes \textit{Church}, \textit{Garbage truck} and \textit{English springer} from ILSVRC2012~\cite{imagenet_cvpr09}. %
  }
  \label{fig:app:craft:qualitative}
\end{figure*}

\subsubsection{Most important concepts.} We show more example of the 4 most importants concepts for 6 classes: `Chain saw', `English springer', `Gas pump', `Golf ball', `French horn' and `Garbage Truck' (Figure~\ref{fig:craft:segments_all}).%

\begin{figure*}[ht]
    \centering
      \includegraphics[width=\textwidth]{assets/craft/segments/all.jpg}
      \caption{ \textbf{CRAFT most important concepts}. The 4 most important concepts ranked by importance (left to right) for the following classes: `English springer', `Chain saw',  `Gas pump', `Golf ball', `French horn',  and `Garbage truck'.
      }
      \label{fig:craft:segments_all}
\end{figure*}
    



\clearpage

\subsubsection{Feature Visualization validation} \label{app:craft:feature-viz-val}

Another way of interpreting concepts -- as per~\cite{kim2018interpretability} -- is to employ feature visualization methods: through optimization, find an image that maximizes an activation pattern.
In our case, we used the set of regularization and constraints proposed by \cite{olah2017feature}, which allow us to successfully obtain realistic images. In Figures~[\ref{fig:craft:feature_viz_chainsaw}-\ref{fig:craft:feature_viz_golf}], we showcase these synthetic images obtained through feature visualization, along with the segments that maximize the target concept. We observe that they do reflect the underlying concepts of interest.

Concretely, to produce those feature visualization, we are looking for an image $\vx^*$ that is optimized to correspond to a concept from the concept bank $\m{W}_i$. We use the so called `dot-cossim' loss proposed by ~\cite{olah2017feature}, which give the following objective:

\begin{equation*}
    \vx^* = \argmax_{\vx \in \mathcal{X}} ~ \langle \bm{g}(\vx), \m{W}_i \rangle 
\frac{\langle \bm{g}(\vx), \m{W}_i \rangle^2}{||\bm{g}(\vx)|| ~ ||\m{W}_i||  } - \mathcal{R}(\vx)    
\end{equation*}

With $\mathcal{R}(\cdot)$, the regularizations applied to $\vx$  -- the default regularizations in the \textbf{Xplique} library~\cite{xplique}. As for the specific parameters, we used Fourier preconditioning on the image with a decay rate of $0.8$ and an Adam optimizer ($lr = 1e-1$). 

\begin{figure}[ht]
\centering
  \includegraphics[width=0.45\textwidth]{assets/craft/feature_viz/chainsaw.jpg}
  \caption{ \textbf{Feature visualization for chainsaw CRAFT concepts.}
  }
  \label{fig:craft:feature_viz_chainsaw}
\end{figure}

\begin{figure}[ht]
\centering
  \includegraphics[width=0.45\textwidth]{assets/craft/feature_viz/english.jpg}
  \caption{ \textbf{Feature visualization for english springer CRAFT concepts.}
  }
  \label{fig:craft:feature_viz_englishspringer}
\end{figure}

\begin{figure}[ht]
\centering
  \includegraphics[width=0.45\textwidth]{assets/craft/feature_viz/golf.jpg}
  \caption{ \textbf{Feature visualization for golf CRAFT concepts.} 
  }
  \label{fig:craft:feature_viz_golf}
\end{figure}

\clearpage
\newpage






\subsection{Backpropagating through the NMF block}

\subsubsection{Alternating Direction Method of Multipliers (ADMM) for NMF}

We recall that NMF decomposes the positive features vector $\m{A} \in \mathbb{R}^{n \times p}$ of $n$ examples lying in dimension $p$, into a product of positive low rank matrices $\m{U}(\m{A})\in\mathbb{R}^{n\times r}$ and $\m{W}(\m{A})\in\mathbb{R}^{p\times r}$ (with $r<<\min(n,p)$), i.e the solution to the problem:
\begin{align}\label{apeq:craft:nmf}
\min_{\m{U}\geq 0,\m{W}\geq 0} & \frac{1}{2}\|\m{A}-\m{U}\m{W}^T\|^2_F. %
\end{align}

For simplicity we used a non-regularized version of the NMF objective, following Algorithms 1 and 3 in paper~\cite{huang2016flexible}, based on ADMM~\cite{boyd2011distributed}. This algorithm transforms the non-linear equality constraints into indicator functions $\bm{\delta}$. Auxiliary variables $\tilde{\m{U}},\tilde{\m{W}}$ are also introduced to separate the optimization of the objective on the one side, and the satisfaction of the constraint on $\m{U}, \m{W}$ on the other side. The equality constraints $\tilde{\m{U}}=\m{U},\tilde{\m{W}}=\m{W}$ are linear and easily handled by the ADMM framework through the associated dual variables $\bar{\m{U}},\bar{\m{W}}$. In our case, the problem in Equation~\ref{apeq:craft:nmf} is transformed into:
  
\begin{equation}
\begin{aligned}\label{apeq:craft:admm}
\min_{\m{U},\tilde{\m{U}}, \m{W},\tilde{\m{W}}} & \frac{1}{2}\|\m{A}-\tilde{\m{U}} \tilde{\m{W}}^T\|^2_F+\bm{\delta}(\m{U})+\bm{\delta}(\m{W}), 
\\%\quad&\textcolor{red}{\text{non convex, NP hard}}\\
~ ~ s.t. ~ ~ &\tilde{\m{U}}=\m{U}, \tilde{\m{W}}=\m{W} \\
     \text{with} ~ & \bm{\delta}(\bm{H})=\begin{cases}
                            0 \text{ if } \bm{H} \geq 0,\\
                            +\infty \text{ otherwise.}
                            \end{cases}
\end{aligned}
\end{equation}

Note that $\tilde{\m{U}}$ and $\m{U}$ (resp. $\tilde{\m{W}}$ and $\m{W}$) seem redundant: they are meant to be equal thanks to constraints $\tilde{\m{U}}=\m{U}, \tilde{\m{W}}=\m{W}$. This is standard practice within ADMM framework: introducing redundancies allows to disentangle the (unconstrained) optimization of the objective on one side (with $\tilde{\m{U}}$ and $\tilde{\m{W}}$) and constraint satisfaction on the other side with $\m{U}$ and $\m{W}$. During the optimization process the variables $\tilde{\m{U}},\m{U}$ (resp. $\tilde{\m{W}},\m{W}$) are different, and only become equal in the limit at convergence. The dual variables $\bar{\m{U}},\bar{\m{W}}$ control the balance between optimization of the objective $\frac{1}{2}\|\m{A}-\tilde{\m{U}} \tilde{\m{W}}^T\|^2_F$ and constraint satisfaction $\tilde{\m{U}}=\m{U}, \tilde{\m{W}}=\m{W}$. The constraints are simplified at the cost of a non-smooth (and even a non-finite) objective function $\frac{1}{2}\|\m{A} -\bar{\m{U}} \bar{\m{W}}^T\|^2_F+\bm{\delta}(\m{U})+\bm{\delta}(\m{W})$ due to the term $\bm{\delta}(\m{U})+\bm{\delta}(\m{W})$. ADMM proceeds to create a so-called \textit{augmented Lagrangian} with $l_2$ regularization $\rho>0$:
\begin{equation}
    \begin{aligned}
    \Lagrangian&(\m{A},\m{U},\m{W},\tilde{\m{U}},\tilde{\m{W}},\bar{\m{U}},\bar{\m{W}})=\\
    &\frac{1}{2}\|\m{A}-\tilde{\m{U}}\tilde{\m{W}}^T\|^2_F+\bm{\delta}(\m{U})+\bm{\delta}(\m{W})\\
    &+\bar{\m{U}}^T(\tilde{\m{U}}-\m{U})+\bar{\m{W}}^T(\tilde{\m{W}}-\m{W})\\
    &+\frac{\rho}{2}\left(\|\tilde{\m{U}}-\m{U}\|_2^2+\|\tilde{\m{W}}-\m{W}\|_2^2\right).
    \end{aligned}
\end{equation}

This regularization ensures that the dual problem is well posed and that it remain convex, even with the non smooth and infinite terms $\bm{\delta}(\m{U})+\bm{\delta}(\m{W})$. Once again, this is standard practice within ADMM framework. The (regularized) problem associated to this Lagrangian is decomposed into a sequence of convex problems that alternate minimization over the $\m{U},\tilde{\m{U}},\bar{\m{U}}$ and the $\m{W},\tilde{\m{W}},\bar{\m{W}}$ triplets.
  
\begin{align}\label{apeq:craft:pairnnls}
\m{U}_{t+1}&=\argmin_{\m{U}=\tilde{\m{U}}} \frac{1}{2}\|\m{A}-\tilde{\m{U}}\m{W}_t^T\|^2_F+\bm{\delta}(\m{U})+\frac{\rho}{2}\|\tilde{\m{U}}-\m{U}\|_2^2. %
\\
\m{W}_{t+1}&=\argmin_{\m{W}=\tilde{\m{W}}} \frac{1}{2}\|\m{A}-\m{U}_t\tilde{\m{W}}^T\|^2_F+\bm{\delta}(\m{W})+\frac{\rho}{2}\|\tilde{\m{W}}-\m{W}\|_2^2.%
\end{align}

This guarantees a monotonic decrease of the objective function $\|\m{A}-\tilde{\m{U}}_t\tilde{\m{W}}_t^T\|_F^2$. Each of these sub-problems is thus solved with ADMM separately, by alternating minimization steps of $\frac{1}{2}\|\m{A}-\tilde{\m{U}}\m{W}_t^T\|^2_F+\bar{\m{U}}^T(\tilde{\m{U}}-\m{U})+\frac{\rho}{2}\|\m{U}-\tilde{\m{U}}\|_2^2$ over $\tilde{\m{U}}$ (\textbf{\textit{i}}), with minimization steps of $\bm{\delta}(\m{U})+\frac{\rho}{2}\|\m{U}-\tilde{\m{U}}\|_2^2$ over $\m{U}$ (\textbf{\textit{ii}}), and gradient ascent steps (\textbf{\textit{iii}}) on the dual variable $\bar{\m{U}}\gets \bar{\m{U}}+(\tilde{\m{U}}-\m{U})$. A similar scheme is used for $\m{W}$ updates. Step (\textbf{\textit{i}}) is a simple convex quadratic program with equality constraints, whose KKT~\cite{karush1939minima,kuhn1951nonlinear} conditions yield a linear system with a Positive Semi-Definite (PSD) matrix. Step (\textbf{\textit{ii}}) is a simple projection of $\tilde{\m{U}}$ onto the convex set $\bm{\delta}^{-1}(\bm{0})$. Finally, step (\textbf{\textit{iii}}) is inexpensive.

Concretely, we solved the quadratic program using Conjugate Gradient, from \textit{jax.scipy.sparse.linalg.cg}. This indirect method only involves \textit{matrix-vector} products and can be more GPU-efficient than methods that are based on matrix factorization (such as Cholesky decomposition). Also, we re-implemented the pseudo code of~\cite{huang2016flexible} in \textit{Jax} for a fully GPU-compatible program. We used the primal variables $\m{U}_0,\m{W}_0$ returned by \textit{sklearn.decompose.nmf} as a \textit{warm start} for ADMM and observe that the high quality initialization of these primal variables considerably speeds up the convergence of the dual variables.

\subsubsection{Implicit differentiation}\label{app:craft:implicit}

The Lagrangian of the NMF problem reads $\mathcal{L}(\m{U},\m{W},\bar{\m{U}},\bar{\m{W}})=\frac{1}{2}\|\m{A}-\m{U}\m{W}^T\|_F^2-\bar{\m{U}}^T\m{U}-\bar{\m{W}}^T\m{W}$, with dual variables $\bar{\m{U}}$ and $\bar{\m{W}}$ associated to the constraints $\m{U}\geq 0, \m{W} \geq 0$. It yields a function $\bm{F}$ based on the KKT conditions~\cite{karush1939minima,kuhn1951nonlinear} whose optimal tuple $\m{U},\m{W},\bar{\m{U}},\bar{\m{W}}$ is a root.  
  
For single NNLS problem (for example, with optimization over $\m{U}$) the KKT conditions are:

\begin{equation} %
    \begin{cases}
    \nabla_{\m{U}}\left(\frac{1}{2}\|\m{A}-\tilde{\m{U}} \tilde{\m{W}}^T\|^2_F+\bar{\m{U}}^T(-\m{U})\right)    =0, \text{ stationarity,}\\
    -\m{U}\leq 0, \text{ primal feasability,}\\
    \bar{\m{U}} \odot \m{U}   =0, \text{ complementary slackness,}\\
    \bar{\m{U}}   \geq 0, \text{ dual feasability.}\\
\end{cases}
\label{apeq:craft:optimality_fun}
\end{equation}

By stacking the KKT conditions of the NNLS problems the we obtain the so-called \textit{optimality function} $\bm{F}$:

\begin{equation} %
    \bm{F}((\m{U},\m{W},\bar{\m{U}},\bar{\m{W}}),\m{A})=\begin{cases}
    (\m{U}\m{W}^T-\m{A})\m{W}-\bar{\m{U}}    ,& \\ %
    (\m{W}\m{U}^T-\m{A}^T)\m{U}-\bar{\m{W}}  ,& \\ %
    \bar{\m{U}} \odot \m{U}   ,& \\ %
    \bar{\m{W}} \odot \m{W}   .& \\ %
\end{cases}
\label{eq:craft:optimality_fun_2}
\end{equation}

The implicit function theorem~\cite{griewank2008evaluating} allows us to use implicit differentiation~\cite{krantz2002implicit,griewank2008evaluating,bell2008algorithmic} to efficiently compute the Jacobians $\frac{\partial \m{U}}{\partial \m{A}}$ and $\frac{\partial \m{W}}{\partial \m{A}}$ without requiring to back-propagate through each of the iterations of the NMF solver:
\begin{equation}
    \frac{\partial (\m{U},\m{W},\bar{\m{U}},\bar{\m{W}})}{\partial \m{A}}=-(\partial_1 \bm{F})^{-1}\partial_2 \bm{F}.
\end{equation}

Implicit differentiation requires access to the dual variables of the optimization problem in equation~\ref{eq:craft:nmf}, which are not computed by Scikit-learn's popular implementation. Scikit-learn uses Block coordinate descent algorithm~\cite{cichocki2009fast,fevotte2011algorithms}, with a randomized SVD initialization. Consequently, we leverage our implementation in Jax based on ADMM~\cite{boyd2011distributed}.

Concretely, we perform a two-stage backpropagation \textit{Jax (2)}$\to$\textit{Tensorflow (1)} to leverage the advantage of each framework. The lower stage (1) corresponds to feature extraction $\m{A}=\v{h}_l(\vx)$ from crops of images $\vx$, and upper stage (2) computes NMF $\m{A} \approx \m{U}\m{W}^T$.  
  
We use the \textit{Jaxopt}~\cite{blondel2021implicitdiff} library that allows efficient computation of $\frac{\partial (\m{U},\m{W},\bar{\m{U}},\bar{\m{W}})}{\partial \m{A}}=-(\partial_1 \bm{F})^{-1}\partial_2 \bm{F}$. The matrix $(\partial_1 \bm{F})^{-1}$ is never explicitly computed -- that would be too costly. Instead, the system $\partial_1 \bm{F}\frac{\partial (\m{U},\m{W},\bar{\m{U}},\bar{\m{W}})}{\partial \m{A}}=-\partial_2 \bm{F}$ is solved with Conjugate Gradient through the use of Jacobian Vector Products (JVP) $\bm{v}\mapsto (\partial_1 \bm{F})\bm{v}$.  
  
The chain rule yields:
$$\frac{\partial \m{U}}{\partial \vx}=\frac{\partial \m{A}}{\partial \vx}\frac{\partial \m{U}}{\partial \m{A}}.$$

Usually, most Autodiff frameworks (e.g Tensorflow, Pytorch, Jax) handle it automatically. Unfortunately, combining two of those framework raises a new difficulty since they are not compatible. Hence, we re-implement manually the two stages auto-differentiation.  
  
Since $r$ is far smaller ($r=25$ in all our experiments) than input dimension $\vx$ (typically $224\times 244$ for ImageNet images), back-propagation is the preferred algorithm in this setting over forward-propagation. We start by computing sequentially the gradients $\nabla_{\vx} \m{U}_i$ for all concepts $1\leq i\leq r$. This amounts to compute $\bm{v}=\nabla_{\m{A}} \m{U}_i$ with Implicit Differentiation in Jax, convert the Jax array $\bm{v}$ into Tensorflow tensor, and then to compute $\nabla_{\vx} \m{U}_i=\frac{\partial \m{A}}{\partial \vx}\nabla_{\m{A}} \m{U}_i=\nabla_{\vx} (\v{h}_l(\vx) \cdot \bm{v})$. The latter is easily done in Tensorflow. Finally we stack the gradients $\nabla_{\vx} \m{U}_i$ to obtain the Jacobian $\frac{\partial \m{U}}{\partial \vx}$.



\subsection{Sobol indices for concepts} \label{apdx:sobol}

We propose to formally derive the Sobol indices for the estimation of the importance of concepts.
Let us define a probability space  $(\Omega, \mathcal{F}, \mathbb{P})$ of possible concept perturbations. In order to build these concept perturbations, we start from an original vector of concepts coefficient\footnote{We choose to name $\vx$ the concept coefficient vector here, instead to avoid any confusion with $\v{u}$ that will be the set of indices.} $\v{x} \in \mathbb{R}^r$ and use i.i.d. stochastic masks $\rv{m} = (\r{m}_1, ..., \r{m}_r) \sim \mathcal{U}([0, 1]^r)$, as well as a perturbation operator $\bm{\tau}(\cdot)$ to create stochastic perturbation of $\vx$ that we call concept perturbation $\rvx = \bm{\tau}(\vx, \rv{m})$. 


Concretely, to create our concept perturbation we consider the inpainting function as our perturbation operator (as in \cite{ribeiro2016lime, petsiuk2018rise, fel2021sobol}) : $\bm{\tau}(\vx, \rv{m}) = \vx \odot \rv{m} + (\bm{1} - \rv{m}) \mu$ with $\odot$ the Hadamard product and $\mu \in \mathbb{R}$ a baseline value, here zero.
For the sake of notation, we will note $\pred$ the function mapping a random concept perturbation $\rvx$ from an intermediat layer to the output $\pred(\rvx)$ (e.g., the final layer if we do the concept extraction on the penultimate layer).
We denote the set $\mathcal{U} = \{1, ..., r\}$, $\bm{u}$ a subset of $\mathcal{U}$, its complementary $\sim \bm{u}$ and $\mathbb{E}(\cdot)$ the expectation over the perturbation space.
Finally, we assume that $\pred \in \mathbb{L}^2(\mathcal{F}, \mathbb{P})$ i.e. $|\mathbb{E}(\pred(\rvx))| < + \infty$.

The Hoeffding decomposition allows us to express the function $\pred$ into summands of increasing dimension, denoting $\pred_{\bm{u}}$ the partial contribution of the concepts $\rvx_{\bm{u}} = (\r{x}_i)_{i\in \bm{u}}$ to the score $\pred(\rvx)$:
\begin{equation}
    \label{eq:craft:anova}
    \begin{aligned}
    \bm{f}(\rvx) &= \bm{f}_{\emptyset}
     + \sum_i^r \bm{f}_i(\r{x}_i)
     + \sum_{1 \leqslant i < j \leqslant r} \bm{f}_{i,j}(\r{x}_i, \r{x}_j)
     + \cdots 
     + \bm{f}_{1,...,r}(\rvx) \\
    &= \sum_{\substack{\bm{u} \subseteq \mathcal{U}}} \bm{f}_{\bm{u}}(\rvx_{\bm{u}}).
    \end{aligned}
\end{equation}

Eq.~\ref{eq:craft:anova} consists of $2^r$ terms and is unique under the following orthogonality constraint:
\begin{equation}
    \label{eq:craft:anova_ortho}
    \begin{aligned}
    \forall (\bm{u},\bm{v}) \subseteq \mathcal{U}^2 \; s.t. \;  \bm{u} \neq \bm{v}, \;\; \mathbb{E}\big(\bm{f}_{\bm{u}}(\rvx_{\bm{u}}) \bm{f}_{\bm{v}}(\rvx_{\bm{v}})\big) = 0.
    \end{aligned}
\end{equation}

Furthermore, orthogonality yields the characterization $\bm{f}_{\bm{u}}(\rvx_{\bm{u}}) = \mathbb{E}(\bm{f}(\rvx)|\rvx_{\bm{u}}) - \sum_{\bm{v}\subset \bm{u}}\bm{f}_{\bm{v}}(\rvx_{\bm{v}})$ and allows us to decompose the model variance as:
\begin{equation}
    \label{eq:craft:var_decomposition}
    \begin{aligned}
        \V(\pred(\rvx)) &= \sum_i^r \V(\bm{f}_i(\r{x}_i)) 
        +\sum_{1 \leqslant i < j \leqslant r} \V(\bm{f}_{i,j}(\r{x}_i, \r{x}_j))
        + ... + \V(\bm{f}_{1,...,r}(\rvx)) \\
        &=\sum_{\substack{\bm{u} \subseteq \mathcal{U}}} \V(\bm{f}_{\bm{u}}(\rvx_{\bm{u}})).
        \end{aligned}
\end{equation}

Building from Eq.~\ref{eq:craft:var_decomposition}, it is natural to characterize the influence of any subset of concepts $\bm{u}$ as its own variance w.r.t. the total variance. This yields, after normalization by $\V(\bm{f}(\rvx))$, the general definition of Sobol' indices.
\begin{definition}[Sobol indices~\cite{sobol1993sensitivity}]
The sensitivity index $\mathcal{S}_{\bm{u}}$ which measures the contribution of the concept set $\rvx_{\bm{u}}$ to the model response $\bm{f}(\rvx)$ in terms of fluctuation is given by:
\begin{equation}\label{eq:craft:sobol_indice}
\begin{aligned}
    \mathcal{S}_{\bm{u}}  &= \frac{ \V(\bm{f}_{\bm{u}}(\rvx_{\bm{u}})) }{ \V(\pred(\rvx)) }\\
    &= \frac{ \V(\mathbb{E}(\bm{f}(\rvx) | \rvx_{\bm{u}})) - \sum_{\bm{v}\subset \bm{u}}\V(\mathbb{E}(\bm{f}(\rvx) | \rvx_{\bm{v}} ))}{ \V(\bm{f}(\rvx)) }.
\end{aligned}
\end{equation}
\end{definition}

Sobol indices give a quantification of the importance of any subset of concepts with respect to the model decision, in the form of a normalized measure of the model output deviation from $\bm{f}(\rvx)$. Thus, Sobol indices sum to one : $\sum_{\bm{u} \subseteq \mathcal{U}} \mathcal{S}_{\bm{u}} = 1$. 

\vspace{2mm}
Furthermore, the framework of Sobol' indices enables us to easily capture higher-order interactions between features. Thus, we can view the Total Sobol indices defined in \ref{eq:craft:total_sobol} as the sum of of all the Sobol indices containing the concept $i$ : $\mathcal{S}^{T}_i = \sum_{\bm{u} \subseteq \mathcal{U}, i \in \bm{u}} \mathcal{S}_{\bm{u}}$. Concretely, we estimate the total Sobol indices using the Jansen estimator~\cite{janon2014asymptotic} and Quasi-Monte carlo Sequence (Sobol $LP_{\tau}$ sequence).

\clearpage

\subsection{Human experiments}\label{app:craft:human-exp}

We first describe how participants were enrolled in our studies, then the general experimental design they went through.


\subsubsection{Utility evaluation}
\label{app:craft:utility}

\paragraph{Participants}
The participants that went through our experiments are users from the online platform Amazon Mechanical Turk (AMT), specifically, we recruit users with high qualifications (number of HIT completed $=5 000$ and HIT accepted $> 98 \%$). All participants provided informed consent electronically in order to perform the experiment ($\sim 5 - 8$ min), for which they received 1.4\$.\\

For the \textit{Husky vs. Wolf} scenario, $n=84$ participants passed all our screening and filtering process, respectively $n=32$ for CRAFT, $n=22$ for ACE and $n=22$ for CRAFTCO.


For the \textit{Leaves} scenario, after filtering, we analyzed data from $n=87$ participants, respectively $n=32$ for CRAFT, $n=24$ for ACE and $n=31$ for CRAFTCO.


For the \textit{"Kit Fox" vs. "Red Fox"} scenario, the results come from $n=79$ participants who passed all our screening processes, respectively $n=22$ for CRAFT, $n=31$ for ACE and $n=26$ for CRAFTCO.

\paragraph{General study design}
We followed the experimental design described in \autoref{sec:meta_pred}, in which explanations are evaluated according to their ability to help training participants at getting better at predicting their models' decisions on unseen images.

Each of those participants are only tested on a single condition to avoid possible experimental confounds. 

The main experiment is divided into 3 training sessions (with 5 training samples in each) each followed by a brief test. In each individual training trial, an image was presented with the associated prediction of the model, together with an explanation. After a brief training phase (5 samples), participants' ability to predict the classifier's output was evaluated on 7 new samples during a test phase. During the test phase, no explanation was provided.
We also use the reservoir that subjects can refer to during the testing phase to minimize memory load as a confounding factor.

We implement the same 3-stage screening process: First we filter participants not successful at the practice session done prior to the main experiment used to teach them the task, then we have them go through a quiz to make sure they understood the instructions. Finally, we add a catch trial in each testing phase --that users paying attention are expected to be correct on-- allowing us to catch uncooperative participants.

\begin{figure*}[hb]
    \centering
    \begin{subfigure}{0.9\textwidth}
        \centering
        \includegraphics[width=0.45\textwidth]{assets/craft/website/utility_husky_study.png}
        \includegraphics[width=0.45\textwidth]{assets/craft/website/utility_leaves_study.png}
        \caption{\textbf{Utility experiment.} Training trials taken from the \textit{Husky vs. Wolf} scenario (left) and the \textit{Leaves} scenario (right).}
        \label{fig:craft:website_utility}
    \end{subfigure}
\end{figure*}


\subsubsection{Validation of Recursivity}

\paragraph{Participants} Behavioral accuracy data were gathered from $n=73$ participants. All participants provided informed consent electronically in order to perform the experiment ($\sim 4 - 6$ min). The protocol was approved by the University IRB and was carried out in accordance with the provisions of the World Medical Association Declaration of Helsinki. 
For each of the 2 experiment tested, we had prepared filtering criteria for uncooperative people (namely based on time), but all participants passed these filters.

\paragraph{General study design}

For the first experiment -- consisting in finding the intruder among elements of the same concept and an element from a different concept (but of the same class, see Figure~\ref{fig:craft:website_intruder}) -- the order of presentation is randomized across participants so that it does not bias the results.
Moreover, in order to avoid any bias coming from the participants themselves (one group being more successful than the other) all participants went through both conditions of finding intruders in batches of images coming from either concepts or sub-concepts.
Concerning experiment 2, the order was also randomized (see Figure~\ref{fig:craft:website_choice}).

The participants had to successively find 30 intruders (15 block concepts and 15 block sub-concepts) for experiment 1 and then make 15 choices (sub-concept vs concept) for experiment 2, see Figure~\ref{fig:craft:website}.

The expert participants are people working in machine learning (researchers, software developers, engineers) and have participated in the study following an announcement in the authors' laboratory/company. The other participants (Laymen) have no expertise in machine learning.

\begin{figure*}[ht]
    \centering
    \begin{subfigure}{0.95\textwidth}
        \includegraphics[width=\textwidth]{assets/craft/figure_website.jpg}
        \caption{\textbf{Recursivity Experiment Website.}}
        \label{fig:craft:website}
      
    \end{subfigure}
    \begin{subfigure}{0.95\textwidth}
        \centering
        \includegraphics[width=0.32\textwidth]{assets/craft/website/intruder2.png}
        \includegraphics[width=0.32\textwidth]{assets/craft/website/intruder3.png}
        \includegraphics[width=0.32\textwidth]{assets/craft/website/intruder1.png}
        \caption{\textbf{Binary choice experiment.}}
        \label{fig:craft:website_intruder}
    \end{subfigure}
    
    \begin{subfigure}{0.95\textwidth}
        \centering
        \includegraphics[width=0.32\textwidth]{assets/craft/website/choice3.png}
        \includegraphics[width=0.32\textwidth]{assets/craft/website/choice1.png}
        \includegraphics[width=0.32\textwidth]{assets/craft/website/choice2.png}
        \caption{\textbf{Intruder experiment.}}
        \label{fig:craft:website_choice}
    \end{subfigure}
\end{figure*}

\clearpage

\subsection{Fidelity experiments}\label{app:craft:fidelity}

\begin{figure}[ht]
  \includegraphics[width=.99\linewidth]{assets/craft/deletion_full.jpg}
  \includegraphics[width=.99\linewidth]{assets/craft/insertion_full.jpg}
  \caption{ \textbf{(1) Deletion curves} for different concept extraction methods, Sobol outperforms TCAV not only for NMF to correctly estimate concept importance (lower is better). \textbf{(2) Insertion curves} for different concept extraction methods, Sobol outperforms TCAV to correctly estimate concept importance (higher is better).}
  \label{fig:craft:deletion_full}
\end{figure}

For our experiments on the concept importance measure, we focused on certain classes of ILSRVC2012~\cite{imagenet_cvpr09} and used a ResNet50V2~\cite{he2016deep} that had already been trained on this dataset. Just like in~\cite{ghorbani2017interpretation, zhang2021invertible}, we measure the insertion and deletion metrics for our concept extraction technique -- as well as concepts vectors extracted using PCA, ICA and RCA as dimensionality reduction algorithms, see Figure~\ref{fig:craft:deletion_full} -- and we compare them when we add/remove the concepts as ranked by the TCAV score~\cite{kim2018interpretability} and by the Sobol importance score. As originally explained in~\cite{petsiuk2018rise}, the objective of these metrics is to add/remove parts of the input according to how much an explainability method considers that it is influential and looking at the speed at which the logit for the predicted class increases/decreases.

In particular, for our experimental evaluations, we have randomly chosen 100000 images from ILSVRC2012~\cite{imagenet_cvpr09} and computed the deletion and insertion metrics for 5 different seeds -- for a total of half a million images. In Figure~\ref{fig:craft:deletion_full}, the shade around the curves represent the standard deviation over these 5 experiments.

\clearpage

\subsection{Sanity Check}
\label{app:craft:sanity-checks}

Following the work from~\cite{adebayo2018sanity}, we performed a sanity check on our method, by running the concept extraction pipeline on a randomized model. This procedure was performed on a ResNet-50v2 model with randomized weights. As showcased in Figure~\ref{fig:craft:sanity_check}, the concepts drastically differ from trained models, thus proving that CRAFT passes the sanity check.

\begin{figure}[h]
    \centering
    \includegraphics[width=0.95\linewidth]{assets/craft/sanity_check/rand_c1.png}
    \includegraphics[width=0.95\linewidth]{assets/craft/sanity_check/rand_c2.png}
    \includegraphics[width=0.95\linewidth]{assets/craft/sanity_check/rand_c3.png}
    \caption{\textbf{Sanity check of the method:} we ran the method on a Resnet50 with randomized weights, and extracted the 3 most relevant concepts for the class `Chain saw'. When weights are randomized, concepts are mainly based on color histograms.}
    \label{fig:craft:sanity_check}
\end{figure}

\clearpage

\section{Application: FRSign}
\label{sec:concepts:frsign}
In this comprehensive examination, we extend our investigation into the utility of concept-based methods applied to models trained on the FRSign dataset~\cite{2020frsign}, aiming to delve deeper than conventional attribution methods allow. This inquiry builds upon our previous work outlined in~\autoref{sec:attribution:frsign}, where we expressed reservations about the strategies the model employs, particularly concerning the interpretation of white signals.

In this section, our exploration is structured in three parts: initially, we conduct a review of classes for which we hypothesize the model's behavior aligns closely with expectations. Subsequently, we direct our focus towards the more enigmatic white signal. Finally, we venture further by examining secondary concepts, leading us to propose a hypothesis we term ``support concepts''.

\begin{figure}[ht]
\centering
\includegraphics[width=0.9\textwidth]{assets/frsign/frsign_concepts_red.jpg}
\includegraphics[width=0.9\textwidth]{assets/frsign/frsign_concepts_orange.jpg}
\caption{\textbf{Most Important Concepts for Red and Orange Class.} Applying~\craft method, we extracted the most significant concepts for the ResNet50 model trained on the FRSign dataset. Consistent with findings from~\autoref{sec:attribution:frsign}, the key concepts appear reasonable and are focused on the traffic light itself, reaffirming the model's attention to relevant features.}
\label{fig:frsign:good_concepts}
\end{figure}


\subsection{Visual inspection using concepts}

We commence with a visual inspection of the model's behavior across various classes using our concept-based method, \craft. We visualize the most important concepts that the ResNet50 model -- as detailed in~\autoref{sec:intro:frsign} --  leverages. 

The~\autoref{fig:frsign:good_concepts} illustrates these concepts which, as hypothesized in~\autoref{sec:attribution:frsign}, appear aligned and plausible. In this instance, the concepts do not seem to offer substantial new insights at first glance. We will now proceed to address the challenging case highlighted in the previous section: the interpretation of white signals.

\begin{figure}[ht]
\centering
\includegraphics[width=0.48\textwidth]{assets/frsign/concept_white_signals.jpg}
\includegraphics[width=0.48\textwidth]{assets/frsign/second_concept_white.jpg}
\caption{\textbf{Most and second most important concepts for White class.} The predominant concept for the white signal class seems to be the shear effect on image borders. This finding not only confirms our earlier hypothesis but also sheds additional light on the significance of the frame's edge in the model's decision-making process.}
\label{fig:frsign:white_concepts}
\end{figure}

\subsection{Understanding the White Signal Case}

Our analysis takes a deeper dive into the peculiar case of the white signal, where the primary concept identified appears to be the shear effect on images around the edges of the frame (\autoref{fig:frsign:white_concepts}). This observation supports and further illuminates our previous hypothesis from~\autoref{sec:attribution:frsign}, suggesting that the frame's edge plays a crucial role in the model's interpretation.


\begin{figure}[ht]
\centering
\includegraphics[width=0.45\textwidth]{assets/frsign/concept_orange_curve.jpg}
\includegraphics[width=0.45\textwidth]{assets/frsign/concept_red_curve.jpg}
\includegraphics[width=0.45\textwidth]{assets/frsign/blur_concept.jpg}
\includegraphics[width=0.45\textwidth]{assets/frsign/catenair_concept.jpg}
\caption{\textbf{Examples of Four Secondary, Yet Significant, Concepts.} These concepts are used for the orange, red, yellow, and violet logits, respectively. Although they are not the top-1 concepts, this does not mean they do not contribute to the logits. A portion of the logits is influenced by these concepts, underscoring their importance in the model's decision-making process.}
\label{fig:frsign:support}
\end{figure}


\subsection{Hypothesis: Support Concepts}

In an effort to further our understanding, we investigate secondary, yet influential concepts, which we refer to as "support concepts." These concepts, while not being the top-1 most important for a given class, still significantly contribute to the model's logits for various signals, as illustrated in \autoref{fig:frsign:support}.

We can conjecture that, more problematically, attribution methods that highlight the most important pixels or areas may overlook several features that also drive decision-making and could represent shortcuts. Thus, these concepts could be "hidden" by the attribution maps but still present internally. We introduce the idea of "support concepts" as an avenue for future work, suggesting that a deeper exploration into these underlying influences could unveil additional layers of model reasoning not immediately apparent through conventional attribution techniques.

\subsection{Conclusion}

The application of concept-based explanations has provided us with a more granular understanding of the model's behavior, particularly elucidating the case of the white signal. It appears that biases, possibly inherent in the dataset, necessitate a broader collection of images to mitigate such issues. Alarmingly, our analysis confirms that "support concepts," while not paramount for a class, play a critical role in achieving high performance levels. This revelation affirms the pervasive nature of biases and shortcuts in model training. Consequently, concept-based explainability holds promising potential for unveiling these complexities in model interpretations, offering a path towards more transparent and interpretable machine learning models.

\clearpage

\section{Unifying Automatic Concept Extraction and Concept Importance Estimation}
\label{sec:concepts:holistic}
\definecolor{color1}{RGB}{67, 160, 71}
\definecolor{color2}{RGB}{53, 183, 121}
\definecolor{color3}{RGB}{62, 74, 137}

\newcommand{\ACE}{\textbf{\textcolor{color1}{ACE}}}
\newcommand{\ICE}{\textbf{\textcolor{color2}{ICE}}}

\newcommand{\fa}{\bm{g}}
\newcommand{\fb}{\bm{h}}

In the first section (\autoref{sec:concepts:craft}), we have introduced a first framework able to automatically extract concept and estimate their importance. Recently, other approaches have been proposed, either for concept extraction or concept importance estimation. However no proper metric, benchmark or theoretical framework have been proposed. In this section, we start by noticing that all current concept-based approaches seek discover intelligible visual ``concepts'' buried within the complex patterns of activations using two key steps: (1) concept extraction followed by (2) importance estimation. Again, while these two steps are shared across methods, they all differ in their specific implementations.

Starting from that, we introduce a unifying theoretical framework that recast the first step -- concept extraction problem -- as a special case of \textbf{dictionary learning}, and we formalize the second step -- concept importance estimation -- as a more general form of \textbf{attribution method}.
This framework offers several advantages as it allows us: \tbi{i} to propose new evaluation metrics for comparing different concept extraction approaches; \tbi{ii} to leverage modern attribution methods and evaluation metrics to extend and systematically evaluate state-of-the-art concept-based approaches and importance estimation techniques; \tbi{iii}  to derive theoretical guarantees regarding the optimality of such methods.

We further leverage our framework to try to tackle a crucial question in explainability: how to \textit{efficiently} identify clusters of data points that are classified based on a similar shared strategy.
To illustrate these findings and to highlight the main strategies of a model, we introduce a visual representation called the strategic cluster graph.


\subsection{Introduction}

One promising set of explainability methods to address the issue posed in~\autoref{hyp:what} includes concept-based explainability methods, which are methods that aim to identify high-level concepts within the activation space of ANNs~\cite{kim2018interpretability}. These methods have recently gained renewed interest due to their success in providing human-interpretable explanations~\cite{ghorbani2019towards, zhang2021invertible, fel2023craft, graziani2023concept}. However, concept-based explainability methods are still in the early stages, and progress relies largely on researchers' intuitions rather than well-established theoretical foundations.  A key challenge lies in formalizing the notion of concept itself~\cite{genone2012concept}. 
Researchers have proposed desiderata such as meaningfulness, coherence, and importance~\cite{ghorbani2019towards} but the lack of formalism in concept definition has hindered the derivation of appropriate metrics for comparing different methods.

This section presents a theoretical framework to unify and characterize current concept-based explainability methods. Our approach builds on the fundamental observation that all concept-based explainability methods share two key steps: (1) concepts are extracted, and (2) importance scores are assigned to these concepts based on their contribution to the model's decision~\cite{ghorbani2019towards}. Here, we show how the first extraction step can be formulated as a dictionary learning problem while the second importance scoring step can be formulated as an attribution problem in the concept space. To summarize, our contributions are as follows:

\setlist[itemize]{leftmargin=5.5mm}
\begin{itemize}
    \item We describe a novel framework that unifies all modern concept-based explainability methods and we borrow metrics from different fields (such as sparsity, reconstruction, stability, FID, or OOD scores) to evaluate the effectiveness of those methods. 
    \item We leverage modern attribution methods to derive seven novel concept importance estimation methods and provide theoretical guarantees regarding their optimality. 
    Additionally, we show how standard faithfulness evaluation metrics used to evaluate attribution methods (i.e., Insertion, Deletion~\cite{petsiuk2018rise}, and $\mu$Fidelity~\cite{aggregating2020}) can be adapted to serve as benchmarks for concept importance scoring.
    In particular, we demonstrate that Integrated Gradients, Gradient Input, RISE, and Occlusion achieve the highest theoretical scores for 3 faithfulness metrics when the concept decomposition is on the penultimate layer. 
    \item We introduce the notion of local concept importance to address a significant challenge in explainability: the identification of image clusters that reflect a shared strategy by the model (see Figure~\ref{fig:holistic:clustering_graph}). We show how the corresponding cluster plots can be used as visualization tools to help with the identification of the main visual strategies used by a model to help explain false positive classifications. 
\end{itemize}



\begin{figure}[ht]
\begin{center}
   \includegraphics[width=.99\textwidth]{assets/holistic/clustering_graph.jpg}
\end{center}
   \caption{\textbf{Strategic cluster graphs for the espresso and zucchini classes.}
    The framework presented in this section provides a comprehensive approach to uncover local importance using any attribution methods. 
    Consequently, it allow us to estimate the critical concepts influencing the model's decision for each image.
    As a results, we introduced the Strategic cluster graph, which offers a visual representation of the main strategies employed by the model in recognizing an entire object class.
    For espresso (left), the main strategies for classification appear to be: \textcolor{anthracite}{$\bullet$} bubbles and foam on the coffee, \textcolor{green}{$\bullet$} Latte art, \textcolor{yellow}{$\bullet$} transparent cups with foam and black liquid, \textcolor{red}{$\bullet$} the handle of the coffee cup, and finally \textcolor{blue}{$\bullet$} the coffee in the cup, which appears to be the predominant strategy.
    As for zucchini, the strategies are: \textcolor{blue}{$\bullet$} a zucchini in a vegetable garden, \textcolor{yellow}{$\bullet$} the corolla of the zucchini flower, \textcolor{red}{$\bullet$} sliced zucchini, \textcolor{green}{$\bullet$} the spotted pattern on the zucchini skin and \textcolor{anthracite}{$\bullet$} stacked zucchini.
   }
\label{fig:holistic:clustering_graph}
\end{figure}

\subsection{A Unifying perspective}

\paragraph{Notations.} Throughout, $||\cdot||_2$ and $||\cdot||_F$ represent the $\ell_2$ and Frobenius norm, respectively. 
We consider a general supervised learning setting, where a classifier $\f : \sx \to \sy$ maps inputs from an input space $\mathcal{X} \subseteq \Real^d$ to an output space $\sy \subseteq \Real^c$. 
For any matrix $\mx \in \Real^{n \times d}$, $\vx_i$ denotes the $i^{th}$ row of $\mx$, where $i \in \{1, \ldots, n \}$ and $\vx_i \in \Real^{d}$.
Without loss of generality, we assume that $\f$ admits an intermediate space $\s{I} \subseteq \Real^p$. In this setup, $\fa : \sx \to \s{I}$ maps inputs to the intermediate space, and $\fb : \s{I} \to \sy$ takes the intermediate space to the output. Consequently, $\f(\vx) = (\fb \circ \fa)(\vx)$. Additionally, let $\bm{a} = \fa(\vx) \in \s{I}$ represent the activations of $\vx$ in this intermediate space.
We also abuse notation slightly: $\f(\mx) = (\fb \circ \fa)(\mx)$ denotes the vectorized application of $\f$ on each element $\vx$ of $\mx$, resulting in $(\f(\vx_1),\ldots, \f(\vx_n))$.

\paragraph{2 Fundamental steps.} Prior methods for concept extraction, namely \ACE~\cite{ghorbani2019towards}, \ICE~\cite{zhang2021invertible}~and \CRAFT~\cite{fel2023craft}, can be distilled into two fundamental steps:

\begin{enumerate}[label=(\textit{\textbf{\roman*}}), labelindent=0pt,leftmargin=5mm]

\item {\bf Concept extraction:} A set of images $\mx \in \Real^{n\times d}$ belonging to the same class is sent to the intermediate space giving activations $\m{A} = \fa(\mx) \in \Real^{n \times p}$.
These activations are used to extract a set of $k$ CAVs using K-Means~\cite{ghorbani2019towards}, PCA (or SVD)~\cite{zhang2021invertible, graziani2023concept} or NMF~\cite{zhang2021invertible, fel2023craft}. Each CAV is denoted $\cav_i$ and $\m{V} = (\cav_1, \ldots, \cav_k) \in \Real^{p \times k}$ forms the dictionary of concepts.

\item {\bf Concept importance scoring:} 
It involves calculating a set of $k$ global scores, which provides an importance measure of each concept $\cav_i$ to the class as a whole. Specifically, it quantifies the influence of each concept $\cav_i$ on the final classifier prediction for the given set of points  $\mx$. Prominent measures for concept importance include TCAV~\cite{kim2018interpretability} and the Sobol indices~\cite{fel2023craft}. 

\end{enumerate}


The two-step process described above is repeated for all classes. In the following subsections, we theoretically demonstrate that the concept extraction step \tbi{i} could be recast as a dictionary learning problem (see~\ref{sec:dico_learning}). It allows us to reformulate and generalize the concept importance step \tbi{ii} using attribution methods (see~\ref{sec:importance}). 

\subsubsection{Concept Extraction}

\paragraph{A dictionary learning perspective.}
\label{sec:dico_learning} 
The purpose of this section is to redefine all current concept extraction methods as a problem within the framework of dictionary learning. Given the necessity for clearer formalization and metrics in the field of concept extraction, integrating concept extraction with dictionary learning enables us to employ a comprehensive set of metrics and obtain valuable theoretical insights from a well-established and extensively researched domain. 

The goal of concept extraction is to find a small set of interpretable CAVs (i.e., $\m{V}$) that allows us to faithfully interpret the activation $\m{A}$. By preserving a linear relationship during the reconstruction, from $\m{U}$ to $\m{A}$ (and not necessarily from $\m{A}$ to $\m{U}$), we facilitate the understanding and interpretability of the learned concepts~\cite{kim2018interpretability, elhage2022superposition}. Therefore, we look for a coefficient matrix $\m{U} \in \Real^{n \times k}$ (also called loading matrix) and a set of CAVs $\m{V}$, so that $\m{A} \approx \m{U} \m{V}^\tr$.
In this approximation of $\m{A}$ using the two low-rank matrices $(\m{U}, \m{V})$,
$\m{V}$ represents the concept basis used to reinterpret our samples, and $\m{U}$ are the coordinates of the activation in this new basis. Interestingly, such a formulation allows a recast of the concept extraction problem as an instance of dictionary learning problem~\cite{mairal2014sparse} %
in which all known concept-based explainability methods fall:%

\begin{numcases}{(\m{U}^\star, \m{V}^\star) = \argmin_{\m{U},\m{V}} || \m{A} - \m{U} \m{V}^\tr ||^2_F ~~s.t~~}
 \forall ~ i, \v{u}_i \in \{ \e_1, \ldots, \e_k \} ~~ \text{\small\cite{ghorbani2019towards})}, \label{eq:holistic:dico_kmeans}\nonumber\\
  \m{V}^\tr \m{V} = \mathbf{I} ~~~ \text{\small(\cite{graziani2023concept,zhang2021invertible})}, \label{eq:holistic:dico_pca}\nonumber\\
 \m{U} \geq 0, \m{V} \geq 0 ~~~ \text{\CRAFT} \nonumber\\
 \m{U} = \bm{\psi}(\m{A}), ||\m{U}||_0 \leq \kappa  ~~\text{\small \cite{makhzani2013k}} \label{eq:holistic:dico_nmf}\nonumber
\end{numcases}

with $\bm{e}_i$ the $i$-th element of the canonical basis, $\mathbf{I}$ the identity matrix and $\bm{\psi}$ any neural network. 
In this context, $\m{V}$ is the \emph{dictionary} and $\m{U}$ the \emph{representation} of $\m{A}$ with the atoms of $\m{V}$. $\v{u}_i$ denote the $i$-th row of $\m{U}$. 
These methods extract the concept banks $\m{V}$ differently, thereby necessitating different interpretations\footnote{Concept extractions are typically overcomplete dictionaries, meaning that if the dictionary for each class is combined, $k >> p$, as noted in our previous section. The collapse problem in \autoref{sec:concepts:craft}, and a more detailed work \cite{bricken2023monosemanticity} suggest that overcomplete dictionaries are serious candidates to the superposition problem~\cite{elhage2022superposition}.}. 

In \ACE, the CAVs are defined as the centroids of the clusters found by the K-means algorithm.
Specifically, a concept vector $\cav_i$ in the matrix $\m{V}$ indicates a dense concentration of points associated with the corresponding concept, implying a repeated activation pattern. 
The main benefit of ACE comes from its reconstruction process, involving projecting activations onto the nearest centroid, which ensures that the representation will lie within the observed distribution (no out-of-distribution instances). %
However, its limitation lies in its lack of expressivity, as each activation representation is restricted to a single concept ($||\v{u}||_{0}=1$). As a result, it cannot capture compositions of concepts, leading to sub-optimal representations that fail to fully grasp the richness of the underlying data distribution.

On the other hand, the PCA benefits from superior reconstruction performance due to its lower constraints, as stated by the Eckart-Young-Mirsky~\cite{eckart1936approximation} theorem. %
The CAVs are the eigenvector of the covariance matrix: they indicate the direction in which the data variance is maximal. %
An inherent limitation is that the PCA will not be able to properly capture stable concepts that do not contribute to the sample variability (e.g. the dog-head concept might not be considered important by the PCA to explain the dog class if it is present across all examples).
Neural networks are known to cluster together the points belonging to the same category in the last layer to achieve linear separability (\cite{paypan2020collapse, fel2023craft}). Thus, the orthogonality constraint in the PCA might not be suitable to correctly interpret the manifold of the deep layer induced by points from the same class (it is interesting to note that this limitation can be of interest when studying all classes at once).
Also, unlike K-means, which produces strictly positive clusters if all points are positive (e.g., the output of ReLU), PCA has no sign constraint and can undesirably reconstruct out-of-distribution (OOD) activations, including negative values after ReLU. %

In contrast to K-Means, which induces extremely sparse representations, and PCA, which generates dense representations, the NMF (used in \CRAFT~and \ICE) strikes a harmonious balance as it provides moderately sparse representation. This is due to NMF relaxing the constraints imposed by the K-means algorithm (adding an orthogonality constraint on $\m{V}$ such that $\m{V} \m{V}^\tr = \mathbf{I}$ would yield an equivalent solution to K-means clustering~\cite{ding2005equivalence}). This sparsity facilitates the encoding of compositional representations that are particularly valuable when an image encompasses multiple concepts. Moreover, by allowing only additive linear combinations of components with non-negative coefficients, %
NMF inherently fosters a parts-based representation. This distinguishes NMF from PCA, which offers a holistic representation model. Interestingly, the NMF is known to yield representations that are interpretable by humans~\cite{zhang2021invertible, fel2023craft}.
Finally, the non-orthogonality of these concepts presents an advantage as it accommodates the phenomenon of superposition~\cite{elhage2022superposition}, wherein neurons within a layer may contribute to multiple distinct concepts simultaneously.

To summarize, we have explored three approaches to concept extraction, each necessitating a unique interpretation of the resulting Concept Activation Vectors (CAVs). Among these methods, NMF (used in \CRAFT~ and \ICE) emerges as a promising middle ground between PCA and K-means. Leveraging its capacity to capture intricate patterns, along with its ability to facilitate compositional representations and intuitive parts-based interpretations (as demonstrated in Figure~\ref{fig:holistic:qualitative_comparison}), NMF stands out as a compelling choice for extracting meaningful concepts from high-dimensional data. These advantages have been underscored by our human studies, and also evidenced by works such as~\cite{zhang2021invertible}.

\begin{table*}[h!]
\centering
\scalebox{0.76}{\begin{tabular}{l c c c c c}
\toprule
& \multicolumn{1}{c}{Relative $\ell_2$ ($\downarrow$)} 
& \multicolumn{1}{c}{Sparsity ($\uparrow$)} 
& \multicolumn{1}{c}{Stability ($\downarrow$)} 
& \multicolumn{1}{c}{FID ($\downarrow$)} 
& \multicolumn{1}{c}{OOD ($\downarrow$)} \\
 
\cmidrule(lr){2-2}
\cmidrule(lr){3-3}
\cmidrule(lr){4-4}
\cmidrule(lr){5-5}
\cmidrule(lr){6-6}

& 
Eff / R50 / Mob &
Eff / R50 / Mob &
Eff / R50 / Mob &
Eff / R50 / Mob &
Eff / R50 / Mob 
\\

\midrule
PCA 
   & 0.60 / 0.54 / 0.73 
   & 0.00 / 0.00 / 0.0
   & 0.41 / 0.38 / 0.43
   & 0.47 / 0.17 / 0.24
   & 2.44 / 0.36 / 0.16
\\
KMeans 
   & 0.72 / 0.66 / 0.84 
   & 0.95 / 0.95 / 0.95
   & 0.07 / 0.08 / 0.04
   & 0.46 / 0.21 / 0.33
   & 1.76 / 0.29 / 0.15
\\
NMF 
   & 0.63 / 0.57 / 0.75 
   & 0.68 / 0.44 / 0.64
   & 0.17 / 0.14 / 0.16
   & 0.38 / 0.21 / 0.24
   & 1.98 / 0.29 / 0.15
\\
\bottomrule
\end{tabular}}
\caption{\textbf{Concept extraction comparison.} Eff, R50 and Mob denote EfficientNetV2~\cite{zhang2018efficient}, ResNet50~\cite{he2016deep}, MobileNetV2~\cite{sandler2018mobilenetv2}. The concept extraction methods are applied on the last layer of the networks. Each results is averaged across 10 classes of ImageNet and obtained from a set of 16k images for each class.
}\label{tab:quantitative_comparison}
\vspace{-6mm}
\end{table*}



\begin{figure}[t]
\begin{center}
   \includegraphics[width=.99\textwidth]{assets/holistic/qualitative_extraction.jpg}
\end{center}
   \caption{\textbf{Most important concepts extracted for the studied methods.} This qualitative example shows the three most important concepts extracted for the 'rabbit' class using a ResNet50 trained on ImageNet. The crops correspond to those maximizing each concepts $i$ (i.e., $\vx$ where $\m{U}(\vx)_i$ is maximal). As demonstrated in previous works \cite{zhang2021invertible,fel2023craft,parekh2022listen}, NMF (requiring positive activations) produces particularly interpretable concepts despite poorer reconstruction than PCA and being less sparse than K-Means. Details for the sparse Autoencoder architecture are provided in the appendix.}
\label{fig:holistic:qualitative_comparison}
\end{figure}

\paragraph{Evaluation of concept extraction}
Following the theoretical discussion of the various concept extraction methods, we conduct an empirical investigation of the previously discussed properties to gain deeper insights into their distinctions and advantages. In our experiment, we apply the PCA, K-Means, and NMF concept extraction methods on the penultimate layer of three state-of-the-art models. We subsequently evaluate the concepts using five different metrics (see Table \ref{tab:quantitative_comparison}). 
All five metrics are connected with the desired characteristics of a dictionary learning method. They include achieving a high-quality reconstruction (Relative l2), sparse encoding of concepts (Sparsity), ensuring the stability of the concept base in relation to $\m{A}$ (Stability), performing reconstructions within the intended domain (avoiding OOD), and maintaining the overall distribution during the reconstruction process (FID).
All the results come from 10 classes of ImageNet (the one used in Imagenette \cite{imagenette}), and are obtained using $n=16k$ images for each class. 



We begin our empirical investigation by using a set of standard metrics derived from the dictionary learning literature, namely Relative $l_2$ and Sparsity. 
Concerning the Relative $\ell_2$, PCA achieves the highest score among the three considered methods, confirming the theoretical expectations based on the Eckart–Young–Mirsky theorem~\cite{eckart1936approximation}, followed by NMF.
Concerning the sparsity of the underlying representation $\v{u}$, we compute the proportion of non-zero elements $||\v{u}||_0 / k$. Since K-means inherently has a sparsity of $1 / k$ (as induced by equation \ref{eq:holistic:dico_kmeans}), it naturally performs better in terms of sparsity, followed by NMF.


We deepen our investigation by proposing three additional metrics that offer complementary insights into the extracted concepts. Those metrics are the Stability, the FID, and the OOD score.
The Stability (as it can be seen as a loose approximation of algorithmic stability~\cite{bousquet2002stability}) measures how consistent concepts remain when they are extracted from different subsets of the data.
To evaluate Stability, we perform the concept extraction methods $N$ times on $K$-fold subsets of the data. Then, we map the extracted concepts together using a Hungarian loss function and measure the cosine similarity of the CAVs. If a method is stable, it should yield the same concepts (up to permutation) across each $K$-fold, where each fold consists of $1000$ images.
K-Means and NMF demonstrate the highest stability, while PCA appears to be highly unstable, which can be problematic for interpreting the results and may undermine confidence in the extracted concepts.

The last two metrics, FID and OOD, are complementary in that they measure: (i) how faithful the representations extracted are w.r.t the original distribution, and (ii) the ability of the method to generate points lying in the data distribution (non-OOD).
Formally, the FID quantifies the 1-Wasserstein distance~\cite{villani2009optimal} $\mathcal{W}_1$ between the empirical distribution of activation $\m{A}$, denoted $\mu_{\bm{a}}$, and the empirical distribution of the reconstructed activation $\m{U}\m{V}^\tr$ denoted $\mu_{\bm{u}}$. Thus, FID is calculated as $\text{FID} = \mathcal{W}_1(\mu_{\bm{a}}, \mu_{\bm{u}})$.
On the other hand, the OOD score measures the plausibility of the reconstruction by leveraging Deep-KNN~\cite{sun2022out}, a recent state-of-the-art OOD metric. More specifically,  we use the Deep-KNN score to evaluate the deviation of a reconstructed point from the closest original point. In summary, a good reconstruction method is capable of accurately representing the original distribution (as indicated by FID) while ensuring that the generated points remain within the model's domain (non-OOD). 
K-means leads to the best OOD scores because each instance is reconstructed as a centroid, resulting in proximity to in-distribution (ID) instances. However, this approach collapses the distribution to a limited set of points, resulting in low FID. On the other hand, PCA may suffer from mapping to negative values, which can adversely affect the OOD score. Nevertheless, PCA is specifically optimized to achieve the best average reconstructions. NMF, with fewer stringent constraints, strikes a balance by providing in-distribution reconstructions at both the sample and population levels.

In conclusion, the results clearly demonstrate NMF as a method that strikes a balance between the two approaches as NMF demonstrates promising performance across all tested metrics. Henceforth, we will use the NMF to extract concepts without mentioning it.



\paragraph{The Last Layer as a Promising Direction}
The various methods examined, namely \ACE, \ICE, and \CRAFT, generally rely on a deep layer to perform their decomposition without providing quantitative or theoretical justifications for their choice. 
To explore the validity of this choice, we apply the aforementioned metrics to each block's output in a ResNet50 model.
Figure~\ref{fig:holistic:metrics_across_layer} illustrates the metric evolution across different blocks, revealing a trend that favors the last layer for the decomposition. This empirical finding aligns with the practical implementations discussed above.


\begin{figure}[ht]
\begin{center}
   \includegraphics[width=.99\textwidth]{assets/holistic/metrics_across_layer.pdf}
\end{center}
   \caption{\textbf{Concept extraction metrics across layers.} The concept extraction methods are applied on activations probed on different blocks of a ResNet50 (B2 to B5). Each point is averaged over 10 classes of ImageNet using $16$k images for each class. We evaluate $3$ concept extraction methods: PCA (\dashed), NMF (\full), and KMeans (\dotted).
   }
\label{fig:holistic:metrics_across_layer}

\end{figure}
















\subsubsection{Concept importance}\label{sec:importance}


In this section, we leverage our framework to unify concept importance scoring using the existing attribution methods. Furthermore, we demonstrate that specifically in the case of decomposition in the penultimate layer, it exists optimal methods for importance estimation, namely RISE~\cite{petsiuk2018rise}, Integrated Gradients~\cite{sundararajan2017axiomatic}, Gradient-Input~\cite{shrikumar2017learning}, and Occlusion~\cite{zeiler2014visualizing}. We provide theoretical evidence to support the optimality of these methods.

\paragraph{From concept importance to attribution methods}
The dictionary learning formulation allows us to define the concepts $\m{V}$ in such a way that they are optimal to reconstruct the activation, i.e., $\m{A} \approx \m{U} \m{V}^\tr$. Nevertheless, this does not guarantee that those concepts are important for the model's prediction. For example, the ``grass'' concept might be important to characterize the activations of a neural network when presented with a 
St-Bernard image, but it might not be crucial for the network to classify the same image as a St Bernard~\cite{kim2018interpretability, adebayo2018sanity, ghorbani2017interpretation}. The notion of concept importance is precisely introduced to avoid such a confirmation bias and to identify the concepts used to classify among all detected concepts. 


We use the notion of Concept ATtribution methods (which we denote as \emph{CAT}s) to assess the concept importance score. The CATs are a generalization of the attribution methods: 
while attribution methods assess the sensitivity of the model output to a change in the pixel space, the concept importance evaluates the sensitivity to a change in the concept space. To compute the CATs methods, it is necessary to link the activation $\v{a} \in \Real^p$ to the concept base $\m{V}$ and the model prediction $\v{y}$. To do so, we feed the second part of the network ($\fb$) with the activation reconstruction ($\v{u} \m{V}^\tr \approx \v{a}$) so that $\v{y} = \fb(\v{u}\m{V}^\tr)$. Intuitively, a CAT method quantifies how a variation of  $\v{u}$ will impact $\v{y}$. 
We denote $\cam_i(\bm{u})$ the $i$-th coordinate of $\cam(\bm{u})$, so that it represents the importance of the $i$-th concept in the representation $\bm{u}$.  Equipped with these notations, we can leverage the sensitivity metrics introduced in standard attribution methods to re-define the current measures of concept importance, as well as introduce the new CATs borrowed from the attribution methods literature:


\scalebox{0.9}{\parbox{\linewidth}{%
\begin{empheq}[left={\cam_{i}(\v{u}) =\empheqlbrace}]{alignat=1}
&\nabla_{\v{u}_i} \fb(\v{u} \m{V}^\tr)
~~
\text{(\small TCAV: \cite{ghorbani2019towards,zhang2021invertible,graziani2021sharpening})}, \nonumber\\
&\displaystyle \frac{ \mathbb{E}_{\mathbf{m}_{\sim i}}( \mathbb{V}_{\mathbf{m}}( \fb( (\v{u} \odot \mathbf{m} ) \m{V}^\tr ) | \mathbf{m}_{\sim i} ) ) }{ \mathbb{V}( \fb( (\v{u} \odot \mathbf{m} ) \m{V}^\tr)) }
\qquad \qquad \qquad \qquad ~~~~~~~~~ \text{\small(Sobol: \CRAFT),}  \nonumber\\
&(\v{u}_i - \v{u}_i')  \times \int_0^1\nabla_{\v{u}_i}\fb((\v{u}' \alpha + (1 - \alpha)(\v{u} - \v{u}'))\m{V}^\tr) d\alpha
\qquad \text{\small(Int.Gradients)}, \nonumber\\
&\displaystyle \underset{\bm{\delta} \sim \mathcal{N}(0, \mathbf{I}\sigma)}{\mathbb{E}}(\nabla_{\v{u}_i} \fb( (\v{u} + \bm{\delta})\m{V}^T) )
\qquad \qquad \qquad \qquad \qquad \qquad ~~ \text{\small(Smoothgrad)}. \nonumber \\
\ldots \nonumber
\end{empheq} 
}}
\vspace{3mm}



The complete derivation of the 7 new CATs is provided in the appendix. 
In the derivations, $\nabla_{\v{u}_i}$ denotes the gradient with respect to the $i$-th coordinate of $\v{u}$, while $\mathbb{E}$ and $\mathbb{V}$ represent the expectation and variance, respectively, $\mathbf{m}$ is a mask of real-valued random variable between $0$ and $1$ (i.e $\mathbf{m}\sim\mathcal{U}([0,1]^p)$). We note that, when we use the gradient (w.r.t to $\v{u}_i$) as an importance score, we end up with the directional derivative used in the TCAV metric~\cite{kim2018interpretability}. In other words, one could say that TCAV is the Saliency of the \textit{Concept Attribution} methods. 

\CRAFT~leverages the Sobol-Hoeffding decomposition (used in sensitivity analysis), to estimate the concept importance. The Sobol indices measure the contribution of a concept as well as its interaction of any order with any other concepts to the output variance. Intuitively, the numerator for the Sobol importance formula is the expected variance that would be left if all variables but $\v{u}_i$ were to be fixed.

\begin{figure}[ht]
\begin{subfigure}[b]{0.49\textwidth}
\includegraphics[width=.99\textwidth]{assets/holistic/deletion_curves.jpg}
    \caption{}
\end{subfigure}
\hfil
\begin{subfigure}[b]{0.49\textwidth}
\includegraphics[width=.99\textwidth]{assets/holistic/deletion_layers_craft.png}
    \caption{}
\end{subfigure}
\caption{\textbf{(a) C-Deletion, C-Insertion curves.} Fidelity curves for C-Deletion depict the model's score as the most important concepts are removed. The results are averaged across 10 classes of ImageNet using a ResNet50 model.
\textbf{(b) C-Deletion, C-Insertion and C-$\mu$Fidelity across layer.} 
We report the $3$ metrics to evaluate CATs for each block (from B2 to B5) of a ResNet50. 
We evaluate $8$ Concept Attribution methods, all represented with different colors (see legend in  Figure~\ref{fig:holistic:deletion_curves}(a). The average trend of these eight methods is represented by the black dashed line (\dashed). Lower C-Deletion is better, higher C-Insertion and C-$\mu$Fidelity is better. Overall, it appears that the estimation of importance becomes more faithful towards the end of the model.
}
\label{fig:holistic:deletion_curves}
\end{figure}


\paragraph{Evaluation of concept importance methods}

Our generalization of the concept importance score, using the Concept ATtributions (CATs), allows us to observe that current concept-based explainability methods are only leveraging a small subset of concept importance methods. In Appendix~\ref{sup:holistic:all_cams}, we provide the complete derivation of $7$ new CATs based on the following existing attribution methods, notably: Gradient input~\cite{shrikumar2017learning}, Smooth grad~\cite{smilkov2017smoothgrad}, Integrated Gradients~\cite{sundararajan2017axiomatic}, VarGrad~\cite{hooker2018benchmark}, Occlusion~\cite{zeiler2014visualizing}, HSIC~\cite{novello2022making} and RISE~\cite{petsiuk2018rise}.



With the concept importance scoring now formulated as a generalization of attribution methods, we can borrow the metrics from the attribution domain to evaluate the faithfulness~\cite{jacovi2020towards,petsiuk2018rise,aggregating2020} of concept importance methods. In particular, 
we adapt three distinct metrics %
to evaluate the significance of concept importance scores: the C-Deletion~\cite{petsiuk2018rise}, C-Insertion~\cite{petsiuk2018rise}, and C-$\mu$Fidelity~\cite{aggregating2020} metrics.
In C-Deletion, we gradually remove the concepts (as shown in Figure \ref{fig:holistic:deletion_curves}), in decreasing order of importance, and we report the network's output each time a concept is removed. When a concept is removed in C-Deletion, the corresponding coordinate in the representation is set to $\bm{0}$. 
The final C-Deletion metrics are computed as the area under the curve in Figure~\ref{fig:holistic:deletion_curves}. For C-Insertion, this is the opposite: we start from a representation vector filled with zero, and we progressively add more concepts, following an increasing order of importance. 


For the C-$\mu$Fidelity, we calculate the correlation between the model's output when concepts are randomly removed and the importance assigned to those specific concepts.
The results across layers for a ResNet50 model are depicted in Figure \ref{fig:holistic:deletion_curves}b. We observe that decomposition towards the end of the model is preferred across all the metrics. As a result, in the next section, we will specifically examine the case of the penultimate layer.




\paragraph{A note on the last layer}
Based on our empirical results, it appears that the last layer is preferable for both improved concept extraction and more accurate estimation of importance. 
Herein, we derive theoretical guarantees about the optimality of concept importance methods in the penultimate layer. %
Without loss of generality, we assume $y \in \Real$ the logits of the class of interest. In the penultimate layer, the score $y$ is a linear combination of activations: $y=\bm{a}\m{W}+\bias$ for weight matrix $\m{W}$ and bias $\bias$. 
In this particular case, all CATs have a closed-form (see appendix~\ref{sup:holistic:closed_form}), that allows us to derive $2$ theorems. The first theorem tackles the CATs optimality for the C-Deletion and C-Insertion methods (demonstration in Appendix~\ref{sup:holistic:matroid}). We observe that the C-Deletion and C-Insertion problems can be represented as weighted matroids. Therefore the greedy algorithms lead to optimal solutions for CATs and a similar theorem could be derived for C-$\mu$Fidelity.
\begin{theorem}[Optimal C-Deletion, C-Insertion in the penultimate layer]
When decomposing in the penultimate layer,~\textbf{Gradient Input}, \textbf{Integrated Gradients}, \textbf{Occlusion}, and \textbf{Rise} yield the optimal solution for the C-Deletion and C-Insertion metrics.
More generally, any method $\cam(\v{u})$ that satisfies the condition 
$\forall (i, j) \in \{1, \ldots, k\}^2, 
(\v{u} \odot \e_i) \m{V}^\tr\m{W} \geq (\v{u} \odot \e_j) \m{V}^\tr \m{W}
\implies 
\cam(\v{u})_i \geq \cam(\v{u})_j 
$ yields the optimal solution.
\end{theorem}
\begin{theorem}[Optimal C-$\mu$Fidelity in the penultimate layer]
When decomposing in the penultimate layer,~\textbf{Gradient Input}, \textbf{Integrated Gradients}, \textbf{Occlusion}, and \textbf{Rise} yield the optimal solution for the C-$\mu$Fidelity metric.
\end{theorem}


\begin{figure}[ht]
    \centering
    \includegraphics[width=0.9\textwidth]{assets/holistic/reliability.jpg}
    \caption{\textbf{From global (class-based) to local (image-based) importance.} Global importance can be decomposed into \textit{reliability} and \textit{prevalence} scores. Prevalence quantifies how frequently a concept is encountered, and reliability indicates how diagnostic a concept is for the class. The bar-charts are computed for the class ``Espresso'' on a ResNet50 (see Figure~\ref{fig:holistic:clustering_graph}, left panel)
    }
    \label{fig:holistic:barchart}
\end{figure}

Therefore, for all $3$ metrics, the concept importance methods based on Gradient Input, Integrated Gradient, Occlusion, and Rise are optimal, when used in the penultimate layer.

In summary, our investigation of concept extraction methods from the perspective of dictionary learning demonstrates that the NMF approach, specifically when extracting concepts from the penultimate layer, presents the most appealing trade-off compared to PCA and K-Means methods. In addition, our formalization of concept importance using attribution methods provided us with a theoretical guarantee for $4$ different CATs. Henceforth, we will then consider the following setup: a NMF on the penultimate layer to extract the concepts, combined with a concept importance method based on Integrated Gradient.

\subsubsection{Unveiling main strategies}

So far, the concept-based explainability methods have mainly focused on evaluating the global importance of concepts, i.e., the importance of concepts for an entire class~\cite{kim2018interpretability,fel2023craft}. This point can be limiting when studying misclassified data points, as we can speculate that the most important concepts for a given class might not hold for an individual sample (local importance). Fortunately, our formulation of concept importance using attribution methods gives us access to importance scores at the level of individual samples (\textit{i.e.,} $\cam(\v{u})$). Here, we show how to use these local importance scores to efficiently cluster data points based on the strategy used for their classification. 

The local (or image-based) importance of concepts can be integrated into global measures of importance for the entire class with the notion of \textit{prevalence} and \textit{reliability} (see Figure~\ref{fig:holistic:barchart}). A concept is said to be prevalent at the class level when it appears very frequently. A \textit{prevalence} score is computed based on the number of times a concept is identified as the most important one, i.e., $\argmax \cam(\v{u})$. At the same time, a concept is said to be reliable if it is very likely to trigger a correct prediction. The \textit{reliability} is quantified using the mean classification accuracy on samples sharing the same most important concept.


\paragraph{Strategic cluster graph.} In the strategic cluster graph (Figure~\ref{fig:holistic:clustering_graph} and Figure~\ref{fig:holistic:lemon}), we combine the notions of concept \textit{prevalence} and \textit{reliability} to reveal the main strategies of a model for a given category, more precisely, we reveal their repartition across the different samples of the class.
We use a dimensionality reduction technique (UMAP~\cite{mcinnes2018umap}) to arrange the data points based on the concept importance vector $\cam(\v{u})$ of each sample. Data points are colored according to the associated concept with the highest importance -- $\argmax \cam(\v{u})$. 
Interestingly, one can see in Figure~\ref{fig:holistic:clustering_graph} and Figure~\ref{fig:holistic:lemon} that spatially close points represent samples classified using \textit{similar strategies} -- as they exhibit similar concept importance -- and not necessarily similar embeddings.
For example, for the ``lemon'' object category (Figure \ref{fig:holistic:lemon}), the texture of the lemon peel is the most \textit{prevalent} concept, as it appears to be the dominant concept in $90\%$ of the samples (see the green cluster in Figure~\ref{fig:holistic:lemon}). We also observe that the concept ``pile of round, yellow objects'' is not reliable for the network to properly classify a lemon as it results in a mean classification accuracy of $40\%$ only (see top-left graph in Figure~\ref{fig:holistic:lemon}).

In Figure~\ref{fig:holistic:lemon} (right panel), we have exploited the strategic cluster graph to understand the classification strategies leading to bad classifications. For example, an orange ($1^{st}$ image, $1^{st}$ row) was classified as a lemon because of the peel texture they both share. Similarly, a cathedral roof was classified as a lemon because of the wedge-shaped structure of the structure ($4^{th}$ image, $1^{st}$ row). 




\begin{figure}[ht]
\begin{center}
   \includegraphics[width=1\textwidth]{assets/holistic/lemon.jpg}
\end{center}
   \caption{\textbf{Strategic cluster graph for the lemon category.} \textbf{Left}: U-MAP of lemon samples, in the concept space. Each concept is represented with its own color and is exemplified with example belonging to the cluster. The concepts are  \textcolor{red}{$\bullet$} the lemon wedge shape, \textcolor{yellow}{$\bullet$} a pile of round, yellow objects, \textcolor{blue}{$\bullet$} green objects hanging on a tree, and finally \textcolor{green}{$\bullet$} the peel texture, which is the predominant strategy. The reliability of each concept is shown in the top-left bar-chart. \textbf{Right}: 
   Example of images predicted as lemon along with their corresponding explanations. These misclassified images are recognized as lemons through the implementation of strategies that are captured by our proposed strategic cluster graph.
   }
\label{fig:holistic:lemon}
\end{figure}


\subsection{Discussion}

In this section, we have introduced a theoretical framework that unifies all modern concept-based explainability methods. Breaking down and formalizing the two essential steps in these methods, concept extraction and concept importance scoring, allowed us to better understand the underlying principles driving concept-based explainability. We leveraged this unified framework to propose new evaluation metrics for assessing the quality of extracted concepts. Through experimental and theoretical analyses, we justified the standard use of the last layer of an ANN for concept-based explanation. Finally, we harnessed the parallel between concept importance and attribution methods to gain insights into global concept importance (at the class level) by examining local concept importance (for individual samples). We proposed the strategic cluster graph, which provides insights into the strategy used by an ANN to classify images. We have provided an example use of this approach to better understand the failure cases of a system. Overall, our work demonstrates the potential benefits of the \textbf{dictionary learning} framework for automatic concept extraction and we hope this work will pave the way for further advancements and methodologies in the field au concept-based explainability.


In this research, we deliberately overlooked a particular challenge associated with the automatic concept approach, namely, the comprehensibility of the features extracted by dictionary-based methods. Indeed, relying solely on image segments to elucidate a concept could be restrictive. In the next section, we will examine an alternative approach to visualize concepts with feature visualization.

\clearpage
  
\section{Modern Feature Visualization with MACO}
\label{sec:concepts:maco}


The last section of this chapter will be dedicated to a novel method that will enable us one problem that we identify in \autoref{sec:concepts:craft}: the visualization of concept. Feature visualization -- defined in \autoref{def:intro:feature_viz} -- has gained substantial popularity, particularly after the seminal and influential work of the Clarity team~\cite{olah2017feature}, which established it as a crucial tool for explainability.
However, its widespread adoption has been limited due to a reliance on tricks to generate interpretable images, and corresponding challenges in scaling it to deeper neural networks.
Here, we will introduce \magfv, a simple approach to address these shortcomings.
The main idea is to generate images by optimizing the phase spectrum while keeping the magnitude constant to ensure that generated explanations lie in the space of natural images. Our approach yields significantly better results -- both qualitatively and quantitatively -- and unlocks efficient and interpretable feature visualizations for large state-of-the-art neural networks.
We also show that our approach exhibits an attribution mechanism allowing us to augment feature visualizations with spatial importance.


Overall, our approach unlocks, for the first time, feature visualizations for large, state-of-the-art deep neural networks without resorting to any parametric prior image model.

\begin{figure}[ht]
\begin{center}
   \includegraphics[width=.99\textwidth]{assets/maco/big_picture.jpg}
\end{center}

\caption{\textbf{Comparison between feature visualization methods for ``White Shark'' classification.}
\textbf{(Top)} Standard Fourier preconditioning-based method for feature visualization~\cite{olah2017feature}.
\textbf{(Bottom)} Proposed approach, \magfv, which incorporates a Fourier spectrum magnitude constraint. %
}
\label{fig:maco:logits_fail}

\end{figure}

\subsection{Introduction}

As discussed in \autoref{chap:attributions}, the initial tools in the explainability toolkit were primarily attribution methods~\cite{simonyan2014deep,smilkov2017smoothgrad,selvaraju2017gradcam,fel2021sobol,novello2022making,sundararajan2017axiomatic,zeiler2014visualizing,shrikumar2017learning,Fong_2017,graziani2021sharpening}. We also seen in ~\autoref{sec:attributions:metapred} that those approaches only offer a partial understanding of the learned decision processes as they aim to identify the location of the most discriminative features in an image, the ``\where'', leaving open the ``\what'' question, \textit{i.e.} the semantic meaning of those features.


Feature visualization methods, which aim to bridge this gap, involve formulating and solving an optimization problem to identify an input image that maximizes the activation of a specific target element (be it a neuron, layer, or the entire model)~\cite{zeiler2014visualizing}. Most of the approaches developed in the field fall along a spectrum based on how strongly they regularize the model. At one end of the spectrum, if no regularization is used, the optimization process can search the whole image space, but this tends to produce noisy images and nonsensical high-frequency patterns~\cite{erhan2009visualizing}. To circumvent this issue, researchers have proposed to penalize high-frequency in  the resulting images -- either by reducing the variance between neighboring pixels~\cite{mahendran2015understanding}, by imposing constraints on the image's total variation~\cite{nguyen2016synthesizing,nguyen2017plug,simonyan2014deep}, or by blurring the image at each optimization step~\cite{nguyen2015deep}. However, in addition to rendering images of debatable validity, these approaches also suppress genuine, interesting high-frequency features, including edges. To mitigate this issue, a bilateral filter may be used instead of blurring, as it has been shown to preserve edges and improve the overall result~\cite{tyka2016class}. Other studies have described a similar technique to decrease high frequencies by operating directly on the gradient, with the goal of preventing their accumulation in the resulting visualization~\cite{AudunGoogleNet}. One advantage of reducing high frequencies present in the gradient, as opposed to the visualization itself, is that it resists the amplification of high frequencies while still allowing them to manifest when consistently promoted by the gradient.
This process, known as "preconditioning" in optimization, can greatly simplify the optimization problem. The Fourier transform has been shown to be a successful preconditioner as it forces the optimization to be performed in a decorrelated and whitened image space~\cite{olah2017feature}. 

The emergence of high-frequency patterns in the absence of regularization is associated with a lack of robustness and sensitivity of the neural network to adversarial examples~\cite{szegedy2013intriguing}, and consequently, these patterns are less often observed in adversarially robust models~\cite{engstrom2019adversarial, santurkar2019image, tsipras2018robustness}. An alternative strategy to promote robustness involves enforcing small perturbations, such as jittering, rotating, or scaling, in the visualization process~\cite{mordvintsev2015inceptionism}, which, when combined with a frequency penalty~\cite{olah2017feature}, has been proved to greatly enhance the generated images.

Unfortunately, previous methods in the field of feature visualization have been limited in their ability to generate visualizations for newer architectures beyond VGG, resulting in a lack of interpretable visualizations for larger networks like ResNets~\cite{olah2017feature}. Consequently, researchers have shifted their focus to approaches that leverage statistically learned priors to produce highly realistic visualizations. One such approach involves training a generator, like a GAN~\cite{nguyen2016synthesizing} or an autoencoder~\cite{wang2022traditional, nguyen2017plug}, to map points from a latent space to realistic examples and optimizing within that space. Alternatively, a prior can be learned to provide the gradient (w.r.t the input) of the probability and optimize both the prior and the objective jointly~\cite{nguyen2017plug, tyka2016class}. Another method involves approximating a generative model prior by penalizing the distance between output patches and the nearest patches retrieved from a database of image patches collected from the training data~\cite{wei2015understanding}.
Although it is well-established that learning an image prior produces realistic visualizations, it is difficult to distinguish between the contributions of the generative models and that of the neural network under study. Hence, in this work, we focus on the development of visualization methods that rely on minimal priors to yield the least biased visualizations.


Our proposed approach, called MAgnitude Constrained Optimization (\magfv), builds on the seminal work by Olah et al. We propose a straightforward re-parametrization that essentially relies on exploiting the phase/magnitude decomposition of the Fourier spectrum, to exclusively optimizing the image's phase while keeping its magnitude constant.
Such a constraint is motivated by psychophysics experiments that have shown that humans are more sensitive to differences in phase than in magnitude~\cite{oppenheim1981importance,caelli1982visual,guyader2004image,joubert2009rapid, gladilin2015role}. Our contributions are threefold:

\begin{enumerate}[label=(\textit{\textbf{\roman*}})]

\item{We unlock feature visualizations for large modern CNNs without resorting to any strong parametric image prior (see Figure~\ref{fig:maco:logits_fail}).}

\item{We describe how to leverage the gradients obtained throughout our optimization process to combine feature visualization with attribution methods, thereby explaining both ``\what'' activates a neuron and ``\where'' it is located in an image.}

\item{We introduce new metrics to compare the feature visualizations produced with \magfv~to those generated with other methods.}
\end{enumerate}
As an application of our approach, we propose feature visualizations for FlexViT \cite{beyer2022flexivit} and ViT \cite{Dosovitskiy2021-zy} (logits and intermediate layers;  see Figure~\ref{fig:maco:logits_and_internal}).  We also employ our approach on a feature inversion task to generate images that yield the same activations as target images to better understand what information is getting propagated through the network and which parts of the image are getting discarded by the model (on ViT, see Figure~\ref{fig:maco:inversion}).
Finally, we will make a link with our work introduced in \autoref{sec:concepts:craft} and show how to combine our work with \craft (see Figure~\ref{fig:maco:concepts}). As feature visualization can be used to optimize in directions in the network's representation space, we employ \magfv~to generate concept visualizations, thus allowing us to improve the human interpretability of concepts and reducing the risk of confirmation bias. 

\subsection{Magnitude-Constrained Feature Visualization}




\paragraph{Notations}

Throughout, we consider a general supervised learning setting, with an input space $\sx \subseteq \Real^{h \times w}$, an output space $\sy \subseteq \Real^c$, and a classifier $\f : \sx \to \sy$ that maps inputs $\vx \in \sx$ to a prediction $\v{y} \in \sy$.
Without loss of generality, we assume that $\f$ admits a series of $L$ intermediate spaces $\s{A}_\ell \subseteq \Real^{p_\ell}, 1 < \ell < L$.
In this setup, $\f_\ell : \sx \to \s{A}_\ell$ maps an input to an intermediate activation $\v{v} = (v_1, \ldots, v_{p_\ell})^\intercal \in \s{A}_\ell$ of $\f$.
We respectively denote $\fourier$ and $\fourier^{-1}$ as the 2-D Discrete Fourier Transform (DFT) on $\sx$ and its inverse.








\paragraph{Optimization Criterion.}
The primary goal of a feature visualization method is to produce an image $\vx^\star$ that maximizes a given criterion $\mathcal{L}_{\v{v}}(\vx) \in \Real$; usually some value aggregated over a subset of weights in a neural network $\f$ (neurons, channels, layers, logits).
A concrete example consists in finding a natural "prototypical" image $\vx^\star$ of a class $k \in \llbracket 1, K \rrbracket$ without using a dataset or generative models.
However, optimizing in the pixel space $\Real^{W \times H}$ is known to produce noisy, adversarial-like $\vx^\star$. Therefore, the optimization is constrained using a regularizer $\Omega: \sx \to \Real^+$ to penalize unrealistic images:
\begin{equation}
\vx^\star = \argmax_{\vx \in \sx} \mathcal{L}_{\v{v}}(\vx) - \lambda \Omega(\vx).
\label{eq:maco:general}
\end{equation}
In Eq.~\ref{eq:maco:general}, $\lambda$ is a hyperparameter used to balance the main optimization criterion $\mathcal{L}_{\v{v}}$ and the regularizer $\Omega(\cdot)$. Finding a regularizer that perfectly matches the structure of natural images is hard, so  proxies have to be used instead. Previous studies have explored various forms of regularization spanning from total variation, $\ell_1$, or $\ell_2$ loss~\cite{nguyen2016synthesizing,nguyen2017plug,simonyan2014deep}. More successful attempts rely on the reparametrization of the optimization problem in the Fourier domain rather than on regularization.


\subsubsection{A Fourier perspective}

Mordvintsev et al.~\cite{mordvintsev2018differentiable} noted in their seminal work that one could use differentiable image parametrizations to facilitate the maximization of $\mathcal{L}_{\v{v}}$. Olah et al.~\cite{olah2017feature} proposed to re-parametrize the images using their Fourier spectrum. Such a parametrization allows amplifying the low frequencies using a scalar $\v{w}$. Formally, the prototypal image $\vx^\star$ can be written as $\vx^\star = \fourier^{-1}(\v{z}^\star \odot \v{w})$ with:

$$ \v{z}^\star = \argmax_{\v{z} \in \mathbb{C}^{W \times H}} \mathcal{L}_{\v{v}}(\fourier^{-1}(\v{z} \odot \v{w})).$$

Finding $\vx^\star$ boils down to optimizing a Fourier buffer
$\v{z} = \bm{a} + i \bm{b}$ together with boosting the low-frequency components and then recovering the final image by inverting the optimized Fourier buffer using inverse Fourier transform.

\begin{figure}
\centering
\includegraphics[width=0.9\textwidth]{assets/maco/leakage.jpg};
\caption{\textbf{Comparison between Fourier FV and natural image power spectrum.} In \textbf{(left)}, the power spectrum is averaged over $10$ different logits visualizations for each of the $1000$ classes of ImageNet. The visualizations are obtained using the \textbf{Fourier FV}Fourier FV method to maximize the logits of a ViT network~\citep{olah2017feature}. In \textbf{(right)} the spectrum is averaged over all training images of the ImageNet dataset.}
\label{fig:maco:leakage}
\end{figure}






However, multiple studies have shown that the resulting images are not sufficiently robust, in the sense that a small change in the image can cause the criterion $ \mathcal{L}_{\v{v}}$ to drop. Therefore, it is common to see robustness transformations applied to candidate images throughout the optimization process. In other words, the goal is to ensure that the generated image satisfies the criterion even if it is rotated by a few degrees or jittered by a few pixels. Formally, given a set of possible transformation functions -- sometimes called augmentations -- that we denote $\mathcal{T}$ such that for any transformation $\augmentation \sim \mathcal{T}$, we have $\augmentation(\vx) \in \sx$, the optimization becomes:

$$ 
\v{z}^\star = \argmax_{\v{z} \in \mathbb{C}^{W \times H}}
\mathbb{E}_{\augmentation \sim \mathcal{T}}(\mathcal{L}_{\v{v}}((\augmentation \circ \fourier^{-1})(\v{z} \odot \v{w})).
$$


Empirically, it is common knowledge that the deeper the models are, the more transformations are needed and the greater their magnitudes should be. To make their approach work on models like VGG, Olah et al.~\cite{olah2017feature} used no less than a dozen transformations. However, this method fails for modern architectures, no matter how many transformations are applied. We argue that this may come from the low-frequency scalar (or booster) no longer working with models that are too deep. For such models, high frequencies eventually come through, polluting the resulting images with high-frequency content -- making them impossible to interpret by humans. %
To empirically illustrate this phenomenon, we compute the $k$ logit visualizations obtained by maximizing each of the logits corresponding to the $k$ classes of a ViT using the parameterization used by Olah et al.~ In Figure~\ref{fig:maco:leakage} (left), we show the average of the spectrum of these generated visualizations over all classes: $\frac{1}{k} \sum_{i=1}^k |\fourier(\vx^\star_i)|$. We compare it with the average spectrum of images on the ImageNet dataset (denoted $\mathcal{D}$): $\mathbb{E}_{\vx \sim \mathcal{D}}(|\fourier(\vx)|)$ (Figure~\ref{fig:maco:leakage}, right panel).
We observe that the images obtained through optimization put much more energy into high frequencies compared to natural images. Note that we did not observe this phenomenon in older models such as LeNet or VGG.

In the following section, we introduce our method named~\magfv, which is motivated by this observation. We constrain the magnitude of the visualization to a natural value, enabling natural visualization for any contemporary model, and reducing the number of required transformations to only two.


\subsubsection{\magfv: from Regularization to Constraint}
\begin{figure}[t!]
\center
\includegraphics[width=1\textwidth]{assets/maco/method.pdf}
\caption{\textbf{Overview of the approach:} \textbf{(a)}  Current Fourier parameterization approaches optimize the entire spectrum (yellow arrow). \textbf{(b)}  In contrast,  the optimization flow in our approach (green arrows) goes from the network activation ($\v{y}$) to the phase of the spectrum ($\v{\varphi}$) of the input image ($\vx$).}

\label{fig:maco:method}
\end{figure}

Parameterizing the image in the Fourier space makes it possible to directly manipulate the image in the frequency domain. We propose to take a step further and decompose the Fourier spectrum $\v{z}$ into its polar form $\v{z} = \v{r} e^{i \v{\varphi}}$ instead of its cartesian form $\v{z} = \bm{a} + i \bm{b}$, which allows us to disentangle the magnitude ($\v{r}$) and the phase ($\v{\varphi}$).

It is known that human recognition of objects in images is driven not by magnitude but by phase~\cite{oppenheim1981importance,caelli1982visual,guyader2004image,joubert2009rapid, gladilin2015role}. Motivated by this, we propose to optimize the phase of the Fourier spectrum while fixing its magnitude to a typical value of a natural image (with few high frequencies). In particular, the magnitude is kept constant at the average magnitude computed over a set of natural images (such as ImageNet), so $\v{r} = \mathbb{E}_{\vx \sim \mathcal{D}}(|\fourier(\vx)|)$. Note that this spectrum needs to be calculated only once and can be used at will for other tasks.

\begin{figure}[ht]
\section{The general case: Proof of \texorpdfstring{\Cref{thm:main-decomp}}{Theorem 1.6}}\label{sec:algo}

First, we show that data structure of \Cref{l:max_min_query} can be used to compute distances witnessed by shortest paths that pass through a constant-size separator.

\begin{lemma}\label{l:single_adhesion}
Fix a constant $k \in \mathbb{N}$. There exists an algorithm which as the input receives an edge-weighted graph $G$ on $n$ vertices and $m$ edges together with a partition of its vertices into three sets $A, B, C$ such that $|B| \leq k$ and there are no edges between $A$ and $C$, and as the output computes $\max_{c \in C} \dist(a, c)$ for every $a \in A$. The running time is $\Oh(m \log n + n \log^{k - 1} n)$.
\end{lemma}

\begin{proof}
Let $B = \{b_1, \ldots, b_k\}$. For any $a \in A, c \in C$, we have $\dist(a, c) = \min_{i \in [k]} \dist(a, b_i) + \dist(c, b_i)$. First, we run Dijkstra's algorithm from every vertex in $B$ to find $\dist(v, b_i)$ for every $v \in V(G)$ and $i \in [k]$. Next, we use \Cref{l:max_min_query} to construct a data structure $\mathbb{D}$ for the point set $\{(\dist(c, b_1), \dots, \dist(c, b_k))\colon c\in C\}\subseteq \mathbb{R}^k$. Now, the value $\max_{c \in C} \dist(a, c)$ for any given $a$ is equal to the answer of $\mathbb{D}$ to the query with argument $(\dist(a, b_1), \dots, \dist(a, b_k))$.
\end{proof}

After computing the distances over a constant-size separator, we will use the following observation to simplify one of the sides of the separation.

\begin{lemma}\label{l:inserting_paths}
Let $G$ be a edge-weighted connected graph and let $A, B, C$ be a partition of its vertices such that there are no edges between $A$ and $C$. For every pair of vertices $u, v \in B$, let $P_{u, v}$ be any shortest path from $u$ to $v$ with all internal vertices in $C$ (assuming such a path exists).

Let $G'$ denote a graph obtained from $G[A \cup B]$ by adding an edge from $u$ to $v$ of weight equal to the length of $P_{u, v}$, for all $u, v \in B$ for which $P_{u, v}$ exists. Then,  $$\dist_G(s, t) = \dist_{G'}(s, t)\qquad\textrm{for all }s,t\in A\cup B.$$
\end{lemma}
\begin{proof}
Let $G''$ be the graph obtained by adding new edges of $G'$ to $G$.
Fix any $s, t \in A \cup B$ and let $P$ denote the shortest path from $s$ to $t$ in $G''$ which minimizes the number of vertices from $C$ visited. Naturally, the weight of $P$ is equal $\dist_G(s, t)$. Assume that such path visits at least one vertex of $C$. Then, the path $P$ is of the form $s \xrightarrow{P_1} x \xrightarrow{P_2} y \xrightarrow{P_3} t$, where $x, y \in B$ and all the internal vertices of $P_2$ are in $C$. By the construction of $G'$, $P_2$ can be replaced with a direct edge from $x$ to $y$ of the same weight. We obtain a same weight path with a smaller number of vertices of $C$ visited, which is a contradiction. Therefore, $P$ is entirely contained in $A \cup B$, hence it exists in $G'$. This shows that $\dist_G(s, t) = \dist_{G'}(s, t)$.
\end{proof}


The next lemma encapsulates the main algorithmic content of the proof of \Cref{thm:main-decomp}. The algorithm will split the tree decomposition provided on input into smaller parts for which the eccentricities are easier to calculate. We use the following lemma to handle a single such part.
\begin{lemma}\label{l:star}
Fix constants $k, g \in \mathbb{N}, 0 < \delta < \frac{1}{54}$. Assume we are given $n \in \mathbb{N}$, an edge-weighted graph $G$ on at most $n$ vertices with a weight function $w \colon E(G) \to \mathbb{N}$, a vertex subset $A$ and a collection of non-empty vertex subsets $V_0, V_1, \dots, V_\ell$ satisfying the following conditions:
\begin{itemize}[nosep]
	\item The sum of weights of all the edges in $G$ is bounded by $\Oh(n)$.
	\item $V(G) \setminus A = V_0 \cup V_1 \cup \dots \cup V_\ell$.
	\item $|A| \leq k$.
	\item For every $i \in [\ell]$, $G[V_i \setminus V_0]$ is connected, $N_G(V_i \setminus V_0) = V_i \cap V_0$, $|V_i| = \Oh(n^\delta)$, and $|V_0 \cap V_i| \leq 4$.
	\item For all $i, j \in [\ell], i \neq j$, $V_i \setminus V_0$ and $V_j \setminus V_0$ are disjoint and non-adjacent in $G$.
	\item Every edge $uv \in E(G)$ with $u, v \not\in A$ is contained in $G[V_i]$ for some $i\in \{0,1,\ldots,\ell\}$.
	\item The graph obtained by taking $G[V_0]$ and adding a clique on $V_0 \cap V_i$ for every $i \in [\ell]$ has Euler genus bounded by $g$.
\end{itemize}
Then, we can compute the eccentricity of every vertex of $G$ in time $\Oh \left( n^{1 + \frac{150 + 54 \delta}{151}} \log^k n \right)$.
\end{lemma}

\begin{proof}
Fix $\delta' = \frac{1 + 97 \delta}{151}$; we have $\delta' - \delta = \frac{1 - 54\delta}{151} > 0$.
Let $E_i$ denote the set of edges with one endpoint in $V_i$ and the other endpoint in $V_i \setminus V_0$. For $i \in [\ell]$, we shall say that $V_i$ is {\em{heavy}} if the sum of weights of $E_i$ is larger than $n^{\delta'}$. Since the sets $E_i$ are pairwise disjoint and the total sum of weights of all the edges is bounded by $\Oh(n)$, the number of heavy subsets is bounded by $\Oh(n^{1 - \delta'})$. Without loss of generality, we may assume that $V_{\ell' + 1}, \dots, V_\ell$ are heavy and $V_1, \dots, V_{\ell'}$ are not, for some $\ell'\in \{0,\ldots,\ell\}$.


For any source vertex $s$, we can calculate distances from $s$ to every vertex of $G$  using breadth first search in time $\Oh(\sum_{e \in E(G)} w(e)) = \Oh(n)$.
In particular, for every $\ell' < i \leq \ell$, we can compute the distances from every vertex of $V_i$ to every vertex of $G$ in total time $\Oh(n^{2 - \delta' + \delta})$, because $$|V_{\ell'+1}\cup \ldots\cup V_{\ell}|\leq n^{1-\delta'}\cdot \Oh(n^\delta)=\Oh(n^{1-\delta'+
\delta}).$$
Additionally, we calculate distances $\dist_G(a, v)$ for every $a \in A, v \in V(G)$ in time $O(n)$.

For every $i \in [\ell]$ and $u,v \in V_0 \cap V_i$, there exists a shortest path $P_{i,u,v}$ from $u$ to $v$ with all internal vertices belonging to $V_i - V_0$ due to the assumption that $G[V_i - V_0]$ is connected and $N_G(V_i - V_0) = V_i \cap V_0$. Therefore, the distance from $u$ to $v$ is bounded by the sum of weights of edges in $E_i$. In particular, for $i \in [\ell']$, $\dist_G(u, v) \leq n^{\delta'}$.

We define $\widetilde{G}$ to be the graph obtained by taking $G[A \cup V_0 \cup \dots \cup V_{\ell'}]$ and applying the following operation for every $i \in \{\ell' + 1, \dots, \ell\}$:
for each pair of vertices $u, v \in A \cup (V_0 \cap V_i)$, add an edge in $\widetilde{G}$ between $u$ and $v$ with weight equal to the total weight of $P_{i,u,v}$. For a fixed $i, u$, we can find $P_{i, u, v}$ for all $v$ using breadth first search in time $\Oh(n)$. Taking a sum over all $i, u$, we get that $\tilde{G}$ can be computed in total time $\Oh(n^{2 - \delta'})$.


\begin{claim}\label{cl:wG}
The sum of the edge weights in $\widetilde{G}$ is $\Oh(n)$. Moreover, for all $u, v \in V(\widetilde{G})$, we have $\dist_{\widetilde{G}}(u, v) = \dist_{G}(u, v)$.
\end{claim}

\begin{proof}
Consider $i \in \{\ell' + 1, \dots, \ell\}$ and any $u, v \in A \cup (V_0 \cap V_i)$ for which we added an edge. Its weight is bounded by the sum of weights of edges in $E_i$. Therefore, the total weight of all edges added is at most
$$
\sum_{i \in \{\ell' + 1, \dots, \ell\}} \left( |A \cup (V_0 \cap V_i)|^2 \sum_{e \in E_i} w(e) \right) \leq (4 + k)^2 \sum_{e \in E(G)} w(e) = \Oh(n).
$$
This proves the first part of the claim.

For the second part of the claim, consider any $i \in \{\ell' + 1, \dots, \ell \}$ and observe that by our assumptions, $A \cup (V_0 \cap V_i)$ separates $(V_0 \cup \dots \cup V_{\ell'} \cup V_{i + 1} \cup \dots \cup V_\ell) \setminus V_i$ from $V_i \setminus V_0$. Hence it suffices to repeatedly apply \Cref{l:inserting_paths}.
\end{proof}

For every $u \in V(\widetilde{G})$, we have $\ecc_G(u) = \max(\ecc_{\widetilde{G}}(v), \max_{v \in V(G) \setminus V(\widetilde{G})} \dist_G(u, v))$. Note, that we already know all the distances $\dist_G(u, v)$ for $v \in V(G) \setminus V(\widetilde{G})$. Similarly, we can already compute $\ecc_G(u)$ for every $u \in V(G) \setminus V(\widetilde{G})$. Therefore, it remains to compute $\ecc_{\widetilde{G}}(v)$ for each $v \in V(\widetilde{G})$. Our goal is to show that this can be done efficiently using \Cref{l:main_ecc}.

Now, let $G'$ be the graph obtained from $\tilde{G}$ by replacing every edge $e$ non-indicent to $A$ with $w(e)\geq 2$ with a path of length $w(e)$ consisting of unit-weight edges. This operation again preserves the distances. Since the sum of edge weights in $\tilde{G}$ is of $\Oh(n)$, the total number of vertices in $G'$ is of $\Oh(n)$. For $0 \leq i \leq \ell'$, we write $V'_i$ to denote the set $V_i$ together with all the vertices added as a part of a path between two endpoints in $V_i$.
As $V_i$ is not heavy for each $i\in [\ell']$, we have
$$
|V'_i \setminus V'_0| \leq |V_i| + \sum_{e \in E_i} w(e) = \Oh(n^{\delta'})\qquad \textrm{for all }i\in [\ell'].
$$

Let $G_0$ denote the graph $G'[V'_0]$ and let $G_0^*$ denote the graph $G'- A$ with $V'_i - V'_0$ contracted to a single vertex $v_i^*$, for each $i \in [\ell']$; note that, all edges of $G_0$ and $G_0^*$ have unit weight.

\begin{claim}
	The graph $G_0^*$ is does not contain $K_{t}$ as a minor, where $t = \Oh(\sqrt{g})$.
\end{claim}

\begin{proof}
Let $\bar{G}_0$ denote the graph obtained by taking $G_0$ and adding a clique on $V_0 \cap V_i$ for every $i \in [\ell']$.
By lemma assumptions and the fact that subdividing edges does not increase the Euler genus, $\bar{G}_0$ has Euler genus at most $g$. In particular, $\bar{G}_0$ is $K_{t'}$-minor-free for some $t' = \Oh(\sqrt{g})$, because the Euler genus of $K_{t'}$ is $\Omega({t'}^2)$.

Similarly, let $\bar{G}_0^*$ be the graph obtained by taking $G_0^*$ and adding a clique on each $V_0 \cap V_i$.
Note, that $\bar{G}_0^* - \{v_1^*, \dots, v_{\ell'}^*\}$ is precisely $\bar{G}_0$. Let $t = \max(t', 6)$.
Recall that a minor model of a clique $K_t$ consists of $t$ pairwise vertex-disjoint connected subgraphs, called
branch sets, such that there is at least one edge between each pair of the branch sets.
Consider a minor model $\varphi$ of $K_{t}$ in $\bar{G}^*_0$.
Note that $\varphi$ cannot contain any singleton branch set of the form $\{v^*_i\}$, for the degree of $v^*_i$ in $\bar{G}^*_0$ is at most $4 < t - 1$. Furthermore, since $N_{\bar{G}^*_0}(v^*_i) = V_0 \cap V_i$, any branch set containing $v^*_i$ and at least one other vertex contains some $u \in V_0 \cap V_i$, and $N_{\bar{G}^*_0}(v^*_i)\subseteq N_{\bar{G}^*_0}(u)$, hence removing $v^*_i$ from this branch set preserves the model. Therefore, we can assume without loss of generality that all branch sets of $\varphi$ are disjoint from $\{v^*_1, \dots, v^*_{\ell'}\}$, hence $\varphi$ is a minor model of $K_{t}$ in $\bar{G}_0$. This is a contradiction, as $t \geq t'$ and $\bar{G}_0$ is $K_{t'}$-minor-free. Therefore, $\bar{G}_0^*$ is $K_t$-minor-free, hence $G_0^*$ also.
\end{proof}

Let $\rho' = \frac{2 - 108 \delta}{151} > 0$. The graph $G^*_0$ is a unit-weight graph and is $K_{t}$-minor-free.
Hence, by applying \Cref{t:r_division} to $G^*_0$ (with $\varepsilon = \rho'/2$)
we obtain an $n^{\rho'}$-division $\mathcal{R}_0$ in time $\Oh(n^{1 + \rho'})$.
We extend it to $G' - A$ by mapping every contracted vertex $v^*_i$ to $N_{G' - A}[V'_i - V'_0] = (V'_i - V'_0) \cup (V_0 \cap V_i)$. Formally, we put $V''_i \coloneqq N_{G' - A}[V'_i - V'_0]$ and 
$$
\mathcal{R} \coloneqq \left\{ (R_0 \cap V'_0) \cup \bigcup_{i \colon v^*_i \in R_0} V''_i \colon R_0 \in \mathcal{R}_0 \right\}.
$$

Now, we argue that $\mathcal{R}$ is a reasonable division of $G' - A$. Clearly, all sets $R \in \mathcal{R}$ are connected in $G' - A$. Pick any $R \in \mathcal{R}$ and let $R_0$ be its corresponding set in $\mathcal{R}_0$.
Every vertex $v^*_i$ is mapped to a set of size $\Oh(n^{\delta'})$, therefore
$$|R| \leq |R_0| \cdot \Oh(n^{\delta'}) = \Oh(n^{\rho' + \delta'}).$$

By our construction, for every $i \in [\ell']$, $R$ is either disjoint from $V'_i - V'_0$ or contains whole $N_{G' - A}[V'_i - V'_0]$. This means that no vertex belonging to any $V'_i - V'_0$ can be in $\partial R$, hence $\partial R \subseteq V'_0$.

Pick any $u \in \partial R \cap R_0$. Assume that $u \not\in \partial R_0$. Then every vertex of $N_{G_0^*}(u)$ must be in $R_0$, hence $N_{G - A'}(u) \subseteq R$, which is a contradiction. This means that $\partial R \cap R_0 \subseteq \partial R_0$.

Pick any $u \in \partial R - R_0$. Then, $u \in V_0 \cap V_i$ for some $i \in [\ell']$ such that $v_i^* \in R_0$. Moreover, $v_i^* \in \partial R_0$ and is adjacent to $u$ in $G_0^*$. The number of such $u$ is bounded by $4 |\partial R_0 \cap \{ v_1^*, \dots, v_{\ell'}^* \}|$.

Putting two cases together, we obtain:
$$
\sum_{R \in \mathcal{R}} |\partial R| = \sum_{R \in \mathcal{R}} \left(|\partial R \cap R_0| + |\partial R - R_0|\right) \leq \sum_{R_0 \in \mathcal{R}_0} \left(|\partial R_0| + 4 |\partial R_0 \cap \{ v_1^*, \dots, v_{\ell'}^* \}|\right) = \Oh(n^{1 - \frac{1}{2}\rho'}).
$$

It remains to show the following claim.

\begin{claim}
Pick any $R \in \mathcal{R}, s_R \in R$. The number of different distance profiles on $R$ relative to $s_R$ in $G' - A$ is of $\Oh(n^{48\rho' + 54\delta'})$.
\end{claim}
\begin{proof}
We look at every vertex $v \in V(G') \setminus A$ and consider three cases: $v \in R$, $v \in V'_0$, and $v \in V'_i \setminus (V'_0 \cup R)$ for some $i \in [\ell']$. By our construction, $R \cap V'_0$ is non-empty, hence w.l.o.g. we can assume that $s_R \in V'_0$ as whether two vertices have the same profile on $R$ is independent of the choice of the pivot vertex.

In the first case, there are at most $|R| = \Oh(n^{\rho' + \delta'})$ such vertices, hence they realise at most that many profiles.

In the second case, we want to observe that profile of any vertex $v \in V'_0$ on $R$ depends only on its profile on $R \cap V'_0$ (relative to $s_R$). Pick any $t \in R - V'_0$. Then $t \in V'_i - V'_0$ for some $i \in [\ell']$, $V_i \cap V_0 \subseteq R \cap V'_0$, and every path from $v$ to $t$ intersects $V_i \cap V_0$. In particular, distances from $v$ to vertices of $V_i \cap V_0$ determine its distance to $t$, which proves the observation.

Let $\tilde{G}_0$ denote the graph obtained by taking $G'[V'_0]$ and for every $i \in [\ell'], u, v \in V_0 \cap V_i$ adding a disjoint path from $u$ to $v$ of length $\dist(u, v)$. Let $P_i$ denote the vertex set of paths added between $V_0 \cap V_i$. For every $t \in V'_0$ we have $\dist_{G' - A}(v, t) = \dist_{\tilde{G}_0}(v, t)$, so it suffices to bound the number of profiles on $R \cap V'_0$ in $\tilde{G}_0$. By our assumptions, $\tilde{G}_0$ has Euler genus bounded by $g$ and all $P_i$ are of size $\Oh(n^{\delta'})$.

Let $R_0$ be the set of $\mathcal{R}_0$ corresponding to $R$. Let $\tilde{R}_0$ denote the set $(R \cap V'_0) \cup \bigcup_{i : v^*_i \in R_0} P_i$. Such set is connected in $\tilde{G}_0$. Moreover, similarly to $R$, its size is $\Oh(n^{\rho' + \delta'})$. Applying \Cref{thm:distprofiles}, we get that the number of distance profiles on $\tilde{R}_0$ in $\tilde{G}_0$ is $\Oh(n^{12(\rho' + \delta')})$, which also bounds the number of profiles on $R$ in $G' - A$ realised by $V'_0$.

For the third case, assume $v \in V'_i \setminus (V'_0 \cup R)$ for some $i\in [\ell']$. Every path from $v$ to any vertex of $R$ in $G' - A$ intersects $V_i \cap V_0$. Let $v_1, \dots v_p$ be the vertices of $V_i \cap V_0$, where $p \leq 4$. The profile of $v$ on $R$ is then determined by the following:
\begin{itemize}[nosep]
 \item[(a)] the profile of each $v_j$ on $R$,
 \item[(b)] $\dist_{G' - A}(v, v_j) - \dist_{G' - A}(v, v_1)$ for each $2 \leq j \leq p$, and
 \item[(c)] $\dist_{G' - A}(s_R, v_j) - \dist_{G' - A}(s_R, v_1)$ for each $2 \leq j \leq p$ where $s_R$ is some pivot vertex of $R$.
\end{itemize}
By the previous case, the number of distance profiles of each $v_j$ is $\Oh(n^{12(\rho' + \delta')})$. The distances between $v$ and $v_j$ are bounded by $|V'_i|$, hence each quantity described in (b) can take $\Oh(n^{\delta'})$ different possible values. Similarly, since $v_1$ and $v_j$ are connected via $V'_i$, $|\dist_{G' - A}(s_R, v_j) - \dist_{G' - A}(s_R, v_1)| \leq \Oh(n^{\delta'})$. The number of different possible profiles of such $v$ is therefore bounded by $\Oh(n^{48(\rho' + \delta') + 6\delta'}) = \Oh(n^{48\rho' + 54\delta'})$. This finishes the proof of the claim.
\end{proof}

Now we can apply \Cref{l:main_ecc} to graph $G'$ with apex set $A$, $X = V(\widetilde{G})$, and the following constants: $$\rho = \rho' + \delta',\qquad \gamma = 1 - \frac{1}{2}\rho',\quad \textrm{and}\quad \alpha = 48\rho' + 54 \delta'.$$ This allows us to calculate all $V(\widetilde{G})$-eccentricities in $G'$ in time
$$
\Oh \left( \left(
	n^{ 2 - \frac{1}{2} \rho' } +
	n^{ 1 + 48\rho' + 54 \delta' }
\right) \log^k n \right) =
\Oh \left( n^{1 + \frac{150 + 54 \delta}{151}} \log^k n \right).
$$
Since for each $v\in V(\widetilde{G})$ we have $\ecc_{\widetilde{G}}(v) = \max_{u \in V(\widetilde{G})} \dist_{\widetilde{G}}(v, u) = \max_{u \in V(\widetilde{G})} \dist_{G'}(v, u)$, this means that we have successfully computed all the eccentricities in $\widetilde{G}$; and as we argued, this is enough to compute all the eccentricities in $G$ as well.

Finally, the total running time of the algorithm is
$$
\Oh \left( n^{1 + \frac{150 + 54 \delta}{151}} \log^k n + n^{2 - \delta' + \delta} \right) =
\Oh \left( n^{1 + \frac{150 + 54 \delta}{151}} \log^k n \right).
$$\qedhere
\end{proof}


\begin{lemma}\label{l:star2}
Fix constants $k, g \in \mathbb{N}, 0 < \delta < \frac{1}{54}$. Assume we are given $n \in \mathbb{N}$, an edge-weighted graph $G$ on at most $n$ vertices with a weight function $w \colon E(G) \to \mathbb{N}$, a vertex subset $A$ and a collection of non-empty vertex subsets $V_0, V_1, \dots, V_\ell$ satisfying the same conditions as in \Cref{l:star} with the following differences:
\begin{itemize}
	\item we don't require $G[V_i - V_0]$ to be connected and $V_i - V_0$ to be adjacent to whole $V_i \cap V_0$;
	\item instead of $|V_0 \cap V_i| \leq 4$, we require $|V_0 \cap V_i| \leq k$.
\end{itemize}
Then, we can compute the eccentricity of every vertex of $G$ in time $\Oh \left( n^{1 + \frac{150 + 54 \delta}{151}} \log^{k + 5g} n \right)$.
\end{lemma}

\begin{proof}
We will reduce our input to one which will satisfy the conditions of \Cref{l:star}. We start by addressing the adhesions $V_0 \cap V_i$ containing too many vertices.

Let $G_0$ denote the graph $G[V_0]$ with cliques placed at $V_0 \cap V_i$ for every $i \in [\ell]$.
For every $i \in [\ell]$ we repeat the following procedure: while $|V_0 \cap V_i| > 4$,
remove arbitrary $5$ vertices from $V_0 \cap V_i$. Since $|V_0 \cap V_i| \leq k$ for each $i\in [\ell]$,
this procedure can be implemented in total time $\Oh(n)$. As a result, at the end we have $|V_0 \cap V_i| \leq 4$ for all $i \in [\ell]$. Let $M$ be the set of all the removed vertices. By our assumptions, $G_0$ has Euler genus bounded by $g$, hence it cannot contain $g + 1$ pairwise disjoint copies of $K_5$
(as the Euler genus of a graph is the sum of the Euler genera of its 2-connected components~\cite{StahlB77} and $K_5$ is not planar). Each removed quintiple of vertices induces a $K_5$ in $G_0$, hence we have $|M| \leq 5g$. We set $A' = A \cup M$ and may thus assume that $V_i$ is disjoint from $A'$ for all $0 \leq i \leq \ell$.

Now, fix $i \in [\ell]$. Let $C^i_1, \dots, C^i_{r_i}$ denote the connected components of $V_i - V_0$ in $G - A'$. We define $W^i_j := N_{G - A'}[C^i_j]$ for every $j \in [r_i]$. Clearly, all $W^i_j$ induce a connected subgraph of $G$ and satisfy $N_{G - A'}(W^i_j - V_0) = W^i_j \cap V_0$. We put $V'_0 := V_0$ and enumerate
$$
\{V'_1, V'_2, \dots V'_{\ell'}\} := \{ W^i_j \colon i \in [\ell], j \in [r_i] \}.
$$
It is easy to verify that the sets $A'$ and $V'_0, V'_1, \dots, V'_{\ell'}$ satisfy the conditions of \Cref{l:star}. We apply said lemma to calculate the eccentricity of every vertex of $G$ in the desired time.
\end{proof}



The next statement is a reformulation of \Cref{thm:main-decomp}.

\begin{theorem}
Fix constants $k, g \in \mathbb{N}$. Assume we are given a graph $G$ on $n$ vertices together with its tree decomposition $(T, \beta)$ and a set of private apices $A_t \subseteq \beta(t)$ for each node $t\in V(T)$ such that the following conditions hold:
\begin{itemize}[nosep]
 \item For every node $t \in V(T)$, we have $|A_t| \leq k$.
 \item For every edge $st \in E(T)$,  we have $|\beta(v) \cap \beta(u)|\leq k$.
 \item For every node $t \in V(T)$, graph obtained by taking $G[\beta(t)] - A_t$ and turning  $(\beta(t) \cap \beta(s))\setminus A_t$ into a clique for every edge $st \in E(T)$ has Euler genus bounded by $g$.
\end{itemize}
Then, we can compute the eccentricity of every vertex of $G$ in time $\Oh \left( n^{1 + \frac{355}{356}} \log^{k + 5g} n \right)$.
\end{theorem}

\begin{proof}
We may assume that $|V(T)|\leq n$, for every tree decomposition with no two bags comparable by inclusion has this property; and adjacent comparable bags can be merged by contracting the edge between them.

For a node $t\in V(T)$, by the {\em{weight}} of $t$ we mean the size of the corresponding bag, that is, $|\beta(t)|$. For any subset of nodes $S \subseteq V(T)$, we define $\beta(S) \coloneqq \bigcup_{t \in S} \beta(t)$ By the {\em{weight}} of $S$, we mean the total weight of the elements of $S$, that is, $\sum_{t\in S} |\beta(t)|$. 

\begin{claim}\label{cl:weight-T}
The weight of $V(T)$ is of $\Oh(n)$.
\end{claim}

\begin{proof}
The sets $\beta'(t) := \beta(t) - \bigcup_{s \in N_T(t)} \beta(s)$ are pairwise disjoint. We have
$$
\sum_{t \in V(T)} |\beta(t)| =
\sum_{t \in V(T)} |\beta'(t)| + 2 \cdot \sum_{st \in E(T)} |\beta(s) \cap \beta(t)| \leq
|V(T)| + 2k|E(T)| = \Oh(n).
$$
\end{proof}

Since every bag induces a graph of bounded Euler genus, the number of edges contained in a bag is linear in its size. In particular, this implies that the total number of edges of $G$ is also bounded by $\Oh(n)$.

We set $$\delta \coloneqq \frac{1}{356}\qquad\textrm{and}\qquad \Delta \coloneqq \frac{355}{356}.$$ Root the tree $T$ in an arbitrarily chosen node; this naturally imposes an ancestor-descendant relation in $T$ (for convenience, every node is considered its own ancestor and descendant).

We start by partitioning $T$ into connected subtrees using the following procedure.
We proceed bottom-up over $T$, processing nodes in any order so that a node is processed after all its strict descendants have been processed. Along the way, we mark some nodes and split the edges of $T$ into heavy and light. Let $t \in V(T)$ be the currently processed non-root node of $T$ and let $e \in E(T)$ be the edge connecting $t$ with its parent. If the total weight of all the unmarked nodes that are descendants of $t$ is at least $n^\delta$ (recall that this includes $t$ itself as well), then we declare $e$ heavy and mark all the descendants of $t$ that were unmarked so far. Otherwise, the edge $e$ is declared light and the procedure proceeds to further nodes of $T$.

Observe that
removing all heavy edges splits $T$ into connected subtrees, say $T'_1, \cdots T'_m$. All of the subtrees, except for possibly the subtree containing the root node, are of weight at least $n^\delta$. In particular, the number of subtrees $m$, and therefore the number of heavy edges, is  bounded by $\Oh(n^{1 - \delta})$. Moreover, in every subtree $T'_i$, removing the node closest to the root splits $T'_i$ into smaller components, each of weight less than $n^\delta$.

Fix a heavy edge $e$ and let $T^e_1$ and $T^e_2$ be the two subtrees into which $T$ splits after removing~$e$. Let $X^e_i = \beta(T^e_i)$ for $i \in \{1, 2\}$. Put $A_e = X^e_1 \setminus X^e_2$, $C_e = X^e_2 \setminus X^e_1$, and $B_e = X^e_1 \cap X^e_2$. By the properties of tree decompositions, such choice of $A_e, B_e, C_e$ satisfies the conditions of \Cref{l:single_adhesion}, hence in time $\Oh(n \log^{k - 1} n)$ we can compute $\max_{v \in X^e_2} \dist_G(u,v)$ for every $u \in X^e_1$, and $\max_{u \in X^e_1} \dist_G(u,v)$ for every $v \in X^e_2$. Computing this for every heavy edge $e$ takes total time $\Oh(n^{2 - \delta} \log^{k - 1} n)$.

Fix any subtree $T'=T'_j$. Let $e_1 = t^{e_1}_1t^{e_1}_2, e_2 = t^{e_2}_1 t^{e_2}_2, \dots, e_\ell = t^{e_\ell}_1 t^{e_\ell}_2$ denote the heavy edges incident to $T'$, where $t^{e_i}_1 \in V(T')$ and $V(T') \subseteq V(T_1^{e_i})$ for every $i \in [\ell]$.
For a vertex $v \in \beta(T')$, let
$$d_0(v) = \max_{u \in \beta(T')} \dist_G(v, u)\qquad\textrm{and}\qquad d_i(v) = \max_{u \in X_2^{e_i}}\dist_G(v,u),\quad\textrm{for } i \in [\ell].$$ We have $\ecc(v) = \max \{ d_i(v)\colon i\in \{0,1,\ldots,\ell\}\}$.The values of $d_i(v)$ are already calculated for all $i\in [\ell]$, hence it remains to compute $d_0(v)$.

For every $i \in [\ell]$ and every pair of vertices $u, v \in \beta(t^{e_i}_1) \cap \beta(t^{e_i}_2)$ we find a shortest path between $u$ and $v$ with all internal vertices inside $X^{e_i}_2$ (or determine that it doesn't exist). For a fixed $u, v$ this can be done in time $\Oh(n)$. Since in total we perform this step at most $2k^2$ times per heavy edge, it takes $\Oh(n^{2 - \delta})$ time in total. Let $P_{i, u, v}$ denote such path, assuming it exists.

Let $G'$ denote the graph obtained from $G[\beta(T')]$ by taking every $i, u, v$ for which $P_{i, u, v}$ exists and adding an edge between $u$ and $v$ of weight equal to the total weight of $P_{i, u, v}$.
The weight of every edge inserted in $\beta(t^{e_i}_1) \cap \beta(t^{e_i}_2)$ is bounded by $|X^{e_i}_2|+1$. The total weight of all edges inserted is therefore at most
$$
\sum_{i \in [\ell]} |\beta(t^{e_i}_1) \cap \beta(t^{e_i}_2)|^2 \cdot (|X^{e_i}_2|+1) \leq
k^2 \sum_{i \in [\ell]} (|X^{e_i}_2|+1) = \Oh(n),
$$
where the last equality follows from the fact that all the trees $T^{e_i}_2$ are pairwise disjoint.
By \Cref{l:inserting_paths}, we have $\dist_{G'}(u, v) = \dist_G(u, v)$ for each $u, v \in \beta(T')$. Hence, computing $d_0(v)$ for every $v \in \beta(T')$ is equivalent to computing the eccentricity of every vertex in $G'$.

If the size of $\beta(T')$ is smaller than $n^\Delta$, we compute the eccentricities naively in time $\Oh(|\beta(T')|^2)$, 
noting that $G'$ has $\Oh(|\beta(T')|)$ edges (thanks to Claim~\ref{cl:weight-T} and bounded genus assumption 
of the last bullet of the theorem statement). Otherwise, we argue that we can use the algorithm in \Cref{l:star} as follows.

Let $t$ be the node of $T'$ closest to the root. Let $s_1, \dots, s_p$ be the children of $t$ in $T$ and let $T''_i$ denote the connected component of $T' - \{t\}$ containing $s_i$. Set $V_0 = \beta(t)$ and $V_i = \beta(T''_i)$ for $i \in [p]$.

It is now easy to verify that $G'$ and sets $A, \{V_i\colon 0\leq i\leq p\}$ selected this way satisfy the assumptions of \Cref{l:star2}. This allows us to use it to compute the eccentricities in $G'$ in time
$$
\Oh \left( n^{1 + \frac{150 + 54\delta}{151}} \log^{k + 5g} n \right) =
\Oh \left( n^{1 + \frac{354}{356}} \log^{k + 5g} n \right).
$$
As we argued, from these eccentricities, we may easily compute all the eccentricities in $G$.

Now, let us analyse the total running time of the whole algorithm. We invoke \Cref{l:star} $\Oh(n^{1 - \Delta})$ times, since we apply it only to subtrees $T'_i$ of size at least $n^\Delta$. The total running time of those applications is hence
$$
\Oh \left( n^{2 + \frac{354}{356} - \Delta} \log^{k + 5g} n \right) =
\Oh \left( n^{1 + \frac{355}{356}} \log^{k + 5g} n \right).
$$
We compute the eccentricities naively for subtrees smaller than $n^\Delta$, hence the total running time of this computation is
$$
\sum_{i \in [m] \colon |\beta(T'_i)| \leq n^\Delta} |\beta(T'_i)|^2 \leq
n^\Delta \cdot \sum_{i \in m} |\beta(T'_i)| = \Oh(n^{1 + \Delta})=\Oh\left(n^{1+\frac{355}{356}}\right).
$$
The rest of computation can be done in $\Oh(n^{2 - \delta} \log^k n)$. Therefore, the whole algorithm runs in time $\Oh \left( n^{1 + \frac{355}{356}} \log^{k + 5g} n \right)$.
\end{proof}

\end{figure}


Therefore, our method does not backpropagate through the entire Fourier spectrum but only through the phase (Figure~\ref{fig:maco:method}), thus reducing the number of parameters to optimize by half. Since the magnitude of our spectrum is constrained, we no longer need hyperparameters such as $\lambda$ or scaling factors, and the generated image at each step is naturally plausible in the frequency domain.
We also enhance the quality of our visualizations via two data augmentations: random crop and additive uniform noise.
To the best of our knowledge, our approach is the first to completely alleviate the need for explicit regularization -- using instead a hard constraint on the solution of the optimization problem for feature visualization.
To summarize, we formally introduce our method:

\begin{definition}[\textbf{\magfv}]
The feature visualization results from optimizing the parameter vector $\v{\varphi}$  such that:
$$
\v{\varphi}^\star = \argmax_{\v{\varphi} \in \Real^{W \times H}}
\mathbb{E}_{\augmentation \sim \mathcal{T}}(\mathcal{L}_{\v{v}}((\augmentation \circ \fourier^{-1})(\v{r} e^{i \v{\varphi}})) ~~~\text{where}~~~ \v{r} = \mathbb{E}_{\vx \sim \mathcal{D}}(|\fourier(\vx)|)
$$
The feature visualization is then obtained by applying the inverse Fourier transform to the optimal complex-valued spectrum: $\vx^\star = \fourier^{-1}((\v{r} e^{i \v{\varphi}^\star})$
\end{definition}









\paragraph{Transparency for free:}\label{sec:maco:transparency}
Visualizations often suffer from repeated patterns or unimportant elements in the generated images. This can lead to readability problems or confirmation biases~\cite{borowski2020exemplary}. It is important to ensure that the user is looking at what is truly important in the feature visualization. The concept of transparency, introduced in \cite{mordvintsev2018differentiable}, addresses this issue but induces additional implementation efforts and computational costs.

We propose an effective approach, which leverages attribution methods -- specifically a variant of Smoothgrad seen in \autoref{chap:attributions}) -- that yields a transparency map $\v{\alpha}$ for the associated feature visualization without any additional cost. Our solution takes advantage of the fact that during backpropagation, we can obtain the intermediate gradients on the input $\partial \mathcal{L}_{\v{v}}( \vx) / \partial \vx$ for free as $\frac{\partial \mathcal{L}_{\v{v}}( \vx)}{\partial \v{\varphi}} =  \frac{\partial \mathcal{L}_{\v{v}}( \vx)}{\partial \vx} \frac{\partial \vx}{\partial \v{\varphi}}$. We store these gradients throughout the optimization process and then average them, as done in SmoothGrad, to identify the areas that have been modified/attended to by the model the most during the optimization process. We note that a similar technique has recently been used to explain diffusion models \cite{boutin2023diffusion}. In Algorithm \ref{alg:maco:cap}, we provide pseudo-code for \magfv~and an example of the transparency maps in Figure~\ref{fig:maco:inversion} (third column).




\begin{figure}
    \centering
    \includegraphics[width=0.98\textwidth]{assets/maco/qualitative_internal.jpg}
    \caption{\textbf{(left) Logits and (right) internal representations of FlexiViT.}  \magfv~was used to maximize the activations of \textbf{(left)} logit units and \textbf{(right)} specific channels located in different blocks of the FlexViT (blocks 1, 2, 6 and 10 from left to right).}
    \label{fig:maco:logits_and_internal}
\end{figure}





\subsection{Evaluation}
\label{section:maco:evaluation}
We now describe and compute three different scores to compare the different feature visualization methods: Fourier (Olah et al.), CBR (optimization in the pixel space), and \magfv~(ours). It is important to note that these scores are only applicable to output logit visualizations. We will then demonstrate how we can use our method to perform concept visualization. %
To keep a fair comparison, we restrict the benchmark to methods that do not rely on any learned image priors. Indeed, methods with learned prior will inevitably yield lower FID scores (and lower plausibility score) as the prior forces the generated visualizations to lie on the manifold of natural images.




\paragraph*{Plausibility score.} We consider a feature visualization plausible when it is similar to the distribution of images belonging to the class it represents.
We quantify the plausibility through an OOD metric (Deep-KNN, recently used in~\cite{sun2022out}): it measures how far a feature visualization deviates from the corresponding ImageNet object category images based on their representation in the network's intermediate layers (see Table~\ref{table:maco:ood_fid}).



\paragraph{FID score.} The FID quantifies the similarity between the distribution of the feature visualizations and that of natural images for the same object category. Importantly, the FID measures the distance between two distributions, while the plausibility score quantifies the distance from a sample to a distribution. To compute the FID,  we used images from the ImageNet validation set and used the Inception v3 last layer (see Table~\ref{table:maco:ood_fid}). Additionally, we center-cropped our $512\times 512$ images to $299\times 299$ images to avoid the center-bias problem~\cite{nguyen2016multifaceted}.



\paragraph{Transferability score.} This score measures how consistent the feature visualizations are with other pre-trained classifiers. To compute the transferability score, we feed the obtained feature visualizations into 6 additional pre-trained classifiers (MobileNet~\cite{howard2017mobilenets}, VGG16~\cite{simonyan2014deep}, Xception~\cite{chollet2017xception}, EfficientNet~\cite{tan2019efficientnet}, Tiny ConvNext~\cite{liu2022convnet} and Densenet~\cite{huang2017densely}), and we report their classification accuracy (see Table~\ref{table:maco:transferability}).

All scores are computed using 500 feature visualizations, each of them maximizing the logit of one of the ImageNet classes obtained on the FlexiViT~\cite{beyer2022flexivit}, ViT\cite{kolesnikov2020bit}, and ResNetV2\cite{he2016deep} models. For the feature visualizations derived from Olah et al.~ \cite{olah2017feature}, we used all 10 transformations set from the Lucid library\footnote{\href{https://github.com/tensorflow/lucid}{https://github.com/tensorflow/lucid}}.
CBR denotes an optimization in pixel space and using the same 10 transformations, as described in~\cite{nguyen2015deep}.
For \magfv, $\augmentation$ only consists of two transformations; first we add uniform noise $\bm{\delta} \sim \mathcal{U}([-0.1, 0.1])^{W \times H}$ and crops and resized the image with a crop size drawn from the normal distribution $\mathcal{N}(0.25, 0.1)$, which corresponds on average to 25\% of the image.
We used the NAdam optimizer \cite{dozat2016incorporating} with $lr=1.0$ and $N = 256$ optimization steps. Finally, we used the implementation of \cite{olah2017feature} and CBR which are available in the Xplique library~\cite{fel2022xplique} \footnote{\href{https://github.com/deel-ai/xplique}{https://github.com/deel-ai/xplique}} which is based on Lucid.

\begin{table}[ht]
\centering
        \begin{tabular}{lccc}
            & FlexiViT & ViT & ResNetV2\\
            \hline
            \multicolumn{4}{l}{$\bullet$\;\textbf{Plausibility score} (1-KNN) ($\downarrow$)}\\

            \magfv & {\bf 1473} & {\bf 1097 } & {\bf 1248} \\%\\
            Fourier~\cite{olah2017feature} & 1815 &  1817 & 1837 \\
            CBR~\cite{nguyen2015deep} &  1866 & 1920 & 1933 \\
            \hline
            \multicolumn{4}{l}{$\bullet$\;\textbf{FID Score}  ($\downarrow$)}\\
            \magfv & {\bf 230.68} & {\bf 241.68} & {\bf 312.66} \\
            Fourier~\cite{olah2017feature} &  250.25 & 257.81 & 318.15 \\
            CBR~\cite{nguyen2015deep} &  247.12 & 268.59 & 346.41 \\
            \hline
        \end{tabular}
        
        \caption{Plausibility and FID scores for different feature visualization methods applied on FlexiVIT, ViT and ResNetV2}
    \label{table:maco:ood_fid}
\end{table}

\begin{table}[ht]
\centering
\begin{tabular}{lccc}
    & FlexiViT & ViT & ResNetV2 \\
    \hline
    \multicolumn{4}{l}{$\bullet$\;\textbf{Transferability score($\uparrow$)}: \magfv / Fourier~\cite{olah2017feature}} \\

    MobileNet  & {\bf 68} \slash~38 & {\bf 48}\slash~37  & {\bf 93} \slash~36 \\
    VGG16         & {\bf 64} \slash~30 & {\bf 50} \slash~30 & {\bf 90} \slash~20 \\
    Xception      & {\bf 85} \slash~61 & {\bf 73} \slash~62 & {\bf 97} \slash~64 \\
    Eff. Net  & {\bf 88} \slash~25 & {\bf 63} \slash~25 & {\bf 82} \slash~21 \\
    ConvNext & {\bf 96} \slash~52 & {\bf 84} \slash~55 & {\bf 93} \slash~60\\
    DenseNet      & {\bf 84} \slash~32 & {\bf 66} \slash~31 & {\bf 93} \slash~25 \\
    \hline
    \\
    \end{tabular}
        \caption{Transferability scores for different feature visualization methods applied on FlexiVIT, ViT and ResNetV2.}
        \label{table:maco:transferability}
\end{table}


























For all tested metrics, we observe that \magfv~produces better feature visualizations than those generated by Olah et al.~\cite{olah2017feature} and CBR~\cite{nguyen2015deep}. We would like to emphasize that our proposed evaluation scores represent the first attempt to provide a systematic evaluation of feature visualization methods, but we acknowledge that each individual metric on its own is insufficient and cannot provide a comprehensive assessment of a method's performance. However, when taken together, the three proposed scores provide a more complete and accurate evaluation of the feature visualization methods.

\subsubsection{Human psychophysics study}
Ultimately, the goal of any feature visualization method is to demystify the CNN's underlying decision process in the eyes of human users. To evaluate \magfv~'s ability to do this, we closely followed the psychophysical paradigm introduced in~\cite{zimmermann2021well}. In this paradigm, the participants are presented with examples of a model's ``favorite'' inputs (i.e., feature visualization generated for a given unit) in addition to two query inputs. Both queries represent the same natural image, but have a different part of the image hidden from the model by a square occludor. The task for participants is to judge which of the two queries would be ``favored by the model'' (i.e., maximally activate the unit). The rationale here is that a good feature visualization method would enable participants to more accurately predict the model's behavior. Here, we compared four visualization conditions (manipulated between subjects): Olah~\cite{olah2017feature}, \magfv~with the transparency mask (the transparency mask is decribed in \ref{sec:maco:transparency}), \magfv~without the transparency mask, and a control condition in which no visualizations were provided. In addition, the network (VGG16, ResNet50, ViT) was a within-subject variable. The units to be understood were taken from the output layer.

\begin{figure}[ht]
\includegraphics[width=0.9\textwidth]{assets/maco/human_exp.png}
\caption{\textbf{Human causal understanding of model activations}. We follow the experimental procedure introduced in~\cite{zimmermann2021well} to evaluate Olah and \magfv~visualizations on $3$ different networks. The control condition is when the participant did not see any feature visualization. 
}
\label{fig:maco:human_results}    
\end{figure}

Based on the data of 174 participants on Prolific (\url{www.prolific.com}), we found both visualization and network to significantly predict the logodds of choosing the right query (Fig.~\ref{fig:maco:human_results}). That is, the logodds were significantly higher for participants in both the \magfv~conditions compared to Olah. On the other hand, our tests did not yield a significant difference between Olah and the control condition, or between the two \magfv~conditions. Finally, we found that, overall, ViT was significantly harder to interpret than ResNet50 and VGG16, with no significant difference observed between the latter two networks. Full experiment and analysis details can be found in the supplementary materials, section~\ref{sup:maco:psychophysics}. 

However, it should be noted that investigating the effect on a neuron-by-neuron basis, as in the original setup, may not be advisable for the issues outlined in \autoref{sec:concepts:craft} and referenced in \cite{elhage2022superposition}. Conducting a parallel study that confirms this by utilizing meaningful directions in the latent space -- e.g., with \craft -- instead of individual neurons would be of interest.

\subsubsection{Ablation study}

    \begin{table}%
        \centering
        \begin{tabular}{lccc}
            FlexiViT & Plausibility ($\downarrow$) & FID ($\downarrow$) & logit magnitude ($\uparrow$) \\
            \hline
            \magfv  & 571.68 & 211.0 & 5.12 \\
            - transparency & 617.9 (+46.2) & 208.1 (-2.9) & 5.05 (-0.1)\\
            - crop & 680.1 (+62.2) & 299.2 (-91.1) & 8.18 (+3.1)\\
            - noise & 707.3 (+27.1) & 324.5 (-25.3) & 11.7 (+3.5)\\
            \hline
            Fourier~\cite{olah2017feature} & 673.3 & 259.0 & 3.22\\
            - augmentations & 735.9 (+62.6) &  312.5 (+53.5) & 12.4 (+9.2)\\
        \end{tabular}
        \caption{\textbf{Ablation study on the FlexiViT model:} This reveals that 1. augmentations help to have better FID and Plausibility scores, but lead to lesser salients visualizations (softmax value), 2. Fourier~\cite{olah2017feature} benefits less from augmentations than \magfv.}
        \label{table:maco:ablation}
    \end{table}


    To disentangle the effects of the various components of \magfv, we perform an ablation study on the feature visualization applications. We consider the following components: (1) the use of a magnitude constraint, (2) the use of the random crop, (3) the use of the noise addition, and (4) the use of the transparency mask. We perform the ablation study on the FlexiViT model, and the results are presented in Table~\ref{table:maco:ablation}. We observe an inherent tradeoff between optimization quality (measured by logit magnitude) on one side, and the plausibility (and FID) scores on the other side. This reveals that plausible images which are close to the natural image distribution do not necessarily maximize the logit.
    Finally, we observe that the transparency mask does not significantly affect any of the scores confirming that it is mainly a post-processing step that does not affect the feature visualization itself.


\subsection{Applications}

We demonstrate the versatility of the proposed \magfv~technique by applying it to three different XAI applications:

\paragraph{Logit and internal state visualization.} For logit visualization, the optimization objective is to maximize the activation of a specific unit in the logits vector of a pre-trained neural network (here a FlexiViT\cite{beyer2022flexivit}). The resulting visualizations provide insights into the features that contribute the most to a class prediction (refer to Figure~\ref{fig:maco:logits_and_internal}a). For internal state visualization, the optimization objective is to maximize the activation of specific channels located in various intermediate blocks of the network (refer to Figure~\ref{fig:maco:logits_and_internal}b). This visualization allows us to better understand the kind of features these blocks -- of a FlexiViT\cite{beyer2022flexivit} in the figure -- are sensitive to.

\paragraph{Feature inversion.} The goal of this application is to find an image that produces an activation pattern similar to that of a reference image. By maximizing the similarity to reference activations, we are able to generate images representing the same semantic information at the target layer but without the parts of the original image that were discarded in the previous stages of the network, which allows us to better understand how the model operates.
Figure~\ref{fig:maco:inversion}a displays the images (second column) that match the activation pattern of the penultimate layer of a VIT when given the images from the first column. We also provide examples of transparency masks based on attribution (third column), which we apply to the feature visualizations to enhance them (fourth column).



\begin{figure}
    \centering
    \includegraphics[width=1.0\textwidth]{assets/maco/inversion.jpg}
    \caption{\textbf{Feature inversion.} Images in the second column match the activation pattern of the penultimate layer of a ViT when fed with the images of the first column. In the third column, we show their corresponding attribution-based transparency masks, leading to better feature visualization when applied (fourth column).}
    \label{fig:maco:inversion}
\end{figure}


\paragraph{Concept visualization.} Herein we combine \magfv~with concept-based explainability. Such methods aim to increase the interpretability of activation patterns by decomposing them into a set of concepts~\cite{ghorbani2019towards}. In this work, we leverage our \craft~concept-based explainability method~\cite{fel2023craft}, which uses Non-negative Matrix Factorization to decompose activation patterns into main directions -- that are called concepts --, and then, we apply \magfv~to visualize these concepts in the pixel space. To do so, we optimize the visualization such that it matches the concept activation patterns. In Figure~\ref{fig:maco:concepts}b, we present the top $2$ most important concepts (one concept per column) for five different object categories (one category per row) in a ResNet50 trained on ImageNet. The concepts' visualizations are followed by a mosaic of patches extracted from natural images: the patches that maximally activate the corresponding concept. 

\begin{figure}
    \centering
    \includegraphics[width=1.0\textwidth]{assets/maco/concept_maco.jpg}
    \caption{\textbf{Concept visualization.} \magfv~is used to visualize concept vectors extracted with the \craft~ method~\autoref{sec:concepts:craft}. The concepts are extracted from a ResNet50 trained on ImageNet.}
    \label{fig:maco:concepts}
\end{figure}











\subsection{Limitations}\label{sec:maco:limitations}
We have demonstrated the generation of realistic explanations for large neural networks by imposing constraints on the magnitude of the spectrum. However, it is important to note that generating realistic images does not necessarily imply effective explanation of the neural networks. The metrics introduced in this section allow us to claim that our generated images are closer to natural images in latent space, that our feature visualizations are more plausible and better reflect the original distribution. However, they do not necessarily indicate that these visualizations helps humans in effectively communicating with the models or conveying information easily to humans.
Furthermore, in order for a feature visualization to provide informative insights about the model, including spurious features, it may need to generate visualizations that deviate from the spectrum of natural images. Consequently, these visualizations might yield lower scores using our proposed metrics.
Simultaneously, several interesting studies have highlighted the weaknesses and limitations of feature visualizations~\cite{borowski2020exemplary,geirhos2023dont,zimmermann2021well}. One prominent criticism is their lack of interpretability for humans, with research demonstrating that dataset examples are more useful than feature visualizations in understanding convolutional neural networks (CNNs)~\cite{borowski2020exemplary}. This can be attributed to the lack of realism in feature visualizations and their isolated use as an explainability technique.
With our approach, \magfv~, we take an initial step towards addressing this limitation by introducing magnitude constraints, which lead to qualitative and quantitative improvements. Additionally, we promote the use of feature visualizations as a supportive and complementary tool alongside other methods such as concept-based explainability, exemplified by \craft. We emphasize the importance of feature visualizations in combating confirmation bias and encourage their integration within a comprehensive explainability framework.



\subsection{Discussion}

In this section, we introduced a novel approach, \magfv, for efficiently generating feature visualizations in modern deep neural networks based on \tbi{i} a hard constraint on the magnitude of the spectrum to ensure that the generated visualizations lie in the space of natural images, and \tbi{ii} a new attribution-based transparency mask to augment these feature visualizations with the notion of spatial importance. This enhancement allowed us to scale up and unlock feature visualizations on large modern CNNs and vision transformers without the need for strong -- and possibly misleading -- parametric priors.
We also complement our method with a set of three metrics to assess the quality of the visualizations. Combining their insights offers a way to compare the techniques developed in this branch of XAI more objectively. We illustrated the scalability of \magfv~ with feature visualizations of large models like ViT, but also feature inversion and, critically, concept visualization.

Indeed, this tool integrates seamlessly with concept extraction methods, enabling the visualization of extracted concepts without resorting to image cropping. This approach offers a clearer, more causal view of the mechanisms that activate a given concept, thereby contributing significantly to our understanding of the internal workings of neural networks.

\clearpage

\section{Conclusion}

The conclusion of this chapter serves as an opportune moment for reflection and synthesis. Our research has led us through an in-depth examination of Hypothesis~\ref{hyp:what}, which posited that existing attribution methods fall short, as they primarily reveal ~\where but overlook the crucial aspect of the \what.

This chapter was dedicated to developing appropriate tools to address this issue. We began by constructing \craft, a method for decomposing the activations of a model into a set of concepts, demonstrating indeed its enhanced utility for human understanding compared to traditional attribution methods.
We decided to go one step further, in \autoref{sec:concepts:holistic}, where we established a theoretical framework that make: \tbi{i} show that concept extraction is \textbf{Dictionary learning}, and \tbi{ii} make a link between attribution methods and concept importance. The formulas used to determine the importance of a pixel, as seen in \autoref{chap:attributions}, are identical to those applied in evaluating the significance of concepts after decomposition. In the final section, we explored concept visualization as a way to visualize concept by introducing \maco. 

To summarize our novel framework, it consists in reinterpreting the intricate latent space of neural networks through a collection of atomic units termed concepts. While these concepts are mathematically abstract, we employed two methods to imbue them with meaning: maximally activating crops and feature visualization techniques. Additionally, it became evident that among these concepts, some offer greater utility than others, with attribution methods precisely identifying the most relevant ones.

\paragraph{A New Synergetic Approach to Explainability.} This new framework is distinct in its ability to \textit{synthesize all existing tools for explainability into a cohesive and synergetic system}. Our goal was to demonstrate the potential of this approach -- and the powerful synergy it creates -- through the visual demonstration offered by \Lens~(illustrated in \autoref{fig:concepts:lens}).

\begin{figure}[ht]
    \centering
    \includegraphics[width=0.49\textwidth]{assets/lens_website.jpg}
    \includegraphics[width=0.49\textwidth]{assets/lens_click.jpg}
    \caption{\textbf{LENS Project.} Example of results from the LENS demo for the espresso class. \textbf{(Left)} The page displays the top 10 concepts, ranked from most to least important for the class. These concepts are extracted using \craft~and visualized with \maco; their importance is calculated using the optimal formula found in \autoref{sec:concepts:holistic}. \textbf{(Right)} Clicking on a feature visualization that illustrates a concept reveals the image crops that most strongly activate the concept.}
    \label{fig:concepts:lens}
\end{figure}

This platform organizes, for each of the 1000 ImageNet classes, the ten most significant concepts, along with their feature visualizations and respective importance. 

\paragraph{Perspective.} While the potential of concept-based methods is clear, it is now critical to establish distinct research directions to fully unlock their potential in the wake of preliminary studies. Four key areas emerge, meriting further exploration:

\begin{itemize}

    \item \textbf{Revisiting Dictionary Learning:} The evident parallels between concept extraction and dictionary learning highlight a pressing need for the XAI community to reassess and tailor dictionary learning methodologies for application in explainability. This adaptation could bridge gaps in our understanding and application of these techniques within XAI.

    \item \textbf{Beyond Classification:} The necessity of extending our investigative scope beyond mere classification tasks is crucial. Diverse models, including bounding box detection, segmentation, generative and Vision-Language models present intricate challenges and vast opportunities for enhancing explainability. Diversifying our focus will enable a general comprehension of AI systems, integrating a broader spectrum of tasks and functionalities, thus deepening the XAI field with richer insights and more adaptable explainability tools. An illustrative example is given in \autoref{fig:holistc:conceptbbox}.
    
    \item \textbf{Exploring Hierarchical Concepts and Compositionality:} Investigating hierarchical concepts and their compositionality also offers a very promising path to deepen our understanding of how neural networks operate. Recent research has highlighted that models can exhibit compositional behaviors~\cite{lepori2024break}. Understanding the ways in which concepts are combined and interact at various abstraction levels could offer a nuanced perspective on decision-making processes within models, paving the way for more refined interpretability strategies.
    
    \item \textbf{Expanding on Synergies:} The demonstrated synergy among different explainability methods within our framework suggests a fertile area for research. A comprehensive examination of how these methods can be cohesively integrated, and the resultant synergistic effects could lead to groundbreaking insights and the development of potent tools for explainability.
    
\end{itemize}

\begin{figure}
    \centering
    \includegraphics[width=1.05\textwidth]{assets/holistic/concept_bbox.jpg}
    \caption{\textbf{Concepts of \craft~on ResNet50 Trained on Different Tasks.} The concepts extracted for classification tasks, specifically for the St. Bernard class, seem to focus on the head and spots of the St. Bernard. In the case of the bounding box (bbox) model, the legs are also deemed important, possibly because they help delineate the edges of the bounding boxes? Interestingly, for the CLIP model, the dog head concept is also activated by human heads, suggesting that despite visual differences (in the pixel space), the concept of 'head' seems present for models trained with language components like CLIP.}
    \label{fig:holistc:conceptbbox}
\end{figure}

While this framework does not solve all the challenges presented in the \autoref{chap:intro}, and there remains a significant journey toward fully understanding models such as ResNet50 or ViT, it opens a novel avenue. We encourage the academic community to explore the synergies between attribution methods, concepts, and feature visualization for deeper explainability.


