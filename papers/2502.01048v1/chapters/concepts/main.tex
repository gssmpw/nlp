\chapter{From Pixels to Features: Towards Deeper Explainability with Concepts}
\chaptermark{\protect\parbox{.5\textwidth}{From Pixels to Features}}
\label{chap:concepts}

\begin{chapterabstract}
\textit{
In this chapter, we address a challenge identified in \autoref{chap:attributions}: Is it possible to transcend attributions methods to forge methods that do more than just spotlight where a model directs its attention -- \where~the model is looking -- but also clarify \what~exactly it perceives? Essentially, existing methods primarily disclose the ``\where'' in terms of the model's focus, rather than elucidating the "\what" it discerns, in terms of feature. The question then becomes, how can we define and characterize this ``\what''? This is the subject of this chapter that aims to extend beyond attribution methods to lay a more robust foundation for a deeper and more precise Explainability. \\
Our exploration begins in \autoref{sec:concepts:craft}, which propose a significant advancement in concept-based explainability by introducing an automated method, \craft, for extracting a model's learned concepts. We demonstrate that it is feasible to easily assess the significance of these derived concepts using Sobol indices presented in \autoref{sec:attributions:sobol}. The findings from this work substantially improve upon the benchmarks established in \autoref{sec:attributions:metapred}, and offer new avenues for addressing complex scenarios requiring in-depth explainability.
Progressing to \autoref{sec:concepts:holistic}, the cornerstone of this chapter, we show \tbi{i} how \craft~and related research fit within a broader framework of dictionary learning. We propose a unified framework for concept extraction, paving the way for new methodologies. Further, \tbi{ii} we establish a link between concept importance estimation and traditional attribution methods, demonstrating that concept importance estimation methods can be viewed as attribution methods recontextualized within the concept space for evaluative purposes.
With this framework in place, we find it possible to derive insightful answers to literature questions such as ``where should concept decomposition be performed?'' or ``which importance method to choose''. Furthermore, we delve into the importance measure of concepts, revealing that this information can be utilized to address a significant open problem in Explainability: ``how to identify points classified for similar reasons'', by proposing the strategic clustering plot.
The final section of this chapter, \autoref{sec:concepts:maco}, is dedicated to scaling feature visualization through a reformulation of the optimization problem within the Fourier space, by constraining magnitude. This new module allows for the use of feature visualization to create prototypes of the concepts extracted with \craft.
In conclusion, we will showcase the powerful synergies this new framework offers with \Lens, a demo that enables the visualization of the concepts used by a ResNet50 model for the 1000 ImageNet classes.
In sum, this chapter not only tackles foundational questions within the domain of machine learning explainability, but also sets forth a comprehensive framework that integrates advanced methodologies for concept extraction and importance estimation.
}
\end{chapterabstract}

The work in this chapter has led to the publication of the following conference papers:
{\small{
\begin{itemize}

    \item \textbf{Thomas Fel}\equal, Agustin Picard\equal, Louis Bethune\equal, Thibaut Boissin\equal, David Vigouroux, Julien Colin, Rémi Cadène, Thomas Serre, (2023). \textit{``CRAFT: Concept Recursive Activation FacTorization for Explainability''.} In: \textit{IEEE/CVF Conference on Computer Vision and Pattern Recognition} (\textcolor{confcolor}{CVPR})
    
    \item \textbf{Thomas Fel}\equal, Victor Boutin\equal, Mazda Moayeri, Rémi Cadène, Louis Bethune, Mathieu Chalvidal, Thomas Serre (2023). \textit{``A Holistic Approach to Unifying Automatic Concept Extraction and Concept Importance Estimation''.} In: \textit{Advances in Neural Information Processing Systems}  (\textcolor{confcolor}{NeurIPS})
    
    \item \textbf{Thomas Fel}\equal, Thibaut Boissin\equal, Victor Boutin\equal, Agustin Picard\equal, Paul Novello\equal, Julien Colin, Drew Linsley, Tom Rousseau, Rémi Cadène, Lore Goetschalckx, Thomas Serre (2024). \textit{``Unlocking feature visualization for deep network with MAgnitude constrained optimization''.} In: \textit{Advances in Neural Information Processing Systems}  (\textcolor{confcolor}{NeurIPS})

\end{itemize}
}}

\minitoc
\clearpage

\section{Introduction}
\section{Introduction}


\begin{figure}[t]
\centering
\includegraphics[width=0.6\columnwidth]{figures/evaluation_desiderata_V5.pdf}
\vspace{-0.5cm}
\caption{\systemName is a platform for conducting realistic evaluations of code LLMs, collecting human preferences of coding models with real users, real tasks, and in realistic environments, aimed at addressing the limitations of existing evaluations.
}
\label{fig:motivation}
\end{figure}

\begin{figure*}[t]
\centering
\includegraphics[width=\textwidth]{figures/system_design_v2.png}
\caption{We introduce \systemName, a VSCode extension to collect human preferences of code directly in a developer's IDE. \systemName enables developers to use code completions from various models. The system comprises a) the interface in the user's IDE which presents paired completions to users (left), b) a sampling strategy that picks model pairs to reduce latency (right, top), and c) a prompting scheme that allows diverse LLMs to perform code completions with high fidelity.
Users can select between the top completion (green box) using \texttt{tab} or the bottom completion (blue box) using \texttt{shift+tab}.}
\label{fig:overview}
\end{figure*}

As model capabilities improve, large language models (LLMs) are increasingly integrated into user environments and workflows.
For example, software developers code with AI in integrated developer environments (IDEs)~\citep{peng2023impact}, doctors rely on notes generated through ambient listening~\citep{oberst2024science}, and lawyers consider case evidence identified by electronic discovery systems~\citep{yang2024beyond}.
Increasing deployment of models in productivity tools demands evaluation that more closely reflects real-world circumstances~\citep{hutchinson2022evaluation, saxon2024benchmarks, kapoor2024ai}.
While newer benchmarks and live platforms incorporate human feedback to capture real-world usage, they almost exclusively focus on evaluating LLMs in chat conversations~\citep{zheng2023judging,dubois2023alpacafarm,chiang2024chatbot, kirk2024the}.
Model evaluation must move beyond chat-based interactions and into specialized user environments.



 

In this work, we focus on evaluating LLM-based coding assistants. 
Despite the popularity of these tools---millions of developers use Github Copilot~\citep{Copilot}---existing
evaluations of the coding capabilities of new models exhibit multiple limitations (Figure~\ref{fig:motivation}, bottom).
Traditional ML benchmarks evaluate LLM capabilities by measuring how well a model can complete static, interview-style coding tasks~\citep{chen2021evaluating,austin2021program,jain2024livecodebench, white2024livebench} and lack \emph{real users}. 
User studies recruit real users to evaluate the effectiveness of LLMs as coding assistants, but are often limited to simple programming tasks as opposed to \emph{real tasks}~\citep{vaithilingam2022expectation,ross2023programmer, mozannar2024realhumaneval}.
Recent efforts to collect human feedback such as Chatbot Arena~\citep{chiang2024chatbot} are still removed from a \emph{realistic environment}, resulting in users and data that deviate from typical software development processes.
We introduce \systemName to address these limitations (Figure~\ref{fig:motivation}, top), and we describe our three main contributions below.


\textbf{We deploy \systemName in-the-wild to collect human preferences on code.} 
\systemName is a Visual Studio Code extension, collecting preferences directly in a developer's IDE within their actual workflow (Figure~\ref{fig:overview}).
\systemName provides developers with code completions, akin to the type of support provided by Github Copilot~\citep{Copilot}. 
Over the past 3 months, \systemName has served over~\completions suggestions from 10 state-of-the-art LLMs, 
gathering \sampleCount~votes from \userCount~users.
To collect user preferences,
\systemName presents a novel interface that shows users paired code completions from two different LLMs, which are determined based on a sampling strategy that aims to 
mitigate latency while preserving coverage across model comparisons.
Additionally, we devise a prompting scheme that allows a diverse set of models to perform code completions with high fidelity.
See Section~\ref{sec:system} and Section~\ref{sec:deployment} for details about system design and deployment respectively.



\textbf{We construct a leaderboard of user preferences and find notable differences from existing static benchmarks and human preference leaderboards.}
In general, we observe that smaller models seem to overperform in static benchmarks compared to our leaderboard, while performance among larger models is mixed (Section~\ref{sec:leaderboard_calculation}).
We attribute these differences to the fact that \systemName is exposed to users and tasks that differ drastically from code evaluations in the past. 
Our data spans 103 programming languages and 24 natural languages as well as a variety of real-world applications and code structures, while static benchmarks tend to focus on a specific programming and natural language and task (e.g. coding competition problems).
Additionally, while all of \systemName interactions contain code contexts and the majority involve infilling tasks, a much smaller fraction of Chatbot Arena's coding tasks contain code context, with infilling tasks appearing even more rarely. 
We analyze our data in depth in Section~\ref{subsec:comparison}.



\textbf{We derive new insights into user preferences of code by analyzing \systemName's diverse and distinct data distribution.}
We compare user preferences across different stratifications of input data (e.g., common versus rare languages) and observe which affect observed preferences most (Section~\ref{sec:analysis}).
For example, while user preferences stay relatively consistent across various programming languages, they differ drastically between different task categories (e.g. frontend/backend versus algorithm design).
We also observe variations in user preference due to different features related to code structure 
(e.g., context length and completion patterns).
We open-source \systemName and release a curated subset of code contexts.
Altogether, our results highlight the necessity of model evaluation in realistic and domain-specific settings.





\clearpage

\section{CRAFT : Concept Recursive Activation FacTorization}
\label{sec:concepts:craft}
\definecolor{metalgreen}{RGB}{51, 157, 144}
\definecolor{metalorange}{RGB}{244, 161, 97}

We propose to directly start with our first work, \craft. As we have seen in \autoref{chap:attributions}, Attribution methods employ heatmaps to identify the most influential regions of an image that impact model decisions, and those methods have gained widespread popularity as a type of explainability method.
However, they only reveal~\where~the model looks, failing to elucidate \what~the model sees in those areas.
In this section, we will try to fill in this gap with \craft -- a novel approach to identify both ``\what'' and ``\where'' by generating concept-based explanations.
We introduce 3 new ingredients to the automatic concept extraction literature: (\textbf{i}) a recursive strategy to detect and decompose concepts across layers, (\textbf{ii}) a novel method for a more faithful estimation of concept importance using Sobol indices, and (\textbf{iii}) the use of implicit differentiation to unlock Concept Attribution Maps.

We conduct both human and computer vision experiments, specifically the one proposed in \autoref{sec:attributions:metapred}, to demonstrate the benefits of the proposed approach. We show that the proposed concept importance estimation technique -- based on Sobol indices -- is more faithful to the model than previous methods. Moreover, we have open-sourced our code at
\href{https://github.com/deel-ai/Craft}{\nolinkurl{github.com/deel-ai/Craft}}, and also in the \href{https://github.com/deel-ai/xplique}{Xplique} library.

\begin{figure}[ht]\centering
\includegraphics[width=0.99\textwidth]{assets/craft/moon.pdf}
\caption{\textbf{The ``Man on the Moon'' incorrectly classified as a ``shovel'' by an ImageNet-trained ResNet50.} Heatmap generated by a classic attribution method~\cite{petsiuk2018rise} (left) vs.  \textit{concept attribution maps} generated with the proposed \craft~approach (right) which highlights the two most influential concepts that drove the ResNet50's decision along with their corresponding locations. 
\craft~suggests that the neural net arrived at its decision because it identified the concept of ``dirt'' \textcolor{green}{$\bullet$} commonly found in members of the image class ``shovel'' and the concept of ``ski pants'' \textcolor{violet}{$\bullet$} typically worn by people clearing snow from their driveway with a shovel instead the correct concept of astronaut's pants (which was probably never seen during training).
}
\label{fig:craft:shovel}
\end{figure}


\subsection{Background}

\begin{figure*}[t!]\centering
\includegraphics[width=0.99\textwidth]{assets/craft/big_picture.pdf}
\caption{\textbf{\craft~results for the prediction ``chain saw''.} 
First, our method uses Non-Negative Matrix Factorization (NMF) to extract the most relevant concepts used by the network (ResNet50V2) from the train set (ILSVRC2012~\cite{imagenet_cvpr09}). The global influence of these concepts on the predictions is then measured using Sobol indices (right panel). Finally, the method provides local explanations through \textit{concept attribution maps} (heatmaps associated with a concept, and computed using grad-CAM by backpropagating through the NMF concept values with implicit differentiation).
Besides, concepts can be interpreted by looking at crops that maximize the NMF coefficients. For the class ``chain saw'', the detected concepts seem to be:
\textcolor{blue}{$\bullet$} the chainsaw engine, 
\textcolor{purple}{$\bullet$} the saw blade, 
\textcolor{green}{$\bullet$} the human head, 
\textcolor{orange}{$\bullet$} the vegetation, 
\textcolor{red}{$\bullet$} the jeans and
\textcolor{dark}{$\bullet$} the tree trunk.
}
\label{fig:craft:craft_demo}
\end{figure*}


\paragraph{Attribution methods}

Attribution methods are widely used as post-hoc explainability techniques to determine the input variables that contribute to a model's prediction by generating importance maps, such as the ones shown in Fig.\ref{fig:craft:shovel}. The first attribution method, Saliency, introduced in~\cite{zeiler2014visualizing}, generates a heatmap by utilizing the gradient of a given classification score with respect to the pixels. This method was later improved upon in the context of deep convolutional networks for classification in subsequent studies, such as~\cite{zeiler2013visualizing, springenberg2014striving, sundararajan2017axiomatic, smilkov2017smoothgrad}.

Unfortunately, a severe limitation of these approaches -- apart from the fact that they only show the ``\where'' -- is that they are subject to confirmation bias: while they may appear to offer useful explanations to a user, sometimes these explanations are actually incorrect~\cite{adebayo2018sanity, ghorbani2017interpretation,slack2020fooling}.
These limitations raise questions about their usefulness, as recent research has shown by using human-centered experiments to evaluate the utility of attribution~\cite{hase2020evaluating,nguyen2021effectiveness,fel2021cannot,kim2021hive,shen2020useful}.

In particular, in our previous \autoref{sec:attributions:metapred}, we have proposed a protocol to measure the usefulness of explanations, corresponding to how much they help users identify rules driving a model's predictions (correct or incorrect) that transfer to unseen data -- using the concept of meta-predictor (also called simulatability)~\cite{kim2016examples,doshivelez2017rigorous,fong2017meaningful}. 
The main idea is to train users to predict the output of the system using a small set of images along with associated model predictions and corresponding explanations. 
A method that performs well on this this benchmark is said useful, as it help users better predict the output of the model by providing meaningful information about the internal functioning of the model.
This framework being agnostic to the type of explainability method, we have chosen to use it in Section~\ref{sec:craft:exp} in order to compare \craft~with attribution methods.




\paragraph{Concepts-based methods}
\cite{kim2018interpretability} introduced a method aimed at providing explanations that go beyond attribution-based approaches by measuring the impact of pre-selected concepts on a model's outputs. Although this method appears more interpretable to human users than standard attribution techniques, it requires a database of images describing the relevant concepts to be manually curated.
Ghorbani et al.~\cite{ghorbani2019towards} further extended the approach to extract concepts  without the need for human supervision. The approach, called ACE~\cite{ghorbani2019towards}, uses a segmentation scheme on images, that belong to an image class of interest. %
The authors leveraged the intermediate activations of a neural network for specific image segments. These segments were resized to the appropriate input size and filled with a baseline value. The resulting activations were clustered to produce prototypes, which they referred to as "concepts". However, some concepts contained background segments, leading to the inclusion of uninteresting and outlier concepts. To address this, the authors implemented a postprocessing cleanup step to remove these concepts, including those that were present in only one image of the class and were not representative. While this improved the interpretability of their explanations to human subjects, the use of a baseline value filled around the segments could introduce biases in the explanations~\cite{hsieh2020evaluations,sturmfels2020visualizing,haug2021baselines,kindermans2019reliability}.

Zhang et al.~\cite{zhang2021invertible} developed a solution to the unsupervised concept discovery problem by using matrix factorizations in the latent spaces of neural networks. However, one major drawback of this method is that it operates at the level of convolutional kernels, leading to the discovery of localized concepts. For example, the concept of "grass" at the bottom of the image is considered distinct from the concept of "grass" at the top of the image.


Here, we try to fill these gaps with a novel method called \craft~which uses Non-Negative Matrix Factorization (NMF)~\cite{lee1999learning} for concept discovery. In contrast to other concept-based explanation methods, our approach provides an explicit link between their global and local explanations (Fig.~\ref{fig:craft:craft_demo}) and identifies the relevant layer(s) to use to represent individual concepts (Fig.~\ref{fig:craft:collapse}). Our main contributions can be described as follows:

{\textbf{(i)}} A novel approach for the automated extraction of high-level concepts learned by deep neural networks. We validate its practical utility to users with human psychophysics experiments.

{\textbf{(ii)}} A recursive procedure to automatically identify concepts and sub-concepts at the right level of granularity -- starting with our decomposition at the top of the model and working our way upstream. We validate the benefit of this approach with human psychophysics experiments showing that (i) the decomposition of a concept yields more coherent sub-concepts and (ii) that the groups of points formed by these sub-concepts are more refined and appear meaningful to humans.

{\textbf{(iii)}} A novel technique to quantify the importance of individual concepts for a model's prediction using Sobol indices~\cite{sobol1993sensitivity,da2013efficient,sobol2001,sobol2005global,saltelli2002} -- a technique borrowed from Sensitivity Analysis.

{\textbf{(iv)}} The first concept-based explainability method which produces  \textit{concept attribution maps} by backpropagating  concept scores  into the pixel space by leveraging the implicit function theorem in order to localize the pixels associated with the concept of a given input image. This effectively opens up the toolbox of both white-box~\cite{smilkov2017smoothgrad, zeiler2014visualizing, sundararajan2017axiomatic, selvaraju2017gradcam, springenberg2014striving, eva2} and black-box~\cite{ribeiro2016lime, lundberg2017unified, petsiuk2018rise, fel2021sobol} explainability methods to derive concept-wise attribution maps.


\begin{figure*}[t!]
\centering\includegraphics[width=1.0\textwidth]{assets/craft/figure2.pdf}
\caption{ 
\textbf{(1) Neural collapse (amalgamation).}
A classifier needs to be able to linearly separate classes by the final layer. It is commonly assumed that in order to achieve this, image activations from the same class get progressively ``merged'' such that these image activations converge to a one-hot vector associated with the class at the level of the logits layer~\cite{paypan2020collapse}. 
In practice, this means that different concepts get ultimately blended together along the way. 
\textbf{(2) Recursive process.} When a concept is not understood (e.g., $\mathcal{C}$), we propose to decompose it into multiple sub-concepts (e.g., $\mathcal{C}_{\textcolor{green}{1}}, \mathcal{C}_{\textcolor{purple}{2}}, \mathcal{C}_{\textcolor{blue}{3}}$) using the activations from an earlier layer to overcome the aforementioned neural collapse issue.
\textbf{(3) Example of recursive concept decomposition} using \craft~on the ImageNet class ``parachute''.
}
\label{fig:craft:collapse}
\end{figure*}


\subsection{Overview of the method} \label{sec:craft:method}

In this section, we first describe our concept activations factorization method. Below we highlight the main differences with related work.
We then proceed to introduce the three novel ingredients that make up \craft: %
(1) a method to recursively decompose concepts into sub-concepts, 
(2) a method to better estimate the importance of extracted concepts, and 
(3) a method to use any attribution method to create \textit{concept attribution maps}, using implicit differentiation~\cite{krantz2002implicit,griewank2008evaluating,blondel2021implicitdiff}.%



\paragraph{Notations}
In this work, we consider a general supervised learning setting, where $(\vx_1, ..., \vx_n) \in \sx^n \subseteq \Real^{n \times d}$ are $n$ inputs images and $(y_1, ..., y_n) \in \sy^n$ their associated labels. 
We are given a (machine-learnt) black-box predictor $\f : \sx \to \sy$, which at some test input $\vx$ predicts the output $\f(\vx)$.
Without loss of generality, we establish that $\f$ is a neural network that can be decomposed into two distinct components. The first component is a function $\v{g}$ that maps the input to an intermediate state, and the second component is $\v{h}$, which takes this intermediate state to the output, such that $\f(\vx) = (\v{h} \circ \v{g})(\vx)$. In this context, $\v{g}(\vx) \subseteq \Real^p$ represents the intermediate activations of $\vx$ within the network.
Further, we will assume non-negative activations: $ \v{g}(\vx) \geq 0$. In particular, this assumption is verified by any architecture that utilizes \textit{ReLU}, but any non-negative activation function works. 



\subsubsection{Concept activation factorization.}\label{subsec:caf}

We use Non-negative matrix factorization to identify a basis for concepts based on a network's activations (Fig.\ref{fig:craft:craft}). Inspired by the approach taken in ACE~\cite{ghorbani2019towards}, we will use image sub-regions to try to identify coherent concepts. 

The first step involves gathering a set of images that one wishes to explain, such as the dataset, in order to generate associated concepts. In our examples, to explain a specific class $y \in \sy$, we selected the set of points $\mathcal{C}$ from the dataset for which the model's predictions matched a specific class $\mathcal{C} = \{ \vx_i : \f(\vx_i) = y, 1 \leq i \leq n \}$.
It is important to emphasize that this choice is significant. The goal is not to understand how humans labeled the data, but rather to comprehend the model itself. By only selecting correctly classified images, important biases and failure cases may be missed, preventing a complete understanding of our model.

Now that we have defined our set of images, we will proceed with selecting sub-regions of those images to identify specific concepts within a localized context. It has been observed that the implementation of segmentation masks suggested in ACE can lead to the introduction of artifacts due to the associated inpainting with a baseline value.
In contrast, our proposed method takes advantage of the prevalent use of modern data augmentation techniques such as randaugment, mixup, and cutmix during the training of current models.
These techniques involve the current practice of models being trained on image crops, which enables us to leverage a straightforward crop and resize function denoted by $\bm{\pi}(\cdot)$ to create sub-regions (illustrated in Fig.\ref{fig:craft:craft}). By applying $\bm{\pi}$ function to each image in the set $\mathcal{C}$, we obtain an auxiliary dataset $\mx \in \Real^{n \times d}$ such that each entries $\mx_i = \bm{\pi}(\vx_i)$ is an image crop.

To discover the concept basis, we start by obtaining the activations for the random crops $\m{A} = \v{g}(\mx) \in \Real^{n \times p}$.
In the case where $\f$ is a convolutional neural network, a global average pooling is applied to the activations.


We are now ready to apply Non-negative Matrix Factorization (NMF) to decompose  positive activations $\m{A}$ into a product of non-negative, low-rank matrices $\m{U} \in \Real^{n \times r}$ and $\m{W} \in \Real^{p \times r}$ by solving:

\begin{equation}
\label{eq:craft:nmf}
(\m{U}, \m{W}) = \argmin_{\m{U} \geq 0, \m{W} \geq 0} ~ \frac{1}{2}\|\m{A} - \m{U}\m{W}^\tr \|^2_{F}, %
\end{equation}  
where $||\cdot||_F$ denotes the Frobenius norm.

This decomposition of our activations $\m{A}$ yields two matrices: $\m{W}$ containing our Concept Activation Vectors (CAVs) and $\m{U}$ that redefines the data points in our dataset according to this new basis. Moreover, this decomposition in this new basis has some interesting properties that go beyond the simple low-rank factorization -- since $r \ll \min(n,p)$.
First, NMF can be understood as the joint learning of a dictionary of Concept Activation Vectors -- called a ``concept bank'' in Fig.~\ref{fig:craft:craft} -- that maps a $\Real^p$ basis onto $\Real^r$, and $\m{U}$ the coefficients of the vectors $\m{A}$ expressed in this new basis. 
The minimization of the reconstruction error $\frac{1}{2}\|\m{A} - \m{U}\m{W}\|^2_F$ ensures that the new basis contains (mostly) relevant concepts. Intuitively, the non-negativity constraints $\m{U} \geq 0, \m{W} \geq 0$ encourage (\textbf{\textit{i}}) $\m{W}$ to be sparse (useful for creating disentangled concepts), (\textbf{\textit{ii}})  $\m{U}$ to be sparse (convenient for selecting a minimal set of useful concepts)  and (\textbf{\textit{iii}})  missing data to be imputed~\cite{ren2020using}, which corresponds to the sparsity pattern of \textit{post-ReLU} activations $\m{A}$. 

It is worth noting that each input $\mx_i$ can be expressed as a linear combination of concepts denoted as $\m{A}_i = \sum_{j=1}^r U_{i,j} \m{W}_j^\tr$.  This approach is advantageous because it allows us to interpret each input as a composition of the underlying concepts. Furthermore, the strict positivity of each term -- NMF is working over the anti-negative semiring, -- enhances the interpretability of the decomposition. Another interesting interpretation could be that each input is represented as a superposition of concepts~\cite{elhage2022superposition}.

While other methods in the literature solve a similar problem (such as low-rank factorization using SVD or ICA), the NMF is both fast and effective and is known to yield concepts that are meaningful to humans~\cite{fu2019nonnegative,zhang2021invertible}. Finally, once the concept bank $\m{W}$ has been precomputed, we can associate the concept coefficients $\bm{u} \in \Real^r$ to any new input $\vx$ (e.g., a full image) by solving the underlying Non-Negative Least Squares (NNLS) problem $\min_{\bm{u} \geq 0} ~ \frac{1}{2}\|\v{g}(\vx) - \bm{u}\m{W}^\tr\|^2_{F}$, and therefore recover its decomposition in the concept basis.


\begin{figure*}[t!]
\centering
\centering
\includegraphics[width=1.0\textwidth]{assets/craft/figure3.pdf}

\caption{
\textbf{Overview of \craft.}
Starting from a set of crops $\vx$ containing a concept $\mathcal{C}$ (e.g., crops images of the class ``parachute''), we compute activations $\v{g}(\vx)$ corresponding to an intermediate layer from a neural network for random image crops. 
We then factorize these activations into two lower-rank matrices, $(\textcolor{metalgreen}{\m{U}}, \textcolor{metalorange}{\m{W}})$. $\textcolor{metalorange}{\m{W}}$ is what we call a ``concept bank'' and is a new basis used to express the activations, while $\textcolor{metalgreen}{\m{U}}$ corresponds to the corresponding coefficients in this new basis.
We then extend the method with 3 new ingredients: (1) recursivity -- by proposing to re-decompose a concept (e.g., take a new set of images containing $\mathcal{C}_{\textcolor{red}{1}}$) at an earlier layer, (2) a better importance estimation using Sobol indices and (3) an approach to leverage implicit differentiation to generate \textit{concept attribution maps} to localize concepts in an image.
}
\label{fig:craft:craft}
\end{figure*}

In essence, the core of our method can be summarized as follows: using a set of images, the idea is to re-interpret their embedding at a given layer as a composition of concepts that humans can easily understand. 
In the next section, we show how one can recursively apply concept activation factorizations to preceding layer for an image containing a previously computed concept.



\subsubsection{Ingredient 1: A pinch of recursivity}\label{subsec:rec}



One of the most apparent issues in previous work~\cite{ghorbani2019towards,zhang2021invertible} is the need for choosing a priori a layer at which the activation maps are computed. This choice will critically affect the concepts that are identified  because certain concepts get amalgamated~\cite{paypan2020collapse} into one at different layers of the neural network, resulting in incoherent and indecipherable clusters, as illustrated in Fig.~\ref{fig:craft:collapse}. We posit that this can be solved by iteratively applying our decomposition at different layer depths, and for the concepts that remain difficult to understand, by looking for their sub-concepts in earlier layers by isolating the images that contain them. This allows us to build hierarchies of concepts for each class. 


We offer a simple solution consisting of reapplying our method to a concept by performing a second step of concept activation factorization on a set of images that contain the concept $\mathcal{C}$ in order to refine it and create sub-concepts (e.g., decompose $\mathcal{C}$ into $\{ \mathcal{C}_1,\mathcal{C}_2,\mathcal{C}_3 \}$) see Fig.~\ref{fig:craft:collapse} for an illustrative example. 
Note that we generalize current methods in the sense that taking images $(\vx_1, ..., \vx_n)$ that are clustered in the logits layer (belonging to the same class) and decomposing them in a previous layer -- as done in \cite{ghorbani2019towards, zhang2021invertible} -- is a valid recursive step.
For a more general case, let us assume that a set of images that contain a common concept is obtained using the first step of concept activation factorization. 

We will then take a subset of the auxiliary dataset points to refine any concept $j$. To do this, we select the subset of points that contain the concept $\mathcal{C}_j = \{\vx_i : U_{i,j} > \lambda_j, 1 \leq i \leq n \}$, where $\lambda_j$ is the 90th percentile of the values of the concept $\m{U}_{:,j}$ across the $n$ points. In other words, the 10\% of images that activate the concept $j$ the most are selected for further refinement into sub-concepts.
Given this new set of points, we can then re-apply the Concept Matrix Factorization method to an earlier layer to obtain the sub-concepts decomposition from the initial concept -- as illustrated in Fig.\ref{fig:craft:collapse}.



\subsubsection{Ingredient 2: A dash of sensitivity analysis}\label{subsec:sobol}



A major concern with concept extraction methods is that concepts that makes sense to humans are not necessarily the same as those being used by a model to classify images.
In order to prevent such confirmation bias during our concept analysis phase, a faithful estimate the overall importance of the extracted concepts is crucial. 
Kim et al.~\cite{kim2018interpretability} proposed an importance estimator based on directional derivatives: the partial derivative of the model output with respect to the vector of concepts. 
While this measure is theoretically grounded, it relies on the same principle as gradient-based methods, and thus, suffers from the same pitfalls: neural network models have noisy gradients~\cite{smilkov2017smoothgrad,sundararajan2017axiomatic}. Hence, the farther the chosen layer is from the output, the noisier the directional derivative score will be.


Since we essentially want to know which concept has the greatest effect on the output of the model, it is natural to consider the field of sensitivity analysis~\cite{sobol2005global, sobol1993sensitivity, sobol2001, cukier1973study,idrissi2021developments}.
In this section, we briefly recall the classic ``total Sobol indices'' on wich we based our previous method in the \autoref{sec:attributions:sobol}, and how to apply them to our problem. The complete derivation of the Sobol-Hoeffding decomposition for concepts is presented in Section~\ref{apdx:sobol} of the supplementary materials.
Formally, a natural way to estimate the importance of a concept $i$ is to measure the fluctuations of the model's output $\v{h}(\m{U} \m{W}^\tr)$ in response to meaningful perturbations of the concept coefficient $\m{U}_{:,i}$ across the $n$ points.
Concretely, we will use perturbation masks $\rm{M}  = (\r{M}_1, ..., \r{M}_r) \sim \mathcal{U}([0, 1]^r)$, here an i.i.d sequence of real-valued random variables, we introduce a concept fluctuation to generate a perturbed activation $\rm{A} = (\m{U} \odot \rm{M})\m{W}^\tr$ where $\odot$ denote the Hadamard product (e.g., the masks can be used to remove a concept by setting its value to zero). We can then propagate this perturbed activation to the model output and get the associated random output $\rm{Y} = \v{h}(\rm{A})$.
Simply put, removing or applying perturbation of an important concept will result in a substantial variation in the output, whereas an unused concept will have minimal effect on the output.

Finally, we can capture the importance that a concept might have as a main effect -- along with its interactions with other concepts -- on the model's output by calculating the expected variance that would remain if all the concepts except the $i$ were to be fixed. This yields the general definition of the total Sobol indices.


\begin{definition}[\textbf{Total Sobol indices for Concept}]
\textit{The total Sobol index $\mathcal{S}^T_i$, which measures the contribution of a concept $i$ as well as its interactions of any order with any other concepts to the model output variance, is given by:}

\begin{align}
\label{eq:craft:total_sobol}
\mathcal{S}^T_i 
& = \frac{ \E_{\rm{M}_{\sim i}}( \V_{M_i} ( \rm{Y} | \rm{M}_{\sim i} )) }{ \V(\rm{Y}) } \\
& = \frac{ \E_{\bm{M}_{\sim i}}( \V_{M_i} ( \v{h}((\m{U} \odot \rm{M})\m{W}^\tr) | \rm{M}_{\sim i} )) }{ \V( \v{h}((\m{U} \odot \rm{M})\m{W}^\tr)) }.
\end{align}
\end{definition}



In practice, this index can be calculated very efficiently~\cite{saltelli2010variance, marrel2009calculations, janon2014asymptotic, owen2013better, tarantola2006random}, more details on the Quasi-Monte Carlo sampling and the estimator used are left in appendix~\ref{apdx:sobol}.


\begin{figure*}[ht]
\centering
\includegraphics[width=0.95\textwidth]{assets/craft/qualitative.jpg}
\caption{
\textbf{Qualitative Results:} \craft~results on 6 classes of ILSVRC2012~\cite{imagenet_cvpr09} for a trained ResNet50V2. The results showcase the top 3 most important concepts for each class. This is done by displaying crop images that activate the concept the most (using $\m{U}$) and also feature visualization~\cite{olah2017feature} of the associated CAVs (using $\m{W}$). 
}
\label{fig:craft:qualitative}
\end{figure*}

\subsubsection{Ingredient 3: A smidgen of implicit differentiation}\label{subsec:cam}

Attribution methods are useful for determining the regions deemed important by a model for its decision, but they lack information about what exactly triggered it.
We have seen that we can already extract this information from the matrices $\m{U}$ and $\m{W}$, but as it is, we do not know in what part of an image a given concept is represented.
In this section, we will show how we can leverage attribution methods (forward and backward modes) to find where a concept is located in the input image (see Fig.~\ref{fig:craft:craft_demo}). Forward attribution methods do not rely on any gradient computation as they only use inference processes, whereas backward methods require back-propagating through a network's layers. By application of the chain rule, computing $\partial \m{U} / \partial \mx$ requires access to $\partial \m{U} /\partial \m{A}$.  





To do so, one could be tempted to solve the linear system $\m{U}\m{W}^\tr=\m{A}$. 
However, this problem is ill-posed since $\m{W}^\tr$ is low rank. A standard approach is to calculate the Moore-Penrose pseudo-inverse $(\m{W}^\tr)^\pinv$, which solves rank deficient systems by looking at the minimum norm solution~\cite{barata2012moore}. In practice, $(\m{W}^\tr)^{\dagger}$ is computed with the Singular Value Decomposition (SVD) of $\m{W}^\tr$. Unfortunately, SVD is also the solution to the \textit{unstructured minimization} of $\frac{1}{2}\|\m{A}-\m{U}\m{W}^\tr\|^2_F$ by the Eckart-\-Young-\-Mirsky theorem~\cite{eckart1936approximation}. Hence, the non-negativity constraints of the NMF are ignored, which prevents such approaches from succeeding. Other issues stem from the fact that the $\m{U},\m{W}$ decomposition is generally not unique.



Our third contribution consists of tackling this problem to allow the use of attribution methods, i.e., \textit{concept attribution maps}, by proposing a strategy to differentiate through the NMF block.

\paragraph{Implicit differentiation of NMF block}


The NMF problem~\ref{eq:craft:nmf} is NP-hard~\cite{vavasis2010complexity}, and it is not convex with respect to the input pair $(\m{U},\m{W})$. However, fixing the value of one of the two factors and optimizing the other turns the NMF formulation into a pair of Non-Negative Least Squares (NNLS) problems, which are convex. This ensures that alternating minimization (a standard approach for NMF) of $(\m{U},\m{W})$ factors will eventually reach a local minimum.
Each of this alternating NNLS problems fulfills the Karush-–Kuhn-–Tucker (KKT) conditions~\cite{karush1939minima,kuhn1951nonlinear}, which can be encoded in the so-called \textit{optimality function} $\implicit$ from \cite{blondel2021implicitdiff}, see Eq.~\ref{apeq:craft:optimality_fun} Appendix~\ref{app:craft:implicit}. The implicit function theorem~\cite{griewank2008evaluating} allows us to use implicit differentiation~\cite{krantz2002implicit,griewank2008evaluating,bell2008algorithmic} to efficiently compute the Jacobians $\partial \m{U}/ \partial \m{A}$ and $\partial \m{W} / \partial \m{A}$ without requiring to back-propagate through each of the iterations of the NMF solver:

Let the optimality function $\implicit$, as introduced in Blondel et al. (2021) and based on the Karush-Kuhn-Tucker (KKT) conditions (Karush, 1939; Kuhn and Tucker, 1951), encapsulate the optimality conditions of the Non-negative Matrix Factorization (NMF) problem as formulated in Equation \ref{eq:craft:nmf}. The function $\implicit$ is defined for a given matrix $\m{A}$ and the tuple of matrices $(\m{U},\m{W},\bar{\m{U}},\bar{\m{W}})$ as follows:

\begin{theorem}[Implicit differentiation of NMF.] Let the optimality function $\implicit$ as introduced in~\cite{blondel2021implicitdiff} adapted for the Karush-Kuhn-Tucker (KKT) conditions~\cite{karush1939minima,kuhn1951nonlinear} capturing the optimality conditions of the problem~\ref{eq:craft:nmf} reads:

\begin{equation}
    \implicit((\m{U},\m{W},\bar{\m{U}},\bar{\m{W}}),\m{A})=
    \begin{cases}
    (\m{U}\m{W}^T-\m{A})\m{W}-\bar{\m{U}}    ,& \\ 
    (\m{W}\m{U}^T-\m{A}^T)\m{U}-\bar{\m{W}}  ,& \\ 
    \bar{\m{U}} \odot \m{U}   ,& \\ 
    \bar{\m{W}} \odot \m{W}   .& \\
    \end{cases}
\end{equation}

Given the optimal tuple $(\m{U},\m{W},\bar{\m{U}},\bar{\m{W}})$ that constitutes a root of $\implicit$ which is a root of $\implicit$, then, the implicit differentiation yields:

\begin{equation}
\frac{\partial (\m{U},\m{W},\bar{\m{U}},\bar{\m{W}})}{\partial \m{A}}=-(\partial_1 \implicit)^{-1}\partial_2 \implicit.
\end{equation}
\end{theorem}

See \autoref{app:craft:implicit} for full derivation. In particular, this requires the dual variables $\bar{\m{U}}$ and $\bar{\m{W}}$, which are not computed in scikit-learn's~\cite{pedregosa2011scikit} popular implementation\footnote{Scikit-learn uses a block coordinate descent algorithm~\cite{cichocki2009fast,fevotte2011algorithms}, with a randomized SVD initialization.}. Consequently, we leverage the work of~\cite{huang2016flexible} and we re-implement our own solver with Jaxopt~\cite{blondel2021implicitdiff} based on ADMM~\cite{boyd2011distributed}, a GPU friendly algorithm (see Appendix~\ref{app:craft:implicit}).



To ensure we audit the privacy of synthetic text data in a realistic setup, the synthetic data needs to bear high utility. We measure the synthetic data utility by comparing the downstream classification performance of RoBERTa-base~\citep{DBLP:journals/corr/abs-1907-11692} when fine-tuned exclusively on real or synthetic data. We fine-tune models for binary (SST-2) and multi-class classification (AG News) for 1 epoch on the same number of real or synthetic data records using a batch size of $16$ and learning rate $\eta = \num{1e-5}$. We report the macro-averaged AUC score and accuracy on a held-out test dataset of real records. 

Table~\ref{tab:utility_no_canaries} summarizes the results for synthetic data generated based on original data which does not contain any canaries. While we do see a slight drop in downstream performance when considering synthetic data instead of the original data, AUC and accuracy remain high for both tasks. 

\begin{table}[ht]
    \centering
    \begin{tabular}{ccrr}
    \toprule
        & \multirow{2}{*}{Fine-tuning data} & \multicolumn{2}{c}{Classification} \\
        \cmidrule(lr){3-4}
        Dataset &  & AUC & Accuracy \\
        \midrule 
        \multirow{2}{*}{\parbox{2cm}{\centering SST-2}} & Real & $0.984$ & \SI{92.3}{\percent} \\ 
         & Synthetic & $0.968$ & \SI{91.5}{\percent} \\
         \midrule
        \multirow{2}{*}{\parbox{2cm}{\centering AG News}} & Real & $0.992$ & \SI{94.4}{\percent} \\ 
         & Synthetic & $0.978$ & \SI{90.0}{\percent} \\ 
        \bottomrule
    \end{tabular}
    \caption{Utility of synthetic data generated from real data \emph{without} canaries. We compare the performance of text classifiers trained on real or synthetic data---both evaluated on real, held-out test data.}
    \label{tab:utility_no_canaries}
\end{table}

We further measure the synthetic data utility when the original data contains standard canaries (see Sec.~\ref{sec:baseline_results}). Specifically, we consider synthetic data generated from a target model trained on data containing \num{500} canaries repeated $n_\textrm{rep} = 12$ times, so \num{6000} data records. When inserting canaries with an artificial label, we remove all synthetic data associated with labels not present originally when fine-tuning the RoBERTa-base model. 

\begin{table}[h]
    \centering
    \begin{tabular}{ccc@{\hskip 15pt}rr}
    \toprule
        & \multicolumn{2}{c}{Canary injection} & \multicolumn{2}{c}{Classification}\\
        \cmidrule(lr){2-3} \cmidrule(lr){4-5}
        Dataset & Source & Label & AUC & Accuracy \\
        \midrule
        \multirow{3}{*}{\parbox{1cm}{\centering SST-2}} & \multicolumn{2}{l}{In-distribution} & $0.972$ & \SI{91.6}{\percent} \\ 
        \cmidrule{2-5}
         & \multirow{2}{*}{\parbox{1.8cm}{Synthetic}} & Natural & $0.959$ & \SI{89.3}{\percent} \\ 
         & & Artificial & $0.962$ & \SI{89.9}{\percent} \\ 
        \midrule
        \multirow{3}{*}{\parbox{2cm}{\centering AG News}} & \multicolumn{2}{l}{In-distribution} & $0.978$ & \SI{89.8}{\percent}\\ 
        \cmidrule{2-5} 
         & \multirow{2}{*}{\parbox{1.8cm}{Synthetic}} & Natural & $0.977$ & \SI{88.6}{\percent} \\ 
         & & Artificial & $0.980$ & \SI{90.1}{\percent} \\         
         \bottomrule
    \end{tabular}
    \caption{Utility of synthetic data generated from real data \emph{with} canaries ($n_\textrm{rep}=12$). We compare the performance of text classifiers trained on real or synthetic data---both evaluated on real, held-out test data.}
    \label{tab:utility_canaries}
\end{table}

Table~\ref{tab:utility_canaries} summarizes the results. Across all canary injection methods, we find limited impact of canaries on the downstream utility of synthetic data. While the difference is minor, the natural canary labels lead to the largest utility degradation. This makes sense, as the high perplexity synthetic sequences likely distort the distribution of synthetic text associated with a certain real label. In contrast, in-distribution canaries can be seen as up-sampling certain real data points during fine-tuning, while canaries with artificial labels merely reduce the capacity of the model to learn from real data and do not interfere with this process as much as canaries with natural labels do.


Concretely, given our concepts bank $\m{W}$, the concept attribution maps of a new input $\vx$ are calculated by solving the NNLS problem $\min_{\m{U} \geq 0} \frac{1}{2}\|\v{g}(\vx)-\m{U}\m{W}^\tr\|^2_F$. The implicit differentiation of the NMF block $\partial \m{U} / \partial \m{A}$ is integrated into the classic back-propagation to obtain $\partial \m{U} / \partial \vx$. Most interestingly, this technical advance enables the use of all white-box explainability methods~\cite{smilkov2017smoothgrad, zeiler2014visualizing, sundararajan2017axiomatic, selvaraju2017gradcam, springenberg2014striving} to generate concept-wise attribution maps and trace the part of an image that triggered the detection of the concept by the network. Additionally, it is even possible to employ black-box methods~\cite{ribeiro2016lime, petsiuk2018rise, lundberg2017unified, fel2021sobol} since it only amounts to solving an NNLS problem. %



\subsection{Experimental evaluation}\label{sec:craft:exp}


In order to evaluate the interest and the benefits brought by \craft, we start in Section~\ref{sec:craft:expUtility} by assessing the practical utility of the method on a human-centered benchmark composed of 3 XAI scenarios and presented in~\autoref{sec:attributions:metapred}.

After demonstrating the usefulness of the method using these human experiments, we independently validate the 3 proposed ingredients.
First, we provide evidence that recursivity allows refining concepts, making them more meaningful to humans using two additional human experiments in Section \ref{sec:craft:expRecursivity}.
Next, we evaluate our new Sobol estimator and show quantitatively that it provides a more faithful assessment of concept importance in Section~\ref{sec:craft:expSobol}.
Finally, we run an ablation experiment that measures the interest of local explanations based on concept attribution maps coupled with global explanations.
Additional experiments, including a sanity check and an example of deep dreams applied on the concept bank, as well as many other examples of local explanations for randomly picked images from ILSVRC2012, are included in Section~\ref{apx:craft:more-craft} of the supplementary materials.
We leave the discussion on the limitations of this method and on the broader impact in appendix~\ref{apx:craft:limitations}.

\subsubsection{Utility Evaluation}
\label{sec:craft:expUtility}


As emphasized by Doshi-Velez et al.~\cite{doshivelez2017rigorous}, the goal of XAI should be to develop methods that help a user better understand the behavior of deep neural network models. An instantiation of this idea was proposed in \autoref{sec:attributions:metapred} where we described an experimental framework to quantitatively measure the practical usefulness of explainability methods in real-world scenarios. In the initial setup, we recruited  $n=1,150$ online participants (evaluated over 8 unique conditions and 3 AI scenarios) -- making it the largest benchmark to date in XAI. Here, we extend our framework to allow for the robust evaluation of the utility of our proposed \craft~method and the related ACE.
The 3 representative real-world scenarios are: (1) identifying bias in an AI system (using Husky vs Wolf dataset from~\cite{ribeiro2016lime}), (2) characterizing the visual strategy that are too difficult for an untrained non-expert human observer (using  the Paleobotanical dataset from \cite{wilf2016computer}), (3) understanding complex failure cases (using ImageNet ``Red fox'' vs ``Kit fox'' binary classification).
Using this benchmark, we evaluate \craft, ACE, as well as \craft~with only the global concepts (\craft CO) to allow for a fair comparison with ACE.
To the best of our knowledge, we are the first to systematically evaluate concept-based methods against attribution methods.

Results are shown in Table~\ref{tab:utility} and demonstrate the benefit of \craft, which achieves higher scores than all of the attribution methods tested as well as ACE in the first two scenarios. To date, no method appears to exceed the baseline on the third scenario suggesting that additional work is required.
We also note that, in the first two scenarios, \craft CO is one of the best-performing methods and it always outperforms ACE -- meaning that even without the local explanation of the concept attribution maps, \craft~largely outperforms ACE. Examples of concepts produced by \craft~are shown in the Appendix~\ref{app:craft:utility}.


\subsubsection{Validation of Recursivity}
\label{sec:craft:expRecursivity}

\begin{table}
\centering
\begin{tabular}{lll}
\toprule
& Experts ($n=36$) & Laymen ($n=37$)\\
\cmidrule[0.1pt](lr){2-3}
\textit{Intruder}  \\
\cmidrule[0.1pt](lr){1-3}
Acc. Concept     & 70.19\%  & 61.08\%     \\
Acc. Sub-Concept & 74.81\% ($p = 0.18$)  & \textbf{67.03}\% ($p = 0.043$)      \\
\midrule
\textit{Binary choice} \\
\cmidrule[0.1pt](lr){1-3}
Sub-Concept & \textbf{76.1}\% ($p < 0.001$) & \textbf{74.95}\% ($p < 0.001$)\\
Odds Ratios & $3.53$ & $2.99$\\
\bottomrule
\end{tabular}
\caption{\textbf{Results from the psychophysics experiments to validate the recursivity ingredient. }}\label{tab:results}
\end{table}


To evaluate the meaningfulness of the extracted high-level concepts, we performed psychophysics experiments with human subjects, whom we asked to answer a survey in two phases. Furthermore, we distinguished two different audiences: on the one hand, experts in machine learning, and on the other hand, people with no particular knowledge of computer vision. Both groups of participants were volunteers and did not receive any monetary compensation. Some examples of the developed interface are available the appendix~\ref{app:craft:human-exp}. It is important to note that this experiment was carried out independently from the utility evaluation and thus it was setup differently.
\newline\textbf{Intruder detection experiment} First, we ask users to identify the intruder out of a series of five image crops belonging to a certain class, with the odd one being taken from a different concept but still from the same class. Then, we compare the results of this intruder detection with another intruder detection, this time, using a concept (e.g., $\mathcal{C}_1$) coming from a layer $l$ and one of its sub-concepts (e.g., $\mathcal{C}_{12}$ in Fig.\ref{fig:craft:collapse}) extracted using our recursive method. If the concept (or sub-concept) is coherent, then it should be easy for the users to find the intruder.
Table~\ref{tab:results} summarizes our results, showing that indeed both concepts and sub-concepts are coherent, and that recursivity can lead to a slightly higher understanding of the generated concepts (significant for non-experts, but not for experts) and might suggest a way to make concepts more interpretable.
\newline\textbf{Binary choice experiment} In order to test the improvement of coherence of the sub-concept generated by recursivity with respect to the larger parent concept, we showed participants an image crop belonging to both a subcluster and a parent cluster (e.g.,  $\bm{\pi}(\vx) \in \mathcal{C}_{11} \subset \mathcal{C}_1$) and asked them which of the two clusters (i.e., $\mathcal{C}_{11}$ or $\mathcal{C}_{1}$) seemed to accommodate the image the best. If our hypothesis is correct, then the concept refinement brought by recursivity should help form more coherent clusters.
The results in Table~\ref{tab:results} are satisfying since in both the expert and non-expert groups, the participants chose the sub-cluster more than 74\% of the time. We measure the significance of our results by fitting a binomial logistic regression to our data, and we find that both groups are more likely to choose the sub-concept cluster (at a $p < 0.001$).

\subsubsection{Fidelity analysis} \label{sec:craft:expSobol}

We propose to simultaneously verify that identified concepts are faithful to the model and that the concept importance estimator performs better than that used in TCAV~\cite{kim2018interpretability} by using the fidelity metrics introduced in \cite{ghorbani2019towards, zhang2021invertible}. These metrics are similar to the ones used for attribution methods, which consist of studying the change of the logit score when removing/adding pixels considered important. Here, we do not introduce these perturbations in the pixel space but in the concept space: once $\m{U}$ and $\m{W}$ are computed, we reconstruct the matrix $\m{A}\approx \m{U}\m{W}^\tr$ using only the most important concept (or removing the most important concept for deletion) and compute the resulting change in the output of the model.  As can be seen from Fig.~\ref{fig:craft:deletion}%
, ranking the extracted concepts using Sobol's importance score results in steeper curves than when they are sorted by their TCAV scores. %
We confirm that these results generalize with other matrix factorization techniques (PCA, ICA, RCA) in Section~\ref{app:craft:fidelity} of the Appendix.

\begin{figure}[ht]
\includegraphics[width=\linewidth]{assets/craft/nmf_fidelity.png}
\caption{
\textbf{(Left)} Deletion curves (lower is better). \textbf{(Right)} Insertion curves (higher is better). 
For both the deletion or insertion metrics, Sobol indices lead to better estimates (calculated on >100K images) of important concepts. %
}
\label{fig:craft:deletion}
\end{figure}




\subsection{Conclusion}

In this first section, we introduced \craft, a method for automatically extracting human-interpretable concepts from deep networks. Our method aims to explain a pre-trained model's decisions both on a per-class and per-image basis by highlighting both ``\what'' the model saw and ``\where'' it saw it  -- with complementary benefits. The approach relies on 3 novel ingredients: \tbi{i} a recursive formulation of concept discovery to identify the correct level of granularity for which individual concepts are understandable; \tbi{ii} a novel method for measuring concept importance through Sobol indices to more accurately identify which concepts influence a model's decision for a given class; and \tbi{iii} the use of implicit differentiation methods to backpropagate through non-negative matrix factorization (NMF) blocks to allow the generation of concept-wise local explanations or \textit{concept attribution maps} independently of the attribution method used. Using our previously introduced human-centered utility benchmark, we conducted psychophysics experiments to confirm the validity of the approach: and that the concepts identified by \craft~are useful and meaningful to human experimenters. 

\clearpage

\section{Application: FRSign}
\label{sec:concepts:frsign}
In this section, we examine the application of attribution methods to models trained on the FRSign dataset~\cite{2020frsign}, and use our recently introduced Sobol method. This dataset, containing images of French railway signals, serves as a practical case for assessing our attribution technique's effectiveness in making models more transparent. Detailed setup is documented in~\autoref{sec:intro:frsign}. Our analysis primarily features results from ResNet50, yet findings are applicable to VGG and ViT models.

For this application, our focus will be twofold: firstly, we will analyze fidelity scores to determine which attribution methods is more faithful; secondly, we aim to understand the model's strategies for the classes under study. We will observe that for most classes, the model appears to use plausible features. However, for one class, the attributions are somewhat mysterious. To have deeper understanding, we will employ feature visualization to formulate a diagnosis and hypotheses.

\subsection{Fidelity Scores}

We begin with a fidelity measure to identify which attribution methods best transcribe the model's behavior. \autoref{tab:frsign:fidelity} displays the results computed from 100 randomly selected images from the test dataset\footnote{It should be noted that the question of whether it is relevant to apply explainability to the training set remains open. Up to my knowledge, I see no a priori issues with it, but out of an abundance of caution and to ensure that nothing is overlooked, we will exclusively conduct explainability analyses on the test set.}.

\begin{table}[h]
\centering
\begin{tabular}{l c c}
\textbf{Attribution Method}&\textbf{Deletion Score}&\textbf{Insertion Score}\\
\hline
Sobol&\textbf{0.329}&0.377\\
RISE&0.348&\textbf{0.396}\\
Saliency&0.402&0.325\\
Integrated Gradient&0.396&0.348\\
Grad-CAM&0.419&0.372\\
SmoothGrad&\underline{0.338}&0.363\\
Occlusion&0.345&\underline{0.380}\\
\\
\hline
\end{tabular}
\caption{\textbf{Insertion and Deletion Scores for Seven Attribution Methods on the FRSign Dataset.} This table presents the scores for each attribution method according to the fidelity metrics of insertion and deletion. It's important to remember that a lower deletion score is preferable, and a higher insertion score is considered better. The best method is highlighted in \textbf{bold}, while the second best is \underline{underlined}. The methods that appear to be the most effective are RISE, Sobol, and SmoothGrad.}
\label{tab:frsign:fidelity}
\end{table}

In \autoref{tab:frsign:fidelity}, we observe that the method we previously introduced also achieves favorable deletion scores. This finding is reassuring as it suggests that Sobol's performance generalizes beyond the datasets studied earlier. Additionally, RISE and SmoothGrad both exhibit strong performance across both metrics. Therefore, for the remainder of our study, we will primarily focus on these three methods to draw our conclusions.

\begin{figure}[ht!]
\centering
\includegraphics[width=0.9\textwidth]{assets/frsign/good_attributions1.jpg}
\includegraphics[width=0.9\textwidth]{assets/frsign/good_attributions2.jpg}
\caption{\textbf{Comparative Visual Analysis of Attribution Methods.} We applied seven attribution methods to our model trained on the FRSign dataset. According to~\autoref{tab:frsign:fidelity}, the most faithful methods are Sobol, RISE, and SmoothGrad. Remarkably, the model tends to focus on the areas it should, specifically the traffic lights (or light) for the target class, which is reassuring.}
\label{fig:frsign:good_attributions}
\end{figure}

\subsection{Comparative Visual Analysis}

After computing the fidelity scores, we have a clearer understanding of which attribution methods more accurately reflect the model's decisions. This allows us to place greater trust in certain methods over others based on these initial tests. Sobol, RISE, and SmoothGrad emerge as the top three methods. However, we will continue to consider all methods to comprehensively assess our results. An interesting observation is that when all methods achieve good fidelity scores but highlight different areas of importance, this could be interpreted as indicating multiple ways to explain the model's reasoning. This is an intriguing aspect to explore~\footnote{Some preliminary remarks have been done on this topic in~\cite{aggregating2020}, but the ``diversity'' of explanation and the capability to aggregate them is still an interesting open questions.}. Nonetheless, it is important to remember that methods with lower fidelity scores should be approached with caution.

Figure~\autoref{fig:frsign:good_attributions} displays examples of attributions for each class that appear to be accurate. For critical signals such as violet, red, and yellow lights, the model seems to focus on the specific light or lights it is supposed to, which could increase our confidence in the model's decision-making for these types of signals.

These examples focus solely on instances where the model's predictions are correct. Next, we will apply our attribution methods to investigate failure cases, that is, instances where the model has made incorrect predictions.

\subsection{Explaining Failure Cases}

Despite the ResNet-50 model being our most performant, with an accuracy above 90\%, it is not without its share of incorrect predictions. Figure \autoref{fig:frsign:bad_attributions} presents several examples of explanations for misclassified points.

\begin{figure}[ht!]
\centering
\includegraphics[width=0.95\textwidth]{assets/frsign/bad_labels.jpg}
\caption{\textbf{Attribution Methods on wrongly classified points.} We applied the same seven attribution methods to points where the model's predictions were incorrect. ``P'' denotes the model's prediction, $\f(\vx)$, and ``GT'' for the label $y$. Upon review, the human eye tends to agree with the model's predictions, which might lead us to suspect incorrect labeling. However, for some images, the labeling is indeed accurate, and it is light aberrations or capture problem that obscure the correct ground truth from view.}
\label{fig:frsign:bad_attributions}
\end{figure}

Upon further analysis, it appears that a portion of the data points were indeed incorrectly labeled, while a significant number are correctly labeled, although human observation alone may not always accurately identify the correct label due to noise, errors, or anomalies in the image capture process. This leads to an intriguing question that extends beyond the scope of this thesis: whether the model or the label is at fault. In other words, if in reality a signal was violet but appears red in our images, should we expect the model to perceive it as humans do, with all associated biases, or should it interpret the data optimally for the task at hand, potentially employing mechanisms or perceptions different from those of humans? These considerations open up a broader discourse, yet there is one final observation to be made before concluding this section.

The analysis of failure cases does not encompass the entirety of our observations. There remains one particularly perplexing scenario, observed post-analysis: the case of the white signals.

\subsection{White Signal}

The interpretability of white signals poses a challenge, as the focal points of the model remain unclear. This is illustrated in Figure \autoref{fig:frsign:white_signals}.

\begin{figure}[ht!]
\centering
\includegraphics[width=0.95\textwidth]{assets/frsign/white_fire_attributions.jpg}
\caption{\textbf{Attribution Methods on White Signals.} The set of explainability methods applied to images correctly predicted as white signals is concerning. The model appears to focus on areas other than the traffic lights; however, it is unclear what specifically garners the model's attention.}
\label{fig:frsign:white_signals}
\end{figure}

The areas of attention for white signals appear cryptic and are not consistently focused on the lights. Furthermore, the focus does not always seem to be located in the same manner, which prevents a clear understanding of what the model is observing or relying upon for its decisions. We will now employ feature visualizations to delve deeper into this issue.

\paragraph{Feature Visualization.} To gain a better understanding of the potential strategies employed by our model, we utilized feature visualization. The results are shown in \autoref{fig:frsign:fviz}.

\begin{figure}[ht]
\centering
\includegraphics[width=0.95\textwidth]{assets/frsign/fviz_fire_1.jpg}
\includegraphics[width=0.95\textwidth]{assets/frsign/fviz_fire_2.jpg}
\caption{\textbf{Feature Visualization for the Logits of $\f$.} The images represent the results of two settings of feature visualization (in Fourier space) for the image that maximizes the logits for the classes yellow, red, green, white, and violet.}
\label{fig:frsign:fviz}
\end{figure}

For crucial traffic lights such as red, yellow, and violet, the feature visualizations seem to make sense, which is reassuring. However, for white, the interpretations remain somewhat cryptic. Nonetheless, we can hypothesize that the model focuses on the frame (contour of the light) as indicated in the top feature visualization for the white light in \autoref{fig:frsign:fviz}.

\subsection{Conclusion}

Attribution methods serve as a valuable tool for understanding the model and verifying that it relies on plausible features. They provide reassurance in most cases by ensuring that the areas most important to the model are also those containing information meaningful to humans.

However, two main issues arise. Firstly, we wish to extend our methods to offer stronger guarantees; that is, to establish confidence bounds around our explanations to ensure the model's reliance on these interpretations. Secondly, in some instances, the features the model focuses on remain ambiguous, such as with the case of white signals. This observation suggests that further research is necessary to make attribution methods both safer and more informative.

\clearpage

\section{Unifying Automatic Concept Extraction and Concept Importance Estimation}
\label{sec:concepts:holistic}
\subsection{Attribution methods for Concepts}\label{sup:holistic:all_cams}

In the following section, we will re-derive the different attribution methods in the literature. We use the Xplique library and adapted each methods~\cite{fel2022xplique}.
We quickly recall that we seek to estimate the importance of each concept for a set of concept coefficients $\v{u} = (\v{u}_1, \ldots, \v{u}_k) \in \mathbb{R}^k$ in the concept basis $\m{V} \in \mathbb{R}^{p \times k}$. This concept basis is a re-interpretation of a latent space (in $\mathbb{R}^{p}$) and the function $\fb: \mathbb{R}^{p} \to \mathbb{R}$ is a signal used to compute importance from (e.g., logits value, cosine similarity with a sentence...). Each Attributions method will map a set of concept values to an importance score $\cam: \mathbb{R}^k \to \mathbb{R}^k$, a greater score $\cam(\v{u})_i$ indicates that a concept $\v{u}_i$ is more important. 

\textbf{Saliency (SA)}~\cite{simonyan2013deep} was originally a visualization technique based on the gradient of a class score relative to the input, indicating in an infinitesimal neighborhood, which pixels must be modified to most affect the score of the class of interest. In our case, it indicates which concept in an infinitesimal neighborhood has the most influence on the output:

$$ \cam^{(SA)}(\v{u}) = \nabla_{\v{u}} \fb(\v{u} V^\tr) .$$

\textbf{Gradient $\odot$ Input (GI)}~\cite{shrikumar2017learning} is based on the gradient of a class score relative to the input, element-wise with the input, it was introduced to improve the sharpness of the attribution maps. A theoretical analysis conducted by~\cite{ancona2017better} showed that Gradient $\odot$ Input is equivalent to $\epsilon$-LRP and DeepLIFT~\cite{shrikumar2017learning} methods under certain conditions -- using a baseline of zero, and with all biases to zero. In our case, it boils down to:

$$ \cam^{(GI)}(\v{u}) = \v{u} \odot \nabla_{\v{u}} \fb(\v{u} \m{V}^\tr) .$$

\textbf{Integrated Gradients (IG)}~\cite{sundararajan2017axiomatic} consists of summing the gradient values along the path from a baseline state to the current value. The baseline $\v{u}_0$ used is zero. This integral can be approximated with a set of $m$ points at regular intervals between the baseline and the point of interest. In order to approximate from a finite number of steps, we use a trapezoidal rule and not a left-Riemann summation, which allows for more accurate results and improved performance (see~\cite{sotoudeh2019computing} for a comparison). For all the experiments $m = 30$.

$$ \cam^{(IG)}(\v{u}) = (\v{u} - \v{u}_0) \int_0^1 \nabla_{\v{u}} \fb((\v{u}_0 + \alpha(\v{u} - \v{u}_0))\m{V}^\tr) \dif\alpha. $$

\textbf{SmoothGrad (SG)}~\cite{smilkov2017smoothgrad} is also a gradient-based explanation method, which, as the name suggests, averages the gradient at several points corresponding to small perturbations (drawn i.i.d from an isotropic normal distribution of standard deviation $\sigma$) around the point of interest. The smoothing effect induced by the average helps to reduce the visual noise, and hence improves the explanations. In our case, the attribution is obtained after averaging $m$ points with noise added to the concept coefficients. For all the experiments, we took $m = 30$ and $\sigma = 0.1$.

$$ \cam^{(SG)}(\v{u}) = \underset{\bm{\delta} \sim \mathcal{N}(0, \mathbf{I}\sigma)}{\mathbb{E}}(\nabla_{\v{u}} \fb( \v{u} + \bm{\delta}) ).
$$

\textbf{VarGrad (VG)}~\cite{hooker2018benchmark} was proposed as an alternative to SmoothGrad as it employs the same methodology to construct the attribution maps: using a set of $m$ noisy inputs, it aggregates the gradients using the variance rather than the mean. For the experiment, $m$ and $\sigma$ are the same as SmoothGrad. Formally:

$$ \cam^{(VG)}(\v{u}) = \underset{\bm{\delta} \sim \mathcal{N}(0, \mathbf{I}\sigma)}{\mathbb{V}}(\nabla_{\v{u}} \fb( \v{u} + \bm{\delta}) ).
$$

\textbf{Occlusion (OC)}~\cite{zeiler2013visualizing} is a simple -- yet effective -- sensitivity method that sweeps a patch that occludes pixels over the images using a baseline state and use the variations of the model prediction to deduce critical areas. In our case, we simply omit each concept one-at-a-time to deduce the concept's importance. For all the experiments, the baseline state $\v{u}_0$ was zero.

$$ \cam^{(OC)}(\v{u})_i = \fb(\v{u} \m{V}^\tr) - \fb(\v{u}_{[i = \v{u}_0]} \m{V}^\tr)  $$


\textbf{Sobol Attribution Method (SM)}~\cite{fel2021sobol} then used for estimating concept importance in \cite{fel2023craft} is a black-box attribution method grounded in Sensitivity Analysis. Beyond modeling the individual contributions of image regions, Sobol indices provide an efficient way to capture higher-order interactions between image regions and their contributions to a neural network’s prediction through the lens of variance. In our case, the score for a concept $\v{u}_i$ is the expected variance that would be left if all variables but $i$ were to be fixed : 

$$ \cam^{(SM)}(\v{u})_i = \frac{ \mathbb{E}( \mathbb{V}( \fb( (\v{u} \odot \mathbf{M} ) \m{V}^\tr ) | \mathbf{M}_{\sim i} ) ) }{ \mathbb{V}( \fb( (\v{u} \odot \mathbf{M} ) \m{V}^\tr)) } . $$

With $\mathbf{M} \sim \mathcal{U}([0, 1])^k$. For all the experiments, the number of designs was $32$ and we use the Jansen estimator of the Xplique library.

\textbf{HSIC Attribution Method (HS)}~\cite{novello2022making} seeks to explain a neural network's prediction for a given input image by assessing the dependence between the output and patches of the input. In our case, we randomly mask/remove concepts and measure the dependence between the output and the presence of each concept through $N$ binary masks. Formally:

$$ \cam^{(HS)}(\v{u}) = \frac{1}{(N-1)^2} \mathrm{Tr}(KHLH). $$

With $H, L, K \in \mathbb{R}^{N \times N}$ and $K_{ij} = k(\mathbf{M}_i, \mathbf{M}_j)$, $L_{ij} = l(\vy_i, \vy_j)$ and $H_{ij} = \delta(i=j)-N^{-1}$. Here, $k(\cdot, \cdot)$ and $l(\cdot, \cdot)$ denote the chosen kernels and $\mathbf{M} \sim \{0, 1\}^p$ the binary mask applied to the input $\v{u}$.

\textbf{RISE (RI)}~\cite{petsiuk2018rise} is also a black-box attribution method that probes the model with multiple version of a masked input to model the most important features. Formally, with $\bm{m} \sim \mathcal{U}([0, 1])^k$. : 

$$ \cam^{(RI)}_i(\v{u}) =  
\mathbb{E}(\fb( \v{u} \odot \bm{m} ) | \bm{m}_i = 1).
$$

\subsection{Closed-form of Attributions for the last layer}\label{sup:holistic:closed_form}

Without loss of generality, we focus on the decomposition in the last layer, that is $\v{a} = \v{u}\m{V}^\tr$ with parameters $(\m{W}, \bias)$ for the weight and the bias respectively, hence we obtain $\vy = (\v{u} \m{V}^\tr)\m{W} + \bias$ with $\m{W} \in \mathbb{R}^{p}$ and $\bias \in \mathbb{R}$.



We start by deriving the closed form of Saliency (SA) and naturally Gradient-Input (GI):

\begin{flalign*}
\cam^{(SA)}(\v{u}) 
&= \nabla_{\v{u}} \fb(\v{u} \m{V}^\tr)
= \nabla_{\v{u}} (\v{u} \m{V}^\tr \m{W} + \bias) &\\
&= \m{W}^\tr \m{V}&.
\end{flalign*}
\begin{flalign*}
\cam^{(GI)}(\v{u}) 
&= \nabla_{\v{u}} \fb(\v{u} \m{V}^\tr) \odot \v{u} 
= \nabla_{\v{u}} (\v{u} \m{V}^\tr \m{W} + \bias) \odot \v{u} &\\
&= \m{W}^\tr \m{V} \odot \v{u} &.
\end{flalign*}

We observe two different forms that will in fact be repeated for the other methods, for example with Integrated-Gradient (IG) which will take the form of Gradient-Input, while SmoothGrad (SG) will take the form of Saliency.

\begin{flalign*}
\cam^{(IG)}(\v{u})
 &= (\v{u} - \v{u}_0) \odot \int_0^1 \nabla_{\v{u}} \fb((\v{u}_0 + \alpha (\v{u} - \v{u}_0)) \m{V}^\tr) \dif \alpha &\\
 &= \v{u} \odot \int_0^1 \nabla_{\v{u}}((\alpha \v{u})) \m{V}^\tr\m{W} + \bias + (\alpha-1)\v{u}_0\m{V}^\tr\m{W}) \dif \alpha &\\
 &= \v{u} \odot \int_0^1 \alpha\m{W}^\tr \dif \alpha = \v{u} \odot \m{W}^\tr \m{V} \left[\frac{1}{2}\alpha^2\right]_0^1\\
 &= \frac{1}{2}\v{u} \odot \m{W}^\tr \m{V}.
\end{flalign*}

\begin{flalign*}
\cam^{(SG)}(\v{u})
&= \underset{\bm{\delta} \sim \mathcal{N}(0, \mathbf{I}\sigma)}{\mathbb{E}}(\nabla_{\v{u}} \fb( \v{u} + \bm{\delta}) ) 
= \underset{\bm{\delta} \sim \mathcal{N}(0, \mathbf{I}\sigma)}{\mathbb{E}}(\nabla_{\v{u}}( (\v{u} + \bm{\delta}) \m{V}^\tr\m{W} + \bias) ) & \\
& = \underset{\bm{\delta} \sim \mathcal{N}(0, \mathbf{I}\sigma)}{\mathbb{E}}(\nabla_{\v{u}}(\v{u} \m{V}^\tr\m{W})) & \\
& = \m{W}^\tr \m{V} &.
\end{flalign*}

The case of VarGrad is specific, as the gradient of a linear system being constant, its variance is null.

\begin{flalign*}        
\cam^{(VG)}(\v{u})
&= \underset{\bm{\delta} \sim \mathcal{N}(0, \mathbf{I}\sigma)}{\mathbb{V}}(\nabla_{\v{u}} \fb( \v{u} + \bm{\delta}) )
= \underset{\bm{\delta} \sim \mathcal{N}(0, \mathbf{I}\sigma)}{\mathbb{V}}(\nabla_{\v{u}} ( (\v{u} + \bm{\delta}) \m{V}^\tr \m{W} + \bias) ) & \\
&= \underset{\bm{\delta} \sim \mathcal{N}(0, \mathbf{I}\sigma)}{\mathbb{V}}(\m{W}^\tr \m{V}) &\\
&= 0&.
\end{flalign*}

Finally, for Occlusion (OC) and RISE (RI), we fall back on the Gradient Input form (with multiplicative and additive constant for RISE).

\begin{flalign*}
\cam^{(OC)}_i(\v{u})
&= \fb(\v{u} \m{V}^\tr) - \fb(\v{u}_{[i = \v{u}_0]} \m{V}^\tr)
= \v{u} \m{V}^\tr\m{W} + \bias - (\v{u}_{[i = \v{u}_0]} \m{V}^\tr\m{W} + \bias) & \\
& = (\sum_{j}^{r} \v{u}_j \m{V}_j^\tr)\m{W} - (\sum_{j \neq i}^{r} \v{u}_j \m{V}_j^\tr)\m{W} &\\
& = \v{u}_i \m{V}_i^\tr \m{W} &
\end{flalign*}
thus $\cam^{(OC)}(\v{u}) = \v{u} \odot \m{W}^\tr \m{V}$

\begin{flalign*}
\cam^{(RI)}_i(\v{u})
&= \mathbb{E}(\fb( \v{u} \odot \bm{m} ) | \bm{m}_i = 1)
= \mathbb{E}( (\v{u}\odot\bm{m}) \m{V}^\tr\m{W} + \bias | \bm{m}_i = 1) & \\
& = \bias + \sum_{j \neq i}^r \v{u}_j \mathbb{E}(\bm{m}_j) \m{V}_j^\tr\m{W} + \v{u}_i \m{V}_i^\tr\m{W} &\\
& = \bias + \frac{1}{2} (\v{u}\m{V}^\tr\m{W} + \v{u}_i \m{V}_i^\tr\m{W})&
\end{flalign*}

\subsection{Fidelity optimality}\label{sup:holistic:fidelity_theorem}

Before showing that some methods are optimal with regard to C-Deletion and C-Insertion, we start with a first metric that studies the fidelity of the importance of concepts: $\mu$Fidelity, whose definition we recall

$$
\mu F = \underset{\substack{S \subseteq \{1, \ldots, k\} \\ |S| = m} }{\rho}(
\sum_{i \in S} \cam(\v{u})_i,
\fb(\v{u}) - \fb(\v{u}_{[\v{u}_i = \v{u}_0, i \in S]})
)
$$

With $\rho$ the Pearson correlation and $\v{u}_{[\v{u}_i = \v{u}_0, i \in S]}$ means that all $i$ components of $\v{u}$ are set to zero.

\begin{theorem}[Optimal $\mu$Fidelity in the last layer]
When decomposing in the last layer,~\textbf{Gradient Input}, \textbf{Integrated Gradients}, \textbf{Occlusion}, and \textbf{Rise} yield the optimal solution for the $\mu$Fidelity metric.
In a more general sense, any method $\cam(\v{u})$ that is of the form
$\cam_{i}(\v{u}) = a (\v{u}_i\m{V}_i^\tr \m{W}) + b $ with $a \in \mathbb{R}^+, b \in \mathbb{R}$ yield the optimal solution, thus having a correlation of 1.
\end{theorem}
\begin{proof}
In the last layer case, $\mu$Fidelity boils down to:

\begin{flalign*}
\mu F &= \underset{\substack{S \subseteq \{1, \ldots, k\} \\ |S| = m} }{\rho}\big(\sum_{i \in S} \cam(\v{u})_i,
\v{u} \m{V}^\tr \m{W} + \bias - ( \sum_{i \notin S} \v{u}_i \m{V}_i^\tr \m{W}) - \bias
\big) & \\
&= \underset{\substack{S \subseteq \{1, \ldots, k\} \\ |S| = m} }{\rho}\big(\sum_{i \in S} \cam(\v{u})_i,
\sum_{i \in S} \v{u}_i \m{V}_i^\tr \m{W}
\big) &
\end{flalign*}

We recall that for \textbf{Gradient Input}, \textbf{Integrated Gradients}, \textbf{Occlusion}, $\cam_i(\v{u}) \propto \v{u}_i \m{V}_i^\tr \m{W}$, thus 
\begin{flalign*}
\mu F &= \underset{\substack{S \subseteq \{1, \ldots, k\} \\ |S| = m} }{\rho}\big(
\sum_{i \in S} \v{u}_i \m{V}_i^\tr \m{W},
\sum_{i \in S} \v{u}_i \m{V}_i^\tr \m{W}
\big) = 1 &
\end{flalign*}
For \textbf{RISE}, we get the following characterization:
\begin{flalign*}
\mu F &= \underset{\substack{S \subseteq \{1, \ldots, k\} \\ |S| = m} }{\rho}\big(
\sum_{i \in S} \bias + \frac{1}{2} (\v{u}\m{V}^\tr\m{W} + \v{u}_i \m{V}_i^\tr\m{W})
,
\sum_{i \in S} \v{u}_i \m{V}_i^\tr \m{W}
\big) & \\
&= \underset{\substack{S \subseteq \{1, \ldots, k\} \\ |S| = m} }{\rho}\big(
|S|(\bias + \frac{1}{2} (\v{u}\m{V}^\tr\m{W})) + 
\sum_{i \in S} \frac{1}{2} \v{u}_i \m{V}_i^\tr\m{W}
,
\sum_{i \in S} \v{u}_i \m{V}_i^\tr \m{W}
\big) & \\
&= \underset{\substack{S \subseteq \{1, \ldots, k\} \\ |S| = m} }{\rho}\big(
a(  
\sum_{i \in S} \v{u}_i \m{V}_i^\tr\m{W}) + b
,
\sum_{i \in S} \v{u}_i \m{V}_i^\tr \m{W}
\big)  = 1 & \\
\end{flalign*}

with $a = \frac{1}{2}, b = m(\bias + \frac{1}{2} (\v{u}\m{V}^\tr\m{W}))$. 

\end{proof}

\subsection{Optimality for C-Insertion and C-Deletion}\label{sup:holistic:matroid}

In order to prove the optimality of some attribution methods on the C-Insertion and C-Deletion metrics, we will use the Matroid theory of which we recall some fundamentals.




Matroids were introduced by Whitney in 1935~\cite{whitney1992abstract}. 
It was quickly realized that they unified properties of various domains such as graph theory, linear algebra or geometry. 
Later, in the '60s, a connection was made with combinatorial optimization, nothing that they also played a central role in combinatorial optimization. 

The power of this tool is that it allows us to show easily that greedy algorithms are optimal with respect to some criterion on a broad range of problems. Here, we show that insertion is a greedy algorithm (since the concepts inserted are chosen sequentially based on the model score).

For the rest of this section, we assume $E = \{ e_1, \ldots, e_k \}$ the set of the canonical vectors in $\mathbb{R}^k$, with $e_i$ being the element associated with the $i^{th}$ concept.

\begin{definition}[Matroid] A matroid $M$ is a tuple $(E, \mathcal{J})$, where E is a finite ground set and $\mathcal{J} \subseteq 2^E$ is the power set of $E$, a collection of independent sets, such that:

\begin{enumerate}
  \item $\mathcal{J}$ is nonempty, $\emptyset \in \mathcal{J}$.
  \item $\mathcal{J}$ is downward closed; i.e., if $S \in \mathcal{J}$ and $S' \subseteq S$, then $S' \in \mathcal{J}$ 
  \item If $S, S' \in \mathcal{J}^2$ and $|S| < |S'|$, then $\exists s \in S' \setminus S$ such that $S \cup \{s\} \in \mathcal{J}$
\end{enumerate}

\end{definition}

In particular, we will need uniform matroids: 

\begin{definition}[Uniform Matroid] 
\label{def:matroid}
Let $E$ be a set of size $k$ and let $n \in \{1, \ldots, k \}$. If $\mathcal{J}$ is the collection of all subsets of $E$ of size at most $n$, then $(E, \mathcal{J})$ is a matroid, called a uniform matroid and denoted $M^{(n)}$.
\end{definition}


Finally, we need to characterize the concept set chosen at each step.

\begin{definition}[Base of Matroid] 
Let $M = (E, \mathcal{J})$ be a matroid. A subset $B$ of $E$ is called a basis of $M$ if and only if:
\begin{enumerate}
  \item $B \in \mathcal{J}$
  \item $\forall e \in E \setminus B, ~ B \cup \{e\} \notin \mathcal{J}$
\end{enumerate}
Moreover, we denote $\mathcal{B}(M)$ the set of all the basis of $M$.
\end{definition}


At each step, the insertion metric selects the concepts of maximum score given a cardinality constraint. At each new step, the concepts from the previous step are selected and it add a new concept from the whole available set, the one not selected so far with the highest score.  
This criterion requires an additional ingredient: the \emph{weight} associated to each element of the matroid - here an element of the matroid is a concept.

\paragraph{Ponderated Matroid}

Let $M^{(n)} = (E, \mathcal{J})$ be a uniform matroid and $w : E \to \mathbb{R}$ a weighting function associated to an element of $E$ (a concept).
The goal of C-Insertion at step $n$ is to find a basis (a set of concepts) $B^\star$ subject to $|B| = n$, that maximizes the weighting function : 

$$
\forall B \in \mathcal{J}, ~~ \sum_{e \in B^\star} w(e) \geq \sum_{e \in B} w(e).
$$

Such a basis is called the basis of maximum weights (MW) of the weighted matroid $M^{(n)}$. We will see that the greedy algorithm associated with this weighting function gives the optimal solution to the MW problem on C-Insertion. First, let's define the \emph{Greedy algorithm}.

\begin{algorithm}[ht]
\caption{Greedy algorithm}\label{alg:greedy_matroide}
\begin{algorithmic}
  \REQUIRE A $n$-uniform weighted matroid $M^{(n)} = (E, \mathcal{J}, w)$
  \STATE Sort the concepts by their weight $w(e_i)$ in non-increasing order, and store them in a list $\bar{e}$ such that~${\forall (i, j) \subseteq \{1, \ldots, k\}^2, w(\bar{e}_i) \geq w(\bar{e}_j) ~ \text{if} ~ i < j}$.
  \STATE $B^{\star} = \{\}$
  \FOR{$k = 1$ to $n$}
    \STATE $B^{\star} = B^{\star} \cup \bar{e}_k$ %
  \ENDFOR
  \STATE \textbf{Return} $B^{\star}$
\end{algorithmic}
\end{algorithm}


\begin{theorem}[Greedy Algorithm is an optimal solution to MW.] Let $M = (E, \mathcal{J}, w)$ a weighted matroid. The greedy Algorithm~\ref{alg:greedy_matroide} returns a maximum basis of $M$.
\end{theorem}

\begin{proof}
First, by definition, $B^\star$ is a basis and thus an independent set, i.e., $B^\star \in \mathcal{B}(M)$ (as $\forall (e,e') \in E^2, ~ \langle e,e' \rangle = 0$).
Now, suppose by contradiction that there exists a base $B'$ with a weight strictly greater than $B^\star$. We will obtain a contradiction with respect to the augmentation axiom of the matroid definition.
Let $e_1, \ldots, e_k$ be the elements of $M$ sorted such that $w(e_i) > w(e_j)$ whenever $i < j$. 
Let $n$ be the rank of our weighted uniform matroid $M^{(n)}$. 
Then we can write $B^\star = (e_{i_1}, \ldots, e_{i_n})$ and $B' = (e_{j_1}, \ldots, e_{j_n})$ with $j_k < j_l$ and $i_k < i_l$ for any $k < l$.

Let $\ell$ be the smallest positive integer such that $i_\ell$ > $j_\ell$. In particular, $\ell$ exists and is at most $n$ by assumption. Consider the independent set $S_{\ell-1} = \{e_{i_1}, \ldots e_{\ell-1}\}$ (in particular, $S_{\ell-1} = \emptyset$ if $\ell =1$). According to the augmentation axiom (Definition \ref{def:matroid}, I3), there exist $k \in \{1, \ldots, \ell \}$ such that $S_{\ell-1} + e_{j_k} \in \mathcal{J}$ and $e_{j_k} \notin S_{\ell-1}$. However, $j_k \leq j_\ell < i_\ell$, thus $w(e_{j_k}) \leq w(e_{j_\ell}) <w(e_{i_\ell})$. This contradicts the definition of the greedy algorithm.
\end{proof}

Now, we notice that for the last layer, Insertion is a weighted matroid. We insist that this result is \emph{only true for the concepts in the penultimate layer}, as our demonstrations rely on the linearity of the decomposition. Here, the weight is given by the score of the model, which is a linear combination of concepts.  


\begin{theorem}[Optimal Insertion in the last layer]
When decomposing in the last layer,~\textbf{Gradient Input}, \textbf{Integrated Gradients}, \textbf{Occlusion}, and \textbf{Rise} yield the optimal solution for the C-Insertion metric.
In a more general sense, any method $\cam(\v{u})$ that  satisfies the condition 
$\forall (i, j) \in \{1, \ldots, k\}^2, 
(\v{u} \odot \e_i) \m{V}^\tr\m{W} \geq (\v{u} \odot \e_j) \m{V}^\tr \m{W}
\implies 
\cam(\v{u})_i \geq \cam(\v{u})_j 
$ yield the optimal solution.
\end{theorem}

\begin{proof}
Each $n$ step of the C-Insertion algorithm corresponds to the $n$-uniform weighted matroid with weighting function $w(e_i) = (\v{u} \odot e_i) \m{V}^\tr\m{W} + b = \v{u}_i \m{V}^\tr\m{W} + b$. Therefore, any $\cam(\cdot)$ method that produces the same ordering as $w(\cdot)$ will yield the optimal solution. 
It easily follows that \textbf{Gradient Input}, \textbf{Integrated Gradients}, \textbf{Occlusion} are optimal as they all boil down to $\cam_i(\v{u}) = \v{u}_i \m{V}^\tr\m{W}+b$.
Concerning RISE, suppose that $w(e_i) \geq w(e_j)$, then $\v{u}_i \m{V}_i^\tr\m{W} + b \geq \v{u}_j \m{V}_j^\tr\m{W} + b$, and  
$\cam_i^{(RI)}(\v{u}) - \cam_j^{(RI)}(\v{u})
= \bias + \frac{1}{2} (\v{u}\m{V}^\tr\m{W} + \v{u}_i \m{V}_i^\tr\m{W}) - \bias + \frac{1}{2} (\v{u}\m{V}^\tr\m{W} + \v{u}_j \m{V}_j^\tr\m{W})
= \v{u}_i \m{V}_i^\tr\m{W} - \v{u}_j \m{V}_j^\tr\m{W}
\geq 0.
$ Thus, RISE importance will order in the same manner and is also optimal.
\end{proof}

\begin{corollary}[Optimal Deletion in the last layer]
When decomposing in the last layer,~\textbf{Gradient Input}, \textbf{Integrated Gradients}, \textbf{Occlusion}, and \textbf{Rise} yield the optimal solution for the C-Deletion metric.
\end{corollary}
\begin{proof}
It is simply observed that the C-Deletion problem seeks a minimum weight basis and corresponds to the same weighted matroid with weighting function $w'(\cdot) = -w(\cdot)$.
\end{proof}


\subsection{Sparse Autoencoder}

As a remainder, a general method (as it encompasses both PCA and K-means) to obtain the loading-dictionary pair and achieve a matrix reconstruction $\mathbf{A} = \mathbf{U} \mathbf{V}^\tr$ is to train a neural network to obtain $\mathbf{U}$ from $\mathbf{A}$ such that the reconstruction of $\mathbf{A}$ is linear in $\mathbf{U}$. This can be formally represented as:

$$
(\bm{\psi}^\star, \mathbf{V}^\star) = \arg\min_{\bm{\psi},\mathbf{V}} \| \mathbf{A} - \bm{\psi}(\mathbf{A}) \mathbf{V}^\top \|_F^2
$$

Here, $\mathbf{U}^\star = \bm{\psi}^\star(\mathbf{A}).$ An interesting characteristic of NMF and K-means is the non-linear relationship between $\mathbf{A}$ and $\mathbf{U}$. Specifically, the transformation from $\mathbf{A}$ to $\mathbf{U}$ is non-linear, while the transformation from $\mathbf{U}$ to $\mathbf{A}$ is linear, as explained in \cite{fel2022xplique}, which need to introduce a method based on implicit differentiation to obtain the gradient of $\mathbf{U}$ with respect to $\mathbf{A}$. Indeed, the sequence of operations to optimize $\mathbf{U}$ causes us to lose information about which elements of $\mathbf{A}$ contributed to obtaining $\mathbf{U}$. We believe that this non-linear relationship (absent in PCA) may be an essential ingredient for effective concept extraction.

Finally, as described in this article, other characteristics that appear to make it interpretable include its compositionality (due to non-extreme sparsity), good reconstruction, and positivity, which aids in interpretation. Thus, the architecture of $\bm{\psi}$ used for Figure~\ref{fig:holistic:qualitative_comparison} consists of a sequence of dense layers and batch normalization with ReLU activation to obtain positive scores and sparsity similar to NMF, without imposing constraints on $\mathbf{V}$. More formally, $\bm{\psi}$ is a sequence of layers as follows:

$$
\textsc{Dense(128) - BatchNormalization - ReLU}
$$
$$
\textsc{Dense(64) - BatchNormalization - ReLU}
$$
$$
\textsc{Dense(10) - BatchNormalization - ReLU}
$$

While the vector $\m{V}$ is initialized using a truncated SVD~\cite{fathi2023initialization}. We used Adam optimizer\cite{kingma2014adam} with a learning rate of $1e^{-3}$. However, it's worth noting that there is a wealth of literature on dictionary learning that remains to be explored for the task of concept extraction~\cite{dumitrescu2018dictionary}.









\clearpage
  
\section{Modern Feature Visualization with MACO}
\label{sec:concepts:maco}


The last section of this chapter will be dedicated to a novel method that will enable us one problem that we identify in \autoref{sec:concepts:craft}: the visualization of concept. Feature visualization -- defined in \autoref{def:intro:feature_viz} -- has gained substantial popularity, particularly after the seminal and influential work of the Clarity team~\cite{olah2017feature}, which established it as a crucial tool for explainability.
However, its widespread adoption has been limited due to a reliance on tricks to generate interpretable images, and corresponding challenges in scaling it to deeper neural networks.
Here, we will introduce \magfv, a simple approach to address these shortcomings.
The main idea is to generate images by optimizing the phase spectrum while keeping the magnitude constant to ensure that generated explanations lie in the space of natural images. Our approach yields significantly better results -- both qualitatively and quantitatively -- and unlocks efficient and interpretable feature visualizations for large state-of-the-art neural networks.
We also show that our approach exhibits an attribution mechanism allowing us to augment feature visualizations with spatial importance.


Overall, our approach unlocks, for the first time, feature visualizations for large, state-of-the-art deep neural networks without resorting to any parametric prior image model.

\begin{figure}[ht]
\begin{center}
   \includegraphics[width=.99\textwidth]{assets/maco/big_picture.jpg}
\end{center}

\caption{\textbf{Comparison between feature visualization methods for ``White Shark'' classification.}
\textbf{(Top)} Standard Fourier preconditioning-based method for feature visualization~\cite{olah2017feature}.
\textbf{(Bottom)} Proposed approach, \magfv, which incorporates a Fourier spectrum magnitude constraint. %
}
\label{fig:maco:logits_fail}

\end{figure}

\subsection{Introduction}

As discussed in \autoref{chap:attributions}, the initial tools in the explainability toolkit were primarily attribution methods~\cite{simonyan2014deep,smilkov2017smoothgrad,selvaraju2017gradcam,fel2021sobol,novello2022making,sundararajan2017axiomatic,zeiler2014visualizing,shrikumar2017learning,Fong_2017,graziani2021sharpening}. We also seen in ~\autoref{sec:attributions:metapred} that those approaches only offer a partial understanding of the learned decision processes as they aim to identify the location of the most discriminative features in an image, the ``\where'', leaving open the ``\what'' question, \textit{i.e.} the semantic meaning of those features.


Feature visualization methods, which aim to bridge this gap, involve formulating and solving an optimization problem to identify an input image that maximizes the activation of a specific target element (be it a neuron, layer, or the entire model)~\cite{zeiler2014visualizing}. Most of the approaches developed in the field fall along a spectrum based on how strongly they regularize the model. At one end of the spectrum, if no regularization is used, the optimization process can search the whole image space, but this tends to produce noisy images and nonsensical high-frequency patterns~\cite{erhan2009visualizing}. To circumvent this issue, researchers have proposed to penalize high-frequency in  the resulting images -- either by reducing the variance between neighboring pixels~\cite{mahendran2015understanding}, by imposing constraints on the image's total variation~\cite{nguyen2016synthesizing,nguyen2017plug,simonyan2014deep}, or by blurring the image at each optimization step~\cite{nguyen2015deep}. However, in addition to rendering images of debatable validity, these approaches also suppress genuine, interesting high-frequency features, including edges. To mitigate this issue, a bilateral filter may be used instead of blurring, as it has been shown to preserve edges and improve the overall result~\cite{tyka2016class}. Other studies have described a similar technique to decrease high frequencies by operating directly on the gradient, with the goal of preventing their accumulation in the resulting visualization~\cite{AudunGoogleNet}. One advantage of reducing high frequencies present in the gradient, as opposed to the visualization itself, is that it resists the amplification of high frequencies while still allowing them to manifest when consistently promoted by the gradient.
This process, known as "preconditioning" in optimization, can greatly simplify the optimization problem. The Fourier transform has been shown to be a successful preconditioner as it forces the optimization to be performed in a decorrelated and whitened image space~\cite{olah2017feature}. 

The emergence of high-frequency patterns in the absence of regularization is associated with a lack of robustness and sensitivity of the neural network to adversarial examples~\cite{szegedy2013intriguing}, and consequently, these patterns are less often observed in adversarially robust models~\cite{engstrom2019adversarial, santurkar2019image, tsipras2018robustness}. An alternative strategy to promote robustness involves enforcing small perturbations, such as jittering, rotating, or scaling, in the visualization process~\cite{mordvintsev2015inceptionism}, which, when combined with a frequency penalty~\cite{olah2017feature}, has been proved to greatly enhance the generated images.

Unfortunately, previous methods in the field of feature visualization have been limited in their ability to generate visualizations for newer architectures beyond VGG, resulting in a lack of interpretable visualizations for larger networks like ResNets~\cite{olah2017feature}. Consequently, researchers have shifted their focus to approaches that leverage statistically learned priors to produce highly realistic visualizations. One such approach involves training a generator, like a GAN~\cite{nguyen2016synthesizing} or an autoencoder~\cite{wang2022traditional, nguyen2017plug}, to map points from a latent space to realistic examples and optimizing within that space. Alternatively, a prior can be learned to provide the gradient (w.r.t the input) of the probability and optimize both the prior and the objective jointly~\cite{nguyen2017plug, tyka2016class}. Another method involves approximating a generative model prior by penalizing the distance between output patches and the nearest patches retrieved from a database of image patches collected from the training data~\cite{wei2015understanding}.
Although it is well-established that learning an image prior produces realistic visualizations, it is difficult to distinguish between the contributions of the generative models and that of the neural network under study. Hence, in this work, we focus on the development of visualization methods that rely on minimal priors to yield the least biased visualizations.


Our proposed approach, called MAgnitude Constrained Optimization (\magfv), builds on the seminal work by Olah et al. We propose a straightforward re-parametrization that essentially relies on exploiting the phase/magnitude decomposition of the Fourier spectrum, to exclusively optimizing the image's phase while keeping its magnitude constant.
Such a constraint is motivated by psychophysics experiments that have shown that humans are more sensitive to differences in phase than in magnitude~\cite{oppenheim1981importance,caelli1982visual,guyader2004image,joubert2009rapid, gladilin2015role}. Our contributions are threefold:

\begin{enumerate}[label=(\textit{\textbf{\roman*}})]

\item{We unlock feature visualizations for large modern CNNs without resorting to any strong parametric image prior (see Figure~\ref{fig:maco:logits_fail}).}

\item{We describe how to leverage the gradients obtained throughout our optimization process to combine feature visualization with attribution methods, thereby explaining both ``\what'' activates a neuron and ``\where'' it is located in an image.}

\item{We introduce new metrics to compare the feature visualizations produced with \magfv~to those generated with other methods.}
\end{enumerate}
As an application of our approach, we propose feature visualizations for FlexViT \cite{beyer2022flexivit} and ViT \cite{Dosovitskiy2021-zy} (logits and intermediate layers;  see Figure~\ref{fig:maco:logits_and_internal}).  We also employ our approach on a feature inversion task to generate images that yield the same activations as target images to better understand what information is getting propagated through the network and which parts of the image are getting discarded by the model (on ViT, see Figure~\ref{fig:maco:inversion}).
Finally, we will make a link with our work introduced in \autoref{sec:concepts:craft} and show how to combine our work with \craft (see Figure~\ref{fig:maco:concepts}). As feature visualization can be used to optimize in directions in the network's representation space, we employ \magfv~to generate concept visualizations, thus allowing us to improve the human interpretability of concepts and reducing the risk of confirmation bias. 

\subsection{Magnitude-Constrained Feature Visualization}




\paragraph{Notations}

Throughout, we consider a general supervised learning setting, with an input space $\sx \subseteq \Real^{h \times w}$, an output space $\sy \subseteq \Real^c$, and a classifier $\f : \sx \to \sy$ that maps inputs $\vx \in \sx$ to a prediction $\v{y} \in \sy$.
Without loss of generality, we assume that $\f$ admits a series of $L$ intermediate spaces $\s{A}_\ell \subseteq \Real^{p_\ell}, 1 < \ell < L$.
In this setup, $\f_\ell : \sx \to \s{A}_\ell$ maps an input to an intermediate activation $\v{v} = (v_1, \ldots, v_{p_\ell})^\intercal \in \s{A}_\ell$ of $\f$.
We respectively denote $\fourier$ and $\fourier^{-1}$ as the 2-D Discrete Fourier Transform (DFT) on $\sx$ and its inverse.








\paragraph{Optimization Criterion.}
The primary goal of a feature visualization method is to produce an image $\vx^\star$ that maximizes a given criterion $\mathcal{L}_{\v{v}}(\vx) \in \Real$; usually some value aggregated over a subset of weights in a neural network $\f$ (neurons, channels, layers, logits).
A concrete example consists in finding a natural "prototypical" image $\vx^\star$ of a class $k \in \llbracket 1, K \rrbracket$ without using a dataset or generative models.
However, optimizing in the pixel space $\Real^{W \times H}$ is known to produce noisy, adversarial-like $\vx^\star$. Therefore, the optimization is constrained using a regularizer $\Omega: \sx \to \Real^+$ to penalize unrealistic images:
\begin{equation}
\vx^\star = \argmax_{\vx \in \sx} \mathcal{L}_{\v{v}}(\vx) - \lambda \Omega(\vx).
\label{eq:maco:general}
\end{equation}
In Eq.~\ref{eq:maco:general}, $\lambda$ is a hyperparameter used to balance the main optimization criterion $\mathcal{L}_{\v{v}}$ and the regularizer $\Omega(\cdot)$. Finding a regularizer that perfectly matches the structure of natural images is hard, so  proxies have to be used instead. Previous studies have explored various forms of regularization spanning from total variation, $\ell_1$, or $\ell_2$ loss~\cite{nguyen2016synthesizing,nguyen2017plug,simonyan2014deep}. More successful attempts rely on the reparametrization of the optimization problem in the Fourier domain rather than on regularization.


\subsubsection{A Fourier perspective}

Mordvintsev et al.~\cite{mordvintsev2018differentiable} noted in their seminal work that one could use differentiable image parametrizations to facilitate the maximization of $\mathcal{L}_{\v{v}}$. Olah et al.~\cite{olah2017feature} proposed to re-parametrize the images using their Fourier spectrum. Such a parametrization allows amplifying the low frequencies using a scalar $\v{w}$. Formally, the prototypal image $\vx^\star$ can be written as $\vx^\star = \fourier^{-1}(\v{z}^\star \odot \v{w})$ with:

$$ \v{z}^\star = \argmax_{\v{z} \in \mathbb{C}^{W \times H}} \mathcal{L}_{\v{v}}(\fourier^{-1}(\v{z} \odot \v{w})).$$

Finding $\vx^\star$ boils down to optimizing a Fourier buffer
$\v{z} = \bm{a} + i \bm{b}$ together with boosting the low-frequency components and then recovering the final image by inverting the optimized Fourier buffer using inverse Fourier transform.

\begin{figure}
\centering
\includegraphics[width=0.9\textwidth]{assets/maco/leakage.jpg};
\caption{\textbf{Comparison between Fourier FV and natural image power spectrum.} In \textbf{(left)}, the power spectrum is averaged over $10$ different logits visualizations for each of the $1000$ classes of ImageNet. The visualizations are obtained using the \textbf{Fourier FV}Fourier FV method to maximize the logits of a ViT network~\citep{olah2017feature}. In \textbf{(right)} the spectrum is averaged over all training images of the ImageNet dataset.}
\label{fig:maco:leakage}
\end{figure}






However, multiple studies have shown that the resulting images are not sufficiently robust, in the sense that a small change in the image can cause the criterion $ \mathcal{L}_{\v{v}}$ to drop. Therefore, it is common to see robustness transformations applied to candidate images throughout the optimization process. In other words, the goal is to ensure that the generated image satisfies the criterion even if it is rotated by a few degrees or jittered by a few pixels. Formally, given a set of possible transformation functions -- sometimes called augmentations -- that we denote $\mathcal{T}$ such that for any transformation $\augmentation \sim \mathcal{T}$, we have $\augmentation(\vx) \in \sx$, the optimization becomes:

$$ 
\v{z}^\star = \argmax_{\v{z} \in \mathbb{C}^{W \times H}}
\mathbb{E}_{\augmentation \sim \mathcal{T}}(\mathcal{L}_{\v{v}}((\augmentation \circ \fourier^{-1})(\v{z} \odot \v{w})).
$$


Empirically, it is common knowledge that the deeper the models are, the more transformations are needed and the greater their magnitudes should be. To make their approach work on models like VGG, Olah et al.~\cite{olah2017feature} used no less than a dozen transformations. However, this method fails for modern architectures, no matter how many transformations are applied. We argue that this may come from the low-frequency scalar (or booster) no longer working with models that are too deep. For such models, high frequencies eventually come through, polluting the resulting images with high-frequency content -- making them impossible to interpret by humans. %
To empirically illustrate this phenomenon, we compute the $k$ logit visualizations obtained by maximizing each of the logits corresponding to the $k$ classes of a ViT using the parameterization used by Olah et al.~ In Figure~\ref{fig:maco:leakage} (left), we show the average of the spectrum of these generated visualizations over all classes: $\frac{1}{k} \sum_{i=1}^k |\fourier(\vx^\star_i)|$. We compare it with the average spectrum of images on the ImageNet dataset (denoted $\mathcal{D}$): $\mathbb{E}_{\vx \sim \mathcal{D}}(|\fourier(\vx)|)$ (Figure~\ref{fig:maco:leakage}, right panel).
We observe that the images obtained through optimization put much more energy into high frequencies compared to natural images. Note that we did not observe this phenomenon in older models such as LeNet or VGG.

In the following section, we introduce our method named~\magfv, which is motivated by this observation. We constrain the magnitude of the visualization to a natural value, enabling natural visualization for any contemporary model, and reducing the number of required transformations to only two.


\subsubsection{\magfv: from Regularization to Constraint}
\begin{figure}[t!]
\center
\includegraphics[width=1\textwidth]{assets/maco/method.pdf}
\caption{\textbf{Overview of the approach:} \textbf{(a)}  Current Fourier parameterization approaches optimize the entire spectrum (yellow arrow). \textbf{(b)}  In contrast,  the optimization flow in our approach (green arrows) goes from the network activation ($\v{y}$) to the phase of the spectrum ($\v{\varphi}$) of the input image ($\vx$).}

\label{fig:maco:method}
\end{figure}

Parameterizing the image in the Fourier space makes it possible to directly manipulate the image in the frequency domain. We propose to take a step further and decompose the Fourier spectrum $\v{z}$ into its polar form $\v{z} = \v{r} e^{i \v{\varphi}}$ instead of its cartesian form $\v{z} = \bm{a} + i \bm{b}$, which allows us to disentangle the magnitude ($\v{r}$) and the phase ($\v{\varphi}$).

It is known that human recognition of objects in images is driven not by magnitude but by phase~\cite{oppenheim1981importance,caelli1982visual,guyader2004image,joubert2009rapid, gladilin2015role}. Motivated by this, we propose to optimize the phase of the Fourier spectrum while fixing its magnitude to a typical value of a natural image (with few high frequencies). In particular, the magnitude is kept constant at the average magnitude computed over a set of natural images (such as ImageNet), so $\v{r} = \mathbb{E}_{\vx \sim \mathcal{D}}(|\fourier(\vx)|)$. Note that this spectrum needs to be calculated only once and can be used at will for other tasks.

\begin{figure}[ht]
\begin{algorithm}[ht!]
\caption{\textit{NovelSelect}}
\label{alg:novelselect}
\begin{algorithmic}[1]
\State \textbf{Input:} Data pool $\mathcal{X}^{all}$, data budget $n$
\State Initialize an empty dataset, $\mathcal{X} \gets \emptyset$
\While{$|\mathcal{X}| < n$}
    \State $x^{new} \gets \arg\max_{x \in \mathcal{X}^{all}} v(x)$
    \State $\mathcal{X} \gets \mathcal{X} \cup \{x^{new}\}$
    \State $\mathcal{X}^{all} \gets \mathcal{X}^{all} \setminus \{x^{new}\}$
\EndWhile
\State \textbf{return} $\mathcal{X}$
\end{algorithmic}
\end{algorithm}

\end{figure}


Therefore, our method does not backpropagate through the entire Fourier spectrum but only through the phase (Figure~\ref{fig:maco:method}), thus reducing the number of parameters to optimize by half. Since the magnitude of our spectrum is constrained, we no longer need hyperparameters such as $\lambda$ or scaling factors, and the generated image at each step is naturally plausible in the frequency domain.
We also enhance the quality of our visualizations via two data augmentations: random crop and additive uniform noise.
To the best of our knowledge, our approach is the first to completely alleviate the need for explicit regularization -- using instead a hard constraint on the solution of the optimization problem for feature visualization.
To summarize, we formally introduce our method:

\begin{definition}[\textbf{\magfv}]
The feature visualization results from optimizing the parameter vector $\v{\varphi}$  such that:
$$
\v{\varphi}^\star = \argmax_{\v{\varphi} \in \Real^{W \times H}}
\mathbb{E}_{\augmentation \sim \mathcal{T}}(\mathcal{L}_{\v{v}}((\augmentation \circ \fourier^{-1})(\v{r} e^{i \v{\varphi}})) ~~~\text{where}~~~ \v{r} = \mathbb{E}_{\vx \sim \mathcal{D}}(|\fourier(\vx)|)
$$
The feature visualization is then obtained by applying the inverse Fourier transform to the optimal complex-valued spectrum: $\vx^\star = \fourier^{-1}((\v{r} e^{i \v{\varphi}^\star})$
\end{definition}









\paragraph{Transparency for free:}\label{sec:maco:transparency}
Visualizations often suffer from repeated patterns or unimportant elements in the generated images. This can lead to readability problems or confirmation biases~\cite{borowski2020exemplary}. It is important to ensure that the user is looking at what is truly important in the feature visualization. The concept of transparency, introduced in \cite{mordvintsev2018differentiable}, addresses this issue but induces additional implementation efforts and computational costs.

We propose an effective approach, which leverages attribution methods -- specifically a variant of Smoothgrad seen in \autoref{chap:attributions}) -- that yields a transparency map $\v{\alpha}$ for the associated feature visualization without any additional cost. Our solution takes advantage of the fact that during backpropagation, we can obtain the intermediate gradients on the input $\partial \mathcal{L}_{\v{v}}( \vx) / \partial \vx$ for free as $\frac{\partial \mathcal{L}_{\v{v}}( \vx)}{\partial \v{\varphi}} =  \frac{\partial \mathcal{L}_{\v{v}}( \vx)}{\partial \vx} \frac{\partial \vx}{\partial \v{\varphi}}$. We store these gradients throughout the optimization process and then average them, as done in SmoothGrad, to identify the areas that have been modified/attended to by the model the most during the optimization process. We note that a similar technique has recently been used to explain diffusion models \cite{boutin2023diffusion}. In Algorithm \ref{alg:maco:cap}, we provide pseudo-code for \magfv~and an example of the transparency maps in Figure~\ref{fig:maco:inversion} (third column).




\begin{figure}
    \centering
    \includegraphics[width=0.98\textwidth]{assets/maco/qualitative_internal.jpg}
    \caption{\textbf{(left) Logits and (right) internal representations of FlexiViT.}  \magfv~was used to maximize the activations of \textbf{(left)} logit units and \textbf{(right)} specific channels located in different blocks of the FlexViT (blocks 1, 2, 6 and 10 from left to right).}
    \label{fig:maco:logits_and_internal}
\end{figure}





\subsection{Evaluation}
\label{section:maco:evaluation}
We now describe and compute three different scores to compare the different feature visualization methods: Fourier (Olah et al.), CBR (optimization in the pixel space), and \magfv~(ours). It is important to note that these scores are only applicable to output logit visualizations. We will then demonstrate how we can use our method to perform concept visualization. %
To keep a fair comparison, we restrict the benchmark to methods that do not rely on any learned image priors. Indeed, methods with learned prior will inevitably yield lower FID scores (and lower plausibility score) as the prior forces the generated visualizations to lie on the manifold of natural images.




\paragraph*{Plausibility score.} We consider a feature visualization plausible when it is similar to the distribution of images belonging to the class it represents.
We quantify the plausibility through an OOD metric (Deep-KNN, recently used in~\cite{sun2022out}): it measures how far a feature visualization deviates from the corresponding ImageNet object category images based on their representation in the network's intermediate layers (see Table~\ref{table:maco:ood_fid}).



\paragraph{FID score.} The FID quantifies the similarity between the distribution of the feature visualizations and that of natural images for the same object category. Importantly, the FID measures the distance between two distributions, while the plausibility score quantifies the distance from a sample to a distribution. To compute the FID,  we used images from the ImageNet validation set and used the Inception v3 last layer (see Table~\ref{table:maco:ood_fid}). Additionally, we center-cropped our $512\times 512$ images to $299\times 299$ images to avoid the center-bias problem~\cite{nguyen2016multifaceted}.



\paragraph{Transferability score.} This score measures how consistent the feature visualizations are with other pre-trained classifiers. To compute the transferability score, we feed the obtained feature visualizations into 6 additional pre-trained classifiers (MobileNet~\cite{howard2017mobilenets}, VGG16~\cite{simonyan2014deep}, Xception~\cite{chollet2017xception}, EfficientNet~\cite{tan2019efficientnet}, Tiny ConvNext~\cite{liu2022convnet} and Densenet~\cite{huang2017densely}), and we report their classification accuracy (see Table~\ref{table:maco:transferability}).

All scores are computed using 500 feature visualizations, each of them maximizing the logit of one of the ImageNet classes obtained on the FlexiViT~\cite{beyer2022flexivit}, ViT\cite{kolesnikov2020bit}, and ResNetV2\cite{he2016deep} models. For the feature visualizations derived from Olah et al.~ \cite{olah2017feature}, we used all 10 transformations set from the Lucid library\footnote{\href{https://github.com/tensorflow/lucid}{https://github.com/tensorflow/lucid}}.
CBR denotes an optimization in pixel space and using the same 10 transformations, as described in~\cite{nguyen2015deep}.
For \magfv, $\augmentation$ only consists of two transformations; first we add uniform noise $\bm{\delta} \sim \mathcal{U}([-0.1, 0.1])^{W \times H}$ and crops and resized the image with a crop size drawn from the normal distribution $\mathcal{N}(0.25, 0.1)$, which corresponds on average to 25\% of the image.
We used the NAdam optimizer \cite{dozat2016incorporating} with $lr=1.0$ and $N = 256$ optimization steps. Finally, we used the implementation of \cite{olah2017feature} and CBR which are available in the Xplique library~\cite{fel2022xplique} \footnote{\href{https://github.com/deel-ai/xplique}{https://github.com/deel-ai/xplique}} which is based on Lucid.

\begin{table}[ht]
\centering
        \begin{tabular}{lccc}
            & FlexiViT & ViT & ResNetV2\\
            \hline
            \multicolumn{4}{l}{$\bullet$\;\textbf{Plausibility score} (1-KNN) ($\downarrow$)}\\

            \magfv & {\bf 1473} & {\bf 1097 } & {\bf 1248} \\%\\
            Fourier~\cite{olah2017feature} & 1815 &  1817 & 1837 \\
            CBR~\cite{nguyen2015deep} &  1866 & 1920 & 1933 \\
            \hline
            \multicolumn{4}{l}{$\bullet$\;\textbf{FID Score}  ($\downarrow$)}\\
            \magfv & {\bf 230.68} & {\bf 241.68} & {\bf 312.66} \\
            Fourier~\cite{olah2017feature} &  250.25 & 257.81 & 318.15 \\
            CBR~\cite{nguyen2015deep} &  247.12 & 268.59 & 346.41 \\
            \hline
        \end{tabular}
        
        \caption{Plausibility and FID scores for different feature visualization methods applied on FlexiVIT, ViT and ResNetV2}
    \label{table:maco:ood_fid}
\end{table}

\begin{table}[ht]
\centering
\begin{tabular}{lccc}
    & FlexiViT & ViT & ResNetV2 \\
    \hline
    \multicolumn{4}{l}{$\bullet$\;\textbf{Transferability score($\uparrow$)}: \magfv / Fourier~\cite{olah2017feature}} \\

    MobileNet  & {\bf 68} \slash~38 & {\bf 48}\slash~37  & {\bf 93} \slash~36 \\
    VGG16         & {\bf 64} \slash~30 & {\bf 50} \slash~30 & {\bf 90} \slash~20 \\
    Xception      & {\bf 85} \slash~61 & {\bf 73} \slash~62 & {\bf 97} \slash~64 \\
    Eff. Net  & {\bf 88} \slash~25 & {\bf 63} \slash~25 & {\bf 82} \slash~21 \\
    ConvNext & {\bf 96} \slash~52 & {\bf 84} \slash~55 & {\bf 93} \slash~60\\
    DenseNet      & {\bf 84} \slash~32 & {\bf 66} \slash~31 & {\bf 93} \slash~25 \\
    \hline
    \\
    \end{tabular}
        \caption{Transferability scores for different feature visualization methods applied on FlexiVIT, ViT and ResNetV2.}
        \label{table:maco:transferability}
\end{table}


























For all tested metrics, we observe that \magfv~produces better feature visualizations than those generated by Olah et al.~\cite{olah2017feature} and CBR~\cite{nguyen2015deep}. We would like to emphasize that our proposed evaluation scores represent the first attempt to provide a systematic evaluation of feature visualization methods, but we acknowledge that each individual metric on its own is insufficient and cannot provide a comprehensive assessment of a method's performance. However, when taken together, the three proposed scores provide a more complete and accurate evaluation of the feature visualization methods.

\subsubsection{Human psychophysics study}
Ultimately, the goal of any feature visualization method is to demystify the CNN's underlying decision process in the eyes of human users. To evaluate \magfv~'s ability to do this, we closely followed the psychophysical paradigm introduced in~\cite{zimmermann2021well}. In this paradigm, the participants are presented with examples of a model's ``favorite'' inputs (i.e., feature visualization generated for a given unit) in addition to two query inputs. Both queries represent the same natural image, but have a different part of the image hidden from the model by a square occludor. The task for participants is to judge which of the two queries would be ``favored by the model'' (i.e., maximally activate the unit). The rationale here is that a good feature visualization method would enable participants to more accurately predict the model's behavior. Here, we compared four visualization conditions (manipulated between subjects): Olah~\cite{olah2017feature}, \magfv~with the transparency mask (the transparency mask is decribed in \ref{sec:maco:transparency}), \magfv~without the transparency mask, and a control condition in which no visualizations were provided. In addition, the network (VGG16, ResNet50, ViT) was a within-subject variable. The units to be understood were taken from the output layer.

\begin{figure}[ht]
\includegraphics[width=0.9\textwidth]{assets/maco/human_exp.png}
\caption{\textbf{Human causal understanding of model activations}. We follow the experimental procedure introduced in~\cite{zimmermann2021well} to evaluate Olah and \magfv~visualizations on $3$ different networks. The control condition is when the participant did not see any feature visualization. 
}
\label{fig:maco:human_results}    
\end{figure}

Based on the data of 174 participants on Prolific (\url{www.prolific.com}), we found both visualization and network to significantly predict the logodds of choosing the right query (Fig.~\ref{fig:maco:human_results}). That is, the logodds were significantly higher for participants in both the \magfv~conditions compared to Olah. On the other hand, our tests did not yield a significant difference between Olah and the control condition, or between the two \magfv~conditions. Finally, we found that, overall, ViT was significantly harder to interpret than ResNet50 and VGG16, with no significant difference observed between the latter two networks. Full experiment and analysis details can be found in the supplementary materials, section~\ref{sup:maco:psychophysics}. 

However, it should be noted that investigating the effect on a neuron-by-neuron basis, as in the original setup, may not be advisable for the issues outlined in \autoref{sec:concepts:craft} and referenced in \cite{elhage2022superposition}. Conducting a parallel study that confirms this by utilizing meaningful directions in the latent space -- e.g., with \craft -- instead of individual neurons would be of interest.

\subsubsection{Ablation study}

    \begin{table}%
        \centering
        \begin{tabular}{lccc}
            FlexiViT & Plausibility ($\downarrow$) & FID ($\downarrow$) & logit magnitude ($\uparrow$) \\
            \hline
            \magfv  & 571.68 & 211.0 & 5.12 \\
            - transparency & 617.9 (+46.2) & 208.1 (-2.9) & 5.05 (-0.1)\\
            - crop & 680.1 (+62.2) & 299.2 (-91.1) & 8.18 (+3.1)\\
            - noise & 707.3 (+27.1) & 324.5 (-25.3) & 11.7 (+3.5)\\
            \hline
            Fourier~\cite{olah2017feature} & 673.3 & 259.0 & 3.22\\
            - augmentations & 735.9 (+62.6) &  312.5 (+53.5) & 12.4 (+9.2)\\
        \end{tabular}
        \caption{\textbf{Ablation study on the FlexiViT model:} This reveals that 1. augmentations help to have better FID and Plausibility scores, but lead to lesser salients visualizations (softmax value), 2. Fourier~\cite{olah2017feature} benefits less from augmentations than \magfv.}
        \label{table:maco:ablation}
    \end{table}


    To disentangle the effects of the various components of \magfv, we perform an ablation study on the feature visualization applications. We consider the following components: (1) the use of a magnitude constraint, (2) the use of the random crop, (3) the use of the noise addition, and (4) the use of the transparency mask. We perform the ablation study on the FlexiViT model, and the results are presented in Table~\ref{table:maco:ablation}. We observe an inherent tradeoff between optimization quality (measured by logit magnitude) on one side, and the plausibility (and FID) scores on the other side. This reveals that plausible images which are close to the natural image distribution do not necessarily maximize the logit.
    Finally, we observe that the transparency mask does not significantly affect any of the scores confirming that it is mainly a post-processing step that does not affect the feature visualization itself.


\subsection{Applications}

We demonstrate the versatility of the proposed \magfv~technique by applying it to three different XAI applications:

\paragraph{Logit and internal state visualization.} For logit visualization, the optimization objective is to maximize the activation of a specific unit in the logits vector of a pre-trained neural network (here a FlexiViT\cite{beyer2022flexivit}). The resulting visualizations provide insights into the features that contribute the most to a class prediction (refer to Figure~\ref{fig:maco:logits_and_internal}a). For internal state visualization, the optimization objective is to maximize the activation of specific channels located in various intermediate blocks of the network (refer to Figure~\ref{fig:maco:logits_and_internal}b). This visualization allows us to better understand the kind of features these blocks -- of a FlexiViT\cite{beyer2022flexivit} in the figure -- are sensitive to.

\paragraph{Feature inversion.} The goal of this application is to find an image that produces an activation pattern similar to that of a reference image. By maximizing the similarity to reference activations, we are able to generate images representing the same semantic information at the target layer but without the parts of the original image that were discarded in the previous stages of the network, which allows us to better understand how the model operates.
Figure~\ref{fig:maco:inversion}a displays the images (second column) that match the activation pattern of the penultimate layer of a VIT when given the images from the first column. We also provide examples of transparency masks based on attribution (third column), which we apply to the feature visualizations to enhance them (fourth column).



\begin{figure}
    \centering
    \includegraphics[width=1.0\textwidth]{assets/maco/inversion.jpg}
    \caption{\textbf{Feature inversion.} Images in the second column match the activation pattern of the penultimate layer of a ViT when fed with the images of the first column. In the third column, we show their corresponding attribution-based transparency masks, leading to better feature visualization when applied (fourth column).}
    \label{fig:maco:inversion}
\end{figure}


\paragraph{Concept visualization.} Herein we combine \magfv~with concept-based explainability. Such methods aim to increase the interpretability of activation patterns by decomposing them into a set of concepts~\cite{ghorbani2019towards}. In this work, we leverage our \craft~concept-based explainability method~\cite{fel2023craft}, which uses Non-negative Matrix Factorization to decompose activation patterns into main directions -- that are called concepts --, and then, we apply \magfv~to visualize these concepts in the pixel space. To do so, we optimize the visualization such that it matches the concept activation patterns. In Figure~\ref{fig:maco:concepts}b, we present the top $2$ most important concepts (one concept per column) for five different object categories (one category per row) in a ResNet50 trained on ImageNet. The concepts' visualizations are followed by a mosaic of patches extracted from natural images: the patches that maximally activate the corresponding concept. 

\begin{figure}
    \centering
    \includegraphics[width=1.0\textwidth]{assets/maco/concept_maco.jpg}
    \caption{\textbf{Concept visualization.} \magfv~is used to visualize concept vectors extracted with the \craft~ method~\autoref{sec:concepts:craft}. The concepts are extracted from a ResNet50 trained on ImageNet.}
    \label{fig:maco:concepts}
\end{figure}











\subsection{Limitations}\label{sec:maco:limitations}
We have demonstrated the generation of realistic explanations for large neural networks by imposing constraints on the magnitude of the spectrum. However, it is important to note that generating realistic images does not necessarily imply effective explanation of the neural networks. The metrics introduced in this section allow us to claim that our generated images are closer to natural images in latent space, that our feature visualizations are more plausible and better reflect the original distribution. However, they do not necessarily indicate that these visualizations helps humans in effectively communicating with the models or conveying information easily to humans.
Furthermore, in order for a feature visualization to provide informative insights about the model, including spurious features, it may need to generate visualizations that deviate from the spectrum of natural images. Consequently, these visualizations might yield lower scores using our proposed metrics.
Simultaneously, several interesting studies have highlighted the weaknesses and limitations of feature visualizations~\cite{borowski2020exemplary,geirhos2023dont,zimmermann2021well}. One prominent criticism is their lack of interpretability for humans, with research demonstrating that dataset examples are more useful than feature visualizations in understanding convolutional neural networks (CNNs)~\cite{borowski2020exemplary}. This can be attributed to the lack of realism in feature visualizations and their isolated use as an explainability technique.
With our approach, \magfv~, we take an initial step towards addressing this limitation by introducing magnitude constraints, which lead to qualitative and quantitative improvements. Additionally, we promote the use of feature visualizations as a supportive and complementary tool alongside other methods such as concept-based explainability, exemplified by \craft. We emphasize the importance of feature visualizations in combating confirmation bias and encourage their integration within a comprehensive explainability framework.



\subsection{Discussion}

In this section, we introduced a novel approach, \magfv, for efficiently generating feature visualizations in modern deep neural networks based on \tbi{i} a hard constraint on the magnitude of the spectrum to ensure that the generated visualizations lie in the space of natural images, and \tbi{ii} a new attribution-based transparency mask to augment these feature visualizations with the notion of spatial importance. This enhancement allowed us to scale up and unlock feature visualizations on large modern CNNs and vision transformers without the need for strong -- and possibly misleading -- parametric priors.
We also complement our method with a set of three metrics to assess the quality of the visualizations. Combining their insights offers a way to compare the techniques developed in this branch of XAI more objectively. We illustrated the scalability of \magfv~ with feature visualizations of large models like ViT, but also feature inversion and, critically, concept visualization.

Indeed, this tool integrates seamlessly with concept extraction methods, enabling the visualization of extracted concepts without resorting to image cropping. This approach offers a clearer, more causal view of the mechanisms that activate a given concept, thereby contributing significantly to our understanding of the internal workings of neural networks.

\clearpage

\section{Conclusion}

The conclusion of this chapter serves as an opportune moment for reflection and synthesis. Our research has led us through an in-depth examination of Hypothesis~\ref{hyp:what}, which posited that existing attribution methods fall short, as they primarily reveal ~\where but overlook the crucial aspect of the \what.

This chapter was dedicated to developing appropriate tools to address this issue. We began by constructing \craft, a method for decomposing the activations of a model into a set of concepts, demonstrating indeed its enhanced utility for human understanding compared to traditional attribution methods.
We decided to go one step further, in \autoref{sec:concepts:holistic}, where we established a theoretical framework that make: \tbi{i} show that concept extraction is \textbf{Dictionary learning}, and \tbi{ii} make a link between attribution methods and concept importance. The formulas used to determine the importance of a pixel, as seen in \autoref{chap:attributions}, are identical to those applied in evaluating the significance of concepts after decomposition. In the final section, we explored concept visualization as a way to visualize concept by introducing \maco. 

To summarize our novel framework, it consists in reinterpreting the intricate latent space of neural networks through a collection of atomic units termed concepts. While these concepts are mathematically abstract, we employed two methods to imbue them with meaning: maximally activating crops and feature visualization techniques. Additionally, it became evident that among these concepts, some offer greater utility than others, with attribution methods precisely identifying the most relevant ones.

\paragraph{A New Synergetic Approach to Explainability.} This new framework is distinct in its ability to \textit{synthesize all existing tools for explainability into a cohesive and synergetic system}. Our goal was to demonstrate the potential of this approach -- and the powerful synergy it creates -- through the visual demonstration offered by \Lens~(illustrated in \autoref{fig:concepts:lens}).

\begin{figure}[ht]
    \centering
    \includegraphics[width=0.49\textwidth]{assets/lens_website.jpg}
    \includegraphics[width=0.49\textwidth]{assets/lens_click.jpg}
    \caption{\textbf{LENS Project.} Example of results from the LENS demo for the espresso class. \textbf{(Left)} The page displays the top 10 concepts, ranked from most to least important for the class. These concepts are extracted using \craft~and visualized with \maco; their importance is calculated using the optimal formula found in \autoref{sec:concepts:holistic}. \textbf{(Right)} Clicking on a feature visualization that illustrates a concept reveals the image crops that most strongly activate the concept.}
    \label{fig:concepts:lens}
\end{figure}

This platform organizes, for each of the 1000 ImageNet classes, the ten most significant concepts, along with their feature visualizations and respective importance. 

\paragraph{Perspective.} While the potential of concept-based methods is clear, it is now critical to establish distinct research directions to fully unlock their potential in the wake of preliminary studies. Four key areas emerge, meriting further exploration:

\begin{itemize}

    \item \textbf{Revisiting Dictionary Learning:} The evident parallels between concept extraction and dictionary learning highlight a pressing need for the XAI community to reassess and tailor dictionary learning methodologies for application in explainability. This adaptation could bridge gaps in our understanding and application of these techniques within XAI.

    \item \textbf{Beyond Classification:} The necessity of extending our investigative scope beyond mere classification tasks is crucial. Diverse models, including bounding box detection, segmentation, generative and Vision-Language models present intricate challenges and vast opportunities for enhancing explainability. Diversifying our focus will enable a general comprehension of AI systems, integrating a broader spectrum of tasks and functionalities, thus deepening the XAI field with richer insights and more adaptable explainability tools. An illustrative example is given in \autoref{fig:holistc:conceptbbox}.
    
    \item \textbf{Exploring Hierarchical Concepts and Compositionality:} Investigating hierarchical concepts and their compositionality also offers a very promising path to deepen our understanding of how neural networks operate. Recent research has highlighted that models can exhibit compositional behaviors~\cite{lepori2024break}. Understanding the ways in which concepts are combined and interact at various abstraction levels could offer a nuanced perspective on decision-making processes within models, paving the way for more refined interpretability strategies.
    
    \item \textbf{Expanding on Synergies:} The demonstrated synergy among different explainability methods within our framework suggests a fertile area for research. A comprehensive examination of how these methods can be cohesively integrated, and the resultant synergistic effects could lead to groundbreaking insights and the development of potent tools for explainability.
    
\end{itemize}

\begin{figure}
    \centering
    \includegraphics[width=1.05\textwidth]{assets/holistic/concept_bbox.jpg}
    \caption{\textbf{Concepts of \craft~on ResNet50 Trained on Different Tasks.} The concepts extracted for classification tasks, specifically for the St. Bernard class, seem to focus on the head and spots of the St. Bernard. In the case of the bounding box (bbox) model, the legs are also deemed important, possibly because they help delineate the edges of the bounding boxes? Interestingly, for the CLIP model, the dog head concept is also activated by human heads, suggesting that despite visual differences (in the pixel space), the concept of 'head' seems present for models trained with language components like CLIP.}
    \label{fig:holistc:conceptbbox}
\end{figure}

While this framework does not solve all the challenges presented in the \autoref{chap:intro}, and there remains a significant journey toward fully understanding models such as ResNet50 or ViT, it opens a novel avenue. We encourage the academic community to explore the synergies between attribution methods, concepts, and feature visualization for deeper explainability.


