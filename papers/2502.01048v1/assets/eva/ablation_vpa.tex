\begin{table}[h]
    \centering
    \scalebox{0.9}{
        \begin{tabular}{l c P{5mm}P{5mm}P{5mm}P{5mm}}
        \toprule
        & Tightness$\downarrow$ & Del.$\downarrow$ & Ins.$\uparrow$ & Fid.$\uparrow$ & Rob.$\downarrow$ \\
        \midrule
        IBP & 4.58 & .148 & .588 & .222 & .077 \\
        Forward & 2.66 & .150 & .580 & .209 & .078 \\
        Backward & \underline{2.36} &  \underline{.115} & \underline{.607} & \underline{.274} & \underline{.074} \\
        IBP + Fo. + Ba. & \textbf{1.55} & \textbf{.089} & \textbf{.736} & \textbf{.428} & \textbf{.069} \\
        \bottomrule
        \vspace{-4mm}
        \end{tabular}
    }
    \caption{\textbf{Impact of the verified perturbation analysis method on EVA.}
    Results of \eva on Tightness, Deletion (Del.), Insertion (Ins.), Fidelity (Fid.) and $\rsr$ (Rob.) metrics obtained on MNIST. The Tightness score corresponds to the average adversarial surface. A lower Tightness score indicates that the method is more precise: it reaches tighter bound, resulting in better explanations and superior scores on the other metrics. The first and second best results are respectively in \textbf{bold} and \underline{underlined}.
    \vspace{-4mm}
    }
    \label{tab:eva:ablation_vpa}
\end{table}
