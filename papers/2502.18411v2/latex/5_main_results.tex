\section{Evaluation Results}

\subsection{SFT with OmniAlign-V }

We conduct extensive experiments to demonstrate OmniAlign-V's effectiveness. 
We combine OmniAlign-V with LLaVA-Next-778k (excluding text-only samples), 
creating OmniAlign-V$_{mix}$ with 946K training samples. 
We evaluate various MLLMs tuned on OmniAlign-V against their counterparts tuned on LLaVA-Next-778k.

Our evaluation spans multiple multi-modal benchmarks, 
including standard VQA benchmarks~\cite{yu2023mm,liu2024mmbench,liu2023hidden,yue2023mmmu,kembhavi2016diagram} and human-preference alignment benchmarks: MM-AlignBench, WildVision ~\cite{lu2024wildvision}, and MIA-Bench~\cite{qian2024mia}. 
Results in \cref{tab: main} show that OmniAlign-V significantly improves human alignment across all benchmarks. 
Moreover, our training data improves general multi-modal capabilities, particularly on MMVet and MMMU, demonstrating a trend distinct from text-only data. 

Notably, despite excluding language samples from training data, models maintain stronger language alignment than those trained on LLaVA-Next-778k, as shown in Table~\ref{tab: main_lang}. 
This suggests that while high-quality language data alone may not significantly impact multi-modal capabilities, 
enhancing multi-modal data quality can improve both language and multi-modal performance, highlighting the crucial role of high-quality, human-aligned multi-modal training data.


% We conduct several experiments to demonstrate the efficacy of our training data. We remove the pure-text samples from LLaVA-Next-778k and combined the remaining data with our own, resulting in a total of 946k training samples OmniAlign-V$_{mix}$. We first evaluate on Multi-Modal benchmarks, encompassing both common VQA benchmarks~\cite{yu2023mm,liu2024mmbench,liu2023hidden,yue2023mmmu,kembhavi2016diagram} and those that evaluate alignment of models, MM-AlignBench, WildVision~\cite{lu2024wildvision}, and MIA-Bench~\cite{qian2024mia} which focus on the multi-modality instruction-following performance. The results are shown in Tab.~\ref{tab: main}.

% It can be observed that incorporating our training data significantly enhances the alignment of models with human. This improvement is evident not only on the MM-AlignBench but also manifests as a substantial zero-shot increase on WildVision and MIA-Bench.
% Moreover, the model's general multi-modal capabilities improve with our training data, particularly notable on MMVet and MMMU, which demonstrates a markedly different trend compared to language data.
% % This observation underscores the high quality of our data.
% \begin{table}[t]
    \centering
    \resizebox{.5\textwidth}{!}{%
    \tablestyle{3pt}{1.1}
    \begin{tabular}{l|cc|ccccc}
    % \toprule[0.15em]
    \Xhline{0.15em}
        \textbf{Model} &\textbf{Win Rate} $\uparrow$ &\textbf{Reward}$\uparrow$ &B+ & B& T& W& W+\\
        % \midrule
        \hline
        % GPT-4o & & &  /  /  /  / \\
        Claude3.5V-Sonnet & 84.9 & +51.4 & 70 & 144 & 12 & 31 & 4 \\ 
        GPT-4o &81.3 &+49.0 & 81 & 124 & 12 & 31 & 4\\
        GPT-4V &82.5 &+46.0 & 57 & 157 & 12 & 31 & 1\\
        GeminiFlash1.5-002 &77.0 &+39.1 & 56 & 138 & 14 & 35 & 9\\
        \rowcolor{gray!20}
        \textbf{LLaVANext-OA-32B-DPO} & 74.2 & +36.9 & 49 & 138 & 20 & 40 & 5 \\
        Qwen2VL-72B   &61.5 &+21.6 & 43 & 112 & 15 & 75 & 7\\
        \rowcolor{gray!20}
        \textbf{LLaVANext-OA-32B} & 62.3 & +19.4 & 31 & 126 & 19 & 62 & 14 \\
        \hdashline
        Claude-3V-Sonnet & 50 & 0  & - & - & - & - & -  \\
        \hdashline
        Qwen2VL-7B   &44.4 & -5.8 & 28 & 84 & 5 & 101 & 34\\
        InternVL2-72B &44.4 &-6.9 & 19 & 93 & 8 & 98 & 34\\
        InternVL2-8B-MPO &40.1 & -10.9 & 26 & 75 & 10 & 100 & 41\\
        InternVL2-8B &31.3 & -21.8 & 18 & 61 & 15 & 109 & 49\\
        LLaMA3.2-Vision-11B & 27.8&-33.7 & 18 & 52 & 4 & 98 & 80\\
        \rowcolor{gray!20}
        \textbf{LLaVANext-Qwen32B} & 26.6 & -29.0 & 16 & 51 & 10 & 121 & 54 \\
        LLaVA-OneVision-7B &23.8 &-46.2 & 14 & 46 & 1 & 75 & 116 \\
        MiniCPM-V-2.5 &12.7 & -53.0 & 9 & 23 & 8 & 116 & 96 \\
        Xcomposer2.5-7B &7.5 & -74.0 & 5 & 14 & 3 & 63 & 167\\
        Idefics3-8B     &2.7 & -92.3 & 3 & 4 & 0 & 15 & 230\\
        \Xhline{0.15em}
    % \bottomrule[0.1em]
    \end{tabular}
    }%
    \caption{\textbf{Performance of existing MLLMs on MM-AlignBench}. B+/B/T/W/W+ denotes MuchBetter/Better/Tie/Worse/MuchWorse.  Our LLaVA-Next-OmniAlign(OA)-32B-DPO, trained with OmniAlign-V and applied DPO with OmniAlign-V-DPO, demonstrates outstanding performance, surpassing a wide range of strong MLLMs, even Qwen2VL-72B.}
    \label{tab: bench}
    \vspace{-10pt}
\end{table}
% Furthermore, we observe that despite the absence of language samples in training data, the alignment of model's language capability remains stronger than that of LLaVA-Next-778k dataset. The results are shown in Table~\ref{tab: main_lang}.
% This phenomenon suggests that high-quality language training data alone may not significantly influence multi-modal capabilities. However, enhancing the quality of multi-modal data can lead to improvements in both language and multi-modal performance. This underscores the essential role of high-quality, human-aligned multi-modal training data.
\section{\dpo}

\begin{figure*}
    \centering
    \includegraphics[width=\linewidth]{imgs/dpo.pdf}
    \caption{\textbf{Overview of the \dpo framework,} The dynamic reward scaling mechanism adjusts the update strength based on the reward margin, improving optimization stability and robustness.}
    \label{fig:dpo_alg}
\end{figure*}

In this section, we propose \dpo, an extension of the traditional DPO framework. \dpo introduces Dynamic Reward Scaling, which dynamically adjusts the update strength based on the confidence of training pairs, ensuring effective utilization of high-quality samples while mitigating the impact of noisy or low-confidence data.

\subsection{Background: Direct Preference Optimization}
The DPO framework is a preference-based learning method that optimizes model parameters $\theta$ by aligning model outputs with human preferences. Given a query $\mathbf{x}$ and corresponding responses $y_w$ (positive) and $y_l$ (negative), the DPO loss is defined as:
\begin{equation}
\ell_{\text{DPO}}(\theta) = 
\mathbb{E}_{\mathbf{x}, y_w, y_l} 
\Big[ 
    - \log \sigma \Big( 
        \beta \Big( 
            \log \frac{\pi_\theta(y_w | \mathbf{x})}{\pi_{\text{ref}}(y_w | \mathbf{x})} 
            - 
            \log \frac{\pi_\theta(y_l | \mathbf{x})}{\pi_{\text{ref}}(y_l | \mathbf{x})} 
        \Big)
    \Big)
\Big],
\end{equation}
where $\pi_\theta$ is the model's predicted probability distribution, $\pi_{\text{ref}}$ is a reference policy, $\beta$ is a scaling factor, and $\sigma(\cdot)$ is the sigmoid function. Traditional DPO treats all training pairs equally, regardless of their quality differences. This uniform scaling fails to prioritize high-quality pairs with clear preference distinctions, leading to inefficient use of informative samples and suboptimal optimization.

\subsection{\dpo: Key Contributions and Improvements}
\paragraph{Training on all possible comparison pairs instead of the hardest pairs}.  
Unlike many recent MLLM alignment approaches that prioritize training on the hardest comparison pairs, \dpo incorporates all possible comparison pairs for a single query into the training process. Specifically, for any query with multiple responses, every response pair with differing ranks is treated as a valid comparison pair. This comprehensive approach captures more nuanced ranking information, allowing the model to learn from a broader set of preferences. However, this strategy also introduces a challenge: pairs involving responses with similar ranks (e.g., rank 3 and rank 4) often have lower reward margins compared to pairs with more distinct rankings (e.g., rank 1 and rank 4). Treating all pairs equally, as in traditional DPO, exacerbates the issue of uniform scaling and underutilizes the high-confidence information contained in larger reward margins. To address this, \dpo introduces Dynamic Reward Scaling, which dynamically adjusts the update strength based on the reward margin to prioritize high-confidence training pairs.

\begin{wrapfigure}{r}{0.34\linewidth}
\vspace{-0.7cm}
  \begin{center}
    \includegraphics[width=\linewidth]{imgs/beta_curve.pdf}
\vspace{-0.4cm}
\caption{Effect of $k$ on $1 - e^{-k \delta}$.}
\label{fig:beta_func}
\end{center}
\vspace{-0.4cm}
\end{wrapfigure}
\paragraph{Definition of dynamic reward scaling}. Reward models can naturally provide a pairwise reward margin, which serves as a straightforward signal for scaling. However, two critical aspects must be addressed: (1) ensuring the signal quality is sufficiently high, and (2) bounding the signal to prevent overly aggressive updates that might destabilize training.


Regarding the first aspect, our experiments reveal that publicly available models, such as GPT-4o and LLaVA-Critic, perform inadequately in scoring our dataset. Conversely, our \abbr-Reward-7B model surpasses several publicly available 72B models, offering a reliable and robust reward signal. We use this model to compute the reward margin: 
 $\delta = r(y_w) - r(y_l),$
where $r(y_w)$ and $r(y_l)$ are the scores assigned to the positive and negative samples.

For the second factor, we control the scaling factor $\beta(\delta)$ using the following formulation:
\[
\beta(\delta) = \beta_{\text{ori}} \Big( 1 + w \big( 1 - e^{-k \delta} \big) \Big),
\]

where $\beta_{\text{ori}}$ is the initial default scaling factor, $w$ is a parameter balancing the dynamic component's contribution, and $k$ is a tunable hyperparameter that adjusts $\beta(\delta)$'s sensitivity to changes in $\delta$. The function $1 - e^{-k \delta}$ is bounded between $[0, 1]$, {as illustrated in Figure~\ref{fig:beta_func}}. A smaller $k$ value keeps most $\beta(\delta)$ values near $\beta_{\text{ori}}$, with slow growth as $\delta$ increases. In contrast, a larger $k$ makes $\beta(\delta)$ highly responsive to changes in $\delta$, quickly reaching its maximum. To avoid overly aggressive updates, we constrain $\beta(\delta)$ within $[\beta_{\text{ori}}, (1 + w) \beta_{\text{ori}}]$. Overall, Dynamic Reward Scaling significantly enhances \dpo by leveraging high-quality reward signals and tailoring optimization steps to the confidence level of training pairs. This results in improved robustness, efficiency, and overall effectiveness of the framework. We discuss the similarities and differing perspectives between our approach and existing methods in Appendix~\ref{sec:app_com_beta}.


\subsection{DPO with OmniAlign-V-DPO}
We conduct DPO post-training on three models: 
LLaVA-Next trained with LLaVA-Next-778k, 
LLaVA-Next trained with OmniAlign-V$_{mix}$, 
and InternVL2-8B. 
Results in \cref{tab: dpo} show that DPO tuning significantly improves performance on real-world questions (WildVision) across all models. 
While the baseline trained solely on LLaVA-Next-778k shows minimal improvement on MM-AlignBench, 
models incorporating OmniAlign-V during SFT demonstrate substantial gains after DPO. 
Similarly, InternVL2-8B, a state-of-the-art MLLM partially trained on proprietary multi-modal corpora, shows significant improvement on MM-AlignBench post-DPO. 
% We hypothesize that sufficient and diverse multi-modal training during SFT is a prerequisite for OmniAlign-V DPO's effectiveness. 
This indicates that if a model has been trained on data aligned with human preferences, such as open-ended or long-context data  during SFT phase, the subsequent DPO training on high-quality human-aligned data can significantly activate the model’s ability, leading to a considerable improvement in alignment performance. 
In contrast, if the model has not been exposed to such alignment-focused datasets during SFT, training with open-ended data alone via DPO will not significantly improve its capabilities of alignment.
These findings demonstrate the value of OmniAlign-V in both SFT and DPO stages for enhancing human preference alignment.

% We conduct DPO post-training experiments on the following models: LLaVA-Next trained with LLaVA-Next-778k; LLaVA-Next trained with OmniAlign-V$_{mix}$; InternVL2-8B, the results are shown in Table.~\ref{tab: dpo}.

% It can be observed that after the DPO stage, all models exhibit better performance on questions in the wild(WildVision).
% On MM-AlignBench, the LLaVANext-InternLM trained solely with LLaVA-Next-778k does not show significant improvement.
% However, integrating our data during the SFT stage leads to further substantial enhancements in performance after the DPO stage.
% InternVL2-8B also shows a significant improvement on MM-AlignBench, which can be attributed to the large volume of context-rich image-text interleaved data used during SFT stage.
% These findings indicate that our data is not only effective when used in the SFT stage but also enhances model alignment with human preferences when integrated into the DPO post-training process.
\begin{table}[t]
    \centering
    \resizebox{.5\textwidth}{!}{%
    \tablestyle{3pt}{1.1}
    \begin{tabular}{l|cc|ccccc}
    % \toprule[0.15em]
    \Xhline{0.15em}
        \textbf{Model} &\textbf{Win Rate} $\uparrow$ &\textbf{Reward}$\uparrow$ &B+ & B& T& W& W+\\
        % \midrule
        \hline
        % GPT-4o & & &  /  /  /  / \\
        Claude3.5V-Sonnet & 84.9 & +51.4 & 70 & 144 & 12 & 31 & 4 \\ 
        GPT-4o &81.3 &+49.0 & 81 & 124 & 12 & 31 & 4\\
        GPT-4V &82.5 &+46.0 & 57 & 157 & 12 & 31 & 1\\
        GeminiFlash1.5-002 &77.0 &+39.1 & 56 & 138 & 14 & 35 & 9\\
        \rowcolor{gray!20}
        \textbf{LLaVANext-OA-32B-DPO} & 74.2 & +36.9 & 49 & 138 & 20 & 40 & 5 \\
        Qwen2VL-72B   &61.5 &+21.6 & 43 & 112 & 15 & 75 & 7\\
        \rowcolor{gray!20}
        \textbf{LLaVANext-OA-32B} & 62.3 & +19.4 & 31 & 126 & 19 & 62 & 14 \\
        \hdashline
        Claude-3V-Sonnet & 50 & 0  & - & - & - & - & -  \\
        \hdashline
        Qwen2VL-7B   &44.4 & -5.8 & 28 & 84 & 5 & 101 & 34\\
        InternVL2-72B &44.4 &-6.9 & 19 & 93 & 8 & 98 & 34\\
        InternVL2-8B-MPO &40.1 & -10.9 & 26 & 75 & 10 & 100 & 41\\
        InternVL2-8B &31.3 & -21.8 & 18 & 61 & 15 & 109 & 49\\
        LLaMA3.2-Vision-11B & 27.8&-33.7 & 18 & 52 & 4 & 98 & 80\\
        \rowcolor{gray!20}
        \textbf{LLaVANext-Qwen32B} & 26.6 & -29.0 & 16 & 51 & 10 & 121 & 54 \\
        LLaVA-OneVision-7B &23.8 &-46.2 & 14 & 46 & 1 & 75 & 116 \\
        MiniCPM-V-2.5 &12.7 & -53.0 & 9 & 23 & 8 & 116 & 96 \\
        Xcomposer2.5-7B &7.5 & -74.0 & 5 & 14 & 3 & 63 & 167\\
        Idefics3-8B     &2.7 & -92.3 & 3 & 4 & 0 & 15 & 230\\
        \Xhline{0.15em}
    % \bottomrule[0.1em]
    \end{tabular}
    }%
    \caption{\textbf{Performance of existing MLLMs on MM-AlignBench}. B+/B/T/W/W+ denotes MuchBetter/Better/Tie/Worse/MuchWorse.  Our LLaVA-Next-OmniAlign(OA)-32B-DPO, trained with OmniAlign-V and applied DPO with OmniAlign-V-DPO, demonstrates outstanding performance, surpassing a wide range of strong MLLMs, even Qwen2VL-72B.}
    \label{tab: bench}
    \vspace{-10pt}
\end{table}

\subsection{MM-AlignBench}
We evaluate various state-of-the-art MLLMs~\cite{2022chatgpt,team2023gemini,Claude3,bai2023qwen,chen2024far,li2024llava,minicpm2024,internlmxcomposer2,LLaMA32Vision,laurençon2024building,wang2024enhancing} on MM-Alignbench, with results shown in \cref{tab: bench}. 
Closed-source models like GPT, Claude, and Gemini demonstrate strong alignment in responding to open-ended questions. 
In contrast, while Qwen2-VL and InternVL2 excel at common VQA tasks, they show relatively lower human preference alignment. 
This highlights the importance of prioritizing MLLM alignment for improved daily human interactions. 
Our LLaVA-OA-32B, trained with OmniAlign-V, achieves exceptional performance, outperforming numerous strong MLLMs and nearly matching Qwen2VL-72B. 
After applying DPO with OmniAlign-V-DPO, \textbf{LLaVA-OA-32B-DPO achieves winning rate of 72.6 with an average reward of +33.5,  surpassing the performance of Qwen2VL-72B}. 
These results highlight the high quality and effectiveness of the OmniAlign-V dataset in improving model alignment with human preferences.


% We evaluate SOTA MLLMs on MM-Alignbench, with the results presented in Tab.~\ref{tab: bench}. It is evident that closed-source models, such as GPT, Claude, and Gemini, exhibit strong alignment when responding to open-ended questions. In contrast, Qwen2VL and InternVL2 perform exceptionally well on common VQA tasks but show comparatively lower alignment with human preferences. This suggests that the alignment of MLLMs should be prioritized to enhance their utility in daily human interactions. Our LLaVA-OA-32B, trained with OmniAlign-V, demonstrates outstanding performance, surpassing a wide range of strong MLLMs and nearly matching the performance of Qwen2VL-72B, shows the great quality of our data.

\subsection{Ablation Study}
We conduct an ablation study to evaluate each subset of OmniAlign-V, 
reporting results on MM-Alignbench, WildVision, and MMVet in \cref{tab: ablation}. 
Performance improves progressively as different tasks from OmniAlign-V are incorporated. 
Notably, Instruction-Following data significantly enhances performance across all three benchmarks, 
demonstrating its crucial role. 
The creation data subset uniquely improves performance on MM-Alignbench but not on WildVision and MMVet.
This discrepancy can be attributed to the absence of multi-modal creative question types in these two benchmarks, suggesting their incompleteness in capturing full spectrum of alignment challenges.
\begin{table*}[h]
    \centering
    \resizebox{\textwidth}{!}{
    \renewcommand{\arraystretch}{1.1}
    \renewcommand{\tabcolsep}{4pt}
    \begin{tabular}{l c c c c a c c c c a c c c c a c c c}
    \toprule
        &
        \multicolumn{5}{c}{\textsc{Structural}} &
        \multicolumn{5}{c}{\textsc{Chemical}} &
        \multicolumn{5}{c}{\textsc{Biological}} &
        \multicolumn{3}{c}{\textsc{PAMPA}} \\
    \cmidrule(l{2pt}r{2pt}){2-6}
    \cmidrule(l{2pt}r{2pt}){7-11}
    \cmidrule(l{2pt}r{2pt}){12-16}
    \cmidrule(l{2pt}r{2pt}){17-19}
        Models & Help. & Relev. & Acc. & Details & Overall & Help. & Relev. & Acc. & Details & Overall & Help. & Relev. & Acc. & Details & Overall & Default & CoT & w/ Task Info.\\
    \midrule
    S & 1.124 & 1.119 & 1.128 & 1.123 & 1.119 & 1.127 & 1.183 & 1.189 & 1.133 & 1.166 & 1.201 & 1.320 & 1.499 & 1.310 & 1.328 & 37.35 & 32.92 & 56.02 \\
    S+S2F & 1.176 & 1.163 & 1.189 & 1.184 & 1.172 & 1.249 & 1.286 & 1.325 & 1.259 & 1.285 & 1.604 & 1.814 & 2.005 & 1.608 & 1.754 & 66.83 & 64.13 & 57.88\\
    Conv. & 1.169 & 1.152 & 1.172 & 1.176 & 1.166 & 0.674 & 0.814 & 0.787 & 0.519 & 0.689 & 0.861 & 1.079 & 1.067 & 0.589 & 0.887 & 41.03 & 45.95 & 83.05 \\
    Full & 1.126 & 1.145 & 1.154 & 1.090 & 1.125 & 1.224 & 1.266 & 1.302 & 1.211 & 1.251 & 1.578 & 1.840 & 2.030 & 1.528 & 1.744 & 63.55 & 64.37 & 72.48 \\
    \bottomrule
    \end{tabular}}
    \vspace{-0.1in}
    \caption{Ablation studies on data types. S denotes the detailed structural descriptions, S2F denotes the structure-to-feature relationship explanations, and Conv. denotes the comprehensive conversations. We report all scores including helpfulness relevance, accuracy, level of detail, and overall score following the settings in Section~\ref{sec:quantitative}. Additionally, we report the accuracy on PAMPA task.}
    \vspace{-0.07in}
    \label{app:tab:ablation_data}
\end{table*}


% We conduct an ablation study to evaluate the impact of each component of our training data and report the results on MM-Alignbench, WildVision, and MMVet, as detailed in Table~\ref{tab: ablation}. The performance on these benchmarks improves progressively by incorporating a series of tasks within OminiAlign-V. Specifically, the integration of Instruction-Following significantly enhances performance across all three benchmarks, highlighting its importance. Additionally, we observe that the inclusion of creation data leads to performance gains only on MM-Alignbench, likely due to its distinct distribution compared to WildVision and MMVet.

\section{Conclusion}
In this paper, we introduce \textbf{OmniAlign-V}, 
a dataset designed to enhance the alignment of MLLMs with human preferences, 
as well as \textbf{MM-AlignBench}, 
a high-quality, specific-purpose benchmark for evaluating such alignment. 
We investigate the impact of both language and multi-modal training data, 
emphasizing the critical role of multi-modal open-ended training data. 
By incorporating OmniAlign-V into SFT and DPO stages, we achieve significant improvements in the alignment of MLLMs. 
% OmniAlign-V is poised to become resource in advancing the development of MLLMs.

\section{Limitation}
% Although OmniAlign-V  demonstrates strong performance in improving the alignment of MLLMs, the scale of our dataset remains limited and may not be sufficient for large-scale training. Additionally, the selected tasks may lack generality and fail to encompass all aspects of daily life. Deeper exploration into the alignment of MLLMs is still needed to address these limitations and further advance the field.
Although the OmniAlign-V pipeline can be easily scaled to support larger datasets, the scale of the dataset used in this paper may be insufficient for large-scale training due to cost limitations. 
% Besides, MM-Alignbench utilizes GPT-4o for evaluation, which may not fully align with human preferences.
Deeper exploration into the alignment of MLLMs is still needed to address these limitations and further advance the field.