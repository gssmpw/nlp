\newpage
\newpage
\newpage
\appendix

\section{Alignment of MLLMs with Human Preference}
\label{appd: align}
Although open-source MLLMs have already matched or even surpassed proprietary models like the GPT and Claude series in common VQA tasks like OCR and Visual Perception, a significant gap remains in their alignment with human preferences. When posed with open-ended questions that require knowledge-rich responses, even the most advanced open-source MLLM, InternVL2-76B, struggles to provide comprehensive answers with high readability, as illustrated in Fig. \ref{fig:internvl}. In contrast, GPT-4o not only accurately identifies the main objects relevant to the question but also provides well-structured responses enriched with comprehensive contextual knowledge, achieving a high level of alignment with human preferences.

\section{Image sources}
\label{sources}
We carefully select image sources for \textit{Arts}, \textit{Charts}, \textit{Diagrams}, and \textit{Posters} tasks.
For \textit{Arts}, WikiArt~\cite{wikiart} is selected as the image source, offering a diverse range of painting styles. We uniformly sample 2000 images across all painting styles to ensure diversity in the dataset.
For \textit{Charts}, we select ChartQA~\cite{masry2022chartqa}, a dataset featuring charts that contain substantial statistical information. ChartQA includes several subcategories, from which we filter out simplistic charts with only two columns and retain those that contain charts with rich contextual information and diverse types.
For \textit{Diagrams}, we choose images from TextbookVQA~\cite{wikiart}, which provides diagrams rich in natural content and detailed explanations. We exclusively utilize the question images and teaching images from the image sources, as they meet our specific requirements.
For \textit{Posters}, we utilize InfographicsVQA~\cite{mathew2022infographicvqa}, a dataset containing high-quality posters with intricate designs and informative content.  

% \section{Image Examples}
% For instance, an image densely populated with bananas may be classified as having high complexity due to the abundance of objects but may lack meaningful semantic content. 

\section{Chart Post-Refine Pipeline}
\label{appd: chart post}
After curating images along with high-quality question, we first employ the current most powerful models(GPT-4o, InternVL2-72B, Qwen2VL-72B) to generate answers separately. 
A powerful LLM(Qwen2.5-72B) is then utilized to extract objective facts from the chart within each generated answer. 
The facts extracted from the answers of multiple models are compared for consistency. If the facts differ significantly and lead to entirely different conclusions, such responses are flagged for further review or discarded to avoid misinformation.
For cases where the facts exhibit only minor differences, we merge the detailed factual content from Qwen2VL-72B into the comprehensive explanations provided by GPT-4o. 
The merged answers are further reviewed and supervised by human experts to ensure their quality and consistency. 
\begin{figure}[t]
    \centering
    \includegraphics[width=\linewidth]{figs/appendix_sft.pdf}
    \caption{Examples of limitation with current multi-modal instruction tuning dataset.}
    \label{fig:badsft}
\end{figure}
\section{Training Details}
Our training strategy largely follows the approaches adopted by LLaVA and LLaVANext. CLIP-Large-336-Patch14 is employed as the visual encoder. In line with the LLaVA training strategy, we first conduct a pretraining stage where both the visual encoder and the LLM are frozen. 
We utilize the LLaVA-pretrain-558k  and ALLaVA-pretrain-728k  datasets for pretraining. The batch size is uniformly set to 256 and learning rate is set to 1e-3.
During this phase, images are resized to 336×336, and no image-splitting method is applied. 

For SFT stage, we unfreeze the LLM for LLaVA and further unfreeze the visual encoder for LLaVANext. 
In the case of LLaVANext, we apply the image-splitting method, setting the maximum split size to 3×3. The batch size is uniformly set to 128 and learning rate is set to 2e-5.. LLaVA-InternLM2.5-7B  is trained using 8×A800 GPUs for 12 hours. LLaVANext-InternLM2.5-7B  is trained using 16×A800 GPUs for 13 hours. LLaVANext-Qwen2.5-32B  is trained using 32×A800 GPUs for 24 hours.

\begin{table}[t]
    \centering
    \resizebox{.5\textwidth}{!}{%
    \begin{tabular}{l|ccc}
    \Xhline{0.15em}
        \textbf{Model}  &\textbf{MM-AlignBench}  &\textbf{WildVision} &\textbf{MMVet}\\
        \hline
        LLaVANext-77k                   & 15.1/-52.6  & 13.6 / -63.1 & 37.7\\
        + 33k w.o. Imager Filter              & 31.4 / -42.3 & 22.0 / -42.3 & 42.0\\
        % inf $\rightarrow$ refined inf  \\
        + 33k w. Imager Filter            & 35.3 / -41.0 & 22.6 / -37.5 & 44.4\\
        % replace chart with refined       \\
    \Xhline{0.15em}
    \end{tabular}
    }%
    \caption{Ablation study on the impact of utilizing image filter.}
    \label{tab: ablation image}
    \vspace{-10pt}
\end{table}
\section{Ablation on Image Filter}
\label{appx: filterablation}
We conduct ablation studies to evaluate the impact of using our image filter. We randomly sample 77k images from the LLaVA-Next-778K SFT dataset as a baseline and separately select images from CC3M both with and without applying our image filter. 
Subsequently, the selected images are used to generate Knowledge and Inferential Question-Answer pairs following the pipeline described in \cref{sec: sft generate}.  
We then assess the effect of these datasets on the performance of LLaVA-Next-InternLM2.5-7B, with the results presented in \cref{tab: ablation image}. 
It can be observed that by utilizing the image filter, the selected images contain richer semantic information, leading to the generation of higher-quality image-question pairs. This, in turn, enhances the model's performance in terms of alignment with human preferences.

\section{Human Experts}
In this study, two authors (both PhD students in computer science) serve as human experts. They are responsible for reviewing and refining the questions in MM-AlignBench, as well as evaluating and filtering incorrect merged cases following the Chart Post-Processing Refinement stage.

\section{License}
The InternLM and Qwen  models are licensed under the Apache-2.0  license. The ChartQA dataset is distributed under the GNU General Public License v3.0. The remaining datasets are licensed under CC BY-NC 4.0, which permits only non-commercial use.

\begin{figure*}[t]
    \centering
    \includegraphics[width=\linewidth]{figs/appendix_internvl.pdf}
    \caption{GPT-4o shows superior alignment with human preference than InternVL2-76B.}
    \label{fig:internvl}
\end{figure*}
\begin{figure*}[t]
    \centering
    \includegraphics[width=\linewidth]{figs/appendix_image.pdf}
    \caption{Demonstration examples of our image filter.}
    \label{fig:imagefilter}
\end{figure*}
\begin{figure*}[t]
    \centering
    \includegraphics[width=\linewidth]{figs/appendix_wildvision.pdf}
    \caption{Examples of limitation within current multi-modal benchmark for alignment.}
    \label{fig:badbench}
\end{figure*}
\begin{figure*}[t]
    \centering
    \includegraphics[width=\linewidth]{figs/appendix_omini.pdf}
    \caption{Examples of each task in OmniAlign-V.}
    \label{fig:datasample1}
\end{figure*}
\begin{figure*}[t]
    \centering
    \includegraphics[width=\linewidth]{figs/appendix_omni2.pdf}
    \caption{Examples of each task in OmniAlign-V.}
    \label{fig:datasample2}
\end{figure*}
\begin{figure*}[t]
    \centering
    \includegraphics[width=\linewidth]{figs/appendix_omni3.pdf}
    \caption{Examples of each task in OmniAlign-V.}
    \label{fig:datasample3}
\end{figure*}
\begin{figure*}[t]
    \centering
    \includegraphics[width=\linewidth]{figs/appendix_omni4.pdf}
    \caption{Examples of each task in OmniAlign-V.}
    \label{fig:datasample4}
\end{figure*}


\begin{figure*}[!ht] 
\begin{AIbox}{Prompt for Knowledge Task}
{Examine the image provided and generate a knowledge-based and exploratory question based on the content of the image and supply corresponding detailed answers.\\
 Question Guidelines:\\
- Your question should invite insightful discussion on the types of elements in the image, such as:\\
  - **Objects**: For example, animals, plants, food, or products.\\
    - E.g., "What breed of dog is in the picture, and what are their characteristics?", "Can you give me a recipe for the food in image?", "Write a Product Description for the product in the image.".\\
  - **Locations and Features**: Relevant to countries, landmarks, famous people, or scenic spots.\\
    - E.g., "Please introduce the history of the landmark in the picture.", "How did the states in image get their names?", "Who is the person in the image and what is him famous for?".\\
  - **Activities and Technologies**: Related to sports, machines, technology, and environmental details.\\
    - E.g., "Can you explain how the game in image is played?", "How is the machine shown in the image operated?"\\
  - **Events, Literature, and Media**: Concerning books, movies, or series in the image.\\
    - E.g., "Write a short description about the movie or series in the image.", "Think of books that would be enjoyable for someone who liked the books in the image."\\

 Answer Guidelines:\\
- Ensure your answers are factual and comprehensive.\\
- Please use Markdown formatting in your text to enhance the content, making it visually appealing and easy to read. Include appropriate headings, subheadings, lists, code blocks, and other Markdown elements to optimize your answers.\\
 Output Format:\\
Your response should strictly follow this format:\\
```json\\
\{\\
  "question": "Question text",\\
  "answer": "Answer text"\\
\}\\
```
}
\end{AIbox} 
\caption{\textbf{An Example of the prompt for Knowledge task generation. }}
\label{fig: prompt_kn}
% \vspace{-5mm}
\end{figure*}

\begin{figure*}[!ht] 
% \vspace{-5mm}
\begin{AIbox}{Prompt for Creative tasks}
{You are a skilled writer with a talent for crafting insightful and engaging questions based on the content of a given image.\\
**Task Guidelines**:\\
- Analyze the content of the provided image and select one of the question types listed below. Use it to create engaging questions that lead the viewer to explore and interpret the image in different level of complexity.\\
    - Be closely tied to the content of the image, emphasizing its primary visual or thematic elements.\\
    - Your questions should **avoid directly referencing specific details** in the image. Instead, they should encourage deeper reflection, ensuring the question cannot be answered without seeing the image.\\
    - Avoid overly rigid or direct phrasing, focusing instead on open-ended exploration.\\
**Question Types**\\
Your questions can be from the following types, each followed by an example with a different level of complexity.\\
- Simple (basic observation or initial reflection)\\
- Moderate (more thought-provoking, requiring a deeper understanding and more structured response)\\
- Difficult (complex or abstract, requiring analysis and strict formatting) \\
\textcolor{Red}{\{Match Types\}} \\
**Output Format**:\\
Your response should strictly follow this format:\\
```json\\
\{\\
  "question": "Question text",\\
  "type": "Question type",\\
  "level": "Question level"\\
\}
```
}
\end{AIbox} 
\caption{\textbf{An Example of the prompt for Creative task generation. }}
\label{fig: prompt_creation}
% \vspace{-5mm}
\end{figure*}

\begin{figure*}[!ht] 
% \vspace{-5mm}
\begin{AIbox}{Prompt for Inferential tasks}
{You are an image analysis expert skilled in posing high-quality **inferential questions**. Please provide 2-5 of the most insightful questions you can think of, following these guidelines:\\
For Questions:\\
- **Focus on image-based questions:** Ensure that your question cannot be answered without analyzing the image. **You should not directly provide image's data or details in your generated questions.** For example, "What might be the impact on the radio industry due to 34 stations having their licenses revoked?" includes the data in the image and can be answered without analyzing the image, so it is a bad question.\\
- Ensure that questions are natural, not overly rigid. **You should be quite certain and confident about the questions you pose and their answers, and avoid using words like "possibly", "maybe" or "might be" in both questions and answers.**\\
- Your questions must require reasoning beyond the direct content of the image, making reasonable inferences based on the information presented.
- The scope of the questions should not be overly broad or delve into political, philosophical, speculative, sensitive, or controversial topics. Stay within the context of the scene and elements inferred from it.\\
 For Answers:\\
- You should provide a clear and concise answer to the question.\\
Good Examples:\\
- What precautions are the people on the boat taking to stay comfortable during the trip?
- Is there anything else on the table other than the pizza?
- Why do these people choose to dress in this style?
- What decorative element is present in this public restroom that is not typical?
Bad Examples:\\
- What might indicate \\
}
\end{AIbox} 
\caption{\textbf{An Example of the prompt for Inferential task generation. }}
\label{fig: prompt_infer}
% \vspace{-5mm}
\end{figure*}

\begin{figure*}[!ht] 
% \vspace{-5mm}
\begin{AIbox}{Prompt for Chart tasks}
{You're a great image analyst. You need to analyze the image provided and generate some insightful questions based on the content of the image. \\
Question generation guidelines:\\
- Ensure that your questions require the image to be answered and do not include explicit information from the image. Instead, pose questions that prompt the respondent to analyze the image to find the answer.\\
- Your question should be explainable and require some reasoning to answer.\\
- Your question could contain different analytical perspectives, such as trends, comparisons, causal inference, etc.\\
- - Your questions should be insightful but also clear and straightforward. Avoid overly complex or niche questions.\\
Bad question examples:\\
- "What might be some factors contributing to the significantly higher private health expenditure per person in Argentina compared to Fiji and Benin?" (This includes specific details from the chart.) Its correct clarification should be "Is there any difference in private health expenditure per person between Argentina, Fiji, and Benin? If so, what might cause the difference?"\\
- "What trends can be observed in private health expenditures per person among the three countries shown?" (This question is unclear because 'trends between countries' is not a standard analytical concept. Trends typically refer to patterns over time or categories, not direct cross-entity comparisons.)\\
Output format:\\
Your response should strictly follow this format:\\
\{\\
"questions": [\\
\{\\
 "question": "Question text"\\
\},\\
\{\\
"question": "Another question text"\\
\}\\
]\\
\}\\
}
\end{AIbox} 
\caption{\textbf{An Example of the prompt for Chart task generation. }}
\label{fig: prompt_chart}
% \vspace{-5mm}
\end{figure*}

\begin{figure*}[!ht] 
% \vspace{-5mm}
\begin{AIbox}{Prompt for Poster tasks}
{You're an excellent image analyst and good at generating insightful questions about the image. You need to analyze the image provided and generate some insightful questions based on the content of the image, and you should answer the questions you generated.\\
Possible Categories Reference:\\
1. **Cultural and Social Context**\\
2. **Analysis and Inference**.\\
3. **Visual Elements and Design Techniques**.\\
Question Guidelines:\\
- **Focus on image-based questions:** Ensure that your question cannot be answered without analyzing the image. **You should not directly provide image's data or details in your generated questions.** For example, "What might be the impact on the radio industry due to 34 stations having their licenses revoked?" includes the data in the image and can be answered without analyzing the image, so it is a bad question.\\
- **Encourage thoughtful, structured responses:** Your question should be explainable and need some reasoning to answer. You should not generate questions that are just extracting information from the image. For example, "What percentage of organizations verify the past employment records of new employees according to the image?" is a bad question.\\
- **Ensure diversity in the questions:** Cover different aspects of the image, encouraging multiple perspectives. You can choose some appropriate categories from the possible categories reference. **For the same category, you can generate multiple questions.**\\
- **Generate high-quality questions:** You can choose to generate challenging questions, but their answers should be able to clearly explain. \\

Output Format:\\
Your response should strictly follow this format:\\

"questions": [\\
\{\\
"question": "Question text"\\
\}\\
]
}
\end{AIbox} 
\caption{\textbf{An Example of the prompt for Diagram task generation. }}
\label{fig: prompt_info}
% \vspace{-5mm}
\end{figure*}

\begin{figure*}[!ht] 
% \vspace{-5mm}
\begin{AIbox}{Prompt for Diagram tasks}
{You're a great diagram analyst. You need to analyze the diagram provided and generate 2-4 insightful questions based on the content of the diagram.\\
Question Guidelines:\\
- Your questions should be guiding and should not directly point to the content. For example: "How does acid rain affect water bodies, soil, and plant life?" should be changed to "How does the process in image affect water bodies, soil, and plant life?"\\
- Your question should invite insightful discussion, such as:\\
  - **Interpretation**: Symbol Interpretation, Data Extraction\\
    - **Examples**:\\
      - "What is the role of the cytokine-producing cell in the process shown?"\\
      - "Enumerate the steps outlined in the flowchart."\\
  - **Relation Analysis**:\\
    - **Examples**:\\
      - "How does variable A affect variable B in the diagram?"\\
      - "How many ways can A to B be achieved in the diagram?"\\
  - **Inference**:\\
    - **Examples**:\\
      - "What can be inferred about the system's stability from the diagram?"\\
      - "What does the bacterium do once it has the hybrid plasmid?"\\
Output Format:\\
Your response should strictly follow this format:\\
```json\\
\{\\
  "question":["Question text 1", "Question text 2", ...],\\
\}\\
```
}
\end{AIbox} 
\caption{\textbf{An Example of the prompt for Diagram task generation. }}
\label{fig: prompt_diagram}
% \vspace{-5mm}
\end{figure*}