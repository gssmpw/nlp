\section{Background and Motivation}
\label{sec:motivation}
 
In this section, we describe the definition of \brt (\S\ref{sec:movtivation:def}), the availability of \brt in the industrial setting (\S\ref{sec:movtivation:available}), and the need for automated \brt generation (\S\ref{sec:movtivation:need}).

\subsection{\brt Definition}
\label{sec:movtivation:def}

A \brt is a test that exhibits a Fail-to-Pass (\failtopass) behavior: the test fails when executed against a buggy codebase, but passes when executed against the fixed codebase~\cite{mundler2024swt}. 
The implementation of the \brt is useful for understanding and localizing the reproduced bug, aiding the development of an effective fix patch to the bug, while the \failtopass behavior of \brt provides necessary evidence that the proposed fix patch to the reproduced bug is correct~\cite{ko2008debugging,kang2023large,mundler2024swt,yang2024swe,nayrolles2015jcharming,rondon2025passerine,Ernst2014Defects4J,saha2017elixir,koyuncu2019ifixr,le2012systematic}. 
Program repair literature often denotes ``fix patch'' as ``patch''; however, to avoid ambiguity with the test patch in \brt development, we denote \textit{fix patch} for a bug as \textit{fix}.


\subsection{\brt Availability}
\label{sec:movtivation:available}

Prior studies have shown that \brt{}s are scarce at time of bug reporting in open-source projects~\cite{koyuncu2019ifixr, mundler2024swt} (\S\ref{sec:intro}).
In the industrial setting, the common expectation is that \brt{}(s) are developed alongside the fix\Space{, after the bug report is filed}\Space{, transitioning from a failing to a passing state (\failtopass)}\footnote{A \brt could be available as evidence in the bug report from certain \textit{machine-reported} bug categories, e.g., bugs reported by dynamic analyses like AddressSanitizer~\cite{addresssanitizer}.}.
The practice of deferring \brt development to the fix development stage stems from the fact that bug reports originate from sources that often do not readily prioritize the task of, or possess the knowledge of, developing \brt{}s at the time of bug reporting. 

Specifically, typical bug report sources are team-internal developers, team-external developers, and end users. 
Team-internal developers understand the codebase but may not prioritize writing a \brt immediately when encountering an issue, particularly during active development or when dealing with issues surfaced by production monitoring systems. 
Team-external developers understand the system's expected functionality but may not have the expertise to write a bug-targeted test.
End users have limited knowledge of the underlying code, and focus primarily on describing the observed issue\Space{ instead of writing tests}. 
It is also important to note that issues reported by end users are typically not classified as internal bugs in our context.\Comment{\samcheng{why is this note relevant?} \mt{the origin of this note was to highlight the fact that in our study we do not consider user-submitted bug reports, but only internal bugs, i.e., Googlers submitted bugs.}}


Regardless of the bug report source's exposure to the codebase, writing an effective \brt is  challenging and time-consuming.
It requires sufficient understanding of the targeted bug's root cause and code segment, as well as elaborated engineering effort to create a test failure that captures the buggy behaviors\Space{ with existing testing utilities}, all of which could not be immediately carried out as soon as the bug report is filed.
As a result, bug report sources often prioritize to describing the buggy behaviors over developing a \brt.
Manual effort of developing \brt{}s can also substantially delay fix implementation\Space{, especially when dealing with complex or subtle bugs\samcheng{we didnt show our agent generates brt for complex bug}}.




\subsection{Automated \brt Generation}
\label{sec:movtivation:need}

The importance of {\brt}s in software debugging (\S\ref{sec:movtivation:def}) and the challenge in \brt development (\S\ref{sec:movtivation:available}) motivate the need for automated \brt generation.
Automated \brt generation enables developing \brt{}s for filed bug reports \textit{at scale}, which is especially important in an industrial environment where a large number of bug reports are handled per day.
The generated \brt{}s can then be used by \autopr systems and human developers for bug reproduction, root cause analysis, fix development, and fix validation.
We list below the benefits of automated \brt generation from our experience.

\subsubsection{Improving \autopr Effectiveness}
A prior study on \passerine~\cite{rondon2025passerine}, an \autopr system at \google, suggests that the availability of \brt{}s could aid its fix generation process\Space{for machine-reported bugs}. 
Bug reports of machine-reported bugs often contain \brt information\Space{ (in the form of executable test command)} and textual description of the bug.
\passerine frequently runs the provided \brt and leverages the fault localization information in the \brt while generating fixes for these bugs, resulting in a higher fix rate than on human-reported bugs that have no \brt information.
As our experiment on human-reported bugs will show, generated \brt{}s can also substantially improve \autopr effectiveness by providing valuable context of human-reported bugs and  means to evaluate the generated fixes (\S\ref{sec:results}).


\subsubsection{Enhancing Fix Validation}
Generated \brt{}s can help\Space{ as a reliable mechanism to} validate the effectiveness of proposed fixes, increasing confidence that the bug is truly resolved and preventing regressions~\cite{mundler2024swt,kang2023large}.



\subsubsection{Accelerating Manual Bug Resolution}
Generated \brt{}s provide developers with a readily-usable tool for understanding and diagnosing bugs\Comment{\pmr{``Fixing'' doesn't seem relevant here?}\mt{replaced with "diagnosing", which should fit this paragraph better?}}, which can help reduce their manual reproduction and investigation time~\cite{beller2018dichotomy}.
Automated \brt generation is especially helpful when it transforms a production crash or failure into a reproducible unit test~\cite{soltani2018single,nayrolles2015jcharming}.
Developers can then use familiar debugging tools\Space{ and techniques} to step through the test execution, inspect variables, and isolate the root cause more efficiently and conveniently, compared to directly debugging in a complex production environment or from a limited bug report.








