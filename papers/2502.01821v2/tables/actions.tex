\begin{table*}[htbp]
\centering
\begin{adjustbox}{max width=\textwidth}
\begin{tabular}{p{0.17\textwidth} p{0.83\textwidth}}
\toprule
\textbf{Action} & \textbf{Description} \\ \midrule
\CodeIn{cat [path]} & Displays the content of the file at the specified path. This allows the agent to inspect code files. \\ \midrule
\CodeIn{code\_search [text]} & Searches \google's internal code repositories for code snippets matching the given query. This enables the agent to find relevant code examples or identify potential locations of the bug. \\ \midrule
\CodeIn{edit [path] [prompt]} & \Space{A specialized command that t}Triggers the code editing process\Space{, as detailed in step 4 below}. It is designed to prompt an \llm fine-tuned for editing in \google's codebase. To avoid introducing errors to non-test code, we restrict \texttt{edit} to only apply to test files identified based on regex\Space{\CodeIn{*[tT]est\..*}}. \\ \midrule
\CodeIn{bazel test [path]} & Executes the \CodeIn{bazel test} command on a specified target via \google's internal testing framework. This allows agent to run tests and observe the results. We also de-duplicate test output if the test log is too long for \llm input context. \\ \midrule
\CodeIn{finish} & Signals the agent's belief that it has either successfully generated a \brt or that it cannot create a \brt. The agent is restricted from using \CodeIn{finish} before running any tests. \\
\bottomrule
\end{tabular}
\end{adjustbox}
\caption{\tool actions.\label{tab:actions}}
\vspace{-10pt}
\end{table*}


