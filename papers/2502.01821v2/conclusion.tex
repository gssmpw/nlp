\section{Conclusion}
\label{sec:conclusion}

This paper investigates the potential for automated \brtfull (\brt) generation within a large-scale industrial setting, specifically at \google. 
We adapt and evaluate an existing \llm-based approach for \brt generation, specifically \libro \cite{kang2023large}, to function within \google's complex internal development environment.
We also introduce our \llm-agent-based \brt generation approach, designed to work with \google's development environment. 
\tool, our \brt generation technique that is built on top of a \gemini model fine-tuned on \google's codebase, significantly outperforms the adapted \libro built on top of the same \gemini model, demonstrating the efficacy of agent-based system for effective \brt generation.


Our empirical results show that \tool can generate plausible \brt{}s for more bugs (28\% compared to 10\% by \libro). 
These generated plausible \brt{}s can improve the performance of \passerine, \google's industrial-scale \autopr system, in generating plausible fixes for more bugs more efficiently. 
Finally, we show that candidate \brt{}s with \failtoany behavior can also be used to effectively rank and select plausible fixes generated by \autopr leveraging our proposed \enpassratefull metric.

These findings underscore the practical value of \brt{}s and highlight the importance of developing robust, industry-aware \brt generation techniques. 
We hope this research contributes to bridging the gap between academic research and industrial needs, offering a practical solution for improving software quality and accelerating bug resolution in large-scale, complex software ecosystems.
