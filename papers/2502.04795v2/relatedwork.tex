\section{Related Work}
\subsection{Critical Period for Language Acquisition}
\label{subsec:cp}

The CPH posits that language acquisition is most efficient within a specific developmental window, after which it declines. 
CP effects are observed in both \lone and \ltwo acquisition, suggesting a shared underlying mechanism. 


\paragraph{Critical Period for \lone Acquisition}
Research in neurolinguistics and cognitive science suggests that there is a biologically determined CP for acquiring an \lone, beyond which full native-like proficiency is unattainable if exposure to language is delayed. 
Studies on late \lone learners, such as deaf individuals who acquire sign language after early childhood, indicate severe deficits in grammatical proficiency compared to those exposed to language from birth~\cite{mayberry1989looking, NEWPORT199011}. 
These findings suggest that neural plasticity, essential for \lone acquisition, diminishes with age, limiting the ability to develop full linguistic competence.
From a theoretical perspective, the existence of the CP for \lone acquisition is often attributed to biological constraints. 
Nativist theories propose that \lone acquisition relies on an innate language faculty that operates most effectively during the CP~\cite{penfield1965conditioning,chomsky1965,pinker1994language}. 
On the other hand, empiricist perspectives argue that the decline in \lone learning ability may result from environmental factors, such as a reduced need for language learning mechanisms once fundamental linguistic structures have been internalized~\cite{elman1996rethinking,seidenberg2006connectionist}. 
Despite extensive research, the precise boundary and mechanisms of the CP for \lone remain a subject of debate.




\paragraph{Critical Period for \ltwo Acquisition}
CP effects are also observed in \ltwo acquisition, where late learners struggle with pronunciation, morphology, and syntax~\cite{JOHNSON198960,hartshorne2018critical}. 
While biological constraints play a role, entrenchment—where prior exposure to \lone limits flexibility in learning new linguistic structures—is also a factor~\cite{ellis2000age, seidenberg2006connectionist}. 
Although the CP for \ltwo acquisition is an important topic, this study focuses on the CP for \lone acquisition, since our goal is to design data-efficient LMs by exploring the mechanisms of CP in \lone acquisition.






\subsection{The Role of Language Models in Acquisition Theories}


In recent years, computational models have played a crucial role in elucidating the mechanisms of language acquisition. 
These models enable controlled investigations of learning mechanisms and environments, which are difficult to achieve with human participants, and they are used to test theoretical claims such as the ``poverty of the stimulus''~\cite{Clark-etal-2011}. 
For instance, \citet{McCoy-etal-2020}, \citet{Wilcox-etal-2024}, and~\citet{warstadt-etal-2023-findings} have employed LMs to directly test hypotheses about language acquisition, demonstrating that such models can provide proof-of-concept evidence for \textit{learnability}. 
These studies have attracted attention as efforts to deepen theoretical discussions on language acquisition through computational modeling, including research on the CP.


\citet{Constantinescu-tacl2025} investigated CP phenomena in \ltwo acquisition and \lone attrition,\footnote{The phenomenon in which earlier cessation of \lone exposure increases the likelihood of \lone forgetting.} assuming a shared underlying mechanism for CP effects across \lone and \ltwo. 
They simulated \ltwo exposure at varying ages to examine how LMs differ from human learners, finding that LMs do not naturally exhibit CP effects. 
To artificially induce such effects, they employed Elastic Weight Consolidation~\cite{Kirkpatrick-etal-2017}, a regularization method for mitigating catastrophic forgetting, thereby mimicking a maturational decline in plasticity. 
Their findings suggest that CP effects are not an inevitable outcome of statistical learning but may instead involve innate mechanisms.


While this study shares the broader objective of enhancing the cognitive plausibility of LMs as models of human language acquisition, it differs from \citet{Constantinescu-tacl2025} in both \textit{focus} and \textit{methodology}. 
Rather than modeling CP effects through dataset manipulation or post-CP plasticity constraints, this study explicitly addresses the \textbf{developmental processes unfolding during the CP itself}. 
Specifically, we integrate a mechanism to simulate the progressive growth of working memory capacity throughout the CP, a factor considered crucial for \lone acquisition but previously unmodeled in LM-based research. 
By incorporating developmental constraints, this study aims to provide a more fine-grained computational model of early \lone acquisition and its cognitive underpinnings, advancing the developmental plausibility of LMs.