\documentclass[10pt,journal,compsoc]{IEEEtran}
\IEEEoverridecommandlockouts

% *** CITATION PACKAGES ***
%
\ifCLASSOPTIONcompsoc
  % The IEEE Computer Society needs nocompress option
  % requires cite.sty v4.0 or later (November 2003)
  \usepackage[nocompress]{cite}
\else
  % normal IEEE
  \usepackage{cite}
\fi
\usepackage{amsmath,amssymb,amsfonts}
\usepackage{algorithmic}
\usepackage{graphicx}
\usepackage{textcomp}
\usepackage[absolute]{textpos}


\usepackage{enumitem}
\usepackage{color}
\usepackage{multirow}
\usepackage{lipsum}
\usepackage{listings}
% \usepackage{subfigure}
\usepackage{graphicx}
\usepackage{pdfpages}
\usepackage{verbatim}
\usepackage{pifont}
\usepackage{etoolbox}
\usepackage{textcomp}
% \usepackage{ulem}
\usepackage{hyperref}
\usepackage{url}
\usepackage{caption}
\usepackage{booktabs} % For nice tables
\usepackage{siunitx} % To align table numbers by unit
\usepackage{etoolbox}

% \usepackage{booktabs}
% \usepackage{multirow}
% \usepackage{graphicx}
\usepackage[normalem]{ulem}
\useunder{\uline}{\ul}{}

\usepackage[switch]{lineno} % default option is 'left'

\usepackage[linesnumbered,titlenumbered,ruled,vlined,resetcount,algosection]{algorithm2e}
\usepackage{xcolor}
\usepackage{color}
\def\BibTeX{{\rm B\kern-.05em{\sc i\kern-.025em b}\kern-.08em
    T\kern-.1667em\lower.7ex\hbox{E}\kern-.125emX}}

\usepackage{amsthm}
\newtheorem{theorem}{Theorem}
%\newtheorem{theorem}{Theorem}[section]
%\newtheorem{corollary}{Corollary}[theorem]
\newtheorem{corollary}{Corollary}
%\newtheorem{lemma}[theorem]{Lemma}
\newtheorem{lemma}{Lemma}

\usepackage{hhline}
\usepackage{xcolor}
\usepackage{colortbl}
\usepackage{threeparttable}
% \usepackage{minted}
\usepackage{subfig}
\usepackage{comment}
\usepackage{enumitem}
% \usepackage[ruled,vlined]{algorithm2e}
% Circled number
\usepackage{tikz}
\newcommand*\circled[1]{\tikz[baseline=(char.base)]{
            \node[shape=circle,fill,inner sep=0.8pt] (char) {\footnotesize\textcolor{white}{#1}};}}
\newcommand*\redcircled[1]{\tikz[baseline=(char.base)]{
            \node[shape=circle,fill=red,inner sep=1.3pt] (char) {\textcolor{white}{#1}};}}
            
\def\BibTeX{{\rm B\kern-.05em{\sc i\kern-.025em b}\kern-.08em
    T\kern-.1667em\lower.7ex\hbox{E}\kern-.125emX}}

% \setlength{\floatsep}{1em}
% \setlength{\textfloatsep}{0.8em}
% \setlength{\intextsep}{0.1em}
% \setlength{\belowcaptionskip}{0.2em} 

\newcommand{\yguo}[1]{
  \begin{framed}
    \noindent{\textcolor{blue}{\bf\em #1 -- Yanfei Guo}}
  \end{framed}
}

\newcommand{\yujia}[1]{\textcolor{red}{#1}} % Track changes from Yujia

\newcommand{\fn}[1]{{\tt\small #1}}
%%

% \newcommand{\pname}{\textit{gZCCL}}
\newcommand{\pname}{\mbox{ZCCL}}
% \newcommand{\pname}{\mbox{\textbf{\texttt{ZCCL}}}}
\newcommand{\newtext}[1]{\textcolor{blue}{#1}}

\newcommand{\fzlight}{\mbox{\textit{fZ}-light}}


\begin{document}

\Urlmuskip=0mu plus 1mu


\title{ZCCL: Significantly Improving Collective Communication With Error-Bounded Lossy Compression}

\author{Jiajun Huang, Sheng Di,~\IEEEmembership{Senior Member,~IEEE}, Xiaodong Yu, Yujia Zhai, Zhaorui Zhang, Jinyang Liu, Xiaoyi Lu, Ken Raffenetti, Hui Zhou, Kai Zhao, Khalid Alharthi, Zizhong Chen,~\IEEEmembership{Senior Member,~IEEE}, Franck Cappello,~\IEEEmembership{Fellow,~IEEE}, Yanfei Guo, Rajeev Thakur,~\IEEEmembership{Fellow,~IEEE}
% Xiaodong Yu,\IEEEauthorrefmark{3}
% Yujia Zhai,\IEEEauthorrefmark{1}
% Zhaorui Zhang,\IEEEauthorrefmark{4}
% Jinyang Liu,\IEEEauthorrefmark{1}
% Xiaoyi Lu,\IEEEauthorrefmark{5}
% Ken Raffenetti\IEEEauthorrefmark{2}
% Hui Zhou\IEEEauthorrefmark{2}
% Kai Zhao\IEEEauthorrefmark{6}
% Zizhong Chen\IEEEauthorrefmark{1}
% Franck Cappello\IEEEauthorrefmark{2}
% Yanfei Guo\IEEEauthorrefmark{2}
% Rajeev Thakur\IEEEauthorrefmark{2}

\IEEEcompsocitemizethanks{
\IEEEcompsocthanksitem Jiajun Huang, Yujia Zhai, and Zizhong Chen are affiliated with the University of California, Riverside, CA 92521. Sheng Di, Ken Raffenetti, Hui Zhou, Franck Cappello, Yanfei Guo, and Rajeev Thakur are affiliated with Argonne National Laboratory, Lemont, IL 60439. Xiaodong Yu is affiliated with Stevens Institute of Technology, Hoboken, NJ 07030. Zhaorui Zhang is affiliated with The Hong Kong Polytechnic University, Kowloon, Hong Kong. Jinyang Liu is affiliated with University of Houston,
Houston, TX 77204. Xiaoyo Lu is is affiliated with University of California, Merced, CA 95343. Kai Zhao is affiliated with Florida State University, Tallahassee, FL 32306. Khalid Alharthi is affiliated with Department Of Computer Science, College of Computing And Information Technology, University Of Bisha, Bisha 61922, P.O. Box 551, Saudi Arabia. 
% \IEEEcompsocthanksitem A shortened version of this paper was presented at 2024 IEEE International Parallel
% and Distributed Processing Symposium (IPDPS) and was published in its \href{https://doi.ieeecomputersociety.org/10.1109/IPDPS57955.2024.00072}{proceedings}.
}

}

% The paper headers
% \markboth{Journal of \LaTeX\ Class Files,~Vol.~14, No.~8, August~2021}%
% {Shell \MakeLowercase{\textit{et al.}}: A Sample Article Using IEEEtran.cls for IEEE Journals}

% \IEEEpubid{0000--0000/00\$00.00~\copyright~2021 IEEE}

% for Computer Society papers, we must declare the abstract and index terms
% PRIOR to the title within the \IEEEtitleabstractindextext IEEEtran
% command as these need to go into the title area created by \maketitle.
% As a general rule, do not put math, special symbols or citations
% in the abstract or keywords.
\IEEEtitleabstractindextext{%

\begin{abstract}



With the ever-increasing computing power of supercomputers and the growing scale of scientific applications, the efficiency of MPI collective communication turns out to be a critical bottleneck in large-scale distributed and parallel processing. The large message size in MPI collectives is particularly concerning because it can significantly degrade overall parallel performance. To address this issue, prior research simply applies off-the-shelf fixed-rate lossy compressors in the MPI collectives, leading to suboptimal performance, limited generalizability, and unbounded errors. In this paper, we propose a novel solution, called {\pname}, which leverages error-bounded lossy compression to significantly reduce the message size, resulting in a substantial reduction in communication costs. The key contributions are three-fold. (1) We develop two general, optimized lossy-compression-based frameworks for both types of MPI collectives (collective data movement as well as collective computation), based on their particular characteristics. Our framework not only reduces communication costs but also preserves data accuracy. (2) We customize {\fzlight}, an ultra-fast error-bounded lossy compressor, to meet the specific needs of collective communication. (3) We integrate {\pname} into multiple collectives, such as Allgather, Allreduce, Scatter, and Broadcast, and perform a comprehensive evaluation based on real-world scientific application datasets. Experiments show that our solution outperforms the original MPI collectives as well as multiple baselines by 1.9--8.9$\times$.

\end{abstract}

\begin{IEEEkeywords}
Error-bounded Lossy Compression, Collective Communication, Distributed Computing, Parallel Algorithm
\end{IEEEkeywords}
}
\maketitle

%%
%% This command processes the author and affiliation and title
%% information and builds the first part of the formatted document.
\maketitle

\documentclass[../main.tex]{subfiles}
\graphicspath{{../images/}}
\makeatletter
\def\input@path{{../images/}}
\makeatother
\begin{document}
\section{Introduction}
\begin{figure}
\centering
\begin{tikzpicture}
\node[inner sep=0pt] (ws) at (0, 0) {
\includegraphics[height=.4\textwidth, trim={10cm 0 10cm 0},clip]{world_space.png}};
\node[inner sep=0pt] (cs) at (6,0) {\includegraphics[height=.4\textwidth, trim={10cm 1cm 10cm 4cm},clip]{conf_space.png}};
\end{tikzpicture}
\vspace{-5pt}
\label{fig:pbrm_intro}
\caption{\textbf{Left}: Shows world space obstacles as grey spheres. Robots start and goal configuration is colored red and green, respectively. Configurations along the computed path are colored transparent blue. \textbf{Right:} Mapped world space scenario to configuration space. Obstacle region is the grey mesh. Red spheres are collision-free regions computed by the neural SCDF. The optimized shortest path in the convex corridor is the blue curve.}
\vspace{-25pt}
\end{figure}
Motion planning is the problem of finding a collision-free trajectory that connects a given start and goal configuration. The planning takes place in the configuration space of the robot. For single body robots, like mobile robots or drones, the configuration space and the world space are usually the same. This simplifies the planning, since explicit obstacle representations are available which enables geometrical tools like separating hyperplanes, smallest distance to obstacles etc., to be used when designing motion planning algorithms. For multi-body robots like manipulators, the situation is completely different. The world space obstacles are usually mapped to non-convex regions, and to make the problem even harder, the mapping is usually not known. Forming explicit representations of the obstacle region in the configuration space is usually too expensive or intractable. Despite all of this, sampling based planners are used with great success, which mainly is due to their use of implicit representations of the obstacle region. The basic idea is to construct a graph in the configuration space that covers and connects the collision-free region. From this graph, a path can be extracted that connects a given start and goal configuration. The approach is computationally expensive, since the graph is constructed with the smallest geometrical building block available, points, which represents a collision-check. Furthermore, the extracted paths from the graph are non-smooth and jagged due to the stochastic nature of the approach. This adds an additional post-processing step to the process, where the paths are shortcutted and smoothened, before the path can be used for tracking. Clearly a lot of time is invested to form this graph and produce smooth paths. Thus, if the obstacles start to move, then all of this work is done in no use, since all points that make up this graph need to be re-verified, which is simply too time consuming to be done in real time.
\\\\
In this work, we want to address the existing drawbacks of the sampling based planners. Our main contribution is an improved motion planner where each vertex in the graph covers a collision-free region in the form of a sphere instead of a point and where the edges are formed with neighboring intersecting spheres. This representation has the advantage of instead of returning piecewise linear paths, returning a sequence of overlapping spheres, i.e. a convex corridor, that connects a given start and goal configuration, illustrated in Figure \ref{fig:pbrm_intro}. This convex corridor allows us to use convex optimization to produce smooth trajectories, instead of computationally expensive post-processing methods. The representation further allows us to estimate the coverage of the collision-free space, which gives us awareness and feedback in the offline roadmap construction phase. Finally, our representation is simple to adapt to moving obstacles, simply requery for the new radii and recheck for intersections. 
\\\\
The spherical collision-free regions are formed using a signed distance function (SDF), which is a function that returns the smallest distance from an arbitrary point to the boundary of an obstacle. As the name implies, the distance is signed, thus if the point is inside the obstacle it is negative otherwise positive. If the distance is positive, a sphere with radius equal to the distance is guaranteed to cover a collision-free region. Using an SDF in motion planning is not new, but what is novel about our approach is that we express the distance in the configuration space instead of the world space and by doing so allows us to form these convex collision-free regions. We refer to the resulting SDF as a signed configuration distance function (SCDF). Computing an SCDF analytically is non-trivial, our approach is therefore to parameterize the SCDF with a deep neural network and learn the mapping by supervised learning. Our resulting neural SCDF can compute distances for different parameter values of obstacle shapes and we also show how multiple distances can be combined, thus making our approach flexible.
\section{Related work}
Motion planning algorithms can roughly be divided into three families, grid-based, sampling based and optimization based methods. Grid-based methods (GBM) discretize the planning space from which a graph is then compiled. A standard search method is A$^\star$ \citep{a_star}, which is classified as an \textit{informed} search method, since it employs a heuristic function to speed up the search. A$^\star$ guarantees to return an optimal path at the level of discretization used. GBMs usually discretize the planning space by a regular lattice and this limits the GBMs to problems with low dimensionality due to the curse of dimensionality. Thus, GBMs are usually limited to single-body robots where the degrees of freedom (DOF) are low. To overcome the inherent scaling problem with the GBMs, stochastic methods are usually used for multi-body robots. These methods are termed as sampling-based methods (SBM) and core members within this family are the rapidly-exploring random trees (RRT) \citep{rrt} and the probabilistic roadmap (PRM) \citep{prm}. RRT grows a tree from the start configuration and explores the collision-free region in a rapid way until it is able to connect to the goal region. RRT is usually improved by bi-directional planning \citep{rrt_connect}, i.e. an additional tree is grown from the goal configuration and the trees are tested for connection after any tree has been expanded. RRT is a single-query method, thus it searches for a path from scratch each time it is queried. Contrary to this, PRM is a multi-query method, which solves for multiple queries without starting from scratch. PRM does this by creating a roadmap (graph) that covers the collision-free space as an offline step. The graph is then used to solve for multiple queries. PRMs are used in cases where the environment does not change since the extra offline step is too computationally costly and needs to be re-done if the environment is changed. In our work, we address this inherent issue by using a different roadmap representation. Our vertices in the graph cover a collision-free region in the form of spheres and we form the edges by checking for intersecting spheres. If something in the environment changes, we recompute the spheres radii and recheck the intersections, without relying on collision detection. We use a trained neural network to compute the sphere radius, therefore querying for the radius can be done fast, hence our representation enables the PRM for dynamic environments.
\\\\
In the recent decades, optimization based methods (OBM) \citep{chomp, schulman, itomp, stomp} have been introduced as an alternative to SBM for multi-body robots. Like the SBM, the OBMs scale well to higher dimensional problems and produce smoother motion. It is common to use a SDF in the optimization since it is a smooth function, thus enabling gradient-based methods. However, the standard way of expressing the SDF is in world space. The distance therefore needs to be mapped to the configuration space by the forward kinematics. This mapping makes the optimization problem a non-linear program (NLP), which is computationally expensive to solve. Recently, a different approach has been proposed. In \cite{mp_gcs} motion planning is formulated as a convex optimization problem by using the graph of convex sets framework \citep{gcs}. The underlying idea is to decompose the collision-free space into intersecting convex sets from which a convex optimization problem is formulated. In cases where an explicit representation of the obstacles in the configuration space exists, like for single-body robots, creating collision-free convex regions can be done fast \citep{iris}. For multi-body robots, this is non-trivial. Existing work does this successfully \citep{iris_nlp, iris_c} by an optimization based approach, but the methods are still too time consuming to be used in the presence of moving obstacles. Our approach is instead to use deep learning to learn an SDF expressed in the configuration space. With this, we can query for shortest distances to the collision boundary, which allows us to expand spherical regions which are collision-free. Our approach is fast and therefore enables our suggested roadmap planner to be used in dynamic environments.
\\\\
Recent research has focused on learning collision detection \citep{fk_kernel_distance, diffco, graphdistnet} by predicting the signed distance between the robot links and the surrounding obstacles in the world space. The learned SDF is used in trajectory optimization but since the distance is expressed in the world space, the problem becomes an NLP and therefore takes a long time to solve. We take a novel approach and suggest to instead express the signed distance in the configuration space. This allows us to improve the PRM at the same time as it enables convex optimization for trajectory optimization, which runs faster and is more reliable than NLP solvers. In \cite{cspf} a learned signed distance function in the configuration space is proposed similar to our approach. However, their approach is restricted to point cloud representations, while we propose to represent the obstacles as parameterized geometric shapes, e.g. spheres. Furthermore, we also show how to use our learned SCDF to improve an existing roadmap planner.
\section{Problem formulation}
A robot is located in the world space, $\W \subset \R^3 $. The unique location of the robot is given by its configuration $\q \in \C$, where $\C$ is the configuration space. The set of points covered by the robots bodies at a certain configuration is expressed as $\B(\q) \subset \W$. The robot is surrounded by $\NrObst$ obstacles $\O = \bigcup_{i=1}^{\NrObst} \O_i$, where  $\O_i \subset \W$. The representation of the obstacle in the configuration space is the set $\C\O_i = \{\q \in \C \: |\: \B(\q) \cap \O_i \neq \emptyset \}$. The obstacle space is formed as $\Co = \bigcup_{i=1}^{\NrObst} \C \O_i$. The complement is referred to as the free space, $\Cf = \C \setminus \Co$. The path planning problem is a tuple, ($\Cf$, $\qStart$, $\qGoal$), where we want to connect a query pair, consisting of a start, $\qStart$, and goal configuration, $\qGoal$, with a geometric path, $\q(s): [0, 1] \mapsto \Cf$, such that $\q(0)=\qStart$ and $\q(1)=\qGoal$, or report correctly when such a path does not exist.
\end{document}

% \section{Motivation}
\label{sec:motivation}



% In LLM inference, not only does weight matter, but the memory requirements of the KV Cache are also considerable.
In this section, we first demonstrate that the emerging paradigm of group quantization demands a high level of adaptivity, which current adaptive methods lack.
We then discuss how adapting these methods to group quantization could compromise their efficiency.
Given that LLMs generate KV caches during runtime, real-time quantization capability is crucial.
These challenges lead to our proposal of a mathematical adaptive numerical type (\texttt{MANT}), which we will detail later.



\begin{figure}[t]
    \centering
    \begin{minipage}[t]{0.48\columnwidth}
      \centering
      \includegraphics[width=\columnwidth]{fig/moti_group_ppl.pdf}
      \caption{LLM accuracy with different quantization granularities. We report the perplexity (PPL) metric (lower is better).}\label{fig:moti_group_ppl} 
    \end{minipage}
    \hspace{2pt}
    \begin{minipage}[t]{0.48\columnwidth}
      \centering
      \includegraphics[width=\columnwidth]{fig/motivation_adaptive_ppl.pdf}
      \caption{Accuracy loss for \texttt{INT}, \texttt{ANT}, and Ideal (clustering algorithm K-Means) adaptive methods in group quantization. }\label{fig:moti_ppl} 
    \end{minipage}
    % \vspace*{-0.3cm}
\end{figure}




\subsection{Group Quantization Accuracy Analysis}
\label{sec:acc_analysis}

In this subsection, we begin by comparing the accuracy of traditional channel-wise quantization with group-wise quantization~\cite{shao2024omniquant,zhao2023atom,liu2024kivi,sheng2023flexgen,lin2023awq,zhao2023atom}, establishing the baseline for group-wise quantization in this study.
We then delve into the use of various adaptive data types in group quantization, emphasizing the necessity for full adaptivity.



\Fig{fig:moti_group_ppl} illustrates the perplexity when quantizing the LLaMA-7B model~\cite{touvron2023llama} with various granularities using the \texttt{INT4}-based symmetric quantization.
Channel-wise quantization significantly worsens the perplexity of the examined LLM, increasing it from 5.68 to 6.85.
Conversely, group-wise quantization mitigates this loss in perplexity with a group size of 128, corresponding to an average of 4.125 bits per element (16-bit scaling factor).
Additionally, we observe that a smaller group size of 32 offers only a slight improvement in perplexity, but the scaling factor overhead increases by $4\times$.



Given this analysis, we adopt a group size of 128 as our standard configuration for the remainder of this section.
Previous research indicates that the \texttt{INT} data type is not optimal for accuracy since tensors or channels exhibit varied distributions, leading to the proposal of various adaptive data types~\cite{guo2022ant, guo2023olive, zadeh2020gobo, zadeh2022mokey}.
We evaluate their efficacy in the context of group quantization, which falls into two main categories: data-type-based and clustering-based.



\textbf{Data-type-based adaptive methods} select data types from discrete sets based on tensor data distribution.
ANT~\cite{guo2022ant} is a representative example of the data-type-based method.
ANT packages several different data types for selection, including \texttt{INT} for the uniform distribution, \texttt{PoT} (Power of Two) for the Laplace distribution, and \texttt{flint} for the Gaussian distribution.
%ANT designed \texttt{flint} for Gaussian distributions.

\textbf{Clustering-based adaptive methods} utilize clustering algorithms to generate centroids that align with the data distribution and provide considerable adaptivity. 
Mokey~\cite{zadeh2022mokey} and GOBO~\cite{zadeh2020gobo} exemplify this approach, though they focus on tensor- or channel-wise quantization. In our study, we adapt them to group quantization through per-group clustering.

%Clustering-based methods employ clustering algorithms to generate centroids that fit the data distribution, demonstrating sufficient adaptivity.
%Mokey~\cite{zadeh2022mokey} and GOBO~\cite{zadeh2020gobo} are such presentative works, but only target tensor- or channel-wise quantization.
%In our work, we modify those works to support group quantization by performing per-group clustering.
\Fig{fig:moti_ppl} compares the accuracy of the methods described above for the LLaMA-7B model under 4-bit group-wise quantization. 
The group-wise \texttt{ANT} method outperforms the \texttt{INT} type by dynamically selecting from three data types to better match the value distribution, resulting in reduced perplexity (PPL) loss. 
Moreover, per-group clustering adjusts more effectively to the value distribution of each group, establishing itself as the accuracy-optimal and ideal adaptive method. 
This approach achieves nearly lossless 4-bit quantization, equivalent to 16 centroids per group. 
However, this ideal scenario is impractical due to the significant overhead associated with storing per-group centroids, effectively rendering it a 6-bit quantization.

\begin{figure}[t] 
    \centering 
    \includegraphics[width=1.0\linewidth]{fig/intro_cdf.pdf}  
    \caption{The cumulative distribution function (CDF) of the tensor, channel, and group, respectively. The tensor data were taken from layers 8 to 23, while the 16 channel and group data were sampled from one tensor with specific strides.}\label{fig:moti_dist} 
    %  \vspace*{-0.3cm}
\end{figure}

To illustrate the group-wise diversity in data distribution, we sampled the weights of the Q and V tensors in LLaMA-7B model. 
We normalized all sampled data to their absolute maximum values, which ranged from -1 to 1. \Fig{fig:moti_dist} displays the cumulative distribution function (CDF) for the tensor, channel, and group levels, respectively. 
We observed that the diversity at the group level is significantly higher than at the tensor level. 
In simpler terms, while different tensors exhibit similar distributions, groups can have markedly different distributions. This finding underscores the necessity for full adaptivity in group quantization to fully realize its potential.
\paragraph{Takeaway 1.} The group quantization is an emerging paradigm to accelerate LLMs, and the significant group-level diversity requires a high level of adaptivity to fully unleash its potential.

\subsection{Group Quantization Efficiency Analysis}
\label{subsec:efficiency}


In this subsection, we provide a detailed efficiency analysis for the above adaptive quantization methods.
In \Tbl{intro:dtype}, we compare OliVe~\cite{guo2023olive}, ANT~\cite{guo2022ant}, GOBO~\cite{zadeh2020gobo}, and Mokey~\cite{zadeh2022mokey} with \texttt{INT} regarding the efficiency of computation, encoding, and decoding. 
In this paper, we use the term encoding (decoding) interchangeably with quantization (dequantization).
 

Data-type-based adaptive methods such as ANT~\cite{guo2022ant} and Olive~\cite{guo2023olive} achieve computational efficiency comparable to \texttt{INT}. 
Both utilize specialized decoders that decode these data types prior to computation, resulting in high decoding efficiency. 
However, as previously demonstrated, these methods suffer from limited adaptivity in the group quantization paradigm. 
A straightforward approach to enhance adaptivity is to expand their set of data types. 
However, incorporating new data types necessitates additional decoders, escalating hardware design costs. 
Additionally, compatibility issues between new and existing data types may reduce computational efficiency. 
For instance, the \texttt{NF4} data type~\cite{dettmers2023qlora} requires an FP16 MAC unit, which is incompatible with existing \texttt{ANT} data types.


\paragraph{Takeaway 2.} Enhancing the data-type-based adaptive method for group quantization is challenging and requires a careful balance for the computation and decoding efficiency.

Clustering-based adaptive methods like GOBO~\cite{zadeh2020gobo} and Mokey~\cite{zadeh2022mokey} can sufficiently adapt to various distributions at the group level. 
However, they require codebooks for quantization and dequantization, leading to high adaptivity at the expense of encoding and computational efficiency. 
For instance, a 16-entry codebook with 8 bits per entry requires 128 bits per group, creating an inevitable trade-off between adaptivity and memory overhead. GOBO~\cite{zadeh2020gobo} employs the K-means algorithm to quantize weights and requires dequantization to \texttt{FP16} using a codebook lookup table before computation, resulting in high adaptivity but low computational efficiency. 
Conversely, Mokey~\cite{zadeh2022mokey} enhances the computation of clustering-based methods by using indices for centroid values via approximate calculations, though matrix multiplication still relies on floating-point units, increasing overhead compared to integer units. 
Furthermore, Mokey creates one \texttt{golden dictionary} for all activations and weights, akin to using a single data type in quantization, thus reducing adaptivity.


\paragraph{Takeaway 3.} Deploying the clustering-based adaptive methods under group quantization is challenging owing to the low encoding and computation efficiency. 


\begin{table}[t]
    \centering
    \small
    \renewcommand{\arraystretch}{1.2}
    \caption[]{Features of DNN accelerators with adaptive and flexible data types are summarized. Here, `Effi.' stands for efficiency, `Med.' for medium, and `LUT' for lookup table.}
  
    \resizebox{1.0\columnwidth}{!}{
      \begin{tabular}{c|cc|ccc|cc|c}
        \Xhline{1.2pt}
        \multirow{2}{*}{Architecture} & \multicolumn{2}{c|}{Encode} & \multicolumn{3}{c|}{Computation} & \multicolumn{2}{c|}{Decode} & \multirow{2}{*}{Adaptivity} \\ \cline{2-8}
        & Method & Effi. & Method & Bit & Effi. & Method & Effi. \\
        \Xhline{1.2pt}
        \texttt{INT} & Round & High & INT & 4 \& 8 & High & Calculation & High & Low \\ 
        OliVe~\cite{guo2023olive} & Search & Med. & INT & 4 \& 8 & High & Decoder & High & Med. \\ 
        ANT~\cite{guo2022ant} & Search & Med. & INT & 4 \& 8 & High & Decoder & High & Med. \\ 
        Mokey~\cite{zadeh2022mokey} & Cluster & Med. & Float & 4 \& 8 & Med. & Calculation & Med. & Low \\ 
        GOBO~\cite{zadeh2020gobo} & Cluster & Low & Float & 16 & Low & LUT & Med. & High \\ 
        \hline
        \multirow{2}{*}{\proj}  & Search  & Med.  & \multirow{2}{*}{INT} & \multirow{2}{*}{4 \& 8} & \multirow{2}{*}{High} & \multirow{2}{*}{Calculation} & \multirow{2}{*}{High} & \multirow{2}{*}{High} \\ \cline{2-3}
        &  Map &  High &  &&&\\ 
        \Xhline{1.2pt}
    \end{tabular}
    }
    \vspace*{0.1cm}
    \label{intro:dtype}
    \vspace*{-0.2cm}
  \end{table}

\subsection{Support for Real-time Quantization}
\label{sec:moti_kvcache}

The above group-wise diversity presents a challenge for both weights and KV cache.
In addition, KV cache faces challenges in real-time group-wise quantization because the KV cache is generated dynamically during LLM inference.


To facilitate low-precision computation in group-wise quantization, it is necessary to quantize K and V along the inner dimension. 
This requirement stems from the support for matrix inner product operations in most GPUs and TPUs. 
During these operations, the group-wise scaling factor can be extracted from the multiply-accumulate process. 
\Fig{fig:kv_process} depicts the computation process of K and V during the decode stage. We define the dimension used for matrix inner product operations as the inner dimension. 
The inner dimensions of the K and V caches differ; the K cache requires a transpose operation, whereas the V cache does not, complicating the situation.


In the prefill stage, K and V can easily compute the scaling factor for each group. 
During the decode stage, the newly generated K vector is concatenated along the inner dimension of the K cache, enabling immediate quantization. 
However, the newly generated V vector is associated with different groups, with only one element per group produced per iteration. This process prevents the scaling factor for the entire group from being obtained in a single iteration, posing a significant challenge for the real-time quantization of the V cache.


\begin{figure}[t] 
  \centering 
  % \includegraphics[width=1.0\linewidth]{fig/dse_kv_process.pdf}  
  \includegraphics[width=0.9\linewidth]{fig/moti_kv_dimension.pdf}  
  \caption{\small Comparison of group-wise K and V cache quantization. They have different inner dimensions due to the transposition of K (key).}

  \label{fig:kv_process}
  % \vspace*{-0.4cm}
\end{figure}


Given those challenges, we propose \proj with a mathematical encoding format that can fuse with integer computation and enhance the decoding efficiency.
In addition, this encoding format provides sufficient adaptivity for group-wise quantization.
Regarding the challenge in KV cache, \proj employs a real-time quantization engine that ensures efficient encoding and decoding for KV cache.
By addressing these challenges, \proj enables efficient low-bit group-wise quantization.



\section{Basic Background: Supervised Learning and the PAC Model}
\label{sec:background}

At this point almost everyone has heard of machine learning (ML). Anyone likely to stumble upon this article will have also heard of its most influential special case, supervised learning, and those theoretically inclined will also be familiar with the PAC model. Nonetheless, I will set the stage by  recapping the basics.

\subsection{Basics of Supervised Learning}%Let's set the stage in any case

\emph{Supervised Learning} is the task of ``coming up'' with a function $f: \X \to \Y$ to ``explain'' or ``fit'' a sequence of input/output examples   $(x_1,y_1), \ldots, (x_n,y_n)$, with $x_i \in \X$ and $y_i \in \Y$.  Here $\X$ is a \emph{data domain} consisting of \emph{datapoints} $x \in \X$, $\Y$ is a \emph{label set} consisting of \emph{labels} $y \in \Y$, and the sequence $(x_1,y_1),\ldots,(x_n,y_n)$ is the \emph{training data} consisting of \emph{labeled examples (a.k.a. samples)}~$(x_i,y_i)$.  I~will refer to the chosen function $f$ as a \emph{predictor}, and to $n$ as the \emph{sample size}. A \emph{learning algorithm} takes as input training data, and outputs (some representation of) a predictor $f \in \Y^\X$.\footnote{Note that this describes the usual \emph{batch}, a.k.a.~\emph{offline}, setting of supervised learning. I do not discuss other paradigms such as online or active learning in this article.} 



Success in supervised learning is defined as \emph{generalization} to  future examples: For a typical \emph{test example}  $(x_{\tst},y_{\tst})$, the predicted label $y'_{\tst}=f(x_{\tst})$ should ``equal'' $y_{\tst}$, perhaps approximately. We usually assume the test example is drawn from the same  ``source'' as the training data  --- commonly, i.i.d.~from the same distribution. The quality of the prediction is quantified by $\ell(y'_{\tst},y_{\tst})$, where $\ell:~\Y~\times~\Y \to \RR_{\geq 0}$ is a \emph{loss function} chosen as part of the problem definition. Common loss functions include the 0-1 loss $\ell_{0-1}(y',y) = [y' \neq y]$ for \emph{classification} problems,\footnote{The notation $[P]$ denotes $1$ when predicate $P$ is true, and denotes $0$ when $P$ is false.} as well as the absolute loss $|y'-y|$ or squared loss $(y'-y)^2$ for \emph{regression problems} featuring $\Y  \sse \RR$.

Nontrivial generalization properties are typically only possible if one assumes something about the data.\footnote{The need for such an assumption is formalized by the  \emph{no free lunch theorems} of supervised learning \cite{wolpert_connection_1992,wolpert_lack_1996,schaffer_conservation_1994}.} The Bayesian approach to  machine learning, common in many applications, assumes some parametric form for the distribution generating the data, and postulates a prior on the parameters. This is not the approach I will take in this article. Instead, I will focus on the frequentist --- and some would say ``worst-case'' or ``adversarial'' ---  approach that is common in the computational learning theory community, embodied by the PAC model. Here we assume that the (training and test) data can be explained, perhaps approximately, by a function in some ``simple enough to learn'' class of functions $\H \sse \Y^\X$, often called the \emph{hypotheses}. Equivalently, we  seek a predictor which explains the unseen data roughly  as well as the best hypothesis $h^* \in \H$, whether or not we assume that $h^*$ itself provides a perfect explanation.



 \paragraph{Common Algorithmic Templates.} Perhaps the best known general-purpose supervised learning algorithm is \emph{empirical risk minimization (ERM)}, which chooses as its predictor a hypothesis $f \in \H$ minimizing $\frac{1}{n} \sum_{i=1}^n \ell(f(x_i),y_i)$ --- a quantity called the \emph{training error}, \emph{empirical error}, or \emph{empirical risk} of $f$. %\footnote{When multiple hypotheses minimize the empirical risk, we assume ERM breaks ties arbitrarily.}
A common template for generalizing ERM involves adding a \emph{regularization term} $\psi(f)$ to the  objective function, typically chosen to measure some notion of ``hypothesis complexity.'' An algorithm instantiating this template is known as a \emph{structural risk minimizer (SRM)}, and chooses as its predictor the hypothesis $f \in \H$ minimizing the \emph{structural risk} $\frac{1}{n} \sum_{i=1}^n \ell(f(x_i),y_i) + \psi(f)$. Other well-known algorithms, such as gradient descent and its variations,  can frequently be interpreted as approximate implementations of ERM or SRM.


\paragraph{Proper vs Improper Learning.} A learning algorithm is said to be \emph{proper} if its predictor $f$ is always chosen from the hypothesis class, i.e., $f \in \H$, otherwise it is said to be \emph{improper}. ERM  is an example of a proper learning algorithm, as are SRM algorithms of the form described above.  In the \emph{proper regime} of learning, algorithms are required to be proper. This article will be concerned with the more flexible \emph{improper regime} (a.k.a \emph{representation-independent learning}), where no such constraint is placed on the learner. In other words, all we care about is predictive power at test time, rather than any insights derived from the functional form or representation of the predictor~itself.


\subsection{The PAC Model}
A standard mathematical setup for evaluation of supervised learning algorithms, at least in the theoretical computer science community, is Valiant's \emph{Probably Approximately Correct (PAC) model} of learning (see e.g.~\cite{kearns_introduction_1994,mohri_foundations_2018}). Here, we assume there is an unknown distribution $\D$ on $\X \times \Y$ from which training and test data are  drawn.  Specifically, the labeled datapoints of the training set  $(x_1,y_1), \ldots, (x_n,y_n)$, as well as the test data  $(x_\tst,y_\tst)$, are i.i.d.~from $\D$. Often it is assumed that $\D$ lies in some class of distributions of interest. The \emph{true expected loss}, or simply \emph{loss}, of a predictor $f: \X \to \Y$ is the expected loss it incurs on draws from $\D$, written $L_\D(f) = \Ex_{(x,y) \sim \D} \ell(f(x),y)$.


There are two main ``settings'' in PAC learning. The  \emph{realizable setting} only requires that the data be perfectly explained by some hypothesis in $\H$. More generally, the \emph{agnostic setting} makes no assumption relating the data to the hypotheses, but shifts the goalposts as necessary to allow nontrivial guarantees: the expected loss at test time is evaluated only ``relative'' to that of the best hypothesis $h^* \in \H$. There are other settings which make more nuanced assumptions, such as $\D$ being of a particular parametric form or its support living in some (unknown) lower-dimensional space, etc. I will mostly discuss the realizable and agnostic settings in this article, those being the simplest and most studied from a theoretical perspective. %TODO:We will briefly discuss other settings in Section ??

The PAC model demands high probability guarantees of learners, in the worst case over distributions of interest. Consider first the realizable setting, where $\D$ is such that $\min_{h \in \H} L_{\D}(h) = 0$. A PAC learner has \emph{error} $\epsilon=\epsilon(n)$ and \emph{confidence} $\delta=\delta(n)$ if, when training data consists of $n$ i.i.d~samples from a realizable distribution $\D$, it produces a predictor $f$  satisfying $L_\D(f) \leq \epsilon$ with probability at least $1-\delta$. In the agnostic setting, where $\D$ can be arbitrary, we require $L_\D(f) - \min_{h \in \H} L_\D(h) \leq \epsilon$ with probability $1-\delta$.

In both the realizable and agnostic settings, we look for PAC learners with small $\epsilon$ and $\delta$ as a function of the sample size $n$. An equivalent perspective looks at the sample complexity $m(\epsilon,\delta)$, which is the minimum sample size which guarantees error  at most $\epsilon$ with probability at least $1-\delta$. We say a problem is \emph{PAC learnable} if its PAC sample complexity is finite whenever $\epsilon,\delta > 0$.

For most PAC learning problems, learnability and sample complexity are characterized in terms of a  ``dimension'' of the hypothesis class. Most prominently this is the \emph{VC dimension} for binary classification, the \emph{fat shattering dimension} for agnostic regression, and the \emph{DS dimension} for multiclass classification (see \cite{anthony_neural_1999,daniely_optimal_2014,brukhim_characterization_2022}). Treatment of these is beyond the scope of this article. The unfamiliar reader need not worry, however,  as dimensions will feature only tangentially in our~discussion.




%\paragraph{Learning settings: Realizable, Agnostic, etc.} In learning theory, evaluating a supervised learning algorithm requires specifying a data model and an objective. We will leave the details of the data model flexible for now, to allow for both the PAC model and the adversarial transductive model. Nonetheless we will describe two variations, which we call ``settings'', which cut across different models. The  \emph{realizable setting}  requires only that the data be perfectly explained by some hypothesis $h \in \H$ --- i.e., there exists a hypothesis which is guaranteed to suffer a loss of $0$ on training and test data. The performance of the learning algorithm is its expected loss at test time for some ``worst case'' realizable instance. More generally, the \emph{agnostic setting} makes no assumption relating the data to the hypotheses, but shifts the goalposts as necessary to allow nontrivial guarantees: the expected loss at test time is evaluated only ``relative'' to that of the best hypothesis $h^* \in \H$, again for some ``worst case'' instance. There are other settings which make more nuanced assumptions about the data, such as it is drawn from a distribution of a particular parametric form, or that it lives in some (unknown) lower-dimensional space, etc. We will mostly discuss the realizable and agnostic settings, those being the simplest and most studied from a theoretical perspective.




%%% Local Variables:
%%% mode: latex
%%% TeX-master: "learning_matching"
%%% End:

\section{Design}\label{sec:design}

%%%%%%%%%%%%%%%%%%%%%%%%%%%%%%


\begin{figure*}[t]
    \centering
    \includegraphics[trim = 15 530 15 15, width=1\textwidth]{Algorithm_drawio.pdf}
    \caption{Overview of KiSS}
    \label{fig:overview}
\end{figure*}


The results we gleaned from the previous section (see Section~\ref{sec:work_anly}) helped in developing our policy: KiSS. The KiSS or \textbf{Keep it Separated Serverless} policy aims to address critical challenges in Function-as-a-Service (FaaS) platforms, particularly in edge computing environments, by achieving the following objectives:

\begin{itemize}
    \item \textbf{Reduced Cold Start Latency:} Prioritizes high-frequency functions to minimize delays in real-time applications.
    \item \textbf{Improved Resource Efficiency:} Optimizes memory and compute usage while avoiding unnecessary overhead from static warm states.
    \item \textbf{Minimized Inter-Function Interference:} Enhances throughput and scalability through modular resource partitioning.
    \item \textbf{Improved Function Service Rate:} Adopts resource-aware policies to reduce dropped requests and maximize system reliability.
\end{itemize}


\subsection{KiSS Policy Overview}

KiSS introduces a modular, data-driven orchestration strategy designed to optimize serverless execution in resource-constrained environments, particularly at the edge. By leveraging our workload analysis (refer Section 2.5), our policy segments functions based on key metrics—memory footprint, invocation frequency, and execution time—to optimize performance across diverse workloads.

The edge computing context introduces unique challenges like limited memory, heterogeneous resources, and dynamic workloads. Generalized cloud strategies often fail to adapt to such constraints. KiSS addresses this gap by analyzing workload characteristics and implementing a resource-efficient, modular strategy that aligns with edge-specific demands.

\subsection{Components of KiSS Policy Design}
Figure~\ref{fig:overview} shows the overall architecture of KiSS. 
The incoming \textit{FaaS traffic} will include both small and large functions. 
The \textit{request handler} accepts the incoming functions and shares the function information to the workload analyzer. 
The \textit{workload analyser} processes the function information to profile the incoming function traffic information and generate data such as invocation frequency, memory footprint etc.
The \textit{KiSS policy} uses this data to estimate where this function will be placed between the two different warm pool partitions.

The \textit{load balancer} implements a partitioning logic where functions are allocated to distinct warm pools using (\textit{invoker 1} and \textit{invoker 2}) based on profiling thresholds:

(i)~Small Functions Pool: Dedicated to high-frequency, low-memory functions to ensure low latency, and (ii)~Large Functions Pool: Allocated for low-frequency, memory-intensive functions, minimizing contention with smaller containers.
Each warm pool operates autonomously achieving Policy Independence.
The \textit{Warm Pool Replacement Policy} for each warm container pool can independently implement different workload-specific strategies to reduce contention and enhance temporal locality.


These factors form the foundation of KiSS’s multi-tiered warm pool framework, allowing it to effectively manage serverless resources and enhance performance in edge computing. By addressing these challenges, KiSS positions itself as a practical and scalable solution for FaaS platforms in environments with diverse and demanding resource constraints.


\subsection{Innovations of KiSS Policy}

One of the most innovative features of KiSS is its multi-level warm pool partitioning, which isolates high- and low-frequency functions into separate pools. This design eliminates inefficiencies inherent in monolithic resource strategies by ensuring that small, frequently invoked functions are always ready to execute, while larger, less frequent functions remain accessible without competing for resources. This adaptability extends to the ability to add more pools as workload patterns evolve, making KiSS a flexible and future-proof solution. Moreover, its modular architecture supports diverse deployment scenarios, from centralized clouds to resource-constrained edge environments. Integration with traffic-aware schedulers ensures that KiSS maintains scalability and responsiveness even under fluctuating workloads.


\subsubsection{Advantages of KiSS}

The advantages of KiSS are particularly pronounced in edge environments. By keeping frequently accessed containers in warm states, it drastically reduces cold start latency, which is critical for real-time applications such as IoT and AI analytics. Static warm pool partitioning, based on workload analysis, optimizes memory usage by eliminating unnecessary overhead, ensuring that resources are used efficiently even in environments with stringent memory constraints. This strategy not only enhances performance but also reduces operational costs by consolidating memory usage and minimizing cold starts. KiSS’s platform-agnostic design further enhances its versatility, enabling seamless deployment across various serverless frameworks.


% \renewcommand\cellset{\renewcommand\arraystretch{0.7}}
\begin{figure*}
    \centering
    \resizebox{\textwidth}{!}{
    \begin{tabular}{c|c|c}
    \toprule
    \scriptsize \textbf{} & \small \textbf{$2\times$ length extrapolation} & 
    \small \makecell{\textbf{$2\times$ spatial extrapolation}}  \\ \midrule 
    \multirow{2}{*}{\makecell[t]{\small \textbf{Normal} \\ \textbf{length}}} 
    &
    \begin{minipage}{0.75\textwidth}
    \centering
\includegraphics[width=0.95\textwidth]{images/challenge/ref_vid.pdf}
    %\vspace{.1cm}
    \end{minipage}
    & 
    \begin{minipage}{0.2\textwidth}
    \centering   
    \includegraphics[height=0.25\textwidth]{images/challenge/ref_img.pdf}
    %%\vspace{.1cm}
    \end{minipage}
    \\ 
    & \small{Video of $49$ frames} &  \small{Image of 1K resolution} \\ \midrule
    % Extrapolation 
    \multirow{2}{*}{\makecell[t]{\small\textbf{PE}}} &
    \begin{minipage}{0.75\textwidth}
    \centering
    \includegraphics[width=0.95\textwidth]{images/challenge/PE_vid.pdf}
    %\vspace{.1cm}
    \end{minipage}
    & 
    \begin{minipage}{0.15\textwidth}
    \centering   
    \includegraphics[height=0.68\textwidth]{images/challenge/PE_img.pdf}
    %\vspace{.1cm}
    \end{minipage}
    \\
     &  \small{(a) Temporal repetition} & \small{(d) Spatial repetition} \\ 
    % Interpolation 
    \multirow{2}{*}{\makecell[t]{\small\textbf{PI}}} &
    \begin{minipage}{0.75\textwidth}
    %\vspace{.1cm}
    \centering
    \includegraphics[width=0.95\textwidth]{images/challenge/PI_vid.pdf}
    %\vspace{.1cm}
    \end{minipage}
    & 
    \begin{minipage}{0.15\textwidth}
    \centering 
    %\vspace{.1cm}
    \includegraphics[width=0.68\textwidth]{images/challenge/PI_img.pdf}
    %\vspace{.1cm}
    \end{minipage}
    \\
     &  \small{(b) Slower motion} & \small{(e)  Blurred details
} \\ 
    % Time-Aware RoPE 
    \multirow{2}{*}{\makecell[t]{\small\textbf{NTK}}} &
    \begin{minipage}{0.75\textwidth}
     %\vspace{.1cm}
     \centering
    \includegraphics[width=0.95\textwidth, height=50pt]{images/challenge/TA_vid.pdf}
     %\vspace{.1cm}
     \end{minipage}
    & 
    \begin{minipage}{0.15\textwidth}
    \centering   
     %\vspace{.1cm}
    \includegraphics[width=0.68\textwidth]{images/challenge/TA_img.pdf}
     %\vspace{.1cm}
    \end{minipage}
    \\
     &  \small{(c) Temporal repetition} & \small{(f) Spatial repetition} \\ \bottomrule
    \end{tabular}
    }
    \caption{\textbf{Visualization of existing methods for 2$\times$ extrapolation in video and image generation.} The base models CogVideoX-5B~\cite{yang2024cogvideox} and Lumina-Next~\cite{zhuo2024lumina} are trained to sample videos of up to 49 frames and images of up to 1K resolution, respectively. Existing methods lead to \textit{temporal repetition} or \textit{slower motion} in video extrapolation and \textit{spatial repetition} or \textit{blurred content} in image extrapolation, respectively. Please refer to Appendix~\ref{sec: existing failure} for more results and details. 
    }
    \label{fig:challenge}
    %%\vspace{-0.5cm}
\end{figure*}
% \subsection{Results on PPAC Optimization with 2 Silicon Dies}
Table~\ref{tab:2die_optimization} shows the results for the three techniques: Baseline, MMFP-SA and MMFP-RL when the number of dies $\mathcal{D}$ is 2 (one die with 7nm, and the other with 45nm). The results indicate that MMFP-SA achieves average reductions of 6\%, 8\% and 3.6\%, while MMFP-RL achieves reductions of 8.8\%, 11.7\% and 5.7\% in area, HPWL and cost, respectively, compared to the baseline. In terms of post-placement TNS, MMFP-SA and MMFP-RL show average improvements of 17.2\% and 21.7\% respectively. Moreover, MMFP-SA and MMFP-RL achieve average savings in dynamic power of 6.4\% and 8.1\% respectively. In terms of CPU runtime, MMFP-RL is on average 0.8$\times$ slower, and MMFP-SA is 0.9$\times$ slower than the baseline. However, MMFP-RL is faster than MMFP-SA and the baseline as the number of interconnected blocks $\mathcal{B}$ increases. Figure~\ref{fig:objective_iteration} shows the objective function $f$ across iterations during optimization on the \emph{netcard} design. MMFP-RL requires $293$ fewer iterations than MMFP-SA and achieves a better objective value.

\begin{figure}[ht] 
\centering
\includegraphics[width=.95\linewidth]{figures/objective_iterations1.png}
\caption{Objective function $f$ value for MMFP-SA/RL across iterations.}
\label{fig:objective_iteration}
\end{figure}



\subsection{Results on PPAC Optimization with 4 Silicon Dies}
%In heterogeneous integration, the number of dies $\mathcal{D}$ is typically more than 2. 
Table~\ref{tab:4die_optimization} shows the results for the \mbox{\emph{leon3-avnet}} design when the number of dies $\mathcal{D}$ is 4. The results show that MMFP-SA achieves average reductions of 7.2\%, 6.5\% and 3.2\% in area, HPWL and cost, respectively, compared to the baseline. MMFP-RL further improves the reductions achieving 9.7\%, 9\% and 4.3\%. Post-placement TNS is improved by 7.3\% for MMFP-SA, and 11.1\% for MMFP-RL. Furthermore, MMFP-SA achieves 7\% power savings, while MMFP-RL achieves 8.6\%. In terms of CPU runtime, MMFP-RL is 1.19$\times$ faster than the baseline, while MMFP-SA is 0.95$\times$ slower.

\begin{table*}[!ht]
\centering
\caption{PPAC optimization results for leon3-avnet design in 4 silicon dies.}
\begin{tabular}{cclrrrrrrr}
\hline
\multicolumn{2}{c}{\# dies} & \multirow{2}{*}{Method} & Area & HPWL & Cost & \multicolumn{2}{c}{Timing (ns)} & Power & CPU \\
\cline{1-2} \cline{7-8}
7nm & 45nm & & ($\times10^3 \mu$m$^2$) & ($\mu$m) & ($\times10^{-3}$) & TNS & WNS & (mW) & (sec) \\
\hline
\hline
\multirow{3}{*}{1} & \multirow{3}{*}{3} & Baseline & 1409.29 & 17495.01 & 3173 & -1253.06 & -1.948 & 1397.8 & 2104 \\
& & MMFP-SA & 1320.57 & 16029.53 & 3086 & -1171.95 & -1.872 & 1320.6 & 2308 \\
& & MMFP-RL & 1296.30 & 15702.68 & 3069 & -1128.43 & -1.830 & 1315.1 & 1963 \\
\hline
\multirow{3}{*}{2} & \multirow{3}{*}{2} & Baseline & 831.06 & 13972.70 & 2880 & -810.57 & -1.304 & 1150.3 & 2376 \\
& & MMFP-SA & 759.25 & 13295.18 & 2755 & -752.86 & -1.258 & 1051.3 & 2450 \\
& & MMFP-RL & 738.14 & 12960.91 & 2716 & -719.25 & -1.230 & 1016.3 & 2003 \\
\hline
\multirow{3}{*}{3} & \multirow{3}{*}{1} & Baseline & 590.71 & 9713.55 & 2674 & -603.74 & -1.071 & 674.9 & 2502 \\
& & MMFP-SA & 552.05 & 9101.84 & 2603 & -552.83 & -0.994 & 628.1 & 2581 \\
& & MMFP-RL & 531.92 & 8782.73 & 2568 & -530.06 & -0.976 & 619.3 & 1924 \\
\hline
\hline
& & Baseline & 1 & 1 & 1 & 1 & 1 & 1 & 1 \\
\multicolumn{2}{c}{Norm. Avg.}  & MMFP-SA & 0.928 & 0.935 & 0.968 & 0.927 & 0.951 & 0.930 & 0.950 \\
& & MMFP-RL & 0.903 & 0.910 & 0.957 & 0.889 & 0.931 & 0.914 & 1.186 \\
\hline
\end{tabular}
\label{tab:4die_optimization}
\end{table*}
\section{Evaluation}


\begin{table}[t]
    \centering
    % \vspace{-0.1in}
    \scalebox{0.78}{
    % \begin{small}
        \begin{tabular}{lccc}
            \toprule
            \multirow{2}*{\textbf{MoE Models}} & \textbf{Parameters} & \textbf{Experts Per Layer} & \textbf{Num. of} \\
            & \textbf{(active / total)} & \textbf{(active / total)} & \textbf{Layers} \\
            \otoprule 
            \mixtral~\cite{jiang2024mixtral} & 12.9B / 46.7B & 2 / 8 & 32 \\
            % \hline
            \qwen~\cite{yang2024qwen2} & 2.7B / 14.3B & 4 / 60 & 24 \\
            \phimoe~\cite{abdin2024phi} & 6.6B / 42B & 2 / 16 & 32 \\
            \bottomrule 
        \end{tabular}
    % \end{small}
    }
    \caption{Characteristics of three \MoE models in evaluation.}
    \vspace{-0.2in}
    \label{table:eval-moe-models}
\end{table}








\subsection{Experimental Setup}
\label{subsec:eval-setup}


% \begin{figure*}[t]
%     \centering
%     \begin{subfigure}[t]{0.48\textwidth}
%         \centering
%         \includegraphics[width=.9\linewidth]{figs/eval-overall-lmsys.pdf}
%         \caption{Serving three \MoE models with LMSYS-Chat-1M dataset.}
%     \end{subfigure}
%     \begin{subfigure}[t]{0.48\textwidth}
%         \centering
%         \includegraphics[width=.9\linewidth]{figs/eval-overall-sharegpt.pdf}
%         \caption{Serving three \MoE models with ShareGPT dataset.}
%     \end{subfigure}
%     \caption{Overall performance of prefill and decode stages for \sys and other four baselines.}
%     \label{fig:eval-overall.pdf}
% \end{figure*}


\noindent \textbf{Testbed.}
We conduct all experiments on a six-GPU testbed, where each GPU is an NVIDIA GeForce RTX 3090 with 24 GB GPU memory. 
%
All GPUs are inter-connected using pairwise NVLinks and connected to the CPU memory using PCIe 4.0 with 32GB/s bandwidth. 
%
Additionally, the testbed has a total of 32 AMD Ryzen Threadripper PRO 3955WX CPU cores and 480 GB CPU memory.


\noindent \textbf{Models.}
We employ three popular \MoE-based \LLMs in our evaluation: \mixtral~\cite{jiang2024mixtral}, \qwen~\cite{yang2024qwen2}, and \phimoe~\cite{abdin2024phi}.
Table~\ref{table:eval-moe-models} describes the parameters, number of \MoE layers, and number of experts per layer for the three models.
Following the evaluation of existing works~\cite{song2024promoe}, we profile the models to set the optimal prefetch distance $d$ to three before evaluation.
% We set $d$ of \mixtral, \qwen, and \phimoe to \todo{$xxx$}, \todo{$xxx$}, and \todo{$xxx$}, respectively.


\noindent \textbf{Datasets and traces.}
We employ two real-world prompt datasets commonly used for \LLM evaluation: LMSYS-Chat-1M~\cite{zheng2023lmsys} and ShareGPT~\cite{sharegpt}.
%
For most experiments, we split the sampled datasets in a standard 7:3 ratio, where 70\% of the prompts' context data (\ie, semantic embeddings and expert maps) are stored in \sys's Expert Map Store, and 30\% of the prompts are used for testing. 
%
For online serving experiments, we empty the Expert Map Store and use real-world \LLM inference traces~\cite{patel2024splitwise,stojkovic2025dynamollm} released by Microsoft Azure to set input and generation lengths and drive invocations.

\noindent \textbf{Baselines.}
We compare \sys against four \SOTA \MoE serving baselines:
1) \textbf{MoE-Infinity}~\cite{xue2024moe} uses coarse-grained request-level expert activation patterns and synchronous expert prediction and prefetching for \MoE serving. 
We prepare the expert activation matrix collection for MoE-Infinity before evaluation for a fair comparison.
%
% However, the open-sourced MoE-Infinity codebase~\cite{moe-infinity-code} lacks some features described in its original paper, we had to modify
%y 
2) \textbf{ProMoE}~\cite{song2024promoe} employs a stride-based speculative expert prefetching approach for \MoE serving. Since the codebase of ProMoE is not open-sourced and requires training predictors for each \MoE model, we reproduced a prototype of ProMoE on top of MoE-Infinity in our best effort.
%
3) \textbf{Mixtral-Offloading}~\cite{eliseev2023fast} combines a layer-wise speculative expert prefetching and a \LRU-based expert cache. 
%
4) \textbf{DeepSpeend-Inference} employs an expert-agnostic layer-wise parameter offloading approach, which uses pure on-demand loading and does not support prefetching. 
%
We implement the offloading logic of DeepSpeed-Inference in the MoE-Infinity codebase and add an expert cache for a fair comparison.
We enable all baselines to serve \MoE models from HuggingFace Transformer~\cite{wolf2020huggingface}. 


\noindent \textbf{Metrics.}
Following the standard evaluation methodology of existing works~\cite{song2024promoe,xue2024moe,zhong2024distserve,agrawal2024taming} on \LLM serving, we report the performance of the prefill and decode stages separately. 
We measure Time-to-First-Token (TTFT) for the prefill stage and Time-Per-Output-Token (TPOT) for the decode stage.
Additionally, we also report other system metrics, such as expert hit rate and overheads, for detailed evaluation.


% \noindent \textbf{\sys's setting.}
% The hyperparameters of \sys containing the prefetch distance $d$ for each \MoE model, Expert Map Store capacity $C$, and Expert Cache memory limit $M$.
% For most experiments, we profile the \MoE models and set the prefetch distance $d$ to their optimal values. The Expert Map Store capacity $C$ is set to \todo{$xxx$} expert maps. We configure the Expert Cache memory limit to \todo{$xxx$} GB.
% The hyperparameter sensitivity is analyzed in \S\ref{subsec:eval-sensitivity}.


\begin{figure}[t]
  \centering
  \includegraphics[width=.95\linewidth]{figs/eval-overall-arxiv.pdf}
  \vspace{-0.15in}
  \caption{Overall performance of prefill and decode stages for \sys and other four baselines.}
  \vspace{-0.2in}
  \label{fig:eval-overall}
\end{figure}


\subsection{Overall Performance}
\label{subsec:eval-overall}



We first evaluate the performance of prefill and decode stages when running \sys and other baselines with the three \MoE models, where we measure Time-To-First-Token (TTFT) and Time-Per-Output-Token (TPOT) for each stage.
Note that the inference latency with expert offloading tends to be higher than no offloading due to two reasons: 
1) During inference, an excessive amount of parameters in \MoE models are loaded and offloaded, which prolongs the inference latency.
2) All baselines and \sys are implemented on top of the MoE-Infinity codebase~\cite{moe-infinity-code}, whose inference latency is inherently impacted by MoE-Infinity's implementation.
Nevertheless, comparing \sys and baselines is fair with the same experimental setup.

Figure~\ref{fig:eval-overall} shows the \TTFT, \TPOT, and expert hit rate of \sys and other four baselines when serving three \MoE models with LMSYS-Chat-1M and ShareGPT datasets, respectively.
DeepSpeed has both the worst \TTFT and \TPOT due to expert-agnostic offloading and lacking expert prefetching.
While Mixtral-Offloading, ProMoE, and MoE-Infinity perform better than DeepSpeed-Inference, they are underperformed by \sys because of coarse-grained offloading designs.
Compared to DeepSpeed-Inference, Mixtral-Offloading, ProMoE, and MoE-Infinity, our \sys reduces the average \TTFT by 44\%, 35\%, 33\%, 30\%, and reduces the average \TPOT by 70\%, 61\%, 55\%, 48\%, across three \MoE models.
%
% Figure~\ref{fig:eval-overall} also reports the expert hit rate of \sys and each baseline. 
For expert hit rate, Mixtral-Offloading achieves a higher hit rate than the other three baselines because of its synchronous speculative prefetching with a prefetch distance of 1. However, due to synchronous prefetching, its \TTFT and \TPOT are worse than others except DeepSpeed-Inference.
\sys improves the average expert hit rate by 147\%, 11\%, 34\%, and 63\% over DeepSpeed-Inference, Mixtral-Offloading, ProMoE, and MoE-Infinity, respectively.

% \begin{figure}[t]
%   \centering
%   \includegraphics[width=.9\linewidth]{figs/eval-overall-sharegpt.pdf}
%   % \vspace{-0.15in}
%   \caption{}
%   % \vspace{-0.25in}
%   \label{fig:eval-overall-sharegpt.pdf}
% \end{figure}




\subsection{Online Serving Performance}
\label{subsec:eval-online}


Except for the offline evaluation (\ie, Expert Map Store in full capacity before serving), we also evaluate \sys against other baselines in online serving settings.
We empty the Expert Map Store of \sys and the expert activation matrix collection of MoE-Infinity for the online serving experiment.
%
The request traces are derived from Azure \LLM inference traces~\cite{patel2024splitwise,stojkovic2025dynamollm}, with 64 requests randomly sampled to drive LMSYS-Chat-1M prompts for each \MoE model serving. 
To ensure consistency, \sys and all baselines input and generate the exact number of tokens specified in the traces.
%
Figure~\ref{fig:eval-online-serve} illustrates the CDF of end-to-end request latency across three \MoE models. The results demonstrate that \sys significantly reduces overall request latency compared to other baselines in online serving scenarios.


\begin{figure}[t]
  \centering
  \includegraphics[width=.95\linewidth]{figs/eval-online-serve-arxiv.pdf}
  \vspace{-0.15in}
  \caption{CDF of request latency for \MoE online serving.}
  \vspace{-0.2in}
  \label{fig:eval-online-serve}
\end{figure}



\subsection{Impact of Expert Cache Limits}



We measure the \TPOT of \sys and other baselines by limiting the expert cache memory budget to investigate their performance in the latency-memory trade-off (\S\ref{subsec:bg-latency-memory-tradeoff}).
We mainly focus on \TPOT to show the end-to-end performance impacted by varying cache limits.
Figure~\ref{fig:eval-cache-limit.pdf} shows the \TPOT of \sys and other four baselines when serving three \MoE models under different expert cache limits.
We gradually increase the GPU memory allocated for caching experts from 6 GB to 96 GB while employing the same experimental setting in \S\ref{subsec:eval-overall}.
Similarly, DeepSpeed-Inference has the worst \TPOT due to being expert-agnostic.
\sys consistently outperforms Mixtral-Offloading, ProMoE, and MoE-Infinity under varying expert cache limits.
Especially for limited GPU memory sizes (\eg, 6GB), \sys reduces the \TPOT by 32\%, 24\%, 18\%, and 18\%, compared to DeepSpeed-Inference, Mixtral-Offloading, ProMoE, and MoE-Infinity, across three \MoE models, respectively.
With fine-grained expert offloading, \sys significantly reduces the expert on-demand loading latency while maintaining a lower GPU memory footprint, therefore achieving a better spot in the latency-memory trade-off of \MoE serving.

% \subsection{Impact of Inference Batch Size}

\subsection{Ablation Study}
\label{subsec:eval-ablation}


% \begin{figure}[t]
%   \centering
%   \includegraphics[width=.95\linewidth]{figs/eval-expert-tracking.pdf}
%   % \vspace{-0.15in}
%   \caption{Expert hit rate of different expert pattern tracking approaches.}
%   % \vspace{-0.25in}
%   \label{fig:eval-expert-tracking}
% \end{figure}



We present the ablation study of \sys's design.


\textbf{Effectiveness of expert map search.}
One of \sys's key designs is the expert map, which tracks expert selection preferences in fine granularity.
We evaluate the effectiveness of the expert map against five expert pattern-tracking approaches as follows.
%
1) \textbf{Speculate}: speculative prediction used by Mixtral-Offloading~\cite{eliseev2023fast} and ProMoE~\cite{song2024promoe}, 
%
2) \textbf{Hit count}: request-level expert hit count used by MoE-Infinity~\cite{xue2024moe}, 
%
3) \textbf{Map (T)}: expert map with only trajectory similarity search,
4) \textbf{Map (T+S)}: expert map with both trajectory and semantic similarity search,
%
and
5) \textbf{Map (T+S+$\delta$)}: expert map with full features enabled, including trajectory and semantic similarity search (\S\ref{subsec:design-similarity-match}) and dynamic expert selection (\S\ref{subsec:design-expert-prefetch}).
%
We implement the above methods in \sys's Expert Map Matcher for a fair comparison.
Figure~\ref{fig:eval-expert-tracking} shows the expert hit rate of the above expert pattern tracking methods.
%
Speculative prediction is effective due to the widespread presence of residual connections in Transformer blocks. However, its effectiveness decreases drastically as prefetch distance increases~\cite{song2024promoe}.
%
The request-level expert activation count has the worst performance due to coarse granularity.
%
As features are incrementally restored to \sys's expert map, the expert hit rate gradually increases, demonstrating its effectiveness.

% \textbf{Effectiveness of asynchronous map matching.}




\begin{figure}[t]
  \centering
  \includegraphics[width=.9\linewidth]{figs/eval-cache-limit-arxiv.pdf}
  \vspace{-0.15in}
  \caption{Performance of \sys and other four baselines under varying expert cache limits.}
  \vspace{-0.1in}
  \label{fig:eval-cache-limit.pdf}
\end{figure}

\begin{figure}[!t]
    \centering
    \begin{subfigure}[t]{0.585\linewidth}
        \centering
        \includegraphics[width=\linewidth]{figs/eval-expert-tracking.pdf}
        \caption{Expert pattern tracking approaches.}
        \label{fig:eval-expert-tracking}
    \end{subfigure}
    % \hspace{0.02in}
    \begin{subfigure}[t]{0.385\linewidth}
        \centering
        \includegraphics[width=\linewidth]{figs/eval-prefetch-and-cache-arxiv.pdf}
        \caption{Prefetch and caching.}
        \label{fig:eval-prefetch-and-cache}
    \end{subfigure}
    \vspace{-0.1in}
    \caption{Ablation study of \sys.}
    \label{fig:eval-ablation}
    \vspace{-0.2in}
\end{figure}

\textbf{Effectiveness of expert prefetching and caching.}
We evaluate \sys's expert prefetching and caching against two caching algorithms:
1) \textbf{\LRU} used by Mixtral-Offloading~\cite{eliseev2023fast}
and 
2) \textbf{\LFU} used by MoE-Infinity~\cite{xue2024moe}.
%
Figure~\ref{fig:eval-prefetch-and-cache} depicts the expert hit rate of \sys and two baselines.
The results show that \LRU performs poorly in expert offloading scenarios. Though \LFU achieves a higher hit rate than \LRU, \sys surpasses both, achieving the highest expert hit rate.

\subsection{Sensitivity Analysis}
\label{subsec:eval-sensitivity}


\begin{figure}[t]
  \centering
  \includegraphics[width=.9\linewidth]{figs/eval-prefetch-distance.pdf}
  \vspace{-0.15in}
  \caption{Performance of \sys serving \MoE models with different prefetch distances.}
  \vspace{-0.1in}
  \label{fig:eval-prefetch-distance}
\end{figure}

% \begin{figure}[t]
%   \centering
%   \includegraphics[width=.9\linewidth]{figs/eval-store-capacity.pdf}
%   % \vspace{-0.15in}
%   \caption{Semantic and trajectory similarity lower bounds in \sys's serving with different Expert Map Store capacity.}
%   % \vspace{-0.25in}
%   \label{fig:eval-store-capacity}
% \end{figure}

\begin{figure}[t]
    \centering
    \begin{subfigure}[t]{0.55\linewidth}
        \centering
        \includegraphics[width=\linewidth]{figs/eval-store-capacity.pdf}
        \caption{Expert Map Store capacity.}
        \label{fig:eval-store-capacity}
    \end{subfigure}
    % \hspace{0.02in}
    \begin{subfigure}[t]{0.435\linewidth}
        \centering
        \includegraphics[width=\linewidth]{figs/eval-batch-size-arxiv.pdf}
        \caption{Inference batch size.}
        \label{fig:eval-batch-size}
    \end{subfigure}
    \vspace{-0.1in}
    \caption{Sensitivity analysis of \sys.}
    \vspace{-0.2in}
    \label{fig:eval-sensitivity}
\end{figure}


We analyze the sensitivity of three hyperparameters: prefetch distance of \MoE models, the capacity of Expert Map Store, and inference batch size.


\textbf{Prefetch distance of \MoE models.}
Figure~\ref{fig:eval-prefetch-distance} shows the \TTFT and \TPOT of \sys when serving three \MoE models with different prefetch distances.
%
We have demonstrated that the expert hit rate decreases when gradually increasing the prefetch distance (Figure~\ref{fig:bg-hit-distance}).
%
When the prefetch distance is small ($<3$), \sys cannot perfectly hide its system delay from the inference process, such as the map matching and expert prefetching, leading to the increase of inference latency.
%
With larger prefetch distances ($>3$), \sys has worse expert hit rates that also degrade the performance. 
Therefore, we set the prefetch distance $d$ to 3 for evaluating \sys.


\textbf{Capacity of Expert Map Store.}
We measure the mean semantic and trajectory similarity scores searched in \sys's expert map matching for \MoE model serving.
%
Figure~\ref{fig:eval-store-capacity} presents the mean semantic and trajectory similarity scores of \sys with different Expert Map Store capacity sizes.
%
Both semantic and trajectory similarity scores improve as the store capacity increases.
%
While the similarity scores exhibit a significant increase with capacities below 1K, further capacity expansion yields diminishing similarity gains. 
To minimize \sys's memory overhead, we set \sys's Expert Map Store capacity to 1K in evaluation.


\textbf{Inference batch size.}
We investigate the impact of inference batch size on \sys and three baselines using \mixtral with LMSYS-Chat-1M.
%
Figure~\ref{fig:eval-batch-size} presents the performance of \sys, Mixtral-Offloading, ProMoE, and MoE-Infinity as the batch size increases from one to four. \sys achieves the lowest \TTFT and \TPOT in most cases.


% \textbf{Inference batch size.}


% \subsection{Scalability}
% \label{subsec:eval-scalability}
% From one to six GPUs


\begin{figure}[t]
  \centering
  \includegraphics[width=.92\linewidth]{figs/eval-overhead-latency.pdf}
  \vspace{-0.15in}
  \caption{Latency breakdown of \sys's one inference iteration with three \MoE models.}
  \vspace{-0.1in}
  \label{fig:eval-overhead-latency.pdf}
\end{figure}





\subsection{System Overheads}
\label{subsec:eval-overhead}


\noindent \textbf{Latency overheads of \sys's operations.}
Figure~\ref{fig:eval-overhead-latency.pdf} shows the latency breakdown of one inference iteration in \sys when serving the three \MoE models.
We report any operations of \sys in \S\ref{subsec:eval-overall} that may incur a significant latency delay, including context collection, map matching, expert on-demand loading, expert prefetching, and map update after the iteration completes.
\qwen has lower end-to-end iteration latency than \mixtral and \phimoe because of significantly fewer parameters.
Note that expert prefetching, map matching, and map update tasks are executed asynchronously, aside from the inference process. Hence, they do not contribute to the end-to-end iteration latency.
Excluding three asynchronous tasks, the total delay incurred by other operations is consistently less than 30ms (5\% of the iteration) across three \MoE models, which is negligible compared to the inference latency.


\noindent \textbf{Memory overheads of \sys's Expert Map Store.}
Figure~\ref{fig:eval-overhead-memory.pdf} shows the CPU memory footprint of \sys's Expert Map Store when varying the store capacity from 1K to 32K maps.
The memory needed to store expert maps for \qwen is more than \mixtral and \phimoe because it has more experts per layer over the other two models, which increases the map shape.
Even for the largest capacity (32K), the Expert Map Store requires less than 200MB of memory to store the maps, which is trivial since modern GPU servers usually have abundant CPU memory (\eg, p4d.24xlarge on AWS EC2~\cite{aws-ec2} has over 1100 GB of CPU memory).
In the evaluation, \sys's map store capacity with 1K maps is sufficient for maintaining performance (\S\ref{subsec:eval-sensitivity}), resulting in minimal memory overhead.



\begin{figure}[t]
  \centering
  \includegraphics[width=.85\linewidth]{figs/eval-overhead-memory.pdf}
  % \vspace{-0.1in}
  \caption{CPU memory footprint of \sys's Expert Map Store with different capacity.}
  \vspace{-0.1in}
  \label{fig:eval-overhead-memory.pdf}
\end{figure}

% 
\section{Related Work} \label{sec:related}

% \textbf{Adversarial Attack}
\textbf{Attacks on SLAM.} 
%With the rise of machine learning, 
The robustness of computer vision systems is being actively investigated. With the emergence of adversarial images in the digital domain by adding optimized noise directly to images~\cite{szegedy2013intriguing,carlini2017towards}, researchers find that such attacks also exist physically in the real world \cite{eykholt2018robust,song2018physical,zhao2019seeing}. To fill the gap between attacks in the digital and physical worlds, recent studies have demonstrated that attacks on real-world computer vision systems are practical \cite{eykholt2018robust,li2019adversarial,man2020ghostimage,sharif2016accessorize,zhao2019seeing,zhou2018invisible}. However, attacks on traditional computer vision methods such as SLAM are relatively less explored. \cite{yoshida2022adversarial} proposes an attack against the scan matching algorithm in LiDAR-based SLAM, while most SLAMs in AR/VR devices rely on different sensors like RGB/depth cameras and IMUs. \cite{ikram2022perceptual} and \cite{chen2024adversary} mislead visual SLAM by poisoning the images with special patterns, and \cite{wang2021can} causes the camera to fail using infrared light. In our work, we demonstrate attacks on Visual-Inertial SLAM (VI-SLAM) by perturbing the IMU readings, rather than cameras, and showing its impact on XR user experience. 

\textbf{Acoustic Injection Attacks.} Among various physical attacks, acoustic injection attacks are attractive due to their low cost. Son~\etal~\cite{son2015rocking} were the first to introduce acoustic attacks on MEMS gyroscopes, demonstrating how these attacks could lead to sensor denial-of-service and result in drone crashes. WALNUT~\cite{trippel2017walnut} expanded on this by developing output biasing and control attacks that enable precise manipulation of MEMS accelerometer outputs using modulated sound waves. Wang et al.~\cite{wang2017sonic} demonstrated a sonic gun, showcasing the vulnerability of various smart devices (\eg drones and self-balancing vehicles) to acoustic attacks. Tu et al. \cite{tu2018injected} designed side-swing and switching attacks to alter the outputs of MEMS gyroscopes and accelerometers. Furthermore, Ji et al. \cite{ji2021poltergeist} fool the object detectors by applying acoustic attack to the image stabilizers commonly used in modern cameras. However, none of the existing works study the relationship between the acoustic injections and SLAM outputs on recent XR devices. 

% \zijian{Do we need one session about security in AR/VR?}
% \yicheng{TODO}
%\jiasi{cite the AIVR paper (UMass Amherst?) paper is we have not already. They add IMU perturbation but w/o SLAM, iirc} \yicheng{Cited}

\textbf{XR Security and Privacy.} 
%Security and privacy concerns in XR systems have gained significant attention. 
For single-user XR systems, researchers have demonstrated various side-channel attacks to extract sensitive information (\eg keystrokes) through video feeds~\cite{ling2019know}, head movements~\cite{nair2023unique, slocum2023going}, architectural hints~\cite{zhang2023its,shang2020arspy}, power usage~\cite{li2024dangers}, and EM side-channel leakages~\cite{al2021vr}. In multi-user XR systems, Su et al.~\cite{su2024remote} use avatar motion data to infer keystrokes in shared VR environments. Slocum et al.~\cite{slocum2024doesn} reveal vulnerabilities in the shared state frameworks of multi-user AR. Similarly, Lebeck et al.~\cite{lebeck2017securing} highlight risks like deceptive virtual objects and emphasize access control for managing shared physical and virtual spaces. Ruth et al.~\cite{ruth2019secure} further propose a secure multi-user AR framework focusing on content sharing and permissions.
Chandio et al.~\cite{chandio2024stealthy} %introduced a multi-modal spatiotemporal attack that 
simultaneously manipulated visual and inertial sensors to disrupt XR pose estimation. However, their study evaluated the attack using offline datasets and assumed the attacker's capability to manipulate IMU data streams through acoustic means, without real experiments. Ours is the first to demonstrate acoustic injection attacks on recent XR devices, like the Hololens 2, in the real world.
 




In this paper we have described our efforts in mechanizing the strand spaces framework~\cite{FHG98} in Coq.
To assess the flexibility of the approach and the usability of the library and of the proofs we have analyzed a variety of examples: a basic authentication protocol and some of its variants, the classical Needham-Schroeder-Lowe authentication protocol, and a recent key management API equipped with a key management policy.

Wherever possible, our mechanization remains faithful to the original pen-and-paper development of strand spaces.
At the same time, we put a lot of engineering effort to make the code and the proofs reusable.
For that, we have made the framework modular and parametric in the terms and the penetrator.
Additionally, we have developed a number of strands-specific tactics whose goal is to make the life of the protocol's analyst easier by removing some of the burden of these kinds of proofs.
Indeed, the tactics automate a number of intermediate steps enabling, in some cases, easy proof reuse.
For instance, the proof of the NSL responder's nonce secrecy
 required just one hour of work using the initiator's nonce secrecy.
The mechanization
gives the freedom to experiment with protocols and their properties, while retaining the unique ability of strand spaces-based analyses to give interesting insights on the inner workings of protocols.
With our experiments, we uncovered
and fixed issues, discarded
redundant or unused requirements, and significantly improved previous results on the analysis of key management policies, making it possible to formally prove the security of the \emph{secure templates} policy from \cite{BCFS-ccs10} (\cref{sec:casestudies}).

\cref{tab:simpleauth,tab:nsl} in \cref{sec:summary}  summarize the premises for each security property across the analyzed protocol variants. These premises are essential for our security proofs and offer important insights into the assumptions required to make a security protocol correct. The strand spaces model highlights this aspect, and the use of Coq and the \easystrands{} library further clarifies the minimal and necessary nature of these assumptions, reinforcing the model's ability to accurately capture security requirements.
With the insights from these experiments we also developed a new proof technique which we call \emph{protected predicate} technique that, in certain situations, simplifies the proofs making some previously challenging cases trivial.


Another advantage of having this mechanized platform is that it opens up new and interesting avenues of research.
\ifdefined\COLORDIFF
    \color{cbred}
\else
\fi
For instance, an intriguing enhancement to our framework would be the inclusion of algebraic intruders. We believe they can be implemented using at least two approaches, which we briefly outline below.

Given an equational theory $E$ over a signature $\mathit{FS}$, the first approach requires implementing $E$ as a (terminating and confluent) rewriting system \lstinline{rew_E}, and allow penetrators to use \lstinline{rew_E} to manipulate terms containing symbols of $\mathit{FS}$.
More concretely, we first need to create an instance of \easystrands{} terms with support for function symbols in $\mathit{FS}$, then we can extend the penetrator as:
\begin{lstlisting}
Inductive penetrator_strand : Σ -> Prop := ...
| PT_Eqn : forall (g h : 𝔸) i, replace g h rew_E  -> penetrator_strand (i, [⊖ g; ⊕ h]).
\end{lstlisting}
where \lstinline{replace g h rew_E} holds iff \lstinline{g} can be rewritten as \lstinline{h} under \lstinline{rew_E}.
This approach is inspired by that of Tamarin \cite{MSCB13}.

The second approach aligns  with the method used in DY*~\cite{DY}, where cryptographic primitives are modeled as functions that symbolically represent the actual primitives, e.g., \lstinline{dec (c, k) = (if c = enc (m, k) then m else Error)}.
With these definitions, the equational theory $E$ could be defined using Coq Setoids and used for terms in place of Leibniz equality.
This has the advantage to allow both honest parties and the intruder to transparently use the equational theory.
However, as observed by~\citet{DY}, this approach requires proving (at least) that $E$ is an equivalence relation respected by all functions, predicates, and protocol specifications which can be lengthy and tedious.
\ifdefined\COLORDIFF
    \color{black}
\else
\fi

Despite their age, strand spaces have been a catalyst for extensive research, leading to notable extensions that include authentication tests~\cite{guttman2000authentication}, process algebraic-style choice operators~\cite{YEMMS16},
 compositionality \cite{StrandComposition,StrandIndependence,StrandMixed}, and stateful protocols \cite{J12}.
Many of these advancements are crucial for enhancing the expressiveness and usability of the model.
Our plan is to enhance \easystrands{} by integrating these extensions, thereby enabling scalability to more realistic protocols.
Ultimately, this will help narrow the gap with state-of-the-art tools such as DY* \cite{DY}.
In terms of foundational research, an intriguing avenue involves closely examining the relationship between Paulson's inductive method \cite{Paulson94} and strand spaces. We plan to mechanize Paulson's method in Coq and conduct a comparative analysis to assess the relative merits of these two inductive methods.

Finally, we defined a maximal penetrator as the set of strands that do not violate sensitive cryptographic operations required for protocol security. This method is inspired by the approach in \cite{banaSymbolic} to achieve computational soundness and, to our knowledge, has not been explored in a purely symbolic context before. It allows for proving injective agreement without explicitly defining the Dolev-Yao attacker, which we showed to be \diff{strictly} subsumed by the maximal penetrator. Notably, this approach facilitates the composition of protocols proven secure under their respective maximal penetrators, provided they adhere to each other's constraints. We are currently extending this technique to protocols like NSL, where security relies on decryption capabilities.


\section*{Acknowledgements}
This is acknowledgment.

% \bibliographystyle{IEEEtran}
\documentclass[10pt,journal,compsoc]{IEEEtran}
\IEEEoverridecommandlockouts

% *** CITATION PACKAGES ***
%
\ifCLASSOPTIONcompsoc
  % The IEEE Computer Society needs nocompress option
  % requires cite.sty v4.0 or later (November 2003)
  \usepackage[nocompress]{cite}
\else
  % normal IEEE
  \usepackage{cite}
\fi
\usepackage{amsmath,amssymb,amsfonts}
\usepackage{algorithmic}
\usepackage{graphicx}
\usepackage{textcomp}
\usepackage[absolute]{textpos}


\usepackage{enumitem}
\usepackage{color}
\usepackage{multirow}
\usepackage{lipsum}
\usepackage{listings}
% \usepackage{subfigure}
\usepackage{graphicx}
\usepackage{pdfpages}
\usepackage{verbatim}
\usepackage{pifont}
\usepackage{etoolbox}
\usepackage{textcomp}
% \usepackage{ulem}
\usepackage{hyperref}
\usepackage{url}
\usepackage{caption}
\usepackage{booktabs} % For nice tables
\usepackage{siunitx} % To align table numbers by unit
\usepackage{etoolbox}

% \usepackage{booktabs}
% \usepackage{multirow}
% \usepackage{graphicx}
\usepackage[normalem]{ulem}
\useunder{\uline}{\ul}{}

\usepackage[switch]{lineno} % default option is 'left'

\usepackage[linesnumbered,titlenumbered,ruled,vlined,resetcount,algosection]{algorithm2e}
\usepackage{xcolor}
\usepackage{color}
\def\BibTeX{{\rm B\kern-.05em{\sc i\kern-.025em b}\kern-.08em
    T\kern-.1667em\lower.7ex\hbox{E}\kern-.125emX}}

\usepackage{amsthm}
\newtheorem{theorem}{Theorem}
%\newtheorem{theorem}{Theorem}[section]
%\newtheorem{corollary}{Corollary}[theorem]
\newtheorem{corollary}{Corollary}
%\newtheorem{lemma}[theorem]{Lemma}
\newtheorem{lemma}{Lemma}

\usepackage{hhline}
\usepackage{xcolor}
\usepackage{colortbl}
\usepackage{threeparttable}
% \usepackage{minted}
\usepackage{subfig}
\usepackage{comment}
\usepackage{enumitem}
% \usepackage[ruled,vlined]{algorithm2e}
% Circled number
\usepackage{tikz}
\newcommand*\circled[1]{\tikz[baseline=(char.base)]{
            \node[shape=circle,fill,inner sep=0.8pt] (char) {\footnotesize\textcolor{white}{#1}};}}
\newcommand*\redcircled[1]{\tikz[baseline=(char.base)]{
            \node[shape=circle,fill=red,inner sep=1.3pt] (char) {\textcolor{white}{#1}};}}
            
\def\BibTeX{{\rm B\kern-.05em{\sc i\kern-.025em b}\kern-.08em
    T\kern-.1667em\lower.7ex\hbox{E}\kern-.125emX}}

% \setlength{\floatsep}{1em}
% \setlength{\textfloatsep}{0.8em}
% \setlength{\intextsep}{0.1em}
% \setlength{\belowcaptionskip}{0.2em} 

\newcommand{\yguo}[1]{
  \begin{framed}
    \noindent{\textcolor{blue}{\bf\em #1 -- Yanfei Guo}}
  \end{framed}
}

\newcommand{\yujia}[1]{\textcolor{red}{#1}} % Track changes from Yujia

\newcommand{\fn}[1]{{\tt\small #1}}
%%

% \newcommand{\pname}{\textit{gZCCL}}
\newcommand{\pname}{\mbox{ZCCL}}
% \newcommand{\pname}{\mbox{\textbf{\texttt{ZCCL}}}}
\newcommand{\newtext}[1]{\textcolor{blue}{#1}}

\newcommand{\fzlight}{\mbox{\textit{fZ}-light}}


\begin{document}

\Urlmuskip=0mu plus 1mu


\title{ZCCL: Significantly Improving Collective Communication With Error-Bounded Lossy Compression}

\author{Jiajun Huang, Sheng Di,~\IEEEmembership{Senior Member,~IEEE}, Xiaodong Yu, Yujia Zhai, Zhaorui Zhang, Jinyang Liu, Xiaoyi Lu, Ken Raffenetti, Hui Zhou, Kai Zhao, Khalid Alharthi, Zizhong Chen,~\IEEEmembership{Senior Member,~IEEE}, Franck Cappello,~\IEEEmembership{Fellow,~IEEE}, Yanfei Guo, Rajeev Thakur,~\IEEEmembership{Fellow,~IEEE}
% Xiaodong Yu,\IEEEauthorrefmark{3}
% Yujia Zhai,\IEEEauthorrefmark{1}
% Zhaorui Zhang,\IEEEauthorrefmark{4}
% Jinyang Liu,\IEEEauthorrefmark{1}
% Xiaoyi Lu,\IEEEauthorrefmark{5}
% Ken Raffenetti\IEEEauthorrefmark{2}
% Hui Zhou\IEEEauthorrefmark{2}
% Kai Zhao\IEEEauthorrefmark{6}
% Zizhong Chen\IEEEauthorrefmark{1}
% Franck Cappello\IEEEauthorrefmark{2}
% Yanfei Guo\IEEEauthorrefmark{2}
% Rajeev Thakur\IEEEauthorrefmark{2}

\IEEEcompsocitemizethanks{
\IEEEcompsocthanksitem Jiajun Huang, Yujia Zhai, and Zizhong Chen are affiliated with the University of California, Riverside, CA 92521. Sheng Di, Ken Raffenetti, Hui Zhou, Franck Cappello, Yanfei Guo, and Rajeev Thakur are affiliated with Argonne National Laboratory, Lemont, IL 60439. Xiaodong Yu is affiliated with Stevens Institute of Technology, Hoboken, NJ 07030. Zhaorui Zhang is affiliated with The Hong Kong Polytechnic University, Kowloon, Hong Kong. Jinyang Liu is affiliated with University of Houston,
Houston, TX 77204. Xiaoyo Lu is is affiliated with University of California, Merced, CA 95343. Kai Zhao is affiliated with Florida State University, Tallahassee, FL 32306. Khalid Alharthi is affiliated with Department Of Computer Science, College of Computing And Information Technology, University Of Bisha, Bisha 61922, P.O. Box 551, Saudi Arabia. 
% \IEEEcompsocthanksitem A shortened version of this paper was presented at 2024 IEEE International Parallel
% and Distributed Processing Symposium (IPDPS) and was published in its \href{https://doi.ieeecomputersociety.org/10.1109/IPDPS57955.2024.00072}{proceedings}.
}

}

% The paper headers
% \markboth{Journal of \LaTeX\ Class Files,~Vol.~14, No.~8, August~2021}%
% {Shell \MakeLowercase{\textit{et al.}}: A Sample Article Using IEEEtran.cls for IEEE Journals}

% \IEEEpubid{0000--0000/00\$00.00~\copyright~2021 IEEE}

% for Computer Society papers, we must declare the abstract and index terms
% PRIOR to the title within the \IEEEtitleabstractindextext IEEEtran
% command as these need to go into the title area created by \maketitle.
% As a general rule, do not put math, special symbols or citations
% in the abstract or keywords.
\IEEEtitleabstractindextext{%

\begin{abstract}



With the ever-increasing computing power of supercomputers and the growing scale of scientific applications, the efficiency of MPI collective communication turns out to be a critical bottleneck in large-scale distributed and parallel processing. The large message size in MPI collectives is particularly concerning because it can significantly degrade overall parallel performance. To address this issue, prior research simply applies off-the-shelf fixed-rate lossy compressors in the MPI collectives, leading to suboptimal performance, limited generalizability, and unbounded errors. In this paper, we propose a novel solution, called {\pname}, which leverages error-bounded lossy compression to significantly reduce the message size, resulting in a substantial reduction in communication costs. The key contributions are three-fold. (1) We develop two general, optimized lossy-compression-based frameworks for both types of MPI collectives (collective data movement as well as collective computation), based on their particular characteristics. Our framework not only reduces communication costs but also preserves data accuracy. (2) We customize {\fzlight}, an ultra-fast error-bounded lossy compressor, to meet the specific needs of collective communication. (3) We integrate {\pname} into multiple collectives, such as Allgather, Allreduce, Scatter, and Broadcast, and perform a comprehensive evaluation based on real-world scientific application datasets. Experiments show that our solution outperforms the original MPI collectives as well as multiple baselines by 1.9--8.9$\times$.

\end{abstract}

\begin{IEEEkeywords}
Error-bounded Lossy Compression, Collective Communication, Distributed Computing, Parallel Algorithm
\end{IEEEkeywords}
}
\maketitle

%%
%% This command processes the author and affiliation and title
%% information and builds the first part of the formatted document.
\maketitle

\documentclass[../main.tex]{subfiles}
\graphicspath{{../images/}}
\makeatletter
\def\input@path{{../images/}}
\makeatother
\begin{document}
\section{Introduction}
\begin{figure}
\centering
\begin{tikzpicture}
\node[inner sep=0pt] (ws) at (0, 0) {
\includegraphics[height=.4\textwidth, trim={10cm 0 10cm 0},clip]{world_space.png}};
\node[inner sep=0pt] (cs) at (6,0) {\includegraphics[height=.4\textwidth, trim={10cm 1cm 10cm 4cm},clip]{conf_space.png}};
\end{tikzpicture}
\vspace{-5pt}
\label{fig:pbrm_intro}
\caption{\textbf{Left}: Shows world space obstacles as grey spheres. Robots start and goal configuration is colored red and green, respectively. Configurations along the computed path are colored transparent blue. \textbf{Right:} Mapped world space scenario to configuration space. Obstacle region is the grey mesh. Red spheres are collision-free regions computed by the neural SCDF. The optimized shortest path in the convex corridor is the blue curve.}
\vspace{-25pt}
\end{figure}
Motion planning is the problem of finding a collision-free trajectory that connects a given start and goal configuration. The planning takes place in the configuration space of the robot. For single body robots, like mobile robots or drones, the configuration space and the world space are usually the same. This simplifies the planning, since explicit obstacle representations are available which enables geometrical tools like separating hyperplanes, smallest distance to obstacles etc., to be used when designing motion planning algorithms. For multi-body robots like manipulators, the situation is completely different. The world space obstacles are usually mapped to non-convex regions, and to make the problem even harder, the mapping is usually not known. Forming explicit representations of the obstacle region in the configuration space is usually too expensive or intractable. Despite all of this, sampling based planners are used with great success, which mainly is due to their use of implicit representations of the obstacle region. The basic idea is to construct a graph in the configuration space that covers and connects the collision-free region. From this graph, a path can be extracted that connects a given start and goal configuration. The approach is computationally expensive, since the graph is constructed with the smallest geometrical building block available, points, which represents a collision-check. Furthermore, the extracted paths from the graph are non-smooth and jagged due to the stochastic nature of the approach. This adds an additional post-processing step to the process, where the paths are shortcutted and smoothened, before the path can be used for tracking. Clearly a lot of time is invested to form this graph and produce smooth paths. Thus, if the obstacles start to move, then all of this work is done in no use, since all points that make up this graph need to be re-verified, which is simply too time consuming to be done in real time.
\\\\
In this work, we want to address the existing drawbacks of the sampling based planners. Our main contribution is an improved motion planner where each vertex in the graph covers a collision-free region in the form of a sphere instead of a point and where the edges are formed with neighboring intersecting spheres. This representation has the advantage of instead of returning piecewise linear paths, returning a sequence of overlapping spheres, i.e. a convex corridor, that connects a given start and goal configuration, illustrated in Figure \ref{fig:pbrm_intro}. This convex corridor allows us to use convex optimization to produce smooth trajectories, instead of computationally expensive post-processing methods. The representation further allows us to estimate the coverage of the collision-free space, which gives us awareness and feedback in the offline roadmap construction phase. Finally, our representation is simple to adapt to moving obstacles, simply requery for the new radii and recheck for intersections. 
\\\\
The spherical collision-free regions are formed using a signed distance function (SDF), which is a function that returns the smallest distance from an arbitrary point to the boundary of an obstacle. As the name implies, the distance is signed, thus if the point is inside the obstacle it is negative otherwise positive. If the distance is positive, a sphere with radius equal to the distance is guaranteed to cover a collision-free region. Using an SDF in motion planning is not new, but what is novel about our approach is that we express the distance in the configuration space instead of the world space and by doing so allows us to form these convex collision-free regions. We refer to the resulting SDF as a signed configuration distance function (SCDF). Computing an SCDF analytically is non-trivial, our approach is therefore to parameterize the SCDF with a deep neural network and learn the mapping by supervised learning. Our resulting neural SCDF can compute distances for different parameter values of obstacle shapes and we also show how multiple distances can be combined, thus making our approach flexible.
\section{Related work}
Motion planning algorithms can roughly be divided into three families, grid-based, sampling based and optimization based methods. Grid-based methods (GBM) discretize the planning space from which a graph is then compiled. A standard search method is A$^\star$ \citep{a_star}, which is classified as an \textit{informed} search method, since it employs a heuristic function to speed up the search. A$^\star$ guarantees to return an optimal path at the level of discretization used. GBMs usually discretize the planning space by a regular lattice and this limits the GBMs to problems with low dimensionality due to the curse of dimensionality. Thus, GBMs are usually limited to single-body robots where the degrees of freedom (DOF) are low. To overcome the inherent scaling problem with the GBMs, stochastic methods are usually used for multi-body robots. These methods are termed as sampling-based methods (SBM) and core members within this family are the rapidly-exploring random trees (RRT) \citep{rrt} and the probabilistic roadmap (PRM) \citep{prm}. RRT grows a tree from the start configuration and explores the collision-free region in a rapid way until it is able to connect to the goal region. RRT is usually improved by bi-directional planning \citep{rrt_connect}, i.e. an additional tree is grown from the goal configuration and the trees are tested for connection after any tree has been expanded. RRT is a single-query method, thus it searches for a path from scratch each time it is queried. Contrary to this, PRM is a multi-query method, which solves for multiple queries without starting from scratch. PRM does this by creating a roadmap (graph) that covers the collision-free space as an offline step. The graph is then used to solve for multiple queries. PRMs are used in cases where the environment does not change since the extra offline step is too computationally costly and needs to be re-done if the environment is changed. In our work, we address this inherent issue by using a different roadmap representation. Our vertices in the graph cover a collision-free region in the form of spheres and we form the edges by checking for intersecting spheres. If something in the environment changes, we recompute the spheres radii and recheck the intersections, without relying on collision detection. We use a trained neural network to compute the sphere radius, therefore querying for the radius can be done fast, hence our representation enables the PRM for dynamic environments.
\\\\
In the recent decades, optimization based methods (OBM) \citep{chomp, schulman, itomp, stomp} have been introduced as an alternative to SBM for multi-body robots. Like the SBM, the OBMs scale well to higher dimensional problems and produce smoother motion. It is common to use a SDF in the optimization since it is a smooth function, thus enabling gradient-based methods. However, the standard way of expressing the SDF is in world space. The distance therefore needs to be mapped to the configuration space by the forward kinematics. This mapping makes the optimization problem a non-linear program (NLP), which is computationally expensive to solve. Recently, a different approach has been proposed. In \cite{mp_gcs} motion planning is formulated as a convex optimization problem by using the graph of convex sets framework \citep{gcs}. The underlying idea is to decompose the collision-free space into intersecting convex sets from which a convex optimization problem is formulated. In cases where an explicit representation of the obstacles in the configuration space exists, like for single-body robots, creating collision-free convex regions can be done fast \citep{iris}. For multi-body robots, this is non-trivial. Existing work does this successfully \citep{iris_nlp, iris_c} by an optimization based approach, but the methods are still too time consuming to be used in the presence of moving obstacles. Our approach is instead to use deep learning to learn an SDF expressed in the configuration space. With this, we can query for shortest distances to the collision boundary, which allows us to expand spherical regions which are collision-free. Our approach is fast and therefore enables our suggested roadmap planner to be used in dynamic environments.
\\\\
Recent research has focused on learning collision detection \citep{fk_kernel_distance, diffco, graphdistnet} by predicting the signed distance between the robot links and the surrounding obstacles in the world space. The learned SDF is used in trajectory optimization but since the distance is expressed in the world space, the problem becomes an NLP and therefore takes a long time to solve. We take a novel approach and suggest to instead express the signed distance in the configuration space. This allows us to improve the PRM at the same time as it enables convex optimization for trajectory optimization, which runs faster and is more reliable than NLP solvers. In \cite{cspf} a learned signed distance function in the configuration space is proposed similar to our approach. However, their approach is restricted to point cloud representations, while we propose to represent the obstacles as parameterized geometric shapes, e.g. spheres. Furthermore, we also show how to use our learned SCDF to improve an existing roadmap planner.
\section{Problem formulation}
A robot is located in the world space, $\W \subset \R^3 $. The unique location of the robot is given by its configuration $\q \in \C$, where $\C$ is the configuration space. The set of points covered by the robots bodies at a certain configuration is expressed as $\B(\q) \subset \W$. The robot is surrounded by $\NrObst$ obstacles $\O = \bigcup_{i=1}^{\NrObst} \O_i$, where  $\O_i \subset \W$. The representation of the obstacle in the configuration space is the set $\C\O_i = \{\q \in \C \: |\: \B(\q) \cap \O_i \neq \emptyset \}$. The obstacle space is formed as $\Co = \bigcup_{i=1}^{\NrObst} \C \O_i$. The complement is referred to as the free space, $\Cf = \C \setminus \Co$. The path planning problem is a tuple, ($\Cf$, $\qStart$, $\qGoal$), where we want to connect a query pair, consisting of a start, $\qStart$, and goal configuration, $\qGoal$, with a geometric path, $\q(s): [0, 1] \mapsto \Cf$, such that $\q(0)=\qStart$ and $\q(1)=\qGoal$, or report correctly when such a path does not exist.
\end{document}

% \section{Motivation}
\label{sec:motivation}



% In LLM inference, not only does weight matter, but the memory requirements of the KV Cache are also considerable.
In this section, we first demonstrate that the emerging paradigm of group quantization demands a high level of adaptivity, which current adaptive methods lack.
We then discuss how adapting these methods to group quantization could compromise their efficiency.
Given that LLMs generate KV caches during runtime, real-time quantization capability is crucial.
These challenges lead to our proposal of a mathematical adaptive numerical type (\texttt{MANT}), which we will detail later.



\begin{figure}[t]
    \centering
    \begin{minipage}[t]{0.48\columnwidth}
      \centering
      \includegraphics[width=\columnwidth]{fig/moti_group_ppl.pdf}
      \caption{LLM accuracy with different quantization granularities. We report the perplexity (PPL) metric (lower is better).}\label{fig:moti_group_ppl} 
    \end{minipage}
    \hspace{2pt}
    \begin{minipage}[t]{0.48\columnwidth}
      \centering
      \includegraphics[width=\columnwidth]{fig/motivation_adaptive_ppl.pdf}
      \caption{Accuracy loss for \texttt{INT}, \texttt{ANT}, and Ideal (clustering algorithm K-Means) adaptive methods in group quantization. }\label{fig:moti_ppl} 
    \end{minipage}
    % \vspace*{-0.3cm}
\end{figure}




\subsection{Group Quantization Accuracy Analysis}
\label{sec:acc_analysis}

In this subsection, we begin by comparing the accuracy of traditional channel-wise quantization with group-wise quantization~\cite{shao2024omniquant,zhao2023atom,liu2024kivi,sheng2023flexgen,lin2023awq,zhao2023atom}, establishing the baseline for group-wise quantization in this study.
We then delve into the use of various adaptive data types in group quantization, emphasizing the necessity for full adaptivity.



\Fig{fig:moti_group_ppl} illustrates the perplexity when quantizing the LLaMA-7B model~\cite{touvron2023llama} with various granularities using the \texttt{INT4}-based symmetric quantization.
Channel-wise quantization significantly worsens the perplexity of the examined LLM, increasing it from 5.68 to 6.85.
Conversely, group-wise quantization mitigates this loss in perplexity with a group size of 128, corresponding to an average of 4.125 bits per element (16-bit scaling factor).
Additionally, we observe that a smaller group size of 32 offers only a slight improvement in perplexity, but the scaling factor overhead increases by $4\times$.



Given this analysis, we adopt a group size of 128 as our standard configuration for the remainder of this section.
Previous research indicates that the \texttt{INT} data type is not optimal for accuracy since tensors or channels exhibit varied distributions, leading to the proposal of various adaptive data types~\cite{guo2022ant, guo2023olive, zadeh2020gobo, zadeh2022mokey}.
We evaluate their efficacy in the context of group quantization, which falls into two main categories: data-type-based and clustering-based.



\textbf{Data-type-based adaptive methods} select data types from discrete sets based on tensor data distribution.
ANT~\cite{guo2022ant} is a representative example of the data-type-based method.
ANT packages several different data types for selection, including \texttt{INT} for the uniform distribution, \texttt{PoT} (Power of Two) for the Laplace distribution, and \texttt{flint} for the Gaussian distribution.
%ANT designed \texttt{flint} for Gaussian distributions.

\textbf{Clustering-based adaptive methods} utilize clustering algorithms to generate centroids that align with the data distribution and provide considerable adaptivity. 
Mokey~\cite{zadeh2022mokey} and GOBO~\cite{zadeh2020gobo} exemplify this approach, though they focus on tensor- or channel-wise quantization. In our study, we adapt them to group quantization through per-group clustering.

%Clustering-based methods employ clustering algorithms to generate centroids that fit the data distribution, demonstrating sufficient adaptivity.
%Mokey~\cite{zadeh2022mokey} and GOBO~\cite{zadeh2020gobo} are such presentative works, but only target tensor- or channel-wise quantization.
%In our work, we modify those works to support group quantization by performing per-group clustering.
\Fig{fig:moti_ppl} compares the accuracy of the methods described above for the LLaMA-7B model under 4-bit group-wise quantization. 
The group-wise \texttt{ANT} method outperforms the \texttt{INT} type by dynamically selecting from three data types to better match the value distribution, resulting in reduced perplexity (PPL) loss. 
Moreover, per-group clustering adjusts more effectively to the value distribution of each group, establishing itself as the accuracy-optimal and ideal adaptive method. 
This approach achieves nearly lossless 4-bit quantization, equivalent to 16 centroids per group. 
However, this ideal scenario is impractical due to the significant overhead associated with storing per-group centroids, effectively rendering it a 6-bit quantization.

\begin{figure}[t] 
    \centering 
    \includegraphics[width=1.0\linewidth]{fig/intro_cdf.pdf}  
    \caption{The cumulative distribution function (CDF) of the tensor, channel, and group, respectively. The tensor data were taken from layers 8 to 23, while the 16 channel and group data were sampled from one tensor with specific strides.}\label{fig:moti_dist} 
    %  \vspace*{-0.3cm}
\end{figure}

To illustrate the group-wise diversity in data distribution, we sampled the weights of the Q and V tensors in LLaMA-7B model. 
We normalized all sampled data to their absolute maximum values, which ranged from -1 to 1. \Fig{fig:moti_dist} displays the cumulative distribution function (CDF) for the tensor, channel, and group levels, respectively. 
We observed that the diversity at the group level is significantly higher than at the tensor level. 
In simpler terms, while different tensors exhibit similar distributions, groups can have markedly different distributions. This finding underscores the necessity for full adaptivity in group quantization to fully realize its potential.
\paragraph{Takeaway 1.} The group quantization is an emerging paradigm to accelerate LLMs, and the significant group-level diversity requires a high level of adaptivity to fully unleash its potential.

\subsection{Group Quantization Efficiency Analysis}
\label{subsec:efficiency}


In this subsection, we provide a detailed efficiency analysis for the above adaptive quantization methods.
In \Tbl{intro:dtype}, we compare OliVe~\cite{guo2023olive}, ANT~\cite{guo2022ant}, GOBO~\cite{zadeh2020gobo}, and Mokey~\cite{zadeh2022mokey} with \texttt{INT} regarding the efficiency of computation, encoding, and decoding. 
In this paper, we use the term encoding (decoding) interchangeably with quantization (dequantization).
 

Data-type-based adaptive methods such as ANT~\cite{guo2022ant} and Olive~\cite{guo2023olive} achieve computational efficiency comparable to \texttt{INT}. 
Both utilize specialized decoders that decode these data types prior to computation, resulting in high decoding efficiency. 
However, as previously demonstrated, these methods suffer from limited adaptivity in the group quantization paradigm. 
A straightforward approach to enhance adaptivity is to expand their set of data types. 
However, incorporating new data types necessitates additional decoders, escalating hardware design costs. 
Additionally, compatibility issues between new and existing data types may reduce computational efficiency. 
For instance, the \texttt{NF4} data type~\cite{dettmers2023qlora} requires an FP16 MAC unit, which is incompatible with existing \texttt{ANT} data types.


\paragraph{Takeaway 2.} Enhancing the data-type-based adaptive method for group quantization is challenging and requires a careful balance for the computation and decoding efficiency.

Clustering-based adaptive methods like GOBO~\cite{zadeh2020gobo} and Mokey~\cite{zadeh2022mokey} can sufficiently adapt to various distributions at the group level. 
However, they require codebooks for quantization and dequantization, leading to high adaptivity at the expense of encoding and computational efficiency. 
For instance, a 16-entry codebook with 8 bits per entry requires 128 bits per group, creating an inevitable trade-off between adaptivity and memory overhead. GOBO~\cite{zadeh2020gobo} employs the K-means algorithm to quantize weights and requires dequantization to \texttt{FP16} using a codebook lookup table before computation, resulting in high adaptivity but low computational efficiency. 
Conversely, Mokey~\cite{zadeh2022mokey} enhances the computation of clustering-based methods by using indices for centroid values via approximate calculations, though matrix multiplication still relies on floating-point units, increasing overhead compared to integer units. 
Furthermore, Mokey creates one \texttt{golden dictionary} for all activations and weights, akin to using a single data type in quantization, thus reducing adaptivity.


\paragraph{Takeaway 3.} Deploying the clustering-based adaptive methods under group quantization is challenging owing to the low encoding and computation efficiency. 


\begin{table}[t]
    \centering
    \small
    \renewcommand{\arraystretch}{1.2}
    \caption[]{Features of DNN accelerators with adaptive and flexible data types are summarized. Here, `Effi.' stands for efficiency, `Med.' for medium, and `LUT' for lookup table.}
  
    \resizebox{1.0\columnwidth}{!}{
      \begin{tabular}{c|cc|ccc|cc|c}
        \Xhline{1.2pt}
        \multirow{2}{*}{Architecture} & \multicolumn{2}{c|}{Encode} & \multicolumn{3}{c|}{Computation} & \multicolumn{2}{c|}{Decode} & \multirow{2}{*}{Adaptivity} \\ \cline{2-8}
        & Method & Effi. & Method & Bit & Effi. & Method & Effi. \\
        \Xhline{1.2pt}
        \texttt{INT} & Round & High & INT & 4 \& 8 & High & Calculation & High & Low \\ 
        OliVe~\cite{guo2023olive} & Search & Med. & INT & 4 \& 8 & High & Decoder & High & Med. \\ 
        ANT~\cite{guo2022ant} & Search & Med. & INT & 4 \& 8 & High & Decoder & High & Med. \\ 
        Mokey~\cite{zadeh2022mokey} & Cluster & Med. & Float & 4 \& 8 & Med. & Calculation & Med. & Low \\ 
        GOBO~\cite{zadeh2020gobo} & Cluster & Low & Float & 16 & Low & LUT & Med. & High \\ 
        \hline
        \multirow{2}{*}{\proj}  & Search  & Med.  & \multirow{2}{*}{INT} & \multirow{2}{*}{4 \& 8} & \multirow{2}{*}{High} & \multirow{2}{*}{Calculation} & \multirow{2}{*}{High} & \multirow{2}{*}{High} \\ \cline{2-3}
        &  Map &  High &  &&&\\ 
        \Xhline{1.2pt}
    \end{tabular}
    }
    \vspace*{0.1cm}
    \label{intro:dtype}
    \vspace*{-0.2cm}
  \end{table}

\subsection{Support for Real-time Quantization}
\label{sec:moti_kvcache}

The above group-wise diversity presents a challenge for both weights and KV cache.
In addition, KV cache faces challenges in real-time group-wise quantization because the KV cache is generated dynamically during LLM inference.


To facilitate low-precision computation in group-wise quantization, it is necessary to quantize K and V along the inner dimension. 
This requirement stems from the support for matrix inner product operations in most GPUs and TPUs. 
During these operations, the group-wise scaling factor can be extracted from the multiply-accumulate process. 
\Fig{fig:kv_process} depicts the computation process of K and V during the decode stage. We define the dimension used for matrix inner product operations as the inner dimension. 
The inner dimensions of the K and V caches differ; the K cache requires a transpose operation, whereas the V cache does not, complicating the situation.


In the prefill stage, K and V can easily compute the scaling factor for each group. 
During the decode stage, the newly generated K vector is concatenated along the inner dimension of the K cache, enabling immediate quantization. 
However, the newly generated V vector is associated with different groups, with only one element per group produced per iteration. This process prevents the scaling factor for the entire group from being obtained in a single iteration, posing a significant challenge for the real-time quantization of the V cache.


\begin{figure}[t] 
  \centering 
  % \includegraphics[width=1.0\linewidth]{fig/dse_kv_process.pdf}  
  \includegraphics[width=0.9\linewidth]{fig/moti_kv_dimension.pdf}  
  \caption{\small Comparison of group-wise K and V cache quantization. They have different inner dimensions due to the transposition of K (key).}

  \label{fig:kv_process}
  % \vspace*{-0.4cm}
\end{figure}


Given those challenges, we propose \proj with a mathematical encoding format that can fuse with integer computation and enhance the decoding efficiency.
In addition, this encoding format provides sufficient adaptivity for group-wise quantization.
Regarding the challenge in KV cache, \proj employs a real-time quantization engine that ensures efficient encoding and decoding for KV cache.
By addressing these challenges, \proj enables efficient low-bit group-wise quantization.



\section{Basic Background: Supervised Learning and the PAC Model}
\label{sec:background}

At this point almost everyone has heard of machine learning (ML). Anyone likely to stumble upon this article will have also heard of its most influential special case, supervised learning, and those theoretically inclined will also be familiar with the PAC model. Nonetheless, I will set the stage by  recapping the basics.

\subsection{Basics of Supervised Learning}%Let's set the stage in any case

\emph{Supervised Learning} is the task of ``coming up'' with a function $f: \X \to \Y$ to ``explain'' or ``fit'' a sequence of input/output examples   $(x_1,y_1), \ldots, (x_n,y_n)$, with $x_i \in \X$ and $y_i \in \Y$.  Here $\X$ is a \emph{data domain} consisting of \emph{datapoints} $x \in \X$, $\Y$ is a \emph{label set} consisting of \emph{labels} $y \in \Y$, and the sequence $(x_1,y_1),\ldots,(x_n,y_n)$ is the \emph{training data} consisting of \emph{labeled examples (a.k.a. samples)}~$(x_i,y_i)$.  I~will refer to the chosen function $f$ as a \emph{predictor}, and to $n$ as the \emph{sample size}. A \emph{learning algorithm} takes as input training data, and outputs (some representation of) a predictor $f \in \Y^\X$.\footnote{Note that this describes the usual \emph{batch}, a.k.a.~\emph{offline}, setting of supervised learning. I do not discuss other paradigms such as online or active learning in this article.} 



Success in supervised learning is defined as \emph{generalization} to  future examples: For a typical \emph{test example}  $(x_{\tst},y_{\tst})$, the predicted label $y'_{\tst}=f(x_{\tst})$ should ``equal'' $y_{\tst}$, perhaps approximately. We usually assume the test example is drawn from the same  ``source'' as the training data  --- commonly, i.i.d.~from the same distribution. The quality of the prediction is quantified by $\ell(y'_{\tst},y_{\tst})$, where $\ell:~\Y~\times~\Y \to \RR_{\geq 0}$ is a \emph{loss function} chosen as part of the problem definition. Common loss functions include the 0-1 loss $\ell_{0-1}(y',y) = [y' \neq y]$ for \emph{classification} problems,\footnote{The notation $[P]$ denotes $1$ when predicate $P$ is true, and denotes $0$ when $P$ is false.} as well as the absolute loss $|y'-y|$ or squared loss $(y'-y)^2$ for \emph{regression problems} featuring $\Y  \sse \RR$.

Nontrivial generalization properties are typically only possible if one assumes something about the data.\footnote{The need for such an assumption is formalized by the  \emph{no free lunch theorems} of supervised learning \cite{wolpert_connection_1992,wolpert_lack_1996,schaffer_conservation_1994}.} The Bayesian approach to  machine learning, common in many applications, assumes some parametric form for the distribution generating the data, and postulates a prior on the parameters. This is not the approach I will take in this article. Instead, I will focus on the frequentist --- and some would say ``worst-case'' or ``adversarial'' ---  approach that is common in the computational learning theory community, embodied by the PAC model. Here we assume that the (training and test) data can be explained, perhaps approximately, by a function in some ``simple enough to learn'' class of functions $\H \sse \Y^\X$, often called the \emph{hypotheses}. Equivalently, we  seek a predictor which explains the unseen data roughly  as well as the best hypothesis $h^* \in \H$, whether or not we assume that $h^*$ itself provides a perfect explanation.



 \paragraph{Common Algorithmic Templates.} Perhaps the best known general-purpose supervised learning algorithm is \emph{empirical risk minimization (ERM)}, which chooses as its predictor a hypothesis $f \in \H$ minimizing $\frac{1}{n} \sum_{i=1}^n \ell(f(x_i),y_i)$ --- a quantity called the \emph{training error}, \emph{empirical error}, or \emph{empirical risk} of $f$. %\footnote{When multiple hypotheses minimize the empirical risk, we assume ERM breaks ties arbitrarily.}
A common template for generalizing ERM involves adding a \emph{regularization term} $\psi(f)$ to the  objective function, typically chosen to measure some notion of ``hypothesis complexity.'' An algorithm instantiating this template is known as a \emph{structural risk minimizer (SRM)}, and chooses as its predictor the hypothesis $f \in \H$ minimizing the \emph{structural risk} $\frac{1}{n} \sum_{i=1}^n \ell(f(x_i),y_i) + \psi(f)$. Other well-known algorithms, such as gradient descent and its variations,  can frequently be interpreted as approximate implementations of ERM or SRM.


\paragraph{Proper vs Improper Learning.} A learning algorithm is said to be \emph{proper} if its predictor $f$ is always chosen from the hypothesis class, i.e., $f \in \H$, otherwise it is said to be \emph{improper}. ERM  is an example of a proper learning algorithm, as are SRM algorithms of the form described above.  In the \emph{proper regime} of learning, algorithms are required to be proper. This article will be concerned with the more flexible \emph{improper regime} (a.k.a \emph{representation-independent learning}), where no such constraint is placed on the learner. In other words, all we care about is predictive power at test time, rather than any insights derived from the functional form or representation of the predictor~itself.


\subsection{The PAC Model}
A standard mathematical setup for evaluation of supervised learning algorithms, at least in the theoretical computer science community, is Valiant's \emph{Probably Approximately Correct (PAC) model} of learning (see e.g.~\cite{kearns_introduction_1994,mohri_foundations_2018}). Here, we assume there is an unknown distribution $\D$ on $\X \times \Y$ from which training and test data are  drawn.  Specifically, the labeled datapoints of the training set  $(x_1,y_1), \ldots, (x_n,y_n)$, as well as the test data  $(x_\tst,y_\tst)$, are i.i.d.~from $\D$. Often it is assumed that $\D$ lies in some class of distributions of interest. The \emph{true expected loss}, or simply \emph{loss}, of a predictor $f: \X \to \Y$ is the expected loss it incurs on draws from $\D$, written $L_\D(f) = \Ex_{(x,y) \sim \D} \ell(f(x),y)$.


There are two main ``settings'' in PAC learning. The  \emph{realizable setting} only requires that the data be perfectly explained by some hypothesis in $\H$. More generally, the \emph{agnostic setting} makes no assumption relating the data to the hypotheses, but shifts the goalposts as necessary to allow nontrivial guarantees: the expected loss at test time is evaluated only ``relative'' to that of the best hypothesis $h^* \in \H$. There are other settings which make more nuanced assumptions, such as $\D$ being of a particular parametric form or its support living in some (unknown) lower-dimensional space, etc. I will mostly discuss the realizable and agnostic settings in this article, those being the simplest and most studied from a theoretical perspective. %TODO:We will briefly discuss other settings in Section ??

The PAC model demands high probability guarantees of learners, in the worst case over distributions of interest. Consider first the realizable setting, where $\D$ is such that $\min_{h \in \H} L_{\D}(h) = 0$. A PAC learner has \emph{error} $\epsilon=\epsilon(n)$ and \emph{confidence} $\delta=\delta(n)$ if, when training data consists of $n$ i.i.d~samples from a realizable distribution $\D$, it produces a predictor $f$  satisfying $L_\D(f) \leq \epsilon$ with probability at least $1-\delta$. In the agnostic setting, where $\D$ can be arbitrary, we require $L_\D(f) - \min_{h \in \H} L_\D(h) \leq \epsilon$ with probability $1-\delta$.

In both the realizable and agnostic settings, we look for PAC learners with small $\epsilon$ and $\delta$ as a function of the sample size $n$. An equivalent perspective looks at the sample complexity $m(\epsilon,\delta)$, which is the minimum sample size which guarantees error  at most $\epsilon$ with probability at least $1-\delta$. We say a problem is \emph{PAC learnable} if its PAC sample complexity is finite whenever $\epsilon,\delta > 0$.

For most PAC learning problems, learnability and sample complexity are characterized in terms of a  ``dimension'' of the hypothesis class. Most prominently this is the \emph{VC dimension} for binary classification, the \emph{fat shattering dimension} for agnostic regression, and the \emph{DS dimension} for multiclass classification (see \cite{anthony_neural_1999,daniely_optimal_2014,brukhim_characterization_2022}). Treatment of these is beyond the scope of this article. The unfamiliar reader need not worry, however,  as dimensions will feature only tangentially in our~discussion.




%\paragraph{Learning settings: Realizable, Agnostic, etc.} In learning theory, evaluating a supervised learning algorithm requires specifying a data model and an objective. We will leave the details of the data model flexible for now, to allow for both the PAC model and the adversarial transductive model. Nonetheless we will describe two variations, which we call ``settings'', which cut across different models. The  \emph{realizable setting}  requires only that the data be perfectly explained by some hypothesis $h \in \H$ --- i.e., there exists a hypothesis which is guaranteed to suffer a loss of $0$ on training and test data. The performance of the learning algorithm is its expected loss at test time for some ``worst case'' realizable instance. More generally, the \emph{agnostic setting} makes no assumption relating the data to the hypotheses, but shifts the goalposts as necessary to allow nontrivial guarantees: the expected loss at test time is evaluated only ``relative'' to that of the best hypothesis $h^* \in \H$, again for some ``worst case'' instance. There are other settings which make more nuanced assumptions about the data, such as it is drawn from a distribution of a particular parametric form, or that it lives in some (unknown) lower-dimensional space, etc. We will mostly discuss the realizable and agnostic settings, those being the simplest and most studied from a theoretical perspective.




%%% Local Variables:
%%% mode: latex
%%% TeX-master: "learning_matching"
%%% End:

\section{Design}\label{sec:design}

%%%%%%%%%%%%%%%%%%%%%%%%%%%%%%


\begin{figure*}[t]
    \centering
    \includegraphics[trim = 15 530 15 15, width=1\textwidth]{Algorithm_drawio.pdf}
    \caption{Overview of KiSS}
    \label{fig:overview}
\end{figure*}


The results we gleaned from the previous section (see Section~\ref{sec:work_anly}) helped in developing our policy: KiSS. The KiSS or \textbf{Keep it Separated Serverless} policy aims to address critical challenges in Function-as-a-Service (FaaS) platforms, particularly in edge computing environments, by achieving the following objectives:

\begin{itemize}
    \item \textbf{Reduced Cold Start Latency:} Prioritizes high-frequency functions to minimize delays in real-time applications.
    \item \textbf{Improved Resource Efficiency:} Optimizes memory and compute usage while avoiding unnecessary overhead from static warm states.
    \item \textbf{Minimized Inter-Function Interference:} Enhances throughput and scalability through modular resource partitioning.
    \item \textbf{Improved Function Service Rate:} Adopts resource-aware policies to reduce dropped requests and maximize system reliability.
\end{itemize}


\subsection{KiSS Policy Overview}

KiSS introduces a modular, data-driven orchestration strategy designed to optimize serverless execution in resource-constrained environments, particularly at the edge. By leveraging our workload analysis (refer Section 2.5), our policy segments functions based on key metrics—memory footprint, invocation frequency, and execution time—to optimize performance across diverse workloads.

The edge computing context introduces unique challenges like limited memory, heterogeneous resources, and dynamic workloads. Generalized cloud strategies often fail to adapt to such constraints. KiSS addresses this gap by analyzing workload characteristics and implementing a resource-efficient, modular strategy that aligns with edge-specific demands.

\subsection{Components of KiSS Policy Design}
Figure~\ref{fig:overview} shows the overall architecture of KiSS. 
The incoming \textit{FaaS traffic} will include both small and large functions. 
The \textit{request handler} accepts the incoming functions and shares the function information to the workload analyzer. 
The \textit{workload analyser} processes the function information to profile the incoming function traffic information and generate data such as invocation frequency, memory footprint etc.
The \textit{KiSS policy} uses this data to estimate where this function will be placed between the two different warm pool partitions.

The \textit{load balancer} implements a partitioning logic where functions are allocated to distinct warm pools using (\textit{invoker 1} and \textit{invoker 2}) based on profiling thresholds:

(i)~Small Functions Pool: Dedicated to high-frequency, low-memory functions to ensure low latency, and (ii)~Large Functions Pool: Allocated for low-frequency, memory-intensive functions, minimizing contention with smaller containers.
Each warm pool operates autonomously achieving Policy Independence.
The \textit{Warm Pool Replacement Policy} for each warm container pool can independently implement different workload-specific strategies to reduce contention and enhance temporal locality.


These factors form the foundation of KiSS’s multi-tiered warm pool framework, allowing it to effectively manage serverless resources and enhance performance in edge computing. By addressing these challenges, KiSS positions itself as a practical and scalable solution for FaaS platforms in environments with diverse and demanding resource constraints.


\subsection{Innovations of KiSS Policy}

One of the most innovative features of KiSS is its multi-level warm pool partitioning, which isolates high- and low-frequency functions into separate pools. This design eliminates inefficiencies inherent in monolithic resource strategies by ensuring that small, frequently invoked functions are always ready to execute, while larger, less frequent functions remain accessible without competing for resources. This adaptability extends to the ability to add more pools as workload patterns evolve, making KiSS a flexible and future-proof solution. Moreover, its modular architecture supports diverse deployment scenarios, from centralized clouds to resource-constrained edge environments. Integration with traffic-aware schedulers ensures that KiSS maintains scalability and responsiveness even under fluctuating workloads.


\subsubsection{Advantages of KiSS}

The advantages of KiSS are particularly pronounced in edge environments. By keeping frequently accessed containers in warm states, it drastically reduces cold start latency, which is critical for real-time applications such as IoT and AI analytics. Static warm pool partitioning, based on workload analysis, optimizes memory usage by eliminating unnecessary overhead, ensuring that resources are used efficiently even in environments with stringent memory constraints. This strategy not only enhances performance but also reduces operational costs by consolidating memory usage and minimizing cold starts. KiSS’s platform-agnostic design further enhances its versatility, enabling seamless deployment across various serverless frameworks.


% \renewcommand\cellset{\renewcommand\arraystretch{0.7}}
\begin{figure*}
    \centering
    \resizebox{\textwidth}{!}{
    \begin{tabular}{c|c|c}
    \toprule
    \scriptsize \textbf{} & \small \textbf{$2\times$ length extrapolation} & 
    \small \makecell{\textbf{$2\times$ spatial extrapolation}}  \\ \midrule 
    \multirow{2}{*}{\makecell[t]{\small \textbf{Normal} \\ \textbf{length}}} 
    &
    \begin{minipage}{0.75\textwidth}
    \centering
\includegraphics[width=0.95\textwidth]{images/challenge/ref_vid.pdf}
    %\vspace{.1cm}
    \end{minipage}
    & 
    \begin{minipage}{0.2\textwidth}
    \centering   
    \includegraphics[height=0.25\textwidth]{images/challenge/ref_img.pdf}
    %%\vspace{.1cm}
    \end{minipage}
    \\ 
    & \small{Video of $49$ frames} &  \small{Image of 1K resolution} \\ \midrule
    % Extrapolation 
    \multirow{2}{*}{\makecell[t]{\small\textbf{PE}}} &
    \begin{minipage}{0.75\textwidth}
    \centering
    \includegraphics[width=0.95\textwidth]{images/challenge/PE_vid.pdf}
    %\vspace{.1cm}
    \end{minipage}
    & 
    \begin{minipage}{0.15\textwidth}
    \centering   
    \includegraphics[height=0.68\textwidth]{images/challenge/PE_img.pdf}
    %\vspace{.1cm}
    \end{minipage}
    \\
     &  \small{(a) Temporal repetition} & \small{(d) Spatial repetition} \\ 
    % Interpolation 
    \multirow{2}{*}{\makecell[t]{\small\textbf{PI}}} &
    \begin{minipage}{0.75\textwidth}
    %\vspace{.1cm}
    \centering
    \includegraphics[width=0.95\textwidth]{images/challenge/PI_vid.pdf}
    %\vspace{.1cm}
    \end{minipage}
    & 
    \begin{minipage}{0.15\textwidth}
    \centering 
    %\vspace{.1cm}
    \includegraphics[width=0.68\textwidth]{images/challenge/PI_img.pdf}
    %\vspace{.1cm}
    \end{minipage}
    \\
     &  \small{(b) Slower motion} & \small{(e)  Blurred details
} \\ 
    % Time-Aware RoPE 
    \multirow{2}{*}{\makecell[t]{\small\textbf{NTK}}} &
    \begin{minipage}{0.75\textwidth}
     %\vspace{.1cm}
     \centering
    \includegraphics[width=0.95\textwidth, height=50pt]{images/challenge/TA_vid.pdf}
     %\vspace{.1cm}
     \end{minipage}
    & 
    \begin{minipage}{0.15\textwidth}
    \centering   
     %\vspace{.1cm}
    \includegraphics[width=0.68\textwidth]{images/challenge/TA_img.pdf}
     %\vspace{.1cm}
    \end{minipage}
    \\
     &  \small{(c) Temporal repetition} & \small{(f) Spatial repetition} \\ \bottomrule
    \end{tabular}
    }
    \caption{\textbf{Visualization of existing methods for 2$\times$ extrapolation in video and image generation.} The base models CogVideoX-5B~\cite{yang2024cogvideox} and Lumina-Next~\cite{zhuo2024lumina} are trained to sample videos of up to 49 frames and images of up to 1K resolution, respectively. Existing methods lead to \textit{temporal repetition} or \textit{slower motion} in video extrapolation and \textit{spatial repetition} or \textit{blurred content} in image extrapolation, respectively. Please refer to Appendix~\ref{sec: existing failure} for more results and details. 
    }
    \label{fig:challenge}
    %%\vspace{-0.5cm}
\end{figure*}
% \subsection{Results on PPAC Optimization with 2 Silicon Dies}
Table~\ref{tab:2die_optimization} shows the results for the three techniques: Baseline, MMFP-SA and MMFP-RL when the number of dies $\mathcal{D}$ is 2 (one die with 7nm, and the other with 45nm). The results indicate that MMFP-SA achieves average reductions of 6\%, 8\% and 3.6\%, while MMFP-RL achieves reductions of 8.8\%, 11.7\% and 5.7\% in area, HPWL and cost, respectively, compared to the baseline. In terms of post-placement TNS, MMFP-SA and MMFP-RL show average improvements of 17.2\% and 21.7\% respectively. Moreover, MMFP-SA and MMFP-RL achieve average savings in dynamic power of 6.4\% and 8.1\% respectively. In terms of CPU runtime, MMFP-RL is on average 0.8$\times$ slower, and MMFP-SA is 0.9$\times$ slower than the baseline. However, MMFP-RL is faster than MMFP-SA and the baseline as the number of interconnected blocks $\mathcal{B}$ increases. Figure~\ref{fig:objective_iteration} shows the objective function $f$ across iterations during optimization on the \emph{netcard} design. MMFP-RL requires $293$ fewer iterations than MMFP-SA and achieves a better objective value.

\begin{figure}[ht] 
\centering
\includegraphics[width=.95\linewidth]{figures/objective_iterations1.png}
\caption{Objective function $f$ value for MMFP-SA/RL across iterations.}
\label{fig:objective_iteration}
\end{figure}



\subsection{Results on PPAC Optimization with 4 Silicon Dies}
%In heterogeneous integration, the number of dies $\mathcal{D}$ is typically more than 2. 
Table~\ref{tab:4die_optimization} shows the results for the \mbox{\emph{leon3-avnet}} design when the number of dies $\mathcal{D}$ is 4. The results show that MMFP-SA achieves average reductions of 7.2\%, 6.5\% and 3.2\% in area, HPWL and cost, respectively, compared to the baseline. MMFP-RL further improves the reductions achieving 9.7\%, 9\% and 4.3\%. Post-placement TNS is improved by 7.3\% for MMFP-SA, and 11.1\% for MMFP-RL. Furthermore, MMFP-SA achieves 7\% power savings, while MMFP-RL achieves 8.6\%. In terms of CPU runtime, MMFP-RL is 1.19$\times$ faster than the baseline, while MMFP-SA is 0.95$\times$ slower.

\begin{table*}[!ht]
\centering
\caption{PPAC optimization results for leon3-avnet design in 4 silicon dies.}
\begin{tabular}{cclrrrrrrr}
\hline
\multicolumn{2}{c}{\# dies} & \multirow{2}{*}{Method} & Area & HPWL & Cost & \multicolumn{2}{c}{Timing (ns)} & Power & CPU \\
\cline{1-2} \cline{7-8}
7nm & 45nm & & ($\times10^3 \mu$m$^2$) & ($\mu$m) & ($\times10^{-3}$) & TNS & WNS & (mW) & (sec) \\
\hline
\hline
\multirow{3}{*}{1} & \multirow{3}{*}{3} & Baseline & 1409.29 & 17495.01 & 3173 & -1253.06 & -1.948 & 1397.8 & 2104 \\
& & MMFP-SA & 1320.57 & 16029.53 & 3086 & -1171.95 & -1.872 & 1320.6 & 2308 \\
& & MMFP-RL & 1296.30 & 15702.68 & 3069 & -1128.43 & -1.830 & 1315.1 & 1963 \\
\hline
\multirow{3}{*}{2} & \multirow{3}{*}{2} & Baseline & 831.06 & 13972.70 & 2880 & -810.57 & -1.304 & 1150.3 & 2376 \\
& & MMFP-SA & 759.25 & 13295.18 & 2755 & -752.86 & -1.258 & 1051.3 & 2450 \\
& & MMFP-RL & 738.14 & 12960.91 & 2716 & -719.25 & -1.230 & 1016.3 & 2003 \\
\hline
\multirow{3}{*}{3} & \multirow{3}{*}{1} & Baseline & 590.71 & 9713.55 & 2674 & -603.74 & -1.071 & 674.9 & 2502 \\
& & MMFP-SA & 552.05 & 9101.84 & 2603 & -552.83 & -0.994 & 628.1 & 2581 \\
& & MMFP-RL & 531.92 & 8782.73 & 2568 & -530.06 & -0.976 & 619.3 & 1924 \\
\hline
\hline
& & Baseline & 1 & 1 & 1 & 1 & 1 & 1 & 1 \\
\multicolumn{2}{c}{Norm. Avg.}  & MMFP-SA & 0.928 & 0.935 & 0.968 & 0.927 & 0.951 & 0.930 & 0.950 \\
& & MMFP-RL & 0.903 & 0.910 & 0.957 & 0.889 & 0.931 & 0.914 & 1.186 \\
\hline
\end{tabular}
\label{tab:4die_optimization}
\end{table*}
\section{Evaluation}


\begin{table}[t]
    \centering
    % \vspace{-0.1in}
    \scalebox{0.78}{
    % \begin{small}
        \begin{tabular}{lccc}
            \toprule
            \multirow{2}*{\textbf{MoE Models}} & \textbf{Parameters} & \textbf{Experts Per Layer} & \textbf{Num. of} \\
            & \textbf{(active / total)} & \textbf{(active / total)} & \textbf{Layers} \\
            \otoprule 
            \mixtral~\cite{jiang2024mixtral} & 12.9B / 46.7B & 2 / 8 & 32 \\
            % \hline
            \qwen~\cite{yang2024qwen2} & 2.7B / 14.3B & 4 / 60 & 24 \\
            \phimoe~\cite{abdin2024phi} & 6.6B / 42B & 2 / 16 & 32 \\
            \bottomrule 
        \end{tabular}
    % \end{small}
    }
    \caption{Characteristics of three \MoE models in evaluation.}
    \vspace{-0.2in}
    \label{table:eval-moe-models}
\end{table}








\subsection{Experimental Setup}
\label{subsec:eval-setup}


% \begin{figure*}[t]
%     \centering
%     \begin{subfigure}[t]{0.48\textwidth}
%         \centering
%         \includegraphics[width=.9\linewidth]{figs/eval-overall-lmsys.pdf}
%         \caption{Serving three \MoE models with LMSYS-Chat-1M dataset.}
%     \end{subfigure}
%     \begin{subfigure}[t]{0.48\textwidth}
%         \centering
%         \includegraphics[width=.9\linewidth]{figs/eval-overall-sharegpt.pdf}
%         \caption{Serving three \MoE models with ShareGPT dataset.}
%     \end{subfigure}
%     \caption{Overall performance of prefill and decode stages for \sys and other four baselines.}
%     \label{fig:eval-overall.pdf}
% \end{figure*}


\noindent \textbf{Testbed.}
We conduct all experiments on a six-GPU testbed, where each GPU is an NVIDIA GeForce RTX 3090 with 24 GB GPU memory. 
%
All GPUs are inter-connected using pairwise NVLinks and connected to the CPU memory using PCIe 4.0 with 32GB/s bandwidth. 
%
Additionally, the testbed has a total of 32 AMD Ryzen Threadripper PRO 3955WX CPU cores and 480 GB CPU memory.


\noindent \textbf{Models.}
We employ three popular \MoE-based \LLMs in our evaluation: \mixtral~\cite{jiang2024mixtral}, \qwen~\cite{yang2024qwen2}, and \phimoe~\cite{abdin2024phi}.
Table~\ref{table:eval-moe-models} describes the parameters, number of \MoE layers, and number of experts per layer for the three models.
Following the evaluation of existing works~\cite{song2024promoe}, we profile the models to set the optimal prefetch distance $d$ to three before evaluation.
% We set $d$ of \mixtral, \qwen, and \phimoe to \todo{$xxx$}, \todo{$xxx$}, and \todo{$xxx$}, respectively.


\noindent \textbf{Datasets and traces.}
We employ two real-world prompt datasets commonly used for \LLM evaluation: LMSYS-Chat-1M~\cite{zheng2023lmsys} and ShareGPT~\cite{sharegpt}.
%
For most experiments, we split the sampled datasets in a standard 7:3 ratio, where 70\% of the prompts' context data (\ie, semantic embeddings and expert maps) are stored in \sys's Expert Map Store, and 30\% of the prompts are used for testing. 
%
For online serving experiments, we empty the Expert Map Store and use real-world \LLM inference traces~\cite{patel2024splitwise,stojkovic2025dynamollm} released by Microsoft Azure to set input and generation lengths and drive invocations.

\noindent \textbf{Baselines.}
We compare \sys against four \SOTA \MoE serving baselines:
1) \textbf{MoE-Infinity}~\cite{xue2024moe} uses coarse-grained request-level expert activation patterns and synchronous expert prediction and prefetching for \MoE serving. 
We prepare the expert activation matrix collection for MoE-Infinity before evaluation for a fair comparison.
%
% However, the open-sourced MoE-Infinity codebase~\cite{moe-infinity-code} lacks some features described in its original paper, we had to modify
%y 
2) \textbf{ProMoE}~\cite{song2024promoe} employs a stride-based speculative expert prefetching approach for \MoE serving. Since the codebase of ProMoE is not open-sourced and requires training predictors for each \MoE model, we reproduced a prototype of ProMoE on top of MoE-Infinity in our best effort.
%
3) \textbf{Mixtral-Offloading}~\cite{eliseev2023fast} combines a layer-wise speculative expert prefetching and a \LRU-based expert cache. 
%
4) \textbf{DeepSpeend-Inference} employs an expert-agnostic layer-wise parameter offloading approach, which uses pure on-demand loading and does not support prefetching. 
%
We implement the offloading logic of DeepSpeed-Inference in the MoE-Infinity codebase and add an expert cache for a fair comparison.
We enable all baselines to serve \MoE models from HuggingFace Transformer~\cite{wolf2020huggingface}. 


\noindent \textbf{Metrics.}
Following the standard evaluation methodology of existing works~\cite{song2024promoe,xue2024moe,zhong2024distserve,agrawal2024taming} on \LLM serving, we report the performance of the prefill and decode stages separately. 
We measure Time-to-First-Token (TTFT) for the prefill stage and Time-Per-Output-Token (TPOT) for the decode stage.
Additionally, we also report other system metrics, such as expert hit rate and overheads, for detailed evaluation.


% \noindent \textbf{\sys's setting.}
% The hyperparameters of \sys containing the prefetch distance $d$ for each \MoE model, Expert Map Store capacity $C$, and Expert Cache memory limit $M$.
% For most experiments, we profile the \MoE models and set the prefetch distance $d$ to their optimal values. The Expert Map Store capacity $C$ is set to \todo{$xxx$} expert maps. We configure the Expert Cache memory limit to \todo{$xxx$} GB.
% The hyperparameter sensitivity is analyzed in \S\ref{subsec:eval-sensitivity}.


\begin{figure}[t]
  \centering
  \includegraphics[width=.95\linewidth]{figs/eval-overall-arxiv.pdf}
  \vspace{-0.15in}
  \caption{Overall performance of prefill and decode stages for \sys and other four baselines.}
  \vspace{-0.2in}
  \label{fig:eval-overall}
\end{figure}


\subsection{Overall Performance}
\label{subsec:eval-overall}



We first evaluate the performance of prefill and decode stages when running \sys and other baselines with the three \MoE models, where we measure Time-To-First-Token (TTFT) and Time-Per-Output-Token (TPOT) for each stage.
Note that the inference latency with expert offloading tends to be higher than no offloading due to two reasons: 
1) During inference, an excessive amount of parameters in \MoE models are loaded and offloaded, which prolongs the inference latency.
2) All baselines and \sys are implemented on top of the MoE-Infinity codebase~\cite{moe-infinity-code}, whose inference latency is inherently impacted by MoE-Infinity's implementation.
Nevertheless, comparing \sys and baselines is fair with the same experimental setup.

Figure~\ref{fig:eval-overall} shows the \TTFT, \TPOT, and expert hit rate of \sys and other four baselines when serving three \MoE models with LMSYS-Chat-1M and ShareGPT datasets, respectively.
DeepSpeed has both the worst \TTFT and \TPOT due to expert-agnostic offloading and lacking expert prefetching.
While Mixtral-Offloading, ProMoE, and MoE-Infinity perform better than DeepSpeed-Inference, they are underperformed by \sys because of coarse-grained offloading designs.
Compared to DeepSpeed-Inference, Mixtral-Offloading, ProMoE, and MoE-Infinity, our \sys reduces the average \TTFT by 44\%, 35\%, 33\%, 30\%, and reduces the average \TPOT by 70\%, 61\%, 55\%, 48\%, across three \MoE models.
%
% Figure~\ref{fig:eval-overall} also reports the expert hit rate of \sys and each baseline. 
For expert hit rate, Mixtral-Offloading achieves a higher hit rate than the other three baselines because of its synchronous speculative prefetching with a prefetch distance of 1. However, due to synchronous prefetching, its \TTFT and \TPOT are worse than others except DeepSpeed-Inference.
\sys improves the average expert hit rate by 147\%, 11\%, 34\%, and 63\% over DeepSpeed-Inference, Mixtral-Offloading, ProMoE, and MoE-Infinity, respectively.

% \begin{figure}[t]
%   \centering
%   \includegraphics[width=.9\linewidth]{figs/eval-overall-sharegpt.pdf}
%   % \vspace{-0.15in}
%   \caption{}
%   % \vspace{-0.25in}
%   \label{fig:eval-overall-sharegpt.pdf}
% \end{figure}




\subsection{Online Serving Performance}
\label{subsec:eval-online}


Except for the offline evaluation (\ie, Expert Map Store in full capacity before serving), we also evaluate \sys against other baselines in online serving settings.
We empty the Expert Map Store of \sys and the expert activation matrix collection of MoE-Infinity for the online serving experiment.
%
The request traces are derived from Azure \LLM inference traces~\cite{patel2024splitwise,stojkovic2025dynamollm}, with 64 requests randomly sampled to drive LMSYS-Chat-1M prompts for each \MoE model serving. 
To ensure consistency, \sys and all baselines input and generate the exact number of tokens specified in the traces.
%
Figure~\ref{fig:eval-online-serve} illustrates the CDF of end-to-end request latency across three \MoE models. The results demonstrate that \sys significantly reduces overall request latency compared to other baselines in online serving scenarios.


\begin{figure}[t]
  \centering
  \includegraphics[width=.95\linewidth]{figs/eval-online-serve-arxiv.pdf}
  \vspace{-0.15in}
  \caption{CDF of request latency for \MoE online serving.}
  \vspace{-0.2in}
  \label{fig:eval-online-serve}
\end{figure}



\subsection{Impact of Expert Cache Limits}



We measure the \TPOT of \sys and other baselines by limiting the expert cache memory budget to investigate their performance in the latency-memory trade-off (\S\ref{subsec:bg-latency-memory-tradeoff}).
We mainly focus on \TPOT to show the end-to-end performance impacted by varying cache limits.
Figure~\ref{fig:eval-cache-limit.pdf} shows the \TPOT of \sys and other four baselines when serving three \MoE models under different expert cache limits.
We gradually increase the GPU memory allocated for caching experts from 6 GB to 96 GB while employing the same experimental setting in \S\ref{subsec:eval-overall}.
Similarly, DeepSpeed-Inference has the worst \TPOT due to being expert-agnostic.
\sys consistently outperforms Mixtral-Offloading, ProMoE, and MoE-Infinity under varying expert cache limits.
Especially for limited GPU memory sizes (\eg, 6GB), \sys reduces the \TPOT by 32\%, 24\%, 18\%, and 18\%, compared to DeepSpeed-Inference, Mixtral-Offloading, ProMoE, and MoE-Infinity, across three \MoE models, respectively.
With fine-grained expert offloading, \sys significantly reduces the expert on-demand loading latency while maintaining a lower GPU memory footprint, therefore achieving a better spot in the latency-memory trade-off of \MoE serving.

% \subsection{Impact of Inference Batch Size}

\subsection{Ablation Study}
\label{subsec:eval-ablation}


% \begin{figure}[t]
%   \centering
%   \includegraphics[width=.95\linewidth]{figs/eval-expert-tracking.pdf}
%   % \vspace{-0.15in}
%   \caption{Expert hit rate of different expert pattern tracking approaches.}
%   % \vspace{-0.25in}
%   \label{fig:eval-expert-tracking}
% \end{figure}



We present the ablation study of \sys's design.


\textbf{Effectiveness of expert map search.}
One of \sys's key designs is the expert map, which tracks expert selection preferences in fine granularity.
We evaluate the effectiveness of the expert map against five expert pattern-tracking approaches as follows.
%
1) \textbf{Speculate}: speculative prediction used by Mixtral-Offloading~\cite{eliseev2023fast} and ProMoE~\cite{song2024promoe}, 
%
2) \textbf{Hit count}: request-level expert hit count used by MoE-Infinity~\cite{xue2024moe}, 
%
3) \textbf{Map (T)}: expert map with only trajectory similarity search,
4) \textbf{Map (T+S)}: expert map with both trajectory and semantic similarity search,
%
and
5) \textbf{Map (T+S+$\delta$)}: expert map with full features enabled, including trajectory and semantic similarity search (\S\ref{subsec:design-similarity-match}) and dynamic expert selection (\S\ref{subsec:design-expert-prefetch}).
%
We implement the above methods in \sys's Expert Map Matcher for a fair comparison.
Figure~\ref{fig:eval-expert-tracking} shows the expert hit rate of the above expert pattern tracking methods.
%
Speculative prediction is effective due to the widespread presence of residual connections in Transformer blocks. However, its effectiveness decreases drastically as prefetch distance increases~\cite{song2024promoe}.
%
The request-level expert activation count has the worst performance due to coarse granularity.
%
As features are incrementally restored to \sys's expert map, the expert hit rate gradually increases, demonstrating its effectiveness.

% \textbf{Effectiveness of asynchronous map matching.}




\begin{figure}[t]
  \centering
  \includegraphics[width=.9\linewidth]{figs/eval-cache-limit-arxiv.pdf}
  \vspace{-0.15in}
  \caption{Performance of \sys and other four baselines under varying expert cache limits.}
  \vspace{-0.1in}
  \label{fig:eval-cache-limit.pdf}
\end{figure}

\begin{figure}[!t]
    \centering
    \begin{subfigure}[t]{0.585\linewidth}
        \centering
        \includegraphics[width=\linewidth]{figs/eval-expert-tracking.pdf}
        \caption{Expert pattern tracking approaches.}
        \label{fig:eval-expert-tracking}
    \end{subfigure}
    % \hspace{0.02in}
    \begin{subfigure}[t]{0.385\linewidth}
        \centering
        \includegraphics[width=\linewidth]{figs/eval-prefetch-and-cache-arxiv.pdf}
        \caption{Prefetch and caching.}
        \label{fig:eval-prefetch-and-cache}
    \end{subfigure}
    \vspace{-0.1in}
    \caption{Ablation study of \sys.}
    \label{fig:eval-ablation}
    \vspace{-0.2in}
\end{figure}

\textbf{Effectiveness of expert prefetching and caching.}
We evaluate \sys's expert prefetching and caching against two caching algorithms:
1) \textbf{\LRU} used by Mixtral-Offloading~\cite{eliseev2023fast}
and 
2) \textbf{\LFU} used by MoE-Infinity~\cite{xue2024moe}.
%
Figure~\ref{fig:eval-prefetch-and-cache} depicts the expert hit rate of \sys and two baselines.
The results show that \LRU performs poorly in expert offloading scenarios. Though \LFU achieves a higher hit rate than \LRU, \sys surpasses both, achieving the highest expert hit rate.

\subsection{Sensitivity Analysis}
\label{subsec:eval-sensitivity}


\begin{figure}[t]
  \centering
  \includegraphics[width=.9\linewidth]{figs/eval-prefetch-distance.pdf}
  \vspace{-0.15in}
  \caption{Performance of \sys serving \MoE models with different prefetch distances.}
  \vspace{-0.1in}
  \label{fig:eval-prefetch-distance}
\end{figure}

% \begin{figure}[t]
%   \centering
%   \includegraphics[width=.9\linewidth]{figs/eval-store-capacity.pdf}
%   % \vspace{-0.15in}
%   \caption{Semantic and trajectory similarity lower bounds in \sys's serving with different Expert Map Store capacity.}
%   % \vspace{-0.25in}
%   \label{fig:eval-store-capacity}
% \end{figure}

\begin{figure}[t]
    \centering
    \begin{subfigure}[t]{0.55\linewidth}
        \centering
        \includegraphics[width=\linewidth]{figs/eval-store-capacity.pdf}
        \caption{Expert Map Store capacity.}
        \label{fig:eval-store-capacity}
    \end{subfigure}
    % \hspace{0.02in}
    \begin{subfigure}[t]{0.435\linewidth}
        \centering
        \includegraphics[width=\linewidth]{figs/eval-batch-size-arxiv.pdf}
        \caption{Inference batch size.}
        \label{fig:eval-batch-size}
    \end{subfigure}
    \vspace{-0.1in}
    \caption{Sensitivity analysis of \sys.}
    \vspace{-0.2in}
    \label{fig:eval-sensitivity}
\end{figure}


We analyze the sensitivity of three hyperparameters: prefetch distance of \MoE models, the capacity of Expert Map Store, and inference batch size.


\textbf{Prefetch distance of \MoE models.}
Figure~\ref{fig:eval-prefetch-distance} shows the \TTFT and \TPOT of \sys when serving three \MoE models with different prefetch distances.
%
We have demonstrated that the expert hit rate decreases when gradually increasing the prefetch distance (Figure~\ref{fig:bg-hit-distance}).
%
When the prefetch distance is small ($<3$), \sys cannot perfectly hide its system delay from the inference process, such as the map matching and expert prefetching, leading to the increase of inference latency.
%
With larger prefetch distances ($>3$), \sys has worse expert hit rates that also degrade the performance. 
Therefore, we set the prefetch distance $d$ to 3 for evaluating \sys.


\textbf{Capacity of Expert Map Store.}
We measure the mean semantic and trajectory similarity scores searched in \sys's expert map matching for \MoE model serving.
%
Figure~\ref{fig:eval-store-capacity} presents the mean semantic and trajectory similarity scores of \sys with different Expert Map Store capacity sizes.
%
Both semantic and trajectory similarity scores improve as the store capacity increases.
%
While the similarity scores exhibit a significant increase with capacities below 1K, further capacity expansion yields diminishing similarity gains. 
To minimize \sys's memory overhead, we set \sys's Expert Map Store capacity to 1K in evaluation.


\textbf{Inference batch size.}
We investigate the impact of inference batch size on \sys and three baselines using \mixtral with LMSYS-Chat-1M.
%
Figure~\ref{fig:eval-batch-size} presents the performance of \sys, Mixtral-Offloading, ProMoE, and MoE-Infinity as the batch size increases from one to four. \sys achieves the lowest \TTFT and \TPOT in most cases.


% \textbf{Inference batch size.}


% \subsection{Scalability}
% \label{subsec:eval-scalability}
% From one to six GPUs


\begin{figure}[t]
  \centering
  \includegraphics[width=.92\linewidth]{figs/eval-overhead-latency.pdf}
  \vspace{-0.15in}
  \caption{Latency breakdown of \sys's one inference iteration with three \MoE models.}
  \vspace{-0.1in}
  \label{fig:eval-overhead-latency.pdf}
\end{figure}





\subsection{System Overheads}
\label{subsec:eval-overhead}


\noindent \textbf{Latency overheads of \sys's operations.}
Figure~\ref{fig:eval-overhead-latency.pdf} shows the latency breakdown of one inference iteration in \sys when serving the three \MoE models.
We report any operations of \sys in \S\ref{subsec:eval-overall} that may incur a significant latency delay, including context collection, map matching, expert on-demand loading, expert prefetching, and map update after the iteration completes.
\qwen has lower end-to-end iteration latency than \mixtral and \phimoe because of significantly fewer parameters.
Note that expert prefetching, map matching, and map update tasks are executed asynchronously, aside from the inference process. Hence, they do not contribute to the end-to-end iteration latency.
Excluding three asynchronous tasks, the total delay incurred by other operations is consistently less than 30ms (5\% of the iteration) across three \MoE models, which is negligible compared to the inference latency.


\noindent \textbf{Memory overheads of \sys's Expert Map Store.}
Figure~\ref{fig:eval-overhead-memory.pdf} shows the CPU memory footprint of \sys's Expert Map Store when varying the store capacity from 1K to 32K maps.
The memory needed to store expert maps for \qwen is more than \mixtral and \phimoe because it has more experts per layer over the other two models, which increases the map shape.
Even for the largest capacity (32K), the Expert Map Store requires less than 200MB of memory to store the maps, which is trivial since modern GPU servers usually have abundant CPU memory (\eg, p4d.24xlarge on AWS EC2~\cite{aws-ec2} has over 1100 GB of CPU memory).
In the evaluation, \sys's map store capacity with 1K maps is sufficient for maintaining performance (\S\ref{subsec:eval-sensitivity}), resulting in minimal memory overhead.



\begin{figure}[t]
  \centering
  \includegraphics[width=.85\linewidth]{figs/eval-overhead-memory.pdf}
  % \vspace{-0.1in}
  \caption{CPU memory footprint of \sys's Expert Map Store with different capacity.}
  \vspace{-0.1in}
  \label{fig:eval-overhead-memory.pdf}
\end{figure}

% 
\section{Related Work} \label{sec:related}

% \textbf{Adversarial Attack}
\textbf{Attacks on SLAM.} 
%With the rise of machine learning, 
The robustness of computer vision systems is being actively investigated. With the emergence of adversarial images in the digital domain by adding optimized noise directly to images~\cite{szegedy2013intriguing,carlini2017towards}, researchers find that such attacks also exist physically in the real world \cite{eykholt2018robust,song2018physical,zhao2019seeing}. To fill the gap between attacks in the digital and physical worlds, recent studies have demonstrated that attacks on real-world computer vision systems are practical \cite{eykholt2018robust,li2019adversarial,man2020ghostimage,sharif2016accessorize,zhao2019seeing,zhou2018invisible}. However, attacks on traditional computer vision methods such as SLAM are relatively less explored. \cite{yoshida2022adversarial} proposes an attack against the scan matching algorithm in LiDAR-based SLAM, while most SLAMs in AR/VR devices rely on different sensors like RGB/depth cameras and IMUs. \cite{ikram2022perceptual} and \cite{chen2024adversary} mislead visual SLAM by poisoning the images with special patterns, and \cite{wang2021can} causes the camera to fail using infrared light. In our work, we demonstrate attacks on Visual-Inertial SLAM (VI-SLAM) by perturbing the IMU readings, rather than cameras, and showing its impact on XR user experience. 

\textbf{Acoustic Injection Attacks.} Among various physical attacks, acoustic injection attacks are attractive due to their low cost. Son~\etal~\cite{son2015rocking} were the first to introduce acoustic attacks on MEMS gyroscopes, demonstrating how these attacks could lead to sensor denial-of-service and result in drone crashes. WALNUT~\cite{trippel2017walnut} expanded on this by developing output biasing and control attacks that enable precise manipulation of MEMS accelerometer outputs using modulated sound waves. Wang et al.~\cite{wang2017sonic} demonstrated a sonic gun, showcasing the vulnerability of various smart devices (\eg drones and self-balancing vehicles) to acoustic attacks. Tu et al. \cite{tu2018injected} designed side-swing and switching attacks to alter the outputs of MEMS gyroscopes and accelerometers. Furthermore, Ji et al. \cite{ji2021poltergeist} fool the object detectors by applying acoustic attack to the image stabilizers commonly used in modern cameras. However, none of the existing works study the relationship between the acoustic injections and SLAM outputs on recent XR devices. 

% \zijian{Do we need one session about security in AR/VR?}
% \yicheng{TODO}
%\jiasi{cite the AIVR paper (UMass Amherst?) paper is we have not already. They add IMU perturbation but w/o SLAM, iirc} \yicheng{Cited}

\textbf{XR Security and Privacy.} 
%Security and privacy concerns in XR systems have gained significant attention. 
For single-user XR systems, researchers have demonstrated various side-channel attacks to extract sensitive information (\eg keystrokes) through video feeds~\cite{ling2019know}, head movements~\cite{nair2023unique, slocum2023going}, architectural hints~\cite{zhang2023its,shang2020arspy}, power usage~\cite{li2024dangers}, and EM side-channel leakages~\cite{al2021vr}. In multi-user XR systems, Su et al.~\cite{su2024remote} use avatar motion data to infer keystrokes in shared VR environments. Slocum et al.~\cite{slocum2024doesn} reveal vulnerabilities in the shared state frameworks of multi-user AR. Similarly, Lebeck et al.~\cite{lebeck2017securing} highlight risks like deceptive virtual objects and emphasize access control for managing shared physical and virtual spaces. Ruth et al.~\cite{ruth2019secure} further propose a secure multi-user AR framework focusing on content sharing and permissions.
Chandio et al.~\cite{chandio2024stealthy} %introduced a multi-modal spatiotemporal attack that 
simultaneously manipulated visual and inertial sensors to disrupt XR pose estimation. However, their study evaluated the attack using offline datasets and assumed the attacker's capability to manipulate IMU data streams through acoustic means, without real experiments. Ours is the first to demonstrate acoustic injection attacks on recent XR devices, like the Hololens 2, in the real world.
 




In this paper we have described our efforts in mechanizing the strand spaces framework~\cite{FHG98} in Coq.
To assess the flexibility of the approach and the usability of the library and of the proofs we have analyzed a variety of examples: a basic authentication protocol and some of its variants, the classical Needham-Schroeder-Lowe authentication protocol, and a recent key management API equipped with a key management policy.

Wherever possible, our mechanization remains faithful to the original pen-and-paper development of strand spaces.
At the same time, we put a lot of engineering effort to make the code and the proofs reusable.
For that, we have made the framework modular and parametric in the terms and the penetrator.
Additionally, we have developed a number of strands-specific tactics whose goal is to make the life of the protocol's analyst easier by removing some of the burden of these kinds of proofs.
Indeed, the tactics automate a number of intermediate steps enabling, in some cases, easy proof reuse.
For instance, the proof of the NSL responder's nonce secrecy
 required just one hour of work using the initiator's nonce secrecy.
The mechanization
gives the freedom to experiment with protocols and their properties, while retaining the unique ability of strand spaces-based analyses to give interesting insights on the inner workings of protocols.
With our experiments, we uncovered
and fixed issues, discarded
redundant or unused requirements, and significantly improved previous results on the analysis of key management policies, making it possible to formally prove the security of the \emph{secure templates} policy from \cite{BCFS-ccs10} (\cref{sec:casestudies}).

\cref{tab:simpleauth,tab:nsl} in \cref{sec:summary}  summarize the premises for each security property across the analyzed protocol variants. These premises are essential for our security proofs and offer important insights into the assumptions required to make a security protocol correct. The strand spaces model highlights this aspect, and the use of Coq and the \easystrands{} library further clarifies the minimal and necessary nature of these assumptions, reinforcing the model's ability to accurately capture security requirements.
With the insights from these experiments we also developed a new proof technique which we call \emph{protected predicate} technique that, in certain situations, simplifies the proofs making some previously challenging cases trivial.


Another advantage of having this mechanized platform is that it opens up new and interesting avenues of research.
\ifdefined\COLORDIFF
    \color{cbred}
\else
\fi
For instance, an intriguing enhancement to our framework would be the inclusion of algebraic intruders. We believe they can be implemented using at least two approaches, which we briefly outline below.

Given an equational theory $E$ over a signature $\mathit{FS}$, the first approach requires implementing $E$ as a (terminating and confluent) rewriting system \lstinline{rew_E}, and allow penetrators to use \lstinline{rew_E} to manipulate terms containing symbols of $\mathit{FS}$.
More concretely, we first need to create an instance of \easystrands{} terms with support for function symbols in $\mathit{FS}$, then we can extend the penetrator as:
\begin{lstlisting}
Inductive penetrator_strand : Σ -> Prop := ...
| PT_Eqn : forall (g h : 𝔸) i, replace g h rew_E  -> penetrator_strand (i, [⊖ g; ⊕ h]).
\end{lstlisting}
where \lstinline{replace g h rew_E} holds iff \lstinline{g} can be rewritten as \lstinline{h} under \lstinline{rew_E}.
This approach is inspired by that of Tamarin \cite{MSCB13}.

The second approach aligns  with the method used in DY*~\cite{DY}, where cryptographic primitives are modeled as functions that symbolically represent the actual primitives, e.g., \lstinline{dec (c, k) = (if c = enc (m, k) then m else Error)}.
With these definitions, the equational theory $E$ could be defined using Coq Setoids and used for terms in place of Leibniz equality.
This has the advantage to allow both honest parties and the intruder to transparently use the equational theory.
However, as observed by~\citet{DY}, this approach requires proving (at least) that $E$ is an equivalence relation respected by all functions, predicates, and protocol specifications which can be lengthy and tedious.
\ifdefined\COLORDIFF
    \color{black}
\else
\fi

Despite their age, strand spaces have been a catalyst for extensive research, leading to notable extensions that include authentication tests~\cite{guttman2000authentication}, process algebraic-style choice operators~\cite{YEMMS16},
 compositionality \cite{StrandComposition,StrandIndependence,StrandMixed}, and stateful protocols \cite{J12}.
Many of these advancements are crucial for enhancing the expressiveness and usability of the model.
Our plan is to enhance \easystrands{} by integrating these extensions, thereby enabling scalability to more realistic protocols.
Ultimately, this will help narrow the gap with state-of-the-art tools such as DY* \cite{DY}.
In terms of foundational research, an intriguing avenue involves closely examining the relationship between Paulson's inductive method \cite{Paulson94} and strand spaces. We plan to mechanize Paulson's method in Coq and conduct a comparative analysis to assess the relative merits of these two inductive methods.

Finally, we defined a maximal penetrator as the set of strands that do not violate sensitive cryptographic operations required for protocol security. This method is inspired by the approach in \cite{banaSymbolic} to achieve computational soundness and, to our knowledge, has not been explored in a purely symbolic context before. It allows for proving injective agreement without explicitly defining the Dolev-Yao attacker, which we showed to be \diff{strictly} subsumed by the maximal penetrator. Notably, this approach facilitates the composition of protocols proven secure under their respective maximal penetrators, provided they adhere to each other's constraints. We are currently extending this technique to protocols like NSL, where security relies on decryption capabilities.


\section*{Acknowledgements}
This is acknowledgment.

% \bibliographystyle{IEEEtran}
\documentclass[10pt,journal,compsoc]{IEEEtran}
\IEEEoverridecommandlockouts

% *** CITATION PACKAGES ***
%
\ifCLASSOPTIONcompsoc
  % The IEEE Computer Society needs nocompress option
  % requires cite.sty v4.0 or later (November 2003)
  \usepackage[nocompress]{cite}
\else
  % normal IEEE
  \usepackage{cite}
\fi
\usepackage{amsmath,amssymb,amsfonts}
\usepackage{algorithmic}
\usepackage{graphicx}
\usepackage{textcomp}
\usepackage[absolute]{textpos}


\usepackage{enumitem}
\usepackage{color}
\usepackage{multirow}
\usepackage{lipsum}
\usepackage{listings}
% \usepackage{subfigure}
\usepackage{graphicx}
\usepackage{pdfpages}
\usepackage{verbatim}
\usepackage{pifont}
\usepackage{etoolbox}
\usepackage{textcomp}
% \usepackage{ulem}
\usepackage{hyperref}
\usepackage{url}
\usepackage{caption}
\usepackage{booktabs} % For nice tables
\usepackage{siunitx} % To align table numbers by unit
\usepackage{etoolbox}

% \usepackage{booktabs}
% \usepackage{multirow}
% \usepackage{graphicx}
\usepackage[normalem]{ulem}
\useunder{\uline}{\ul}{}

\usepackage[switch]{lineno} % default option is 'left'

\usepackage[linesnumbered,titlenumbered,ruled,vlined,resetcount,algosection]{algorithm2e}
\usepackage{xcolor}
\usepackage{color}
\def\BibTeX{{\rm B\kern-.05em{\sc i\kern-.025em b}\kern-.08em
    T\kern-.1667em\lower.7ex\hbox{E}\kern-.125emX}}

\usepackage{amsthm}
\newtheorem{theorem}{Theorem}
%\newtheorem{theorem}{Theorem}[section]
%\newtheorem{corollary}{Corollary}[theorem]
\newtheorem{corollary}{Corollary}
%\newtheorem{lemma}[theorem]{Lemma}
\newtheorem{lemma}{Lemma}

\usepackage{hhline}
\usepackage{xcolor}
\usepackage{colortbl}
\usepackage{threeparttable}
% \usepackage{minted}
\usepackage{subfig}
\usepackage{comment}
\usepackage{enumitem}
% \usepackage[ruled,vlined]{algorithm2e}
% Circled number
\usepackage{tikz}
\newcommand*\circled[1]{\tikz[baseline=(char.base)]{
            \node[shape=circle,fill,inner sep=0.8pt] (char) {\footnotesize\textcolor{white}{#1}};}}
\newcommand*\redcircled[1]{\tikz[baseline=(char.base)]{
            \node[shape=circle,fill=red,inner sep=1.3pt] (char) {\textcolor{white}{#1}};}}
            
\def\BibTeX{{\rm B\kern-.05em{\sc i\kern-.025em b}\kern-.08em
    T\kern-.1667em\lower.7ex\hbox{E}\kern-.125emX}}

% \setlength{\floatsep}{1em}
% \setlength{\textfloatsep}{0.8em}
% \setlength{\intextsep}{0.1em}
% \setlength{\belowcaptionskip}{0.2em} 

\newcommand{\yguo}[1]{
  \begin{framed}
    \noindent{\textcolor{blue}{\bf\em #1 -- Yanfei Guo}}
  \end{framed}
}

\newcommand{\yujia}[1]{\textcolor{red}{#1}} % Track changes from Yujia

\newcommand{\fn}[1]{{\tt\small #1}}
%%

% \newcommand{\pname}{\textit{gZCCL}}
\newcommand{\pname}{\mbox{ZCCL}}
% \newcommand{\pname}{\mbox{\textbf{\texttt{ZCCL}}}}
\newcommand{\newtext}[1]{\textcolor{blue}{#1}}

\newcommand{\fzlight}{\mbox{\textit{fZ}-light}}


\begin{document}

\Urlmuskip=0mu plus 1mu


\title{ZCCL: Significantly Improving Collective Communication With Error-Bounded Lossy Compression}

\author{Jiajun Huang, Sheng Di,~\IEEEmembership{Senior Member,~IEEE}, Xiaodong Yu, Yujia Zhai, Zhaorui Zhang, Jinyang Liu, Xiaoyi Lu, Ken Raffenetti, Hui Zhou, Kai Zhao, Khalid Alharthi, Zizhong Chen,~\IEEEmembership{Senior Member,~IEEE}, Franck Cappello,~\IEEEmembership{Fellow,~IEEE}, Yanfei Guo, Rajeev Thakur,~\IEEEmembership{Fellow,~IEEE}
% Xiaodong Yu,\IEEEauthorrefmark{3}
% Yujia Zhai,\IEEEauthorrefmark{1}
% Zhaorui Zhang,\IEEEauthorrefmark{4}
% Jinyang Liu,\IEEEauthorrefmark{1}
% Xiaoyi Lu,\IEEEauthorrefmark{5}
% Ken Raffenetti\IEEEauthorrefmark{2}
% Hui Zhou\IEEEauthorrefmark{2}
% Kai Zhao\IEEEauthorrefmark{6}
% Zizhong Chen\IEEEauthorrefmark{1}
% Franck Cappello\IEEEauthorrefmark{2}
% Yanfei Guo\IEEEauthorrefmark{2}
% Rajeev Thakur\IEEEauthorrefmark{2}

\IEEEcompsocitemizethanks{
\IEEEcompsocthanksitem Jiajun Huang, Yujia Zhai, and Zizhong Chen are affiliated with the University of California, Riverside, CA 92521. Sheng Di, Ken Raffenetti, Hui Zhou, Franck Cappello, Yanfei Guo, and Rajeev Thakur are affiliated with Argonne National Laboratory, Lemont, IL 60439. Xiaodong Yu is affiliated with Stevens Institute of Technology, Hoboken, NJ 07030. Zhaorui Zhang is affiliated with The Hong Kong Polytechnic University, Kowloon, Hong Kong. Jinyang Liu is affiliated with University of Houston,
Houston, TX 77204. Xiaoyo Lu is is affiliated with University of California, Merced, CA 95343. Kai Zhao is affiliated with Florida State University, Tallahassee, FL 32306. Khalid Alharthi is affiliated with Department Of Computer Science, College of Computing And Information Technology, University Of Bisha, Bisha 61922, P.O. Box 551, Saudi Arabia. 
% \IEEEcompsocthanksitem A shortened version of this paper was presented at 2024 IEEE International Parallel
% and Distributed Processing Symposium (IPDPS) and was published in its \href{https://doi.ieeecomputersociety.org/10.1109/IPDPS57955.2024.00072}{proceedings}.
}

}

% The paper headers
% \markboth{Journal of \LaTeX\ Class Files,~Vol.~14, No.~8, August~2021}%
% {Shell \MakeLowercase{\textit{et al.}}: A Sample Article Using IEEEtran.cls for IEEE Journals}

% \IEEEpubid{0000--0000/00\$00.00~\copyright~2021 IEEE}

% for Computer Society papers, we must declare the abstract and index terms
% PRIOR to the title within the \IEEEtitleabstractindextext IEEEtran
% command as these need to go into the title area created by \maketitle.
% As a general rule, do not put math, special symbols or citations
% in the abstract or keywords.
\IEEEtitleabstractindextext{%

\begin{abstract}



With the ever-increasing computing power of supercomputers and the growing scale of scientific applications, the efficiency of MPI collective communication turns out to be a critical bottleneck in large-scale distributed and parallel processing. The large message size in MPI collectives is particularly concerning because it can significantly degrade overall parallel performance. To address this issue, prior research simply applies off-the-shelf fixed-rate lossy compressors in the MPI collectives, leading to suboptimal performance, limited generalizability, and unbounded errors. In this paper, we propose a novel solution, called {\pname}, which leverages error-bounded lossy compression to significantly reduce the message size, resulting in a substantial reduction in communication costs. The key contributions are three-fold. (1) We develop two general, optimized lossy-compression-based frameworks for both types of MPI collectives (collective data movement as well as collective computation), based on their particular characteristics. Our framework not only reduces communication costs but also preserves data accuracy. (2) We customize {\fzlight}, an ultra-fast error-bounded lossy compressor, to meet the specific needs of collective communication. (3) We integrate {\pname} into multiple collectives, such as Allgather, Allreduce, Scatter, and Broadcast, and perform a comprehensive evaluation based on real-world scientific application datasets. Experiments show that our solution outperforms the original MPI collectives as well as multiple baselines by 1.9--8.9$\times$.

\end{abstract}

\begin{IEEEkeywords}
Error-bounded Lossy Compression, Collective Communication, Distributed Computing, Parallel Algorithm
\end{IEEEkeywords}
}
\maketitle

%%
%% This command processes the author and affiliation and title
%% information and builds the first part of the formatted document.
\maketitle

\documentclass[../main.tex]{subfiles}
\graphicspath{{../images/}}
\makeatletter
\def\input@path{{../images/}}
\makeatother
\begin{document}
\section{Introduction}
\begin{figure}
\centering
\begin{tikzpicture}
\node[inner sep=0pt] (ws) at (0, 0) {
\includegraphics[height=.4\textwidth, trim={10cm 0 10cm 0},clip]{world_space.png}};
\node[inner sep=0pt] (cs) at (6,0) {\includegraphics[height=.4\textwidth, trim={10cm 1cm 10cm 4cm},clip]{conf_space.png}};
\end{tikzpicture}
\vspace{-5pt}
\label{fig:pbrm_intro}
\caption{\textbf{Left}: Shows world space obstacles as grey spheres. Robots start and goal configuration is colored red and green, respectively. Configurations along the computed path are colored transparent blue. \textbf{Right:} Mapped world space scenario to configuration space. Obstacle region is the grey mesh. Red spheres are collision-free regions computed by the neural SCDF. The optimized shortest path in the convex corridor is the blue curve.}
\vspace{-25pt}
\end{figure}
Motion planning is the problem of finding a collision-free trajectory that connects a given start and goal configuration. The planning takes place in the configuration space of the robot. For single body robots, like mobile robots or drones, the configuration space and the world space are usually the same. This simplifies the planning, since explicit obstacle representations are available which enables geometrical tools like separating hyperplanes, smallest distance to obstacles etc., to be used when designing motion planning algorithms. For multi-body robots like manipulators, the situation is completely different. The world space obstacles are usually mapped to non-convex regions, and to make the problem even harder, the mapping is usually not known. Forming explicit representations of the obstacle region in the configuration space is usually too expensive or intractable. Despite all of this, sampling based planners are used with great success, which mainly is due to their use of implicit representations of the obstacle region. The basic idea is to construct a graph in the configuration space that covers and connects the collision-free region. From this graph, a path can be extracted that connects a given start and goal configuration. The approach is computationally expensive, since the graph is constructed with the smallest geometrical building block available, points, which represents a collision-check. Furthermore, the extracted paths from the graph are non-smooth and jagged due to the stochastic nature of the approach. This adds an additional post-processing step to the process, where the paths are shortcutted and smoothened, before the path can be used for tracking. Clearly a lot of time is invested to form this graph and produce smooth paths. Thus, if the obstacles start to move, then all of this work is done in no use, since all points that make up this graph need to be re-verified, which is simply too time consuming to be done in real time.
\\\\
In this work, we want to address the existing drawbacks of the sampling based planners. Our main contribution is an improved motion planner where each vertex in the graph covers a collision-free region in the form of a sphere instead of a point and where the edges are formed with neighboring intersecting spheres. This representation has the advantage of instead of returning piecewise linear paths, returning a sequence of overlapping spheres, i.e. a convex corridor, that connects a given start and goal configuration, illustrated in Figure \ref{fig:pbrm_intro}. This convex corridor allows us to use convex optimization to produce smooth trajectories, instead of computationally expensive post-processing methods. The representation further allows us to estimate the coverage of the collision-free space, which gives us awareness and feedback in the offline roadmap construction phase. Finally, our representation is simple to adapt to moving obstacles, simply requery for the new radii and recheck for intersections. 
\\\\
The spherical collision-free regions are formed using a signed distance function (SDF), which is a function that returns the smallest distance from an arbitrary point to the boundary of an obstacle. As the name implies, the distance is signed, thus if the point is inside the obstacle it is negative otherwise positive. If the distance is positive, a sphere with radius equal to the distance is guaranteed to cover a collision-free region. Using an SDF in motion planning is not new, but what is novel about our approach is that we express the distance in the configuration space instead of the world space and by doing so allows us to form these convex collision-free regions. We refer to the resulting SDF as a signed configuration distance function (SCDF). Computing an SCDF analytically is non-trivial, our approach is therefore to parameterize the SCDF with a deep neural network and learn the mapping by supervised learning. Our resulting neural SCDF can compute distances for different parameter values of obstacle shapes and we also show how multiple distances can be combined, thus making our approach flexible.
\section{Related work}
Motion planning algorithms can roughly be divided into three families, grid-based, sampling based and optimization based methods. Grid-based methods (GBM) discretize the planning space from which a graph is then compiled. A standard search method is A$^\star$ \citep{a_star}, which is classified as an \textit{informed} search method, since it employs a heuristic function to speed up the search. A$^\star$ guarantees to return an optimal path at the level of discretization used. GBMs usually discretize the planning space by a regular lattice and this limits the GBMs to problems with low dimensionality due to the curse of dimensionality. Thus, GBMs are usually limited to single-body robots where the degrees of freedom (DOF) are low. To overcome the inherent scaling problem with the GBMs, stochastic methods are usually used for multi-body robots. These methods are termed as sampling-based methods (SBM) and core members within this family are the rapidly-exploring random trees (RRT) \citep{rrt} and the probabilistic roadmap (PRM) \citep{prm}. RRT grows a tree from the start configuration and explores the collision-free region in a rapid way until it is able to connect to the goal region. RRT is usually improved by bi-directional planning \citep{rrt_connect}, i.e. an additional tree is grown from the goal configuration and the trees are tested for connection after any tree has been expanded. RRT is a single-query method, thus it searches for a path from scratch each time it is queried. Contrary to this, PRM is a multi-query method, which solves for multiple queries without starting from scratch. PRM does this by creating a roadmap (graph) that covers the collision-free space as an offline step. The graph is then used to solve for multiple queries. PRMs are used in cases where the environment does not change since the extra offline step is too computationally costly and needs to be re-done if the environment is changed. In our work, we address this inherent issue by using a different roadmap representation. Our vertices in the graph cover a collision-free region in the form of spheres and we form the edges by checking for intersecting spheres. If something in the environment changes, we recompute the spheres radii and recheck the intersections, without relying on collision detection. We use a trained neural network to compute the sphere radius, therefore querying for the radius can be done fast, hence our representation enables the PRM for dynamic environments.
\\\\
In the recent decades, optimization based methods (OBM) \citep{chomp, schulman, itomp, stomp} have been introduced as an alternative to SBM for multi-body robots. Like the SBM, the OBMs scale well to higher dimensional problems and produce smoother motion. It is common to use a SDF in the optimization since it is a smooth function, thus enabling gradient-based methods. However, the standard way of expressing the SDF is in world space. The distance therefore needs to be mapped to the configuration space by the forward kinematics. This mapping makes the optimization problem a non-linear program (NLP), which is computationally expensive to solve. Recently, a different approach has been proposed. In \cite{mp_gcs} motion planning is formulated as a convex optimization problem by using the graph of convex sets framework \citep{gcs}. The underlying idea is to decompose the collision-free space into intersecting convex sets from which a convex optimization problem is formulated. In cases where an explicit representation of the obstacles in the configuration space exists, like for single-body robots, creating collision-free convex regions can be done fast \citep{iris}. For multi-body robots, this is non-trivial. Existing work does this successfully \citep{iris_nlp, iris_c} by an optimization based approach, but the methods are still too time consuming to be used in the presence of moving obstacles. Our approach is instead to use deep learning to learn an SDF expressed in the configuration space. With this, we can query for shortest distances to the collision boundary, which allows us to expand spherical regions which are collision-free. Our approach is fast and therefore enables our suggested roadmap planner to be used in dynamic environments.
\\\\
Recent research has focused on learning collision detection \citep{fk_kernel_distance, diffco, graphdistnet} by predicting the signed distance between the robot links and the surrounding obstacles in the world space. The learned SDF is used in trajectory optimization but since the distance is expressed in the world space, the problem becomes an NLP and therefore takes a long time to solve. We take a novel approach and suggest to instead express the signed distance in the configuration space. This allows us to improve the PRM at the same time as it enables convex optimization for trajectory optimization, which runs faster and is more reliable than NLP solvers. In \cite{cspf} a learned signed distance function in the configuration space is proposed similar to our approach. However, their approach is restricted to point cloud representations, while we propose to represent the obstacles as parameterized geometric shapes, e.g. spheres. Furthermore, we also show how to use our learned SCDF to improve an existing roadmap planner.
\section{Problem formulation}
A robot is located in the world space, $\W \subset \R^3 $. The unique location of the robot is given by its configuration $\q \in \C$, where $\C$ is the configuration space. The set of points covered by the robots bodies at a certain configuration is expressed as $\B(\q) \subset \W$. The robot is surrounded by $\NrObst$ obstacles $\O = \bigcup_{i=1}^{\NrObst} \O_i$, where  $\O_i \subset \W$. The representation of the obstacle in the configuration space is the set $\C\O_i = \{\q \in \C \: |\: \B(\q) \cap \O_i \neq \emptyset \}$. The obstacle space is formed as $\Co = \bigcup_{i=1}^{\NrObst} \C \O_i$. The complement is referred to as the free space, $\Cf = \C \setminus \Co$. The path planning problem is a tuple, ($\Cf$, $\qStart$, $\qGoal$), where we want to connect a query pair, consisting of a start, $\qStart$, and goal configuration, $\qGoal$, with a geometric path, $\q(s): [0, 1] \mapsto \Cf$, such that $\q(0)=\qStart$ and $\q(1)=\qGoal$, or report correctly when such a path does not exist.
\end{document}

% \section{Motivation}
\label{sec:motivation}



% In LLM inference, not only does weight matter, but the memory requirements of the KV Cache are also considerable.
In this section, we first demonstrate that the emerging paradigm of group quantization demands a high level of adaptivity, which current adaptive methods lack.
We then discuss how adapting these methods to group quantization could compromise their efficiency.
Given that LLMs generate KV caches during runtime, real-time quantization capability is crucial.
These challenges lead to our proposal of a mathematical adaptive numerical type (\texttt{MANT}), which we will detail later.



\input{figure_text/fig_moti_group_ppl.tex}




\subsection{Group Quantization Accuracy Analysis}
\label{sec:acc_analysis}

In this subsection, we begin by comparing the accuracy of traditional channel-wise quantization with group-wise quantization~\cite{shao2024omniquant,zhao2023atom,liu2024kivi,sheng2023flexgen,lin2023awq,zhao2023atom}, establishing the baseline for group-wise quantization in this study.
We then delve into the use of various adaptive data types in group quantization, emphasizing the necessity for full adaptivity.



\Fig{fig:moti_group_ppl} illustrates the perplexity when quantizing the LLaMA-7B model~\cite{touvron2023llama} with various granularities using the \texttt{INT4}-based symmetric quantization.
Channel-wise quantization significantly worsens the perplexity of the examined LLM, increasing it from 5.68 to 6.85.
Conversely, group-wise quantization mitigates this loss in perplexity with a group size of 128, corresponding to an average of 4.125 bits per element (16-bit scaling factor).
Additionally, we observe that a smaller group size of 32 offers only a slight improvement in perplexity, but the scaling factor overhead increases by $4\times$.



Given this analysis, we adopt a group size of 128 as our standard configuration for the remainder of this section.
Previous research indicates that the \texttt{INT} data type is not optimal for accuracy since tensors or channels exhibit varied distributions, leading to the proposal of various adaptive data types~\cite{guo2022ant, guo2023olive, zadeh2020gobo, zadeh2022mokey}.
We evaluate their efficacy in the context of group quantization, which falls into two main categories: data-type-based and clustering-based.



\textbf{Data-type-based adaptive methods} select data types from discrete sets based on tensor data distribution.
ANT~\cite{guo2022ant} is a representative example of the data-type-based method.
ANT packages several different data types for selection, including \texttt{INT} for the uniform distribution, \texttt{PoT} (Power of Two) for the Laplace distribution, and \texttt{flint} for the Gaussian distribution.
%ANT designed \texttt{flint} for Gaussian distributions.

\textbf{Clustering-based adaptive methods} utilize clustering algorithms to generate centroids that align with the data distribution and provide considerable adaptivity. 
Mokey~\cite{zadeh2022mokey} and GOBO~\cite{zadeh2020gobo} exemplify this approach, though they focus on tensor- or channel-wise quantization. In our study, we adapt them to group quantization through per-group clustering.

%Clustering-based methods employ clustering algorithms to generate centroids that fit the data distribution, demonstrating sufficient adaptivity.
%Mokey~\cite{zadeh2022mokey} and GOBO~\cite{zadeh2020gobo} are such presentative works, but only target tensor- or channel-wise quantization.
%In our work, we modify those works to support group quantization by performing per-group clustering.
\Fig{fig:moti_ppl} compares the accuracy of the methods described above for the LLaMA-7B model under 4-bit group-wise quantization. 
The group-wise \texttt{ANT} method outperforms the \texttt{INT} type by dynamically selecting from three data types to better match the value distribution, resulting in reduced perplexity (PPL) loss. 
Moreover, per-group clustering adjusts more effectively to the value distribution of each group, establishing itself as the accuracy-optimal and ideal adaptive method. 
This approach achieves nearly lossless 4-bit quantization, equivalent to 16 centroids per group. 
However, this ideal scenario is impractical due to the significant overhead associated with storing per-group centroids, effectively rendering it a 6-bit quantization.

\input{figure_text/fig_moti_cdf.tex}

To illustrate the group-wise diversity in data distribution, we sampled the weights of the Q and V tensors in LLaMA-7B model. 
We normalized all sampled data to their absolute maximum values, which ranged from -1 to 1. \Fig{fig:moti_dist} displays the cumulative distribution function (CDF) for the tensor, channel, and group levels, respectively. 
We observed that the diversity at the group level is significantly higher than at the tensor level. 
In simpler terms, while different tensors exhibit similar distributions, groups can have markedly different distributions. This finding underscores the necessity for full adaptivity in group quantization to fully realize its potential.
\paragraph{Takeaway 1.} The group quantization is an emerging paradigm to accelerate LLMs, and the significant group-level diversity requires a high level of adaptivity to fully unleash its potential.

\subsection{Group Quantization Efficiency Analysis}
\label{subsec:efficiency}


In this subsection, we provide a detailed efficiency analysis for the above adaptive quantization methods.
In \Tbl{intro:dtype}, we compare OliVe~\cite{guo2023olive}, ANT~\cite{guo2022ant}, GOBO~\cite{zadeh2020gobo}, and Mokey~\cite{zadeh2022mokey} with \texttt{INT} regarding the efficiency of computation, encoding, and decoding. 
In this paper, we use the term encoding (decoding) interchangeably with quantization (dequantization).
 

Data-type-based adaptive methods such as ANT~\cite{guo2022ant} and Olive~\cite{guo2023olive} achieve computational efficiency comparable to \texttt{INT}. 
Both utilize specialized decoders that decode these data types prior to computation, resulting in high decoding efficiency. 
However, as previously demonstrated, these methods suffer from limited adaptivity in the group quantization paradigm. 
A straightforward approach to enhance adaptivity is to expand their set of data types. 
However, incorporating new data types necessitates additional decoders, escalating hardware design costs. 
Additionally, compatibility issues between new and existing data types may reduce computational efficiency. 
For instance, the \texttt{NF4} data type~\cite{dettmers2023qlora} requires an FP16 MAC unit, which is incompatible with existing \texttt{ANT} data types.


\paragraph{Takeaway 2.} Enhancing the data-type-based adaptive method for group quantization is challenging and requires a careful balance for the computation and decoding efficiency.

Clustering-based adaptive methods like GOBO~\cite{zadeh2020gobo} and Mokey~\cite{zadeh2022mokey} can sufficiently adapt to various distributions at the group level. 
However, they require codebooks for quantization and dequantization, leading to high adaptivity at the expense of encoding and computational efficiency. 
For instance, a 16-entry codebook with 8 bits per entry requires 128 bits per group, creating an inevitable trade-off between adaptivity and memory overhead. GOBO~\cite{zadeh2020gobo} employs the K-means algorithm to quantize weights and requires dequantization to \texttt{FP16} using a codebook lookup table before computation, resulting in high adaptivity but low computational efficiency. 
Conversely, Mokey~\cite{zadeh2022mokey} enhances the computation of clustering-based methods by using indices for centroid values via approximate calculations, though matrix multiplication still relies on floating-point units, increasing overhead compared to integer units. 
Furthermore, Mokey creates one \texttt{golden dictionary} for all activations and weights, akin to using a single data type in quantization, thus reducing adaptivity.


\paragraph{Takeaway 3.} Deploying the clustering-based adaptive methods under group quantization is challenging owing to the low encoding and computation efficiency. 


\input{figure_text/tbl_moti_accelerator_feature.tex}

\subsection{Support for Real-time Quantization}
\label{sec:moti_kvcache}

The above group-wise diversity presents a challenge for both weights and KV cache.
In addition, KV cache faces challenges in real-time group-wise quantization because the KV cache is generated dynamically during LLM inference.


To facilitate low-precision computation in group-wise quantization, it is necessary to quantize K and V along the inner dimension. 
This requirement stems from the support for matrix inner product operations in most GPUs and TPUs. 
During these operations, the group-wise scaling factor can be extracted from the multiply-accumulate process. 
\Fig{fig:kv_process} depicts the computation process of K and V during the decode stage. We define the dimension used for matrix inner product operations as the inner dimension. 
The inner dimensions of the K and V caches differ; the K cache requires a transpose operation, whereas the V cache does not, complicating the situation.


In the prefill stage, K and V can easily compute the scaling factor for each group. 
During the decode stage, the newly generated K vector is concatenated along the inner dimension of the K cache, enabling immediate quantization. 
However, the newly generated V vector is associated with different groups, with only one element per group produced per iteration. This process prevents the scaling factor for the entire group from being obtained in a single iteration, posing a significant challenge for the real-time quantization of the V cache.


\input{figure_text/fig_moti_kv_dimension.tex}


Given those challenges, we propose \proj with a mathematical encoding format that can fuse with integer computation and enhance the decoding efficiency.
In addition, this encoding format provides sufficient adaptivity for group-wise quantization.
Regarding the challenge in KV cache, \proj employs a real-time quantization engine that ensures efficient encoding and decoding for KV cache.
By addressing these challenges, \proj enables efficient low-bit group-wise quantization.



\section{Basic Background: Supervised Learning and the PAC Model}
\label{sec:background}

At this point almost everyone has heard of machine learning (ML). Anyone likely to stumble upon this article will have also heard of its most influential special case, supervised learning, and those theoretically inclined will also be familiar with the PAC model. Nonetheless, I will set the stage by  recapping the basics.

\subsection{Basics of Supervised Learning}%Let's set the stage in any case

\emph{Supervised Learning} is the task of ``coming up'' with a function $f: \X \to \Y$ to ``explain'' or ``fit'' a sequence of input/output examples   $(x_1,y_1), \ldots, (x_n,y_n)$, with $x_i \in \X$ and $y_i \in \Y$.  Here $\X$ is a \emph{data domain} consisting of \emph{datapoints} $x \in \X$, $\Y$ is a \emph{label set} consisting of \emph{labels} $y \in \Y$, and the sequence $(x_1,y_1),\ldots,(x_n,y_n)$ is the \emph{training data} consisting of \emph{labeled examples (a.k.a. samples)}~$(x_i,y_i)$.  I~will refer to the chosen function $f$ as a \emph{predictor}, and to $n$ as the \emph{sample size}. A \emph{learning algorithm} takes as input training data, and outputs (some representation of) a predictor $f \in \Y^\X$.\footnote{Note that this describes the usual \emph{batch}, a.k.a.~\emph{offline}, setting of supervised learning. I do not discuss other paradigms such as online or active learning in this article.} 



Success in supervised learning is defined as \emph{generalization} to  future examples: For a typical \emph{test example}  $(x_{\tst},y_{\tst})$, the predicted label $y'_{\tst}=f(x_{\tst})$ should ``equal'' $y_{\tst}$, perhaps approximately. We usually assume the test example is drawn from the same  ``source'' as the training data  --- commonly, i.i.d.~from the same distribution. The quality of the prediction is quantified by $\ell(y'_{\tst},y_{\tst})$, where $\ell:~\Y~\times~\Y \to \RR_{\geq 0}$ is a \emph{loss function} chosen as part of the problem definition. Common loss functions include the 0-1 loss $\ell_{0-1}(y',y) = [y' \neq y]$ for \emph{classification} problems,\footnote{The notation $[P]$ denotes $1$ when predicate $P$ is true, and denotes $0$ when $P$ is false.} as well as the absolute loss $|y'-y|$ or squared loss $(y'-y)^2$ for \emph{regression problems} featuring $\Y  \sse \RR$.

Nontrivial generalization properties are typically only possible if one assumes something about the data.\footnote{The need for such an assumption is formalized by the  \emph{no free lunch theorems} of supervised learning \cite{wolpert_connection_1992,wolpert_lack_1996,schaffer_conservation_1994}.} The Bayesian approach to  machine learning, common in many applications, assumes some parametric form for the distribution generating the data, and postulates a prior on the parameters. This is not the approach I will take in this article. Instead, I will focus on the frequentist --- and some would say ``worst-case'' or ``adversarial'' ---  approach that is common in the computational learning theory community, embodied by the PAC model. Here we assume that the (training and test) data can be explained, perhaps approximately, by a function in some ``simple enough to learn'' class of functions $\H \sse \Y^\X$, often called the \emph{hypotheses}. Equivalently, we  seek a predictor which explains the unseen data roughly  as well as the best hypothesis $h^* \in \H$, whether or not we assume that $h^*$ itself provides a perfect explanation.



 \paragraph{Common Algorithmic Templates.} Perhaps the best known general-purpose supervised learning algorithm is \emph{empirical risk minimization (ERM)}, which chooses as its predictor a hypothesis $f \in \H$ minimizing $\frac{1}{n} \sum_{i=1}^n \ell(f(x_i),y_i)$ --- a quantity called the \emph{training error}, \emph{empirical error}, or \emph{empirical risk} of $f$. %\footnote{When multiple hypotheses minimize the empirical risk, we assume ERM breaks ties arbitrarily.}
A common template for generalizing ERM involves adding a \emph{regularization term} $\psi(f)$ to the  objective function, typically chosen to measure some notion of ``hypothesis complexity.'' An algorithm instantiating this template is known as a \emph{structural risk minimizer (SRM)}, and chooses as its predictor the hypothesis $f \in \H$ minimizing the \emph{structural risk} $\frac{1}{n} \sum_{i=1}^n \ell(f(x_i),y_i) + \psi(f)$. Other well-known algorithms, such as gradient descent and its variations,  can frequently be interpreted as approximate implementations of ERM or SRM.


\paragraph{Proper vs Improper Learning.} A learning algorithm is said to be \emph{proper} if its predictor $f$ is always chosen from the hypothesis class, i.e., $f \in \H$, otherwise it is said to be \emph{improper}. ERM  is an example of a proper learning algorithm, as are SRM algorithms of the form described above.  In the \emph{proper regime} of learning, algorithms are required to be proper. This article will be concerned with the more flexible \emph{improper regime} (a.k.a \emph{representation-independent learning}), where no such constraint is placed on the learner. In other words, all we care about is predictive power at test time, rather than any insights derived from the functional form or representation of the predictor~itself.


\subsection{The PAC Model}
A standard mathematical setup for evaluation of supervised learning algorithms, at least in the theoretical computer science community, is Valiant's \emph{Probably Approximately Correct (PAC) model} of learning (see e.g.~\cite{kearns_introduction_1994,mohri_foundations_2018}). Here, we assume there is an unknown distribution $\D$ on $\X \times \Y$ from which training and test data are  drawn.  Specifically, the labeled datapoints of the training set  $(x_1,y_1), \ldots, (x_n,y_n)$, as well as the test data  $(x_\tst,y_\tst)$, are i.i.d.~from $\D$. Often it is assumed that $\D$ lies in some class of distributions of interest. The \emph{true expected loss}, or simply \emph{loss}, of a predictor $f: \X \to \Y$ is the expected loss it incurs on draws from $\D$, written $L_\D(f) = \Ex_{(x,y) \sim \D} \ell(f(x),y)$.


There are two main ``settings'' in PAC learning. The  \emph{realizable setting} only requires that the data be perfectly explained by some hypothesis in $\H$. More generally, the \emph{agnostic setting} makes no assumption relating the data to the hypotheses, but shifts the goalposts as necessary to allow nontrivial guarantees: the expected loss at test time is evaluated only ``relative'' to that of the best hypothesis $h^* \in \H$. There are other settings which make more nuanced assumptions, such as $\D$ being of a particular parametric form or its support living in some (unknown) lower-dimensional space, etc. I will mostly discuss the realizable and agnostic settings in this article, those being the simplest and most studied from a theoretical perspective. %TODO:We will briefly discuss other settings in Section ??

The PAC model demands high probability guarantees of learners, in the worst case over distributions of interest. Consider first the realizable setting, where $\D$ is such that $\min_{h \in \H} L_{\D}(h) = 0$. A PAC learner has \emph{error} $\epsilon=\epsilon(n)$ and \emph{confidence} $\delta=\delta(n)$ if, when training data consists of $n$ i.i.d~samples from a realizable distribution $\D$, it produces a predictor $f$  satisfying $L_\D(f) \leq \epsilon$ with probability at least $1-\delta$. In the agnostic setting, where $\D$ can be arbitrary, we require $L_\D(f) - \min_{h \in \H} L_\D(h) \leq \epsilon$ with probability $1-\delta$.

In both the realizable and agnostic settings, we look for PAC learners with small $\epsilon$ and $\delta$ as a function of the sample size $n$. An equivalent perspective looks at the sample complexity $m(\epsilon,\delta)$, which is the minimum sample size which guarantees error  at most $\epsilon$ with probability at least $1-\delta$. We say a problem is \emph{PAC learnable} if its PAC sample complexity is finite whenever $\epsilon,\delta > 0$.

For most PAC learning problems, learnability and sample complexity are characterized in terms of a  ``dimension'' of the hypothesis class. Most prominently this is the \emph{VC dimension} for binary classification, the \emph{fat shattering dimension} for agnostic regression, and the \emph{DS dimension} for multiclass classification (see \cite{anthony_neural_1999,daniely_optimal_2014,brukhim_characterization_2022}). Treatment of these is beyond the scope of this article. The unfamiliar reader need not worry, however,  as dimensions will feature only tangentially in our~discussion.




%\paragraph{Learning settings: Realizable, Agnostic, etc.} In learning theory, evaluating a supervised learning algorithm requires specifying a data model and an objective. We will leave the details of the data model flexible for now, to allow for both the PAC model and the adversarial transductive model. Nonetheless we will describe two variations, which we call ``settings'', which cut across different models. The  \emph{realizable setting}  requires only that the data be perfectly explained by some hypothesis $h \in \H$ --- i.e., there exists a hypothesis which is guaranteed to suffer a loss of $0$ on training and test data. The performance of the learning algorithm is its expected loss at test time for some ``worst case'' realizable instance. More generally, the \emph{agnostic setting} makes no assumption relating the data to the hypotheses, but shifts the goalposts as necessary to allow nontrivial guarantees: the expected loss at test time is evaluated only ``relative'' to that of the best hypothesis $h^* \in \H$, again for some ``worst case'' instance. There are other settings which make more nuanced assumptions about the data, such as it is drawn from a distribution of a particular parametric form, or that it lives in some (unknown) lower-dimensional space, etc. We will mostly discuss the realizable and agnostic settings, those being the simplest and most studied from a theoretical perspective.




%%% Local Variables:
%%% mode: latex
%%% TeX-master: "learning_matching"
%%% End:

\section{Design}\label{sec:design}

%%%%%%%%%%%%%%%%%%%%%%%%%%%%%%


\begin{figure*}[t]
    \centering
    \includegraphics[trim = 15 530 15 15, width=1\textwidth]{Algorithm_drawio.pdf}
    \caption{Overview of KiSS}
    \label{fig:overview}
\end{figure*}


The results we gleaned from the previous section (see Section~\ref{sec:work_anly}) helped in developing our policy: KiSS. The KiSS or \textbf{Keep it Separated Serverless} policy aims to address critical challenges in Function-as-a-Service (FaaS) platforms, particularly in edge computing environments, by achieving the following objectives:

\begin{itemize}
    \item \textbf{Reduced Cold Start Latency:} Prioritizes high-frequency functions to minimize delays in real-time applications.
    \item \textbf{Improved Resource Efficiency:} Optimizes memory and compute usage while avoiding unnecessary overhead from static warm states.
    \item \textbf{Minimized Inter-Function Interference:} Enhances throughput and scalability through modular resource partitioning.
    \item \textbf{Improved Function Service Rate:} Adopts resource-aware policies to reduce dropped requests and maximize system reliability.
\end{itemize}


\subsection{KiSS Policy Overview}

KiSS introduces a modular, data-driven orchestration strategy designed to optimize serverless execution in resource-constrained environments, particularly at the edge. By leveraging our workload analysis (refer Section 2.5), our policy segments functions based on key metrics—memory footprint, invocation frequency, and execution time—to optimize performance across diverse workloads.

The edge computing context introduces unique challenges like limited memory, heterogeneous resources, and dynamic workloads. Generalized cloud strategies often fail to adapt to such constraints. KiSS addresses this gap by analyzing workload characteristics and implementing a resource-efficient, modular strategy that aligns with edge-specific demands.

\subsection{Components of KiSS Policy Design}
Figure~\ref{fig:overview} shows the overall architecture of KiSS. 
The incoming \textit{FaaS traffic} will include both small and large functions. 
The \textit{request handler} accepts the incoming functions and shares the function information to the workload analyzer. 
The \textit{workload analyser} processes the function information to profile the incoming function traffic information and generate data such as invocation frequency, memory footprint etc.
The \textit{KiSS policy} uses this data to estimate where this function will be placed between the two different warm pool partitions.

The \textit{load balancer} implements a partitioning logic where functions are allocated to distinct warm pools using (\textit{invoker 1} and \textit{invoker 2}) based on profiling thresholds:

(i)~Small Functions Pool: Dedicated to high-frequency, low-memory functions to ensure low latency, and (ii)~Large Functions Pool: Allocated for low-frequency, memory-intensive functions, minimizing contention with smaller containers.
Each warm pool operates autonomously achieving Policy Independence.
The \textit{Warm Pool Replacement Policy} for each warm container pool can independently implement different workload-specific strategies to reduce contention and enhance temporal locality.


These factors form the foundation of KiSS’s multi-tiered warm pool framework, allowing it to effectively manage serverless resources and enhance performance in edge computing. By addressing these challenges, KiSS positions itself as a practical and scalable solution for FaaS platforms in environments with diverse and demanding resource constraints.


\subsection{Innovations of KiSS Policy}

One of the most innovative features of KiSS is its multi-level warm pool partitioning, which isolates high- and low-frequency functions into separate pools. This design eliminates inefficiencies inherent in monolithic resource strategies by ensuring that small, frequently invoked functions are always ready to execute, while larger, less frequent functions remain accessible without competing for resources. This adaptability extends to the ability to add more pools as workload patterns evolve, making KiSS a flexible and future-proof solution. Moreover, its modular architecture supports diverse deployment scenarios, from centralized clouds to resource-constrained edge environments. Integration with traffic-aware schedulers ensures that KiSS maintains scalability and responsiveness even under fluctuating workloads.


\subsubsection{Advantages of KiSS}

The advantages of KiSS are particularly pronounced in edge environments. By keeping frequently accessed containers in warm states, it drastically reduces cold start latency, which is critical for real-time applications such as IoT and AI analytics. Static warm pool partitioning, based on workload analysis, optimizes memory usage by eliminating unnecessary overhead, ensuring that resources are used efficiently even in environments with stringent memory constraints. This strategy not only enhances performance but also reduces operational costs by consolidating memory usage and minimizing cold starts. KiSS’s platform-agnostic design further enhances its versatility, enabling seamless deployment across various serverless frameworks.


% \renewcommand\cellset{\renewcommand\arraystretch{0.7}}
\begin{figure*}
    \centering
    \resizebox{\textwidth}{!}{
    \begin{tabular}{c|c|c}
    \toprule
    \scriptsize \textbf{} & \small \textbf{$2\times$ length extrapolation} & 
    \small \makecell{\textbf{$2\times$ spatial extrapolation}}  \\ \midrule 
    \multirow{2}{*}{\makecell[t]{\small \textbf{Normal} \\ \textbf{length}}} 
    &
    \begin{minipage}{0.75\textwidth}
    \centering
\includegraphics[width=0.95\textwidth]{images/challenge/ref_vid.pdf}
    %\vspace{.1cm}
    \end{minipage}
    & 
    \begin{minipage}{0.2\textwidth}
    \centering   
    \includegraphics[height=0.25\textwidth]{images/challenge/ref_img.pdf}
    %%\vspace{.1cm}
    \end{minipage}
    \\ 
    & \small{Video of $49$ frames} &  \small{Image of 1K resolution} \\ \midrule
    % Extrapolation 
    \multirow{2}{*}{\makecell[t]{\small\textbf{PE}}} &
    \begin{minipage}{0.75\textwidth}
    \centering
    \includegraphics[width=0.95\textwidth]{images/challenge/PE_vid.pdf}
    %\vspace{.1cm}
    \end{minipage}
    & 
    \begin{minipage}{0.15\textwidth}
    \centering   
    \includegraphics[height=0.68\textwidth]{images/challenge/PE_img.pdf}
    %\vspace{.1cm}
    \end{minipage}
    \\
     &  \small{(a) Temporal repetition} & \small{(d) Spatial repetition} \\ 
    % Interpolation 
    \multirow{2}{*}{\makecell[t]{\small\textbf{PI}}} &
    \begin{minipage}{0.75\textwidth}
    %\vspace{.1cm}
    \centering
    \includegraphics[width=0.95\textwidth]{images/challenge/PI_vid.pdf}
    %\vspace{.1cm}
    \end{minipage}
    & 
    \begin{minipage}{0.15\textwidth}
    \centering 
    %\vspace{.1cm}
    \includegraphics[width=0.68\textwidth]{images/challenge/PI_img.pdf}
    %\vspace{.1cm}
    \end{minipage}
    \\
     &  \small{(b) Slower motion} & \small{(e)  Blurred details
} \\ 
    % Time-Aware RoPE 
    \multirow{2}{*}{\makecell[t]{\small\textbf{NTK}}} &
    \begin{minipage}{0.75\textwidth}
     %\vspace{.1cm}
     \centering
    \includegraphics[width=0.95\textwidth, height=50pt]{images/challenge/TA_vid.pdf}
     %\vspace{.1cm}
     \end{minipage}
    & 
    \begin{minipage}{0.15\textwidth}
    \centering   
     %\vspace{.1cm}
    \includegraphics[width=0.68\textwidth]{images/challenge/TA_img.pdf}
     %\vspace{.1cm}
    \end{minipage}
    \\
     &  \small{(c) Temporal repetition} & \small{(f) Spatial repetition} \\ \bottomrule
    \end{tabular}
    }
    \caption{\textbf{Visualization of existing methods for 2$\times$ extrapolation in video and image generation.} The base models CogVideoX-5B~\cite{yang2024cogvideox} and Lumina-Next~\cite{zhuo2024lumina} are trained to sample videos of up to 49 frames and images of up to 1K resolution, respectively. Existing methods lead to \textit{temporal repetition} or \textit{slower motion} in video extrapolation and \textit{spatial repetition} or \textit{blurred content} in image extrapolation, respectively. Please refer to Appendix~\ref{sec: existing failure} for more results and details. 
    }
    \label{fig:challenge}
    %%\vspace{-0.5cm}
\end{figure*}
% \subsection{Results on PPAC Optimization with 2 Silicon Dies}
Table~\ref{tab:2die_optimization} shows the results for the three techniques: Baseline, MMFP-SA and MMFP-RL when the number of dies $\mathcal{D}$ is 2 (one die with 7nm, and the other with 45nm). The results indicate that MMFP-SA achieves average reductions of 6\%, 8\% and 3.6\%, while MMFP-RL achieves reductions of 8.8\%, 11.7\% and 5.7\% in area, HPWL and cost, respectively, compared to the baseline. In terms of post-placement TNS, MMFP-SA and MMFP-RL show average improvements of 17.2\% and 21.7\% respectively. Moreover, MMFP-SA and MMFP-RL achieve average savings in dynamic power of 6.4\% and 8.1\% respectively. In terms of CPU runtime, MMFP-RL is on average 0.8$\times$ slower, and MMFP-SA is 0.9$\times$ slower than the baseline. However, MMFP-RL is faster than MMFP-SA and the baseline as the number of interconnected blocks $\mathcal{B}$ increases. Figure~\ref{fig:objective_iteration} shows the objective function $f$ across iterations during optimization on the \emph{netcard} design. MMFP-RL requires $293$ fewer iterations than MMFP-SA and achieves a better objective value.

\begin{figure}[ht] 
\centering
\includegraphics[width=.95\linewidth]{figures/objective_iterations1.png}
\caption{Objective function $f$ value for MMFP-SA/RL across iterations.}
\label{fig:objective_iteration}
\end{figure}



\subsection{Results on PPAC Optimization with 4 Silicon Dies}
%In heterogeneous integration, the number of dies $\mathcal{D}$ is typically more than 2. 
Table~\ref{tab:4die_optimization} shows the results for the \mbox{\emph{leon3-avnet}} design when the number of dies $\mathcal{D}$ is 4. The results show that MMFP-SA achieves average reductions of 7.2\%, 6.5\% and 3.2\% in area, HPWL and cost, respectively, compared to the baseline. MMFP-RL further improves the reductions achieving 9.7\%, 9\% and 4.3\%. Post-placement TNS is improved by 7.3\% for MMFP-SA, and 11.1\% for MMFP-RL. Furthermore, MMFP-SA achieves 7\% power savings, while MMFP-RL achieves 8.6\%. In terms of CPU runtime, MMFP-RL is 1.19$\times$ faster than the baseline, while MMFP-SA is 0.95$\times$ slower.

\begin{table*}[!ht]
\centering
\caption{PPAC optimization results for leon3-avnet design in 4 silicon dies.}
\begin{tabular}{cclrrrrrrr}
\hline
\multicolumn{2}{c}{\# dies} & \multirow{2}{*}{Method} & Area & HPWL & Cost & \multicolumn{2}{c}{Timing (ns)} & Power & CPU \\
\cline{1-2} \cline{7-8}
7nm & 45nm & & ($\times10^3 \mu$m$^2$) & ($\mu$m) & ($\times10^{-3}$) & TNS & WNS & (mW) & (sec) \\
\hline
\hline
\multirow{3}{*}{1} & \multirow{3}{*}{3} & Baseline & 1409.29 & 17495.01 & 3173 & -1253.06 & -1.948 & 1397.8 & 2104 \\
& & MMFP-SA & 1320.57 & 16029.53 & 3086 & -1171.95 & -1.872 & 1320.6 & 2308 \\
& & MMFP-RL & 1296.30 & 15702.68 & 3069 & -1128.43 & -1.830 & 1315.1 & 1963 \\
\hline
\multirow{3}{*}{2} & \multirow{3}{*}{2} & Baseline & 831.06 & 13972.70 & 2880 & -810.57 & -1.304 & 1150.3 & 2376 \\
& & MMFP-SA & 759.25 & 13295.18 & 2755 & -752.86 & -1.258 & 1051.3 & 2450 \\
& & MMFP-RL & 738.14 & 12960.91 & 2716 & -719.25 & -1.230 & 1016.3 & 2003 \\
\hline
\multirow{3}{*}{3} & \multirow{3}{*}{1} & Baseline & 590.71 & 9713.55 & 2674 & -603.74 & -1.071 & 674.9 & 2502 \\
& & MMFP-SA & 552.05 & 9101.84 & 2603 & -552.83 & -0.994 & 628.1 & 2581 \\
& & MMFP-RL & 531.92 & 8782.73 & 2568 & -530.06 & -0.976 & 619.3 & 1924 \\
\hline
\hline
& & Baseline & 1 & 1 & 1 & 1 & 1 & 1 & 1 \\
\multicolumn{2}{c}{Norm. Avg.}  & MMFP-SA & 0.928 & 0.935 & 0.968 & 0.927 & 0.951 & 0.930 & 0.950 \\
& & MMFP-RL & 0.903 & 0.910 & 0.957 & 0.889 & 0.931 & 0.914 & 1.186 \\
\hline
\end{tabular}
\label{tab:4die_optimization}
\end{table*}
\section{Evaluation}


\begin{table}[t]
    \centering
    % \vspace{-0.1in}
    \scalebox{0.78}{
    % \begin{small}
        \begin{tabular}{lccc}
            \toprule
            \multirow{2}*{\textbf{MoE Models}} & \textbf{Parameters} & \textbf{Experts Per Layer} & \textbf{Num. of} \\
            & \textbf{(active / total)} & \textbf{(active / total)} & \textbf{Layers} \\
            \otoprule 
            \mixtral~\cite{jiang2024mixtral} & 12.9B / 46.7B & 2 / 8 & 32 \\
            % \hline
            \qwen~\cite{yang2024qwen2} & 2.7B / 14.3B & 4 / 60 & 24 \\
            \phimoe~\cite{abdin2024phi} & 6.6B / 42B & 2 / 16 & 32 \\
            \bottomrule 
        \end{tabular}
    % \end{small}
    }
    \caption{Characteristics of three \MoE models in evaluation.}
    \vspace{-0.2in}
    \label{table:eval-moe-models}
\end{table}








\subsection{Experimental Setup}
\label{subsec:eval-setup}


% \begin{figure*}[t]
%     \centering
%     \begin{subfigure}[t]{0.48\textwidth}
%         \centering
%         \includegraphics[width=.9\linewidth]{figs/eval-overall-lmsys.pdf}
%         \caption{Serving three \MoE models with LMSYS-Chat-1M dataset.}
%     \end{subfigure}
%     \begin{subfigure}[t]{0.48\textwidth}
%         \centering
%         \includegraphics[width=.9\linewidth]{figs/eval-overall-sharegpt.pdf}
%         \caption{Serving three \MoE models with ShareGPT dataset.}
%     \end{subfigure}
%     \caption{Overall performance of prefill and decode stages for \sys and other four baselines.}
%     \label{fig:eval-overall.pdf}
% \end{figure*}


\noindent \textbf{Testbed.}
We conduct all experiments on a six-GPU testbed, where each GPU is an NVIDIA GeForce RTX 3090 with 24 GB GPU memory. 
%
All GPUs are inter-connected using pairwise NVLinks and connected to the CPU memory using PCIe 4.0 with 32GB/s bandwidth. 
%
Additionally, the testbed has a total of 32 AMD Ryzen Threadripper PRO 3955WX CPU cores and 480 GB CPU memory.


\noindent \textbf{Models.}
We employ three popular \MoE-based \LLMs in our evaluation: \mixtral~\cite{jiang2024mixtral}, \qwen~\cite{yang2024qwen2}, and \phimoe~\cite{abdin2024phi}.
Table~\ref{table:eval-moe-models} describes the parameters, number of \MoE layers, and number of experts per layer for the three models.
Following the evaluation of existing works~\cite{song2024promoe}, we profile the models to set the optimal prefetch distance $d$ to three before evaluation.
% We set $d$ of \mixtral, \qwen, and \phimoe to \todo{$xxx$}, \todo{$xxx$}, and \todo{$xxx$}, respectively.


\noindent \textbf{Datasets and traces.}
We employ two real-world prompt datasets commonly used for \LLM evaluation: LMSYS-Chat-1M~\cite{zheng2023lmsys} and ShareGPT~\cite{sharegpt}.
%
For most experiments, we split the sampled datasets in a standard 7:3 ratio, where 70\% of the prompts' context data (\ie, semantic embeddings and expert maps) are stored in \sys's Expert Map Store, and 30\% of the prompts are used for testing. 
%
For online serving experiments, we empty the Expert Map Store and use real-world \LLM inference traces~\cite{patel2024splitwise,stojkovic2025dynamollm} released by Microsoft Azure to set input and generation lengths and drive invocations.

\noindent \textbf{Baselines.}
We compare \sys against four \SOTA \MoE serving baselines:
1) \textbf{MoE-Infinity}~\cite{xue2024moe} uses coarse-grained request-level expert activation patterns and synchronous expert prediction and prefetching for \MoE serving. 
We prepare the expert activation matrix collection for MoE-Infinity before evaluation for a fair comparison.
%
% However, the open-sourced MoE-Infinity codebase~\cite{moe-infinity-code} lacks some features described in its original paper, we had to modify
%y 
2) \textbf{ProMoE}~\cite{song2024promoe} employs a stride-based speculative expert prefetching approach for \MoE serving. Since the codebase of ProMoE is not open-sourced and requires training predictors for each \MoE model, we reproduced a prototype of ProMoE on top of MoE-Infinity in our best effort.
%
3) \textbf{Mixtral-Offloading}~\cite{eliseev2023fast} combines a layer-wise speculative expert prefetching and a \LRU-based expert cache. 
%
4) \textbf{DeepSpeend-Inference} employs an expert-agnostic layer-wise parameter offloading approach, which uses pure on-demand loading and does not support prefetching. 
%
We implement the offloading logic of DeepSpeed-Inference in the MoE-Infinity codebase and add an expert cache for a fair comparison.
We enable all baselines to serve \MoE models from HuggingFace Transformer~\cite{wolf2020huggingface}. 


\noindent \textbf{Metrics.}
Following the standard evaluation methodology of existing works~\cite{song2024promoe,xue2024moe,zhong2024distserve,agrawal2024taming} on \LLM serving, we report the performance of the prefill and decode stages separately. 
We measure Time-to-First-Token (TTFT) for the prefill stage and Time-Per-Output-Token (TPOT) for the decode stage.
Additionally, we also report other system metrics, such as expert hit rate and overheads, for detailed evaluation.


% \noindent \textbf{\sys's setting.}
% The hyperparameters of \sys containing the prefetch distance $d$ for each \MoE model, Expert Map Store capacity $C$, and Expert Cache memory limit $M$.
% For most experiments, we profile the \MoE models and set the prefetch distance $d$ to their optimal values. The Expert Map Store capacity $C$ is set to \todo{$xxx$} expert maps. We configure the Expert Cache memory limit to \todo{$xxx$} GB.
% The hyperparameter sensitivity is analyzed in \S\ref{subsec:eval-sensitivity}.


\begin{figure}[t]
  \centering
  \includegraphics[width=.95\linewidth]{figs/eval-overall-arxiv.pdf}
  \vspace{-0.15in}
  \caption{Overall performance of prefill and decode stages for \sys and other four baselines.}
  \vspace{-0.2in}
  \label{fig:eval-overall}
\end{figure}


\subsection{Overall Performance}
\label{subsec:eval-overall}



We first evaluate the performance of prefill and decode stages when running \sys and other baselines with the three \MoE models, where we measure Time-To-First-Token (TTFT) and Time-Per-Output-Token (TPOT) for each stage.
Note that the inference latency with expert offloading tends to be higher than no offloading due to two reasons: 
1) During inference, an excessive amount of parameters in \MoE models are loaded and offloaded, which prolongs the inference latency.
2) All baselines and \sys are implemented on top of the MoE-Infinity codebase~\cite{moe-infinity-code}, whose inference latency is inherently impacted by MoE-Infinity's implementation.
Nevertheless, comparing \sys and baselines is fair with the same experimental setup.

Figure~\ref{fig:eval-overall} shows the \TTFT, \TPOT, and expert hit rate of \sys and other four baselines when serving three \MoE models with LMSYS-Chat-1M and ShareGPT datasets, respectively.
DeepSpeed has both the worst \TTFT and \TPOT due to expert-agnostic offloading and lacking expert prefetching.
While Mixtral-Offloading, ProMoE, and MoE-Infinity perform better than DeepSpeed-Inference, they are underperformed by \sys because of coarse-grained offloading designs.
Compared to DeepSpeed-Inference, Mixtral-Offloading, ProMoE, and MoE-Infinity, our \sys reduces the average \TTFT by 44\%, 35\%, 33\%, 30\%, and reduces the average \TPOT by 70\%, 61\%, 55\%, 48\%, across three \MoE models.
%
% Figure~\ref{fig:eval-overall} also reports the expert hit rate of \sys and each baseline. 
For expert hit rate, Mixtral-Offloading achieves a higher hit rate than the other three baselines because of its synchronous speculative prefetching with a prefetch distance of 1. However, due to synchronous prefetching, its \TTFT and \TPOT are worse than others except DeepSpeed-Inference.
\sys improves the average expert hit rate by 147\%, 11\%, 34\%, and 63\% over DeepSpeed-Inference, Mixtral-Offloading, ProMoE, and MoE-Infinity, respectively.

% \begin{figure}[t]
%   \centering
%   \includegraphics[width=.9\linewidth]{figs/eval-overall-sharegpt.pdf}
%   % \vspace{-0.15in}
%   \caption{}
%   % \vspace{-0.25in}
%   \label{fig:eval-overall-sharegpt.pdf}
% \end{figure}




\subsection{Online Serving Performance}
\label{subsec:eval-online}


Except for the offline evaluation (\ie, Expert Map Store in full capacity before serving), we also evaluate \sys against other baselines in online serving settings.
We empty the Expert Map Store of \sys and the expert activation matrix collection of MoE-Infinity for the online serving experiment.
%
The request traces are derived from Azure \LLM inference traces~\cite{patel2024splitwise,stojkovic2025dynamollm}, with 64 requests randomly sampled to drive LMSYS-Chat-1M prompts for each \MoE model serving. 
To ensure consistency, \sys and all baselines input and generate the exact number of tokens specified in the traces.
%
Figure~\ref{fig:eval-online-serve} illustrates the CDF of end-to-end request latency across three \MoE models. The results demonstrate that \sys significantly reduces overall request latency compared to other baselines in online serving scenarios.


\begin{figure}[t]
  \centering
  \includegraphics[width=.95\linewidth]{figs/eval-online-serve-arxiv.pdf}
  \vspace{-0.15in}
  \caption{CDF of request latency for \MoE online serving.}
  \vspace{-0.2in}
  \label{fig:eval-online-serve}
\end{figure}



\subsection{Impact of Expert Cache Limits}



We measure the \TPOT of \sys and other baselines by limiting the expert cache memory budget to investigate their performance in the latency-memory trade-off (\S\ref{subsec:bg-latency-memory-tradeoff}).
We mainly focus on \TPOT to show the end-to-end performance impacted by varying cache limits.
Figure~\ref{fig:eval-cache-limit.pdf} shows the \TPOT of \sys and other four baselines when serving three \MoE models under different expert cache limits.
We gradually increase the GPU memory allocated for caching experts from 6 GB to 96 GB while employing the same experimental setting in \S\ref{subsec:eval-overall}.
Similarly, DeepSpeed-Inference has the worst \TPOT due to being expert-agnostic.
\sys consistently outperforms Mixtral-Offloading, ProMoE, and MoE-Infinity under varying expert cache limits.
Especially for limited GPU memory sizes (\eg, 6GB), \sys reduces the \TPOT by 32\%, 24\%, 18\%, and 18\%, compared to DeepSpeed-Inference, Mixtral-Offloading, ProMoE, and MoE-Infinity, across three \MoE models, respectively.
With fine-grained expert offloading, \sys significantly reduces the expert on-demand loading latency while maintaining a lower GPU memory footprint, therefore achieving a better spot in the latency-memory trade-off of \MoE serving.

% \subsection{Impact of Inference Batch Size}

\subsection{Ablation Study}
\label{subsec:eval-ablation}


% \begin{figure}[t]
%   \centering
%   \includegraphics[width=.95\linewidth]{figs/eval-expert-tracking.pdf}
%   % \vspace{-0.15in}
%   \caption{Expert hit rate of different expert pattern tracking approaches.}
%   % \vspace{-0.25in}
%   \label{fig:eval-expert-tracking}
% \end{figure}



We present the ablation study of \sys's design.


\textbf{Effectiveness of expert map search.}
One of \sys's key designs is the expert map, which tracks expert selection preferences in fine granularity.
We evaluate the effectiveness of the expert map against five expert pattern-tracking approaches as follows.
%
1) \textbf{Speculate}: speculative prediction used by Mixtral-Offloading~\cite{eliseev2023fast} and ProMoE~\cite{song2024promoe}, 
%
2) \textbf{Hit count}: request-level expert hit count used by MoE-Infinity~\cite{xue2024moe}, 
%
3) \textbf{Map (T)}: expert map with only trajectory similarity search,
4) \textbf{Map (T+S)}: expert map with both trajectory and semantic similarity search,
%
and
5) \textbf{Map (T+S+$\delta$)}: expert map with full features enabled, including trajectory and semantic similarity search (\S\ref{subsec:design-similarity-match}) and dynamic expert selection (\S\ref{subsec:design-expert-prefetch}).
%
We implement the above methods in \sys's Expert Map Matcher for a fair comparison.
Figure~\ref{fig:eval-expert-tracking} shows the expert hit rate of the above expert pattern tracking methods.
%
Speculative prediction is effective due to the widespread presence of residual connections in Transformer blocks. However, its effectiveness decreases drastically as prefetch distance increases~\cite{song2024promoe}.
%
The request-level expert activation count has the worst performance due to coarse granularity.
%
As features are incrementally restored to \sys's expert map, the expert hit rate gradually increases, demonstrating its effectiveness.

% \textbf{Effectiveness of asynchronous map matching.}




\begin{figure}[t]
  \centering
  \includegraphics[width=.9\linewidth]{figs/eval-cache-limit-arxiv.pdf}
  \vspace{-0.15in}
  \caption{Performance of \sys and other four baselines under varying expert cache limits.}
  \vspace{-0.1in}
  \label{fig:eval-cache-limit.pdf}
\end{figure}

\begin{figure}[!t]
    \centering
    \begin{subfigure}[t]{0.585\linewidth}
        \centering
        \includegraphics[width=\linewidth]{figs/eval-expert-tracking.pdf}
        \caption{Expert pattern tracking approaches.}
        \label{fig:eval-expert-tracking}
    \end{subfigure}
    % \hspace{0.02in}
    \begin{subfigure}[t]{0.385\linewidth}
        \centering
        \includegraphics[width=\linewidth]{figs/eval-prefetch-and-cache-arxiv.pdf}
        \caption{Prefetch and caching.}
        \label{fig:eval-prefetch-and-cache}
    \end{subfigure}
    \vspace{-0.1in}
    \caption{Ablation study of \sys.}
    \label{fig:eval-ablation}
    \vspace{-0.2in}
\end{figure}

\textbf{Effectiveness of expert prefetching and caching.}
We evaluate \sys's expert prefetching and caching against two caching algorithms:
1) \textbf{\LRU} used by Mixtral-Offloading~\cite{eliseev2023fast}
and 
2) \textbf{\LFU} used by MoE-Infinity~\cite{xue2024moe}.
%
Figure~\ref{fig:eval-prefetch-and-cache} depicts the expert hit rate of \sys and two baselines.
The results show that \LRU performs poorly in expert offloading scenarios. Though \LFU achieves a higher hit rate than \LRU, \sys surpasses both, achieving the highest expert hit rate.

\subsection{Sensitivity Analysis}
\label{subsec:eval-sensitivity}


\begin{figure}[t]
  \centering
  \includegraphics[width=.9\linewidth]{figs/eval-prefetch-distance.pdf}
  \vspace{-0.15in}
  \caption{Performance of \sys serving \MoE models with different prefetch distances.}
  \vspace{-0.1in}
  \label{fig:eval-prefetch-distance}
\end{figure}

% \begin{figure}[t]
%   \centering
%   \includegraphics[width=.9\linewidth]{figs/eval-store-capacity.pdf}
%   % \vspace{-0.15in}
%   \caption{Semantic and trajectory similarity lower bounds in \sys's serving with different Expert Map Store capacity.}
%   % \vspace{-0.25in}
%   \label{fig:eval-store-capacity}
% \end{figure}

\begin{figure}[t]
    \centering
    \begin{subfigure}[t]{0.55\linewidth}
        \centering
        \includegraphics[width=\linewidth]{figs/eval-store-capacity.pdf}
        \caption{Expert Map Store capacity.}
        \label{fig:eval-store-capacity}
    \end{subfigure}
    % \hspace{0.02in}
    \begin{subfigure}[t]{0.435\linewidth}
        \centering
        \includegraphics[width=\linewidth]{figs/eval-batch-size-arxiv.pdf}
        \caption{Inference batch size.}
        \label{fig:eval-batch-size}
    \end{subfigure}
    \vspace{-0.1in}
    \caption{Sensitivity analysis of \sys.}
    \vspace{-0.2in}
    \label{fig:eval-sensitivity}
\end{figure}


We analyze the sensitivity of three hyperparameters: prefetch distance of \MoE models, the capacity of Expert Map Store, and inference batch size.


\textbf{Prefetch distance of \MoE models.}
Figure~\ref{fig:eval-prefetch-distance} shows the \TTFT and \TPOT of \sys when serving three \MoE models with different prefetch distances.
%
We have demonstrated that the expert hit rate decreases when gradually increasing the prefetch distance (Figure~\ref{fig:bg-hit-distance}).
%
When the prefetch distance is small ($<3$), \sys cannot perfectly hide its system delay from the inference process, such as the map matching and expert prefetching, leading to the increase of inference latency.
%
With larger prefetch distances ($>3$), \sys has worse expert hit rates that also degrade the performance. 
Therefore, we set the prefetch distance $d$ to 3 for evaluating \sys.


\textbf{Capacity of Expert Map Store.}
We measure the mean semantic and trajectory similarity scores searched in \sys's expert map matching for \MoE model serving.
%
Figure~\ref{fig:eval-store-capacity} presents the mean semantic and trajectory similarity scores of \sys with different Expert Map Store capacity sizes.
%
Both semantic and trajectory similarity scores improve as the store capacity increases.
%
While the similarity scores exhibit a significant increase with capacities below 1K, further capacity expansion yields diminishing similarity gains. 
To minimize \sys's memory overhead, we set \sys's Expert Map Store capacity to 1K in evaluation.


\textbf{Inference batch size.}
We investigate the impact of inference batch size on \sys and three baselines using \mixtral with LMSYS-Chat-1M.
%
Figure~\ref{fig:eval-batch-size} presents the performance of \sys, Mixtral-Offloading, ProMoE, and MoE-Infinity as the batch size increases from one to four. \sys achieves the lowest \TTFT and \TPOT in most cases.


% \textbf{Inference batch size.}


% \subsection{Scalability}
% \label{subsec:eval-scalability}
% From one to six GPUs


\begin{figure}[t]
  \centering
  \includegraphics[width=.92\linewidth]{figs/eval-overhead-latency.pdf}
  \vspace{-0.15in}
  \caption{Latency breakdown of \sys's one inference iteration with three \MoE models.}
  \vspace{-0.1in}
  \label{fig:eval-overhead-latency.pdf}
\end{figure}





\subsection{System Overheads}
\label{subsec:eval-overhead}


\noindent \textbf{Latency overheads of \sys's operations.}
Figure~\ref{fig:eval-overhead-latency.pdf} shows the latency breakdown of one inference iteration in \sys when serving the three \MoE models.
We report any operations of \sys in \S\ref{subsec:eval-overall} that may incur a significant latency delay, including context collection, map matching, expert on-demand loading, expert prefetching, and map update after the iteration completes.
\qwen has lower end-to-end iteration latency than \mixtral and \phimoe because of significantly fewer parameters.
Note that expert prefetching, map matching, and map update tasks are executed asynchronously, aside from the inference process. Hence, they do not contribute to the end-to-end iteration latency.
Excluding three asynchronous tasks, the total delay incurred by other operations is consistently less than 30ms (5\% of the iteration) across three \MoE models, which is negligible compared to the inference latency.


\noindent \textbf{Memory overheads of \sys's Expert Map Store.}
Figure~\ref{fig:eval-overhead-memory.pdf} shows the CPU memory footprint of \sys's Expert Map Store when varying the store capacity from 1K to 32K maps.
The memory needed to store expert maps for \qwen is more than \mixtral and \phimoe because it has more experts per layer over the other two models, which increases the map shape.
Even for the largest capacity (32K), the Expert Map Store requires less than 200MB of memory to store the maps, which is trivial since modern GPU servers usually have abundant CPU memory (\eg, p4d.24xlarge on AWS EC2~\cite{aws-ec2} has over 1100 GB of CPU memory).
In the evaluation, \sys's map store capacity with 1K maps is sufficient for maintaining performance (\S\ref{subsec:eval-sensitivity}), resulting in minimal memory overhead.



\begin{figure}[t]
  \centering
  \includegraphics[width=.85\linewidth]{figs/eval-overhead-memory.pdf}
  % \vspace{-0.1in}
  \caption{CPU memory footprint of \sys's Expert Map Store with different capacity.}
  \vspace{-0.1in}
  \label{fig:eval-overhead-memory.pdf}
\end{figure}

% 
\section{Related Work} \label{sec:related}

% \textbf{Adversarial Attack}
\textbf{Attacks on SLAM.} 
%With the rise of machine learning, 
The robustness of computer vision systems is being actively investigated. With the emergence of adversarial images in the digital domain by adding optimized noise directly to images~\cite{szegedy2013intriguing,carlini2017towards}, researchers find that such attacks also exist physically in the real world \cite{eykholt2018robust,song2018physical,zhao2019seeing}. To fill the gap between attacks in the digital and physical worlds, recent studies have demonstrated that attacks on real-world computer vision systems are practical \cite{eykholt2018robust,li2019adversarial,man2020ghostimage,sharif2016accessorize,zhao2019seeing,zhou2018invisible}. However, attacks on traditional computer vision methods such as SLAM are relatively less explored. \cite{yoshida2022adversarial} proposes an attack against the scan matching algorithm in LiDAR-based SLAM, while most SLAMs in AR/VR devices rely on different sensors like RGB/depth cameras and IMUs. \cite{ikram2022perceptual} and \cite{chen2024adversary} mislead visual SLAM by poisoning the images with special patterns, and \cite{wang2021can} causes the camera to fail using infrared light. In our work, we demonstrate attacks on Visual-Inertial SLAM (VI-SLAM) by perturbing the IMU readings, rather than cameras, and showing its impact on XR user experience. 

\textbf{Acoustic Injection Attacks.} Among various physical attacks, acoustic injection attacks are attractive due to their low cost. Son~\etal~\cite{son2015rocking} were the first to introduce acoustic attacks on MEMS gyroscopes, demonstrating how these attacks could lead to sensor denial-of-service and result in drone crashes. WALNUT~\cite{trippel2017walnut} expanded on this by developing output biasing and control attacks that enable precise manipulation of MEMS accelerometer outputs using modulated sound waves. Wang et al.~\cite{wang2017sonic} demonstrated a sonic gun, showcasing the vulnerability of various smart devices (\eg drones and self-balancing vehicles) to acoustic attacks. Tu et al. \cite{tu2018injected} designed side-swing and switching attacks to alter the outputs of MEMS gyroscopes and accelerometers. Furthermore, Ji et al. \cite{ji2021poltergeist} fool the object detectors by applying acoustic attack to the image stabilizers commonly used in modern cameras. However, none of the existing works study the relationship between the acoustic injections and SLAM outputs on recent XR devices. 

% \zijian{Do we need one session about security in AR/VR?}
% \yicheng{TODO}
%\jiasi{cite the AIVR paper (UMass Amherst?) paper is we have not already. They add IMU perturbation but w/o SLAM, iirc} \yicheng{Cited}

\textbf{XR Security and Privacy.} 
%Security and privacy concerns in XR systems have gained significant attention. 
For single-user XR systems, researchers have demonstrated various side-channel attacks to extract sensitive information (\eg keystrokes) through video feeds~\cite{ling2019know}, head movements~\cite{nair2023unique, slocum2023going}, architectural hints~\cite{zhang2023its,shang2020arspy}, power usage~\cite{li2024dangers}, and EM side-channel leakages~\cite{al2021vr}. In multi-user XR systems, Su et al.~\cite{su2024remote} use avatar motion data to infer keystrokes in shared VR environments. Slocum et al.~\cite{slocum2024doesn} reveal vulnerabilities in the shared state frameworks of multi-user AR. Similarly, Lebeck et al.~\cite{lebeck2017securing} highlight risks like deceptive virtual objects and emphasize access control for managing shared physical and virtual spaces. Ruth et al.~\cite{ruth2019secure} further propose a secure multi-user AR framework focusing on content sharing and permissions.
Chandio et al.~\cite{chandio2024stealthy} %introduced a multi-modal spatiotemporal attack that 
simultaneously manipulated visual and inertial sensors to disrupt XR pose estimation. However, their study evaluated the attack using offline datasets and assumed the attacker's capability to manipulate IMU data streams through acoustic means, without real experiments. Ours is the first to demonstrate acoustic injection attacks on recent XR devices, like the Hololens 2, in the real world.
 




In this paper we have described our efforts in mechanizing the strand spaces framework~\cite{FHG98} in Coq.
To assess the flexibility of the approach and the usability of the library and of the proofs we have analyzed a variety of examples: a basic authentication protocol and some of its variants, the classical Needham-Schroeder-Lowe authentication protocol, and a recent key management API equipped with a key management policy.

Wherever possible, our mechanization remains faithful to the original pen-and-paper development of strand spaces.
At the same time, we put a lot of engineering effort to make the code and the proofs reusable.
For that, we have made the framework modular and parametric in the terms and the penetrator.
Additionally, we have developed a number of strands-specific tactics whose goal is to make the life of the protocol's analyst easier by removing some of the burden of these kinds of proofs.
Indeed, the tactics automate a number of intermediate steps enabling, in some cases, easy proof reuse.
For instance, the proof of the NSL responder's nonce secrecy
 required just one hour of work using the initiator's nonce secrecy.
The mechanization
gives the freedom to experiment with protocols and their properties, while retaining the unique ability of strand spaces-based analyses to give interesting insights on the inner workings of protocols.
With our experiments, we uncovered
and fixed issues, discarded
redundant or unused requirements, and significantly improved previous results on the analysis of key management policies, making it possible to formally prove the security of the \emph{secure templates} policy from \cite{BCFS-ccs10} (\cref{sec:casestudies}).

\cref{tab:simpleauth,tab:nsl} in \cref{sec:summary}  summarize the premises for each security property across the analyzed protocol variants. These premises are essential for our security proofs and offer important insights into the assumptions required to make a security protocol correct. The strand spaces model highlights this aspect, and the use of Coq and the \easystrands{} library further clarifies the minimal and necessary nature of these assumptions, reinforcing the model's ability to accurately capture security requirements.
With the insights from these experiments we also developed a new proof technique which we call \emph{protected predicate} technique that, in certain situations, simplifies the proofs making some previously challenging cases trivial.


Another advantage of having this mechanized platform is that it opens up new and interesting avenues of research.
\ifdefined\COLORDIFF
    \color{cbred}
\else
\fi
For instance, an intriguing enhancement to our framework would be the inclusion of algebraic intruders. We believe they can be implemented using at least two approaches, which we briefly outline below.

Given an equational theory $E$ over a signature $\mathit{FS}$, the first approach requires implementing $E$ as a (terminating and confluent) rewriting system \lstinline{rew_E}, and allow penetrators to use \lstinline{rew_E} to manipulate terms containing symbols of $\mathit{FS}$.
More concretely, we first need to create an instance of \easystrands{} terms with support for function symbols in $\mathit{FS}$, then we can extend the penetrator as:
\begin{lstlisting}
Inductive penetrator_strand : Σ -> Prop := ...
| PT_Eqn : forall (g h : 𝔸) i, replace g h rew_E  -> penetrator_strand (i, [⊖ g; ⊕ h]).
\end{lstlisting}
where \lstinline{replace g h rew_E} holds iff \lstinline{g} can be rewritten as \lstinline{h} under \lstinline{rew_E}.
This approach is inspired by that of Tamarin \cite{MSCB13}.

The second approach aligns  with the method used in DY*~\cite{DY}, where cryptographic primitives are modeled as functions that symbolically represent the actual primitives, e.g., \lstinline{dec (c, k) = (if c = enc (m, k) then m else Error)}.
With these definitions, the equational theory $E$ could be defined using Coq Setoids and used for terms in place of Leibniz equality.
This has the advantage to allow both honest parties and the intruder to transparently use the equational theory.
However, as observed by~\citet{DY}, this approach requires proving (at least) that $E$ is an equivalence relation respected by all functions, predicates, and protocol specifications which can be lengthy and tedious.
\ifdefined\COLORDIFF
    \color{black}
\else
\fi

Despite their age, strand spaces have been a catalyst for extensive research, leading to notable extensions that include authentication tests~\cite{guttman2000authentication}, process algebraic-style choice operators~\cite{YEMMS16},
 compositionality \cite{StrandComposition,StrandIndependence,StrandMixed}, and stateful protocols \cite{J12}.
Many of these advancements are crucial for enhancing the expressiveness and usability of the model.
Our plan is to enhance \easystrands{} by integrating these extensions, thereby enabling scalability to more realistic protocols.
Ultimately, this will help narrow the gap with state-of-the-art tools such as DY* \cite{DY}.
In terms of foundational research, an intriguing avenue involves closely examining the relationship between Paulson's inductive method \cite{Paulson94} and strand spaces. We plan to mechanize Paulson's method in Coq and conduct a comparative analysis to assess the relative merits of these two inductive methods.

Finally, we defined a maximal penetrator as the set of strands that do not violate sensitive cryptographic operations required for protocol security. This method is inspired by the approach in \cite{banaSymbolic} to achieve computational soundness and, to our knowledge, has not been explored in a purely symbolic context before. It allows for proving injective agreement without explicitly defining the Dolev-Yao attacker, which we showed to be \diff{strictly} subsumed by the maximal penetrator. Notably, this approach facilitates the composition of protocols proven secure under their respective maximal penetrators, provided they adhere to each other's constraints. We are currently extending this technique to protocols like NSL, where security relies on decryption capabilities.


\section*{Acknowledgements}
This is acknowledgment.

% \bibliographystyle{IEEEtran}
\documentclass[10pt,journal,compsoc]{IEEEtran}
\IEEEoverridecommandlockouts

% *** CITATION PACKAGES ***
%
\ifCLASSOPTIONcompsoc
  % The IEEE Computer Society needs nocompress option
  % requires cite.sty v4.0 or later (November 2003)
  \usepackage[nocompress]{cite}
\else
  % normal IEEE
  \usepackage{cite}
\fi
\usepackage{amsmath,amssymb,amsfonts}
\usepackage{algorithmic}
\usepackage{graphicx}
\usepackage{textcomp}
\usepackage[absolute]{textpos}


\usepackage{enumitem}
\usepackage{color}
\usepackage{multirow}
\usepackage{lipsum}
\usepackage{listings}
% \usepackage{subfigure}
\usepackage{graphicx}
\usepackage{pdfpages}
\usepackage{verbatim}
\usepackage{pifont}
\usepackage{etoolbox}
\usepackage{textcomp}
% \usepackage{ulem}
\usepackage{hyperref}
\usepackage{url}
\usepackage{caption}
\usepackage{booktabs} % For nice tables
\usepackage{siunitx} % To align table numbers by unit
\usepackage{etoolbox}

% \usepackage{booktabs}
% \usepackage{multirow}
% \usepackage{graphicx}
\usepackage[normalem]{ulem}
\useunder{\uline}{\ul}{}

\usepackage[switch]{lineno} % default option is 'left'

\usepackage[linesnumbered,titlenumbered,ruled,vlined,resetcount,algosection]{algorithm2e}
\usepackage{xcolor}
\usepackage{color}
\def\BibTeX{{\rm B\kern-.05em{\sc i\kern-.025em b}\kern-.08em
    T\kern-.1667em\lower.7ex\hbox{E}\kern-.125emX}}

\usepackage{amsthm}
\newtheorem{theorem}{Theorem}
%\newtheorem{theorem}{Theorem}[section]
%\newtheorem{corollary}{Corollary}[theorem]
\newtheorem{corollary}{Corollary}
%\newtheorem{lemma}[theorem]{Lemma}
\newtheorem{lemma}{Lemma}

\usepackage{hhline}
\usepackage{xcolor}
\usepackage{colortbl}
\usepackage{threeparttable}
% \usepackage{minted}
\usepackage{subfig}
\usepackage{comment}
\usepackage{enumitem}
% \usepackage[ruled,vlined]{algorithm2e}
% Circled number
\usepackage{tikz}
\newcommand*\circled[1]{\tikz[baseline=(char.base)]{
            \node[shape=circle,fill,inner sep=0.8pt] (char) {\footnotesize\textcolor{white}{#1}};}}
\newcommand*\redcircled[1]{\tikz[baseline=(char.base)]{
            \node[shape=circle,fill=red,inner sep=1.3pt] (char) {\textcolor{white}{#1}};}}
            
\def\BibTeX{{\rm B\kern-.05em{\sc i\kern-.025em b}\kern-.08em
    T\kern-.1667em\lower.7ex\hbox{E}\kern-.125emX}}

% \setlength{\floatsep}{1em}
% \setlength{\textfloatsep}{0.8em}
% \setlength{\intextsep}{0.1em}
% \setlength{\belowcaptionskip}{0.2em} 

\newcommand{\yguo}[1]{
  \begin{framed}
    \noindent{\textcolor{blue}{\bf\em #1 -- Yanfei Guo}}
  \end{framed}
}

\newcommand{\yujia}[1]{\textcolor{red}{#1}} % Track changes from Yujia

\newcommand{\fn}[1]{{\tt\small #1}}
%%

% \newcommand{\pname}{\textit{gZCCL}}
\newcommand{\pname}{\mbox{ZCCL}}
% \newcommand{\pname}{\mbox{\textbf{\texttt{ZCCL}}}}
\newcommand{\newtext}[1]{\textcolor{blue}{#1}}

\newcommand{\fzlight}{\mbox{\textit{fZ}-light}}


\begin{document}

\Urlmuskip=0mu plus 1mu


\title{ZCCL: Significantly Improving Collective Communication With Error-Bounded Lossy Compression}

\author{Jiajun Huang, Sheng Di,~\IEEEmembership{Senior Member,~IEEE}, Xiaodong Yu, Yujia Zhai, Zhaorui Zhang, Jinyang Liu, Xiaoyi Lu, Ken Raffenetti, Hui Zhou, Kai Zhao, Khalid Alharthi, Zizhong Chen,~\IEEEmembership{Senior Member,~IEEE}, Franck Cappello,~\IEEEmembership{Fellow,~IEEE}, Yanfei Guo, Rajeev Thakur,~\IEEEmembership{Fellow,~IEEE}
% Xiaodong Yu,\IEEEauthorrefmark{3}
% Yujia Zhai,\IEEEauthorrefmark{1}
% Zhaorui Zhang,\IEEEauthorrefmark{4}
% Jinyang Liu,\IEEEauthorrefmark{1}
% Xiaoyi Lu,\IEEEauthorrefmark{5}
% Ken Raffenetti\IEEEauthorrefmark{2}
% Hui Zhou\IEEEauthorrefmark{2}
% Kai Zhao\IEEEauthorrefmark{6}
% Zizhong Chen\IEEEauthorrefmark{1}
% Franck Cappello\IEEEauthorrefmark{2}
% Yanfei Guo\IEEEauthorrefmark{2}
% Rajeev Thakur\IEEEauthorrefmark{2}

\IEEEcompsocitemizethanks{
\IEEEcompsocthanksitem Jiajun Huang, Yujia Zhai, and Zizhong Chen are affiliated with the University of California, Riverside, CA 92521. Sheng Di, Ken Raffenetti, Hui Zhou, Franck Cappello, Yanfei Guo, and Rajeev Thakur are affiliated with Argonne National Laboratory, Lemont, IL 60439. Xiaodong Yu is affiliated with Stevens Institute of Technology, Hoboken, NJ 07030. Zhaorui Zhang is affiliated with The Hong Kong Polytechnic University, Kowloon, Hong Kong. Jinyang Liu is affiliated with University of Houston,
Houston, TX 77204. Xiaoyo Lu is is affiliated with University of California, Merced, CA 95343. Kai Zhao is affiliated with Florida State University, Tallahassee, FL 32306. Khalid Alharthi is affiliated with Department Of Computer Science, College of Computing And Information Technology, University Of Bisha, Bisha 61922, P.O. Box 551, Saudi Arabia. 
% \IEEEcompsocthanksitem A shortened version of this paper was presented at 2024 IEEE International Parallel
% and Distributed Processing Symposium (IPDPS) and was published in its \href{https://doi.ieeecomputersociety.org/10.1109/IPDPS57955.2024.00072}{proceedings}.
}

}

% The paper headers
% \markboth{Journal of \LaTeX\ Class Files,~Vol.~14, No.~8, August~2021}%
% {Shell \MakeLowercase{\textit{et al.}}: A Sample Article Using IEEEtran.cls for IEEE Journals}

% \IEEEpubid{0000--0000/00\$00.00~\copyright~2021 IEEE}

% for Computer Society papers, we must declare the abstract and index terms
% PRIOR to the title within the \IEEEtitleabstractindextext IEEEtran
% command as these need to go into the title area created by \maketitle.
% As a general rule, do not put math, special symbols or citations
% in the abstract or keywords.
\IEEEtitleabstractindextext{%

\begin{abstract}



With the ever-increasing computing power of supercomputers and the growing scale of scientific applications, the efficiency of MPI collective communication turns out to be a critical bottleneck in large-scale distributed and parallel processing. The large message size in MPI collectives is particularly concerning because it can significantly degrade overall parallel performance. To address this issue, prior research simply applies off-the-shelf fixed-rate lossy compressors in the MPI collectives, leading to suboptimal performance, limited generalizability, and unbounded errors. In this paper, we propose a novel solution, called {\pname}, which leverages error-bounded lossy compression to significantly reduce the message size, resulting in a substantial reduction in communication costs. The key contributions are three-fold. (1) We develop two general, optimized lossy-compression-based frameworks for both types of MPI collectives (collective data movement as well as collective computation), based on their particular characteristics. Our framework not only reduces communication costs but also preserves data accuracy. (2) We customize {\fzlight}, an ultra-fast error-bounded lossy compressor, to meet the specific needs of collective communication. (3) We integrate {\pname} into multiple collectives, such as Allgather, Allreduce, Scatter, and Broadcast, and perform a comprehensive evaluation based on real-world scientific application datasets. Experiments show that our solution outperforms the original MPI collectives as well as multiple baselines by 1.9--8.9$\times$.

\end{abstract}

\begin{IEEEkeywords}
Error-bounded Lossy Compression, Collective Communication, Distributed Computing, Parallel Algorithm
\end{IEEEkeywords}
}
\maketitle

%%
%% This command processes the author and affiliation and title
%% information and builds the first part of the formatted document.
\maketitle

\input{tex/introduction}
% \input{tex/motivation}

\input{tex/background}
\input{tex/design}
% \input{tex/challenge}
% \input{tex/optimization}
\input{tex/eval}
% \input{tex/related}
\input{tex/concl}
\input{tex/ack.tex}

% \bibliographystyle{IEEEtran}
\input{ACMmain.bbl}

\begin{IEEEbiography}
[{\includegraphics[width=1in,height=1.25in,clip,keepaspectratio]{picture/authors/jiajun.jpg}}]{Jiajun Huang} is a Ph.D. candidate in Computer Science at the University of California, Riverside, and a long-term visiting student at Argonne National Laboratory. He received his bachelor's degree in Electronic Information Engineering from the University of Electronic Science and Technology of China (UESTC) and the University of Glasgow (Honors of the First Class), in 2021. His research interests include distributed and parallel computing/systems, high-performance computing, and general artificial intelligence. Email: jhuan380@ucr.edu
\end{IEEEbiography}
\vskip -2\baselineskip plus -1fil

\begin{IEEEbiography}[{\includegraphics[width=1in,height=1.25in,clip,keepaspectratio]{picture/authors/Di.jpg}}]{Sheng Di}
(Senior Member, IEEE) received his master's degree from Huazhong University of Science and Technology in 2007 and Ph.D. degree from the University of Hong Kong in 2011. He is currently a computer scientist at Argonne National Laboratory. 
His research interests involve resilience on high-performance computing (such as silent data corruption, optimization checkpoint model, and in situ data compression) and broad research topics on cloud computing.
He is working on multiple HPC projects, such as detection of silent data corruption, characterization of failures and faults for HPC systems, and optimization of multilevel checkpoint models. He is the recipient of a DOE 2021 Early Career Research Program award. Email: sdi@anl.gov.
\end{IEEEbiography}

\vskip -2\baselineskip plus -1fil

\begin{IEEEbiography}[{\includegraphics[width=1in,height=1.25in,clip,keepaspectratio]{picture/authors/YuXiaodong.jpg}}]{Xiaodong Yu}
is an Assistant Professor in the Computer Science Department at Stevens Institute of Technology. He was an Assistant Computer Scientist at Argonne National Laboratory. He earned his Ph.D. in Computer Science from Virginia Tech in 2019. His research areas span parallel and distributed computing, next-generation AI hardware, and machine learning privacy and security. Email: xyu38@stevens.edu.
\end{IEEEbiography}

\vskip -2\baselineskip plus -1fil

\begin{IEEEbiography}[{\includegraphics[width=1in,height=1.25in,clip,keepaspectratio]{picture/authors/Zhai.jpg}}]{Yujia Zhai} received his bachelor's degree from the University of Science and Technology of China in 2016, a master's degree from Duke University in 2018, and a Ph.D. degree from the University of California, Riverside in 2023. He is interested in performance optimization for math libraries on GPUs. Email: yzhai015@ucr.edu.
\end{IEEEbiography}


\vskip -2\baselineskip plus -1fil

\begin{IEEEbiography}[{\includegraphics[width=1in,height=1.25in,clip,keepaspectratio]{picture/authors/Zhaorui.jpg}}]{Zhaorui Zhang} is currently a research assistant professor in the Department of Computing at The Hong Kong Polytechnic University. She received her Ph.D. from the Department of Computer Science at The University of Hong Kong, Hong Kong, and her BSc degree in computer science from Xi'an Jiaotong University. Her research interests include distributed machine learning systems, distributed systems, HPC, cloud computing, and data reduction. Email: zhaorui.zhang@polyu.edu.hk.
\end{IEEEbiography}


\vskip -2\baselineskip plus -1fil

\begin{IEEEbiography}[{\includegraphics[width=1in,height=1.25in,clip,keepaspectratio]{picture/authors/Jinyang.jpg}}]{Jinyang Liu} is an assistant professor at the department of Computer Science in the University of Houston. Jinyang’s research lies in the interdisciplinary areas of High-Performance Computing, Scientific Data Management, and Artificial Intelligence. He has multiple published or accepted works in various highly prestigious conferences and journals such as ACM SIGMOD, IEEE/ACM SC, ACM ICS (one paper in the best paper finalist), IEEE ICDE, IEEE Cluster, IEEE BigData, IEEE TPDS, etc.
\end{IEEEbiography}

\vskip -2\baselineskip plus -1fil

\begin{IEEEbiography}[{\includegraphics[width=1in,height=1.25in,clip,keepaspectratio]{picture/authors/luxi.jpg}}]{Xiaoyi Lu} is an Associate Professor in the Department of Computer Science and Engineering at the University of California, Merced. His current research interests include parallel and distributed computing, high-performance networking and I/O technologies, big data analytics, cloud computing, and deep learning. He has published one book and more than 150 papers in prestigious international conferences, workshops, and journals with multiple Best (Student) Paper Awards or Nominations. Dr. Lu has received the NSF CAREER Award and other research awards from Meta, Amazon, and Google. Email:  xiaoyi.lu@ucmerced.edu.
\end{IEEEbiography}


\vskip -2\baselineskip plus -1fil

\begin{IEEEbiography}[{\includegraphics[width=1in,height=1.25in,clip,keepaspectratio]{picture/authors/Raffenetti_Kenneth.jpeg}}]{Ken Raffenetti} is a Principal Software Development Specialist in the Programming Models and Runtime Systems group at Argonne National Laboratory. He is a core MPICH developer and active participant in several HPC industry working groups. Prior to Argonne, Ken earned a B.S. in Computer Science from the University of Illinois Urbana-Champaign.
\end{IEEEbiography}

\vskip -2\baselineskip plus -1fil


\begin{IEEEbiography}[{\includegraphics[width=1in,height=1.25in,clip,keepaspectratio]{picture/authors/ZhouHui-profile.jpg}}]{Hui Zhou} is a Principal Research Software Engineer at Argonne National Laboratory and a core member of the MPICH development team. His research focuses on runtime systems for high-performance computing, accessible parallel computing, and scalable software development. He has a particular interest in enhancing interoperability between runtime systems.
\end{IEEEbiography}


\vskip -2\baselineskip plus -1fil

\begin{IEEEbiography}[{\includegraphics[width=1in,height=1.25in,clip,keepaspectratio]{picture/authors/Zhao.png}}]{Kai Zhao} is an assistant professor in the computer science department at Florida State University. He
received his bachelor's degree from Peking University in 2014 and his Ph.D. degree from the University of California, Riverside in 2022.
His research interests include high-performance computing and scientific data management. Email: kai.zhao@fsu.edu.
\end{IEEEbiography}

\vskip -2\baselineskip plus -1fil

\begin{IEEEbiography}[{\includegraphics[width=0.9in,height=1.25in,clip,keepaspectratio]{picture/authors/Khalid_Alharthi.png}}]{Khalid Ayed Alharthi} is an assistant professor at the University of Bisha, KSA. He received his bachelor's degree from King Khalid University in 2008, his master's degree from Kent State University, USA in 2013, and his Ph.D. from the University of Warwick, UK in 2023. He is a long-term intern at UChicago Argonne National Laboratory and the Alan Turing Institute, UK. His research interests include AI, NLP, and AI in supporting resilience in HPC systems. Email: kharthi@ub.edu.sa.
\end{IEEEbiography}


\vskip -2\baselineskip plus -1fil

\begin{IEEEbiography}[{\includegraphics[width=1in,height=1.25in,clip,keepaspectratio]{picture/authors/Chen.jpg}}]{Zizhong Chen}
% \begin{IEEEbiographynophoto}{Zizhong Chen} 
(Senior Member, IEEE) received a bachelor's degree in mathematics from Beijing Normal University, a master's degree in economics from the Renmin University of China, and a Ph.D. degree in computer science from the University of Tennessee, Knoxville. He is a professor of computer science at the University of California, Riverside. 
His research interests include high-performance computing, parallel and distributed systems, big data analytics, cluster and cloud computing, algorithm-based fault tolerance, power and energy efficient computing, numerical algorithms and software, and large-scale computer simulations. His research has been supported by the National Science Foundation, Department of Energy, CMG Reservoir Simulation Foundation, Abu Dhabi National Oil Company, Nvidia, and Microsoft Corporation. 
He received a CAREER Award from the US National Science Foundation and a Best Paper Award from the International Supercomputing Conference. He is a Senior Member of the IEEE and a Life Member of the ACM. Email: chen@cs.ucr.edu.
\end{IEEEbiography}

\vskip -2\baselineskip plus -1fil

\begin{IEEEbiography}[{\includegraphics[width=1in,height=1.25in,clip,keepaspectratio]{picture/authors/Cappello.jpg}}]{Franck Cappello}
% \begin{IEEEbiographynophoto}{Franck Cappello}
%(Fellow, IEEE) 
is the director of the Joint-Laboratory on Extreme Scale Computing gathering six of the leading high-performance computing institutions in the world: Argonne National Laboratory, National Center for Scientific Applications, Inria, Barcelona Supercomputing Center, Julich Supercomputing Center, and Riken AICS. He is a senior computer scientist at Argonne National Laboratory and an adjunct associate professor in the Department of Computer Science at the University of Illinois at Urbana-Champaign. He is an expert in resilience and fault tolerance for scientific computing and data analytics. Recently he started investigating lossy compression for scientific datasets to respond to the pressing needs of scientist performing large-scale simulations and experiments. His contribution to this domain is one of the best lossy compressors for scientific datasets respecting user-set error bounds. He is a member of the editorial board of the \textit{IEEE Transactions on Parallel and Distributed Computing} and of the \textit{ACM HPDC} and \textit{IEEE CCGRID} steering committees. He is a fellow of the IEEE. Email: cappello@mcs.anl.gov.
\end{IEEEbiography}



\vskip -2\baselineskip plus -1fil

\begin{IEEEbiography}[{\includegraphics[width=1in,height=1.25in,clip,keepaspectratio]{picture/authors/yguo.jpg}}] {Yanfei Guo} holds an appointment as an Computer Scientist
at the Argonne National Laboratory. He is a member of the Programming
Models and the Runtime Systems group. He has been working on multiple
software projects including MPI, Yaksa and OSHMPI. His research
interests include parallel programming models and runtime systems in
extreme-scale supercomputing systems, data-intensive computing and cloud
computing systems. Yanfei has received the best paper award at the
USENIX International Conference on Autonomic Computing 2013 (ICAC'13).
His work on programming models and runtime systems has been published on
peer-reviewed conferences and journals including the ACM/IEEE
Supercomputing Conference (SC'14, SC'15) and IEEE Transactions on
Parallel and Distributed Systems (TPDS). Yanfei have delivered eight
tutorials on MPI to various audience levels from university students to
researchers. Yanfei served as reviewers and technical committee members
in many journals and conferences. He is a member of the IEEE and a
member the ACM.
\end{IEEEbiography}

\vskip -2\baselineskip plus -1fil

\begin{IEEEbiography}[{\includegraphics[width=1in,height=1.25in,clip,keepaspectratio]{picture/authors/rajeev.jpg}}]{Rajeev Thakur} is an Argonne Distinguished Fellow and Deputy Director of the Data Science and Learning Division at Argonne National Laboratory. He received a Ph.D. in Computer Engineering from Syracuse University. His research interests are in high-performance computing, parallel programming models, runtime systems, communication libraries, scalable parallel I/O, and artificial intelligence and machine learning. He is a Fellow of IEEE.
\end{IEEEbiography}


\end{document}
\endinput
%%
%% End of file `sample-sigconf.tex'.


\begin{IEEEbiography}
[{\includegraphics[width=1in,height=1.25in,clip,keepaspectratio]{picture/authors/jiajun.jpg}}]{Jiajun Huang} is a Ph.D. candidate in Computer Science at the University of California, Riverside, and a long-term visiting student at Argonne National Laboratory. He received his bachelor's degree in Electronic Information Engineering from the University of Electronic Science and Technology of China (UESTC) and the University of Glasgow (Honors of the First Class), in 2021. His research interests include distributed and parallel computing/systems, high-performance computing, and general artificial intelligence. Email: jhuan380@ucr.edu
\end{IEEEbiography}
\vskip -2\baselineskip plus -1fil

\begin{IEEEbiography}[{\includegraphics[width=1in,height=1.25in,clip,keepaspectratio]{picture/authors/Di.jpg}}]{Sheng Di}
(Senior Member, IEEE) received his master's degree from Huazhong University of Science and Technology in 2007 and Ph.D. degree from the University of Hong Kong in 2011. He is currently a computer scientist at Argonne National Laboratory. 
His research interests involve resilience on high-performance computing (such as silent data corruption, optimization checkpoint model, and in situ data compression) and broad research topics on cloud computing.
He is working on multiple HPC projects, such as detection of silent data corruption, characterization of failures and faults for HPC systems, and optimization of multilevel checkpoint models. He is the recipient of a DOE 2021 Early Career Research Program award. Email: sdi@anl.gov.
\end{IEEEbiography}

\vskip -2\baselineskip plus -1fil

\begin{IEEEbiography}[{\includegraphics[width=1in,height=1.25in,clip,keepaspectratio]{picture/authors/YuXiaodong.jpg}}]{Xiaodong Yu}
is an Assistant Professor in the Computer Science Department at Stevens Institute of Technology. He was an Assistant Computer Scientist at Argonne National Laboratory. He earned his Ph.D. in Computer Science from Virginia Tech in 2019. His research areas span parallel and distributed computing, next-generation AI hardware, and machine learning privacy and security. Email: xyu38@stevens.edu.
\end{IEEEbiography}

\vskip -2\baselineskip plus -1fil

\begin{IEEEbiography}[{\includegraphics[width=1in,height=1.25in,clip,keepaspectratio]{picture/authors/Zhai.jpg}}]{Yujia Zhai} received his bachelor's degree from the University of Science and Technology of China in 2016, a master's degree from Duke University in 2018, and a Ph.D. degree from the University of California, Riverside in 2023. He is interested in performance optimization for math libraries on GPUs. Email: yzhai015@ucr.edu.
\end{IEEEbiography}


\vskip -2\baselineskip plus -1fil

\begin{IEEEbiography}[{\includegraphics[width=1in,height=1.25in,clip,keepaspectratio]{picture/authors/Zhaorui.jpg}}]{Zhaorui Zhang} is currently a research assistant professor in the Department of Computing at The Hong Kong Polytechnic University. She received her Ph.D. from the Department of Computer Science at The University of Hong Kong, Hong Kong, and her BSc degree in computer science from Xi'an Jiaotong University. Her research interests include distributed machine learning systems, distributed systems, HPC, cloud computing, and data reduction. Email: zhaorui.zhang@polyu.edu.hk.
\end{IEEEbiography}


\vskip -2\baselineskip plus -1fil

\begin{IEEEbiography}[{\includegraphics[width=1in,height=1.25in,clip,keepaspectratio]{picture/authors/Jinyang.jpg}}]{Jinyang Liu} is an assistant professor at the department of Computer Science in the University of Houston. Jinyang’s research lies in the interdisciplinary areas of High-Performance Computing, Scientific Data Management, and Artificial Intelligence. He has multiple published or accepted works in various highly prestigious conferences and journals such as ACM SIGMOD, IEEE/ACM SC, ACM ICS (one paper in the best paper finalist), IEEE ICDE, IEEE Cluster, IEEE BigData, IEEE TPDS, etc.
\end{IEEEbiography}

\vskip -2\baselineskip plus -1fil

\begin{IEEEbiography}[{\includegraphics[width=1in,height=1.25in,clip,keepaspectratio]{picture/authors/luxi.jpg}}]{Xiaoyi Lu} is an Associate Professor in the Department of Computer Science and Engineering at the University of California, Merced. His current research interests include parallel and distributed computing, high-performance networking and I/O technologies, big data analytics, cloud computing, and deep learning. He has published one book and more than 150 papers in prestigious international conferences, workshops, and journals with multiple Best (Student) Paper Awards or Nominations. Dr. Lu has received the NSF CAREER Award and other research awards from Meta, Amazon, and Google. Email:  xiaoyi.lu@ucmerced.edu.
\end{IEEEbiography}


\vskip -2\baselineskip plus -1fil

\begin{IEEEbiography}[{\includegraphics[width=1in,height=1.25in,clip,keepaspectratio]{picture/authors/Raffenetti_Kenneth.jpeg}}]{Ken Raffenetti} is a Principal Software Development Specialist in the Programming Models and Runtime Systems group at Argonne National Laboratory. He is a core MPICH developer and active participant in several HPC industry working groups. Prior to Argonne, Ken earned a B.S. in Computer Science from the University of Illinois Urbana-Champaign.
\end{IEEEbiography}

\vskip -2\baselineskip plus -1fil


\begin{IEEEbiography}[{\includegraphics[width=1in,height=1.25in,clip,keepaspectratio]{picture/authors/ZhouHui-profile.jpg}}]{Hui Zhou} is a Principal Research Software Engineer at Argonne National Laboratory and a core member of the MPICH development team. His research focuses on runtime systems for high-performance computing, accessible parallel computing, and scalable software development. He has a particular interest in enhancing interoperability between runtime systems.
\end{IEEEbiography}


\vskip -2\baselineskip plus -1fil

\begin{IEEEbiography}[{\includegraphics[width=1in,height=1.25in,clip,keepaspectratio]{picture/authors/Zhao.png}}]{Kai Zhao} is an assistant professor in the computer science department at Florida State University. He
received his bachelor's degree from Peking University in 2014 and his Ph.D. degree from the University of California, Riverside in 2022.
His research interests include high-performance computing and scientific data management. Email: kai.zhao@fsu.edu.
\end{IEEEbiography}

\vskip -2\baselineskip plus -1fil

\begin{IEEEbiography}[{\includegraphics[width=0.9in,height=1.25in,clip,keepaspectratio]{picture/authors/Khalid_Alharthi.png}}]{Khalid Ayed Alharthi} is an assistant professor at the University of Bisha, KSA. He received his bachelor's degree from King Khalid University in 2008, his master's degree from Kent State University, USA in 2013, and his Ph.D. from the University of Warwick, UK in 2023. He is a long-term intern at UChicago Argonne National Laboratory and the Alan Turing Institute, UK. His research interests include AI, NLP, and AI in supporting resilience in HPC systems. Email: kharthi@ub.edu.sa.
\end{IEEEbiography}


\vskip -2\baselineskip plus -1fil

\begin{IEEEbiography}[{\includegraphics[width=1in,height=1.25in,clip,keepaspectratio]{picture/authors/Chen.jpg}}]{Zizhong Chen}
% \begin{IEEEbiographynophoto}{Zizhong Chen} 
(Senior Member, IEEE) received a bachelor's degree in mathematics from Beijing Normal University, a master's degree in economics from the Renmin University of China, and a Ph.D. degree in computer science from the University of Tennessee, Knoxville. He is a professor of computer science at the University of California, Riverside. 
His research interests include high-performance computing, parallel and distributed systems, big data analytics, cluster and cloud computing, algorithm-based fault tolerance, power and energy efficient computing, numerical algorithms and software, and large-scale computer simulations. His research has been supported by the National Science Foundation, Department of Energy, CMG Reservoir Simulation Foundation, Abu Dhabi National Oil Company, Nvidia, and Microsoft Corporation. 
He received a CAREER Award from the US National Science Foundation and a Best Paper Award from the International Supercomputing Conference. He is a Senior Member of the IEEE and a Life Member of the ACM. Email: chen@cs.ucr.edu.
\end{IEEEbiography}

\vskip -2\baselineskip plus -1fil

\begin{IEEEbiography}[{\includegraphics[width=1in,height=1.25in,clip,keepaspectratio]{picture/authors/Cappello.jpg}}]{Franck Cappello}
% \begin{IEEEbiographynophoto}{Franck Cappello}
%(Fellow, IEEE) 
is the director of the Joint-Laboratory on Extreme Scale Computing gathering six of the leading high-performance computing institutions in the world: Argonne National Laboratory, National Center for Scientific Applications, Inria, Barcelona Supercomputing Center, Julich Supercomputing Center, and Riken AICS. He is a senior computer scientist at Argonne National Laboratory and an adjunct associate professor in the Department of Computer Science at the University of Illinois at Urbana-Champaign. He is an expert in resilience and fault tolerance for scientific computing and data analytics. Recently he started investigating lossy compression for scientific datasets to respond to the pressing needs of scientist performing large-scale simulations and experiments. His contribution to this domain is one of the best lossy compressors for scientific datasets respecting user-set error bounds. He is a member of the editorial board of the \textit{IEEE Transactions on Parallel and Distributed Computing} and of the \textit{ACM HPDC} and \textit{IEEE CCGRID} steering committees. He is a fellow of the IEEE. Email: cappello@mcs.anl.gov.
\end{IEEEbiography}



\vskip -2\baselineskip plus -1fil

\begin{IEEEbiography}[{\includegraphics[width=1in,height=1.25in,clip,keepaspectratio]{picture/authors/yguo.jpg}}] {Yanfei Guo} holds an appointment as an Computer Scientist
at the Argonne National Laboratory. He is a member of the Programming
Models and the Runtime Systems group. He has been working on multiple
software projects including MPI, Yaksa and OSHMPI. His research
interests include parallel programming models and runtime systems in
extreme-scale supercomputing systems, data-intensive computing and cloud
computing systems. Yanfei has received the best paper award at the
USENIX International Conference on Autonomic Computing 2013 (ICAC'13).
His work on programming models and runtime systems has been published on
peer-reviewed conferences and journals including the ACM/IEEE
Supercomputing Conference (SC'14, SC'15) and IEEE Transactions on
Parallel and Distributed Systems (TPDS). Yanfei have delivered eight
tutorials on MPI to various audience levels from university students to
researchers. Yanfei served as reviewers and technical committee members
in many journals and conferences. He is a member of the IEEE and a
member the ACM.
\end{IEEEbiography}

\vskip -2\baselineskip plus -1fil

\begin{IEEEbiography}[{\includegraphics[width=1in,height=1.25in,clip,keepaspectratio]{picture/authors/rajeev.jpg}}]{Rajeev Thakur} is an Argonne Distinguished Fellow and Deputy Director of the Data Science and Learning Division at Argonne National Laboratory. He received a Ph.D. in Computer Engineering from Syracuse University. His research interests are in high-performance computing, parallel programming models, runtime systems, communication libraries, scalable parallel I/O, and artificial intelligence and machine learning. He is a Fellow of IEEE.
\end{IEEEbiography}


\end{document}
\endinput
%%
%% End of file `sample-sigconf.tex'.


\begin{IEEEbiography}
[{\includegraphics[width=1in,height=1.25in,clip,keepaspectratio]{picture/authors/jiajun.jpg}}]{Jiajun Huang} is a Ph.D. candidate in Computer Science at the University of California, Riverside, and a long-term visiting student at Argonne National Laboratory. He received his bachelor's degree in Electronic Information Engineering from the University of Electronic Science and Technology of China (UESTC) and the University of Glasgow (Honors of the First Class), in 2021. His research interests include distributed and parallel computing/systems, high-performance computing, and general artificial intelligence. Email: jhuan380@ucr.edu
\end{IEEEbiography}
\vskip -2\baselineskip plus -1fil

\begin{IEEEbiography}[{\includegraphics[width=1in,height=1.25in,clip,keepaspectratio]{picture/authors/Di.jpg}}]{Sheng Di}
(Senior Member, IEEE) received his master's degree from Huazhong University of Science and Technology in 2007 and Ph.D. degree from the University of Hong Kong in 2011. He is currently a computer scientist at Argonne National Laboratory. 
His research interests involve resilience on high-performance computing (such as silent data corruption, optimization checkpoint model, and in situ data compression) and broad research topics on cloud computing.
He is working on multiple HPC projects, such as detection of silent data corruption, characterization of failures and faults for HPC systems, and optimization of multilevel checkpoint models. He is the recipient of a DOE 2021 Early Career Research Program award. Email: sdi@anl.gov.
\end{IEEEbiography}

\vskip -2\baselineskip plus -1fil

\begin{IEEEbiography}[{\includegraphics[width=1in,height=1.25in,clip,keepaspectratio]{picture/authors/YuXiaodong.jpg}}]{Xiaodong Yu}
is an Assistant Professor in the Computer Science Department at Stevens Institute of Technology. He was an Assistant Computer Scientist at Argonne National Laboratory. He earned his Ph.D. in Computer Science from Virginia Tech in 2019. His research areas span parallel and distributed computing, next-generation AI hardware, and machine learning privacy and security. Email: xyu38@stevens.edu.
\end{IEEEbiography}

\vskip -2\baselineskip plus -1fil

\begin{IEEEbiography}[{\includegraphics[width=1in,height=1.25in,clip,keepaspectratio]{picture/authors/Zhai.jpg}}]{Yujia Zhai} received his bachelor's degree from the University of Science and Technology of China in 2016, a master's degree from Duke University in 2018, and a Ph.D. degree from the University of California, Riverside in 2023. He is interested in performance optimization for math libraries on GPUs. Email: yzhai015@ucr.edu.
\end{IEEEbiography}


\vskip -2\baselineskip plus -1fil

\begin{IEEEbiography}[{\includegraphics[width=1in,height=1.25in,clip,keepaspectratio]{picture/authors/Zhaorui.jpg}}]{Zhaorui Zhang} is currently a research assistant professor in the Department of Computing at The Hong Kong Polytechnic University. She received her Ph.D. from the Department of Computer Science at The University of Hong Kong, Hong Kong, and her BSc degree in computer science from Xi'an Jiaotong University. Her research interests include distributed machine learning systems, distributed systems, HPC, cloud computing, and data reduction. Email: zhaorui.zhang@polyu.edu.hk.
\end{IEEEbiography}


\vskip -2\baselineskip plus -1fil

\begin{IEEEbiography}[{\includegraphics[width=1in,height=1.25in,clip,keepaspectratio]{picture/authors/Jinyang.jpg}}]{Jinyang Liu} is an assistant professor at the department of Computer Science in the University of Houston. Jinyang’s research lies in the interdisciplinary areas of High-Performance Computing, Scientific Data Management, and Artificial Intelligence. He has multiple published or accepted works in various highly prestigious conferences and journals such as ACM SIGMOD, IEEE/ACM SC, ACM ICS (one paper in the best paper finalist), IEEE ICDE, IEEE Cluster, IEEE BigData, IEEE TPDS, etc.
\end{IEEEbiography}

\vskip -2\baselineskip plus -1fil

\begin{IEEEbiography}[{\includegraphics[width=1in,height=1.25in,clip,keepaspectratio]{picture/authors/luxi.jpg}}]{Xiaoyi Lu} is an Associate Professor in the Department of Computer Science and Engineering at the University of California, Merced. His current research interests include parallel and distributed computing, high-performance networking and I/O technologies, big data analytics, cloud computing, and deep learning. He has published one book and more than 150 papers in prestigious international conferences, workshops, and journals with multiple Best (Student) Paper Awards or Nominations. Dr. Lu has received the NSF CAREER Award and other research awards from Meta, Amazon, and Google. Email:  xiaoyi.lu@ucmerced.edu.
\end{IEEEbiography}


\vskip -2\baselineskip plus -1fil

\begin{IEEEbiography}[{\includegraphics[width=1in,height=1.25in,clip,keepaspectratio]{picture/authors/Raffenetti_Kenneth.jpeg}}]{Ken Raffenetti} is a Principal Software Development Specialist in the Programming Models and Runtime Systems group at Argonne National Laboratory. He is a core MPICH developer and active participant in several HPC industry working groups. Prior to Argonne, Ken earned a B.S. in Computer Science from the University of Illinois Urbana-Champaign.
\end{IEEEbiography}

\vskip -2\baselineskip plus -1fil


\begin{IEEEbiography}[{\includegraphics[width=1in,height=1.25in,clip,keepaspectratio]{picture/authors/ZhouHui-profile.jpg}}]{Hui Zhou} is a Principal Research Software Engineer at Argonne National Laboratory and a core member of the MPICH development team. His research focuses on runtime systems for high-performance computing, accessible parallel computing, and scalable software development. He has a particular interest in enhancing interoperability between runtime systems.
\end{IEEEbiography}


\vskip -2\baselineskip plus -1fil

\begin{IEEEbiography}[{\includegraphics[width=1in,height=1.25in,clip,keepaspectratio]{picture/authors/Zhao.png}}]{Kai Zhao} is an assistant professor in the computer science department at Florida State University. He
received his bachelor's degree from Peking University in 2014 and his Ph.D. degree from the University of California, Riverside in 2022.
His research interests include high-performance computing and scientific data management. Email: kai.zhao@fsu.edu.
\end{IEEEbiography}

\vskip -2\baselineskip plus -1fil

\begin{IEEEbiography}[{\includegraphics[width=0.9in,height=1.25in,clip,keepaspectratio]{picture/authors/Khalid_Alharthi.png}}]{Khalid Ayed Alharthi} is an assistant professor at the University of Bisha, KSA. He received his bachelor's degree from King Khalid University in 2008, his master's degree from Kent State University, USA in 2013, and his Ph.D. from the University of Warwick, UK in 2023. He is a long-term intern at UChicago Argonne National Laboratory and the Alan Turing Institute, UK. His research interests include AI, NLP, and AI in supporting resilience in HPC systems. Email: kharthi@ub.edu.sa.
\end{IEEEbiography}


\vskip -2\baselineskip plus -1fil

\begin{IEEEbiography}[{\includegraphics[width=1in,height=1.25in,clip,keepaspectratio]{picture/authors/Chen.jpg}}]{Zizhong Chen}
% \begin{IEEEbiographynophoto}{Zizhong Chen} 
(Senior Member, IEEE) received a bachelor's degree in mathematics from Beijing Normal University, a master's degree in economics from the Renmin University of China, and a Ph.D. degree in computer science from the University of Tennessee, Knoxville. He is a professor of computer science at the University of California, Riverside. 
His research interests include high-performance computing, parallel and distributed systems, big data analytics, cluster and cloud computing, algorithm-based fault tolerance, power and energy efficient computing, numerical algorithms and software, and large-scale computer simulations. His research has been supported by the National Science Foundation, Department of Energy, CMG Reservoir Simulation Foundation, Abu Dhabi National Oil Company, Nvidia, and Microsoft Corporation. 
He received a CAREER Award from the US National Science Foundation and a Best Paper Award from the International Supercomputing Conference. He is a Senior Member of the IEEE and a Life Member of the ACM. Email: chen@cs.ucr.edu.
\end{IEEEbiography}

\vskip -2\baselineskip plus -1fil

\begin{IEEEbiography}[{\includegraphics[width=1in,height=1.25in,clip,keepaspectratio]{picture/authors/Cappello.jpg}}]{Franck Cappello}
% \begin{IEEEbiographynophoto}{Franck Cappello}
%(Fellow, IEEE) 
is the director of the Joint-Laboratory on Extreme Scale Computing gathering six of the leading high-performance computing institutions in the world: Argonne National Laboratory, National Center for Scientific Applications, Inria, Barcelona Supercomputing Center, Julich Supercomputing Center, and Riken AICS. He is a senior computer scientist at Argonne National Laboratory and an adjunct associate professor in the Department of Computer Science at the University of Illinois at Urbana-Champaign. He is an expert in resilience and fault tolerance for scientific computing and data analytics. Recently he started investigating lossy compression for scientific datasets to respond to the pressing needs of scientist performing large-scale simulations and experiments. His contribution to this domain is one of the best lossy compressors for scientific datasets respecting user-set error bounds. He is a member of the editorial board of the \textit{IEEE Transactions on Parallel and Distributed Computing} and of the \textit{ACM HPDC} and \textit{IEEE CCGRID} steering committees. He is a fellow of the IEEE. Email: cappello@mcs.anl.gov.
\end{IEEEbiography}



\vskip -2\baselineskip plus -1fil

\begin{IEEEbiography}[{\includegraphics[width=1in,height=1.25in,clip,keepaspectratio]{picture/authors/yguo.jpg}}] {Yanfei Guo} holds an appointment as an Computer Scientist
at the Argonne National Laboratory. He is a member of the Programming
Models and the Runtime Systems group. He has been working on multiple
software projects including MPI, Yaksa and OSHMPI. His research
interests include parallel programming models and runtime systems in
extreme-scale supercomputing systems, data-intensive computing and cloud
computing systems. Yanfei has received the best paper award at the
USENIX International Conference on Autonomic Computing 2013 (ICAC'13).
His work on programming models and runtime systems has been published on
peer-reviewed conferences and journals including the ACM/IEEE
Supercomputing Conference (SC'14, SC'15) and IEEE Transactions on
Parallel and Distributed Systems (TPDS). Yanfei have delivered eight
tutorials on MPI to various audience levels from university students to
researchers. Yanfei served as reviewers and technical committee members
in many journals and conferences. He is a member of the IEEE and a
member the ACM.
\end{IEEEbiography}

\vskip -2\baselineskip plus -1fil

\begin{IEEEbiography}[{\includegraphics[width=1in,height=1.25in,clip,keepaspectratio]{picture/authors/rajeev.jpg}}]{Rajeev Thakur} is an Argonne Distinguished Fellow and Deputy Director of the Data Science and Learning Division at Argonne National Laboratory. He received a Ph.D. in Computer Engineering from Syracuse University. His research interests are in high-performance computing, parallel programming models, runtime systems, communication libraries, scalable parallel I/O, and artificial intelligence and machine learning. He is a Fellow of IEEE.
\end{IEEEbiography}


\end{document}
\endinput
%%
%% End of file `sample-sigconf.tex'.


\begin{IEEEbiography}
[{\includegraphics[width=1in,height=1.25in,clip,keepaspectratio]{picture/authors/jiajun.jpg}}]{Jiajun Huang} is a Ph.D. candidate in Computer Science at the University of California, Riverside, and a long-term visiting student at Argonne National Laboratory. He received his bachelor's degree in Electronic Information Engineering from the University of Electronic Science and Technology of China (UESTC) and the University of Glasgow (Honors of the First Class), in 2021. His research interests include distributed and parallel computing/systems, high-performance computing, and general artificial intelligence. Email: jhuan380@ucr.edu
\end{IEEEbiography}
\vskip -2\baselineskip plus -1fil

\begin{IEEEbiography}[{\includegraphics[width=1in,height=1.25in,clip,keepaspectratio]{picture/authors/Di.jpg}}]{Sheng Di}
(Senior Member, IEEE) received his master's degree from Huazhong University of Science and Technology in 2007 and Ph.D. degree from the University of Hong Kong in 2011. He is currently a computer scientist at Argonne National Laboratory. 
His research interests involve resilience on high-performance computing (such as silent data corruption, optimization checkpoint model, and in situ data compression) and broad research topics on cloud computing.
He is working on multiple HPC projects, such as detection of silent data corruption, characterization of failures and faults for HPC systems, and optimization of multilevel checkpoint models. He is the recipient of a DOE 2021 Early Career Research Program award. Email: sdi@anl.gov.
\end{IEEEbiography}

\vskip -2\baselineskip plus -1fil

\begin{IEEEbiography}[{\includegraphics[width=1in,height=1.25in,clip,keepaspectratio]{picture/authors/YuXiaodong.jpg}}]{Xiaodong Yu}
is an Assistant Professor in the Computer Science Department at Stevens Institute of Technology. He was an Assistant Computer Scientist at Argonne National Laboratory. He earned his Ph.D. in Computer Science from Virginia Tech in 2019. His research areas span parallel and distributed computing, next-generation AI hardware, and machine learning privacy and security. Email: xyu38@stevens.edu.
\end{IEEEbiography}

\vskip -2\baselineskip plus -1fil

\begin{IEEEbiography}[{\includegraphics[width=1in,height=1.25in,clip,keepaspectratio]{picture/authors/Zhai.jpg}}]{Yujia Zhai} received his bachelor's degree from the University of Science and Technology of China in 2016, a master's degree from Duke University in 2018, and a Ph.D. degree from the University of California, Riverside in 2023. He is interested in performance optimization for math libraries on GPUs. Email: yzhai015@ucr.edu.
\end{IEEEbiography}


\vskip -2\baselineskip plus -1fil

\begin{IEEEbiography}[{\includegraphics[width=1in,height=1.25in,clip,keepaspectratio]{picture/authors/Zhaorui.jpg}}]{Zhaorui Zhang} is currently a research assistant professor in the Department of Computing at The Hong Kong Polytechnic University. She received her Ph.D. from the Department of Computer Science at The University of Hong Kong, Hong Kong, and her BSc degree in computer science from Xi'an Jiaotong University. Her research interests include distributed machine learning systems, distributed systems, HPC, cloud computing, and data reduction. Email: zhaorui.zhang@polyu.edu.hk.
\end{IEEEbiography}


\vskip -2\baselineskip plus -1fil

\begin{IEEEbiography}[{\includegraphics[width=1in,height=1.25in,clip,keepaspectratio]{picture/authors/Jinyang.jpg}}]{Jinyang Liu} is an assistant professor at the department of Computer Science in the University of Houston. Jinyang’s research lies in the interdisciplinary areas of High-Performance Computing, Scientific Data Management, and Artificial Intelligence. He has multiple published or accepted works in various highly prestigious conferences and journals such as ACM SIGMOD, IEEE/ACM SC, ACM ICS (one paper in the best paper finalist), IEEE ICDE, IEEE Cluster, IEEE BigData, IEEE TPDS, etc.
\end{IEEEbiography}

\vskip -2\baselineskip plus -1fil

\begin{IEEEbiography}[{\includegraphics[width=1in,height=1.25in,clip,keepaspectratio]{picture/authors/luxi.jpg}}]{Xiaoyi Lu} is an Associate Professor in the Department of Computer Science and Engineering at the University of California, Merced. His current research interests include parallel and distributed computing, high-performance networking and I/O technologies, big data analytics, cloud computing, and deep learning. He has published one book and more than 150 papers in prestigious international conferences, workshops, and journals with multiple Best (Student) Paper Awards or Nominations. Dr. Lu has received the NSF CAREER Award and other research awards from Meta, Amazon, and Google. Email:  xiaoyi.lu@ucmerced.edu.
\end{IEEEbiography}


\vskip -2\baselineskip plus -1fil

\begin{IEEEbiography}[{\includegraphics[width=1in,height=1.25in,clip,keepaspectratio]{picture/authors/Raffenetti_Kenneth.jpeg}}]{Ken Raffenetti} is a Principal Software Development Specialist in the Programming Models and Runtime Systems group at Argonne National Laboratory. He is a core MPICH developer and active participant in several HPC industry working groups. Prior to Argonne, Ken earned a B.S. in Computer Science from the University of Illinois Urbana-Champaign.
\end{IEEEbiography}

\vskip -2\baselineskip plus -1fil


\begin{IEEEbiography}[{\includegraphics[width=1in,height=1.25in,clip,keepaspectratio]{picture/authors/ZhouHui-profile.jpg}}]{Hui Zhou} is a Principal Research Software Engineer at Argonne National Laboratory and a core member of the MPICH development team. His research focuses on runtime systems for high-performance computing, accessible parallel computing, and scalable software development. He has a particular interest in enhancing interoperability between runtime systems.
\end{IEEEbiography}


\vskip -2\baselineskip plus -1fil

\begin{IEEEbiography}[{\includegraphics[width=1in,height=1.25in,clip,keepaspectratio]{picture/authors/Zhao.png}}]{Kai Zhao} is an assistant professor in the computer science department at Florida State University. He
received his bachelor's degree from Peking University in 2014 and his Ph.D. degree from the University of California, Riverside in 2022.
His research interests include high-performance computing and scientific data management. Email: kai.zhao@fsu.edu.
\end{IEEEbiography}

\vskip -2\baselineskip plus -1fil

\begin{IEEEbiography}[{\includegraphics[width=0.9in,height=1.25in,clip,keepaspectratio]{picture/authors/Khalid_Alharthi.png}}]{Khalid Ayed Alharthi} is an assistant professor at the University of Bisha, KSA. He received his bachelor's degree from King Khalid University in 2008, his master's degree from Kent State University, USA in 2013, and his Ph.D. from the University of Warwick, UK in 2023. He is a long-term intern at UChicago Argonne National Laboratory and the Alan Turing Institute, UK. His research interests include AI, NLP, and AI in supporting resilience in HPC systems. Email: kharthi@ub.edu.sa.
\end{IEEEbiography}


\vskip -2\baselineskip plus -1fil

\begin{IEEEbiography}[{\includegraphics[width=1in,height=1.25in,clip,keepaspectratio]{picture/authors/Chen.jpg}}]{Zizhong Chen}
% \begin{IEEEbiographynophoto}{Zizhong Chen} 
(Senior Member, IEEE) received a bachelor's degree in mathematics from Beijing Normal University, a master's degree in economics from the Renmin University of China, and a Ph.D. degree in computer science from the University of Tennessee, Knoxville. He is a professor of computer science at the University of California, Riverside. 
His research interests include high-performance computing, parallel and distributed systems, big data analytics, cluster and cloud computing, algorithm-based fault tolerance, power and energy efficient computing, numerical algorithms and software, and large-scale computer simulations. His research has been supported by the National Science Foundation, Department of Energy, CMG Reservoir Simulation Foundation, Abu Dhabi National Oil Company, Nvidia, and Microsoft Corporation. 
He received a CAREER Award from the US National Science Foundation and a Best Paper Award from the International Supercomputing Conference. He is a Senior Member of the IEEE and a Life Member of the ACM. Email: chen@cs.ucr.edu.
\end{IEEEbiography}

\vskip -2\baselineskip plus -1fil

\begin{IEEEbiography}[{\includegraphics[width=1in,height=1.25in,clip,keepaspectratio]{picture/authors/Cappello.jpg}}]{Franck Cappello}
% \begin{IEEEbiographynophoto}{Franck Cappello}
%(Fellow, IEEE) 
is the director of the Joint-Laboratory on Extreme Scale Computing gathering six of the leading high-performance computing institutions in the world: Argonne National Laboratory, National Center for Scientific Applications, Inria, Barcelona Supercomputing Center, Julich Supercomputing Center, and Riken AICS. He is a senior computer scientist at Argonne National Laboratory and an adjunct associate professor in the Department of Computer Science at the University of Illinois at Urbana-Champaign. He is an expert in resilience and fault tolerance for scientific computing and data analytics. Recently he started investigating lossy compression for scientific datasets to respond to the pressing needs of scientist performing large-scale simulations and experiments. His contribution to this domain is one of the best lossy compressors for scientific datasets respecting user-set error bounds. He is a member of the editorial board of the \textit{IEEE Transactions on Parallel and Distributed Computing} and of the \textit{ACM HPDC} and \textit{IEEE CCGRID} steering committees. He is a fellow of the IEEE. Email: cappello@mcs.anl.gov.
\end{IEEEbiography}



\vskip -2\baselineskip plus -1fil

\begin{IEEEbiography}[{\includegraphics[width=1in,height=1.25in,clip,keepaspectratio]{picture/authors/yguo.jpg}}] {Yanfei Guo} holds an appointment as an Computer Scientist
at the Argonne National Laboratory. He is a member of the Programming
Models and the Runtime Systems group. He has been working on multiple
software projects including MPI, Yaksa and OSHMPI. His research
interests include parallel programming models and runtime systems in
extreme-scale supercomputing systems, data-intensive computing and cloud
computing systems. Yanfei has received the best paper award at the
USENIX International Conference on Autonomic Computing 2013 (ICAC'13).
His work on programming models and runtime systems has been published on
peer-reviewed conferences and journals including the ACM/IEEE
Supercomputing Conference (SC'14, SC'15) and IEEE Transactions on
Parallel and Distributed Systems (TPDS). Yanfei have delivered eight
tutorials on MPI to various audience levels from university students to
researchers. Yanfei served as reviewers and technical committee members
in many journals and conferences. He is a member of the IEEE and a
member the ACM.
\end{IEEEbiography}

\vskip -2\baselineskip plus -1fil

\begin{IEEEbiography}[{\includegraphics[width=1in,height=1.25in,clip,keepaspectratio]{picture/authors/rajeev.jpg}}]{Rajeev Thakur} is an Argonne Distinguished Fellow and Deputy Director of the Data Science and Learning Division at Argonne National Laboratory. He received a Ph.D. in Computer Engineering from Syracuse University. His research interests are in high-performance computing, parallel programming models, runtime systems, communication libraries, scalable parallel I/O, and artificial intelligence and machine learning. He is a Fellow of IEEE.
\end{IEEEbiography}


\end{document}
\endinput
%%
%% End of file `sample-sigconf.tex'.
