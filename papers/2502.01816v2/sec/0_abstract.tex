\begin{abstract}
The tradeoff between reconstruction quality and compute required for video super-resolution (VSR) remains a formidable challenge in its adoption for deployment on resource-constrained edge devices. While transformer-based VSR models have set new benchmarks for reconstruction quality in recent years, these require substantial computational resources. On the other hand, lightweight models that have been introduced even recently struggle to deliver state-of-the-art reconstruction. We propose a novel lightweight and parameter-efficient neural architecture for VSR that achieves state-of-the-art reconstruction accuracy with just 2.3 million parameters. Our model enhances information utilization based on several architectural attributes. Firstly, it uses 2D wavelet decompositions  strategically interlayered with learnable convolutional layers to utilize the inductive prior of spatial sparsity of edges in visual data. Secondly, it uses a single memory tensor to capture inter-frame temporal information while avoiding the computational cost of previous memory-based schemes. Thirdly, it uses residual deformable convolutions for implicit inter-frame object alignment that improve upon deformable convolutions by enhancing spatial information in inter-frame feature differences. Architectural insights from our model can pave the way for real-time VSR on the edge, such as display devices for streaming data.

%Transformer-based video super-resolution (VSR) models have set new benchmarks for reconstruction quality in recent years, but their substantial computational demands make most of them unsuitable for deployment on resource-constrained devices. Achieving a balance between model complexity and output quality remains a formidable challenge in VSR. Although lightweight models have been introduced to address this issue, they often struggle to deliver state-of-the-art reconstruction. We propose a novel lightweight, parameter-efficient deep residual deformable convolution network for VSR. Unlike prior methods, our model enhances feature utilization through residual connections and employs deformable convolution for precise frame alignment, addressing motion dynamics effectively. Furthermore, we introduce a single memory tensor to capture information accrued from the past frames and improve motion estimation across frames along with a wavelet-based enhancement to strengthen frequency content in super-resolved frames. This design enables an efficient balance between computational cost and reconstruction quality. With just 2.3 million parameters, our model achieves state-of-the-art SSIM of 0.9175 on the REDS4 dataset, surpassing existing lightweight and many heavy models in both accuracy and resource efficiency. Architectural insights from our model pave the way for real-time VSR on streaming data.

\end{abstract}