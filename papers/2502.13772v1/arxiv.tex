\documentclass[11pt]{article}
\usepackage{fullpage}
\usepackage{amsmath,amssymb,graphicx,amsthm,dsfont}
\usepackage{color,soul}
\usepackage{xspace}
\usepackage{enumerate}
\usepackage{expdlist}
\usepackage{mathtools}
\usepackage{bbm}

\usepackage{multicol}
\usepackage{booktabs}
\usepackage[round]{natbib}%[numbers]
\usepackage{float}
\usepackage{graphicx}
\usepackage{tikz-cd}
\usepackage{tikz}
\usetikzlibrary{positioning,chains,fit,shapes,calc,snakes}
\usepackage[normalem]{ulem}
\usepackage[ruled]{algorithm2e} % For algorithms
%Problem names
\usepackage{cleveref}
\usepackage{xcolor}


\newcommand{\rep}{\ensuremath{\text{rep}}}
\newcommand{\h}{\ensuremath{\mathbf{h}}}
\newcommand{\x}{\ensuremath{\mathbf{x}}}
\newcommand{\y}{\ensuremath{\mathbf{y}}}
\newcommand{\rr}{\ensuremath{\mathbf{r}}}
\newcommand{\rank}{\ensuremath{\text{rank}}}
\newcommand{\profile}{\boldsymbol{\succ}}
\newcommand{\plu}{\ensuremath{\text{plu}}}
\newcommand{\LP}{\ensuremath{\text{LP}}}
% the following package is optional:
%\usepackage{latexsym}

% See https://www.overleaf.com/learn/latex/theorems_and_proofs
% for a nice explanation of how to define new theorems, but keep
% in mind that the amsthm package is already included in this
% template and that you must *not* alter the styling.
\newtheorem{theorem}{Theorem}
\newtheorem{proposition}[theorem]{Proposition}
\newtheorem{lemma}[theorem]{Lemma}
\newtheorem{claim}[theorem]{Claim}
\newtheorem{corollary}[theorem]{Corollary}
\newtheorem{definition}[theorem]{Definition}
\newtheorem{observation}[theorem]{Observation}
\newtheorem*{example}{Example}
\newtheorem*{remark}{Remark}
\newtheorem{question}{Open Problem}

\newcommand{\R}{\ensuremath{\mathds{R}}}
\newcommand{\Q}{\ensuremath{\mathds{Q}}}

\newcommand{\pref}{\ensuremath{{\succ}}}
\newcommand{\wpref}{\ensuremath{{\succeq}}}
\newcommand{\indif}{\ensuremath{\sim}}

\newcommand{\val}{\ensuremath{\mathsf{val}}}
\newcommand{\util}{\ensuremath{\mathcal{U}}}
\newcommand{\Ordinalstability}{Ordinal stability\xspace}
\newcommand{\oblocking}{ordinally blocking\xspace}

\newcommand{\csm}{cardinal stable matching\xspace}
\newcommand{\osm}{ordinal stable matching\xspace}
\newcommand{\med}{median\xspace}





\newcommand{\SR}[1]{{\color{red}{SR:#1}}}


\title{Quantile agent utility and implications to randomized social choice}
\author{Ioannis Caragiannis\thanks{Department of Computer Science, Aarhus University, Denmark. Email: iannis@cs.au.dk.} \and Sanjukta Roy\thanks{Indian Statistical Institute, Kolkata, India. Email: sanjukta@isical.ac.in}}
\date{}


\begin{document}

\maketitle

\begin{abstract}
We initiate a novel direction in randomized social choice by proposing a new definition of agent utility for randomized outcomes. Each agent has a preference over all outcomes and a {\em quantile} parameter. Given a {\em lottery} over the outcomes, an agent gets utility from a particular {\em representative}, defined as the least preferred outcome that can be realized so that the probability that any worse-ranked outcome can be realized is at most the agent's quantile value. 

In contrast to other utility models that have been considered in randomized social choice (e.g., stochastic dominance, expected utility), our {\em quantile agent utility} compares two lotteries for an agent by just comparing the representatives, as is done for deterministic outcomes.

We revisit questions in randomized social choice using the new utility definition. We study the compatibility of efficiency and strategyproofness for randomized voting rules, efficiency and fairness for randomized one-sided matching mechanisms, and efficiency, stability, and strategyproofness for lotteries over two-sided matchings. In contrast to well-known impossibilities in randomized social choice, we show that satisfying the above properties simultaneously can be possible.
\end{abstract}


\section{Introduction}
%We introduce a new utility model and study its implications to randomized social choice~\cite{B17}. In particular, we consider three different settings, namely voting, one-sided matching, and two-sided matching.
%
Randomized mechanisms are pivotal in social choice and matching theory, offering innovative solutions to challenges such as fairness, efficiency, and strategyproofness. Recent global events have further underscored their importance. For instance, during the COVID-19 pandemic, randomized algorithms played a critical role in vaccine allocation to ensure equitable distribution across diverse populations. Today, randomized mechanisms are increasingly employed to assign limited resources such as hospital beds, educational opportunities, and public housing to applicants in a manner that balances fairness and efficiency. In voting theory, randomized mechanisms have been used to break ties or select representatives fairly in multi-agent decision-making environments, such as participatory budgeting or citizen assemblies, where no deterministic method can satisfy all stakeholders. 
%These examples highlight how randomized approaches can address deeply complex and high-stakes resource allocation and decision-making problems.

Traditional approaches in {\em randomized social choice} have often grappled with inherent incompatibilities. For instance, achieving important axioms such as both efficiency and strategyproofness simultaneously has been shown to be fundamentally problematic~\citep{a18}. Gibbard's theorem identifies (randomized versions of) dictatorships as the only voting systems that satisfy the two axioms simultaneously~\citep{G77chance}. However,  such a negative result strongly depends on assumptions regarding how the voters (or agents) evaluate the voting outcome. 
%demonstrates that strategy-proofness in voting systems often leads to dictatorial outcomes in deterministic settings, motivating the use of randomization to achieve fairness and avoid strategic manipulation. Moreover, while randomized mechanisms possess desirable normative properties, their implementation raises practical concerns about verifiability and determinism, as evidenced by debates surrounding fairness in randomized school choice programs and resource-sharing platforms.

In this paper, we propose a novel definition of agent utility that is specifically tailored to randomized outcomes. Our framework shifts the focus from traditional comparisons, such as stochastic dominance or expected utility (e.g., see~\citealt{B17}), to defining utility based on an agent's least preferred outcome that meets their probabilistic expectations. By adopting this perspective, we aim to reconcile the often conflicting objectives of efficiency, stability, and strategyproofness in social choice mechanisms.

More specifically, the study of axiomatic properties in social choice theory has been traditionally based on how agents compare different outcomes. For example, in a classical voting scenario, one would like to aggregate preferences over alternatives submitted by the voters (or agents) into a single winning alternative. An agent's preference (usually, a strict ranking) over the alternatives determines how the agent evaluates different deterministic outcomes that may be returned by a voting rule and compares them to each other. Similarly, in a two-sided matching instance, we are given two disjoint sets of agents (e.g., men and women in a stable marriage instance), with each agent having a preference on the agents of the other side. Important properties like stability of matchings are defined using the agents' preferences.

The relatively recent trend of randomized social choice aims to study the role of randomization in the procedures above. How do agents evaluate the outcome of randomized voting rules, random assignments in one-sided matching instances, or fractional two-sided matchings? For example, in the voting scenario mentione above, a randomized voting rule takes as input a profile of agent preferences and returns a random alternative according to a probability distribution. In other words, a randomized voting rule takes as input a profile of agent preferences (rankings over the alternatives) and returns a {\em lottery} over the alternatives. 

The important question now is: how can an agent decide which among two lotteries is better using her preference ranking of the alternatives? The answer is not trivial anymore. For example, how can an agent with preference $a\succ b \succ c$ over three alternatives $a$, $b$, and $c$ compare the lottery that returns equiprobably the three alternatives with the lottery that returns alternative $b$ with probability $1$? There have been several answers in the literature. One approach is to assume that the agent has a valuation function that assigns a cardinal utility for each alternative that respect the preference relation of the agent for the alternatives. In our example, the utility for alternative $a$ has to be higher than that for alternative $b$, which is turn has to be higher than that for alternative $c$. The two lotteries can now be compared according to the expected utility of the agent for the alternative they return. This approach requires the existence of underlying cardinal utilities, which the agents should be able to compute; this is not always realistic.

Another prominent approach in randomized social choice is to compare lotteries using the notion of stochastic dominance. According to it, a lottery is better than another for an agent if it yields higher expected utility for her for every possible underlying utilities for the alternatives that respect her preference ranking of the agent. This is a very stringent definition which may result in two lotteries being incomparable. For example, the comparison between the two lotteries mentioned above gives a different outcome for the underlying utilities $(1, 0.2, 0)$ and $(1, 0.8, 0)$ for alternatives $a$, $b$, and $c$, respectively. The expected utilities for the two lotteries are $0.4$ and $0.2$ for the former and $0.6$ and $0.8$ for the latter.

In this work, we present a new agent utility model for the comparison of lotteries. Each agent has a {\em quantile parameter} (a scalar between $0$ and $1$), and associates each lottery over alternatives with the highest-ranked alternative that has the following properties: it has positive probability and it is such that the less preferred alternatives have total probability that does not exceed the quantile parameter. We use the term {\em representative} to refer to this alternative. Then, the comparison of two lotteries by an agent is simply the comparison of their representative alternatives according to the agent's ranking. We use the term {\em quantile agent utility} to refer to our new utility model.


Our quantile agent utility model has several advantages. First, the comparison between lotteries can take place in the very same way as in the case of deterministic outcomes. Second, as we will see, it turns out that stochastic dominance is as stringent in the comparison between two lotteries as our utility definition would be by considering all possible quantile parameters. And, of course, there are no underlying cardinal utilities for the alternatives that the agent needs to be able to evaluate the lotteries.

The different quantile parameters can model agents with different risk levels in their interpretation of a lottery. As should be expected, different quantile parameter values can lead to different results. For example, if all agents have their quantile parameter equal to $0$, we usually recover the impossibilities for deterministic mechanisms. If, instead, all quantile parameters are equal to $1$, trivial mechanisms like uniform lotteries are usually ideal. Intermediate quantile parameter values, possibly different for each agent, allow for a spectrum of positive and negative results. 

\subsection{Our contribution}
The definition of the quantile agent utility model is our main conceptual contribution (see Section~\ref{sec:concept}). More importantly, we explore its implications to three social choice settings: voting, one-, and two-sided matchings.

\paragraph{Voting.} In Section~\ref{sec:voting}, we consider the classical ranking-based voting scenario, in which a voting rule aims to select a winning alternative (candidate) taking as input rankings of the available alternatives that are submitted as votes by a set of agents (voters). The voting rule is randomized, and the agents have quantile utilities. We adapt the definition of (Pareto) efficiency in this context and first observe that no lottery can be efficient for all vectors of quantile parameters for the agents unless all agents agree on their top-ranked alternative. This justifies our choice to focus on scenarios in which each agent has their quantile parameter, which in turn can be used in the definition of the lottery returned by the voting rule. We then adapt the notion of strategyproofness and explore its relation to monotonicity in the new context for the case of two alternatives. We conclude this section with the more intricate case of three alternatives and profiles in which all agents have the same quantile parameter. For high values of the quantile parameter, we present non-dictatorial voting rules that are efficient and strategyproof. For small values, we show that deterministic dictatorships are the only efficient and strategyproof rules. Extending our characterization for all possible quantile parameter vectors is left as an (apparently challenging) important open problem.

\paragraph{One-sided matching.} In Section~\ref{sec:matching}, we consider scenarios with a set of agents that have to be matched (in a one-to-one manner) to a set of items of equal size. Each agent has a ranking of the items representing her preference over them and a quantile parameter. We consider randomized matching mechanisms that return lotteries over matchings. These lotteries define a representative item for each agent, the position of this item in the agent's ranking indicates how desirable the lottery is for the agent. We begin by considering a simple variant of social welfare in this context and show that optimal lotteries over matchings can be easily computed. We also discuss more elaborate definitions of social welfare, which seem to be more challenging to optimize. Again, we adapt the notions of efficiency and strategyproofness and demonstrate how to adapt serial dictatorship in our context using linear programming. Next, motivated by the fair division literature, we consider fairness properties. We define the analog of proportionality and show how to adapt serial dictatorships to get a matching mechanism that computes efficient and proportional lotteries. Unfortunately, our rule is not strategyproof. Whether matching mechanisms that are simultaneously efficient, strategyproof, and proportional exist is another open problem for the one-sided matching setting. Furthermore, we give a natural definition of envy-freeness. Our definitions guarantee that envy-freeness implies proportionality (in sync with the classical fair division literature). We observe that envy-freeness may not be consistent with efficiency; whether envy-free and efficient lotteries can be computed in polynomial time is another open problem for one-sided matchings.

\paragraph{Two-sided matching.} In Section~\ref{sec:stable-matching}, we consider another matching scenario with two disjoint sets of agents in which each agent has a quantile parameter, and every agent in one set has a ranking of the agents in the other set as their preference. We briefly consider a variant of social welfare but move quickly to redefining the notion of stability. We present structural observations for stable lotteries over matchings and discuss their relations with integral stable matchings and ordinal stability from the classical stable matchings literature. We demonstrate that stability and efficiency are not compatible for small values of the quantile parameters and present possibility results for a restricted version of efficiency, strategyproofness, and stability when the quantile parameter is above $1/2$. Determining the range of quantile parameters that allow for the (unrestricted version of the) three properties simultaneously is the main open problem for two-sided matchings under agents with quantile utilities.

\subsection{Related work}
Lotteries over alternatives and were first formally studied by~\citet{zeckhauser1969majority}, \citet{fishburn1972lotteries}, and \citet{intriligator1973probabilistic}.
Randomization in voting processes have been considered to address impossibility results in deterministic social choice mechanisms, such as those established by Arrow’s Impossibility Theorem~\citep{arrow1951}. One of the most famous results in this area is by~\citet{Gib73} and~\citet{Sat75}, who showed that deterministic strategy-proof mechanisms must be dictatorial, inspiring the use of randomization to achieve strategy-proofness and fairness. \citet{bogomolnaia2001} 
proposed the Random Priority (RP) mechanism, ensuring fairness and strategy-proofness in resource allocation and voting.
\citet{procaccia2010} designed randomized voting rules that are approximation of score-based deterministic voting rules to achieve strategy-proofness. 
\citet{a18} show that the efficiency and strategyproofness of a system can vary depending on how preferences over alternatives are extended to preferences over lotteries. They consider preference extensions using stochastic dominance (SD),  pairwise comparisons (PC), bilinear dominance (BD), and sure-thing principle (ST). They showed that while random serial dictatorships are PC-strategy-proof, they only achieve ex post efficiency, strict maximal lotteries are both PC-efficient and ST-strategy-proof, and show multiple incompatibility results. See also the survey of related results by~\citet{B17}.

In resource allocation problems consisting of agents that need to be matched to items, envy-freeness~\citep{foley1966resource,varian1973equity} and proportionality~\citep{steinhaus1948problem} have been widely studied in both economics and computer science as measures of fairness.  
\citet{hylland1979efficient} focused on the probabilistic allocations of indivisible items without relying on a priority structure and showed that this approach is more efficient than randomization over deterministic methods based on priority. \citet{abdulkadirouglu2003ordinal} studied lotteries in the house allocation problem, which is another variation of the problem of matching under one-sided preferences.
\citet{aziz2019probabilistic} explored the advantages of applying randomization in social choice scenarios, such as fair division, and examined the associated challenges. \citet{caragiannis2021interim} studied interim envy-freeness for lotteries. For resource allocation, obtaining both ex-post and ex-ante guarantees simultaneously has also been considered~\citep{babaioff2022best,aziz2024best}.  

In the two-sided matching literature, a version of the random-matching problem, where both sides of the market have strict preferences, was studied by \citet{RothRothblumVate1993lsm-lattice}, who showed that the set of fractional stable matchings has a lattice structure. One can interpret a fractional deterministic matching as a random matching using Birkhoff-von Neumann's theorem \citep[Theorem~3.2.6]{horn_topics_1991}.  
Thus, stable fractional matchings has been studied as a relaxation of the integrality constraint in the linear equations encoding stability~\citep{vandevate_linear_1989,abeledo_stable_1994,teo1998geometry} which were forerunner of numerous other works~\citep{BiroCechFleiner08stablehalfmatchings,KiPRST2013osmhypegraphic,KU2015,IK2018osm-hypergraphic,aziz_random_2019,CFKV20j,dogan_efficiency_2016,manjunath2013stability,manjunath2014markets}.
In particular, stable fractional matchings have been considered by \citet{AF03Scarfs}, who proposed the definition of ordinal stability. We remark that, in our setting with agents having quantile utilities, stability generalizes both ordinal and integral stability. \citet{AF03Scarfs} studied ordinal stability in the hypergraphic setting and found that Scarf’s lemma from game theory guarantees the existence of ordinally stable matchings.
Later, \citet{aziz_random_2019} studied multiple notions of stability and the relations between them.
\citet{alkan2003stable} studied a type of stability for fractional matching when both sides
are equipped with complex preference structures.
\citet{CFKV20j} defined cardinal stable lotteries when both sides have cardinal preferences as opposed to rankings or ordinal preferences. \citet{chen2020fractional} studied both cardinal and ordinal stability. 
Lotteries have been extensively studied for another application of matching under preferences, namely, the school choice problem \citep{abdulkadirouglu2003school,KU2015}. In particular, \citet{KU2015} studied ordinal stability, which they refer to as ex-ante stability.
 

\section{The quantile utility model }\label{sec:prelim}\label{sec:concept}
We define the quantile utility model as follows. Let $O$ denote a finite set of options. Consider an agent equipped with a strict preference order $\succ$ among the options of $O$ and a parameter $h\in [0,1]$. Given a {\em lottery} (or probability distribution) $x$ over the options in $O$, we call option $o\in O$ is the $h$-quantile representative of the agent if $h=1$ and $o$ is the highest-ranked option according to $\succ$ with $x(o)>0$; or $h\in [0,1)$ and $\sum_{o\succ o'}{x(o')}\leq h < \sum_{o\succeq o'}{x(o')}$. The $h$-quantile representative of the agent in lottery $x$ is denoted by $\rep(x,\succ,h)$.

Given two lotteries $x$ and $y$ over the options in $O$, we say that the agent prefers $x$ to $y$ if and only if $\rep(x,\succ,h)\succ \rep(y,\succ,h)$. We say that the agent weakly prefers $x$ to $y$ if and only if $\rep(x,\succ,h)\succeq \rep(y,\succ,h)$

We now explore the connection of our lottery extensions to the notion of {\em stochastic dominance}.
\begin{definition}[stochastic dominance]
We say that lottery $x$ over the set of options $O$ is not stochastically dominated by lottery $y$ with respect to a preference $\succ$ over the options (and write $x\succsim^{\text{sd}} y$) if and only if 
\begin{align*}
    \sum_{o':o\succeq o'}{x(o')} &\leq \sum_{o':o\succeq o'}{y(o')}
\end{align*}
for every option $o\in O$. 
\end{definition}

It turns out that stochastic dominance is much more stringent than our new notion.

\begin{theorem}
    Let $O$ be a set of options, $x$ and $y$ be lotteries over the options in $O$, and $\succ$ be a preference order over $O$. Then, $x\succsim^{\text{sd}} y$ if and only if $\rep(x,\succ,h)\succeq_i \rep(y,\succ,h)$ for every $h\in [0,1]$.
\end{theorem}

\begin{proof}
    Assume that $x\succsim^{\text{sd}} y$ and, for the sake of contradiction, $a=\rep(y,\succ,h^*)\succ \rep(x,\succ,h^*)=b$ for some $h^*\in [0,1]$. This means that
    \begin{align*}
        \sum_{o':b\succeq o'}{x(o')} > h^*
    \end{align*}
    and 
    \begin{align*}
        h^* &\geq \sum_{o':a\succ o'}{y(o')}\geq \sum_{o':b\succeq o'}{y(o')}.
    \end{align*}
    In the above equation the first inequality follows from the definition of $h^*$-quantile representative. The second inequality holds true as $a \succ b$.
    Hence, $\sum_{o':b\succeq o'}{x(o')} > \sum_{o':b\succeq o'}{y(o')}$, contradicting the assumption $x\succsim^{\text{sd}} y$. Thus, we prove the forward direction.

    Now assume that $\rep(x,\succ,h)\succeq \rep(y,\succ,h)$ for every $h\in [0,1]$. For the sake of contradiction, assume that $x\not\succsim^{\text{sd}} y$ and let $o\in O$ be such that 
    \begin{align}\label{eq:not-1}
        \sum_{o':o\succeq o'}{x(o')} &>\sum_{o':o\succeq o'}{y(o')}
    \end{align}
    and 
    \begin{align}\label{eq:minimal}
        \sum_{o':o\succ o'}{x(o')} &\leq \sum_{o':o\succ o'}{y(o')}.
    \end{align}
    Notice that such an option $o$ clearly exists since $x\not\succsim^{\text{sd}} y$.

    Define 
    \begin{align}\label{eq:def-h-star}
        h^*=\sum_{o':o\succeq o'}{y(o')}.
    \end{align}
    By equations (\ref{eq:not-1}), (\ref{eq:minimal}), and (\ref{eq:def-h-star}), we get that $\rep(x,\succ,h^*)=o$. Now, let $a$ be the least preferred option in $O$ that satisfies $a\succ o$ and $y(a)>0$. Such an option exists since inequality (\ref{eq:not-1}) implies that $\sum_{o':o\succeq o'}{y(o')}<1$. We obtain that $\rep(y,\succ,h^*)=a\succ o$, a contradiction.
\end{proof}

\section{Voting}\label{sec:voting}
We assume that we have a set $A$ of $m$ alternatives and a set $N$ of $n$ agents (voters). Each agent $i\in N$ has a quantile parameter $h_i\in [0,1]$. We use $\h=(h_1, ..., h_n)$ to denote the vector of the quantile parameters. A voting profile $\profile=(\succ_1, ..., \succ_n)$ consists of the preference $\succ_i$ of each agent; $\succ_i$ is a strict ordering over the alternatives. We use the term {\em voting instance} and identify the above scenario with the tuple $(N,A,\profile,\h)$. Next we define an efficient lottery.

\begin{definition}[Efficiency]
    Given a voting instance $(N,A,\succ,\h)$, a lottery $x$ over the alternatives of $A$ is {\em efficient} if there is no other lottery $y$ such that $\rep(y,\succ_i,h_i)\succeq_i \rep(x,\succ_i,h_i)$ for every $i\in N$ and there is an agent $i^*\in N$ such that $\rep(y,\succ_{i^*},h_{i^*})\succ_{i^*} \rep(x,\succ_{i^*},h_{i^*})$.
\end{definition}
We answer the question about existence of a stable lottery in the following theorem.
\begin{theorem}
Consider a set $N$ of agents, a set $A$ of alternatives, and a preference profile $\profile$ with the preferences of the agents in $N$ over the alternatives in $A$. There is a lottery that is efficient for the voting instance $(N,A,\profile,\h)$ for every quantile parameter vector $\h$ if and only if all agents have the same alternative ranked first.
\end{theorem}

\begin{proof}
First, assume that the same alternative $a\in A$ is in the top position of the preference $\succ_i$ of every agent $i\in N$. Then, the lottery $x$ with $x(a)=1$ and $x(o)=0$ for $o\in A\setminus\{a\}$ makes alternative $a$ the representative of each agent $i$ for every value of her quantile parameter $h_i$. Clearly, the lottery $x$ is efficient.

Now, let $L$ be the set of top-ranked alternatives and assume that $|L|\geq 2$. 
First, consider the quantile vector $\h^1=(0,0,...,0)$. We claim that a lottery is efficient for $(N,A,\profile,\h^1)$ only if it returns one of the alternatives with probability $1$. Indeed, assume otherwise and let $S\subseteq A$ be the set of alternatives with positive probability under lottery $x$, i.e., $x(o)>0$ for $o\in S$ and $x(o)=0$ for $o\in A\setminus S$, with $|S|\geq 2$. By the definition of the representative, we have that the representative $\rep(x,\succ_i,0)$ of agent $i\in N$ is the alternative in $S$ that is ranked lowest among the alternatives in $S$. Then, lottery $x$ is dominated by the lottery $x'$ defined as $x'(a)=1$ for the alternative $a\in S$ that is ranked higher than any other alternative in $S$ by agent $1$ and $x'(o)=0$ for $o\in A\setminus\{a\}$. In this way, agent $1$ has a strictly better representative under $x'$ compared to $x$, while the representative under $x'$ of any other agent is at least as good as the one under $x$.

Next, consider the quantile vector $\h^2=(1-1/\ell, 1-1/\ell, ..., 1-1/\ell)$. There exists a unique lottery $y$ with $y(o)=1/\ell$ for $o\in L$ and $y(o)=0$ for $o\in A\setminus L$, which yields the top-ranked alternative of each agent as her representative. This lottery dominates every other lottery, including the set of (deterministic) lotteries that are efficient for the quantile vector $\h^1$.
\end{proof}

A (randomized) voting rule $R$\footnote{Randomized voting rules are also called {\em social decision schemes} or {\em probabilistic social choice functions} in the literature; e.g., see~\cite{B17}.} takes as input a profile and returns a lottery over the alternatives in $A$. Given a profile $\profile$ and an alternative $a\in A$, we denote by $R_a(\profile)$ the probability assigned to alternative $a$ when the voting rule $R$ is applied on the profile $\profile$.  A  randomized voting rule is efficient if it produces an efficient lottery. Next, we define strategyproof rules.

\begin{definition}[Strategyproofness]
Given a set $N$ of $n$ agents with quantile parameter vector $\h$ and a set $A$ of $m$ alternatives, the voting rule $R$ is strategyproof for the domain $(N,A,\cdot,\h)$ if for every preference profile $\profile$, agent $i$, and preference $\succ'_i$, it holds
$\rep(R(\profile), \succ_i,h_i)\succeq_i \rep(R(\profile_{-i},\succ'_i), \succ_i,h_i)$.
\end{definition}

 A natural question is if we can characterize strategyproof rules in quantile utility model. Towards this, we first define monotone rules.

\begin{definition}[Monotonicity]
    Consider the voting rule $R$ applied on profiles consisting of the preferences of a set $N$ agents and a set $A=\{a,b\}$ of two alternatives. The rule $R$ is monotonic if $R_a(\succ) \geq R_a(\succ_{-i},\succ_i)$ for every preference profile $\succ$, every agent $i\in N$ with $a\succ_i b$, and every preference $\succ'_i$ with $b\succ'_i a$. 
\end{definition}

We show the relation between monotone rules and stregyproofness.

\begin{theorem}
    A monotone voting rule for two alternatives is strategyproof for any values of quantile parameters. For any non-monotone rule $R$, there exists quantile parameter vector $\h$ so that $R$ is not strategyproof.  
\end{theorem}

\begin{proof}
For every agent $i\in N$ with $a\succ_i b$, it suffices to consider the case where $\rep(R(\succ),\succ_i,h_i)=b$ and show that $\rep(R(\succ_{-i},\succ'_i),\succ_i,h_i)=b$ as well, where $b\succ'_i a$. Assume otherwise that $\rep(R(\succ_{-i},\succ'_i),\succ_i,h_i)=a$. By the definition of $\rep(R(\succ),\succ_i,h_i)$, we have that $R_b(\succ)>h_i$, i.e., $R_a(\succ)<1-h_i$. By the definition of $\rep(R(\succ_{-i},\succ'_i),\succ_i,h_i)$, we have that $R_b(\succ_{-i},\succ'_i)\leq h_i$, i.e., $R_a(\succ_{-i},\succ'_i) \geq 1-h_i$. Thus, $R_a(\succ)<R_a(\succ_{-i},\succ'_i)$, contradicting the monotonicity of $R$.

Consider a non-monotone voting rule $R$, a preference profile $\succ$, an agent $i$ with $a\succ_i b$, another preference $\succ'_i$ with $b\succ'_i a$ such that $t_1=R_a(\succ)<R_a(\succ_{-i},\succ'_i)=t_2$. By setting $h_i=1-\frac{t_1+t_2}{2}$, we have $\rep(R(\succ),\succ_i,h_i)=b$ and $\rep(R(\succ_{-i},\succ'_i),\succ_i,h_i)=a$. Indeed, $R_b(\succ)=1-R_a(\succ)=1-t_1>1-\frac{t_1+t_2}{2}=h_i$ and $R_a(\succ_{-i},\succ'_i)=t_2>t_1\geq 0$ and $R_b(\succ_{-i},\succ'_i)=1-R_a(\succ_{-i},\succ'_i)=1-t_2\leq 1-\frac{t_1+t_2}{2}=h_i$.
\end{proof}



For the classical randomized voting rules, the celebrated result of \citet{G77chance} has shown that random dictatorships are the only strategyproof and efficient \footnote{Efficiency, here, refers to ex post efficiency. A rule is dictatorship if it depends only on the preference of one person.} rules. We ask the same question of existence of strategyproof and efficient rules in our model.

For two alternatives, we define the following randomized voting rule $R\text{-}plurality$ that will be used to prove the next theorem. Consider the voting instance $(N,\{a,b\},\profile,\h)$ with two alternatives $\{a,b\}$ and, without loss of generality, assume that the plurality winner in $\profile$ is alternative $a$. Define $R\text{-}plurality_a(\profile)=1-\min_{i:a\succ_i b}{h_i}$ (and $R\text{-}plurality_b(\profile)=1-R\text{-}plurality_a(\profile)$). 


% \begin{algorithm}[t]
%   \KwIn{A voting instance $(N,\{a,b\},\profile,\h)$}
%   \KwOut{An efficient lottery $x$}
%    Let alternative $a$ denote the plurality winner in $\profile$\;
%        Define $R_a(\profile)=1-\min_{i:a\succ_i b}{h_i}$\;
%         ~~~~~~~~~~~$R_b(\profile)=1-R_a(\profile)$ \;
%     \textbf{Return} the lottery produced by $R$
%   \caption{Efficient and Strategy-proof Voting with Two Alternatives} \label{alg:effSP2alt}
% \end{algorithm}

\begin{theorem}
The rule $R\text{-}plurality$ is an efficient and strategyproof rule for voting instances with two alternatives exists.
\end{theorem}
\begin{proof}

We first prove that the rule is efficient. Its definition is such that all agents $j\in N$ with $b\succ_j a$ with $h_j\geq R_a(\profile)$ as well as all agents $j\in N$ with $a\succ_j b$ have their top alternative as representative. Consider an agent $j^*\in N$ with $b\succ_{j^*} a$ and $h_{j^*}<\min_{i:a\succ_i b}{h_i}$. This means that $R_a(\profile)>h_{j^*}$, i.e., 
\begin{align}\label{eq:unhappy-agent}
    1-\min_{i:a\succ_i b}{h_i}&>h_{j^*}. 
\end{align}
Let $i^*\in N$ be an agent with preference $a\succ_{i^*} b$ and $h_{i^*}=\min_{i:a\succ_i b}{h_i}$. Under lottery $R(\profile)$, agents $j^*$ and $i^*$ have alternative $a$ as representative, which is their bottom- and top-ranked one, respectively. We will show that any better lottery for agent $j^*$ is worse for agent $i^*$, proving the efficiency of $R(\profile)$. Indeed, any lottery $x$ that makes alternative $b$ representative of agent $j^*$ must satisfy $x(a)\leq h_{j^*}$ which together with inequality (\ref{eq:unhappy-agent}) yields
\begin{align*}
    x(a)&\leq h_{j^*}<1-\min_{i:a\succ_i b}{h_i}=1-h_{i^*}. 
\end{align*}
Hence, $x(b)>h_{i^*}$, meaning that alternative $b$ is the representative of $i^*$ under $x$.

To prove strategyproofness, it suffices to consider a profile $\profile$ in which alternative $a$ is the plurality winner, and an agent $j\in N$ with preference $b\succ_i a$ who has the bottom-ranked alternative $a$ as representative under lottery $R(\profile)$, and show that the alternative $a$ is still her representative under the lottery $R(\profile_{-j},\succ'_j)$ for the deviation to preference $a\succ'_j b$. Since $\rep(R(\profile),\succ_j,h_j)=a$, it is $R_a(\profile)>h_j$. Clearly, alternative $a$ is still the plurality winner in profile $(\profile_{-j},\succ_j)$. By definition, 
\begin{align*}
    R_a(\profile_{-j},\succ'_j)&=1-\min\{h_j,\min_{i:a\succ_i b}{h_i}\} \geq h_j,
\end{align*}
i.e., $R_b(\profile_{-j},\succ'_j)\leq h_j$, implying that $\rep(R(\profile_{-j},\succ'_j),\succ_j,h_j)=a$, as desired.
\end{proof}

Although, we were able to design a simple rule that is efficient and strategyproof for two alternatives, unsurprisingly, a similar result cannot hold for three alternatives as we show in the following theorem. 

\begin{theorem}\label{thm:voting}
Consider a set $N$ of agents with the same quantile parameter $h\in [0,1]$ and a set of alternatives $A=\{a,b,c\}$. 
\begin{enumerate}
    \item[(a)] For $0\leq h< 1/3$, any efficient and strategyproof voting rule is a deterministic dictatorship. 
    \item[(b)] For $1/3\leq h <1/2$, any efficient and strategyproof voting rule among two agents is a deterministic dictatorship.  
    \item[(c)] For $1/2\leq h<2/3$, the voting rule that returns each of the two alternatives with the highest plurality score with probability $1/2$ is efficient and strategyproof. 
    \item[(d)] For $2/3\leq h\leq 1$, the voting rule that returns each alternative with probability $1/3$ is efficient and strategyproof.
\end{enumerate}
\end{theorem}

\begin{proof}
{\it (a).} Consider any lottery $x$ over the alternatives in $A$ assume that $x(a)=\max_{o\in A}{x(o)}$. Then, $x(a)\geq 1/3$ and $\sum_{o:a\succeq_i o}{x(o)}\geq 1/3>h$ for every $i\in N$. This implies that $\rep(x,\succ_i,h)\succeq_i a$, meaning that the lottery $x$ is (weakly) dominated by the deterministic lottery $y(a)=1$ (and $y(b)=y(c)=0$). Hence, any efficient voting rule is deterministic in this case and the statement follows by the Gibbard-Satterthwaite theorem~\citep{Gib73,Sat75}.

\paragraph{(b)} We first observe that if a lottery $x$ makes the top-ranked alternative $a$ of an agent $i\in N$ representative, then $a$ is the representative of every agent; this observation holds for profiles with any number of agents. By the definition of the representatives, the above condition implies that $x(b)+x(c)\leq h$ and $x(a)\geq 1-h>h$. Clearly, alternative $a$ is the representative of every agent with $a$ as their top preference. If $a$ is second-ranked at an agent $i'$, e.g., with preference $b\succ_{i'} a\succ_{i'} c$, we have $x(c)\leq h$ and $x(a)+x(c)\geq x(a)>h$, which implies that $a$ is the representative of agent $i'$. If alternative $a$ is bottom-ranked at an agent $i''$, the fact $x(a)>h$ implies that $a$ is again the representative of agent $i''$. Furthermore, there exist lotteries that make the second-ranked alternative in every agent her representative, e.g., the lottery that assigns probability $1/3$ to all alternatives is such a lottery.

Hence, we may assume that any efficient and strategyproof voting rule returns either some alternative with probability $1$ or returns each alternative equiprobably; in the following, we use $q$ to refer to this lottery. Notice that in profiles with a dominated alternative, any efficient voting rule should return (deterministically) some of the other alternatives. The only case where no dominated alternative exists is for profiles like profile $\profile^{11}$ defined as follows: $\profile^{11}=(a\succ_{11} c\succ_{11} b, b\succ_{21} c \succ_{21} a)$. 

Observe that in profile $\profile^{12}=(a\succ_{11} c \succ_{11} b, b\succ_{22} a\succ_{22} c$, alternative $c$ is dominated by alternative $a$ and the lottery that returns alternative $a$ dominates lottery $q$. Due to its efficiency, the voting rule should return either alternative $a$ or alternative $b$ in profile $\profile^{12}$; without loss of generality, assume that it returns $a$.

Then, alternative $a$ should also be returned in profile $\profile^{22}=(a\succ_{12} b \succ_{12} c, b \succ_{22} a \succ_{22} c)$. Indeed, notice that alternatives $a$ and $b$ are undominated and returning any of them deterministically dominates lottery $q$ as well. Hence, efficiency implies that one of these alternatives should be returned by the voting rule. Due its strategyproofness, if the voting rule returned alternative $b$ for profile $\profile^{22}$, the deviation of agent $1$ to preference $\succ_{11}$ would yield profile $\profile^{12}$ with alternative $a$ as outcome of the voting rule, which is more preferable for agent $1$ than alternative $b$ at profile $\profile^{22}$. Hence, the voting rule returns alternative $a$ in profile $\profile^{22}$ as well. 

Then, alternative $a$ should also be returned in profile $\profile^{21}=(a\succ_{12} b \succ_{12} c, b \succ_{21} c \succ_{21} a)$. Indeed, alternative $b$ dominates alternative $c$ and the lottery that returns it dominates lottery $q$. Due to its efficiency, the voting rule should return (deterministically) one of the alternatives $a$ or $b$. By its strategyproofness, if the voting rule returned alternative $b$ for profile $\profile^{21}$, the deviation of agent $2$ from preference $\succ_{22}$ in profile $\profile^{22}$ to preference $\succ_{21}$ would yield profile $\profile^{21}$ with alternative $b$ as outcome of the voting rule, which is more preferable for agent $2$ than alternative $a$ at profile $\profile^{22}$. Hence, the voting rule returns alternative $a$ in profile $\profile^{21}$ as well. 

Now, consider the lottery returned by the voting rule in profile $\profile^{11}$. If it does not make alternative $a$ the representative of agent $1$, then the deviation to preference $\succ_{12}$ would yield profile $\profile^{21}$ in which alternative $a$ is returned deterministically. Hence, by our observation in the first paragraph above, alternative $a$ should be returned deterministically by the voting rule on profile $\profile^{11}$. Overall, the voting rule is deterministic and the statement in (b) follows again by the Gibbard-Satterthwaite theorem~\citep{Gib73,Sat75}.

\paragraph{(c)} Consider a profile $\profile$ and let $a$ and $b$ be the two alternatives with the highest plurality score. If alternative $c$ is not top-ranked by any agent, the lottery returned when applying the rule on $\profile$ is efficient (since every agent has her top-ranked alternative as representative). Otherwise, the lottery would be dominated by a lottery $x$ which makes the top-ranked alternative in every agent her representative. Let $o\in A$ be the top-ranked alternative of agent $i\in N$ satisfying $\rep(x,\succ_i,h)=o$. Then, $x(o)=1-\sum_{o':o\succ_i o'}{x(o')}\geq 1-h>1/3$. Hence, $x(o)>1/3$ for every $o\in A$, a contradiction. Thus, the lottery returned is never dominated, and the rule is efficient.

To prove strategyproofness, it suffices to consider a profile $\profile$ in which alternatives $a$ and $b$ have the highest plurality score and an agent $i\in N$ with $c$ as top-ranked alternative and (without loss of generality) alternative $a$ ranked next. So, agent $i$ has alternative $a$ as representative under the voting rule. Now, observe that any deviating preference $\succ'_i$ by an agent $i\in N$ who has the plurality loser as her top-ranked alternative in $\succ_i$ cannot increase the plurality score of $c$ or decrease the plurality score of $a$ and $b$ in $(\profile_{-i},\succ_i)$ compared to $\profile$. Thus, $a$ and $b$ are the two alternatives with the highest plurality score and the lottery returned by the rule is the same.
    
\paragraph{(d)} Finally, the top-ranked alternative is the representative under the lottery $x(a)=x(b)=x(c)=1/3$. Hence, this rule is trivially efficient and strategyproof.
\end{proof}
We conclude this section with the following challenging open problem.
\begin{question}
Extend the characterization of efficient and strategyproof randomized voting rules for all possible vectors of quantile parameters and at least three alternatives.
\end{question}

\section{One-sided matching}\label{sec:matching}
In this section we assume that we have a set $N$ of $n$ agents and a set $M$ of $n$ items. Each agent $i \in N$ has a quantile parameter $h_i\in [0,1]$ and a preference order $\succ_i$ over the items in $M$. A matching mechanism takes as input a preference profile $\succ$ and the vector $\h$ of quantile parameters, and returns a lottery over perfect matchings between the sets $N$ and $M$. Thus, a lottery $x$ assigns  probability $x_{ig}$ for agent $i \in N$ and item $g \in M$ such that $\sum_{i\in N} x_{ig} = 1$ for each $g \in M$ and $\sum_{g\in M} x_{ig}=1$ for each $i \in N$.

We begin our study of lotteries over matchings by focusing on the maximization of (a variant of) social welfare.

\begin{theorem}
    Given a matching instance $(N,M,\profile,\h)$, a lottery that maximizes the number of agents having their top-ranked item as representative can be computed in polynomial time.
\end{theorem}

\begin{proof}
For each item $g\in M$, denote by $S_g$ the set of agents having item $g$ as their top choice. Clearly, for any lottery, only agents from set $S_g$ can have item $g$ as representative. Furthermore, an agent from $S_g$ can have item $g$ as representative only if the lottery assigns it to item $g$ with probability at least $1-h_i$. Denote by $\overline{S}_g$ a subset of $S_g$ of maximum cardinality that satisfies $\sum_{i\in \overline{S}_g}{(1-h_i)}\leq 1$; this condition is necessary so that there is a lottery that makes item $g$ representative for all agents in $\overline{S}_g$. Then, the maximum number of agents having their top item as representative is at most $\sum_{g\in M}{|\overline{S}_g|}$.

We now present a lottery under which a maximum number of $\sum_{g\in M}{|\overline{S}_g|}$ agents have their top item as representative. Notice that for every two different items $g_1,g_2\in M$, the sets $\overline{S}_{g_1}$ and $\overline{S}_{g_2}$ are disjoint. Thus, by setting $x_{ig}=1-h_i$ for every $g\in M$ and $i\in \overline{S}_g$, we get that all agents in $\cup_{g\in M}{\overline{S}_g}$ have their top item as representative. By the definition of set $\overline{S}_g$, we have the condition $\sum_{i\in \overline{S}_g}{x_{ig}}\leq 1$ for every item $g$. Hence, we can trivially complete $x$ and get a valid lottery by setting the values $x_{ig}$ for every other agent-item pair $(i,g)$ with $i\not\in \overline{S}_g$ so that $\sum_{i\in N}{x_{ig}}=1$ for every item $g\in M$ and $\sum_{g\in M}{x_{ig}}=1$ for every agent $i\in N$.

Notice that, for a given item $g\in M$, the set $\overline{S}_g$ can be easily computed by starting with the empty set, considering the agents in set $S_g$ in monotone non-increasing order of their quantile parameter, and including an agent in $\overline{S}_g$ as long as the sum of the quantities $1-h_i$ in $\overline{S}_g$ does not exceed $1$.
\end{proof}

One can consider other social welfare objectives for lotteries over matchings. We believe that the following two problems are worth studying. 
\begin{question}\label{ques:one-sided-rank-maximization}
    What is the complexity of the following two problems? 
    \begin{itemize}
        \item Given a matching instance $(N,M,\profile,\h)$ and an integer rank requirement $r_i$ for every agent $i\in N$, compute a lottery $\x$ that maximizes $|\left\{i\in N: \rank_i(\rep(x_i,\succ_i,h_i))\in [r_i]\right\}|$, i.e., the number of agents with a representative meeting their rank requirement.
        \item \begin{sloppypar}Given a matching instance $(N,M,\profile,\h)$, compute a lottery $\x$ that minimizes $\sum_{i\in N}{\rank_i(\rep(x_i,\succ_i,h_i))}$, i.e., the sum of ranks of the representatives over all agents.
        \end{sloppypar}
    \end{itemize}
\end{question}

We now adapt the definitions of efficiency and strategyproofness for lotteries over matchings and matching mechanisms. 
\begin{definition}[Efficiency for lotteries over matchings]
    Given a one-sided instance $(N,M,\profile,\h)$, a lottery $\x$ over matchings is {\em efficient} if there is no other lottery $\y$ such that $\rep(y_i,\succ_i,h_i)\succeq_i \rep(x_i,\succ_i,h_i)$ for every $i\in N$ and there is an agent $i^*\in N$ such that $\rep(y_{i^*},\succ_{i^*},h_{i^*})\succ_{i^*} \rep(x_{i^*},\succ_{i^*},h_{i^*})$.
\end{definition}

\begin{definition}[Strategyproofness of matching mechanisms]
Given a set $N$ of $n$ agents with quantile parameter vector $\h$ and a set $M$ of $n$ items, the matching mechanism $R$ is strategyproof for the domain $(N,M,\cdot,\h)$ if for every preference profile $\profile$, agent $i\in N$, and preference $\succ'_i$, it holds
$\rep(R(\profile), \succ_i,h_i)\succeq_i \rep(R(\profile_{-i},\succ'_i), \succ_i,h_i)$.
\end{definition}

We will prove that efficiency and strategyproofness are compatible for agents with quantile utilities. The main component of our proof is a simple set of linear inequalities that check whether lotteries over matchings that satisfy given requirements for the ranks of the representatives of the agents exist. For a given one-sided matching instance $(N,M,\profile,\h)$ and a vector of rank requirements $\rr=(r_1, r_2, ..., r_n)$ with integer $r_i\in [n]$, the following linear program is feasible if and only if there is a lottery over matchings that satisfy the rank requirement $r_i$ for the representative of each agent $i\in N$.
\begin{align*}
    \sum_{g:\rank_i(g)\leq r_i}{x_{it}} \geq 1-h_i,& \quad \forall i\in N\\
    \sum_{g\in M}{x_{ig}}=1, &\quad \forall i\in N\\
    \sum_{i\in N}{x_{ig}}=1, &\quad \forall g\in M
\end{align*}
Dropping $N$, $M$, and $\h$ from notation, we will refer to this linear program as $\LP(\profile;\rr)$.

The compatibility of efficiency and strategyproofness can be obtained via the following variant of the serial dictatorship mechanism.

\paragraph{Serial dictatorship (SD) mechanism:} On input a one-sided matching instance $(N,M,\profile,\h)$ with $n$ agents/items, mechanism SD starts with a vector $\rr=(n, ..., n)$ of minimum rank requirements for all agents and considers the agents one by one in increasing order of their ids. When the agent $i\in N$ is considered, the mechanism computes the minimum rank $t$ for agent $i$ so that the linear program $\LP(\profile;\rr_{-i},t)$ is feasible and updates $\rr$ by setting $r_i=t$. An arbitrary lottery over matchings that satisfy $\LP(\profile;\rr)$ is returned as the output of the mechanism.

\begin{theorem}\label{thm:one-sided:eff+sp}
Mechanism SD is efficient and strategyproof.
\end{theorem}

\begin{proof}
The crucial property of mechanism SD is that the final linear program $\LP(\profile;\rr)$ has as solutions all those lotteries which make representative for agent $1$ her top-ranked item and representative for agent $i\geq 2$ her highest-ranked item under rank-constraints for the representative items for agents $1, ..., i-1$. 

To prove efficiency, let $\x$ be the lottery returned by the mechanism as the solution of the linear program $\LP(\profile;\rr)$. For the sake of contradiction, assume that there is another lottery $\y$ such that $\rank_i(y_i,\succ_i,h_i)\leq \rank_i(x_i,\succ_i,h_i)$ for every agent $i\in N$ and $\rank_{i^*}(y_{i^*},\succ_{i^*},h_{i^*})<\rank_{i^*}(x_{i^*},\succ_{i^*},h_{i^*})$ for some agent $i^*\in N$. Let $i'\in N$ be the agent of minimum id with $r'_{i'}=\rank_{i'}(y_{i'},\succ_{i'},h_{i'})< r_i$, i.e., $\rank_{i}(y_{i},\succ_{i},h_{i})= r_i$ for $i=1, ..., i'-1$. This means that the linear program $\LP(\profile;(r_1, ..., r_{i'-1}, r'_{i'},n, ..., n))$ has lottery $\y$ as solution, contradicting the definition of $r_{i'}$ at round $i'$ of the mechanism.

Strategyproofness follows since when it considers agent $i\in N$, the mechanism restricts the candidate output lotteries to those that make the best possible item a representative for agent $i\in N$, given decisions about the representatives of agents $1, ..., i-1$ in previous rounds. Hence, unilateral misreporting by agent $i$ cannot result to a final lottery that improves her representative item further.
\end{proof}

In the following, we define two fairness properties for lotteries over matchings and study their interplay with efficiency (and strategyproofness). We begin with the definition of {\em proportionality}.

\begin{definition}[proportionality]\label{defn:prop}
    A lottery $x$ for the one-sided matching instance $(N,M,\profile,\h)$ is proportional if for every agent $i\in N$, $\rank_i(\rep(x_i,\succ_i,h_i))=1$ if $h_i=1$ and $\rank_i(\rep(x_j,\succ_i,h_i))\leq \lceil n(1-h_i)\rceil$ if $h_i\in [0,1)$.
\end{definition}

Note that our definition implies that the uniform lottery is proportional. The first question that arises is whether proportionality is compatible with efficiency. We prove that this is indeed the case, using a straightforward restriction of the serial dictatorship mechanism. In its definition below, we use $p_i$ as a shorthand of the rank requirement for agent $i\in N$ from Definition~\ref{defn:prop}, i.e., $p_i=\max\left\{1,\lceil n(1-h_i)\rceil\right\}$.

\paragraph{Proportionality-constraint serial dictatorship (PSD):} On input a one-sided matching instance $(N,M,\profile,\h)$ with $n$ agents/items, mechanism PSD starts with a vector $\rr=(p_1, ..., p_n)$ of the rank requirements for proportionality for all agents and considers the agents one by one in increasing order of their ids. When the agent $i\in N$ is considered, the mechanism computes the minimum rank for agent $i$ so that the linear program $\LP(\profile;\rr_{-i},t)$ is feasible and updates $\rr$ by setting $r_i=t$. An arbitrary lottery over matchings that satisfy $\LP(\profile;\rr)$ is returned as the output of the mechanism.

\begin{theorem}\label{thm:one-sided-eff+prop}
    Mechanism PSD returns efficient and proportional lotteries.
\end{theorem}

\begin{proof}
The crucial property of mechanism PSD is that the final linear program $\LP(\profile;\rr)$ has as solutions all those lotteries which make representative for agent $1$ her top-ranked item and representative for agent $i\geq 2$ her highest-ranked item under rank-constraints for the representative items for agents $1, ..., i-1$ and proportionality rank-constraints for agents $i+1,...,n$. 

The proportionality of the final lottery is guaranteed by the definition of the mechanism. Initially, before the execution of the first round, the linear program $\LP(\profile, p_1, ..., p_n)$ is feasible (e.g., it is satisfied by the uniform lottery). When considering agent $i\in N$, the linear program $\LP(\profile;(r_1,r_2, ..., r_{i-1},p_i, ...., p_n))$ is guaranteed to be feasible after the execution of phase $i-1$. Hence, $r_i\geq p_i$ and $\LP(\profile;\rr)$ has proportional lotteries as feasible solutions. 

Efficiency follows by slightly adapting the argument we used in the proof of Theorem~\ref{thm:one-sided:eff+sp}; we include it here for completeness. To prove efficiency, let $\x$ be the lottery returned by the mechanism as the solution of the linear program $\LP(\profile;\rr)$. For the sake of contradiction, assume that there is another lottery $\y$ such that $\rank_i(y_i,\succ_i,h_i)\leq \rank_i(x_i,\succ_i,h_i)$ for every agent $i\in N$ and $\rank_{i^*}(y_{i^*},\succ_{i^*},h_{i^*})<\rank_{i^*}(x_{i^*},\succ_{i^*},h_{i^*})$ for some agent $i^*\in N$. Let $i'\in N$ be the agent of minimum id with $r'_{i'}=\rank_{i'}(y_{i'},\succ_{i'},h_{i'})< r_i$, i.e., $\rank_{i}(y_{i},\succ_{i},h_{i})= r_i$ for $i=1, ..., i'-1$. This means that the linear program $\LP(\profile;(r_1, ..., r_{i'-1}, r'_{i'},p_{i'+1}, ..., p_n))$ has lottery $\y$ as solution, contradicting the definition of $r_{i'}$ at round $i'$ of the mechanism.
\end{proof}

Due to its similarity with mechanics SD, it is tempting to assume that the PSD mechanism is also strategy-proof. Surprisingly, this is not the case; the restriction of serial dictatorships to proportional lotteries violates strategy-proofness.

\begin{theorem}\label{thm:psd-not-sp}
Mechanism PSD is not strategy-proof.
\end{theorem}

\begin{proof}
Consider a one-sided matching instance with three agents $1$, $2$, and $3$ and three items $a$, $b$, and $c$. Agent $1$ has $h_1=0$ and preference $a\succ_1 b\succ_1 c$. Agents $2$ and $3$ have $h_2=h_3=1/3$ and identical preference $\succ_2=\succ_3$ with $b\succ_2 c \succ_2 a$. 

Proportionality restricts the outcome of mechanism PSD to lotteries in which some of the items $b$ and $c$ are representative of agents $2$ and $3$, while any item can be the representative of agent $1$. Under this restriction, the lottery selected by the mechanism yields the representatives $a$, $b$, and $c$ for the three agents, respectively. E.g., such a lottery has $x_{1a}=1$, $x_{2b}=x_{3c}=2/3$, $x_{2c}=x_{3b}=1/3$, and $x_{1b}=x_{1c}=x_{2a}=x_{3a}=0$. Notice that making item $b$ representative in both agents $2$ and $3$ is infeasible since both $x_{2b}$ and $x_{3b}$ should be at least $2/3$ violating the matching condition.

Now, assume that agent $3$ misreports the preference $a\succ'_3 b\succ'_3 c$. Proportionality restricts the outcome of mechanism PSD to lotteries in which the representative of agent $2$ is some of the items $b$ and $c$ and the representative of agent $3$ is some of the items $a$ and $b$. Among them, the set of lotteries that make item $a$ representative of agent $1$ is non-empty, and the matching and proportionality condition yield that items $c$ and $b$ are the representatives of agents $2$ and $3$, respectively. Indeed, such a lottery $x'$ should satisfy $x'_{1a}=1$ and, by the matching condition, $x'_{3a}=0$. Hence, item $a$ cannot be the representative of agent $3$. To make item $b$ her representative as proportionality requires, it must be $x'_{3b}\geq 2/3$. But then, it must also be $x'_{2b}\leq 1/3$, which implies that item $b$ cannot be the representative of agent $2$. The lottery $x'$ with $x'_{1a}=1$, $x'_{2b}=x'_{3c}=1/3$, $x'_{2c}=x'_{3b}=2/3$, and $x'_{1b}=x'_{1c}=x'_{2a}=x'_{3a}=0$ makes items $a$, $c$, and $b$ representatives in agents $1$, $2$, and $3$, respectively.

Thus, when agent $3$ misreports $\succ'_3$ instead of her true preference $\succ_2$, mechanism PSD returns a lottery with a strictly better representative for her, violating strategy-proofness.  
\end{proof}

Now, Theorems~\ref{thm:one-sided-eff+prop} and~\ref{thm:psd-not-sp} beg the following question.

\begin{question}
Are there efficient, strategy-proof, and proportional mechanisms for one-sided matching instances?
\end{question}

We now define {\em envy-freeness}, our second fairness property.
\begin{definition}[envy-freeness]
    A lottery $x$ for the one-sided matching instance $(N,M,\profile,\h)$ is envy-free if $\rep(x_i,\succ_i,h_i)\succeq_i \rep(x_i,\succ_i,h_i)$ for every pair of agents $i,j\in N$.
\end{definition}

The definition of envy-freeness is a very natural one. One may wonder why the particular definition of proportionality is well justified. The next statement indicates that, similar to the relations of these concepts in the literature on fair division, our definitions are such that envy-freeness implies proportionality.

\begin{theorem}
    Envy-freeness implies proportionality.
\end{theorem}

\begin{proof}
Consider a matching instance and let $\x$ be an envy-free lottery over matchings. Abbreviate the rank requirement for agent $i\in N$ for proportionality by $p_i=\max\left\{1,\left\lceil n(1-h_i)\right\rceil\right\}$ and assume that $\x$ is not proportional. Then, there exists an agent $i^*\in N$ such that
    \begin{align*}
        \rank_{i^*}(\rep(x_{i^*},\succ_{i^*},h_{i^*})) &\geq p_{i^*}+1.
    \end{align*}
    Since $\x$ is envy-free, we have
    \begin{align}\nonumber
        \rank_{i^*}(\rep(x_i,\succ_{i^*},h_{i^*})) &\geq \rank_{i^*}(\rep(x_{i^*},\succ_{i^*},h_{i^*}))\\\label{eq:ef}
        &\geq p_{i^*}+1,
    \end{align}
    for every agent $i\in N$. Denote by $g^*$ the item ranked $p_{i^*}$-th by agent ${i^*}$. By the definition of $\rep(x_i,\succ_{i^*},h_{i^*})$ and inequality (\ref{eq:ef}), we have
    \begin{align*}
        \sum_{g\in M:g^*\succ_{i^*} g}{x_{ig}} &> h_{i^*}
    \end{align*}
    for every agent $i\in N$. Using this inequality, we have
    \begin{align*}
        \sum_{g\in M:g^*\succ_{i^*} g}{\sum_{i\in N}{x_{ig}}} & =\sum_{i\in N}{\sum_{g\in M:g^*\succ_{i^*} g}{x_{ig}}} > n\cdot h_{i^*}.
    \end{align*}
    Now, notice that there are $n-p_{i^*} =n-\max\left\{\lceil n(1-h_{i^*})\rceil\right\} \leq n\cdot h_{i^*}$ items ranked below the item $g^*$ by agent $i^*$ and, hence, at most $n\cdot h_{i^*}$ terms in the outer sum of the LHS of the last inequality. Thus, there exists one such term corresponding to an item $g'\in M$ such that $\sum_{i\in N}{x_{ig'}}>1$, violating the validity of lottery $\x$.
\end{proof}

Unfortunately, envy-freeness and efficiency may not be possible simultaneously, as the next simple example illustrates.
\begin{example}
Consider an instance with two agents $1$ and $2$ with identical preference $a\succ b$ for two items $a$ and $b$ and identical quantile parameter $h<1/2$. First, observe that the same item cannot become the representative of both agents. If item $a$ is the representative for both agents, this means that $x_1(a)\geq 1-h>1/2$ and $x_2(a)\geq 1-h>1/2$, violating the matching constraint $x_1(a)+x_2(a)\leq 1$. The lottery that makes item $b$ representative for both agents (e.g., the uniform lottery) is dominated by the lotteries that make item $a$ representative for agent $1$ and item $b$ representative for agent $2$ (e.g., $x_1(a)=x_2(b)=1$ and $x_1(b)=x_2(a)=0$) and vice-versa. These two classes of efficient lotteries are clearly not envy-free. \qed
\end{example}

The following question is, hence, interesting to study.
\begin{question}
    What is the complexity of deciding whether an efficient and envy-free lottery exists for a given one-sided matching instance?
\end{question}

\section{Two-sided and stable matchings}\label{sec:stable-matching}
In this section we assume that we have two sets of agents $N$ and $M$, each has $n$ agents. Each agent $i \in N$ (or, $i\in M$) has a quantile parameter $h_i\in [0,1]$ and a preference order $\succ_i$ over the agents in $M$ (resp, over the agents $N$). Similar to the one-sides scenario, a matching mechanism takes as input a preference profile $\profile$ and the vector $\h$ of quantile parameters, and returns a lottery over perfect matchings between the sets $N$ and $M$.
We begin by studying a variation of social welfare.

\paragraph{Top-choice maximization } Given an instance $(N,M,\profile,\h)$ with $h_i=h$ for every $i\in N\cup M$,  we design the following algorithm based on maximum weight b-matching (see \Cref{alg:2sidedtop-choice}).
We construct a complete weighted bipartite graph $G$ between the agent sets $N$ and $M$. Each edge $(i,j)$ such that $i$ is the top choice of agent $j$ and $j$ is the top choice of agent $i$ has weight $2$. Each edge $(i,j)$ such that $i\in N$ is the top choice of agent $j\in M$ but agent $j$ is not the top choice of agent $i$ or $i\in N$ is not the top choice of agent $j\in M$ but agent $j$ is the top choice of agent $i$ has weight $1$. All other edges have a weight $0$. 
Set $b=\lfloor \frac{1}{1-h}\rfloor$. Then, we find a maximum weight $b$-matching in graph $G$. The edges of the $b$-matching will each have probability $1-h$ and the probabilities of the remaining edges are completed appropriately to form a lottery. 

\begin{algorithm}[t]
  \KwIn{A two-sided matching instance $(N,M,\profile,\h)$ with $h_i=h$ for every $i\in N\cup M$}
  \KwOut{A lottery $x$ that maximize the number of agents that has top-choice as representative}
   Construct a complete weighted bipartite graph $G = (N \cup M, E, wt)$ where $E = \{\{i,j\}| i \in N, j \in M\}$ and the weight $wt: E \rightarrow \{2,1,0\}$ is defined as:\\ 
   \For{$\{i,j\} \in E$}{
    $wt(\{i,j\})=2 $ if agent $i$ and $j$ are each other's top-choice\;
     $wt(\{i,j\})=1$, else if  agent $i$ is the top-choice of agent $j$ or agent $j$ is the top-choice of $i$\;
     $wt(\{i,j\})=0$, otherwise\;
     }
     Set $b=\lfloor \frac{1}{1-h}\rfloor$\;
     Let $\mu$ = a maximum weight $b$-matching in $G$\;
     \For{edge $\{i,j\} \in \mu$}{
      \tcc{\small Construct lottery $x$}
      Set $x_{ij} = 1-h$\;
     }
      Assign probabilities to the remaining edges appropriately such that $x$ is a lottery\;
    \textbf{Return} $x$\;
  \caption{Maximize the number of agents that has top-choice as representative} \label{alg:2sidedtop-choice}
\end{algorithm}

\begin{theorem}
    Given a two-sided matching instance $(N,M,\profile,\h)$ with $h_i=h$ for every $i\in N\cup M$, the lottery computed in \Cref{alg:2sidedtop-choice} maximizes the number of agents with their top-choice as representative can be solved in polynomial time.
\end{theorem}

\begin{proof}
    Let $x$ be the lottery returned by the algorithm. Suppose that $x$ does not maximize the number of agents with top-choice as representative. Let $y$ be another lottery that maximizes the number of agents with top-choice as representative. Construct a matching $\mu'$ from $y$ as follows.
    % For each pair of agents $i$ and $j$, $i \in N, j \in M$, such that they are each other's top-choice and representatives in $y$, add the edge $i,j$ to the matching $\mu'$.  Next 
    For each agent $i$, such that its top-choice agent $j$ is its representative in $y$, add the edge $\{i,j\}$ to $\mu'$. Towards contradiction, we will show that $\mu'$ is a $b$-matching with larger weight than that of $\mu$. Note that  $b=\lfloor \frac{1}{1-h}\rfloor$ and each edge that is added to $\mu'$ corresponds to a representative. Then, from the definition of representatives, we have that $\mu'$ is a $b$-matching. Now we show that weight of $\mu'$ is grater than that of $\mu$.
    From the construction of the graph $G$,  we have that if both agents in an edge are each other's top-choice, then the edge contributes weight $2$ to the total weight and if only one agent in an edge has top-choice as its representative, then it adds weight $1$ to the weight of $\mu'$. Thus, for each agent that has its top-choice as representative contributes one to the weight of the $b$-matching. Therefore, since more agents has top-choice as representative in $y$, weight of $\mu'$ is more than $\mu$, a contradiction. Finally, the algorithm runs in polynomial time as we can construct the graph $G$ and find a weighted $b$-matching \citep{kuhn1955hungarian} in polynomial time.
\end{proof}
Before we can ask a question similar to \Cref{ques:one-sided-rank-maximization}, we ask the following, more basic, question that we could answer in polynomial time in the one-sided setting.
\begin{question}
    What about general values of $\h$? Is the problem still polynomial-time solvable or NP-hard?
\end{question}

% Note that answering the above question is not as simple as solving a $b$-matching with different values of $b$ for different vertices, as the probability assigned to one edge in the matching decides the representative for both agents in the edge.
In the following, we focus on stability.

\begin{definition}[Stability]\label{def:blocking-pair}
    Given an instance $(N,M,\profile,\h)$ of two-sided matching, a lottery $x$ over matchings is stable if for every pair of agents $i\in N$ and $j\in M$, it holds that $\rep(x_i,\succ_i,h_i)\succeq_i j$ or $\rep(x_j,\succ_j,h_j)\succeq_j i$. If this is not the case, i.e., it is $j \succ_i \rep(x_i,\succ_i,h_i)$ and $i \succ_j \rep(x_j,\succ_j,h_j)$, we say that the pair $(i,j)$ is a blocking pair for lottery $x$.
\end{definition}
We explore the relations between the stability definitions.
\begin{lemma}\label{lem:integral_is_stable_lottery}
    In every instance $(N,M,\profile,\h)$, any stable integral matching is also a stable lottery. 
\end{lemma}

\begin{proof}
    Let $\mu$ be an integral stable matching for the instance $(N,M, \succ)$.
We define a lottery $x$ from the matching $\mu$ by assigning $x(i,j) =1$ if $\mu(i)=j$, and assigning zero otherwise, where $i \in N$ and $j \in M$. 
We show that $x$ is a stable lottery for the instance $(N,M, \succ, \h)$.
Suppose not and $(i,j)$ is a blocking pair for $x$. Then, since the preferences are strict ordering, by negation of the condition in \Cref{def:blocking-pair}  we get that there exists a pair $i\in N, j \in M$ such that $j \succeq_i\rep(x_i,\succ_i,h_i) $ and $i \succeq_{j} \rep(x_{j},\succ_{j},h_{j})$. Observe that $\rep(x_i,\succ_i,h_i) = \mu(i)$ and $\rep(x_{j},\succ_{j},h_{j}) = \mu(j)$. Therefore, we get that $j \succeq_i\mu(i) $ and $i \succeq_{j} \mu(j)$. Then $(i,j)$ is a blocking pair for $\mu$, a contradiction.
\end{proof}
% \todo{Discuss relations to ordinal stability ($h=0$) and integral stability ($h=1$).}

However, a stable lottery captures much more than stable integral matchings.
\begin{lemma}
 There is a two-sided matching instance $(N,M,\profile,\h)$ with $h_i=1/2$ for every $i\in N\cup M$
 such that a stable lottery has an unstable integral matchings in its support.
\end{lemma}
\begin{proof}
 Consider an instance of two sided matching where $N=\{m_1,m_2\}$, $M=\{w_1,w_2\}$,  $h_i =1/2$ for each agent $i \in N \cup M$, and the preference profile is as follows:
 \begin{center}
 \begin{tabular}{ccc  ccc}
   $m_1$:& $w_2$ & $\succ_{m_1}$ $w_1$   & \quad \quad \quad $w_1$: & $m_1$&$\succ_{w_1}$ $m_2$  \\
    $m_2$:& $w_2$ &$\succ_{m_2}$ $w_1$   & \quad \quad \quad $w_2$: & $m_2$&$\succ_{w_2}$ $m_1$ 
 \end{tabular}
 \end{center}

The lottery $x$ is defined as $x_{m_1w_2}=1/3$, $x_{m_1w_1}=2/3$,
$x_{m_2w_2}=2/3$, and
$x_{m_2w_1}=1/3$. Thus, by symmetry we get, 
$x_{w_1m_1}=2/3$, $x_{w_1m_2}=1/3$, $x_{w_2m_2}=2/3$, and $x_{w_2m_1}=1/3$.
% The lottery $x$ is defined as $x_{m_1w_2}=1-h_{m_1} - \epsilon$, $x_{m_1w_1}=h_{m_1} + \epsilon$,
% $x_{m_2w_2}=h_{m_2} + \epsilon$, and
% $x_{m_2w_1}=1-h_{m_2} - \epsilon$. 
 Observe that $x$ is stable. 
 However, in a decomposition of the lottery as a convex combination of integral matchings, there must be an integral matching $\mu$ that matches $m_2$ to $w_1$ and $m_1$ to $w_2$. However, $\mu$ is not an integral stable since $(m_2,w_2)$ is blocking.
\end{proof}


 We observe the relation between ordinally stable \citep{AF03Scarfs} (also known as fractionally stable; see \citealt{aziz_random_2019}) matchings and stable lotteries that follows directly from the definitions.

\begin{observation}\label{obs:ordinal-stableh=0}
    Given a two-sided matching instance $(N,M,\profile,\h)$ with $h_i=0$ for every $i\in N\cup M$, a lottery is stable if and only if it is ordinally stable. 
\end{observation}

We restate the definition of efficiency which is the same as defined in the one-sided setting.

\begin{definition}[Efficiency for lotteries over two-sided matchings]
    Given a two-sided instance $(N,M,\profile,\h)$, a lottery $\x$ over matchings is {\em efficient} if there is no other lottery $\y$ such that $\rep(y_i,\succ_i,h_i)\succeq_i \rep(x_i,\succ_i,h_i)$ for every $i\in N \cup M$ and there is an agent $i^*\in N \cup M$ such that $\rep(y_{i^*},\succ_{i^*},h_{i^*})\succ_{i^*} \rep(x_{i^*},\succ_{i^*},h_{i^*})$.
\end{definition}

Given two lotteries $x$ and $y$, if  representative of every agent in $y$ is at least as good as it is in $x$ and there is an agent who's representative strictly improves in $y$ then $y$ is said to Pareto dominate $x$. Additionally, $y$ is a \emph{Pareto improvement} over the lottery $x$.

We show that stability is compatible with efficiency.

\begin{theorem}
    Given a two-sided matching instance $(N,M,\profile,\h)$, an efficient, stable lottery can be computed in polynomial time.
\end{theorem}

\begin{proof}
Observe that an integral stable matching is a stable lottery (\Cref{lem:integral_is_stable_lottery}). Thus, it suffices to show that a stable lottery remains stable after a Pareto improvement.

 Let $y$ denotes a lottery that is a Pareto improvement over a stable lottery $x$. That is, where an agent $i^* \in N \cup M$ has strictly better representative in $y$ than in $x$ and the representative of an agent $i' \in N \cup M$ is not worse off in $y$ compared to $x$. Since $x$ is a stable lottery, we know that for every pair of agents $i\in N$ and $j\in M$, it holds that $\rep(x_i,\succ_i,h_i)\succeq_i j$ or $\rep(x_j,\succ_j,h_j)\succeq_j i$. Without loss of generality suppose $i^* \in N$. Therefore, for every agent $j\in M$, it holds that $\rep(y_{i^*},\succ_{i^*},h_{i^*})\succeq_{i^*} j$. Thus, there in no blocking pair for $y$ containing $i^*$. The representative of an agent $i' \in N \cup M$ is not worse off in $y$ compared to $x$. Thus, from stability of $x$, it holds that for every pair of agents $i\in N \setminus \{i^*\}$ and $j\in M$, it holds that $\rep(y_i,\succ_i,h_i)\succeq_i j$ or $\rep(y_j,\succ_j,h_j)\succeq_j i$. Therefore, $y$ is a stable lottery. 

Finally to prove the theorem, we employee the following algorithm: Given an instance $(N,M,\profile, \h)$, we first compute a stable integral matching $\mu$ for the instance $(N,M,\succ)$ and compute $x$ as described above. Then solve the following LP to check if there exists a Pareto improvement of $x$. Clearly, this process runs in polynomial time.
%
% \begin{algorithm}[t]
%   \KwIn{An instance $(N,M,\profile, \h)$ }
%   \KwOut{A efficient stable lottery $x$}
%     Find a stable matching $\mu$ of the instance $(N,M,\succ)$;
%     Initialize $x(i,j) =1$ if $\mu(i) =j$;
%     \While{}
%     Update $\N \gets \N \setminus \{i,j\}$;
%     }
%    \caption{Efficient Stable lottery}\label{alg:greedy}
% \end{algorithm}
%
%
    %Start with an integral stable matching and perform Pareto-improvements as long as this is possible. The main argument is that, after an improvement, no agent pair $(i,j)$ can become blocking.
\end{proof}

In the following, we consider tradeoffs between efficiency, stability, and strategyproofness.
\begin{theorem}
    Given a two-sided matching instance $(N,M,\profile,\h)$ with $h_i=0$ for every $i\in N\cup M$, no stable lottery mechanism can be simultaneously efficient and strategyproof.
\end{theorem}

% Before we prove the above theorem 

% Now we are ready to prove the theorem.

\begin{proof}
It is known that an ordinally stable lottery can be written as convex combination of integral stable matchings and each integrally stable matchings in the support of a ordinally stable lottery is integrally stable and there is a matching $\mu$ in the support that is efficient \citep{chen2020fractional}. Thus, using \Cref{obs:ordinal-stableh=0}, we have that each integral matching in the support is of a stable lottery $x$ is integral stable. Next, we show that if $x$ is efficient, then $x$ is integral, i.e., there is one integral stable, efficient matching $\mu$ such that $x = \mu$.
For every matching $\mu$ in the support of a lottery, if there is another matching $\mu'$ so that an agent $i$ is worse-off in $\mu'$ as compared to $\mu$, then  the representative of  agent  $i$ is the worse mate she has in $\mu'$ as $h_i=0$. Hence, if the two matchings in the support of $x$ are different, then at least one of them Pareto dominates the lottery $x$ and $x$ is not efficient.  Thus, any lottery that is efficient  is equivalent to an integral matching when $h_i=0$ for each agent $i$. Then, the theorem follows by the well-known impossibility result for integral matching mechanisms~\cite{R82}.
% \todo{is this true?}
    %The main argument is that efficiency with $h_i=0$ implies that the mechanism returns deterministic lotteries. Then, the impossibility of strategyproofness follows by known impossibility results for integral matching mechanisms~\cite{R82}.
\end{proof}

To overcome the impossibility result and show a positive result for higher values of the quantile parameter, we define a restricted version of lotteries and  efficiency for these restricted lotteries. A lottery $x$  has \emph{distinct representatives} if representative of each agent in $x$ is distinct, i.e., for each two distinct agents $i,i' \in N\cup M$ it holds that $\rep(x_i,\succ_i,h_i)\neq \rep(x_{i'},\succ_{i'},h_{i'})$.

\begin{definition}[Distinct Representative Efficient Lotteries]
There is a two-sided matching instance $(N,M,\profile,\h)$, a lottery $x$ is Distinct Representative Efficient (DR-Efficient) if $x$ has distinct representatives and for each agent there is no other lottery $y$ with distinct representatives such that $\rep(y_i,\succ_i,h_i)\succeq_i \rep(x_i,\succ_i,h_i)$ for every $i\in N$ and there is an agent $i^*\in N$ such that $\rep(y,\succ_i,h_i)\succ_i \rep(x,\succ_i,h_i)$.
\end{definition}

To construct a stable lottery that is DR-efficient we will use the deferred acceptance algorithm, arguably, the most common stable matching algorithm.

\paragraph{Deferred Acceptance Algorithm}
The \emph{Deferred Acceptance} (DA) algorithm takes as input the preferences of two sets of agents over each other (commonly referred to as men and women),
In DA for men, it proceeds in rounds where each round consists of two phases. (i) a proposal phase:  Each man without a partner proposes to his most preferred woman from those who have not yet rejected him, and (ii) a rejection phase: each woman receiving multiple proposals rejects all but the one she prefers the most. The algorithm terminates when every agent has a partner. Analogously, one can define DA for women where women are proposing and men accept or reject the proposals.

\citet{GaleShapley1962} showed that DA always terminates with a stable matching. Moreover, the matching obtained by the men-proposing algorithm is the best for all men among all stable matchings, while being the worst for all women \citep{mcvitie1971stable}.

Thus, we say, for the two set of agents $M$ and $N$, a DA algorithm for $M$ (resp. $N$) produces a stable matching that is the \emph{optimal} for all agents in $M$ (resp. all agents in $N$) among all stable matchings. Now we are ready to state the following rule that will be used in the next theorem.

We define a mechanism \emph{half-DA} that produces a lottery computed by running deferred acceptance for the set $N$ and for the set $M$ and taking the two stable matchings, each with probability $1/2$.

\begin{theorem}\label{thm:half-halfSP}
    For two-sided matching instances with $h_i\geq 1/2$ for every agent $i$, the mechanism half-DA is strategyproof and returns a DR-efficient, stable lottery in polynomial time.
\end{theorem}
\begin{proof}
  Let $x$ denote the lottery produced by half-DA. We show that lottery $x$ satisfies the stated properties.

 \textbf{DR-efficiency.} First observe that  Since $h_i \geq 1/2$ $x$ has distinct representatives  and for each agent $i$, from the definition of $x$, we have that each agent's representative is their best stable partner. Suppose there exists another lottery $y$ with distinct representatives where an agent $i^*$ gets a better representative than $x$ and no agent $i' \in N\cup M$ is worse off. Without loss of generality, suppose that $i^* \in N$, then we construct a matching $\mu^*$ as follows each agent $i \in N$ is matched to its representative in $y$. Since no agent $i' \in N\cup M$ is worse off than $x$, then $\mu^*$ is stable, contradicting the fact that deferred acceptance for $N$ produces an optimal stable matching for $N$.

 \textbf{Strategyproofness.} It is well known that no agent from the proposing side can obtain a better partner in the outcome deferred acceptance algorithm by misreporting their preference~\citep{GusfieldIrving1989}. Observe that each agent in $M$ (and $N$) receives their optimal partner in $M$-proposing DA (resp. $N$-proposing DA). 
 % Moreover, the matching induced by an optimal manipulation is stable with respect to the true preferences, i.e., if, by misreporting, the manipulator secures the best possible partner (according to her true preferences) \cite{vaish2017manipulating}. 
 Moreover, in the $N$-proposing DA, the partner of an agent $j \in M$ cannot be better than $j$'s partner in $M$-proposing DA \citep{mcvitie1971stable}. Thus, in the lottery $x$, the representative of $j$ remains the same in a lottery produced by our mechanism even after misreporting. This holds true for any agent in $N\cup M$. Thus, no agent has incentive to misreport.

 \textbf{Stability.} It is easy to observe that $x$ is stable since the representative of each agent is their best stable integral partner. Thus, we prove the theorem.
\end{proof}

\begin{remark}
    Although the mechanism in \Cref{thm:half-halfSP} is DR-efficient, it is not efficient. Consider the following example. Let $N =\{m_1,m_2\} $, $M = \{w_1,w_2\}$, $h_i =1/2$ for each $i \in N \cup M$, and the preference profile $\profile$ is as follows:
    \[m_1: w_1\succ_{m_1}w_2; \quad \quad \quad w_1:m_1\succ_{w_1} m_2\]
    \[m_2: w_1\succ_{m_2}w_2; \quad \quad \quad w_2:m_1\succ_{w_2} m_2\]
    Then the lottery $x'$ Pareto dominates the lottery $x$ produced by the half-DA mechanism, where $x'_{ij}=1/2$ for each $i \in N$ and $j \in M$.
\end{remark}
 
 We identified that distinct representative property ensures the existence of an efficiency  and  strategy-proofness mechanism to compute a stable matching. However, we do not know if it is a necessary condition. Thus, we conclude the section with the following question.
\begin{question}
    What is an, as general as possible, sufficient and necessary condition that allows for efficiency, stability, and strategyproofness?
\end{question}



 \section*{Acknowledgements}
We thank Felix Brandt and Herve Moulin for helpful discussions and pointers to the literature. The work of Caragiannis has been partially supported by Independent Research Fund Denmark (DFF) under grant 2032-00185B.

\bibliographystyle{plainnat}
\bibliography{arxiv}

\end{document}


