\section{Related work}
Lotteries over alternatives and were first formally studied by~\citet{zeckhauser1969majority}, \citet{fishburn1972lotteries}, and \citet{intriligator1973probabilistic}.
Randomization in voting processes have been considered to address impossibility results in deterministic social choice mechanisms, such as those established by Arrow’s Impossibility Theorem~\citep{arrow1951}. One of the most famous results in this area is by~\citet{Gib73} and~\citet{Sat75}, who showed that deterministic strategy-proof mechanisms must be dictatorial, inspiring the use of randomization to achieve strategy-proofness and fairness. \citet{bogomolnaia2001} 
proposed the Random Priority (RP) mechanism, ensuring fairness and strategy-proofness in resource allocation and voting.
\citet{procaccia2010} designed randomized voting rules that are approximation of score-based deterministic voting rules to achieve strategy-proofness. 
\citet{a18} show that the efficiency and strategyproofness of a system can vary depending on how preferences over alternatives are extended to preferences over lotteries. They consider preference extensions using stochastic dominance (SD),  pairwise comparisons (PC), bilinear dominance (BD), and sure-thing principle (ST). They showed that while random serial dictatorships are PC-strategy-proof, they only achieve ex post efficiency, strict maximal lotteries are both PC-efficient and ST-strategy-proof, and show multiple incompatibility results. See also the survey of related results by~\citet{B17}.

In resource allocation problems consisting of agents that need to be matched to items, envy-freeness~\citep{foley1966resource,varian1973equity} and proportionality~\citep{steinhaus1948problem} have been widely studied in both economics and computer science as measures of fairness.  
\citet{hylland1979efficient} focused on the probabilistic allocations of indivisible items without relying on a priority structure and showed that this approach is more efficient than randomization over deterministic methods based on priority. \citet{abdulkadirouglu2003ordinal} studied lotteries in the house allocation problem, which is another variation of the problem of matching under one-sided preferences.
\citet{aziz2019probabilistic} explored the advantages of applying randomization in social choice scenarios, such as fair division, and examined the associated challenges. \citet{caragiannis2021interim} studied interim envy-freeness for lotteries. For resource allocation, obtaining both ex-post and ex-ante guarantees simultaneously has also been considered~\citep{babaioff2022best,aziz2024best}.  

In the two-sided matching literature, a version of the random-matching problem, where both sides of the market have strict preferences, was studied by \citet{RothRothblumVate1993lsm-lattice}, who showed that the set of fractional stable matchings has a lattice structure. One can interpret a fractional deterministic matching as a random matching using Birkhoff-von Neumann's theorem \citep[Theorem~3.2.6]{horn_topics_1991}.  
Thus, stable fractional matchings has been studied as a relaxation of the integrality constraint in the linear equations encoding stability~\citep{vandevate_linear_1989,abeledo_stable_1994,teo1998geometry} which were forerunner of numerous other works~\citep{BiroCechFleiner08stablehalfmatchings,KiPRST2013osmhypegraphic,KU2015,IK2018osm-hypergraphic,aziz_random_2019,CFKV20j,dogan_efficiency_2016,manjunath2013stability,manjunath2014markets}.
In particular, stable fractional matchings have been considered by \citet{AF03Scarfs}, who proposed the definition of ordinal stability. We remark that, in our setting with agents having quantile utilities, stability generalizes both ordinal and integral stability. \citet{AF03Scarfs} studied ordinal stability in the hypergraphic setting and found that Scarf’s lemma from game theory guarantees the existence of ordinally stable matchings.
Later, \citet{aziz_random_2019} studied multiple notions of stability and the relations between them.
\citet{alkan2003stable} studied a type of stability for fractional matching when both sides
are equipped with complex preference structures.
\citet{CFKV20j} defined cardinal stable lotteries when both sides have cardinal preferences as opposed to rankings or ordinal preferences. \citet{chen2020fractional} studied both cardinal and ordinal stability. 
Lotteries have been extensively studied for another application of matching under preferences, namely, the school choice problem \citep{abdulkadirouglu2003school,KU2015}. In particular, \citet{KU2015} studied ordinal stability, which they refer to as ex-ante stability.