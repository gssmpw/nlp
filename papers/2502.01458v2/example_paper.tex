 %%%%%%%% ICML 2024 EXAMPLE LATEX SUBMISSION FILE %%%%%%%%%%%%%%%%%

\documentclass{article}

% Recommended, but optional, packages for figures and better typesetting:
\usepackage{microtype}
\usepackage{graphicx}
\usepackage{subfigure}
\usepackage{booktabs} % for professional tables
\usepackage{subcaption}
\usepackage{float}

% hyperref makes hyperlinks in the resulting PDF.
% If your build breaks (sometimes temporarily if a hyperlink spans a page)
% please comment out the following usepackage line and replace
% \usepackage{icml2024} with \usepackage[nohyperref]{icml2024} above.
\usepackage{hyperref}


% Attempt to make hyperref and algorithmic work together better:
\newcommand{\theHalgorithm}{\arabic{algorithm}}


% \usepackage{icml2024}
\usepackage[accepted]{icml2024}

% For theorems and such
\usepackage{amsmath}
\usepackage{amssymb}
\usepackage{mathtools}
\usepackage{amsthm}



\usepackage{titlesec}


% 目录行距
\makeatletter
\renewcommand{\tableofcontents}{%
    \section*{\contentsname}%
    \begingroup
    \setlength{\baselineskip}{10pt} % 调整行距,值越小行距越窄
    \@starttoc{toc}%
    \endgroup
}
\makeatother


% 重新定义 \paragraph{} 的间距
\titleformat{\paragraph}[runin] % 设置段落标题格式
{\normalfont\normalsize\bfseries}{\theparagraph}{0.5em}{}
\titlespacing{\paragraph}{0pt}{0.7ex}{0.5em} 



\usepackage[capitalize,noabbrev]{cleveref}

% \theoremstyle{plain}
\newtheorem{theorem}{Theorem}
\newtheorem{corollary}{Corollary}
\newtheorem{proposition}{Proposition}
\newtheorem{definition}{Definition}
\newtheorem{assumption}{Assumption}
\newtheorem{claim}{Claim}
\newtheorem{lemma}{Lemma}
\newtheorem*{remark}{Remark}



\newcommand{\R}{\mathbb{R}}
\newcommand{\E}{\mathbb{E}}
\newcommand{\cX}{\mathcal{X}}
\newcommand{\cY}{\mathcal{Y}}
\newcommand{\cI}{\mathcal{I}}
\newcommand{\cC}{\mathcal{C}}
\newcommand{\cE}{\mathcal{E}}
\newcommand{\expect}{\mathbb{E}}
\newcommand{\prob}{\mathbb{P}}
\newcommand{\cO}{\mathcal{O}}
\newcommand{\cH}{\mathcal{H}}
\newcommand{\cF}{\mathcal{F}}
\newcommand{\cP}{\mathcal{P}}
\newcommand{\bE}{\mathbb{E}}
\newcommand{\argmax}{\text{argmax}}
\newcommand{\argmin}{\text{argmin}}
\newcommand{\eps}{\varepsilon}
\newcommand{\kl}{\mathrm{d}_{\mathrm{KL}}}
\newcommand{\tv}{\mathrm{D}_{\mathrm{TV}}}
\newcommand{\sq}{\mathrm{D}_{\mathrm{SQ}}}
\newcommand{\is}{\mathrm{D}_{\mathrm{IS}}}
\newcommand{\wis}{\mathrm{D}_{\mathrm{WIS}}}
\newcommand{\dist}{d_\mathcal{P}}
\newcommand{\disthat}{\hat{d}_\mathcal{P}}


\allowdisplaybreaks[4]

\usepackage{url}
\def\UrlBreaks{\do\A\do\B\do\C\do\D\do\E\do\F\do\G\do\H\do\I\do\J\do\K\do\L\do\M\do\N\do\O\do\P\do\Q\do\R\do\S\do\T\do\U\do\V\do\W\do\X\do\Y\do\Z\do\[\do\\\do\]\do\^\do\_\do\`\do\a\do\b\do\c\do\d\do\e\do\f\do\g\do\h\do\i\do\j\do\k\do\l\do\m\do\n\do\o\do\p\do\q\do\r\do\s\do\t\do\u\do\v\do\w\do\x\do\y\do\z\do\.\do\@\do\\\do\/\do\!\do\_\do\|\do\;\do\>\do\]\do\)\do\,\do\?\do\'\do+\do\=\do\#}
\urlstyle{same}


\usepackage{xcolor}
\newcommand{\yw}[1]{\textcolor{red}{#1}}


% Todonotes is useful during development; simply uncomment the next line
%    and comment out the line below the next line to turn off comments
%\usepackage[disable,textsize=tiny]{todonotes}
\usepackage[textsize=tiny]{todonotes}







% The \icmltitle you define below is probably too long as a header.
% Therefore, a short form for the running title is supplied here:
\icmltitlerunning{Understanding the Capabilities and Limitations of Weak-to-Strong Generalization}



\begin{document}

\twocolumn[
\icmltitle{Understanding the Capabilities and Limitations of \\ Weak-to-Strong Generalization}
% List of affiliations: The first argument should be a (short)
% identifier you will use later to specify author affiliations
% Academic affiliations should list Department, University, City, Region, Country
% Industry affiliations should list Company, City, Region, Country
% You can specify symbols, otherwise they are numbered in order.
% Ideally, you should not use this facility. Affiliations will be numbered in order of appearance and this is the preferred way.










% camera ready :)
\icmlsetsymbol{equal}{*}

\begin{icmlauthorlist}
\icmlauthor{Wei Yao}{ruc}
\icmlauthor{Wenkai Yang}{ruc} 
\icmlauthor{Ziqiao Wang}{tongji}
\icmlauthor{Yankai Lin}{ruc}
\icmlauthor{Yong Liu}{ruc}
\end{icmlauthorlist}

\icmlaffiliation{ruc}{Gaoling School of Artificial Intelligence, Renmin University of China, Beijing, China}
\icmlaffiliation{tongji}{School of Computer Science and Technology, Tongji University, Shanghai, China}

\icmlcorrespondingauthor{Yong Liu}{liuyonggsai@ruc.edu.cn}


\icmlkeywords{Weak-to-Strong Generalization, AI Alignment}
\vskip 0.3in
]


% # camera ready :)
\printAffiliationsAndNotice{}
% \printAffiliationsAndNotice{\icmlEqualContribution} 



\begin{abstract}
Weak-to-strong generalization, where weakly supervised strong models outperform their weaker teachers, offers a promising approach to aligning superhuman models with human values. 
To deepen the understanding of this approach, we provide theoretical insights into its capabilities and limitations. 
First, in the classification setting, we establish upper and lower generalization error bounds for the strong model, identifying the primary limitations as stemming from the weak model's generalization error and the optimization objective itself.
Additionally, we derive lower and upper bounds on the calibration error of the strong model. 
These theoretical bounds reveal two critical insights: (1) the weak model should demonstrate strong generalization performance and maintain well-calibrated predictions, and (2) the strong model's training process must strike a careful balance, as excessive optimization could undermine its generalization capability by over-relying on the weak supervision signals.
Finally, in the regression setting, we extend the work of~\citet{charikar2024quantifying} to a loss function based on KL divergence, offering guarantees that the strong student can outperform its weak teacher by at least the magnitude of their disagreement. 
The theory is validated through sufficient experiments.

\end{abstract}





\section{Introduction}
\label{sec:introduction}
The business processes of organizations are experiencing ever-increasing complexity due to the large amount of data, high number of users, and high-tech devices involved \cite{martin2021pmopportunitieschallenges, beerepoot2023biggestbpmproblems}. This complexity may cause business processes to deviate from normal control flow due to unforeseen and disruptive anomalies \cite{adams2023proceddsriftdetection}. These control-flow anomalies manifest as unknown, skipped, and wrongly-ordered activities in the traces of event logs monitored from the execution of business processes \cite{ko2023adsystematicreview}. For the sake of clarity, let us consider an illustrative example of such anomalies. Figure \ref{FP_ANOMALIES} shows a so-called event log footprint, which captures the control flow relations of four activities of a hypothetical event log. In particular, this footprint captures the control-flow relations between activities \texttt{a}, \texttt{b}, \texttt{c} and \texttt{d}. These are the causal ($\rightarrow$) relation, concurrent ($\parallel$) relation, and other ($\#$) relations such as exclusivity or non-local dependency \cite{aalst2022pmhandbook}. In addition, on the right are six traces, of which five exhibit skipped, wrongly-ordered and unknown control-flow anomalies. For example, $\langle$\texttt{a b d}$\rangle$ has a skipped activity, which is \texttt{c}. Because of this skipped activity, the control-flow relation \texttt{b}$\,\#\,$\texttt{d} is violated, since \texttt{d} directly follows \texttt{b} in the anomalous trace.
\begin{figure}[!t]
\centering
\includegraphics[width=0.9\columnwidth]{images/FP_ANOMALIES.png}
\caption{An example event log footprint with six traces, of which five exhibit control-flow anomalies.}
\label{FP_ANOMALIES}
\end{figure}

\subsection{Control-flow anomaly detection}
Control-flow anomaly detection techniques aim to characterize the normal control flow from event logs and verify whether these deviations occur in new event logs \cite{ko2023adsystematicreview}. To develop control-flow anomaly detection techniques, \revision{process mining} has seen widespread adoption owing to process discovery and \revision{conformance checking}. On the one hand, process discovery is a set of algorithms that encode control-flow relations as a set of model elements and constraints according to a given modeling formalism \cite{aalst2022pmhandbook}; hereafter, we refer to the Petri net, a widespread modeling formalism. On the other hand, \revision{conformance checking} is an explainable set of algorithms that allows linking any deviations with the reference Petri net and providing the fitness measure, namely a measure of how much the Petri net fits the new event log \cite{aalst2022pmhandbook}. Many control-flow anomaly detection techniques based on \revision{conformance checking} (hereafter, \revision{conformance checking}-based techniques) use the fitness measure to determine whether an event log is anomalous \cite{bezerra2009pmad, bezerra2013adlogspais, myers2018icsadpm, pecchia2020applicationfailuresanalysispm}. 

The scientific literature also includes many \revision{conformance checking}-independent techniques for control-flow anomaly detection that combine specific types of trace encodings with machine/deep learning \cite{ko2023adsystematicreview, tavares2023pmtraceencoding}. Whereas these techniques are very effective, their explainability is challenging due to both the type of trace encoding employed and the machine/deep learning model used \cite{rawal2022trustworthyaiadvances,li2023explainablead}. Hence, in the following, we focus on the shortcomings of \revision{conformance checking}-based techniques to investigate whether it is possible to support the development of competitive control-flow anomaly detection techniques while maintaining the explainable nature of \revision{conformance checking}.
\begin{figure}[!t]
\centering
\includegraphics[width=\columnwidth]{images/HIGH_LEVEL_VIEW.png}
\caption{A high-level view of the proposed framework for combining \revision{process mining}-based feature extraction with dimensionality reduction for control-flow anomaly detection.}
\label{HIGH_LEVEL_VIEW}
\end{figure}

\subsection{Shortcomings of \revision{conformance checking}-based techniques}
Unfortunately, the detection effectiveness of \revision{conformance checking}-based techniques is affected by noisy data and low-quality Petri nets, which may be due to human errors in the modeling process or representational bias of process discovery algorithms \cite{bezerra2013adlogspais, pecchia2020applicationfailuresanalysispm, aalst2016pm}. Specifically, on the one hand, noisy data may introduce infrequent and deceptive control-flow relations that may result in inconsistent fitness measures, whereas, on the other hand, checking event logs against a low-quality Petri net could lead to an unreliable distribution of fitness measures. Nonetheless, such Petri nets can still be used as references to obtain insightful information for \revision{process mining}-based feature extraction, supporting the development of competitive and explainable \revision{conformance checking}-based techniques for control-flow anomaly detection despite the problems above. For example, a few works outline that token-based \revision{conformance checking} can be used for \revision{process mining}-based feature extraction to build tabular data and develop effective \revision{conformance checking}-based techniques for control-flow anomaly detection \cite{singh2022lapmsh, debenedictis2023dtadiiot}. However, to the best of our knowledge, the scientific literature lacks a structured proposal for \revision{process mining}-based feature extraction using the state-of-the-art \revision{conformance checking} variant, namely alignment-based \revision{conformance checking}.

\subsection{Contributions}
We propose a novel \revision{process mining}-based feature extraction approach with alignment-based \revision{conformance checking}. This variant aligns the deviating control flow with a reference Petri net; the resulting alignment can be inspected to extract additional statistics such as the number of times a given activity caused mismatches \cite{aalst2022pmhandbook}. We integrate this approach into a flexible and explainable framework for developing techniques for control-flow anomaly detection. The framework combines \revision{process mining}-based feature extraction and dimensionality reduction to handle high-dimensional feature sets, achieve detection effectiveness, and support explainability. Notably, in addition to our proposed \revision{process mining}-based feature extraction approach, the framework allows employing other approaches, enabling a fair comparison of multiple \revision{conformance checking}-based and \revision{conformance checking}-independent techniques for control-flow anomaly detection. Figure \ref{HIGH_LEVEL_VIEW} shows a high-level view of the framework. Business processes are monitored, and event logs obtained from the database of information systems. Subsequently, \revision{process mining}-based feature extraction is applied to these event logs and tabular data input to dimensionality reduction to identify control-flow anomalies. We apply several \revision{conformance checking}-based and \revision{conformance checking}-independent framework techniques to publicly available datasets, simulated data of a case study from railways, and real-world data of a case study from healthcare. We show that the framework techniques implementing our approach outperform the baseline \revision{conformance checking}-based techniques while maintaining the explainable nature of \revision{conformance checking}.

In summary, the contributions of this paper are as follows.
\begin{itemize}
    \item{
        A novel \revision{process mining}-based feature extraction approach to support the development of competitive and explainable \revision{conformance checking}-based techniques for control-flow anomaly detection.
    }
    \item{
        A flexible and explainable framework for developing techniques for control-flow anomaly detection using \revision{process mining}-based feature extraction and dimensionality reduction.
    }
    \item{
        Application to synthetic and real-world datasets of several \revision{conformance checking}-based and \revision{conformance checking}-independent framework techniques, evaluating their detection effectiveness and explainability.
    }
\end{itemize}

The rest of the paper is organized as follows.
\begin{itemize}
    \item Section \ref{sec:related_work} reviews the existing techniques for control-flow anomaly detection, categorizing them into \revision{conformance checking}-based and \revision{conformance checking}-independent techniques.
    \item Section \ref{sec:abccfe} provides the preliminaries of \revision{process mining} to establish the notation used throughout the paper, and delves into the details of the proposed \revision{process mining}-based feature extraction approach with alignment-based \revision{conformance checking}.
    \item Section \ref{sec:framework} describes the framework for developing \revision{conformance checking}-based and \revision{conformance checking}-independent techniques for control-flow anomaly detection that combine \revision{process mining}-based feature extraction and dimensionality reduction.
    \item Section \ref{sec:evaluation} presents the experiments conducted with multiple framework and baseline techniques using data from publicly available datasets and case studies.
    \item Section \ref{sec:conclusions} draws the conclusions and presents future work.
\end{itemize}

\section{RELATED WORK}
\label{sec:relatedwork}
In this section, we describe the previous works related to our proposal, which are divided into two parts. In Section~\ref{sec:relatedwork_exoplanet}, we present a review of approaches based on machine learning techniques for the detection of planetary transit signals. Section~\ref{sec:relatedwork_attention} provides an account of the approaches based on attention mechanisms applied in Astronomy.\par

\subsection{Exoplanet detection}
\label{sec:relatedwork_exoplanet}
Machine learning methods have achieved great performance for the automatic selection of exoplanet transit signals. One of the earliest applications of machine learning is a model named Autovetter \citep{MCcauliff}, which is a random forest (RF) model based on characteristics derived from Kepler pipeline statistics to classify exoplanet and false positive signals. Then, other studies emerged that also used supervised learning. \cite{mislis2016sidra} also used a RF, but unlike the work by \citet{MCcauliff}, they used simulated light curves and a box least square \citep[BLS;][]{kovacs2002box}-based periodogram to search for transiting exoplanets. \citet{thompson2015machine} proposed a k-nearest neighbors model for Kepler data to determine if a given signal has similarity to known transits. Unsupervised learning techniques were also applied, such as self-organizing maps (SOM), proposed \citet{armstrong2016transit}; which implements an architecture to segment similar light curves. In the same way, \citet{armstrong2018automatic} developed a combination of supervised and unsupervised learning, including RF and SOM models. In general, these approaches require a previous phase of feature engineering for each light curve. \par

%DL is a modern data-driven technology that automatically extracts characteristics, and that has been successful in classification problems from a variety of application domains. The architecture relies on several layers of NNs of simple interconnected units and uses layers to build increasingly complex and useful features by means of linear and non-linear transformation. This family of models is capable of generating increasingly high-level representations \citep{lecun2015deep}.

The application of DL for exoplanetary signal detection has evolved rapidly in recent years and has become very popular in planetary science.  \citet{pearson2018} and \citet{zucker2018shallow} developed CNN-based algorithms that learn from synthetic data to search for exoplanets. Perhaps one of the most successful applications of the DL models in transit detection was that of \citet{Shallue_2018}; who, in collaboration with Google, proposed a CNN named AstroNet that recognizes exoplanet signals in real data from Kepler. AstroNet uses the training set of labelled TCEs from the Autovetter planet candidate catalog of Q1–Q17 data release 24 (DR24) of the Kepler mission \citep{catanzarite2015autovetter}. AstroNet analyses the data in two views: a ``global view'', and ``local view'' \citep{Shallue_2018}. \par


% The global view shows the characteristics of the light curve over an orbital period, and a local view shows the moment at occurring the transit in detail

%different = space-based

Based on AstroNet, researchers have modified the original AstroNet model to rank candidates from different surveys, specifically for Kepler and TESS missions. \citet{ansdell2018scientific} developed a CNN trained on Kepler data, and included for the first time the information on the centroids, showing that the model improves performance considerably. Then, \citet{osborn2020rapid} and \citet{yu2019identifying} also included the centroids information, but in addition, \citet{osborn2020rapid} included information of the stellar and transit parameters. Finally, \citet{rao2021nigraha} proposed a pipeline that includes a new ``half-phase'' view of the transit signal. This half-phase view represents a transit view with a different time and phase. The purpose of this view is to recover any possible secondary eclipse (the object hiding behind the disk of the primary star).


%last pipeline applies a procedure after the prediction of the model to obtain new candidates, this process is carried out through a series of steps that include the evaluation with Discovery and Validation of Exoplanets (DAVE) \citet{kostov2019discovery} that was adapted for the TESS telescope.\par
%



\subsection{Attention mechanisms in astronomy}
\label{sec:relatedwork_attention}
Despite the remarkable success of attention mechanisms in sequential data, few papers have exploited their advantages in astronomy. In particular, there are no models based on attention mechanisms for detecting planets. Below we present a summary of the main applications of this modeling approach to astronomy, based on two points of view; performance and interpretability of the model.\par
%Attention mechanisms have not yet been explored in all sub-areas of astronomy. However, recent works show a successful application of the mechanism.
%performance

The application of attention mechanisms has shown improvements in the performance of some regression and classification tasks compared to previous approaches. One of the first implementations of the attention mechanism was to find gravitational lenses proposed by \citet{thuruthipilly2021finding}. They designed 21 self-attention-based encoder models, where each model was trained separately with 18,000 simulated images, demonstrating that the model based on the Transformer has a better performance and uses fewer trainable parameters compared to CNN. A novel application was proposed by \citet{lin2021galaxy} for the morphological classification of galaxies, who used an architecture derived from the Transformer, named Vision Transformer (VIT) \citep{dosovitskiy2020image}. \citet{lin2021galaxy} demonstrated competitive results compared to CNNs. Another application with successful results was proposed by \citet{zerveas2021transformer}; which first proposed a transformer-based framework for learning unsupervised representations of multivariate time series. Their methodology takes advantage of unlabeled data to train an encoder and extract dense vector representations of time series. Subsequently, they evaluate the model for regression and classification tasks, demonstrating better performance than other state-of-the-art supervised methods, even with data sets with limited samples.

%interpretation
Regarding the interpretability of the model, a recent contribution that analyses the attention maps was presented by \citet{bowles20212}, which explored the use of group-equivariant self-attention for radio astronomy classification. Compared to other approaches, this model analysed the attention maps of the predictions and showed that the mechanism extracts the brightest spots and jets of the radio source more clearly. This indicates that attention maps for prediction interpretation could help experts see patterns that the human eye often misses. \par

In the field of variable stars, \citet{allam2021paying} employed the mechanism for classifying multivariate time series in variable stars. And additionally, \citet{allam2021paying} showed that the activation weights are accommodated according to the variation in brightness of the star, achieving a more interpretable model. And finally, related to the TESS telescope, \citet{morvan2022don} proposed a model that removes the noise from the light curves through the distribution of attention weights. \citet{morvan2022don} showed that the use of the attention mechanism is excellent for removing noise and outliers in time series datasets compared with other approaches. In addition, the use of attention maps allowed them to show the representations learned from the model. \par

Recent attention mechanism approaches in astronomy demonstrate comparable results with earlier approaches, such as CNNs. At the same time, they offer interpretability of their results, which allows a post-prediction analysis. \par



% !TEX root =  ../main.tex
\section{Background on causality and abstraction}\label{sec:preliminaries}

This section provides the notation and key concepts related to causal modeling and abstraction theory.

\spara{Notation.} The set of integers from $1$ to $n$ is $[n]$.
The vectors of zeros and ones of size $n$ are $\zeros_n$ and $\ones_n$.
The identity matrix of size $n \times n$ is $\identity_n$. The Frobenius norm is $\frob{\mathbf{A}}$.
The set of positive definite matrices over $\reall^{n\times n}$ is $\pd^n$. The Hadamard product is $\odot$.
Function composition is $\circ$.
The domain of a function is $\dom{\cdot}$ and its kernel $\ker$.
Let $\mathcal{M}(\mathcal{X}^n)$ be the set of Borel measures over $\mathcal{X}^n \subseteq \reall^n$. Given a measure $\mu^n \in \mathcal{M}(\mathcal{X}^n)$ and a measurable map $\varphi^{\V}$, $\mathcal{X}^n \ni \mathbf{x} \overset{\varphi^{\V}}{\longmapsto} \V^\top \mathbf{x} \in \mathcal{X}^m$, we denote by $\varphi^{\V}_{\#}(\mu^n) \coloneqq \mu^n(\varphi^{\V^{-1}}(\mathbf{x}))$ the pushforward measure $\mu^m \in \mathcal{M}(\mathcal{X}^m)$. 


We now present the standard definition of SCM.

\begin{definition}[SCM, \citealp{pearl2009causality}]\label{def:SCM}
A (Markovian) structural causal model (SCM) $\scm^n$ is a tuple $\langle \myendogenous, \myexogenous, \myfunctional, \zeta^\myexogenous \rangle$, where \emph{(i)} $\myendogenous = \{X_1, \ldots, X_n\}$ is a set of $n$ endogenous random variables; \emph{(ii)} $\myexogenous =\{Z_1,\ldots,Z_n\}$ is a set of $n$ exogenous variables; \emph{(iii)} $\myfunctional$ is a set of $n$ functional assignments such that $X_i=f_i(\parents_i, Z_i)$, $\forall \; i \in [n]$, with $ \parents_i \subseteq \myendogenous \setminus \{ X_i\}$; \emph{(iv)} $\zeta^\myexogenous$ is a product probability measure over independent exogenous variables $\zeta^\myexogenous=\prod_{i \in [n]} \zeta^i$, where $\zeta^i=P(Z_i)$. 
\end{definition}
A Markovian SCM induces a directed acyclic graph (DAG) $\mathcal{G}_{\scm^n}$ where the nodes represent the variables $\myendogenous$ and the edges are determined by the structural functions $\myfunctional$; $ \parents_i$ constitutes then the parent set for $X_i$. Furthermore, we can recursively rewrite the set of structural function $\myfunctional$ as a set of mixing functions $\mymixing$ dependent only on the exogenous variables (cf. \cref{app:CA}). A key feature for studying causality is the possibility of defining interventions on the model:
\begin{definition}[Hard intervention, \citealp{pearl2009causality}]\label{def:intervention}
Given SCM $\scm^n = \langle \myendogenous, \myexogenous, \myfunctional, \zeta^\myexogenous \rangle$, a (hard) intervention $\iota = \operatorname{do}(\myendogenous^{\iota} = \mathbf{x}^{\iota})$, $\myendogenous^{\iota}\subseteq \myendogenous$,
is an operator that generates a new post-intervention SCM $\scm^n_\iota = \langle \myendogenous, \myexogenous, \myfunctional_\iota, \zeta^\myexogenous \rangle$ by replacing each function $f_i$ for $X_i\in\myendogenous^{\iota}$ with the constant $x_i^\iota\in \mathbf{x}^\iota$. 
Graphically, an intervention mutilates $\mathcal{G}_{\mathsf{M}^n}$ by removing all the incoming edges of the variables in $\myendogenous^{\iota}$.
\end{definition}

Given multiple SCMs describing the same system at different levels of granularity, CA provides the definition of an $\alpha$-abstraction map to relate these SCMs:
\begin{definition}[$\abst$-abstraction, \citealp{rischel2020category}]\label{def:abstraction}
Given low-level $\mathsf{M}^\ell$ and high-level $\mathsf{M}^h$ SCMs, an $\abst$-abstraction is a triple $\abst = \langle \Rset, \amap, \alphamap{} \rangle$, where \emph{(i)} $\Rset \subseteq \datalow$ is a subset of relevant variables in $\mathsf{M}^\ell$; \emph{(ii)} $\amap: \Rset \rightarrow \datahigh$ is a surjective function between the relevant variables of $\mathsf{M}^\ell$ and the endogenous variables of $\mathsf{M}^h$; \emph{(iii)} $\alphamap{}: \dom{\Rset} \rightarrow \dom{\datahigh}$ is a modular function $\alphamap{} = \bigotimes_{i\in[n]} \alphamap{X^h_i}$ made up by surjective functions $\alphamap{X^h_i}: \dom{\amap^{-1}(X^h_i)} \rightarrow \dom{X^h_i}$ from the outcome of low-level variables $\amap^{-1}(X^h_i) \in \datalow$ onto outcomes of the high-level variables $X^h_i \in \datahigh$.
\end{definition}
Notice that an $\abst$-abstraction simultaneously maps variables via the function $\amap$ and values through the function $\alphamap{}$. The definition itself does not place any constraint on these functions, although a common requirement in the literature is for the abstraction to satisfy \emph{interventional consistency} \cite{rubenstein2017causal,rischel2020category,beckers2019abstracting}. An important class of such well-behaved abstractions is \emph{constructive linear abstraction}, for which the following properties hold. By constructivity, \emph{(i)} $\abst$ is interventionally consistent; \emph{(ii)} all low-level variables are relevant $\Rset=\datalow$; \emph{(iii)} in addition to the map $\alphamap{}$ between endogenous variables, there exists a map ${\alphamap{}}_U$ between exogenous variables satisfying interventional consistency \cite{beckers2019abstracting,schooltink2024aligning}. By linearity, $\alphamap{} = \V^\top \in \reall^{h \times \ell}$ \cite{massidda2024learningcausalabstractionslinear}. \cref{app:CA} provides formal definitions for interventional consistency, linear and constructive abstraction.

\section{Universal Results in WTSG} \label{section:universal_result}

In this section, we consider the classification problem, where $\dist$ is the KL divergence loss defined in~\cref{def:cross_entropy}.
We first establish lower and upper generalization error bounds of the strong model in WTSG in~\cref{section_lower_upper}.
Then the lower and upper calibration error bounds are shown in~\cref{subsec:suff_nece_condition}.



\subsection{Lower and Upper Bound} \label{section_lower_upper}



\begin{theorem}[Proved in \cref{proof_lemma_inf}] \label{lemma:upper_lower_inf}
% Let $\dist$ be the KL divergence loss.
Given the data domain $\cX$, output domain $\cY$ and models $F_{sw}, F_w, F^\star$ defined above. 
Then there holds
\begin{multline}
    \dist\left( F^\star, F_w \right) - C_1\sqrt{\dist(F_w, F_{sw})} \\ \le \dist(F^\star, F_{sw}) \le \\ \dist\left( F^\star, F_w \right) + C_1\sqrt{\dist(F_w, F_{sw})},
\end{multline}
% \begin{align}
%     \left| \dist(F^\star, F_{sw}) - \dist\left( F^\star, F_w \right) \right| \le \sqrt{C_\gamma \dist(F_{sw}, F_w)},
% \end{align}
where $C_1$ is a positive constant.
\end{theorem}

\begin{remark}
    The proof can be also extended to the regression setting, which is provided in~\cref{proof:lower_upper}.
\end{remark}

% \begin{remark}
%     The lower bound can be tighter by reducing $C_1$, which is proved in~\cref{constant:theorem}.
% \end{remark}

% \begin{remark}
%     Our proof can also cover reverse KL divergence~\citep{malinin2019reverse,he2024training,shi2024choosy}, with the corresponding proof in~\cref{proof_lemma_inf}.
% \end{remark}
% , which is also used in AI alignment literature~\citep{wang2023beyond}


\cref{lemma:upper_lower_inf} provides a quantitative framework for assessing the performance gap between weak model and strong model in WTSG.
Specifically, the value of $\dist(F^\star, F_{sw})$ is constrained by two terms: 
(1) $\dist(F^\star, F_{sw})$, which reflects the performance of the weak model, and 
(2) $\dist(F_w, F_{sw})$, which is decided by the optimization result in~\cref{eqn:fsw-population-minimizer} and measures how the strong model learns to imitate the weak supervisor.
This result is examined from two complementary perspectives: a lower bound and an upper bound.
They offer insights into the fundamental limitation and theoretical guarantee for WTSG. 

\noindent \textbf{Lower bound.}
The lower bound indicates the fundamental limitation: $\dist(F^\star, F_{sw})$ cannot be arbitrarily reduced.
Firstly, a minimal $\dist(F^\star, F_{sw})$ is intrinsically tied to the weak model performance $\dist\left( F^\star, F_w \right)$.
To improve the strong model, the weak model becomes critical---that is, $\dist\left( F^\star, F_w \right)$ should be as small as possible. It underscores the importance of \textbf{\textit{carefully selecting the weak model}}~\citep{burns2023weak,yang2024super}.
Secondly, the performance improvement of strong model over the weak model cannot exceed $\cO \left(\sqrt{\dist(F_w, F_{sw})} \right)$.
In WTSG, while the student-supervisor disagreement $\dist(F_w, F_{sw})$ is minimized in~\cref{eqn:fsw-population-minimizer}, we anticipate $\cO \left(\sqrt{\dist(F_w, F_{sw})} \right)$ to remain relatively small.
However, a paradox arises: achieving a smaller $\dist\left( F^\star, F_{sw} \right)$ necessitates a larger $\dist(F_w, F_{sw})$.
This implies that \textbf{\textit{the performance improvement of WTSG is probably constrained by its own optimization objective}}.





\noindent \textbf{Upper bound.}
The upper bound provides a theoretical guarantee for WTSG by ensuring that $\dist(F^\star, F_{sw})$ remains bounded and does not grow arbitrarily large.
Firstly, effective WTSG requires choosing a weak model that produces supervision signal closely aligned with the true score, i.e., achieving a small $\dist\left( F^\star, F_w\right)$. 
To this end, employing a stronger weak model is crucial to obtain a tighter upper bound of $\dist(F^\star, F_{sw})$.
Secondly, the worst-case performance of the strong model is constrained by the sum of $\dist\left( F^\star, F_w\right)$ and $\cO \left(\sqrt{\dist(F_w, F_{sw})} \right)$.
By appropriately selecting the weak model and determining the minimizer of~\cref{eqn:fsw-population-minimizer}, both $\dist\left( F^\star, F_w\right)$ and $\cO \left(\sqrt{\dist(F_w, F_{sw})} \right)$ can be kept small, ensuring the effectiveness and practicality of the strong model.
















\subsection{Calibration in Weak-to-strong Generalization} \label{subsec:suff_nece_condition}

% In the previous section, \cref{lemma:upper_lower_inf} does not ensure that the strong model will necessarily surpass the performance of its weak supervisor, i.e., $\dist(F^\star, F_{sw}) \le \dist\left( F^\star, F_w \right)$.
% Establishing such a bound would provide a stronger theoretical foundation for weak-to-strong generalization. 
% This raises the question of whether a tighter upper bound can be derived, such as $\dist(F^\star, F_{sw}) \le \dist\left( F^\star, F_w \right) - \dist(F_w, F_{sw})$.
% The previous section sheds light on upper and lower bounds of the strong model performance.
In this section, we further explore WTSG through the lens of calibration~\citep{kumar2019verified}, which requires that the prediction confidence should match the actual outcome.
We first state the definition of Marginal Calibration Error (MCE)~\citep{kumar2019verified}, which is an extended version of Expected Calibration Error (ECE)~\citep{guo2017calibration} designed for multi-class classification.
In particular, we use an $\ell_1$ version of it, with the weight constant $\frac{1}{k}$ omitted.
\begin{definition}[Marginal Calibration Error~\citep{kumar2019verified}] \label{def:cal:mce}
Let $x \in \cX$, ground truth $y=[y_1, \cdots, y_k]^T \in \{ 0,1 \}^k$ where $\sum_{i=1}^k y_i=1$, and a model $f: \cX \to \cY$.
Define the marginal calibration error of $f$ as:
\begin{align} \label{def:cal_err}
    \textit{MCE}(f) = \sum_{i=1}^k \expect_x \left| [f(x)]_i-\prob[y_i=1|[f(x)]_i] \right|.
\end{align}
\end{definition}

\vspace{-3pt}
It measures the difference between model confidence and actual outcome, and $\textit{MCE}(f) \in [0,2]$.
For binary classification, $\textit{MCE}$ is twice the $\textit{ECE}$.
We shed light on upper and lower bounds of calibration of the strong model.

\begin{theorem}[Proved in~\cref{proof:calibration}] \label{theorem:calibration}
% Consider $\dist$ as the KL divergence loss.
Let $\text{MCE}(\cdot)$ be the marginal calibration error in~\cref{def:cal:mce}.
Then there holds
\begin{multline}
    \textit{MCE}(F_w) - 2 \cdot \sqrt{1-\exp{\left(-\dist(F_w,F_{sw})\right)}} \\ \le \textit{MCE}(F_{sw}) \le \\ \textit{MCE}(F_w) + 2 \cdot \sqrt{1-\exp{\left(-\dist(F_w,F_{sw})\right)}}.
\end{multline}
\end{theorem}


\cref{theorem:calibration} demonstrates that the calibration error of $F_{sw}$ is influenced by two key factors: (1) the calibration error of $F_w$, and (2) the teacher-student disagreement, as characterized by the optimization result in~\cref{eqn:fsw-population-minimizer}.
% In our experiments on large language models
% the practical use of this bound depends on $\textit{MCE}(F_w)$ (usually 0.08-0.24), and $\dist(F_w,F_{sw})$ (which is usually 0.0004-0.04, making $2 \cdot \sqrt{1-\exp{\left(-\dist(F_w,F_{sw})\right)}}$ falls into 0.04-0.4).
This theoretical result yields two insights.
First, to achieve a strong model with acceptable calibration, the weak teacher should also exhibit acceptable calibration. Otherwise, the strong model will inherit a non-trivial calibration error from the weak teacher as $\dist(F_w,F_{sw})$ goes to zero.
Second, closely imitating the weak supervisor minimizes $\dist(F_w,F_{sw})$, causing the calibration errors of the strong and weak models to converge. 
Taking them together, to ensure WTSG with reasonable calibration and prevent a poorly-calibrated $F_{sw}$, it is crucial to \textit{\textbf{avoid using a poorly-calibrated weak model with an overfitted strong model}}.
Additionally, since models with larger capacity may exhibit higher calibration errors~\citep{guo2017calibration}, \textit{\textbf{a potential trade-off exists between the weak model's calibration error and the teacher-student disagreement}}.
In other words, $\textit{MCE}(F_w)$ and $\sqrt{1-\exp{\left(-\dist(F_w,F_{sw})\right)}}$ may not be minimized simultaneously, posing a challenge in selecting the weak model and designing an effective optimization strategy to achieve better calibration in the strong model.


% Note that the upper bounds derived in~\cref{lemma:upper_lower_inf} and~\cref{theorem:calibration} do not guarantee that the strong model will outperform the weak model in terms of both generalization performance and calibration properties in WTSG.
% The intuition behind this is that if the strong model overfits the weak supervision, it will closely mimic the weak model's generalization and calibration behavior, potentially performing equally poorly or even worse.
% It will be validated in our experiments below.




% First, employing a better-calibrated weak model in WTSG leads to a better-calibrated strong model. 
% Given that recent studies have shown that aligned large language models exhibit non-trivial calibration errors~\citep{zhu2023calibration, tian2023just}, we emphasize the importance of carefully considering the calibration properties of weak models used to align strong models. Both the accuracy and calibration of the weak model are critical in WTSG.


% We demonstrate that a better-calibrated $F_{sw}$ contributes to an improved $\dist(F_w, F_{sw})$, thereby mitigating the issue of overfitting to weak supervision~\citep{burns2023weak}, where the strong model overly imitates the weak supervisor.

% \cref{theorem:calibration} offers a potential strategy to ensure that $\dist(F_w, F_{sw})$ does not asymptotically approach zero.
% Specifically, if $F_{sw}$ achieves a lower calibration error than $F_w$, $\dist(F_w, F_{sw})$ will remain positive, thereby potentially mitigating overfitting to weak supervision.
% This observation highlights a promising direction for enhancing weak-to-strong generalization: improving the calibration of the strong model. 
% Such an approach opens new avenues for algorithmic advancements in this area. 
% Moreover, this perspective sheds light on the overfitting mitigation effect of the auxiliary loss employed in~\citep{burns2023weak}, which strengthens the strong model’s confidence in its predictions.




% To further investigate the benefits of calibration, we decompose $\dist(F^\star, F_{sw})$ in the following equation:

% \begin{proposition}[Proved in~\cref{proof:general_equation_true}] \label{general_equation_true}
% % Let $\dist$ be the KL divergence loss.
% Given the models $F_{sw}, F_w, F^\star$ defined above. Then there holds
% \begin{multline} \label{eq:general_equation_true}
%     \dist(F^\star, F_{sw}) = \dist(F^\star, F_w) - \underbrace{\left \langle F^\star, \log{\frac{F_{sw}}{F_w}} \right \rangle_E}_{R}.
% \end{multline}
% % If $\dist$ is the output distribution divergence, the same result can be recovered by replacing $\left \langle \cdot, \cdot \right \rangle_E$ with $\left \langle \cdot, \cdot \right \rangle$.
% \end{proposition}


% Therefore, $\dist(F^\star, F_{sw}) \le \dist(F^\star, F_w)$ holds if and only if the remainder term $R \ge 0$.
% Furthermore, an increasing $R$ contributes to a smaller $\dist(F^\star, F_{sw})$, indicating an improved strong model.
% Intuitively, $R$ is more likely to increase if $\forall x \in \cX$ and $i \in \{ 1, \cdots, k \}$: 
% \begin{itemize}
%     \item when $[F^\star(x)]_i$ is small, $[F_{sw}(x)]_i \le [F_w(x)]_i$;
%     \item when $[F^\star(x)]_i$ is large, $[F_{sw}(x)]_i \ge [F_w(x)]_i$.
% \end{itemize}
% In other words, we expect the confidence of $F_{sw}$ aligns closely with the true outcome $F^\star$, suggesting a strong calibration property of $F_{sw}$.
% Additionally, if a promising calibration property is achieved and the remainder $R \ge 0$ is successfully satisfied, we prove that the performance improvement of the strong model relative to the weak model is bounded by $\cO \left( \sqrt{\dist(F_w, F_{sw})} \right)$:

% \begin{theorem}[Proved in~\cref{proof:theorem:residue}] \label{theorem:residue}
% % Let $\dist$ be the KL divergence loss.
% Given $F_{sw}, F_w, F^\star$ defined above. 
% Denote $R=\left \langle F^\star, \log{\frac{F_{sw}}{F_w}} \right \rangle_E$. 
% Then $R \le C'\sqrt{\dist(F_w, F_{sw})}$, where $C'$ is a positive constant.
% \end{theorem}

% % \begin{remark}
% %     We also prove that if $R \ge 0$ and $\dist(F_w, F_{sw}) \ge \sqrt{2}C'$, then $R \le \dist(F_w, F_{sw})$, which means that there holds $\dist(F^\star, F_{sw}) \ge \dist\left( F^\star, F_w \right) - \dist(F_w, F_{sw})$.
% % \end{remark}

% \cref{theorem:residue} and~\cref{general_equation_true} collectively states that
% \begin{align*}
%     \dist(F^\star, F_{sw}) \ge \dist\left( F^\star, F_w \right) - C'\sqrt{\dist(F_w, F_{sw})}.
% \end{align*}

% It provides an alternative perspective to~\cref{lemma:upper_lower_inf} with different proof techniques, demonstrating that the maximum potential gain of the strong model over the weak model is related to $\sqrt{\dist(F_w, F_{sw})}$.
% It reinforces that the key bottleneck for performance improvement over $F_w$ arises from the optimization objective's inherent nature.




\section{Experiments}
\label{sec:exp}
Following the settings in Section \ref{sec:existing}, we evaluate \textit{NovelSum}'s correlation with the fine-tuned model performance across 53 IT datasets and compare it with previous diversity metrics. Additionally, we conduct a correlation analysis using Qwen-2.5-7B \cite{yang2024qwen2} as the backbone model, alongside previous LLaMA-3-8B experiments, to further demonstrate the metric's effectiveness across different scenarios. Qwen is used for both instruction tuning and deriving semantic embeddings. Due to resource constraints, we run each strategy on Qwen for two rounds, resulting in 25 datasets. 

\subsection{Main Results}

\begin{table*}[!t]
    \centering
    \resizebox{\linewidth}{!}{
    \begin{tabular}{lcccccccccc}
    \toprule
    \multirow{3}*{\textbf{Diversity Metrics}} & \multicolumn{10}{c}{\textbf{Data Selection Strategies}} \\
    \cmidrule(lr){2-11}
    & \multirow{2}*{\textbf{K-means}} & \multirow{2}*{\vtop{\hbox{\textbf{K-Center}}\vspace{1mm}\hbox{\textbf{-Greedy}}}}  & \multirow{2}*{\textbf{QDIT}} & \multirow{2}*{\vtop{\hbox{\textbf{Repr}}\vspace{1mm}\hbox{\textbf{Filter}}}} & \multicolumn{5}{c}{\textbf{Random}} & \multirow{2}{*}{\textbf{Duplicate}} \\ 
    \cmidrule(lr){6-10}
    & & & & & \textbf{$\mathcal{X}^{all}$} & ShareGPT & WizardLM & Alpaca & Dolly &  \\
    \midrule
    \rowcolor{gray!15} \multicolumn{11}{c}{\textit{LLaMA-3-8B}} \\
    Facility Loc. $_{\times10^5}$ & \cellcolor{BLUE!40} 2.99 & \cellcolor{ORANGE!10} 2.73 & \cellcolor{BLUE!40} 2.99 & \cellcolor{BLUE!20} 2.86 & \cellcolor{BLUE!40} 2.99 & \cellcolor{BLUE!0} 2.83 & \cellcolor{BLUE!30} 2.88 & \cellcolor{BLUE!0} 2.83 & \cellcolor{ORANGE!20} 2.59 & \cellcolor{ORANGE!30} 2.52 \\    
    DistSum$_{cosine}$  & \cellcolor{BLUE!30} 0.648 & \cellcolor{BLUE!60} 0.746 & \cellcolor{BLUE!0} 0.629 & \cellcolor{BLUE!50} 0.703 & \cellcolor{BLUE!10} 0.634 & \cellcolor{BLUE!40} 0.656 & \cellcolor{ORANGE!30} 0.578 & \cellcolor{ORANGE!10} 0.605 & \cellcolor{ORANGE!20} 0.603 & \cellcolor{BLUE!10} 0.634 \\
    Vendi Score $_{\times10^7}$ & \cellcolor{BLUE!30} 1.70 & \cellcolor{BLUE!60} 2.53 & \cellcolor{BLUE!10} 1.59 & \cellcolor{BLUE!50} 2.23 & \cellcolor{BLUE!20} 1.61 & \cellcolor{BLUE!30} 1.70 & \cellcolor{ORANGE!10} 1.44 & \cellcolor{ORANGE!20} 1.32 & \cellcolor{ORANGE!10} 1.44 & \cellcolor{ORANGE!30} 0.05 \\
    \textbf{NovelSum (Ours)} & \cellcolor{BLUE!60} 0.693 & \cellcolor{BLUE!50} 0.687 & \cellcolor{BLUE!30} 0.673 & \cellcolor{BLUE!20} 0.671 & \cellcolor{BLUE!40} 0.675 & \cellcolor{BLUE!10} 0.628 & \cellcolor{BLUE!0} 0.591 & \cellcolor{ORANGE!10} 0.572 & \cellcolor{ORANGE!20} 0.50 & \cellcolor{ORANGE!30} 0.461 \\
    \midrule    
    \textbf{Model Performance} & \cellcolor{BLUE!60}1.32 & \cellcolor{BLUE!50}1.31 & \cellcolor{BLUE!40}1.25 & \cellcolor{BLUE!30}1.05 & \cellcolor{BLUE!20}1.20 & \cellcolor{BLUE!10}0.83 & \cellcolor{BLUE!0}0.72 & \cellcolor{ORANGE!10}0.07 & \cellcolor{ORANGE!20}-0.14 & \cellcolor{ORANGE!30}-1.35 \\
    \midrule
    \midrule
    \rowcolor{gray!15} \multicolumn{11}{c}{\textit{Qwen-2.5-7B}} \\
    Facility Loc. $_{\times10^5}$ & \cellcolor{BLUE!40} 3.54 & \cellcolor{ORANGE!30} 3.42 & \cellcolor{BLUE!40} 3.54 & \cellcolor{ORANGE!20} 3.46 & \cellcolor{BLUE!40} 3.54 & \cellcolor{BLUE!30} 3.51 & \cellcolor{BLUE!10} 3.50 & \cellcolor{BLUE!10} 3.50 & \cellcolor{ORANGE!20} 3.46 & \cellcolor{BLUE!0} 3.48 \\ 
    DistSum$_{cosine}$ & \cellcolor{BLUE!30} 0.260 & \cellcolor{BLUE!60} 0.440 & \cellcolor{BLUE!0} 0.223 & \cellcolor{BLUE!50} 0.421 & \cellcolor{BLUE!10} 0.230 & \cellcolor{BLUE!40} 0.285 & \cellcolor{ORANGE!20} 0.211 & \cellcolor{ORANGE!30} 0.189 & \cellcolor{ORANGE!10} 0.221 & \cellcolor{BLUE!20} 0.243 \\
    Vendi Score $_{\times10^6}$ & \cellcolor{ORANGE!10} 1.60 & \cellcolor{BLUE!40} 3.09 & \cellcolor{BLUE!10} 2.60 & \cellcolor{BLUE!60} 7.15 & \cellcolor{ORANGE!20} 1.41 & \cellcolor{BLUE!50} 3.36 & \cellcolor{BLUE!20} 2.65 & \cellcolor{BLUE!0} 1.89 & \cellcolor{BLUE!30} 3.04 & \cellcolor{ORANGE!30} 0.20 \\
    \textbf{NovelSum (Ours)}  & \cellcolor{BLUE!40} 0.440 & \cellcolor{BLUE!60} 0.505 & \cellcolor{BLUE!20} 0.403 & \cellcolor{BLUE!50} 0.495 & \cellcolor{BLUE!30} 0.408 & \cellcolor{BLUE!10} 0.392 & \cellcolor{BLUE!0} 0.349 & \cellcolor{ORANGE!10} 0.336 & \cellcolor{ORANGE!20} 0.320 & \cellcolor{ORANGE!30} 0.309 \\
    \midrule
    \textbf{Model Performance} & \cellcolor{BLUE!30} 1.06 & \cellcolor{BLUE!60} 1.45 & \cellcolor{BLUE!40} 1.23 & \cellcolor{BLUE!50} 1.35 & \cellcolor{BLUE!20} 0.87 & \cellcolor{BLUE!10} 0.07 & \cellcolor{BLUE!0} -0.08 & \cellcolor{ORANGE!10} -0.38 & \cellcolor{ORANGE!30} -0.49 & \cellcolor{ORANGE!20} -0.43 \\
    \bottomrule
    \end{tabular}
    }
    \caption{Measuring the diversity of datasets selected by different strategies using \textit{NovelSum} and baseline metrics. Fine-tuned model performances (Eq. \ref{eq:perf}), based on MT-bench and AlpacaEval, are also included for cross reference. Darker \colorbox{BLUE!60}{blue} shades indicate higher values for each metric, while darker \colorbox{ORANGE!30}{orange} shades indicate lower values. While data selection strategies vary in performance on LLaMA-3-8B and Qwen-2.5-7B, \textit{NovelSum} consistently shows a stronger correlation with model performance than other metrics. More results are provided in Appendix \ref{app:results}.}
    \label{tbl:main}
    \vspace{-4mm}
\end{table*}


\begin{table}[t!]
\centering
\resizebox{\linewidth}{!}{
\begin{tabular}{lcccc}
\toprule
\multirow{2}*{\textbf{Diversity Metrics}} & \multicolumn{3}{c}{\textbf{LLaMA}} & \textbf{Qwen}\\
\cmidrule(lr){2-4} \cmidrule(lr){5-5} 
& \textbf{Pearson} & \textbf{Spearman} & \textbf{Avg.} & \textbf{Avg.} \\
\midrule
TTR & -0.38 & -0.16 & -0.27 & -0.30 \\
vocd-D & -0.43 & -0.17 & -0.30 & -0.31 \\
\midrule
Facility Loc. & 0.86 & 0.69 & 0.77 & 0.08 \\
Entropy & 0.93 & 0.80 & 0.86 & 0.63 \\
\midrule
LDD & 0.61 & 0.75 & 0.68 & 0.60 \\
KNN Distance & 0.59 & 0.80 & 0.70 & 0.67 \\
DistSum$_{cosine}$ & 0.85 & 0.67 & 0.76 & 0.51 \\
Vendi Score & 0.70 & 0.85 & 0.78 & 0.60 \\
DistSum$_{L2}$ & 0.86 & 0.76 & 0.81 & 0.51 \\
Cluster Inertia & 0.81 & 0.85 & 0.83 & 0.76 \\
Radius & 0.87 & 0.81 & 0.84 & 0.48 \\
\midrule
NovelSum & \textbf{0.98} & \textbf{0.95} & \textbf{0.97} & \textbf{0.90} \\
\bottomrule
\end{tabular}
}
\caption{Correlations between different metrics and model performance on LLaMA-3-8B and Qwen-2.5-7B.  “Avg.” denotes the average correlation (Eq. \ref{eq:cor}).}
\label{tbl:correlations}
\vspace{-2mm}
\end{table}

\paragraph{\textit{NovelSum} consistently achieves state-of-the-art correlation with model performance across various data selection strategies, backbone LLMs, and correlation measures.}
Table \ref{tbl:main} presents diversity measurement results on datasets constructed by mainstream data selection methods (based on $\mathcal{X}^{all}$), random selection from various sources, and duplicated samples (with only $m=100$ unique samples). 
Results from multiple runs are averaged for each strategy.
Although these strategies yield varying performance rankings across base models, \textit{NovelSum} consistently tracks changes in IT performance by accurately measuring dataset diversity. For instance, K-means achieves the best performance on LLaMA with the highest NovelSum score, while K-Center-Greedy excels on Qwen, also correlating with the highest NovelSum. Table \ref{tbl:correlations} shows the correlation coefficients between various metrics and model performance for both LLaMA and Qwen experiments, where \textit{NovelSum} achieves state-of-the-art correlation across different models and measures.

\paragraph{\textit{NovelSum} can provide valuable guidance for data engineering practices.}
As a reliable indicator of data diversity, \textit{NovelSum} can assess diversity at both the dataset and sample levels, directly guiding data selection and construction decisions. For example, Table \ref{tbl:main} shows that the combined data source $\mathcal{X}^{all}$ is a better choice for sampling diverse IT data than other sources. Moreover, \textit{NovelSum} can offer insights through comparative analyses, such as: (1) ShareGPT, which collects data from real internet users, exhibits greater diversity than Dolly, which relies on company employees, suggesting that IT samples from diverse sources enhance dataset diversity \cite{wang2024diversity-logD}; (2) In LLaMA experiments, random selection can outperform some mainstream strategies, aligning with prior work \cite{xia2024rethinking,diddee2024chasing}, highlighting gaps in current data selection methods for optimizing diversity.



\subsection{Ablation Study}


\textit{NovelSum} involves several flexible hyperparameters and variations. In our main experiments, \textit{NovelSum} uses cosine distance to compute $d(x_i, x_j)$ in Eq. \ref{eq:dad}. We set $\alpha = 1$, $\beta = 0.5$, and $K = 10$ nearest neighbors in Eq. \ref{eq:pws} and \ref{eq:dad}. Here, we conduct an ablation study to investigate the impact of these settings based on LLaMA-3-8B.

\begin{table}[ht!]
\centering
\resizebox{\linewidth}{!}{
\begin{tabular}{lccc}
\toprule
\textbf{Variants} & \textbf{Pearson} & \textbf{Spearman} & \textbf{Avg.} \\
\midrule
NovelSum & 0.98 & 0.96 & 0.97 \\
\midrule
\hspace{0.10cm} - Use $L2$ distance & 0.97 & 0.83 & 0.90\textsubscript{↓ 0.08} \\
\hspace{0.10cm} - $K=20$ & 0.98 & 0.96 & 0.97\textsubscript{↓ 0.00} \\
\hspace{0.10cm} - $\alpha=0$ (w/o proximity) & 0.79 & 0.31 & 0.55\textsubscript{↓ 0.42} \\
\hspace{0.10cm} - $\alpha=2$ & 0.73 & 0.88 & 0.81\textsubscript{↓ 0.16} \\
\hspace{0.10cm} - $\beta=0$ (w/o density) & 0.92 & 0.89 & 0.91\textsubscript{↓ 0.07} \\
\hspace{0.10cm} - $\beta=1$ & 0.90 & 0.62 & 0.76\textsubscript{↓ 0.21} \\
\bottomrule
\end{tabular}
}
\caption{Ablation Study for \textit{NovelSum}.}
\label{tbl:ablation}
\vspace{-2mm}
\end{table}

In Table \ref{tbl:ablation}, $\alpha=0$ removes the proximity weights, and $\beta=0$ eliminates the density multiplier. We observe that both $\alpha=0$ and $\beta=0$ significantly weaken the correlation, validating the benefits of the proximity-weighted sum and density-aware distance. Additionally, improper values for $\alpha$ and $\beta$ greatly reduce the metric's reliability, highlighting that \textit{NovelSum} strikes a delicate balance between distances and distribution. Replacing cosine distance with Euclidean distance and using more neighbors for density approximation have minimal impact, particularly on Pearson's correlation, demonstrating \textit{NovelSum}'s robustness to different distance measures.







\section{Results Beyond Squared Loss} \label{subsec:recover_quantify}

In regression problems under some assumptions, \citet{charikar2024quantifying} proves that the strong model’s error is smaller than the weak model’s, with the gap at least the strong model’s error on the weak labels.
This observation naturally raises the following question:
\textit{Can their proof be extended from squared loss to output distribution divergence?}
In this section, we show how to theoretically bridge the gap between squared loss and KL divergence within the overall proof framework established in~\citet{charikar2024quantifying}.
% Interestingly, our analysis reveals that employing KL divergence as the loss function can potentially lead to a reduction in the reverse KL divergence, and vice versa.
To begin with, we restate an assumption used in previous study.
\begin{assumption}[Convexity Assumption~\citep{charikar2024quantifying}] \label{convex_set}
The strong model learns fine-tuning tasks from a function class $\cF_{s}$, which is a convex set. 
\end{assumption}
\vspace{-5pt}
It requires that, for any $f, g \in \cF_s$, and for any $\lambda \in [0,1]$, there exists $h \in \cF$ such that for all $z \in \R^{d_s}$, $h(z) = \lambda f(z) + (1-\lambda) g(z)$. 
% And we do not assume anything about either $f^\star$ or $f_w$, which need not belong to $\cF$. 
To satisfy the convex set assumption, $\cF_s$ can be the class of all linear functions.
In these cases, $\cF_s$ is a convex set. 
Note that it is validated by practice: a popular way to fine-tune a pre-trained model on task-specific data is by tuning the weights of only the last linear layer of the model~\citep{howard2018universal,kumar2022fine}.

\subsection{Upper Bound (Realizability)} \label{subsub:realize}

Firstly, we consider the case where $\exists f_s \in \cF_s$ such that $F_s = F^\star$ (also called ``Realizability''~\citep{charikar2024quantifying}).
It means we can find a $f_s$ such that $f_s \circ h_s = f^\star \circ h^\star$.
This assumption implicitly indicates the strong power of pre-training. 
It requires that the representation $h_s$ has learned extremely enough information during pre-training, which is reasonable in modern large language models pre-trained on very large corpus~\citep{touvron2023llama,achiam2023gpt}.
The scale and diversity of the corpus ensure that the model is exposed to a broad spectrum of lexical, syntactic, and semantic structures, enhancing its ability to generalize effectively across varied language tasks.

We state our result in the realizable setting, which corresponds to Theorem 1 in~\citet{charikar2024quantifying}.
\begin{theorem}[Proved in \cref{proof_theorem_1-main}]
\label{thm:realizable-main}

Given $F^\star$, $F_w$ and $F_{sw}$ defined above.
Consider $\cF_s$ that satisfies Assumption~\ref{convex_set}. 
Consider WTSG using reverse KL divergence loss:
\begin{align*}
    f_{sw} = \argmin_{f \in \cF_{s}}\; \dist(f \circ h_s, f_w \circ h_w).
\end{align*}
Assume that $\exists f_s \in \cF_s$ such that $F_s = F^\star$.
Then
\begin{align} \label{eqn:realizable-main}
    \dist(F^\star, F_{sw}) \le \dist(F^\star, F_w) - \dist(F_{sw}, F_w).
\end{align}
\end{theorem}

\begin{remark}
    The corresponding theorem and proof in the case of forward KL divergence loss is provided in~\cref{thm:realizable} from~\cref{proof_theorem_1}, under an additional assumption.
\end{remark}

In contrast to the symmetric squared loss studied in prior work~\citep{charikar2024quantifying}, the emergence of the reverse KL divergence is inherently tied to the asymmetric properties of the KL divergence.
Although extending previous work to both forward and reverse KL divergences presents significant technical challenges, our results demonstrate the theoretical guarantees of WTSG in these settings.
In Inequality~\eqref{eqn:realizable-main}, the left-hand side represents the error of the weakly-supervised strong model on the true data. 
On the right-hand side, the first term denotes the true error of the weak model, while the second term captures the disagreement between the strong and weak models, which also serves as the minimization objective in WTSG. 
This inequality indicates that the weakly-supervised strong model improves upon the weak model by at least the magnitude of their disagreement, $\dist(F_{sw}, F_w)$.
To reduce the error of $F_{sw}$, \cref{thm:realizable-main} aligns with~\cref{lemma:upper_lower_inf}, highlighting the importance of selecting an effective weak model and the inherent limitations of the optimization objective in WTSG.


% Notice that the error of weak model and strong model in~\cref{thm:realizable-main} is the reverse version, which fundamentally stems from the asymmetric properties of KL divergence.
% Despite this subtle difference, our empirical results in the experiments demonstrate consistent trends between forward and reverse KL divergence.











\subsection{Upper Bound (Non-Realizability)}

Now we relax the ``realizability'' condition and draw $n$ i.i.d. samples to perform WTSG.
We provide the result in the ``unrealizable'' setting, where the condition $F_s = F^\star$ may not be satisfied for any $f_s \in \cF_s$.
It corresponds to Theorem 2 in~\citet{charikar2024quantifying}.

\begin{theorem}[Proved in~\cref{proof_non-realizable-main}] \label{thm:non-realizable-finite-samples-main}
Given $F^\star$, $F_w$ and $F_{sw}$ defined above.
Consider $\cF_s$ that satisfies~\cref{convex_set}.
Consider weak-to-strong generalization using reverse KL:
\begin{align*}
    & f_{sw} = \argmin_{f \in \cF_{s}}\; \dist(f \circ h_s, f_w \circ h_w),
    \\ & \hat{f}_{sw} = \argmin_{f \in \cF_{s}}\; \hat{d}_{\cP}(f \circ h_s, f_w \circ h_w),
\end{align*}
Denote $\dist(F^\star, F_s) = \eps$. 
With probability at least $1-\delta$ over the draw of $n$ i.i.d. samples, there holds
\begin{multline} 
\dist(F^\star, \hat{F}_{sw}) \le \dist(F^\star, F_w) - \dist(\hat{F}_{sw}, F_w) + \\ \cO(\sqrt{\eps}) +  \cO\left(\sqrt{\frac{\cC_{\cF_s}}{n}}\right) + \cO\left(\sqrt{\frac{\log(1/\delta)}{n}}\right),
\end{multline}
where $\cC_{\cF_s}$ is a constant capturing the complexity of the function class $\cF_s$, and the asymptotic notation is with respect to $\eps \to 0, n \to \infty$.
\end{theorem}

\begin{remark}
    The extension to forward KL divergence loss is provided in~\cref{thm:non-realizable-finite-samples} from~\cref{proof_non-realizable}, under an additional assumption.
\end{remark}


Compared to Inequality~\eqref{eqn:realizable-main}, this bound introduces two another error terms: the first term of $\cO(\sqrt{\eps})$ arises due to the non-realizability assumption, and diminishes as the strong ceiling model $F_s$ becomes more expressive.
The remaining two error terms arise from the strong model $\hat{F}_{sw}$ being trained on a finite weakly-labeled sample. They also asymptotically approach zero as the sample size increases.







\begin{figure*}[t]
  \centering
  \subfigure[Realizable (pre-training).]{
    \includegraphics[width=0.31\textwidth]{images/exp_2/realizable-pretrain.pdf}
  }
  \label{fig3:a}
  \subfigure[Non-realizable (pre-training).]{
    \includegraphics[width=0.31\textwidth]{images/exp_2/unrealizable-pretrain.pdf}
  }
  \subfigure[Non-realizable (perturbation).]{
    \includegraphics[width=0.31\textwidth]{images/exp_2/perturb.pdf}
  }
  \subfigure[Realizable (pre-training).]{
    \includegraphics[width=0.31\textwidth]{images/exp_2/realizable-pretrain_forward.pdf}
  }
  \subfigure[Non-realizable (pre-training).]{
    \includegraphics[width=0.31\textwidth]{images/exp_2/unrealizable-pretrain_forward.pdf}
  }
  \subfigure[Non-realizable (perturbation).]{
    \includegraphics[width=0.31\textwidth]{images/exp_2/perturb_forward.pdf}
  }
  \vspace{-5pt}
  \caption{Experiments on synthetic data using reverse KL divergence loss (\textbf{a-c}) and forward KL divergence loss (\textbf{d-f}). 
  Each point corresponds to a task and the gray dotted line represents $y=x$. 
  $h^{\star}$ is a 16-layer MLP. 
  (\textbf{a,d}) Realizable (pre-training): $h_s=h^\star$, and $h_w$ is a 2-layer MLP obtained by pre-training. (\textbf{b,e}) Non-realizable (pre-training): $h_s$ is an 8-layer MLP, and $h_w$ is a 2-layer MLP. Both $h_s$ and $h_w$ are obtained by pre-training. (\textbf{c,f}) Non-realizable (perturbation): Both $h_s$ and $h_w$ are obtained by directly perturbing the weights in $h^{\star}$:  $h_s=h^{\star}+ \mathcal{N}\left(0,0.01\right)$, and $h_w=h^{\star}+\mathcal{N}\left(0,9\right)$.}
  \label{syn_result:reverse}
  \vspace{-10pt}
\end{figure*}




\subsection{Synthetic Experiments} \label{section:syn_exp}
In this section, we conduct experiments on synthetically generated data to validate the theoretical results in~\cref{subsec:recover_quantify}.
While drawing inspiration from the theoretical framework of~\citet{charikar2024quantifying}, we extend their synthetic experiments by replacing the squared loss used in their work with the output distribution divergence defined in~\cref{def:kl_dist_emp}.


\subsubsection{Experimental Setting}

In our setup, The data distribution $\mathcal{P}$ is chosen as $\mathcal{N}(0, \sigma^2 I)$, with $\sigma=500$ to ensure the data is well-dispersed.
The ground truth representation $h^\star:\R^8 \to \R^{16}$ is implemented as a randomly initialized 16-layer multi-layer perceptron (MLP) with ReLU activations.
Let the weak model and strong model representations $h_w, h_s: \R^8 \to \R^{16}$ be 2-layer and 8-layer MLP with ReLU activations, respectively.
Given $h_w$ and $h_s$ frozen, both the strong and weak models learn from the fine-tuning task class $\cF_s$, which consists of linear functions mapping $\R^{16}\to\R$.
This makes $\cF_s$ a convex set. 

For the ``realizable'' setting, we set $h_s = h^\star$.
For the ``unrealizable'' setting, we adopt the approach of~\citet{charikar2024quantifying} and investigate two methods for generating weak and strong representations: (1) \textbf{Pre-training}: 20 models $f_1^\star,\dots,f_{20}^\star: \R^8 \to \R^{16}$ are randomly sampled as fine-tuning tasks. 2000 data points are independently generated from $\cP$ for these tasks. Accordingly, $h_w$ and $h_s$ are obtained by minimizing the average output distribution divergence between ground truth label ($f_t^\star \circ h^\star$) and model prediction ($f_t^\star \circ h_w$ and $f_t^\star \circ h_s$) over the 20 tasks.
(2) \textbf{Perturbations}: As an alternative, we directly perturb the parameters of $h^\star$ to obtain the weak and strong representations. 
Specifically, we add independent Gaussian noises $\mathcal{N}(0, \sigma_s^2)$ and $\mathcal{N}(0, \sigma_w^2)$ to every parameter in $h^\star$ to generate $h_s$ and $h_w$, respectively.
% We add independent Gaussian noise $\mathcal{N}(0, \sigma_s^2)$ to every parameter in $h^\star$ to generate $h_s$. Similarly, we perturb $h^\star$ with $\mathcal{N}(0, \sigma_w^2)$ to generate $h_w$. 
To ensure the strong representation $h_s$ is closer to $h^\star$ than $h_w$~\citep{charikar2024quantifying}, we set $\sigma_s=0.1$ and $\sigma_w=3$.



\noindent \textbf{Weak Model Finetuning.} 
We freeze the weak model representation $h_w$ and train the weak models on new fine-tuning tasks.
We randomly sample 100 new fine-tuning tasks $f_{21}^\star,\dots,f_{120}^\star: \R^8 \to \R^{16}$, and independently generate another 2000 data points from $\mathcal{P}$. 
For each task $t \in \{ 21, \cdots, 120 \}$, the corresponding weak model is obtained by minimizing the output distribution divergence between ground truth label and weak model prediction.

\noindent \textbf{Weak-to-Strong Supervision.} 
Using the trained weak models, we generate weakly labeled data to supervise the strong model.
Specifically, we first independently generate another 2000 data points from $\mathcal{P}$.
Then for each task $t \in \{ 21, \cdots, 120 \}$, the strong model is obtained by minimizing the output distribution divergence between weak model supervision and strong model prediction.
At this stage, the weak-to-strong training procedure is complete. The detailed introduction of above is in~\cref{appendix:syn_train}.

\noindent \textbf{Evaluation.}
We independently draw an additional 2000 samples from $\cP$ to construct the test set.
They are used to estimate $\dist(F^\star, F_{sw})$, $\dist(F^\star, F_w)$ and $\dist(F_{sw}, F_w)$ for each task $t \in \{ 21, \cdots, 120 \}$.
We estimate these quantities using their empirical counterparts: $\disthat(F^\star, F_w)$, $\disthat(F^\star, F_{sw})$, and $\disthat(F_{sw}, F_w)$.
To validate~\cref{thm:realizable-main}-\ref{thm:non-realizable-finite-samples-main} and visualize the trend clearly, we plot $\disthat(F^\star, F_w)-\disthat(F^\star, F_{sw})$ on the $x$-axis versus $\disthat(F_{sw}, F_w)$ on the $y$-axis. The results are presented in~\cref{syn_result:reverse}(a)-(c).
We also examine forward KL divergence loss. 
To validate~\cref{thm:realizable}-\ref{thm:non-realizable-finite-samples}, which extend~\cref{thm:realizable-main}-\ref{thm:non-realizable-finite-samples-main} to the case of using forward KL divergence loss in WTSG,
we plot $\disthat(F_w, F^\star)-\disthat(F_{sw}, F^\star)$ on the $x$-axis versus $\disthat(F_w, F_{sw})$ on the $y$-axis. 
The results are presented in~\cref{syn_result:reverse}(d)-(f).


\subsubsection{Results and Analysis}

\noindent \textbf{Reverse KL divergence loss.}
Similar to previous results of squared loss~\citep{charikar2024quantifying}, the points in our experiments also cluster around the line $y=x$.
This suggests that 
$\disthat(F^\star, F_w)-\disthat(F^\star, F_{sw}) \approx \disthat(F_{sw}, F_w)$.
It is consistent with our theoretical framework, suggesting that the improvement over the weak teacher can be quantified by the disagreement between strong and weak models.

\noindent \textbf{Forward KL divergence loss.}
The observed trend closely mirrors that of reverse KL. 
The dots are generally around the line $y=x$.
It suggest that the relationship 
$\disthat(F_w, F^\star)-\disthat(F_{sw}, F^\star) \approx \disthat(F_w, F_{sw})$
may also hold, indicating a similar theoretical guarantee for forward KL in WTSG.











% \begin{algorithm}[ht!]
\caption{\textit{NovelSelect}}
\label{alg:novelselect}
\begin{algorithmic}[1]
\State \textbf{Input:} Data pool $\mathcal{X}^{all}$, data budget $n$
\State Initialize an empty dataset, $\mathcal{X} \gets \emptyset$
\While{$|\mathcal{X}| < n$}
    \State $x^{new} \gets \arg\max_{x \in \mathcal{X}^{all}} v(x)$
    \State $\mathcal{X} \gets \mathcal{X} \cup \{x^{new}\}$
    \State $\mathcal{X}^{all} \gets \mathcal{X}^{all} \setminus \{x^{new}\}$
\EndWhile
\State \textbf{return} $\mathcal{X}$
\end{algorithmic}
\end{algorithm}


\section{Conclusion}
In this work, we propose a simple yet effective approach, called SMILE, for graph few-shot learning with fewer tasks. Specifically, we introduce a novel dual-level mixup strategy, including within-task and across-task mixup, for enriching the diversity of nodes within each task and the diversity of tasks. Also, we incorporate the degree-based prior information to learn expressive node embeddings. Theoretically, we prove that SMILE effectively enhances the model's generalization performance. Empirically, we conduct extensive experiments on multiple benchmarks and the results suggest that SMILE significantly outperforms other baselines, including both in-domain and cross-domain few-shot settings.

\section*{Broader Impact and Ethics Statement}

This work on weak-to-strong generalization aims to improve the alignment of superhuman models with human values. 
While our theoretical and empirical insights highlight the potential of this approach, we acknowledge the risks of propagating biases or errors from the weak model to the strong model. 
To address these concerns, we emphasize the importance of ensuring the weak model's generalization and calibration, as well as carefully balancing the strong model's optimization to avoid over-reliance on weak supervision. 
We encourage rigorous testing, transparency, and ongoing monitoring in real-world applications to ensure the safe and ethical deployment of such systems. 
Our work contributes to the broader effort of aligning advanced AI with human values, but its implementation must prioritize fairness, accountability, and safety.



\section*{Acknowledgements}
We are deeply grateful to Shaojie Li, Chenyu Zheng, Gengze Xu for their  constructive suggestions.







% Acknowledgements should only appear in the accepted version.
% \section*{Acknowledgements}



\bibliography{example_paper}
\bibliographystyle{icml2024}

\newpage
\onecolumn
\appendix




%%%%%%%%%%%%%%%%%%%%%%%%%%%%%%%%%%%%%%%%%%%%%%%%%%%%%%%%%%%%%%%%%%%%%%%%%%%%%%%
%%%%%%%%%%%%%%%%%%%%%%%%%%%%%%%%%%%%%%%%%%%%%%%%%%%%%%%%%%%%%%%%%%%%%%%%%%%%%%%
% APPENDIX
%%%%%%%%%%%%%%%%%%%%%%%%%%%%%%%%%%%%%%%%%%%%%%%%%%%%%%%%%%%%%%%%%%%%%%%%%%%%%%%
%%%%%%%%%%%%%%%%%%%%%%%%%%%%%%%%%%%%%%%%%%%%%%%%%%%%%%%%%%%%%%%%%%%%%%%%%%%%%%%



\clearpage
\tableofcontents
\newpage

{\LARGE \centering \textbf{Appendix} \par}
\subsection{Lloyd-Max Algorithm}
\label{subsec:Lloyd-Max}
For a given quantization bitwidth $B$ and an operand $\bm{X}$, the Lloyd-Max algorithm finds $2^B$ quantization levels $\{\hat{x}_i\}_{i=1}^{2^B}$ such that quantizing $\bm{X}$ by rounding each scalar in $\bm{X}$ to the nearest quantization level minimizes the quantization MSE. 

The algorithm starts with an initial guess of quantization levels and then iteratively computes quantization thresholds $\{\tau_i\}_{i=1}^{2^B-1}$ and updates quantization levels $\{\hat{x}_i\}_{i=1}^{2^B}$. Specifically, at iteration $n$, thresholds are set to the midpoints of the previous iteration's levels:
\begin{align*}
    \tau_i^{(n)}=\frac{\hat{x}_i^{(n-1)}+\hat{x}_{i+1}^{(n-1)}}2 \text{ for } i=1\ldots 2^B-1
\end{align*}
Subsequently, the quantization levels are re-computed as conditional means of the data regions defined by the new thresholds:
\begin{align*}
    \hat{x}_i^{(n)}=\mathbb{E}\left[ \bm{X} \big| \bm{X}\in [\tau_{i-1}^{(n)},\tau_i^{(n)}] \right] \text{ for } i=1\ldots 2^B
\end{align*}
where to satisfy boundary conditions we have $\tau_0=-\infty$ and $\tau_{2^B}=\infty$. The algorithm iterates the above steps until convergence.

Figure \ref{fig:lm_quant} compares the quantization levels of a $7$-bit floating point (E3M3) quantizer (left) to a $7$-bit Lloyd-Max quantizer (right) when quantizing a layer of weights from the GPT3-126M model at a per-tensor granularity. As shown, the Lloyd-Max quantizer achieves substantially lower quantization MSE. Further, Table \ref{tab:FP7_vs_LM7} shows the superior perplexity achieved by Lloyd-Max quantizers for bitwidths of $7$, $6$ and $5$. The difference between the quantizers is clear at 5 bits, where per-tensor FP quantization incurs a drastic and unacceptable increase in perplexity, while Lloyd-Max quantization incurs a much smaller increase. Nevertheless, we note that even the optimal Lloyd-Max quantizer incurs a notable ($\sim 1.5$) increase in perplexity due to the coarse granularity of quantization. 

\begin{figure}[h]
  \centering
  \includegraphics[width=0.7\linewidth]{sections/figures/LM7_FP7.pdf}
  \caption{\small Quantization levels and the corresponding quantization MSE of Floating Point (left) vs Lloyd-Max (right) Quantizers for a layer of weights in the GPT3-126M model.}
  \label{fig:lm_quant}
\end{figure}

\begin{table}[h]\scriptsize
\begin{center}
\caption{\label{tab:FP7_vs_LM7} \small Comparing perplexity (lower is better) achieved by floating point quantizers and Lloyd-Max quantizers on a GPT3-126M model for the Wikitext-103 dataset.}
\begin{tabular}{c|cc|c}
\hline
 \multirow{2}{*}{\textbf{Bitwidth}} & \multicolumn{2}{|c|}{\textbf{Floating-Point Quantizer}} & \textbf{Lloyd-Max Quantizer} \\
 & Best Format & Wikitext-103 Perplexity & Wikitext-103 Perplexity \\
\hline
7 & E3M3 & 18.32 & 18.27 \\
6 & E3M2 & 19.07 & 18.51 \\
5 & E4M0 & 43.89 & 19.71 \\
\hline
\end{tabular}
\end{center}
\end{table}

\subsection{Proof of Local Optimality of LO-BCQ}
\label{subsec:lobcq_opt_proof}
For a given block $\bm{b}_j$, the quantization MSE during LO-BCQ can be empirically evaluated as $\frac{1}{L_b}\lVert \bm{b}_j- \bm{\hat{b}}_j\rVert^2_2$ where $\bm{\hat{b}}_j$ is computed from equation (\ref{eq:clustered_quantization_definition}) as $C_{f(\bm{b}_j)}(\bm{b}_j)$. Further, for a given block cluster $\mathcal{B}_i$, we compute the quantization MSE as $\frac{1}{|\mathcal{B}_{i}|}\sum_{\bm{b} \in \mathcal{B}_{i}} \frac{1}{L_b}\lVert \bm{b}- C_i^{(n)}(\bm{b})\rVert^2_2$. Therefore, at the end of iteration $n$, we evaluate the overall quantization MSE $J^{(n)}$ for a given operand $\bm{X}$ composed of $N_c$ block clusters as:
\begin{align*}
    \label{eq:mse_iter_n}
    J^{(n)} = \frac{1}{N_c} \sum_{i=1}^{N_c} \frac{1}{|\mathcal{B}_{i}^{(n)}|}\sum_{\bm{v} \in \mathcal{B}_{i}^{(n)}} \frac{1}{L_b}\lVert \bm{b}- B_i^{(n)}(\bm{b})\rVert^2_2
\end{align*}

At the end of iteration $n$, the codebooks are updated from $\mathcal{C}^{(n-1)}$ to $\mathcal{C}^{(n)}$. However, the mapping of a given vector $\bm{b}_j$ to quantizers $\mathcal{C}^{(n)}$ remains as  $f^{(n)}(\bm{b}_j)$. At the next iteration, during the vector clustering step, $f^{(n+1)}(\bm{b}_j)$ finds new mapping of $\bm{b}_j$ to updated codebooks $\mathcal{C}^{(n)}$ such that the quantization MSE over the candidate codebooks is minimized. Therefore, we obtain the following result for $\bm{b}_j$:
\begin{align*}
\frac{1}{L_b}\lVert \bm{b}_j - C_{f^{(n+1)}(\bm{b}_j)}^{(n)}(\bm{b}_j)\rVert^2_2 \le \frac{1}{L_b}\lVert \bm{b}_j - C_{f^{(n)}(\bm{b}_j)}^{(n)}(\bm{b}_j)\rVert^2_2
\end{align*}

That is, quantizing $\bm{b}_j$ at the end of the block clustering step of iteration $n+1$ results in lower quantization MSE compared to quantizing at the end of iteration $n$. Since this is true for all $\bm{b} \in \bm{X}$, we assert the following:
\begin{equation}
\begin{split}
\label{eq:mse_ineq_1}
    \tilde{J}^{(n+1)} &= \frac{1}{N_c} \sum_{i=1}^{N_c} \frac{1}{|\mathcal{B}_{i}^{(n+1)}|}\sum_{\bm{b} \in \mathcal{B}_{i}^{(n+1)}} \frac{1}{L_b}\lVert \bm{b} - C_i^{(n)}(b)\rVert^2_2 \le J^{(n)}
\end{split}
\end{equation}
where $\tilde{J}^{(n+1)}$ is the the quantization MSE after the vector clustering step at iteration $n+1$.

Next, during the codebook update step (\ref{eq:quantizers_update}) at iteration $n+1$, the per-cluster codebooks $\mathcal{C}^{(n)}$ are updated to $\mathcal{C}^{(n+1)}$ by invoking the Lloyd-Max algorithm \citep{Lloyd}. We know that for any given value distribution, the Lloyd-Max algorithm minimizes the quantization MSE. Therefore, for a given vector cluster $\mathcal{B}_i$ we obtain the following result:

\begin{equation}
    \frac{1}{|\mathcal{B}_{i}^{(n+1)}|}\sum_{\bm{b} \in \mathcal{B}_{i}^{(n+1)}} \frac{1}{L_b}\lVert \bm{b}- C_i^{(n+1)}(\bm{b})\rVert^2_2 \le \frac{1}{|\mathcal{B}_{i}^{(n+1)}|}\sum_{\bm{b} \in \mathcal{B}_{i}^{(n+1)}} \frac{1}{L_b}\lVert \bm{b}- C_i^{(n)}(\bm{b})\rVert^2_2
\end{equation}

The above equation states that quantizing the given block cluster $\mathcal{B}_i$ after updating the associated codebook from $C_i^{(n)}$ to $C_i^{(n+1)}$ results in lower quantization MSE. Since this is true for all the block clusters, we derive the following result: 
\begin{equation}
\begin{split}
\label{eq:mse_ineq_2}
     J^{(n+1)} &= \frac{1}{N_c} \sum_{i=1}^{N_c} \frac{1}{|\mathcal{B}_{i}^{(n+1)}|}\sum_{\bm{b} \in \mathcal{B}_{i}^{(n+1)}} \frac{1}{L_b}\lVert \bm{b}- C_i^{(n+1)}(\bm{b})\rVert^2_2  \le \tilde{J}^{(n+1)}   
\end{split}
\end{equation}

Following (\ref{eq:mse_ineq_1}) and (\ref{eq:mse_ineq_2}), we find that the quantization MSE is non-increasing for each iteration, that is, $J^{(1)} \ge J^{(2)} \ge J^{(3)} \ge \ldots \ge J^{(M)}$ where $M$ is the maximum number of iterations. 
%Therefore, we can say that if the algorithm converges, then it must be that it has converged to a local minimum. 
\hfill $\blacksquare$


\begin{figure}
    \begin{center}
    \includegraphics[width=0.5\textwidth]{sections//figures/mse_vs_iter.pdf}
    \end{center}
    \caption{\small NMSE vs iterations during LO-BCQ compared to other block quantization proposals}
    \label{fig:nmse_vs_iter}
\end{figure}

Figure \ref{fig:nmse_vs_iter} shows the empirical convergence of LO-BCQ across several block lengths and number of codebooks. Also, the MSE achieved by LO-BCQ is compared to baselines such as MXFP and VSQ. As shown, LO-BCQ converges to a lower MSE than the baselines. Further, we achieve better convergence for larger number of codebooks ($N_c$) and for a smaller block length ($L_b$), both of which increase the bitwidth of BCQ (see Eq \ref{eq:bitwidth_bcq}).


\subsection{Additional Accuracy Results}
%Table \ref{tab:lobcq_config} lists the various LOBCQ configurations and their corresponding bitwidths.
\begin{table}
\setlength{\tabcolsep}{4.75pt}
\begin{center}
\caption{\label{tab:lobcq_config} Various LO-BCQ configurations and their bitwidths.}
\begin{tabular}{|c||c|c|c|c||c|c||c|} 
\hline
 & \multicolumn{4}{|c||}{$L_b=8$} & \multicolumn{2}{|c||}{$L_b=4$} & $L_b=2$ \\
 \hline
 \backslashbox{$L_A$\kern-1em}{\kern-1em$N_c$} & 2 & 4 & 8 & 16 & 2 & 4 & 2 \\
 \hline
 64 & 4.25 & 4.375 & 4.5 & 4.625 & 4.375 & 4.625 & 4.625\\
 \hline
 32 & 4.375 & 4.5 & 4.625& 4.75 & 4.5 & 4.75 & 4.75 \\
 \hline
 16 & 4.625 & 4.75& 4.875 & 5 & 4.75 & 5 & 5 \\
 \hline
\end{tabular}
\end{center}
\end{table}

%\subsection{Perplexity achieved by various LO-BCQ configurations on Wikitext-103 dataset}

\begin{table} \centering
\begin{tabular}{|c||c|c|c|c||c|c||c|} 
\hline
 $L_b \rightarrow$& \multicolumn{4}{c||}{8} & \multicolumn{2}{c||}{4} & 2\\
 \hline
 \backslashbox{$L_A$\kern-1em}{\kern-1em$N_c$} & 2 & 4 & 8 & 16 & 2 & 4 & 2  \\
 %$N_c \rightarrow$ & 2 & 4 & 8 & 16 & 2 & 4 & 2 \\
 \hline
 \hline
 \multicolumn{8}{c}{GPT3-1.3B (FP32 PPL = 9.98)} \\ 
 \hline
 \hline
 64 & 10.40 & 10.23 & 10.17 & 10.15 &  10.28 & 10.18 & 10.19 \\
 \hline
 32 & 10.25 & 10.20 & 10.15 & 10.12 &  10.23 & 10.17 & 10.17 \\
 \hline
 16 & 10.22 & 10.16 & 10.10 & 10.09 &  10.21 & 10.14 & 10.16 \\
 \hline
  \hline
 \multicolumn{8}{c}{GPT3-8B (FP32 PPL = 7.38)} \\ 
 \hline
 \hline
 64 & 7.61 & 7.52 & 7.48 &  7.47 &  7.55 &  7.49 & 7.50 \\
 \hline
 32 & 7.52 & 7.50 & 7.46 &  7.45 &  7.52 &  7.48 & 7.48  \\
 \hline
 16 & 7.51 & 7.48 & 7.44 &  7.44 &  7.51 &  7.49 & 7.47  \\
 \hline
\end{tabular}
\caption{\label{tab:ppl_gpt3_abalation} Wikitext-103 perplexity across GPT3-1.3B and 8B models.}
\end{table}

\begin{table} \centering
\begin{tabular}{|c||c|c|c|c||} 
\hline
 $L_b \rightarrow$& \multicolumn{4}{c||}{8}\\
 \hline
 \backslashbox{$L_A$\kern-1em}{\kern-1em$N_c$} & 2 & 4 & 8 & 16 \\
 %$N_c \rightarrow$ & 2 & 4 & 8 & 16 & 2 & 4 & 2 \\
 \hline
 \hline
 \multicolumn{5}{|c|}{Llama2-7B (FP32 PPL = 5.06)} \\ 
 \hline
 \hline
 64 & 5.31 & 5.26 & 5.19 & 5.18  \\
 \hline
 32 & 5.23 & 5.25 & 5.18 & 5.15  \\
 \hline
 16 & 5.23 & 5.19 & 5.16 & 5.14  \\
 \hline
 \multicolumn{5}{|c|}{Nemotron4-15B (FP32 PPL = 5.87)} \\ 
 \hline
 \hline
 64  & 6.3 & 6.20 & 6.13 & 6.08  \\
 \hline
 32  & 6.24 & 6.12 & 6.07 & 6.03  \\
 \hline
 16  & 6.12 & 6.14 & 6.04 & 6.02  \\
 \hline
 \multicolumn{5}{|c|}{Nemotron4-340B (FP32 PPL = 3.48)} \\ 
 \hline
 \hline
 64 & 3.67 & 3.62 & 3.60 & 3.59 \\
 \hline
 32 & 3.63 & 3.61 & 3.59 & 3.56 \\
 \hline
 16 & 3.61 & 3.58 & 3.57 & 3.55 \\
 \hline
\end{tabular}
\caption{\label{tab:ppl_llama7B_nemo15B} Wikitext-103 perplexity compared to FP32 baseline in Llama2-7B and Nemotron4-15B, 340B models}
\end{table}

%\subsection{Perplexity achieved by various LO-BCQ configurations on MMLU dataset}


\begin{table} \centering
\begin{tabular}{|c||c|c|c|c||c|c|c|c|} 
\hline
 $L_b \rightarrow$& \multicolumn{4}{c||}{8} & \multicolumn{4}{c||}{8}\\
 \hline
 \backslashbox{$L_A$\kern-1em}{\kern-1em$N_c$} & 2 & 4 & 8 & 16 & 2 & 4 & 8 & 16  \\
 %$N_c \rightarrow$ & 2 & 4 & 8 & 16 & 2 & 4 & 2 \\
 \hline
 \hline
 \multicolumn{5}{|c|}{Llama2-7B (FP32 Accuracy = 45.8\%)} & \multicolumn{4}{|c|}{Llama2-70B (FP32 Accuracy = 69.12\%)} \\ 
 \hline
 \hline
 64 & 43.9 & 43.4 & 43.9 & 44.9 & 68.07 & 68.27 & 68.17 & 68.75 \\
 \hline
 32 & 44.5 & 43.8 & 44.9 & 44.5 & 68.37 & 68.51 & 68.35 & 68.27  \\
 \hline
 16 & 43.9 & 42.7 & 44.9 & 45 & 68.12 & 68.77 & 68.31 & 68.59  \\
 \hline
 \hline
 \multicolumn{5}{|c|}{GPT3-22B (FP32 Accuracy = 38.75\%)} & \multicolumn{4}{|c|}{Nemotron4-15B (FP32 Accuracy = 64.3\%)} \\ 
 \hline
 \hline
 64 & 36.71 & 38.85 & 38.13 & 38.92 & 63.17 & 62.36 & 63.72 & 64.09 \\
 \hline
 32 & 37.95 & 38.69 & 39.45 & 38.34 & 64.05 & 62.30 & 63.8 & 64.33  \\
 \hline
 16 & 38.88 & 38.80 & 38.31 & 38.92 & 63.22 & 63.51 & 63.93 & 64.43  \\
 \hline
\end{tabular}
\caption{\label{tab:mmlu_abalation} Accuracy on MMLU dataset across GPT3-22B, Llama2-7B, 70B and Nemotron4-15B models.}
\end{table}


%\subsection{Perplexity achieved by various LO-BCQ configurations on LM evaluation harness}

\begin{table} \centering
\begin{tabular}{|c||c|c|c|c||c|c|c|c|} 
\hline
 $L_b \rightarrow$& \multicolumn{4}{c||}{8} & \multicolumn{4}{c||}{8}\\
 \hline
 \backslashbox{$L_A$\kern-1em}{\kern-1em$N_c$} & 2 & 4 & 8 & 16 & 2 & 4 & 8 & 16  \\
 %$N_c \rightarrow$ & 2 & 4 & 8 & 16 & 2 & 4 & 2 \\
 \hline
 \hline
 \multicolumn{5}{|c|}{Race (FP32 Accuracy = 37.51\%)} & \multicolumn{4}{|c|}{Boolq (FP32 Accuracy = 64.62\%)} \\ 
 \hline
 \hline
 64 & 36.94 & 37.13 & 36.27 & 37.13 & 63.73 & 62.26 & 63.49 & 63.36 \\
 \hline
 32 & 37.03 & 36.36 & 36.08 & 37.03 & 62.54 & 63.51 & 63.49 & 63.55  \\
 \hline
 16 & 37.03 & 37.03 & 36.46 & 37.03 & 61.1 & 63.79 & 63.58 & 63.33  \\
 \hline
 \hline
 \multicolumn{5}{|c|}{Winogrande (FP32 Accuracy = 58.01\%)} & \multicolumn{4}{|c|}{Piqa (FP32 Accuracy = 74.21\%)} \\ 
 \hline
 \hline
 64 & 58.17 & 57.22 & 57.85 & 58.33 & 73.01 & 73.07 & 73.07 & 72.80 \\
 \hline
 32 & 59.12 & 58.09 & 57.85 & 58.41 & 73.01 & 73.94 & 72.74 & 73.18  \\
 \hline
 16 & 57.93 & 58.88 & 57.93 & 58.56 & 73.94 & 72.80 & 73.01 & 73.94  \\
 \hline
\end{tabular}
\caption{\label{tab:mmlu_abalation} Accuracy on LM evaluation harness tasks on GPT3-1.3B model.}
\end{table}

\begin{table} \centering
\begin{tabular}{|c||c|c|c|c||c|c|c|c|} 
\hline
 $L_b \rightarrow$& \multicolumn{4}{c||}{8} & \multicolumn{4}{c||}{8}\\
 \hline
 \backslashbox{$L_A$\kern-1em}{\kern-1em$N_c$} & 2 & 4 & 8 & 16 & 2 & 4 & 8 & 16  \\
 %$N_c \rightarrow$ & 2 & 4 & 8 & 16 & 2 & 4 & 2 \\
 \hline
 \hline
 \multicolumn{5}{|c|}{Race (FP32 Accuracy = 41.34\%)} & \multicolumn{4}{|c|}{Boolq (FP32 Accuracy = 68.32\%)} \\ 
 \hline
 \hline
 64 & 40.48 & 40.10 & 39.43 & 39.90 & 69.20 & 68.41 & 69.45 & 68.56 \\
 \hline
 32 & 39.52 & 39.52 & 40.77 & 39.62 & 68.32 & 67.43 & 68.17 & 69.30  \\
 \hline
 16 & 39.81 & 39.71 & 39.90 & 40.38 & 68.10 & 66.33 & 69.51 & 69.42  \\
 \hline
 \hline
 \multicolumn{5}{|c|}{Winogrande (FP32 Accuracy = 67.88\%)} & \multicolumn{4}{|c|}{Piqa (FP32 Accuracy = 78.78\%)} \\ 
 \hline
 \hline
 64 & 66.85 & 66.61 & 67.72 & 67.88 & 77.31 & 77.42 & 77.75 & 77.64 \\
 \hline
 32 & 67.25 & 67.72 & 67.72 & 67.00 & 77.31 & 77.04 & 77.80 & 77.37  \\
 \hline
 16 & 68.11 & 68.90 & 67.88 & 67.48 & 77.37 & 78.13 & 78.13 & 77.69  \\
 \hline
\end{tabular}
\caption{\label{tab:mmlu_abalation} Accuracy on LM evaluation harness tasks on GPT3-8B model.}
\end{table}

\begin{table} \centering
\begin{tabular}{|c||c|c|c|c||c|c|c|c|} 
\hline
 $L_b \rightarrow$& \multicolumn{4}{c||}{8} & \multicolumn{4}{c||}{8}\\
 \hline
 \backslashbox{$L_A$\kern-1em}{\kern-1em$N_c$} & 2 & 4 & 8 & 16 & 2 & 4 & 8 & 16  \\
 %$N_c \rightarrow$ & 2 & 4 & 8 & 16 & 2 & 4 & 2 \\
 \hline
 \hline
 \multicolumn{5}{|c|}{Race (FP32 Accuracy = 40.67\%)} & \multicolumn{4}{|c|}{Boolq (FP32 Accuracy = 76.54\%)} \\ 
 \hline
 \hline
 64 & 40.48 & 40.10 & 39.43 & 39.90 & 75.41 & 75.11 & 77.09 & 75.66 \\
 \hline
 32 & 39.52 & 39.52 & 40.77 & 39.62 & 76.02 & 76.02 & 75.96 & 75.35  \\
 \hline
 16 & 39.81 & 39.71 & 39.90 & 40.38 & 75.05 & 73.82 & 75.72 & 76.09  \\
 \hline
 \hline
 \multicolumn{5}{|c|}{Winogrande (FP32 Accuracy = 70.64\%)} & \multicolumn{4}{|c|}{Piqa (FP32 Accuracy = 79.16\%)} \\ 
 \hline
 \hline
 64 & 69.14 & 70.17 & 70.17 & 70.56 & 78.24 & 79.00 & 78.62 & 78.73 \\
 \hline
 32 & 70.96 & 69.69 & 71.27 & 69.30 & 78.56 & 79.49 & 79.16 & 78.89  \\
 \hline
 16 & 71.03 & 69.53 & 69.69 & 70.40 & 78.13 & 79.16 & 79.00 & 79.00  \\
 \hline
\end{tabular}
\caption{\label{tab:mmlu_abalation} Accuracy on LM evaluation harness tasks on GPT3-22B model.}
\end{table}

\begin{table} \centering
\begin{tabular}{|c||c|c|c|c||c|c|c|c|} 
\hline
 $L_b \rightarrow$& \multicolumn{4}{c||}{8} & \multicolumn{4}{c||}{8}\\
 \hline
 \backslashbox{$L_A$\kern-1em}{\kern-1em$N_c$} & 2 & 4 & 8 & 16 & 2 & 4 & 8 & 16  \\
 %$N_c \rightarrow$ & 2 & 4 & 8 & 16 & 2 & 4 & 2 \\
 \hline
 \hline
 \multicolumn{5}{|c|}{Race (FP32 Accuracy = 44.4\%)} & \multicolumn{4}{|c|}{Boolq (FP32 Accuracy = 79.29\%)} \\ 
 \hline
 \hline
 64 & 42.49 & 42.51 & 42.58 & 43.45 & 77.58 & 77.37 & 77.43 & 78.1 \\
 \hline
 32 & 43.35 & 42.49 & 43.64 & 43.73 & 77.86 & 75.32 & 77.28 & 77.86  \\
 \hline
 16 & 44.21 & 44.21 & 43.64 & 42.97 & 78.65 & 77 & 76.94 & 77.98  \\
 \hline
 \hline
 \multicolumn{5}{|c|}{Winogrande (FP32 Accuracy = 69.38\%)} & \multicolumn{4}{|c|}{Piqa (FP32 Accuracy = 78.07\%)} \\ 
 \hline
 \hline
 64 & 68.9 & 68.43 & 69.77 & 68.19 & 77.09 & 76.82 & 77.09 & 77.86 \\
 \hline
 32 & 69.38 & 68.51 & 68.82 & 68.90 & 78.07 & 76.71 & 78.07 & 77.86  \\
 \hline
 16 & 69.53 & 67.09 & 69.38 & 68.90 & 77.37 & 77.8 & 77.91 & 77.69  \\
 \hline
\end{tabular}
\caption{\label{tab:mmlu_abalation} Accuracy on LM evaluation harness tasks on Llama2-7B model.}
\end{table}

\begin{table} \centering
\begin{tabular}{|c||c|c|c|c||c|c|c|c|} 
\hline
 $L_b \rightarrow$& \multicolumn{4}{c||}{8} & \multicolumn{4}{c||}{8}\\
 \hline
 \backslashbox{$L_A$\kern-1em}{\kern-1em$N_c$} & 2 & 4 & 8 & 16 & 2 & 4 & 8 & 16  \\
 %$N_c \rightarrow$ & 2 & 4 & 8 & 16 & 2 & 4 & 2 \\
 \hline
 \hline
 \multicolumn{5}{|c|}{Race (FP32 Accuracy = 48.8\%)} & \multicolumn{4}{|c|}{Boolq (FP32 Accuracy = 85.23\%)} \\ 
 \hline
 \hline
 64 & 49.00 & 49.00 & 49.28 & 48.71 & 82.82 & 84.28 & 84.03 & 84.25 \\
 \hline
 32 & 49.57 & 48.52 & 48.33 & 49.28 & 83.85 & 84.46 & 84.31 & 84.93  \\
 \hline
 16 & 49.85 & 49.09 & 49.28 & 48.99 & 85.11 & 84.46 & 84.61 & 83.94  \\
 \hline
 \hline
 \multicolumn{5}{|c|}{Winogrande (FP32 Accuracy = 79.95\%)} & \multicolumn{4}{|c|}{Piqa (FP32 Accuracy = 81.56\%)} \\ 
 \hline
 \hline
 64 & 78.77 & 78.45 & 78.37 & 79.16 & 81.45 & 80.69 & 81.45 & 81.5 \\
 \hline
 32 & 78.45 & 79.01 & 78.69 & 80.66 & 81.56 & 80.58 & 81.18 & 81.34  \\
 \hline
 16 & 79.95 & 79.56 & 79.79 & 79.72 & 81.28 & 81.66 & 81.28 & 80.96  \\
 \hline
\end{tabular}
\caption{\label{tab:mmlu_abalation} Accuracy on LM evaluation harness tasks on Llama2-70B model.}
\end{table}

%\section{MSE Studies}
%\textcolor{red}{TODO}


\subsection{Number Formats and Quantization Method}
\label{subsec:numFormats_quantMethod}
\subsubsection{Integer Format}
An $n$-bit signed integer (INT) is typically represented with a 2s-complement format \citep{yao2022zeroquant,xiao2023smoothquant,dai2021vsq}, where the most significant bit denotes the sign.

\subsubsection{Floating Point Format}
An $n$-bit signed floating point (FP) number $x$ comprises of a 1-bit sign ($x_{\mathrm{sign}}$), $B_m$-bit mantissa ($x_{\mathrm{mant}}$) and $B_e$-bit exponent ($x_{\mathrm{exp}}$) such that $B_m+B_e=n-1$. The associated constant exponent bias ($E_{\mathrm{bias}}$) is computed as $(2^{{B_e}-1}-1)$. We denote this format as $E_{B_e}M_{B_m}$.  

\subsubsection{Quantization Scheme}
\label{subsec:quant_method}
A quantization scheme dictates how a given unquantized tensor is converted to its quantized representation. We consider FP formats for the purpose of illustration. Given an unquantized tensor $\bm{X}$ and an FP format $E_{B_e}M_{B_m}$, we first, we compute the quantization scale factor $s_X$ that maps the maximum absolute value of $\bm{X}$ to the maximum quantization level of the $E_{B_e}M_{B_m}$ format as follows:
\begin{align}
\label{eq:sf}
    s_X = \frac{\mathrm{max}(|\bm{X}|)}{\mathrm{max}(E_{B_e}M_{B_m})}
\end{align}
In the above equation, $|\cdot|$ denotes the absolute value function.

Next, we scale $\bm{X}$ by $s_X$ and quantize it to $\hat{\bm{X}}$ by rounding it to the nearest quantization level of $E_{B_e}M_{B_m}$ as:

\begin{align}
\label{eq:tensor_quant}
    \hat{\bm{X}} = \text{round-to-nearest}\left(\frac{\bm{X}}{s_X}, E_{B_e}M_{B_m}\right)
\end{align}

We perform dynamic max-scaled quantization \citep{wu2020integer}, where the scale factor $s$ for activations is dynamically computed during runtime.

\subsection{Vector Scaled Quantization}
\begin{wrapfigure}{r}{0.35\linewidth}
  \centering
  \includegraphics[width=\linewidth]{sections/figures/vsquant.jpg}
  \caption{\small Vectorwise decomposition for per-vector scaled quantization (VSQ \citep{dai2021vsq}).}
  \label{fig:vsquant}
\end{wrapfigure}
During VSQ \citep{dai2021vsq}, the operand tensors are decomposed into 1D vectors in a hardware friendly manner as shown in Figure \ref{fig:vsquant}. Since the decomposed tensors are used as operands in matrix multiplications during inference, it is beneficial to perform this decomposition along the reduction dimension of the multiplication. The vectorwise quantization is performed similar to tensorwise quantization described in Equations \ref{eq:sf} and \ref{eq:tensor_quant}, where a scale factor $s_v$ is required for each vector $\bm{v}$ that maps the maximum absolute value of that vector to the maximum quantization level. While smaller vector lengths can lead to larger accuracy gains, the associated memory and computational overheads due to the per-vector scale factors increases. To alleviate these overheads, VSQ \citep{dai2021vsq} proposed a second level quantization of the per-vector scale factors to unsigned integers, while MX \citep{rouhani2023shared} quantizes them to integer powers of 2 (denoted as $2^{INT}$).

\subsubsection{MX Format}
The MX format proposed in \citep{rouhani2023microscaling} introduces the concept of sub-block shifting. For every two scalar elements of $b$-bits each, there is a shared exponent bit. The value of this exponent bit is determined through an empirical analysis that targets minimizing quantization MSE. We note that the FP format $E_{1}M_{b}$ is strictly better than MX from an accuracy perspective since it allocates a dedicated exponent bit to each scalar as opposed to sharing it across two scalars. Therefore, we conservatively bound the accuracy of a $b+2$-bit signed MX format with that of a $E_{1}M_{b}$ format in our comparisons. For instance, we use E1M2 format as a proxy for MX4.

\begin{figure}
    \centering
    \includegraphics[width=1\linewidth]{sections//figures/BlockFormats.pdf}
    \caption{\small Comparing LO-BCQ to MX format.}
    \label{fig:block_formats}
\end{figure}

Figure \ref{fig:block_formats} compares our $4$-bit LO-BCQ block format to MX \citep{rouhani2023microscaling}. As shown, both LO-BCQ and MX decompose a given operand tensor into block arrays and each block array into blocks. Similar to MX, we find that per-block quantization ($L_b < L_A$) leads to better accuracy due to increased flexibility. While MX achieves this through per-block $1$-bit micro-scales, we associate a dedicated codebook to each block through a per-block codebook selector. Further, MX quantizes the per-block array scale-factor to E8M0 format without per-tensor scaling. In contrast during LO-BCQ, we find that per-tensor scaling combined with quantization of per-block array scale-factor to E4M3 format results in superior inference accuracy across models. 





\end{document}

