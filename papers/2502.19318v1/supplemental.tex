% ---------------------------------------------------------------------------
% Author guideline and sample document for EG publication using LaTeX2e input
% D.Fellner, v1.20, Jan 18, 2023

\documentclass{egpubl}
\usepackage[pdftex]{graphicx}
\usepackage{tcolorbox}
\usepackage{colortbl} % For coloring table cells
\usepackage{xcolor}   % For defining custom colors
\usepackage{eg2025}
% --- for  Annual CONFERENCE
%\ConferenceSubmission   % uncomment for Conference submission
\ConferencePaper        % uncomment for (final) Conference Paper
% \STAR                   % uncomment for STAR contribution
% \Tutorial               % uncomment for Tutorial contribution
% \ShortPresentation      % uncomment for (final) Short Conference Presentation
% \Areas                  % uncomment for Areas contribution
% \Education              % uncomment for Education contribution
% \Poster                 % uncomment for Poster contribution
% \DC                     % uncomment for Doctoral Consortium
%
% --- for  CGF Journal
% \JournalSubmission    % uncomment for submission to Computer Graphics Forum
% \JournalPaper         % uncomment for final version of Journal Paper
%
% --- for  CGF Journal: special issue
% \SpecialIssueSubmission    % uncomment for submission to , special issue
% \SpecialIssuePaper         % uncomment for final version of Computer Graphics Forum, special issue
%                          % EuroVis, SGP, Rendering, PG
% --- for  EG Workshop Proceedings
% \WsSubmission      % uncomment for submission to EG Workshop
% \WsPaper           % uncomment for final version of EG Workshop contribution
% \WsSubmissionJoint % for joint events, for example ICAT-EGVE
% \WsPaperJoint      % for joint events, for example ICAT-EGVE
% \WsPoster          % uncomment for Poster contribution
% \WsShortPaper      % uncomment for Short Paper contribution
%  \WsWiP     % uncomment for Work in Progress contribution
% \Expressive        % for SBIM, CAe, NPAR
% \DigitalHeritagePaper
% \PaperL2P          % for events EG only asks for License to Publish

% --- for EuroVis 
% for full papers use \SpecialIssuePaper
% \STAREurovis   % for EuroVis additional material 
% \EuroVisPoster % for EuroVis additional material 
% \EuroVisShort  % for EuroVis additional material
% \MedicalPrize  % uncomment for Medical Prize (Dirk Bartz) contribution, since 2021 part of EuroVis

% Licences: for CGF Journal (EG conf. full papers and STARs, EuroVis conf. full papers and STARs, SR, SGP, PG)
% please choose the correct license
%\CGFStandardLicense
\CGFccby
%\CGFccbync
%\CGFccbyncnd

% !! *please* don't change anything above
% !! unless you REALLY know what you are doing
% ------------------------------------------------------------------------
\usepackage[T1]{fontenc}
\usepackage{dfadobe}  

\usepackage{cite}  % comment out for biblatex with backend=biber
% ---------------------------
%\biberVersion
\BibtexOrBiblatex
%\usepackage[backend=biber,bibstyle=EG,citestyle=alphabetic,backref=true]{biblatex} 
%\addbibresource{egbibsample.bib}
% ---------------------------  
\electronicVersion
\PrintedOrElectronic
% for including postscript figures
% mind: package option 'draft' will replace PS figure by a filename within a frame
\ifpdf \usepackage[pdftex]{graphicx} \pdfcompresslevel=9
\else \usepackage[dvips]{graphicx} \fi

\usepackage{egweblnk} 
\usepackage{amsmath}
\usepackage{amssymb}
\usepackage{booktabs}
\usepackage{caption}
\usepackage{subcaption}
\usepackage{gensymb}
\usepackage{overpic}
\usepackage{longtable}
\usepackage{array}
\usepackage{tikz}
\usetikzlibrary{arrows.meta}
\usepackage{graphicx}
\usepackage{bm}
\usepackage[normalem]{ulem}

\captionsetup{labelfont=bf,textfont=it}
% end of prologue

% For creating the colored boxes
\newcommand{\revision}[1]{#1}

\newcommand{\GD}[1]{{\color[rgb]{0,0.7,0.0}\textbf{GD: {#1}}}} % Comment for the final version, to raise errors.
\newcommand{\GDmodif}[1]{{\color[rgb]{1,0.2,0.0}\textbf{{#1}}}}
\newcommand{\BKmodif}[1]{{\color[rgb]{1,0,1.0}\textbf{{#1}}}}
% Comment for the final version, to raise errors.
\newcommand{\GK}[1]{{\color[rgb]{0,0.2,1.0}\textbf{{GK: #1}}}} % Comment for the final version, to raise errors.
\newcommand{\TODO}[1]{{\color{red}\textbf{TODO: {#1}}}} % Comment for the final version, to raise errors.
\newcommand{\todo}[1]{{\color{red}\textbf{TODO: {#1}}}} % Comment for the final version, to raise errors.
\newcommand{\reftodo}[1]{{\color{red}\textbf{REF: {#1}}}} % Comment for the final version, to raise errors.
\newcommand{\MW}[1]{{\color[rgb]{0,0,0.7}\textbf{(MW: {#1})}}} % Comment for the final version, to raise errors.
\newcommand{\AC}[1]{{\color{magenta}{#1}}} % Comment for the final version, to raise errors.
\newcommand{\ACdel}[1]{{\color{magenta}{\sout{#1}}}} % Comment for the final version, to raise errors.
\newcommand{\abs}[1]{\left| #1 \right|}
\newcommand{\tAbs}[1]{| #1 |}
\newcommand{\G}{\ensuremath{\mathcal{G}}}
\newcommand{\Dim}{\ensuremath{\mathcal{D}}}
\newcommand{\vr}[1]{\ensuremath{\bm{#1}}}
\newcommand{\mat}[1]{\ensuremath{\bm{#1}}}
\newcommand*\diff{\mathop{}\!\mathrm{d}}
% \newcommand{\colvecXYZ}{%
%   \begin{pmatrix}
%   x \\
%   y \\
%   z
%   \end{pmatrix}
% }
% \newcommand{\colvecXY}{%
%   \begin{pmatrix}
%   x \\
%   y
%   \end{pmatrix}
% }
\newcommand{\colvecXYZ}{(x, y, z)^T}
\newcommand{\colvecXY}{(x, y)^T}

\newcommand{\avec}[0]{\ensuremath{\vr{a}}}
\newcommand{\bvec}[0]{\ensuremath{\vr{b}}}
\newcommand{\xvec}[0]{\ensuremath{\vr{x}}}
\newcommand{\yvec}[0]{\ensuremath{\vr{y}}}
\newcommand{\zvec}[0]{\ensuremath{\vr{z}}}
\newcommand{\ray}[0]{\ensuremath{\mathcal{R}}}
\newcommand{\pixel}[0]{\ensuremath{\vr{p}}}

\newcommand{\Ctns}[0]{\ensuremath{\tr{C}}}
\newcommand{\Gammatns}[0]{\ensuremath{\tr{\Gamma}}}

% ---------------------------------------------------------------------

\title[Supplemental Material for ``Does 3D Gaussian Splatting Need Accurate Volumetric Rendering ?'']{Supplemental Material for ``Does 3D Gaussian Splatting Need Accurate Volumetric Rendering ?''}
%%\title[Harmonizing 3D Gaussian Splatting with Volume Rendering]%
%%{Harmonizing 3D Gaussian Splatting with Volume Rendering}

% for anonymous conference submission please enter your SUBMISSION ID
\author[A. Celarek, G. Kopanas, G. Drettakis, M. Wimmer, B. Kerbl]
{\parbox{\textwidth}{\centering A. Celarek$^{1}$, G. Kopanas$^{2, 3, 4}$\orcid{0009-0002-5829-2192}, G. Drettakis$^{3, 4}$\orcid{0000-0002-9254-4819}, M. Wimmer$^{1}$\orcid{0000-0002-9370-2663} and B. Kerbl$^{1}$\orcid{0000-0002-5168-8648}
		%        S. Spencer$^2$\thanks{Chairman Siggraph Publications Board}
	}
	\\
	% For Computer Graphics Forum: Please use the abbreviation of your first name.
	{\parbox{\textwidth}{\centering $^1$TU Wien, Austria, $^2$Google, United Kingdom, $^3$Inria, France, $^4$Université Côte d'Azur, France
			%        $^2$ Another Department to illustrate the use in /papers from authors
			%             with different affiliations
		}
	}
}
% instead of the author's name (and leave the affiliation blank) !!
% for final version: please provide your *own* ORCID in the brackets following \orcid; see https://orcid.org/ for more details.
%\author[D. Fellner \& S. Behnke]
%{\parbox{\textwidth}{\centering D.\,W. Fellner\thanks{Chairman Eurographics Publications Board}$^{1,2}$\orcid{0000-0001-7756-0901}
		%		and S. Behnke$^{2}$\orcid{0000-0001-5923-423X} 
		%		%        S. Spencer$^2$\thanks{Chairman Siggraph Publications Board}
		%	}
	%	\\
	%	% For Computer Graphics Forum: Please use the abbreviation of your first name.
	%	{\parbox{\textwidth}{\centering $^1$TU Darmstadt \& Fraunhofer IGD, Germany\\
			%			$^2$Graz University of Technology, Institute of Computer Graphics and Knowledge Visualization, Austria
			%			%        $^2$ Another Department to illustrate the use in papers from authors
			%			%             with different affiliations
			%		}
		%	}
	%}
% ------------------------------------------------------------------------

% if the Editors-in-Chief have given you the data, you may uncomment
% the following five lines and insert it here
%
% \volume{36}   % the volume in which the issue will be published;
% \issue{1}     % the issue number of the publication
% \pStartPage{1}      % set starting page


%-------------------------------------------------------------------------
\begin{document}
	

\maketitle

\section{Proof for Equation used for Self-Attenuation}
\label{ap:proof_volumetric_equation_single_gaussian_closed_form}

\revision{
Given the computation:
\begin{align}
	I(\pixel) = c_0(\ray)\int_{-\infty}^{\infty} \mathcal{G}_3^n(\ray(t), 0) e^{-\int_{-\infty}^t \mathcal{G}_3^n(\ray(\tau), 0)\diff \tau} \diff t,
\end{align}
where $\G^n$ are normalised Gaussians, $f_0$ is a 2D extinction function:
\begin{align}
	f_0(\vr{p}) = \G_2^n(\pixel, w_0, \vr{\mu}'_0, \vr{\Sigma}'_0) = \int_{-\infty}^{\infty} \G_3^n(\ray(t), w_0, \vr{\mu}_0, \vr{\Sigma}_0)\diff t,
\end{align}
and $\ray$ is the ray going through pixel $\pixel$.
\\
\\
1. \textbf{Define the inner integral}:
\[
H(t) = \int_{-\infty}^t \mathcal{G}_3^n(\ray(\tau), 0) d\tau
\]
2. \textbf{Differentiate \( H(t) \)}.
By the Fundamental Theorem of Calculus:
\[
\frac{dH(t)}{dt} = \mathcal{G}_3^n(\ray(\tau), 0)
\]
3. \textbf{Rewrite the integral}.
Substitute \( \mathcal{G}_3^n(\ray(\tau), 0) dt \) with \( dH(t) \):
\[
\int_{-\infty}^{\infty} \mathcal{G}_3^n(\ray(\tau), 0) e^{-H(t)} dt = \int_{-\infty}^{\infty} e^{-H(t)} dH(t)
\]
%4. \textbf{Substitute the Gaussian function}.
%Given \( g_0(\ray(t)) = A e^{-\frac{(\ray(t) - \vr{\mu})^2}{2\mat{\Sigma}^2}} \), we can rewrite \( H(t) \):
%\[
%H(t) = \int_{-\infty}^t A e^{-\frac{(\ray(\tau) - \vr{\mu})^2}{2\mat{\Sigma}^2}} d\tau
%\]
%5. \textbf{Evaluate \( H(t) \)}.
%Since \( g_0(\ray(t)) \) is a Gaussian function, its integral from \(-\infty\) to \( t \) will be a cumulative distribution function (CDF) of the normal distribution, scaled by \( w_0 \):
%\[
%H(t) = w_0 \int_{-\infty}^t \frac{1}{w_0} g_0(\ray(t)) d\tau
%\]
4. \textbf{Integral evaluation}:
\[
\int_{-\infty}^{\infty} e^{-H(t)} dH(t)
\]
Given that \( H(t) \) for the Gaussian distribution is bounded and converges from $0$ to \( f_0(\pixel) \) as \( t \) goes from \(-\infty\) to \( \infty \), we can write:
\[
\int_{0}^{f_0(\pixel)} e^{-u} du
\]
5. \textbf{Evaluate the integral}:
\[
\int_{0}^{f_0(\pixel)} e^{-u} du = \left[ -e^{-u} \right]_{0}^{f_0(\pixel)} = -e^{-f_0(\pixel)} + e^{0} = 1 - e^{-f_0(\pixel)}
\]
Thus, the integral evaluates to \( 1 - e^{-f_0(\pixel)} \).
}

% Adding the second term, we get:
% \[
% (1 - e^{-w_0}) + c_b e^{-w_0(x)}
% \]
% Therefore, the closed-form solution to the given formulation, under the assumption that \( g_0(x(t)) \) is a Gaussian function that integrates to \( w_0 \), is:
% \[
% c_0(x)(1 - e^{-w_0}) + c_b e^{-w_0(x)}
% \]

%\subsection{old notation}
%($m_0$ became $w_0$)
%
%Given the computation:
%\[
%c_0(\pixel)\int_{-\infty}^{\infty} g_0(\ray(t)) e^{-\int_{-\infty}^t g_0(x(\tau)) d\tau} dt + c_b e^{-m_0(x)},
%\]
%assume \( g_0(x(t)) \) is a Gaussian function that integrates to \( m_0 \):
%\[
%g_0(x(t)) = A e^{-\frac{(x(t) - \vr{\mu})^2}{2\mat{\Sigma}^2}}
%\]
%where \( A \) is a constant such that:
%\[
%\int_{-\infty}^{\infty} g_0(x(t)) dt = m_0
%\]
%1. \textbf{Define the inner integral}:
%   \[
%   H(t) = \int_{-\infty}^t g_0(x(\tau)) d\tau
%   \]
%2. \textbf{Differentiate \( H(t) \)}.
%   By the Fundamental Theorem of Calculus:
%   \[
%   \frac{dH(t)}{dt} = g_0(x(t))
%   \]
%3. \textbf{Rewrite the integral}.
%   Substitute \( g_0(x(t)) dt \) with \( dH(t) \):
%   \[
%   \int_{-\infty}^{\infty} g_0(x(t)) e^{-H(t)} dt = \int_{-\infty}^{\infty} e^{-H(t)} g_0(x(t)) dt = \int_{-\infty}^{\infty} e^{-H(t)} dH(t)
%   \]
%4. \textbf{Substitute the Gaussian function}.
%   Given \( g_0(x(t)) = A e^{-\frac{(x(t) - \vr{\mu})^2}{2\mat{\Sigma}^2}} \), we can rewrite \( H(t) \):
%   \[
%   H(t) = \int_{-\infty}^t A e^{-\frac{(x(\tau) - \vr{\mu})^2}{2\mat{\Sigma}^2}} d\tau
%   \]
%5. \textbf{Evaluate \( H(t) \)}.
%   Since \( g_0(x(t)) \) is a Gaussian function, its integral from \(-\infty\) to \( t \) will be a cumulative distribution function (CDF) of the normal distribution, scaled by \( m_0 \):
%   \[
%   H(t) = m_0 \int_{-\infty}^t \frac{1}{m_0} A e^{-\frac{(x(\tau) - \vr{\mu})^2}{2\mat{\Sigma}^2}} d\tau
%   \]
%6. \textbf{Integral evaluation}:
%   \[
%   \int_{-\infty}^{\infty} e^{-H(t)} dH(t)
%   \]
%   Given that \( H(t) \) for the Gaussian distribution is bounded and converges from 0 to \( m_0 \) as \( t \) goes from \(-\infty\) to \( \infty \), we can write:
%   \[
%   \int_{0}^{m_0} e^{-u} du
%   \]
%7. \textbf{Evaluate the integral}:
%   \[
%   \int_{0}^{m_0} e^{-u} du = \left[ -e^{-u} \right]_{0}^{m_0} = -e^{-m_0} + e^{0} = 1 - e^{-m_0}
%   \]
%Thus, the integral evaluates to \( 1 - e^{-m_0} \). 
%Adding the second term, we get:
%\[
%(1 - e^{-m_0}) + c_b e^{-m_0(x)}
%\]
%Therefore, the closed-form solution to the given formulation, under the assumption that \( g_0(x(t)) \) is a Gaussian function that integrates to \( m_0 \), is:
%\[
%c_0(x)(1 - e^{-m_0}) + c_b e^{-m_0(x)}
%\]

\revision{

\section{Implementation Details}

For initialization, we use the same policy that the original 3DGS codebase employed for training Nerf-Synthetic: here, the unit cube around the scene origin was initialized with 100k Gaussians with random positions and uniform opacity, i.e, all Gaussians were initialized to the same opacity / OTS weight.

We initialize the setup with 4k/12k/36k, etc., Gaussians, according to the experiment. We also compute a single value for all Gaussians to be used, however, ours is derived via the power function: Due to their different image formation solutions, we use $\frac{2}{N^{0.35}}$ and $\frac{2}{N^{0.55}}$ for 3DGS and OTS-based methods, respectively, where $N$ is the total number of Gaussians. 

\section{Results on Real-World Scenes}
To provide tentative evaluations on real-world datasets, we propose an exploration of larger scenes with the following experiment: we adapt the densification variables in our most principled splatting technique, OTS+SAtn, to match 3DGS as closely as possible and stop densification after reaching 100k/500k/1m/5m Gaussians. While this procedure is not as controlled as our main setup, it suffices to eliminate egregious discrepancies in primitive count, while still making use of densification.

Evaluating the outcomes on the MipNeRF360 dataset, we did not find significant trends for quality: across all tested sizes and scenes, 3DGS and OTS+SAtn differ by at most 4\% PSNR. Average training times were slightly lower (2\%) for OTS+SAtn, across all sizes. Similarly, rendering times are balanced, although with comparably high variance (OTS+SAtn 10\% faster in \textsc{Room}, 15\% slower in \textsc{Bonsai}). This suggests that the approximations used in the 3DGS visibility model also do not impact its effectiveness and robustness in larger, real-world scenarios.
}

\section{Details of the Backward Pass for Self-Attenuation Based Splatting}
\label{ap:self_attenuation_details}
3DGS implements the backward pass exactly.
That is, in forward mode the Gaussians are blended front-to-back, and in backward mode the gradient is computed back-to-front.
For this to work, the index of the last Gaussian and the remaining opacity is stored in forward mode.
Gaussians retrieved via this index, and remaining opacity are then used in the backward pass to recover all intermediate values.
In order to keep the numerical error in check, the forward mode blending is stopped \emph{before} the remaining opacity becomes too small.
However, this approach causes heavy artefacts if self-attenuation is used.
The reason is, that the linear falloff of the Taylor approximation is replaced with a true exponential falloff.
Therefore, we compute the gradient in the backward pass with a front-to-back strategy.
This works, because volumetric rendering can be performed either front-to-back or back-to-front, hence we can also compute the gradient in either direction.

\section{Details of the Ray Marching Algorithm}
\label{ap:ray_marching_details}
\paragraph*{Batching.}
For computational efficiency, we divide the work into batches of e.g. 128 bins.
Within each batch, the mixture is iterated twice as described in the main text.
The first batch starts at $t=0$ and ends at an position adapted to the data.
The next batch is started at the end-border of the last bin.
This repeats until we run out of Gaussians or accumulated opacity reaches a high enough threshold.

\paragraph*{Adaptive binning.}
This process starts by creating a fixed-size \emph{density section buffer}, where a sections are defined by start, end, and density.
Then, we iterate over all Gaussians to fill the buffer.
Initially, each Gaussian produces one sampling section by taking its position $\pm$ 3 standard deviations for start and end, and computing the density such that a pre-defined number of bins fit in this space.
We then add this section to the density section buffer.
When adding a section, existing and new sections are combined in such a way, that the resulting sections are in order and without overlap.
If sections do overlap during construction, the section with higher sampling density wins in the overlap area.
Accordingly, combining 2 sections can create up to 3 new sections.
Sections that fall out of the buffer are deferred to the next batch.

Once the buffer is filled, the bins are created by traversing the sections such that the size of each bin is inversely proportional to the density.

\paragraph*{Blending}
We then iterate over Gaussians again to compute integrals for each bin.
This results in 4 values per bin: red, green, blue, and integrated extinction (opacity).
Following Eq.~10 of the paper, colours are multiplied with per Gaussian integrated extinction.
Finally, the bin evaluations are blended together using
\begin{align}
	I(\vr{x}) = \sum_{i=0}^\infty \gamma_i \prod_{j=0}^{i-1} e^{- \rho_j} + c_b \prod_{j=0}^\infty e^{-\rho_j},
\end{align}
where the sum and products iterate over bins, $\gamma$ is one of the colours and $\rho$ the integrated extinction (opacity).
$\prod_{j=0}^{i-1} e^{- \rho_j}$ is the transmittance factor.
It is cached between iteration such that the product is computed only once.

\paragraph*{Gradient computation.}
The gradient was implemented manually using a similar front-to-back strategy as described in Sec. \ref{ap:self_attenuation_details}.
Fundamentally, there are two methods to compute a gradient for numerical integration methods: attached and detached \cite{zeltner2021monte}.
An attached gradient takes the numerical integration procedure into account.
In Monte Carlo integration that is sampling probabilities, or in our case the border positions.
A detached gradient, on the other hand, assumes that the numerical method can be ignored.
It therefore makes sampling or border positions constant with respect to the differentiation.
We implemented both variants and found, that the detached variant performs significantly better.
All gradient computations were verified by testing against a numerical variant.

\bibliographystyle{eg-alpha-doi} 
\bibliography{article}      

\onecolumn

\section{Aggregated Numerical Results}
\label{ap:avg_results}
In the the tables we show averages for various error metrics over all, the nerf, and the volumetric scenes.
Full numerical results, and error plots for each scene are further below.

\subsection{PSNR}
\begin{table}[h!]
\centering
\caption{Average PSNR results for all scenes.}
\label{tab:psnr_avg}
\begin{tabular}{lrrrrrr}
\toprule
& \multicolumn{6}{c}{Number of Gaussians} \\
Algorithm & 4000 & 12000 & 36000 & 108000 & 324000 & 972000 \\

\midrule
3DGS            & 31.63 & 33.22 & 34.73 & 36.05 & 37.02 & 37.58 \\
3DGS + STP      & 31.69 & 33.30 & 34.81 & 36.15 & 37.10 & 37.62 \\
OTS             & 32.17 & 33.66 & 34.83 & 35.78 & 36.52 & 36.93 \\
OTS + SAtn      & 32.24 & 33.72 & 34.90 & 35.81 & 36.51 & 36.93 \\
3DGS marcher    & 32.23 & 33.77 & 35.08 & 36.11 & 36.79 & 37.14 \\
OTS marcher     & 32.28 & 33.77 & 34.94 & 35.85 & 36.45 & 36.77 \\
\bottomrule
\end{tabular}
\end{table}

\begin{figure}[h!]
	\centering
	\includegraphics[width=0.7\linewidth]{tables/psnr_avg_all_plot}
	\caption{PSNR average over all scenes}
	\label{fig:ap_psnr_avg_all_plot}
\end{figure}
\pagebreak

\begin{table}[h!]
\centering
\caption{Average PSNR results for volumetric scenes}
\label{tab:psnr_avg_interest}
\begin{tabular}{lrrrrrr}
\toprule
& \multicolumn{6}{c}{Number of Gaussians} \\
Algorithm & 4000 & 12000 & 36000 & 108000 & 324000 & 972000 \\

\midrule
3DGS & 39.37 & 40.78 & 42.15 & 43.29 & 44.01 & 44.33 \\
3DGS + STP & 39.44 & 40.85 & 42.23 & 43.43 & 44.12 & 44.43 \\
OTS & 39.59 & 40.98 & 42.06 & 42.91 & 43.43 & 43.71 \\
OTS + SAtn & 39.66 & 41.04 & 42.15 & 42.93 & 43.38 & 43.71 \\
3DGS marcher & 39.51 & 41.03 & 42.31 & 43.15 & 43.59 & 43.66 \\
OTS marcher & 39.66 & 41.04 & 42.20 & 42.98 & 43.34 & 43.48 \\
\bottomrule
\end{tabular}
\end{table}

\begin{figure}[h!]
	\centering
	\includegraphics[width=0.7\linewidth]{tables/psnr_avg_volumetric_plot}
	\caption{PSNR average over volumetric scenes}
	\label{fig:ap_psnr_avg_vol_plot}
\end{figure}
\pagebreak

\begin{table}[h!]
\centering
\caption{Average PSNR results for Nerf scenes}
\label{tab:psnr_avg_others}
\begin{tabular}{lrrrrrr}
\toprule
& \multicolumn{6}{c}{Number of Gaussians} \\
Algorithm & 4000 & 12000 & 36000 & 108000 & 324000 & 972000 \\

\midrule
3DGS & 25.82 & 27.56 & 29.16 & 30.63 & 31.77 & 32.52 \\
3DGS + STP & 25.87 & 27.63 & 29.25 & 30.69 & 31.83 & 32.51 \\
OTS & 26.60 & 28.16 & 29.40 & 30.43 & 31.34 & 31.85 \\
OTS + SAtn & 26.67 & 28.24 & 29.45 & 30.47 & 31.36 & 31.85 \\
3DGS marcher & 26.77 & 28.33 & 29.66 & 30.83 & 31.70 & 32.25 \\
OTS marcher & 26.76 & 28.31 & 29.50 & 30.50 & 31.27 & 31.74 \\
\bottomrule
\end{tabular}
\end{table}

\begin{figure}[h!]
	\centering
	\includegraphics[width=0.7\linewidth]{tables/psnr_avg_nerf_plot}
	\caption{PSNR average over Nerf scenes}
	\label{fig:ap_psnr_avg_nerf_plot}
\end{figure}
\pagebreak

\subsection{SSIM}
\begin{table}[h!]
\centering
\caption{Average SSIM results for all scenes.}
\label{tab:ssim_avg}
\begin{tabular}{lrrrrrr}
\toprule
& \multicolumn{6}{c}{Number of Gaussians} \\
Algorithm & 4000 & 12000 & 36000 & 108000 & 324000 & 972000 \\

\midrule
3DGS            & 0.9320 & 0.9471 & 0.9596 & 0.9684 & 0.9736 & 0.9760 \\
3DGS + STP      & 0.9322 & 0.9476 & 0.9601 & 0.9688 & 0.9738 & 0.9760 \\
OTS             & 0.9394 & 0.9524 & 0.9619 & 0.9686 & 0.9729 & 0.9747 \\
OTS + SAtn      & 0.9394 & 0.9526 & 0.9621 & 0.9688 & 0.9730 & 0.9748 \\
3DGS marcher    & 0.9401 & 0.9536 & 0.9635 & 0.9701 & 0.9739 & 0.9756 \\
OTS marcher     & 0.9402 & 0.9536 & 0.9629 & 0.9694 & 0.9732 & 0.9749 \\
\bottomrule
\end{tabular}
\end{table}

\begin{figure}[h!]
	\centering
	\includegraphics[width=0.7\linewidth]{tables/ssim_avg_all_plot}
	\caption{SSIM average over all scenes}
	\label{fig:ap_ssim_avg_all_plot}
\end{figure}
\pagebreak

\begin{table}[h!]
\centering
\caption{Average SSIM results for volumetric scenes}
\label{tab:ssim_avg_interest}
\begin{tabular}{lrrrrrr}
\toprule
& \multicolumn{6}{c}{Number of Gaussians} \\
Algorithm & 4000 & 12000 & 36000 & 108000 & 324000 & 972000 \\

\midrule
3DGS & 0.9733 & 0.9801 & 0.9853 & 0.9882 & 0.9896 & 0.9900 \\
3DGS + STP & 0.9733 & 0.9801 & 0.9852 & 0.9883 & 0.9897 & 0.9900 \\
OTS & 0.9754 & 0.9821 & 0.9862 & 0.9884 & 0.9893 & 0.9895 \\
OTS + SAtn & 0.9756 & 0.9823 & 0.9863 & 0.9885 & 0.9894 & 0.9895 \\
3DGS marcher & 0.9746 & 0.9816 & 0.9860 & 0.9883 & 0.9892 & 0.9893 \\
OTS marcher & 0.9755 & 0.9821 & 0.9863 & 0.9885 & 0.9893 & 0.9894 \\
\bottomrule
\end{tabular}
\end{table}

\begin{figure}[h!]
	\centering
	\includegraphics[width=0.7\linewidth]{tables/ssim_avg_volumetric_plot}
	\caption{SSIM average over volumetric scenes}
	\label{fig:ap_ssim_avg_vol_plot}
\end{figure}
\pagebreak

\begin{table}[h!]
\centering
\caption{Average SSIM results for Nerf scenes}
\label{tab:ssim_avg_others}
\begin{tabular}{lrrrrrr}
\toprule
& \multicolumn{6}{c}{Number of Gaussians} \\
Algorithm & 4000 & 12000 & 36000 & 108000 & 324000 & 972000 \\

\midrule
3DGS & 0.9010 & 0.9223 & 0.9404 & 0.9535 & 0.9616 & 0.9655 \\
3DGS + STP & 0.9013 & 0.9233 & 0.9412 & 0.9542 & 0.9619 & 0.9655 \\
OTS & 0.9124 & 0.9302 & 0.9438 & 0.9538 & 0.9606 & 0.9636 \\
OTS + SAtn & 0.9122 & 0.9304 & 0.9440 & 0.9540 & 0.9607 & 0.9637 \\
3DGS marcher & 0.9143 & 0.9326 & 0.9466 & 0.9565 & 0.9624 & 0.9654 \\
OTS marcher & 0.9138 & 0.9322 & 0.9454 & 0.9552 & 0.9611 & 0.9640 \\
\bottomrule
\end{tabular}
\end{table}

\begin{figure}[h!]
	\centering
	\includegraphics[width=0.7\linewidth]{tables/ssim_avg_nerf_plot}
	\caption{SSIM average over Nerf scenes}
	\label{fig:ap_ssim_avg_nerf_plot}
\end{figure}
\pagebreak

\subsection{LPIPS}
\begin{table}[h!]
\centering
\caption{Average LPIPS results for all scenes.}
\label{tab:lpips_avg}
\begin{tabular}{lrrrrrr}
\toprule
& \multicolumn{6}{c}{Number of Gaussians} \\
Algorithm & 4000 & 12000 & 36000 & 108000 & 324000 & 972000 \\

\midrule
3DGS            & 0.1231 & 0.1007 & 0.0800 & 0.0632 & 0.0523 & 0.0467 \\
3DGS + STP      & 0.1222 & 0.0995 & 0.0790 & 0.0625 & 0.0520 & 0.0467 \\
OTS             & 0.1115 & 0.0906 & 0.0742 & 0.0611 & 0.0517 & 0.0471 \\
OTS + SAtn      & 0.1122 & 0.0913 & 0.0744 & 0.0611 & 0.0518 & 0.0471 \\
3DGS marcher    & 0.1115 & 0.0908 & 0.0736 & 0.0606 & 0.0521 & 0.0477 \\
OTS marcher     & 0.1116 & 0.0908 & 0.0743 & 0.0610 & 0.0523 & 0.0481 \\
\bottomrule
\end{tabular}
\end{table}

\begin{figure}[h!]
	\centering
	\includegraphics[width=0.7\linewidth]{tables/lpips_avg_all_plot}
	\caption{LPIPS average over all scenes}
	\label{fig:ap_lpips_avg_all_plot}
\end{figure}
\pagebreak

\begin{table}[h!]
\centering
\caption{Average LPIPS results for volumetric scenes}
\label{tab:lpips_avg_interest}
\begin{tabular}{lrrrrrr}
\toprule
& \multicolumn{6}{c}{Number of Gaussians} \\
Algorithm & 4000 & 12000 & 36000 & 108000 & 324000 & 972000 \\

\midrule
3DGS & 0.1200 & 0.1023 & 0.0853 & 0.0721 & 0.0641 & 0.0600 \\
3DGS + STP & 0.1192 & 0.1015 & 0.0849 & 0.0720 & 0.0641 & 0.0602 \\
OTS & 0.1146 & 0.0954 & 0.0803 & 0.0685 & 0.0613 & 0.0574 \\
OTS + SAtn & 0.1146 & 0.0955 & 0.0800 & 0.0683 & 0.0611 & 0.0573 \\
3DGS marcher & 0.1173 & 0.0986 & 0.0832 & 0.0719 & 0.0647 & 0.0609 \\
OTS marcher & 0.1148 & 0.0959 & 0.0809 & 0.0691 & 0.0623 & 0.0586 \\
\bottomrule
\end{tabular}
\end{table}

\begin{figure}[h!]
	\centering
	\includegraphics[width=0.7\linewidth]{tables/lpips_avg_volumetric_plot}
	\caption{LPIPS average over volumetric scenes}
	\label{fig:ap_lpips_avg_vol_plot}
\end{figure}
\pagebreak

\begin{table}[h!]
\centering
\caption{Average LPIPS results for Nerf scenes}
\label{tab:lpips_avg_others}
\begin{tabular}{lrrrrrr}
\toprule
& \multicolumn{6}{c}{Number of Gaussians} \\
Algorithm & 4000 & 12000 & 36000 & 108000 & 324000 & 972000 \\

\midrule
3DGS & 0.1254 & 0.0995 & 0.0760 & 0.0566 & 0.0435 & 0.0366 \\
3DGS + STP & 0.1244 & 0.0980 & 0.0746 & 0.0553 & 0.0429 & 0.0366 \\
OTS & 0.1092 & 0.0871 & 0.0696 & 0.0555 & 0.0446 & 0.0394 \\
OTS + SAtn & 0.1105 & 0.0881 & 0.0703 & 0.0557 & 0.0448 & 0.0396 \\
3DGS marcher & 0.1071 & 0.0850 & 0.0664 & 0.0521 & 0.0426 & 0.0378 \\
OTS marcher & 0.1091 & 0.0869 & 0.0694 & 0.0549 & 0.0449 & 0.0403 \\
\bottomrule
\end{tabular}
\end{table}

\begin{figure}[h!]
	\centering
	\includegraphics[width=0.7\linewidth]{tables/lpips_avg_nerf_plot}
	\caption{LPIPS average over Nerf scenes}
	\label{fig:ap_lpips_avg_nerf_plot}
\end{figure}
\pagebreak

\section{Full Numerical Results}
Here we show the full numerical results, and per-scene error plots.
\label{ap:full_results}
\subsection{PSNR}
\begin{longtable}[H]{llrrrrrr}
\toprule
& & \multicolumn{6}{c}{Number of Gaussians} \\
Scene & Algorithm & 4000 & 12000 & 36000 & 108000 & 324000 & 972000 \\
\midrule \endhead
Burning Ficus & 3DGS & 29.61 & 32.35 & 35.29 & 37.97 & 39.85 & 40.57 \\
 & 3DGS + STP & 29.75 & 32.34 & 35.30 & 38.18 & 40.01 & 40.64 \\
 & OTS & 30.07 & 32.96 & 34.66 & 35.64 & 36.20 & 36.43 \\
 & OTS + SAtn & 30.36 & 33.06 & 34.85 & 35.74 & 36.24 & 36.43 \\
 & 3DGS marcher & 29.99 & 33.28 & 35.48 & 36.93 & 37.79 & 38.11 \\
 & OTS marcher & 30.15 & 32.80 & 34.59 & 35.55 & 36.05 & 36.20 \\
Chair & 3DGS & 27.79 & 29.16 & 30.41 & 31.70 & 32.89 & 33.96 \\
 & 3DGS + STP & 27.68 & 29.11 & 30.37 & 31.58 & 32.95 & 33.99 \\
 & OTS & 28.32 & 29.74 & 30.79 & 31.91 & 33.14 & 33.97 \\
 & OTS + SAtn & 28.38 & 29.87 & 30.86 & 31.93 & 33.13 & 33.92 \\
 & 3DGS marcher & 28.23 & 29.18 & 30.25 & 31.90 & 33.01 & 33.96 \\
 & OTS marcher & 28.44 & 29.90 & 30.80 & 31.94 & 33.00 & 33.83 \\
Coloured Wdas & 3DGS & 37.19 & 38.13 & 39.02 & 39.65 & 39.99 & 40.17 \\
 & 3DGS + STP & 37.16 & 38.15 & 39.00 & 39.63 & 40.01 & 40.19 \\
 & OTS & 37.46 & 38.18 & 38.67 & 39.10 & 39.43 & 39.79 \\
 & OTS + SAtn & 37.48 & 38.22 & 38.67 & 39.09 & 39.43 & 39.78 \\
 & 3DGS marcher & 37.18 & 38.02 & 38.80 & 39.34 & 39.72 & 39.94 \\
 & OTS marcher & 37.32 & 37.92 & 38.35 & 38.85 & 39.19 & 39.56 \\
Drums & 3DGS & 21.72 & 23.32 & 24.45 & 25.22 & 25.68 & 25.98 \\
 & 3DGS + STP & 21.80 & 23.47 & 24.49 & 25.24 & 25.67 & 25.91 \\
 & OTS & 22.13 & 23.78 & 24.63 & 25.16 & 25.51 & 25.69 \\
 & OTS + SAtn & 22.18 & 23.77 & 24.66 & 25.17 & 25.52 & 25.69 \\
 & 3DGS marcher & 22.53 & 23.98 & 24.89 & 25.42 & 25.75 & 25.95 \\
 & OTS marcher & 22.26 & 23.83 & 24.69 & 25.17 & 25.49 & 25.64 \\
Explosion & 3DGS & 32.79 & 36.26 & 39.81 & 42.67 & 44.47 & 45.49 \\
 & 3DGS + STP & 32.92 & 36.41 & 39.96 & 42.75 & 44.48 & 45.43 \\
 & OTS & 33.29 & 36.92 & 40.46 & 43.36 & 44.96 & 45.62 \\
 & OTS + SAtn & 33.33 & 37.01 & 40.64 & 43.38 & 44.93 & 45.62 \\
 & 3DGS marcher & 33.43 & 37.10 & 40.88 & 43.46 & 44.52 & 44.49 \\
 & OTS marcher & 33.65 & 37.60 & 41.54 & 44.13 & 44.84 & 44.70 \\
Ficus & 3DGS & 25.78 & 28.30 & 30.93 & 33.39 & 35.12 & 35.86 \\
 & 3DGS + STP & 25.73 & 28.39 & 31.14 & 33.64 & 35.37 & 36.02 \\
 & OTS & 26.49 & 28.61 & 30.21 & 31.11 & 31.67 & 31.94 \\
 & OTS + SAtn & 26.57 & 28.84 & 30.46 & 31.26 & 31.74 & 31.95 \\
 & 3DGS marcher & 26.95 & 29.71 & 31.69 & 32.95 & 33.72 & 34.05 \\
 & OTS marcher & 26.58 & 28.71 & 30.36 & 31.27 & 31.71 & 31.87 \\
Hotdog & 3DGS & 30.96 & 32.91 & 34.44 & 35.56 & 36.32 & 36.86 \\
 & 3DGS + STP & 31.20 & 33.10 & 34.63 & 35.66 & 36.33 & 36.77 \\
 & OTS & 31.71 & 33.42 & 34.65 & 35.42 & 36.00 & 36.40 \\
 & OTS + SAtn & 32.04 & 33.58 & 34.73 & 35.46 & 36.04 & 36.42 \\
 & 3DGS marcher & 31.87 & 33.43 & 34.67 & 35.49 & 36.02 & 36.28 \\
 & OTS marcher & 32.18 & 33.64 & 34.55 & 35.28 & 35.59 & 35.77 \\
Lego & 3DGS & 24.59 & 26.55 & 28.45 & 30.63 & 32.69 & 34.14 \\
 & 3DGS + STP & 24.81 & 26.69 & 28.72 & 30.72 & 32.67 & 34.03 \\
 & OTS & 25.57 & 27.22 & 28.96 & 30.81 & 32.59 & 33.65 \\
 & OTS + SAtn & 25.54 & 27.30 & 28.97 & 30.76 & 32.57 & 33.67 \\
 & 3DGS marcher & 25.72 & 27.45 & 29.31 & 31.22 & 33.02 & 34.24 \\
 & OTS marcher & 25.61 & 27.38 & 29.17 & 30.79 & 32.51 & 33.53 \\
Materials & 3DGS & 22.48 & 24.42 & 26.23 & 27.74 & 28.81 & 29.54 \\
 & 3DGS + STP & 22.46 & 24.50 & 26.34 & 27.84 & 28.85 & 29.54 \\
 & OTS & 23.65 & 25.62 & 27.28 & 28.51 & 29.43 & 29.95 \\
 & OTS + SAtn & 23.59 & 25.61 & 27.26 & 28.59 & 29.44 & 29.96 \\
 & 3DGS marcher & 23.39 & 25.52 & 27.08 & 28.21 & 29.09 & 29.66 \\
 & OTS marcher & 23.68 & 25.68 & 27.24 & 28.58 & 29.45 & 29.94 \\
Mic & 3DGS & 28.47 & 29.93 & 31.31 & 32.64 & 33.64 & 34.26 \\
 & 3DGS + STP & 28.51 & 30.03 & 31.38 & 32.74 & 33.78 & 34.27 \\
 & OTS & 29.45 & 30.47 & 31.35 & 32.26 & 33.43 & 33.90 \\
 & OTS + SAtn & 29.62 & 30.54 & 31.34 & 32.35 & 33.45 & 33.95 \\
 & 3DGS marcher & 29.70 & 30.61 & 31.65 & 32.90 & 33.73 & 34.27 \\
 & OTS marcher & 29.65 & 30.62 & 31.41 & 32.38 & 33.38 & 33.92 \\
Ship & 3DGS & 24.78 & 25.87 & 27.06 & 28.15 & 29.01 & 29.58 \\
 & 3DGS + STP & 24.76 & 25.78 & 26.92 & 28.12 & 28.99 & 29.53 \\
 & OTS & 25.46 & 26.46 & 27.36 & 28.25 & 28.97 & 29.28 \\
 & OTS + SAtn & 25.45 & 26.41 & 27.33 & 28.27 & 28.96 & 29.26 \\
 & 3DGS marcher & 25.74 & 26.78 & 27.75 & 28.57 & 29.24 & 29.61 \\
 & OTS marcher & 25.66 & 26.71 & 27.74 & 28.60 & 29.03 & 29.41 \\
Wdas Cloud 1 & 3DGS & 48.87 & 49.05 & 49.11 & 49.17 & 49.17 & 49.23 \\
 & 3DGS + STP & 48.87 & 49.11 & 49.28 & 49.38 & 49.36 & 49.37 \\
 & OTS & 48.82 & 49.04 & 49.36 & 49.59 & 49.71 & 49.99 \\
 & OTS + SAtn & 48.96 & 49.12 & 49.43 & 49.51 & 49.70 & 49.95 \\
 & 3DGS marcher & 48.88 & 48.91 & 49.17 & 49.14 & 49.17 & 49.22 \\
 & OTS marcher & 49.03 & 49.31 & 49.51 & 49.68 & 50.00 & 50.17 \\
Wdas Cloud 2 & 3DGS & 45.29 & 45.57 & 45.80 & 46.08 & 46.20 & 46.14 \\
 & 3DGS + STP & 45.32 & 45.61 & 45.90 & 46.23 & 46.29 & 46.26 \\
 & OTS & 45.36 & 45.60 & 45.77 & 45.89 & 46.00 & 46.09 \\
 & OTS + SAtn & 45.36 & 45.63 & 45.79 & 45.90 & 45.93 & 46.06 \\
 & 3DGS marcher & 45.03 & 45.50 & 45.74 & 45.90 & 46.03 & 45.88 \\
 & OTS marcher & 45.42 & 45.68 & 45.86 & 46.00 & 46.13 & 46.15 \\
Wdas Cloud 3 & 3DGS & 42.48 & 43.32 & 43.86 & 44.20 & 44.40 & 44.39 \\
 & 3DGS + STP & 42.65 & 43.50 & 43.95 & 44.40 & 44.59 & 44.69 \\
 & OTS & 42.57 & 43.22 & 43.46 & 43.91 & 44.30 & 44.34 \\
 & OTS + SAtn & 42.49 & 43.17 & 43.54 & 43.93 & 44.06 & 44.42 \\
 & 3DGS marcher & 42.53 & 43.36 & 43.80 & 44.10 & 44.30 & 44.31 \\
 & OTS marcher & 42.36 & 42.95 & 43.38 & 43.70 & 43.85 & 44.08 \\
\bottomrule
\caption{PSNR results for all scenes}
\end{longtable}

\begin{figure}[h!]
\centering
\includegraphics[width=0.8\textwidth]{tables/psnr_Burning_Ficus_plot.pdf}
\caption{PSNR results for Burning Ficus}
\label{fig:ap_psnr_Burning Ficus_plot}
\end{figure}

\begin{figure}[h!]
\centering
\includegraphics[width=0.8\textwidth]{tables/psnr_Coloured_Wdas_plot.pdf}
\caption{PSNR results for Coloured Wdas}
\label{fig:ap_psnr_Coloured Wdas_plot}
\end{figure}

\begin{figure}[h!]
\centering
\includegraphics[width=0.8\textwidth]{tables/psnr_Explosion_plot.pdf}
\caption{PSNR results for Explosion}
\label{fig:ap_psnr_Explosion_plot}
\end{figure}

\begin{figure}[h!]
\centering
\includegraphics[width=0.8\textwidth]{tables/psnr_Wdas_Cloud_1_plot.pdf}
\caption{PSNR results for Wdas Cloud 1}
\label{fig:ap_psnr_Wdas Cloud 1_plot}
\end{figure}

\begin{figure}[h!]
\centering
\includegraphics[width=0.8\textwidth]{tables/psnr_Wdas_Cloud_2_plot.pdf}
\caption{PSNR results for Wdas Cloud 2}
\label{fig:ap_psnr_Wdas Cloud 2_plot}
\end{figure}

\begin{figure}[h!]
\centering
\includegraphics[width=0.8\textwidth]{tables/psnr_Wdas_Cloud_3_plot.pdf}
\caption{PSNR results for Wdas Cloud 3}
\label{fig:ap_psnr_Wdas Cloud 3_plot}
\end{figure}

\begin{figure}[h!]
\centering
\includegraphics[width=0.8\textwidth]{tables/psnr_Chair_plot.pdf}
\caption{PSNR results for Chair}
\label{fig:ap_psnr_Chair_plot}
\end{figure}

\begin{figure}[h!]
\centering
\includegraphics[width=0.8\textwidth]{tables/psnr_Drums_plot.pdf}
\caption{PSNR results for Drums}
\label{fig:ap_psnr_Drums_plot}
\end{figure}

\begin{figure}[h!]
\centering
\includegraphics[width=0.8\textwidth]{tables/psnr_Ficus_plot.pdf}
\caption{PSNR results for Ficus}
\label{fig:ap_psnr_Ficus_plot}
\end{figure}

\begin{figure}[h!]
\centering
\includegraphics[width=0.8\textwidth]{tables/psnr_Hotdog_plot.pdf}
\caption{PSNR results for Hotdog}
\label{fig:ap_psnr_Hotdog_plot}
\end{figure}

\begin{figure}[h!]
\centering
\includegraphics[width=0.8\textwidth]{tables/psnr_Lego_plot.pdf}
\caption{PSNR results for Lego}
\label{fig:ap_psnr_Lego_plot}
\end{figure}

\begin{figure}[h!]
\centering
\includegraphics[width=0.8\textwidth]{tables/psnr_Materials_plot.pdf}
\caption{PSNR results for Materials}
\label{fig:ap_psnr_Materials_plot}
\end{figure}

\begin{figure}[h!]
\centering
\includegraphics[width=0.8\textwidth]{tables/psnr_Mic_plot.pdf}
\caption{PSNR results for Mic}
\label{fig:ap_psnr_Mic_plot}
\end{figure}

\begin{figure}[h!]
\centering
\includegraphics[width=0.8\textwidth]{tables/psnr_Ship_plot.pdf}
\caption{PSNR results for Ship}
\label{fig:ap_psnr_Ship_plot}
\end{figure}

\subsection{SSIM}
\begin{longtable}[H]{llrrrrrr}
\toprule
& & \multicolumn{6}{c}{Number of Gaussians} \\
Scene & Algorithm & 4000 & 12000 & 36000 & 108000 & 324000 & 972000 \\
\midrule \endhead
Burning Ficus & 3DGS & 0.9431 & 0.9610 & 0.9761 & 0.9851 & 0.9895 & 0.9908 \\
 & 3DGS + STP & 0.9436 & 0.9613 & 0.9762 & 0.9857 & 0.9899 & 0.9909 \\
 & OTS & 0.9506 & 0.9674 & 0.9769 & 0.9821 & 0.9848 & 0.9855 \\
 & OTS + SAtn & 0.9512 & 0.9678 & 0.9776 & 0.9826 & 0.9851 & 0.9856 \\
 & 3DGS marcher & 0.9477 & 0.9674 & 0.9789 & 0.9851 & 0.9880 & 0.9885 \\
 & OTS marcher & 0.9496 & 0.9660 & 0.9766 & 0.9823 & 0.9848 & 0.9852 \\
Chair & 3DGS & 0.9198 & 0.9381 & 0.9547 & 0.9686 & 0.9774 & 0.9829 \\
 & 3DGS + STP & 0.9186 & 0.9398 & 0.9553 & 0.9687 & 0.9779 & 0.9831 \\
 & OTS & 0.9268 & 0.9462 & 0.9579 & 0.9694 & 0.9782 & 0.9829 \\
 & OTS + SAtn & 0.9270 & 0.9462 & 0.9581 & 0.9694 & 0.9780 & 0.9827 \\
 & 3DGS marcher & 0.9276 & 0.9438 & 0.9575 & 0.9694 & 0.9775 & 0.9826 \\
 & OTS marcher & 0.9290 & 0.9472 & 0.9586 & 0.9699 & 0.9781 & 0.9825 \\
Coloured Wdas & 3DGS & 0.9788 & 0.9810 & 0.9833 & 0.9849 & 0.9859 & 0.9862 \\
 & 3DGS + STP & 0.9786 & 0.9810 & 0.9832 & 0.9849 & 0.9860 & 0.9862 \\
 & OTS & 0.9800 & 0.9822 & 0.9841 & 0.9856 & 0.9865 & 0.9866 \\
 & OTS + SAtn & 0.9800 & 0.9824 & 0.9841 & 0.9857 & 0.9865 & 0.9866 \\
 & 3DGS marcher & 0.9787 & 0.9807 & 0.9827 & 0.9842 & 0.9852 & 0.9857 \\
 & OTS marcher & 0.9795 & 0.9815 & 0.9830 & 0.9848 & 0.9860 & 0.9865 \\
Drums & 3DGS & 0.8872 & 0.9108 & 0.9292 & 0.9410 & 0.9478 & 0.9512 \\
 & 3DGS + STP & 0.8878 & 0.9121 & 0.9298 & 0.9419 & 0.9483 & 0.9513 \\
 & OTS & 0.8992 & 0.9206 & 0.9332 & 0.9411 & 0.9463 & 0.9485 \\
 & OTS + SAtn & 0.8987 & 0.9200 & 0.9332 & 0.9414 & 0.9463 & 0.9486 \\
 & 3DGS marcher & 0.9029 & 0.9233 & 0.9369 & 0.9451 & 0.9494 & 0.9519 \\
 & OTS marcher & 0.9000 & 0.9211 & 0.9338 & 0.9418 & 0.9466 & 0.9486 \\
Explosion & 3DGS & 0.9515 & 0.9703 & 0.9829 & 0.9892 & 0.9918 & 0.9929 \\
 & 3DGS + STP & 0.9515 & 0.9701 & 0.9827 & 0.9890 & 0.9917 & 0.9929 \\
 & OTS & 0.9528 & 0.9724 & 0.9843 & 0.9901 & 0.9922 & 0.9929 \\
 & OTS + SAtn & 0.9535 & 0.9728 & 0.9846 & 0.9902 & 0.9923 & 0.9929 \\
 & 3DGS marcher & 0.9545 & 0.9732 & 0.9852 & 0.9905 & 0.9923 & 0.9925 \\
 & OTS marcher & 0.9549 & 0.9745 & 0.9863 & 0.9912 & 0.9926 & 0.9926 \\
Ficus & 3DGS & 0.9306 & 0.9534 & 0.9702 & 0.9816 & 0.9871 & 0.9890 \\
 & 3DGS + STP & 0.9309 & 0.9543 & 0.9715 & 0.9825 & 0.9878 & 0.9895 \\
 & OTS & 0.9442 & 0.9604 & 0.9705 & 0.9764 & 0.9799 & 0.9814 \\
 & OTS + SAtn & 0.9442 & 0.9613 & 0.9716 & 0.9770 & 0.9803 & 0.9815 \\
 & 3DGS marcher & 0.9463 & 0.9648 & 0.9758 & 0.9823 & 0.9855 & 0.9868 \\
 & OTS marcher & 0.9435 & 0.9608 & 0.9715 & 0.9778 & 0.9808 & 0.9818 \\
Hotdog & 3DGS & 0.9503 & 0.9610 & 0.9693 & 0.9754 & 0.9795 & 0.9818 \\
 & 3DGS + STP & 0.9512 & 0.9617 & 0.9698 & 0.9756 & 0.9792 & 0.9814 \\
 & OTS & 0.9545 & 0.9632 & 0.9703 & 0.9751 & 0.9786 & 0.9808 \\
 & OTS + SAtn & 0.9560 & 0.9641 & 0.9707 & 0.9753 & 0.9787 & 0.9809 \\
 & 3DGS marcher & 0.9564 & 0.9643 & 0.9708 & 0.9755 & 0.9790 & 0.9811 \\
 & OTS marcher & 0.9577 & 0.9654 & 0.9707 & 0.9752 & 0.9781 & 0.9799 \\
Lego & 3DGS & 0.8703 & 0.9029 & 0.9320 & 0.9558 & 0.9706 & 0.9778 \\
 & 3DGS + STP & 0.8739 & 0.9058 & 0.9355 & 0.9574 & 0.9711 & 0.9777 \\
 & OTS & 0.8839 & 0.9130 & 0.9373 & 0.9565 & 0.9695 & 0.9756 \\
 & OTS + SAtn & 0.8837 & 0.9129 & 0.9374 & 0.9561 & 0.9698 & 0.9759 \\
 & 3DGS marcher & 0.8898 & 0.9198 & 0.9454 & 0.9633 & 0.9741 & 0.9791 \\
 & OTS marcher & 0.8879 & 0.9182 & 0.9435 & 0.9607 & 0.9722 & 0.9770 \\
Materials & 3DGS & 0.8713 & 0.9019 & 0.9280 & 0.9448 & 0.9541 & 0.9589 \\
 & 3DGS + STP & 0.8705 & 0.9029 & 0.9287 & 0.9455 & 0.9545 & 0.9592 \\
 & OTS & 0.8913 & 0.9180 & 0.9380 & 0.9505 & 0.9581 & 0.9614 \\
 & OTS + SAtn & 0.8904 & 0.9180 & 0.9379 & 0.9513 & 0.9582 & 0.9615 \\
 & 3DGS marcher & 0.8887 & 0.9185 & 0.9377 & 0.9495 & 0.9562 & 0.9597 \\
 & OTS marcher & 0.8912 & 0.9193 & 0.9384 & 0.9519 & 0.9586 & 0.9616 \\
Mic & 3DGS & 0.9573 & 0.9687 & 0.9779 & 0.9839 & 0.9872 & 0.9886 \\
 & 3DGS + STP & 0.9578 & 0.9692 & 0.9784 & 0.9845 & 0.9877 & 0.9888 \\
 & OTS & 0.9636 & 0.9699 & 0.9775 & 0.9827 & 0.9868 & 0.9881 \\
 & OTS + SAtn & 0.9644 & 0.9696 & 0.9771 & 0.9830 & 0.9869 & 0.9882 \\
 & 3DGS marcher & 0.9650 & 0.9717 & 0.9802 & 0.9855 & 0.9879 & 0.9889 \\
 & OTS marcher & 0.9653 & 0.9708 & 0.9778 & 0.9839 & 0.9873 & 0.9886 \\
Ship & 3DGS & 0.8214 & 0.8419 & 0.8616 & 0.8769 & 0.8887 & 0.8939 \\
 & 3DGS + STP & 0.8199 & 0.8403 & 0.8609 & 0.8775 & 0.8886 & 0.8929 \\
 & OTS & 0.8353 & 0.8502 & 0.8653 & 0.8784 & 0.8876 & 0.8903 \\
 & OTS + SAtn & 0.8333 & 0.8507 & 0.8656 & 0.8787 & 0.8877 & 0.8906 \\
 & 3DGS marcher & 0.8377 & 0.8547 & 0.8685 & 0.8812 & 0.8895 & 0.8931 \\
 & OTS marcher & 0.8359 & 0.8545 & 0.8687 & 0.8802 & 0.8874 & 0.8917 \\
Wdas Cloud 1 & 3DGS & 0.9951 & 0.9951 & 0.9951 & 0.9951 & 0.9951 & 0.9950 \\
 & 3DGS + STP & 0.9951 & 0.9951 & 0.9951 & 0.9951 & 0.9951 & 0.9950 \\
 & OTS & 0.9953 & 0.9954 & 0.9954 & 0.9954 & 0.9954 & 0.9952 \\
 & OTS + SAtn & 0.9953 & 0.9954 & 0.9954 & 0.9954 & 0.9954 & 0.9952 \\
 & 3DGS marcher & 0.9951 & 0.9951 & 0.9951 & 0.9951 & 0.9950 & 0.9950 \\
 & OTS marcher & 0.9953 & 0.9954 & 0.9954 & 0.9954 & 0.9954 & 0.9953 \\
Wdas Cloud 2 & 3DGS & 0.9881 & 0.9885 & 0.9888 & 0.9890 & 0.9890 & 0.9889 \\
 & 3DGS + STP & 0.9882 & 0.9885 & 0.9888 & 0.9890 & 0.9890 & 0.9889 \\
 & OTS & 0.9888 & 0.9892 & 0.9895 & 0.9896 & 0.9896 & 0.9894 \\
 & OTS + SAtn & 0.9888 & 0.9892 & 0.9895 & 0.9896 & 0.9896 & 0.9894 \\
 & 3DGS marcher & 0.9882 & 0.9886 & 0.9888 & 0.9888 & 0.9887 & 0.9886 \\
 & OTS marcher & 0.9888 & 0.9892 & 0.9894 & 0.9896 & 0.9896 & 0.9895 \\
Wdas Cloud 3 & 3DGS & 0.9831 & 0.9845 & 0.9854 & 0.9861 & 0.9862 & 0.9860 \\
 & 3DGS + STP & 0.9832 & 0.9845 & 0.9854 & 0.9862 & 0.9864 & 0.9862 \\
 & OTS & 0.9847 & 0.9861 & 0.9868 & 0.9875 & 0.9875 & 0.9872 \\
 & OTS + SAtn & 0.9848 & 0.9861 & 0.9869 & 0.9875 & 0.9876 & 0.9872 \\
 & 3DGS marcher & 0.9833 & 0.9847 & 0.9854 & 0.9860 & 0.9860 & 0.9856 \\
 & OTS marcher & 0.9848 & 0.9860 & 0.9868 & 0.9875 & 0.9876 & 0.9872 \\
\bottomrule
\caption{SSIM results for all scenes}
\end{longtable}

\begin{figure}[h!]
\centering
\includegraphics[width=0.8\textwidth]{tables/ssim_Burning_Ficus_plot.pdf}
\caption{SSIM results for Burning Ficus}
\label{fig:ap_ssim_Burning Ficus_plot}
\end{figure}

\begin{figure}[h!]
\centering
\includegraphics[width=0.8\textwidth]{tables/ssim_Coloured_Wdas_plot.pdf}
\caption{SSIM results for Coloured Wdas}
\label{fig:ap_ssim_Coloured Wdas_plot}
\end{figure}

\begin{figure}[h!]
\centering
\includegraphics[width=0.8\textwidth]{tables/ssim_Explosion_plot.pdf}
\caption{SSIM results for Explosion}
\label{fig:ap_ssim_Explosion_plot}
\end{figure}

\begin{figure}[h!]
\centering
\includegraphics[width=0.8\textwidth]{tables/ssim_Wdas_Cloud_1_plot.pdf}
\caption{SSIM results for Wdas Cloud 1}
\label{fig:ap_ssim_Wdas Cloud 1_plot}
\end{figure}

\begin{figure}[h!]
\centering
\includegraphics[width=0.8\textwidth]{tables/ssim_Wdas_Cloud_2_plot.pdf}
\caption{SSIM results for Wdas Cloud 2}
\label{fig:ap_ssim_Wdas Cloud 2_plot}
\end{figure}

\begin{figure}[h!]
\centering
\includegraphics[width=0.8\textwidth]{tables/ssim_Wdas_Cloud_3_plot.pdf}
\caption{SSIM results for Wdas Cloud 3}
\label{fig:ap_ssim_Wdas Cloud 3_plot}
\end{figure}

\begin{figure}[h!]
\centering
\includegraphics[width=0.8\textwidth]{tables/ssim_Chair_plot.pdf}
\caption{SSIM results for Chair}
\label{fig:ap_ssim_Chair_plot}
\end{figure}

\begin{figure}[h!]
\centering
\includegraphics[width=0.8\textwidth]{tables/ssim_Drums_plot.pdf}
\caption{SSIM results for Drums}
\label{fig:ap_ssim_Drums_plot}
\end{figure}

\begin{figure}[h!]
\centering
\includegraphics[width=0.8\textwidth]{tables/ssim_Ficus_plot.pdf}
\caption{SSIM results for Ficus}
\label{fig:ap_ssim_Ficus_plot}
\end{figure}

\begin{figure}[h!]
\centering
\includegraphics[width=0.8\textwidth]{tables/ssim_Hotdog_plot.pdf}
\caption{SSIM results for Hotdog}
\label{fig:ap_ssim_Hotdog_plot}
\end{figure}

\begin{figure}[h!]
\centering
\includegraphics[width=0.8\textwidth]{tables/ssim_Lego_plot.pdf}
\caption{SSIM results for Lego}
\label{fig:ap_ssim_Lego_plot}
\end{figure}

\begin{figure}[h!]
\centering
\includegraphics[width=0.8\textwidth]{tables/ssim_Materials_plot.pdf}
\caption{SSIM results for Materials}
\label{fig:ap_ssim_Materials_plot}
\end{figure}

\begin{figure}[h!]
\centering
\includegraphics[width=0.8\textwidth]{tables/ssim_Mic_plot.pdf}
\caption{SSIM results for Mic}
\label{fig:ap_ssim_Mic_plot}
\end{figure}

\begin{figure}[h!]
\centering
\includegraphics[width=0.8\textwidth]{tables/ssim_Ship_plot.pdf}
\caption{SSIM results for Ship}
\label{fig:ap_ssim_Ship_plot}
\end{figure}

\subsection{LPIPS}
\begin{longtable}[H]{llrrrrrr}
\toprule
& & \multicolumn{6}{c}{Number of Gaussians} \\
Scene & Algorithm & 4000 & 12000 & 36000 & 108000 & 324000 & 972000 \\
\midrule \endhead
Burning Ficus & 3DGS & 0.1082 & 0.0805 & 0.0534 & 0.0336 & 0.0227 & 0.0187 \\
 & 3DGS + STP & 0.1052 & 0.0789 & 0.0522 & 0.0322 & 0.0222 & 0.0187 \\
 & OTS & 0.0920 & 0.0641 & 0.0456 & 0.0339 & 0.0274 & 0.0253 \\
 & OTS + SAtn & 0.0923 & 0.0654 & 0.0459 & 0.0341 & 0.0272 & 0.0253 \\
 & 3DGS marcher & 0.0954 & 0.0648 & 0.0443 & 0.0311 & 0.0238 & 0.0217 \\
 & OTS marcher & 0.0934 & 0.0669 & 0.0476 & 0.0347 & 0.0283 & 0.0268 \\
Chair & 3DGS & 0.0976 & 0.0780 & 0.0588 & 0.0407 & 0.0276 & 0.0195 \\
 & 3DGS + STP & 0.0980 & 0.0761 & 0.0583 & 0.0399 & 0.0265 & 0.0192 \\
 & OTS & 0.0860 & 0.0666 & 0.0530 & 0.0387 & 0.0252 & 0.0192 \\
 & OTS + SAtn & 0.0877 & 0.0681 & 0.0534 & 0.0384 & 0.0258 & 0.0196 \\
 & 3DGS marcher & 0.0861 & 0.0687 & 0.0542 & 0.0392 & 0.0285 & 0.0215 \\
 & OTS marcher & 0.0865 & 0.0671 & 0.0539 & 0.0387 & 0.0263 & 0.0206 \\
Coloured Wdas & 3DGS & 0.1184 & 0.1058 & 0.0900 & 0.0764 & 0.0662 & 0.0592 \\
 & 3DGS + STP & 0.1184 & 0.1058 & 0.0903 & 0.0769 & 0.0664 & 0.0592 \\
 & OTS & 0.1156 & 0.1008 & 0.0861 & 0.0731 & 0.0638 & 0.0572 \\
 & OTS + SAtn & 0.1152 & 0.1003 & 0.0860 & 0.0726 & 0.0637 & 0.0572 \\
 & 3DGS marcher & 0.1192 & 0.1078 & 0.0941 & 0.0813 & 0.0708 & 0.0628 \\
 & OTS marcher & 0.1179 & 0.1055 & 0.0929 & 0.0789 & 0.0685 & 0.0606 \\
Drums & 3DGS & 0.1386 & 0.1081 & 0.0841 & 0.0656 & 0.0533 & 0.0467 \\
 & 3DGS + STP & 0.1364 & 0.1056 & 0.0827 & 0.0633 & 0.0522 & 0.0466 \\
 & OTS & 0.1170 & 0.0895 & 0.0738 & 0.0630 & 0.0543 & 0.0506 \\
 & OTS + SAtn & 0.1204 & 0.0926 & 0.0764 & 0.0639 & 0.0548 & 0.0511 \\
 & 3DGS marcher & 0.1130 & 0.0885 & 0.0706 & 0.0576 & 0.0499 & 0.0452 \\
 & OTS marcher & 0.1192 & 0.0924 & 0.0761 & 0.0641 & 0.0551 & 0.0521 \\
Explosion & 3DGS & 0.1639 & 0.1148 & 0.0728 & 0.0476 & 0.0354 & 0.0298 \\
 & 3DGS + STP & 0.1614 & 0.1118 & 0.0709 & 0.0470 & 0.0352 & 0.0300 \\
 & OTS & 0.1613 & 0.1090 & 0.0684 & 0.0440 & 0.0333 & 0.0289 \\
 & OTS + SAtn & 0.1613 & 0.1085 & 0.0670 & 0.0435 & 0.0331 & 0.0287 \\
 & 3DGS marcher & 0.1596 & 0.1090 & 0.0668 & 0.0446 & 0.0352 & 0.0328 \\
 & OTS marcher & 0.1584 & 0.1038 & 0.0623 & 0.0410 & 0.0334 & 0.0316 \\
Ficus & 3DGS & 0.0740 & 0.0528 & 0.0341 & 0.0201 & 0.0134 & 0.0112 \\
 & 3DGS + STP & 0.0740 & 0.0517 & 0.0328 & 0.0193 & 0.0128 & 0.0109 \\
 & OTS & 0.0600 & 0.0423 & 0.0305 & 0.0242 & 0.0208 & 0.0203 \\
 & OTS + SAtn & 0.0598 & 0.0419 & 0.0303 & 0.0242 & 0.0209 & 0.0205 \\
 & 3DGS marcher & 0.0557 & 0.0378 & 0.0259 & 0.0188 & 0.0154 & 0.0145 \\
 & OTS marcher & 0.0598 & 0.0424 & 0.0308 & 0.0239 & 0.0210 & 0.0207 \\
Hotdog & 3DGS & 0.0840 & 0.0661 & 0.0521 & 0.0406 & 0.0316 & 0.0261 \\
 & 3DGS + STP & 0.0817 & 0.0641 & 0.0505 & 0.0396 & 0.0317 & 0.0266 \\
 & OTS & 0.0745 & 0.0607 & 0.0490 & 0.0398 & 0.0324 & 0.0274 \\
 & OTS + SAtn & 0.0730 & 0.0596 & 0.0486 & 0.0399 & 0.0323 & 0.0276 \\
 & 3DGS marcher & 0.0709 & 0.0588 & 0.0481 & 0.0398 & 0.0330 & 0.0291 \\
 & OTS marcher & 0.0705 & 0.0575 & 0.0487 & 0.0405 & 0.0339 & 0.0306 \\
Lego & 3DGS & 0.1674 & 0.1315 & 0.0953 & 0.0609 & 0.0372 & 0.0246 \\
 & 3DGS + STP & 0.1650 & 0.1303 & 0.0920 & 0.0601 & 0.0369 & 0.0251 \\
 & OTS & 0.1463 & 0.1142 & 0.0841 & 0.0573 & 0.0364 & 0.0264 \\
 & OTS + SAtn & 0.1494 & 0.1167 & 0.0855 & 0.0591 & 0.0367 & 0.0265 \\
 & 3DGS marcher & 0.1440 & 0.1100 & 0.0746 & 0.0488 & 0.0315 & 0.0243 \\
 & OTS marcher & 0.1489 & 0.1148 & 0.0809 & 0.0541 & 0.0340 & 0.0254 \\
Materials & 3DGS & 0.1463 & 0.1124 & 0.0832 & 0.0613 & 0.0476 & 0.0398 \\
 & 3DGS + STP & 0.1463 & 0.1105 & 0.0809 & 0.0589 & 0.0461 & 0.0388 \\
 & OTS & 0.1290 & 0.0965 & 0.0724 & 0.0557 & 0.0452 & 0.0398 \\
 & OTS + SAtn & 0.1294 & 0.0975 & 0.0730 & 0.0550 & 0.0450 & 0.0394 \\
 & 3DGS marcher & 0.1277 & 0.0944 & 0.0708 & 0.0547 & 0.0445 & 0.0390 \\
 & OTS marcher & 0.1276 & 0.0967 & 0.0738 & 0.0552 & 0.0454 & 0.0401 \\
Mic & 3DGS & 0.0560 & 0.0400 & 0.0252 & 0.0161 & 0.0112 & 0.0099 \\
 & 3DGS + STP & 0.0553 & 0.0387 & 0.0243 & 0.0148 & 0.0106 & 0.0096 \\
 & OTS & 0.0423 & 0.0325 & 0.0231 & 0.0159 & 0.0107 & 0.0101 \\
 & OTS + SAtn & 0.0414 & 0.0324 & 0.0235 & 0.0155 & 0.0106 & 0.0100 \\
 & 3DGS marcher & 0.0415 & 0.0307 & 0.0203 & 0.0132 & 0.0109 & 0.0107 \\
 & OTS marcher & 0.0406 & 0.0311 & 0.0224 & 0.0145 & 0.0104 & 0.0099 \\
Ship & 3DGS & 0.2395 & 0.2073 & 0.1750 & 0.1472 & 0.1263 & 0.1154 \\
 & 3DGS + STP & 0.2385 & 0.2072 & 0.1753 & 0.1463 & 0.1263 & 0.1160 \\
 & OTS & 0.2183 & 0.1941 & 0.1711 & 0.1495 & 0.1317 & 0.1214 \\
 & OTS + SAtn & 0.2227 & 0.1959 & 0.1718 & 0.1498 & 0.1319 & 0.1219 \\
 & 3DGS marcher & 0.2178 & 0.1907 & 0.1668 & 0.1446 & 0.1275 & 0.1179 \\
 & OTS marcher & 0.2202 & 0.1928 & 0.1690 & 0.1486 & 0.1329 & 0.1229 \\
Wdas Cloud 1 & 3DGS & 0.0741 & 0.0723 & 0.0703 & 0.0680 & 0.0670 & 0.0675 \\
 & 3DGS + STP & 0.0741 & 0.0720 & 0.0705 & 0.0681 & 0.0673 & 0.0675 \\
 & OTS & 0.0734 & 0.0709 & 0.0690 & 0.0651 & 0.0623 & 0.0611 \\
 & OTS + SAtn & 0.0733 & 0.0708 & 0.0689 & 0.0650 & 0.0623 & 0.0608 \\
 & 3DGS marcher & 0.0739 & 0.0719 & 0.0695 & 0.0667 & 0.0646 & 0.0647 \\
 & OTS marcher & 0.0734 & 0.0709 & 0.0692 & 0.0649 & 0.0624 & 0.0610 \\
Wdas Cloud 2 & 3DGS & 0.1211 & 0.1149 & 0.1082 & 0.1006 & 0.0960 & 0.0948 \\
 & 3DGS + STP & 0.1214 & 0.1150 & 0.1084 & 0.1012 & 0.0963 & 0.0952 \\
 & OTS & 0.1174 & 0.1106 & 0.1040 & 0.0975 & 0.0917 & 0.0888 \\
 & OTS + SAtn & 0.1173 & 0.1107 & 0.1039 & 0.0975 & 0.0915 & 0.0885 \\
 & 3DGS marcher & 0.1215 & 0.1144 & 0.1080 & 0.1007 & 0.0955 & 0.0928 \\
 & OTS marcher & 0.1176 & 0.1109 & 0.1044 & 0.0971 & 0.0914 & 0.0883 \\
Wdas Cloud 3 & 3DGS & 0.1343 & 0.1252 & 0.1171 & 0.1064 & 0.0971 & 0.0902 \\
 & 3DGS + STP & 0.1349 & 0.1253 & 0.1173 & 0.1067 & 0.0972 & 0.0904 \\
 & OTS & 0.1280 & 0.1170 & 0.1083 & 0.0974 & 0.0891 & 0.0832 \\
 & OTS + SAtn & 0.1282 & 0.1171 & 0.1079 & 0.0969 & 0.0889 & 0.0832 \\
 & 3DGS marcher & 0.1341 & 0.1239 & 0.1168 & 0.1072 & 0.0983 & 0.0905 \\
 & OTS marcher & 0.1282 & 0.1176 & 0.1087 & 0.0982 & 0.0898 & 0.0831 \\
\bottomrule
\caption{LPIPS results for all scenes}
\end{longtable}

\begin{figure}[h!]
\centering
\includegraphics[width=0.8\textwidth]{tables/lpips_Burning_Ficus_plot.pdf}
\caption{LPIPS results for Burning Ficus}
\label{fig:ap_lpips_Burning Ficus_plot}
\end{figure}

\begin{figure}[h!]
\centering
\includegraphics[width=0.8\textwidth]{tables/lpips_Coloured_Wdas_plot.pdf}
\caption{LPIPS results for Coloured Wdas}
\label{fig:ap_lpips_Coloured Wdas_plot}
\end{figure}

\begin{figure}[h!]
\centering
\includegraphics[width=0.8\textwidth]{tables/lpips_Explosion_plot.pdf}
\caption{LPIPS results for Explosion}
\label{fig:ap_lpips_Explosion_plot}
\end{figure}

\begin{figure}[h!]
\centering
\includegraphics[width=0.8\textwidth]{tables/lpips_Wdas_Cloud_1_plot.pdf}
\caption{LPIPS results for Wdas Cloud 1}
\label{fig:ap_lpips_Wdas Cloud 1_plot}
\end{figure}

\begin{figure}[h!]
\centering
\includegraphics[width=0.8\textwidth]{tables/lpips_Wdas_Cloud_2_plot.pdf}
\caption{LPIPS results for Wdas Cloud 2}
\label{fig:ap_lpips_Wdas Cloud 2_plot}
\end{figure}

\begin{figure}[h!]
\centering
\includegraphics[width=0.8\textwidth]{tables/lpips_Wdas_Cloud_3_plot.pdf}
\caption{LPIPS results for Wdas Cloud 3}
\label{fig:ap_lpips_Wdas Cloud 3_plot}
\end{figure}

\begin{figure}[h!]
\centering
\includegraphics[width=0.8\textwidth]{tables/lpips_Chair_plot.pdf}
\caption{LPIPS results for Chair}
\label{fig:ap_lpips_Chair_plot}
\end{figure}

\begin{figure}[h!]
\centering
\includegraphics[width=0.8\textwidth]{tables/lpips_Drums_plot.pdf}
\caption{LPIPS results for Drums}
\label{fig:ap_lpips_Drums_plot}
\end{figure}

\begin{figure}[h!]
\centering
\includegraphics[width=0.8\textwidth]{tables/lpips_Ficus_plot.pdf}
\caption{LPIPS results for Ficus}
\label{fig:ap_lpips_Ficus_plot}
\end{figure}

\begin{figure}[h!]
\centering
\includegraphics[width=0.8\textwidth]{tables/lpips_Hotdog_plot.pdf}
\caption{LPIPS results for Hotdog}
\label{fig:ap_lpips_Hotdog_plot}
\end{figure}

\begin{figure}[h!]
\centering
\includegraphics[width=0.8\textwidth]{tables/lpips_Lego_plot.pdf}
\caption{LPIPS results for Lego}
\label{fig:ap_lpips_Lego_plot}
\end{figure}

\begin{figure}[h!]
\centering
\includegraphics[width=0.8\textwidth]{tables/lpips_Materials_plot.pdf}
\caption{LPIPS results for Materials}
\label{fig:ap_lpips_Materials_plot}
\end{figure}

\begin{figure}[h!]
\centering
\includegraphics[width=0.8\textwidth]{tables/lpips_Mic_plot.pdf}
\caption{LPIPS results for Mic}
\label{fig:ap_lpips_Mic_plot}
\end{figure}

\begin{figure}[h!]
\centering
\includegraphics[width=0.8\textwidth]{tables/lpips_Ship_plot.pdf}
\caption{LPIPS results for Ship}
\label{fig:ap_lpips_Ship_plot}
\end{figure}


\end{document}

