\section{Related Work}\label{sec:related_work}
In the following, we structure and discuss related work.

\textbf{Learned Cost Models (\lcms).}
As discussed in \Cref{sec:introduction}, existing works on \lcms do not evaluate against query optimization tasks.
Only a few evaluations exist, that in contrast to our evaluation lack either the consideration of diverse query optimization tasks or analyze only a limited number of learned approaches.
For example, an early analysis \cite{wu2013} also compares traditional to ML-based cost models. 
However, only simple ML approaches were examined.
Moreover, \cite{leis_how_2015} analyzes the quality of cardinality estimates and cost models of traditional query optimizers, but they do not evaluate learned approaches.
Another line of work evaluates individual aspects, such as the impact of different query plan representation methods and featurizations on cost estimation \cite{zhao2024, chang2024}. 

\textbf{Learned Cardinalities (LCE).}
The idea of learning cardinalities has been widely studied \cite{kipf2019, hilprecht2020deepdb, naru2019, yang2020} and often evaluated.
\cite{wang2021} showed that LCE approaches are often more accurate than traditional methods but come with high training and inference efforts.
\cite{sun2021} revisits learned cardinalities and proposes a unified design space.
However, neither study evaluates the effects of query optimization.
Different is \cite{kim2022}, which showed that the overall runtime using LCE approaches can significantly reduce against PostgreSQL in many cases.
Another recent work \cite{reiner2023} showed that a similar bias occurs for LCE as we show in our work for \lcms, as they typically rely on training samples that are provided by the PostgreSQL optimizer and thus near-optimal.
They address this bias by proposing a novel geometric LCE approach that requires fewer training samples.
However, these approaches cannot be easily transferred to \lcms.

\textbf{Learned Query Optimization (LQO).}
The idea of LQO is to directly predict an optimal query execution plan or execution hints given a SQL query without the use of a cost model \cite{marcus_bao_2021, yang_balsa_2022, chen2023, zhu2023}.
While these approaches improve the overall query performance, the generality of these results is limited, as shown by \cite{lehmann2024}.
Their work found that PostgreSQL still outperforms recent LQO approaches in many cases, especially for the end-to-end execution time, including inference and plan selection.
Similar to our work, they point out that learned database components often do not behave as expected and highlight biased evaluation strategies.
However, they focus on reinforcement learning approaches and the effect of different sampling strategies for obtaining splits of the training and test data.