\begin{abstract}
Traditionally, query optimizers rely on cost models to choose the best execution plan from several candidates, making precise cost estimates critical for efficient query execution. 
In recent years, cost models based on machine learning have been proposed to overcome the weaknesses of traditional cost models. 
While these models have been shown to provide better prediction accuracy, only limited efforts have been made to investigate how well \lcms actually perform in query optimization and how they affect overall query performance. 
In this paper, we address this by a systematic study evaluating \lcms on three of the core query optimization tasks: \emph{join ordering}, \emph{access path selection}, and \emph{physical operator selection}. 
In our study, we compare seven state-of-the-art \lcms to a traditional cost model and, surprisingly, find that the traditional model often still outperforms \lcms in these tasks.
We conclude by highlighting major takeaways and recommendations to guide future research toward making \lcms more effective for query optimization.
\end{abstract}
\maketitle
