\documentclass{article}

% if you need to pass options to natbib, use, e.g.:
    \PassOptionsToPackage{numbers, compress}{natbib}
% before loading neurips_2024


% ready for submission
% \usepackage{neurips_2024}


% to compile a preprint version, e.g., for submission to arXiv, add add the
% [preprint] option:
    \usepackage[preprint]{neurips_2024}


% to compile a camera-ready version, add the [final] option, e.g.:
% \usepackage[final]{neurips_2024}


% to avoid loading the natbib package, add option nonatbib:
%    \usepackage[nonatbib]{neurips_2024}


\usepackage[utf8]{inputenc} % allow utf-8 input
\usepackage[T1]{fontenc}    % use 8-bit T1 fonts
\usepackage{hyperref}       % hyperlinks
\usepackage{url}            % simple URL typesetting
\usepackage{booktabs}       % professional-quality tables
\usepackage{amsfonts}       % blackboard math symbols
\usepackage{nicefrac}       % compact symbols for 1/2, etc.
\usepackage{microtype}      % microtypography
\usepackage{xcolor}         % colors
\usepackage{multirow}
\usepackage{amsmath}
\usepackage{graphicx}
\usepackage{colortbl}
\usepackage{pifont}
\usepackage{caption}


\title{Ask Patients with Patience: Enabling LLMs for Human-Centric Medical Dialogue with Grounded Reasoning}


% The \author macro works with any number of authors. There are two commands
% used to separate the names and addresses of multiple authors: \And and \AND.
%
% Using \And between authors leaves it to LaTeX to determine where to break the
% lines. Using \AND forces a line break at that point. So, if LaTeX puts 3 of 4
% authors names on the first line, and the last on the second line, try using
% \AND instead of \And before the third author name.

\author{Jiayuan Zhu \\
University of Oxford\\
\texttt{jiayuan.zhu@ieee.org} \\
\And
Junde Wu  \thanks{corresponding author} \\
University of Oxford\\
\texttt{jundewu@ieee.org} \\
}


% \author{%
%   David S.~Hippocampus\thanks{Use footnote for providing further information
%     about author (webpage, alternative address)---\emph{not} for acknowledging
%     funding agencies.} \\
%   Department of Computer Science\\
%   Cranberry-Lemon University\\
%   Pittsburgh, PA 15213 \\
%   \texttt{hippo@cs.cranberry-lemon.edu} \\
%   % examples of more authors
%   % \And
%   % Coauthor \\
%   % Affiliation \\
%   % Address \\
%   % \texttt{email} \\
%   % \AND
%   % Coauthor \\
%   % Affiliation \\
%   % Address \\
%   % \texttt{email} \\
%   % \And
%   % Coauthor \\
%   % Affiliation \\
%   % Address \\
%   % \texttt{email} \\
%   % \And
%   % Coauthor \\
%   % Affiliation \\
%   % Address \\
%   % \texttt{email} \\
% }


\begin{document}


\maketitle


\begin{abstract}
Accurate and efficient diagnosis in online medical consultations remains a challenge for current large language models. These models often rely on single-turn interactions and lack the ability to refine their predictions through follow-up questions. Additionally, their responses frequently contain complex medical terminology, making them less accessible to non-medical users and creating barriers to effective communication. In this paper, we introduce Ask Patients with Patience (APP), the first multi-turn dialogue that enables LLMs to iteratively refine diagnoses based on grounded reasoning. By integrating medical guidelines and entropy minimization, APP improves both diagnostic accuracy and efficiency. Furthermore, it features human-centric communication that bridges the gap between user comprehension and medical terminology, significantly enhancing user accessibility and engagement. We evaluated APP using a subset of the ReMeDi dataset, comparing it with single-turn and traditional multi-turn LLM baselines. APP achieved higher similarity scores in diagnosis predictions, demonstrating better alignment with ground truth diagnoses. Entropy analysis showed that APP reduces diagnostic uncertainty more rapidly across iterations, increasing confidence in its predictions. APP also excels in user accessibility and empathy, further bridging the gap between complex medical language and user understanding. Code will be released at: \url{https://github.com/SuperMedIntel/AskPatients}.

\end{abstract} \label{abstract}


%%%%%%%%%%%%%%%%%%%%%%%%%%%%%%%%%%%%%%%%%%%%%%%%%%
\section{Introduction}
People often face prolonged waiting times to consult a general practitioner (GP) for medical inquiries in traditional healthcare settings. For example, in Australia, the average waiting time for a GP consultation is approximately four days \cite{swami2018hours}. Such delays can be frustrating and potentially detrimental, particularly for individuals with urgent health concerns. Online medical consultation (OMC) \cite{al2015online, kessler2023online} has emerged as a viable alternative, especially in regions with limited healthcare resources. By leveraging remote platforms, OMC enhances accessibility and provides a more timely way for individuals to seek medical advice, reducing the burden on traditional healthcare systems.

Large language models (LLMs), such as the GPT \cite{achiam2023gpt, brown2020language, ouyang2022training, radford2018improving, radford2019language} %and LLaMA \cite{touvron2023llama, touvron2023llama2, dubey2024llama} 
series, have significantly advanced artificial intelligence. In the medical domain, LLMs have been widely applied to tasks such as encoding medical knowledge \cite{singhal2023large, vaid2023using}, retrieving relevant clinical texts \cite{wu2024medical, xiong2024benchmarking}, generating medical reports \cite{peng2023study, tie2024personalized}, and supporting clinical decision-making through reasoning \cite{lievin2024can, lucas2024reasoning}. Notably, models such as GPT-4 with Medprompt \cite{nori2023can} and Med-Gemini-L 1.0 \cite{saab2024capabilities} have achieved expert-level performance on benchmarks like MedQA (USMLE) \cite{jin2021disease}, surpassing human experts in structured evaluations. %These advancements suggest their potential to support OMC by providing automated clinical insights. 


%-------------------------------------------------------------------------
\begin{figure*}[hbt!]
    \centering
    \includegraphics[width=0.98\linewidth]{fig1_overview.png} %width=0.75\linewidth
    \caption{(a) Existing LLMs follow a single-turn diagnostic approach, generating multiple possible diseases without asking follow-up questions. (b) While LLMs can be instructed to ask follow-up questions, they often overwhelm users with excessive inquiries, potentially disrupting the dialogue and reducing engagement. (c) Our human-centric multi-turn dialogue with grounded reasoning approach, APP, achieves accurate and efficient diagnoses by structuring follow-up questions in a logical sequence, guided by medical protocols and entropy minimization. It also incorporates user-friendly features, such as easily accessible questions to encourage engagement and a friendly tone to reduce user pressure and anxiety. \textcolor[RGB]{78, 149, 217}{Blue} represents user-described symptoms, \textcolor[RGB]{255, 133, 21}{Orange} indicates user questions, \textcolor[RGB]{240, 14, 77}{Red} highlights the diagnosis, and \textcolor[RGB]{184, 82, 255}{Purple} shows human-friendly features.}
    \label{fig:overflow}
\vspace{-10pt}
\end{figure*}
%-------------------------------------------------------------------------

Despite advancements, LLMs still face significant challenges in clinical applications, especially in real-world OMC settings \cite{goh2024large, hager2024evaluation}. Currently, LLMs follow a single-turn diagnostic approach, generating disease probabilities based solely on the user's initial input without follow-up questions. However, accurate diagnosis requires detailed information, which users often struggle to provide in one attempt. This poor inquiry quality limits diagnostic effectiveness regardless of the model’s capability \cite{liu2025exploring}. While LLMs can be instructed to ask follow-up questions, they often overwhelm users with excessive or irrelevant inquiries, potentially disrupting the dialogue and reducing engagement \cite{huang2024does}. To improve diagnosis accuracy, follow-up questions should be asked one at a time and guided by grounded medical guidelines to ensure relevance. To enhance efficiency, question order should be prioritized based on user input and optimized using entropy minimization to gather critical information effectively. Therefore, a multi-turn OMC system that adapts and optimizes  follow-up questioning becomes essential for improving both diagnostic accuracy and efficiency.

Another critical challenge in integrating LLMs into OMC is the gap between users’ understanding and the medical information required for accurate diagnosis. Many individuals struggle with medical terminology and may find diagnostic questions confusing without additional explanation. This can lead to frustration and miscommunication, as people often lack the knowledge to describe their symptoms accurately or even fail to recognize %certain physiological abnormalities as clinically relevant. 
clinically relevant abnormalities. For instance, an individual with lactose intolerance may report general discomfort without realizing it is triggered by dairy consumption. These challenges not only limit the accuracy of LLM-based diagnoses but can also leave users feeling misunderstood. To address this, an effective system must incorporate human-centric features. User-friendly language and guided questioning is helpful for users to express their symptoms more clearly. By fostering an intuitive and empathetic dialogue, such a system can bridge the communication gap, improve user accessibility and engagement, ultimately enhancing diagnosis accuracy.


In this paper, we propose \textbf{A}sk \textbf{P}atients with \textbf{P}atience (\textbf{APP}), a novel approach to enable LLMs for human-centric medical dialogue with grounded reasoning . By structuring follow-up questions in a logical sequence based on medical guidelines and entropy minimization, the system enhances diagnostic accuracy and efficiency. Our method also provides a user-friendly experience by bridging the gap between user understanding and medical terminology. It presents medical questions in human understandable language while enabling LLMs to elicit user symptoms more effectively, improving user accessibility and engagement. In summary, our key contributions are as follows:

\begin{itemize}
\item We introduce APP, the first human-centric multi-turn dialogue that enables LLMs to iteratively refine diagnoses based on grounded reasoning.

\item APP optimizes question sequencing by integrating medical guidelines and entropy minimization, enhancing both diagnostic accuracy and efficiency.

\item APP features human-centric communication that bridges the gap between user comprehension and medical terminology, significantly improving user accessibility and engagement.

\item APP demonstrates strong performance from both clinical and user perspectives, achieving SOTA diagnostic accuracy and efficiency while maintaining a high-quality and empathetic user experience.
\end{itemize}




%-------------------------------------------------------------------------
\begin{figure*}[hbt!]
    \centering
    \includegraphics[width=0.98\linewidth]{fig2_workflow.png} %width=0.75\linewidth
    \caption{APP Workflow. The system maps dialogue $S_t$ to standardized medical knowledge based on MSD Manual \cite{ noauthor_msd_nodate-1, noauthor_msd_nodate}. It then iteratively updates disease probabilities and selects optimal follow-up questions using entropy minimization.}
    \label{fig:workflow}
\vspace{-10pt}
\end{figure*}
%-------------------------------------------------------------------------
%%%%%%%%%%%%%%%%%%%%%%%%%%%%%%%%%%%%%%%%


%%%%%%%%%%%%%%%%%%%%%%%%%%%%%%%%%%%%%%%%
% \section{Related Work}

%%%%%%%%%%%%%%%%%%%%%%%%%%%%%%%%%%%%%%%%
\section{Methodology}
\subsection{Framework Overview}
Traditional single-turn diagnostic models often suffer from incomplete and imprecise user input, as people may fail to provide all relevant information in a single interaction. Additionally, existing LLMs lack the ability to iteratively refine diagnoses based on grounded reasoning. To address these limitations, we propose APP, a human-centric multi-turn diagnostic framework that dynamically queries users to enhance diagnostic accuracy based on medical guidelines.

Formally, let $S_{1}$ represents the initial conversation between the user and APP, where APP asks the first question $q_1$ and the user provides the corresponding response $r_1$. This interaction captures the initial user-provided information, where each $s_1^n \in \{q_1, r_1\}$ corresponds to an observed symptom, event, or relevant medical history. Due to limited user input, typically, $n<3$. The objective of APP is to determine the most probable diagnosis $d^{*} \in D$, where $D=\{d_i\}_{i=1}^{I}$ represents the set of all possible diseases under consideration. The system is designed to minimize the number of interactions required to accurately infer $d^*$, ensuring both diagnostic accuracy and efficiency. 

Let $S_t$ denotes the dialogue between the user and APP after $t$ iterations: $S_t=\{(q_1,r_1), (q_2,r_2), \dots, (q_t, r_t)\}$. At each iteration $t$,  the probability distribution $P_t(D)$ is updated based on $S_t$, and a question pool $Q_{t+1}$ is generated based on medical guidance to ensure relevance. The system selects the optimal follow-up question $q_{t+1}^*$ that maximizes information gain for the next iteration, ensuring efficient refinement. After receiving the user response $r_{t+1}$, the probability distribution is iteratively updated from $P_t(D)$ to $P_{t+1}(D)$ using the Bayes' Theorem and Law of Total Probability, thereby continuously refining the diagnosis.

To ensure the relevance and reliability of follow-up questions, APP leverages the MSD Manual \cite{ noauthor_msd_nodate-1, noauthor_msd_nodate} as its primary knowledge source, incorporating both professional and consumer versions. The professional version offers structured clinical definitions, diagnostic criteria, and treatment guidelines, ensuring medical precision. Meanwhile, the consumer version presents simplified medical concepts, improving accessibility for general users. By integrating both, APP remains grounded in authoritative medical knowledge while ensuring interpretability for non-expert users.

Specifically, APP involves following steps (Figure \ref{fig:workflow}): mapping initial information to MSD (Section \ref{sec:initial_map_msd}); updating potential diagnosis probability (Section \ref{sec:potential_diagnosis_probability}); determining follow-up question (Section \ref{sec:question}); and incorporating human-centric communication (Section~\ref{sec:human_friendly}).


\subsection{Initial information mapping into MSD}
\label{sec:initial_map_msd}
The initial APP-user dialogue, represented as $S_1=\{q_1, r_1\}$, is often limited and imprecise, as users may use non-standard terminology or provide vague descriptions that do not directly align with clinical definitions. We assume that $S_1$ reflects the most crucial and obvious symptoms, making it a natural starting point for diagnosis. Mapping this information to the standardized symptoms in the MSD Manual not only streamlines the diagnostic process but also aligns with clinical practices, where clinicians begin with the most prominent symptom to guide decision-making. 

To standardize and enhance reliability, the initial user-reported information $S_1$ is mapped to the MSD-defined symptom set $S_{MSD}$. During this process, key points for each symptom, retrieved from the MSD webpage, support the alignment between  $S_1$ and  $S_{MSD}$. This refinement improves accuracy, particularly when user descriptions do not directly correspond to predefined MSD symptoms. APP ensures a comprehensive representation by mapping $S_1$ to one symptom in the professional and one in the consumer symptom list of the MSD Manual: $S_{MSD} = \{S_{prof}, S_{cons}\}$. 



\subsection{Potential Diagnosis Probability}
\label{sec:potential_diagnosis_probability}
Given $S_{MSD}$, we can access the detailed symptom page, which provides information on causes, pathophysiology, and etiology of the symptom. In most cases, the page includes an additional table listing potential causes, descriptions, and further clinical comments. While explicit disease probabilities are not provided, qualitative indicators such as ``most common causes'' and ``less likely causes'' are available, which can be leveraged to estimate the likelihood of different conditions. We represent this set of relevant medical information as $\Gamma(S_{MSD})$. 

This reliable medical knowledge, combined with the current available dialogue $S_t$, serves as the foundation for generating the potential disease probability distribution:
\begin{multline}
P_t(D \mid \Gamma(S_{MSD}), S_t) = 
\{P_t(d_i \mid \Gamma(S_{MSD}), S_t) \mid d_i \in D, \sum_{i=1}^{I} P_t(d_i \mid \Gamma(S_{MSD}), S_t) = 1\}
\end{multline}
where $P_t(d_i\mid \Gamma(S_{MSD}), S_t)$ \footnote{For brevity, $P_t(D \mid \Gamma(S_{MSD}), S_t)$ and $P_t(d_i\mid \Gamma(S_{MSD}), S_t)$ are referred to $P_t(D)$ and $P_t(d_i)$, respectively.} represents the estimated probability of disease $d_i$ at iteration $t$, given the medical knowledge from $\Gamma(S_{MSD})$ and the cumulative dialogue $S_t$. 



\subsection{Follow-up Question}
\label{sec:question}
At each iteration $t$, APP generates a question pool $Q_{t+1}$, leveraging structured guidance from the MSD Manual. Relevant content, denoted as $\Upsilon(S_{MSD})$, is extracted from sections such as ``Diagnosis'' and ``What a doctor does'', ensuring that each generated question is both clinically reliable and symptom-specific. The set of candidate diagnostic questions are represented as: $Q_{t+1} = \{q_1, q_2, \dots, q_{K}\}$, %K \leq 5$ 
where $K$ is the maximum number of questions considered per iteration. For each candidate question $q_{k} \in Q_{t+1}$,  a set of plausible responses are generated based on current dialogue $S_t$. The set of responses for question $q_k$ is denoted as $R_{k} = \{r_{k}^{1}, r_{k}^{2}, \dots, r_{k}^{L}\}$, %2 \leq L \leq 5$, 
where $R_{k}$ represents the possible responses for question $q_k$, and $L$ is the number of generated responses.

Once the possible responses are generated, APP updates the disease probability distribution using Bayesian inference. For each disease $d_i \in D$, the conditional probability of receiving a specific response $r_{k}^{l}$ is computed as $P(r_{k}^{l} \mid \Gamma(d_i))$, where $\Gamma(d_i)$ represents the relevant medical information for disease $d_i$ extracted from MSD Manual. Using Bayes' Theorem, the joint probability of observing both the response $r_{k}^{l}$ and the disease $d_i$ is calculated as: 
\begin{equation}
    P(r_{k}^{l}, d_i) = P(r_{k}^{l} \mid \Gamma(d_i)) \cdot P_t(d_i)
\end{equation}
Applying the law of total probability, the posterior probability of each disease $d_i$ after receiving the responses to question $q_k$ is updated as:
\begin{equation}
    P(d_i \mid {q_k}) = \frac{\sum_{l=1}^{L}P(r_{k}^{l}, d_i)}{\sum_{j=1}^I \sum_{l=1}^{L}P(r_{k}^{l}, d_j)}
\end{equation}

To select the optimal follow-up question $q_{t+1}^*$ for next iteration, APP evaluates the expected entropy for each candidate question $q_k$:
\begin{equation}
    H_{q_k} = -\sum_{i=1}^I P(d_i \mid q_k) \cdot logP(d_i\mid q_k)
\end{equation}
The follow-up question is then selected by minimizing entropy, ensuring that the question yields the greatest information gain: 
\begin{equation}
    q_{t+1}^* = \arg\min_{q_{k}\in Q_{t+1}} H_{q_k}
\end{equation}
After asking the optimal question $q_{t+1}^*$, the user's response $r_{t+1}$ is incorporated into the dialogue, forming $S_{t+1}$. By repeatedly generating the potential diagnosis probability $P_{t+1}(D)$ and determining the optimal follow-up question, APP reaches a optimal diagnosis $d^*$, ensuring both accuracy and efficiency in the diagnostic workflow. 


\subsection{Human-Centric Communication}
\label{sec:human_friendly}
To make the diagnostic process more accessible for individuals without a medical background, APP simplifies complex medical terminology and symptom descriptions. When asking each optimal question $q_t^*$, Dr.Know uses clear, easy-to-understand language to reduce the chance of users missing critical details, ensuring effective communication and minimizing misunderstandings.

Individuals may not always recognize or articulate abnormal behaviors or symptoms from a clinical perspective. To address this, APP guides users with contextual hints that prompt them to recall relevant information they might otherwise overlook. For example, instead of asking a broad question like \textit{``Have you eaten anything unusual?''}, the system offers specific cues such as \textit{``Have you consumed foods like milk or beverages like soda (e.g., Coke)?''} This approach helps users remember details that could be related to their symptoms, improving the accuracy of the information provided.

Even with simplified yes/no questions, users may struggle with medical terminology or subtle differences in symptom descriptions. To mitigate this, APP formulates specific, descriptive questions. For instance, rather than asking \textit{``Do you feel dizzy?''}, the system refines the inquiry to: \textit{``Are you experiencing a feeling of losing balance, or does it seem like your surroundings are spinning or moving, even when everything is still?''} This ensures users can accurately identify and describe their symptoms, leading to more precise and efficient communication.




\section{Experiment}
\subsection{Dataset}
To evaluate the performance of our proposed approach, APP, we use a subset of the ReMeDi \cite{yan2022remedi} dataset. The ReMeDi dataset consists of natural multi-turn conversations between doctors and patients, collected from ChunYuYiSheng \cite{noauthor_chunyu}, a Chinese online medical community. This ensures that the dialogues reflect realistic, natural interactions, capturing the inherent variability and complexity of user-provided information. We use ReMeDi-base, which originally contained 1,557 labeled dialogues, as the foundation of our dataset. In this dataset, doctors' responses are annotated with seven different action labels: ``Informing'', ``Inquiring'', ``Chitchat'', ``QA'', ``Recommendation'', ``Diagnosis'', and ``Others''. For our study, we extracted dialogues that exclusively contain the ``Diagnosis'' label, resulting in 329 real-world, multi-turn diagnostic conversations between doctors and patients. At the first stage, we randomly selected 70 dialogues, covering 58 distinct diseases across 15 specialties, such as Orthopedics (e.g. osteoarthritis), Gynecology (e.g. polycystic ovary syndrome), Dermatology (e.g. androgenetic alopecia).


%-------------------------------------------------------------------------
\begin{figure*}[hbt!]
    \centering
    \includegraphics[width=0.98\linewidth]{fig3_case_study.png} %width=0.75\linewidth
    \caption{An APP case study of human-centric multi-turn dialogue based on medical guildlines. The ground truth is Diffuse Otitis Externa, where our diagnosis is Otitis Externa. \textcolor[RGB]{78, 149, 217}{Blue} represents user-described symptoms, \textcolor[RGB]{255, 133, 21}{Orange} indicates patient questions, \textcolor[RGB]{240, 14, 77}{Red} highlights the diagnosis, and \textcolor[RGB]{184, 82, 255}{Purple} shows human-friendly features.}
    \label{fig:case_study}
\vspace{-10pt}
\end{figure*}
%-------------------------------------------------------------------------


\subsection{Experimental Setup}
Since the ground truth is derived from a Chinese online medical community, we first use GPT-4o to translate disease names into English. As the questions asked by APP may differ from those in the ReMeDi-base, this could lead to variations in patient responses. To generate natural patient responses in our dialogues, we use the original conversations as a foundation and prompt DeepSeek-v3 to summarize the patient’s personality traits, symptoms, intentions, background etc. The patient is simulated to respond to APP based on this information. To further enhance the natural flow of patient response, we also prompt to include details like ``adding daily life elements that fit the character's personality and background.'' We primarily use DeepSeek-v3 for internal processes, applying GPT-4o only for generating English sentence outputs.

\subsection{Evaluation Matrix}
\subsubsection{Similarity}
Since both the ground truth and predictions are medical terms in text format, translation imperfections may lead to discrepancies in term mapping. To address this, we generate embeddings of the medical text using OpenAIEmbeddings \cite{noauthor_openai_nodate}. These embeddings convert the text into numerical representations that capture the semantic meaning of medical terms, enabling accurate similarity comparisons. We then apply cosine similarity to compare the embeddings and calculate a similarity score. This approach allows us to effectively assess the alignment between our predictions and the ground truth.

\subsubsection{Entropy}
Given the current probability distribution of potential diseases $P_t(D)$, we aim for the system to increase confidence in certain diagnoses and rule out less likely conditions through multi-turn dialogue. We use entropy as a quantitative measure to assess diagnostic confidence\footnote{Figure \ref{fig:overflow}(a) presents potential diseases without indicating their likelihood, while (c) shows how APP distinguishes between more and less probable diseases.}. The entropy at iteration $t$ is calculated as: $H_t = - \sum_{i=1}^{I}P_t(d_i)\cdot logP_t(d_i)$, where $P_t(d_i)$ is the probability of disease $d_i$ and $I$ is the total number of possible diseases at iteration $t$. A reduction in entropy over successive dialogue turns indicates increased diagnostic confidence. %This reflects the system’s ability to assign higher probabilities to the correct diagnosis while reducing the likelihood of incorrect ones, leading to more accurate predictions.

\subsubsection{Human-Centric}
\textbf{Accessibility Score}
To assess whether the questions posed by APP are easy for users without medical background to understand, we evaluate the language accessibility using GPT-4o. The model rate the clarity and simplicity of the doctor's language on a scale from 0 to 1.

\textbf{Empathy Score}
This score reflects the level of empathy demonstrated by the APP during the conversation with the user. The degree of empathy is rated on a scale from 0 to 1 using GPT-4o, with higher values indicating more empathetic communication.

\textbf{Relevant Response Rate}
In some cases, users may ask the doctor follow-up questions. Ideally, the doctor should address these concerns before proceeding with the next question. This metric evaluates whether the doctor’s response directly answers the user’s question, with GPT-4o assigning a score of 0 or 1.



\subsection{Similarity Analysis versus Baseline across Iterations}
To evaluate the diagnostic accuracy of APP, we compared it with two baseline models: a Single-Turn LLM and a Multi-Turn LLM. In the Single-Turn LLM setup, the model was provided with the user's initial input and asked to generate the most likely disease without any follow-up questions. For the Multi-Turn LLM, the model was explicitly prompted to ask one follow-up question, with user responses simulated using the same strategy as APP’s patient simulator.

Table \ref{table:similarity} presents the similarity scores between the predicted diagnoses and the ground truth across multiple iterations. The Single-Turn LLM achieved a score of 83.7\%, reflecting its limited diagnostic accuracy without follow-up questioning. The Multi-Turn LLM showed slight improvements over iterations, peaking at 84.5\% by the third iteration before declining slightly to 83.8\% at the sixth iteration. In contrast, APP consistently outperformed both baselines. Starting with a similarity score of 84.2\% in the first iteration, it steadily improved, reaching 85.7\% by the sixth iteration. This demonstrates APP' ability to effectively leverage multi-turn dialogues to refine diagnoses, highlighting its efficiency in integrating user feedback. Overall, these results show that APP not only improves diagnostic accuracy but also maintains consistent performance across iterations, outperforming standard LLM approaches in interactive medical diagnosis.


%-------------------------------------------------------------------------
\begin{figure}[ht]
    \begin{minipage}{0.5\linewidth}
        \centering
        \captionof{table}{Similarity scores between predicted diagnoses and ground truth across iterations for Single-Turn LLM, Multi-Turn LLM, and APP. APP demonstrates consistently higher similarity, improving with each iteration, highlighting the effectiveness of its multi-turn dialogue approach in refining diagnoses.}
        \resizebox{0.95\textwidth}{!}{
        \begin{tabular}{c|cccccc}
    Iteration       & 1      & 2      & 3      & 4      & 5      & 6               \\ \hline
    Single-Turn LLM & 83.7\% &   -     &     -   &   -     &     -   &     -            \\
    Multi-Turn LLM  &   83.7\%     &   84.4\%     &    84.5\%    &   84.0\%     &   84.2\%     &    83.8\%             \\ 
    APP        & 84.2\% & 85.5\% & 85.4\% & 85.5\% & 85.6\% & \textbf{85.7\%}
    \end{tabular}}\label{table:similarity}
    % \vspace{-15pt}
    \end{minipage}\,
    \hfill
\begin{minipage}{0.5\linewidth}
            \centering
            \includegraphics[scale=0.25]{entropy.png}
            \vspace{-5pt}
             \caption{Entropy comparison between APP and the Multi-turn Baseline across iterations. APP consistently demonstrates a sharper decrease in entropy, indicating increased diagnostic confidence and efficiency in refining predictions through iterative dialogues.}
            \label{fig:entropy}
\end{minipage}
\end{figure}


%-------------------------------------------------------------------------


\subsection{Entropy Analysis across Iterations}
Figure \ref{fig:entropy} illustrates the evolution of diagnostic confidence over multiple dialogue iterations by comparing the entropy values of APP and the Multi-turn Baseline. From the initial iteration, APP exhibits lower diagnostic uncertainty, with an entropy of 2.85, compared to 3.29 for the Multi-turn Baseline. This suggests that even before follow-up interactions, APP is better at handling initial user input, providing more confident predictions.

As iterations progress, APP shows a sharper and more consistent decline in entropy. By the third iteration, APP reduces its entropy to 2.51, while the Multi-turn Baseline remains higher at 3.23, reflecting APP' superior ability to refine potential diagnoses. After six iterations, APP reaches its lowest entropy of 1.95, indicating a high degree of certainty in its final predictions. In contrast, the Multi-turn Baseline retains an entropy of 3.18, suggesting persistent uncertainty. This consistent reduction in entropy highlights the effectiveness of APP’ multi-turn dialogue framework in refining diagnostic accuracy and enhancing confidence, outperforming traditional multi-turn approaches in managing diagnostic uncertainty.



\subsection{Human-Centric Analysis with Real-world Dialogue}
Our multi-turn dialogue system, APP, shows notable performance in user accessibility, question empathy and relevance compared to original dialogues collected from real-world online platform. In terms of accessibility, APP achieved an average score of 0.91, outperforming the original dialogues, which scored 0.85. This highlights the system’s ability to present medical information in a way that is easier for users to understand. For empathy, APP scored 0.66, compared to 0.50 in the original dialogues. This indicates that our system encourages more compassionate and human-centric dialogues, helping to reduce user anxiety and create a better overall experience. Regarding relevance, APP maintained a high score of 0.79, closely aligning with the original dialogues' score of 0.82. Overall, these results demonstrate that APP enhances human-friendly communication, leading to better user understanding and engagement.


\section{Conclusion}
In this study, we introduce APP, a human-centric multi-turn dialogue designed to enhance the diagnostic capabilities based on grounded reasoning. By integrating structured medical guidelines and entropy minimization, APP effectively improves diagnostic accuracy and efficiency through iterative user interactions. Our findings demonstrate that adaptive follow-up questioning significantly enhances both diagnostic accuracy and efficiency compared to single-turn and traditional multi-turn LLM baselines. APP consistently achieves higher similarity scores in diagnosis prediction, indicating better alignment with ground truth diagnoses. Furthermore, entropy analysis shows that APP rapidly reduces diagnostic uncertainty over successive iterations, reflecting increased confidence in its predictions. Importantly, APP excels in user accessibility and empathy, bridging the gap between complex medical terminology and human comprehension. This fosters greater user engagement and trust. Overall, APP enhances diagnostic accuracy through multi-turn dialogue guided by medical guidelines while fostering a human-centric communication environment.

















\clearpage
\bibliography{neurips_2024}
\bibliographystyle{abbrv}
\medskip







%%%%%%%%%%%%%%%%%%%%%%%%%%%%%%%%%%%%%%%%%%%%%%%%%%%%%%%%%%%%

% \newpage
% \section*{NeurIPS Paper Checklist}

% %%% BEGIN INSTRUCTIONS %%%
% The checklist is designed to encourage best practices for responsible machine learning research, addressing issues of reproducibility, transparency, research ethics, and societal impact. Do not remove the checklist: {\bf The papers not including the checklist will be desk rejected.} The checklist should follow the references and follow the (optional) supplemental material.  The checklist does NOT count towards the page
% limit. 

% Please read the checklist guidelines carefully for information on how to answer these questions. For each question in the checklist:
% \begin{itemize}
%     \item You should answer \answerYes{}, \answerNo{}, or \answerNA{}.
%     \item \answerNA{} means either that the question is Not Applicable for that particular paper or the relevant information is Not Available.
%     \item Please provide a short (1–2 sentence) justification right after your answer (even for NA). 
%    % \item {\bf The papers not including the checklist will be desk rejected.}
% \end{itemize}

% {\bf The checklist answers are an integral part of your paper submission.} They are visible to the reviewers, area chairs, senior area chairs, and ethics reviewers. You will be asked to also include it (after eventual revisions) with the final version of your paper, and its final version will be published with the paper.

% The reviewers of your paper will be asked to use the checklist as one of the factors in their evaluation. While "\answerYes{}" is generally preferable to "\answerNo{}", it is perfectly acceptable to answer "\answerNo{}" provided a proper justification is given (e.g., "error bars are not reported because it would be too computationally expensive" or "we were unable to find the license for the dataset we used"). In general, answering "\answerNo{}" or "\answerNA{}" is not grounds for rejection. While the questions are phrased in a binary way, we acknowledge that the true answer is often more nuanced, so please just use your best judgment and write a justification to elaborate. All supporting evidence can appear either in the main paper or the supplemental material, provided in appendix. If you answer \answerYes{} to a question, in the justification please point to the section(s) where related material for the question can be found.

% IMPORTANT, please:
% \begin{itemize}
%     \item {\bf Delete this instruction block, but keep the section heading ``NeurIPS paper checklist"},
%     \item  {\bf Keep the checklist subsection headings, questions/answers and guidelines below.}
%     \item {\bf Do not modify the questions and only use the provided macros for your answers}.
% \end{itemize} 
 

% %%% END INSTRUCTIONS %%%


% \begin{enumerate}

% \item {\bf Claims}
%     \item[] Question: Do the main claims made in the abstract and introduction accurately reflect the paper's contributions and scope?
%     \item[] Answer: \answerYes{} % Replace by \answerYes{}, \answerNo{}, or \answerNA{}.
%     \item[] Justification: \justificationTODO{}
%     \item[] Guidelines:
%     \begin{itemize}
%         \item The answer NA means that the abstract and introduction do not include the claims made in the paper.
%         \item The abstract and/or introduction should clearly state the claims made, including the contributions made in the paper and important assumptions and limitations. A No or NA answer to this question will not be perceived well by the reviewers. 
%         \item The claims made should match theoretical and experimental results, and reflect how much the results can be expected to generalize to other settings. 
%         \item It is fine to include aspirational goals as motivation as long as it is clear that these goals are not attained by the paper. 
%     \end{itemize}

% \item {\bf Limitations}
%     \item[] Question: Does the paper discuss the limitations of the work performed by the authors?
%     \item[] Answer: \answerTODO{} % Replace by \answerYes{}, \answerNo{}, or \answerNA{}.
%     \item[] Justification: \justificationTODO{}
%     \item[] Guidelines:
%     \begin{itemize}
%         \item The answer NA means that the paper has no limitation while the answer No means that the paper has limitations, but those are not discussed in the paper. 
%         \item The authors are encouraged to create a separate "Limitations" section in their paper.
%         \item The paper should point out any strong assumptions and how robust the results are to violations of these assumptions (e.g., independence assumptions, noiseless settings, model well-specification, asymptotic approximations only holding locally). The authors should reflect on how these assumptions might be violated in practice and what the implications would be.
%         \item The authors should reflect on the scope of the claims made, e.g., if the approach was only tested on a few datasets or with a few runs. In general, empirical results often depend on implicit assumptions, which should be articulated.
%         \item The authors should reflect on the factors that influence the performance of the approach. For example, a facial recognition algorithm may perform poorly when image resolution is low or images are taken in low lighting. Or a speech-to-text system might not be used reliably to provide closed captions for online lectures because it fails to handle technical jargon.
%         \item The authors should discuss the computational efficiency of the proposed algorithms and how they scale with dataset size.
%         \item If applicable, the authors should discuss possible limitations of their approach to address problems of privacy and fairness.
%         \item While the authors might fear that complete honesty about limitations might be used by reviewers as grounds for rejection, a worse outcome might be that reviewers discover limitations that aren't acknowledged in the paper. The authors should use their best judgment and recognize that individual actions in favor of transparency play an important role in developing norms that preserve the integrity of the community. Reviewers will be specifically instructed to not penalize honesty concerning limitations.
%     \end{itemize}

% \item {\bf Theory Assumptions and Proofs}
%     \item[] Question: For each theoretical result, does the paper provide the full set of assumptions and a complete (and correct) proof?
%     \item[] Answer: \answerTODO{} % Replace by \answerYes{}, \answerNo{}, or \answerNA{}.
%     \item[] Justification: \justificationTODO{}
%     \item[] Guidelines:
%     \begin{itemize}
%         \item The answer NA means that the paper does not include theoretical results. 
%         \item All the theorems, formulas, and proofs in the paper should be numbered and cross-referenced.
%         \item All assumptions should be clearly stated or referenced in the statement of any theorems.
%         \item The proofs can either appear in the main paper or the supplemental material, but if they appear in the supplemental material, the authors are encouraged to provide a short proof sketch to provide intuition. 
%         \item Inversely, any informal proof provided in the core of the paper should be complemented by formal proofs provided in appendix or supplemental material.
%         \item Theorems and Lemmas that the proof relies upon should be properly referenced. 
%     \end{itemize}

%     \item {\bf Experimental Result Reproducibility}
%     \item[] Question: Does the paper fully disclose all the information needed to reproduce the main experimental results of the paper to the extent that it affects the main claims and/or conclusions of the paper (regardless of whether the code and data are provided or not)?
%     \item[] Answer: \answerTODO{} % Replace by \answerYes{}, \answerNo{}, or \answerNA{}.
%     \item[] Justification: \justificationTODO{}
%     \item[] Guidelines:
%     \begin{itemize}
%         \item The answer NA means that the paper does not include experiments.
%         \item If the paper includes experiments, a No answer to this question will not be perceived well by the reviewers: Making the paper reproducible is important, regardless of whether the code and data are provided or not.
%         \item If the contribution is a dataset and/or model, the authors should describe the steps taken to make their results reproducible or verifiable. 
%         \item Depending on the contribution, reproducibility can be accomplished in various ways. For example, if the contribution is a novel architecture, describing the architecture fully might suffice, or if the contribution is a specific model and empirical evaluation, it may be necessary to either make it possible for others to replicate the model with the same dataset, or provide access to the model. In general. releasing code and data is often one good way to accomplish this, but reproducibility can also be provided via detailed instructions for how to replicate the results, access to a hosted model (e.g., in the case of a large language model), releasing of a model checkpoint, or other means that are appropriate to the research performed.
%         \item While NeurIPS does not require releasing code, the conference does require all submissions to provide some reasonable avenue for reproducibility, which may depend on the nature of the contribution. For example
%         \begin{enumerate}
%             \item If the contribution is primarily a new algorithm, the paper should make it clear how to reproduce that algorithm.
%             \item If the contribution is primarily a new model architecture, the paper should describe the architecture clearly and fully.
%             \item If the contribution is a new model (e.g., a large language model), then there should either be a way to access this model for reproducing the results or a way to reproduce the model (e.g., with an open-source dataset or instructions for how to construct the dataset).
%             \item We recognize that reproducibility may be tricky in some cases, in which case authors are welcome to describe the particular way they provide for reproducibility. In the case of closed-source models, it may be that access to the model is limited in some way (e.g., to registered users), but it should be possible for other researchers to have some path to reproducing or verifying the results.
%         \end{enumerate}
%     \end{itemize}


% \item {\bf Open access to data and code}
%     \item[] Question: Does the paper provide open access to the data and code, with sufficient instructions to faithfully reproduce the main experimental results, as described in supplemental material?
%     \item[] Answer: \answerTODO{} % Replace by \answerYes{}, \answerNo{}, or \answerNA{}.
%     \item[] Justification: \justificationTODO{}
%     \item[] Guidelines:
%     \begin{itemize}
%         \item The answer NA means that paper does not include experiments requiring code.
%         \item Please see the NeurIPS code and data submission guidelines (\url{https://nips.cc/public/guides/CodeSubmissionPolicy}) for more details.
%         \item While we encourage the release of code and data, we understand that this might not be possible, so “No” is an acceptable answer. Papers cannot be rejected simply for not including code, unless this is central to the contribution (e.g., for a new open-source benchmark).
%         \item The instructions should contain the exact command and environment needed to run to reproduce the results. See the NeurIPS code and data submission guidelines (\url{https://nips.cc/public/guides/CodeSubmissionPolicy}) for more details.
%         \item The authors should provide instructions on data access and preparation, including how to access the raw data, preprocessed data, intermediate data, and generated data, etc.
%         \item The authors should provide scripts to reproduce all experimental results for the new proposed method and baselines. If only a subset of experiments are reproducible, they should state which ones are omitted from the script and why.
%         \item At submission time, to preserve anonymity, the authors should release anonymized versions (if applicable).
%         \item Providing as much information as possible in supplemental material (appended to the paper) is recommended, but including URLs to data and code is permitted.
%     \end{itemize}


% \item {\bf Experimental Setting/Details}
%     \item[] Question: Does the paper specify all the training and test details (e.g., data splits, hyperparameters, how they were chosen, type of optimizer, etc.) necessary to understand the results?
%     \item[] Answer: \answerTODO{} % Replace by \answerYes{}, \answerNo{}, or \answerNA{}.
%     \item[] Justification: \justificationTODO{}
%     \item[] Guidelines:
%     \begin{itemize}
%         \item The answer NA means that the paper does not include experiments.
%         \item The experimental setting should be presented in the core of the paper to a level of detail that is necessary to appreciate the results and make sense of them.
%         \item The full details can be provided either with the code, in appendix, or as supplemental material.
%     \end{itemize}

% \item {\bf Experiment Statistical Significance}
%     \item[] Question: Does the paper report error bars suitably and correctly defined or other appropriate information about the statistical significance of the experiments?
%     \item[] Answer: \answerTODO{} % Replace by \answerYes{}, \answerNo{}, or \answerNA{}.
%     \item[] Justification: \justificationTODO{}
%     \item[] Guidelines:
%     \begin{itemize}
%         \item The answer NA means that the paper does not include experiments.
%         \item The authors should answer "Yes" if the results are accompanied by error bars, confidence intervals, or statistical significance tests, at least for the experiments that support the main claims of the paper.
%         \item The factors of variability that the error bars are capturing should be clearly stated (for example, train/test split, initialization, random drawing of some parameter, or overall run with given experimental conditions).
%         \item The method for calculating the error bars should be explained (closed form formula, call to a library function, bootstrap, etc.)
%         \item The assumptions made should be given (e.g., Normally distributed errors).
%         \item It should be clear whether the error bar is the standard deviation or the standard error of the mean.
%         \item It is OK to report 1-sigma error bars, but one should state it. The authors should preferably report a 2-sigma error bar than state that they have a 96\% CI, if the hypothesis of Normality of errors is not verified.
%         \item For asymmetric distributions, the authors should be careful not to show in tables or figures symmetric error bars that would yield results that are out of range (e.g. negative error rates).
%         \item If error bars are reported in tables or plots, The authors should explain in the text how they were calculated and reference the corresponding figures or tables in the text.
%     \end{itemize}

% \item {\bf Experiments Compute Resources}
%     \item[] Question: For each experiment, does the paper provide sufficient information on the computer resources (type of compute workers, memory, time of execution) needed to reproduce the experiments?
%     \item[] Answer: \answerTODO{} % Replace by \answerYes{}, \answerNo{}, or \answerNA{}.
%     \item[] Justification: \justificationTODO{}
%     \item[] Guidelines:
%     \begin{itemize}
%         \item The answer NA means that the paper does not include experiments.
%         \item The paper should indicate the type of compute workers CPU or GPU, internal cluster, or cloud provider, including relevant memory and storage.
%         \item The paper should provide the amount of compute required for each of the individual experimental runs as well as estimate the total compute. 
%         \item The paper should disclose whether the full research project required more compute than the experiments reported in the paper (e.g., preliminary or failed experiments that didn't make it into the paper). 
%     \end{itemize}
    
% \item {\bf Code Of Ethics}
%     \item[] Question: Does the research conducted in the paper conform, in every respect, with the NeurIPS Code of Ethics \url{https://neurips.cc/public/EthicsGuidelines}?
%     \item[] Answer: \answerTODO{} % Replace by \answerYes{}, \answerNo{}, or \answerNA{}.
%     \item[] Justification: \justificationTODO{}
%     \item[] Guidelines:
%     \begin{itemize}
%         \item The answer NA means that the authors have not reviewed the NeurIPS Code of Ethics.
%         \item If the authors answer No, they should explain the special circumstances that require a deviation from the Code of Ethics.
%         \item The authors should make sure to preserve anonymity (e.g., if there is a special consideration due to laws or regulations in their jurisdiction).
%     \end{itemize}


% \item {\bf Broader Impacts}
%     \item[] Question: Does the paper discuss both potential positive societal impacts and negative societal impacts of the work performed?
%     \item[] Answer: \answerTODO{} % Replace by \answerYes{}, \answerNo{}, or \answerNA{}.
%     \item[] Justification: \justificationTODO{}
%     \item[] Guidelines:
%     \begin{itemize}
%         \item The answer NA means that there is no societal impact of the work performed.
%         \item If the authors answer NA or No, they should explain why their work has no societal impact or why the paper does not address societal impact.
%         \item Examples of negative societal impacts include potential malicious or unintended uses (e.g., disinformation, generating fake profiles, surveillance), fairness considerations (e.g., deployment of technologies that could make decisions that unfairly impact specific groups), privacy considerations, and security considerations.
%         \item The conference expects that many papers will be foundational research and not tied to particular applications, let alone deployments. However, if there is a direct path to any negative applications, the authors should point it out. For example, it is legitimate to point out that an improvement in the quality of generative models could be used to generate deepfakes for disinformation. On the other hand, it is not needed to point out that a generic algorithm for optimizing neural networks could enable people to train models that generate Deepfakes faster.
%         \item The authors should consider possible harms that could arise when the technology is being used as intended and functioning correctly, harms that could arise when the technology is being used as intended but gives incorrect results, and harms following from (intentional or unintentional) misuse of the technology.
%         \item If there are negative societal impacts, the authors could also discuss possible mitigation strategies (e.g., gated release of models, providing defenses in addition to attacks, mechanisms for monitoring misuse, mechanisms to monitor how a system learns from feedback over time, improving the efficiency and accessibility of ML).
%     \end{itemize}
    
% \item {\bf Safeguards}
%     \item[] Question: Does the paper describe safeguards that have been put in place for responsible release of data or models that have a high risk for misuse (e.g., pretrained language models, image generators, or scraped datasets)?
%     \item[] Answer: \answerTODO{} % Replace by \answerYes{}, \answerNo{}, or \answerNA{}.
%     \item[] Justification: \justificationTODO{}
%     \item[] Guidelines:
%     \begin{itemize}
%         \item The answer NA means that the paper poses no such risks.
%         \item Released models that have a high risk for misuse or dual-use should be released with necessary safeguards to allow for controlled use of the model, for example by requiring that users adhere to usage guidelines or restrictions to access the model or implementing safety filters. 
%         \item Datasets that have been scraped from the Internet could pose safety risks. The authors should describe how they avoided releasing unsafe images.
%         \item We recognize that providing effective safeguards is challenging, and many papers do not require this, but we encourage authors to take this into account and make a best faith effort.
%     \end{itemize}

% \item {\bf Licenses for existing assets}
%     \item[] Question: Are the creators or original owners of assets (e.g., code, data, models), used in the paper, properly credited and are the license and terms of use explicitly mentioned and properly respected?
%     \item[] Answer: \answerTODO{} % Replace by \answerYes{}, \answerNo{}, or \answerNA{}.
%     \item[] Justification: \justificationTODO{}
%     \item[] Guidelines:
%     \begin{itemize}
%         \item The answer NA means that the paper does not use existing assets.
%         \item The authors should cite the original paper that produced the code package or dataset.
%         \item The authors should state which version of the asset is used and, if possible, include a URL.
%         \item The name of the license (e.g., CC-BY 4.0) should be included for each asset.
%         \item For scraped data from a particular source (e.g., website), the copyright and terms of service of that source should be provided.
%         \item If assets are released, the license, copyright information, and terms of use in the package should be provided. For popular datasets, \url{paperswithcode.com/datasets} has curated licenses for some datasets. Their licensing guide can help determine the license of a dataset.
%         \item For existing datasets that are re-packaged, both the original license and the license of the derived asset (if it has changed) should be provided.
%         \item If this information is not available online, the authors are encouraged to reach out to the asset's creators.
%     \end{itemize}

% \item {\bf New Assets}
%     \item[] Question: Are new assets introduced in the paper well documented and is the documentation provided alongside the assets?
%     \item[] Answer: \answerTODO{} % Replace by \answerYes{}, \answerNo{}, or \answerNA{}.
%     \item[] Justification: \justificationTODO{}
%     \item[] Guidelines:
%     \begin{itemize}
%         \item The answer NA means that the paper does not release new assets.
%         \item Researchers should communicate the details of the dataset/code/model as part of their submissions via structured templates. This includes details about training, license, limitations, etc. 
%         \item The paper should discuss whether and how consent was obtained from people whose asset is used.
%         \item At submission time, remember to anonymize your assets (if applicable). You can either create an anonymized URL or include an anonymized zip file.
%     \end{itemize}

% \item {\bf Crowdsourcing and Research with Human Subjects}
%     \item[] Question: For crowdsourcing experiments and research with human subjects, does the paper include the full text of instructions given to participants and screenshots, if applicable, as well as details about compensation (if any)? 
%     \item[] Answer: \answerTODO{} % Replace by \answerYes{}, \answerNo{}, or \answerNA{}.
%     \item[] Justification: \justificationTODO{}
%     \item[] Guidelines:
%     \begin{itemize}
%         \item The answer NA means that the paper does not involve crowdsourcing nor research with human subjects.
%         \item Including this information in the supplemental material is fine, but if the main contribution of the paper involves human subjects, then as much detail as possible should be included in the main paper. 
%         \item According to the NeurIPS Code of Ethics, workers involved in data collection, curation, or other labor should be paid at least the minimum wage in the country of the data collector. 
%     \end{itemize}

% \item {\bf Institutional Review Board (IRB) Approvals or Equivalent for Research with Human Subjects}
%     \item[] Question: Does the paper describe potential risks incurred by study participants, whether such risks were disclosed to the subjects, and whether Institutional Review Board (IRB) approvals (or an equivalent approval/review based on the requirements of your country or institution) were obtained?
%     \item[] Answer: \answerTODO{} % Replace by \answerYes{}, \answerNo{}, or \answerNA{}.
%     \item[] Justification: \justificationTODO{}
%     \item[] Guidelines:
%     \begin{itemize}
%         \item The answer NA means that the paper does not involve crowdsourcing nor research with human subjects.
%         \item Depending on the country in which research is conducted, IRB approval (or equivalent) may be required for any human subjects research. If you obtained IRB approval, you should clearly state this in the paper. 
%         \item We recognize that the procedures for this may vary significantly between institutions and locations, and we expect authors to adhere to the NeurIPS Code of Ethics and the guidelines for their institution. 
%         \item For initial submissions, do not include any information that would break anonymity (if applicable), such as the institution conducting the review.
%     \end{itemize}

% \end{enumerate}


\end{document}