\section{Related Works and Discussions}
\label{sec:related_works}

The proposed Orlicz-Sobolev transport (OST) generalizes GST~\citep{le2024generalized} for unbalanced measures supported on a graph (see Proposition~\ref{prop:relation_OST_GST}). We emphasize that unlike OT, it is nontrivial to incorporate mass constraints for GST, since as its root, the definition of GST is steamed from the Kantorovich duality of $1$-order Wasserstein, and it optimizes the critic function  with constraints in the graph-based Orlicz-Sobolev space. Therefore, it is essential to take a detour to consider EPT problem for unbalanced measures on a graph~\citep{le2023scalable}, then leverage~\citet{CM}'s observations to derive a corresponding standard complete OT problem, and incorporate back the Orlicz geometric structure for the proposed Orlicz-EPT. We further note that \citet{CM}'s observations may not be applicable for some other certain formulations of unbalanced optimal transport (UOT) such as those in~\citet{frogner2015learning, chizat2018unbalanced, sejourne2019sinkhorn}.


Moreover, Orlicz-EPT is formulated as a two-level optimization problem (Equation~\eqref{eq:OrliczEPT}) which leads to a high-computational cost and hinders its practical applications, similar to OW for the case of balanced measures. By leveraging~\citet{le2021ept}'s observations and exploiting graph structure, we propose novel regularization for critic function within the Orlicz-Sobolev space, and develop OST. Our key result is to show that one can simply solving a univariate optimization problem (Theorem~\ref{thm:OST_computation}) for OST computation, and make it more practical for applications. 



%%%%%%%%%%%%%%%%%%%%%%%%%%%%%%%%%%%%%%%%%%%%%%%%%%%%
%%%%%%%%%%%%%%%%%%%%%%%%%%%%%%%%%%%%%%%%%%%%%%%%%%%%