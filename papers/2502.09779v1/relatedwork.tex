\section{Related Works}
\subsection{Body composition analysis using CT}
Body composition plays a crucial role in influencing physical performance \citep{hernandez2024relationship, falsarella2015body}, metabolic health \citep{trouwborst2024body, kakinami2022body}, and disease outcomes \citep{rutten2016loss}. Imaging offers an objective, quantitative approach to its analysis through various techniques, including CT, magnetic resonance imaging (MRI), and ultrasonography \citep{hou2024enhanced, tan2024quantitative, sharafi2024quantitative, xu2024value}. Among all the modalities, CT offers high spatial resolution, faster acquisition times, and superior contrast between tissues \citep{zhang2021improving}, making it particularly suitable for assessing visceral and subcutaneous fat, skeletal muscle, and organ-specific fat deposits \citep{wathen2013vivo}. 

During body composition calculation on CT, several key metrics are frequently utilized, including muscle density, the VAT/SAT ratio, muscle area/volume, and the SMI. Muscle density in CT provides insights into muscle quality, which is linearly influenced by muscular fat content \citep{engelke2018quantitative}. A reduction in muscle density is often associated with increased fat infiltration within the muscle, known as myosteatosis \citep{chang2024prognostic}, which compromises muscle function and structural integrity. This reduction serves as a critical indicator of sarcopenia, a condition characterized by the progressive loss of skeletal muscle mass and strength, as well as frailty and diminished physical performance, particularly in aging populations \citep{cawthon2015assessment}. 

The VAT/SAT ratio, on the other hand, is a key metric for assessing metabolic risk \citep{kaess2012ratio, oh2017visceral}. While visceral adipose tissue (VAT) is strongly associated with metabolic disturbances and cardiovascular risk \citep{vasamsetti2023regulation}, its volume alone may reflect both overall fat mass and an individual's tendency to store fat viscerally \citep{kaess2012ratio}. In contrast, the VAT/SAT ratio offers a more precise assessment, as it accounts for the balance between visceral and subcutaneous fat, providing insight that is independent of total body fat percentage \citep{kaess2012ratio}.

Muscle area/volume and SMI are essential measurements of total muscle quantity and its proportionality to body size. These metrics provide critical information about an individual’s muscle reserves, which are vital for mobility, metabolic function, and overall health status \citep{chen2023really}. Studies highlight them as significant markers of nutritional status \citep{risch2022assessment}, which are crucial for recovery from illness, mortality, and treatment-related complications, such as the length of hospital stays and the rate of readmissions \citep{schuetz2021management, kaegi2021evaluation, guenter2021malnutrition}. Furthermore, they also serve as important factors in assessing metabolic health \citep{cruz2019sarcopenia, prado2014lean, martin2013cancer, dodds2015epidemiology}, as lower muscle mass is associated with insulin resistance and impaired glucose metabolism.

The collection of these metrics pictures the clear body condition of patients, showcasing a comprehensive overview of their muscle composition, fat distribution, and overall physiological status.

\subsection{Traditional methods for body composition segmentation}
Most early studies on body composition analysis rely on semi-automated threshold-based segmentation using predefined Hounsfield unit (HU) ranges to differentiate lean muscle mass from adipose tissue \citep{lee2017pixel, ji2022thresholds}. Despite its simplicity, threshold-based segmentation presents significant challenges due to the overlapping HU values between different tissue types, such as SAT and skin, as well as muscle and adjacent organs \citep{lee2017pixel}. The method is also highly susceptible to image noise \citep{sehgal2022ct, diwakar2020ct}, which can significantly compromise tissue classification accuracy, particularly in low-quality or artifact-prone scans. As a result, the method typically requires manual correction based on visual analysis by highly skilled radiologists and is impractical on large datasets due to the expense and time required.

To overcome these limitations, researchers have developed various advanced segmentation algorithms, including rule-based \citep{kamiya2009automated, kamiya2011automated}, clustering-based \citep{positano2009accurate, positano2004accurate, christ2011fuzzy}, and finite-element-method-based \citep{popuri2015body} approaches. Kamiya et al. proposed a rule-based expert system for segmenting the psoas major and rectus abdominis muscles from CT images, approximating muscle shapes with simple quadratic functions \citep{kamiya2009automated, kamiya2011automated}. Positano et al. utilize a fuzzy c-mean algorithm to make unsupervised classification of image pixels on MRI \citep{positano2009accurate, positano2004accurate}. Karteek and the team developed a novel FEM deformable model for muscle and fat segmentation from CT \citep{popuri2015body}.

However, these methods primarily focus on extracting specific muscle groups from CT or MRI scans and are unable to differentiate between visceral fat, subcutaneous fat, and intramuscular fat—an essential distinction in many body composition analysis tasks \citep{staley2019visceral, torres2013nutritional, iwase2016impact}. A potential approach to address these challenges is the use of deep learning-based segmentation algorithms.

\subsection{Deep learning-based models for body composition segmentation}
Deep learning-based segmentation has been proven to be a reliable technique in various clinical applications \citep{gu2024segmentanybone, dong2024segment, wasserthal2023totalsegmentator, mazurowski2023segment}. While networks offer high accuracy, reduce human labor, and provide greater generalizability compared to traditional segmentation algorithms, it is straightforward to apply deep learning-based segmentation algorithms for body composition analysis.

The majority of current deep learning-based segmentation models for body composition are still based on convolutional neural networks (CNNs) \citep{nowak2020fully}. U-Net and its variants are among the most widely used architectures in this domain, providing precise segmentation of body composition components such as skeletal muscle, SAT, and VAT \citep{paris2020body, weston2019automated}. However, these models are typically not publicly accessible and are often designed for specific tasks \citep{mai2023systematic}. While a few commercial models are available \citep{cespedes2020evaluation, mai2023systematic, lee2021deep}, they are often associated with high costs and limited customization options. TotalSegmentator \citep{wasserthal2023totalsegmentator}, a recently published general CT segmentation model based on nnU-Net, also supports muscle and fat segmentation. However, studies have shown that its performance in segmenting muscle, SAT, and VAT can be further improved, and its non-commercial license restricts broader usage. Therefore, there remains a significant need for publicly accessible, transparent, and generalizable segmentation models for body composition analysis.
% A methodological, model, or similar section often comes here.