\section{Related Work}
\label{sec:related_work}

\begin{figure*}[htb!]
    \centering 
    \includegraphics[width=1\linewidth]{figures/f1_4.png}
    \caption{The overall framework for generating the KMI dataset. The context data and dialogue history are originally in Korean but have been translated into English for the figure.} 
    \label{fig:framework}
\end{figure*}


\paragraph{Dialogue Generation Using LLMs}
Recent research has increasingly focused on using LLMs to generate dialogues for various applications.
\citet{kim2022prosocialdialog} proposed a human-machine collaborative framework to build a large-scale dialogue dataset for training conversational agents to handle problematic content appropriately.
\citet{chen2023places} used a small number of expert-written conversations as in-context examples to create synthetic multi-party conversations.
\citet{chen2023controllable} proposed LLM prompting methods to generate mixed-initiative dialogues.
\citet{kim2023soda} created a large-scale social dialogue dataset by distilling conversations from LLMs.
\citet{macina2023mathdial} paired human teachers with an LLM student to generate teacher-student tutoring dialogues grounded in math reasoning problems.



\paragraph{NLP Applications in MI}
MI was developed as a technique to assist individuals in resolving ambivalence and committing to change \citep{Miller_1983}, representing an evolution of client-centered therapy. 
Along with the advancement in NLP, considerable research efforts have been underway to apply NLP techniques in the field of MI. 
Several studies have been proposed to automatically classify a given utterance into one of the MI behavioral codes.
While earlier approaches utilize linguistic features \citep{perez2017predicting} and recurrent neural network architectures \citep{tanana2015recursive, xiao2016behavioral, cao2019observing, gibson2019multi}, recent approaches make use of pre-trained language models such as RoBERTa \citep{liu2019roberta, tavabi2021analysis, welivita2023boosting}. 
Some works adopt a multimodal approach, leveraging additional information such as speech features \citep{tavabi2020multimodal} or facial features \citep{nakano2022detecting}.



\citet{welivita2023boosting} demonstrated the potential of LLMs in boosting dialogues using the MI strategy. They observed that among the MI dataset curated from online platforms \citep{welivita2022curating}, 92.86\% of the advice given by peers fall into the \textit{Advise without permission} category, which is MI non-adherent. To make the dataset more MI-consistent, they fine-tuned BlenderBot \citep{roller2021recipes} and GPT-3 \citep{brown2020language} to rephrase these responses into more MI-adherent \textit{Advise with permission} responses. Although this work demonstrated the possibility of leveraging LLMs in MI, the impact of rephrasing remains marginal, considering that its capability is restricted to revising the manner of speech and has limitations in modifying the content itself.
In our work, we further exploit the generative ability of LLMs to generate the entire dataset from scratch.




\begin{table*}[hbt!]
    \begin{center}
    \scriptsize
    \setlength{\tabcolsep}{10pt}
    \renewcommand{\arraystretch}{1.2}
    \begin{tabular}{p{2.2cm}p{6cm}p{5.7cm}}
    \toprule
    \textbf{MI Label} & \textbf{Description}  & \textbf{Examples} \\ 
    \midrule
    1. Simple Reflection & Repetition, rephrasing, or paraphrasing of the speaker’s previous statement. & It sounds like you’re feeling worried. \\
    2. Complex Reflection & Repeating or rephrasing the previous statement of the speaker but adding substantial meaning/emphasis to it. & \parbox[t]{5.6cm}{\textit{Speaker}: Mostly, I would change for future generations.\\ \textit{Listener}: It sounds like you have a strong feeling of responsibility.} \\
    3. Open Question & Questions that allow a wide range of possible answers. & What is your take on that? \\
    4. Closed Question & Questions that can be answered with a yes/no response or a very restricted range of answers. & Do you think this is an advantage? \\
    5. Affirm & Encouraging the speaker by saying something positive or complimentary. & You should be proud of yourself for your past efforts. \\
    6. Give Information & Educating, providing feedback, or giving an opinion without advising. & Logging your cravings is important as cravings often lead to relapses. \\
    7. Advise & Making suggestions, offering solutions or possible actions. & We could try to brainstorm some ideas that might help. \\
    8. Other & Statements that are not classified under the above codes. & Hi there.
     \\
    \bottomrule
    \end{tabular}
    \end{center}
    \caption{MI labels derived from the MITI code. The descriptions and examples of each label are taken from \citet{welivita2022curating}.}
    \label{tab:mi_label}
\end{table*}