\section{Related Works}
In this work, we consider the problem of 6D object pose tracking problem, which has been widely studied as a visual problem____.
In particular, we focus on model-based tracking approaches, which assume access to the object's CAD model. 
While model-free approaches____ exist, they fall outside the scope of this work.
Visual pose tracking has achieved significant progress on established benchmarks, such as BOP____.
Despite these successes, deploying such systems in real-world robotic applications remains challenging, especially under scenarios with high occlusion and dynamic interactions, such as in-hand manipulation tasks.

To address these challenges, prior research has explored combining visual and tactile information to improve pose tracking robustness____.
These approaches leverage learning-based techniques to estimate object poses by fusing visuo-tactile inputs. 
However, these methods estimate poses on a per-frame basis, which lacks temporal coherence. 
Additionally, cross-embodiment and domain generalization remain significant hurdles, limiting their scalability and practicality for broad deployment.

More recent works aim to overcome some of these limitations. 
For example, ____ proposes an optimization-based approach that integrates tactile data with visual pose tracking using an ad-hoc slippage detector and velocity predictor. 
____ extend the model-free tracking frameworks BundleTrack____ and BundleSDF____ by combining visual and tactile point clouds within a pose graph optimization framework. 
However, these approaches are only validated on a single embodiment and suffer from computational inefficiencies____, which present challenges for real-time deployment in dynamic manipulation tasks.