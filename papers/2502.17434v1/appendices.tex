\onecolumn
\appendices

\begin{figure*}[t!]
    \centering
    \includegraphics[width=\textwidth]{Figures/dataset-proportion.png}
    \caption{
    \textbf{Distributions of embodiments and objects.}
    The novel gripper and object have fewer samples as they are only used for evaluation and not during training.
    }
    \label{fig:dataset-proportion}
\end{figure*}

\begin{figure*}
    \centering
    \includegraphics[width=\textwidth]{Figures/dataset.png}
    \caption{
    \textbf{Dataset distributions}.
    We view the occlusion rate, position, and rotation distribution of our data samples.
    }
    \label{fig:dataset-analysis}
\end{figure*}

\section{Dataset}
In Fig.~\ref{fig:dataset-analysis}, we provide a visualization of the dataset's distribution. 
The dataset focuses on in-hand object pose tracking and addresses challenges such as heavy occlusion during in-hand manipulation. 
Our visualization reveals that the positions and rotations of the data samples are well-distributed, following either normal or uniform distributions, ensuring a comprehensive evaluation of pose tracking performance.


\section{Experiments}

\begin{figure}
    \centering
    \includegraphics[width=\textwidth]{Figures/AUC-curve.png}
    \caption{
    \textbf{Accuracy-threshold curve~\cite{xiang_posecnn_2018} on our dataset.}
    V-HOP consistently demonstrates stronger or similar performance as FoundationPose (FP) under various thresholds.
    }
    \label{fig:enter-label}
\end{figure}

\begin{figure*}[t!]\centering
\noindent 
\begin{tabularx}{\textwidth}{c *{10}{>{\centering\arraybackslash}X}}

    \rotatebox[origin=c]{90}{FP} &
    \raisebox{-0.5\height}{\includegraphics[width=.95\textwidth]{Figures/real-qual/FoundationPose_004_sugar_box.png}} \\

    \rotatebox[origin=c]{90}{\textbf{VH6T}} &
    \raisebox{-0.5\height}{\includegraphics[width=.95\textwidth]{Figures/real-qual/Ours_004_sugar_box.png}} \\

    \vspace{0.1cm} \\
    
    \rotatebox[origin=c]{90}{FP} &
    \raisebox{-0.5\height}{\includegraphics[width=.95\textwidth]{Figures/real-qual/FoundationPose_005_tomato_soup_can.png}} \\

    \rotatebox[origin=c]{90}{\textbf{VH6T}} &
    \raisebox{-0.5\height}{\includegraphics[width=.95\textwidth]{Figures/real-qual/Ours_005_tomato_soup_can.png}} \\
    
\end{tabularx}
\caption{\textbf{Qualitative results of pose tracking sequences}. 
We perform qualitative comparisons on more objects.
Our results demonstrate that V-HOP consistently outperforms FP by a large margin.
}
\label{fig: real-qual-exp-appendix}
\end{figure*}

\begin{figure}[t!]
    \centering
    \includegraphics[width=\textwidth]{Figures/real-manipulation/handover-supp.png}
    \caption{
    \textbf{Bimanual handover experiments.}
    We perform bimanual handover experiments on more objects.
    }
    \label{fig:handover-supp}
\end{figure}
