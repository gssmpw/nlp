\section{Related Works}
In this work, we consider the problem of 6D object pose tracking problem, which has been widely studied as a visual problem~\cite{wen_se3-tracknet_2020, li_deepim_2018, wen_foundationpose_2024, deng_poserbpf_2021}.
In particular, we focus on model-based tracking approaches, which assume access to the object's CAD model. 
While model-free approaches~\cite{wen_bundletrack_2021, wen_bundlesdf_2023, suresh_neuralfeels_2024} exist, they fall outside the scope of this work.
Visual pose tracking has achieved significant progress on established benchmarks, such as BOP~\cite{hodan_bop_2024}.
Despite these successes, deploying such systems in real-world robotic applications remains challenging, especially under scenarios with high occlusion and dynamic interactions, such as in-hand manipulation tasks.

To address these challenges, prior research has explored combining visual and tactile information to improve pose tracking robustness~\cite{li_vihope_2023, suresh_neuralfeels_2024, dikhale_visuotactile_2022, wan_vint-6d_2024, rezazadeh_hierarchical_2023, tu_posefusion_2023, gao_-hand_2023, li_hypertaxel_2024}.
These approaches leverage learning-based techniques to estimate object poses by fusing visuo-tactile inputs. 
However, these methods estimate poses on a per-frame basis, which lacks temporal coherence. 
Additionally, cross-embodiment and domain generalization remain significant hurdles, limiting their scalability and practicality for broad deployment.

More recent works aim to overcome some of these limitations. 
For example, \citet{liu_enhancing_2024} proposes an optimization-based approach that integrates tactile data with visual pose tracking using an ad-hoc slippage detector and velocity predictor. 
\citet{suresh_neuralfeels_2024} extend the model-free tracking frameworks BundleTrack~\cite{wen_bundletrack_2021} and BundleSDF~\cite{wen_bundlesdf_2023} by combining visual and tactile point clouds within a pose graph optimization framework. 
However, these approaches are only validated on a single embodiment and suffer from computational inefficiencies~\cite{suresh_neuralfeels_2024}, which present challenges for real-time deployment in dynamic manipulation tasks.