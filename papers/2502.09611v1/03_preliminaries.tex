\section{Preliminaries}
We begin by introducing Continuous Normalizing Flow~\cite{chen2019neural} in Sec.~\ref{sec:probability_paths} and Flow Matching~\cite{lipman2022flow} in Sec.~\ref{sec:flow_matching}. This will motivate our approach, detailed in Sec.~\ref{sec:method}, which defines an
informative conditional prior distribution on a conditional flow model. 

\subsection{Continuous Normalizing Flows}
\label{sec:probability_paths}

A probability density function over a manifold $\gM$ is a continuous non-negative function $\rho: \gM\rightarrow\R_+$ such that $\int \rho(x)dx = 1$. We set $\gP$ to be the space of such probability densities on $\gM$. A \emph{probability path} $\rho_t: [0,1]\rightarrow\gP$ is a curve in probability space connecting two densities $\rho_0,\rho_1 \in \gP$ at endpoints $t=0, t=1$. A \emph{flow} $\psi_t: [0,1] \times \gM \to \gM$ is a time-dependent diffeomorphism defined to be the solution to the Ordinary Differential Equation (ODE):
\begin{equation}\label{eq_ode_cnf}
    \frac{d}{dt} \psi_t(x) = u_t \left(\psi_t(x) \right), \quad \psi_0(x) = x
\end{equation}
subject to initial conditions where $u_t: [0,1] \times \gM \to \gT\gM$ is a time-dependent smooth vector field on the collection of all tangent planes on the manifold  $\gT\gM$ (\emph{tangent bundle}). A flow $\psi_t$ is said to generate a probability path $\rho_t$ if it `pushes' $\rho_0$ forward to $\rho_1$ following the time-dependent vector field $u_t$. The path is denoted by:
\begin{equation}
    \rho_t = [\psi_t]_\#\rho_0 := \rho_0(\psi_{t}^{-1}(x))\det \Big|\frac{d\psi_{t}^{-1}}{dx}(x)\Big|
\end{equation}
where $\#$ is the standard push-forward operation. Previously, \citep{chen2019neural} proposed to model the flow $\psi_t$ implicitly by parameterizing the vector field $u_t$, to produce $\rho_t$, in a method called \emph{Continuous Normalizing Flows} (CNF).

\subsection{Flow Matching}
\label{sec:flow_matching}
Flow Matching (FM) \citep{lipman2022flow} is a simulation-free method for training CNFs that avoids likelihood computation during training, which can be expensive.  It does so by fitting a vector field $v_{t}^\theta$ with parameters $\theta$
and regressing vector fields $u_t$ that are known \emph{a priori} to generate a probability path $\rho_t \in \gP$ satisfying the boundary conditions:
\begin{equation}\label{eq:boundary_conditions}
    \rho_0=p, \quad \rho_1=q
\end{equation} 
Note that $u_t$ is generally intractable. However, a key insight of ~\cite{lipman2022flow}, is that this vector field can be constructed based on conditional vector fields $u_t(x|x_1)$ that generate conditional probability paths $\rho_t(x|x_1)$. The push-forward of the conditional flow $\psi_t(x|x_1)$, start at $\rho_t$ and concentrate the density around $x=x_1 \in \gM$ at $t=1$. Marginalizing over the target distribution $q$ recovers the unconditional probability path and unconditional vector field:
\begin{equation}
    \rho_t(x) = \int_{\gM} \rho_t(x|x_1)q(x_1)dx_1
\end{equation} 
\begin{equation}
    u_t(x) = \int_{\gM} u_t(x|x_1)\frac{\rho_t(x|x_1)q(x_1)}{\rho_t(x)}dx_1
\end{equation}

This vector field can be matched by a parameterized vector field $v_\theta$ using the $\gL_{\rm cfm}(\theta)$ objective:
\begin{equation}\label{eq_cfm_objective}
    \mathbb{E}_{t\sim \mathcal{U}(0,1), q(x_1), \rho_t(x | x_1)} \|v_\theta(t, x) - u_t(x | x_1)\|^2
\end{equation}
where $\|\cdot\|$ is a norm on $\gT\gM$.
One particular choice of a conditional probability path $\rho_t(x|x_1)$ is to use the flow corresponding the optimal transport displacement interpolant \citep{McCann1997ACP} between Gaussian distributions. Specifically, in the context of the conditional probability path, $\rho_0(x|x_1)$ is the standard Gaussian, a common convention in generative modeling, and $\rho_1(x|x_1)$ is a small Gaussian centered around $x_1$.
The conditional flow interpolating these distributions is given by:
\begin{equation}\label{eq_ot_flow}
    x_t = \psi_t(x|x_1) = (1-t)x_0 + tx_1
\end{equation}
which results in the following conditional vector field:
\begin{equation}\label{eq:ot_vector_field}
    u_t(x|x_1) = \frac{x_1-x}{1-t}
\end{equation}
which is marginalized in Eq.~\ref{eq_cfm_objective}.
Substituting Eq.~\ref{eq_ot_flow} to Eq.~\ref{eq:ot_vector_field}, one can also express the value of this vector field using a simpler expression:
\begin{equation}\label{eq_ot_final_vector_field}
    u_t(x_t|x_1) = x_1 - x_0
\end{equation}

\noindent \textbf{Conditional Generation via Flow Matching.} \quad Flow matching (FM) has been extended to conditional generative modeling in several works \citep{zheng2023guidedflowsgenerativemodeling, dao2023flowmatchinglatentspace,atanackovic2024metaflowmatchingintegrating, isobe2024extendedflowmatchingmethod}. In contrast to the original FM formulation of Eq.~\ref{eq:ot_vector_field}, one first samples a condition $c$. One then produces samples from $p_t(x | c)$ by passing $c$ as input to the parametric vector field $v_\theta$. The \emph{Conditional Generative Flow Matching} (CGFM) objective $\gL_{\rm cgfm}(\theta)$ is: 
\begin{equation}\label{eq_conditional_generative_fm_objective}
    \mathbb{E}_{t\sim \mathcal{U}(0,1), q(x_1, c), \rho_t(x | x_1)} \|v_\theta(t, c, x) - u_t(x | x_1)\|^2
\end{equation}

In practice, $c$ is incorporated by embedding it into some representation space and then using cross-attention between it and the features of $v_\theta$ as in \cite{rombach2022high}.  


\noindent \textbf{Flow Matching with Joint Distributions.} \quad \label{sec_joint_flow_matching}
While ~\cite{lipman2022flow} considered the setting of independently sampled $x_0$ and $x_1$, recently, ~\cite{pooladian2023multisample, tong2023improving} generalized the FM framework to an arbitrary joint distribution of $\rho(x_0, x_1)$ in the unconditional generation setting. This construction satisfies the following marginal constraints, i.e.
\begin{equation}
    \int \rho(x_0, x_1)dx_1 = q(x_0), \int \rho(x_0, x_1)dx_0 = q(x_1)
\end{equation}

\citet{pooladian2023multisample} modified the conditional probability path construction so at $t=0$, $\rho_0(x_0|x_1) = p(x_0|x_1)$, 
where $p(x_0|x_1)$ is the conditional distribution $\frac{\rho(x_0,x_1)}{q(x1)}$. 
The \emph{Joint Conditional Flow Matching} (JCFM) objective is:
\begin{equation}
    \gL_{\rm jcfm}(\theta) = \mathbb{E}_{t\sim \mathcal{U}(0,1), \rho(x_0,x_1)} \|v_\theta(t, x) - u_t(x | x_1)\|^2
\end{equation}




