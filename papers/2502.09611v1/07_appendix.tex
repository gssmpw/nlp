\appendix

\section{Additional Quantitative Results}

\begin{table}\centering


    \caption{\textbf{Numerical evaluation.} Quality of generated samples (FID, KID), and conditional fidelity (CLIP-Score) for our method in comparison to baselines, for the ImageNet-64 dataset for 15 NFEs. We consider CondOT~\cite{lipman2022flow}, BatchOT~\cite{pooladian2023multisample} and DDPM ~\cite{ho2020denoising}. 
    }
    \vskip 0.05in
    \centering
    \begin{tabular}{lccc}
    \toprule
               & FID $\downarrow$ & KID$\downarrow$ & CLIP$ \uparrow$\\
            \midrule
            DDPM & 47.51 & 6.74 & 17.71 \\
            CondOT & 16.16 & 1.96 & 18.02 \\
            BatchOT & 16.10 & 1.43 & 17.72\\
            Ours & \bf 13.62 & \bf 0.83 & \bf 18.05 \\

\bottomrule
    \end{tabular}
\vspace{-0.4cm}
\label{tab:nfe_results}
\end{table}

In Tab.\ref{tab:nfe_results}, we present additional metrics (FID, KID, and CLIP-Score) for ImageNet-64 with 15 NFEs. We compare the performance of CondOT~\cite{lipman2022flow}, BatchOT~\cite{pooladian2023multisample} and  DDPM~\cite{ho2020denoising}.
As shown, our model delivers significant improvements over the baselines.
\section{Visual Results}
\label{sec:visuals_appendix}

\begin{figure*}[]
\centering



\begin{tabular}{c@{}c@{}}



\includegraphics[width=0.5\textwidth]{figures/coco_generation/non_curated_samples/images_normal.png}  & 
\includegraphics[width=0.5\textwidth]{figures/coco_generation/non_curated_samples/images_ours.png}  \\
Flow Matching & Ours 

\end{tabular}
\caption{Visual comparison of randomly generated samples for prompts from the MS-COCO validation set using our method, in comparison to flow matching, for a model trained on MS-COCO. } %

\label{fig:non_curated_coco}
\end{figure*}


In Fig.~\ref{fig:non_curated_coco}, we provide additional visual results for our method in comparison to standard flow matching for a model trained on MS-COCO.

\section{Implementation Details}
\label{sec:implementation_details}
\section{Hyperparameter Search}\label{app:hype}
\normalsize
We exclusively conduct hyperparameter search on fold 0. 
For \textbf{GraFITi}~\citep{Yalavarthi2024.GraFITi} the hyperparameters for the search are as follows:
\begin{itemize}
    \item The number of layers, with possible values [1, 2, 3, 4].
    \item The number of attention heads, with possible values [1, 2, 4].
    \item The latent dimension, with possible values [16, 32, 64, 128, 256].
\end{itemize}

For the \textbf{LinODEnet} model~\citep{Scholz2022.Latenta} we search the hyperparameters from:
\begin{itemize}
    \item The hidden dimension, with possible values [16, 32, 64, 128].
    \item The latent dimension, with possible values [64, 128, 192, 256].
\end{itemize}

For \textbf{GRU-ODE-Bayes}~\citep{DeBrouwer2019.GRUODEBayesd} we tune the hidden size from [16, 32, 64, 128, 256]

For \textbf{Neural Flows}~\citep{Bilos2021.Neurald} we define the hyperparameter spaces for the search are as follows:
\begin{itemize}
    \item The number of flow layers, with possible values [1, 2, 4].
    \item The hidden dimension, with possible values [16, 32, 64, 128, 256].
    \item The flow model type, with possible values [GRU, ResNet].
\end{itemize}

For the \textbf{CRU}~\citep{Schirmer2022.Modelingb} the hyperparameter space is as follows:
\begin{itemize}
    \item The latent state dimension, with possible values [10, 20, 30].
    \item The number of basis functions, with possible values [10, 20].
    \item The bandwidth with possible values [3, 10].
\end{itemize}

We report the hyper-parameters used in Table ~\ref{tab:hyper-params}. All models were trained using the Adam optimizer ~\cite{kingma2017adammethodstochasticoptimization} with the following parameters: $\beta_1 = 0.9$, $\beta_2=0.999$, weight decay = 0.0, and $\epsilon = 1e{-8}$. 
All methods we trained (\emph{i.e.} Ours, CondOT, BatchOT, DDPM) using  identical architectures, specifically, the standard Unet ~\cite{ronneberger2015unetconvolutionalnetworksbiomedical} architecture from the \texttt{diffusers} ~\cite{von-platen-etal-2022-diffusers} library with the same number of parameters ($872M$) for the the same number of Epochs (see Table \ref{tab:hyper-params} for details). For all methods and datasets, we utilize a pre-trained Auto-Encoder ~\cite{oord2018neuraldiscreterepresentationlearning} and perform the flow/diffusion in its latent space.

In the case of text-to-image generation, we encode the text prompt using a pre-trained CLIP network and pass to the velocity $v_\theta$ using the standard UNet condition mechanism. In the class-conditional setting, we create the prompt `an image of a $\langle class \rangle$' and use it for the same conditioning scheme as in text conditional generation.

For the mapper $\gP_\theta$ from Sec~\ref{sec:prior_distribution} we use a network consisting a linear layer and 2 ResNet blocks with $11M$ parameters.

When using an adaptive step size sampler, we use \texttt{dopri5} with \texttt{atol=rtol=1e-5} from the \texttt{torchdiffeq} ~\citep{torchdiffeq} library.

Regarding the toy example Sec.~\ref{sec:toy_example}, we use a 4 layer MLP with ReLU activation as the velocity $v_\theta$. In this setup, we incorporate the condition by using positional embedding ~\cite{vaswani2023attentionneed} on the mean of each conditional mode and pass it to the velocity $v_\theta$ by concatenating it to its input. 


