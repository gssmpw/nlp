\begin{figure}[]
\centering

\begin{tabular}{c@{}c@{}c@{}}
\includegraphics[trim={0 3cm 0 3cm},clip,width=0.15\textwidth]{figures/toy_multimodal/standard.png}  & 
\includegraphics[trim={0 3cm 0 3cm},clip,width=0.15\textwidth]{figures/toy_multimodal/ours_1.png} & 
\includegraphics[trim={0 3cm 0 3cm},clip,width=0.15\textwidth]{figures/toy_multimodal/ours_2.png}  \\
\small{\emph{(a)}} & \small{\emph{(b)}} & \small{\emph{(c)}} 

\end{tabular}
\vskip -0.1in
\caption{\textbf{Multi-modal classes.} 
A toy example illustrating multi-modal classes with intersections in the prior. Each color represents a class (class A or B), with samples as points and the prior distribution as contour lines. 
\emph{(a)} shows a standard Gaussian prior (in black), while \emph{(b)} and \emph{(c)} show class-specific priors. 
While the mean each class falls on samples from the other class, our method results in an improved MMD score.  %
}
\vspace{-0.1cm}
\label{fig:multimodal}
\end{figure}

