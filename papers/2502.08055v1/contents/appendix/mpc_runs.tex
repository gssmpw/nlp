\section{MPC Benchmark of \ours}
\label{app:mpc-benchmark}




\mypara{Setup.} The framework is instantiated using building blocks from MP-SPDZ~\cite{CCS:Keller20}. Benchmarks were conducted on a MacBook Pro with an Apple Silicon M4 Pro processor and 48 GB RAM. We simulated the network setup locally, and ran all the parties on the same machine, with 4 threads. For the LAN case, we simulated a network with 1 Gbps bandwidth, and 1 ms latency, and for WAN, we used 200 Mbps and 20 ms respectively.

We benchmark the online phases of three components of our MPC protocol -- the norm bound check, the cross-client check, and selecting the top-k gradients separately. In reality, the normbound and cross-client check can be run in parallel, with top-k run on the outputs of these checks. All the components are instantiated using a semi-honest 3PC protocol from MP-SPDZ, specifically, replicated-ring-party.x which implements the replicated secret-sharing based protocol from \cite{CCS:AFLNO16}. The protocols are run with a 128-bit ring, although the larger ring size compared to the typical 64-bit ring is only to the size of the norm. It is possible to operate over a 64-bit ring, and switch to a 128-bit ring only for the norm, although we do not implement this optimization. 

For most cases, max-prob combined with normbound gives a good balance between robustness and runtime efficiency. However, in some scenarios the full accuracy check provides a nontrivial amount of robustness gain, which is why we also report the times for it. Note that normbound is generally faster than both versions of the cross-client checks, except in the case of ResNet-18. This is because the vector for which the norm is computed is of size 11,173,962 in the case of ResNet-18, and due to a limitation with the amount of RAM available, we could not take full advantage of parallelism.

