\begin{figure*}[ht]
\captionsetup[subfigure]{justification=centering}
    \centering
    \begin{subfigure}{0.24\textwidth}
        \includegraphics[width=\textwidth]{figures/mnist_adaptive.png}
        \caption{LeNet-5, MNIST}
        \label{subfig:adaptive-mnist}
    \end{subfigure}
    \hspace{.2mm}
    \begin{subfigure}{0.24\textwidth}
        \includegraphics[width=\textwidth]{figures/fmnist_adaptive.png}
        \caption{LeNet-5, FMNIST}
        \label{subfig:adaptive-fmnist}
    \end{subfigure}
    \hspace{.2mm}
    \begin{subfigure}{0.24\textwidth}
        \includegraphics[width=\textwidth]{figures/svhn_adaptive.png}
        \caption{ResNet-20, SVHN}
        \label{subfig:adaptive-svhn}
    \end{subfigure}
    \hspace{.2mm}
    \begin{subfigure}{0.24\textwidth}
        \includegraphics[width=\textwidth]{figures/cifar10_adaptive.png}
        \caption{ResNet-20, CIFAR-10}
        \label{subfig:adaptive-cifar10}
    \end{subfigure}
    \caption{\textbf{Robustness against Adaptive Attack with 20\% malicious clients.} Both \ours~(acc) and \ours~(prob) demonstrate competitive convergence performance against adaptive attacks. In contrast, other baselines that rely on static public validation datasets (\eg Norm Bound ($\calD_{val}$), Norm Ball, Cosine Similarity) or those that are adaptive but rely on simple validation checks (\eg Norm Bound (adaptive)) become vulnerable.
    }
    \label{fig:adaptiveattack}
\end{figure*}
