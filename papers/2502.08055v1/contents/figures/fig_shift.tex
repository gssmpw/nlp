


\begin{figure*}[t]
\captionsetup[subfigure]{justification=centering}
    \centering
    \begin{subfigure}{\textwidth}
        \centering
        \begin{subfigure}{0.49\textwidth}
            \centering
            \includegraphics[width=0.48\textwidth]{figures/mnist_svhn_evolving_period100.png}
            \includegraphics[width=0.48\textwidth]{figures/mnist_svhn_evolving_period250.png}
            \caption{Scenario 1: evolving clients}
            \label{subfig:scenario-evolving}
        \end{subfigure}
        \hfill
        \begin{subfigure}{0.49\textwidth}
            \centering
            \includegraphics[width=0.48\textwidth]{figures/mnist_svhn_new_period100.png}
            \includegraphics[width=0.48\textwidth]{figures/mnist_svhn_new_period250.png}
            \caption{Scenario 2: new clients}
            \label{subfig:scenario-new}
        \end{subfigure}
    \end{subfigure}
    
    \caption{\textbf{Adaptability across the two distribution shift scenarios.} We start with 100 clients working on MNIST data. Every 100 (or 250) rounds (marked by dotted lines), either 20 random clients transition to SVHN data (\ref{subfig:scenario-evolving}), or an additional 20 clients with SVHN data join the communication (\ref{subfig:scenario-new}). Our model consistently adapts and improves accuracy throughout the communication rounds, outperforming other aggregation protocols that struggle to adjust to distribution shifts.}
    \label{fig:adaptability}
\end{figure*}
