\section{Experiments}
\label{sec:experiment}



In this section, we employ experiments and show that~\ours~has (1) better robustness in the presence of malicious clients (Section~\ref{sec:empirical-robustness}; Table~\ref{tab:robustness}), (2) better adaptability to the change in the underlying distribution of client data (Section~\ref{sec: empirical-adaptability}; Figure~\ref{fig:adaptability}), and (3) reasonable computation and communication overhead that can be significantly improved via parallelism (Section~\ref{sec: mpc-benchmark}; Table~\ref{tab:mpc-benchmark}).



\subsection{Setup}
\label{sec:exp-setup}

We briefly describe the experimental setup following the conventions in the FL literature~\cite{fang2020local,CCS:CGJv22}.
See Appendix~\ref{app:setup} for full descriptions.

\mypara{Models and Datasets.} We consider the following model architectures and datasets: LeNet-5 for MNIST and Fashion-MNIST (FMNIST), ResNet-18 for SVHN and CIFAR-10.
We follow the given train vs. test set split and further split each train set, balanced across classes. We follow the setup in~\cite{CCS:CGJv22} and reserve random 10k training samples as the public validation set ($\calD_{pubval}$) for aggregation baselines that leverage public validation datasets. The remaining training samples are partitioned across clients by a Dirichlet distribution $\text{Dir}_K(\alpha)$ with $\alpha = 0.5$ to emulate data heterogeneity in realistic FL scenarios~\cite{cao2020fltrust}.
We assume there are 100 clients where $10\%$ or $20\%$ of them are malicious clients.



\mypara{Defense baselines.} We compare the performance of \ours with the following aggregation methods that are most commonly considered in Byzantine-robust secure FL literature: 
\begin{enumerate}
    \item \textit{Norm Bound (adaptive)}~\cite{rofl, elsa} accepts a client update if the update is bounded by a $\tau$: $\msf{Chk}(\bfu_i) = \indicator{||\bfu_i^{(t)}||_2 < \tau}$.
    At the start of the $t$th communication round, the server computes the threshold as $\tau = \lambda \times \text{median}(\bfu_1^{(r-1)}, \cdots, \bfu_m^{(t-1)})$ adaptively, and then broadcast it to all clients. Hence, each client can submit the norm-bound check result based on the threshold value.

    \item \textit{Norm Bound ($\calD_{pubval}$ or public data)}~\cite{CCS:CGJv22} is identical to the above Norm Bound (adaptive) except that the threshold is computed by referring to the public validation dataset: $\tau = ||\bfu_{pubval}||_2$ where $\bfu_{pubval}$ is the previous round's global model update computed with the public validation data $\calD_{pubval}$. $\tau$ can be computed by anyone participating in the communication.

    \item \textit{Norm Ball}~\cite{steinhardt2017normball,CCS:CGJv22} accepts a client update if the update is within a spherical radius from the reference %
    update computed using the public validation dataset: $\msf{Chk}(\bfu_i, \calD_{pubval}) = \indicator{||\bfu_i^{(t)} - \bfu_{pubval}||_2 < \tau}$ where $\tau = \lambda \times ||\bfu_{pubval}||_2$. $\bfu_{pubval}$ and the value for $\tau$ is available to anyone participating in the communication.

    \item \textit{Cosine Similarity}~\cite{cao2020fltrust,CCS:CGJv22} checks the cosine similarity between each client update and the global model update from the previous round $\bfu^{(t-1)}$: $\msf{Chk}(\bfu_i, \calD_{pubval}) = \indicator{\cos(\bfu_i^{(t)}, \bfu^{(t-1)}) < \tau}$ where $\cos(u, v) = \frac{<u, v>}{||u||_2 ||v||_2}$. The threshold is computed as $\tau = \lambda \times \cos(\bfu_{pubval}, \bfu^{(t-1)})$, and again, can be computed by anyone in the communication.
\end{enumerate}

$\lambda$ is a constant where a larger $\lambda$ allows for more gradient updates, leading to faster convergence but reduced robustness.


\mypara{Attack baselines.}
We evaluate each aggregation method against the following attacks:
\begin{enumerate}
    \item \textit{Additive Noise}~\cite{li2019additive} adds Gaussian noise to a malicious client update.
    \item \textit{Sign Flipping}~\cite{damaskinos2018signflipping} flips the sign of a client update: $\bfu'_i = - \bfu_i$, for a client $\calC_i$.
    \item \textit{Label Flipping}~\cite{fang2020local} is a data poisoning attack that flips the label of each training instance. Specifically, it flips a label $l \in \{0, 1, \dots, L-1\}$ into $L-l-1$ where $L$ is the number of classes.
    \item \textit{Adaptive Attack} is considered the strongest attack adaptive to each aggregation as described in Section~\ref{sec:adaptiveattack}.
\end{enumerate}


\section{Loss Robustness}
\label{sec:robustness}

% We extend the concept of label noise to the autoregressive language modeling domain, focusing on asymmetric or class-conditional noise. Specifically, at each step $t$, the label $\xbm_t$ in the training data of the black-box model is flipped to  $\tilde \xbm_t \in V$ with probability $p^*(\tilde \xbm_t|\xbm_t)$, while the feature vectors or preceding tokens $(\xbm_{t-1:1})$ remain unchanged. Consequently, the black-box model observes samples from a noisy distribution given by  $p^*(\tilde \xbm_t, \xbm_{t-1:1}) = \sum_{\xbm_t}p^*(\tilde \xbm_t | \xbm_t)p^*(\xbm_t|\xbm_{t-1:1})p^*(\xbm_{t-1:1})$.

% Denote by $T_t  \in [0, 1]^{|V|\times |V|}$, the noise transition matrix at step $t$ specifying the probability of one label being flipped to another, so that $\forall i, j \;\; T_{t_{ij}}=p^*(\tilde \xbm_t = \ebm^j | \xbm_t = \ebm^i)$. The matrix is row-stochastic and not necessarily symmetric across the classes. 

% To address asymmetric label noise, we modify the loss $\bm{\ell}$ to ensure robustness. Initially, assuming the noise transition matrix $T_t$ is known, we apply a loss correction inspired by prior work~\citep{patrini2017making, sukhbaatar2015training}. Subsequently, we relax this assumption and estimate $T_t$ directly, forming the foundation of our plugin model approach.

We extend label noise modeling to the autoregressive language setting, focusing on asymmetric or class-conditional noise. At each step $t$, the label $\xbm_t$ in the black-box model’s training data is flipped to $\tilde \xbm_t \in V$ with probability $p^*(\tilde \xbm_t|\xbm_t)$, while preceding tokens $(\xbm_{t-1:1})$ remain unchanged. As a result, the black-box model observes samples from a noisy distribution: $p^*(\tilde \xbm_t, \xbm_{t-1:1}) = \sum_{\xbm_t} p^*(\tilde \xbm_t | \xbm_t) p^*(\xbm_t|\xbm_{t-1:1}) p^*(\xbm_{t-1:1}).$

We define the noise transition matrix $T_t \in [0,1]^{|V|\times |V|}$ at step $t$, where each entry $T_{t_{ij}} = p^*(\tilde \xbm_t = \ebm^j | \xbm_t = \ebm^i)$ represents the probability of label flipping. This matrix is row-stochastic but not necessarily symmetric.

To handle asymmetric label noise, we modify the loss $\bm{\ell}$ for robustness. Initially, assuming a known $T_t$, we apply a loss correction inspired by~\citep{patrini2017making, sukhbaatar2015training}. We then relax this assumption by estimating $T_t$ directly, forming the basis of our \textit{Plugin} model approach.

We observe that a language model trained with no loss correction would result in a predictor for noisy labels $b(\tilde \xbm_t | \xbm_{t-1:1})$. We can make explicit the dependence on $T_t$. For example, with cross-entropy we have:

\begin{align*}
&\ell(\ebm^i, b(\tilde \xbm_t | \xbm_{t-1:1})) = -\log b(\tilde\xbm_t = \ebm^i | \xbm_{t-1:1}) \\
&= -\log \sum_{j=1}^{|V|} p^*(\tilde\xbm_t = \ebm^i | \xbm_t = \ebm^j) b(\xbm_t = \ebm^j | \xbm_{t-1:1}) \\ 
&= -\log \sum_{j=1}^{|V|} T_{t_{ji}} {b}(\xbm_t = \ebm^j | \xbm_{t-1:1}), \numberthis
\label{eq:fc}
\end{align*}
or in matrix form
\begin{equation}
    \label{eq:fc-mat}
    \bm{\ell}(b(\tilde \xbm_t|\xbm_{t-1:1})) = -\log T_t^\top b(\xbm_t|\xbm_{t-1:1}).
\end{equation}

% This loss compares the noisy label $\tilde \xbm_t$ to the noisy predictions averaged using the transition matrix $T_t$ at step $t$. Note that the cross-entropy loss is commonly employed for next-token prediction tasks. Cross-entropy is a \emph{proper composite loss} with the softmax function as its \emph{inverse link function}~\citep{patrini2017making}. Consequently, from Theorem 2 of~\citep{patrini2017making}, the minimizer of the \emph{forwardly-corrected} loss in Equation~\eqref{eq:fc-mat} for noisy data corresponds to the minimizer of the actual loss for clean data. Formally, this can be expressed as:

% \begin{align*}
%     \label{eq:loss-minimizers}
%     & \argmin_{w} E^*_{\tilde \xbm_t,\xbm_{t-1:1}}\Big[\bm{\ell}(\xbm_t, T_t^T b(\xbm_t|\xbm_{t-1:1})) \Big] \\ &= 
%     \argmin_{w} E^*_{\xbm_t,\xbm_{t-1:1}}\Big[\bm{\ell}(\xbm_t, b(\xbm_t|\xbm_{t-1:1})) \Big],
% \end{align*}
% where $w$ are the weights of the language model, and their dependence is implicitly embedded in the definition of the softmax output $b$ from the black-box language model. This result indicates that if the transition matrix $T_t$ were known, we could transform the softmax output $b(\bm{x}_t \mid \bm{x}_{t-1:1})$ using $T_t^T$, use the transformed predictions as the final outputs, and re-train the black-box model accordingly with the corrected loss. However, the transition matrix $T_t$ is not known a priori, and we do not have access to the training data. Thus, estimating $T_t$ from clean data becomes a crucial step in our approach.

This loss compares the noisy label $\tilde \xbm_t$ to the noisy predictions averaged via the transition matrix $T_t$ at step $t$. Cross-entropy loss, commonly used for next-token prediction, is a \emph{proper composite loss} with the softmax function as its \emph{inverse link function}~\citep{patrini2017making}. Consequently, from Theorem 2 of~\citet{patrini2017making}, the minimizer of the \emph{forwardly-corrected} loss in Equation~\eqref{eq:fc-mat} on noisy data aligns with the minimizer of the true loss on clean data, i.e., 
\begin{align*}
    \label{eq:loss-minimizers}
    & \argmin_{w} E^*_{\tilde \xbm_t,\xbm_{t-1:1}}\Big[\bm{\ell}(\xbm_t, T_t^\top b(\xbm_t|\xbm_{t-1:1})) \Big] \\ &= 
    \argmin_{w} E^*_{\xbm_t,\xbm_{t-1:1}}\Big[\bm{\ell}(\xbm_t, b(\xbm_t|\xbm_{t-1:1})) \Big],
\end{align*}
where $w$ are the language model’s weights, implicitly embedded in the softmax output $b$ from the black-box model. This result suggests that if $T_t$ were known, we could transform the softmax output $b(\xbm_t \mid \xbm_{t-1:1})$ using $T_t^T$, use the transformed predictions as final outputs, and retrain the model accordingly. However, since $T_t$ is unknown and training data is inaccessible, estimating $T_t$ from clean data is essential to our approach.


\subsection{Estimation of Transition Matrix}
\label{ssec:estimatingT}

% In our problem setup, we assume access to a small amount of clean language data for the task. Under the assumption that the black-box model is expressive enough to model $p^*(\tilde{\bm{x}}_t \mid \bm{x}_{t-1:1})$ (Assumption (2) in Theorem 3 of~\citep{patrini2017making}), the transition matrix $T_t$ can be estimated using this clean data. Considering the supervised classification problem at step $t$, let $\mathcal{X}_t^i$ denote all samples in the clean data where $\bm{x}_t = \bm{e}^i$ and the preceding tokens are $(\bm{x}_{t-1}, \dots, \bm{x}_1)$. A naive estimate of the transition matrix can be computed as follows:

We assume access to a small amount of target language data for the task. Given that the black-box model is expressive enough to approximate $p^*(\tilde{\xbm}_t \mid \xbm_{t-1:1})$ (Assumption (2) in Theorem 3 of~\citet{patrini2017making}), the transition matrix $T_t$ can be estimated from this target data. Considering the supervised classification setting at step $t$, let $\mathcal{X}_t^i$ represent all target data samples where $\xbm_t = \ebm^i$ and the preceding tokens are $(\xbm_{t-1:1})$. A naive estimate of the transition matrix is: $\hat T_{t_{ij}}=b(\tilde \xbm_t = \ebm^j|\xbm_t=\ebm^i)=\frac{1}{|\mathcal{X}_t^i|}\sum_{x\in\mathcal{X}_t^i}b(\tilde \xbm_t = \ebm^j|\xbm_{t-1:1})$.


While this setup works for a single step $t$, there are two key challenges in extending it across all steps in the token prediction task:

\begin{enumerate}[leftmargin=0.4cm]
    \item \textbf{Limited sample availability:} The number of samples where $\bm{x}_t = \bm{e}^i$ and the preceding tokens $(\bm{x}_{t-1}, \dots, \bm{x}_1)$ match exactly is limited in the clean data, especially with large vocabulary sizes (e.g., $|V| = O(100K)$ for LLaMA~\citep{dubey2024llama}). This necessitates modeling the transition matrix as a function of features derived from $\bm{x}_{t-1:1}$, akin to text-based autoregressive models.
    \item \textbf{Large parameter space:} With a vocabulary size of $|V| = O(100K)$, the transition matrix $T_t$ at step $t$ contains approximately 10 billion parameters. This scale may exceed the size of the closed-source LLM and cannot be effectively learned from limited target data. Therefore, structural restrictions must be imposed on $T_t$ to reduce its complexity.
\end{enumerate}

To address these challenges, we impose the restriction that the transition matrix $T_t$ is diagonal. While various constraints could be applied to simplify the problem, assuming $T_t$ is diagonal offers two key advantages. First, it allows the transition matrix—effectively a vector in this case—to be modeled using standard autoregressive language models, such as a \emph{GPT-2 model with $k$ transformer blocks}, a \emph{LLaMA model with $d$-dimensional embeddings}, or a fine-tuned \emph{GPT-2-small} model. These architectures can be adjusted based on the size of the target data. Second, a diagonal transition matrix corresponds to a symmetric or class-independent label noise setup, where $\xbm_t = \ebm^i$ flips to any other class with equal probability in the training data. This assumption, while simplifying, remains realistic within the framework of label noise models.

By enforcing this diagonal structure, we ensure efficient estimation of the transition matrix while maintaining practical applicability within our framework. In the next section, we outline our approach for adapting closed-source language models to target data.
















\subsection{Robustness against Poisoning Attacks}
\label{sec:empirical-robustness}

\mypara{\ours~is more Byzantine-robust.}
Table~\ref{tab:robustness} summarizes the robust accuracy of each aggregation method under attacks.
According to our goal in Equation~\ref{obj:defense},
All aggregation methods demonstrate comparable robust accuracies against most cases of non-adaptive, common attack baselines (\ie Additive Noise, Labelflip, and Signflip), varying by only a small margin of $2 \mhyphen 3\%$ accuracy in each column.
However, all other defense baselines suffer from significant accuracy drops under adaptive attacks, and the robustness gap compared to \ours~becomes more pronounced with more complex tasks. For example, while the accuracy gap on MNIST is relatively modest, it widens to approximately 50\% on CIFAR-10.
In other words, unlike other aggregation baselines that are more susceptible to certain attacks (\eg Non-adaptive vs. Adaptive) or fail on certain data sets (\eg Norm Ball on CIFAR-10), \ours~shows consistent robustness against all attacks on all data sets.
Such consistency makes it a more desirable choice in real-world uncertainty. An adversary may employ the strongest possible attack, which is the adaptive attack in most cases, it is crucial to provide robustness even under those attacks; specifically, as in Figure~\ref{fig:adaptiveattack}, in the worst case, the adversary can only pull down the performance of FL with \ours~to ~97\% on MNIST, ~87\% on FMNIST, ~83\% on SVHN, and ~72\% on CIFAR-10 after 1000 rounds.



\begin{figure*}[ht]
\captionsetup[subfigure]{justification=centering}
    \centering
    \begin{subfigure}{0.24\textwidth}
        \includegraphics[width=\textwidth]{figures/mnist_adaptive.png}
        \caption{LeNet-5, MNIST}
        \label{subfig:adaptive-mnist}
    \end{subfigure}
    \hspace{.2mm}
    \begin{subfigure}{0.24\textwidth}
        \includegraphics[width=\textwidth]{figures/fmnist_adaptive.png}
        \caption{LeNet-5, FMNIST}
        \label{subfig:adaptive-fmnist}
    \end{subfigure}
    \hspace{.2mm}
    \begin{subfigure}{0.24\textwidth}
        \includegraphics[width=\textwidth]{figures/svhn_adaptive.png}
        \caption{ResNet-20, SVHN}
        \label{subfig:adaptive-svhn}
    \end{subfigure}
    \hspace{.2mm}
    \begin{subfigure}{0.24\textwidth}
        \includegraphics[width=\textwidth]{figures/cifar10_adaptive.png}
        \caption{ResNet-20, CIFAR-10}
        \label{subfig:adaptive-cifar10}
    \end{subfigure}
    \caption{\textbf{Robustness against Adaptive Attack with 20\% malicious clients.} Both \ours~(acc) and \ours~(prob) demonstrate competitive convergence performance against adaptive attacks. In contrast, other baselines that rely on static public validation datasets (\eg Norm Bound ($\calD_{val}$), Norm Ball, Cosine Similarity) or those that are adaptive but rely on simple validation checks (\eg Norm Bound (adaptive)) become vulnerable.
    }
    \label{fig:adaptiveattack}
\end{figure*}




\begin{figure*}[t]
\captionsetup[subfigure]{justification=centering}
    \centering
    \begin{subfigure}{\textwidth}
        \centering
        \begin{subfigure}{0.49\textwidth}
            \centering
            \includegraphics[width=0.48\textwidth]{figures/mnist_svhn_evolving_period100.png}
            \includegraphics[width=0.48\textwidth]{figures/mnist_svhn_evolving_period250.png}
            \caption{Scenario 1: evolving clients}
            \label{subfig:scenario-evolving}
        \end{subfigure}
        \hfill
        \begin{subfigure}{0.49\textwidth}
            \centering
            \includegraphics[width=0.48\textwidth]{figures/mnist_svhn_new_period100.png}
            \includegraphics[width=0.48\textwidth]{figures/mnist_svhn_new_period250.png}
            \caption{Scenario 2: new clients}
            \label{subfig:scenario-new}
        \end{subfigure}
    \end{subfigure}
    
    \caption{\textbf{Adaptability across the two distribution shift scenarios.} We start with 100 clients working on MNIST data. Every 100 (or 250) rounds (marked by dotted lines), either 20 random clients transition to SVHN data (\ref{subfig:scenario-evolving}), or an additional 20 clients with SVHN data join the communication (\ref{subfig:scenario-new}). Our model consistently adapts and improves accuracy throughout the communication rounds, outperforming other aggregation protocols that struggle to adjust to distribution shifts.}
    \label{fig:adaptability}
\end{figure*}



\subsection{Adaptability to Distribution Shifts}
\label{sec: empirical-adaptability}

In this section, we examine the adaptability of \ours~along with other baseline aggregation protocols in the face of distribution shifts.
We take into account two practical scenarios that may frequently occur in real-world FL settings:


\begin{enumerate}
    \item \textbf{Scenario 1: evolving clients} (Figure~\ref{subfig:scenario-evolving}).
    Some clients involved in the communication send different data over time to the central server. Consider a cross-silo FL system for predicting patient outcomes in hospitals. Over time, a hospital's patient demographic may shift due to various factors such as changes in the local population, disease outbreaks, or the introduction of new healthcare programs. This could lead to an evolution in the type of patient data the hospital, or a subset of hospitals send to the central server, reflecting these changes.

    \item \textbf{Scenario 2: new clients} (Figure~\ref{subfig:scenario-new}). Along with the existing clients, new clients join the communication, but their data differs slightly from the existing clients. For instance, in the above FL example for a healthcare application, the existing clients could be hospitals from a different geographic location, or specialized hospitals and clinics that want to join the system to solve the same task. The new clients' data coming from such sources may differ due to their specialized nature of care, different patient demographics, or unique healthcare practices.
\end{enumerate}

We simulate these scenarios using MNIST for existing clients and SVHN for evolving or new clients, with LeNet-5 as the model. The server initially communicates with 100 MNIST clients and uses a static MNIST validation dataset. Distribution shifts occur every 100 or 250 rounds, where clients transition to SVHN (Scenario 1) or new SVHN clients join (Scenario 2). We measure global model accuracy on MNIST initially and on the combined MNIST-SVHN test set thereafter. Further details are provided in Appendix~\ref{app:setup}.
We aim to evaluate how cross-client checks contribute to adaptability under changing data distributions. Therefore, we disable the norm-bound check in \ours~and rely exclusively on cross-client checks to address the occurrence of distribution shifts.


\mypara{\ours~can adapt to changing client data distributions.}
Figure~\ref{fig:adaptability} shows the convergence of the global model against the number of communication rounds.
We observe that baseline aggregation protocols relying on a static public validation dataset suffer significant performance degradation under distribution shifts.
Specifically, Norm Bound ($\calD_{pubval}$) and Norm Ball reject every update from SVHN clients because the magnitude of updates computed on SVHN far exceeds the threshold determined on the static MNIST validation set.
The same observation applies to Norm Bound (adaptive) that uses the median of gradient updates to adaptively set the threshold since the gradient updates computed with new SVHN data are likely to fall above the median. One could relax the threshold by multiplying larger constant values to the computed median of gradients, but again, this highlights that appropriate threshold parameter setting is crucial to ensure robustness and adaptability under changing distributions, which requires non-trivial efforts, especially in secure FL settings.
On the contrary, \ours~is capable of dynamically incorporating the new shifted local data into validation. Consequently, its choice of updates balances the performance on both MNIST and SVHN, which leads to much stronger adaptability without any parameter tuning.



\subsection{MPC Benchmarks}
\label{sec: mpc-benchmark}
We benchmark the computation and communication overhead of our MPC implementation. The framework is instantiated using building blocks from MP-SPDZ~\cite{CCS:Keller20}. Benchmarks were conducted on a MacBook Pro with an Apple Silicon M4 Pro processor and 48 GB RAM. We simulated the network setup locally, and ran all the parties on the same machine, with 4 threads. For the LAN case, we simulated a network with 1 Gbps bandwidth, and 1 ms latency, and for WAN, we used 200 Mbps and 20 ms respectively.

We benchmark the three components of our MPC protocol -- the norm bound check, the cross-client check, and selecting the top-$k$ model updates separately. In reality, the normbound and cross-client check can be run in parallel, with top-$k$ run on the outputs of these checks. All the components are instantiated using a semi-honest 3PC protocol from MP-SPDZ, specifically, replicated-ring-party.x which implements the replicated secret-sharing based protocol from \cite{CCS:AFLNO16}. The protocols are run with a 128-bit ring, although the larger ring size compared to the typical 64-bit ring is only to the size of the norm. It is possible to operate over a 64-bit ring, and switch to a 128-bit ring only for the norm, although we do not implement this optimization. 

\begin{table}[b!]
\centering
\caption{\textbf{MPC benchmarks.} With $\cor = 10\%$ corruptions, $2\cor + 1$ validation pairs, validation dataset size of 10.}

\begin{adjustbox}{width=\columnwidth,center}
\begin{tabular}{cc|cc|cc|cc|cc}
\toprule
     & & \multicolumn{2}{c|}{Normbound} & \multicolumn{2}{c|}{Cross-client check (acc)} & \multicolumn{2}{c|}{Cross-client check (prob)} & \multicolumn{2}{c}{Top-k filtering} \\ \cline{3-10}
     & & LAN & WAN & LAN & WAN & LAN & WAN & LAN & WAN \\
     \hline \hline
    \multirow{2}{*}{LeNet-5, 10 clients} & time (s) & 1.22 & 6.73 & 13.14 & 109.62 & 7.10 & 60.67 & 2.64 & 23.46 \\
    & data (MB) & \multicolumn{2}{c|}{26.40} & \multicolumn{2}{c|}{42.32} & \multicolumn{2}{c|}{21.51} & \multicolumn{2}{c}{0.25} \\ \cline{1-10}
    \multirow{2}{*}{LeNet-5, 20 clients} & time (s) & 1.78 & 8.92 & 20.20 & 174.90 & 10.89 & 93.39 & 2.65 & 23.53 \\
    & data (MB) & \multicolumn{2}{c|}{84.10} & \multicolumn{2}{c|}{70.08} & \multicolumn{2}{c|}{35.40} & \multicolumn{2}{c}{0.73} \\ \cline{1-10}
    \multirow{2}{*}{ResNet-18, 10 clients} & time (s) & 207.26 & 418.55 & 261.78 & 2084.90 & 131.30 & 1056.52 & 2.64 & 23.46 \\
    & data (MB) & \multicolumn{2}{c|}{4791.75} & \multicolumn{2}{c|}{2010.03} & \multicolumn{2}{c|}{1098.59} & \multicolumn{2}{c}{0.25} \\ \cline{1-10}
    \multirow{2}{*}{ResNet-18, 20 clients} & time (s) & 370.49 & 818.43 & 429.15 & 3471.01 & 217.45 & 1746.22 & 2.65 & 23.53 \\
    & data (MB) & \multicolumn{2}{c|}{9583.53} & \multicolumn{2}{c|}{2150.21} & \multicolumn{2}{c|}{1137.50} & \multicolumn{2}{c}{0.73} \\
   
\bottomrule
\end{tabular}
\end{adjustbox}

\label{tab:mpc-benchmark}
\end{table}



We report the MPC benchmarks in Table~\ref{tab:mpc-benchmark}. Compared to previous FL schemes, \ours\ unsurprisingly incurs increasing overhead as it supports private collection and computation of richer information via cross-client check. Nevertheless, the checks can be completed in reasonable time. Moreover, a large portion of the computations, \eg $\score{i}{j}$ for different $(i,j)$ pairs, are highly parallelizable, suggesting substantial speed-ups with more powerful infrastructure than our lab computer.



