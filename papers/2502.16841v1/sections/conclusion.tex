\section{Conclusion}

FMs represent a transformative advancement in medical imaging analysis, yet their implementation presents both opportunities and challenges for achieving equitable healthcare delivery. Our comprehensive review demonstrates that effective bias mitigation in FMs requires systematic interventions throughout the development pipeline, from data curation to deployment protocols. While technical innovations in training methodologies show promise for enhancing fairness without relying on sensitive attributes, the substantial computational and data demands of these models risk exacerbating global inequalities. The emergence of regulatory frameworks such as the EU AI Act reflects growing recognition of FMs societal impact and the need for governance structures that ensure responsible development. However, significant disparities persist, particularly in Global South nations where limited access to essential resources including specialized workforce, datasets, and computational infrastructure hinders both development and implementation of fair FMs. Moving forward, addressing these challenges requires coordinated action between technologists, healthcare providers, and policymakers to develop accessible solutions and appropriate frameworks for low-resource countries and institutions. As FMs continue to evolve, their successful implementation in healthcare will depend on our ability to balance technical innovation with ethical principles, ultimately working toward reducing rather than amplifying existing healthcare disparities.