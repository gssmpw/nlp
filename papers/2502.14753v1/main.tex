\documentclass[10pt, letter, onecolumn]{main}

% \usepackage{csquotes}
% \usepackage[bibstyle=nature,citestyle=numeric-comp,
%             natbib=true,backend=biber,maxbibnames=5,
%             giveninits=false,sorting=none,defernumbers=true]{biblatex}
% \usepackage[superscript,biblabel]{cite}
% note: sorting=none gives refs in order they appear
\usepackage[pdftex]{graphicx}
\usepackage{pifont}
\usepackage{amssymb}
\usepackage{multirow}
\usepackage{array}
\usepackage{dblfloatfix}
\usepackage{titlesec}
\usepackage[numbers,sort&compress]{natbib}
\usepackage{nameref}
\usepackage{varioref}
% \usepackage{hyperref}
\usepackage{etoc}


% \addbibresource{main.bib}
% \renewcommand*{\bibfont}{\linespread{0.8}\footnotesize}


%%%%%%%%%%%%%%%%%%%%%%%%%%%%%%%%%%
%%% user added packages / commands

\setlength{\abovecaptionskip}{10pt plus 3pt minus 2pt} % add spacing b/w figure and table
\newlength{\cboxlength}
\settowidth{\cboxlength}{7.1 $\pm$ 13.2} % The widest entry

% for nice-looking up arrows
\usepackage{amsmath,xparse,mleftright}
\NewDocumentCommand{\up}{som}{%
  \IfBooleanTF{#1}
    {\upext{#3}}
    {#3\IfNoValueTF{#2}{\mathord}{#2}\uparrow}%
}
\NewDocumentCommand{\upext}{m}{%
  \mleft.\kern-\nulldelimiterspace#1\mright\uparrow
}

\usepackage{xcolor} % For text color
\usepackage{soul} % For highlighting

\usepackage{adjustbox}
\usepackage[most]{tcolorbox}
\usepackage{float}
\usepackage{xspace}
\usepackage[symbol]{footmisc}
\usepackage{lineno}

\tcbset{
  aibox/.style={
    width=394.18663pt,
    top=10pt,
    colback=white,
    colframe=black,
    colbacktitle=black,
    enhanced,
    center,
    attach boxed title to top left={yshift=-0.1in,xshift=0.15in},
    boxed title style={boxrule=0pt,colframe=white,},
  }
}
\newtcolorbox{AIbox}[2][]{aibox,title=#2,#1}

% preliminaries
\ifdefined\red
    \renewcommand{\red}[1]{\textcolor{red}{#1}}
\else
    \newcommand{\red}[1]{\textcolor{red}{#1}}
\fi
\ifdefined\blue
    \renewcommand{\blue}[1]{\textcolor{blue}{#1}}
\else
    \newcommand{\blue}[1]{\textcolor{blue}{#1}}
\fi
% \newcommand{\hhll}[1]{\sethlcolor{yellow}\hl{#1}} % highlights on
\newcommand{\hhll}[1]{#1} % highlights off
\newcommand{\hhllred}[1]{\sethlcolor{red}\hl{#1}} 

% small caps formatting of model names
\newcommand{\flantfive}{\textsc{FLAN-T5}}
\newcommand*{\eg}{e.g.\@\xspace}

% \usepackage{todonotes}

% \newcommand{\conditionalprintbibliography}[1]{
%   \begingroup
%     \def\do#1{\ifcsname blx@bsee@#1\endcsname\else
%       \global\cslet{blx@bsee@#1}\relax
%       \printbibliography[heading=subbibintoc, segment=#1, resetnumbers=false]
%     \fi}
%     \do{#1}
%   \endgroup
% }

%%%%%%%%%%%%%%%%%%%%%%%%%%%%%%%%%%
\newcommand{\maya}[1]{{\noindent\color{blue}[Maya: #1]}}
\newcommand{\name}[1][]{\textsc{CompRx}}

\makeatletter
\renewcommand\AB@authnote[1]{}
\renewcommand\AB@affilnote[1]{}
\makeatother

\usepackage{anyfontsize} % allows for precise control over title font size
\usepackage{titlesec}
\titleformat{\section}{\normalfont\Large\bfseries}{\thesection}{1em}{#1}

% -------------  Title  ---------------------- 
\title{{\fontsize{16.5pt}{15.5pt}\selectfont MedVAE: Efficient Automated Interpretation of Medical Images with Large-Scale Generalizable Autoencoders}}


% -------------  Authors ----------------------

\author[]{Maya Varma$^{1,*}$, Ashwin Kumar$^{1,*}$, Rogier van der Sluijs$^{1,*}$, Sophie Ostmeier$^{1}$, Louis Blankemeier$^{1}$, Pierre Chambon$^{1}$, Christian Bluethgen$^{1}$, Jip Prince$^{2}$, Curtis Langlotz$^{1}$, Akshay Chaudhari$^{1}$}

\affil{\footnotesize{$^1$Stanford Center for Artificial Intelligence in Medicine and Imaging, Stanford University, Palo Alto, CA, USA. $^2$UMC Utrecht, Utrecht, Netherlands}}

% \affil{}
\renewcommand{\correspondingauthor}[1]{$\ast$~Equal contributions.}
\usepackage{xcolor}
\usepackage{hyperref}

\hypersetup{
    colorlinks=true,   % Enables colored links
    urlcolor=blue      % Ensures URLs are blue
}
\usepackage[noabbrev,capitalize]{cleveref}


\begin{document}
\begin{abstract}
Medical images are acquired at high resolutions with large fields of view in order to capture fine-grained features necessary for clinical decision-making. Consequently, training deep learning models on medical images can incur large computational costs. In this work, we address the challenge of downsizing medical images in order to improve downstream computational efficiency while preserving clinically-relevant features. We introduce \textit{MedVAE}, a family of six large-scale 2D and 3D autoencoders capable of encoding medical images as downsized latent representations and decoding latent representations back to high-resolution images. We train MedVAE autoencoders using a novel two-stage training approach with 1,052,730 medical images. Across diverse tasks obtained from 20 medical image datasets, we demonstrate that (1) utilizing MedVAE latent representations in place of high-resolution images when training downstream models can lead to efficiency benefits (up to 70x improvement in throughput) while simultaneously preserving clinically-relevant features and (2) MedVAE can decode latent representations back to high-resolution images with high fidelity. Our work demonstrates that large-scale, generalizable autoencoders can help address critical efficiency challenges in the medical domain. Our code is available at \href{https://github.com/StanfordMIMI/MedVAE}{https://github.com/StanfordMIMI/MedVAE}. 
\end{abstract}

\maketitle


\vspace{10mm}
% \begin{refsegment}
% \defbibfilter{notother}{not segment=\therefsegment}
\nolinenumbers
\section{Introduction}
Backdoor attacks pose a concealed yet profound security risk to machine learning (ML) models, for which the adversaries can inject a stealth backdoor into the model during training, enabling them to illicitly control the model's output upon encountering predefined inputs. These attacks can even occur without the knowledge of developers or end-users, thereby undermining the trust in ML systems. As ML becomes more deeply embedded in critical sectors like finance, healthcare, and autonomous driving \citep{he2016deep, liu2020computing, tournier2019mrtrix3, adjabi2020past}, the potential damage from backdoor attacks grows, underscoring the emergency for developing robust defense mechanisms against backdoor attacks.

To address the threat of backdoor attacks, researchers have developed a variety of strategies \cite{liu2018fine,wu2021adversarial,wang2019neural,zeng2022adversarial,zhu2023neural,Zhu_2023_ICCV, wei2024shared,wei2024d3}, aimed at purifying backdoors within victim models. These methods are designed to integrate with current deployment workflows seamlessly and have demonstrated significant success in mitigating the effects of backdoor triggers \cite{wubackdoorbench, wu2023defenses, wu2024backdoorbench,dunnett2024countering}.  However, most state-of-the-art (SOTA) backdoor purification methods operate under the assumption that a small clean dataset, often referred to as \textbf{auxiliary dataset}, is available for purification. Such an assumption poses practical challenges, especially in scenarios where data is scarce. To tackle this challenge, efforts have been made to reduce the size of the required auxiliary dataset~\cite{chai2022oneshot,li2023reconstructive, Zhu_2023_ICCV} and even explore dataset-free purification techniques~\cite{zheng2022data,hong2023revisiting,lin2024fusing}. Although these approaches offer some improvements, recent evaluations \cite{dunnett2024countering, wu2024backdoorbench} continue to highlight the importance of sufficient auxiliary data for achieving robust defenses against backdoor attacks.

While significant progress has been made in reducing the size of auxiliary datasets, an equally critical yet underexplored question remains: \emph{how does the nature of the auxiliary dataset affect purification effectiveness?} In  real-world  applications, auxiliary datasets can vary widely, encompassing in-distribution data, synthetic data, or external data from different sources. Understanding how each type of auxiliary dataset influences the purification effectiveness is vital for selecting or constructing the most suitable auxiliary dataset and the corresponding technique. For instance, when multiple datasets are available, understanding how different datasets contribute to purification can guide defenders in selecting or crafting the most appropriate dataset. Conversely, when only limited auxiliary data is accessible, knowing which purification technique works best under those constraints is critical. Therefore, there is an urgent need for a thorough investigation into the impact of auxiliary datasets on purification effectiveness to guide defenders in  enhancing the security of ML systems. 

In this paper, we systematically investigate the critical role of auxiliary datasets in backdoor purification, aiming to bridge the gap between idealized and practical purification scenarios.  Specifically, we first construct a diverse set of auxiliary datasets to emulate real-world conditions, as summarized in Table~\ref{overall}. These datasets include in-distribution data, synthetic data, and external data from other sources. Through an evaluation of SOTA backdoor purification methods across these datasets, we uncover several critical insights: \textbf{1)} In-distribution datasets, particularly those carefully filtered from the original training data of the victim model, effectively preserve the model’s utility for its intended tasks but may fall short in eliminating backdoors. \textbf{2)} Incorporating OOD datasets can help the model forget backdoors but also bring the risk of forgetting critical learned knowledge, significantly degrading its overall performance. Building on these findings, we propose Guided Input Calibration (GIC), a novel technique that enhances backdoor purification by adaptively transforming auxiliary data to better align with the victim model’s learned representations. By leveraging the victim model itself to guide this transformation, GIC optimizes the purification process, striking a balance between preserving model utility and mitigating backdoor threats. Extensive experiments demonstrate that GIC significantly improves the effectiveness of backdoor purification across diverse auxiliary datasets, providing a practical and robust defense solution.

Our main contributions are threefold:
\textbf{1) Impact analysis of auxiliary datasets:} We take the \textbf{first step}  in systematically investigating how different types of auxiliary datasets influence backdoor purification effectiveness. Our findings provide novel insights and serve as a foundation for future research on optimizing dataset selection and construction for enhanced backdoor defense.
%
\textbf{2) Compilation and evaluation of diverse auxiliary datasets:}  We have compiled and rigorously evaluated a diverse set of auxiliary datasets using SOTA purification methods, making our datasets and code publicly available to facilitate and support future research on practical backdoor defense strategies.
%
\textbf{3) Introduction of GIC:} We introduce GIC, the \textbf{first} dedicated solution designed to align auxiliary datasets with the model’s learned representations, significantly enhancing backdoor mitigation across various dataset types. Our approach sets a new benchmark for practical and effective backdoor defense.



\clearpage
\section{Results}
\subsection{Training MedVAE autoencoders}

Autoencoding methods are capable of encoding high-resolution images as downsized latent representations. For a given 2D input image with dimensions $H \times W$ with $B$ channels, an autoencoding method will output a downsized latent representation of size $H/(\sqrt{f}) \times (W/\sqrt{f}) \times C$. Here, $f$ represents the downsizing factor applied to the 2D area of the image and $C$ represents a pre-specified number of latent channels. 3D autoencoding methods follow a similar formulation, where input images are 3D in nature with dimensions $H \times W \times S$ with $B$ channels. Here, the downsizing factor $f$ is applied to the 3D volume of the image; as a result, the latent representation will have dimensions $(H/(\sqrt[3]{f}) \times (W/\sqrt[3]{f}) \times (S/(\sqrt[3]{f}) \times C$. Autoencoding methods are also capable of decoding latent representations back to reconstructed high-resolution images. 

We aim to develop large-scale, generalizable medical image autoencoders capable of preserving diverse clinically-relevant features in both latent representations and reconstructions. To this end, we first collect a large-scale training dataset with 1,021,356 2D images and 31,374 3D images curated from 19 multi-institutional, open-source datasets~\cite{johnson2019mimic,feng2021candid,jeong2022emory,sorkhei2021csaw,rsnamammo,nguyen2022vindrmammo,moreira2012inbreast,cai2023online,jack2008alzheimer,dagley2017harvard,insel2020a4,lamontagne2019oasis,bien2018deep,hooper2021impact,chilamkurthy2018development,wasserthal2023totalsegmentator,ji2022amos,armato2011lung,stanfordaimi_coca_2024}. Images are obtained from two chest X-ray datasets, six full-field digital mammogram (FFDM) datasets, four T1- and T2-weighted head magnetic resonance imaging (MRI) datasets, one knee MRI dataset, two head/neck CT datasts, two whole-body CT datasets, and two chest CT datasets.

% ******* Figure ********
\begin{figure}[ht]
\centering
\includegraphics[width=\textwidth, trim=0 0 0 0]{figures/method.pdf}
\caption{\textbf{Overview of training pipeline and evaluation tasks for MedVAE, a suite of large-scale autoencoders for medical images.} \textbf{a,} We train MedVAE autoencoders using a two-stage process. The first stage involves training base autoencoders using 2D images. \textbf{b,} The second stage of training aims to further refine the latent space such that clinically-relevant features are preserved across modalities. We introduce separate training procedures for 2D images (e.g. X-rays, mammograms) and 3D volumes (e.g. CT scans, MRI). \textbf{c,} We evaluate medical image autoencoders with respect to latent representation quality, storage and efficiency benefits arising from using latent representations rather than high-resolution images in downstream CAD pipelines, and reconstructed image quality.}
\label{fig:method}
\end{figure}


We then utilize this dataset to train a family of generalizable autoencoders for medical images. Motivated by prior work on natural images~\cite{rombach2022high}, we utilize variational autoencoders (VAEs) as the model backbone. We perform model training using a novel two-stage training scheme designed to optimize quality of latent representations and decoded reconstructions. Specifically, the first stage involves training base autoencoders using 2D images (Fig.~\ref{fig:method}a); we maximize the perceptual similarity between input images and reconstructed images using a perceptual loss~\cite{lpips}, a patch-based adversarial objective~\cite{isola2018patchgan}, and a domain-specific embedding consistency loss. Whereas existing works on autoencoders train using only this stage, the medical image domain introduces the added complexity of subtle, fine-grained features required for clinical interpretation; thus, we introduce a second stage of training, which aims to further refine the latent space such that clinically-relevant features are preserved across various modalities (Fig.~\ref{fig:method}b). Specifically, in the context of 2D imaging modalities (e.g. X-rays, mammograms), the second training stage takes the form of a lightweight training approach that leverages the embedding space of BiomedCLIP, a recently-developed medical vision-language foundation model~\cite{zhang2023biomedclip}, to enforce feature consistency between input images and latent representations. In the context of 3D imaging modalities (e.g. CT scans, MRI), the second training stage involves lifting the 2D autoencoder architecture to 3D and performing continued fine-tuning with 3D images. In Appendix Table~\ref{table:ablations2d} and Appendix Table~\ref{table:ablations3d}, we analyze the effects of each stage of training on latent representation quality. In total, the MedVAE family includes 4 large-scale 2D autoencoders and 2 large-scale 3D autoencoders trained with various downsizing factors $f$ and latent channels $C$. 

In order to assess the capabilities of MedVAE, we evaluate the extent to which latent representations and reconstructed images generated by MedVAE autoencoders can contribute to downstream storage and efficiency benefits while simultaneously preserving clinically-relevant features (Fig.~\ref{fig:method}c). Specifically, we assess (1) whether downsized latent representations can effectively replace high-resolution images in CAD pipelines while maintaining performance; (2) whether latent representations can reduce storage requirements and improve downstream efficiency; and (3) whether decoded reconstructions effectively preserve clinically-relevant features necessary for radiologist interpretation. 

\subsection{Latent representation quality}


\begin{table*}[t]
% \scriptsize

\centering
\resizebox{\linewidth}{!}{%
{%\renewcommand{\arraystretch}{1.2}%
\begin{tabular}{ lcccccccc }
\toprule
\textbf{}
& \multicolumn{2}{c}{\textbf{}}
& \multicolumn{5}{c}{\textbf{AUROC} $\uparrow$}
& \textbf{}
\\
\cmidrule(l{3pt}r{0pt}){4-8}
\cmidrule(l{3pt}r{0pt}){9-9}
% "
\textbf{Method}
& \textbf{$f$}
& \textbf{$C$}
& \small Malignancy
& \small Calcification
& \small BI-RADS
& \small Bone Age
& \small Wrist Fracture
& Average
\\ 
 & & &  \small (FFDM) & \small  (FFDM) & \small  (FFDM) &  (X-ray) &  (X-ray)
\\
\midrule
\small High-Resolution & 1 & 1 & \textbf{66.1$_{\pm0.5}$} & \textbf{62.4$_{\pm0.6}$} & \textbf{63.4$_{\pm0.1}$} & \textbf{80.2$_{\pm0.1}$}  & \textbf{73.7$_{\pm0.0}$} & \textbf{69.2}\\

\midrule
\small Nearest & 16 & 1   &  65.5$_{\pm0.1}$ & 59.7$_{\pm0.3}$ & 62.4$_{\pm0.1}$ & \textcolor{blue}{81.6$_{\pm0.1}$}  & 70.5$_{\pm0.0}$ & 67.9\\
\small Bilinear & 16 & 1    &65.5$_{\pm0.1}$ & 58.1$_{\pm0.3}$ & 61.1$_{\pm0.2}$ & \textcolor{blue}{81.6$_{\pm0.0}$}  & \textbf{71.2$_{\pm0.1}$} & 67.5 \\
\small Bicubic & 16 & 1   &  65.5$_{\pm0.4}$ & 58.5$_{\pm0.5}$ & 61.1$_{\pm0.0}$ & \textcolor{blue}{81.8$_{\pm0.2}$}  & 71.1$_{\pm0.1}$ & 67.6\\
\small KL-VAE & 16 & 3   & 59.7$_{\pm0.2}$ & 59.1$_{\pm0.3}$ & 58.5$_{\pm0.1}$ & 74.3$_{\pm0.1}$ & 64.5$_{\pm0.1}$ & 63.2\\
\small VQ-GAN & 16 & 3   & 57.4$_{\pm0.3}$  & 58.2$_{\pm0.4}$ & 62.3$_{\pm0.1}$ & 79.1$_{\pm0.2}$ & 65.8$_{\pm0.1}$ & 64.6\\
\small 2D MedVAE & 16 & 1   &  63.6$_{\pm0.6}$ & \textcolor{blue}{\textbf{63.9$_{\pm0.4}$}} & \textcolor{blue}{\textbf{65.3$_{\pm0.2}$}} & \textcolor{blue}{\textbf{84.6$_{\pm0.1}$}} & 70.3$_{\pm0.1}$ & \textcolor{blue}{\textbf{69.5}}\\
\small 2D MedVAE & 16  & 3 &  \textcolor{blue}{\textbf{66.1$_{\pm0.2}$}} &   61.7$_{\pm0.2}$ & 62.3$_{\pm0.1}$ & \textcolor{blue}{82.1$_{\pm0.1}$}  &   70.6$_{\pm0.1}$ & 68.6\\
\midrule

\small Nearest & 64 & 1    & 63.0$_{\pm0.1}$ & 58.8$_{\pm0.2}$ & 60.0$_{\pm0.2}$ & 72.1$_{\pm0.0}$  & 65.1$_{\pm0.1}$ & 63.8	\\
\small Bilinear & 64 & 1   & 61.5$_{\pm0.3}$ & 56.9$_{\pm0.4}$ & \textbf{61.3$_{\pm0.1}$} & 72.8$_{\pm0.5}$  & \textbf{67.9$_{\pm0.1}$} & 64.1\\
\small Bicubic & 64 & 1   & 61.2$_{\pm0.5}$ & 57.6$_{\pm0.4}$ & 61.1$_{\pm0.1}$ & 72.8$_{\pm0.2}$  & 67.9$_{\pm0.2}$ & 64.1\\
\small KL-VAE & 64 & 4   & 62.2$_{\pm0.7}$ &  55.8$_{\pm0.4}$ & 56.8$_{\pm0.1}$ & 65.7$_{\pm0.0}$ & 58.8$_{\pm0.0}$ & 59.9\\
\small VQ-GAN & 64 & 4    & 64.5$_{\pm0.5}$ & 57.3$_{\pm0.3}$ &  56.6$_{\pm0.1}$ & 67.6$_{\pm0.1}$  & 61.6$_{\pm0.2}$ & 61.5 \\
\small 2D MedVAE & 64 & 1  & 59.0$_{\pm0.3}$ & \textbf{59.4$_{\pm0.7}$} & 60.7$_{\pm0.1}$ & \textbf{73.5$_{\pm0.2}$} & 64.3$_{\pm0.1}$ & 63.4\\
\small 2D MedVAE & 64 & 4   & \textbf{64.9$_{\pm0.2}$} &  58.5$_{\pm0.3}$ & 60.6$_{\pm0.0}$ & 73.0$_{\pm0.2}$ & 66.7$_{\pm0.1}$ & \textbf{64.7}\\

\bottomrule
\end{tabular}
}
}
\caption{\textbf{Evaluating latent representation quality with 2D CAD tasks.} We evaluate the 2D MedVAE autoencoders on five 2D CAD tasks, and we report the mean AUROC and standard deviation across three random seeds. We compare MedVAE with three interpolation methods (nearest, bilinear, bicubic) and two natural image autoencoders (KL-VAE and VQ-GAN). Here, $f$ represents the downsizing factor applied to the 2D area of the input image and $C$ represents the number of latent channels. The best performing models on each task are bolded. We highlight methods that perfectly preserve clinically-relevant features (i.e. performance equals or exceeds performance when training with high-resolution images) in \textcolor{blue}{\textbf{blue}}.}
\label{table:image_classification}
\vspace{-1mm}
\end{table*}


\begin{table*}[ht]
% \scriptsize
\centering
{%
\begin{tabular}{ lcccccc }
\toprule
\textbf{}
& \multicolumn{2}{c}{\textbf{}}
& \multicolumn{3}{c}{\textbf{AUROC} $\uparrow$}
& \textbf{}
\\
\cmidrule(l{3pt}r{0pt}){4-6}
\cmidrule(l{3pt}r{0pt}){7-7}
\textbf{Method}
& \textbf{$f$}
& \textbf{$C$}
& \small Spine Fractures
& \small Skull Fractures
& \small Knee Injury
& Average
\\ 
 & & &  \small (CT) & \small (CT) & \small (MRI) &
\\
\midrule
\small High-Resolution & 1 & 1  & \textbf{82.9$_{\pm2.2}$} & \textbf{63.9$_{\pm6.3}$} & \textbf{69.9$_{\pm0.6}$} & \textbf{72.2}\\

\midrule
\small Bicubic & 64 & 1  &  77.3$_{\pm4.1}$ & \textcolor{blue}{64.8$_{\pm4.0}$} & 66.4$_{\pm2.3}$ & 69.5\\
\small KL-VAE & 64 & 3   & 68.8$_{\pm2.1}$ & 40.7$_{\pm9.1}$ & 63.9$_{\pm8.2}$  &  57.8\\
\small VQ-GAN & 64 & 3 & 73.2$_{\pm2.0}$  & 75.5$_{\pm14.8}$ & 63.6$_{\pm10.5}$ &   70.8 \\
\small 3D MedVAE & 64 & 1  & \textcolor{blue}{\textbf{83.7$_{\pm2.8}$}} & \textcolor{blue}{\textbf{87.0$_{\pm7.3}$}} & \textbf{68.4$_{\pm2.4}$} & \textcolor{blue}{\textbf{79.7}}\\
\midrule
\small Bicubic & 512 & 1 & \textbf{72.3$_{\pm2.2}$} & 38.4$_{\pm24.5}$ & \textbf{59.4$_{\pm2.5}$} & 56.7\\
\small KL-VAE & 512 & 4  & 67.7$_{\pm3.9}$ & 42.6$_{\pm4.0}$ & 50.9$_{\pm5.1}$ & 53.7\\
\small VQ-GAN & 512 & 4  & 68.9$_{\pm7.0}$ & 30.6$_{\pm12.5}$ & 57.4$_{\pm5.0}$ & 52.3 \\
\small 3D MedVAE & 512 & 1  & 72.0$_{\pm3.8}$ & \textbf{49.1$_{\pm19.8}$} & 58.2$_{\pm1.7}$ & \textbf{59.8} \\
\bottomrule
\end{tabular}
}
\caption{\textbf{Evaluating latent representation quality with 3D CAD tasks.} We evaluate the 3D MedVAE autoencoders on three 3D CAD tasks, and we report the mean AUROC and standard deviation across three random seeds. We compare MedVAE with one interpolation method (bicubic) and two natural image 2D autoencoders (KL-VAE and VQ-GAN). For 2D baselines, we stitch 2D latent representations together across slices such that the size of the 2D latent representation matches those generated by 3D models. Here, $f$ represents the downsizing factor applied to the 3D volume of the input image and $C$ represents the number of latent channels. The best performing models on each task are bolded. We highlight methods that perfectly preserve clinically-relevant features (i.e. performance equals or exceeds performance when training with high-resolution volumes) in \textcolor{blue}{\textbf{blue}}.}
\label{table:3D_latent_cls}
\vspace{-1mm}
\end{table*}





We first evaluate whether clinically-relevant features are preserved in MedVAE latent representations. To this end, we measure the extent to which latent representations can serve as drop-in replacements for high-resolution input images in CAD pipelines \textit{without} any customization or modifications to CAD model architectures. 

We evaluate latent representation quality using the following 8 CAD tasks: malignancy detection on 2D FFDMs~\cite{cai2023online}, calcification detection on 2D FFDMs~\cite{cai2023online}, BI-RADS prediction on 2D FFDMs~\cite{nguyen2022vindrmammo}, bone age prediction on 2D X-rays~\cite{rsnaboneage}, fracture detection on 2D wrist X-rays~\cite{Nagy2022wristfrac}, fracture detection on 3D spine CTs~\cite{loffler2020vertebral}, fracture classification on 3D head CTs~\cite{chilamkurthy2018development}, and anterior cruciate liagment (ACL) and meniscal tear detection on 3D sagittal knee MRIs~\cite{bien2018deep}. In order to perform each of these CAD tasks, a model must rely on fine-grained, clinically-relevant features.% we quantitatively evaluate this by selecting tasks where using naive image downsizing techniques (i.e. interpolation methods) contributes to degraded performance when compared to using high-resolution images. 

For each CAD task, we train a classifier (HRNet~\cite{wang2020hrnet} in 2D settings and SEResNet~\cite{hu2018squeeze} in 3D settings) on a training set consisting of latent representations. We then measure the difference in classification performance between models trained directly on latent representations and models trained using original, high-resolution images; this serves as an indicator of latent representation quality (e.g. a small performance difference indicates that the downsizing approach preserves diagnostic features). We compute AUROC for all binary tasks and macro AUROC for all multi-class tasks. We train each classifier with three random seeds, and we report results as mean AUROC $\pm$ standard deviation.

We compare MedVAE with two categories of image downsizing methods: (1) interpolation methods (nearest, bilinear, and bicubic), which are the de-facto gold standard for medical image downsizing as demonstrated by the quantity of prior work leveraging this approach~\cite{wantlin2023benchmd, Varma2019, convirt, Huang_2021_ICCV, miura2021improving, Tiu2022}, and (2) recently-introduced large-scale natural image autoencoders (KL-VAE and VQ-GAN)~\cite{rombach2022high}. Due to the fact that prior work on developing large-scale 3D autoencoders has been limited, we compare our 3D MedVAE models with 2D methods by stitching 2D latent representations together across slices such that the size of the 2D latent representation matches those generated by 3D models. 

We provide results for 2D CAD tasks in Table \ref{table:image_classification} and 3D CAD tasks in Table \ref{table:3D_latent_cls}. Our results demonstrate that the MedVAE training approach yields high-quality latent representations for both 2D and 3D images. At a downsizing factor of $f=16$, 2D MedVAE perfectly preserves clinically-relevant features on four out of five 2D classification tasks. Similarly, at a downsizing factor of $f=64$, 3D MedVAE perfectly preserves relevant clinical information on two out of three 3D classification tasks (spine and skull CT fracture detection). In these cases, performance equals or exceeds performance when training with original, high-resolution images. The average performance of 2D MedVAE with $f=16$ and 3D MedVAE with $f=64$ across all tasks also exceeds the average performance when training with high-resolution images.  

We observe that MedVAE consistently outperforms the natural image autoencoders KL-VAE and VQ-GAN on all classification tasks, demonstrating the utility of the MedVAE training procedure. On average across the five 2D classification tasks, 2D MedVAE demonstrates a 10.0\% improvement over KL-VAE at a downsizing factor of $f=16$ and a 8.0\% improvement at a downsizing factor of $f=64$. Similar trends are noted for VQ-GAN. %, with MedVAE demonstrating a 7.6\% improvement at $f=16$ and a 5.2\% improvement at $f=64$.
In particular, 2D MedVAE outperforms KL-VAE and VQ-GAN on the two musculoskeletal tasks (bone age prediction and wrist fracture detection) despite the fact that no musculoskeletal radiographs are used during MedVAE training; this suggests effective generalization to other types of medical images. On average across the three 3D tasks, 3D MedVAE demonstrates a 37.9\% improvement over KL-VAE at a downsizing factor of $f=64$ and a 11.4\% improvement over KL-VAE at a downsizing factor of $f=512$. Our findings suggest that 3D training of autoencoders leads to high-quality latent representations due to preservation of volumetric information (e.g. fractures spanning multiple slices), particularly at $f=64$. These findings are corroborated by results in Appendix Table~\ref{table:2d3dcad}, which compares performance of 2D MedVAE and 3D MedVAE on 3D CAD tasks.

Additionally, we note that MedVAE outperforms the interpolation methods across most tasks, but interpolation methods are a competitive baseline. Overall, our findings suggest that our MedVAE training procedure yields downsized latent representations that can be used as drop-in replacements for high-resolution input images in CAD pipelines. 

\subsection{Storage and efficiency benefits of latent representations}

Next, we evaluate the extent to which downsized MedVAE latent representations can reduce storage requirements and improve downstream efficiency of CAD pipelines when compared to high-resolution input images. Using a 2D high-resolution network (HRNet\_w64) and 3D squeeze-excitation network (SEResNet-152) as our base CAD architectures, we report latency, throughput, and maximum batch size. Latency is the time (in milliseconds) to perform a forward pass of the network on one batch. Throughput is the number of samples that can be evaluated by the network in one second. Finally, we report the maximum batch size (in powers of 2) for a forward pass that will fit on a single A100 GPU (2D) and an A6000 GPU (3D). We assume a high-resolution input image size of $1024 \times 1024$ with 1 channel for 2D settings and an input volume size of $256 \times 256 \times 256$ with 1 channel for 3D settings.

% ******* Figure ********
\begin{figure}[ht]
\centering
\includegraphics[width=\textwidth, trim=0 0 0 0]{figures/efficiency.pdf}
\caption{\textbf{CAD model efficiency.} Here, we compare the efficiency of CAD models trained with downsized latent representations to CAD models trained with high-resolution images. $f$ represents the downsizing factor applied to the 2D area or 3D volume of the input image. We report latency (in milliseconds), throughput (in samples per second), and the maximum batch size (in powers of 2) that will fit on one GPU.}
\label{fig:efficiency}
\end{figure}

Results are provided in Figure \ref{fig:efficiency}. We demonstrate that training CAD models directly on downsized latent representations can lead to large improvements in model efficiency. In the 2D setting, we observe that as the downsizing factor increases to $f=64$, the latency decreases by 69x, the throughput increases by 70x, and the maximum batch size increases by 32x. In the 3D setting, as the downsizing factor increases to $f=512$, the latency decreases by 62x, the throughput increases by 55x, and the maximum batch size increases by 512x. Storage costs decrease proportionally with the downsizing factor (i.e. 64x for 2D and 512x for 3D).

\subsection{Reconstructed image quality}

We evaluate whether clinically-relevant features are preserved in reconstructed images using both automated and manual perceptual quality evaluations. These evaluations quantify the extent to which the encoding and subsequent decoding processes retain relevant features.

For automated evaluations, we use perceptual metrics to compare reconstructed images with the original inputs. We report peak signal-to-noise ratio (PSNR) and the multi-scale structural similarity index measure (MS-SSIM). For 2D evaluations, we measure perceptual quality on X-rays~\cite{feng2021candid,johnson2019mimic}; FFDMs~\cite{jeong2022emory,sorkhei2021csaw,rsnamammo,nguyen2022vindrmammo,moreira2012inbreast,cai2023online}; and musculoskeletal X-rays~\cite{Nagy2022wristfrac}. In addition to full-image evaluations, we additionally include a fine-grained perceptual quality assessment, where we extract regions containing wrist fractures by using bounding boxes~\cite{Nagy2022wristfrac}; then, the original region and reconstructed region are compared using perceptual metrics. For 3D evaluations, we compute metrics on brain MRIs~\cite{jack2008alzheimer,dagley2017harvard,insel2020a4,lamontagne2019oasis}; head CTs~\cite{chilamkurthy2018development}; abdomen CTs~\cite{ji2022amos}; CTs from a wide range of anatomies~\cite{wasserthal2023totalsegmentator}; lung CTs~\cite{armato2011lung}; and knee MRIs~\cite{bien2018deep}.


\begin{table*}[t]
% \scriptsize
\centering
\resizebox{\linewidth}{!}{
\begin{tabular}{lccccccccc}
\toprule
\textbf{Method}
& \textbf{$f$}
& \textbf{$C$}
& \multicolumn{2}{c}{\textbf{Mammograms}}
& \multicolumn{2}{c}{\textbf{Chest X-rays}}
& \multicolumn{2}{c}{\textbf{Musculoskeletal X-rays}}
& \multicolumn{1}{c}{\textbf{Wrist X-rays (FG)}}
\\
\cmidrule(l{3pt}r{0pt}){4-5}
\cmidrule(l{3pt}r{0pt}){6-7}
\cmidrule(l{3pt}r{0pt}){8-9}
\cmidrule(l{3pt}r{0pt}){10-10}
% "
& 
&
& \small PSNR $\uparrow$
& \small MS-SSIM $\uparrow$
& \small PSNR $\uparrow$
& \small MS-SSIM $\uparrow$
& \small PSNR $\uparrow$
& \small MS-SSIM $\uparrow$
& \small PSNR $\uparrow$
\\ 
\midrule
\small Nearest & 16 & 1    &  25.95$_{\pm0.06}$  & 0.846$_{\pm0.00}$  & 29.87$_{\pm0.04}$ & 0.942$_{\pm0.00}$   &  24.06$_{\pm0.02}$  & 0.890$_{\pm0.00}$  & 26.11$_{\pm0.02}$\\
\small Bilinear & 16 & 1    & 30.18$_{\pm0.07}$  & 0.936$_{\pm0.00}$ & 34.23$_{\pm0.03}$ & 0.981$_{\pm0.00}$ & 28.75$_{\pm0.02}$  & 0.959$_{\pm0.00}$ & 30.92$_{\pm0.03}$ \\
\small Bicubic & 16 & 1    & 31.69$_{\pm0.07}$  & 0.961$_{\pm0.00}$ & 35.48$_{\pm0.03}$ & 0.989$_{\pm0.00}$ & 30.18$_{\pm0.02}$  & 0.974$_{\pm0.00}$ & 32.65$_{\pm0.04}$  \\
\small KL-VAE & 16 & 3  & 36.11$_{\pm0.07}$  & 0.989$_{\pm0.00}$ & 41.45$_{\pm0.04}$ & 0.996$_{\pm0.00}$ & 38.29$_{\pm0.03}$  &0.992$_{\pm0.00}$  & 36.55$_{\pm0.03}$\\
\small VQ-GAN & 16 & 3  &   35.55$_{\pm0.07}$  & 0.986$_{\pm0.00}$ & 37.80$_{\pm0.03}$ & 0.995$_{\pm0.00}$ & 36.41$_{\pm0.02}$  & 0.990$_{\pm0.00}$  & 34.19$_{\pm0.04}$\\
\small 2D MedVAE  & 16 & 1  & 32.34$_{\pm0.07}$ &  0.969$_{\pm0.00}$ & 38.44$_{\pm0.02}$ & 0.990$_{\pm0.00}$ & 33.97$_{\pm0.03}$ & 0.973$_{\pm0.00}$  &  31.97$_{\pm0.03}$ \\
\small 2D MedVAE  & 16 & 3  & \textbf{37.57}$_{\pm0.08}$ &  \textbf{0.993}$_{\pm0.00}$ & \textbf{43.55 }$_{\pm0.02}$& \textbf{0.997}$_{\pm0.00}$ & \textbf{39.41}$_{\pm0.04}$ & \textbf{0.994}$_{\pm0.00}$  &  \textbf{37.61}$_{\pm0.02}$  \\
\midrule

\small Nearest & 64 & 1   & 22.46$_{\pm0.05}$  & 0.669$_{\pm0.00}$ & 26.22$_{\pm0.03}$ & 0.858$_{\pm0.00}$ & 19.93$_{\pm0.02}$  & 0.756$_{\pm0.00}$  & 22.14$_{\pm0.04}$  \\
\small Bilinear & 64 & 1    & 26.81$_{\pm0.06}$  & 0.837$_{\pm0.00}$ & 31.18$_{\pm0.03}$ & 0.949$_{\pm0.00}$ & 24.89$_{\pm0.01}$  & 0.898$_{\pm0.00}$ & 27.12$_{\pm0.03}$  \\
\small Bicubic & 64 & 1    & 27.84$_{\pm0.06}$  & 0.874$_{\pm0.00}$ & 32.09$_{\pm0.03}$ & 0.962$_{\pm0.00}$  & 25.92$_{\pm0.01}$  & 0.922$_{\pm0.00}$  & 28.54$_{\pm0.03}$ \\
\small KL-VAE & 64 & 4    & 31.88$_{\pm0.07}$  & 0.959$_{\pm0.00}$ &36.37$_{\pm0.01}$ &0.987$_{\pm0.00}$ &33.49$_{\pm0.02}$ &0.966$_{\pm0.00}$ & 31.04$_{\pm0.03}$ \\
\small VQ-GAN & 64 & 4   & 30.13$_{\pm0.06}$  & 0.938$_{\pm0.00}$ & 34.87$_{\pm0.02}$ & 0.980$_{\pm0.00}$ & 32.00$_{\pm0.02}$  & 0.953$_{\pm0.0}$  & 29.92$_{\pm0.02}$ \\
\small 2D MedVAE  & 64 & 1   & 28.00$_{\pm0.07}$ & 0.872$_{\pm0.00}$ & 31.92$_{\pm0.04}$ &  0.962$_{\pm0.00}$ & 28.27$_{\pm0.02}$&  0.917$_{\pm0.00}$  & 28.03$_{\pm0.01}$\\
\small 2D MedVAE  & 64 & 4  & \textbf{33.13}$_{\pm0.07}$  & \textbf{0.969}$_{\pm0.00}$ & \textbf{38.88}$_{\pm0.03}$ &  \textbf{0.990}$_{\pm0.00}$ & \textbf{34.73}$_{\pm0.02}$&  \textbf{0.972}$_{\pm0.00}$& \textbf{32.30}$_{\pm0.02}$\\
\bottomrule
\end{tabular}
}
\caption{\textit{Evaluating reconstruction quality on 2D datasets.} We evaluate 2D MedVAE with perceptual quality metrics on mammograms and chest X-rays, which we classify as \textit{in-distribution}, since the MedVAE training set includes mammograms and chest X-rays. We also evaluate MedVAE on musculoskeletal X-rays and wrist X-rays (fine-grained), which we classify as \textit{out-of-distribution}. Here, $f$ represents the downsizing factor applied to the 2D area of the input image and $C$ represents the number of latent channels. The best performing models are bolded. We calculate PSNR and MS-SSIM using a random sample of 1000 images for each image type; we report mean and standard deviations across four runs with different random seeds.}
\label{table:perceptualid}
\end{table*}




\begin{table*}[t]
% \scriptsize
\centering
\resizebox{\linewidth}{!}{
\begin{tabular}{lcccccccccccccc}
\toprule
\textbf{Method}
& \textbf{$f$}
& \textbf{$C$}
& \multicolumn{2}{c}{\textbf{Brain MRIs}}
& \multicolumn{2}{c}{\textbf{Head CTs}}
& \multicolumn{2}{c}{\textbf{Abdomen CTs}}
& \multicolumn{2}{c}{\textbf{TS CTs}}
& \multicolumn{2}{c}{\textbf{Lung CTs}}
& \multicolumn{2}{c}{\textbf{Knee MRIs}}
\\
\cmidrule(l{3pt}r{0pt}){4-5}
\cmidrule(l{3pt}r{0pt}){6-7}
\cmidrule(l{3pt}r{0pt}){8-9}
\cmidrule(l{3pt}r{0pt}){10-11}
\cmidrule(l{3pt}r{0pt}){12-13}
\cmidrule(l{3pt}r{0pt}){14-15}
% "
& 
&
& \small PSNR $\uparrow$
& \small MS-SSIM $\uparrow$
& \small PSNR $\uparrow$
& \small MS-SSIM $\uparrow$
& \small PSNR $\uparrow$
& \small MS-SSIM $\uparrow$
& \small PSNR $\uparrow$
& \small MS-SSIM $\uparrow$
& \small PSNR $\uparrow$
& \small MS-SSIM $\uparrow$
& \small PSNR $\uparrow$
& \small MS-SSIM $\uparrow$
\\ 
\midrule
\small Bicubic & 16 & 1 & 29.27 & 0.975 & 36.21 & 0.996 & 33.81 & 0.989 & 27.33 & 0.972 & 28.00 & 0.973 & 26.37 & 0.986 \\
\small KL-VAE & 16 & 3 & 33.23 & {\textbf{0.994}} & 47.65 & {\textbf{1.000}} & 43.51 & 0.998 & 34.14 & 0.994 & 32.62 & {\textbf{0.989}} & 31.31 & {\textbf{0.998}} \\
\small VQ-GAN & 16 & 3 & 32.72 & 0.992 & 42.87 & 0.999 & 40.85 & 0.997 & 33.55 & 0.993 & 32.20 & {\textbf{0.989}} & 30.75 & 0.997 \\
\small 2D MedVAE  & 16 & 1 & 29.48 & 0.980 & 39.71 & 0.997 & 33.45 & 0.983 & 29.70 & 0.983 & 28.40 & 0.973 & 27.38 & 0.990 \\
\small 2D MedVAE  & 16 & 3 & {\textbf{33.99}} & {\textbf{0.994}} & {\textbf{48.56}} & {\textbf{1.000}} & {\textbf{44.95}} & {\textbf{0.999}} & {\textbf{34.83}} & {\textbf{0.995}} & {\textbf{33.34}} & {\textbf{0.989}} & {\textbf{31.52}} & 0.997 \\
\small 3D MedVAE  & 64 & 1 & 29.52 & 0.983 & 39.03 & 0.999 & 36.61 & 0.993 & 31.35 & 0.987 & 28.79 & 0.975 & 28.25 & 0.994 \\
\midrule
\small Bicubic & 64 & 1 & 26.25 & 0.911 & 30.11 & 0.980 & 28.84 & 0.955  & 24.24 & 0.914 & 24.40 & 0.928 & 24.11 & 0.956  \\
\small KL-VAE & 64 & 3 & 29.32 & {\textbf{0.977}} & 40.95 & 0.997 & 38.07 & {\textbf{0.995}} & 29.85 & 0.982 & 28.83 & 0.974 & 27.68 & {\textbf{0.993}} \\
\small VQ-GAN & 64 & 3 & 27.43 & 0.967 & 39.02 & 0.997 & 36.25 & 0.991 & 27.47 & 0.972 & 26.66 & 0.964 & 25.95 & 0.990 \\
\small 2D MedVAE  & 64 & 1 & 25.66 & 0.920 & 33.10 & 0.988 & 29.51 & 0.967  & 24.50 & 0.922 & 24.39 & 0.933 & 24.48 & 0.973 \\
\small 2D MedVAE  & 64 & 3 & {\textbf{29.34}} & 0.976 & {\textbf{41.98}} & {\textbf{0.999}} & {\textbf{39.49}} & {\textbf{0.995}} & {\textbf{30.35}} & {\textbf{0.984}} & {\textbf{29.59}} & {\textbf{0.977}} & {\textbf{28.05}} & {\textbf{0.993}} \\
\small 3D MedVAE  & 512 & 1 & 26.23 & 0.937 & 30.85 & 0.991 & 29.47 & 0.960 & 26.34 & 0.949 & 24.76 & 0.934 & 24.36 & 0.977 \\

\bottomrule
\end{tabular}
}
\caption{\textit{Evaluating reconstruction quality on 3D datasets.} We evaluate 3D MedVAE with perceptual quality metrics on head MRIs, head CTs, abdomen CTs, various high-resolution CTs (TS), lung CTs, and knee MRIs. $f$ represents the downsizing factor applied to the input volume and $C$ represents the number of latent channels. The best performing models are bolded. We compare 3D MedVAE with several 2D methods, including 2D MedVAE, KL-VAE, and VQ-GAN.}
\label{table:3dperceptual}
\end{table*}


In Table \ref{table:perceptualid}, we compare 2D MedVAE with interpolation methods and large-scale natural image autoencoders across four types of 2D images. We find that 2D MedVAE achieves the highest perceptual quality across all evaluated image types. In particular, our evaluations with wrist X-rays explore generalization of MedVAE to unseen anatomical features; notably, MedVAE achieves the highest PSNR scores on this task, despite the fact that MedVAE was not trained on musculoskeletal X-rays. We also note a general trend that increasing the number of latent channels $C$ improves perceptual quality of the reconstructed image. 

In Table \ref{table:3dperceptual}, we compare MedVAE with interpolation methods and large-scale natural image autoencoders across six types of 3D volumes. Due to the absence of existing large-scale 3D autoencoder baselines, we compare our 3D MedVAE models with 2D methods by performing downsizing on individual 2D slices and then stitching slices together to form the reconstructed 3D volume. We again find that MedVAE reconstructions demonstrate superior perceptual quality when compared to baselines. In particular, 2D MedVAE achieves the highest perceptual quality across almost all evaluated image types, despite the fact that no MRI or CT slices were included in the 2D MedVAE training set. We also observe that 3D MedVAE achieves competitive performance, despite utilizing a significantly higher downsizing factor than comparable 2D methods (i.e. downsizing across all three dimensions rather than just two). In Appendix Table~\ref{table:decoder}, we compare 3D MedVAE with a model referred to as 2D MedVAE-Decoder, which has a comparable downsizing factor $f$. The 2D MedVAE-Decoder model performs downsizing on individual 2D slices, which are then stitched and interpolated together to form a latent representation of equivalent size to the 3D MedVAE model; we then perform fine-tuning of the decoder using our curated dataset of 3D volumes. The superiority of 3D MedVAE to the 2D MedVAE-Decoder approach demonstrates the utility of 3D training of autoencoders, which enables the model to capture important volumetric patterns. 

% ******* Figure ********
\begin{figure}[h]
\centering
\includegraphics[width=0.9\textwidth, trim=0 0 0 0]{figures/readerstudy.pdf}
\caption{\textbf{Manual perceptual quality evaluations with expert readers.} We report the mean scores from three expert readers on three criteria: fidelity, preservation of relevant features, and artifacts. We compare 2D MedVAE with ($f=16,C=3$) and ($f=64,C=4$) with bicubic interpolation, a standard and widely-used approach for downsizing medical images. Error bars represent 95\% confidence intervals.}
\label{fig:readerstudy}
\end{figure}


Qualitative reader studies by domain experts are critical for ensuring clinical usability of developed methods. We supplement our automated evaluations of reconstructed image quality with a manual reader study. Each reader is presented with a pair of chest X-rays, consisting of an original high-resolution image on the left and a reconstructed image on the right. A total of 50 unique chest X-rays with fractures, randomly sampled from CANDID-PTX, are selected and presented in a randomized order~\cite{feng2021candid}. The reconstructed images are scored on a 5-point Likert scale ranging from -2 to 2 based on three main criteria: image fidelity, preservation of diagnostic features, and the presence of artifacts. Our study involved three radiologists as expert readers. We compared 2D MedVAE with bicubic interpolation, a standard and widely-used approach for downsizing medical images (Figure~\ref{fig:readerstudy}).

For manual evaluations of reconstructed image quality, readers rated image fidelity for 2D MedVAE to be 2.8 points higher than bicubic interpolation averaged across the two downsizing factors. 2D MedVAE also better preserved clinically-relevant features (2.8 points). Artifacts (e.g. blurring, hallucinations) were more frequent in interpolated images (2.6 points), which severely suffered from blurring artifacts with increasing downsizing factors. In summary, our results suggest that 2D MedVAE better preserves diagnostic features than interpolation. In Figure \ref{fig:qualitative}, we provide qualitative examples of a reconstructed chest X-ray and a reconstructed T1-weighted brain MRI slice. 


% ******* Figure ********
\begin{figure}[h]
\centering
\includegraphics[width=0.9\textwidth, trim=0 0 0 0]{figures/qualitative.pdf}
\caption{\textbf{Qualitative examples of reconstructed medical images.} The top section provides qualitative examples of a reconstructed chest X-ray. The bottom section provides qualitative examples of a reconstructed brain MRI slice. Residual figures show pixel-level differences between reconstructed images and original, high-resolution images; brighter colors represent larger differences.}
\label{fig:qualitative}
\end{figure}
\clearpage
\section{Discussion}

High-resolution medical images can result in large data storage costs and increased or intractable computational complexity for trained models. As the volume of data stored by hospitals continues to increase and large-scale foundation models become more commonplace, methods for inexpensively storing and efficiently processing high-resolution medical images become a critical necessity. In this work, we aim to address this need by introducing MedVAE, a family of 6 large-scale autoencoders for medical images developed using a novel two-stage training procedure. MedVAE encodes high-resolution medical images as downsized latent representations. We demonstrate with extensive evaluations that (1) downsized latent representations can effectively replace high-resolution images in CAD pipelines while maintaining or exceeding performance, (2) downsized latent representations reduce storage requirements (up to 512x) and improve downstream efficiency (up to 70x in model throughput) when compared to high-resolution input images, and (3) reconstructed images effectively preserve relevant features necessary for clinical interpretation by radiologists.

Several prior works have introduced powerful autoencoders capable of generating downsized latents for images. In particular, recent work on latent diffusion models has involved the development of several large-scale autoencoders, such as VQ-GANs and VAEs, trained on eight million natural images~\cite{rombach2022high,kingma2013vae,esser2021taming,openimages}; downsized latents generated by these models were shown to capture relevant spatial structure as well as improve efficiency of downstream diffusion model training~\cite{rombach2022high}. However, recent works have demonstrated that models trained on natural images often generalize poorly to medical images due to significant distribution shift~\cite{guan2022,van2023exploring,chambon2022adapting}, suggesting that existing natural image autoencoders may not be well-suited for the complexity of the medical image domain. Our evaluations on both latent representations and reconstructed images support this point, demonstrating that existing large-scale natural image autoencoders consistently underperform our domain-specific medical image autoencoders. These findings demonstrate the need for domain-specific models capable of understanding complex and fine-grained patterns across diverse imaging modalities and anatomical regions.

Our work aims to reduce computational costs associated with automated medical image interpretation by proposing the use of training datasets comprised of downsized MedVAE latent representations rather than high-resolution medical images. For instance, given a chest X-ray training dataset with images of size $1024 \times 1024$ with 1 channel, our 2D MedVAE model with $f=64$ and $C=1$ can generate downsized latent representations of size $128 \times 128$ with 1 channel, contributing to substantial downstream efficiency and storage benefits. We demonstrate with eight CAD tasks that latent representations do not result in the loss of clinically-important information; at a 2D downsizing factor of $f=16$ and a 3D downsizing factor of $f=64$, we observe equivalent or better performance than high-resolution images with substantial improvements over multiple existing downsizing methods. MedVAE models can also generalize beyond the images included in the training set, as shown by performance on 2D musculoskeletal X-rays and 3D spine CTs. Importantly, the efficiency benefits of using latent representations are significant; in particular, using latent representations can contribute to large increases in batch sizes, which can be particularly useful in the modern era of self-supervised foundation models that rely heavily on the use of large batch sizes during training. 

The MedVAE autoencoder family includes two 3D autoencoders that are explicitly designed to downsize 3D medical imaging modalities (e.g. CT, MRI), a previously underresearched setting. Our results demonstrate that at a 3D downsizing factor of $f=64$, the volumetric latent representations generated by 3D MedVAE are substantially higher quality than those generated by stitching together 2D slices downsized using 2D baselines. This suggests that 3D autoencoders can better capture clinically-important volumetric patterns, such as fractures that span multiple slices. Efficiency benefits in the 3D setting are also notable, particularly since training downstream CAD models on high-resolution 3D volumes is often computationally expensive or intractable. At significantly higher downsizing factors ($f=512$), we observe the benefits of 3D autoencoder training to be less pronounced, suggesting that users will need to carefully consider the tradeoffs between latent representation quality and desired downstream efficiency when selecting a MedVAE model.

In addition to generating high-quality latent representations, MedVAE models also include a trained decoder, which can reconstruct the original high-resolution image from the downsized latent. This is a particularly useful capability in the medical imaging domain, since high-resolution images are necessary for effective clinical interpretation by radiologists. We demonstrate with a reader study consisting of three radiologists that reconstructed images can effectively preserve clinically-relevant signal needed for diagnoses; in this setting, fine-grained fractures in chest X-rays were preserved through the encoding and decoding process.

Our study presents several opportunities for future work. First, additional research into model architectures, data augmentation architectures, and training strategies would be useful for building effective downstream CAD models that can learn from latent representations. In addition, the batch size and efficiency benefits afforded by latent representations raise the possibility of training large-scale foundation models using downsized latent representations. Whereas foundation models traditionally require significant computational resources and training time, utilizing downsized latent representations that preserve diagnostic features can greatly accelerate model training, particularly in resource-constrained settings. Future work can explore foundation model performance and scaling laws in this context. Finally, future work can explore additional autoencoder training strategies to better preserve clinically-relevant features at high downsizing factors. 

Overall, our work demonstrates the potential that large-scale, generalizable autoencoders hold in addressing critical storage and efficiency challenges in the medical domain. 

\clearpage
\section{Methodology}
\label{sec:method}
\begin{figure}[!ht]
% \vspace{-1em}
\centering
    \includegraphics[width=0.90\columnwidth]{figures/HIM-framework.pdf}
    % \vspace{-1em}
    \caption{The overall framework of HIM.}
    \label{fig:HIM}
\end{figure}
% \vspace{-2em}
\subsection{Overview}
This work aims to address the IM problem from a new perspective.
We encode potential influence spread trends into hyperbolic representations for the effective selection of highly influential seed users.
Our motivation has two key points.
(1) We aim for a diffusion model agnostic method that solves the IM problem without relying on any assumptions on diffusion parameters.
(2) The influenc trend of users can be efficiently approximated by directly utilizing the properties of learned representations.
To this end, we leverage the benefits of hyperbolic geometry to propose a novel method for IM.

We use the social network and the graph set of influence propagation instances as learning data and apply hyperbolic network embedding to construct user representations.
Instead of explicitly computing users' influence spread, we implicitly estimate their influence spread with the learned representations.
The distance information of the representations can effectively measure the influence spread of seed user nodes.

Specifically, a novel hyperbolic spread learning method HIM is proposed, as is shown in Figure~\ref{fig:HIM}. HIM mainly consists of two modules: (1) \textit{Hyperbolic Influence Representation} aims to learn user representations in the hyperbolic space. (2) \textit{Adaptive Seed Selection} selects target seed users based on learned hyperbolic representations via an adaptive algorithm. 

\subsection{Hyperbolic Influence Representation}
We first encode influence spread features from social influence data to construct user representations in hyperbolic space. The social influence data includes social networks and influence propagation instances, as mentioned in Section~\ref{sec:assume}.
The structural information of the network and the historical spread patterns of propagation instances are crucial for estimating the influence spread of seed users. Both should be effectively integrated into user representations.

% Social connections can be easily obtained from a given social network.
% However, the propagation relations are relatively complicated. Instead of relying on specific diffusion models, we attempt to learn the influence propagation information from the observed data. Given the historical diffusion cascades, we can obtain their propagation instances~\cite{ICDE_feng2018inf2vec}. 
% As mentioned, each instance can be viewed as a directed subgraph $G_D$ of the social network $G$.
% Each instance can be viewed as a directed propagation subgraph of the social network, where an edge $(u \to v)$ denotes that user $u$ influences user $v$. 

The learning process follows a shallow embedding approach.
Both types of data naturally form a graph, enabling effective representation learning on edge sets.
We do not adopt more complex embedding methods as~\cite{KDD2016_grover_node2vec, KDD2017_ribeiro_struc2vec} as we aim to intuitively demonstrate that influence spread can be estimated based on hyperbolic representations learned from social influence data, which is previously unexplored.
Meanwhile, this approach maintains computational efficiency, making it scalable for large-scale social networks.

Given social influence data, we propose a rotation-based Lorentz model to learn hyperbolic user representations. 
Note that this preprocess is model-agnostic, making it adaptable to various diffusion models and practical applications. 

At first, given a social network $G = (V, E)$, we assign each user $u \in V$ an initial representation $\mathbf{x}_u \in \mathbb{L}^{n}_{\gamma}$, initialized via hyperbolic Gaussian sampling as in work~\cite{sun2021hgcf}.

\subsubsection{Rotation Operation.}
We apply hyperbolic rotation operation~\cite{ICLR19rotate, ACL20_chami2020low} to assist in integrating structure and influence spread information for effective representation learning. 
By adjusting angles, various rotation operations capture different types of information, ensuring seamless integration into unified user representations.

In detail, we use two sets of rotation matrices $(\mathbf{Rot}^{S}_{s}, \mathbf{Rot}^{T}_{s})$ and $(\mathbf{Rot}^{S}_{d}, \mathbf{Rot}^{T}_{d})$ to assist in representation learning. Here, $s$ denotes the social relation, while $d$ denotes the propagation relation. $S$ and $T$ denote the rotation operations applied to head nodes and tail nodes, respectively.
The rotation operation further brings extra benefits for IM.
% Employing rotation transformation offers several benefits to learning influence representations for the IM problem.
% First, rotations can capture various symmetric and asymmetric relations among users~\cite{ICLR19rotate,ACL20_chami2020low}.
The rotation operation in representation learning adjusts vectors' angles to bring related user representations closer while preserving their distances, therefore maintaining hierarchical information.
Besides, It is also efficient and easy to implement.

% -------------------------------------------------------------------------------------
% Learn Static Influence
% -------------------------------------------------------------------------------------

\subsubsection{Network Structure Learning.}

In this part, we deduce structure influence from the social connections present in the social network by modeling the edges within the given graph $G=(V, E)$. 
The core idea is to maximize the joint probability of observing all edges in the graph to learn node embeddings.

Specifically, given an observed edge $(u \rightarrow v) \in E$, the probability $\Pr(v|u)$ can be estimated by a score function based on the squared Lorentzian distance:
% $\small \Pr(v|u) = \frac{ \exp(\mathcal{V}^{S}_{uv}) } { Z(u) }$,
$\small \Pr(v|u) = \exp(\mathcal{V}^{S}_{uv})  /  Z(u) $,
% \begin{equation} \small \Pr(v|u) = \frac{ \exp(\mathcal{V}^{S}_{uv}) } { Z(u) }, \label{eq:prob-u-v} \end{equation}
where $Z(u) = \sum_{ o \in V } \exp(\mathcal{V}^{S}_{uo})$, and
edge score $\mathcal{V}^{S}_{uv} $ is defined as:
\begin{equation} 
\small \mathcal{V}^{S}_{uv} = - w_{uv} \cdot d^2_{\mathcal{L}}\left(\mathbf{x}^S_u, \mathbf{x}^T_v\right) + b_u + b_v,
\label{eq:relation-score}
\end{equation}
where $ w_{uv} > 0 $ is the coefficient associated with the edge $(u \rightarrow v)$.
Generally, we set $w_{uv} = 1/d_{u}$.
$b_u$ and $b_v$ represent biases of node $u$ and node $v$, respectively.
$\mathbf{x}^S_u = \mathbf{Rot}_{s}^S(\mathbf{x}_u)$ and $\mathbf{x}^T_v = \mathbf{Rot}_{s}^T(\mathbf{x}_v)$ are the rotated representations. 
Since the normalization term $Z(u)$ is expensive to compute, we approximate it via a negative sampling strategy~\cite{mikolov2013neg-sampling}.
% Note that the normalization term $Z(u)$ is expensive to compute, we approximate it via a new negative sampling strategy: We first divide all nodes into $L$ ranges according to their degrees. 
% When sampling negative nodes for given users, we carry out sample selection in the corresponding range according to their degrees, making refined distinctions among users with similar degrees. 
Therefore, we estimate $\Pr(v|u)$ in the log form as:
\begin{equation}
\small \log P(v|u) \approx \log \varphi \left(\mathcal{V}_{uv} \right) + \sum_{o \in \mathcal{N}_u} \log \varphi \left( - \mathcal{V}_{uo}\right),
\label{eq:log_p_u_v}
\end{equation}
where $\varphi(x) = 1/(1+e^{-x})$ is the Sigmoid function and $\mathcal{N}_u$ is the set of sampled negative nodes.
Assuming they are independent of each other, the joint probability of all social connections can be calculated as:
\begin{equation} \small \mathcal{P} = \sum_{(u,v)\in E} \log P(v|u). \end{equation}
By maximizing this joint probability, we encode the structure information of the social network into user representations.
% Accordingly, our goal is to capture the static influence of all users by maximizing this joint probability.

% -------------------------------------------------------------------------------------
% Learn Dynamic Influence
% -------------------------------------------------------------------------------------

\subsubsection{Influence Propagation Learning.}
Here, we extract historical influence spread patterns from the propagation instance graph sets $\mathcal{G}_D$.
Similarly, given any propagation graph $G^i_D \in \mathcal{G}_D$, we maximize the joint probability of observing influence activations in the $G^i_D$ to encode spread patterns into user embeddings.

In detail, given $G^i_D = (V^i_D, E^i_D)$, the edge probability of $(u \rightarrow v) \in E^i_D$ can be calculated similar to Eq. (\ref{eq:log_p_u_v}) as:
\begin{equation}
\small \log P(v|u) \approx \log \varphi \left(\mathcal{V}^{D}_{uv}\right) + \sum_{o \in \mathcal{N}_u} \log \varphi \left( - \mathcal{V}^{D}_{uo}\right),
\end{equation}
% \;\:
\begin{equation}
\small \mathcal{V}^{D}_{uv} = - w_{uv} \cdot d^2_{\mathcal{L}}\left(\mathbf{x}^S_u, \mathbf{x}^T_v\right) + b_u + b_v,
\label{eq:propagation_score}
\end{equation}
where $ w_{uv} = 1/d_u $ is the coefficient, $\mathbf{x}^S_u = \mathbf{Rot}_{d}^S(\mathbf{x}_u)$ and $\mathbf{x}^T_v = \mathbf{Rot}_{d}^T(\mathbf{x}_v)$ are the rotated user representations. 
The joint probability of all edges in $G^{i}_D$ can be calculated as:
\begin{equation} \small \mathcal{P}_{G^i_D} = \sum_{(u,v)\in E^{i}_D} \log P(v|u). \end{equation}

During the propagation process, once a user $u$ triggers influence activation, we want to assign a bonus to highlight this user’s tendency to positively influence others. 
Inspired by the approach in~\cite{ICML2023_Yang}, we address this intuitively by reducing the hyperbolic distance of the related user representations from the origin in the embedding space.
Thus, for all influence activations in $G^i_D$, we propose a proactive influence regularization term:
\begin{equation}
 \mathcal{I}_{G^i_D} = \sum_{(u,v)\in G^i_D} \alpha_u \cdot \log \varphi \left(d^2_{\mathcal{L}}(\mathbf{x_u}, \mathbf{o}_{\mathcal{L}})\right).
\end{equation}
where $\mathbf{o}_{\mathcal{L}}$ is the origin of the Lorentz model and $\alpha_u$ is calculated as $\sqrt{d_u/d_{\text{max}}}$.
This term further pulls high-influence users closer to the origin in the representation space.
The illustration of learning an observed influence instance $(u \rightarrow v)$ is shown in Figure~\ref{fig:emb}.
\begin{figure}[h]
  \centering
  \includegraphics[width=0.90\columnwidth]{figures/method/do_emb.pdf}
  \caption{ Illustration of the influence propagation learning. The propagation relation between user $u$ and $v$ is depicted by the distance $d^2_{\mathcal{L}}$ between their rotated embeddings. }
  \label{fig:emb}
\end{figure}

For simplicity, we define $LDO$ as the squared Lorentzian distance from a given representation to the origin. Specifically, for user $u$, the $LDO_u$ is defined as $LDO_u = d^2_{\mathcal{L}}(\mathbf{x}_u, \mathbf{o}_{\mathcal{L}})$.
Previous studies~\cite{nickel2017poincare, ICML2023_Yang, feng2022role} have shown that hierarchical information can be effectively inferred from $LDO$s. In our method, user nodes with smaller $LDO$ values are more likely to be influential in social networks. 
We will later design seed selection strategies based on $LDO$.

% -------------------------------------------------------------------------------------
% Objective Function
% -------------------------------------------------------------------------------------

\subsubsection{Objective Function.}

Combining above two parts, the overall loss function is calculated as:
\begin{equation}
\small
\mathcal{L} = - \left( \mathcal{P} + \sum_{G^i_D \in \mathcal{G}_D}\left(\mathcal{P}_{G^{i}_D} + \mathcal{I}_{G^i_D}\right)\right). 
\label{eq:over_loss}
\end{equation}
Optimizing Eq. (\ref{eq:over_loss}) brings relevant nodes closer together while keeping irrelevant nodes as far apart as possible. Meanwhile, users involved in more influence activations tend to have their representations move closer to the origin, indicating potential higher influence spread.
The time complexity can be found in Appendix.

Once the learning process is complete, users with strong spread relations will be clustered together in the embedding space, and highly influential users tend to be located near the origin, which helps to identify seed users for the IM problem.

\subsection{Adaptive Seed Selection} 
\begin{algorithm}[H]
% \small
\caption{Adaptive Sliding Window (ASW)}\label{alg:ASW}
\begin{algorithmic}[1]
\Statex \textbf{Input:} social graph $G$, user representations $\mathbf{X}$, seed number $k$ and window size coefficient $\beta$ 
\Statex \textbf{Output:} $S^*$ with $k$ seed users
\State $S^* \gets$ an empty set, window size $w \gets \beta \cdot k$
\State $D \gets$ compute $ LDO_u = d^2_{\mathcal{L}}(\mathbf{x}_u, \mathbf{o}_{\mathcal{L}}) \text{ for each } u \text{ in } V$ 
\State $\mathcal{Z} \gets \text{sort } D  \text{ in ascending order} $
\State $c \gets$ select the $u$ with minimum $\mathcal{Z}_u$
\State $Q \gets$ a priority queue initialized with key-value pairs $(u, \mathcal{Z}_u)$ for the next $w$ users in $\mathcal{Z}$.
\While{$|S^*| < k$}
\State add $c$ to $S^*$ and find $N_c$ the neighbors of $c$ from $G$
\State $\mathcal{C} = N_c \cap Q_{keys}$
\If{$ \mathcal{C} = \emptyset $}
\State $c = Q.$pop and add the next $(u, \mathcal{Z}_u)$ in $\mathcal{Z}$ to $Q$ 
\Else
\State compute $\mathcal{Z}'_v$ according to Eq. (\ref{eq:update_score}) 
\State update $Q$ with $(v, \mathcal{Z}'_v)$ for each $v$ in $\mathcal{C}$
\State $c = Q.$pop and add the next $(u, \mathcal{Z}_u)$ in $\mathcal{Z}$ to $Q$ 
\EndIf
\EndWhile \textbf{ and return $S^*$} 
\end{algorithmic}
\end{algorithm}

After integrating social influence information into the hyperbolic representations, the next step involves designing strategies to select target seed users based on these learned representations. Specifically, we propose adaptive seed selection, which aims to leverage the geometric properties of the hyperbolic representations to effectively find seed users who possess large influence spread.

In practice, users with high influence might have overlapping areas of influence. Independently selecting highly influential users may not result in optimal overall performance due to the submodularity of social influence~\cite{kempe2003im}. Additionally, the submodularity property of the IM problem implies diminishing marginal gains from seed users~\cite{TKDE18_li2018influence_survey}, particularly for users who are close to the already selected seed users. Therefore, it is crucial to consider these spread relations among users when selecting seed nodes. Previous methods required traversing all nodes, leading to high computational costs. Given that influence strength can be estimated by the distance of representations from the origin and that the spread relations among users can be measured by the distance between their representation vectors, we have designed a new algorithm for seed set selection, which is shown in Algorithm~\ref{alg:ASW}.

The key idea of our strategy is to assign each user an initial score and dynamically adjust these scores during the selection process. Therefore, we could determine the final seed set by considering the spread relation among users. Specifically, we first assign each user with a score $\mathcal{Z}_u = LDO_u = d^2_{\mathcal{L}}(\mathbf{x}_u, \mathbf{o}_{\mathcal{L}})$. We sort all scores $\mathcal{Z}$ and select the node $c$ with the smallest $\mathcal{Z}_c$ as the first seed user. Instead of directly choosing the node with the second lowest $LDO$, the next $w$ nodes in the sorted list are viewed as candidate nodes, where $w$ is the size of a sliding window $W$. The $W$ is used to explore a wider range of candidate nodes while maintaining computational efficiency. We determine the window size $w$ based on $k$ as $w = \beta \cdot k$, allowing it to adaptively adjust its size for different data scales.
In Algorithm~\ref{alg:ASW}, the sliding window $W$ is implemented by a priority queue $Q$.
Next, we find the intersection $\mathcal{C}$ of the current seed node's neighbors with the candidate nodes.
Accordingly, we update the scores of the nodes in $\mathcal{C}$. 
For a user $u \in \mathcal{C}$, the updated score $\mathcal{Z}'_u$ is calculated as:
\begin{equation}
\small
\mathcal{Z}'_u = \mathcal{Z}_u + \frac{w_{c,u}}{d_c} \cdot  \mathcal{Z}_{c},
\label{eq:update_score}
\end{equation}
\begin{equation}
% \;\;
\small
w_{c,u} = \frac{
\exp(1/d^2_{\mathcal{L}}(\mathbf{x}_c, \mathbf{x}_u))
}{ \sum_{v \in \mathcal{C}} \exp(1/d^2_{\mathcal{L}}(\mathbf{x}_c, \mathbf{x}_v))}.
\label{eq:update_score_2}
\end{equation}
Here, $c$ denotes the recently selected node, $d_c$ is the degree of node $c$, and $w_{c,u}$ means the weight between them. Intuitively, a node closer to node $c$ may have a larger spread overlap with $c$, leading to a larger penalty from node $c$ and thus increasing its score.
In this way, the candidates' scores in the sliding window will be updated. After that, we select the node with the lowest score. At each iteration, the chosen node is removed from the window, and the next node from the sorted $LDO$ list is added to the window. This process is repeated until $k$ seed users are selected. 
Due to space limitations, the time complexity analysis can be found in Appendix.

% \subsubsection{Discussion.} 

% The influence strength of user nodes can be effectively measured by the distance of their representations from the space's origin. Meanwhile, the relationship between two users can be efficiently measured by the distance between their representations. Compared to other methods that utilize graph properties, such as the shortest path, to measure the relationship between two nodes, calculating the distance between representation vectors can greatly enhance computational efficiency. Representations in hyperbolic space can effectively measure both the influence of individual users and the social relations among them, enabling the design of efficient algorithms for classic IM problems. 
% Indeed, how to effectively select seed nodes based on the learned hyperbolic representations remains an open question worth further exploration.

% The influence strength of user nodes can be effectively measured by the distance of their representations from the origin in hyperbolic space. Similarly, the relationship between two users can be efficiently assessed by the distance between their respective representations. Compared to traditional methods that rely on graph properties, such as the shortest path, calculating the distance between representation vectors significantly enhances computational efficiency. Thus, we argue that hyperbolic space representations are particularly well-suited for measuring both individual user influence and social relationships, thereby facilitating the design of efficient algorithms for the IM problem.

% Applying two proposed strategies to HIM, we obtain two specific IM methods: HIM-MD and HIM-ASW.
% Later, in the experimental section, we will evaluate the performance of two methods.

% Section Transition
\clearpage

\subsection*{Acknowledgments}
MV is supported by graduate fellowship awards from the Department of Defense (NDSEG), the Knight-Hennessy Scholars program at Stanford University, and the Quad program. AK is supported by graduate fellowships from Tau Beta Pi and the Knight-Hennessy Scholars program at Stanford University. RS was supported by the Rubicon fellowship of the Dutch National Research Council (NWO). This work was supported in part by NIH grants R01 HL155410, R01 HL157235, by AHRQ grant R18HS026886, and by the Gordon and Betty Moore Foundation. CL is supported by the Medical Imaging and Data Resource Center (MIDRC), which is funded by the National Institute of Biomedical Imaging and Bioengineering (NIBIB) under contract 75N92020C00021 and through The Advanced Research Projects Agency for Health (ARPA-H). AC is supported by NIH grants R01 HL167974, R01HL169345, R01 AR077604, R01 EB002524, R01 AR079431, P41 EB027060, AY2AX000045, and 1AYSAX0000024-01; and NIH contracts 75N92020C00008 and 75N92020C00021. 

\subsection*{Author contributions}
M.V., A.K., and R.S. designed the study, constructed models, and performed technical evaluations. R.S., C.B., and J.P. carried out the reader study. M.V., A.K., R.S., S.O., L.B., and P.C. collected data and analyzed model performance. All authors contributed to technical discussions and the drafting and revision of the manuscript. C.L. and A.C. supervised, funded, and guided the research.

\clearpage
\nolinenumbers
% \setlength\bibitemsep{3pt}
\bibliographystyle{plain} 
\bibliography{main}      
% \end{refsegment}
\clearpage
\renewcommand{\thefigure}{A\arabic{figure}}
\renewcommand{\thetable}{A\arabic{table}}
\renewcommand{\theequation}{A\arabic{equation}}
\setcounter{figure}{0}
\setcounter{table}{0}
\setcounter{equation}{0}

Our Appendix is organized as follows. First, we present the pseudocode for the key components of iGCT. We also include the proof for unit variance and boundary conditions in preconditioning iGCT's noiser. Next, we detail the training setups for our CIFAR-10 and ImageNet64 experiments. Additionally, we provide ablation studies on using guided synthesized images as data augmentation in image classification. Finally, we present more uncurated results comparing iGCT and CFG-EDM on inversion, editing and guidance, thoroughly of iGCT.

\vspace{-0.2cm}
\label{appendix:iGCT}
\section{Pseudocode for iGCT}
\vspace{-0.2cm}

iGCT is trained under a continuous-time scheduler similar to the one proposed by ECT \cite{ect}. Our noise sampling function follows a lognormal distribution, \(p(t) = \textit{LogNormal}(P_\textit{mean}, P_\textit{std})\), with \(P_\textit{mean}=-1.1\) and \( P_\textit{std}=2.0\). At training, the sampled noise is clamped at \(t_\text{min} = 0.002\) and \(t_\text{max} = 80.0\). Step function \(\Delta t (t)=\frac{t}{2^{\left\lfloor k/d \right\rfloor}}n(t)\), is used to compute the step size from the sampled noise \(t\), with \(k,d\) being the current training iteration and the number of iterations for halfing \(\Delta t\), and \(n(t) = 1 + 8 \sigma(-t)\) is a sigmoid adjusting function. 

In Guided Consistency Training, the guidance mask function determines whether the sampled noise \( t \) should be supervised for guidance training. With probability \( q(t) \in [0,1] \), the update is directed towards the target sample \( \boldsymbol{x}_0^{\text{tar}} \); otherwise, no guidance is applied. In practice, \( q(t) \) is higher in noisier regions and zero in low-noise regions, 
\begin{equation}
    q(t) = 0.9 \cdot \left( \text{clamp} \left( \frac{t - t_{\text{low}}}{t_{\text{high}} - t_{\text{low}}}, 0, 1 \right) \right)^2,
\end{equation}
where \( t_{\text{low}} = 11.0 \) and \( t_{\text{high}} = 14.3 \). For the range of guidance strength, we set \(w_\text{min} = 1\) and \(w_\text{max} = 15\). Guidance strengths are sampled uniformly at training, with \(w_\text{min} = 1\) means no guidance applied. 


\begin{algorithm}
\caption{Guided Consistency Training}
\label{alg:GCT}
\begin{algorithmic}[1]  % Adds line numbers
\setlength{\baselineskip}{0.9\baselineskip} % Adjust line spacing
\INPUT Dataset $\mathcal{D}$, weighting function $\lambda(t)$, noise sampling function $p(t)$, noise range $[t_\text{min}, t_\text{max}]$, step function $\Delta t(t)$, guidance mask function $q(t)$, guidance range $[w_\text{min}, w_\text{max}]$, denoiser $D_\theta$
\STATE \rule{0.96\textwidth}{0.45pt} 
\STATE Sample $(\boldsymbol{x}_0^{\text{src}}, c^{\text{src}}), (\boldsymbol{x}_0^{\text{tar}}, c^{\text{tar}}) \sim \mathcal{D}$ 
\STATE Sample noise $\boldsymbol{z} \sim \mathcal{N}(\boldsymbol{0},\mathbf{I})$, time step $t \sim p(t)$, and guidance weight $w \sim \mathcal{U}(w_\text{min}, w_\text{max})$
\STATE Clamp $t \leftarrow \text{clamp}(t,t_\text{min}, t_\text{max})$
\STATE Compute noisy sample: $\boldsymbol{x}_t = \boldsymbol{x}_0^{\text{src}} + t\boldsymbol{z}$
\STATE Sample $\rho \sim \mathcal{U}(0,1)$  
\vspace{0.3em}
\IF{$\rho > q(t)$}
    \STATE Compute step as normal CT: $\boldsymbol{x}_r = \boldsymbol{x}_t - \Delta t(t) \boldsymbol{z}$
    \STATE Set target class: $c \leftarrow c^{\text{src}}$
\ELSE
    \STATE Compute guided noise: $\boldsymbol{z}^* = (\boldsymbol{x}_t - \boldsymbol{x}_0^{\text{tar}}) / t$
    \STATE Compute guided step: $\boldsymbol{x}_r = \boldsymbol{x}_t - \Delta t(t) [w \boldsymbol{z}^* + (1-w)\boldsymbol{z}]$
    \STATE Set target class: $c \leftarrow c^{\text{tar}}$
\ENDIF
\vspace{0.3em} % Reduces extra vertical space before the loss line
\STATE Compute loss: 
\[
\mathcal{L}_\text{gct} = \lambda(t) \, d(D_{\theta}(\boldsymbol{x}_t, t, c, w), D_{{\theta}^-}(\boldsymbol{x}_r, r, c, w))
\]
\STATE Return $\mathcal{L}_\text{gct}$ 
\end{algorithmic}
\end{algorithm}



A \textit{noiser} trained under \textit{Inverse Consistency Training} maps an image to its latent noise in a single step. In contrast, DDIM Inversion requires multiple steps with a diffusion model to accurately produce an image's latent representation. Since the boundary signal is reversed, spreading from \( t_\text{max} \) down to \( t_\text{min} \), we design the importance weighting function \( \lambda'(t) \) to emphasize higher noise regions, defined as:
\begin{equation}
    \lambda'(t) = \frac{\Delta t (t)}{t_\text{max}},
\end{equation}
where the step size \( \Delta t (t) \) is proportional to the sampled noise level \(t\), and \( t_\text{max} \) is a constant that normalizes the scale of the inversion loss. The noise sampling function \( p(t) \) and the step function \( \Delta t (t) \) used in computing both \(\mathcal{L}_\text{gct}\) and \(\mathcal{L}_\text{ict}\) are the same.



\begin{algorithm}
\caption{Inverse Consistency Training}
\label{alg:iCT}
\begin{algorithmic}[1]  % Adds line numbers
\setlength{\baselineskip}{0.9\baselineskip} % Adjust line spacing
\INPUT Dataset $\mathcal{D}$, weighting function $\lambda'(t)$, noise sampling function $p(t)$, noise range $[t_\text{min}, t_\text{max}]$, step function $\Delta t(t)$, noiser $N_\varphi$
\STATE \rule{0.96\textwidth}{0.45pt} 
\STATE Sample $\boldsymbol{x}_0, c \sim \mathcal{D}$ 
\STATE Sample noise $\boldsymbol{z} \sim \mathcal{N}(\boldsymbol{0},\mathbf{I})$, time step $t \sim p(t)$
\STATE Clamp $t \leftarrow \text{clamp}(t,t_\text{min}, t_\text{max})$
\STATE Compute noisy sample: $\boldsymbol{x}_t = \boldsymbol{x}_0 + t\boldsymbol{z}$
\STATE Compute cleaner sample: $\boldsymbol{x}_r = \boldsymbol{x}_t - \Delta t(t) \boldsymbol{z}$
\vspace{0.3em} 
\STATE Compute loss: 
\[
\mathcal{L}_\text{ict} = \lambda'(t) \, d(N_{\varphi}(\boldsymbol{x}_r, r, c), D_{{\varphi}^-}(\boldsymbol{x}_t, t, c))
\]
\STATE Return $\mathcal{L}_\text{ict}$ 
\end{algorithmic}
\end{algorithm}

Together, iGCT jointly optimizes the two consistency objectives \(\mathcal{L}_\text{gct}, \mathcal{L}_\text{ict}\), and aligns the noiser and denoiser via a reconstruction loss, \(\mathcal{L}_\text{recon}\). To improve training efficiency, \(\mathcal{L}_\text{recon}\) is computed every \(i_\text{skip}\), reducing the computational cost of back-propagation through both the weights of the \textit{denoiser} \(\theta\) and the \textit{noiser} \(\varphi\). Alg. \ref{alg:iGCT} provides an overview of iGCT. 

\begin{algorithm}
\caption{iGCT}
\label{alg:iGCT}
\begin{algorithmic}[1]  % Adds line numbers
\setlength{\baselineskip}{0.9\baselineskip} % Adjust line spacing
\INPUT Dataset $\mathcal{D}$, learning rate $\eta$, weighting functions $\lambda'(t), \lambda(t), \lambda_{\text{recon}}$, noise sampling function $p(t)$, noise range $[t_\text{min}, t_\text{max}]$, step function $\Delta t(t)$, guidance mask function $q(t)$, guidance range $[w_\text{min}, w_\text{max}]$, reconstruction skip iters $i_\text{skip}$, models $N_\varphi, D_\theta$
\STATE \rule{0.9\textwidth}{0.45pt}  % Horizontal line to separate input from main algorithm
\STATE \textbf{Init:} $\theta, \varphi$, $\text{Iters} = 0$
\REPEAT
\STATE Do guided consistency training 
\[
\mathcal{L}_\text{gct}(\theta;\mathcal{D},\lambda(t),p(t),t_\text{min},t_\text{max},\Delta t(t),q(t),w_\text{min},w_\text{max})
\]
\STATE Do inverse consistency training
\[
\mathcal{L}_\text{ict}(\varphi;\mathcal{D},\lambda'(t),p(t),t_\text{min},t_\text{max},\Delta t(t))
\]
\IF{$(\text{Iters} \ \% \ i_\text{skip}) == 0$}
\STATE Compute reconstruction loss
\[
\mathcal{L}_\text{recon} = d(D_{\theta}(N_{\varphi}(\boldsymbol{x}_0,t_\text{min},c),t_\text{max},c,0), \boldsymbol{x}_0)
\]
\ELSE
\STATE \[
\mathcal{L}_\text{recon} = 0
\]
\ENDIF
\STATE Compute total loss: 
\[
\mathcal{L} = \mathcal{L}_\text{gct} + \mathcal{L}_\text{ict} + \lambda_{\text{recon}}\mathcal{L}_\text{recon}
\]
\STATE $\theta \leftarrow \theta - \eta \nabla_{\theta} \mathcal{L}, \ \varphi \leftarrow \varphi - \eta \nabla_{\varphi} \mathcal{L}$
\STATE $\text{Iters} = \text{Iters} + 1$
\UNTIL{$\Delta t \rightarrow dt$}
\end{algorithmic}
\end{algorithm}



\vspace{-0.3cm}
\section{Preconditioning for Noiser}
\label{appendix:unit-variance}
\vspace{-0.1cm}

We define 
\begin{equation}
    N_{\varphi}(\boldsymbol{x}_t, t, c) = c_\text{skip}(t) \, \boldsymbol{x}_t + c_\text{out}(t) \, F_{\varphi}(c_\text{in}(t) \, \boldsymbol{x}_t, t, c),
\end{equation}
where \( c_\text{in}(t) = \frac{1}{\sqrt{t^2 + \sigma_\text{data}^2}} \), \( c_\text{skip}(t) = 1 \), and \( c_\text{out}(t) = t_\text{max} - t \). This setup naturally serves as a boundary condition. Specifically:

\begin{itemize}
    \item When \( t = 0 \),
    \begin{equation}
        c_\text{out}(0) = t_\text{max} \gg c_\text{skip}(0) = 1,
    \end{equation}
    emphasizing that the model's noise prediction dominates the residual information given a relatively clean sample.

    \item When \( t = t_\text{max} \),
    \begin{equation}
        N_{\varphi}(\boldsymbol{x}_{t_\text{max}}, t_\text{max}, c) = \boldsymbol{x}_{t_\text{max}},
    \end{equation}
    satisfying the condition that \( N_{\varphi} \) outputs \( \boldsymbol{x}_{t_\text{max}} \) at the maximum time step.
\end{itemize}



We show that these preconditions ensure unit variance for the model’s input and target. First, \(\text{Var}_{\boldsymbol{x}_0, z}[\boldsymbol{x}_t] = \sigma_\text{data}^2 + t^2\), so setting \( c_\text{in}(t) = \frac{1}{\sqrt{\sigma_\text{data}^2 + t^2}} \) normalizes the input variance to 1. Second, we require the training target to have unit variance. Given the noise target for \( N_{\varphi} \) is \(\boldsymbol{x}_{t_\text{max}} = \boldsymbol{x}_0 + t_\text{max} z\), by moving of terms, the effective target for \( F_{\varphi} \) can be written as,
\begin{equation}
    \frac{\boldsymbol{x}_{t_\text{max}} - c_\text{skip}(t)\boldsymbol{x}_{t}}{c_\text{out}(t)}
\end{equation}
When \(c_\text{skip}(t) = 1\), \(c_\text{out}(t) = t_\text{max} - t \), we verify that target is unit variance,
\begin{align}
    &\text{Var}_{\boldsymbol{x}_0, \boldsymbol{z}} \left[ \frac{\boldsymbol{x}_{t_\text{max}} - c_\text{skip}(t) \, \boldsymbol{x}_{t}}{c_\text{out}(t)} \right] \\ \notag
    = \ &\text{Var}_{\boldsymbol{x}_0, \boldsymbol{z}} \left[ \frac{\boldsymbol{x}_0 + t_\text{max} \, \boldsymbol{z} - (\boldsymbol{x}_0 + t \, \boldsymbol{z})}{t_\text{max} - t} \right] \notag \\
    = \ &\text{Var}_{\boldsymbol{x}_0, \boldsymbol{z}} \left[ \frac{(t_\text{max} - t) \, \boldsymbol{z}}{t_\text{max} - t} \right] \notag \\
    = \ &\text{Var}_{\boldsymbol{x}_0, \boldsymbol{z}}[\boldsymbol{z}] \notag \\
    = \ &1. \notag
\end{align}

\vspace{-0.3cm}
\section{Baselines \& Training Details}
\label{appendix:bs-config}
\vspace{-0.1cm}

\begin{figure}[t!]  
    \centering
    \begin{subfigure}[b]{0.33\textwidth}
    \includegraphics[width=\textwidth]{fig/appendix/guidance_embed.pdf} 
        \caption{Guidance embedding.}
    \end{subfigure}
    \hfill
    \begin{subfigure}[b]{0.33\textwidth}
    \includegraphics[width=\textwidth]{fig/appendix/adm_arch.pdf} 
        \caption{NCSN++ architecture.}
    \end{subfigure}
    \hfill
    \begin{subfigure}[b]{0.33\textwidth}
    \includegraphics[width=\textwidth]{fig/appendix/ncsnpp_arch.pdf} 
        \caption{ADM architecture.}
    \end{subfigure}
    \hfill
    \caption{Design of guidance embedding, and conditioning under different network architectures.}
    \vspace{-1em}
    \label{fig:guidance_conditioning}
\end{figure}

For our diffusion model baseline, we follow \textit{EDM}'s official repository (\href{https://github.com/NVlabs/edm}{https://github.com/NVlabs/edm}) instructions for training and set \textit{label\_dropout} to 0.1 to optimize a CFG (classifier-free guided) DM. We will use this DM as the teacher model for our consistency model baseline via consistency distillation. 

The consistency model baseline \textit{Guided CD} is trained with a discrete-time schedule. We set the discretization steps \( N = 18 \) and use a Heun ODE solver to predict update directions based on the CFG EDM, as in \cite{song2023consistency}. Following \cite{luo2023latent}, we modify the model's architecture and iGCT's denoiser to accept guidance strength \(w\) by adding an extra linear layer. See the detailed architecture design for guidance conditioning of consistency model in Fig. \ref{fig:guidance_conditioning}. A range of guidance scales \(w \in [1,15]\) is uniformly sampled at training. Following \cite{song2023improved}, we replace LPIPS by Pseudo-Huber loss, with \(c=0.03 \) determining the breadth of the smoothing section between L1 and L2. See Table \ref{tab:training_configs} for a summary of the training configurations for our baseline models.


\begin{table}[t!]
\centering
\renewcommand{\arraystretch}{1.3} % Adjust vertical spacing
\small % Reduce text size
\caption{Summary of training configurations for baseline models.}
\begin{tabular}{lccc}
\toprule
\multirow{2}{*}{} & \multicolumn{2}{c}{\textbf{CIFAR-10}} & \textbf{ImageNet64}  \\
                  & EDM & Guided-CD & EDM \\
\midrule
\multicolumn{4}{l}{\textbf{\small Arch. config.}} \\
\hline
model arch.        & NCSN++ & NCSN++ & ADM     \\
channels mult.     & 2,2,2  & 2,2,2  & 1,2,3,4 \\
UNet size          & 56.4M  & 56.4M  & 295.9M  \\
\midrule
\multicolumn{4}{l}{\textbf{\small Training config.}} \\
\hline
lr             & 1e-3  & 4e-4  & 2e-4 \\
batch          & 512   & 512   & 4096 \\
dropout        & 0.13  & 0     & 0.1 \\
label dropout  & 0.1   & (n.a.) & 0.1 \\
loss           & L2    & Huber & L2    \\
training iterations & 390k  & 800k  & 800K \\
\bottomrule
\end{tabular}
\label{tab:training_configs}
\end{table}


\begin{table}[t!]
\centering
\renewcommand{\arraystretch}{1.3} % Adjust vertical spacing
\small % Reduce text size
\caption{Summary of training configurations for iGCT.}
\begin{tabular}{lcc}
\toprule
\multirow{2}{*}{} & \textbf{CIFAR-10} & \textbf{ImageNet64}  \\
                  & iGCT & iGCT \\
\midrule
\multicolumn{3}{l}{\textbf{\small Arch. config.}} \\
\hline
model arch.        & NCSN++ & ADM \\
channels mult.     & 2,2,2  & 1,2,2,3 \\
UNet size          & 56.4M  & 182.4M \\ 
Total size         & 112.9M & 364.8M \\ 
\midrule
\multicolumn{3}{l}{\textbf{\small Training config.}} \\
\hline
lr              & 1e-4 & 1e-4 \\
batch           & 1024 & 1024 \\
dropout            & 0.2 & 0.3 \\
loss               & Huber   & Huber \\
\(c\)                  & 0.03    &  0.06 \\
\(d\)                  & 40k     &  40k \\
\( P_\textit{mean} \) & -1.1 &  -1.1 \\
\( P_\textit{std} \) &  2.0  &  2.0  \\
\( \lambda_{\text{recon}} \) & 2e-5 & \parbox[t]{3.5cm}{\centering 2e-5, (\(\leq\) 180k)\\ 4e-5, (\(\leq\) 200k)\\ 6e-5, (\(\leq\) 260k) } \\  
\( i_{\text{skip}} \)        & 10 &  10 \\  
training iterations & 360k &  260k \\
\bottomrule
\end{tabular}
\label{tab:igct_training_configs}
\end{table}  

\begin{figure*}[t] 
    \centering
    \includegraphics[width=1.0\textwidth]{fig/appendix/inversion_collapse.pdf} 
    \caption{Inversion collapse observed during training on ImageNet64. The left image shows the input data. The middle image depicts the inversion collapse that occurred at iteration 220k, where leakage of signals in the noise latent can be visualized. The right image shows the inversion results at iteration 220k after appropriately increasing $\lambda_{\text{recon}}$ to 6e-5. The inversion images are generated by scaling the model's outputs by $1/80$, i.e., $ 1/t_\text{max}$, then clipping the values to the range [-3, 3] before denormalizing them to the range [0, 255]. }
    \vspace{-1.5em}
    \label{fig:inversion_collpase}
\end{figure*}

iGCT is trained with a continuous-time scheduler inspired by ECT \cite{ect}. To rigorously assess its independence from diffusion-based models, iGCT is trained from scratch rather than fine-tuned from a pre-trained diffusion model. Consequently, the training curriculum begins with an initial diffusion training stage, followed by consistency training with the step size halved every \(d\) iterations. In practice, we adopt the same noise sampling distribution \(p(t)\), same step function \(\Delta t (t) \), and same distance metric \( d(\cdot, \cdot) \) for both guided consistency training and inverse consistency training. 

For CIFAR-10, iGCT adopts the same UNet architecture as the baseline models. However, the overall model size is doubled, as iGCT comprises two UNets: one for the denoiser and one for the noiser. The Pseudo-Huber loss is employed as the distance metric, with a constant parameter \( c = 0.03 \). Consistency training is organized into nine stages, each comprising 400k iterations with the step size halved from the last stage. We found that training remains stable when the reconstruction weight \( \lambda_{\text{recon}} \) is fixed at \( 2 \times 10^{-5} \) throughout the entire training process.
 
For ImageNet64, iGCT employs a reduced ADM architecture \cite{dhariwal2021diffusionmodelsbeatgans} with smaller channel sizes to address computational constraints. A higher dropout rate and Pseudo-Huber loss with \( c = 0.06 \) is used, following prior works \cite{ect,song2023improved}. During our experiments, we observed that training on ImageNet64 is sensitive to the reconstruction weight. Keeping \(\lambda_{\text{recon}}\) fixed throughout training leads to inversion collapse, with significant signal leaked to the latent noise (see Fig. \ref{fig:inversion_collpase}). We found that increasing \(\lambda_{\text{recon}}\) to \( 4 \times 10^{-5} \) at iteration 1800 and to \( 6 \times 10^{-5} \) at iteration 2000 effectively stabilizes training and prevents collapse. This suggests that the reconstruction loss serves as a regularizer for iGCT. Additionally, we observed diminishing returns when training exceeded 240k iterations, leading us to stop at 260k iterations for our experiments. These findings indicate that alternative training strategies, such as framing iGCT as a multi-task learning problem \cite{kendall2018multi,liu2019loss}, and conducting a more sophisticated analysis of loss weighting, may be necessary to enhance stability and improve convergence. See Table \ref{tab:igct_training_configs} for a summary of the training configurations for iGCT.



\begin{table}[t]
\caption{Comparison of GPU hours across the methods used in our experiments on CIFAR-10.}
\centering
\begin{tabular}{|l|c|}
\hline
\textbf{Methods} & \textbf{A100 (40G) GPU hours} \\ \hline
CFG-EDM \cite{karras2022elucidating} & 312 \\ \hline
Guided-CD \cite{song2023consistency} & 3968 \\ \hline
iGCT (ours) & 2032 \\ \hline
\end{tabular}
\label{table:compute_resources}
\end{table}



\begin{figure*}[t!]  
    \centering
    \begin{subfigure}[b]{0.33\textwidth}
    \includegraphics[width=\textwidth]{fig/cls_exp_w1.png} 
        \caption{Accuracy on various ratios of augmented data, guidance scale w=1.}
    \end{subfigure}
    \begin{subfigure}[b]{0.33\textwidth}
    \includegraphics[width=\textwidth]{fig/cls_exp_w3.png} 
        \caption{Accuracy on various ratios of augmented data, guidance scale w=3.}
    \end{subfigure}
    \begin{subfigure}[b]{0.33\textwidth}
    \includegraphics[width=\textwidth]{fig/cls_exp_w5.png} 
        \caption{Accuracy on various ratios of augmented data, guidance scale w=5.}
    \end{subfigure}
    \begin{subfigure}[b]{0.33\textwidth}
    \includegraphics[width=\textwidth]{fig/cls_exp_w7.png} 
        \caption{Accuracy on various ratios of augmented data, guidance scale w=7.}
    \end{subfigure}
    \begin{subfigure}[b]{0.33\textwidth}
    \includegraphics[width=\textwidth]{fig/cls_exp_w9.png} 
        \caption{Accuracy on various ratios of augmented data, guidance scale w=9.}
    \end{subfigure}
    \caption{Comparison of synthesized methods, CFG-EDM vs iGCT, used for data augmentation in image classification. iGCT consistently improves accuracy. Conversely, augmentation data synthesized from CFG-EDM offers only limited gains.}
    \vspace{-1.5em}
    \label{fig:cls_results}
\end{figure*}


\vspace{-0.1cm}
\section{Application: Data Augmentation Under Different Guidance}
\vspace{-0.2cm}

In this section, we show the effectiveness of data augmentation with diffusion-based models, CFG-EDM and iGCT, across varying guidance scales for image classification on CIFAR-10 \cite{article}. High quality data augmentation has been shown to enhance classification performance \cite{yang2023imagedataaugmentationdeep}. Under high guidance, augmentation data generated from iGCT consistently improves accuracy. Conversely, augmentation data synthesized from CFG-EDM offers only limited gains. We describe the ratios of real to synthesized data, the classifier architecture, and the training setup in the following. 

\noindent{\bf Training Details.} We conduct classification experiments trained on six different mixtures of augmented data synthesized by iGCT and CFG-EDM: \(0\%\), \(20\%\), \(40\%\), \(80\%\), and \(100\%\). The ratio represents \(\textit{synthesized data} / \textit{real data}\). For example, \(0\%\) indicates that the training and validation sets contain only 50k of real samples from CIFAR-10, and \(20\%\) includes 50k real \textit{and} 10k synthesized samples. In terms of guidance scales, we choose \(w=1,3,5,7,9\) to synthesize the augmented data using iGCT and CFG-EDM. 
The augmented dataset is split 80/20 for training and validation. For testing, the model is evaluated on the CIFAR-10 test set with 10k samples and ground truth labels. 

The standard ResNet-18 \cite{he2015deepresiduallearningimage} is used to train on all different augmented datasets. All models are trained for 250 epochs, with batch size 64, using an Adam optimizer \cite{kingma2017adammethodstochasticoptimization}. For each augmentation dataset, we train the model six times under different seeds and report the average classification accuracy.

\noindent{\bf Results.} The classifier's accuracy, trained on augmented data synthesized by CFG-EDM and iGCT, is shown in Fig. \ref{fig:cls_results}. With \(w=1\) (no guidance), both iGCT and CFG-EDM provide comparable performance boosts. As guidance scale increases, iGCT shows more significant improvements than CFG-EDM. At high guidance and augmentation ratios, performance drops, but this effect occurs later for iGCT (e.g., at \(100\%\) augmentation and \(w=9\)), while CFG-EDM stops improving accuracy at \(w=7\). This experiment highlights the importance of high-quality data under high guidance, with iGCT outperforming CFG-EDM in data quality.

\section{Uncurated Results}
In this section, we present additional qualitative results to highlight the performance of our proposed iGCT method compared to the multi-step EDM baseline. These visualizations include both inversion and guidance tasks across the CIFAR-10 and ImageNet64 datasets. The results demonstrate iGCT's ability to maintain competitive quality with significantly fewer steps and minimal artifacts, showcasing the effectiveness of our approach.

\subsection{Inversion Results}
We provide additional visualization of the latent noise on both CIFAR-10 and ImageNet64 datasets. Fig. \ref{fig:CIFAR-10_inversion_reconstruction} and Fig. \ref{fig:im64_inversion_reconstruction} compare our 1-step iGCT with the multi-step EDM on inversion and reconstruction.  

\subsection{Editing Results}
In this section, we dump more uncurated editing results on ImageNet64's subgroups mentioned in Sec. \ref{sec:image-editing}. Fig. \ref{fig:im64_edit_1}--\ref{fig:im64_edit_4} illustrate a comparison between our 1-step iGCT and the multi-step EDM approach.

\subsection{Guidance Results}
In Section \ref{sec:guidance}, we demonstrated that iGCT provides a guidance solution without introducing the high-contrast artifacts commonly observed in CFG-based methods. Here, we present additional uncurated results on CIFAR-10 and ImageNet64. For CIFAR-10, iGCT achieves competitive performance compared to the baseline diffusion model, which requires multiple steps for generation. See Figs. \ref{fig:CIFAR-10_guided_1}--\ref{fig:CIFAR-10_guided_10}. For ImageNet64, although the visual quality of iGCT's generated images falls slightly short of expectations, this can be attributed to the smaller UNet architecture used—only 61\% of the baseline model size—and the need for a more robust training curriculum to prevent collapse, as discussed in Section \ref{appendix:bs-config}. Nonetheless, even at higher guidance levels, iGCT maintains style consistency, whereas CFG-based methods continue to suffer from pronounced high-contrast artifacts. See Figs. \ref{fig:im64_guided_1}--\ref{fig:im64_guided_4}.


\begin{figure*}[t]
    \centering
    \begin{subfigure}{0.48\textwidth}
        \centering
        \includegraphics[width=\linewidth]{fig/appendix/recon_c10_data.png}
        \caption{CIFAR-10: Original data}
    \end{subfigure}
    \begin{subfigure}{0.48\textwidth}
        \centering
        \includegraphics[width=\linewidth]{fig/appendix/recon_im64_data.png}
        \caption{ImageNet64: Original data}
    \end{subfigure}

    \begin{subfigure}{0.48\textwidth}
        \centering
        \includegraphics[width=\linewidth]{fig/appendix/inv_c10_edm.png}
    \end{subfigure}
    \begin{subfigure}{0.48\textwidth}
        \centering
        \includegraphics[width=\linewidth]{fig/appendix/inv_im64_edm.png}
    \end{subfigure}

    \begin{subfigure}{0.48\textwidth}
        \centering
        \includegraphics[width=\linewidth]{fig/appendix/recon_c10_edm.png}
        \caption{CIFAR-10: Inversion + reconstruction, EDM (18 NFE)}
    \end{subfigure}
    \begin{subfigure}{0.48\textwidth}
        \centering
        \includegraphics[width=\linewidth]{fig/appendix/recon_im64_edm.png}
        \caption{ImageNet64: Inversion + reconstruction, EDM (18 NFE)}
    \end{subfigure}

    \begin{subfigure}{0.48\textwidth}
        \centering
        \includegraphics[width=\linewidth]{fig/appendix/inv_c10_igct.png}
    \end{subfigure}
    \begin{subfigure}{0.48\textwidth}
        \centering
        \includegraphics[width=\linewidth]{fig/appendix/inv_im64_igct.png}
    \end{subfigure}

    \begin{subfigure}{0.48\textwidth}
        \centering
        \includegraphics[width=\linewidth]{fig/appendix/recon_c10_igct.png}
        \caption{CIFAR-10: Inversion + reconstruction, iGCT (1 NFE)}
    \end{subfigure}
    \begin{subfigure}{0.48\textwidth}
        \centering
        \includegraphics[width=\linewidth]{fig/appendix/recon_im64_igct.png}
        \caption{ImageNet64: Inversion + reconstruction, iGCT (1 NFE)}
    \end{subfigure}

    \caption{Comparison of inversion and reconstruction for CIFAR-10 (left) and ImageNet64 (right).}
    \label{fig:comparison_CIFAR-10_imagenet64}
\end{figure*}




\begin{figure*}[t]
    \centering

    % Left column: corgi -> golden retriever
    \begin{minipage}{0.48\textwidth}
        \centering
        \begin{subfigure}{0.48\textwidth}
            \includegraphics[width=\linewidth]{fig/appendix_edit_igct/src_corgi.png}
            \caption{Original: "corgi"}
        \end{subfigure}

        \begin{subfigure}{0.48\textwidth}
            \includegraphics[width=\linewidth]{fig/appendix_edit_edm/w=0_src_corgi_tar_golden_retriever.png}
            \caption{EDM (18 NFE), w=1}
        \end{subfigure}
        \begin{subfigure}{0.48\textwidth}
            \includegraphics[width=\linewidth]{fig/appendix_edit_edm/w=6_src_corgi_tar_golden_retriever.png}
            \caption{EDM (18 NFE), w=7}
        \end{subfigure}
        \begin{subfigure}{0.48\textwidth}
            \includegraphics[width=\linewidth]{fig/appendix_edit_igct/w=6_src_corgi_tar_golden_retriever.png}
            \caption{iGCT (1 NFE), w=7}
        \end{subfigure}
        \begin{subfigure}{0.48\textwidth}
            \includegraphics[width=\linewidth]{fig/appendix_edit_igct/w=0_src_corgi_tar_golden_retriever.png}
            \caption{iGCT (1 NFE), w=1}
        \end{subfigure}

        \caption{ImageNet64: "corgi" $\rightarrow$ "golden retriever"}
        \label{fig:im64_edit_1}
    \end{minipage}
    \hfill
    % Right column: zebra -> horse
    \begin{minipage}{0.48\textwidth}
        \centering
        \begin{subfigure}{0.48\textwidth}
            \includegraphics[width=\linewidth]{fig/appendix_edit_igct/src_zebra.png}
            \caption{Original: "zebra"}
        \end{subfigure}

        \begin{subfigure}{0.48\textwidth}
            \includegraphics[width=\linewidth]{fig/appendix_edit_edm/w=0_src_zebra_tar_horse.png}
            \caption{EDM (18 NFE), w=1}
        \end{subfigure}
        \begin{subfigure}{0.48\textwidth}
            \includegraphics[width=\linewidth]{fig/appendix_edit_edm/w=6_src_zebra_tar_horse.png}
            \caption{EDM (18 NFE), w=7}
        \end{subfigure}
        \begin{subfigure}{0.48\textwidth}
            \includegraphics[width=\linewidth]{fig/appendix_edit_igct/w=0_src_zebra_tar_horse.png}
            \caption{iGCT (1 NFE), w=1}
        \end{subfigure}
        \begin{subfigure}{0.48\textwidth}
            \includegraphics[width=\linewidth]{fig/appendix_edit_igct/w=6_src_zebra_tar_horse.png}
            \caption{iGCT (1 NFE), w=7}
        \end{subfigure}

        \caption{ImageNet64: "zebra" $\rightarrow$ "horse"}
        \label{fig:im64_edit_2}
    \end{minipage}

\end{figure*}

\begin{figure*}[t]
    \centering

    % Left column: broccoli -> cauliflower
    \begin{minipage}{0.48\textwidth}
        \centering
        \begin{subfigure}{0.48\textwidth}
            \includegraphics[width=\linewidth]{fig/appendix_edit_igct/src_broccoli.png}
            \caption{Original: "broccoli"}
        \end{subfigure}

        \begin{subfigure}{0.48\textwidth}
            \includegraphics[width=\linewidth]{fig/appendix_edit_edm/w=0_src_broccoli_tar_cauliflower.png}
            \caption{EDM (18 NFE), w=1}
        \end{subfigure}
        \begin{subfigure}{0.48\textwidth}
            \includegraphics[width=\linewidth]{fig/appendix_edit_edm/w=6_src_broccoli_tar_cauliflower.png}
            \caption{EDM (18 NFE), w=7}
        \end{subfigure}
        \begin{subfigure}{0.48\textwidth}
            \includegraphics[width=\linewidth]{fig/appendix_edit_igct/w=0_src_broccoli_tar_cauliflower.png}
            \caption{iGCT (1 NFE), w=1}
        \end{subfigure}
        \begin{subfigure}{0.48\textwidth}
            \includegraphics[width=\linewidth]{fig/appendix_edit_igct/w=6_src_broccoli_tar_cauliflower.png}
            \caption{iGCT (1 NFE), w=7}
        \end{subfigure}

        \caption{ImageNet64: "broccoli" $\rightarrow$ "cauliflower"}
        \label{fig:im64_edit_3}
    \end{minipage}
    \hfill
    % Right column: jaguar -> tiger
    \begin{minipage}{0.48\textwidth}
        \centering
        \begin{subfigure}{0.48\textwidth}
            \includegraphics[width=\linewidth]{fig/appendix_edit_igct/src_jaguar.png}
            \caption{Original: "jaguar"}
        \end{subfigure}

        \begin{subfigure}{0.48\textwidth}
            \includegraphics[width=\linewidth]{fig/appendix_edit_edm/w=0_src_jaguar_tar_tiger.png}
            \caption{EDM (18 NFE), w=1}
        \end{subfigure}
        \begin{subfigure}{0.48\textwidth}
            \includegraphics[width=\linewidth]{fig/appendix_edit_edm/w=6_src_jaguar_tar_tiger.png}
            \caption{EDM (18 NFE), w=7}
        \end{subfigure}
        \begin{subfigure}{0.48\textwidth}
            \includegraphics[width=\linewidth]{fig/appendix_edit_igct/w=0_src_jaguar_tar_tiger.png}
            \caption{iGCT (1 NFE), w=1}
        \end{subfigure}
        \begin{subfigure}{0.48\textwidth}
            \includegraphics[width=\linewidth]{fig/appendix_edit_igct/w=6_src_jaguar_tar_tiger.png}
            \caption{iGCT (1 NFE), w=7}
        \end{subfigure}

        \caption{ImageNet64: "jaguar" $\rightarrow$ "tiger"}
        \label{fig:im64_edit_4}
    \end{minipage}

\end{figure*}






\begin{figure*}[b]
    \centering
    % First image
    \begin{subfigure}{0.25\textwidth}
        \includegraphics[width=\linewidth]{fig/appendix_edm/0_0.0_middle_4x4_grid.png}
        \caption{CFG-EDM (18 NFE), w=1.0}
    \end{subfigure}
    \begin{subfigure}{0.25\textwidth}
        \includegraphics[width=\linewidth]{fig/appendix_edm/0_6.0_middle_4x4_grid.png}
        \caption{CFG-EDM (18 NFE), w=7.0}
    \end{subfigure}
    \begin{subfigure}{0.25\textwidth}
        \includegraphics[width=\linewidth]{fig/appendix_edm/0_12.0_middle_4x4_grid.png}
        \caption{CFG-EDM (18 NFE), w=13.0}
    \end{subfigure}
    \begin{subfigure}{0.25\textwidth}
        \includegraphics[width=\linewidth]{fig/appendix_igct/0_0.0_middle_4x4_grid.png}
        \caption{iGCT (1 NFE), w=1.0}
    \end{subfigure}
    \begin{subfigure}{0.25\textwidth}
        \includegraphics[width=\linewidth]{fig/appendix_igct/0_6.0_middle_4x4_grid.png}
        \caption{iGCT (1 NFE), w=7.0}
    \end{subfigure}
    % Third image
    \begin{subfigure}{0.25\textwidth}
        \includegraphics[width=\linewidth]{fig/appendix_igct/0_12.0_middle_4x4_grid.png}
        \caption{iGCT (1 NFE), w=13.0}
    \end{subfigure}
    \caption{CIFAR-10 "airplane"}
    \label{fig:CIFAR-10_guided_1}
\end{figure*}
\begin{figure*}[t]
    \centering
    % First image
    \begin{subfigure}{0.25\textwidth}
        \includegraphics[width=\linewidth]{fig/appendix_edm/1_0.0_middle_4x4_grid.png}
        \caption{CFG-EDM (18 NFE), w=1.0}
    \end{subfigure}
    \begin{subfigure}{0.25\textwidth}
        \includegraphics[width=\linewidth]{fig/appendix_edm/1_6.0_middle_4x4_grid.png}
        \caption{CFG-EDM (18 NFE), w=7.0}
    \end{subfigure}
    \begin{subfigure}{0.25\textwidth}
        \includegraphics[width=\linewidth]{fig/appendix_edm/1_12.0_middle_4x4_grid.png}
        \caption{CFG-EDM (18 NFE), w=13.0}
    \end{subfigure}
    \begin{subfigure}{0.25\textwidth}
        \includegraphics[width=\linewidth]{fig/appendix_igct/1_0.0_middle_4x4_grid.png}
        \caption{iGCT (1 NFE), w=1.0}
    \end{subfigure}
    % Second image
    \begin{subfigure}{0.25\textwidth}
        \includegraphics[width=\linewidth]{fig/appendix_igct/1_6.0_middle_4x4_grid.png}
        \caption{iGCT (1 NFE), w=7.0}
    \end{subfigure}
    % Third image
    \begin{subfigure}{0.25\textwidth}
        \includegraphics[width=\linewidth]{fig/appendix_igct/1_12.0_middle_4x4_grid.png}
        \caption{iGCT (1 NFE), w=13.0}
    \end{subfigure}
    \caption{CIFAR-10 "car"}
    \label{fig:CIFAR-10_guided_2}
\end{figure*}
\begin{figure*}[t]
    \centering
    % First image
    \begin{subfigure}{0.25\textwidth}
        \includegraphics[width=\linewidth]{fig/appendix_edm/2_0.0_middle_4x4_grid.png}
        \caption{CFG-EDM (18 NFE), w=1.0}
    \end{subfigure}
    \begin{subfigure}{0.25\textwidth}
        \includegraphics[width=\linewidth]{fig/appendix_edm/2_6.0_middle_4x4_grid.png}
        \caption{CFG-EDM (18 NFE), w=7.0}
    \end{subfigure}
    \begin{subfigure}{0.25\textwidth}
        \includegraphics[width=\linewidth]{fig/appendix_edm/2_12.0_middle_4x4_grid.png}
        \caption{CFG-EDM (18 NFE), w=13.0}
    \end{subfigure}
    \begin{subfigure}{0.25\textwidth}
        \includegraphics[width=\linewidth]{fig/appendix_igct/2_0.0_middle_4x4_grid.png}
        \caption{iGCT (1 NFE), w=1.0}
    \end{subfigure}
    % Second image
    \begin{subfigure}{0.25\textwidth}
        \includegraphics[width=\linewidth]{fig/appendix_igct/2_6.0_middle_4x4_grid.png}
        \caption{iGCT (1 NFE), w=7.0}
    \end{subfigure}
    % Third image
    \begin{subfigure}{0.25\textwidth}
        \includegraphics[width=\linewidth]{fig/appendix_igct/2_12.0_middle_4x4_grid.png}
        \caption{iGCT (1 NFE), w=13.0}
    \end{subfigure}
    \caption{CIFAR-10 "bird"}
    \label{fig:CIFAR-10_guided_3}
\end{figure*}
\begin{figure*}[t]
    \centering
    % First image
    \begin{subfigure}{0.25\textwidth}
        \includegraphics[width=\linewidth]{fig/appendix_edm/3_0.0_middle_4x4_grid.png}
        \caption{CFG-EDM (18 NFE), w=1.0}
    \end{subfigure}
    \begin{subfigure}{0.25\textwidth}
        \includegraphics[width=\linewidth]{fig/appendix_edm/3_6.0_middle_4x4_grid.png}
        \caption{CFG-EDM (18 NFE), w=7.0}
    \end{subfigure}
    \begin{subfigure}{0.25\textwidth}
        \includegraphics[width=\linewidth]{fig/appendix_edm/3_12.0_middle_4x4_grid.png}
        \caption{CFG-EDM (18 NFE), w=13.0}
    \end{subfigure}
    \begin{subfigure}{0.25\textwidth}
        \includegraphics[width=\linewidth]{fig/appendix_igct/3_0.0_middle_4x4_grid.png}
        \caption{iGCT (1 NFE), w=1.0}
    \end{subfigure}
    % Second image
    \begin{subfigure}{0.25\textwidth}
        \includegraphics[width=\linewidth]{fig/appendix_igct/3_6.0_middle_4x4_grid.png}
        \caption{iGCT (1 NFE), w=7.0}
    \end{subfigure}
    % Third image
    \begin{subfigure}{0.25\textwidth}
        \includegraphics[width=\linewidth]{fig/appendix_igct/3_12.0_middle_4x4_grid.png}
        \caption{iGCT (1 NFE), w=13.0}
    \end{subfigure}
    \caption{CIFAR-10 "cat"}
    \label{fig:CIFAR-10_guided_4}
\end{figure*}
\begin{figure*}[t]
    \centering
    % First image
    \begin{subfigure}{0.25\textwidth}
        \includegraphics[width=\linewidth]{fig/appendix_edm/4_0.0_middle_4x4_grid.png}
        \caption{CFG-EDM (18 NFE), w=1.0}
    \end{subfigure}
    \begin{subfigure}{0.25\textwidth}
        \includegraphics[width=\linewidth]{fig/appendix_edm/4_6.0_middle_4x4_grid.png}
        \caption{CFG-EDM (18 NFE), w=7.0}
    \end{subfigure}
    \begin{subfigure}{0.25\textwidth}
        \includegraphics[width=\linewidth]{fig/appendix_edm/4_12.0_middle_4x4_grid.png}
        \caption{CFG-EDM (18 NFE), w=13.0}
    \end{subfigure}
    \begin{subfigure}{0.25\textwidth}
        \includegraphics[width=\linewidth]{fig/appendix_igct/4_0.0_middle_4x4_grid.png}
        \caption{iGCT (1 NFE), w=1.0}
    \end{subfigure}
    % Second image
    \begin{subfigure}{0.25\textwidth}
        \includegraphics[width=\linewidth]{fig/appendix_igct/4_6.0_middle_4x4_grid.png}
        \caption{iGCT (1 NFE), w=7.0}
    \end{subfigure}
    % Third image
    \begin{subfigure}{0.25\textwidth}
        \includegraphics[width=\linewidth]{fig/appendix_igct/4_12.0_middle_4x4_grid.png}
        \caption{iGCT (1 NFE), w=13.0}
    \end{subfigure}
    \caption{CIFAR-10 "deer"}
    \label{fig:CIFAR-10_guided_5}
\end{figure*}
\begin{figure*}[t]
    \centering
    % First image
    \begin{subfigure}{0.25\textwidth}
        \includegraphics[width=\linewidth]{fig/appendix_edm/5_0.0_middle_4x4_grid.png}
        \caption{CFG-EDM (18 NFE), w=1.0}
    \end{subfigure}
    \begin{subfigure}{0.25\textwidth}
        \includegraphics[width=\linewidth]{fig/appendix_edm/5_6.0_middle_4x4_grid.png}
        \caption{CFG-EDM (18 NFE), w=7.0}
    \end{subfigure}
    \begin{subfigure}{0.25\textwidth}
        \includegraphics[width=\linewidth]{fig/appendix_edm/5_12.0_middle_4x4_grid.png}
        \caption{CFG-EDM (18 NFE), w=13.0}
    \end{subfigure}
    \begin{subfigure}{0.25\textwidth}
        \includegraphics[width=\linewidth]{fig/appendix_igct/5_0.0_middle_4x4_grid.png}
        \caption{iGCT (1 NFE), w=1.0}
    \end{subfigure}
    % Second image
    \begin{subfigure}{0.25\textwidth}
        \includegraphics[width=\linewidth]{fig/appendix_igct/5_6.0_middle_4x4_grid.png}
        \caption{iGCT (1 NFE), w=7.0}
    \end{subfigure}
    % Third image
    \begin{subfigure}{0.25\textwidth}
        \includegraphics[width=\linewidth]{fig/appendix_igct/5_12.0_middle_4x4_grid.png}
        \caption{iGCT (1 NFE), w=13.0}
    \end{subfigure}
    \caption{CIFAR-10 "dog"}
    \label{fig:CIFAR-10_guided_6}
\end{figure*}
\begin{figure*}[t]
    \centering
    % First image
    \begin{subfigure}{0.25\textwidth}
        \includegraphics[width=\linewidth]{fig/appendix_edm/6_0.0_middle_4x4_grid.png}
        \caption{CFG-EDM (18 NFE), w=1.0}
    \end{subfigure}
    \begin{subfigure}{0.25\textwidth}
        \includegraphics[width=\linewidth]{fig/appendix_edm/6_6.0_middle_4x4_grid.png}
        \caption{CFG-EDM (18 NFE), w=7.0}
    \end{subfigure}
    \begin{subfigure}{0.25\textwidth}
        \includegraphics[width=\linewidth]{fig/appendix_edm/6_12.0_middle_4x4_grid.png}
        \caption{CFG-EDM (18 NFE), w=13.0}
    \end{subfigure}
    \begin{subfigure}{0.25\textwidth}
        \includegraphics[width=\linewidth]{fig/appendix_igct/6_0.0_middle_4x4_grid.png}
        \caption{iGCT (1 NFE), w=1.0}
    \end{subfigure}
    % Second image
    \begin{subfigure}{0.25\textwidth}
        \includegraphics[width=\linewidth]{fig/appendix_igct/6_6.0_middle_4x4_grid.png}
        \caption{iGCT (1 NFE), w=7.0}
    \end{subfigure}
    % Third image
    \begin{subfigure}{0.25\textwidth}
        \includegraphics[width=\linewidth]{fig/appendix_igct/6_12.0_middle_4x4_grid.png}
        \caption{iGCT (1 NFE), w=13.0}
    \end{subfigure}
    \caption{CIFAR-10 "frog"}
    \label{fig:CIFAR-10_guided_7}
\end{figure*}
\begin{figure*}[t]
    \centering
    % First image
    \begin{subfigure}{0.25\textwidth}
        \includegraphics[width=\linewidth]{fig/appendix_edm/7_0.0_middle_4x4_grid.png}
        \caption{CFG-EDM (18 NFE), w=1.0}
    \end{subfigure}
    \begin{subfigure}{0.25\textwidth}
        \includegraphics[width=\linewidth]{fig/appendix_edm/7_6.0_middle_4x4_grid.png}
        \caption{CFG-EDM (18 NFE), w=7.0}
    \end{subfigure}
    \begin{subfigure}{0.25\textwidth}
        \includegraphics[width=\linewidth]{fig/appendix_edm/7_12.0_middle_4x4_grid.png}
        \caption{CFG-EDM (18 NFE), w=13.0}
    \end{subfigure}
    \begin{subfigure}{0.25\textwidth}
        \includegraphics[width=\linewidth]{fig/appendix_igct/7_0.0_middle_4x4_grid.png}
        \caption{iGCT (1 NFE), w=1.0}
    \end{subfigure}
    % Second image
    \begin{subfigure}{0.25\textwidth}
        \includegraphics[width=\linewidth]{fig/appendix_igct/7_6.0_middle_4x4_grid.png}
        \caption{iGCT (1 NFE), w=7.0}
    \end{subfigure}
    % Third image
    \begin{subfigure}{0.25\textwidth}
        \includegraphics[width=\linewidth]{fig/appendix_igct/7_12.0_middle_4x4_grid.png}
        \caption{iGCT (1 NFE), w=13.0}
    \end{subfigure}
    \caption{CIFAR-10 "horse"}
    \label{fig:CIFAR-10_guided_8}
\end{figure*}
\begin{figure*}[t]
    \centering
    % First image
    \begin{subfigure}{0.25\textwidth}
        \includegraphics[width=\linewidth]{fig/appendix_edm/8_0.0_middle_4x4_grid.png}
        \caption{CFG-EDM (18 NFE), w=1.0}
    \end{subfigure}
    \begin{subfigure}{0.25\textwidth}
        \includegraphics[width=\linewidth]{fig/appendix_edm/8_6.0_middle_4x4_grid.png}
        \caption{CFG-EDM (18 NFE), w=7.0}
    \end{subfigure}
    \begin{subfigure}{0.25\textwidth}
        \includegraphics[width=\linewidth]{fig/appendix_edm/8_12.0_middle_4x4_grid.png}
        \caption{CFG-EDM (18 NFE), w=13.0}
    \end{subfigure}
    \begin{subfigure}{0.25\textwidth}
        \includegraphics[width=\linewidth]{fig/appendix_igct/8_0.0_middle_4x4_grid.png}
        \caption{iGCT (1 NFE), w=1.0}
    \end{subfigure}
    % Second image
    \begin{subfigure}{0.25\textwidth}
        \includegraphics[width=\linewidth]{fig/appendix_igct/8_6.0_middle_4x4_grid.png}
        \caption{iGCT (1 NFE), w=7.0}
    \end{subfigure}
    % Third image
    \begin{subfigure}{0.25\textwidth}
        \includegraphics[width=\linewidth]{fig/appendix_igct/8_12.0_middle_4x4_grid.png}
        \caption{iGCT (1 NFE), w=13.0}
    \end{subfigure}
    \caption{CIFAR-10 "ship"}
    \label{fig:CIFAR-10_guided_9}
\end{figure*}
\begin{figure*}[t]
    \centering
    % First image
    \begin{subfigure}{0.25\textwidth}
        \includegraphics[width=\linewidth]{fig/appendix_edm/9_0.0_middle_4x4_grid.png}
        \caption{CFG-EDM (18 NFE), w=1.0}
    \end{subfigure}
    \begin{subfigure}{0.25\textwidth}
        \includegraphics[width=\linewidth]{fig/appendix_edm/9_6.0_middle_4x4_grid.png}
        \caption{CFG-EDM (18 NFE), w=7.0}
    \end{subfigure}
    \begin{subfigure}{0.25\textwidth}
        \includegraphics[width=\linewidth]{fig/appendix_edm/9_12.0_middle_4x4_grid.png}
        \caption{CFG-EDM (18 NFE), w=13.0}
    \end{subfigure}
    \begin{subfigure}{0.25\textwidth}
        \includegraphics[width=\linewidth]{fig/appendix_igct/9_0.0_middle_4x4_grid.png}
        \caption{iGCT (1 NFE), w=1.0}
    \end{subfigure}
    % Second image
    \begin{subfigure}{0.25\textwidth}
        \includegraphics[width=\linewidth]{fig/appendix_igct/9_6.0_middle_4x4_grid.png}
        \caption{iGCT (1 NFE), w=7.0}
    \end{subfigure}
    % Third image
    \begin{subfigure}{0.25\textwidth}
        \includegraphics[width=\linewidth]{fig/appendix_igct/9_12.0_middle_4x4_grid.png}
        \caption{iGCT (1 NFE), w=13.0}
    \end{subfigure}
    \caption{CIFAR-10 "truck"}
    \label{fig:CIFAR-10_guided_10}
\end{figure*}


\begin{figure*}[b]
    \centering
    % First image
    \begin{subfigure}{0.25\textwidth}
        \includegraphics[width=\linewidth]{fig/appendix_im64_edm/edm_class_291_w=0.0.png}
        \caption{CFG-EDM (18 NFE), w=1.0}
    \end{subfigure}
    \begin{subfigure}{0.25\textwidth}
        \includegraphics[width=\linewidth]{fig/appendix_im64_edm/edm_class_291_w=6.0.png}
        \caption{CFG-EDM (18 NFE), w=7.0}
    \end{subfigure}
    \begin{subfigure}{0.25\textwidth}
        \includegraphics[width=\linewidth]{fig/appendix_im64_edm/edm_class_291_w=12.0.png}
        \caption{CFG-EDM (18 NFE), w=13.0}
    \end{subfigure}
    \begin{subfigure}{0.25\textwidth}
        \includegraphics[width=\linewidth]{fig/appendix_im64_igct/class_291_w=0.0.png}
        \caption{iGCT (2 NFE), w=1.0}
    \end{subfigure}
    \begin{subfigure}{0.25\textwidth}
        \includegraphics[width=\linewidth]{fig/appendix_im64_igct/class_291_w=6.0.png}
        \caption{iGCT (2 NFE), w=7.0}
    \end{subfigure}
    % Third image
    \begin{subfigure}{0.25\textwidth}
        \includegraphics[width=\linewidth]{fig/appendix_im64_igct/class_291_w=12.0.png}
        \caption{iGCT (2 NFE), w=13.0}
    \end{subfigure}
    \caption{ImageNet64 "lion"}
    \label{fig:im64_guided_1}
\end{figure*}



\begin{figure*}[b]
    \centering
    % First image
    \begin{subfigure}{0.25\textwidth}
        \includegraphics[width=\linewidth]{fig/appendix_im64_edm/edm_class_292_w=0.0.png}
        \caption{CFG-EDM (18 NFE), w=1.0}
    \end{subfigure}
    \begin{subfigure}{0.25\textwidth}
        \includegraphics[width=\linewidth]{fig/appendix_im64_edm/edm_class_292_w=6.0.png}
        \caption{CFG-EDM (18 NFE), w=7.0}
    \end{subfigure}
    \begin{subfigure}{0.25\textwidth}
        \includegraphics[width=\linewidth]{fig/appendix_im64_edm/edm_class_292_w=12.0.png}
        \caption{CFG-EDM (18 NFE), w=13.0}
    \end{subfigure}
    \begin{subfigure}{0.25\textwidth}
        \includegraphics[width=\linewidth]{fig/appendix_im64_igct/class_292_w=0.0.png}
        \caption{iGCT (2 NFE), w=1.0}
    \end{subfigure}
    \begin{subfigure}{0.25\textwidth}
        \includegraphics[width=\linewidth]{fig/appendix_im64_igct/class_292_w=6.0.png}
        \caption{iGCT (2 NFE), w=7.0}
    \end{subfigure}
    % Third image
    \begin{subfigure}{0.25\textwidth}
        \includegraphics[width=\linewidth]{fig/appendix_im64_igct/class_292_w=12.0.png}
        \caption{iGCT (2 NFE), w=13.0}
    \end{subfigure}
    \caption{ImageNet64 "tiger"}
    \label{fig:im64_guided_2}
\end{figure*}


\begin{figure*}[b]
    \centering
    % First image
    \begin{subfigure}{0.25\textwidth}
        \includegraphics[width=\linewidth]{fig/appendix_im64_edm/edm_class_28_w=0.0.png}
        \caption{CFG-EDM (18 NFE), w=1.0}
    \end{subfigure}
    \begin{subfigure}{0.25\textwidth}
        \includegraphics[width=\linewidth]{fig/appendix_im64_edm/edm_class_28_w=6.0.png}
        \caption{CFG-EDM (18 NFE), w=7.0}
    \end{subfigure}
    \begin{subfigure}{0.25\textwidth}
        \includegraphics[width=\linewidth]{fig/appendix_im64_edm/edm_class_28_w=12.0.png}
        \caption{CFG-EDM (18 NFE), w=13.0}
    \end{subfigure}
    \begin{subfigure}{0.25\textwidth}
        \includegraphics[width=\linewidth]{fig/appendix_im64_igct/class_28_w=0.0.png}
        \caption{iGCT (2 NFE), w=1.0}
    \end{subfigure}
    \begin{subfigure}{0.25\textwidth}
        \includegraphics[width=\linewidth]{fig/appendix_im64_igct/class_28_w=6.0.png}
        \caption{iGCT (2 NFE), w=7.0}
    \end{subfigure}
    % Third image
    \begin{subfigure}{0.25\textwidth}
        \includegraphics[width=\linewidth]{fig/appendix_im64_igct/class_28_w=12.0.png}
        \caption{iGCT (2 NFE), w=13.0}
    \end{subfigure}
    \caption{ImageNet64 "salamander"}
    \label{fig:im64_guided_3}
\end{figure*}


\begin{figure*}[b]
    \centering
    % First image
    \begin{subfigure}{0.25\textwidth}
        \includegraphics[width=\linewidth]{fig/appendix_im64_edm/edm_class_407_w=0.0.png}
        \caption{CFG-EDM (18 NFE), w=1.0}
    \end{subfigure}
    \begin{subfigure}{0.25\textwidth}
        \includegraphics[width=\linewidth]{fig/appendix_im64_edm/edm_class_407_w=6.0.png}
        \caption{CFG-EDM (18 NFE), w=7.0}
    \end{subfigure}
    \begin{subfigure}{0.25\textwidth}
        \includegraphics[width=\linewidth]{fig/appendix_im64_edm/edm_class_407_w=12.0.png}
        \caption{CFG-EDM (18 NFE), w=13.0}
    \end{subfigure}
    \begin{subfigure}{0.25\textwidth}
        \includegraphics[width=\linewidth]{fig/appendix_im64_igct/class_407_w=0.0.png}
        \caption{iGCT (2 NFE), w=1.0}
    \end{subfigure}
    \begin{subfigure}{0.25\textwidth}
        \includegraphics[width=\linewidth]{fig/appendix_im64_igct/class_407_w=6.0.png}
        \caption{iGCT (2 NFE), w=7.0}
    \end{subfigure}
    % Third image
    \begin{subfigure}{0.25\textwidth}
        \includegraphics[width=\linewidth]{fig/appendix_im64_igct/class_407_w=12.0.png}
        \caption{iGCT (2 NFE), w=13.0}
    \end{subfigure}
    \caption{ImageNet64 "ambulance"}
    \label{fig:im64_guided_4}
\end{figure*}

\end{document}
