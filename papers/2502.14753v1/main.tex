\documentclass[10pt, letter, onecolumn]{main}

% \usepackage{csquotes}
% \usepackage[bibstyle=nature,citestyle=numeric-comp,
%             natbib=true,backend=biber,maxbibnames=5,
%             giveninits=false,sorting=none,defernumbers=true]{biblatex}
% \usepackage[superscript,biblabel]{cite}
% note: sorting=none gives refs in order they appear
\usepackage[pdftex]{graphicx}
\usepackage{pifont}
\usepackage{amssymb}
\usepackage{multirow}
\usepackage{array}
\usepackage{dblfloatfix}
\usepackage{titlesec}
\usepackage[numbers,sort&compress]{natbib}
\usepackage{nameref}
\usepackage{varioref}
% \usepackage{hyperref}
\usepackage{etoc}


% \addbibresource{main.bib}
% \renewcommand*{\bibfont}{\linespread{0.8}\footnotesize}


%%%%%%%%%%%%%%%%%%%%%%%%%%%%%%%%%%
%%% user added packages / commands

\setlength{\abovecaptionskip}{10pt plus 3pt minus 2pt} % add spacing b/w figure and table
\newlength{\cboxlength}
\settowidth{\cboxlength}{7.1 $\pm$ 13.2} % The widest entry

% for nice-looking up arrows
\usepackage{amsmath,xparse,mleftright}
\NewDocumentCommand{\up}{som}{%
  \IfBooleanTF{#1}
    {\upext{#3}}
    {#3\IfNoValueTF{#2}{\mathord}{#2}\uparrow}%
}
\NewDocumentCommand{\upext}{m}{%
  \mleft.\kern-\nulldelimiterspace#1\mright\uparrow
}

\usepackage{xcolor} % For text color
\usepackage{soul} % For highlighting

\usepackage{adjustbox}
\usepackage[most]{tcolorbox}
\usepackage{float}
\usepackage{xspace}
\usepackage[symbol]{footmisc}
\usepackage{lineno}

\tcbset{
  aibox/.style={
    width=394.18663pt,
    top=10pt,
    colback=white,
    colframe=black,
    colbacktitle=black,
    enhanced,
    center,
    attach boxed title to top left={yshift=-0.1in,xshift=0.15in},
    boxed title style={boxrule=0pt,colframe=white,},
  }
}
\newtcolorbox{AIbox}[2][]{aibox,title=#2,#1}

% preliminaries
\ifdefined\red
    \renewcommand{\red}[1]{\textcolor{red}{#1}}
\else
    \newcommand{\red}[1]{\textcolor{red}{#1}}
\fi
\ifdefined\blue
    \renewcommand{\blue}[1]{\textcolor{blue}{#1}}
\else
    \newcommand{\blue}[1]{\textcolor{blue}{#1}}
\fi
% \newcommand{\hhll}[1]{\sethlcolor{yellow}\hl{#1}} % highlights on
\newcommand{\hhll}[1]{#1} % highlights off
\newcommand{\hhllred}[1]{\sethlcolor{red}\hl{#1}} 

% small caps formatting of model names
\newcommand{\flantfive}{\textsc{FLAN-T5}}
\newcommand*{\eg}{e.g.\@\xspace}

% \usepackage{todonotes}

% \newcommand{\conditionalprintbibliography}[1]{
%   \begingroup
%     \def\do#1{\ifcsname blx@bsee@#1\endcsname\else
%       \global\cslet{blx@bsee@#1}\relax
%       \printbibliography[heading=subbibintoc, segment=#1, resetnumbers=false]
%     \fi}
%     \do{#1}
%   \endgroup
% }

%%%%%%%%%%%%%%%%%%%%%%%%%%%%%%%%%%
\newcommand{\maya}[1]{{\noindent\color{blue}[Maya: #1]}}
\newcommand{\name}[1][]{\textsc{CompRx}}

\makeatletter
\renewcommand\AB@authnote[1]{}
\renewcommand\AB@affilnote[1]{}
\makeatother

\usepackage{anyfontsize} % allows for precise control over title font size
\usepackage{titlesec}
\titleformat{\section}{\normalfont\Large\bfseries}{\thesection}{1em}{#1}

% -------------  Title  ---------------------- 
\title{{\fontsize{16.5pt}{15.5pt}\selectfont MedVAE: Efficient Automated Interpretation of Medical Images with Large-Scale Generalizable Autoencoders}}


% -------------  Authors ----------------------

\author[]{Maya Varma$^{1,*}$, Ashwin Kumar$^{1,*}$, Rogier van der Sluijs$^{1,*}$, Sophie Ostmeier$^{1}$, Louis Blankemeier$^{1}$, Pierre Chambon$^{1}$, Christian Bluethgen$^{1}$, Jip Prince$^{2}$, Curtis Langlotz$^{1}$, Akshay Chaudhari$^{1}$}

\affil{\footnotesize{$^1$Stanford Center for Artificial Intelligence in Medicine and Imaging, Stanford University, Palo Alto, CA, USA. $^2$UMC Utrecht, Utrecht, Netherlands}}

% \affil{}
\renewcommand{\correspondingauthor}[1]{$\ast$~Equal contributions.}
\usepackage{xcolor}
\usepackage{hyperref}

\hypersetup{
    colorlinks=true,   % Enables colored links
    urlcolor=blue      % Ensures URLs are blue
}
\usepackage[noabbrev,capitalize]{cleveref}


\begin{document}
\begin{abstract}
Medical images are acquired at high resolutions with large fields of view in order to capture fine-grained features necessary for clinical decision-making. Consequently, training deep learning models on medical images can incur large computational costs. In this work, we address the challenge of downsizing medical images in order to improve downstream computational efficiency while preserving clinically-relevant features. We introduce \textit{MedVAE}, a family of six large-scale 2D and 3D autoencoders capable of encoding medical images as downsized latent representations and decoding latent representations back to high-resolution images. We train MedVAE autoencoders using a novel two-stage training approach with 1,052,730 medical images. Across diverse tasks obtained from 20 medical image datasets, we demonstrate that (1) utilizing MedVAE latent representations in place of high-resolution images when training downstream models can lead to efficiency benefits (up to 70x improvement in throughput) while simultaneously preserving clinically-relevant features and (2) MedVAE can decode latent representations back to high-resolution images with high fidelity. Our work demonstrates that large-scale, generalizable autoencoders can help address critical efficiency challenges in the medical domain. Our code is available at \href{https://github.com/StanfordMIMI/MedVAE}{https://github.com/StanfordMIMI/MedVAE}. 
\end{abstract}

\maketitle


\vspace{10mm}
% \begin{refsegment}
% \defbibfilter{notother}{not segment=\therefsegment}
\nolinenumbers
\section{Introduction}
\label{sec:intro}
% Image editing methods in diffusion models depend on user-defined control directions - users can unlock their creativity using these methods by specifying the desired manipulation through prompts~\cite{gandikota2023concept}, reference images~\cite{ruiz2022dreambooth, kumari2022customdiffusion, gal2022image, chen2024trainingfreeregionalpromptingdiffusion}, or attribute vectors~\cite{parmar2023zero,hertz2022prompt}. In this work, we ask a fundamentally different question: \emph{Can we automatically discover the underlying visual structure of a concept within diffusion model's knowledge?} %Rather than requiring user-specified controls, we aim to decompose the model's internal knowledge into meaningful directions.

% This question touches on a fundamental limitation in how we interact with diffusion models. Current control methods ~\cite{zhang2023addingconditionalcontroltexttoimage, gandikota2023concept, ye2023ipadaptertextcompatibleimage,ye2023ipadaptertextcompatibleimage, hertz2024stylealignedimagegeneration, li2023photomaker, shi2024instantbooth, chen2024trainingfreeregionalpromptingdiffusion} require users to specify their desired manipulations in advance, limiting interactive creativity. This contrasts with natural human artistic workflows, where creators dynamically explore creative ideas while jointly refining them toward meaningful artistic outcomes~\cite{hoffmann2016modeling}. This synergy between specification and exploration is not new to generative models. Early GAN architectures naturally developed disentangled latent spaces that enabled continuous\cite{harkonen2020ganspace,radford2015unsupervised, wu2021stylespace, shen2020interfacegan}, compositional control over generated images. Users could explore these spaces to discover interesting variations that would be difficult to describe in words~\cite{wu2021stylespace}, then combine them to achieve their creative goals~\cite{grabe2022towards}. 


% While diffusion models have largely superseded GANs in conditional image synthesis~\cite{dhariwal2021diffusion},  their underlying structure remains less understood. Diffusion models achieve remarkable diversity through high-dimensional latents, unlike GANs' compact latent spaces.  With a single prompt, diffusion models can generate radically different variations through different random initializations of input noise. We ask - Is it possible to discover interpretable structure within this vast space of variations?

Text-to-image diffusion models are capable of generating remarkable visual variations from a single prompt through different random initializations. However, this vast creative potential remains largely opaque to users---while we can generate diverse images, we lack understanding of the underlying structure of these variations. This presents a fundamental challenge: how can we discover and expose the latent visual capabilities encoded within these models?

\let\thefootnote\relax \footnote{$^{*}$Correspondence to \texttt{gandikota.ro@northeastern.edu}}

The challenge touches on a key limitation in how we interact with diffusion models today. Current control methods require users to explicitly specify their desired edits in advance through prompts~\cite{gandikota2023concept}, reference images~\cite{zhang2023addingconditionalcontroltexttoimage, chen2024trainingfreeregionalpromptingdiffusion, ruiz2022dreambooth,kumari2022customdiffusion, Ryu_lora, hu2021lora}, or attribute vectors~\cite{ye2023ipadaptertextcompatibleimage, hertz2024stylealignedimagegeneration, li2023photomaker, shi2024instantbooth,parmar2023zero,hertz2022prompt}. That contrasts sharply with natural human creative workflows, where artists dynamically explore creative ideas and jointly refine them toward meaningful artistic outcomes~\cite{hoffmann2016modeling}. The need for pre-specified controls creates a barrier between users and the full creative potential of these models.

Interestingly, earlier generative models like GANs~\cite{gans,karras2019style,brock2018large} naturally developed more interpretable internal structures. Their compact latent spaces often exhibited emergent disentanglement~\cite{harkonen2020ganspace,radford2015unsupervised, wu2021stylespace, shen2020interfacegan}, enabling continuous and compositional control over generated images. Users could explore these spaces to discover interesting variations that would be difficult to describe in words~\cite{wu2021stylespace}, then combine them to achieve their creative goals~\cite{grabe2022towards}.

Diffusion models have largely superseded GANs in conditional image synthesis~\cite{dhariwal2021diffusion}, achieving greater diversity through much higher-dimensional latents. And yet an understanding of the underlying structure of these larger latent spaces has remained elusive. In this work, we ask a fundamental question: \emph{Can we automatically discover the visual structure within a diffusion model's knowledge of a concept?} Rather than requiring user-specified controls, we aim to decompose the model's internal representations into expressive directions that users can explore and combine.

To address these needs, we present \textbf{SliderSpace}, a framework that brings systematic explorability to diffusion models. Given just a text prompt, SliderSpace discovers a canonical set of meaningful, diverse, and controllable directions within the model's knowledge of that concept. Each direction is implemented as a low-rank adapter~\cite{hu2021lora} that can be scaled and composed with others, allowing users to explore and smoothly combine different aspects of variation, as shown in Figure~\ref{fig:intro}.

We ground SliderSpace discovery in three key requirements for meaningful decomposition of a diffusion model's visual manifold: 
\begin{enumerate}
    \item \textbf{Unsupervised Discovery:} The decomposition process should emerge from the intrinsic structure of the model's learned representation, rather than being guided by predefined attributes. This ensures we capture the true topology of the model's knowledge space rather than projecting our assumptions onto it.
    
    \item \textbf{Semantic Orthogonality:} Each discovered control must represent a distinct semantic direction. This is enforced in a semantic feature space, like CLIP, where every slider has an orthogonal effect in embeddings. This prevents discovering multiple controls that create similar semantic effects, making the system more efficient and easier.
    
    \item \textbf{Distribution Consistency:} Directions must induce consistent transformations across both random seeds and prompt variations. 
\end{enumerate}

These requirements naturally lead to our proposed framework, which we formalize in Section~\ref{sec:method}. As we show in our experiments, SliderSpace is architecture-agnostic, working with both conventional U-Net based models like Stable Diffusion~\cite{rombach2022high, rombach2022sd20, podell2023sdxl, turbo, dmd} and recent transformer-based architectures like Flux~\cite{flux}.

We demonstrate the expressiveness of SliderSpace through three applications: First, we show how SliderSpace can decompose high-level concepts into diverse and expressive components, revealing the natural axes of variation in the model's understanding. Second, we explore artistic style variation, where SliderSpace discovers directions that match or exceed the diversity of manually curated artist lists while being judged more useful by human evaluators. Finally, we show how SliderSpace can help reverse the mode collapse commonly observed in distilled diffusion models, restoring diversity while maintaining generation speed.

Beyond providing practical creative control, SliderSpace opens new avenues for understanding and utilizing the latent capabilities of diffusion models. By mapping these models' visual potential into intuitive, composable directions, we take a step toward making their creative possibilities more accessible and interpretable to users.

% Image editing methods in diffusion models unlock the creativity of users. In this work we ask an alternate question: \emph{Can we organize and expose what of the diffusion model is already capable of?}.
% Existing methods for controlling image generation typically require users to manually specify edit directions for desired changes. This process is time-consuming, requires technical expertise, and limits the spontaneity of the creative process. For instance, if a user wants to adjust the smile of a generated person, they must explicitly request this edit, often through imprecise prompt engineering or model fine-tuning. This approach of predefined controls or manual specifications restricts users from fully exploring the latent capabilities of the model. There may be interesting stylistic variations or attributes that the model can generate, but users have no easy way to discover or utilize these.

% Natural visual disentanglement was an emergent property in the latent space of Generative Adversarial Models (GANs) \cite{harkonen2020ganspace,radford2015unsupervised, wu2021stylespace, shen2020interfacegan}. In particular, it has been observed that StyleGAN~\cite{karras2019style} stylespace neurons offer detailed control over many meaningful aspects of images that would be difficult to describe in words~\cite{wu2021stylespace}. However, diffusion models do not share such a compact latent space~\cite{park2023unsupervised}; and efforts to uncover such a space in the semantic embeddings of the text conditioning have met with limited success \nik{Nick - is there a specific citation you were thinking about?}.

% In this work we introduce \textbf{SliderSpace}, which takes a step towards uncovering an analogous low dimensional representation of diffusion models' visual breadth; in essence treating the diffusion model as many generators sharing parameters, where a particular generator is defined by a specific prompt. For a given prompt we sample many random seeds (and optionally prompt expansions using an LLM), generate the corresponding images, and apply an off the shelf feature extractor (in this work CLIP, but our method can be applied to any differentiable feature extractor). We use PCA to analyze these features, and for each of the leading $k$ principal components we train a LoRA \cite{} which causes the diffusion model to produces images which increase the feature magnitude along that component when passed back through the same feature extractor. This leads to a 'Slider' for each principal component, because each LoRA can be scaled and applied to the original diffusion model, continuously varying those visual features in the generated results (as measured, in our case, by CLIP).

% There are many other works that enhance the controllability of diffusion models. One common approach is enabling users to add spatial constraints to a generation either manually, or via a reference image \cite{zhang2023addingconditionalcontroltexttoimage, chen2024trainingfreeregionalpromptingdiffusion}, a second is leveraging more abstract embeddings (e.g. identity, style) extracted from a reference image \cite{ye2023ipadaptertextcompatibleimage, hertz2024stylealignedimagegeneration, li2023photomaker, shi2024instantbooth}, a third is finetuning a foundation model to better generate a concept important to the user \cite{ruiz2022dreambooth, kumari2022customdiffusion, Ryu_lora, hu2021lora}, and a fourth (most relevant to this work) is finding low-rank adaptors of the model based on a prompt or small training set which can be scaled to provide continous control over one aspect of generated image (e.g. night vs day, basic vs luxury, etc.) \cite{gandikota2023concept}. SliderSpace is complementary to all of these methods and offers something distinct. All of the other methods we are aware require the user (and / or model designer) to know in advance what type of control they want. In contrast SliderSpace assists users in discovering and controlling hidden capabilities present in the diffusion model's distribution of possible generations.

%We propose that truly intuitive creative control in a text-to-image model should meet three key criteria: \emph{discoverability}, \emph{intuitiveness}, and \emph{specificity}. The model should reveal controllable attributes that may not be immediately obvious, offer controls that are easy to understand and manipulate, and ensure each control affects a distinct attribute of the generated image.

% We demonstrate the utility and power of SliderSpace using three applications built on top of SDXL-DMD \cite{dmd}, because its fast generation speed lends itself well to the continuous control offered by SliderSpace.

% First, we study concept decomposition (Section \ref{sec:concept_exp}), where we learn sliders for a specific concept (e.g. 'monster', 'waterfall', 'car'). Through quantitative metrics of diversity and text alignment we demonstrate that the learned sliders dramatically boost the diversity of generations when randomly applied without harming text alignment; we also ask humans to qualitatively judge these results in a user study where they find the SliderSpace results to be more 'Diverse', 'Useful', and 'Creative' than our baselines.

% Second, we attempt to compare the automatic discoveries of SliderSpace to a large scale manual study of artistic styles (Section \ref{sec:art_exp}), open-sourced by ParrotZone \cite{parrotzone}. In this study SDXL was prompted with over 4300 artist names,  and based on visual inspection the cases of successful stylistic mimicry recorded. Quantitatively SliderSpace more closely matches the distribution of artistic variation discovered by ParrotZone than other baselines, and in our user studies was judged to be significantly more 'Diverse' and 'Useful' than the baselines. To our surprise humans even judged SliderSpace results to be slightly more 'Diverse' than the results generated by the manually discovered artist names of \cite{parrotzone}.

% Third, we attempt to use SliderSpace to reverse the mode collapse commonly observed in distilled few-step diffusion models relative to the original teacher model (Section \ref{sec:diverse_exp}). We quantitatively demonstrate that applying SliderSpace to SDXL-DMD leads to more closely matching the distribution of images by the original teacher, SDXL.

%Through extensive experiments on various state-of-the-art text-to-image models, we demonstrate that SliderSpace significantly enhances user control and creative expression in AI-assisted image generation tasks. Our method enables a range of applications, including concept decomposition and control, diversity improvement in generated images, customization dissection and edits, and the exploration of artistic styles inherent in the model.

% SliderSpace goes beyond providing a practical tool for enhanced creative control. By mapping the visual potential of diffusion models it can open new avenues for generative creativity and deepens our understanding of each model's hidden potential.
\clearpage
\section{Results}
\subsection{Training MedVAE autoencoders}

Autoencoding methods are capable of encoding high-resolution images as downsized latent representations. For a given 2D input image with dimensions $H \times W$ with $B$ channels, an autoencoding method will output a downsized latent representation of size $H/(\sqrt{f}) \times (W/\sqrt{f}) \times C$. Here, $f$ represents the downsizing factor applied to the 2D area of the image and $C$ represents a pre-specified number of latent channels. 3D autoencoding methods follow a similar formulation, where input images are 3D in nature with dimensions $H \times W \times S$ with $B$ channels. Here, the downsizing factor $f$ is applied to the 3D volume of the image; as a result, the latent representation will have dimensions $(H/(\sqrt[3]{f}) \times (W/\sqrt[3]{f}) \times (S/(\sqrt[3]{f}) \times C$. Autoencoding methods are also capable of decoding latent representations back to reconstructed high-resolution images. 

We aim to develop large-scale, generalizable medical image autoencoders capable of preserving diverse clinically-relevant features in both latent representations and reconstructions. To this end, we first collect a large-scale training dataset with 1,021,356 2D images and 31,374 3D images curated from 19 multi-institutional, open-source datasets~\cite{johnson2019mimic,feng2021candid,jeong2022emory,sorkhei2021csaw,rsnamammo,nguyen2022vindrmammo,moreira2012inbreast,cai2023online,jack2008alzheimer,dagley2017harvard,insel2020a4,lamontagne2019oasis,bien2018deep,hooper2021impact,chilamkurthy2018development,wasserthal2023totalsegmentator,ji2022amos,armato2011lung,stanfordaimi_coca_2024}. Images are obtained from two chest X-ray datasets, six full-field digital mammogram (FFDM) datasets, four T1- and T2-weighted head magnetic resonance imaging (MRI) datasets, one knee MRI dataset, two head/neck CT datasts, two whole-body CT datasets, and two chest CT datasets.

% ******* Figure ********
\begin{figure}[ht]
\centering
\includegraphics[width=\textwidth, trim=0 0 0 0]{figures/method.pdf}
\caption{\textbf{Overview of training pipeline and evaluation tasks for MedVAE, a suite of large-scale autoencoders for medical images.} \textbf{a,} We train MedVAE autoencoders using a two-stage process. The first stage involves training base autoencoders using 2D images. \textbf{b,} The second stage of training aims to further refine the latent space such that clinically-relevant features are preserved across modalities. We introduce separate training procedures for 2D images (e.g. X-rays, mammograms) and 3D volumes (e.g. CT scans, MRI). \textbf{c,} We evaluate medical image autoencoders with respect to latent representation quality, storage and efficiency benefits arising from using latent representations rather than high-resolution images in downstream CAD pipelines, and reconstructed image quality.}
\label{fig:method}
\end{figure}


We then utilize this dataset to train a family of generalizable autoencoders for medical images. Motivated by prior work on natural images~\cite{rombach2022high}, we utilize variational autoencoders (VAEs) as the model backbone. We perform model training using a novel two-stage training scheme designed to optimize quality of latent representations and decoded reconstructions. Specifically, the first stage involves training base autoencoders using 2D images (Fig.~\ref{fig:method}a); we maximize the perceptual similarity between input images and reconstructed images using a perceptual loss~\cite{lpips}, a patch-based adversarial objective~\cite{isola2018patchgan}, and a domain-specific embedding consistency loss. Whereas existing works on autoencoders train using only this stage, the medical image domain introduces the added complexity of subtle, fine-grained features required for clinical interpretation; thus, we introduce a second stage of training, which aims to further refine the latent space such that clinically-relevant features are preserved across various modalities (Fig.~\ref{fig:method}b). Specifically, in the context of 2D imaging modalities (e.g. X-rays, mammograms), the second training stage takes the form of a lightweight training approach that leverages the embedding space of BiomedCLIP, a recently-developed medical vision-language foundation model~\cite{zhang2023biomedclip}, to enforce feature consistency between input images and latent representations. In the context of 3D imaging modalities (e.g. CT scans, MRI), the second training stage involves lifting the 2D autoencoder architecture to 3D and performing continued fine-tuning with 3D images. In Appendix Table~\ref{table:ablations2d} and Appendix Table~\ref{table:ablations3d}, we analyze the effects of each stage of training on latent representation quality. In total, the MedVAE family includes 4 large-scale 2D autoencoders and 2 large-scale 3D autoencoders trained with various downsizing factors $f$ and latent channels $C$. 

In order to assess the capabilities of MedVAE, we evaluate the extent to which latent representations and reconstructed images generated by MedVAE autoencoders can contribute to downstream storage and efficiency benefits while simultaneously preserving clinically-relevant features (Fig.~\ref{fig:method}c). Specifically, we assess (1) whether downsized latent representations can effectively replace high-resolution images in CAD pipelines while maintaining performance; (2) whether latent representations can reduce storage requirements and improve downstream efficiency; and (3) whether decoded reconstructions effectively preserve clinically-relevant features necessary for radiologist interpretation. 

\subsection{Latent representation quality}


\begin{table*}[t]
% \scriptsize

\centering
\resizebox{\linewidth}{!}{%
{%\renewcommand{\arraystretch}{1.2}%
\begin{tabular}{ lcccccccc }
\toprule
\textbf{}
& \multicolumn{2}{c}{\textbf{}}
& \multicolumn{5}{c}{\textbf{AUROC} $\uparrow$}
& \textbf{}
\\
\cmidrule(l{3pt}r{0pt}){4-8}
\cmidrule(l{3pt}r{0pt}){9-9}
% "
\textbf{Method}
& \textbf{$f$}
& \textbf{$C$}
& \small Malignancy
& \small Calcification
& \small BI-RADS
& \small Bone Age
& \small Wrist Fracture
& Average
\\ 
 & & &  \small (FFDM) & \small  (FFDM) & \small  (FFDM) &  (X-ray) &  (X-ray)
\\
\midrule
\small High-Resolution & 1 & 1 & \textbf{66.1$_{\pm0.5}$} & \textbf{62.4$_{\pm0.6}$} & \textbf{63.4$_{\pm0.1}$} & \textbf{80.2$_{\pm0.1}$}  & \textbf{73.7$_{\pm0.0}$} & \textbf{69.2}\\

\midrule
\small Nearest & 16 & 1   &  65.5$_{\pm0.1}$ & 59.7$_{\pm0.3}$ & 62.4$_{\pm0.1}$ & \textcolor{blue}{81.6$_{\pm0.1}$}  & 70.5$_{\pm0.0}$ & 67.9\\
\small Bilinear & 16 & 1    &65.5$_{\pm0.1}$ & 58.1$_{\pm0.3}$ & 61.1$_{\pm0.2}$ & \textcolor{blue}{81.6$_{\pm0.0}$}  & \textbf{71.2$_{\pm0.1}$} & 67.5 \\
\small Bicubic & 16 & 1   &  65.5$_{\pm0.4}$ & 58.5$_{\pm0.5}$ & 61.1$_{\pm0.0}$ & \textcolor{blue}{81.8$_{\pm0.2}$}  & 71.1$_{\pm0.1}$ & 67.6\\
\small KL-VAE & 16 & 3   & 59.7$_{\pm0.2}$ & 59.1$_{\pm0.3}$ & 58.5$_{\pm0.1}$ & 74.3$_{\pm0.1}$ & 64.5$_{\pm0.1}$ & 63.2\\
\small VQ-GAN & 16 & 3   & 57.4$_{\pm0.3}$  & 58.2$_{\pm0.4}$ & 62.3$_{\pm0.1}$ & 79.1$_{\pm0.2}$ & 65.8$_{\pm0.1}$ & 64.6\\
\small 2D MedVAE & 16 & 1   &  63.6$_{\pm0.6}$ & \textcolor{blue}{\textbf{63.9$_{\pm0.4}$}} & \textcolor{blue}{\textbf{65.3$_{\pm0.2}$}} & \textcolor{blue}{\textbf{84.6$_{\pm0.1}$}} & 70.3$_{\pm0.1}$ & \textcolor{blue}{\textbf{69.5}}\\
\small 2D MedVAE & 16  & 3 &  \textcolor{blue}{\textbf{66.1$_{\pm0.2}$}} &   61.7$_{\pm0.2}$ & 62.3$_{\pm0.1}$ & \textcolor{blue}{82.1$_{\pm0.1}$}  &   70.6$_{\pm0.1}$ & 68.6\\
\midrule

\small Nearest & 64 & 1    & 63.0$_{\pm0.1}$ & 58.8$_{\pm0.2}$ & 60.0$_{\pm0.2}$ & 72.1$_{\pm0.0}$  & 65.1$_{\pm0.1}$ & 63.8	\\
\small Bilinear & 64 & 1   & 61.5$_{\pm0.3}$ & 56.9$_{\pm0.4}$ & \textbf{61.3$_{\pm0.1}$} & 72.8$_{\pm0.5}$  & \textbf{67.9$_{\pm0.1}$} & 64.1\\
\small Bicubic & 64 & 1   & 61.2$_{\pm0.5}$ & 57.6$_{\pm0.4}$ & 61.1$_{\pm0.1}$ & 72.8$_{\pm0.2}$  & 67.9$_{\pm0.2}$ & 64.1\\
\small KL-VAE & 64 & 4   & 62.2$_{\pm0.7}$ &  55.8$_{\pm0.4}$ & 56.8$_{\pm0.1}$ & 65.7$_{\pm0.0}$ & 58.8$_{\pm0.0}$ & 59.9\\
\small VQ-GAN & 64 & 4    & 64.5$_{\pm0.5}$ & 57.3$_{\pm0.3}$ &  56.6$_{\pm0.1}$ & 67.6$_{\pm0.1}$  & 61.6$_{\pm0.2}$ & 61.5 \\
\small 2D MedVAE & 64 & 1  & 59.0$_{\pm0.3}$ & \textbf{59.4$_{\pm0.7}$} & 60.7$_{\pm0.1}$ & \textbf{73.5$_{\pm0.2}$} & 64.3$_{\pm0.1}$ & 63.4\\
\small 2D MedVAE & 64 & 4   & \textbf{64.9$_{\pm0.2}$} &  58.5$_{\pm0.3}$ & 60.6$_{\pm0.0}$ & 73.0$_{\pm0.2}$ & 66.7$_{\pm0.1}$ & \textbf{64.7}\\

\bottomrule
\end{tabular}
}
}
\caption{\textbf{Evaluating latent representation quality with 2D CAD tasks.} We evaluate the 2D MedVAE autoencoders on five 2D CAD tasks, and we report the mean AUROC and standard deviation across three random seeds. We compare MedVAE with three interpolation methods (nearest, bilinear, bicubic) and two natural image autoencoders (KL-VAE and VQ-GAN). Here, $f$ represents the downsizing factor applied to the 2D area of the input image and $C$ represents the number of latent channels. The best performing models on each task are bolded. We highlight methods that perfectly preserve clinically-relevant features (i.e. performance equals or exceeds performance when training with high-resolution images) in \textcolor{blue}{\textbf{blue}}.}
\label{table:image_classification}
\vspace{-1mm}
\end{table*}


\begin{table*}[ht]
% \scriptsize
\centering
{%
\begin{tabular}{ lcccccc }
\toprule
\textbf{}
& \multicolumn{2}{c}{\textbf{}}
& \multicolumn{3}{c}{\textbf{AUROC} $\uparrow$}
& \textbf{}
\\
\cmidrule(l{3pt}r{0pt}){4-6}
\cmidrule(l{3pt}r{0pt}){7-7}
\textbf{Method}
& \textbf{$f$}
& \textbf{$C$}
& \small Spine Fractures
& \small Skull Fractures
& \small Knee Injury
& Average
\\ 
 & & &  \small (CT) & \small (CT) & \small (MRI) &
\\
\midrule
\small High-Resolution & 1 & 1  & \textbf{82.9$_{\pm2.2}$} & \textbf{63.9$_{\pm6.3}$} & \textbf{69.9$_{\pm0.6}$} & \textbf{72.2}\\

\midrule
\small Bicubic & 64 & 1  &  77.3$_{\pm4.1}$ & \textcolor{blue}{64.8$_{\pm4.0}$} & 66.4$_{\pm2.3}$ & 69.5\\
\small KL-VAE & 64 & 3   & 68.8$_{\pm2.1}$ & 40.7$_{\pm9.1}$ & 63.9$_{\pm8.2}$  &  57.8\\
\small VQ-GAN & 64 & 3 & 73.2$_{\pm2.0}$  & 75.5$_{\pm14.8}$ & 63.6$_{\pm10.5}$ &   70.8 \\
\small 3D MedVAE & 64 & 1  & \textcolor{blue}{\textbf{83.7$_{\pm2.8}$}} & \textcolor{blue}{\textbf{87.0$_{\pm7.3}$}} & \textbf{68.4$_{\pm2.4}$} & \textcolor{blue}{\textbf{79.7}}\\
\midrule
\small Bicubic & 512 & 1 & \textbf{72.3$_{\pm2.2}$} & 38.4$_{\pm24.5}$ & \textbf{59.4$_{\pm2.5}$} & 56.7\\
\small KL-VAE & 512 & 4  & 67.7$_{\pm3.9}$ & 42.6$_{\pm4.0}$ & 50.9$_{\pm5.1}$ & 53.7\\
\small VQ-GAN & 512 & 4  & 68.9$_{\pm7.0}$ & 30.6$_{\pm12.5}$ & 57.4$_{\pm5.0}$ & 52.3 \\
\small 3D MedVAE & 512 & 1  & 72.0$_{\pm3.8}$ & \textbf{49.1$_{\pm19.8}$} & 58.2$_{\pm1.7}$ & \textbf{59.8} \\
\bottomrule
\end{tabular}
}
\caption{\textbf{Evaluating latent representation quality with 3D CAD tasks.} We evaluate the 3D MedVAE autoencoders on three 3D CAD tasks, and we report the mean AUROC and standard deviation across three random seeds. We compare MedVAE with one interpolation method (bicubic) and two natural image 2D autoencoders (KL-VAE and VQ-GAN). For 2D baselines, we stitch 2D latent representations together across slices such that the size of the 2D latent representation matches those generated by 3D models. Here, $f$ represents the downsizing factor applied to the 3D volume of the input image and $C$ represents the number of latent channels. The best performing models on each task are bolded. We highlight methods that perfectly preserve clinically-relevant features (i.e. performance equals or exceeds performance when training with high-resolution volumes) in \textcolor{blue}{\textbf{blue}}.}
\label{table:3D_latent_cls}
\vspace{-1mm}
\end{table*}





We first evaluate whether clinically-relevant features are preserved in MedVAE latent representations. To this end, we measure the extent to which latent representations can serve as drop-in replacements for high-resolution input images in CAD pipelines \textit{without} any customization or modifications to CAD model architectures. 

We evaluate latent representation quality using the following 8 CAD tasks: malignancy detection on 2D FFDMs~\cite{cai2023online}, calcification detection on 2D FFDMs~\cite{cai2023online}, BI-RADS prediction on 2D FFDMs~\cite{nguyen2022vindrmammo}, bone age prediction on 2D X-rays~\cite{rsnaboneage}, fracture detection on 2D wrist X-rays~\cite{Nagy2022wristfrac}, fracture detection on 3D spine CTs~\cite{loffler2020vertebral}, fracture classification on 3D head CTs~\cite{chilamkurthy2018development}, and anterior cruciate liagment (ACL) and meniscal tear detection on 3D sagittal knee MRIs~\cite{bien2018deep}. In order to perform each of these CAD tasks, a model must rely on fine-grained, clinically-relevant features.% we quantitatively evaluate this by selecting tasks where using naive image downsizing techniques (i.e. interpolation methods) contributes to degraded performance when compared to using high-resolution images. 

For each CAD task, we train a classifier (HRNet~\cite{wang2020hrnet} in 2D settings and SEResNet~\cite{hu2018squeeze} in 3D settings) on a training set consisting of latent representations. We then measure the difference in classification performance between models trained directly on latent representations and models trained using original, high-resolution images; this serves as an indicator of latent representation quality (e.g. a small performance difference indicates that the downsizing approach preserves diagnostic features). We compute AUROC for all binary tasks and macro AUROC for all multi-class tasks. We train each classifier with three random seeds, and we report results as mean AUROC $\pm$ standard deviation.

We compare MedVAE with two categories of image downsizing methods: (1) interpolation methods (nearest, bilinear, and bicubic), which are the de-facto gold standard for medical image downsizing as demonstrated by the quantity of prior work leveraging this approach~\cite{wantlin2023benchmd, Varma2019, convirt, Huang_2021_ICCV, miura2021improving, Tiu2022}, and (2) recently-introduced large-scale natural image autoencoders (KL-VAE and VQ-GAN)~\cite{rombach2022high}. Due to the fact that prior work on developing large-scale 3D autoencoders has been limited, we compare our 3D MedVAE models with 2D methods by stitching 2D latent representations together across slices such that the size of the 2D latent representation matches those generated by 3D models. 

We provide results for 2D CAD tasks in Table \ref{table:image_classification} and 3D CAD tasks in Table \ref{table:3D_latent_cls}. Our results demonstrate that the MedVAE training approach yields high-quality latent representations for both 2D and 3D images. At a downsizing factor of $f=16$, 2D MedVAE perfectly preserves clinically-relevant features on four out of five 2D classification tasks. Similarly, at a downsizing factor of $f=64$, 3D MedVAE perfectly preserves relevant clinical information on two out of three 3D classification tasks (spine and skull CT fracture detection). In these cases, performance equals or exceeds performance when training with original, high-resolution images. The average performance of 2D MedVAE with $f=16$ and 3D MedVAE with $f=64$ across all tasks also exceeds the average performance when training with high-resolution images.  

We observe that MedVAE consistently outperforms the natural image autoencoders KL-VAE and VQ-GAN on all classification tasks, demonstrating the utility of the MedVAE training procedure. On average across the five 2D classification tasks, 2D MedVAE demonstrates a 10.0\% improvement over KL-VAE at a downsizing factor of $f=16$ and a 8.0\% improvement at a downsizing factor of $f=64$. Similar trends are noted for VQ-GAN. %, with MedVAE demonstrating a 7.6\% improvement at $f=16$ and a 5.2\% improvement at $f=64$.
In particular, 2D MedVAE outperforms KL-VAE and VQ-GAN on the two musculoskeletal tasks (bone age prediction and wrist fracture detection) despite the fact that no musculoskeletal radiographs are used during MedVAE training; this suggests effective generalization to other types of medical images. On average across the three 3D tasks, 3D MedVAE demonstrates a 37.9\% improvement over KL-VAE at a downsizing factor of $f=64$ and a 11.4\% improvement over KL-VAE at a downsizing factor of $f=512$. Our findings suggest that 3D training of autoencoders leads to high-quality latent representations due to preservation of volumetric information (e.g. fractures spanning multiple slices), particularly at $f=64$. These findings are corroborated by results in Appendix Table~\ref{table:2d3dcad}, which compares performance of 2D MedVAE and 3D MedVAE on 3D CAD tasks.

Additionally, we note that MedVAE outperforms the interpolation methods across most tasks, but interpolation methods are a competitive baseline. Overall, our findings suggest that our MedVAE training procedure yields downsized latent representations that can be used as drop-in replacements for high-resolution input images in CAD pipelines. 

\subsection{Storage and efficiency benefits of latent representations}

Next, we evaluate the extent to which downsized MedVAE latent representations can reduce storage requirements and improve downstream efficiency of CAD pipelines when compared to high-resolution input images. Using a 2D high-resolution network (HRNet\_w64) and 3D squeeze-excitation network (SEResNet-152) as our base CAD architectures, we report latency, throughput, and maximum batch size. Latency is the time (in milliseconds) to perform a forward pass of the network on one batch. Throughput is the number of samples that can be evaluated by the network in one second. Finally, we report the maximum batch size (in powers of 2) for a forward pass that will fit on a single A100 GPU (2D) and an A6000 GPU (3D). We assume a high-resolution input image size of $1024 \times 1024$ with 1 channel for 2D settings and an input volume size of $256 \times 256 \times 256$ with 1 channel for 3D settings.

% ******* Figure ********
\begin{figure}[ht]
\centering
\includegraphics[width=\textwidth, trim=0 0 0 0]{figures/efficiency.pdf}
\caption{\textbf{CAD model efficiency.} Here, we compare the efficiency of CAD models trained with downsized latent representations to CAD models trained with high-resolution images. $f$ represents the downsizing factor applied to the 2D area or 3D volume of the input image. We report latency (in milliseconds), throughput (in samples per second), and the maximum batch size (in powers of 2) that will fit on one GPU.}
\label{fig:efficiency}
\end{figure}

Results are provided in Figure \ref{fig:efficiency}. We demonstrate that training CAD models directly on downsized latent representations can lead to large improvements in model efficiency. In the 2D setting, we observe that as the downsizing factor increases to $f=64$, the latency decreases by 69x, the throughput increases by 70x, and the maximum batch size increases by 32x. In the 3D setting, as the downsizing factor increases to $f=512$, the latency decreases by 62x, the throughput increases by 55x, and the maximum batch size increases by 512x. Storage costs decrease proportionally with the downsizing factor (i.e. 64x for 2D and 512x for 3D).

\subsection{Reconstructed image quality}

We evaluate whether clinically-relevant features are preserved in reconstructed images using both automated and manual perceptual quality evaluations. These evaluations quantify the extent to which the encoding and subsequent decoding processes retain relevant features.

For automated evaluations, we use perceptual metrics to compare reconstructed images with the original inputs. We report peak signal-to-noise ratio (PSNR) and the multi-scale structural similarity index measure (MS-SSIM). For 2D evaluations, we measure perceptual quality on X-rays~\cite{feng2021candid,johnson2019mimic}; FFDMs~\cite{jeong2022emory,sorkhei2021csaw,rsnamammo,nguyen2022vindrmammo,moreira2012inbreast,cai2023online}; and musculoskeletal X-rays~\cite{Nagy2022wristfrac}. In addition to full-image evaluations, we additionally include a fine-grained perceptual quality assessment, where we extract regions containing wrist fractures by using bounding boxes~\cite{Nagy2022wristfrac}; then, the original region and reconstructed region are compared using perceptual metrics. For 3D evaluations, we compute metrics on brain MRIs~\cite{jack2008alzheimer,dagley2017harvard,insel2020a4,lamontagne2019oasis}; head CTs~\cite{chilamkurthy2018development}; abdomen CTs~\cite{ji2022amos}; CTs from a wide range of anatomies~\cite{wasserthal2023totalsegmentator}; lung CTs~\cite{armato2011lung}; and knee MRIs~\cite{bien2018deep}.


\begin{table*}[t]
% \scriptsize
\centering
\resizebox{\linewidth}{!}{
\begin{tabular}{lccccccccc}
\toprule
\textbf{Method}
& \textbf{$f$}
& \textbf{$C$}
& \multicolumn{2}{c}{\textbf{Mammograms}}
& \multicolumn{2}{c}{\textbf{Chest X-rays}}
& \multicolumn{2}{c}{\textbf{Musculoskeletal X-rays}}
& \multicolumn{1}{c}{\textbf{Wrist X-rays (FG)}}
\\
\cmidrule(l{3pt}r{0pt}){4-5}
\cmidrule(l{3pt}r{0pt}){6-7}
\cmidrule(l{3pt}r{0pt}){8-9}
\cmidrule(l{3pt}r{0pt}){10-10}
% "
& 
&
& \small PSNR $\uparrow$
& \small MS-SSIM $\uparrow$
& \small PSNR $\uparrow$
& \small MS-SSIM $\uparrow$
& \small PSNR $\uparrow$
& \small MS-SSIM $\uparrow$
& \small PSNR $\uparrow$
\\ 
\midrule
\small Nearest & 16 & 1    &  25.95$_{\pm0.06}$  & 0.846$_{\pm0.00}$  & 29.87$_{\pm0.04}$ & 0.942$_{\pm0.00}$   &  24.06$_{\pm0.02}$  & 0.890$_{\pm0.00}$  & 26.11$_{\pm0.02}$\\
\small Bilinear & 16 & 1    & 30.18$_{\pm0.07}$  & 0.936$_{\pm0.00}$ & 34.23$_{\pm0.03}$ & 0.981$_{\pm0.00}$ & 28.75$_{\pm0.02}$  & 0.959$_{\pm0.00}$ & 30.92$_{\pm0.03}$ \\
\small Bicubic & 16 & 1    & 31.69$_{\pm0.07}$  & 0.961$_{\pm0.00}$ & 35.48$_{\pm0.03}$ & 0.989$_{\pm0.00}$ & 30.18$_{\pm0.02}$  & 0.974$_{\pm0.00}$ & 32.65$_{\pm0.04}$  \\
\small KL-VAE & 16 & 3  & 36.11$_{\pm0.07}$  & 0.989$_{\pm0.00}$ & 41.45$_{\pm0.04}$ & 0.996$_{\pm0.00}$ & 38.29$_{\pm0.03}$  &0.992$_{\pm0.00}$  & 36.55$_{\pm0.03}$\\
\small VQ-GAN & 16 & 3  &   35.55$_{\pm0.07}$  & 0.986$_{\pm0.00}$ & 37.80$_{\pm0.03}$ & 0.995$_{\pm0.00}$ & 36.41$_{\pm0.02}$  & 0.990$_{\pm0.00}$  & 34.19$_{\pm0.04}$\\
\small 2D MedVAE  & 16 & 1  & 32.34$_{\pm0.07}$ &  0.969$_{\pm0.00}$ & 38.44$_{\pm0.02}$ & 0.990$_{\pm0.00}$ & 33.97$_{\pm0.03}$ & 0.973$_{\pm0.00}$  &  31.97$_{\pm0.03}$ \\
\small 2D MedVAE  & 16 & 3  & \textbf{37.57}$_{\pm0.08}$ &  \textbf{0.993}$_{\pm0.00}$ & \textbf{43.55 }$_{\pm0.02}$& \textbf{0.997}$_{\pm0.00}$ & \textbf{39.41}$_{\pm0.04}$ & \textbf{0.994}$_{\pm0.00}$  &  \textbf{37.61}$_{\pm0.02}$  \\
\midrule

\small Nearest & 64 & 1   & 22.46$_{\pm0.05}$  & 0.669$_{\pm0.00}$ & 26.22$_{\pm0.03}$ & 0.858$_{\pm0.00}$ & 19.93$_{\pm0.02}$  & 0.756$_{\pm0.00}$  & 22.14$_{\pm0.04}$  \\
\small Bilinear & 64 & 1    & 26.81$_{\pm0.06}$  & 0.837$_{\pm0.00}$ & 31.18$_{\pm0.03}$ & 0.949$_{\pm0.00}$ & 24.89$_{\pm0.01}$  & 0.898$_{\pm0.00}$ & 27.12$_{\pm0.03}$  \\
\small Bicubic & 64 & 1    & 27.84$_{\pm0.06}$  & 0.874$_{\pm0.00}$ & 32.09$_{\pm0.03}$ & 0.962$_{\pm0.00}$  & 25.92$_{\pm0.01}$  & 0.922$_{\pm0.00}$  & 28.54$_{\pm0.03}$ \\
\small KL-VAE & 64 & 4    & 31.88$_{\pm0.07}$  & 0.959$_{\pm0.00}$ &36.37$_{\pm0.01}$ &0.987$_{\pm0.00}$ &33.49$_{\pm0.02}$ &0.966$_{\pm0.00}$ & 31.04$_{\pm0.03}$ \\
\small VQ-GAN & 64 & 4   & 30.13$_{\pm0.06}$  & 0.938$_{\pm0.00}$ & 34.87$_{\pm0.02}$ & 0.980$_{\pm0.00}$ & 32.00$_{\pm0.02}$  & 0.953$_{\pm0.0}$  & 29.92$_{\pm0.02}$ \\
\small 2D MedVAE  & 64 & 1   & 28.00$_{\pm0.07}$ & 0.872$_{\pm0.00}$ & 31.92$_{\pm0.04}$ &  0.962$_{\pm0.00}$ & 28.27$_{\pm0.02}$&  0.917$_{\pm0.00}$  & 28.03$_{\pm0.01}$\\
\small 2D MedVAE  & 64 & 4  & \textbf{33.13}$_{\pm0.07}$  & \textbf{0.969}$_{\pm0.00}$ & \textbf{38.88}$_{\pm0.03}$ &  \textbf{0.990}$_{\pm0.00}$ & \textbf{34.73}$_{\pm0.02}$&  \textbf{0.972}$_{\pm0.00}$& \textbf{32.30}$_{\pm0.02}$\\
\bottomrule
\end{tabular}
}
\caption{\textit{Evaluating reconstruction quality on 2D datasets.} We evaluate 2D MedVAE with perceptual quality metrics on mammograms and chest X-rays, which we classify as \textit{in-distribution}, since the MedVAE training set includes mammograms and chest X-rays. We also evaluate MedVAE on musculoskeletal X-rays and wrist X-rays (fine-grained), which we classify as \textit{out-of-distribution}. Here, $f$ represents the downsizing factor applied to the 2D area of the input image and $C$ represents the number of latent channels. The best performing models are bolded. We calculate PSNR and MS-SSIM using a random sample of 1000 images for each image type; we report mean and standard deviations across four runs with different random seeds.}
\label{table:perceptualid}
\end{table*}




\begin{table*}[t]
% \scriptsize
\centering
\resizebox{\linewidth}{!}{
\begin{tabular}{lcccccccccccccc}
\toprule
\textbf{Method}
& \textbf{$f$}
& \textbf{$C$}
& \multicolumn{2}{c}{\textbf{Brain MRIs}}
& \multicolumn{2}{c}{\textbf{Head CTs}}
& \multicolumn{2}{c}{\textbf{Abdomen CTs}}
& \multicolumn{2}{c}{\textbf{TS CTs}}
& \multicolumn{2}{c}{\textbf{Lung CTs}}
& \multicolumn{2}{c}{\textbf{Knee MRIs}}
\\
\cmidrule(l{3pt}r{0pt}){4-5}
\cmidrule(l{3pt}r{0pt}){6-7}
\cmidrule(l{3pt}r{0pt}){8-9}
\cmidrule(l{3pt}r{0pt}){10-11}
\cmidrule(l{3pt}r{0pt}){12-13}
\cmidrule(l{3pt}r{0pt}){14-15}
% "
& 
&
& \small PSNR $\uparrow$
& \small MS-SSIM $\uparrow$
& \small PSNR $\uparrow$
& \small MS-SSIM $\uparrow$
& \small PSNR $\uparrow$
& \small MS-SSIM $\uparrow$
& \small PSNR $\uparrow$
& \small MS-SSIM $\uparrow$
& \small PSNR $\uparrow$
& \small MS-SSIM $\uparrow$
& \small PSNR $\uparrow$
& \small MS-SSIM $\uparrow$
\\ 
\midrule
\small Bicubic & 16 & 1 & 29.27 & 0.975 & 36.21 & 0.996 & 33.81 & 0.989 & 27.33 & 0.972 & 28.00 & 0.973 & 26.37 & 0.986 \\
\small KL-VAE & 16 & 3 & 33.23 & {\textbf{0.994}} & 47.65 & {\textbf{1.000}} & 43.51 & 0.998 & 34.14 & 0.994 & 32.62 & {\textbf{0.989}} & 31.31 & {\textbf{0.998}} \\
\small VQ-GAN & 16 & 3 & 32.72 & 0.992 & 42.87 & 0.999 & 40.85 & 0.997 & 33.55 & 0.993 & 32.20 & {\textbf{0.989}} & 30.75 & 0.997 \\
\small 2D MedVAE  & 16 & 1 & 29.48 & 0.980 & 39.71 & 0.997 & 33.45 & 0.983 & 29.70 & 0.983 & 28.40 & 0.973 & 27.38 & 0.990 \\
\small 2D MedVAE  & 16 & 3 & {\textbf{33.99}} & {\textbf{0.994}} & {\textbf{48.56}} & {\textbf{1.000}} & {\textbf{44.95}} & {\textbf{0.999}} & {\textbf{34.83}} & {\textbf{0.995}} & {\textbf{33.34}} & {\textbf{0.989}} & {\textbf{31.52}} & 0.997 \\
\small 3D MedVAE  & 64 & 1 & 29.52 & 0.983 & 39.03 & 0.999 & 36.61 & 0.993 & 31.35 & 0.987 & 28.79 & 0.975 & 28.25 & 0.994 \\
\midrule
\small Bicubic & 64 & 1 & 26.25 & 0.911 & 30.11 & 0.980 & 28.84 & 0.955  & 24.24 & 0.914 & 24.40 & 0.928 & 24.11 & 0.956  \\
\small KL-VAE & 64 & 3 & 29.32 & {\textbf{0.977}} & 40.95 & 0.997 & 38.07 & {\textbf{0.995}} & 29.85 & 0.982 & 28.83 & 0.974 & 27.68 & {\textbf{0.993}} \\
\small VQ-GAN & 64 & 3 & 27.43 & 0.967 & 39.02 & 0.997 & 36.25 & 0.991 & 27.47 & 0.972 & 26.66 & 0.964 & 25.95 & 0.990 \\
\small 2D MedVAE  & 64 & 1 & 25.66 & 0.920 & 33.10 & 0.988 & 29.51 & 0.967  & 24.50 & 0.922 & 24.39 & 0.933 & 24.48 & 0.973 \\
\small 2D MedVAE  & 64 & 3 & {\textbf{29.34}} & 0.976 & {\textbf{41.98}} & {\textbf{0.999}} & {\textbf{39.49}} & {\textbf{0.995}} & {\textbf{30.35}} & {\textbf{0.984}} & {\textbf{29.59}} & {\textbf{0.977}} & {\textbf{28.05}} & {\textbf{0.993}} \\
\small 3D MedVAE  & 512 & 1 & 26.23 & 0.937 & 30.85 & 0.991 & 29.47 & 0.960 & 26.34 & 0.949 & 24.76 & 0.934 & 24.36 & 0.977 \\

\bottomrule
\end{tabular}
}
\caption{\textit{Evaluating reconstruction quality on 3D datasets.} We evaluate 3D MedVAE with perceptual quality metrics on head MRIs, head CTs, abdomen CTs, various high-resolution CTs (TS), lung CTs, and knee MRIs. $f$ represents the downsizing factor applied to the input volume and $C$ represents the number of latent channels. The best performing models are bolded. We compare 3D MedVAE with several 2D methods, including 2D MedVAE, KL-VAE, and VQ-GAN.}
\label{table:3dperceptual}
\end{table*}


In Table \ref{table:perceptualid}, we compare 2D MedVAE with interpolation methods and large-scale natural image autoencoders across four types of 2D images. We find that 2D MedVAE achieves the highest perceptual quality across all evaluated image types. In particular, our evaluations with wrist X-rays explore generalization of MedVAE to unseen anatomical features; notably, MedVAE achieves the highest PSNR scores on this task, despite the fact that MedVAE was not trained on musculoskeletal X-rays. We also note a general trend that increasing the number of latent channels $C$ improves perceptual quality of the reconstructed image. 

In Table \ref{table:3dperceptual}, we compare MedVAE with interpolation methods and large-scale natural image autoencoders across six types of 3D volumes. Due to the absence of existing large-scale 3D autoencoder baselines, we compare our 3D MedVAE models with 2D methods by performing downsizing on individual 2D slices and then stitching slices together to form the reconstructed 3D volume. We again find that MedVAE reconstructions demonstrate superior perceptual quality when compared to baselines. In particular, 2D MedVAE achieves the highest perceptual quality across almost all evaluated image types, despite the fact that no MRI or CT slices were included in the 2D MedVAE training set. We also observe that 3D MedVAE achieves competitive performance, despite utilizing a significantly higher downsizing factor than comparable 2D methods (i.e. downsizing across all three dimensions rather than just two). In Appendix Table~\ref{table:decoder}, we compare 3D MedVAE with a model referred to as 2D MedVAE-Decoder, which has a comparable downsizing factor $f$. The 2D MedVAE-Decoder model performs downsizing on individual 2D slices, which are then stitched and interpolated together to form a latent representation of equivalent size to the 3D MedVAE model; we then perform fine-tuning of the decoder using our curated dataset of 3D volumes. The superiority of 3D MedVAE to the 2D MedVAE-Decoder approach demonstrates the utility of 3D training of autoencoders, which enables the model to capture important volumetric patterns. 

% ******* Figure ********
\begin{figure}[h]
\centering
\includegraphics[width=0.9\textwidth, trim=0 0 0 0]{figures/readerstudy.pdf}
\caption{\textbf{Manual perceptual quality evaluations with expert readers.} We report the mean scores from three expert readers on three criteria: fidelity, preservation of relevant features, and artifacts. We compare 2D MedVAE with ($f=16,C=3$) and ($f=64,C=4$) with bicubic interpolation, a standard and widely-used approach for downsizing medical images. Error bars represent 95\% confidence intervals.}
\label{fig:readerstudy}
\end{figure}


Qualitative reader studies by domain experts are critical for ensuring clinical usability of developed methods. We supplement our automated evaluations of reconstructed image quality with a manual reader study. Each reader is presented with a pair of chest X-rays, consisting of an original high-resolution image on the left and a reconstructed image on the right. A total of 50 unique chest X-rays with fractures, randomly sampled from CANDID-PTX, are selected and presented in a randomized order~\cite{feng2021candid}. The reconstructed images are scored on a 5-point Likert scale ranging from -2 to 2 based on three main criteria: image fidelity, preservation of diagnostic features, and the presence of artifacts. Our study involved three radiologists as expert readers. We compared 2D MedVAE with bicubic interpolation, a standard and widely-used approach for downsizing medical images (Figure~\ref{fig:readerstudy}).

For manual evaluations of reconstructed image quality, readers rated image fidelity for 2D MedVAE to be 2.8 points higher than bicubic interpolation averaged across the two downsizing factors. 2D MedVAE also better preserved clinically-relevant features (2.8 points). Artifacts (e.g. blurring, hallucinations) were more frequent in interpolated images (2.6 points), which severely suffered from blurring artifacts with increasing downsizing factors. In summary, our results suggest that 2D MedVAE better preserves diagnostic features than interpolation. In Figure \ref{fig:qualitative}, we provide qualitative examples of a reconstructed chest X-ray and a reconstructed T1-weighted brain MRI slice. 


% ******* Figure ********
\begin{figure}[h]
\centering
\includegraphics[width=0.9\textwidth, trim=0 0 0 0]{figures/qualitative.pdf}
\caption{\textbf{Qualitative examples of reconstructed medical images.} The top section provides qualitative examples of a reconstructed chest X-ray. The bottom section provides qualitative examples of a reconstructed brain MRI slice. Residual figures show pixel-level differences between reconstructed images and original, high-resolution images; brighter colors represent larger differences.}
\label{fig:qualitative}
\end{figure}
\clearpage
\section{Discussion}

High-resolution medical images can result in large data storage costs and increased or intractable computational complexity for trained models. As the volume of data stored by hospitals continues to increase and large-scale foundation models become more commonplace, methods for inexpensively storing and efficiently processing high-resolution medical images become a critical necessity. In this work, we aim to address this need by introducing MedVAE, a family of 6 large-scale autoencoders for medical images developed using a novel two-stage training procedure. MedVAE encodes high-resolution medical images as downsized latent representations. We demonstrate with extensive evaluations that (1) downsized latent representations can effectively replace high-resolution images in CAD pipelines while maintaining or exceeding performance, (2) downsized latent representations reduce storage requirements (up to 512x) and improve downstream efficiency (up to 70x in model throughput) when compared to high-resolution input images, and (3) reconstructed images effectively preserve relevant features necessary for clinical interpretation by radiologists.

Several prior works have introduced powerful autoencoders capable of generating downsized latents for images. In particular, recent work on latent diffusion models has involved the development of several large-scale autoencoders, such as VQ-GANs and VAEs, trained on eight million natural images~\cite{rombach2022high,kingma2013vae,esser2021taming,openimages}; downsized latents generated by these models were shown to capture relevant spatial structure as well as improve efficiency of downstream diffusion model training~\cite{rombach2022high}. However, recent works have demonstrated that models trained on natural images often generalize poorly to medical images due to significant distribution shift~\cite{guan2022,van2023exploring,chambon2022adapting}, suggesting that existing natural image autoencoders may not be well-suited for the complexity of the medical image domain. Our evaluations on both latent representations and reconstructed images support this point, demonstrating that existing large-scale natural image autoencoders consistently underperform our domain-specific medical image autoencoders. These findings demonstrate the need for domain-specific models capable of understanding complex and fine-grained patterns across diverse imaging modalities and anatomical regions.

Our work aims to reduce computational costs associated with automated medical image interpretation by proposing the use of training datasets comprised of downsized MedVAE latent representations rather than high-resolution medical images. For instance, given a chest X-ray training dataset with images of size $1024 \times 1024$ with 1 channel, our 2D MedVAE model with $f=64$ and $C=1$ can generate downsized latent representations of size $128 \times 128$ with 1 channel, contributing to substantial downstream efficiency and storage benefits. We demonstrate with eight CAD tasks that latent representations do not result in the loss of clinically-important information; at a 2D downsizing factor of $f=16$ and a 3D downsizing factor of $f=64$, we observe equivalent or better performance than high-resolution images with substantial improvements over multiple existing downsizing methods. MedVAE models can also generalize beyond the images included in the training set, as shown by performance on 2D musculoskeletal X-rays and 3D spine CTs. Importantly, the efficiency benefits of using latent representations are significant; in particular, using latent representations can contribute to large increases in batch sizes, which can be particularly useful in the modern era of self-supervised foundation models that rely heavily on the use of large batch sizes during training. 

The MedVAE autoencoder family includes two 3D autoencoders that are explicitly designed to downsize 3D medical imaging modalities (e.g. CT, MRI), a previously underresearched setting. Our results demonstrate that at a 3D downsizing factor of $f=64$, the volumetric latent representations generated by 3D MedVAE are substantially higher quality than those generated by stitching together 2D slices downsized using 2D baselines. This suggests that 3D autoencoders can better capture clinically-important volumetric patterns, such as fractures that span multiple slices. Efficiency benefits in the 3D setting are also notable, particularly since training downstream CAD models on high-resolution 3D volumes is often computationally expensive or intractable. At significantly higher downsizing factors ($f=512$), we observe the benefits of 3D autoencoder training to be less pronounced, suggesting that users will need to carefully consider the tradeoffs between latent representation quality and desired downstream efficiency when selecting a MedVAE model.

In addition to generating high-quality latent representations, MedVAE models also include a trained decoder, which can reconstruct the original high-resolution image from the downsized latent. This is a particularly useful capability in the medical imaging domain, since high-resolution images are necessary for effective clinical interpretation by radiologists. We demonstrate with a reader study consisting of three radiologists that reconstructed images can effectively preserve clinically-relevant signal needed for diagnoses; in this setting, fine-grained fractures in chest X-rays were preserved through the encoding and decoding process.

Our study presents several opportunities for future work. First, additional research into model architectures, data augmentation architectures, and training strategies would be useful for building effective downstream CAD models that can learn from latent representations. In addition, the batch size and efficiency benefits afforded by latent representations raise the possibility of training large-scale foundation models using downsized latent representations. Whereas foundation models traditionally require significant computational resources and training time, utilizing downsized latent representations that preserve diagnostic features can greatly accelerate model training, particularly in resource-constrained settings. Future work can explore foundation model performance and scaling laws in this context. Finally, future work can explore additional autoencoder training strategies to better preserve clinically-relevant features at high downsizing factors. 

Overall, our work demonstrates the potential that large-scale, generalizable autoencoders hold in addressing critical storage and efficiency challenges in the medical domain. 

\clearpage
\section{Methodology}
\label{sec:method}
\begin{figure}[!ht]
% \vspace{-1em}
\centering
    \includegraphics[width=0.90\columnwidth]{figures/HIM-framework.pdf}
    % \vspace{-1em}
    \caption{The overall framework of HIM.}
    \label{fig:HIM}
\end{figure}
% \vspace{-2em}
\subsection{Overview}
This work aims to address the IM problem from a new perspective.
We encode potential influence spread trends into hyperbolic representations for the effective selection of highly influential seed users.
Our motivation has two key points.
(1) We aim for a diffusion model agnostic method that solves the IM problem without relying on any assumptions on diffusion parameters.
(2) The influenc trend of users can be efficiently approximated by directly utilizing the properties of learned representations.
To this end, we leverage the benefits of hyperbolic geometry to propose a novel method for IM.

We use the social network and the graph set of influence propagation instances as learning data and apply hyperbolic network embedding to construct user representations.
Instead of explicitly computing users' influence spread, we implicitly estimate their influence spread with the learned representations.
The distance information of the representations can effectively measure the influence spread of seed user nodes.

Specifically, a novel hyperbolic spread learning method HIM is proposed, as is shown in Figure~\ref{fig:HIM}. HIM mainly consists of two modules: (1) \textit{Hyperbolic Influence Representation} aims to learn user representations in the hyperbolic space. (2) \textit{Adaptive Seed Selection} selects target seed users based on learned hyperbolic representations via an adaptive algorithm. 

\subsection{Hyperbolic Influence Representation}
We first encode influence spread features from social influence data to construct user representations in hyperbolic space. The social influence data includes social networks and influence propagation instances, as mentioned in Section~\ref{sec:assume}.
The structural information of the network and the historical spread patterns of propagation instances are crucial for estimating the influence spread of seed users. Both should be effectively integrated into user representations.

% Social connections can be easily obtained from a given social network.
% However, the propagation relations are relatively complicated. Instead of relying on specific diffusion models, we attempt to learn the influence propagation information from the observed data. Given the historical diffusion cascades, we can obtain their propagation instances~\cite{ICDE_feng2018inf2vec}. 
% As mentioned, each instance can be viewed as a directed subgraph $G_D$ of the social network $G$.
% Each instance can be viewed as a directed propagation subgraph of the social network, where an edge $(u \to v)$ denotes that user $u$ influences user $v$. 

The learning process follows a shallow embedding approach.
Both types of data naturally form a graph, enabling effective representation learning on edge sets.
We do not adopt more complex embedding methods as~\cite{KDD2016_grover_node2vec, KDD2017_ribeiro_struc2vec} as we aim to intuitively demonstrate that influence spread can be estimated based on hyperbolic representations learned from social influence data, which is previously unexplored.
Meanwhile, this approach maintains computational efficiency, making it scalable for large-scale social networks.

Given social influence data, we propose a rotation-based Lorentz model to learn hyperbolic user representations. 
Note that this preprocess is model-agnostic, making it adaptable to various diffusion models and practical applications. 

At first, given a social network $G = (V, E)$, we assign each user $u \in V$ an initial representation $\mathbf{x}_u \in \mathbb{L}^{n}_{\gamma}$, initialized via hyperbolic Gaussian sampling as in work~\cite{sun2021hgcf}.

\subsubsection{Rotation Operation.}
We apply hyperbolic rotation operation~\cite{ICLR19rotate, ACL20_chami2020low} to assist in integrating structure and influence spread information for effective representation learning. 
By adjusting angles, various rotation operations capture different types of information, ensuring seamless integration into unified user representations.

In detail, we use two sets of rotation matrices $(\mathbf{Rot}^{S}_{s}, \mathbf{Rot}^{T}_{s})$ and $(\mathbf{Rot}^{S}_{d}, \mathbf{Rot}^{T}_{d})$ to assist in representation learning. Here, $s$ denotes the social relation, while $d$ denotes the propagation relation. $S$ and $T$ denote the rotation operations applied to head nodes and tail nodes, respectively.
The rotation operation further brings extra benefits for IM.
% Employing rotation transformation offers several benefits to learning influence representations for the IM problem.
% First, rotations can capture various symmetric and asymmetric relations among users~\cite{ICLR19rotate,ACL20_chami2020low}.
The rotation operation in representation learning adjusts vectors' angles to bring related user representations closer while preserving their distances, therefore maintaining hierarchical information.
Besides, It is also efficient and easy to implement.

% -------------------------------------------------------------------------------------
% Learn Static Influence
% -------------------------------------------------------------------------------------

\subsubsection{Network Structure Learning.}

In this part, we deduce structure influence from the social connections present in the social network by modeling the edges within the given graph $G=(V, E)$. 
The core idea is to maximize the joint probability of observing all edges in the graph to learn node embeddings.

Specifically, given an observed edge $(u \rightarrow v) \in E$, the probability $\Pr(v|u)$ can be estimated by a score function based on the squared Lorentzian distance:
% $\small \Pr(v|u) = \frac{ \exp(\mathcal{V}^{S}_{uv}) } { Z(u) }$,
$\small \Pr(v|u) = \exp(\mathcal{V}^{S}_{uv})  /  Z(u) $,
% \begin{equation} \small \Pr(v|u) = \frac{ \exp(\mathcal{V}^{S}_{uv}) } { Z(u) }, \label{eq:prob-u-v} \end{equation}
where $Z(u) = \sum_{ o \in V } \exp(\mathcal{V}^{S}_{uo})$, and
edge score $\mathcal{V}^{S}_{uv} $ is defined as:
\begin{equation} 
\small \mathcal{V}^{S}_{uv} = - w_{uv} \cdot d^2_{\mathcal{L}}\left(\mathbf{x}^S_u, \mathbf{x}^T_v\right) + b_u + b_v,
\label{eq:relation-score}
\end{equation}
where $ w_{uv} > 0 $ is the coefficient associated with the edge $(u \rightarrow v)$.
Generally, we set $w_{uv} = 1/d_{u}$.
$b_u$ and $b_v$ represent biases of node $u$ and node $v$, respectively.
$\mathbf{x}^S_u = \mathbf{Rot}_{s}^S(\mathbf{x}_u)$ and $\mathbf{x}^T_v = \mathbf{Rot}_{s}^T(\mathbf{x}_v)$ are the rotated representations. 
Since the normalization term $Z(u)$ is expensive to compute, we approximate it via a negative sampling strategy~\cite{mikolov2013neg-sampling}.
% Note that the normalization term $Z(u)$ is expensive to compute, we approximate it via a new negative sampling strategy: We first divide all nodes into $L$ ranges according to their degrees. 
% When sampling negative nodes for given users, we carry out sample selection in the corresponding range according to their degrees, making refined distinctions among users with similar degrees. 
Therefore, we estimate $\Pr(v|u)$ in the log form as:
\begin{equation}
\small \log P(v|u) \approx \log \varphi \left(\mathcal{V}_{uv} \right) + \sum_{o \in \mathcal{N}_u} \log \varphi \left( - \mathcal{V}_{uo}\right),
\label{eq:log_p_u_v}
\end{equation}
where $\varphi(x) = 1/(1+e^{-x})$ is the Sigmoid function and $\mathcal{N}_u$ is the set of sampled negative nodes.
Assuming they are independent of each other, the joint probability of all social connections can be calculated as:
\begin{equation} \small \mathcal{P} = \sum_{(u,v)\in E} \log P(v|u). \end{equation}
By maximizing this joint probability, we encode the structure information of the social network into user representations.
% Accordingly, our goal is to capture the static influence of all users by maximizing this joint probability.

% -------------------------------------------------------------------------------------
% Learn Dynamic Influence
% -------------------------------------------------------------------------------------

\subsubsection{Influence Propagation Learning.}
Here, we extract historical influence spread patterns from the propagation instance graph sets $\mathcal{G}_D$.
Similarly, given any propagation graph $G^i_D \in \mathcal{G}_D$, we maximize the joint probability of observing influence activations in the $G^i_D$ to encode spread patterns into user embeddings.

In detail, given $G^i_D = (V^i_D, E^i_D)$, the edge probability of $(u \rightarrow v) \in E^i_D$ can be calculated similar to Eq. (\ref{eq:log_p_u_v}) as:
\begin{equation}
\small \log P(v|u) \approx \log \varphi \left(\mathcal{V}^{D}_{uv}\right) + \sum_{o \in \mathcal{N}_u} \log \varphi \left( - \mathcal{V}^{D}_{uo}\right),
\end{equation}
% \;\:
\begin{equation}
\small \mathcal{V}^{D}_{uv} = - w_{uv} \cdot d^2_{\mathcal{L}}\left(\mathbf{x}^S_u, \mathbf{x}^T_v\right) + b_u + b_v,
\label{eq:propagation_score}
\end{equation}
where $ w_{uv} = 1/d_u $ is the coefficient, $\mathbf{x}^S_u = \mathbf{Rot}_{d}^S(\mathbf{x}_u)$ and $\mathbf{x}^T_v = \mathbf{Rot}_{d}^T(\mathbf{x}_v)$ are the rotated user representations. 
The joint probability of all edges in $G^{i}_D$ can be calculated as:
\begin{equation} \small \mathcal{P}_{G^i_D} = \sum_{(u,v)\in E^{i}_D} \log P(v|u). \end{equation}

During the propagation process, once a user $u$ triggers influence activation, we want to assign a bonus to highlight this user’s tendency to positively influence others. 
Inspired by the approach in~\cite{ICML2023_Yang}, we address this intuitively by reducing the hyperbolic distance of the related user representations from the origin in the embedding space.
Thus, for all influence activations in $G^i_D$, we propose a proactive influence regularization term:
\begin{equation}
 \mathcal{I}_{G^i_D} = \sum_{(u,v)\in G^i_D} \alpha_u \cdot \log \varphi \left(d^2_{\mathcal{L}}(\mathbf{x_u}, \mathbf{o}_{\mathcal{L}})\right).
\end{equation}
where $\mathbf{o}_{\mathcal{L}}$ is the origin of the Lorentz model and $\alpha_u$ is calculated as $\sqrt{d_u/d_{\text{max}}}$.
This term further pulls high-influence users closer to the origin in the representation space.
The illustration of learning an observed influence instance $(u \rightarrow v)$ is shown in Figure~\ref{fig:emb}.
\begin{figure}[h]
  \centering
  \includegraphics[width=0.90\columnwidth]{figures/method/do_emb.pdf}
  \caption{ Illustration of the influence propagation learning. The propagation relation between user $u$ and $v$ is depicted by the distance $d^2_{\mathcal{L}}$ between their rotated embeddings. }
  \label{fig:emb}
\end{figure}

For simplicity, we define $LDO$ as the squared Lorentzian distance from a given representation to the origin. Specifically, for user $u$, the $LDO_u$ is defined as $LDO_u = d^2_{\mathcal{L}}(\mathbf{x}_u, \mathbf{o}_{\mathcal{L}})$.
Previous studies~\cite{nickel2017poincare, ICML2023_Yang, feng2022role} have shown that hierarchical information can be effectively inferred from $LDO$s. In our method, user nodes with smaller $LDO$ values are more likely to be influential in social networks. 
We will later design seed selection strategies based on $LDO$.

% -------------------------------------------------------------------------------------
% Objective Function
% -------------------------------------------------------------------------------------

\subsubsection{Objective Function.}

Combining above two parts, the overall loss function is calculated as:
\begin{equation}
\small
\mathcal{L} = - \left( \mathcal{P} + \sum_{G^i_D \in \mathcal{G}_D}\left(\mathcal{P}_{G^{i}_D} + \mathcal{I}_{G^i_D}\right)\right). 
\label{eq:over_loss}
\end{equation}
Optimizing Eq. (\ref{eq:over_loss}) brings relevant nodes closer together while keeping irrelevant nodes as far apart as possible. Meanwhile, users involved in more influence activations tend to have their representations move closer to the origin, indicating potential higher influence spread.
The time complexity can be found in Appendix.

Once the learning process is complete, users with strong spread relations will be clustered together in the embedding space, and highly influential users tend to be located near the origin, which helps to identify seed users for the IM problem.

\subsection{Adaptive Seed Selection} 
\begin{algorithm}[H]
% \small
\caption{Adaptive Sliding Window (ASW)}\label{alg:ASW}
\begin{algorithmic}[1]
\Statex \textbf{Input:} social graph $G$, user representations $\mathbf{X}$, seed number $k$ and window size coefficient $\beta$ 
\Statex \textbf{Output:} $S^*$ with $k$ seed users
\State $S^* \gets$ an empty set, window size $w \gets \beta \cdot k$
\State $D \gets$ compute $ LDO_u = d^2_{\mathcal{L}}(\mathbf{x}_u, \mathbf{o}_{\mathcal{L}}) \text{ for each } u \text{ in } V$ 
\State $\mathcal{Z} \gets \text{sort } D  \text{ in ascending order} $
\State $c \gets$ select the $u$ with minimum $\mathcal{Z}_u$
\State $Q \gets$ a priority queue initialized with key-value pairs $(u, \mathcal{Z}_u)$ for the next $w$ users in $\mathcal{Z}$.
\While{$|S^*| < k$}
\State add $c$ to $S^*$ and find $N_c$ the neighbors of $c$ from $G$
\State $\mathcal{C} = N_c \cap Q_{keys}$
\If{$ \mathcal{C} = \emptyset $}
\State $c = Q.$pop and add the next $(u, \mathcal{Z}_u)$ in $\mathcal{Z}$ to $Q$ 
\Else
\State compute $\mathcal{Z}'_v$ according to Eq. (\ref{eq:update_score}) 
\State update $Q$ with $(v, \mathcal{Z}'_v)$ for each $v$ in $\mathcal{C}$
\State $c = Q.$pop and add the next $(u, \mathcal{Z}_u)$ in $\mathcal{Z}$ to $Q$ 
\EndIf
\EndWhile \textbf{ and return $S^*$} 
\end{algorithmic}
\end{algorithm}

After integrating social influence information into the hyperbolic representations, the next step involves designing strategies to select target seed users based on these learned representations. Specifically, we propose adaptive seed selection, which aims to leverage the geometric properties of the hyperbolic representations to effectively find seed users who possess large influence spread.

In practice, users with high influence might have overlapping areas of influence. Independently selecting highly influential users may not result in optimal overall performance due to the submodularity of social influence~\cite{kempe2003im}. Additionally, the submodularity property of the IM problem implies diminishing marginal gains from seed users~\cite{TKDE18_li2018influence_survey}, particularly for users who are close to the already selected seed users. Therefore, it is crucial to consider these spread relations among users when selecting seed nodes. Previous methods required traversing all nodes, leading to high computational costs. Given that influence strength can be estimated by the distance of representations from the origin and that the spread relations among users can be measured by the distance between their representation vectors, we have designed a new algorithm for seed set selection, which is shown in Algorithm~\ref{alg:ASW}.

The key idea of our strategy is to assign each user an initial score and dynamically adjust these scores during the selection process. Therefore, we could determine the final seed set by considering the spread relation among users. Specifically, we first assign each user with a score $\mathcal{Z}_u = LDO_u = d^2_{\mathcal{L}}(\mathbf{x}_u, \mathbf{o}_{\mathcal{L}})$. We sort all scores $\mathcal{Z}$ and select the node $c$ with the smallest $\mathcal{Z}_c$ as the first seed user. Instead of directly choosing the node with the second lowest $LDO$, the next $w$ nodes in the sorted list are viewed as candidate nodes, where $w$ is the size of a sliding window $W$. The $W$ is used to explore a wider range of candidate nodes while maintaining computational efficiency. We determine the window size $w$ based on $k$ as $w = \beta \cdot k$, allowing it to adaptively adjust its size for different data scales.
In Algorithm~\ref{alg:ASW}, the sliding window $W$ is implemented by a priority queue $Q$.
Next, we find the intersection $\mathcal{C}$ of the current seed node's neighbors with the candidate nodes.
Accordingly, we update the scores of the nodes in $\mathcal{C}$. 
For a user $u \in \mathcal{C}$, the updated score $\mathcal{Z}'_u$ is calculated as:
\begin{equation}
\small
\mathcal{Z}'_u = \mathcal{Z}_u + \frac{w_{c,u}}{d_c} \cdot  \mathcal{Z}_{c},
\label{eq:update_score}
\end{equation}
\begin{equation}
% \;\;
\small
w_{c,u} = \frac{
\exp(1/d^2_{\mathcal{L}}(\mathbf{x}_c, \mathbf{x}_u))
}{ \sum_{v \in \mathcal{C}} \exp(1/d^2_{\mathcal{L}}(\mathbf{x}_c, \mathbf{x}_v))}.
\label{eq:update_score_2}
\end{equation}
Here, $c$ denotes the recently selected node, $d_c$ is the degree of node $c$, and $w_{c,u}$ means the weight between them. Intuitively, a node closer to node $c$ may have a larger spread overlap with $c$, leading to a larger penalty from node $c$ and thus increasing its score.
In this way, the candidates' scores in the sliding window will be updated. After that, we select the node with the lowest score. At each iteration, the chosen node is removed from the window, and the next node from the sorted $LDO$ list is added to the window. This process is repeated until $k$ seed users are selected. 
Due to space limitations, the time complexity analysis can be found in Appendix.

% \subsubsection{Discussion.} 

% The influence strength of user nodes can be effectively measured by the distance of their representations from the space's origin. Meanwhile, the relationship between two users can be efficiently measured by the distance between their representations. Compared to other methods that utilize graph properties, such as the shortest path, to measure the relationship between two nodes, calculating the distance between representation vectors can greatly enhance computational efficiency. Representations in hyperbolic space can effectively measure both the influence of individual users and the social relations among them, enabling the design of efficient algorithms for classic IM problems. 
% Indeed, how to effectively select seed nodes based on the learned hyperbolic representations remains an open question worth further exploration.

% The influence strength of user nodes can be effectively measured by the distance of their representations from the origin in hyperbolic space. Similarly, the relationship between two users can be efficiently assessed by the distance between their respective representations. Compared to traditional methods that rely on graph properties, such as the shortest path, calculating the distance between representation vectors significantly enhances computational efficiency. Thus, we argue that hyperbolic space representations are particularly well-suited for measuring both individual user influence and social relationships, thereby facilitating the design of efficient algorithms for the IM problem.

% Applying two proposed strategies to HIM, we obtain two specific IM methods: HIM-MD and HIM-ASW.
% Later, in the experimental section, we will evaluate the performance of two methods.

% Section Transition
\clearpage

\subsection*{Acknowledgments}
MV is supported by graduate fellowship awards from the Department of Defense (NDSEG), the Knight-Hennessy Scholars program at Stanford University, and the Quad program. AK is supported by graduate fellowships from Tau Beta Pi and the Knight-Hennessy Scholars program at Stanford University. RS was supported by the Rubicon fellowship of the Dutch National Research Council (NWO). This work was supported in part by NIH grants R01 HL155410, R01 HL157235, by AHRQ grant R18HS026886, and by the Gordon and Betty Moore Foundation. CL is supported by the Medical Imaging and Data Resource Center (MIDRC), which is funded by the National Institute of Biomedical Imaging and Bioengineering (NIBIB) under contract 75N92020C00021 and through The Advanced Research Projects Agency for Health (ARPA-H). AC is supported by NIH grants R01 HL167974, R01HL169345, R01 AR077604, R01 EB002524, R01 AR079431, P41 EB027060, AY2AX000045, and 1AYSAX0000024-01; and NIH contracts 75N92020C00008 and 75N92020C00021. 

\subsection*{Author contributions}
M.V., A.K., and R.S. designed the study, constructed models, and performed technical evaluations. R.S., C.B., and J.P. carried out the reader study. M.V., A.K., R.S., S.O., L.B., and P.C. collected data and analyzed model performance. All authors contributed to technical discussions and the drafting and revision of the manuscript. C.L. and A.C. supervised, funded, and guided the research.

\clearpage
\nolinenumbers
% \setlength\bibitemsep{3pt}
\bibliographystyle{plain} 
\bibliography{main}      
% \end{refsegment}

%%%%%%%%%%%%%%%%%%%%%%%%%%%%%%%%%%%%%%%%%%%%%%%%%%%%%%%%%%%%%%%%%%%%%%%%%%%%%%%
%%%%%%%%%%%%%%%%%%%%%%%%%%%%%%%%%%%%%%%%%%%%%%%%%%%%%%%%%%%%%%%%%%%%%%%%%%%%%%%
% APPENDIX
%%%%%%%%%%%%%%%%%%%%%%%%%%%%%%%%%%%%%%%%%%%%%%%%%%%%%%%%%%%%%%%%%%%%%%%%%%%%%%%
%%%%%%%%%%%%%%%%%%%%%%%%%%%%%%%%%%%%%%%%%%%%%%%%%%%%%%%%%%%%%%%%%%%%%%%%%%%%%%%
\newpage
\appendix
\onecolumn
\section*{Appendix Overview}
\begin{itemize}
    \item Section~\ref{appendix:related}: Related Work.
    \item Section~\ref{appendix:more_dataset}: More Dataset Details.
    \item Section~\ref{appendix:error_analysis}: Error Analysis.
    \item Section~\ref{appendix:more_qualitative}: More Qualitative Examples.
    \item Section~\ref{appendix:eval_setup}: Evaluation Prompts.
\end{itemize}


\section{Related Work}
\label{appendix:related}
\subsection{Large Multimodal Models}
The field of multimodal~\citep{Radford2021LearningTV, li2022blip, openai2023gpt4v, openai2024gpt4o} AI has experienced extraordinary growth, particularly through the development of Large Multimodal Models (LMMs)~\cite{liu2023llava,zhu2023minigpt,lin2023sphinx,Qwen2-VL}. These models build upon the achievements of Large Language Models (LLMs)~\citep{touvron2023llama,qwen2} and advanced vision models~\cite{Radford2021LearningTV}, expanding their capabilities to process multiple kinds of visual input~\cite{li2024llava,guo2023point,li2023videochat}.

Closed-source models, such as OpenAI's GPT-4o~\citep{openai2024gpt4o}, have demonstrated exceptional capabilities in visual understanding and reasoning. However, their closed-source nature creates barriers to widespread adoption and further development by the broader research community. In response, significant progress has been made in developing open-source alternatives. Early approaches like LLaVA~\cite{liu2023llava}, LLaMA-Adapter~\cite{zhang2024llamaadapter}, and MiniGPT-4~\cite{zhu2023minigpt} established a foundation by combining frozen CLIP models for image encoding with LLMs, enabling multimodal instruction tuning. Subsequent developments through projects such as InternVL2~\cite{chen2024far}, Qwen2-VL~\cite{Qwen2-VL}, SPHINX~\cite{gao2024sphinx,lin2023sphinx}, and MiniCPM-V~\cite{yao2024minicpm} have expanded these capabilities by incorporating more diverse visual instruction datasets and broadening application scenarios.

Recently, with the introduction of o1~\cite{o1}, the field of LMMs has also focused on enhancing the reasoning capability. \cite{wang2024enhancing} introduces mixed preference optimization with automatically constructed data. \cite{yao2024mulberry} proposes to leverage collective knowledge from multiple models to identify effective reasoning paths. Besides, several works~\cite{qvq-72b-preview,du2025virgo} have demonstrated the ability to replicate behaviors similar to o1 models, particularly regarding multi-step CoT reasoning with iterative self-reflection and verification processes.

\subsection{Reasoning Evaluation}
Several methods have been developed to evaluate reasoning in natural language processing, including ROSCOE~\cite{golovneva2022roscoe} and ReCEval~\cite{prasad2023receval}, which assess reasoning chains across multiple dimensions such as correctness and informativeness. However, these approaches are limited to text-only scenarios and do not address the unique challenges present in visual reasoning tasks. Furthermore, the emergence of long chain-of-thought (CoT) reasoning has introduced additional considerations, such as output efficiency and reflection quality, which existing evaluation methods do not adequately address.

On the other hand, various multimodal benchmarks have been developed to assess reasoning abilities across specific domains. Current exploration of visual reasoning predominantly focuses on the mathematics~\cite{zhang2024mavis,peng2024chimera} domains. 
MathVista~\cite{Lu2023MathVistaEM} provides a comprehensive collection of mathematical problems that assess mathematical and logical reasoning abilities. 
Building on this, MathVerse~\cite{zhang2024mathverse} introduces a new benchmark by eliminating redundant textual information to evaluate whether LMMs can accurately interpret graphical representations. 
OlympiadBench~\cite{he2024olympiadbench} further raises the complexity bar by incorporating challenging Olympiad-level mathematics and physics problems. Despite these advances in specialized domains, broader applications such as general-scene reasoning remain relatively unexplored.
Recent developments have begun to expand beyond purely scientific reasoning. For instance, M³CoT~\cite{chen-etal-2024-m3cot} and SciVerse~\cite{sciverse} incorporate commonsense tasks alongside scientific reasoning and knowledge-based assessment in the multimodal benchmark. However, most existing benchmarks focus solely on evaluating final answers while overlooking the intermediate steps, thus providing limited insights into the process through which models arrive at their conclusions.


\section{More Dataset Details}
\label{appendix:more_dataset}
\subsection{Data Source Distribution}
We visualize the data source distributions in our benchmark, which consists of 15 sets, including MathVerse~\cite{zhang2024mathverse}, MMMUPro~\cite{yue2024mmmuprorobustmultidisciplinemultimodal}, OlympiadBench~\cite{he2024olympiadbench}, MMT-Bench~\cite{ying2024mmt}, MuirBench~\cite{wang2024muirbench}, ml-rpm-bench~\cite{zhang2024far}, MMSearch~\cite{jiang2024mmsearch}, CharXiv~\cite{wang2024charxiv}, and SciVerse~\cite{sciverse}.

\begin{figure*}[!h]
\centering
\includegraphics[width=0.4\textwidth]{fig/pie_supp.pdf} 
\caption{\textbf{Data Source Distribution of MME-CoT.}}
\label{appendix:more_dataset-source}
\end{figure*}

\newpage

\subsection{Preliminary Categorization Result}
\label{appendix:preliminary_result}
\begin{table}[htbp]
    \centering
    \caption{\textbf{Accuracy of MMT-Bench for different subcategories}. ACT: Action Understanding; AUT: Attribute Similarity; CNT: Cartoon Understanding; CIM: Counting; DOC: Diagram Understanding; EMO: Difference Spotting; HAL: Geographic Understanding; IIT: Image-Text Matching; IRT: Ordering; IQT: Scene Understanding; MEM: Visual Grounding; MIA: Visual Retrieval; OCR: Object Recognition; PLP: Physical Layout Prediction; RRE: Relationship Extraction; TMP: Temporal Reasoning; VCP: Visual Comprehension; VCR: Visual Coherence Reasoning; VGR: Visual Generation; VIL: Visual Identification; VPU: Visual Prediction Understanding; VRE: Visual Reasoning Evaluation.}
    \label{tab:hit_ratio}
    \setlength{\tabcolsep}{4pt} 
    \renewcommand{\arraystretch}{1.2}
    \small 
    \begin{tabularx}{\textwidth}{l *{22}{X}}
        \toprule
        File Name & 
        \rotatebox{90}{ACT} & \rotatebox{90}{AUT} & \rotatebox{90}{CNT} & \rotatebox{90}{CIM} & 
        \rotatebox{90}{DOC} & \rotatebox{90}{EMO} & \rotatebox{90}{HAL} & \rotatebox{90}{IIT} & 
        \rotatebox{90}{IRT} & \rotatebox{90}{IQT} & \rotatebox{90}{MEM} & \rotatebox{90}{MIA} & 
        \rotatebox{90}{OCR} & \rotatebox{90}{PLP} & \rotatebox{90}{RRE} & \rotatebox{90}{TMP} & 
        \rotatebox{90}{VCP} & \rotatebox{90}{VCR} & \rotatebox{90}{VGR} & \rotatebox{90}{VIL} & 
        \rotatebox{90}{VPU} & \rotatebox{90}{VRE} \\
        \midrule
        GPT4o-cot & 0.60 & 0.60 & 0.44 & 0.67 & 0.79 & 0.30 & 0.71 & 0.50 & 0.63 & 0.10 & 0.85 & 0.60 & 0.77 & 0.36 & 0.76 & 0.48 & 0.86 & 0.80 & 0.49 & 0.48 & 0.82 & 0.85 \\
        GPT4-direct & 0.53 & 0.60 & 0.44 & 0.67 & 0.81 & 0.23 & 0.69 & 0.33 & 0.66 & 0.25 & 0.80 & 0.43 & 0.78 & 0.42 & 0.78 & 0.36 & 0.89 & 0.85 & 0.41 & 0.37 & 0.85 & 0.85 \\
        Qwen2-VL-7B-cot & 0.53 & 0.61 & 0.34 & 0.65 & 0.77 & 0.53 & 0.74 & 0.40 & 0.31 & 0.20 & 0.78 & 0.58 & 0.60 & 0.43 & 0.69 & 0.43 & 0.85 & 0.90 & 0.54 & 0.35 & 0.79 & 0.81 \\
        Qwen2-VL-7B-direct & 0.49 & 0.67 & 0.40 & 0.78 & 0.75 & 0.52 & 0.73 & 0.43 & 0.31 & 0.10 & 0.78 & 0.55 & 0.60 & 0.54 & 0.69 & 0.40 & 0.85 & 0.85 & 0.67 & 0.38 & 0.85 & 0.82 \\
        \bottomrule
    \end{tabularx}
\end{table}


\begin{table}[htbp]
    \centering
    \caption{\textbf{Accuracy of MUIRBench for different subcategories}. AU: Action Understanding; AS: Attribute Similarity; CU: Cartoon Understanding; CO: Counting; DU: Diagram Understanding; DS: Difference Spotting; GU: Geographic Understanding; ITM: Image-Text Matching; OR: Ordering; SU: Scene Understanding; VG: Visual Grounding; VR: Visual Retrieval.}

    \label{tab:hit_ratio}
    \setlength{\tabcolsep}{4pt} 
    \renewcommand{\arraystretch}{1.2} 
    \small 
    \begin{tabularx}{\textwidth}{l XXXX XXXX XXXX XXXX}
        \toprule
        File Name & AU & AS & CU & CO & DU & DS & GU & ITM & OR & SU & VG & VR \\
        \midrule
        GPT4o-cot & 0.48 & 0.57 & 0.55 & 0.75 & 0.82 & 0.64 & 0.59 & 0.82 & 0.38 & 0.88 & 0.56 & 0.70 \\
        GPT4o-direct & 0.45 & 0.62 & 0.59 & 0.50 & 0.88 & 0.62 & 0.55 & 0.86 & 0.33 & 0.74 & 0.38 & 0.77 \\
        Qwen2-VL-7B-cot & 0.38 & 0.51 & 0.42 & 0.43 & 0.43 & 0.27 & 0.21 & 0.55 & 0.13 & 0.69 & 0.37 & 0.28 \\
        Qwen2-VL-7B-direct & 0.39 & 0.47 & 0.44 & 0.41 & 0.40 & 0.33 & 0.25 & 0.51 & 0.13 & 0.67 & 0.31 & 0.20 \\
        \bottomrule
    \end{tabularx}
\end{table}



\begin{table}[htbp]
    \centering
    \caption{\textbf{Accuracy of OlympiadBench for the mathematics and physics subcategories}.}
    \label{tab:hit_ratio_oe}
    \small 
    \begin{tabular}{lcc}
        \toprule
        File Name & Mathematics & Physics\\
        \midrule
        GPT4o-cot & 0.25 & 0.04 \\
        GPT4o-direct & 0.07 & 0.03 \\
        Qwen2-VL-7B-cot & 0.05 & 0.01 \\
        Qwen2-VL-7B-direct & 0.07 & 0.01 \\
        \bottomrule
    \end{tabular}
\end{table}

\newpage

\section{Error Analysis}
\label{appendix:error_analysis}
We showcase the examples of the identified error types of reflection in Fig.~\ref{fig:ref_error_example}.
\begin{figure*}[!h]
\centering
\includegraphics[width=\textwidth]{fig/ref_error_example.pdf} 
\caption{\textbf{Examples of Reflection Error Types.}}
\label{fig:ref_error_example}
\end{figure*}


\newpage

\section{More Qualitative Examples}
\label{appendix:more_qualitative}
\begin{figure*}[!h]
\centering
\includegraphics[width=0.6\textwidth]{fig/precision_recall_example_GPT.pdf} 
\caption{\textbf{Examples of Precision and Recall Evaluation.}}
\label{fig:precision_recall_example_GPT}
\end{figure*}
\newpage

\begin{figure*}[!h]
\centering
\includegraphics[width=0.9\textwidth]{fig/precision_recall_example_Qwen.pdf} 
\caption{\textbf{Examples of Precision and Recall Evaluation.}}
\label{fig:precision_recall_example_Qwen}
\end{figure*}
\newpage

\begin{figure*}[!h]
\centering
\includegraphics[width=0.58\textwidth]{fig/precision_recall_example_QVQ.pdf}
\caption{\textbf{Examples of Precision and Recall Evaluation.}}
\label{fig:precision_recall_example_QVQ}
\end{figure*}
\newpage

\begin{figure*}[!h]
\centering
\includegraphics[width=\textwidth]{fig/precision_recall_example_QVQ2.pdf} 
\caption{\textbf{Examples of Precision and Recall Evaluation.}}
\label{fig:precision_recall_example_QVQ2}
\end{figure*}
\newpage

\begin{figure*}[!h]
\centering
\includegraphics[width=0.51\textwidth]{fig/precision_recall_example2_GPT.pdf} 
\caption{\textbf{Examples of Precision and Recall Evaluation.}}
\label{fig:precision_recall_example2_GPT}
\end{figure*}
\newpage

\begin{figure*}[!h]
\centering
\includegraphics[width=0.79\textwidth]{fig/precision_recall_example2_Qwen.pdf} 
\caption{\textbf{Examples of Precision and Recall Evaluation.}}
\label{fig:precision_recall_example2_Qwen}
\end{figure*}
\newpage

\begin{figure*}[!h]
\centering
\includegraphics[width=0.81\textwidth]{fig/precision_recall_example2_QVQ.pdf} 
\caption{\textbf{Examples of Precision and Recall Evaluation.}}
\label{fig:precision_recall_example2_QVQ}
\end{figure*}
\newpage

\begin{figure*}[!h]
\centering
\includegraphics[width=\textwidth]{fig/relevance_example_GPT.pdf} 
\caption{\textbf{Examples of Relevance Rate Evaluation.}}
% \vspace{-1cm}
\label{fig:relevance_example_GPT}
\end{figure*}
\newpage

\begin{figure*}[!h]
\centering
\includegraphics[width=\textwidth]{fig/relevance_example_Qwen.pdf} 
\caption{\textbf{Examples of Relevance Rate Evaluation.}}
% \vspace{-1cm}
\label{fig:relevance_example_Qwen}
\end{figure*}
\newpage

\begin{figure*}[!h]
\centering
\includegraphics[width=\textwidth]{fig/relevance_example_QVQ.pdf} 
\caption{\textbf{Examples of Relevance Rate Evaluation.}}
% \vspace{-1cm}
\label{fig:relevance_example_QVQ}
\end{figure*}
\newpage

\begin{figure*}[!h]
\centering
\includegraphics[width=\textwidth]{fig/ref_example_QVQ.pdf} 
\caption{\textbf{Examples of Reflection Quality Evaluation.}}
% \vspace{-1cm}
\label{fig:ref_example_QVQ}
\end{figure*}
\newpage


\section{Detailed Evaluation Setup}
\label{appendix:eval_setup}
\subsection{CoT Quality Evaluation Prompts}

\begin{tcolorbox}[breakable, colback=gray!5!white, colframe=gray!75!black, 
title=Recall Evaluation Prompt, boxrule=0.5mm, width=\textwidth, arc=3mm, auto outer arc]

You are an expert system to verify solutions to image-based problems. Your task is to match the ground truth middle steps with the provided solution.\\

INPUT FORMAT:\\
1. Problem: The original question/task\\
2. A Solution of a model\\
3. Ground Truth: Essential steps required for a correct answer\\

MATCHING PROCESS:\\

You need to match each ground truth middle step with the solution:\\

Match Criteria:\\
- The middle step should exactly match in the content or is directly entailed by a certain content in the solution\\
- All the details must be matched, including the specific value and content\\
- You should judge all the middle steps for whether there is a match in the solution\\

OUTPUT FORMAT:
\begin{verbatim}
[
  {
    "step_index": \textless integer\textgreater,
    "judgment": "Matched" | "Unmatched"
  }
]
\end{verbatim}

ADDITIONAL RULES:\\
1. Only output the JSON array with no additional information.\\
2. Judge each ground truth middle step in order without omitting any step.\\

Here are the problem, answer, solution, and ground truth middle steps:\\

[Problem]\\

\{question\}\\

[Answer]\\

\{answer\}\\

[Solution]\\

\{solution\}\\

[Ground Truth Information]\\

\{gt\_annotation\}

\end{tcolorbox}

\begin{tcolorbox}[breakable, colback=gray!5!white, colframe=gray!75!black, 
title=Precision Evaluation Prompt, boxrule=0.5mm, width=\textwidth, arc=3mm, auto outer arc]

\# Task Overview\\
Given a solution with multiple reasoning steps for an image-based problem, reformat it into well-structured steps and evaluate their correctness.\\

\# Step 1: Reformatting the Solution\\
Convert the unstructured solution into distinct reasoning steps while:\\
- Preserving all original content and order\\
- Not adding new interpretations\\
- Not omitting any steps\\

\#\# Step Types\\
1. Logical Inference Steps\\
   - Contains exactly one logical deduction\\
   - Must produce a new derived conclusion\\
   - Cannot be just a summary or observation\\
\\
2. Image Observation Steps\\
   - Pure visual observations\\
   - Only includes directly visible elements\\
   - No inferences or assumptions\\
\\
3. Background Information Steps\\
   - External knowledge or question context\\
   - No inference process involved\\

\#\# Step Requirements\\
- Each step must be atomic (one conclusion per step)\\
- No content duplication across steps\\
- Initial analysis counts as background information\\
- Final answer determination counts as logical inference\\

\# Step 2: Evaluating Correctness\\
Evaluate each step against:\\

\#\# Ground Truth Matching\\
For image observations:\\
- Key elements must match ground truth observations\\
\\
For logical inferences:\\
- Conclusion must EXACTLY match or be DIRECTLY entailed by ground truth\\

\#\# Reasonableness Check (if no direct match)\\
Step must:\\
- Premises must not contradict any ground truth or correct answer\\
- Logic is valid\\
- Conclusion must not contradict any ground truth \\
- Conclusion must support or be neutral to correct answer\\

\#\# Judgement Categories\\
- "Match": Aligns with ground truth\\
- "Reasonable": Valid but not in ground truth\\
- "Wrong": Invalid or contradictory\\
- "N/A": For background information steps\\

\# Output Requirements\\
1. The output format must be in valid JSON format without any other content.\\
2. For highly repetitive patterns, output it as a single step.\\
3. Output maximum 40 steps. Always include the final step that contains the answer.\\

Here is the json output format:\\
\#\# Output Format
\begin{verbatim}
[
  {
    "step_type": "image observation|logical inference|background information",
    "premise": "Evidence (only for logical inference)",
    "conclusion": "Step result",
    "judgment": "Match|Reasonable|Wrong|N/A"
  }
]
\end{verbatim}

Here is the problem, and the solution that needs to be reformatted to steps:\\

[Problem]\\

\{question\}\\

[Solution]\\

\{solution\}\\

[Correct Answer]\\

\{answer\}\\

[Ground Truth Information]\\

\{gt\_annotation\}

\end{tcolorbox}

\subsection{CoT Efficiency Prompt}
\begin{tcolorbox}[breakable, colback=gray!5!white, colframe=gray!75!black, 
title=Relevance Rate Evaluation Prompt, boxrule=0.5mm, width=\textwidth, arc=3mm, auto outer arc]
\# Task Overview
Given a solution with multiple reasoning steps for an image-based problem, evaluate the relevance to get a solution (ignore correct or wrong) of each step.\\

\# Step 1: Reformatting the Solution
Convert the unstructured solution into distinct reasoning steps while:\\
- Preserving all original content and order\\
- Not adding new interpretations\\
- Not omitting any steps\\

\#\# Step Types \\
1. Logical Inference Steps\\
  - Contains exactly one logical deduction\\
  - Must produce a new derived conclusion\\
  - Cannot be just a summary or observation

2. Image Description Steps\\
  - Pure visual observations\\
  - Only includes directly visible elements\\
  - No inferences or assumptions

3. Background Information Steps\\
  - External knowledge or question context\\
  - No inference process involved\\

\#\# Step Requirements
- Each step must be atomic (one conclusion per step)\\
- No content duplication across steps\\
- Initial analysis counts as background information\\
- Final answer determination counts as logical inference\\

\# Step 2: Evaluating Relevancy\\
A relevant step is considered as: 75\% content of the step must be related to trying to get a solution (ignore correct or wrong) to the question.\\

IMPORTANT NOTE:\\
Evaluate relevancy independent of correctness. As long as the step is trying to get to a solution, it is considered relevant. Logical fallacy, knowledge mistake, inconsistent with previous steps, or other mistakes do not affect relevance. A logically wrong step can be relevant if the reasoning attempts to address the question.\\

The following behaviour is considered as relevant:\\
i. The step is planning, summarizing, thinking, verifying, calculating, or confirming an intermediate/final conclusion helpful to get a solution.\\
ii. The step is summarizing or reflecting on previously reached conclusion relevant to get a solution.\\
iii. Repeating the information in the question or give the final answer.\\
iv. A relevant image depiction should be in one of following situation:\\
1. help to obtain a conclusion helpful to solve the question later;\\
2. help to identify certain patterns in the image later;\\
3. directly contributes to the answer\\
v. Depicting or analyzing the options of the question is also relevant.\\
vi. Repeating previous relevant steps are also considered relevant.\\

The following behaviour is considered as irrelevant:\\
i. Depicting image information that does not related to what is asking in the question. Example: The question asks how many cars are present in all the images. If the step focuses on other visual elements like the road or building, the step is considered as irrelevant.\\
ii. Self-thought not related to what the question is asking.\\
iii. Other information that is tangential for answering the question.\\

\# Output Format

\begin{verbatim}
[
  {
    "step_type": "image observation|logical inference|background information",
    "conclusion": "A brief summary of step result",
    "relevant": "Yes|No"
  }
]
\end{verbatim}\\

\# Output Rules\\
Direct JSON output without any other output\\
Output at most 40 steps\\

Here is the problem, and the solution that needs to be reformatted to steps:

[Problem]\\

\{question\}\\

[Solution]\\

\{solution\}
\end{tcolorbox}

\begin{tcolorbox}[breakable, colback=gray!5!white, colframe=gray!75!black, 
title=Reflection Quality Evaluation Prompt, boxrule=0.5mm, width=\textwidth, arc=3mm, auto outer arc]

Here\'s a refined prompt that improves clarity and structure:\\

\# Task\\
Evaluate reflection steps in image-based problem solutions, where reflections are self-corrections or reconsideration of previous statements.\\

\# Reflection Step Identification \\
Reflections typically begin with phrases like:\\
- "But xxx"\\
- "Alternatively, xxx" \\
- "Maybe I should"\\
- "Let me double-check"\\
- "Wait xxx"\\
- "Perhaps xxx"\\
It will throw a doubt of its previously reached conclusion or raise a new thought.\\

\# Evaluation Criteria\\
Correct reflections must:\\
1. Reach accurate conclusions aligned with ground truth\\
2. Use new insights to find the mistake of the previous conclusion or verify its correctness. \\

Invalid reflections include:\\
1. Repetition - Restating previous content or method without new insights\\
2. Wrong Conclusion - Reaching incorrect conclusions vs ground truth\\
3. Incompleteness - Proposing but not executing new analysis methods\\
4. Other - Additional error types\\

\# Input Format\\

[Problem]\\

\{question\}\\

[Solution]\\

\{solution\}\\

[Ground Truth]\\

\{gt\_annotation\}\\

\# Output Requirements\\
1. The output format must be in valid JSON format without any other content.\\
2. Output maximum 30 reflection steps.\\

Here is the json output format:\\
\#\# Output Format
\begin{verbatim}
[
  {
    "conclusion": "One-sentence summary of reflection outcome",
    "judgment": "Correct|Wrong",
    "error_type": "N/A|Repetition|Wrong Conclusion|Incompleteness|Other"
  }
]
\end{verbatim}

\# Rules\\
1. Preserve original content and order\\
2. No new interpretations\\
3. Include ALL reflection steps\\
4. Empty list if no reflections found\\
5. Direct JSON output without any other output

\end{tcolorbox}

\subsection{Direct Evaluation Prompt}
\begin{tcolorbox}[breakable, colback=gray!5!white, colframe=gray!75!black, 
title=Answer Extraction Prompt, boxrule=0.5mm, width=\textwidth, arc=3mm, auto outer arc]
You are an AI assistant who will help me to extract an answer of a question. You are provided with a question and a response, and you need to find the final answer of the question. \\

Extract Rule:

[Multiple choice question]

1. The answer could be answering the option letter or the value. You should directly output the choice letter of the answer.

2. You should output a single uppercase character in A, B, C, D, E, F, G, H, I (if they are valid options), and Z.

3. If the meaning of all options are significantly different from the final answer, output Z. \\

[Non Multiple choice question]

1. Output the final value of the answer. It could be hidden inside the last step of calculation or inference. Pay attention to what the question is asking for to extract the value of the answer.

2. The final answer could also be a short phrase or sentence.

3. If the response doesn't give a final answer, output Z.\\

Output Format: 
Directly output the extracted answer of the response. \\

\{In Context Examples\}\\

Question: \{question\}

Answer: \{response\}\\

Your output: 

\end{tcolorbox}

\begin{tcolorbox}[breakable, colback=gray!5!white, colframe=gray!75!black, 
title=Answer Scoring Prompt, boxrule=0.5mm, width=\textwidth, arc=3mm, auto outer arc]

You are an AI assistant who will help me to judge whether two answers are consistent.\\

Input Illustration:
[Standard Answer] is the standard answer to the question. 
[Model Answer] is the answer extracted from a model's output to this question. 

Task Illustration:
Determine whether [Standard Answer] and [Model Answer] are consistent.\\

Consistent Criteria:

[Multiple-Choice questions]

1. If the [Model Answer] is the option letter, then it must completely matches the [Standard Answer].

2. If the [Model Answer] is not an option letter, then the [Model Answer] must completely match the option content of [Standard Answer].

[Nan-Multiple-Choice questions]

1. The [Model Answer] and [Standard Answer] should exactly match.

2. If the meaning is expressed in the same way, it is also considered consistent, for example, 0.5m and 50cm.\\

Output Format: 
1. If they are consistent, output 1; if they are different, output 0.

2. DIRECTLY output 1 or 0 without any other content.

\{In Context Examples\}\\

Question: \{question\}

[Model Answer]: \{extract\_answer\}

[Standard Answer]: \{gt\_answer\}

Your output:

\end{tcolorbox}

\end{document}
