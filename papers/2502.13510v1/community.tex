\subsection{Your software is useful to others only if they know about it}
\label{sec:community}

\paragraph{Background} In the previous recommendations, we discussed various points on how to build high-quality reusable software. However, software quality is meaningless if the community is unaware that the software project exists.
Having external users increases the scientific impact of the software and improves its quality by making it more thoroughly tested with various scenarios and edge cases. Finally, every user of your software is a potential future collaborator.
While \ref{sec:openSource} already discusses the first points as publishing the code open source and registering it in a registry, there are more ways to get attention to your \ac{ERS}.
Therefore, we will provide recommendations on increasing the number of users and overall reach of your \ac{ERS}.
\par 

\paragraph{Recommendations} The natural step to increase the visibility of your software is to publish a paper about it. While we discussed in \ref{sec:documentation} that software papers are not a replacement for documentation, they are great for increasing visibility and motivating your software. Consequently, a software paper should not contain too much technical details from the documentation. Instead, it should explain why the software is important, discuss its high-level capabilities, and convey its scope and vision. We recommend \ac{JOSS}\footref{fn:joss} \cite{jossJournal} as a modern software journal. If you want to discuss domain-specific aspects in more detail, it can also be helpful to publish your software in a conventional domain-specific journal, e.g., as it was done for \textit{pandapower} \cite{pandapower2018} and \textit{renewables.ninja} \cite{pfenninger_long-term_2016}.

\par 
The second essential step is the active engagement with the community.
Especially in the early stage of the software project, it is a good approach to contact domain experts and actively collect early feedback. First, this makes them aware of your software. Second, it prevents the developed tool from missing the needs of the researchers working in the field. Later in the process, you can actively present your software project at a conference, e.g., at the "Open Source Modelling and Simulation of Energy Systems (OSMSES)"\footnote{\url{https://www.osmses2024.org/home}, last access 2025-01-02} conference. 
\par 
If your software project uses GitHub/GitLab, their issue system is a great way to receive feedback of any kind. Feature requests provide direct information about what the users need, and questions about your software indicate where the documentation can be improved. 
In general, the software repository should be structured in a way that facilitates engagement with the community. Create a \texttt{CONTRIBUTING} file to communicate to the community how they can contribute to the software project, including, for example, how to report bugs, style recommendations, or the code of conduct.
Contact information should always be available if external researchers want to contact the maintainers. 

\par 
In summary, when developing reusable software, you should care about users and actively engage with them to improve software quality and maximize scientific impact.  

\recommendation{Grow your community!} 
