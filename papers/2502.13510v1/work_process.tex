\subsection{Being organized boosts your software project}
\label{sec:workProcess}

\paragraph{Background} This recommendation discusses how research software development teams should organize themselves.

The most important aspect here is that there is no one-size-fits-all solution.
A one-person PhD software project should be organized differently than a large team developing software with thousands of potential users.

\paragraph{Recommendations} The main recommendation is to explicitly define the software project's scope and purpose early in the process. What is the software's general use case (see \ref{sec:architecture})? How many users are envisioned (see \ref{sec:community})? 
Is the software intended for research only or for usage by industry as well? What is its expected lifespan? Will complete reimplementation be an option, e.g., for going commercial?

Also, you should get a good understanding of the available team to create the software. 
How many people are working on the software project permanently? What are their skills? 

Explicit answers to these questions are required to make wise decisions about organizing the software project and the team. 
For example, defining the software's scope helps to decide what to implement and what not. 
The team size, on the other hand, is the most important factor for defining the development process. Do we need code reviews? Do we need people responsible for specific tasks like testing, documentation? Especially, large teams must define and enforce the process explicitly to not end in chaos, like outdated documentation (see \ref{sec:documentation}) or failing pipelines (see \ref{sec:testing}).

Generally, you can think of software development organization as a scale from a high degree of agility (e.g., extreme programming~\cite{796139}) to a high degree of planning (e.g., waterfall model \cite{Petersen835760}). Consider which degree is desirable for your project, depending on the answers to the precious questions. Most research projects have a high degree of uncertainty, resulting in frequent changes in software requirements. This would naturally suit a more agile organization.

Another critical question is the developers' software engineering experience. Especially in energy research, the skill levels vary significantly, as we pointed out in \autoref{sec:intro}. 
We researchers are not expected to have the skills of a full-time professional developer. However, we \textbf{are} software developers, which mandates certain minimum skill requirements. Therefore, we should perform a skill assessment of all team members and actively provide training to fill the gaps. 
\par 
Further, it is important to assign clear responsibilities. Primarily the software project lead should be clear. It should be someone who actively participates in the software project and does not have too many other responsibilities. For example, an experienced research assistant is often a better choice than a professor who does not have the time to participate actively.

\par 
Unorganized software projects result in low-quality software, so we should explicitly discuss and define our work process. Since there are no general rules, we need clear leadership and a clear vision of our software project's scope and purpose.

\recommendation{Organize yourself and your team!} 
