\subsection{Research should be publicly available, and your software is part of it}\label{sec:openSource}

\paragraph{Background} Only openly available research software allows other researchers to reproduce your research results. Since research software is mainly publicly funded, it is also fair to make it publicly available. This also enables others to reuse the software, providing additional benefits like further testing of the software and contributing by creating extensions and compatible software. Besides making the software open source, it is also important to register it so other researchers can find it. These aspects are essential to make a research software \ac{FAIR}~\cite{barker2022introducing}. 

\paragraph{Recommendations} We recommend developing your research software open source. By directly starting the development as open source, challenges can be avoided when switching to open source later. Further, it allows other researchers to use your software and actively contribute as early as possible, improving your software's quality and encouraging cooperation. To develop the software open source, a suitable license is required. By choosing a license early each time another software is included, the compatibility of licenses can be checked directly. Some open source licenses are incompatible since they put certain conditions on the reuse of the code, which can conflict between licenses \cite{cui_empirical_2023}.
When choosing a license, you should check your institution's policy, which often already proposes a specific license. Also, various guides help to choose a license\footnote{e.g., \url{https://choosealicense.com/}, last access 2024-12-17.}.

Cooperation and joint research projects with industry are common in energy research. Sometimes, industry partners are critical of developing open source because they want to keep their intellectual property. Generally, we recommend discussing this topic as early as possible to find reasonable compromises. Often, certain parts can be developed open source while others remain closed source. Since open source is open to all researchers and all industries, it can also improve the exchange between industry and research. 

When you develop open source, there are specific approaches to make your code more easily reusable by others. First, the repository should follow programming-language-specific best practices. This can be achieved by starting with templates for the repository\footnote{e.g., cookie-cutter templates at \url{https://cookiecutter.io/templates}, last access 2024-12-17}. Additionally, citing your software can be made easy by including a \ac{CFF} file\footnote{\url{https://citation-file-format.github.io}, last access 2024-12-17.} in your repository
 \cite{druskat_citation_2021}.

Within the repository, it should be indicated if the software will be maintained, if support is available and if the developers are open to joining research projects, including the software. We recommend using the features of the software platform (GitHub/GitLab) to interact with potential users, e.g., by using issues.

Besides making the source code available, the software should also be findable. Therefore, we recommend registering the software in a domain-specific registry like the Open Energy Platform\footref{fn:oep}. Getting a DOI for the software is also helpful, e.g., by archiving versions of the software on Zenodo\footnote{\url{https://zenodo.org/}, last access 2024-12-17.} \cite{zenodo}. 

\par
Energy research is highly interdisciplinary \cite{tijssen_quantitative_1992}.
Therefore, we recommend adding a very general description to your software, allowing all researchers to understand its goals. This way, more researchers can identify whether the software is useful for them, increasing its reusability. GitHub also allows you to provide keywords to your repository, which again improves findability.


\par 
Platforms like GitHub and GitLab make it easy to publish your software under an open source license. Open source research software enables reproducibility and allows reusability, which can also improve your code quality. Therefore, we recommend: 

\recommendation{Develop open source \& make your software findable!}