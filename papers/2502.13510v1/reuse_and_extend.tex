\subsection{Inventing is great, reinventing is a waste of time}
\label{sec:reuseAndExtend}

\paragraph{Background} The core of science is to build on top of other's ideas and contributions. Especially in recent years, more and more research software projects are released as open-source and can be extended and reused by the community. Still, we often feel the necessity to start from scratch and build our own library or framework. For example, a recent review identified 63 different open-source, optimization-based frameworks for energy system modeling~\cite{hoffmann_review_2024}. They all share the same underlying principle of network-based energy flow optimization with similar architectures. Many frameworks are only used by their home institutions, which indicates possible redundancy \cite{hoffmann_review_2024}. However, writing code is only one of our many diverse responsibilities.  
Overall, researchers have too little time to reimplement work already done by others. Instead, we should aim to reuse and extend existing high-quality software and focus on implementing novel research contributions. While familiarizing oneself with the documentation of other tools seems more time-consuming in the beginning, this time will be saved later on because existing code often already comes with a lot of testing, verification, and documentation which you would otherwise need to create for your new code.
This recommendation discusses creating high-quality research software without implementing everything from scratch. 

\paragraph{Recommendations} The first step of reusing software is to find existing software that solves your problem or parts of it. The natural approach is to do a literature survey beforehand. While this is an important step, not all software is published as a conference or journal publication. Instead, software registries and similar platforms can be used to find research software. For energy research,  the Open Energy Platform\footnote{\label{fn:oep}\url{https://openenergyplatform.org/}, last access 2024-12-04.} and DOE CODE\footnote{\url{https://www.osti.gov/doecode/}m last access 2025-01-20} are available. Additionally, the Helmholtz Research Software Directory\footnote{\url{https://helmholtz.software/software?&keywords=\%5B\%22Energy\%22\%5D\&page=0\&rows=12}, last access 2024-12-04.} and the \ac{JOSS}\footnote{\label{fn:joss}\url{https://joss.theoj.org}, last access 2024-12-16.} list research software without a domain focus. 
Software repositories like GitHub\footnote{\url{https://github.com/search?q=\%22power\%20system\%22\&type=repositories}, last access 2024-12-04.} can also be searched. In these cases, energy-specific keywords should be used.
The final recommendation for finding reusable software is to ask experienced colleagues and other researchers who work on similar problems. They can often tell you about hitherto unknown tools, e.g., those under development in other research projects, and which libraries are useful.
\par 
To use external software for your research project, you should check their license and see if it is compatible with your code and its envisioned use case (see also \ref{sec:openSource}). 
If the software does not entirely cover your use case, you can contact the software maintainers. They may be already discussing and planning the feature you need. If they do not want to implement it they may wish to include your software when it is finished. By explicitly communicating with other researchers, redundant software can be prevented, and as a side effect, overall collaboration in science is strengthened. This \textit{collaboration first} thinking should follow through the whole software project. For example, write issues and pull requests when finding bugs in other software you use. Not only does this help other people, but it also improves your software as a side effect.

\par 
However, there are exceptions to the rule of reusing existing software. When the existing solutions are of low quality or outdated, starting a new software from scratch can be a good idea. Also, developing an open-source alternative can have a significant scientific impact if the existing software is proprietary. 
In general, parallel development is perfectly fine to some extent. It results in competition and provides other researchers with options that have respective pros and cons.
\par 
Reusing existing research software results in higher-quality software for less effort. It accelerates scientific progress and supports overall collaboration. Hence, the following recommendation: 
\recommendation{Reuse and extend other software if possible and useful!}     
