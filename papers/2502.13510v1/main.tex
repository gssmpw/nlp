%%
%% Based on `sample-sigconf-authordraft.tex' (https://www.acm.org/publications/proceedings-template),
%%
%% Commands for TeXCount
%TC:macro \cite [option:text,text]
%TC:macro \citep [option:text,text]
%TC:macro \citet [option:text,text]
%TC:envir table 0 1
%TC:envir table* 0 1
%TC:envir tabular [ignore] word
%TC:envir displaymath 0 word
%TC:envir math 0 word
%TC:envir comment 0 0
%%
%%
%% The first command in your LaTeX source must be the \documentclass
%% command.
%%
%% For submission and review of your manuscript please change the
%% command to \documentclass[manuscript, screen, review]{acmart}.
%%
%% When submitting camera ready or to TAPS, please change the command
%% to \documentclass[sigconf]{acmart} or whichever template is required
%% for your publication.
%%
%%

\documentclass[sigconf,nonacm,anonymous=false]{acmart}
\usepackage[nolist,nohyperlinks]{acronym}
%\usepackage{todonotes}
\usepackage[disable]{todonotes}
\usepackage{hyperref}
\usepackage{pifont}% http://ctan.org/pkg/pifont
\newcommand{\cmark}{\ding{51}}%

\newcounter{rec}
\setcounter{rec}{1}

\definecolor{boxgray}{gray}{0.95}
\newcommand{\StaticBox}[1]{%
    \setlength{\fboxsep}{7pt} % Set padding
    \par\noindent
    \colorbox{boxgray}{%
        \parbox[b]{\columnwidth-15pt}{% Subtract double of padding (for both sides) + 1 to allow normal line break afterwards
            \setlength{\parindent}{0pt}%
            \setlength{\parskip}{16pt}%
            \centering
            #1%
        }%
    }%
}

\newcommand{\recommendation}[1]{{
\StaticBox{{{\textbf{(\therec)}}}~\,\textbf{#1}
}
}
\stepcounter{rec}
}

%%
%% \BibTeX command to typeset BibTeX logo in the docs
\AtBeginDocument{%
  \providecommand\BibTeX{{%
    Bib\TeX}}}

%% Rights management information.  This information is sent to you
%% when you complete the rights form.  These commands have SAMPLE
%% values in them; it is your responsibility as an author to replace
%% the commands and values with those provided to you when you
%% complete the rights form.
%\setcopyright{acmlicensed}
%\copyrightyear{2025}
%\acmYear{2025}
%\acmDOI{XXXXXXX.XXXXXXX}

%\acmISBN{978-1-4503-XXXX-X/18/06}


%%
%% Submission ID.
%% Use this when submitting an article to a sponsored event. You'll
%% receive a unique submission ID from the organizers
%% of the event, and this ID should be used as the parameter to this command.
%%\acmSubmissionID{123-A56-BU3}

%%
%% For managing citations, it is recommended to use bibliography
%% files in BibTeX format.
%%
%% You can then either use BibTeX with the ACM-Reference-Format style,
%% or BibLaTeX with the acmnumeric or acmauthoryear sytles, that include
%% support for advanced citation of software artefact from the
%% biblatex-software package, also separately available on CTAN.
%%
%% Look at the sample-*-biblatex.tex files for templates showcasing
%% the biblatex styles.
%%

%%
%% The majority of ACM publications use numbered citations and
%% references.  The command \citestyle{authoryear} switches to the
%% "author year" style.
%%
%% If you are preparing content for an event
%% sponsored by ACM SIGGRAPH, you must use the "author year" style of
%% citations and references.
%% Uncommenting
%% the next command will enable that style.
%%\citestyle{acmauthoryear}

%%
%% end of the preamble, start of the body of the document source.
\begin{document}
{
%%
%% The "title" command has an optional parameter,
%% allowing the author to define a "short title" to be used in page headers.
\title{Ten Recommendations for Engineering Research Software in Energy Research} 
%%
%% The "author" command and its associated commands are used to define
%% the authors and their affiliations.
%% Of note is the shared affiliation of the first two authors, and the
%% "authornote" and "authornotemark" commands
%% used to denote shared contribution to the research.
{
\author{Stephan Ferenz}
%\authornote{Both authors contributed equally to this research.}
\email{stephan.ferenz@uol.de}
\orcid{https://orcid.org/0000-0001-9523-7227}
\affiliation{%
  \institution{Carl von Ossietzky Universität Oldenburg}
  \streetaddress{Ammerländer Heerstraße 114-118}
  \city{Oldenburg}
  \country{Germany}
  \postcode{26129}
}
\affiliation{
  \institution{OFFIS - Institute for Information Technology}
  \streetaddress{Escherweg 2}
  \city{Oldenburg}
  \country{Germany}
  \postcode{26121}
}

\author{Emilie Frost}
\email{emilie.frost@uol.de}
\orcid{https://orcid.org/0000-0003-4791-2333}
\affiliation{%
  \institution{Carl von Ossietzky Universität Oldenburg}
  \streetaddress{Ammerländer Heerstraße 114-118}
  \city{Oldenburg}
  \country{Germany}
  \postcode{26129}
}
\affiliation{
  \institution{OFFIS - Institute for Information Technology}
  \streetaddress{Escherweg 2}
  \city{Oldenburg}
  \country{Germany}
  \postcode{26121}
}

\author{Rico Schrage}
\email{rico.schrage@uol.de}
\orcid{https://orcid.org/0000-0001-5339-6553}
\affiliation{%
  \institution{Carl von Ossietzky Universität Oldenburg}
  \streetaddress{Ammerländer Heerstraße 114-118}
  \city{Oldenburg}
  \country{Germany}
  \postcode{26129}
}
\affiliation{
  \institution{OFFIS - Institute for Information Technology}
  \streetaddress{Escherweg 2}
  \city{Oldenburg}
  \country{Germany}
  \postcode{26121}
}

\author{Thomas Wolgast}
\email{thomas.wolgast@uol.de}
\orcid{https://orcid.org/0000-0002-9042-9964}
\affiliation{%
  \institution{Carl von Ossietzky Universität Oldenburg}
  \streetaddress{Ammerländer Heerstraße 114-118}
  \city{Oldenburg}
  \country{Germany}
  \postcode{26129}
}
\affiliation{
  \institution{OFFIS - Institute for Information Technology}
  \streetaddress{Escherweg 2}
  \city{Oldenburg}
  \country{Germany}
  \postcode{26121}
}




\author{Inga Beyers}
\email{beyers@ifes.uni-hannover.de}
\orcid{https://orcid.org/0000-0002-9566-770X}
\affiliation{%
  \institution{Leibniz University Hanover, Institute of Electric Power Systems}
  \streetaddress{Appelstraße 9A}
  \city{Hannover}
  \country{Germany}
  \postcode{30167}
}


\author{Oliver Karras}
\email{oliver.karras@tib.eu}
\orcid{https://orcid.org/0000-0001-5336-6899}
\affiliation{%
  \institution{TIB - Leibniz Information Centre for Science and Technology}
  \streetaddress{Welfengarten 1B}
  \city{Hannover}
  \country{Germany}
  \postcode{30167}
}

\author{Oliver Werth}
\email{oliver.werth@offis.de}
\orcid{https://orcid.org/0000-0002-6767-5905}
\affiliation{
  \institution{OFFIS - Institute for Information Technology}
  \streetaddress{Escherweg 2}
  \city{Oldenburg}
  \country{Germany}
  \postcode{26121}
}

\author{Astrid Nieße}
\email{astrid.niesse@uol.de}
\orcid{https://orcid.org/0000-0003-1881-9172}
\affiliation{%
  \institution{Carl von Ossietzky Universität Oldenburg}
  \streetaddress{Ammerländer Heerstraße 114-118}
  \city{Oldenburg}
  \country{Germany}
  \postcode{26129}
}
\affiliation{
  \institution{OFFIS - Institute for Information Technology}
  \streetaddress{Escherweg 2}
  \city{Oldenburg}
  \country{Germany}
  \postcode{26121}
}
}

%%
%% By default, the full list of authors will be used in the page
%% headers. Often, this list is too long, and will overlap
%% other information printed in the page headers. This command allows
%% the author to define a more concise list
%% of authors' names for this purpose.
\renewcommand{\shortauthors}{Ferenz et al.}

%%
%% The abstract is a short summary of the work to be presented in the
%% article.
\begin{abstract}

  Energy research software~(ERS) is a central cornerstone to facilitate energy research.
However, ERS is developed by researchers who, in many cases, lack formal training in software engineering. This reduces the quality of ERS, leading to limited reproducibility and reusability. 
  To address these issues, we developed ten central recommendations for the development of ERS, covering areas such as conceptualization, development, testing, and publication of ERS. 
  The recommendations are based on the outcomes of two workshops with a diverse group of energy researchers and aim to improve the awareness of research software engineering in the energy domain. The recommendations should enhance the quality of ERS and, therefore, the reproducibility of energy research.


\end{abstract}

%%
%% The code below is generated by the tool at http://dl.acm.org/ccs.cfm.
%% Please copy and paste the code instead of the example below.
%%
\begin{CCSXML}
<ccs2012>
<concept>
<concept_id>10011007.10011074.10011134.10003559</concept_id>
<concept_desc>Software and its engineering~Open source model</concept_desc>
<concept_significance>300</concept_significance>
</concept>
<concept>
<concept_id>10002951.10003317.10003347</concept_id>
<concept_desc>Information systems~Retrieval tasks and goals</concept_desc>
<concept_significance>300</concept_significance>
</concept>
<concept>
<concept_id>10011007.10011074</concept_id>
<concept_desc>Software and its engineering~Software creation and management</concept_desc>
<concept_significance>500</concept_significance>
</concept>
<concept>
<concept_id>10010405.10010432</concept_id>
<concept_desc>Applied computing~Physical sciences and engineering</concept_desc>
<concept_significance>500</concept_significance>
</concept>
</ccs2012>
\end{CCSXML}

\ccsdesc[300]{Software and its engineering~Open source model}
\ccsdesc[300]{Information systems~Retrieval tasks and goals}
\ccsdesc[500]{Software and its engineering~Software creation and management}
\ccsdesc[500]{Applied computing~Physical sciences and engineering}

%%
%% Keywords. The author(s) should pick words that accurately describe
%% the work being presented. Separate the keywords with commas.
\keywords{Energy Research, Research Software Engineering, Energy Research Software, Software Development, Smart Grid, Power System, Energy System}

%\received{24 January 2025}
%\received[revised]{12 March 2009}
%\received[accepted]{5 June 2009}

%%
%% This command processes the author and affiliation and title
%% information and builds the first part of the formatted document.
\maketitle
}
\section{Introduction}
\label{sec:introduction}
The business processes of organizations are experiencing ever-increasing complexity due to the large amount of data, high number of users, and high-tech devices involved \cite{martin2021pmopportunitieschallenges, beerepoot2023biggestbpmproblems}. This complexity may cause business processes to deviate from normal control flow due to unforeseen and disruptive anomalies \cite{adams2023proceddsriftdetection}. These control-flow anomalies manifest as unknown, skipped, and wrongly-ordered activities in the traces of event logs monitored from the execution of business processes \cite{ko2023adsystematicreview}. For the sake of clarity, let us consider an illustrative example of such anomalies. Figure \ref{FP_ANOMALIES} shows a so-called event log footprint, which captures the control flow relations of four activities of a hypothetical event log. In particular, this footprint captures the control-flow relations between activities \texttt{a}, \texttt{b}, \texttt{c} and \texttt{d}. These are the causal ($\rightarrow$) relation, concurrent ($\parallel$) relation, and other ($\#$) relations such as exclusivity or non-local dependency \cite{aalst2022pmhandbook}. In addition, on the right are six traces, of which five exhibit skipped, wrongly-ordered and unknown control-flow anomalies. For example, $\langle$\texttt{a b d}$\rangle$ has a skipped activity, which is \texttt{c}. Because of this skipped activity, the control-flow relation \texttt{b}$\,\#\,$\texttt{d} is violated, since \texttt{d} directly follows \texttt{b} in the anomalous trace.
\begin{figure}[!t]
\centering
\includegraphics[width=0.9\columnwidth]{images/FP_ANOMALIES.png}
\caption{An example event log footprint with six traces, of which five exhibit control-flow anomalies.}
\label{FP_ANOMALIES}
\end{figure}

\subsection{Control-flow anomaly detection}
Control-flow anomaly detection techniques aim to characterize the normal control flow from event logs and verify whether these deviations occur in new event logs \cite{ko2023adsystematicreview}. To develop control-flow anomaly detection techniques, \revision{process mining} has seen widespread adoption owing to process discovery and \revision{conformance checking}. On the one hand, process discovery is a set of algorithms that encode control-flow relations as a set of model elements and constraints according to a given modeling formalism \cite{aalst2022pmhandbook}; hereafter, we refer to the Petri net, a widespread modeling formalism. On the other hand, \revision{conformance checking} is an explainable set of algorithms that allows linking any deviations with the reference Petri net and providing the fitness measure, namely a measure of how much the Petri net fits the new event log \cite{aalst2022pmhandbook}. Many control-flow anomaly detection techniques based on \revision{conformance checking} (hereafter, \revision{conformance checking}-based techniques) use the fitness measure to determine whether an event log is anomalous \cite{bezerra2009pmad, bezerra2013adlogspais, myers2018icsadpm, pecchia2020applicationfailuresanalysispm}. 

The scientific literature also includes many \revision{conformance checking}-independent techniques for control-flow anomaly detection that combine specific types of trace encodings with machine/deep learning \cite{ko2023adsystematicreview, tavares2023pmtraceencoding}. Whereas these techniques are very effective, their explainability is challenging due to both the type of trace encoding employed and the machine/deep learning model used \cite{rawal2022trustworthyaiadvances,li2023explainablead}. Hence, in the following, we focus on the shortcomings of \revision{conformance checking}-based techniques to investigate whether it is possible to support the development of competitive control-flow anomaly detection techniques while maintaining the explainable nature of \revision{conformance checking}.
\begin{figure}[!t]
\centering
\includegraphics[width=\columnwidth]{images/HIGH_LEVEL_VIEW.png}
\caption{A high-level view of the proposed framework for combining \revision{process mining}-based feature extraction with dimensionality reduction for control-flow anomaly detection.}
\label{HIGH_LEVEL_VIEW}
\end{figure}

\subsection{Shortcomings of \revision{conformance checking}-based techniques}
Unfortunately, the detection effectiveness of \revision{conformance checking}-based techniques is affected by noisy data and low-quality Petri nets, which may be due to human errors in the modeling process or representational bias of process discovery algorithms \cite{bezerra2013adlogspais, pecchia2020applicationfailuresanalysispm, aalst2016pm}. Specifically, on the one hand, noisy data may introduce infrequent and deceptive control-flow relations that may result in inconsistent fitness measures, whereas, on the other hand, checking event logs against a low-quality Petri net could lead to an unreliable distribution of fitness measures. Nonetheless, such Petri nets can still be used as references to obtain insightful information for \revision{process mining}-based feature extraction, supporting the development of competitive and explainable \revision{conformance checking}-based techniques for control-flow anomaly detection despite the problems above. For example, a few works outline that token-based \revision{conformance checking} can be used for \revision{process mining}-based feature extraction to build tabular data and develop effective \revision{conformance checking}-based techniques for control-flow anomaly detection \cite{singh2022lapmsh, debenedictis2023dtadiiot}. However, to the best of our knowledge, the scientific literature lacks a structured proposal for \revision{process mining}-based feature extraction using the state-of-the-art \revision{conformance checking} variant, namely alignment-based \revision{conformance checking}.

\subsection{Contributions}
We propose a novel \revision{process mining}-based feature extraction approach with alignment-based \revision{conformance checking}. This variant aligns the deviating control flow with a reference Petri net; the resulting alignment can be inspected to extract additional statistics such as the number of times a given activity caused mismatches \cite{aalst2022pmhandbook}. We integrate this approach into a flexible and explainable framework for developing techniques for control-flow anomaly detection. The framework combines \revision{process mining}-based feature extraction and dimensionality reduction to handle high-dimensional feature sets, achieve detection effectiveness, and support explainability. Notably, in addition to our proposed \revision{process mining}-based feature extraction approach, the framework allows employing other approaches, enabling a fair comparison of multiple \revision{conformance checking}-based and \revision{conformance checking}-independent techniques for control-flow anomaly detection. Figure \ref{HIGH_LEVEL_VIEW} shows a high-level view of the framework. Business processes are monitored, and event logs obtained from the database of information systems. Subsequently, \revision{process mining}-based feature extraction is applied to these event logs and tabular data input to dimensionality reduction to identify control-flow anomalies. We apply several \revision{conformance checking}-based and \revision{conformance checking}-independent framework techniques to publicly available datasets, simulated data of a case study from railways, and real-world data of a case study from healthcare. We show that the framework techniques implementing our approach outperform the baseline \revision{conformance checking}-based techniques while maintaining the explainable nature of \revision{conformance checking}.

In summary, the contributions of this paper are as follows.
\begin{itemize}
    \item{
        A novel \revision{process mining}-based feature extraction approach to support the development of competitive and explainable \revision{conformance checking}-based techniques for control-flow anomaly detection.
    }
    \item{
        A flexible and explainable framework for developing techniques for control-flow anomaly detection using \revision{process mining}-based feature extraction and dimensionality reduction.
    }
    \item{
        Application to synthetic and real-world datasets of several \revision{conformance checking}-based and \revision{conformance checking}-independent framework techniques, evaluating their detection effectiveness and explainability.
    }
\end{itemize}

The rest of the paper is organized as follows.
\begin{itemize}
    \item Section \ref{sec:related_work} reviews the existing techniques for control-flow anomaly detection, categorizing them into \revision{conformance checking}-based and \revision{conformance checking}-independent techniques.
    \item Section \ref{sec:abccfe} provides the preliminaries of \revision{process mining} to establish the notation used throughout the paper, and delves into the details of the proposed \revision{process mining}-based feature extraction approach with alignment-based \revision{conformance checking}.
    \item Section \ref{sec:framework} describes the framework for developing \revision{conformance checking}-based and \revision{conformance checking}-independent techniques for control-flow anomaly detection that combine \revision{process mining}-based feature extraction and dimensionality reduction.
    \item Section \ref{sec:evaluation} presents the experiments conducted with multiple framework and baseline techniques using data from publicly available datasets and case studies.
    \item Section \ref{sec:conclusions} draws the conclusions and presents future work.
\end{itemize}


\section{Characteristics of ERS} \label{sec:chara}

Research software, in general, has several unique characteristics. It is often highly specific and mainly created to solve one/few particular use case(s). The software is usually not designed for reuse and has a short lifecycle. Generally, it is hard to comprehend the software, even for researchers from the same domain, due to the inherent complexity of the solved problems. Further, most researchers are not explicitly trained in software engineering. Another fundamental characteristic of research software is that the software requirements are not known beforehand but often only evolve during development~\cite{hasselbring_toward_2024,felderer_investigating_2025}. 

\ac{ERS} shares these characteristics of general research software, while some common aspects can be considered more critical in energy research. Additionally, \ac{ERS} also has some special characteristics:

\begin{description}
    \item [Interdisciplinary research] Energy researchers have diverse backgrounds ranging from social science over engineering to different natural sciences and mathematics. As a broad range of domains are involved, it is hard to establish a common language. Therefore, it is difficult to achieve a mutual view of the problems that must be solved as a team. This issue influences the development of \ac{ERS} as well as its capability of being understood by different interested researchers like project partners.
    \item [Applied research] Energy research is an applied research field. Therefore, cooperation and joint projects with industry are widespread. This also has implications on the engineering of the jointly created software projects, e.g., with respect to open source or use of commercial software.
    \item [Changing levels of detail] \ac{ERS} is diverse and complex and needs to support many different analysis types and detail levels (e.g., transient vs. steady-state).
    \item [Complexity] Energy researchers often look into the detailed behavior of larger systems consisting of many components. 
    \item [Data heterogeneity] \ac{ERS} needs a lot of diverse and heterogeneous data for most simulations and often produces data different in structure, type and size, which needs to be managed \cite{zhang2018big}.
    \item [Reliability] Energy systems are critical infrastructures leading to high demands on research software reliability, specifically when conducting field tests. 
    \item [Changing time horizons] \ac{ERS} often considers diverse time horizons, diverse spatial resolutions, and complex optimization problems~\cite{ENGELAND2017600,DECAROLIS2017184}. These aspects can lead to high performance requirements.
    \item [Coupled co-simulations] As huge infrastructure systems are under analysis, there is a need to couple different types of simulation, often comprising communication simulation  \cite{vogt_survey_2018,steinbrink_cpes_2019}.
\end{description}%

We generally conclude that the most essential aspects of \ac{ERS} are the people involved and working on different detail levels, spatial resolutions, and time resolutions in a diverse and highly simulation-driven research area.
Our work draws heavily from the literature on semiparametric inference and double machine learning~\citep{robins1994estimation,robins1995semiparametric,tsiatis2006semiparametric,chernozhukov2018double}. In particular, our estimator is an optimal combination of several Augmented Inverse Probability Weighting~(\aipw) estimators, whose outcome regressions are replaced with foundation models. Importantly, the standard $\aipw$ estimator, which relies on an outcome regression estimated using experimental data alone, is also included in the combination. This approach allows \ours~to significantly reduce finite sample (and potentially asymptotic) variance while attaining the semiparametric \emph{efficiency bound}---the smallest asymptotic variance among all consistent and asymptotically normal estimators of the average treatment effect---even when the foundation models are arbitrarily biased.


\paragraph{Integrating foundation models}
Prediction-powered inference~(\ppi)~\citep{angelopoulos2023prediction} is a statistical framework that constructs valid confidence intervals using a small labeled dataset and a large unlabeled dataset imputed by a foundation model. $\ppi$ has been applied in various domains, including generalization of causal inferences~\citep{demirel24prediction}, large language model evaluation~\citep{fisch2024stratified,dorner2024limitsscalableevaluationfrontier}, and improving the efficiency of social science experiments~\citep{broskamixed,egami2024using}. However, unlike our approach, $\ppi$ requires access to an additional unlabeled dataset from the same distribution as the experimental sample, which may be as costly as labeled data. Recent work by \citet{poulet2025prediction} introduces 
Prediction-powered inference for clinical trials ($\ppct$), an adaptation of $\ppi$ to estimate  average treatment effects in randomized experiments without any additional  external data. $\ppct$ combines the difference in means estimator with an 
$\aipw$ estimator that integrates the same foundation model as the outcome regression for both treatment and control groups. However, our work differs in two key aspects:
(i) $\ppct$ integrates a single foundation model, and (ii) $\ppct$ does not include the standard $\aipw$ estimator with the outcome regression estimated from experimental data. As a result, $\ppct$ cannot achieve the efficiency bound unless the foundation model is almost surely equal to the underlying outcome regression. 


 



\paragraph{Integrating observational data} There is growing interest in augmenting randomized experiments with data from observational studies to improve statistical precision. One approach involves first testing whether the observational data is compatible with the experimental data~\citep{dahabreh2024using}---for instance, using a statistical test to assess if the mean of the outcome conditional on the covariates is invariant across studies \cite{luedtke2019omnibus,hussain2023falsification,de2024detecting}—and then combining the datasets to improve precision, if the test does not reject. These tests, however, have low statistical power, especially when the experimental sample size is small, which is precisely when leveraging observational data would be most beneficial. To overcome this, a recent line of work integrates a prognostic score estimated from observational data as a covariate when estimating the outcome regression~\citep{schuler2022increasing,liao2023prognostic}. However, increasing the dimensionality of the problem---by adding an additional covariate---can increase estimation error and inflate the finite sample variance. Finally, the work most closely related to ours is \citet{karlsson2024robust}, that integrates an outcome regression estimated from observational data into the \aipw~estimator. In contrast, our approach is not constrained by the availability of well-structured observational data, since it leverages black-box foundation models trained on external data sources.
%%%%% Method
\section{Method}\label{sec:method}
\begin{figure}
    \centering
    \includegraphics[width=0.85\textwidth]{imgs/heatmap_acc.pdf}
    \caption{\textbf{Visualization of the proposed periodic Bayesian flow with mean parameter $\mu$ and accumulated accuracy parameter $c$ which corresponds to the entropy/uncertainty}. For $x = 0.3, \beta(1) = 1000$ and $\alpha_i$ defined in \cref{appd:bfn_cir}, this figure plots three colored stochastic parameter trajectories for receiver mean parameter $m$ and accumulated accuracy parameter $c$, superimposed on a log-scale heatmap of the Bayesian flow distribution $p_F(m|x,\senderacc)$ and $p_F(c|x,\senderacc)$. Note the \emph{non-monotonicity} and \emph{non-additive} property of $c$ which could inform the network the entropy of the mean parameter $m$ as a condition and the \emph{periodicity} of $m$. %\jj{Shrink the figures to save space}\hanlin{Do we need to make this figure one-column?}
    }
    \label{fig:vmbf_vis}
    \vskip -0.1in
\end{figure}
% \begin{wrapfigure}{r}{0.5\textwidth}
%     \centering
%     \includegraphics[width=0.49\textwidth]{imgs/heatmap_acc.pdf}
%     \caption{\textbf{Visualization of hyper-torus Bayesian flow based on von Mises Distribution}. For $x = 0.3, \beta(1) = 1000$ and $\alpha_i$ defined in \cref{appd:bfn_cir}, this figure plots three colored stochastic parameter trajectories for receiver mean parameter $m$ and accumulated accuracy parameter $c$, superimposed on a log-scale heatmap of the Bayesian flow distribution $p_F(m|x,\senderacc)$ and $p_F(c|x,\senderacc)$. Note the \emph{non-monotonicity} and \emph{non-additive} property of $c$. \jj{Shrink the figures to save space}}
%     \label{fig:vmbf_vis}
%     \vspace{-30pt}
% \end{wrapfigure}


In this section, we explain the detailed design of CrysBFN tackling theoretical and practical challenges. First, we describe how to derive our new formulation of Bayesian Flow Networks over hyper-torus $\mathbb{T}^{D}$ from scratch. Next, we illustrate the two key differences between \modelname and the original form of BFN: $1)$ a meticulously designed novel base distribution with different Bayesian update rules; and $2)$ different properties over the accuracy scheduling resulted from the periodicity and the new Bayesian update rules. Then, we present in detail the overall framework of \modelname over each manifold of the crystal space (\textit{i.e.} fractional coordinates, lattice vectors, atom types) respecting \textit{periodic E(3) invariance}. 

% In this section, we first demonstrate how to build Bayesian flow on hyper-torus $\mathbb{T}^{D}$ by overcoming theoretical and practical problems to provide a low-noise parameter-space approach to fractional atom coordinate generation. Next, we present how \modelname models each manifold of crystal space respecting \textit{periodic E(3) invariance}. 

\subsection{Periodic Bayesian Flow on Hyper-torus \texorpdfstring{$\mathbb{T}^{D}$}{}} 
For generative modeling of fractional coordinates in crystal, we first construct a periodic Bayesian flow on \texorpdfstring{$\mathbb{T}^{D}$}{} by designing every component of the totally new Bayesian update process which we demonstrate to be distinct from the original Bayesian flow (please see \cref{fig:non_add}). 
 %:) 
 
 The fractional atom coordinate system \citep{jiao2023crystal} inherently distributes over a hyper-torus support $\mathbb{T}^{3\times N}$. Hence, the normal distribution support on $\R$ used in the original \citep{bfn} is not suitable for this scenario. 
% The key problem of generative modeling for crystal is the periodicity of Cartesian atom coordinates $\vX$ requiring:
% \begin{equation}\label{eq:periodcity}
% p(\vA,\vL,\vX)=p(\vA,\vL,\vX+\vec{LK}),\text{where}~\vec{K}=\vec{k}\vec{1}_{1\times N},\forall\vec{k}\in\mathbb{Z}^{3\times1}
% \end{equation}
% However, there does not exist such a distribution supporting on $\R$ to model such property because the integration of such distribution over $\R$ will not be finite and equal to 1. Therefore, the normal distribution used in \citet{bfn} can not meet this condition.

To tackle this problem, the circular distribution~\citep{mardia2009directional} over the finite interval $[-\pi,\pi)$ is a natural choice as the base distribution for deriving the BFN on $\mathbb{T}^D$. 
% one natural choice is to 
% we would like to consider the circular distribution over the finite interval as the base 
% we find that circular distributions \citep{mardia2009directional} defined on a finite interval with lengths of $2\pi$ can be used as the instantiation of input distribution for the BFN on $\mathbb{T}^D$.
Specifically, circular distributions enjoy desirable periodic properties: $1)$ the integration over any interval length of $2\pi$ equals 1; $2)$ the probability distribution function is periodic with period $2\pi$.  Sharing the same intrinsic with fractional coordinates, such periodic property of circular distribution makes it suitable for the instantiation of BFN's input distribution, in parameterizing the belief towards ground truth $\x$ on $\mathbb{T}^D$. 
% \yuxuan{this is very complicated from my perspective.} \hanlin{But this property is exactly beautiful and perfectly fit into the BFN.}

\textbf{von Mises Distribution and its Bayesian Update} We choose von Mises distribution \citep{mardia2009directional} from various circular distributions as the form of input distribution, based on the appealing conjugacy property required in the derivation of the BFN framework.
% to leverage the Bayesian conjugacy property of von Mises distribution which is required by the BFN framework. 
That is, the posterior of a von Mises distribution parameterized likelihood is still in the family of von Mises distributions. The probability density function of von Mises distribution with mean direction parameter $m$ and concentration parameter $c$ (describing the entropy/uncertainty of $m$) is defined as: 
\begin{equation}
f(x|m,c)=vM(x|m,c)=\frac{\exp(c\cos(x-m))}{2\pi I_0(c)}
\end{equation}
where $I_0(c)$ is zeroth order modified Bessel function of the first kind as the normalizing constant. Given the last univariate belief parameterized by von Mises distribution with parameter $\theta_{i-1}=\{m_{i-1},\ c_{i-1}\}$ and the sample $y$ from sender distribution with unknown data sample $x$ and known accuracy $\alpha$ describing the entropy/uncertainty of $y$,  Bayesian update for the receiver is deducted as:
\begin{equation}
 h(\{m_{i-1},c_{i-1}\},y,\alpha)=\{m_i,c_i \}, \text{where}
\end{equation}
\begin{equation}\label{eq:h_m}
m_i=\text{atan2}(\alpha\sin y+c_{i-1}\sin m_{i-1}, {\alpha\cos y+c_{i-1}\cos m_{i-1}})
\end{equation}
\begin{equation}\label{eq:h_c}
c_i =\sqrt{\alpha^2+c_{i-1}^2+2\alpha c_{i-1}\cos(y-m_{i-1})}
\end{equation}
The proof of the above equations can be found in \cref{apdx:bayesian_update_function}. The atan2 function refers to  2-argument arctangent. Independently conducting  Bayesian update for each dimension, we can obtain the Bayesian update distribution by marginalizing $\y$:
\begin{equation}
p_U(\vtheta'|\vtheta,\bold{x};\alpha)=\mathbb{E}_{p_S(\bold{y}|\bold{x};\alpha)}\delta(\vtheta'-h(\vtheta,\bold{y},\alpha))=\mathbb{E}_{vM(\bold{y}|\bold{x},\alpha)}\delta(\vtheta'-h(\vtheta,\bold{y},\alpha))
\end{equation} 
\begin{figure}
    \centering
    \vskip -0.15in
    \includegraphics[width=0.95\linewidth]{imgs/non_add.pdf}
    \caption{An intuitive illustration of non-additive accuracy Bayesian update on the torus. The lengths of arrows represent the uncertainty/entropy of the belief (\emph{e.g.}~$1/\sigma^2$ for Gaussian and $c$ for von Mises). The directions of the arrows represent the believed location (\emph{e.g.}~ $\mu$ for Gaussian and $m$ for von Mises).}
    \label{fig:non_add}
    \vskip -0.15in
\end{figure}
\textbf{Non-additive Accuracy} 
The additive accuracy is a nice property held with the Gaussian-formed sender distribution of the original BFN expressed as:
\begin{align}
\label{eq:standard_id}
    \update(\parsn{}'' \mid \parsn{}, \x; \alpha_a+\alpha_b) = \E_{\update(\parsn{}' \mid \parsn{}, \x; \alpha_a)} \update(\parsn{}'' \mid \parsn{}', \x; \alpha_b)
\end{align}
Such property is mainly derived based on the standard identity of Gaussian variable:
\begin{equation}
X \sim \mathcal{N}\left(\mu_X, \sigma_X^2\right), Y \sim \mathcal{N}\left(\mu_Y, \sigma_Y^2\right) \Longrightarrow X+Y \sim \mathcal{N}\left(\mu_X+\mu_Y, \sigma_X^2+\sigma_Y^2\right)
\end{equation}
The additive accuracy property makes it feasible to derive the Bayesian flow distribution $
p_F(\boldsymbol{\theta} \mid \mathbf{x} ; i)=p_U\left(\boldsymbol{\theta} \mid \boldsymbol{\theta}_0, \mathbf{x}, \sum_{k=1}^{i} \alpha_i \right)
$ for the simulation-free training of \cref{eq:loss_n}.
It should be noted that the standard identity in \cref{eq:standard_id} does not hold in the von Mises distribution. Hence there exists an important difference between the original Bayesian flow defined on Euclidean space and the Bayesian flow of circular data on $\mathbb{T}^D$ based on von Mises distribution. With prior $\btheta = \{\bold{0},\bold{0}\}$, we could formally represent the non-additive accuracy issue as:
% The additive accuracy property implies the fact that the "confidence" for the data sample after observing a series of the noisy samples with accuracy ${\alpha_1, \cdots, \alpha_i}$ could be  as the accuracy sum  which could be  
% Here we 
% Here we emphasize the specific property of BFN based on von Mises distribution.
% Note that 
% \begin{equation}
% \update(\parsn'' \mid \parsn, \x; \alpha_a+\alpha_b) \ne \E_{\update(\parsn' \mid \parsn, \x; \alpha_a)} \update(\parsn'' \mid \parsn', \x; \alpha_b)
% \end{equation}
% \oyyw{please check whether the below equation is better}
% \yuxuan{I fill somehow confusing on what is the update distribution with $\alpha$. }
% \begin{equation}
% \update(\parsn{}'' \mid \parsn{}, \x; \alpha_a+\alpha_b) \ne \E_{\update(\parsn{}' \mid \parsn{}, \x; \alpha_a)} \update(\parsn{}'' \mid \parsn{}', \x; \alpha_b)
% \end{equation}
% We give an intuitive visualization of such difference in \cref{fig:non_add}. The untenability of this property can materialize by considering the following case: with prior $\btheta = \{\bold{0},\bold{0}\}$, check the two-step Bayesian update distribution with $\alpha_a,\alpha_b$ and one-step Bayesian update with $\alpha=\alpha_a+\alpha_b$:
\begin{align}
\label{eq:nonadd}
     &\update(c'' \mid \parsn, \x; \alpha_a+\alpha_b)  = \delta(c-\alpha_a-\alpha_b)
     \ne  \mathbb{E}_{p_U(\parsn' \mid \parsn, \x; \alpha_a)}\update(c'' \mid \parsn', \x; \alpha_b) \nonumber \\&= \mathbb{E}_{vM(\bold{y}_b|\bold{x},\alpha_a)}\mathbb{E}_{vM(\bold{y}_a|\bold{x},\alpha_b)}\delta(c-||[\alpha_a \cos\y_a+\alpha_b\cos \y_b,\alpha_a \sin\y_a+\alpha_b\sin \y_b]^T||_2)
\end{align}
A more intuitive visualization could be found in \cref{fig:non_add}. This fundamental difference between periodic Bayesian flow and that of \citet{bfn} presents both theoretical and practical challenges, which we will explain and address in the following contents.

% This makes constructing Bayesian flow based on von Mises distribution intrinsically different from previous Bayesian flows (\citet{bfn}).

% Thus, we must reformulate the framework of Bayesian flow networks  accordingly. % and do necessary reformulations of BFN. 

% \yuxuan{overall I feel this part is complicated by using the language of update distribution. I would like to suggest simply use bayesian update, to provide intuitive explantion.}\hanlin{See the illustration in \cref{fig:non_add}}

% That introduces a cascade of problems, and we investigate the following issues: $(1)$ Accuracies between sender and receiver are not synchronized and need to be differentiated. $(2)$ There is no tractable Bayesian flow distribution for a one-step sample conditioned on a given time step $i$, and naively simulating the Bayesian flow results in computational overhead. $(3)$ It is difficult to control the entropy of the Bayesian flow. $(4)$ Accuracy is no longer a function of $t$ and becomes a distribution conditioned on $t$, which can be different across dimensions.
%\jj{Edited till here}

\textbf{Entropy Conditioning} As a common practice in generative models~\citep{ddpm,flowmatching,bfn}, timestep $t$ is widely used to distinguish among generation states by feeding the timestep information into the networks. However, this paper shows that for periodic Bayesian flow, the accumulated accuracy $\vc_i$ is more effective than time-based conditioning by informing the network about the entropy and certainty of the states $\parsnt{i}$. This stems from the intrinsic non-additive accuracy which makes the receiver's accumulated accuracy $c$ not bijective function of $t$, but a distribution conditioned on accumulated accuracies $\vc_i$ instead. Therefore, the entropy parameter $\vc$ is taken logarithm and fed into the network to describe the entropy of the input corrupted structure. We verify this consideration in \cref{sec:exp_ablation}. 
% \yuxuan{implement variant. traditionally, the timestep is widely used to distinguish the different states by putting the timestep embedding into the networks. citation of FM, diffusion, BFN. However, we find that conditioned on time in periodic flow could not provide extra benefits. To further boost the performance, we introduce a simple yet effective modification term entropy conditional. This is based on that the accumulated accuracy which represents the current uncertainty or entropy could be a better indicator to distinguish different states. + Describe how you do this. }



\textbf{Reformulations of BFN}. Recall the original update function with Gaussian sender distribution, after receiving noisy samples $\y_1,\y_2,\dots,\y_i$ with accuracies $\senderacc$, the accumulated accuracies of the receiver side could be analytically obtained by the additive property and it is consistent with the sender side.
% Since observing sample $\y$ with $\alpha_i$ can not result in exact accuracy increment $\alpha_i$ for receiver, the accuracies between sender and receiver are not synchronized which need to be differentiated. 
However, as previously mentioned, this does not apply to periodic Bayesian flow, and some of the notations in original BFN~\citep{bfn} need to be adjusted accordingly. We maintain the notations of sender side's one-step accuracy $\alpha$ and added accuracy $\beta$, and alter the notation of receiver's accuracy parameter as $c$, which is needed to be simulated by cascade of Bayesian updates. We emphasize that the receiver's accumulated accuracy $c$ is no longer a function of $t$ (differently from the Gaussian case), and it becomes a distribution conditioned on received accuracies $\senderacc$ from the sender. Therefore, we represent the Bayesian flow distribution of von Mises distribution as $p_F(\btheta|\x;\alpha_1,\alpha_2,\dots,\alpha_i)$. And the original simulation-free training with Bayesian flow distribution is no longer applicable in this scenario.
% Different from previous BFNs where the accumulated accuracy $\rho$ is not explicitly modeled, the accumulated accuracy parameter $c$ (visualized in \cref{fig:vmbf_vis}) needs to be explicitly modeled by feeding it to the network to avoid information loss.
% the randomaccuracy parameter $c$ (visualized in \cref{fig:vmbf_vis}) implies that there exists information in $c$ from the sender just like $m$, meaning that $c$ also should be fed into the network to avoid information loss. 
% We ablate this consideration in  \cref{sec:exp_ablation}. 

\textbf{Fast Sampling from Equivalent Bayesian Flow Distribution} Based on the above reformulations, the Bayesian flow distribution of von Mises distribution is reframed as: 
\begin{equation}\label{eq:flow_frac}
p_F(\btheta_i|\x;\alpha_1,\alpha_2,\dots,\alpha_i)=\E_{\update(\parsnt{1} \mid \parsnt{0}, \x ; \alphat{1})}\dots\E_{\update(\parsn_{i-1} \mid \parsnt{i-2}, \x; \alphat{i-1})} \update(\parsnt{i} | \parsnt{i-1},\x;\alphat{i} )
\end{equation}
Naively sampling from \cref{eq:flow_frac} requires slow auto-regressive iterated simulation, making training unaffordable. Noticing the mathematical properties of \cref{eq:h_m,eq:h_c}, we  transform \cref{eq:flow_frac} to the equivalent form:
\begin{equation}\label{eq:cirflow_equiv}
p_F(\vec{m}_i|\x;\alpha_1,\alpha_2,\dots,\alpha_i)=\E_{vM(\y_1|\x,\alpha_1)\dots vM(\y_i|\x,\alpha_i)} \delta(\vec{m}_i-\text{atan2}(\sum_{j=1}^i \alpha_j \cos \y_j,\sum_{j=1}^i \alpha_j \sin \y_j))
\end{equation}
\begin{equation}\label{eq:cirflow_equiv2}
p_F(\vec{c}_i|\x;\alpha_1,\alpha_2,\dots,\alpha_i)=\E_{vM(\y_1|\x,\alpha_1)\dots vM(\y_i|\x,\alpha_i)}  \delta(\vec{c}_i-||[\sum_{j=1}^i \alpha_j \cos \y_j,\sum_{j=1}^i \alpha_j \sin \y_j]^T||_2)
\end{equation}
which bypasses the computation of intermediate variables and allows pure tensor operations, with negligible computational overhead.
\begin{restatable}{proposition}{cirflowequiv}
The probability density function of Bayesian flow distribution defined by \cref{eq:cirflow_equiv,eq:cirflow_equiv2} is equivalent to the original definition in \cref{eq:flow_frac}. 
\end{restatable}
\textbf{Numerical Determination of Linear Entropy Sender Accuracy Schedule} ~Original BFN designs the accuracy schedule $\beta(t)$ to make the entropy of input distribution linearly decrease. As for crystal generation task, to ensure information coherence between modalities, we choose a sender accuracy schedule $\senderacc$ that makes the receiver's belief entropy $H(t_i)=H(p_I(\cdot|\vtheta_i))=H(p_I(\cdot|\vc_i))$ linearly decrease \emph{w.r.t.} time $t_i$, given the initial and final accuracy parameter $c(0)$ and $c(1)$. Due to the intractability of \cref{eq:vm_entropy}, we first use numerical binary search in $[0,c(1)]$ to determine the receiver's $c(t_i)$ for $i=1,\dots, n$ by solving the equation $H(c(t_i))=(1-t_i)H(c(0))+tH(c(1))$. Next, with $c(t_i)$, we conduct numerical binary search for each $\alpha_i$ in $[0,c(1)]$ by solving the equations $\E_{y\sim vM(x,\alpha_i)}[\sqrt{\alpha_i^2+c_{i-1}^2+2\alpha_i c_{i-1}\cos(y-m_{i-1})}]=c(t_i)$ from $i=1$ to $i=n$ for arbitrarily selected $x\in[-\pi,\pi)$.

After tackling all those issues, we have now arrived at a new BFN architecture for effectively modeling crystals. Such BFN can also be adapted to other type of data located in hyper-torus $\mathbb{T}^{D}$.

\subsection{Equivariant Bayesian Flow for Crystal}
With the above Bayesian flow designed for generative modeling of fractional coordinate $\vF$, we are able to build equivariant Bayesian flow for each modality of crystal. In this section, we first give an overview of the general training and sampling algorithm of \modelname (visualized in \cref{fig:framework}). Then, we describe the details of the Bayesian flow of every modality. The training and sampling algorithm can be found in \cref{alg:train} and \cref{alg:sampling}.

\textbf{Overview} Operating in the parameter space $\bthetaM=\{\bthetaA,\bthetaL,\bthetaF\}$, \modelname generates high-fidelity crystals through a joint BFN sampling process on the parameter of  atom type $\bthetaA$, lattice parameter $\vec{\theta}^L=\{\bmuL,\brhoL\}$, and the parameter of fractional coordinate matrix $\bthetaF=\{\bmF,\bcF\}$. We index the $n$-steps of the generation process in a discrete manner $i$, and denote the corresponding continuous notation $t_i=i/n$ from prior parameter $\thetaM_0$ to a considerably low variance parameter $\thetaM_n$ (\emph{i.e.} large $\vrho^L,\bmF$, and centered $\bthetaA$).

At training time, \modelname samples time $i\sim U\{1,n\}$ and $\bthetaM_{i-1}$ from the Bayesian flow distribution of each modality, serving as the input to the network. The network $\net$ outputs $\net(\parsnt{i-1}^\mathcal{M},t_{i-1})=\net(\parsnt{i-1}^A,\parsnt{i-1}^F,\parsnt{i-1}^L,t_{i-1})$ and conducts gradient descents on loss function \cref{eq:loss_n} for each modality. After proper training, the sender distribution $p_S$ can be approximated by the receiver distribution $p_R$. 

At inference time, from predefined $\thetaM_0$, we conduct transitions from $\thetaM_{i-1}$ to $\thetaM_{i}$ by: $(1)$ sampling $\y_i\sim p_R(\bold{y}|\thetaM_{i-1};t_i,\alpha_i)$ according to network prediction $\predM{i-1}$; and $(2)$ performing Bayesian update $h(\thetaM_{i-1},\y^\calM_{i-1},\alpha_i)$ for each dimension. 

% Alternatively, we complete this transition using the flow-back technique by sampling 
% $\thetaM_{i}$ from Bayesian flow distribution $\flow(\btheta^M_{i}|\predM{i-1};t_{i-1})$. 

% The training objective of $\net$ is to minimize the KL divergence between sender distribution and receiver distribution for every modality as defined in \cref{eq:loss_n} which is equivalent to optimizing the negative variational lower bound $\calL^{VLB}$ as discussed in \cref{sec:preliminaries}. 

%In the following part, we will present the Bayesian flow of each modality in detail.

\textbf{Bayesian Flow of Fractional Coordinate $\vF$}~The distribution of the prior parameter $\bthetaF_0$ is defined as:
\begin{equation}\label{eq:prior_frac}
    p(\bthetaF_0) \defeq \{vM(\vm_0^F|\vec{0}_{3\times N},\vec{0}_{3\times N}),\delta(\vc_0^F-\vec{0}_{3\times N})\} = \{U(\vec{0},\vec{1}),\delta(\vc_0^F-\vec{0}_{3\times N})\}
\end{equation}
Note that this prior distribution of $\vm_0^F$ is uniform over $[\vec{0},\vec{1})$, ensuring the periodic translation invariance property in \cref{De:pi}. The training objective is minimizing the KL divergence between sender and receiver distribution (deduction can be found in \cref{appd:cir_loss}): 
%\oyyw{replace $\vF$ with $\x$?} \hanlin{notations follow Preliminary?}
\begin{align}\label{loss_frac}
\calL_F = n \E_{i \sim \ui{n}, \flow(\parsn{}^F \mid \vF ; \senderacc)} \alpha_i\frac{I_1(\alpha_i)}{I_0(\alpha_i)}(1-\cos(\vF-\predF{i-1}))
\end{align}
where $I_0(x)$ and $I_1(x)$ are the zeroth and the first order of modified Bessel functions. The transition from $\bthetaF_{i-1}$ to $\bthetaF_{i}$ is the Bayesian update distribution based on network prediction:
\begin{equation}\label{eq:transi_frac}
    p(\btheta^F_{i}|\parsnt{i-1}^\calM)=\mathbb{E}_{vM(\bold{y}|\predF{i-1},\alpha_i)}\delta(\btheta^F_{i}-h(\btheta^F_{i-1},\bold{y},\alpha_i))
\end{equation}
\begin{restatable}{proposition}{fracinv}
With $\net_{F}$ as a periodic translation equivariant function namely $\net_F(\parsnt{}^A,w(\parsnt{}^F+\vt),\parsnt{}^L,t)=w(\net_F(\parsnt{}^A,\parsnt{}^F,\parsnt{}^L,t)+\vt), \forall\vt\in\R^3$, the marginal distribution of $p(\vF_n)$ defined by \cref{eq:prior_frac,eq:transi_frac} is periodic translation invariant. 
\end{restatable}
\textbf{Bayesian Flow of Lattice Parameter \texorpdfstring{$\boldsymbol{L}$}{}}   
Noting the lattice parameter $\bm{L}$ located in Euclidean space, we set prior as the parameter of a isotropic multivariate normal distribution $\btheta^L_0\defeq\{\vmu_0^L,\vrho_0^L\}=\{\bm{0}_{3\times3},\bm{1}_{3\times3}\}$
% \begin{equation}\label{eq:lattice_prior}
% \btheta^L_0\defeq\{\vmu_0^L,\vrho_0^L\}=\{\bm{0}_{3\times3},\bm{1}_{3\times3}\}
% \end{equation}
such that the prior distribution of the Markov process on $\vmu^L$ is the Dirac distribution $\delta(\vec{\mu_0}-\vec{0})$ and $\delta(\vec{\rho_0}-\vec{1})$, 
% \begin{equation}
%     p_I^L(\boldsymbol{L}|\btheta_0^L)=\mathcal{N}(\bm{L}|\bm{0},\bm{I})
% \end{equation}
which ensures O(3)-invariance of prior distribution of $\vL$. By Eq. 77 from \citet{bfn}, the Bayesian flow distribution of the lattice parameter $\bm{L}$ is: 
\begin{align}% =p_U(\bmuL|\btheta_0^L,\bm{L},\beta(t))
p_F^L(\bmuL|\bm{L};t) &=\mathcal{N}(\bmuL|\gamma(t)\bm{L},\gamma(t)(1-\gamma(t))\bm{I}) 
\end{align}
where $\gamma(t) = 1 - \sigma_1^{2t}$ and $\sigma_1$ is the predefined hyper-parameter controlling the variance of input distribution at $t=1$ under linear entropy accuracy schedule. The variance parameter $\vrho$ does not need to be modeled and fed to the network, since it is deterministic given the accuracy schedule. After sampling $\bmuL_i$ from $p_F^L$, the training objective is defined as minimizing KL divergence between sender and receiver distribution (based on Eq. 96 in \citet{bfn}):
\begin{align}
\mathcal{L}_{L} = \frac{n}{2}\left(1-\sigma_1^{2/n}\right)\E_{i \sim \ui{n}}\E_{\flow(\bmuL_{i-1} |\vL ; t_{i-1})}  \frac{\left\|\vL -\predL{i-1}\right\|^2}{\sigma_1^{2i/n}},\label{eq:lattice_loss}
\end{align}
where the prediction term $\predL{i-1}$ is the lattice parameter part of network output. After training, the generation process is defined as the Bayesian update distribution given network prediction:
\begin{equation}\label{eq:lattice_sampling}
    p(\bmuL_{i}|\parsnt{i-1}^\calM)=\update^L(\bmuL_{i}|\predL{i-1},\bmuL_{i-1};t_{i-1})
\end{equation}
    

% The final prediction of the lattice parameter is given by $\bmuL_n = \predL{n-1}$.
% \begin{equation}\label{eq:final_lattice}
%     \bmuL_n = \predL{n-1}
% \end{equation}

\begin{restatable}{proposition}{latticeinv}\label{prop:latticeinv}
With $\net_{L}$ as  O(3)-equivariant function namely $\net_L(\parsnt{}^A,\parsnt{}^F,\vQ\parsnt{}^L,t)=\vQ\net_L(\parsnt{}^A,\parsnt{}^F,\parsnt{}^L,t),\forall\vQ^T\vQ=\vI$, the marginal distribution of $p(\bmuL_n)$ defined by \cref{eq:lattice_sampling} is O(3)-invariant. 
\end{restatable}


\textbf{Bayesian Flow of Atom Types \texorpdfstring{$\boldsymbol{A}$}{}} 
Given that atom types are discrete random variables located in a simplex $\calS^K$, the prior parameter of $\boldsymbol{A}$ is the discrete uniform distribution over the vocabulary $\parsnt{0}^A \defeq \frac{1}{K}\vec{1}_{1\times N}$. 
% \begin{align}\label{eq:disc_input_prior}
% \parsnt{0}^A \defeq \frac{1}{K}\vec{1}_{1\times N}
% \end{align}
% \begin{align}
%     (\oh{j}{K})_k \defeq \delta_{j k}, \text{where }\oh{j}{K}\in \R^{K},\oh{\vA}{KD} \defeq \left(\oh{a_1}{K},\dots,\oh{a_N}{K}\right) \in \R^{K\times N}
% \end{align}
With the notation of the projection from the class index $j$ to the length $K$ one-hot vector $ (\oh{j}{K})_k \defeq \delta_{j k}, \text{where }\oh{j}{K}\in \R^{K},\oh{\vA}{KD} \defeq \left(\oh{a_1}{K},\dots,\oh{a_N}{K}\right) \in \R^{K\times N}$, the Bayesian flow distribution of atom types $\vA$ is derived in \citet{bfn}:
\begin{align}
\flow^{A}(\parsn^A \mid \vA; t) &= \E_{\N{\y \mid \beta^A(t)\left(K \oh{\vA}{K\times N} - \vec{1}_{K\times N}\right)}{\beta^A(t) K \vec{I}_{K\times N \times N}}} \delta\left(\parsn^A - \frac{e^{\y}\parsnt{0}^A}{\sum_{k=1}^K e^{\y_k}(\parsnt{0})_{k}^A}\right).
\end{align}
where $\beta^A(t)$ is the predefined accuracy schedule for atom types. Sampling $\btheta_i^A$ from $p_F^A$ as the training signal, the training objective is the $n$-step discrete-time loss for discrete variable \citep{bfn}: 
% \oyyw{can we simplify the next equation? Such as remove $K \times N, K \times N \times N$}
% \begin{align}
% &\calL_A = n\E_{i \sim U\{1,n\},\flow^A(\parsn^A \mid \vA ; t_{i-1}),\N{\y \mid \alphat{i}\left(K \oh{\vA}{KD} - \vec{1}_{K\times N}\right)}{\alphat{i} K \vec{I}_{K\times N \times N}}} \ln \N{\y \mid \alphat{i}\left(K \oh{\vA}{K\times N} - \vec{1}_{K\times N}\right)}{\alphat{i} K \vec{I}_{K\times N \times N}}\nonumber\\
% &\qquad\qquad\qquad-\sum_{d=1}^N \ln \left(\sum_{k=1}^K \out^{(d)}(k \mid \parsn^A; t_{i-1}) \N{\ydd{d} \mid \alphat{i}\left(K\oh{k}{K}- \vec{1}_{K\times N}\right)}{\alphat{i} K \vec{I}_{K\times N \times N}}\right)\label{discdisc_t_loss_exp}
% \end{align}
\begin{align}
&\calL_A = n\E_{i \sim U\{1,n\},\flow^A(\parsn^A \mid \vA ; t_{i-1}),\N{\y \mid \alphat{i}\left(K \oh{\vA}{KD} - \vec{1}\right)}{\alphat{i} K \vec{I}}} \ln \N{\y \mid \alphat{i}\left(K \oh{\vA}{K\times N} - \vec{1}\right)}{\alphat{i} K \vec{I}}\nonumber\\
&\qquad\qquad\qquad-\sum_{d=1}^N \ln \left(\sum_{k=1}^K \out^{(d)}(k \mid \parsn^A; t_{i-1}) \N{\ydd{d} \mid \alphat{i}\left(K\oh{k}{K}- \vec{1}\right)}{\alphat{i} K \vec{I}}\right)\label{discdisc_t_loss_exp}
\end{align}
where $\vec{I}\in \R^{K\times N \times N}$ and $\vec{1}\in\R^{K\times D}$. When sampling, the transition from $\bthetaA_{i-1}$ to $\bthetaA_{i}$ is derived as:
\begin{equation}
    p(\btheta^A_{i}|\parsnt{i-1}^\calM)=\update^A(\btheta^A_{i}|\btheta^A_{i-1},\predA{i-1};t_{i-1})
\end{equation}

The detailed training and sampling algorithm could be found in \cref{alg:train} and \cref{alg:sampling}.





%%%% Guidelines
\section{Recommendations}
\label{sec:guidelines}

\begin{figure*}
  \includegraphics[width=0.75\textwidth]{overview_erse_V4.pdf}
  \caption{Overview of all ten recommendations}
    \Description{overview}
  \label{fig:over}
\end{figure*}

\autoref{fig:over} provides an overview on our recommendations for engineering \ac{ERS}. In this section, we present each recommendation in detail by providing some background on why the recommendation is important, as well as specific advice. The recommendations are ordered by required effort, so the first ones are easier to include in your daily routines than the later ones. Some of the later recommendation may also be more relevant for larger software projects or senior researchers participating in writing research project proposals. While the recommendations give first advice for each topic, we also recommend learning and collecting further information on each topic, e.g., with tutorials.

\subsection{Inventing is great, reinventing is a waste of time}
\label{sec:reuseAndExtend}

\paragraph{Background} The core of science is to build on top of other's ideas and contributions. Especially in recent years, more and more research software projects are released as open-source and can be extended and reused by the community. Still, we often feel the necessity to start from scratch and build our own library or framework. For example, a recent review identified 63 different open-source, optimization-based frameworks for energy system modeling~\cite{hoffmann_review_2024}. They all share the same underlying principle of network-based energy flow optimization with similar architectures. Many frameworks are only used by their home institutions, which indicates possible redundancy \cite{hoffmann_review_2024}. However, writing code is only one of our many diverse responsibilities.  
Overall, researchers have too little time to reimplement work already done by others. Instead, we should aim to reuse and extend existing high-quality software and focus on implementing novel research contributions. While familiarizing oneself with the documentation of other tools seems more time-consuming in the beginning, this time will be saved later on because existing code often already comes with a lot of testing, verification, and documentation which you would otherwise need to create for your new code.
This recommendation discusses creating high-quality research software without implementing everything from scratch. 

\paragraph{Recommendations} The first step of reusing software is to find existing software that solves your problem or parts of it. The natural approach is to do a literature survey beforehand. While this is an important step, not all software is published as a conference or journal publication. Instead, software registries and similar platforms can be used to find research software. For energy research,  the Open Energy Platform\footnote{\label{fn:oep}\url{https://openenergyplatform.org/}, last access 2024-12-04.} and DOE CODE\footnote{\url{https://www.osti.gov/doecode/}m last access 2025-01-20} are available. Additionally, the Helmholtz Research Software Directory\footnote{\url{https://helmholtz.software/software?&keywords=\%5B\%22Energy\%22\%5D\&page=0\&rows=12}, last access 2024-12-04.} and the \ac{JOSS}\footnote{\label{fn:joss}\url{https://joss.theoj.org}, last access 2024-12-16.} list research software without a domain focus. 
Software repositories like GitHub\footnote{\url{https://github.com/search?q=\%22power\%20system\%22\&type=repositories}, last access 2024-12-04.} can also be searched. In these cases, energy-specific keywords should be used.
The final recommendation for finding reusable software is to ask experienced colleagues and other researchers who work on similar problems. They can often tell you about hitherto unknown tools, e.g., those under development in other research projects, and which libraries are useful.
\par 
To use external software for your research project, you should check their license and see if it is compatible with your code and its envisioned use case (see also \ref{sec:openSource}). 
If the software does not entirely cover your use case, you can contact the software maintainers. They may be already discussing and planning the feature you need. If they do not want to implement it they may wish to include your software when it is finished. By explicitly communicating with other researchers, redundant software can be prevented, and as a side effect, overall collaboration in science is strengthened. This \textit{collaboration first} thinking should follow through the whole software project. For example, write issues and pull requests when finding bugs in other software you use. Not only does this help other people, but it also improves your software as a side effect.

\par 
However, there are exceptions to the rule of reusing existing software. When the existing solutions are of low quality or outdated, starting a new software from scratch can be a good idea. Also, developing an open-source alternative can have a significant scientific impact if the existing software is proprietary. 
In general, parallel development is perfectly fine to some extent. It results in competition and provides other researchers with options that have respective pros and cons.
\par 
Reusing existing research software results in higher-quality software for less effort. It accelerates scientific progress and supports overall collaboration. Hence, the following recommendation: 
\recommendation{Reuse and extend other software if possible and useful!}     

\subsection{Track the history and learn for the future}\label{sec:vcs}

\paragraph{Background} Software is constantly changing. New features are implemented, bugs are fixed, and code is improved. Sometimes, these developments also happen in parallel. For example, you are implementing a physical model to analyze the operating behavior of a battery, and you supervise a student tasked to implement a thermal model for the battery and, therefore, give them a copy of your code. When the student finishes their implementation after a couple of months, your own code has evolved, and manually integrating the master student’s code back into your own becomes a messy task of copy, paste, break, fix, test, and retest. This, and many other problems associated with changing code, multiple versions, and coding as a team, can be avoided with version control. While using version control is standard practice in software development, it is not always the case in research software. Tracking different versions of your code enables you and others to keep an overview of which feature was developed and to understand the history of the code, thereby increasing its maintainability. It also allows for the undoing of specific changes when a problem occurs. Additionally, version control enables reproducibility of the research conducted with the code by providing specific links to use  exactly the same version~\cite{sandve2013ten}.

\paragraph{Recommendations} Version control systems can help you track different software versions. The most popular version control system is Git\footnote{\url{https://git-scm.com/}, last access 2024-12-17.}.
The platforms GitHub\footnote{\url{https://github.com/}, last access 2024-12-17.} and GitLab\footnote{\url{https://about.gitlab.com/}, last access 2024-12-18} are based on Git and extend it by a user interface providing a good overview and additional features. If your institution does not allow you to use GitHub or the public GitLab, you can often also use a GitLab instance offered by your institution.

Within Git, changes to the software are included as a \textit{commit}. By adding meaningful messages to each commit, the changes are documented. At least one commit at the end of each day is recommended. 

Version control systems also allow different \textit{branches}, which are parallel versions of the same software. It is helpful to always have a working software version in one branch (usually called \textit{master} or \textit{main}). Also, this way, different developers can work in parallel without directly causing conflicts. A branch can be merged into another by a \textit{pull} or \textit{merge request}. These requests are another good option for additional comments and documentation to make the changes more understandable. In software projects with multiple developers, this is also the step where code reviews should be conducted for quality assurance.

The version control platforms also have additional features that make it easier to organize the software development. \textit{Issues} can be used to track bugs, code contributions, and new potential feature ideas. They also allow others to get an overview of what is planned next. Additionally, they provide options for \ac{CICD}, which is the practice of automating the integration, testing, and deployment of code changes to ensure reliable and efficient software delivery. In the context of research software, \ac{CICD} is especially used for automated testing of each commit (see also \ref{sec:testing}).

\par 
Version control systems are necessary to keep track of software versions while also allowing to organize the software development to a certain extent. If you are not familiar with version control, we recommend taking some basic training on it, e.g., by the software carpentry\footnote{\url{https://swcarpentry.github.io/git-novice/index.html}, last access 2024-12-17}.  

\recommendation{Use version control!}
\subsection{Research should be publicly available, and your software is part of it}\label{sec:openSource}

\paragraph{Background} Only openly available research software allows other researchers to reproduce your research results. Since research software is mainly publicly funded, it is also fair to make it publicly available. This also enables others to reuse the software, providing additional benefits like further testing of the software and contributing by creating extensions and compatible software. Besides making the software open source, it is also important to register it so other researchers can find it. These aspects are essential to make a research software \ac{FAIR}~\cite{barker2022introducing}. 

\paragraph{Recommendations} We recommend developing your research software open source. By directly starting the development as open source, challenges can be avoided when switching to open source later. Further, it allows other researchers to use your software and actively contribute as early as possible, improving your software's quality and encouraging cooperation. To develop the software open source, a suitable license is required. By choosing a license early each time another software is included, the compatibility of licenses can be checked directly. Some open source licenses are incompatible since they put certain conditions on the reuse of the code, which can conflict between licenses \cite{cui_empirical_2023}.
When choosing a license, you should check your institution's policy, which often already proposes a specific license. Also, various guides help to choose a license\footnote{e.g., \url{https://choosealicense.com/}, last access 2024-12-17.}.

Cooperation and joint research projects with industry are common in energy research. Sometimes, industry partners are critical of developing open source because they want to keep their intellectual property. Generally, we recommend discussing this topic as early as possible to find reasonable compromises. Often, certain parts can be developed open source while others remain closed source. Since open source is open to all researchers and all industries, it can also improve the exchange between industry and research. 

When you develop open source, there are specific approaches to make your code more easily reusable by others. First, the repository should follow programming-language-specific best practices. This can be achieved by starting with templates for the repository\footnote{e.g., cookie-cutter templates at \url{https://cookiecutter.io/templates}, last access 2024-12-17}. Additionally, citing your software can be made easy by including a \ac{CFF} file\footnote{\url{https://citation-file-format.github.io}, last access 2024-12-17.} in your repository
 \cite{druskat_citation_2021}.

Within the repository, it should be indicated if the software will be maintained, if support is available and if the developers are open to joining research projects, including the software. We recommend using the features of the software platform (GitHub/GitLab) to interact with potential users, e.g., by using issues.

Besides making the source code available, the software should also be findable. Therefore, we recommend registering the software in a domain-specific registry like the Open Energy Platform\footref{fn:oep}. Getting a DOI for the software is also helpful, e.g., by archiving versions of the software on Zenodo\footnote{\url{https://zenodo.org/}, last access 2024-12-17.} \cite{zenodo}. 

\par
Energy research is highly interdisciplinary \cite{tijssen_quantitative_1992}.
Therefore, we recommend adding a very general description to your software, allowing all researchers to understand its goals. This way, more researchers can identify whether the software is useful for them, increasing its reusability. GitHub also allows you to provide keywords to your repository, which again improves findability.


\par 
Platforms like GitHub and GitLab make it easy to publish your software under an open source license. Open source research software enables reproducibility and allows reusability, which can also improve your code quality. Therefore, we recommend: 

\recommendation{Develop open source \& make your software findable!}
\subsection{Being organized boosts your software project}
\label{sec:workProcess}

\paragraph{Background} This recommendation discusses how research software development teams should organize themselves.

The most important aspect here is that there is no one-size-fits-all solution.
A one-person PhD software project should be organized differently than a large team developing software with thousands of potential users.

\paragraph{Recommendations} The main recommendation is to explicitly define the software project's scope and purpose early in the process. What is the software's general use case (see \ref{sec:architecture})? How many users are envisioned (see \ref{sec:community})? 
Is the software intended for research only or for usage by industry as well? What is its expected lifespan? Will complete reimplementation be an option, e.g., for going commercial?

Also, you should get a good understanding of the available team to create the software. 
How many people are working on the software project permanently? What are their skills? 

Explicit answers to these questions are required to make wise decisions about organizing the software project and the team. 
For example, defining the software's scope helps to decide what to implement and what not. 
The team size, on the other hand, is the most important factor for defining the development process. Do we need code reviews? Do we need people responsible for specific tasks like testing, documentation? Especially, large teams must define and enforce the process explicitly to not end in chaos, like outdated documentation (see \ref{sec:documentation}) or failing pipelines (see \ref{sec:testing}).

Generally, you can think of software development organization as a scale from a high degree of agility (e.g., extreme programming~\cite{796139}) to a high degree of planning (e.g., waterfall model \cite{Petersen835760}). Consider which degree is desirable for your project, depending on the answers to the precious questions. Most research projects have a high degree of uncertainty, resulting in frequent changes in software requirements. This would naturally suit a more agile organization.

Another critical question is the developers' software engineering experience. Especially in energy research, the skill levels vary significantly, as we pointed out in \autoref{sec:intro}. 
We researchers are not expected to have the skills of a full-time professional developer. However, we \textbf{are} software developers, which mandates certain minimum skill requirements. Therefore, we should perform a skill assessment of all team members and actively provide training to fill the gaps. 
\par 
Further, it is important to assign clear responsibilities. Primarily the software project lead should be clear. It should be someone who actively participates in the software project and does not have too many other responsibilities. For example, an experienced research assistant is often a better choice than a professor who does not have the time to participate actively.

\par 
Unorganized software projects result in low-quality software, so we should explicitly discuss and define our work process. Since there are no general rules, we need clear leadership and a clear vision of our software project's scope and purpose.

\recommendation{Organize yourself and your team!} 

\subsection{Building upon research requires reusable software}\label{sec:reuse}

\paragraph{Background} To effectively build upon research, it is often required to build upon research software, and therefore, a software's reusability is a key factor to its scientific success. Research based on openly available and reusable software is cited more often than research where the software is not reusable. To achieve this, it is necessary to consider your software's applicability to more general problems (than your own). This is decided by your software's abstraction level (generalized vs. specific implementation). However, there is a conflict regarding this abstraction level most of the time. There are two extreme ends to this aspect:
\begin{enumerate}
    \item Implementing highly specialized features that serve exactly one use case, e.g., a specific heat pump model that does not enable any parameterization (e.g., to implement different coefficients of performance, capacities, or others). 
    \item Finding an abstraction for every feature, which provides maximal freedom for analyzing other use cases than the one that shall be implemented in the first case, e.g., a generalized heat pump model, which does not even have a specific mathematical model but enables you to provide an arbitrary one by yourself.
\end{enumerate}
Both ends are extreme, and usually, you want to avoid both cases. However, there is a large spectrum of possibilities between these extremes. 

\paragraph{Recommendations} We recommend finding a sensible abstraction level, as it is crucial for the reusability of the software, and good abstractions enable other researchers (including yourself and your group) to build upon your software. At the same time, no unnecessary complexities should be introduced by choosing an overgeneralized abstraction. This might render your software less usable due to the lack of specific implementations and understandability. Consequently, users need to invest more effort, and reusing becomes difficult. 

Another aspect of this is to think about the scope and objective of your software. These determine to which level your software should be abstracted. As a rule of thumb, every artifact should focus on precisely one type of research. This can be on different, relatively macroscopic levels of energy research. For example, you can develop one type of energy component model (i.e., for heat pumps) or one software to simulate decentralized individual behaviors of actors in energy communities using agent-based modeling.

Besides the level of your implementation, meta aspects, like the name of the software, can also impact the reusability of your software. Also, we recommend considering the scientific community you contribute to and how they organize themselves in open software projects. Integrating your contribution in the form of software to these software communities increases your visibility and strengthens the community as a whole. Example communities for \ac{ERS} could be based on research frameworks, e.g., oemof\footnote{\url{https://oemof.org}, last access 2024-12-18.}, or generally communities of your favorite programming language, e.g., the Julia communities\footnote{\url{https://julialang.org/community/organizations/}, last access 2024-12-18}. Furthermore, considering the target audience, which will eventually reuse your software, is essential. To improve the compatibility of the software, you can use ontologies, such as the open energy ontology\footnote{\url{https://openenergyplatform.org/ontology/}, last access 2024-12-18} to develop and validate whether the keywords used in your software and documentation fit those of the broader community.

In short, software reusability is essential for making energy research sustainable and interoperable, not only for you and your working group but also for the broader research community you are working in.

\recommendation{Consider the reuse of your research software!}

%\vspace{-3mm}
\subsection{Multi-destination active message format}
\label{section:message_format}
\vspace{-0.7cm}
\begin{figure}[h!]
	\scriptsize
        \centering
    % \hspace{-1cm}
    \includegraphics[width=1\columnwidth]{diagrams/message_format.pdf}
    \vspace{-0.4cm}
	\caption{Message format} 
	\label{fig:message_format}
	\vspace{-.3cm}
\end{figure}
\textit{Nexus Machine} extends the fundamental Active Message primitives to accommodate a multi-destination based routing mechanism. 
Fig.~\ref{fig:message_format} illustrates the message format: the first 12 bits specify intermediate destinations (\textit{R1}, \textit{R2}, \textit{R3}), based on our workload analysis. 
The next 4 bits contain the Program Counter (PC) for the next instruction (\textit{N\_PC}), followed by 4 bits for the \textit{Opcode}. 
A single bit (\textit{Res\_c}) indicates if the message carries a result. 
The subsequent 2 bits (\textit{Op1\_c} and \textit{Op2\_c}) identify whether \textit{Op1} and \textit{Op2} are addresses or values. 
Depending on \textit{Res\_c}, the \textit{Result} field contains the final result or its address, while the next 16 bits hold data for Operand1 (\textit{Op1}) and Operand2 (\textit{Op2}).

When a message arrives at a router, the first destination (\textit{R1}) is processed by the \textit{Route Computation} logic and then allocated to the appropriate output port. After reaching \textit{R1}, the message is handled by the \textit{Input Network Interface}, and the remaining destinations are cyclically rotated, making \textit{R2} the first and \textit{R3} the second. 

In the \textit{Nexus Machine}, a message is equivalent to a packet or flit (all messages are a single-flit packet).
\begin{comment}
\begin{figure*}[h!]
	\scriptsize
	\centering
	\includegraphics[width=\textwidth]{diagrams/architecture.pdf}
	\caption{\textit{Nexus Machine} microarchitecture. \textit{Nexus Machine} is a fabric of homogenous PEs interconnected by a mesh network for communicating Active messages, enhancing fabric utilization by executing messages en-route.} 
	\label{fig:detail_arch}
	%\vspace{-.5cm}
\end{figure*}
\end{comment}
%\vspace{-3mm}
\subsection{Nexus Machine Micro-architecture}
\begin{figure*}[h!]
	\scriptsize
	\centering
	\includegraphics[width=0.9\textwidth]{diagrams/architecture.pdf}
    \vspace{-.15cm}
	\caption{\textit{Nexus Machine} microarchitecture. A fabric of homogenous PEs interconnected by a mesh network for communicating Active Messages which carry instructions that can be launched en-route at any PE, enhancing fabric utilization and runtime.} 
    \vspace{-0.3cm}
 %\color{red}{\bf Peh: I suggest replacing (d) with one of the Compute Unit, cos it's a major component of Nexus and yet do not feature in any figure. We need to highlight to reviewers that our compute unit consists of ALU :)}} 
	\label{fig:detail_arch}
	%\vspace{-.5cm}
\end{figure*}
%\subsubsection{Top Level}
As presented in Fig.~\ref{fig:detail_arch}(a), the \textit{Nexus Machine}'s fabric comprises homogeneous processing elements (PEs) interconnected with a mesh network, with a global termination detector. Each PE is linked to four neighboring PEs in North, East, South, and West directions.
The off-chip memory is connected to the four PEs located along the left edge.


\subsubsection{Processing Elements (PEs).}
%As presented in figure~\ref{fig:detail_arch}(b), each PE combines a compute unit, a dynamic router for network connectivity with congestion control, a decode unit with local data memory, an Input Network Interface which contains an instruction memory for handling incoming AMs and an AM Network Interface unit for spawning new AMs. \\
As presented in Fig.~\ref{fig:detail_arch}(b), each PE combines a compute unit, a dynamic router for network connectivity with congestion control, a decode unit, and two Network Interface logic.
Specifically, \textit{Input Network Interface} unit is responsible for efficiently handling incoming AMs from the NoC, while the AM Network Interface unit initiates the injection of new messages into the NoC.

\textbf{Input Network Interface.}
%The Input Network Interface logic triggers the loading of subsequent instruction on AM arrival.
%The arrival of a Decode AM triggers loading of the data element 
The \textit{Input Network Interface} unit manages \textit{incoming AMs} to a PE.
Depending on the message, \\%it performs either of these two operations.\\
%(a) It either updates the instruction contained in the message based on the next Program Counter (N\_PC) value provided within the message body.\\
(a) If it pertains to an ALU operation, it is directed to the \textit{Compute Unit} for execution.\\
(b) Alternatively, in case of a memory operation, the message is forwarded to the \textit{Decode} unit. 
This unit initiates a load or store operation, utilizing the operand address information (\textit{Op1} or \textit{Op2}) contained in the message.\\
Once these operations are completed, the resulting \textit{output dynamic AM} is dispatched to the \textit{AM Network Interface} for injecting into the network.
%The message enters the network via the local input port, which feeds the \textit{Compute} unit.

\textbf{Compute Unit.}
The \textit{compute unit} within a PE can perform 16-bit arithmetic operations, logic operations, multiplication, and division on its ALU.

An incoming AM at the \textit{Input Network Interface} dispatches two operands, \textit{Op1} and \textit{Op2} along with the \textit{Opcode} field in the message to the compute unit.
After computation, it generates an output that is combined with the original AM, replacing the \textit{Op1} field in the message.
Finally, this modified AM is forwarded to the \textit{AM Network Interface} for injecting into the network.

%{\bf Peh: There needs to be detailed information on how an AM launches computation! This is the thesis of AM! For instance, what's the format of the AM, when it's received, which field is used to configure the ALU? How is the PC set? What happens in the beginning of execution? read config memory? are there registers? what happens if data operand is not present -- can that happen? stall? Lots of details needed here.}

\textbf{Decode Unit.}
The \textit{Decode Unit}, as shown in Fig.~\ref{fig:detail_arch}(e), can be flexibly configured to operate in dereference and streaming modes.
In \textbf{dereference mode}, the operand address field (\textit{Op1} or \textit{Op2}) in the message triggers the loading of a single element. This gets embedded into the output \textit{dynamic AM}.
Conversely, in \textbf{streaming mode}, the message initiates the loading of multiple elements from memory, generating multiple output AMs.
In this mode, the operand address is considered the base address, along with a count to access and load the elements from memory sequentially.
These two modes suffice for our benchmarks; however, our architecture allows for integration of additional modes if needed.

\textbf{Active Message (AM) Network Interface.}
The \textit{AM Network Interface logic} is responsible for injecting AMs into the network.
%The AM Network Interface logic consists of an AM Queue and a configuration memory.
%The AM Queue is a 1KB FIFO, initialized with 44-bit precompiled entries.
%The configuration memory is 16-bit wide, containing 8 configurations.
This module comprises two primary components: an \textit{AM Queue} and a \textit{configuration memory}. 
The \textit{AM Queue} is a 16KB FIFO initialized with 70-bit precompiled entries. 
The \textit{configuration memory}, 10-bit wide, accommodates 8 distinct configurations.

%Depending on the availability of the output dynamic AM from the \textit{Input Network Interface}, it either
It either performs these two operations, as shown in Fig.~\ref{fig:detail_arch}(b):
(1) If the output \textit{dynamic AM} is available from \textit{Input Network Interface}, the subsequent configuration is loaded from memory based on the \textit{N\_PC} field of the AM (see Fig.~\ref{fig:message_format}). 
This configuration is combined with the output \textit{dynamic AM} and forwarded into the injection port of the router.\\
(2) Alternatively, a \textit{static AM} is injected into the network to keep it occupied. 
This \textit{static AM} is the concatenation of the next precompiled entry from the \textit{AM Queue} with the first configuration loaded from memory.
The generation rate of \textit{static AMs} is determined by the backpressure signal at the router's injection port.

The highlighted blue fields in the message format (see Fig.~\ref{fig:message_format}) depict data from the configuration memory used to construct the subsequent dynamic AM, with fields \textit{Res\_c}, \textit{Op1\_c}, and \textit{Op2\_c} stored to prevent redundancy.
%The AM Network Interface logic consists of a 1KB AM Queue, a FIFO containing 44-bit pre-compiled entries.
%, alongside a 13-bit wide configuration register. 

%As shown in Figure~\ref{fig:detail_arch}(e), the output dynamic AMs from \textit{Input Network Interface} trigger loading the next subsequent configuration from the memory with the N\_PC field of the AM.
%These are further concatenated with the output dynamic AMs and pushed into the injection port of the router.
%The injection rate is managed by the backpressure signal at the injection port of the router.

%To keep the network occupied, static AMs are containing the first precompiled entry from AM queue
%As shown in Figure~\ref{fig:detail_arch}(e), it concatenates an AM Queue entry with the first configuration loaded from the memory to generate a static AM, which is subsequently pushed intothe injection port of the router. 

\subsubsection{Dynamic and Congestion Aware Routing.}
\textit{Nexus Machine} supports turn model routing~\cite{noc_peh}, with each router containing five input and five output ports.
%Specifically, these input ports correspond to AM, local, north, east, south and west, whereas output ports correspond to local and four directions.
Specifically, these input ports are designated for messages coming from \textit{AM Network Interface} unit, as well as north, east, south, and west directions, whereas output ports are designated for messages going to \textit{Input Network Interface} unit and four directions.
%The AM input port receives recently generated messages from the \textit{AM Network Interface unit}, while the local port handles messages coming from the \textit{Input Network Interface}. 
Each input port has a buffer comprising three registers to manage in-flight messages, accompanied by congestion control logic. \textit{Nexus Machine}'s design choice of employing only three registers is motivated by the goal of minimizing overall power consumption.

As presented in Fig.~\ref{fig:detail_arch}(c), each router contains a Route Computation Unit, Separable Allocator, and a Crossbar.

\textbf{Route Computation} logic considers the destination of messages from all the input ports. It compares it with the positional ID of the PE, and calculates the output port to be requested. This is sent as an input to the allocator.

%{\bf Peh: Separable alllocation is a well-known previously proposed technique... so there's no need to elaborate... just cite a NoCs textbook}
A toy example of \textbf{Separable Allocation} process is presented in Fig.~\ref{fig:detail_arch}(d)~\cite{noc_peh}.
\iffalse
The request matrix's rows correspond to input ports, and columns correspond to output ports.
The process consists of two stages of 6:1 and 5:1 fixed priority arbiters. The first stage prunes the matrix to ensure that each output port (or resource) receives requests from at most one input port (or requestor). Subsequently, the backpressure signal is applied to each output port, enabling congestion control, as explained below. The second stage further prunes the matrix to guarantee one grant per input port.
The allocator executes within a single cycle, marking granted requests as issued immediately to prevent them from bidding again.
\fi

\textbf{On/Off Congestion control} involves the transmission of a signal to the upstream router when the count of available buffers falls below a threshold, ensuring all in-flight messages will have buffers on arrival. Each of the five ports transmits an OFF signal when their corresponding available buffer space is reduced to 1, i.e., $T_{OFF} = 1$, and conversely, an ON signal when their buffer space reaches 2, i.e., $T_{ON} = 2$.

The output of the allocator is sent to a 6x5 \textbf{Crossbar}.
%, which forms many-to-many connections among internal and external datapaths.\\ Peh: A crossbar by definition forms many-to-many connections between its input and output ports, so no need to explain. 

\subsubsection{Off-chip Memory Datapath.}
Each off-chip memory port connects to a row of the PE array via an AXI bus, delivering a combined bandwidth of 1.28GBps. During data loading, data transfers from off-chip memory to the \textit{AM queues} and \textit{data memory} in each PE. 
The \textit{AM queues} are actively consumed during execution, effectively hiding data loading latency by performing it concurrently with the execution. 
However, data loading into \textit{data memories} occurs after tile execution is complete.

\subsubsection{Bit-vector Scanners.}
The first sparse operand is encoded in \textit{static AMs}. For subsequent sparse operands, bit-vector scanner hardware assists in efficient iteration, providing coordinates within compressed vectors as described in \cite{capstan}. \textit{Nexus Machine} integrates a modified version of this with its AXI bus controller to obtain these coordinates. It can vectorize 16 non-zeros within 128 elements, allowing it to handle matrices with densities exceeding 12\%.
\vspace{-0.2cm}
\subsection{Deadlock avoidance}
%\textcolor{blue}{\bf Peh: Write up a short blurb on how Nexus machine addresses various deadlock scenarios and explain design choice: (1) Within network: Flow control deadlocks addressed by bubble buffer [cite bubble flow control] instead of VCs so as to minimize buffering; Routing deadlocks addressed by turn model, so as to provide high throughput without complex adaptive routing hardware; (2) AM also introduces potential network-PE deadlocks -- addressed by compiler preventing such cyclic dependencies + runtime timeouts} 

Given the dynamic nature, \textit{Nexus Machine} can potentially encounter deadlock without careful design. We address various deadlock scenarios with these specific design choices: 
(1) To mitigate flow control deadlocks within the network, we adopt the bubble NoC~\cite{bubble_flow} approach over Virtual Channels (VCs), with the aim to minimize buffering. 
(2) Routing deadlocks are mitigated by using the turn model~\cite{noc_peh}, that ensures high throughput without the need for complex adaptive routing hardware.
(3) AMs can potentially create deadlocks between the network and PEs. 
These are effectively mitigated by the compiler through strategic data placement and runtime timeouts.
%\textit{Nexus Machine} currently uses a simple heuristic for data placement strategy within the compiler.
%\textcolor{red}{\bf Peh: Is the simple heuristic related to deadlocks? Cos the above sentence seems to contradict the earlier sentence. Elaborate on the heuristic and timeouts??}
Future research will explore more optimized data placement strategies.
\subsection{Undocumented code can not be used by others, and is therefore near-to useless}\label{sec:documentation}

\paragraph{Background} When writing code, documentation is essential to help you and others understand the code. 
Documentation also helps to ensure that the code is doing what it is supposed to do. Furthermore, it allows the extension and adaption of the code and is fundamental to ensure reproducibility and reusability for others and yourself.\par

\paragraph{Recommendations} We recommend always writing documentation, no matter how large the software project is (even if only one person is involved). Any small documentation is always better than no documentation. However, software papers do not count as documentation since they cannot reflect the ever-changing nature of software.  \par
Multiple types of documentation exist:
\begin{enumerate}
    \item Documentation in the code, e.g., code docstrings, describing methods, classes, or modules. This helps others understand your code.
    \item A README file, which introduces and describes the software. This is necessary to explain the purpose and concept of the software.
    \item A documentation page, like \textit{readthedocs}\footnote{\url{https://about.readthedocs.com/}, last access 2025-01-17}, containing further descriptions of the software, helping others to fully understand the code and use it.
    \item Documentation of tutorials or concrete examples, e.g., included in the documentation page, helps others to easily apply the code.
\end{enumerate}

We recommend to consider your open source strategy (see \ref{sec:openSource}) and try to deduce the applicable documentation style and technique. If it is unclear how the documentation will be processed, you can stick to easy documentation modes. In any case, at least the following should be provided: a README file, including a summary of the purpose of the software, an installation guide, and an example script using your software.

Documentation requires considerable effort and should thus be acknowledged as part of the software development process, e.g., as an own contribution to the respective research project, and relevant resources should be allocated. It is important not to neglect it due to missing resources later on.
When planning the documentation, it is recommended to take into account the different types of documentation and the respective target audience. You can define minimal requirements for your documentation (as requirements and dependencies for the implementation, environment, and decisions made in the development process). Also, we recommend considering tests as part of the documentation, as these are necessary for others to understand your code. To support this understanding, example scenarios can be introduced in the documentation. However, all example scenarios should be fully executable.
To simplify the documentation process, we recommend combining the coding with the documentation (for example, via commit messages, see \ref{sec:vcs}) by writing readable code, including comments. Additionally, you can use templates and standards for your documentation. Many tools are available to make the documentation process easier and to help make your documentation findable. For Python, \textit{Sphinx}\footnote{\url{https://docs.readthedocs.io/en/stable/intro/sphinx.html}, last access 2025-01-17} can be used to create technical documentation, as can \textit{Documenter.jl} for Julia\footnote{\url{https://github.com/JuliaDocs/Documenter.jl}, last access 2025-01-17}.\par 

It also helps to look for best practices. These are for example provided by Github\footnote{\url{https://google.github.io/styleguide/docguide/best_practices.html}, last access 2025-01-17}, but can also be part of recommendations for your respective programming language, as \textit{pep8} for python\footnote{\url{https://peps.python.org/pep-0008/}, last access 2025-01-17}.

Due to the interdisciplinary character of energy systems research, it is recommended that the concept behind the software be explicitly discussed. People from different domains should understand the purpose of your software. Thus, also the concepts and ideas of your software, not only the software itself, should be documented. Additionally, the code itself should be documented so that other people can work with it or even extend it. 
\recommendation{You are not done until the documentation is done!}

\documentclass{article}
\usepackage{xcolor}
\usepackage{graphicx}
\definecolor{darkgreen}{rgb}{0.1, 0.5, 0.1}
\usepackage[a4paper,margin=1in]{geometry}
\usepackage{array}

\begin{document}

\begin{table}[t]
% \centering
\raggedright
\renewcommand{\arraystretch}{2}
\setlength{\tabcolsep}{8pt}
\begin{tabular}{|p{4cm}|p{4cm}|p{4cm}|p{4cm}|}
\hline
\textbf{{\fontsize{12pt}{22pt}\selectfont Comparison with Human Reviews}} & \textbf{{\fontsize{12pt}{22pt}\selectfont Factual \newline Accuracy}} & \textbf{{\fontsize{12pt}{22pt}\selectfont Analytical Depth}} & \textbf{{\fontsize{12pt}{22pt}\selectfont Actionable \newline Insights}} \\ \hline

{\fontsize{12pt}{22pt}\selectfont 
\textcolor{blue}{``The explanation of the results is adequate''} (AI) vs. \textcolor{darkgreen}{``The explanation is too brief and misses key statistical trends in Figure 3, such as the anomaly at epoch 50''} (human).} & 
{\fontsize{12pt}{22pt}\selectfont \textcolor{blue}{``The paper uses supervised learning techniques effectively''} (AI), but the actual technique described is reinforcement learning.} & 
{\fontsize{12pt}{22pt}\selectfont \textcolor{blue}{``The methodology section is sufficient''} (AI), without noting that \textcolor{darkgreen}{``The comparison to baseline models lacks clarity, especially in explaining the choice of hyperparameters''} (human).} & 
{\fontsize{12pt}{22pt}\selectfont \textcolor{blue}{``Provide more examples for better understanding''} (AI), instead of \textcolor{darkgreen}{``Add examples demonstrating how the algorithm performs under different lighting conditions to clarify its robustness''}} \\ \hline

{\fontsize{12pt}{22pt}\selectfont \textcolor{blue}{``Overall, the related work section is relevant''} (AI) vs \textcolor{darkgreen}{``The related work section does not include recent advancements in transformer-based architectures, such as XYZ-2023''} (human)} & 
{\fontsize{12pt}{22pt}\selectfont \textcolor{blue}{``The dataset appears to be balanced''} (AI), but the dataset is actually imbalanced based on the class distributions mentioned in Section 4.2} & 
{\fontsize{12pt}{22pt}\selectfont \textcolor{blue}{``The discussion is clear''} (AI), but it misses feedback like \textcolor{darkgreen}{``The discussion should explore why the proposed approach underperforms on Dataset B, as highlighted in Table 2''} (human)} & 
{\fontsize{12pt}{22pt}\selectfont \textcolor{blue}{``Clarify the introduction''} (AI), rather than \textcolor{darkgreen}{``Reorganize the introduction to define the problem before introducing the contributions, as this will improve flow and reader engagement''}} \\ \hline

{\fontsize{12pt}{22pt}\selectfont \textcolor{blue}{``The conclusion is well-written''} (AI) vs. \textcolor{darkgreen}{``The conclusion does not address limitations, such as the small sample size used in the experiments''} (human)} & 
{\fontsize{12pt}{22pt}\selectfont \textcolor{blue}{``The results suggest strong performance''} (AI), but it incorrectly claims \textcolor{darkgreen}{``The model outperforms all baselines,''} while Table 3 shows it underperforms in some metrics} & 
{\fontsize{12pt}{22pt}\selectfont \textcolor{blue}{``Results are promising''} (AI), lacking human feedback such as \textcolor{darkgreen}{``Consider expanding on the implications of your findings for real-world applications, particularly in autonomous navigation''}} & 
{\fontsize{12pt}{22pt}\selectfont \textcolor{blue}{``Improve the figures for better clarity''} (AI), rather than \textcolor{darkgreen}{``Increase the font size in Figure 4 and add units to the axes labels for better readability''}} \\ \hline
\end{tabular}
% \caption{Concrete Examples of Challenges in AI-Generated Reviews with Color Coding}
\label{tab:review_challenges}
\end{table}

\end{document}


%%%%%%%% ICML 2025 EXAMPLE LATEX SUBMISSION FILE %%%%%%%%%%%%%%%%%

\documentclass[11pt]{article}
\usepackage[review]{acl}
\usepackage{times}
\usepackage{latexsym}
\usepackage[T1]{fontenc}
\usepackage[utf8]{inputenc}
\usepackage{microtype}
\usepackage{inconsolata}
% Recommended, but optional, packages for figures and better typesetting:
\usepackage{tikz}
\newcommand*\circled[1]{\tikz[baseline=(char.base)]{
            \node[shape=circle,draw,inner sep=2pt] (char) {#1};}}
\usepackage{microtype}
\usepackage{graphicx}
\usepackage{subfigure}
\usepackage{booktabs} % for professional tables
\usepackage{algorithm}
\usepackage{algorithmicx}
% \usepackage{algorithmic}

% hyperref makes hyperlinks in the resulting PDF.
% If your build breaks (sometimes temporarily if a hyperlink spans a page)
% please comment out the following usepackage line and replace
% \usepackage{icml2025} with \usepackage[nohyperref]{icml2025} above.
\usepackage{hyperref}
\usepackage[noend]{algpseudocode} % For pseudocode
\usepackage{amsmath} % For math environments



% Attempt to make hyperref and algorithmic work together better:
%\newcommand{\theHalgorithm}{\arabic{algorithm}}

% Use the following line for the initial blind version submitted for review:
%\usepackage{icml2025}


% If accepted, instead use the following line for the camera-ready submission:
% \usepackage[accepted]{icml2025}

% For theorems and such
\usepackage{amsmath}
\usepackage{amssymb}
\usepackage{mathtools}
\usepackage{amsthm}

% if you use cleveref..
\usepackage[capitalize,noabbrev]{cleveref}

\usepackage{xcolor}
\usepackage{amsmath}
\usepackage{listings}
\usepackage{hyperref}
\usepackage{graphicx}
\usepackage{tcolorbox}
%\usepackage{geometry}
% \geometry{margin=1in}

%%%%%%%%%%%%%%%%%%%%%%%%%%%%%%%%
% THEOREMS
%%%%%%%%%%%%%%%%%%%%%%%%%%%%%%%%
\theoremstyle{plain}
\newtheorem{theorem}{Theorem}[section]
\newtheorem{proposition}[theorem]{Proposition}
\newtheorem{lemma}[theorem]{Lemma}
\newtheorem{corollary}[theorem]{Corollary}
\theoremstyle{definition}
\newtheorem{definition}[theorem]{Definition}
\newtheorem{assumption}[theorem]{Assumption}
\theoremstyle{remark}
\newtheorem{remark}[theorem]{Remark}

% Todonotes is useful during development; simply uncomment the next line
%    and comment out the line below the next line to turn off comments
%\usepackage[disable,textsize=tiny]{todonotes}
\usepackage[textsize=tiny]{todonotes}
\usepackage{enumitem}
\usepackage{booktabs}
\usepackage{pifont}
\usepackage{makecell}
\newcommand{\cmark}{\ding{51}}  % Checkmark
\newcommand{\xmark}{\ding{55}}  % Crossmark
% The \icmltitle you define below is probably too long as a header.
% Therefore, a short form for the running title is supplied here:
%\icmltitlerunning{ReviewEval: An Evaluation Framework for AI-Generated Reviews}
\title{ReviewEval: An Evaluation Framework for AI-Generated Reviews}

\begin{document}

%\twocolumn[
%\icmltitle{ReviewEval: An Evaluation Framework for AI-Generated Reviews}

% It is OKAY to include author information, even for blind
% submissions: the style file will automatically remove it for you
% unless you've provided the [accepted] option to the icml2025
% package.

% List of affiliations: The first argument should be a (short)
% identifier you will use later to specify author affiliations
% Academic affiliations should list Department, University, City, Region, Country
% Industry affiliations should list Company, City, Region, Country

% You can specify symbols, otherwise they are numbered in order.
% Ideally, you should not use this facility. Affiliations will be numbered
% in order of appearance and this is the preferred way.
% \icmlsetsymbol{equal}{*}

%\begin{icmlauthorlist}
%\icmlauthor{Firstname1 Lastname1}{equal,yyy}
%\icmlauthor{Firstname2 Lastname2}{equal,yyy,comp}
%\icmlauthor{Firstname3 Lastname3}{comp}
%\icmlauthor{Firstname4 Lastname4}{sch}
%\icmlauthor{Firstname5 Lastname5}{yyy}
%\icmlauthor{Firstname6 Lastname6}{sch,yyy,comp}
% \icmlauthor{Firstname7 Lastname7}{comp}
% %\icmlauthor{}{sch}
% \icmlauthor{Firstname8 Lastname8}{sch}
% \icmlauthor{Firstname8 Lastname8}{yyy,comp}
%\icmlauthor{}{sch}
%\icmlauthor{}{sch}
%\end{icmlauthorlist}

% \icmlaffiliation{yyy}{Department of XXX, University of YYY, Location, Country}
% \icmlaffiliation{comp}{Company Name, Location, Country}
% \icmlaffiliation{sch}{School of ZZZ, Institute of WWW, Location, Country}

% \icmlcorrespondingauthor{Firstname1 Lastname1}{first1.last1@xxx.edu}
% \icmlcorrespondingauthor{Firstname2 Lastname2}{first2.last2@www.uk}

% % You may provide any keywords that you
% % find helpful for describing your paper; these are used to populate
% % the "keywords" metadata in the PDF but will not be shown in the document
% \icmlkeywords{Machine Learning, ICML}

% \vskip 0.3in
% ]

% this must go after the closing bracket ] following \twocolumn[ ...

% This command actually creates the footnote in the first column
% listing the affiliations and the copyright notice.
% The command takes one argument, which is text to display at the start of the footnote.
% The \icmlEqualContribution command is standard text for equal contribution.
% Remove it (just {}) if you do not need this facility.

% \printAffiliationsAndNotice{}  % leave blank if no need to mention equal contribution
% \printAffiliationsAndNotice{\icmlEqualContribution} % otherwise use the standard text.
\maketitle
\begin{abstract}
% We explain the design and implementation of our custom pipeline developed to generate comprehensive reviews for research papers using Large Language Models (LLMs). Our approach leverages Google's latest LLM, Gemini 1.5 Pro, orchestrated through a multi-stage process that includes prompt generation, iterative refinement via reflection loops, and final evaluation to ensure the production of high-quality reviews aligned with specific conference guidelines.
% The exponential growth of academic research has strained traditional peer review processes, prompting the exploration of Large Language Models (LLMs) for automated review generation. While LLMs demonstrate potential in research reviewing, their effectiveness remains limited by issues such as superficial critiques, hallucinations, and lack of actionable insights. Moreover, there has been limited research on developing comprehensive evaluation frameworks to access these LLM based paper reviewers. To address this, we propose a comprehensive evaluation framework that assesses AI-generated reviews across four key dimensions: (1) Alignment with Human Reviews, (2) Factual Accuracy, (3) Analytical Depth, and (4) Actionable Insights. Our goal is to enhance the reliability of AI-generated reviews through a comprehensive evaluation framework and establish standardized metrics for assessing AI-based review systems. Additionally, our approach introduces a novel conference-specific alignment mechanism, addressing the need for adaptable reviewers that cater to varying evaluation priorities across conferences, a gap not previously explored. Furthermore, we propose a self-refinement loop, enabling the LLM to iteratively improve its review prompts, thereby enhancing the quality and depth of AI-generated reviews.

%The escalating volume of academic research, coupled with a shortage of qualified reviewers, necessitates innovative approaches to peer review. While LLMs offer potential for automating this process, their current limitations include superficial critiques, hallucinations, and a lack of actionable insights. This research addresses these challenges by proposing a comprehensive evaluation framework for AI-generated reviews, encompassing alignment with human reviews, factual accuracy, analytical depth, and actionable insights.Furthermore, we introduce a novel conference/journal specific alignment mechanism to adapt LLM-generated reviews to the unique evaluation priorities of different conferences and journals. To enhance the quality of these reviews, we propose a self-refinement loop that iteratively improves the LLM's review prompts. This framework aims to establish standardized metrics for assessing AI-based review systems and enhance the reliability of AI-generated reviews in academic research.

The escalating volume of academic research, coupled with a shortage of qualified reviewers, necessitates innovative approaches to peer review. While large language model (LLMs) offer potential for automating this process, their current limitations include superficial critiques, hallucinations, and a lack of actionable insights. This research addresses these challenges by introducing a comprehensive evaluation framework for AI-generated reviews, that measures alignment with human evaluations, verifies factual accuracy, assesses analytical depth, and identifies actionable insights. We also propose a novel alignment mechanism that tailors LLM-generated reviews to the unique evaluation priorities of individual conferences and journals. To enhance the quality of these reviews, we introduce a self-refinement loop that iteratively optimizes the LLM's review prompts. Our framework establishes standardized metrics for evaluating AI-based review systems, thereby bolstering the reliability of AI-generated reviews in academic research.

%-----Murari----
% The escalating volume of academic research, coupled with a shortage of qualified reviewers, necessitates innovative approaches to peer review. While large language model (LLMs) offer potential for automating this process, their current limitations include superficial critiques, hallucinations, and a lack of actionable insights. This research addresses these challenges by introducing a comprehensive evaluation framework for AI-generated reviews, that measures alignment with human evaluations, verifies factual accuracy, assesses analytical depth, and identifies actionable insights. We also propose a novel alignment mechanism that tailors LLM-generated reviews to the unique evaluation priorities of individual conferences and journals. To enhance the quality of these reviews, we introduce a self-refinement loop that iteratively optimizes the LLM's review prompts. Our framework establishes standardized metrics for evaluating AI-based review systems, thereby bolstering the reliability of AI-generated reviews in academic research.\vspace{-7mm}


% Empirical results demonstrate that our approach significantly improves prompt quality and review rigor. This research advances AI-driven peer review by addressing key shortcomings in evaluation transparency, depth, and adaptability to domain-specific guidelines.
\end{abstract}

\section{Introduction}
The rapid growth of academic research, coupled with a shortage of qualified reviewers, has created an urgent need for scalable and high-quality peer review processes \cite{petrescu2022evolving, schulz2022future, checco2021ai}. Traditional peer review methods are under mounting pressure from the exponentially growing number of submissions, particularly in fields like artificial intelligence, machine learning, and computer vision. This has led to a growing interest in leveraging large language models (LLMs) to automate and enhance various aspects of the peer review process~\cite{robertson2023gpt4, liu2023reviewergpt}.\par

LLMs have shown remarkable potential in automating various natural language processing tasks, such as summarization, translation, and question-answering. However, their effectiveness in serving as reliable and consistent paper reviewers remains a significant challenge. The academic community is already experimenting with AI-assisted reviews, as evidenced by reports that 15.8\% of reviews for ICLR 2024 were generated with AI assistance~\cite{latona2024ai}. While this demonstrates the growing adoption of LLMs in peer review, concerns have been raised regarding their impact on the reliability and fairness of the review process. Specifically, papers reviewed by AI have been perceived to gain an unfair advantage, leading to questions about the integrity of such evaluations. Consequently, research into robust automated review generation systems is crucial, necessitating rigorous evaluation of AI generated reviews to address key challenges.~\cite{zhou2024llm} provide a comprehensive analysis of the use of commercial models, such as GPT-3.5 and GPT-4 \cite{achiam2023gpt}, as research paper reviewers. Their findings highlight key limitations, including the potential for mistakes due to either model hallucinations or an incomplete understanding of the material, as well as the inability to provide critical feedback comparable to human reviewers. Based on our preliminary experiments of using GPT-4 to generate reviews for research papers, we identified additional limitations in AI-generated reviews, including a lack of actionable insights and limited analytical depth, often characterized by generic and vague feedback. These shortcomings stem from the inherent tendency of LLMs to generate superficial reviews.

%\vspace{-2mm}

% The adoption of effective AI-based reviewing systems in the research communities addresses several critical needs \cite{tyser2024ai}: (i) they provide early feedback to authors on their work in progress, facilitating iterative improvements and enhancing the overall quality of research outputs; (ii) they enable conferences to manage the growing volume of submissions more efficiently, ensuring high-quality and timely reviews; and (iii) they contribute to quality control by reducing inconsistencies and potential biases in the peer-review process.

% Should we add more details about us gathering these challenges?

% \begin{table}[h]
%     \centering
%     \renewcommand{\arraystretch}{0.9} % Adjust row height for better readability
%     \begin{tabular}{l>{\centering\arraybackslash}m{2.5cm} >{\centering\arraybackslash}m{2.5cm}}
%         \toprule
%         & \makecell{\textbf{Wider} \\ \textbf{Metric Coverage}} & \makecell{\textbf{Black Box} \\ \textbf{Remedy}} \\
%         \midrule
%         \textbf{Our Framework} & \cmark & \cmark \\
%         \textbf{MARG} & \xmark & \xmark \\
%         \textbf{Aspect-Based} & \xmark & \xmark \\
%         \bottomrule
%     \end{tabular}
%     \caption{Comparison of Evaluation Methods. The evaluation metrics by \cite{d2024marg} (MARG) and \cite{zhou2024llm} (Aspect-Based Comparisons) focus on AI-human review similarity but overlook other key deficiencies in AI reviews. Their heavy reliance on LLMs for end-to-end evaluation creates a black-box system with limited transparency. Our framework addresses these gaps with a more comprehensive and interpretable approach.}
%     \label{tab:evaluation_comparison}
% \end{table}



\begin{figure}[t]
    \centering
    \includegraphics[width=0.5\textwidth, trim=0cm 0cm 0cm 0cm, clip]{table-diagram.pdf} % Adjust width for good fit
    \caption{\small Examples of the challenges and limitations of AI based research paper reviews}
    % \vspace{0.1in} % Adds 0.1 inches of space after the caption
    \label{fig:challenges}
\end{figure}

% There has been limited existing research on developing evaluation metrics to evaluate AI based research paper reviews. \cite{d2024marg} proposed an automated metric to check for approximate matches between AI-generated and human-written review comments using GPT. While GPT-4 is employed iteratively to extract approximate matches and mitigate inconsistencies, it relies completely on GPT-4 for end-to-end evaluation, making the whole metric a black box and unreliable. \cite{zhou2024llm} paper examines aspect coverage
% and similarity to reference reviews through both automatic
% metrics and manual analysis, utilizing the ASAP dataset
% \cite{yuan2022can}, which annotates review sentences with
% labels such as summary, motivation, originality, soundness,
% substance, replicability, meaningful comparison, and clarity.
% However, relying on pre-defined aspect coverage under-
% mines the diversity and depth of AI and human reviews, therefore we recognized the need for a more nuanced and
% dynamic topic coverage metric

% \begin{table}[h]
%     \centering
%     \begin{tabular}{lcc}
%         \toprule
%         & \makecell{\textbf{Dynamic Topic} \\ \textbf{Coverage}} & \makecell{\textbf{Remedying} \\ \textbf{Black Box Nature}} \\
%         \midrule
%         \textbf{Our Evaluation Framework} & \cmark & \cmark \\
%         \textbf{MARG} & \xmark & \xmark \\
%         \textbf{Aspect Coverage Based} & \xmark & \xmark \\
%         \bottomrule
%     \end{tabular}
%     \caption{Comparison of Evaluation Methods}
%     \label{tab:evaluation_comparison}
% \end{table}

Existing research on evaluation metrics for AI-generated research paper reviews remains limited. For instance,~\cite{d2024marg} proposed an automated metric to evaluate approximate matches between AI-generated and human-written review comments using GPT-4~\cite{achiam2023gpt}. Although their method iteratively employs GPT-4 to extract approximate matches and mitigate inconsistencies, its complete reliance on GPT-4 renders the evaluation process a black box, thereby limiting transparency and raising concerns about reliability. Similarly, \cite{zhou2024llm} investigated the aspect coverage and similarity between AI and human reviews through a blend of automatic metrics and manual analysis. Their work leveraged the ASAP dataset \cite{yuan2022can}, which categorizes review sentences into predefined aspects—including summary, motivation, originality, soundness, substance, replicability, meaningful comparison, and clarity, to align AI and human reviews. However, beyond this AI-human comparison, their approach overlooks other critical dimensions where AI reviews may underperform, as highlighted in Figure~\ref{fig:challenges}.\par

% While GPT-4 is employed iteratively to extract approximate matches and mitigate inconsistencies, it relies entirely on GPT-4 for end-to-end evaluation, making the metric a black-box system with limited transparency and reliability. 

%\cite{zhou2024llm} examined aspect coverage and similarity between AI and human reviews through a combination of automatic metrics and manual analysis. Their study utilized the ASAP dataset~\cite{yuan2022can}, which categorizes review sentences into predefined aspects such as summary, motivation, originality, soundness, substance, replicability, meaningful comparison, and clarity, and based on these aspects, matches AI and human reviews together. However, apart from the AI-Human comparison, they are overlooking other critical dimensions where AI reviews may fall short and require assessment (as described in figure \ref{fig:challenges}).

% this reliance on fixed aspect-based evaluation limits the framework’s ability to capture the diversity and depth of both AI and human reviews, thereby restricting its effectiveness in assessing nuanced differences in review quality.

Based on our analysis of the limitations in current AI-generated reviews and the gaps in existing evaluation metrics, we propose a comprehensive evaluation framework designed to assess the quality of AI-generated research paper reviews. Our framework targets four key dimensions (see Figure~\ref{fig:challenges}): \ding{182} \textit{Comparison with Human Reviews}: Evaluates topic coverage and semantic similarity to measure the alignment between AI-generated and human-written feedback. \ding{183}
\textit{Factual Accuracy}: Detects factual errors, including misinterpretations, incorrect claims, and hallucinated information. \ding{184} \textit{Analytical Depth}: Assesses whether the AI’s critique transcends generic commentary to offer in-depth, meaningful engagement with the research. \ding{185} \textit{Actionable Insights}: Measures the ability of the AI to provide specific, constructive suggestions for improving the paper.\par

Recognizing that major conferences and journals have distinct reviewing priorities, a one-size-fits-all approach to AI-driven reviews is insufficient. Recent studies \cite{bauchner2024use, biswas2024ai} underscore the growing importance of aligning reviews with conference-specific evaluation criteria—especially as many venues now require adherence to detailed reporting guidelines. To address this, we introduce a conference-specific AI reviewer that dynamically adapts its review strategy to meet the unique criteria of each target venue. 

Furthermore, inspired by the self-refinement approach of~\cite{madaan2023selfrefine}, our system incorporates a supervisor model that iteratively critiques and refines the instructional prompts guiding the review process. This self-refinement loop promotes deeper analytical assessments and mitigates the tendency of AI-generated reviews to offer only superficial feedback. In summary, our research aims to:
\begin{enumerate}
    \item Develop a comprehensive evaluation framework for LLM-based reviewing systems across five dimensions: (i) alignment with human reviews, (ii) factual accuracy, (iii) analytical depth, (iv) actionable insights, and (v) adherence to Reviewer guidelines.
    
    \item Create an LLM-based reviewer that dynamically aligns with the evaluation criteria of specific conferences/journals and continuously improves its reviewing strategy through an iterative refinement loop.
\end{enumerate}


% Therefore, based on our analysis of the limitations in AI-generated reviews and the gaps in existing research on evaluation metrics, we propose a comprehensive evaluation framework to evaluate the quality of AI-generated research paper reviews across four key dimensions (refer figure~\ref{fig:challenges}):
% \begin{enumerate}
% \item Comparison with Human Reviews: Analyzes topic coverage and semantic similarity to measure alignment between AI and human feedback.
% \item Factual Accuracy: Assesses the presence of factual errors, such as misinterpretations, incorrect claims, or hallucinated information in the reviews.
% \item Analytical Depth: Measures the depth and rigor of the AI’s critique, focusing on whether it moves beyond generic comments to meaningful engagement with the research.
% \item Actionable Insights: Evaluates the AI’s ability to provide specific, constructive feedback for improving the paper.
% \end{enumerate}

% \includegraphics[width=\textwidth, trim=0cm 2cm 0cm 2cm, clip]{yourfile.pdf}
% For example, in AI and machine learning, NeurIPS emphasizes theoretical advancements, ICLR prioritizes deep learning and representation learning, AAAI values practical AI applications, and ACL focuses on linguistic insights in NLP, while ICML covers a broad spectrum of machine learning methodologies and optimization techniques.

% Each major conference/journal has distinct reviewing priorities aligned with its focus areas, making a one-size-fits-all approach to AI-driven paper reviews inadequate. Existing research \cite{bauchner2024use, biswas2024ai} highlights the importance of this direction, noting that many conferences/journals now require authors to indicate adherence to specific reporting guidelines, and AI has the potential to be more effective than human reviewers in assessing such compliance. To address this, we propose a conference-specific AI reviewer, which dynamically aligns AI-generated reviews with the unique evaluation criteria of individual conferences/journals, an area that, to the best of our knowledge, has not been explored before. Additionally, we introduce a self-refinement loop, inspired by \cite{madaan2023selfrefine}, where a supervisor model iteratively critiques and refines the instructional prompts guiding the review process. This ensures deeper, more analytical assessments rather than the superficial feedback that often characterizes AI-generated reviews. 


% Our system dynamically generates review prompts by converting each specific guideline of a target conference into individual, step-by-step instructional prompts designed to guide the LLM through the paper reviewing process. 
% Hence, to achieve this we introduce a self-refinement framework inspired by \cite{madaan2023selfrefine}. Our approach employs a Supervisor LLM that iteratively critiques the review prompts based on clarity, coherence, and alignment with conference guideline. Empirical results show that this refinement process significantly enhances prompt quality (refer figure 2), leading to more rigorous and context-aware paper evaluations. 


% Ultimately, our approach aims to narrow the gap between automated and human-like review processes, improving the analytical depth of LLM-assisted peer review.



% Another key challenge in using LLMs for paper reviews is their tendency to generate shallow or generic feedback. The quality of AI-generated reviews heavily depends on the instructional prompts guiding the model. 
% Since these prompts serve as structured guidelines for navigating the paper, they must closely reflect the cognitive processes of human reviewers to ensure deeper analysis rather than superficial assessments. To achieve this, we introduce a self-refinement framework inspired by \cite{madaan2023selfrefine}. Our approach employs a Supervisor LLM that critiques the review prompts iteratively based on clarity, coherence, and alignment with conference guideline. Empirical results show that this refinement process significantly enhances prompt quality, leading to more rigorous and context-aware paper evaluations (refer figure 2). Ultimately, our approach aims to narrow the gap between automated and human-like review processes, improving the analytical depth of LLM-assisted peer review.


% Each major conference has distinct reviewing priorities that reflect their unique focus areas. For example, within the AI.ML domain, NeurIPS places significant emphasis on theoretical advancements and novelty, while ICLR centers on innovations in deep learning and representations. In contrast, AAAI values practical applications across diverse AI domains, and ACL focuses on linguistic insights in natural language processing. ICML, with its broad scope, prioritizes novel machine learning methodologies and optimization techniques. These varying priorities make a conference-specific LLM reviewer essential, as it tailors the evaluation process to the specific goals and expectations of each conference, ensuring relevant, context-sensitive feedback. Hence, to address the need for peer reviews tailored to the unique requirements of academic conferences, we propose a conference-specific research paper reviewer system powered by Large Language Models. This approach provides a novel solution to align AI-generated reviews with the evaluation criteria of individual conferences, a direction that to the best of our knowledge, has not been explored in prior research. Our system dynamically generates review prompts by converting each specific guideline of a target conference into individual, step-by-step prompts designed to guide the review generation process. These prompts ensure that the AI-generated reviews address all aspects of the conference's expectations in a structured and systematic manner. 

% A key challenge in using LLMs for research paper reviewing is their tendency to produce superficial and insufficiently analytical reviews. We hypothesize that the instructional prompts guiding the LLM play a crucial role in shaping the quality of the generated reviews. Since these prompts serve as structured guidelines for navigating the paper, they must closely reflect the cognitive processes of human reviewers to ensure deeper analysis rather than superficial assessments. To address this, we propose a self-refinement framework inspired by \cite{madaan2023selfrefine}, which enables LLMs to iteratively improve their outputs through feedback-based revision. Specifically, we introduce an iterative prompt refinement mechanism, where an LLM supervisor evaluates the initial prompts generated from conference guidelines. The supervisor provides constructive feedback based on clarity, coherence, and alignment with the guideline’s intent, which is then used by the original LLM to improve its initial prompts. Empirical observations indicate that this iterative refinement process significantly improves prompt quality. As shown in Figure 2, the supervisory LLM identifies major deficiencies, which are then fixed leading to more structured and insightful instructional guidance. This improvement directly impacts the depth and relevance of the LLM-generated reviews. Our approach aims to bridge the gap between automated and human-like review processes, enhancing the analytical rigor of LLM-based research reviewing.

% Furthermore, to address a major concern of using LLMs for research paper reviewing: LLMs generating superficial and insufficient reviews (as will be evaluated by our 3 out of 4 evaluation metrics: Comparison with human reviews, analytical depth, and actionable insights), we realize that the instructional prompts the Reviewer LLM follows plays a critical role in how the review will turn out to be. This instructional prompt plays a crucial role in helping the reviewer navigate the paper, hence this prompt needs to be as close to a human though process while reviewing a paper to ensure that the reviews come out to be less superficial and more analytical and intelligent. Hence, improving the quality of prompts used for reviewing is very important.

% To enhance the quality of our prompts that are generated using conference guidelines, we propose a self-refinement method, inspired by \cite{madaan2023selfrefine}, which suggests that LLMs are capable of revising their own outputs just like humans, through an iterative feedback and refinement
% loop to improve initial outputs from LLMs. The instructional prompt used to generate research reviews using LLM plays a critical role in how well the LLM operates as a Reviewer, and hence, these instructional prompts that are dynamically generated using the conference guidelines need to be refined iteratively to ensure they guide the "LLM as Reviewer" through the a well-defined instructional prompt. We employ a supervisory LLM that is instructed to give constructive feedback on the quality of the initial prompt generated for each guideline, this feedback is based on clarity and coherence of the prompt, its alignment to the guideline, etc. We noticed significant improvement in the quality of prompt generated through refinement loops, where we saw the supervisor LLM pointing out major issues like shown in figure 2. Through our work, we recommend a pipeline that proposes a solution to convert conference guidelines into individual instructional prompts that are refined through reflection loop to make them more comprehensive and "human-like". We aim to propose a system that generates prompts aligning closely with how the though process of a human reviewer will operate.

% To improve the quality and alignment of these prompts, we introduce a refinement loop using a secondary "Judge LLM." The Judge LLM evaluates each prompt for coherence, relevance, and consistency with the original guideline, suggesting refinements to better capture the intended requirements. By systematically aligning reviews with conference-specific guidelines, this framework provides a practical and customizable tool for researchers seeking precise, high-quality, and compliant reviews.

% \usepackage{enumitem} % Add this in the preamble

% In summary, the objectives of the research are listed below.
% \begin{enumerate}[itemsep=0mm, topsep=0mm, parsep=1mm, partopsep=1mm]
%     \item Provide a comprehensive evaluation framework for evaluating LLM based reviewing systems across 5 key dimensions: (1) Alignment with Human Reviews (2) Factual Correctness (3) Depth of Analysis (4) Actionable Insights and (5) Adherence to reviewer guidelines.
%     \item Develop an LLM based reviewer that dynamically aligns itself with target conference guidelines and improves its reviewing strategy iteratively through a refinement loop.
% \end{enumerate}

\begin{table*}[t]
    \centering
    \tiny
    \renewcommand{\arraystretch}{0.8} % Reduce row height slightly
    \begin{tabular}{l>{\centering\arraybackslash}m{2.5cm} >{\centering\arraybackslash}m{2.5cm}}
        \toprule
        & \shortstack{\textbf{Wider} \\ \textbf{Metric Coverage}} 
        & \shortstack{\textbf{Black Box} \\ \textbf{Remedy}} \\
        \midrule
        \cite{d2024marg} & \xmark & \xmark \\
        \midrule
        \cite{zhou2024llm} & \xmark & \xmark \\
        \midrule
        ReviewEval(ours) & \cmark & \cmark \\
        \bottomrule
    \end{tabular}
    \caption{Comparison of Evaluation Methods. The evaluation metrics by \cite{d2024marg} (MARG) and \cite{zhou2024llm} (Aspect-Based Comparisons) focus on AI-human review similarity but overlook other key deficiencies in AI reviews. Their heavy reliance on LLMs for end-to-end evaluation creates a black-box system with limited transparency. Our framework addresses these gaps with a more comprehensive and interpretable approach.}
    \label{tab:evaluation_comparison}
\end{table*}
%\vspace{-2mm}

% In summary, the objectives of the research are listed below.
% \vspace{-5mm}
% \setlist{nolistsep}
% \begin{enumerate}[noitemsep]
% \item Develop an LLM-based system capable of providing detailed, constructive feedback on academic manuscripts.\vspace{-1mm}
% \item Align reviews with specific conference guidelines to ensure relevance.\vspace{-2mm}
% \item Develop an extensive evaluation framework to evaluate several aspects of these AI-generated reviews
% \end{enumerate}

% Furthermore, we identify the requirement of researchers that are targeting a specific research conference, therefore, we propose a system an LLM-based conference specific reviewer. To the best of our knowledge, no other research has effectively explored aligning AI reviews to specific conferences.

% We use the conference specific reviewer guidelines and convert them into individual prompts to generate the reviews. These prompts are instructured to be in format of step by step instructions for the reviewer to follow. These prompts then go through a refinement loop where a Judge LLM checks them coherence, relevance, and miscelaneous improvements to align the prompt/instruction better with the conference guideline it is representing. The final conference specific reviews are then outputted in latex format in a structured format.




% Recent advancements in techniques like meta-prompting and multi-agent systems provide potential solutions to these challenges. Meta-prompting allows LLMs to break down complex tasks into smaller subtasks, handled by expert instances with tailored instructions. This approach has the potential to significantly enhance the quality of AI-based reviews by improving specificity and reducing generic feedback. Similarly, utilizing multi-agent systems for review generation \cite{d2024marg} helps distribute review tasks across specialized LLM instances, enabling detailed and nuanced feedback through simulated internal discussions.

% This paper builds upon these developments, proposing a novel system that integrates iterative refinements, prompt engineering, and section-specific evaluations to create a scalable and effective LLM-based peer review assistant.



% Furthermore, based on our preliminary experiments using these commercial models, we further observed some more shortcomings, like the AI reviews lacking actionable insights as well as AI reviews having limited analytical depth with only generic comments, these all arising because llms have the ability to generate vague reviews. Hence, using the existing literature \cite{zhou2024llm} and our preliminary experiments, we propose an evaluation framework with the following (i) evaluating factual innacuracies (ii) Quantitative analysis and comparison of AI reviews and human reviews (such as topic coverage and semantic similarities) (iii) presence of actionable insights (iv) analysing the depth of analysis in AI reviews.



% To further explore the challenges faced by LLMs as reviewers, we conducted preliminary experiments (refer to Figure 1) using commercial models like GPT-4 and Gemini Pro, with basic prompting strategies (details in Appendix). The results revealed several significant shortcomings that hinder their practical application, including factual inaccuracies (hallucinations), inconsistent review quality relative to human evaluations, limited analytical depth, and a lack of actionable insights, thereby rendering these models ineffective for delivering meaningful feedback.


% \cite{zhou2024llm} conducts a comprehensive analysis of the use of commercial models like gpt-3.5 and gpt-4 as research paper reviewer, and found that its shortcomings are possibility of mistakes during reviewing (caused by either model hallucinations or inability to comprehensively understand the material) and that these models fail to give critical feedback like human reviewers.
% To get a better and in-depth grasp of the reviewer challenges, we also conduct som preliminary experiments (add figure 1) with commercial LLMs such as GPT-4 and Gemini Pro, using basic prompting strategies (see Appendix). It revealed significant shortcomings that currently limit their practical application. These include the generation of factual inaccuracies (hallucinations), inconsistent review quality compared to human evaluations, insufficient depth of analysis, and a lack of actionable insights, rendering such reviews ineffective for providing meaningful feedback.


% Talk about existing review systems

% More recently, their application in academic reviewing has garnered attention, with efforts to align automated reviews with conference-specific guidelines to ensure constructive and high-quality feedback. By reducing the workload on human reviewers, such systems aim to maintain or even improve the quality and consistency of evaluations across submissions.

% Despite their promise, the deployment of LLMs in peer review also introduces challenges. Large language models are prone to hallucinations, generating plausible-sounding but inaccurate information. They also demonstrate the power to persuade humans, which can amplify the risks of inaccuracies in reviews. Studies have shown that controlling the quality and appropriateness of LLM-augmented reviewing is a complex and ongoing challenge.



% The academic community has already begun experimenting with AI-assisted reviews. For instance, 15.8\% of reviews for ICLR 2024 were reportedly written with AI assistance \cite{latona2024ai}. While these developments demonstrate the viability of LLMs in academic reviewing, they also highlight the need for further research to address limitations such as factual accuracy, depth of analysis, and alignment with conference guidelines.

% AI-based reviews present unique advantages for the academic community. They can provide early feedback to authors, allowing them to refine their work before submission. Additionally, these systems can help conferences maintain timely and high-quality reviews despite increasing submission volumes. Lastly, they offer an opportunity for improved quality control of reviews, addressing inconsistencies and biases that may arise in human-only review processes.

% Recent advancements in techniques like meta-prompting and multi-agent systems provide potential solutions to these challenges. Meta-prompting allows LLMs to break down complex tasks into smaller subtasks, handled by expert instances with tailored instructions. This approach has the potential to significantly enhance the quality of AI-based reviews by improving specificity and reducing generic feedback. Similarly, utilizing multi-agent systems for review generation \cite{d2024marg} helps distribute review tasks across specialized LLM instances, enabling detailed and nuanced feedback through simulated internal discussions.


% This paper builds upon these developments, proposing a novel system that integrates iterative refinements, prompt engineering, and section-specific evaluations to create a scalable and effective LLM-based peer review assistant.

% The exponential growth in academic research has led to an increased demand for efficient and high-quality peer review processes. Large Language Models (LLMs) have demonstrated potential in automating various natural language processing tasks, including summarization, translation, and question-answering. This project explores their capability to automate the critical task of reviewing academic papers. By aligning reviews with conference-specific guidelines and generating detailed feedback, this LLM-based research assistant aims to reduce human workload while maintaining the quality and consistency of manuscript evaluations.

% The academic community acknowledges the acute need for
% having foundation models assist reviewing of papers at scale
% (Liu and Shah 2023; Robertson 2023; Petrescu and Krishen
% 2022; Schulz et al. 2022; Checco et al. 2021; Bao, Hong,
% and Li 2021; Vesper 2018; Latona et al. 2024; Kuznetsov
% et al. 2024), along with the risks involved (Kaddour et al.
% 2023; Spitale, Biller-Andorno, and Germani 2023; Zou et al.
% 2023). Previous work addresses the limitations of LLM’s
% ability to perform reviewing (Liu and Shah 2023) and their
% capabilities to review academic papers (Liang et al. 2023).
% Large language models demonstrate surprising creative capabilities in text (Koivisto and Grassini 2023), though they
% may hallucinate (Zhang et al. 2023), and demonstrate the
% power to persuade humans even when inaccurate (Spitale,
% Biller-Andorno, and Germani 2023). This makes controlling
% the quality and appropriateness of LLM-augmented reviewing highly challenging. At least 15.8\% of reviews for ICLR
% 2024 were written with AI assistance (Latona et al. 2024).
% Meta-prompting (Suzgun and Kalai 2024) uses multiple
% LLM instances for managing and integrating multiple independent LLM queries. 

% Utilizing meta-prompting, the LLM
% breaks down complex tasks into smaller subtasks handled
% by expert instances with tailored instructions, significantly
% enhancing performance across various tasks. This approach
% outperforms conventional prompting methods across multiple tasks, enhancing LLM functionality without requiring
% task-specific instructions. Multi-agent review generation for
% scientific papers (D’Arcy et al. 2024) improves LLM reviewing by using multiple instances of LLMs, providing
% more specific and helpful feedback by distributing the text
% across specialized agents that simulate an internal discussion. This reduces generic feedback and increases the generation of good comments. Recent work formulates the peerreview process as a multi-turn dialogue between the different
% roles of authors, reviewers, and decision-makers (Tan et al.
% 2024), and finds that both reviews (Latona et al. 2024) and
% meta-reviews written by LLMs (Santu et al. 2024) are preferred by humans over human reviews and meta-reviews.

% Why do the Artificial Intelligence, Machine Learning, and
% Computer Vision communities need AI-based reviews of papers? (i) AI-based reviews provide early feedback to authors
% for their work in progress, allowing authors to learn and improve their work; (ii) AI-based reviews would help conferences maintain high-quality and timely reviews for the increasing number of papers in these fields, (iii) For quality control of reviews
%\vspace{-3mm}
\section{Related Work}
\textbf{AI based scientific discovery.} The pursuit of automating scientific discovery has been a longstanding goal in artificial intelligence (AI) research. Early efforts in the 1970s introduced expert systems such as DENDRAL~\cite{buchanan1981dendral} and the automated mathematician~\cite{lenat1977automated}, which focused on constrained problem spaces like organic chemistry and theorem proving. More recent advancements in AI-driven research have leveraged LLMs and machine learning techniques to extend beyond structured domains. Notable contributions include AutoML approaches that optimize hyperparameters and architectures~\cite{hutter2019automated, he2021automl} and AI-driven discovery in materials science and synthetic biology~\cite{merchant2023ai, hayes2024simulating}. However, these methods remain largely dependent on human-defined search spaces and predefined evaluation metrics, limiting their potential for open-ended discovery. Recent works~\cite{lu2024ai} aim to automate the entire research cycle, encompassing ideation, experimentation, manuscript generation, and peer review, thus pushing the boundaries of AI-driven scientific inquiry.\par

\textbf{AI based peer-review.} Recent advancements in AI-driven peer review have explored structured, multi-agent approaches for automated evaluation.~\cite{lu2024ai} employ LLMs to autonomously conduct the research pipeline, including peer review. It follows a structured three-stage review process: paper understanding, criterion-based evaluation (aligned with NeurIPS and ICLR guidelines), and final synthesis: assigning scores to key aspects like novelty, clarity, and significance. Evaluation shows its reviews closely match human meta-reviews (2.3/5 vs 2.5/5). MARG~\cite{d2024marg} introduces a multi-agent framework where worker agents review sections, expert agents assess specific aspects, and a leader agent synthesizes feedback. Using BERTScore~\cite{bert-score} and GPT-4-based evaluation, MARG-S improves feedback quality, reducing generic comments (60\% to 29\%) and increasing helpful feedback per paper (1.7 to 3.7). A user study found 47\% of MARG-S feedback helpful, outperforming baseline models (21\%). These studies highlight the potential of AI to enhance peer review through structured automation and multi-agent collaboration.\par

% \textbf{AI based peer-review.} Recent advancements in AI-driven peer review have explored structured, multi-agent approaches for automated evaluation.~\cite{lu2024ai} employ LLMs to autonomously conduct the research pipeline, including peer review. It follows a structured three-stage review process: paper understanding, criterion-based evaluation (aligned with NeurIPS and ICLR guidelines), and final synthesis: assigning scores to key aspects like novelty, clarity, and significance. Evaluation shows its reviews closely match human meta-reviews. MARG~\cite{d2024marg} introduces a multi-agent framework where worker agents review sections, expert agents assess specific aspects, and a leader agent synthesizes feedback. Using BERTScore~\cite{bert-score} and GPT-4-based evaluation, MARG-S improves feedback quality, reducing generic comments (60\% to 29\%) and increasing helpful feedback per paper (1.7 to 3.7). A user study found 47\% of MARG-S feedback helpful, outperforming baseline models (21\%). These studies highlight the potential of AI to enhance peer review through structured automation and multi-agent collaboration.\par


% \subsection{The AI Scientist (by Sakana AI)}
% \vspace{-1mm}
% \subsection{The AI Scientist (by Sakana AI)}
% \cite{lu2024ai} introduce The AI Scientist, an autonomous framework leveraging LLMs to conduct the full research pipeline, from ideation to peer review. The system generates research ideas using chain-of-thought reasoning and self-reflection, refines them via a Semantic Scholar API-assisted literature search, and executes experiments using Aider, an LLM-based coding assistant that iteratively debugs and refines implementations. To automate peer review, the system follows a structured process: (1) Paper Understanding, where the LLM summarizes key contributions, methodology, and results; (2) Criterion-Based Evaluation, assessing novelty, clarity, correctness, significance, and presentation using NeurIPS and ICLR guidelines; and (3) Final Review Synthesis, where individual evaluations are compiled into structured feedback with strengths, weaknesses, and improvement suggestions. The review system assigns numeric scores to each criterion, mimicking human assessment, and aggregates them into an overall Meta-Reviewer Score. AI-generated reviews were compared against NeurIPS 2021 Meta-Reviewer Scores, where human reviews averaged 2.5/5, while the AI system achieved 2.3/5, demonstrating near-human accuracy.

% \subsection{MARG: Multi-Agent Review Generation for Scientific Papers}
% \cite{d2024marg} propose MARG (Multi-Agent Review Generation) that employs multiple worker agents to review different sections, while expert agents specialize in key aspects like experiments, clarity, and impact. A leader agent coordinates the process, ensuring a structured and coherent review. The system uses hardcoded review prompts inspired by ICLR guidelines, which remain static across all review tasks. For evaluation, the authors compare MARG-generated reviews against NeurIPS human reviews using automated alignment metrics and a user study. BERTScore is used to measure semantic similarity between AI-generated and human-written reviews, capturing lexical and contextual overlap. Additionally, GPT-4 is used as a preference model, scoring reviews based on relevance, coherence, and completeness, simulating human judgment. The authors also conduct a user study with NLP and HCI researchers, who assess feedback helpfulness. MARG-S (a specialized agent variant) outperforms single-agent and naive multi-agent baselines, reducing generic feedback from 60\% to 29\% and increasing helpful comments per paper from 1.7 to 3.7. The user study reveals that 47\% of MARG-S feedback was rated as helpful, compared to 21\% from naive multi-agent models.



% introduces a comprehensive framework that automates the entire research pipeline, from idea generation to manuscript writing, using LLMs. A notable component is its foundation model-based automated reviewer, which evaluates generated manuscripts based on NeurIPS guidelines, providing scores and feedback akin to human reviewers. The framework achieves 65\% accuracy for final paper score prediction and demonstrates cost efficiency, but it is limited by its inability to adapt to specific conference guidelines. The paper also doesn't analyze and assess their AI generated reviews using meaningful metrics.
% \cite{d2024marg} presents a multi-agent framework (MARG) that generates peer-review feedback by distributing the task among multiple LLM agents. The system uses hardcoded review prompts inspired by ICLR guidelines, which remain static across all review tasks. MARG significantly reduces generic comments (from 60\% to 29\%) and generates more helpful feedback compared to baseline methods, as shown through a user study. The paper also proposes an automated review evaluation framework to check for approximate matches between AI-generated and human-written review comments using GPT. While GPT-4 is employed iteratively to extract approximate matches and mitigate inconsistencies, we argue that completely relying on GPT for end-to-end evaluation introduces unreliability and opacity. To address this, our evaluation framework emphasizes transparency, striving to develop interpretable metrics for comparing AI and human reviews, moving beyond black-box methodologies.
% \subsection{Is LLM a reliable reviewer? A Comprehensive Evaluation of LLM on Automatic Paper Reviewing Tasks}
\textbf{Evaluation framework for AI based research paper reviews.} There has been limited research on developing evaluation frameworks for evaluating the quality of LLM generated paper reviews.~\cite{zhou2024llm} evaluated GPT models for research paper reviewing across 3 tasks: aspect score prediction, review generation, and review-revision MCQ answering. Using the PeerRead dataset, they assessed LLM's ability to predict review scores based on clarity, originality, and soundness (aspect score prediction). For review generation, they employed the ASAP dataset \cite{yuan2022can} and tested zero-shot, few-shot, and aspect-guided structured prompting, where reviews were generated with explicit tagging for summary, originality, and clarity. The study developed evaluation framework comprising of aspect coverage, ROUGE (lexical overlap), BERTScore (semantic similarity), and BLANC (informativeness), alongside manual analysis. Aspect coverage measured how well the generated reviews addressed key aspects (originality, soundness, substance, replicability, etc.) when compared to the distribution in reference expert reviews. Results showed LLMs overemphasized positive feedback, lacked critical depth, and neglected substance and clarity, despite high lexical similarity to human reviews. \cite{d2024marg} introduced an automated evaluation framework for AI-generated reviews, quantifying similarity to human reviews via recall, precision, and Jaccard index. Their multi-stage GPT-4 \cite{achiam2023gpt} based alignment process first matches semantically equivalent comments using five randomized passes (many-many matching), retaining pairs appearing in at least two runs. Each pair is re-evaluated based on relatedness and relative specificity. Recall measures the fraction of real-reviewer comments with at least one AI match, precision quantifies AI comments aligned with human reviews, and Jaccard index evaluates the intersection-over-union of aligned comments.

% \cite{zhou2024llm} conducted a systematic evaluation of GPT-3.5 and GPT-4 for research paper reviewing task, focusing on three key tasks: aspect score prediction, review generation, and review-revision multiple-choice question answering (RR-MCQ). For aspect score prediction, the study used the PeerRead dataset (ICLR-2017 subset) to assess LLM's ability to infer review scores based on predefined criteria such as clarity, originality, and soundness. For review generation, \cite{zhou2024llm} used the ASAP dataset \cite{yuan2022can} (ICLR-2020 subset), which consists of peer reviews labeled by aspects such as originality, soundness, and clarity. Various prompting strategies—zero-shot, few-shot, and aspect-guided structured generation—were tested to improve review quality. Aspect-guided structured generation organizes reviews by characteristics like summary, originality, and clarity, using models to generate sentences tagged with corresponding aspects. The generated reviews were evaluated using automated metrics and human annotations. Among automated metrics, the study used aspect coverage to measure how well the generated reviews addressed key aspects of the paper, comparing the distribution of positive and negative comments with reference reviews. The results showed that LLMs overemphasized positive feedback, often failing to provide critical assessments. Certain aspects, such as substance and clarity, were frequently neglected, while models disproportionately favored broad statements over precise critiques. Other automated metrics included ROUGE (lexical overlap), BERTScore (semantic similarity), and BLANC (informativeness via a blank-filling task). While these metrics indicated high similarity between LLM-generated and human-written reviews, they failed to capture deficiencies in critical feedback and technical depth. \cite{d2024marg} proposed an automated evaluation framework with the following metrics used to quantify the AI generated reviews using the corresponding human (expert) review comments: Recall, Precision, and Jaccard Index. The paper employs a multi-stage GPT-4-based process to measure the overlap between generated and real reviews by identifying semantically equivalent comments. In the Many-Many Matching Stage, all comments from both sets are input into GPT-4, which performs five passes with randomized comment orders to mitigate inconsistencies. A comment pair is retained if it appears in at least two of five runs. In the Pairwise Stage, each identified pair is re-evaluated, where GPT-4 assigns two scores: Relatedness ("none," "weak," "medium," or "high") and Relative Specificity ("less," "same," or "more"). A match is confirmed if relatedness is at least "medium" and specificity is "same" or "more." This structured alignment enables the calculation of recall, precision, and Jaccard index for assessing review similarity. Recall is the fraction of real-reviewer comments that are aligned to any generated comment (total number of human reviews that have atleast 1 match with AI reviews divided by total number of human reviews). Precision is the fraction of generated comments that are aligned to any real-reviewer comment (total number of AI reviews that have atleast 1 match with human reviews divided by total AI reviews). Jaccard Index is calculated by dividing the intersection of generated comments and real-reviewer comments by the size of the union of the generated comments and real-reviewer comments less the intersection. The Jaccard index measures the similarity between the set of generated comments and the set of real-reviewer comments



% \cite{d2024marg} evaluates the MARG-generated reviews against NeurIPS human reviews using automated alignment metrics and a user study. BERTScore is used to measure semantic similarity between AI-generated and human-written reviews, capturing lexical and contextual overlap. Additionally, GPT-4 is used as a preference model, scoring reviews based on relevance, coherence, and completeness, simulating human judgment. They perform automated evaluation using GPT-4 to extract approximate matches between AI reviews and expert reviews, they run GPT inference iteratively to extract approximate matches.The authors also conduct a user study with NLP and HCI researchers, who assess feedback helpfulness.


% conducts a comprehensive evaluation of large language models (LLMs), specifically GPT-3.5 and GPT-4, in scientific paper reviewing, focusing on their ability to generate review texts. The study examines aspect coverage and similarity to reference reviews through both automatic metrics and manual analysis, utilizing the ASAP dataset \cite{yuan2022can}, which annotates review sentences with labels such as summary, motivation, originality, soundness, substance, replicability, meaningful comparison, and clarity. However, relying on pre-defined aspect coverage undermines the diversity and depth of AI and human reviews, therefore we recognized the need for a more nuanced and dynamic topic coverage metric.
% Papers are split into chunks, processed by worker agents, and analyzed by expert agents specializing in clarity, experiments, or impact, with a leader agent coordinating communication and combining feedback. The refinement stage focuses on improving the generated reviews for clarity, specificity, and validity. 


% \subsection{Existing work}


% % Prior work on LLM academic capabilities suggests that LLMs are now ready for specific reviewing tasks and appear to be more effective for some academic domains and less effective for others (Checco et al.
% % 2021; Schulz et al. 2022; Liu and Shah 2023; Lu et al. 2024)

% % \subsection{Current Applications of LLMs}

% % LLMs are widely used for tasks like language translation, summarization, and question-answering. Their potential for critical assessment of new research remains largely untapped.

% \subsection{Existing Automated Review Systems}

% Sakana AI Scientist: A system performing end-to-end research activities (literature review, hypothesis formulation, experimentation, manuscript writing).

% OpenReviewer: Utilizes GPT-4 for academic paper reviews.

% \subsection{Gaps in Current Systems}

% Existing solutions either aim to generalize across the entire research workflow or lack the depth needed for nuanced peer reviews.



% Note use of \abovespace and \belowspace to get reasonable spacing
% above and below tabular lines. Below is an example table.

% \begin{table}[t]
% \caption{Classification accuracies for naive Bayes and flexible
% Bayes on various data sets.}
% \label{sample-table}
% \vskip 0.15in
% \begin{center}
% \begin{small}
% \begin{sc}
% \begin{tabular}{lcccr}
% \toprule
% Data set & Naive & Flexible & Better? \\
% \midrule
% Breast    & 95.9$\pm$ 0.2& 96.7$\pm$ 0.2& $\surd$ \\
% Cleveland & 83.3$\pm$ 0.6& 80.0$\pm$ 0.6& $\times$\\
% Glass2    & 61.9$\pm$ 1.4& 83.8$\pm$ 0.7& $\surd$ \\
% Credit    & 74.8$\pm$ 0.5& 78.3$\pm$ 0.6&         \\
% Horse     & 73.3$\pm$ 0.9& 69.7$\pm$ 1.0& $\times$\\
% Meta      & 67.1$\pm$ 0.6& 76.5$\pm$ 0.5& $\surd$ \\
% Pima      & 75.1$\pm$ 0.6& 73.9$\pm$ 0.5&         \\
% Vehicle   & 44.9$\pm$ 0.6& 61.5$\pm$ 0.4& $\surd$ \\
% \bottomrule
% \end{tabular}
% \end{sc}
% \end{small}
% \end{center}
% \vskip -0.1in
% \end{table}

%\vspace{-3mm}

%Our experiments utilized 16 papers and their corresponding expert reviews (scraped from OpenReview.net), alongside AI-generated reviews. 

\section{ReviewEval}
We describe the development of our evaluation framework and the LLM-based research paper reviewer. Each review is evaluated on several key parameters to assess the overall quality of the generated feedback. To ensure consistency and reliability, all evaluations for a given metric were performed by LLMs of the same specification and version, thereby maintaining inter-rater reliability and ensuring robust, unbiased comparisons. Our experiments utilized 16 papers and their corresponding expert reviews (scraped from OpenReview.net), alongside AI-generated reviews.\par

%We describe the development of our evaluation framework and the development of our LLM based research paper reviewer. We tested 16 papers and reviews from experts (scraped from openreview.net) alongside AI-generated reviews across several parameters. First, we measured the \textit{alignment between AI-generated and expert reviews} using the cosine similarity of word embeddings. Next, we assessed \textit{topic coverage} by evaluating how comprehensively AI-generated reviews addressed the breadth of topics present in expert reviews using large language models. We also evaluated \textit{factual correctness} through a multi-agent fact-checking framework to verify the accuracy of the claims made. Additionally, we examined \textit{constructiveness} by determining whether the reviews provided actionable insights and recommendations, and finally, we measured the \textit{depth of analysis} by assessing the critical engagement of the reviews with the manuscript's content. To ensure consistency and reliability, all evaluations for a given metric were performed by LLMs of the same specification and version. This approach maintained inter-rater reliability across all samples, ensuring that comparisons and assessments were robust and unbiased.\par

% We tested the 16 papers and reviews from experts (scraped from openreview.net) and AI on the following parameters:
% \item \textbf{Comparing against Expert Reviews} Measured alignment between AI-generated and expert reviews using cosine similarity of word embeddings.
% \item \textbf{Topic Coverage:} Assessed how comprehensively AI-generated reviews covered the breadth of topics in expert reviews using LLMs.
% \item \textbf{Factual Correctness:} Evaluated the accuracy of claims in the reviews through a multi-agent fact-checking framework.
% \item \textbf{Constructiveness:} Determined whether reviews provided actionable insights and recommendations.
% \item \textbf{Depth of Analysis:} Assessed the critical engagement of reviews with the manuscript’s content.

% Adherence to Guidelines: Checked compliance with specific conference review criteria.

%We compared the reviews generated by SakanaAI scientist's review bot with those from our own reviewer bot on the selected dataset.

\subsection{Comparison with Expert Reviews}
We compare the reviews generated by the LLM based reviewer with expert reviews from OpenReview.net. Our primary goal is to gauge how well the AI system replicate or complement expert-level critique. The evaluation is conducted along the following dimensions:

\textbf{Semantic similarity.} 
To assess the alignment between AI-generated and expert reviews, we embed each review \( R \) into a vector space using the OpenAI 
embedding model. The semantic similarity between an AI-generated review \( R_{\text{AI}} \) and an expert review \( R_{\text{Expt}} \) is measured using cosine similarity:
\begin{equation}
S_{\text{sem}}(R_{\text{AI}}, R_{\text{Expt}}) = \frac{e(R_{\text{AI}}) \cdot e(R_{\text{Expt}})}{\|e(R_{\text{AI}})\| \, \|e(R_{\text{Expt}})\|}
\end{equation}

where \( e(R) \) denotes the embedding of review \( R \). A higher cosine similarity indicates a stronger alignment between the AI-generated and expert reviews.\par

% To assess the semantic similarity between AI-generated reviews and expert reviews, we measure the cosine similarity between the word embeddings which are generated by OpenAI Embedding Model. Each review $R$ was embedded as a vector, capturing semantic nuances of the text. We then calculated cosine similarity scores between the expert review embedding and each of the AI-generated review embeddings. It helps us measures the alignment between the AI and expert reviews in the embedding space, providing a numerical representation of their contextual similarity. Higher cosine similarity scores indicate a stronger alignment between the AI-generated content and expert perspectives.

% \begin{equation}
% S_{\text{semantic}} = \frac {e_{R_{AI}}.e_{R_{expert}}} {||e_{R_{AI}}|| ||e_{R_{expert}}||}
% \end{equation}

\textbf{Topic modeling.} We evaluate topic coverage to determine how comprehensively AI-generated reviews address the breadth of topics present in expert reviews. Our approach comprises three steps: \ding{182} \textit{Topic extraction:} Each review \( R \) (either AI-generated or expert) is decomposed into a set of topics: $T_R = \{ t_1, t_2, \dots, t_n \},$ where each topic \( t_i \) is represented by a sentence that captures its core content and context. \ding{183} \textit{Topic similarity:} Let $T_{\text{AI}} = \{ t_1, t_2, \dots, t_m \}$ and  $T_{\text{Expt}} = \{ t'_1, t'_2, \dots, t'_n \}$
denote the topics extracted from the AI and expert reviews, respectively. We define a topic similarity function \( \text{TS}(t_i, t'_j) \) that an LLM assigns on a discrete scale:
\begin{equation}
\begin{split}
\text{TS}(t_i, t'_j) = 3 \cdot \mathbb{I}\{ t_i \sim_{\text{strong}} t'_j \} \\
+ 2 \cdot \mathbb{I}\{ t_i \sim_{\text{moderate}} t'_j \}\\ 
+ 1 \cdot \mathbb{I}\{ t_i \sim_{\text{weak}} t'_j \},
\end{split}
\end{equation}
where $\mathbb{I}$ is the indicator function, $t_i \sim_{\text{strong}} t'_j$, $t_i \sim_{\text{moderate}} t'_j$ , $t_i \sim_{\text{weak}} t'_j$ denote substantial, moderate, and minimal overlap in concepts, respectively. All the conditions are mutually exclusive. We set a similarity threshold \( \tau = 2 \) so that topics with \(\text{TS}(t_i, t'_j) \geq \tau \) are considered aligned. \ding{184} \textit{Coverage ratio:} For each AI-generated review, we construct a topic similarity matrix \( S \) where each element $S[i, j] = \text{TS}(t_i, t'_j)$ represents the similarity between topic \( t_i \) from \( T_{\text{AI}} \) and topic \( t'_j \) from \( T_{\text{Expt}} \). The topic coverage ratio is defined as:
\begin{equation}
S_{\text{coverage}} = \frac{1}{n} \sum_{j=1}^{n} \mathbb{I}\left(\max_{i=1,\dots,m} S[i, j] \geq \tau\right),
\end{equation}
where \( \mathbb{I}(\cdot) \) is the indicator function, and \( n = |T_{\text{Expt}}| \) is the total number of topics extracted from the expert review.

% \underline{\textit{Topic extraction using LLM.}} We employed an LLM for extracting core topics. Each review $R$ (either AI-generated or expert-generated) was decomposed into a set of topics: $T_{R} = \{t_{1}, t_{2}, ..., t_{n} \} $ where each topic $t_i$ is represented by a sentence that clearly captures the essence, key details and provides sufficient context to understand the significance of the topic.\par
% To enhance the representation of topic embeddings, individual word embeddings were weighted based on their relevance to the topic. This weighting ensures that dominant keywords exert greater influence on the final topic representation. Let $w_{ij}$ denote the weight of word $j$ in topic $t_i$, and $e_{w_{ij}}$ be the embedding of the word $j$. The topic embedding $e_{t_i}$ was calculated as: \\
% \begin{equation}
%  e_{t_{i}} = \frac {\sum_{j}w_{ij}e_{w_{ij}}} {\sum_{j}w_{ij}} 
% \end{equation}
% \underline{\textit{Similarity for topic alignment.}} To measure the alignment between AI-generated reviews $R_{\text{AI}}$ and expert reviews $R_{\text{Expert}}$, we leverage advanced contextual understanding of Large Language Models to capture underlying themes in each topic. Let $T_{\text{AI}} = \{ t_1, t_2, \dots, t_m \}$ and $T_{\text{Expert}} = \{t'_1, t'2, \dots, t'n\}$ denote the sets of topics extracted from $R_{\text{AI}}$ and $R_{\text{Expert}}$, respectively. The  similarity score $\text{sim}(t_i, t'_j)$ between two topics $t_i$ and $t'_j$ is given as an output by an LLM on a scale from 0 to 3 as given by the following equation.

% \[ sim(t_{i}, t'_{j}) = \begin{cases} 3, & \text{if substantial overlap in concepts} \\ 2, & \text{if some overlap, but different details} \\ 1, & \text{if little overlap in ideas} \\ 0, & \text{if No meaningful connection.} \end{cases} \]
% A similarity threshold $\tau = 2$ was used to determine topic alignment. Topics with $\text{sim}(t_i, t'_j) \geq \tau$ were deemed sufficiently aligned.

% Uncomment for algorithm
% \begin{algorithm}[t]
% \caption{AI vs. Expert Review Evaluation}
% \label{alg:ai-expert-evaluation}

% {\raggedright
% \begin{algorithmic}[1]
% \Require AI-generated review $R_{AI}$, Expert review $R_{Expert}$, Embedding model $E$, Large Language Model $\mathcal{M}$, Similarity threshold $\tau$
% \Ensure Semantic similarity score $S_{\text{semantic similarity}}$, Topic coverage ratio $S_{\text{coverage ratio}}$

% \Statex
% \Function{EvaluateReviews}{$R_{AI}, R_{Expert}, E, \mathcal{M}, \tau$}

%     \State $e_{R_{AI}} \gets E(R_{AI})$ \Comment{Compute embedding of AI review}
%     \State $e_{R_{Expert}} \gets E(R_{Expert})$ \Comment{Compute embedding of expert review}
%     \State $S_{\text{semantic similarity}} \gets \frac{e_{R_{AI}} \cdot e_{R_{Expert}}}{\|e_{R_{AI}}\| \|e_{R_{Expert}}\|}$ \Comment{Compute cosine similarity}
    
%     \State $T_{AI} \gets \mathcal{M}(\text{``Extract topics from'' } R_{AI})$ 
%     \State $T_{Expert} \gets \mathcal{M}(\text{``Extract topics from'' } R_{Expert})$
    
%     \State $S \in \mathbb{R}^{|T_{AI}| \times |T_{Expert}|}, \quad S[i, j] = 0, \forall i, j$ \Comment{Initialize zero matrix}
%     \For{$i \gets 1$ \textbf{to} $|T_{AI}|$}
%         \For{$j \gets 1$ \textbf{to} $|T_{Expert}|$}
%             \State $S[i,j] \gets \mathcal{M}(\text{``Compare topics'' }, T_{AI}[i], T_{Expert}[j])$ 
%         \EndFor
%     \EndFor
    
%     \Comment{$S \in \{0,1,2,3\}$}
%     \State $count \gets 0$
%     \For{$j \gets 1$ \textbf{to} $|T_{Expert}|$}
%         \If{$\max_{i}(S[i, j]) \geq \tau$}
%             \State $count \gets count + 1$
%         \EndIf
%     \EndFor
%     \State $S_{\text{coverage ratio}} \gets \frac{count}{|T_{Expert}|}$
    
%     \State \Return $S_{\text{semantic similarity}}, S_{\text{coverage ratio}}$
% \EndFunction
% \end{algorithmic}}
% \end{algorithm}

% \underline{\textit{Topic coverage analysis.}} For each AI-generated review, a topic similarity matrix $S$ was constructed, where $S[i, j] = \text{sim}(t_i, t'j)$ represents the similarity between topic $t_i$ from $T_{\text{AI}}$ and topic $t'j$ from $T_{\text{Expert}}$. Using $S$, the \textit{topic coverage ratio}, i.e. the proportion of expert topics  $T_{Expert}$ covered by $T_{AI}$ is calculated as:
% % \[
% %        \text{Coverage Ratio} = \frac{\sum_{j=1}^{n} \mathbb{1}\left(\max_{i=1, \dots, m} S[i, j] \geq \tau\right)}{n}
% % \]
% \begin{equation}
%     S_{\text{coverage ratio}} = \frac{\sum_{j=1}^{n} 1 (\text{max}_{i=1,2,...m} S[i,j] \geq \tau)}{n}
% \end{equation}
% Here, 1(.) is an indicator function that evaluates to 1 if the condition is true, and n = $|T_{Expert}|$ is the number of expert topics. The threshold \( \tau \) (i.e., \( \tau = 2 \)) determines the minimum similarity required for a topic to be considered covered.

% \item Unique Topics in AI-Generated Reviews: Topics in \( T_{\text{AI}} \) that do not have a sufficiently similar counterpart in \( T_{\text{Expert}} \) are defined as unique topics:
% \begin{equation}
% T_{\text{Unique}} = \{t_i \in T_{\text{AI}} : \max_{j=1, \dots, n} S[i, j] < \tau\}
% \end{equation}

% \item Missed Expert Topics: Expert topics not represented in \( T_{\text{AI}} \) are identified as:

% \begin{equation}
% T_{\text{Missed}} = \{t'_j \in T_{\text{Expert}} : \max_{i=1, \dots, m} S[i, j] < \tau\}
% \end{equation}

\subsection{Evaluating Factual Correctness of Reviews}
To address the hallucinations and factual inaccuracies of LLM generated reviews, we propose an automated pipeline that validates the factual correctness of LLM-generated reviews by \textit{simulating the conference rebuttal process}. Our pipeline emulates the traditional rebuttal workflow, where authors clarify or counter reviewer claims using evidence from their work. By automating both the question generation and rebuttal phases, our system produces a robust factual correctness evaluation. The pipeline consists of the following steps:\par
\textit{\textbf{Step 1:} Transforming reviews into structured questions.} 
Each LLM-generated review \(R\) is transformed into a structured question \(Q\) that encapsulates the central claim or critique. For example, consider the following review from the PeerRead dataset for the paper \textit{"Augmenting Negative Representations for Continual Self-Supervised Learning"} \cite{chaaugmenting}:\par
\underline{Review (\(R\)):} ``\texttt{Augmentation represents a crucial area of exploration in self-supervised learning. Given that the authors classify their method as a form of augmentation, it becomes essential to engage in comparisons and discussions with existing augmentation methods.}''

This review is converted into the corresponding question:

\underline{Generated question (\(Q\)):} ``\texttt{Has the paper engaged in comparisons and discussions with existing augmentation methods, given that the authors classify their method as a form of augmentation?}''

\textit{\textbf{Step 2:} Decomposing questions into sub-questions.} The question \(Q\) is then decomposed into a set of sub-questions $\{q_1, q_2, \ldots, q_n\}$ using a dedicated query decomposition engine. This decomposition enables a fine-grained analysis by isolating distinct components of the original question.

\textit{\textbf{Step 3:} Retrieval-augmented generation (RAG) for evidence synthesis.} For each sub-question \(q_i\), we employ a Retrieval-Augmented Generation (RAG) framework to gather and synthesize relevant evidence: \circled{a} \textit{Section retrieval:} For each paper \(P\), we retrieve pertinent text segments \(S\) (approximately 400 tokens) via semantic search. \circled{b} \textit{Parent section extraction:} Using LangChain's Parent Document Retriever, we extract the parent sections \(S_p\) (approximately 4000 tokens) corresponding to each \(S\). Documents are pre-chunked hierarchically, ensuring that each \(S\) is mapped to its contextually relevant \(S_p\). \circled{c} \textit{Answer generation:} With the context provided by \(S_p\), the LLM generates an answer \(A_i\) for each sub-question \(q_i\). These individual answers are then aggregated into a unified, structured response \(A_Q\) addressing the original question \(Q\).\par

\textit{\textbf{Step 4:} Automated rebuttal generation.} The comprehensive answer \(A_Q\) is used to generate an automated rebuttal \(R_b\) for the original review \(R\). This rebuttal is designed to provide evidence-based clarification or counterarguments to the claims in \(R\).

\textit{\textbf{Step 5:} Factual correctness evaluation.} An evaluation agent then assesses the factual correctness of \(R\) by comparing it against the generated rebuttal \(R_b\). The review is deemed: \circled{a} \(\textbf{Valid}\ (\mathcal{V} = \text{True})\) if \(R_b\) substantiates the claims made in \(R\).
\circled{b} \(\textbf{Invalid}\ (\mathcal{V} = \text{False})\) if \(R_b\) reveals factual discrepancies or unsupported claims in \(R\).\par

% LLMs are increasingly being employed in academic peer review processes. However, concerns around hallucinations or factual inaccuracies in LLM-generated reviews necessitate robust validation mechanisms. To address this, we propose a pipeline that automates the process of validating the factual correctness of LLM-generated reviews by simulating the conference rebuttal process. This system emulates the traditional rebuttal workflow where authors clarify or counter reviewer claims by providing evidence or context from their work. The pipeline ensures rigor by automating both the question generation and rebuttal phases, ultimately producing a factual correctness evaluation. Below we explain the proposed pipeline:

% \textbf{Conversion of reviews into questions.} Each LLM-generated review, denoted as $R$ is converted into a structured question $Q$. For example, consider a review from the PeerRead dataset for the paper titled "Augmenting Negative Representations for Continual Self-Supervised Learning" \cite{chaaugmenting}: \\
% Review ($R$): ``Augmentation represents a crucial area of exploration in self-supervised learning. Given that the authors classify their method as a form of augmentation, it becomes essential to engage in comparisons and discussions with existing augmentation methods.''\par

% Generated Question ($Q$):
% ``Has the paper engaged in comparisons and discussions with existing augmentation methods, given that the authors classify their method as a form of augmentation?''\par


% Uncomment for algorithm
% \begin{algorithm}[t]
% \caption{Factual Correctness Evaluation of LLM-Generated Reviews}
% \label{alg:factual-correctness-evaluation}

% {\raggedright
% \begin{algorithmic}[1]
% \Require LLM-generated review $R$, Paper $P$, LLM model $\mathcal{M}$, SubQuestionQueryEngine $\mathcal{Q}$, Semantic Search Engine $\mathcal{S}$, Parent Document Retriever $\mathcal{D}$
% \Ensure Validity score $V$

% \Statex
% \Function{EvaluateReviewCorrectness}{$R, P, \mathcal{M}, \mathcal{Q}, \mathcal{S}, \mathcal{D}$}
%     \Comment{Step 1: Convert Review into Questions}
%     \State $Q \gets \mathcal{M}($``Convert review $R$ into a structured question''$)$

%     \Comment{Step 2: Decompose into Sub-Questions}
%     \State $\{q_1, q_2, ..., q_n\} \gets \mathcal{Q}(Q)$
    
%     \Comment{Step 3: Retrieve Contextual Information}
%     \For{$q_i \in \{q_1, q_2, ..., q_n\}$}
%         \State $S \gets \mathcal{S}(P, q_i, 400)$ \Comment{Retrieve relevant 400-token sections}
%         \State $S_p \gets \mathcal{D}(P, S, 4000)$ \Comment{Retrieve parent 4000-token section}
%         \State $A_i \gets \mathcal{M}($``Answer $q_i$ given context $S_p$''$)$
%     \EndFor
    
%     \Comment{Step 4: Generate Rebuttal}
%     \State $A_Q \gets \text{Concatenate}(A_1, A_2, ..., A_n)$
%     \State $R_b \gets \mathcal{M}($``Generate rebuttal from $A_Q$''$)$
    
%     \Comment{Step 5: Evaluate Review Correctness}
%     \If{$R_b$ agrees with claims in $R$}
%         \State $V \gets \text{True}$ \Comment{Review is valid}
%     \Else
%         \State $V \gets \text{False}$ \Comment{Review is invalid}
%     \EndIf
    
%     \State \Return $V$
% \EndFunction
% \end{algorithmic}}
% \end{algorithm}


% \textbf{Decomposition of questions into sub-questions.} Each question Q is decomposed into a set of sub-questions 
% $\{q_{1}, q_{2}, ....., q_{n}\}$
% using subquestion-based query engine. This decomposition isolates individual components of the original question, facilitating a granular evaluation.\par

% \textbf{Retrieval-Augmented Generation (RAG) for contextual information.} To answer each sub-question $q_{i}$, a RAG pipeline is employed as follows: \ding{182} For each paper $P$, relevant small sections $S$ of 400 tokens are retrieved using semantic search. \ding{183} Parent sections $S_{p}$ of size 4000 tokens that contain $S$ are then extracted using LangChain's Parent Document Retriever. Documents are pre-chunked into hierarchical levels, where each $S$ belongs to a predefined $S_{p}$. Retrieval first selects $S$ based on semantic similarity, and then the corresponding $S_{p}$ is retrieved using this parent-child mapping. \ding{184} Given $S_{p}$, the LLM generates answers $A_{i}$ for each $q_{i}$. These answers are then combined into a single, structured paragraph $A_{Q}$, forming a comprehensive response to $Q$.\par

% \begin{enumerate}
%     \item For each paper $P$, relevant small sections $S$ of 400 tokens are retrieved using semantic search.\vspace{-2mm}
%     \item Parent sections $S_{p}$ of size 4000 tokens that contain $S$ are then extracted using LangChain's Parent Document Retriever to provide the whole context.\vspace{-2mm}
%     \item Given $S_{p}$, the LLM generates answers $A_{i}$ for each $q_{i}$. These answers are then combined into a single, structured paragraph $A_{Q}$, forming a comprehensive response to $Q$.\vspace{-2mm}
% \end{enumerate}
% \textbf{Rebuttal generation.} The combined answer $A_{Q}$ serves as the automated rebuttal $R_{b}$ for the review $R$. This rebuttal provides evidence-based clarification or counters the claims made in $R$.

% \textbf{Evaluating review correctness} An evaluation agent is tasked with determining the validity $V$ of the review $R$ based on the rebuttal $R_{b}$. The review is marked as:

% \begin{enumerate}
%     \item $Valid(V=True$) if $R_{b}$ agrees with the claims made in $R$.
%     \item $Invalid(V=False)$ if $R_{b}$ identifies factual inaccuracies in $R$.
% \end{enumerate}

\subsection{Constructiveness}
We assess review constructiveness by quantifying the presence and quality of actionable insights in AI-generated reviews relative to expert feedback. Our framework begins by extracting key actionable components from each review using an LLM with few-shot examples. Specifically, we identify the following actionable insights: (i) \emph{criticism points} (\(C\)), which capture highlighted flaws or shortcomings in the paper’s content, clarity, novelty, and execution; (ii) \emph{methodological feedback} (\(M\)), which encompasses detailed analysis of experimental design, techniques, and suggestions for methodological improvements; and (iii) \emph{suggestions for improvement} (\(I\)), which consist of broader recommendations for enhancement such as additional experiments, alternative methodologies, or improved clarity.\par

Once these components are extracted, each insight is evaluated along three dimensions: specificity, feasibility, and implementation details. The specificity score \(\sigma\) is defined as 1 if the insight is specific and includes explicit examples, and 0 otherwise; the feasibility score \(\phi\) is set to 1 when the recommendation is practical within the paper's context, and 0 otherwise; and the implementation details score \(\zeta\) is 1 if actionable steps or detailed methodologies are provided, and 0 otherwise. The overall actionability score for an individual insight is then computed as $S_{\text{act},i} = \sigma_i + \phi_i + \zeta_i,$ with an insight considered actionable if \(S_{\text{act},i} > 0\). Finally, we quantify the overall constructiveness of a review by calculating the percentage of actionable insights:
\begin{equation}
 S_{\text{act}} = \frac{1}{N} \sum_{i=1}^{N} \mathbb{I}\big(S_{\text{act},i} > 0\big) \times 100,   
\end{equation}
where \(N\) is the total number of extracted insights and \(\mathbb{I}(\cdot)\) denotes the indicator function. This metric provides a quantitative measure of how effectively a review offers concrete guidance for improving the work.

% To evaluate whether the review carried out includes specific suggestions that are clear and actionable in guiding the authors to improve the paper, we measured the presence of actionable insights within the feedback provided by AI-generated reviews and compared them with the gold standard of expert reviews. Our framework systematically identifies actionable suggestions and assesses their quality through a multi-step process outlined below.

% \textbf{Extracting actionable insights.} The constructiveness evaluation begins by extracting specific components from each review using an LLM backed by few shot examples. \ding{182} \textit{Criticism points ($C$):} Highlighted flaws, shortcomings and aspects that detract the paper quality. This focuses on paper's content, clarity, novelty, and execution.
% \ding{183} \textit{Methodological feedback ($M$):} Feedback on the methodologies used in the paper, including suggestions for improvement, detailed analysis of the paper's experimental design, techniques, or approach, focusing on strengths, weaknesses, and potential areas for improvement.
% \ding{184} \textit{Suggestions ($I$):} Broader recommendations for enhancing the work which can be suggesting additional experiments, alternative methodologies, or improved clarity.\par

% Uncomment for algorithm
% \begin{algorithm}[t]
% \caption{Constructiveness Evaluation}
% \label{alg:constructiveness-evaluation}

% {\raggedright
% \begin{algorithmic}[1]
% \Require Review text $R$, Large Language Model $\mathcal{M}$
% \Ensure Constructiveness score $S_{\text{actionable}} \in [0,1]$

% \Statex
% \Function{EvaluateConstructiveness}{$R, \mathcal{M}$}
%     \State $C \gets \text{ExtractCriticism}(R, \mathcal{M})$ 
%     \State $M \gets \text{ExtractMethodologicalFeedback}(R, \mathcal{M})$ 
%     \State $I \gets \text{ExtractSuggestions}(R, \mathcal{M})$ 
    
%     \State $Insights \gets C \cup M \cup I$ \Comment{Combine extracted insights}
%     \State $N \gets |Insights|$ \Comment{Total number of extracted insights}
%     \State $S \gets [0]^N$ \Comment{Initialize scores array}
    
%     \For{$j \gets 1$ \textbf{to} $N$}
%         \State $\sigma_j \gets \Call{EvaluateSpecificity}{\mathcal{M}, insights_j}$
%         \State $\phi_j \gets \Call{EvaluateFeasibility}{\mathcal{M}, insights_j}$
%         \State $\iota_j \gets \Call{EvaluateImplementationDetails}{\mathcal{M}, insights_j}$
%         \State $S_{\text{actionable},j} \gets \sigma_j + \phi_j + \iota_j$
%     \EndFor
    
%     \State $S_{\text{actionable}} \gets \frac{\sum_{j=1}^{N} \mathbf{1}(S_{\text{actionable},j} > 0)}{N} \times 100$
    
%     \State \Return $S_{\text{actionable}}$
% \EndFunction

% \Statex
% \Function{EvaluateSpecificity}{$\mathcal{M}, C_j$}
%     \State $score \gets \mathcal{M}(\text{``Is insight $C_j$ specific?``})$
%     \State \Return $score$ \Comment{$score \in \{0,1\}$}
% \EndFunction


% \Function{EvaluateFeasibility}{$\mathcal{M}, C_j$}
%     \State $score \gets \mathcal{M}(\text{``Is insight $C_j$ feasible?``})$
%     \State \Return $score$
%     \Comment{$score \in \{0,1\}$}
% \EndFunction

% \Function{EvaluateImplementationDetails}{$\mathcal{M}, C_j$}
%     \State $score \gets \mathcal{M}(\text{``Does insight $C_j$ have implementation details?
%     ``})$
%     \State \Return $score$
%     \Comment{$score \in \{0,1\}$}
% \EndFunction

% \end{algorithmic}}
% \end{algorithm}


% \textbf{Assessing feasibility and specificity.} Once $C$, $M$ and $I$ are extracted, they are assessed for quality along the following dimensions:

% \textit{Specificity:} This metric evaluates whether the feedback addresses a specific aspect of the paper, identifies precise issues or improvement areas, and avoids vague language by providing explicit examples or references. The specificity score $\sigma$ is defined as:
% \[
% \sigma = 
% \begin{cases} 
% 1, & \text{if specific and includes explicit examples.} \\
% 0, & \text{otherwise.}
% \end{cases}
% \]

% \textit{Feasibility:} This measures the practicality of the suggested actions, considering the context of the paper, available resources, and implementation challenges while proposing manageable solutions. The feasibility score $\phi$ is defined as:
% \[
% \phi = 
% \begin{cases} 
% 1, & \text{if suggests practical and feasible actions.} \\
% 0, & \text{otherwise.}
% \end{cases}
% \]

% \textit{Implementation details:} This assesses whether the feedback provides actionable steps, detailed methodologies, or explicit techniques, including references to prior work, tools, or parameters to support implementation. The implementation details score $\iota$ is defined as:
% \[
% \iota = 
% \begin{cases} 
% 1, & \text{if includes actionable steps or methodologies.} \\
% 0, & \text{otherwise.}
% \end{cases}
% \]
% \par

% \textbf{Calculating the actionability score.} Each actionable insight is assigned a score based on the sum of the binary values from the above nodes. The actionability score $S_{\text{actionable}}$ for an individual insight is defined as:
% \[
% S_{\text{actionable},i} = \sigma_{\text{i}} + \phi_{\text{i}} + \iota_{\text{i}}
% \]
% An insight is deemed actionable if $S_{\text{actionable}} > 0$, as even a minimal score can provide a foundation for authors to improve their work.\par

% \textbf{Percentage of actionable insights.} To gauge the overall constructiveness of a review, we calculated the percentage of actionable insights:
% \[
% \text{Percentage of Actionable Insights} = \frac{\sum_{i=1}^{N} S_{\text{actionable},i}}{N} \times 100
% \]

% \noindent where:
% \begin{itemize}
%     \item $N$ is the total number of insights extracted.
%     \item $S_{\text{actionable},i}$ represents the actionable score for the $i$-th insight.
% \end{itemize}
% \[
% S_{\text{actionable}} = \frac{\sum_{i=1}^{N} \mathbf{1}(S_{\text{actionable},i} > 0)}{N} \times 100
% \]

% \noindent where: $N$ is the total number of insights extracted. $\mathbf{1}(\cdot)$ is the indicator function, which equals 1 if $S_{\text{actionable},i} > 0$, and 0 otherwise. Where $S_{\text{actionable},i}$ represents the actionable score for the $i$-th insight. This metric provides a quantitative measure of how well the review guides authors in improving their work.\par

\subsection{Depth of Analysis}
To assess whether a review provides a comprehensive, critical evaluation rather than a superficial commentary, we measure the depth of analysis in AI-generated reviews. This metric captures how thoroughly a review engages with key aspects of a paper, including comparisons with existing literature, identification of logical gaps, methodological scrutiny, interpretation of results, and evaluation of theoretical contributions.\par

Each review is evaluated by multiple LLMs, which assign scores for each of the five dimensions, \(m_i\) (\(i \in \{1,2,3,4,5\}\)), with scores \(S_i \in [0,1]\). Scores in the continuous range allow us to capture nuances in performance. We define the metrics as follows:\par

\textit{Comparison with existing literature (\(m_1\)):} Assesses whether the review critically examines the paper's alignment with prior work, acknowledging relevant studies and identifying omissions. The scoring rubric is:
\begin{equation*}
\resizebox{.9\hsize}{!}{
S_1 = \begin{cases} 
3, & \text{if the review provides a thorough, critical comparison,} \\
2, & \text{if the comparison is meaningful yet shallow,} \\
1, & \text{if the comparison is vague or lacks specific references,} \\
0, & \text{if no comparison is provided.}
\end{cases}
}
\end{equation*}

\textit{Logical gaps identified (\(m_2\)):} Evaluates the review's ability to detect unsupported claims, reasoning flaws, and to offer constructive suggestions:
\begin{equation*}
\resizebox{.9\hsize}{!}{
S_2 = \begin{cases} 
3, & \text{if the review identifies comprehensive gaps and provides suggestions,} \\
2, & \text{if it notes some gaps with unclear recommendations,} \\
1, & \text{if the gaps are vaguely mentioned without solutions,} \\
0, & \text{if no gaps are identified.}
\end{cases}
}
\end{equation*}

\textit{Methodological scrutiny (\(m_3\)):} Measures the depth of critique regarding the paper’s methods, including evaluation of strengths, limitations, and improvement suggestions:
\begin{equation*}
\resizebox{.9\hsize}{!}{
S_3 = \begin{cases} 
3, & \text{if the review delivers a thorough critique with actionable suggestions,} \\
2, & \text{if the critique is meaningful but lacks depth,} \\
1, & \text{if the critique is vague and offers little insight,} \\
0, & \text{if no methodological critique is provided.}
\end{cases}
}
\end{equation*}

\textit{Results interpretation (\(m_4\)):} Assesses how well the review interprets the results, addressing biases, alternative explanations, and broader implications:
\begin{equation*}
\resizebox{.9\hsize}{!}{
S_4 = \begin{cases} 
3, & \text{if the interpretation is detailed and insightful,} \\
2, & \text{if it is meaningful yet shallow,} \\
1, & \text{if the discussion is generic or vague,} \\
0, & \text{if no interpretation is offered.}
\end{cases}
}
\end{equation*}

\textit{Theoretical contribution (\(m_5\)):} Evaluates the assessment of the paper’s theoretical contributions, including its novelty and connections to broader frameworks:
\begin{equation*}
\resizebox{.9\hsize}{!}{
S_5 = \begin{cases} 
3, & \text{if the evaluation is comprehensive and insightful,} \\
2, & \text{if the evaluation is meaningful but lacks depth,} \\
1, & \text{if the critique is vague,} \\
0, & \text{if no theoretical assessment is provided.}
\end{cases}
}
\end{equation*}

The overall depth of analysis score for a review is calculated as the average normalized score across all dimensions:
\begin{equation}
S_{\text{depth}} = \frac{\sum_{i=1}^{5} S_i}{15}.
\end{equation}
A higher \(S_{\text{depth}}\) indicates a more comprehensive and critical engagement with the manuscript.

% To assess whether the review provides a comprehensive evaluation that goes beyond surface-level observations, we measured the depth of analysis within AI-generated reviews. This metric evaluates how thoroughly the review engages with various critical aspects of the paper, such as the comparison with existing literature, identification of logical gaps, methodological scrutiny, results interpretation, and theoretical contribution. Our framework evaluates how well the review engages with key components, ensuring it addresses the significance and implications of the paper.
% Uncomment for algorithm
% \begin{algorithm}[t]
% \caption{Depth of Analysis Evaluation}
% \label{alg:depth-of-analysis}

% {\raggedright
% \begin{algorithmic}[1]
% \Require Review text $R$, Large Language Model $\mathcal{M}$
% \Ensure Depth of analysis score $S_{\text{depth}} \in [0,1]$

% \Statex
% \Function{EvaluateDepthOfAnalysis}{$R, \mathcal{M}$}
%     \State $m[1] \gets \text{Existing Literature Comparison}$
%     \State $m[2] \gets \text{Logical Gaps Identification}$
%     \State $m[3] \gets \text{Methodological Scrutiny}$
%     \State $m[4] \gets \text{Results Interpretation}$
%     \State $m[5] \gets \text{Theoretical Contributions}$


%     \State $N \gets |m|$ \Comment{Number of evaluation criteria}
%     \State $S \gets [0]^N$ \Comment{Initialize score array}
    
%     \For{$i \gets 1$ \textbf{to} $N$}

%         \State ${S}[i] \gets \Call{EvaluateCriterion}{\mathcal{M}, m_i}$ \Comment{Score for criterion $m[i]$}
%     \EndFor
    
%     \State $S_{\text{total}} \gets \sum_{i=1}^N {S}[i]$ \Comment{Aggregate scores}
%     \State $S_{\text{depth}} \gets S_{\text{total}} / (3N)$ \Comment{Normalize score to range $[0,1]$}
    
%     \State \Return $S_{\text{depth}}$
% \EndFunction


% \Statex
% \Function{EvaluateCriterion}{$\mathcal{M}, C_i$}

%     \State $score \gets \mathcal{M}(\text{``Rate $R$ from 0 $\rightarrow$ 3 based on $C_i$.``})$ \Comment{LLM-based evaluation of the review for criterion $C_i$}

    
%     \State \Return $score$
% \EndFunction


% \end{algorithmic}}
% \end{algorithm}
% \textbf{Evaluation metrics for depth of analysis.} The review is processed through multiple LLMs, each independently evaluating the following metrics and assigning a score between 0 and 1, (higher is better). Scores in the range \( (0, 1) \) are also considered to account for the gray area. Mathematically, for each metric \( m_i \), the score \( S_i \) is assigned such that \( S_i \in [0, 1] \), with higher values indicating stronger performance.

% \textit{Comparison with existing literature (\(m_1\)):} Assesses whether the review critically evaluates the alignment or divergence of the paper with prior studies, including the acknowledgment of relevant work and identification of omissions or oversights. Rubrics are defined by the equation below:
%  \[ S_1 = \begin{cases} 3, & \text{if thorough, critical comparison} \\ 2, & \text{if meaningful but shallow comparison} \\ 1, & \text{if vague, lacking specific references} \\ 0, & \text{if no comparison with prior work} \end{cases} \]
% where higher values indicate a more thorough and critical comparison.

% \textit{Logical gaps identified (\(m_2\)):} Evaluates the review's ability to identify unsupported claims, assumptions, gaps in reasoning, or lack of evidence, as well as the provision of constructive suggestions to address these issues. The rubric is defined as follows:
%  \[ S_2 = \begin{cases} 3, & \text{thorough gaps + suggestions} \\ 2, & \text{some gaps, unclear suggestions} \\ 1, & \text{vague gaps, no suggestions} \\ 0, & \text{no gaps identified} \end{cases} \]
 
% \textit{Methodological scrutiny (\(m_3\)):} Assesses the critique of the paper's methodologies, including strengths, weaknesses, limitations, and suggestions for improvement. The rubric is defined as follows:
% \[ S_3 = \begin{cases} 3, & \text{thorough critique with suggestions} \\ 2, & \text{meaningful critique, lacks depth} \\ 1, & \text{vague critique, no insights} \\ 0, & \text{no critique provided} \end{cases} \]

% \textit{Results interpretation (\(m_4\)):} Evaluates the alignment of the results' interpretation with the data, addressing biases, alternative explanations, and broader implications. The rubric is defined as follows:  
% \[ S_4 = \begin{cases} 3, & \text{detailed, insightful, comprehensive} \\ 2, & \text{meaningful but lacks depth} \\ 1, & \text{vague, generic discussion} \\ 0, & \text{no interpretation provided} \end{cases} \]

% \textit{Theoretical contribution (\(m_5\)):} Assesses the evaluation of the paper’s theoretical contributions, including novelty, connections to broader frameworks, and limitations. The rubric is defined as follows:
% \[ S_5 = \begin{cases} 3, & \text{comprehensive, insightful evaluation} \\ 2, & \text{meaningful but lacks depth} \\ 1, & \text{vague, no meaningful critique} \\ 0, & \text{no assessment provided} \end{cases} \]

% Each metric score (\(S_i\)) provides a quantitative evaluation of the depth of analysis in the review, where:  
% \[
% i \in \{1, 2, 3, 4, 5\}
% \]  
% \textbf{Calculating the depth of analysis score.} To calculate the depth of analysis score for a review, we use the formula: 
% \[
% S_{\text{depth}} = \frac{\sum_{i=1}^{M} S_i}{3M}
% \]

% \noindent where, \(M = 5\) is the total number of evaluation metrics. \(S_i\) is the score assigned to the \(i\)-th metric, where \(i \in \{1, 2, 3, 4, 5\}\). This metric quantifies the overall depth of analysis by averaging the scores across all metrics, with higher values indicating a more comprehensive and critical engagement with the manuscript's content.

\subsection{Adherence to Reviewer Guidelines}
To assess whether a review complies with established criteria, we evaluate its adherence to guidelines set by the venue. This metric measures how well the reviewer applies key aspects such as originality, methodology, results, clarity, and ethical considerations, thereby ensuring an objective and structured evaluation process.

%To evaluate the extent to which the review adheres to established guidelines, we assess how well the reviewer follows the criteria and expectations set by the journal or conference. This metric measures whether the reviewer consistently applies the prescribed framework, including key aspects such as originality, methodology, results, clarity, and ethical considerations. Our framework ensures that the review process remains structured and aligned with the goals of thorough, unbiased, and fair evaluation. By testing adherence to these guidelines, we ensure that reviews are objective, consistent, and meaningful, contributing to the overall quality of the academic review process.

% Uncomment for algorithm
% \begin{algorithm}[t]
% \caption{Adherence Evaluation to Reviewer Guidelines}
% \label{alg:adherence-evaluation-v2}

% \begin{algorithmic}[1]
% \Require Review guidelines $G$, review text $R$, LLM $\mathcal{M}$
% \Ensure Adherence score $S_{\text{adherence}} \in [0,1]$

% \Statex
% \Function{EvaluateAdherence}{$G, R, \mathcal{M}$}
%     \State $C \gets \Call{ExtractCriteria}{\mathcal{M}, G}$
%     \Comment{Extract evaluation criteria from guidelines}

%     \State $N \gets |C|$ \Comment{Number of evaluation criteria}
    
%     \State $\mathbf{S} \gets [0]^N$ \Comment{Initialize score array for $N$ criteria}
%     % \\
    
%     \For{$i \gets 1$ \textbf{to} $N$}
%         \State $P_i \gets \Call{GeneratePrompt}{\mathcal{M}, C_i}$ \Comment{Create criterion-specific prompt}
%         \State $\mathbf{S}[i] \gets \Call{Evaluate}{\mathcal{M}, P_i, R}$ \Comment{Get score from LLM where $S_i \in \{0,1,2,3\}$}

%     \EndFor
    
%     \State $S_{\text{total}} \gets \sum_{i=1}^N \mathbf{S}[i]$ \Comment{Aggregate scores}
%     \State $S_{\text{adherence}} \gets S_{\text{total}}/(3N)$ \Comment{Normalize to [0,1] scale}
    
%     \State \Return $S_{\text{adherence}}$
% \EndFunction

% \Statex
% \Function{ExtractCriteria}{$\mathcal{M}, G$}
%     \State \Return $\mathcal{M}$(``Extract evaluation criteria from: $G$'') \Comment{LLM-based extraction}
% \EndFunction

% \Statex
% \Function{GeneratePrompt}{$\mathcal{M}$, $C_i$}
%     \State \Return ``Rate 0-3 how the review addresses: $C_i$. Just give number.'' \Comment{Example prompt template}
% \EndFunction

% \end{algorithmic}
% \end{algorithm}

%\textbf{Extracting criteria for adherence evaluation.} The evaluation begins by identifying criteria $C$ from the review guidelines $G$ which the reviewer is expected to follow. These criteria can be divided into two broad categories: \ding{182} \textit{Subjective Criteria:} These involve qualitative aspects, such as clarity and accuracy in summarizing the paper, or the ability to provide constructive feedback. \ding{183} \textit{Objective Criteria:} These involve quantifiable aspects, such as assigning scores or following a specific rating scale within the review structure. An LLM is tasked with extracting these criteria from the provided guidelines. The extracted criteria form the foundation for the adherence evaluation.

Our approach begins by extracting the criteria \(C\) from the guidelines \(G\). These criteria fall into two broad categories: \ding{182} \emph{subjective} criteria, which involve qualitative judgments (e.g., clarity, constructive feedback), and \ding{183} \emph{objective} criteria, which are quantifiable (e.g., following a prescribed rating scale). For each review \(R\), every extracted criterion \(C_i\) is scored on a 0-3 scale using a dedicated LLM with dynamically generated prompts that include few-shot examples for contextual calibration. For subjective criteria, the score is defined as:
\begin{equation*}
\resizebox{.9\hsize}{!}{
S_i = \begin{cases}
3, & \text{if there is strong adherence with detailed, accurate feedback;} \\
2, & \text{if the review shows reasonable alignment with minor deviations;} \\
1, & \text{if the feedback is incomplete or inaccurate;} \\
0, & \text{if there is no alignment.}
\end{cases}
}
\end{equation*}

For objective criteria, the scoring is binary:
\begin{equation*}
\resizebox{.9\hsize}{!}{
S_i = \begin{cases}
3, & \text{if the review adheres to the required scale and structure;} \\
0, & \text{otherwise.}
\end{cases}
}\end{equation*}

The overall adherence score is then computed as:
\begin{equation}
S_{\text{adherence}} = \frac{\sum_{i=1}^{2} S_i}{6},
\end{equation}

This normalized score provides a quantitative measure of how well the review conforms to the prescribed guidelines.

% \textbf{Assigning scores to criteria.} For review $R$ each extracted criterion $C_i$ is assigned a score $S_{i}$ on a scale of 0 to 3 by a dedicated instance of large language model, where higher scores indicate better adherence. Prompts for this scoring task are dynamically generated by a factory function, ensuring alignment with the specific criterion. The prompts are tailored to assess adherence based on predefined rules, distinguishing between subjective and objective criteria:

% \textit{Subjective criteria scoring:}
% \[ 
% S_i = \begin{cases} 
%     3, & \text{strong adherence with detailed} \hidewidth \\
%        & \text{and accurate feedback} \\
%     2, & \text{reasonable alignment with} \hidewidth \\
%        & \text{minor errors or deviations} \\
%     1, & \text{incomplete or inaccurate feedback,} \hidewidth \\
%        & \text{significant deviation} \\
%     0, & \text{No alignment with the criterion}
% \end{cases}
% \]

% \textit{Objective criteria scoring:}
% \[ S_i = \begin{cases} 3, & \text{adheres to the scale and standards} \\ 0, & \text{fails to rate or uses invalid scale} \end{cases} \] 

% \textbf{Dynamic prompt generation.} The prompts for evaluating each criterion are crafted dynamically by a large language model to ensure that the scoring task is well-defined and contextually appropriate. For subjective criteria, the prompts include specific few shot examples of high, medium, low, and no adherence, to help the model better align its scores to the expectations aiming for better and more consistent results. For objective criteria, the prompts emphasize the importance of following the required scale and structure along with few shot examples.

% \textbf{Calculating adherence score.} The scores $S_i$ assigned to each criterion are aggregated to compute both an overall adherence score $S_{\text{adherence}}$. The overall adherence score is defined as:

% \[
% S_{\text{adherence}} = \frac{\sum_{i=1}^N S_i}{3N}
% \]

% where $N$ is the total number of extracted criteria. $S_i$ is the score assigned to the criterion $C_i$. This score provides a quantifiable measure of guideline adherence ($S_{\text{adherence}}$), enabling comprehensive assessment of review quality.

% \subsection{Conference-Specific Prompt Alignment}
% A key aspect of our approach is dynamically aligning the LLM-based review process to the target conference's reviewing guidelines. In this section we describe the process to achieve this.
\subsection{Conference-Specific Review Alignment}
% To align the review process with conference-specific guidelines, we use the URL of the target conference's official reviewing guideline website. Using Extractor API, we extract the textual content from the webpage, ensuring all relevant criteria are captured. The extracted text is then processed using GPT, which refines the content by removing extraneous information while preserving essential reviewing instructions. Each guideline is subsequently transformed into a step-by-step instructional prompt using GPT, explicitly specifying the sections of the paper it applies to. Since certain review criteria may span multiple sections, prompts are dynamically assigned to relevant sections, with some prompts being applicable to more than one. By performing reviewing in a section-wise manner, we optimize processing efficiency, allowing for independent evaluation of each section before aggregating the final review. The details of all prompts used in this part are defined in the appendix.

To tailor the review process to conference-specific guidelines, we first retrieve the relevant textual content from the target conference's official reviewing website using the Extractor API. The extracted text is then processed using GPT \cite{achiam2023gpt} to filter out extraneous details while preserving essential reviewing instructions. Each guideline \(g_i\) is converted into a step-by-step instructional prompt via GPT:
\[
P_i = \text{GeneratePrompt}(g_i),
\]
where \(P_i\) denotes the prompt corresponding to guideline \(g_i\). Since some criteria apply to multiple sections of a research paper, the prompts are dynamically mapped to the relevant sections. This is achieved using a mapping function:
\[
S_j = \mathcal{M}(P_i),
\]
where \(S_j\) is the set of paper sections associated with prompt \(P_i\). Notably, \(\mathcal{M}\) is a one-to-many mapping, i.e., 
\[
\mathcal{M}: P \to \mathcal{P}(S),
\]
with \(\mathcal{P}(S)\) denoting the power set of all sections. By conducting reviews on a section-wise basis, our framework enhances processing efficiency and allows for independent evaluation of each section prior to aggregating the final review. Detailed descriptions of all prompts are provided in the Appendix.

% To align the review process with conference-specific guidelines, we first retrieve the textual content from the target conference's official reviewing guideline website using the Extractor API. The extracted text is processed using GPT \cite{achiam2023gpt} to remove extraneous information while preserving essential reviewing instructions.  
% Each guideline \( g_i \) is transformed into a step-by-step instructional prompt using GPT \cite{achiam2023gpt}, defined as $P_i = \text{GeneratePrompt}(g_i)$, where \( P_i \) represents the instructional prompt corresponding to guideline \( g_i \).

% Some reviewing criteria may be applicable to multiple sections of a research paper, so the prompts are dynamically assigned to relevant sections using a mapping function defined as $S_j = \mathcal{M}(P_i)$ where \( S_j \) denotes the set of paper sections relevant to prompt \( P_i \). The function \( \mathcal{M} \) extracts relevant sections by querying the prompt $P_i$. Some prompts may apply to multiple sections, making \( \mathcal{M} \) a one-to-many mapping denoted as $\mathcal{M}: P \to \mathcal{P}(S) $, where \( \mathcal{P}(S) \) represents the power set of all possible sections.  
% By performing reviewing in a section-wise manner, we optimize processing efficiency, allowing for independent evaluation of each section before aggregating the final review. The details of all prompts used in this part are defined in the appendix.

% To optimize processing efficiency, the review process is conducted **section-wise**, allowing independent evaluation of each section before aggregating the final review:  

% \begin{equation}
%     R_{\text{final}} = \bigcup_{j} R_j
% \end{equation}  

% where \( R_{\text{final}} \) represents the aggregated review, and \( R_j \) corresponds to the review generated for section \( S_j \).  

% All prompts used in this methodology are detailed in the **Appendix**.



% We first obtain the URL of the
% conference’s official review guidelines. Using a third-party tool, Extractor
% API, we extract the textual content of the guidelines
% webpage, ensuring that all reviewing criteria are cap-
% tured. We guide GPT to structure the relevant reviewer guideline in an easy to read format by removing uncecessary data that might be on the website (refer to prompt in appendix section). Finally, each of these guidelines are then converted into step-by-step instructional prompt using GPT (Refer appendix section). Each reviewing prompt generated from the conference guideline might need to access different sections, and since we have the step by step instructional prompt, it must have which section needs to be checked in it. So, we assign each prompt to individual sections of the paper, with the same prompt maybe going to more than one section. This way, we don't have the process the whole paper altoghether, and we can do reviewing section-wise at first, combining our results later on.

% \begin{enumerate}
%     \item \textbf{Guideline Extraction:} We first obtain the URL of the conference’s official review guidelines.
%     \item \textbf{Content Retrieval:} Using a third-party tool, Extractor API, we extract the textual content of the guidelines webpage, ensuring that all reviewing criteria are captured.
%     \item \textbf{Guideline Structuring:} The extracted content is processed to generate detailed, section-wise reviewing instructions. Each section (e.g., Introduction, Related Work, Methodology, Results) is assigned specific criteria based on conference priorities such as impact assessment, logical soundness, clarity, completeness, and adherence to reproducibility standards.
%     \item \textbf{Instruction Prompt Generation:} The section-wise criteria are transformed into structured step-by-step instructional prompts for the LLM to follow while reviewing the paper.
% \end{enumerate}

% By ensuring that the reviewing process aligns closely with the expectations of the target conference, we enable more accurate and relevant assessments tailored to the research community's standards.

% \subsection{Research Paper Parsing and Preprocessing}
% Before inference, we preprocess the research paper to ensure all relevant content, including text and illustrations, is structured appropriately. \vspace{-3mm}

% \begin{enumerate}
%     \item \textbf{PDF Parsing:} We utilize \texttt{pymupdf} from the \texttt{langchain\_community} library to extract the paper’s text content as a single structured object.\vspace{-1mm}
%     \item \textbf{Mathematical Equations and Figures:} For images, equations, and illustrations, we employ \texttt{pymupdf}'s multimodal LLM-based image parser. This converts images into markdown format with detailed descriptions, preserving contextual information critical for scientific analysis.
% \end{enumerate}


% \subsection{Review Prompt Iterative Refinement}
% To enhance the effectiveness and completeness of the review prompts, we employ an iterative refinement process involving a Supervisor LLM. Given an initial set of prompts generated from the reviewing guidelines, each prompt undergoes structured evaluation and improvement. The Supervisor LLM takes as input the initial prompt along with the corresponding reviewing guideline, which serves as the problem statement. It then critiques the prompt based on multiple key aspects, including clarity, logical consistency, alignment with the guideline, and comprehensiveness. Clarity ensures that the instructions are precise and unambiguous, logical consistency verifies that the reasoning within the prompt is coherent, and alignment ensures that the prompt accurately reflects the intended purpose of the guideline. Additionally, comprehensiveness is assessed by identifying whether the prompt covers all relevant aspects a human reviewer would consider, including those that an AI model might otherwise overlook. The critique provided by the Supervisor LLM is then fed back to the original LLM instance that generated the prompts, which revises the prompt based on the feedback. This iterative refinement loop continues for a fixed number of iterations, set to three in our implementation, ensuring progressive improvement at each step. After completing the refinement process, we obtain a final set of structured, high-quality review prompts that are well-aligned with the reviewing guidelines and optimized for the evaluation process. 

\subsection{Review Prompt Iterative Refinement}
To enhance the quality and completeness of our review prompts, we employ an iterative refinement process using a Supervisor LLM. Starting with an initial set of prompts generated from the reviewing guidelines, the Supervisor LLM evaluates each prompt in conjunction with its corresponding guideline, serving as the problem statement, and provides targeted feedback. This feedback addresses key aspects such as clarity (ensuring precise and unambiguous instructions), logical consistency (ensuring coherent reasoning), alignment with the guideline, and comprehensiveness (ensuring all relevant aspects are covered). The feedback is then used to revise the prompt, and this iterative loop is repeated for a fixed number of iterations (three in our implementation). The result is a final set of structured, high-quality review prompts that are well-aligned with the reviewing guidelines and optimized for the evaluation process.

% The algorithm is explained in algorithm~\ref{alg:refinement}.

% \begin{algorithm}[t]
% \caption{Iterative Refinement of Review Prompts}
% \label{alg:refinement}
% \DontPrintSemicolon

% {\raggedright
% \begin{algorithmic}[1]
% \KwIn{
%     $P = \{P_1, P_2, ..., P_n\}$ \tcp{Initial set of review prompts} \\
%     $G = \{G_1, G_2, ..., G_n\}$ \tcp{Corresponding review guidelines} \\
%     $N$ \tcp{Number of refinement iterations (default: 3)}
% }
% \KwOut{$P^* = \{P_1^*, P_2^*, ..., P_n^*\}$ \tcp{Final refined prompts}}

% \ForEach{$P_i \in P$}{
%     \For{$t = 1$ \KwTo $N$}{
%         \tcp{Step 1: Supervisor LLM evaluates the prompt}
%         $C_i^t \gets \text{SupervisorLLM}(P_i, G_i)$ \tcp{Generate critique}

%         \tcp{Step 2: Original LLM revises the prompt}
%         $P_i \gets \text{PromptGeneratorLLM}(P_i, C_i^t)$ \tcp{Update prompt}
%     }
% }
% \Return $P^* = P$
% \end{algorithmic}}
% \end{algorithm}

% Uncomment for algorithm
% \begin{algorithm}
% \caption{Iterative Refinement of Review Prompts}
% \label{alg:refinement}
% \begin{algorithmic}[1]
% \Require $P = \{P_1, P_2, ..., P_n\}$ \Comment{Initial review prompts}
% \Require $G = \{G_1, G_2, ..., G_n\}$ \Comment{Corresponding guidelines}
% \Require $N$ \Comment{Number of refinement iterations (default: 3)}
% \Ensure $P^* = \{P_1^*, P_2^*, ..., P_n^*\}$ \Comment{Final refined prompts}

% \For {$i = 1$ to $n$} \Comment{Iterate over each prompt}
%     \For {$t = 1$ to $N$} \Comment{Perform iterative refinement}
%         \State $C_i^t \gets \text{SupervisorLLM}(P_i, G_i)$ \Comment{Generate critique}
%         \State $P_i \gets \text{PromptGeneratorLLM}(P_i, C_i^t)$ \Comment{Update prompt}
%     \EndFor
% \EndFor
% \State \Return $P^* = P$
% \end{algorithmic}
% \end{algorithm}




% Once we have a set of initial review prompts, we employ an iterative refinement algorithm to make these instructional prompts more effective and well-rounded.
% Each prompt goes through a refinement
% Supervisor LLM gets input: Initial generated Prompt + the corresponding guideline (the problem statement)
% The supervisor LLM is instructed to critique the prompt regarding multiple key dimensions: clarity, logical consistencies, its alignment to the given problem statement, and any aspect it overlooked for reviewing considering how a human would approach reviewing in a well-rounded manner (covering aspects the AI model would have missed). 
% This critique is then fed back to the main LLM that created the prompts from the guideline. This LLM uses the critique to improve its initial prompt and make it makes changes to it using it. This process repeats for a number of iteration (hardcoded to 3 in our research). After a series of refinements, we have a final set of prompts.




% \subsubsection{Initial Review Generation}
% Using the extracted reviewing criteria, the system generates an initial review for each section by prompting the LLM with the corresponding structured instruction set.

% \subsubsection{Reflection-Based Prompt Refinement}
% To enhance the quality and coherence of the reviews, we employ a structured \textbf{Reflection Loop}, iterating over the review process to refine the responses.

% \begin{enumerate}
%     \item The LLM receives the structured reviewing prompt and generates an initial review for a given section.
%     \item The \textbf{Reflection Prompt Generator LLM} is then used to assess the quality of the generated response.
%     \item The input to this model includes:
%           \begin{itemize}
%               \item The problem statement (i.e., system instruction + review prompt).
%               \item The generated response (initial review).
%           \end{itemize}
%     \item The Reflection Prompt Generator LLM analyzes the response, identifying weaknesses such as lack of depth, missing key aspects, or inconsistencies.
%     \item It then generates an improved prompt with suggestions for refining the review.
%     \item The original reviewing LLM receives this refined prompt along with its previous response and generates an updated review.
%     \item This iterative loop runs for a predefined number of iterations (\texttt{num\_iterations}, currently set to 3), ensuring progressive refinement of the review.
% \end{enumerate}

% At the end of this process, we obtain a structured Python dictionary containing:
% \begin{center}
% \texttt{\{section : guidelines for reviewing\}}
% \end{center}
% which serves as the final reviewing instruction set.

% \subsection{Final Review Generation}
% Once the structured reviewing guidelines have been refined, the research paper undergoes a second inference pass using the final reviewing criteria. The same Reflection Loop is applied to enhance the quality of the generated reviews.

% \subsection{Supervised Validation Attempt}
% We initially explored using a separate "supervisor" LLM instance to validate the reviews. This model was designed to output a binary decision (\texttt{YES/NO}) based on predefined quality criteria. However, this approach often resulted in infinite loops of rejection (\texttt{NO}), indicating the challenges of fully automated review validation. Consequently, this strategy was discarded, reinforcing the need for structured refinement rather than rigid rejection-based feedback.

% \subsection{LaTeX Review Generation}
% The final structured review is compiled into a LaTeX format for direct integration into academic workflows. A separate LLM instance converts the review dictionary into a well-formatted LaTeX document, ensuring that the final output is readily usable by authors and conference organizers.

\section{Experiments \& Results}
\textbf{Dataset.} We selected a sample of 16 papers from the NeurIPS 2024 submissions available on OpenReview.net. These papers were chosen randomly, with a balanced split: 4 papers from each category of accept decision ("oral", "poster" and "spotlight") and 4 that were rejected. To ensure diversity and avoid potential biases in topic/keyword coverage, we specifically included papers that contained unique keywords within each acceptance category. By filtering for papers with distinctive keywords, we aim to include a broad range of research topics, mitigating the risk of over-representation of any single theme or field in our dataset.\\
The papers in dataset are relatively recent, offering a challenging test set that many flagship large language models (LLMs) are unlikely to have encountered given their knowledge cutoffs. This approach enables us to fairly assess each model’s capability to interpret and analyze unseen data effectively.

\subsection{Constructiveness}

\begin{table}[ht]
\centering
\begin{tabular}{|l|c|c|c|c|c|c|c|c|c|}
\hline
\textbf{Model} & \textbf{$\mu + \sigma$}\\
\hline
Ours-GPT-4o & $0.7483 \pm 0.1275$ \\
Ours-GPT-4o-Mini & $0.7385 \pm 0.0952$ \\
Ours-3.5-Sonnet & $0.7513 \pm 0.1669$ \\
Ours-3.5-Haiku & $0.4943 \pm 0.2070$ \\
\hline
Sakana-GPT-4o & $0.7909 \pm 0.1193$ \\
Sakana-GPT-4o-Mini & $0.7916 \pm 0.1246$ \\
Sakana-3.5-Sonnet & $0.6956 \pm 0.1475$ \\
Sakana-3.5-Haiku & $0.4634 \pm 0.1380$ \\
\hline
MARG-GPT-4o & $0.7253 \pm 0.1638$ \\
MARG-GPT-4o-mini & $0.7428 \pm 0.1815$ \\
\hline
Expert & $0.7522 \pm 0.1206$ \\
\hline
\end{tabular}
\caption{Mean scores and standard deviations of different frameworks/models for actionable insights.}
\label{constructiveness}
\end{table}

% \begin{figure}[ht]
% \centering
% \includegraphics[width=\linewidth]{constructiveness.png}
% \caption{Box Plot for presence of actionable insights}
% % \label{fig:example_image} % Assign a label for referencing
% \end{figure}

Table \ref{constructiveness} summarizes the performance of reviews generated using different frameworks with variety of foundation models and expert reviews based on the presence of actionable insights. Expert reviews achieved a mean score of 0.7522, establishing a high-quality benchmark. Notably, Sakana-4o-Mini (0.7916) and Sakana-4o (0.7909) produced feedback comparable to or even exceeding expert performance. Ours-GPT-4o model (0.7483) demonstrated competitive results, closely aligning with expert-level feedback, while Ours-GPT-4o-Mini (0.7385) also performed strongly, indicating the effectiveness of these models in generating actionable insights. In contrast, smaller models like Ours-3.5-Haiku (0.4943) and Sakana-3.5-Haiku (0.4634) lagged, underscoring the importance of model capacity and architecture. Additionally, higher variability in certain models (e.g., Ours-3.5-Sonnet, $\sigma = 0.1669$) indicates inconsistent feedback quality across different papers.

\subsection{Factual Correctness}

\subsection{Adherence to reviewer guidelines}

\begin{table}[ht]
\centering
\begin{tabular}{|l|c|c|c|c|c|c|c|c|c|}
\hline
\textbf{Model} & \textbf{$\mu + \sigma$} \\
\hline
Ours-GPT-4o & $0.6823 \pm 0.0917$ \\
Ours-GPT-4o-Mini & $0.5785 \pm 0.2174$ \\
Ours-3.5-Sonnet & $0.6658 \pm 0.1180$ \\
Ours-3.5-Haiku & $0.4631 \pm 0.1197$ \\
\hline
Sakana-GPT-4o & $0.6327 \pm 0.0808$ \\
Sakana-GPT-4o-Mini & $0.6013 \pm 0.0669$ \\
Sakana-3.5-Sonnet & $0.6400 \pm 0.0297$ \\
Sakana-3.5-Haiku & $0.6313 \pm 0.0398$ \\
\hline
MARG-GPT-4o & $0.3206 \pm 0.0364$ \\
MARG-GPT-4o-mini & $0.3031 \pm 0.0396$ \\
\hline
Expert & $0.5708 \pm 0.1133$ \\
\hline
\end{tabular}
\caption{Mean scores and standard deviations of different frameworks/models for adherence to review guidelines.}
\label{adherence}
\end{table}

% \begin{figure}[ht]
% \centering
% \includegraphics[width=\linewidth]{adherence.png}
% \caption{Box Plot for adherence of reviewer guidelines metric}
% % \label{fig:example_image} % Assign a label for referencing
% \end{figure}

The table \ref{adherence} above presents the performance of various models and expert reviews in adhering to established journal or conference review guidelines. Expert reviews attained a mean score of 0.5708, providing a baseline for high-quality adherence. Among the models, Ours-GPT-4o (0.6823) outperformed the expert benchmark, demonstrating strong adherence to review criteria. Similarly, Ours-3.5-Sonnet (0.6658) and Sakana-3.5-Sonnet (0.6400) showed competitive performance, although it is to be noted that SakanaAi's AI scientist is hardcoded to follow NeurIPS guidlines while our reviewer adjusts itself depending on the given guidelines.
Smaller models have a lower score with higher variability underscoring the significance of model size and architecture in ensuring structured and guideline-compliant academic reviews.

\subsection{Comparing with expert reviews}

\begin{table}[ht]
\centering
\begin{tabular}{|l|c|c|}
\hline
\textbf{Model} & \textbf{$\mu + \sigma$} \\
\hline
Ours-GPT-4o       & $0.6788 \pm 0.2546$ \\
Ours-GPT-4o-Mini  & $0.6924 \pm 0.1650$ \\
Ours-3.5-Sonnet   & $0.6782 \pm 0.1468$ \\
Ours-3.5-Haiku    & $0.8013 \pm 0.1572$ \\
\hline
Sakana-GPT-4o        & $0.6863 \pm 0.1445$ \\
Sakana-GPT-4o-Mini    & $0.7342 \pm 0.1799$ \\
Sakana-3.5-Sonnet & $0.7159 \pm 0.1407$ \\
Sakana-3.5-Haiku  & $0.6641 \pm 0.1467$ \\
\hline
MARG-GPT-4o & $0.6971 \pm 0.1549$ \\
MARG-GPT-4o-mini & $0.5439 \pm 0.1636$ \\
\hline
\end{tabular}
\caption{Mean scores and standard deviations of different frame-
works/models for coverage of expert topics in the review.}
\label{tab:coverage}
\end{table}

\begin{table}[ht]
\centering
\begin{tabular}{|l|c|c|}
\hline
\textbf{Model} & \textbf{$\mu + \sigma$} \\
\hline
Ours-GPT-4o       & $0.8171 \pm 0.0724$ \\
Ours-GPT-4o-Mini  & $0.8170 \pm 0.0346$ \\
Ours-3.5-Sonnet   & $0.8379 \pm 0.0204$ \\
Ours-3.5-Haiku    & $0.8274 \pm 0.0210$ \\
\hline
Sakana-GPT-4o        & $0.8440 \pm 0.0226$ \\
Sakana-GPT-4o-Mini    & $0.8344 \pm 0.0161$ \\
Sakana-3.5-Sonnet & $0.8195 \pm 0.0169$ \\
Sakana-3.5-Haiku  & $0.8130 \pm 0.0177$ \\
\hline
MARG-GPT-4o & $0.8005 \pm 0.0206$ \\
MARG-GPT-4o-mini & $0.7926 \pm 0.0246$ \\
\hline
\end{tabular}
\caption{Mean scores and standard deviations of different frame-
works/models for semantic similarity to the expert reviews}
\label{tab:similarity}
\end{table}

% \begin{figure}[ht]
% \centering
% \includegraphics[width=\linewidth]{topic coverage.png}
% \caption{Box Plot for proportion of expert review topics covered}
% % \label{fig:example_image} % Assign a label for referencing
% \end{figure}

% \begin{figure}[ht]
% \centering
% \includegraphics[width=\linewidth]{semantic similarity.png}
% \caption{Box Plot for semantic similarity with expert reviews}
% % \label{fig:example_image} % Assign a label for referencing
% \end{figure}

The Tables \ref{tab:coverage}, \ref{tab:similarity} show the proportion of topics covered by different frameworks that align with expert reviews in both coverage and semantic similarity. Expert reviews serve as the reference standard.

Ours-GPT-4o (coverage: 0.6788, similarity: 0.8171) and Sakana-4o (coverage: 0.6863, similarity: 0.8440) performed competitively, demonstrating strong alignment with expert feedback. Sakana-4o-Mini (coverage: 0.7342, similarity: 0.8344) even exceeded expert-like coverage in some cases.

Coverage varied across models. Ours-3.5-Haiku (0.8013) and Sakana-3.5-Haiku (0.6641) covered more topics but showed lower similarity, indicating broader yet less expert-aligned feedback. Ours-3.5-Sonnet (0.6782 coverage, 0.8379 similarity) balanced coverage and similarity better but had slightly lower recall.

Standard deviation analysis shows stable similarity scores ($\sigma \sim 0.02$) but greater variation in coverage ($\sigma \leq 0.25$), suggesting models match expert phrasing well but struggle with comprehensive topic selection.

\subsection{Depth of Analysis}
\begin{table}[ht]
\centering
\begin{tabular}{|l|c|c|}
\hline
\textbf{Model} & \textbf{$\mu + \sigma$} \\
\hline
Ours-GPT-4o       & $0.5028 \pm 0.0638$ \\
Ours-GPT-4o-Mini  & $0.5243 \pm 0.0628$ \\
Ours-3.5-Sonnet   & $0.5417 \pm 0.0918$ \\
Ours-3.5-Haiku    & $0.5597 \pm 0.1107$ \\
\hline
Sakana-GPT-4o         & $0.5208 \pm 0.0602$ \\
Sakana-GPT-4o-Mini    & $0.4944 \pm 0.0402$ \\
Sakana-3.5-Sonnet & $0.4875 \pm 0.0737$ \\
Sakana-3.5-Haiku  & $0.4597 \pm 0.0611$ \\
\hline
MARG-GPT-4o & $0.6833 \pm 0.0725$ \\
MARG-GPT-4o-mini & $0.7014 \pm 0.0832$ \\
\hline
Expert            & $0.6264 \pm 0.0722$ \\
\hline
\end{tabular}
\caption{Depth of Analysis scores (mean and standard deviation) for each model across 16 papers.}
\label{tab:depth_analysis}
\end{table}

% \begin{figure}[ht]
% \centering
% \includegraphics[width=\linewidth]{depth.png}
% \caption{Box Plot for depth of analysis done in the research paper review}
% % \label{fig:example_image} % Assign a label for referencing
% \end{figure}

The results in table \ref{tab:depth_analysis} show that expert reviews achieve the highest depth of analysis (0.6264), outperforming all AI models. Among AI-generated reviews, Ours-3.5-Haiku (0.5597) and Ours-3.5-Sonnet (0.5417) provide the most comprehensive evaluations, while Sakana-3.5-Haiku (0.4597) scores the lowest. The Ours-GPT-4o model (0.5028) performs better than most Sakana models but falls short of the top-performing AI models. Notably, Ours-3.5-Haiku has the highest variance (0.1107), indicating inconsistency across papers, whereas Sakana-4o-Mini (0.0402) is more stable but has lower absolute depth.
% Our system consists of four tightly integrated components designed to handle the complex task of research paper reviewing:\\ \textit{Conference Guideline Extraction}, \textit{Customized Paper Parsing}, \textit{Dynamic Prompt Generation and Refinement}, and \textit{Review Generation and Final Compilation}. Below, we describe each component and its significance in ensuring the robustness of the system.

% \subsection*{Step 1: Conference Guideline Extraction}
% Accurate alignment with conference-specific review criteria is critical to ensuring that AI-generated reviews are relevant and valuable. The system takes a conference's URL as input and uses an LLM (e.g., GPT) to extract reviewer guidelines by searching for keywords such as "Reviewer Guidelines," "Review Criteria," and "Instructions for Reviewers."

% \begin{tcolorbox}[colframe=blue!50!black, colback=blue!5, title=Example Guideline Extraction]
% \textbf{Guideline from Conference A:} \textit{Reviews should assess originality, relevance, and technical soundness. Include specific comments on the clarity and potential impact of the work.}
% \end{tcolorbox}

% % \noindent These extracted guidelines are structured into categories such as clarity, originality, and relevance, forming the foundation for dynamic prompt generation (Step 3).

% \subsection*{Step 2: Customized Paper Parsing}
% The system uses a specialized parser based on \texttt{ScienceParser} to extract content from the research paper PDF, ensuring accurate segmentation by sections (e.g., Introduction, Methodology, Results). Key enhancements include:
% \begin{itemize}
%     \item \textbf{Section-Based Parsing:} Content is extracted and organized by section headings for targeted analysis.
%     \item \textbf{Validation by Secondary LLM:} A second LLM reviews the parsed content to ensure accuracy.
% \end{itemize}
% To handle complex content like mathematical formulas, we employ \texttt{Gemini 1.5 Pro}, which converts formulas into LaTeX for consistency and applicability.

% % \begin{tcolorbox}[colframe=red!50!black, colback=red!5, title=Example Output]
% % \textbf{Parsed Content for "Methodology" Section:}
% % \begin{verbatim}
% % \{ "Methodology": "This paper proposes a novel framework using \textbf{X} \
% % and achieves state-of-the-art results. The core algorithm is:\n$F(x) = x^2 + ax + b$.\"\}
% % \end{verbatim}
% % \end{tcolorbox}

% \subsection*{Step 3: Dynamic Prompt Generation and Refinement}
% Based on parsed content and extracted conference guidelines, the system dynamically generates prompts for reviewing each paper section. These prompts include both general review aspects (e.g., clarity, completeness, originality) and conference-specific criteria.

% \noindent The prompt generation process is formalized as:
% \begin{align}
% P_{\text{review}} = P_{\text{generic}} + P_{\text{conference-specific}},
% \end{align}
% where $P_{\text{generic}}$ represents standard review instructions, and $P_{\text{conference-specific}}$ tailors the review to the conference's requirements. 

% These prompts undergo an iterative refinement process using a "judge LLM" to ensure step-by-step clarity and human-like structure, mimicking how expert reviewers analyze papers.

% \begin{tcolorbox}[colframe=green!50!black, colback=green!5, title=Refined Prompt Example (to change)]
% \textit{"For the \textbf{Introduction} section: Assess whether the paper provides sufficient background and motivation for the work. Evaluate clarity, relevance to the conference themes, and the originality of the problem statement."}
% \end{tcolorbox}

% \subsection*{Step 4: Review Generation and Final Compilation}
% Each section is reviewed independently using the refined prompts using separately initialized LLM instances for each section. Reviews are iteratively improved through another round of feedback from the "judge LLM." The final reviews are compiled and formatted into a professional LaTeX document adhering to academic standards.

% \noindent The formatted review includes:
% \begin{itemize}
%     \item Standardized headings for clarity.
%     \item Consistent typography, ensuring readability.
%     \item Inclusion of LaTeX-rendered formulas and equations where applicable.
% \end{itemize}

% \subsection{Overview}

% The system employs a multi-stage process involving:

% Data Ingestion: Extracting textual and structural content from research papers and retrieving conference review guidelines.

% Prompt Generation: Creating tailored prompts for each paper section to guide the review process.

% Reflection Loop: Refining prompts and reviews iteratively to enhance clarity and relevance.

% Section-wise Review Generation: Generating detailed feedback for individual sections using separate LLM instances.

% Presentation: Formatting the final review using LaTeX for professional submission.

% \subsection{Data Ingestion}

% Research Paper Input: Research papers in PDF format are processed to extract textual content and structural details (e.g., headings, figures, references).

% Reviewing Guidelines Retrieval: Conference guidelines are fetched via user-provided URLs, parsed, and structured for integration into the review process.

% \subsection{Prompt Generation}

% An LLM instance (Planner) segments the paper into six key sections: Motivation, Prior Work, Approach, Evidence, Contribution, and Presentation.

% Specific prompts are generated for each section, designed to elicit critical and constructive feedback aligned with the reviewing guidelines.

% \subsection{Reflection Loop}

% A supervisory LLM evaluates and refines the prompts for clarity, alignment, and relevance.

% This iterative refinement ensures high-quality prompts that guide the review generation effectively.

% \subsection{Section-wise Review Generation}

% Each section is reviewed independently by a dedicated LLM instance, ensuring unbiased and focused evaluations.

% The generated reviews undergo iterative refinement to enhance depth and adherence to guidelines.

% \subsection{Presentation}

% Final reviews are compiled and formatted using LaTeX, adhering to academic conventions for professional presentation.

% The formatted review is made available for user download via a web interface built with Streamlit.
\section{Conclusion}
%The experiments where conducted in a controlled environment and we shall test the security against prompt injection attack
% \bibliography{example_paper}
% \bibliographystyle{icml2025}

\bibliography{example_paper}

\appendix
\section{Appendix}
\label{sec:appendix}

\subsection{Limitations}
\subsection{Ethical Considerations}
\subsection{Additional Results}
\end{document}

\subsection{Your software is useful to others only if they know about it}
\label{sec:community}

\paragraph{Background} In the previous recommendations, we discussed various points on how to build high-quality reusable software. However, software quality is meaningless if the community is unaware that the software project exists.
Having external users increases the scientific impact of the software and improves its quality by making it more thoroughly tested with various scenarios and edge cases. Finally, every user of your software is a potential future collaborator.
While \ref{sec:openSource} already discusses the first points as publishing the code open source and registering it in a registry, there are more ways to get attention to your \ac{ERS}.
Therefore, we will provide recommendations on increasing the number of users and overall reach of your \ac{ERS}.
\par 

\paragraph{Recommendations} The natural step to increase the visibility of your software is to publish a paper about it. While we discussed in \ref{sec:documentation} that software papers are not a replacement for documentation, they are great for increasing visibility and motivating your software. Consequently, a software paper should not contain too much technical details from the documentation. Instead, it should explain why the software is important, discuss its high-level capabilities, and convey its scope and vision. We recommend \ac{JOSS}\footref{fn:joss} \cite{jossJournal} as a modern software journal. If you want to discuss domain-specific aspects in more detail, it can also be helpful to publish your software in a conventional domain-specific journal, e.g., as it was done for \textit{pandapower} \cite{pandapower2018} and \textit{renewables.ninja} \cite{pfenninger_long-term_2016}.

\par 
The second essential step is the active engagement with the community.
Especially in the early stage of the software project, it is a good approach to contact domain experts and actively collect early feedback. First, this makes them aware of your software. Second, it prevents the developed tool from missing the needs of the researchers working in the field. Later in the process, you can actively present your software project at a conference, e.g., at the "Open Source Modelling and Simulation of Energy Systems (OSMSES)"\footnote{\url{https://www.osmses2024.org/home}, last access 2025-01-02} conference. 
\par 
If your software project uses GitHub/GitLab, their issue system is a great way to receive feedback of any kind. Feature requests provide direct information about what the users need, and questions about your software indicate where the documentation can be improved. 
In general, the software repository should be structured in a way that facilitates engagement with the community. Create a \texttt{CONTRIBUTING} file to communicate to the community how they can contribute to the software project, including, for example, how to report bugs, style recommendations, or the code of conduct.
Contact information should always be available if external researchers want to contact the maintainers. 

\par 
In summary, when developing reusable software, you should care about users and actively engage with them to improve software quality and maximize scientific impact.  

\recommendation{Grow your community!} 

\subsection{Good software development needs resources}\label{sec:proposal}

\paragraph{Background} Energy research is mainly funded through research projects. Since software engineering is often a big part of research projects it should also be covered in project proposals. Including software engineering aspects in proposals enables researchers working on research projects to develop better research software. Many of the aspects already mentioned can be included in a research project proposal.

\paragraph{Recommendations} Within a proposal, we recommend emphasizing reusing existing frameworks instead of developing new solutions as introduced in \ref{sec:reuseAndExtend}. Even if the reuse of existing software is planned, it is recommended to include some resources for maintaining and adapting the existing software.

Sometimes, it is still necessary to develop new software. In these cases, libraries that can be easily integrated into other software are better suited than complicated frameworks.
Additionally, we recommend
to formulate an overall software strategy, including a software management plan (see \cite{jackson_checklist_2018}). This should include the different development aspects during the research project, e.g., testing. Also, this strategy should cover how the software can be reused after the project. Sometimes, other research groups can even write letters of interest for reusing the software. As discussed in \ref{sec:reuseAndExtend}, we recommend choosing the licenses early, which can already be part of the proposal. 

Especially in energy research, research projects often include cooperation with industry partners. In these cases, we recommend finding a common solution for licensing with all involved partners within the research project proposal. In such cases, it can also be helpful to include resources for legal aspects of the software in the research project proposal.

When new software is developed, relevant resources should be included in the research project proposal. Besides sufficient resources for the general development of the software, including essential steps such as testing and documenting the software, this should also take usability aspects into account since the software should be reusable for other researchers. To achieve reuse, planning  resources for community interaction as described in \ref{sec:community} is also helpful.
Since most researchers are not trained in \ac{RSE}, resources for training and networking with other research software engineers are recommended. 

\par 
Software development plays an important role in research projects. By including different \ac{RSE} aspects in the research project proposal, it becomes easier to cover software engineering aspects and follow our recommendations within a research project.


\recommendation{Include \ac{RSE} in your research project proposals!}

%%%%% Discussion
\section{Discussion of Assumptions}\label{sec:discussion}
In this paper, we have made several assumptions for the sake of clarity and simplicity. In this section, we discuss the rationale behind these assumptions, the extent to which these assumptions hold in practice, and the consequences for our protocol when these assumptions hold.

\subsection{Assumptions on the Demand}

There are two simplifying assumptions we make about the demand. First, we assume the demand at any time is relatively small compared to the channel capacities. Second, we take the demand to be constant over time. We elaborate upon both these points below.

\paragraph{Small demands} The assumption that demands are small relative to channel capacities is made precise in \eqref{eq:large_capacity_assumption}. This assumption simplifies two major aspects of our protocol. First, it largely removes congestion from consideration. In \eqref{eq:primal_problem}, there is no constraint ensuring that total flow in both directions stays below capacity--this is always met. Consequently, there is no Lagrange multiplier for congestion and no congestion pricing; only imbalance penalties apply. In contrast, protocols in \cite{sivaraman2020high, varma2021throughput, wang2024fence} include congestion fees due to explicit congestion constraints. Second, the bound \eqref{eq:large_capacity_assumption} ensures that as long as channels remain balanced, the network can always meet demand, no matter how the demand is routed. Since channels can rebalance when necessary, they never drop transactions. This allows prices and flows to adjust as per the equations in \eqref{eq:algorithm}, which makes it easier to prove the protocol's convergence guarantees. This also preserves the key property that a channel's price remains proportional to net money flow through it.

In practice, payment channel networks are used most often for micro-payments, for which on-chain transactions are prohibitively expensive; large transactions typically take place directly on the blockchain. For example, according to \cite{river2023lightning}, the average channel capacity is roughly $0.1$ BTC ($5,000$ BTC distributed over $50,000$ channels), while the average transaction amount is less than $0.0004$ BTC ($44.7k$ satoshis). Thus, the small demand assumption is not too unrealistic. Additionally, the occasional large transaction can be treated as a sequence of smaller transactions by breaking it into packets and executing each packet serially (as done by \cite{sivaraman2020high}).
Lastly, a good path discovery process that favors large capacity channels over small capacity ones can help ensure that the bound in \eqref{eq:large_capacity_assumption} holds.

\paragraph{Constant demands} 
In this work, we assume that any transacting pair of nodes have a steady transaction demand between them (see Section \ref{sec:transaction_requests}). Making this assumption is necessary to obtain the kind of guarantees that we have presented in this paper. Unless the demand is steady, it is unreasonable to expect that the flows converge to a steady value. Weaker assumptions on the demand lead to weaker guarantees. For example, with the more general setting of stochastic, but i.i.d. demand between any two nodes, \cite{varma2021throughput} shows that the channel queue lengths are bounded in expectation. If the demand can be arbitrary, then it is very hard to get any meaningful performance guarantees; \cite{wang2024fence} shows that even for a single bidirectional channel, the competitive ratio is infinite. Indeed, because a PCN is a decentralized system and decisions must be made based on local information alone, it is difficult for the network to find the optimal detailed balance flow at every time step with a time-varying demand.  With a steady demand, the network can discover the optimal flows in a reasonably short time, as our work shows.

We view the constant demand assumption as an approximation for a more general demand process that could be piece-wise constant, stochastic, or both (see simulations in Figure \ref{fig:five_nodes_variable_demand}).
We believe it should be possible to merge ideas from our work and \cite{varma2021throughput} to provide guarantees in a setting with random demands with arbitrary means. We leave this for future work. In addition, our work suggests that a reasonable method of handling stochastic demands is to queue the transaction requests \textit{at the source node} itself. This queuing action should be viewed in conjunction with flow-control. Indeed, a temporarily high unidirectional demand would raise prices for the sender, incentivizing the sender to stop sending the transactions. If the sender queues the transactions, they can send them later when prices drop. This form of queuing does not require any overhaul of the basic PCN infrastructure and is therefore simpler to implement than per-channel queues as suggested by \cite{sivaraman2020high} and \cite{varma2021throughput}.

\subsection{The Incentive of Channels}
The actions of the channels as prescribed by the DEBT control protocol can be summarized as follows. Channels adjust their prices in proportion to the net flow through them. They rebalance themselves whenever necessary and execute any transaction request that has been made of them. We discuss both these aspects below.

\paragraph{On Prices}
In this work, the exclusive role of channel prices is to ensure that the flows through each channel remains balanced. In practice, it would be important to include other components in a channel's price/fee as well: a congestion price  and an incentive price. The congestion price, as suggested by \cite{varma2021throughput}, would depend on the total flow of transactions through the channel, and would incentivize nodes to balance the load over different paths. The incentive price, which is commonly used in practice \cite{river2023lightning}, is necessary to provide channels with an incentive to serve as an intermediary for different channels. In practice, we expect both these components to be smaller than the imbalance price. Consequently, we expect the behavior of our protocol to be similar to our theoretical results even with these additional prices.

A key aspect of our protocol is that channel fees are allowed to be negative. Although the original Lightning network whitepaper \cite{poon2016bitcoin} suggests that negative channel prices may be a good solution to promote rebalancing, the idea of negative prices in not very popular in the literature. To our knowledge, the only prior work with this feature is \cite{varma2021throughput}. Indeed, in papers such as \cite{van2021merchant} and \cite{wang2024fence}, the price function is explicitly modified such that the channel price is never negative. The results of our paper show the benefits of negative prices. For one, in steady state, equal flows in both directions ensure that a channel doesn't loose any money (the other price components mentioned above ensure that the channel will only gain money). More importantly, negative prices are important to ensure that the protocol selectively stifles acyclic flows while allowing circulations to flow. Indeed, in the example of Section \ref{sec:flow_control_example}, the flows between nodes $A$ and $C$ are left on only because the large positive price over one channel is canceled by the corresponding negative price over the other channel, leading to a net zero price.

Lastly, observe that in the DEBT control protocol, the price charged by a channel does not depend on its capacity. This is a natural consequence of the price being the Lagrange multiplier for the net-zero flow constraint, which also does not depend on the channel capacity. In contrast, in many other works, the imbalance price is normalized by the channel capacity \cite{ren2018optimal, lin2020funds, wang2024fence}; this is shown to work well in practice. The rationale for such a price structure is explained well in \cite{wang2024fence}, where this fee is derived with the aim of always maintaining some balance (liquidity) at each end of every channel. This is a reasonable aim if a channel is to never rebalance itself; the experiments of the aforementioned papers are conducted in such a regime. In this work, however, we allow the channels to rebalance themselves a few times in order to settle on a detailed balance flow. This is because our focus is on the long-term steady state performance of the protocol. This difference in perspective also shows up in how the price depends on the channel imbalance. \cite{lin2020funds} and \cite{wang2024fence} advocate for strictly convex prices whereas this work and \cite{varma2021throughput} propose linear prices.

\paragraph{On Rebalancing} 
Recall that the DEBT control protocol ensures that the flows in the network converge to a detailed balance flow, which can be sustained perpetually without any rebalancing. However, during the transient phase (before convergence), channels may have to perform on-chain rebalancing a few times. Since rebalancing is an expensive operation, it is worthwhile discussing methods by which channels can reduce the extent of rebalancing. One option for the channels to reduce the extent of rebalancing is to increase their capacity; however, this comes at the cost of locking in more capital. Each channel can decide for itself the optimum amount of capital to lock in. Another option, which we discuss in Section \ref{sec:five_node}, is for channels to increase the rate $\gamma$ at which they adjust prices. 

Ultimately, whether or not it is beneficial for a channel to rebalance depends on the time-horizon under consideration. Our protocol is based on the assumption that the demand remains steady for a long period of time. If this is indeed the case, it would be worthwhile for a channel to rebalance itself as it can make up this cost through the incentive fees gained from the flow of transactions through it in steady state. If a channel chooses not to rebalance itself, however, there is a risk of being trapped in a deadlock, which is suboptimal for not only the nodes but also the channel.

\section{Conclusion}
This work presents DEBT control: a protocol for payment channel networks that uses source routing and flow control based on channel prices. The protocol is derived by posing a network utility maximization problem and analyzing its dual minimization. It is shown that under steady demands, the protocol guides the network to an optimal, sustainable point. Simulations show its robustness to demand variations. The work demonstrates that simple protocols with strong theoretical guarantees are possible for PCNs and we hope it inspires further theoretical research in this direction.

\section{Conclusion}
In this work, we propose a simple yet effective approach, called SMILE, for graph few-shot learning with fewer tasks. Specifically, we introduce a novel dual-level mixup strategy, including within-task and across-task mixup, for enriching the diversity of nodes within each task and the diversity of tasks. Also, we incorporate the degree-based prior information to learn expressive node embeddings. Theoretically, we prove that SMILE effectively enhances the model's generalization performance. Empirically, we conduct extensive experiments on multiple benchmarks and the results suggest that SMILE significantly outperforms other baselines, including both in-domain and cross-domain few-shot settings.

%%
%% The acknowledgments section is defined using the "acks" environment
%% (and NOT an unnumbered section). This ensures the proper
%% identification of the section in the article metadata, and the
%% consistent spelling of the heading.

\newacronym{rl}{RL}{Reinforcement Learning}
\newacronym{drl}{DRL}{Deep Reinforcement Learning}
\newacronym{mdp}{MDP}{Markov Decision Process}
\newacronym{ppo}{PPO}{Proximal Policy Optimization}
\newacronym{sac}{SAC}{Soft Actor-Critic}
\newacronym{epvf}{EPVF}{Explicit Policy-conditioned Value Function}
\newacronym{unf}{UNF}{Universal Neural Functional}
\begin{acks}
The authors would like to thank all participants of the two workshops for their time and valuable insights with special thanks to Anna-Lena Lamprecht for her wonderful talk on RSE. Also, the authors would like to thank the German Federal Government, the German State Governments, and the Joint Science Conference (GWK) for their funding and support as part of the NFDI4Energy and NFDI4Ing consortia. The work was partially funded by the German Research Foundation (DFG) – 501865131 and 442146713 within the German National Research Data Infrastructure (NFDI, www.nfdi.de) and 359941476 within the Schwerpunktprogramm 1984.
\end{acks}

%%
%% The next two lines define the bibliography style to be used, and
%% the bibliography file.
\bibliographystyle{ACM-Reference-Format}
\bibliography{references}

\end{document}
\endinput

