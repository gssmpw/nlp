%%
%% Based on `sample-sigconf-authordraft.tex' (https://www.acm.org/publications/proceedings-template),
%%
%% Commands for TeXCount
%TC:macro \cite [option:text,text]
%TC:macro \citep [option:text,text]
%TC:macro \citet [option:text,text]
%TC:envir table 0 1
%TC:envir table* 0 1
%TC:envir tabular [ignore] word
%TC:envir displaymath 0 word
%TC:envir math 0 word
%TC:envir comment 0 0
%%
%%
%% The first command in your LaTeX source must be the \documentclass
%% command.
%%
%% For submission and review of your manuscript please change the
%% command to \documentclass[manuscript, screen, review]{acmart}.
%%
%% When submitting camera ready or to TAPS, please change the command
%% to \documentclass[sigconf]{acmart} or whichever template is required
%% for your publication.
%%
%%

\documentclass[sigconf,nonacm,anonymous=false]{acmart}
\usepackage[nolist,nohyperlinks]{acronym}
%\usepackage{todonotes}
\usepackage[disable]{todonotes}
\usepackage{hyperref}
\usepackage{pifont}% http://ctan.org/pkg/pifont
\newcommand{\cmark}{\ding{51}}%

\newcounter{rec}
\setcounter{rec}{1}

\definecolor{boxgray}{gray}{0.95}
\newcommand{\StaticBox}[1]{%
    \setlength{\fboxsep}{7pt} % Set padding
    \par\noindent
    \colorbox{boxgray}{%
        \parbox[b]{\columnwidth-15pt}{% Subtract double of padding (for both sides) + 1 to allow normal line break afterwards
            \setlength{\parindent}{0pt}%
            \setlength{\parskip}{16pt}%
            \centering
            #1%
        }%
    }%
}

\newcommand{\recommendation}[1]{{
\StaticBox{{{\textbf{(\therec)}}}~\,\textbf{#1}
}
}
\stepcounter{rec}
}

%%
%% \BibTeX command to typeset BibTeX logo in the docs
\AtBeginDocument{%
  \providecommand\BibTeX{{%
    Bib\TeX}}}

%% Rights management information.  This information is sent to you
%% when you complete the rights form.  These commands have SAMPLE
%% values in them; it is your responsibility as an author to replace
%% the commands and values with those provided to you when you
%% complete the rights form.
%\setcopyright{acmlicensed}
%\copyrightyear{2025}
%\acmYear{2025}
%\acmDOI{XXXXXXX.XXXXXXX}

%\acmISBN{978-1-4503-XXXX-X/18/06}


%%
%% Submission ID.
%% Use this when submitting an article to a sponsored event. You'll
%% receive a unique submission ID from the organizers
%% of the event, and this ID should be used as the parameter to this command.
%%\acmSubmissionID{123-A56-BU3}

%%
%% For managing citations, it is recommended to use bibliography
%% files in BibTeX format.
%%
%% You can then either use BibTeX with the ACM-Reference-Format style,
%% or BibLaTeX with the acmnumeric or acmauthoryear sytles, that include
%% support for advanced citation of software artefact from the
%% biblatex-software package, also separately available on CTAN.
%%
%% Look at the sample-*-biblatex.tex files for templates showcasing
%% the biblatex styles.
%%

%%
%% The majority of ACM publications use numbered citations and
%% references.  The command \citestyle{authoryear} switches to the
%% "author year" style.
%%
%% If you are preparing content for an event
%% sponsored by ACM SIGGRAPH, you must use the "author year" style of
%% citations and references.
%% Uncommenting
%% the next command will enable that style.
%%\citestyle{acmauthoryear}

%%
%% end of the preamble, start of the body of the document source.
\begin{document}
{
%%
%% The "title" command has an optional parameter,
%% allowing the author to define a "short title" to be used in page headers.
\title{Ten Recommendations for Engineering Research Software in Energy Research} 
%%
%% The "author" command and its associated commands are used to define
%% the authors and their affiliations.
%% Of note is the shared affiliation of the first two authors, and the
%% "authornote" and "authornotemark" commands
%% used to denote shared contribution to the research.
{
\author{Stephan Ferenz}
%\authornote{Both authors contributed equally to this research.}
\email{stephan.ferenz@uol.de}
\orcid{https://orcid.org/0000-0001-9523-7227}
\affiliation{%
  \institution{Carl von Ossietzky Universität Oldenburg}
  \streetaddress{Ammerländer Heerstraße 114-118}
  \city{Oldenburg}
  \country{Germany}
  \postcode{26129}
}
\affiliation{
  \institution{OFFIS - Institute for Information Technology}
  \streetaddress{Escherweg 2}
  \city{Oldenburg}
  \country{Germany}
  \postcode{26121}
}

\author{Emilie Frost}
\email{emilie.frost@uol.de}
\orcid{https://orcid.org/0000-0003-4791-2333}
\affiliation{%
  \institution{Carl von Ossietzky Universität Oldenburg}
  \streetaddress{Ammerländer Heerstraße 114-118}
  \city{Oldenburg}
  \country{Germany}
  \postcode{26129}
}
\affiliation{
  \institution{OFFIS - Institute for Information Technology}
  \streetaddress{Escherweg 2}
  \city{Oldenburg}
  \country{Germany}
  \postcode{26121}
}

\author{Rico Schrage}
\email{rico.schrage@uol.de}
\orcid{https://orcid.org/0000-0001-5339-6553}
\affiliation{%
  \institution{Carl von Ossietzky Universität Oldenburg}
  \streetaddress{Ammerländer Heerstraße 114-118}
  \city{Oldenburg}
  \country{Germany}
  \postcode{26129}
}
\affiliation{
  \institution{OFFIS - Institute for Information Technology}
  \streetaddress{Escherweg 2}
  \city{Oldenburg}
  \country{Germany}
  \postcode{26121}
}

\author{Thomas Wolgast}
\email{thomas.wolgast@uol.de}
\orcid{https://orcid.org/0000-0002-9042-9964}
\affiliation{%
  \institution{Carl von Ossietzky Universität Oldenburg}
  \streetaddress{Ammerländer Heerstraße 114-118}
  \city{Oldenburg}
  \country{Germany}
  \postcode{26129}
}
\affiliation{
  \institution{OFFIS - Institute for Information Technology}
  \streetaddress{Escherweg 2}
  \city{Oldenburg}
  \country{Germany}
  \postcode{26121}
}




\author{Inga Beyers}
\email{beyers@ifes.uni-hannover.de}
\orcid{https://orcid.org/0000-0002-9566-770X}
\affiliation{%
  \institution{Leibniz University Hanover, Institute of Electric Power Systems}
  \streetaddress{Appelstraße 9A}
  \city{Hannover}
  \country{Germany}
  \postcode{30167}
}


\author{Oliver Karras}
\email{oliver.karras@tib.eu}
\orcid{https://orcid.org/0000-0001-5336-6899}
\affiliation{%
  \institution{TIB - Leibniz Information Centre for Science and Technology}
  \streetaddress{Welfengarten 1B}
  \city{Hannover}
  \country{Germany}
  \postcode{30167}
}

\author{Oliver Werth}
\email{oliver.werth@offis.de}
\orcid{https://orcid.org/0000-0002-6767-5905}
\affiliation{
  \institution{OFFIS - Institute for Information Technology}
  \streetaddress{Escherweg 2}
  \city{Oldenburg}
  \country{Germany}
  \postcode{26121}
}

\author{Astrid Nieße}
\email{astrid.niesse@uol.de}
\orcid{https://orcid.org/0000-0003-1881-9172}
\affiliation{%
  \institution{Carl von Ossietzky Universität Oldenburg}
  \streetaddress{Ammerländer Heerstraße 114-118}
  \city{Oldenburg}
  \country{Germany}
  \postcode{26129}
}
\affiliation{
  \institution{OFFIS - Institute for Information Technology}
  \streetaddress{Escherweg 2}
  \city{Oldenburg}
  \country{Germany}
  \postcode{26121}
}
}

%%
%% By default, the full list of authors will be used in the page
%% headers. Often, this list is too long, and will overlap
%% other information printed in the page headers. This command allows
%% the author to define a more concise list
%% of authors' names for this purpose.
\renewcommand{\shortauthors}{Ferenz et al.}

%%
%% The abstract is a short summary of the work to be presented in the
%% article.
\begin{abstract}

  Energy research software~(ERS) is a central cornerstone to facilitate energy research.
However, ERS is developed by researchers who, in many cases, lack formal training in software engineering. This reduces the quality of ERS, leading to limited reproducibility and reusability. 
  To address these issues, we developed ten central recommendations for the development of ERS, covering areas such as conceptualization, development, testing, and publication of ERS. 
  The recommendations are based on the outcomes of two workshops with a diverse group of energy researchers and aim to improve the awareness of research software engineering in the energy domain. The recommendations should enhance the quality of ERS and, therefore, the reproducibility of energy research.


\end{abstract}

%%
%% The code below is generated by the tool at http://dl.acm.org/ccs.cfm.
%% Please copy and paste the code instead of the example below.
%%
\begin{CCSXML}
<ccs2012>
<concept>
<concept_id>10011007.10011074.10011134.10003559</concept_id>
<concept_desc>Software and its engineering~Open source model</concept_desc>
<concept_significance>300</concept_significance>
</concept>
<concept>
<concept_id>10002951.10003317.10003347</concept_id>
<concept_desc>Information systems~Retrieval tasks and goals</concept_desc>
<concept_significance>300</concept_significance>
</concept>
<concept>
<concept_id>10011007.10011074</concept_id>
<concept_desc>Software and its engineering~Software creation and management</concept_desc>
<concept_significance>500</concept_significance>
</concept>
<concept>
<concept_id>10010405.10010432</concept_id>
<concept_desc>Applied computing~Physical sciences and engineering</concept_desc>
<concept_significance>500</concept_significance>
</concept>
</ccs2012>
\end{CCSXML}

\ccsdesc[300]{Software and its engineering~Open source model}
\ccsdesc[300]{Information systems~Retrieval tasks and goals}
\ccsdesc[500]{Software and its engineering~Software creation and management}
\ccsdesc[500]{Applied computing~Physical sciences and engineering}

%%
%% Keywords. The author(s) should pick words that accurately describe
%% the work being presented. Separate the keywords with commas.
\keywords{Energy Research, Research Software Engineering, Energy Research Software, Software Development, Smart Grid, Power System, Energy System}

%\received{24 January 2025}
%\received[revised]{12 March 2009}
%\received[accepted]{5 June 2009}

%%
%% This command processes the author and affiliation and title
%% information and builds the first part of the formatted document.
\maketitle
}
\documentclass[../main.tex]{subfiles}
\graphicspath{{../images/}}
\makeatletter
\def\input@path{{../images/}}
\makeatother
\begin{document}
\section{Introduction}
\begin{figure}
\centering
\begin{tikzpicture}
\node[inner sep=0pt] (ws) at (0, 0) {
\includegraphics[height=.4\textwidth, trim={10cm 0 10cm 0},clip]{world_space.png}};
\node[inner sep=0pt] (cs) at (6,0) {\includegraphics[height=.4\textwidth, trim={10cm 1cm 10cm 4cm},clip]{conf_space.png}};
\end{tikzpicture}
\vspace{-5pt}
\label{fig:pbrm_intro}
\caption{\textbf{Left}: Shows world space obstacles as grey spheres. Robots start and goal configuration is colored red and green, respectively. Configurations along the computed path are colored transparent blue. \textbf{Right:} Mapped world space scenario to configuration space. Obstacle region is the grey mesh. Red spheres are collision-free regions computed by the neural SCDF. The optimized shortest path in the convex corridor is the blue curve.}
\vspace{-25pt}
\end{figure}
Motion planning is the problem of finding a collision-free trajectory that connects a given start and goal configuration. The planning takes place in the configuration space of the robot. For single body robots, like mobile robots or drones, the configuration space and the world space are usually the same. This simplifies the planning, since explicit obstacle representations are available which enables geometrical tools like separating hyperplanes, smallest distance to obstacles etc., to be used when designing motion planning algorithms. For multi-body robots like manipulators, the situation is completely different. The world space obstacles are usually mapped to non-convex regions, and to make the problem even harder, the mapping is usually not known. Forming explicit representations of the obstacle region in the configuration space is usually too expensive or intractable. Despite all of this, sampling based planners are used with great success, which mainly is due to their use of implicit representations of the obstacle region. The basic idea is to construct a graph in the configuration space that covers and connects the collision-free region. From this graph, a path can be extracted that connects a given start and goal configuration. The approach is computationally expensive, since the graph is constructed with the smallest geometrical building block available, points, which represents a collision-check. Furthermore, the extracted paths from the graph are non-smooth and jagged due to the stochastic nature of the approach. This adds an additional post-processing step to the process, where the paths are shortcutted and smoothened, before the path can be used for tracking. Clearly a lot of time is invested to form this graph and produce smooth paths. Thus, if the obstacles start to move, then all of this work is done in no use, since all points that make up this graph need to be re-verified, which is simply too time consuming to be done in real time.
\\\\
In this work, we want to address the existing drawbacks of the sampling based planners. Our main contribution is an improved motion planner where each vertex in the graph covers a collision-free region in the form of a sphere instead of a point and where the edges are formed with neighboring intersecting spheres. This representation has the advantage of instead of returning piecewise linear paths, returning a sequence of overlapping spheres, i.e. a convex corridor, that connects a given start and goal configuration, illustrated in Figure \ref{fig:pbrm_intro}. This convex corridor allows us to use convex optimization to produce smooth trajectories, instead of computationally expensive post-processing methods. The representation further allows us to estimate the coverage of the collision-free space, which gives us awareness and feedback in the offline roadmap construction phase. Finally, our representation is simple to adapt to moving obstacles, simply requery for the new radii and recheck for intersections. 
\\\\
The spherical collision-free regions are formed using a signed distance function (SDF), which is a function that returns the smallest distance from an arbitrary point to the boundary of an obstacle. As the name implies, the distance is signed, thus if the point is inside the obstacle it is negative otherwise positive. If the distance is positive, a sphere with radius equal to the distance is guaranteed to cover a collision-free region. Using an SDF in motion planning is not new, but what is novel about our approach is that we express the distance in the configuration space instead of the world space and by doing so allows us to form these convex collision-free regions. We refer to the resulting SDF as a signed configuration distance function (SCDF). Computing an SCDF analytically is non-trivial, our approach is therefore to parameterize the SCDF with a deep neural network and learn the mapping by supervised learning. Our resulting neural SCDF can compute distances for different parameter values of obstacle shapes and we also show how multiple distances can be combined, thus making our approach flexible.
\section{Related work}
Motion planning algorithms can roughly be divided into three families, grid-based, sampling based and optimization based methods. Grid-based methods (GBM) discretize the planning space from which a graph is then compiled. A standard search method is A$^\star$ \citep{a_star}, which is classified as an \textit{informed} search method, since it employs a heuristic function to speed up the search. A$^\star$ guarantees to return an optimal path at the level of discretization used. GBMs usually discretize the planning space by a regular lattice and this limits the GBMs to problems with low dimensionality due to the curse of dimensionality. Thus, GBMs are usually limited to single-body robots where the degrees of freedom (DOF) are low. To overcome the inherent scaling problem with the GBMs, stochastic methods are usually used for multi-body robots. These methods are termed as sampling-based methods (SBM) and core members within this family are the rapidly-exploring random trees (RRT) \citep{rrt} and the probabilistic roadmap (PRM) \citep{prm}. RRT grows a tree from the start configuration and explores the collision-free region in a rapid way until it is able to connect to the goal region. RRT is usually improved by bi-directional planning \citep{rrt_connect}, i.e. an additional tree is grown from the goal configuration and the trees are tested for connection after any tree has been expanded. RRT is a single-query method, thus it searches for a path from scratch each time it is queried. Contrary to this, PRM is a multi-query method, which solves for multiple queries without starting from scratch. PRM does this by creating a roadmap (graph) that covers the collision-free space as an offline step. The graph is then used to solve for multiple queries. PRMs are used in cases where the environment does not change since the extra offline step is too computationally costly and needs to be re-done if the environment is changed. In our work, we address this inherent issue by using a different roadmap representation. Our vertices in the graph cover a collision-free region in the form of spheres and we form the edges by checking for intersecting spheres. If something in the environment changes, we recompute the spheres radii and recheck the intersections, without relying on collision detection. We use a trained neural network to compute the sphere radius, therefore querying for the radius can be done fast, hence our representation enables the PRM for dynamic environments.
\\\\
In the recent decades, optimization based methods (OBM) \citep{chomp, schulman, itomp, stomp} have been introduced as an alternative to SBM for multi-body robots. Like the SBM, the OBMs scale well to higher dimensional problems and produce smoother motion. It is common to use a SDF in the optimization since it is a smooth function, thus enabling gradient-based methods. However, the standard way of expressing the SDF is in world space. The distance therefore needs to be mapped to the configuration space by the forward kinematics. This mapping makes the optimization problem a non-linear program (NLP), which is computationally expensive to solve. Recently, a different approach has been proposed. In \cite{mp_gcs} motion planning is formulated as a convex optimization problem by using the graph of convex sets framework \citep{gcs}. The underlying idea is to decompose the collision-free space into intersecting convex sets from which a convex optimization problem is formulated. In cases where an explicit representation of the obstacles in the configuration space exists, like for single-body robots, creating collision-free convex regions can be done fast \citep{iris}. For multi-body robots, this is non-trivial. Existing work does this successfully \citep{iris_nlp, iris_c} by an optimization based approach, but the methods are still too time consuming to be used in the presence of moving obstacles. Our approach is instead to use deep learning to learn an SDF expressed in the configuration space. With this, we can query for shortest distances to the collision boundary, which allows us to expand spherical regions which are collision-free. Our approach is fast and therefore enables our suggested roadmap planner to be used in dynamic environments.
\\\\
Recent research has focused on learning collision detection \citep{fk_kernel_distance, diffco, graphdistnet} by predicting the signed distance between the robot links and the surrounding obstacles in the world space. The learned SDF is used in trajectory optimization but since the distance is expressed in the world space, the problem becomes an NLP and therefore takes a long time to solve. We take a novel approach and suggest to instead express the signed distance in the configuration space. This allows us to improve the PRM at the same time as it enables convex optimization for trajectory optimization, which runs faster and is more reliable than NLP solvers. In \cite{cspf} a learned signed distance function in the configuration space is proposed similar to our approach. However, their approach is restricted to point cloud representations, while we propose to represent the obstacles as parameterized geometric shapes, e.g. spheres. Furthermore, we also show how to use our learned SCDF to improve an existing roadmap planner.
\section{Problem formulation}
A robot is located in the world space, $\W \subset \R^3 $. The unique location of the robot is given by its configuration $\q \in \C$, where $\C$ is the configuration space. The set of points covered by the robots bodies at a certain configuration is expressed as $\B(\q) \subset \W$. The robot is surrounded by $\NrObst$ obstacles $\O = \bigcup_{i=1}^{\NrObst} \O_i$, where  $\O_i \subset \W$. The representation of the obstacle in the configuration space is the set $\C\O_i = \{\q \in \C \: |\: \B(\q) \cap \O_i \neq \emptyset \}$. The obstacle space is formed as $\Co = \bigcup_{i=1}^{\NrObst} \C \O_i$. The complement is referred to as the free space, $\Cf = \C \setminus \Co$. The path planning problem is a tuple, ($\Cf$, $\qStart$, $\qGoal$), where we want to connect a query pair, consisting of a start, $\qStart$, and goal configuration, $\qGoal$, with a geometric path, $\q(s): [0, 1] \mapsto \Cf$, such that $\q(0)=\qStart$ and $\q(1)=\qGoal$, or report correctly when such a path does not exist.
\end{document}



\section{Characteristics of ERS} \label{sec:chara}

Research software, in general, has several unique characteristics. It is often highly specific and mainly created to solve one/few particular use case(s). The software is usually not designed for reuse and has a short lifecycle. Generally, it is hard to comprehend the software, even for researchers from the same domain, due to the inherent complexity of the solved problems. Further, most researchers are not explicitly trained in software engineering. Another fundamental characteristic of research software is that the software requirements are not known beforehand but often only evolve during development~\cite{hasselbring_toward_2024,felderer_investigating_2025}. 

\ac{ERS} shares these characteristics of general research software, while some common aspects can be considered more critical in energy research. Additionally, \ac{ERS} also has some special characteristics:

\begin{description}
    \item [Interdisciplinary research] Energy researchers have diverse backgrounds ranging from social science over engineering to different natural sciences and mathematics. As a broad range of domains are involved, it is hard to establish a common language. Therefore, it is difficult to achieve a mutual view of the problems that must be solved as a team. This issue influences the development of \ac{ERS} as well as its capability of being understood by different interested researchers like project partners.
    \item [Applied research] Energy research is an applied research field. Therefore, cooperation and joint projects with industry are widespread. This also has implications on the engineering of the jointly created software projects, e.g., with respect to open source or use of commercial software.
    \item [Changing levels of detail] \ac{ERS} is diverse and complex and needs to support many different analysis types and detail levels (e.g., transient vs. steady-state).
    \item [Complexity] Energy researchers often look into the detailed behavior of larger systems consisting of many components. 
    \item [Data heterogeneity] \ac{ERS} needs a lot of diverse and heterogeneous data for most simulations and often produces data different in structure, type and size, which needs to be managed \cite{zhang2018big}.
    \item [Reliability] Energy systems are critical infrastructures leading to high demands on research software reliability, specifically when conducting field tests. 
    \item [Changing time horizons] \ac{ERS} often considers diverse time horizons, diverse spatial resolutions, and complex optimization problems~\cite{ENGELAND2017600,DECAROLIS2017184}. These aspects can lead to high performance requirements.
    \item [Coupled co-simulations] As huge infrastructure systems are under analysis, there is a need to couple different types of simulation, often comprising communication simulation  \cite{vogt_survey_2018,steinbrink_cpes_2019}.
\end{description}%

We generally conclude that the most essential aspects of \ac{ERS} are the people involved and working on different detail levels, spatial resolutions, and time resolutions in a diverse and highly simulation-driven research area.
\section{Related Work}\label{sec:relatedwork}

Internet of Things (IoT) has seen rapid advancements in recent years, becoming an integral part of various domains, such as smart industries and homes, and serving as a key enabler in modern society.
However, despite its growth, IoT continues to face numerous security challenges, prompting significant research efforts aimed at improving IoT security.
With the rise of artificial intelligence (AI), machine learning (ML) and deep learning (DL)-based approaches have become increasingly popular in designing defense mechanisms for IoT devices, including malicious traffic classification~\cite{luo2022transformer,shafiq2020corrauc}, malware detection~\cite{vasan2020mthael,chaganti2022deep,aung2022atlas}, vulnerability discovery~\cite{neshenko2019demystifying}, and others~\cite{al2020survey,otoum2022dl,tambe2019detection}.

More recently, inspired by the success of large language models (LLMs), researchers have begun exploring the potential of LLMs to enhance IoT-related security tasks.
For instance, LLMs have been applied to existing IoT security challenges such as threat detection and fuzzing. Ferrag \etal~\cite{sokiotllm} introduced a BERT-based model, SecurityBERT, to achieve better cyber threat detection accuracy over traditional ML and DL-based methods. 
Similarly, Ma \etal~\cite{ma} and Wang \etal~\cite{llmiotfuz} proposed LLM-assisted fuzzing methods to uncover hidden bugs in IoT devices, enabling the detection of complex vulnerabilities that traditional techniques might miss.
Additionally, Yang \etal~\cite{yang2023iot} combined LLMs with static code analysis using prompt engineering to create a cost-effective solution for IoT vulnerability detection.
\cite{ji2024sevenllm} collected cybersecurity raw texts to train cybersecurity LLM to augment the analysis of cybersecurity events, and \cite{llmtikg} made use of a larger LLM to build knowledge graphs from public threat intelligence and use GPT to create datasets to fine-tune a smaller LLM to extract entities and TTPs from attack description.
Ferraris \etal~\cite{ferraris2024ici} proposed utilizing ChatGPT to enhance IoT trust semantics, aligning with W3C Web of Things (WoT) recommendations\footnote{\scriptsize \url{https://www.w3.org/WoT/}}.
This work extends the TrUStAPIS framework~\cite{ferraris2020trustapis}.

Beyond the above tasks, LLMs have been employed in other IoT challenges.
Meyuhas \etal~\cite{meyuhas2024iotlabel} used LLMs to address the problem of labeling previously unseen IoT devices.
\cite{llmiotcontrol,cui2024llmind} explored leveraging LLMs to control IoT devices and facilitate effective collaboration among them.
Mo \etal~\cite{mo2024iot} collected IoT sensor-natural language paired data and trained IoT-LM to interpret and interact with physical IoT sensors.
Xu \etal~\cite{xu2024penetrative} employed ChatGPT to interpret IoT sensor data and reason over tasks in the physical realm, introducing novel ways of integrating human knowledge into cyber-physical systems. 

Recently, Deldari \etal~\cite{deldari2024auditnet} proposed AuditNet, a conversational AI-based security assistant, which is most similar to \chatiot\ and also augmented by external knowledge.
However, AuditNet focused on standards, policies, and regulations of portable document format (PDF), and aimed to reduce the manual effort of security experts involved in compliance checks of IoT. 
On the other hand, we integrate IoT threat intelligence of various sources into \chatiot\ and can assist multiple kinds of users. Besides, we provide an end-to-end toolkit to process data in various formats, not limited to PDF. 

Together, these studies indicate that LLMs have great potential to improve the security of IoT systems in various domains, from vulnerability discovery to trustworthiness management. 
By integrating LLMs with IoT-specific threat intelligence, these models can be guided to meet the unique challenges posed by the IoT ecosystem.
Moreover, the continuous advancements in the LLM community, combined with increasingly accessible IoT datasets, are likely to further drive the adoption of LLMs in IoT-related research and practical applications.

%%%%% Method
% \begin{figure}
%     \centering
%     \includegraphics[width=0.5\linewidth]{Move_teaser.pdf}
%     \caption{Comparison of different dynamic compute approaches. length of arrow indicates residual transformation per token while width indicates velocity of transformation.}
%     \label{fig:enter-label}
% \end{figure}

\section{Method}
\label{sec:method}
Residual connections play a crucial role in shaping token representations, yet their dynamics remain underexplored in the context of efficient decoding. In this work, we delve deeper into transformer residual dynamics and investigate how modulating residual transformation velocity can improve inference efficiency in token-level processing, optimizing both dense and sparse MoE transformers.


\subsection{Residual Dynamics and Motivation for Multi-rate Residuals} \label{sec:motivation}

To analyze how hidden representations evolve across different layers of a transformer architecture, it's crucial to consider the effect of residual connections. Each transformer decoder layer typically has residual connections across attention and MLP submodules. As the residual stream $h_i$ traverses from interval $E_j$ to $E_{j+1}$, it undergoes a residual transformation given by:  
% \begin{equation}
% \label{eq:slow_residual_transformation}
% H_{E_{j+1}} = H_{E_j} \prod_{i=E_j}^{E_{j+1}} \left( I + \mathcal{A}_i \right) \left( I + \mathcal{M}_i \right) \quad \text{where} \quad \mathcal{A}_i = f(c_i, h_{i}), \mathcal{M}_i = g(h_i)
% \end{equation}

\begin{equation} \label{eq:slow_residual_transformation}
h_{E_{j+1}} = h_{E_j} + \sum_{i=E_j}^{E_{j+1}-1} \left( \mathcal{A}_i(h_i) + \mathcal{M}_i(h_i + \mathcal{A}_i(h_i)) \right) \quad \text{where} \quad \mathcal{A}_i = f(c_i, h_{i}), \mathcal{M}_i = g(h_i). 
\end{equation}

Here, \( \mathcal{A}_i \) denotes the non-linear transformation introduced by the multi-head attention mechanism at layer \( i \), while \( \mathcal{M}_i \) corresponds to the non-linear transformation of the MLP block at the same layer. These transformations depend on the input residual stream \( h_i \) and, in the case of \( \mathcal{A}_i \), the previous contextual representation \( c_i \).\footnote{Normalization layers are typically applied in practice but are omitted here for simplicity of the argument.}


% For easy tokens, the magnitude and direction of this delta transformation become progressively smaller with each successive layer as shown in \cref{fig:delta_transformation}. Consequently, it is feasible to predict these tokens after only a few residual connections, whereas harder tokens necessitate more extensive processing through additional layers.

\begin{figure}[ht]
    \centering
    \begin{subfigure}{0.48\textwidth}
        \centering
        \includegraphics[width=\textwidth]{sections/figures/residual_change.pdf}
        \caption{}
        \label{fig:residual_change}
    \end{subfigure}%
    \hfill
    \begin{subfigure}{0.48\textwidth}
        \centering
        \includegraphics[width=\textwidth]{sections/figures/alignment_wrt_dedicated_model.pdf}
        \caption{}
    \label{fig:alignment_wrt_dedicated_model}
    \end{subfigure}
    \caption{(a) As residual streams propagate through the model, the directional shifts in the residuals become progressively smaller. (b) A dedicated model with $k$ layers achieves a faster rate of change in residual streams and higher alignment than base model leveraging early exit mechanisms at layer $k$.}
    \label{fig}
\end{figure}


To examine whether residual transformations can be accelerated across layers, we conducted experiments using a diverse set of prompts on a pre-trained Phi3 model~\cite{phi3_report}. As illustrated in \cref{fig:residual_change}, we measured the directional shift in residual states as \( 1 - \mathcal{C}(h_{i-1}, h_i) \), where \(\mathcal{C}\) denotes normalized cosine similarity. This shift is notably higher in the initial layers, gradually decreasing in subsequent layers. This behavior allows traditional early exit approaches to effectively accelerate decoding by enabling earlier exits for simpler tokens. However, these approaches typically rely on a distance-based approximation, where the full residual transformation of the model is approximated by the residual transformations of the initial layers. To gain deeper insights into the distance versus velocity aspects of residual transformation, we conducted a comparative study. Specifically, we trained an early exit head at layer $k$ of the Phi3 model, which consists of 32 layers, restricting the distance traveled by each token. To accelerate the residual transformation relative to number of layers, we trained a smaller model consisting of only $k$ layers, while keeping all other hyperparameters consistent. We then compared the next-token prediction accuracy of the early exit head of the base model with that of the smaller model. To ensure an equal number of trainable parameters, we inserted low-rank adapters into the smaller model and trained only these adapters, whereas, in the distance-based approach, we trained solely the early exit head. In addition, to accelerate the residual transformation in smaller model, we distilled the residual streams from the larger model by incorporating a distillation loss ~\cite{sanh2019distilbert} between the residual state at layer \(i\) of the smaller model and the residual state at layer \(4 \times i\) of the larger model. As shown in ~\cref{fig:alignment_wrt_dedicated_model} the smaller model demonstrates a significantly faster rate of change in residual streams, leading to higher next token prediction accuracy after $k$ layers compared to the base model that employs traditional early exit mechanisms after $k$ layers \cite{schuster2022confident, chen2023eellm, varshney-etal-2024-investigating}. This experimental setup, which modifies only the rate of change in residual streams while keeping other factors constant, suggests that dense transformers, trained with a fixed number of layers, may inherently possess a slow residual transformation bias.

This observation raises an intriguing question: if the rate of change in residual streams could be accelerated relative to the number of layers, is it possible to facilitate earlier alignment for a greater proportion of tokens? Earlier alignment would be beneficial to not only facilitate dynamic computation but also for generating speculative tokens efficiently with high acceptance rates in speculative decoding setups ~\cite{leviathan2023fast, chen2023accelerating}. 

%thereby enhancing the efficiency of early exiting? 
 % This bias likely constrains the effectiveness of early exiting, particularly for easier tokens. By addressing this limitation through accelerated residual transformations, we hypothesize that it is possible to substantially improve the efficiency and accuracy of early exit strategies in transformer models.

\subsection{Multi-Rate Residual Transformation} \label{m2r2_method}

To address the slow residual transformation bias described in ~\cref{sec:motivation}, we introduce \textit{accelerated residual streams} that operate at rate $R$ relative to original slow residual stream. We pair slow residual stream, $h$ with an accelerated residual stream, $p$, which has an intrinsic bias towards earlier alignment. Relative to ~\cref{eq:slow_residual_transformation}, accelerated residual transformation from interval $E_j$ to $E_{j+1}$ can be represented as: 

% \begin{equation}
% \label{eq:fast_residual_transformation}
% P_{E_{j+1}} = P_{E_j} \prod_{i=E_j}^{E_{j+1}} \left( I + \hat{\mathcal{A}_i} \right) \left( I + \hat{\mathcal{M}_i} \right) \quad \text{where} \quad \hat{\mathcal{A}_i} = \hat{f}(c_i, P_{i}), \hat{\mathcal{M}_i} = \hat{g}(P_{i})
% \end{equation}


\begin{equation} \label{eq:fast_residual_transformation}
p_{E_{j+1}} = p_{E_j} + \sum_{i=E_j}^{E_{j+1}-1} \left( \hat{\mathcal{A}_i}(p_i) + \hat{\mathcal{M}_i}(p_i + \hat{\mathcal{A}_i}(p_i)) \right) \quad \text{where} \quad \hat{\mathcal{A}_i} = \hat{f}(c_i, p_{i}), \hat{\mathcal{M}_i} = \hat{g}(h_i), 
\end{equation}



where $\hat{\mathcal{A}_i}$ and $\hat{\mathcal{M}_i}$ denote non-linear transformation added by layer $i$ to previous accelerated residual $p_{i}$. Similar to $\mathcal{A}_i$, non-linear transformation $\hat{\mathcal{A}_i}$ attends to same context $c_i$ but uses a different transformation $\hat{f}$ for accelerating $p_{E_j}$ relative to $h_{E_j}$. 

We integrate accelerated residual transformation directly into the base network using parallel accelerator adapters such that rank of accelerator adapters $R_p << d$ where $d$ denotes base model hidden dimension. This setup allows the slow residual stream $h_{E_j}$ to pass through the base model layers while the accelerated residual stream $p_{E_j}$ utilizes these parallel adapters as shown in ~\cref{fig:m2r2_main}. Both slow and accelerated residuals are processed in same forward pass via attention masking and incur negligible additional inference latency in memory bound decoding setups, while in compute bound decoding setups where FLOPs optimization is essential, accelerated residual stream utilizes a fraction of attention heads that of slow residual (see ~\cref{sec:flops_optimization}). Additionally, to maximize the utility of accelerated residual transformations without introducing dedicated KV caches, we propose a shared caching mechanism between the slow and accelerated streams which minimally impact alignment benefits of our approach while offering substantial memory savings (see ~\cref{fig:koala_alignment}). Specifically, the attention operation on the slow residuals \( \text{MHA}(h_t, h_{\leq t}, h_{\leq t}) \) is redefined for accelerated residuals as 
\[
\hat{\mathcal{A}} = MHA(p_t, h_{<t} \oplus p_t, h_{<t} \oplus p_t),
\]
where the accelerated residual at time-step $t$, \( p_t \) attends to the slow residual’s KV cache, facilitating the reuse of contextual information across both residual streams without incurring additional caching costs. Here, \(MHA(q, k, v) \) represents multi-head attention between query \( q \), key \( k \), and value \( v \).

\begin{figure}
    \centering
    \includegraphics[width=0.8\linewidth]{sections//figures/m2r2_main2.pdf}
    \caption{Multi-rate Residuals Framework: Slow residual stream of base model is accompanied by a faster stream that operates at a $2-(J+1)\times$ rate relative to the slow stream, undergoing transformations via accelerator adapters as detailed in \cref{m2r2_method}, where J denotes number of early exit intervals. Colors within the slow and fast residual streams indicate similarity, with matching colors representing the most closely aligned residual states. At the beginning of the forward pass and at each exit point, the accelerated residual state is initialized from the corresponding slow residual state to avoid gradient conflict during training (see ~\cref{sec:grad_conflict}). Early exiting decisions are informed by the Accelerated Residual Latent Attention (ARLA) mechanism, described in \cref{method_arla}, which evaluates residual dynamics across consecutive exit gates.}
    \label{fig:m2r2_main}
\end{figure}

% Furthermore. to maximize the benefits of fast residual transformations without using dedicated KV caches, we propose sharing the fast network’s cache with the slow network. Formally speaking, We modify attention operation on slow residuals $MHA(H_t, H_{<=t}, H_{<=t})$ as $MHA(P_{t}, H_{<t} \oplus P_t, H_{<t}  \oplus P_t)$ such that accelerated residuals attend to previous slow context KV cache, where $MHA(q,k,v)$ denotes multi head attention between query, $q$, key $k$ and value $v$.


\subsection{Enhanced Early Residual Alignment}
Early residual alignment is instrumental in optimizing early exiting, speculative decoding, and Mixture-of-Experts (MoE) inference mechanisms. In this section, we provide a detailed analysis of how accelerated residuals enhance these inference setups.

% By aligning the residual states of intermediate layers with the final output representations, the model can maintain high prediction accuracy even when computations are truncated at earlier layers. This enables more reliable early exiting, reducing the overall computational cost while preserving performance. Additionally, in speculative decoding, early residual alignment allows the model to make confident predictions using faster, partial computations, thereby accelerating inference without sacrificing output quality.


\subsubsection{Early Exiting} \label{method_early_exiting}

A prevalent strategy for enabling early exiting at an intermediate layer $E_{j}$ involves approximating the residual transformation between $E_{j}$ and the final layer $N-1$ using a linear, context independent mapping, $\mathcal{T}$, such that $H_{N-1} \approx \mathcal{T}(H_{E_{j}})$. This approximation has been extensively employed in conventional approaches ~\cite{schuster2022confident, chen2023eellm, varshney-etal-2024-investigating}, providing a computationally efficient means to project the output of deeper layers from intermediate states. Specifically, residual state of layer $N-1$ with this approximation can be expressed as:


% \begin{equation}
% \label{eq: vanila_ea_assumption}
% \Phi(H_{E_{j}}) \sim H_{E_{j}} \prod_{i=E_{j}}^{N}\left( I + \mathcal{A}_i \right) \left( I + \mathcal{M}_i \right) \quad \text{where} \quad \Phi \perp C
% \end{equation}

\begin{equation} \label{eq:early_exiting}
h_{E_j} + \sum_{i=E_j}^{N-1} \left( \mathcal{A}_i(h_i) + \mathcal{M}_i(h_i + \mathcal{A}_i(h_i)) \right) \sim \mathcal{T}(h_{E_{j}})  \quad \text{where} \quad \mathcal{T} \perp c. 
\end{equation}


Here, $\mathcal{A}_i$ and $\mathcal{M}_i$ represent the residual contributions of the multi-head attention and MLP layers, respectively, while $\mathcal{T}$ remains independent of $c$, the preceding context.

This approach is inherently limited by two major factors: first, the assumption of linearity between $h_{E_{j}}$ and $h_{N-1}$ may not hold uniformly for all tokens, particularly when $E_j \ll N$. Second, the linear transformation $\mathcal{T}$ disregards the influence of the context $c$ and fails to account for the latent representations of previous contextual states. In contrast, M2R2 accelerated residual states mitigate both of these challenges by approximating the slow residual transformation of all layers via a faster residual transformation of fewer layers as:
% \begin{equation}
% H_{E_j} \prod_{i=E_j}^{N}\left( I + \mathcal{A}_i \right) \left( I + \mathcal{M}_i \right) \sim P_{E_j} \prod_{i=E_j}^{E_j+1}\left( I + \hat{\mathcal{A}_i} \right) \left( I + \hat{\mathcal{M}_i} \right)
% \end{equation}


\begin{equation} \label{eq:m2r2_approximating_ea}
h_{E_j} + \sum_{i=E_j}^{N-1} \left( \mathcal{A}_i(h_i) + \mathcal{M}_i(h_i + \mathcal{A}_i(h_i)) \right) \sim p_{E_j} + \sum_{i=E_j}^{E_{j+1}-1} \left( \hat{\mathcal{A}_i}(p_i) + \hat{\mathcal{M}_i}(p_i + \hat{\mathcal{A}_i}(p_i)) \right), 
\end{equation}

% \begin{equation} \label{eq:fast_residual_transformation}
% p_{E_{j+1}} = p_{E_j} + \sum_{i=E_j}^{E_{j+1}-1} \left( \hat{\mathcal{A}_i}(p_i) + \hat{\mathcal{M}_i}(p_i + \hat{\mathcal{A}_i}(p_i)) \right) \quad \text{where} \quad \hat{\mathcal{A}_i} = \hat{f}(c_i, p_{i}), \hat{\mathcal{M}_i} = \hat{g}(h_i) 
% \end{equation}






where $p_{E_j}$ is initialized from the slow residual state $h_{E_j}$ at each early exit interval $E_j$ using an identity transformation (see ~\cref{fig:m2r2_main}). As shown in ~\cref{fig:m2r2_residual_sim}, accelerated residuals offer a smoother, more consistent shift in residual direction across layers, in contrast to the abrupt changes typically seen at early exit points in standard early exit methods. Moreover, the normalized cosine similarity between accelerated states at early exit intervals and final residual states is substantially higher compared to traditional early exit techniques, highlighting improved alignment with final layer representations. Traditional adaptive compute methods are constrained by two principal factors: the number of tokens eligible for early exit at intermediate layers and the precision of early exit decision. If residual streams fail to saturate early, the majority of tokens remain ineligible for exit, thereby diminishing potential speedups. Additionally, imprecise delineations between tokens suitable for early exit can lead to underthinking (premature exits that adversely affect accuracy) or overthinking (unnecessary processing that compromises efficiency) ~\cite{zhou2020self, dai2020dynamic}. Enhanced early alignment using ~\cref{eq:m2r2_approximating_ea} helps to address  first issue. To address the second issue we introduce Accelerated Residual Latent Attention, which dynamically assesses the saturation of the residual stream, allowing for a more precise differentiation between tokens that can exit early and those requiring further processing.

% This results in uniform change in residual direction    
% % We keep $\mathcal{A} = \hat{\mathcal{A}}$, while $\hat{\mathcal{M}}$ is accelerated by a factor of $2 - (N_{E}+1)X$ relative to the slower residual transformation $\mathcal{M}$, where $N_E$ represents number of early exiting intervals.
% Figure~\cref{fig:rate_change_comparison} illustrates the comparative rate of change between these transformation streams.



% fig:rate_change_comparison
% - grid plot x axis -> layer id (0, 8) , y axis -> layer id -> dark color cell for max similarity , lighter for lower 
% 
-------------------------------------------------------
Let's consider residual stream $h_i$ traverses through interval $E_j$ to $E_{j+1}$ and undergoes residual transformation given by 
\begin{equation}
h_{E_{j+1}} = h_{E_j} \prod_{i=E_j}^{E_{j+1}} \left( 1 + \delta_i \right)    
\end{equation}

where $\delta_i$ denotes non-linear transformation added by layer $i$. Each non-linear transformation of layer $i$ is a function of previous contextual representation, $c_i$ and input residual stream $h_i-1$ as
$\delta_i = f(c_i, h_{i-1})$ 

One way to exit early at exit $E_j+1$ is to assume that residual transformation from $E_j+1$ to final layer $N-1$ can be approximated by a linear function $\phi$ as $h_{N-1} \sim \Phi(h_{E_j+1})$ and most conventional approaches such as \todo{cite EA papers} use this approach. In other words, 

\begin{equation}
\Phi(h_{E_j+1} \sim h_{E_j+1} \prod_{i=E_j+1}^{N} \left( 1 + \delta_i \right)   
\end{equation}

This approach suffers from two primary issues, linearity assumption from $h_E_j+1$ to $H_N-1$ if often incorrect, particularly when $E_j << N$. More importantly, linear transformation $\Phi$ doesn't consider effect of context $C_i$. M2R2  effectively addresses these issues as accelerated residual stream at interval $E_j+1$ can be represented as 

\begin{equation}
r_{E_{j+1}} = r_{E_j} \prod_{i=E_j}^{E_{j+1}} \left( 1 + \gamma_i \right)    
\end{equation}

where $\gamma_i$ denotes non-linear transformation added by layer $i$ to previous accelerated residual $r_i-1$. Similar to $\delta_i$, non-linear transformation $\gamma_i$ considers context $C_i$ as 
$\gamma_i = g(c_i, r_{i-1})$. So in summary, slow residual transformation is approximated by accelerated residual as: 

\begin{equation}
h_{E_j} \prod_{i=E_j}^{N} \left( 1 + \delta_i \right) \sim h_{E_j} \prod_{i=E_j}^{E_j+1} \left( 1 + \gamma_i \right)
\end{equation}

It's worth noting that accelerated residual $r_i$ and slow residual $h_i$ are processed concurrently at layer $i$ by constructing proper attention mask such as attention of slow residual is represented as 

$MHA(H_it, H_{i<=t}, H_{i<=t}$ while attention of fast residual is computed as 

$MHA(r_it, H_{i<=t}, H_{i<=t}$ where $MHA(q,k,v$ denotes multi head attention between query, $q$, key $k$ and value $v$.


------------------------------------------------------------------

Vertical latent attention on accelerated residual is computed as 
$MHA(S_mt, S(Ej<=i<=m)t, S(Ej<=i<=m)t)$ where $Smt$ denotes query/key/value projection in latent domain at layer $m$ at time $t$. 
------------------------------------------------------------------

Gradient conflict Avoidance: 

Let's consider $w_j$ is a trainable parameter that belongs to a layer between $E_j$ and $E_j+1$. Consider early exit loss at gate $E_j+1$, $L_j+1$, gradient propagation of $w_j$ at another trainable parameter $w_j-n$ can be gives as 

$\sum_{k=E_j-n}^{E_j} \beta_k \frac{\partial L_{E_k}}{\partial w_k}$

where $\beta_j$ denotes backward transformation coefficient for weight $w_j$ to reach gate $E_j$. 
 
On the other hand, gradient propagation in proposed approach can be represented as 

\[
\frac{\partial L_{E_j}}{\partial w_j} = 
\begin{cases} 
\beta_j \frac{\partial L_{E_j}}{\partial w_j} & \text{if } E_j \leq w_j \leq E_{j+1} \\
0 & \text{otherwise}
\end{cases}
\]







% \begin{figure}[ht]
%     \centering
%     \includegraphics[width=0.8\textwidth, height=5cm]{rate_change_comparison.png}
%     \caption{Rate of change comparison between fast and slow residual streams.}
%     \label{fig:rate_change_comparison}
% \end{figure}

%vary k and and plot EA accuracy for larger and smaller models. 

% \begin{figure}[ht]
%     \centering
%     \includegraphics[width=0.5\textwidth,height=5cm]{sections/figures/alignment_comparison_dialogsum.pdf}
%     \caption{Alignment of exited tokens for different early exit layers using traditional early exiting heads, dedicated faster networks, and faster residuals.}
%     \label{fig:small_model_early_exiting}
% \end{figure}


\textbf{Accelerated Residual Latent Attention} \label{method_arla}

In the context of residual streams, we observe that the decision to exit at a given layer can be more effectively informed by analyzing the dynamics of residual stream transformations, instead of solely relying on a classification head applied at the early exit interval $E_j$. To capture the subtle dynamics of residual acceleration, we propose a \textit{Accelerated Residual Latent Attention} (ARLA) mechanism. This approach involves making the exit decision at gate $E_j$ by attending to the residuals spanning from gate $E_{j-1}$ to $E_j$, rather than considering only the residual at gate $E_j$. To minimize the computational overhead associated with exit decision-making, the attention mechanism operates within the latent domain as depicted in ~\cref{fig:arla_arch}. Formally, for each interval $[E_j, E_{j+1}]$, the accelerated residuals are projected into Query ($Q^s_{E_j}, \ldots, Q^s_{E_{j+1}}$), Key ($K^s_{E_j}, \ldots, K^s_{E_{j+1}}$), and Value ($V^s_{E_j}, \ldots, V^s_{E_{j+1}}$) vectors, with latent dimension $d^s$ for $Q^s$, $K^s$, and $V^s$ being significantly smaller than hidden dimension of $p$.\footnote{We use $d^s = 64$ for experiments described in ~\cref{sec:experiments}.} Notably, when the router is allowed to make exit decisions at gate $E_j$ based on residual change dynamics, we observe that the attention is not confined to the residual state at $E_j$ but is distributed across residual states from $E_{j-1}$ to $E_j$, %as illustrated in Figure~\ref{fig:vertical_latent_attention_dynamics}. 
This broader focus on residual dynamics significantly reduces decision ambiguity in early exits, as demonstrated in Figure~\ref{fig:roc_arla}, which contrasts routers based on the last hidden state, and the proposed ARLA router.

%show R -> S transformation. 
%show parameter and flop overhead as compared to adapter on last hidden state.

% \begin{figure}[ht]
%     \centering
%     \includegraphics[width=0.5\textwidth,height=5cm]{sections/figures/roc_arla.pdf}
%     \caption{ROC curves of early exit decision strategies: confidence-based methods (CALM/LITE), routers based on the accelerated hidden state, and latent attention routers.}
%     \label{fig:decision_making_comparison}
% \end{figure}

% \begin{figure}[ht]
%     \centering
%     \includegraphics[width=0.5\textwidth,height=5cm]{vertical_latent_attention.png}
%     \caption{Vertical latent attention mechanism for optimizing early exit decisions by considering residuals from gate \(M\) through \(M-1\).}
%     \label{fig:vertical_latent_attention}
% \end{figure}

\begin{figure}[ht]
    \centering
    \begin{subfigure}{0.52\textwidth}
        \centering
        \includegraphics[width=\textwidth, height = 4cm]{sections/figures/arla_arch.pdf}
        \caption{Accelerated Residual Latent Attention (ARLA): Accelerated residuals between early exit gates are projected into latent domain and attention over residual states within the interval is computed to capture residual dynamics and exit decision is made based on residual saturation.}
        \label{fig:arla_arch}
    \end{subfigure}%
    \hfill
    \begin{subfigure}{0.45\textwidth}
        \centering
        \includegraphics[width=\textwidth, height = 4.5cm]{sections/figures/vla_roc.pdf}
        \caption{ROC classification curves of early exit decision strategies using a linear router used on last residual state ~\cite{schuster2022confident, varshney-etal-2024-investigating, chen2023eellm}  and using ARLA approach that considers residual dynamics. }
        \label{fig:roc_arla}
    \end{subfigure}
    \caption{Effectiveness of ARLA in capturing residual dynamics for early exiting decisions.}


\end{figure}



% \begin{figure}[ht]
%     \centering
%     \includegraphics[width=1\textwidth,height=5cm]{sections/figures/arla.pdf}
%     \caption{fig that plots 32 rows 2 cols heatmap showing attention at each gate}
%     \label{fig:vertical_latent_attention_dynamics}
% \end{figure}

\subsubsection{Self Speculative Decoding} \label{method_self_speculative_decoding}

An alternative means to exploit the early alignment properties of our approach is through the use of accelerated residual states for speculative token sampling to accelerate autoregressive decoding. Speculative decoding aims to speed up memory-bound transformer inference by employing a lightweight draft model to predict candidate tokens, while verifying speculated tokens in parallel and advancing token generation by more than one token per full model invocation \cite{leviathan2023fast, chen2023accelerating, xia2023speculative, miao2023specinfer}. Despite its effectiveness in accelerating large language models (LLMs), speculative decoding introduces substantial complexity in both deployment and training. A separate draft model must be specifically trained and aligned with the target model for each application, which increases the training load and operational complexity ~\cite{chen2023accelerating}. Additionally, this approach is resource-inefficient, as it requires both the draft and target models to be simultaneously maintained in memory during inference \cite{leviathan2023fast, chen2023accelerating}. 

One strategy to address this inefficiency is to leverage the initial layers of the target model itself to generate speculative candidates, as depicted in ~\cite{Tang2024}. While this method reduces the autoregressive overhead associated with speculation, it suffers from suboptimal acceptance rates. This occurs because the linear transformation employed for translating hidden states from layer $k$ to the final layer $N$ is typically a poor approximation, as discussed in ~\cref{sec:motivation} and ~\cref{method_early_exiting}. Our approach resolves this limitation by utilizing accelerated residuals, which demonstrate higher fidelity to their slower counterparts. By utilizing accelerated residuals operating at a rate of $N/k$, where $k$ denotes the number of layers used for candidate speculation, we are able to efficiently generate speculative tokens for decoding.\footnote{We typically set $k = 4$ to balance the trade-off between autoregressive drafting overhead and acceptance rate, as discussed in~\cref{sec:experiments}.}
 This technique not only obviates the need for multiple models during inference but also improves the overall efficiency and effectiveness of speculative decoding.

\begin{figure}
    \centering    \includegraphics[width=1\linewidth]{sections/figures/m2r2_aot_loading.pdf}
    \caption{Ahead-of-Time Expert Loading: M2R2 accelerated residual stream predicts experts required for future layers, reducing reliance on on-demand lazy loading. Speculative pre-loading is efficiently overlapped with computation of multi-head attention (MHA) and MLP transformations. Only incorrectly speculated experts are loaded lazily, resulting in faster inference steps and improved computational efficiency. Here, H indicates LBM Host while D indicates HBM Device.}
    \label{fig:moe_expert_aot_loading}
\end{figure}


\subsubsection{Ahead of Time Expert Loading:} \label{method_aot_expert_loading}

Recent advancements in sparse Mixture-of-Experts (MoE) architectures ~\cite{shazeer2017outrageously, fedus2022switch, artetxe2019massively, lepikhin2020gshard, zoph2022designing} have introduced a paradigm shift in token generation by dynamically activating only a subset of experts per input, achieving superior efficiency in comparison to dense models, particularly under memory-bound constraints of autoregressive decoding \cite{fedus2022switch, zoph2022designing}. This sparse activation approach enables MoE-based language models to generate tokens more swiftly, leveraging the efficiency of selective expert usage and avoiding the overhead of full dense layer invocation. In dense transformer models, pre-loading layers is a common strategy to enhance throughput, as computations of current layer can be overlapped with pre-loading of next layer parameters ~\cite{narayanan2021efficient, shoeybi2020megatron}. However, MoE models face a unique challenge: expert selection occurs dynamically based on previous layer’s output, making it infeasible to preload next layer’s experts in parallel. This limitation results in inherent latency, as expert loading becomes a sequential, on-demand process ~\cite{lepikhin2020gshard, fedus2022switch}.

To address this inefficiency, our method introduces a mechanism with \textit{accelerated residuals}, which not only captures key characteristics of base slower residual states but also exhibit high cosine similarity with their final counterparts (as illustrated in \cref{fig:m2r2_residual_sim}). By employing accelerated residual streams, we can effectively predict the necessary experts for future layers well in advance of their actual invocation. Specifically, using a $2\times$ accelerated residual, the experts needed for layers $2i+2$ and $2i+3$ can be identified while still computing in layer $i$, thus overcoming the bottleneck of sequential, on-demand expert selection and mitigating latency in the decoding pipeline, as shown in \cref{fig:moe_expert_aot_loading}. Note that, we use fixed set of accelerator adapters for transforming accelerated residuals (as discussed in ~\cref{m2r2_method}) while slow residual is transformed via expert routing mechanism. 

Furthermore, our approach integrates a Least Recently Used (LRU) caching strategy, which enhances memory efficiency by replacing the least recently used experts with speculated experts that are anticipated to be needed in upcoming layers. This hybrid approach of preemptive expert loading with LRU caching yields substantial improvements over traditional on-demand loading or standalone caching strategies. By minimizing cache misses and efficiently managing memory, this approach addresses both compute and memory bottlenecks, leading to faster, more resource-efficient token generation in MoE architectures. A comprehensive evaluation of this strategy, in relation to state-of-the-art methods, is provided in \cref{experiments_aot}, and the compute and memory traces on an A100 GPU are detailed in \cref{fig:moe_aot_cuda_trace}.



% Recent advancements in sparse Mixture-of-Experts (MoE) architectures have introduced the concept of utilizing distinct computational paths for different tokens \cite{shazeer2017outrageously}. This approach, wherein only a subset of experts are activated per input, enables MoE-based language models to generate tokens more swiftly compared to their dense counterparts due to memory-bound nature of auto-regressive decoding. In dense models, pre-loading layers in advance is a common strategy to enhance computational efficiency. However, this technique is not applicable to MoE models, where expert selection occurs dynamically based on the outputs of previous layers, preventing parallel pre-fetching of experts.

% Our proposed method addresses this inefficiency. Accelerated residuals, which are highly similar to their slower counterparts (see \cref{fig:similarity}), can reliably predict the necessary experts ahead of time. For instance, by utilizing $2X$ accelerated residual stream, we can predict the experts needed for the layer $2i+1$ and $2i+3$ while carrying out computation in layer $i$. This enables us to commence expert loading significantly earlier, as illustrated in \cref{expert_loading}, effectively mitigating the delays observed with the naive on-demand expert loading. Additionally, our method benefits from incorporating a Least Recently Used (LRU) strategy, where speculated experts replace those that are least recently utilized, resulting in improved performance compared to using either strategy alone. For a comprehensive evaluation, refer to \cref{moe_trace}, which provides a CUDA compute and memory trace of our approach executed on <>.



% A naive solution involves using the residual state of the previous layer along with the gating function of the next layer to predict which experts need to be loaded, and initiating the expert loading process in parallel with the attention computation of the next layer. Yet, as shown in \cref{fig:MOE_attn_vs_loading_time}, the attention computation for medium to long contexts is considerably faster than the expert loading time, making this approach inefficient.




\subsection{Training} \label{method_training}
% This approach is feasible due to the absence of gradient conflicts, as discussed in \cref{sec:grad_conflict}.

To accelerate residual streams, we employ parallel accelerator adapters as described in \cref{m2r2_method}.  For the early exiting use-case outlined in \cref{method_early_exiting}, we define the training objective for these adapters using the following loss function, which combines cross-entropy loss at each exit $E_j$ with distillation loss at each layer $i$. Loss weights coefficients $\alpha_0$ and $\alpha_1$ are employed to balance contribution of corresponding losses.

\begin{align} \label{eq:mr_loss}
L_{\text{m2r2}} = \underbrace{-\alpha_0 \sum_{j=1}^{J} \sum_{t=1}^{T} \log p_{\theta} \left( \hat{y}_t^{E_j} \mid y_{<t}, x \right)}_{\text{cross-entropy loss}} 
+ \underbrace{\alpha_1\sum_{i=1}^{E_{J-1}} \sum_{t=1}^{T} \| \mathbf{p}_{t}^{i} - \mathbf{h}_{t}^{((i - E_{j(i)}) \cdot R_i) + E_{j(i)})} \|^2}_{\text{distillation loss}}.
\end{align}

where $\hat{y}_t^{E_j}$ denotes the predictions from the accelerated residual stream at layer $E_j$ and time step $t$, $y_t$ represents the corresponding ground truth tokens, and $x$ indicates previous context tokens. The distillation loss at each layer $i$ is computed by comparing accelerated residuals at layer $i$ with slow residuals at layer $(i - E_{j(i)}) \cdot R_i + E_{j(i)}$, where $R_i$ denotes the rate of accelerated residuals at layer $i$ while $E_{j(i)}$ represents the most recent gate layer index such that $E_{j(i)} <= i$. \( J \) represents the total number of early exit gates, N denotes number of hidden layers and $E_j$ denotes layer index corresponding to gate index $j$ and \( T \) denotes the sequence length. 

In dynamic compute settings, after training of accelerator adapters, we optimize the query, key, and value parameters governing the ARLA routers (see ~\cref{method_arla}) across all exits in parallel on binary cross entropy loss between predicted decision and ground truth exiting decision. The ground truth labels for the router are determined based on whether the application of the final logit head on $\hat{y}_t^{E_j}$ yields the correct next-token prediction. 


% The objective for this optimization is defined by the following loss function:


%TODO are equations required ? 
% \begin{equation} \label{eq:arla_loss_combined}\small
%     L_{\text{arla}} = -\frac{1}{N} \sum_{t=1}^{T} \left( \sum_{j=1}^{E_n} \left[ O_t^{E_j} \log(\hat{O}_t^{E_j}) + (1 - O_t^{E_j}) \log(1 - \hat{O}_t^{E_j}) \right] \right), \quad \text{where} \quad 
%     O_t^{E_j} = \begin{cases} 
%     1, & \text{if } L(\hat{y}_t^{E_j}) = y_t^{E_j} \\
%     0, & \text{otherwise}
%     \end{cases}
% \end{equation}

% where $\hat{O}_t^{E_j}$ represents the binary predicted logits produced by the vertical latent attention router, as described in \cref{sec:arla}, at gate $E_j$ and time step $t$, and $O_t^{E_j}$ denotes the corresponding ground truth labels. The ground truth labels for the router are determined based on whether the application of the logit head on $\hat{y}_t^{E_j}$ yields the correct next-token prediction. The parameters controlling vertical latent attention are trained concurrently to ensure consistency and efficient use of computational resources.

For self-speculative decoding, as described in \cref{method_self_speculative_decoding}, the training objective remains the same as \cref{eq:mr_loss}, but with the number of intervals set to $J = 1$ and the rate of residual transformation set to $R_n = N/k$, where the first $k$ layers generate speculative candidate tokens. In the context of Ahead-of-Time Expert Loading for Mixture-of-Experts (MoE) models (see \cref{method_aot_expert_loading}), setting the rate of residual transformation to $R_n = 2$ typically offers a good trade-off between the accuracy of expert speculation and AoT pre-loading of experts. 

% Thus, we set $J = 1$ and $E_1 = 16$.


~\subsection{FLOPs Optimization} \label{sec:flops_optimization}

Naively implemented, M2R2 incurs higher FLOP overhead compared to traditional speculative decoding and early exiting approaches such as ~\cite{medusa, schuster2022confident, Tang2024}. However, modern accelerators demonstrate compute bandwidth that exceeds memory access bandwidth by an order of magnitude or more~\cite{databricksLLMInference2023, jouppi2021ten}, meaning increased FLOPs do not necessarily translate to increased decoding latency. Nevertheless, to ensure fair comparison and efficiency in compute bound scenarios, we introduce targeted optimizations.

~\textbf{Attention FLOPs Optimization} For medium-to-long context lengths, attention computation dominates FLOPs in the self-attention layer, surpassing the contribution from MLP layers. Specifically, matrix multiplications involving queries, cached keys, and cached values scale with $l_{kv} * l_{q}$ where $l_{kv}$ denotes previous context length and $l_q$ denotes current query length. Since M2R2 pairs accelerated residuals with slow residuals, a naive implementation results in twice the FLOPs consumption compared to a standard attention layer. To address this, we limit the attention of accelerated residual stream to selectively attend to the top-k most relevant tokens, identified by the slow residual stream based on top attention coefficients\footnote{We set to k = 64 and attend to top 64 tokens as identified by the slow residual stream.}. This is possible since slow and accelerated residual streams are processed in same forward pass and accelerated streams have access to attention coefficients of slow stream. Note that, the faster residual stream still retains the flexibility to assign distinct attention coefficients to these tokens. Furthermore, we design the faster residual stream to employ only 8 attention heads, compared to the 32 heads used in the slow residual stream of the Phi-3 model, reducing query, key, value, and output projection FLOPs by a factor of 1/4. ~\cref{fig:m2r2_num_heads_ablation} indicates effect of using a slicker stream on alignment. As depicted, using $\hat{n}_h = 8$ offers a good trade-off between alignment and FLOPs overhead. 

~\textbf{MLP FLOPs Optimization} The accelerator adapters operating on the accelerated residual stream are intentionally designed with lower rank than their counterparts in the base model. This reduces FLOP overhead by a factor proportional to $hiddenSize / rank$. Additionally, since the faster residual stream uses only 8 attention heads (compared to 32 in the slow residual stream of Phi-3), the subsequent MLP layers process a smaller set of activations, further reducing FLOPs by another factor of 1/4.

These optimizations significantly reduce the FLOP overhead per speculative draft generation, as illustrated in ~\cref{fig:flops_optmization}. Notably, while traditional early-exiting speculative approaches such as DEED require propagating the full slow residual state through the initial layers, incurring substantial computational costs, M2R2 achieves efficient token generation via slimmer, low-rank faster residual streams. In contrast, Medusa introduces considerable FLOP overhead due to per-head computations scaling with $d^2+dv$\footnote{Here $d$ denotes hidden state dimension while $v$ denotes vocab size.}, whereas M2R2 employs low-rank layers for both MLP and language modeling heads, maintaining computational efficiency. All experiments involving the M2R2 approach, as detailed in ~\cref{sec:experiments}, are conducted using these FLOPs optimizations.









% \[
% O_t^{E_j} = 
% \begin{cases} 
% 1, & \text{if } L(\hat{y}_t^{E_j}) = y_t^{E_j} \\
% 0, & \text{otherwise}
% \end{cases}
% \]




%add distillation
% We train accelerator adapters described in \cref{m2r2_method} to accelerate residual streams on next token prediction all in parallel since there are no gradient conflict issues as described in \cref{sec:grad_conflict}.

% \begin{align} \label{eq:mr_loss}
% L_{mr} =  & -\sum_{j = 1}^{E_n} (\sum_{t=1}^{T}\log p_{\theta} (\hat{y}_t^{E_j} | \hat{y}_{<t}, x)) \nonumber
% \end{align}

% where $\hat{y_t^{E_j}}$ denotes predicted logits obtained from accelerated residual stream at gate $E_j$ and time-step $t$ while $y_t^{E_j}$ denotes corresponding truth tokens. 

% Upon training of adapters responsible for accelerating residual streams, we train query, key, value parameters responsible for vertical latent attention of all gates in parallel as

% \begin{equation} \label{eq:arla_loss}
%     L_{arla} = -\frac{1}{N} (\sum_{t=1}^{T}(1\sum_{j=1}^{E_n} \left[ O_t^{E_j} \log(\hat{O}_t^{E_j}) + (1 - o_t^{E_j}) \log(1 - \hat{o_t}_{E_j}) \right]))
% \end{equation}

% where $\hat{O_t^{E_j}}$ denotes binary predicted logits obtained from vertical latent attention router described in \cref{sec:arla} at gate $E_j$ and timestep $t$ while $O_t^{E_j}$ denotes corresponding truth label. Truth labels for router are obtained by computing whether logit head application on $\hat{y}_t^j$ results in true next token prediction. Formally speaking, 

% $O_t^{E_j} = 1 if L(\hat{y_t^{E_j}}) == y_t^{E_j} , 0 otherwise$. 

% Parameters responsible for vertical latent attention are also trained in parallel as well. 

%todo: training slow and fast residuals together and distillation can be two training mdoes. 
%Distillation can be an ablation. 




% Although transformer decoding is memory bound on most mainstream accelerators, there could be scenarios where flop savings are crucial. For instance, on on-device settings power consumption is directly correlated with flops per decoding step and reducing flops does help with overall energy consumption. Vanilla early exiting methods help with flop reduction but suffer from mismatch between training and inference due to early exited tokens. If token at decoding step $t$, $T_t$ exited at layer $E_i$, while token $T_{t+k}$ exits at layer $E_j$ such that $E_i < E_j$, hidden state $H_{t+k}l$ does not have corresponding hidden state $H_tl$ to attend to where $E_i < l <= E_j$. One solution that's often used in literature is to rely on last hidden state available, $H_t{E_j}$, however it tends to be sub-optimal and does affect generation quality \cite{ref}.  To alleviate this mismatch while reducing flops, we train router such that attention mask between token $T_{t+k}$ and token $T_{<t+k}$ is given by: 

% \begin{equation}
%     a_{T_{{t+k}{T_{<t+k}}} = 1 if  E_{T_{<t+k}} >= E{T_{t+k}}
%     else 0
% \end{equation}

% This attention mask enables router to account for exited tokens and get trained accordingly. Since attention mechanism during decoding remains exactly same as that during training, impact on generation quality tends to be minimal as noted in \cref{fig:gen_auality_with_and_without_recompute_attention_show_flops}.  Although MoD does not suffer from training and inference mismatch, we observe that it suffers from discountinuity between pre-training and super-vised fine-tuning resulting in sub-optimal perplexity. On the other hand, our method doesn't not require pre-training , doesn't suffer from discountinuity, and achieves much better perplexity in super-vised fine-tuning and instruction tuning setups as shown in \cref{fig:Mod_vs_m2r2_loss_curves}.






% Our techniques are directly applicable in such scenarios.    




%expert loading with cuda streams in experiments

%%%% Guidelines
\section{Recommendations}
\label{sec:guidelines}

\begin{figure*}
  \includegraphics[width=0.75\textwidth]{overview_erse_V4.pdf}
  \caption{Overview of all ten recommendations}
    \Description{overview}
  \label{fig:over}
\end{figure*}

\autoref{fig:over} provides an overview on our recommendations for engineering \ac{ERS}. In this section, we present each recommendation in detail by providing some background on why the recommendation is important, as well as specific advice. The recommendations are ordered by required effort, so the first ones are easier to include in your daily routines than the later ones. Some of the later recommendation may also be more relevant for larger software projects or senior researchers participating in writing research project proposals. While the recommendations give first advice for each topic, we also recommend learning and collecting further information on each topic, e.g., with tutorials.

\subsection{Inventing is great, reinventing is a waste of time}
\label{sec:reuseAndExtend}

\paragraph{Background} The core of science is to build on top of other's ideas and contributions. Especially in recent years, more and more research software projects are released as open-source and can be extended and reused by the community. Still, we often feel the necessity to start from scratch and build our own library or framework. For example, a recent review identified 63 different open-source, optimization-based frameworks for energy system modeling~\cite{hoffmann_review_2024}. They all share the same underlying principle of network-based energy flow optimization with similar architectures. Many frameworks are only used by their home institutions, which indicates possible redundancy \cite{hoffmann_review_2024}. However, writing code is only one of our many diverse responsibilities.  
Overall, researchers have too little time to reimplement work already done by others. Instead, we should aim to reuse and extend existing high-quality software and focus on implementing novel research contributions. While familiarizing oneself with the documentation of other tools seems more time-consuming in the beginning, this time will be saved later on because existing code often already comes with a lot of testing, verification, and documentation which you would otherwise need to create for your new code.
This recommendation discusses creating high-quality research software without implementing everything from scratch. 

\paragraph{Recommendations} The first step of reusing software is to find existing software that solves your problem or parts of it. The natural approach is to do a literature survey beforehand. While this is an important step, not all software is published as a conference or journal publication. Instead, software registries and similar platforms can be used to find research software. For energy research,  the Open Energy Platform\footnote{\label{fn:oep}\url{https://openenergyplatform.org/}, last access 2024-12-04.} and DOE CODE\footnote{\url{https://www.osti.gov/doecode/}m last access 2025-01-20} are available. Additionally, the Helmholtz Research Software Directory\footnote{\url{https://helmholtz.software/software?&keywords=\%5B\%22Energy\%22\%5D\&page=0\&rows=12}, last access 2024-12-04.} and the \ac{JOSS}\footnote{\label{fn:joss}\url{https://joss.theoj.org}, last access 2024-12-16.} list research software without a domain focus. 
Software repositories like GitHub\footnote{\url{https://github.com/search?q=\%22power\%20system\%22\&type=repositories}, last access 2024-12-04.} can also be searched. In these cases, energy-specific keywords should be used.
The final recommendation for finding reusable software is to ask experienced colleagues and other researchers who work on similar problems. They can often tell you about hitherto unknown tools, e.g., those under development in other research projects, and which libraries are useful.
\par 
To use external software for your research project, you should check their license and see if it is compatible with your code and its envisioned use case (see also \ref{sec:openSource}). 
If the software does not entirely cover your use case, you can contact the software maintainers. They may be already discussing and planning the feature you need. If they do not want to implement it they may wish to include your software when it is finished. By explicitly communicating with other researchers, redundant software can be prevented, and as a side effect, overall collaboration in science is strengthened. This \textit{collaboration first} thinking should follow through the whole software project. For example, write issues and pull requests when finding bugs in other software you use. Not only does this help other people, but it also improves your software as a side effect.

\par 
However, there are exceptions to the rule of reusing existing software. When the existing solutions are of low quality or outdated, starting a new software from scratch can be a good idea. Also, developing an open-source alternative can have a significant scientific impact if the existing software is proprietary. 
In general, parallel development is perfectly fine to some extent. It results in competition and provides other researchers with options that have respective pros and cons.
\par 
Reusing existing research software results in higher-quality software for less effort. It accelerates scientific progress and supports overall collaboration. Hence, the following recommendation: 
\recommendation{Reuse and extend other software if possible and useful!}     

\subsection{Track the history and learn for the future}\label{sec:vcs}

\paragraph{Background} Software is constantly changing. New features are implemented, bugs are fixed, and code is improved. Sometimes, these developments also happen in parallel. For example, you are implementing a physical model to analyze the operating behavior of a battery, and you supervise a student tasked to implement a thermal model for the battery and, therefore, give them a copy of your code. When the student finishes their implementation after a couple of months, your own code has evolved, and manually integrating the master student’s code back into your own becomes a messy task of copy, paste, break, fix, test, and retest. This, and many other problems associated with changing code, multiple versions, and coding as a team, can be avoided with version control. While using version control is standard practice in software development, it is not always the case in research software. Tracking different versions of your code enables you and others to keep an overview of which feature was developed and to understand the history of the code, thereby increasing its maintainability. It also allows for the undoing of specific changes when a problem occurs. Additionally, version control enables reproducibility of the research conducted with the code by providing specific links to use  exactly the same version~\cite{sandve2013ten}.

\paragraph{Recommendations} Version control systems can help you track different software versions. The most popular version control system is Git\footnote{\url{https://git-scm.com/}, last access 2024-12-17.}.
The platforms GitHub\footnote{\url{https://github.com/}, last access 2024-12-17.} and GitLab\footnote{\url{https://about.gitlab.com/}, last access 2024-12-18} are based on Git and extend it by a user interface providing a good overview and additional features. If your institution does not allow you to use GitHub or the public GitLab, you can often also use a GitLab instance offered by your institution.

Within Git, changes to the software are included as a \textit{commit}. By adding meaningful messages to each commit, the changes are documented. At least one commit at the end of each day is recommended. 

Version control systems also allow different \textit{branches}, which are parallel versions of the same software. It is helpful to always have a working software version in one branch (usually called \textit{master} or \textit{main}). Also, this way, different developers can work in parallel without directly causing conflicts. A branch can be merged into another by a \textit{pull} or \textit{merge request}. These requests are another good option for additional comments and documentation to make the changes more understandable. In software projects with multiple developers, this is also the step where code reviews should be conducted for quality assurance.

The version control platforms also have additional features that make it easier to organize the software development. \textit{Issues} can be used to track bugs, code contributions, and new potential feature ideas. They also allow others to get an overview of what is planned next. Additionally, they provide options for \ac{CICD}, which is the practice of automating the integration, testing, and deployment of code changes to ensure reliable and efficient software delivery. In the context of research software, \ac{CICD} is especially used for automated testing of each commit (see also \ref{sec:testing}).

\par 
Version control systems are necessary to keep track of software versions while also allowing to organize the software development to a certain extent. If you are not familiar with version control, we recommend taking some basic training on it, e.g., by the software carpentry\footnote{\url{https://swcarpentry.github.io/git-novice/index.html}, last access 2024-12-17}.  

\recommendation{Use version control!}
\subsection{Research should be publicly available, and your software is part of it}\label{sec:openSource}

\paragraph{Background} Only openly available research software allows other researchers to reproduce your research results. Since research software is mainly publicly funded, it is also fair to make it publicly available. This also enables others to reuse the software, providing additional benefits like further testing of the software and contributing by creating extensions and compatible software. Besides making the software open source, it is also important to register it so other researchers can find it. These aspects are essential to make a research software \ac{FAIR}~\cite{barker2022introducing}. 

\paragraph{Recommendations} We recommend developing your research software open source. By directly starting the development as open source, challenges can be avoided when switching to open source later. Further, it allows other researchers to use your software and actively contribute as early as possible, improving your software's quality and encouraging cooperation. To develop the software open source, a suitable license is required. By choosing a license early each time another software is included, the compatibility of licenses can be checked directly. Some open source licenses are incompatible since they put certain conditions on the reuse of the code, which can conflict between licenses \cite{cui_empirical_2023}.
When choosing a license, you should check your institution's policy, which often already proposes a specific license. Also, various guides help to choose a license\footnote{e.g., \url{https://choosealicense.com/}, last access 2024-12-17.}.

Cooperation and joint research projects with industry are common in energy research. Sometimes, industry partners are critical of developing open source because they want to keep their intellectual property. Generally, we recommend discussing this topic as early as possible to find reasonable compromises. Often, certain parts can be developed open source while others remain closed source. Since open source is open to all researchers and all industries, it can also improve the exchange between industry and research. 

When you develop open source, there are specific approaches to make your code more easily reusable by others. First, the repository should follow programming-language-specific best practices. This can be achieved by starting with templates for the repository\footnote{e.g., cookie-cutter templates at \url{https://cookiecutter.io/templates}, last access 2024-12-17}. Additionally, citing your software can be made easy by including a \ac{CFF} file\footnote{\url{https://citation-file-format.github.io}, last access 2024-12-17.} in your repository
 \cite{druskat_citation_2021}.

Within the repository, it should be indicated if the software will be maintained, if support is available and if the developers are open to joining research projects, including the software. We recommend using the features of the software platform (GitHub/GitLab) to interact with potential users, e.g., by using issues.

Besides making the source code available, the software should also be findable. Therefore, we recommend registering the software in a domain-specific registry like the Open Energy Platform\footref{fn:oep}. Getting a DOI for the software is also helpful, e.g., by archiving versions of the software on Zenodo\footnote{\url{https://zenodo.org/}, last access 2024-12-17.} \cite{zenodo}. 

\par
Energy research is highly interdisciplinary \cite{tijssen_quantitative_1992}.
Therefore, we recommend adding a very general description to your software, allowing all researchers to understand its goals. This way, more researchers can identify whether the software is useful for them, increasing its reusability. GitHub also allows you to provide keywords to your repository, which again improves findability.


\par 
Platforms like GitHub and GitLab make it easy to publish your software under an open source license. Open source research software enables reproducibility and allows reusability, which can also improve your code quality. Therefore, we recommend: 

\recommendation{Develop open source \& make your software findable!}
\subsection{Being organized boosts your software project}
\label{sec:workProcess}

\paragraph{Background} This recommendation discusses how research software development teams should organize themselves.

The most important aspect here is that there is no one-size-fits-all solution.
A one-person PhD software project should be organized differently than a large team developing software with thousands of potential users.

\paragraph{Recommendations} The main recommendation is to explicitly define the software project's scope and purpose early in the process. What is the software's general use case (see \ref{sec:architecture})? How many users are envisioned (see \ref{sec:community})? 
Is the software intended for research only or for usage by industry as well? What is its expected lifespan? Will complete reimplementation be an option, e.g., for going commercial?

Also, you should get a good understanding of the available team to create the software. 
How many people are working on the software project permanently? What are their skills? 

Explicit answers to these questions are required to make wise decisions about organizing the software project and the team. 
For example, defining the software's scope helps to decide what to implement and what not. 
The team size, on the other hand, is the most important factor for defining the development process. Do we need code reviews? Do we need people responsible for specific tasks like testing, documentation? Especially, large teams must define and enforce the process explicitly to not end in chaos, like outdated documentation (see \ref{sec:documentation}) or failing pipelines (see \ref{sec:testing}).

Generally, you can think of software development organization as a scale from a high degree of agility (e.g., extreme programming~\cite{796139}) to a high degree of planning (e.g., waterfall model \cite{Petersen835760}). Consider which degree is desirable for your project, depending on the answers to the precious questions. Most research projects have a high degree of uncertainty, resulting in frequent changes in software requirements. This would naturally suit a more agile organization.

Another critical question is the developers' software engineering experience. Especially in energy research, the skill levels vary significantly, as we pointed out in \autoref{sec:intro}. 
We researchers are not expected to have the skills of a full-time professional developer. However, we \textbf{are} software developers, which mandates certain minimum skill requirements. Therefore, we should perform a skill assessment of all team members and actively provide training to fill the gaps. 
\par 
Further, it is important to assign clear responsibilities. Primarily the software project lead should be clear. It should be someone who actively participates in the software project and does not have too many other responsibilities. For example, an experienced research assistant is often a better choice than a professor who does not have the time to participate actively.

\par 
Unorganized software projects result in low-quality software, so we should explicitly discuss and define our work process. Since there are no general rules, we need clear leadership and a clear vision of our software project's scope and purpose.

\recommendation{Organize yourself and your team!} 

\subsection{Building upon research requires reusable software}\label{sec:reuse}

\paragraph{Background} To effectively build upon research, it is often required to build upon research software, and therefore, a software's reusability is a key factor to its scientific success. Research based on openly available and reusable software is cited more often than research where the software is not reusable. To achieve this, it is necessary to consider your software's applicability to more general problems (than your own). This is decided by your software's abstraction level (generalized vs. specific implementation). However, there is a conflict regarding this abstraction level most of the time. There are two extreme ends to this aspect:
\begin{enumerate}
    \item Implementing highly specialized features that serve exactly one use case, e.g., a specific heat pump model that does not enable any parameterization (e.g., to implement different coefficients of performance, capacities, or others). 
    \item Finding an abstraction for every feature, which provides maximal freedom for analyzing other use cases than the one that shall be implemented in the first case, e.g., a generalized heat pump model, which does not even have a specific mathematical model but enables you to provide an arbitrary one by yourself.
\end{enumerate}
Both ends are extreme, and usually, you want to avoid both cases. However, there is a large spectrum of possibilities between these extremes. 

\paragraph{Recommendations} We recommend finding a sensible abstraction level, as it is crucial for the reusability of the software, and good abstractions enable other researchers (including yourself and your group) to build upon your software. At the same time, no unnecessary complexities should be introduced by choosing an overgeneralized abstraction. This might render your software less usable due to the lack of specific implementations and understandability. Consequently, users need to invest more effort, and reusing becomes difficult. 

Another aspect of this is to think about the scope and objective of your software. These determine to which level your software should be abstracted. As a rule of thumb, every artifact should focus on precisely one type of research. This can be on different, relatively macroscopic levels of energy research. For example, you can develop one type of energy component model (i.e., for heat pumps) or one software to simulate decentralized individual behaviors of actors in energy communities using agent-based modeling.

Besides the level of your implementation, meta aspects, like the name of the software, can also impact the reusability of your software. Also, we recommend considering the scientific community you contribute to and how they organize themselves in open software projects. Integrating your contribution in the form of software to these software communities increases your visibility and strengthens the community as a whole. Example communities for \ac{ERS} could be based on research frameworks, e.g., oemof\footnote{\url{https://oemof.org}, last access 2024-12-18.}, or generally communities of your favorite programming language, e.g., the Julia communities\footnote{\url{https://julialang.org/community/organizations/}, last access 2024-12-18}. Furthermore, considering the target audience, which will eventually reuse your software, is essential. To improve the compatibility of the software, you can use ontologies, such as the open energy ontology\footnote{\url{https://openenergyplatform.org/ontology/}, last access 2024-12-18} to develop and validate whether the keywords used in your software and documentation fit those of the broader community.

In short, software reusability is essential for making energy research sustainable and interoperable, not only for you and your working group but also for the broader research community you are working in.

\recommendation{Consider the reuse of your research software!}

\begin{figure*}[t]
\vskip 0.2in
\begin{center}
\centerline{
\includegraphics[width=\textwidth, height=9cm]{figures/architecture_img.pdf}}   
\vspace{-3mm}
\caption{\textbf{Overview of our method at the blending stage. }
% condition
Two input images or concepts are encoded into embeddings, mapped to a shared text space via the Linear Prior Converter from unCLIP~\citep{ramesh2022hierarchical}. These embeddings condition the U-Net: one for downsampling, the other for upsampling.
% module
During the blending stage, a blending latent $L_b$ initialized with Gaussian noise is processed in the Feedback Interpolation Module, conditioned on image embeddings. Noise $\epsilon$ is added to the embeddings to generate initial auxiliary latents, which are interpolated into $L^{(t)}_{b}$ with an increasing weight $p$. The  $L^{(t)}_{a}$ is combined with interpolated latent $L'^{(t)}_{b}$ by proportion $p$. All updated $L'^{(t)}_{a}$ are refined in the auxiliary inference to retain original features using the text prompt for corresponding categories, and $L'^{(t)}_{b}$ is denoised via the blending inference.
% refinement
Finally, the refined $ L_b $ is passed into the VAE decoder to generate the final blending image. 
}
\label{architecture}
\end{center}
% \vskip -0.4in
\vspace{-8mm}
\end{figure*}

\subsection{Undocumented code can not be used by others, and is therefore near-to useless}\label{sec:documentation}

\paragraph{Background} When writing code, documentation is essential to help you and others understand the code. 
Documentation also helps to ensure that the code is doing what it is supposed to do. Furthermore, it allows the extension and adaption of the code and is fundamental to ensure reproducibility and reusability for others and yourself.\par

\paragraph{Recommendations} We recommend always writing documentation, no matter how large the software project is (even if only one person is involved). Any small documentation is always better than no documentation. However, software papers do not count as documentation since they cannot reflect the ever-changing nature of software.  \par
Multiple types of documentation exist:
\begin{enumerate}
    \item Documentation in the code, e.g., code docstrings, describing methods, classes, or modules. This helps others understand your code.
    \item A README file, which introduces and describes the software. This is necessary to explain the purpose and concept of the software.
    \item A documentation page, like \textit{readthedocs}\footnote{\url{https://about.readthedocs.com/}, last access 2025-01-17}, containing further descriptions of the software, helping others to fully understand the code and use it.
    \item Documentation of tutorials or concrete examples, e.g., included in the documentation page, helps others to easily apply the code.
\end{enumerate}

We recommend to consider your open source strategy (see \ref{sec:openSource}) and try to deduce the applicable documentation style and technique. If it is unclear how the documentation will be processed, you can stick to easy documentation modes. In any case, at least the following should be provided: a README file, including a summary of the purpose of the software, an installation guide, and an example script using your software.

Documentation requires considerable effort and should thus be acknowledged as part of the software development process, e.g., as an own contribution to the respective research project, and relevant resources should be allocated. It is important not to neglect it due to missing resources later on.
When planning the documentation, it is recommended to take into account the different types of documentation and the respective target audience. You can define minimal requirements for your documentation (as requirements and dependencies for the implementation, environment, and decisions made in the development process). Also, we recommend considering tests as part of the documentation, as these are necessary for others to understand your code. To support this understanding, example scenarios can be introduced in the documentation. However, all example scenarios should be fully executable.
To simplify the documentation process, we recommend combining the coding with the documentation (for example, via commit messages, see \ref{sec:vcs}) by writing readable code, including comments. Additionally, you can use templates and standards for your documentation. Many tools are available to make the documentation process easier and to help make your documentation findable. For Python, \textit{Sphinx}\footnote{\url{https://docs.readthedocs.io/en/stable/intro/sphinx.html}, last access 2025-01-17} can be used to create technical documentation, as can \textit{Documenter.jl} for Julia\footnote{\url{https://github.com/JuliaDocs/Documenter.jl}, last access 2025-01-17}.\par 

It also helps to look for best practices. These are for example provided by Github\footnote{\url{https://google.github.io/styleguide/docguide/best_practices.html}, last access 2025-01-17}, but can also be part of recommendations for your respective programming language, as \textit{pep8} for python\footnote{\url{https://peps.python.org/pep-0008/}, last access 2025-01-17}.

Due to the interdisciplinary character of energy systems research, it is recommended that the concept behind the software be explicitly discussed. People from different domains should understand the purpose of your software. Thus, also the concepts and ideas of your software, not only the software itself, should be documented. Additionally, the code itself should be documented so that other people can work with it or even extend it. 
\recommendation{You are not done until the documentation is done!}

\documentclass{article}
\usepackage{xcolor}
\usepackage{graphicx}
\definecolor{darkgreen}{rgb}{0.1, 0.5, 0.1}
\usepackage[a4paper,margin=1in]{geometry}
\usepackage{array}

\begin{document}

\begin{table}[t]
% \centering
\raggedright
\renewcommand{\arraystretch}{2}
\setlength{\tabcolsep}{8pt}
\begin{tabular}{|p{4cm}|p{4cm}|p{4cm}|p{4cm}|}
\hline
\textbf{{\fontsize{12pt}{22pt}\selectfont Comparison with Human Reviews}} & \textbf{{\fontsize{12pt}{22pt}\selectfont Factual \newline Accuracy}} & \textbf{{\fontsize{12pt}{22pt}\selectfont Analytical Depth}} & \textbf{{\fontsize{12pt}{22pt}\selectfont Actionable \newline Insights}} \\ \hline

{\fontsize{12pt}{22pt}\selectfont 
\textcolor{blue}{``The explanation of the results is adequate''} (AI) vs. \textcolor{darkgreen}{``The explanation is too brief and misses key statistical trends in Figure 3, such as the anomaly at epoch 50''} (human).} & 
{\fontsize{12pt}{22pt}\selectfont \textcolor{blue}{``The paper uses supervised learning techniques effectively''} (AI), but the actual technique described is reinforcement learning.} & 
{\fontsize{12pt}{22pt}\selectfont \textcolor{blue}{``The methodology section is sufficient''} (AI), without noting that \textcolor{darkgreen}{``The comparison to baseline models lacks clarity, especially in explaining the choice of hyperparameters''} (human).} & 
{\fontsize{12pt}{22pt}\selectfont \textcolor{blue}{``Provide more examples for better understanding''} (AI), instead of \textcolor{darkgreen}{``Add examples demonstrating how the algorithm performs under different lighting conditions to clarify its robustness''}} \\ \hline

{\fontsize{12pt}{22pt}\selectfont \textcolor{blue}{``Overall, the related work section is relevant''} (AI) vs \textcolor{darkgreen}{``The related work section does not include recent advancements in transformer-based architectures, such as XYZ-2023''} (human)} & 
{\fontsize{12pt}{22pt}\selectfont \textcolor{blue}{``The dataset appears to be balanced''} (AI), but the dataset is actually imbalanced based on the class distributions mentioned in Section 4.2} & 
{\fontsize{12pt}{22pt}\selectfont \textcolor{blue}{``The discussion is clear''} (AI), but it misses feedback like \textcolor{darkgreen}{``The discussion should explore why the proposed approach underperforms on Dataset B, as highlighted in Table 2''} (human)} & 
{\fontsize{12pt}{22pt}\selectfont \textcolor{blue}{``Clarify the introduction''} (AI), rather than \textcolor{darkgreen}{``Reorganize the introduction to define the problem before introducing the contributions, as this will improve flow and reader engagement''}} \\ \hline

{\fontsize{12pt}{22pt}\selectfont \textcolor{blue}{``The conclusion is well-written''} (AI) vs. \textcolor{darkgreen}{``The conclusion does not address limitations, such as the small sample size used in the experiments''} (human)} & 
{\fontsize{12pt}{22pt}\selectfont \textcolor{blue}{``The results suggest strong performance''} (AI), but it incorrectly claims \textcolor{darkgreen}{``The model outperforms all baselines,''} while Table 3 shows it underperforms in some metrics} & 
{\fontsize{12pt}{22pt}\selectfont \textcolor{blue}{``Results are promising''} (AI), lacking human feedback such as \textcolor{darkgreen}{``Consider expanding on the implications of your findings for real-world applications, particularly in autonomous navigation''}} & 
{\fontsize{12pt}{22pt}\selectfont \textcolor{blue}{``Improve the figures for better clarity''} (AI), rather than \textcolor{darkgreen}{``Increase the font size in Figure 4 and add units to the axes labels for better readability''}} \\ \hline
\end{tabular}
% \caption{Concrete Examples of Challenges in AI-Generated Reviews with Color Coding}
\label{tab:review_challenges}
\end{table}

\end{document}


%%%%%%%% ICML 2025 EXAMPLE LATEX SUBMISSION FILE %%%%%%%%%%%%%%%%%

\documentclass[11pt]{article}
\usepackage[review]{acl}
\usepackage{times}
\usepackage{latexsym}
\usepackage[T1]{fontenc}
\usepackage[utf8]{inputenc}
\usepackage{microtype}
\usepackage{inconsolata}
% Recommended, but optional, packages for figures and better typesetting:
\usepackage{tikz}
\newcommand*\circled[1]{\tikz[baseline=(char.base)]{
            \node[shape=circle,draw,inner sep=2pt] (char) {#1};}}
\usepackage{microtype}
\usepackage{graphicx}
\usepackage{subfigure}
\usepackage{booktabs} % for professional tables
\usepackage{algorithm}
\usepackage{algorithmicx}
% \usepackage{algorithmic}

% hyperref makes hyperlinks in the resulting PDF.
% If your build breaks (sometimes temporarily if a hyperlink spans a page)
% please comment out the following usepackage line and replace
% \usepackage{icml2025} with \usepackage[nohyperref]{icml2025} above.
\usepackage{hyperref}
\usepackage[noend]{algpseudocode} % For pseudocode
\usepackage{amsmath} % For math environments



% Attempt to make hyperref and algorithmic work together better:
%\newcommand{\theHalgorithm}{\arabic{algorithm}}

% Use the following line for the initial blind version submitted for review:
%\usepackage{icml2025}


% If accepted, instead use the following line for the camera-ready submission:
% \usepackage[accepted]{icml2025}

% For theorems and such
\usepackage{amsmath}
\usepackage{amssymb}
\usepackage{mathtools}
\usepackage{amsthm}

% if you use cleveref..
\usepackage[capitalize,noabbrev]{cleveref}

\usepackage{xcolor}
\usepackage{amsmath}
\usepackage{listings}
\usepackage{hyperref}
\usepackage{graphicx}
\usepackage{tcolorbox}
%\usepackage{geometry}
% \geometry{margin=1in}

%%%%%%%%%%%%%%%%%%%%%%%%%%%%%%%%
% THEOREMS
%%%%%%%%%%%%%%%%%%%%%%%%%%%%%%%%
\theoremstyle{plain}
\newtheorem{theorem}{Theorem}[section]
\newtheorem{proposition}[theorem]{Proposition}
\newtheorem{lemma}[theorem]{Lemma}
\newtheorem{corollary}[theorem]{Corollary}
\theoremstyle{definition}
\newtheorem{definition}[theorem]{Definition}
\newtheorem{assumption}[theorem]{Assumption}
\theoremstyle{remark}
\newtheorem{remark}[theorem]{Remark}

% Todonotes is useful during development; simply uncomment the next line
%    and comment out the line below the next line to turn off comments
%\usepackage[disable,textsize=tiny]{todonotes}
\usepackage[textsize=tiny]{todonotes}
\usepackage{enumitem}
\usepackage{booktabs}
\usepackage{pifont}
\usepackage{makecell}
\newcommand{\cmark}{\ding{51}}  % Checkmark
\newcommand{\xmark}{\ding{55}}  % Crossmark
% The \icmltitle you define below is probably too long as a header.
% Therefore, a short form for the running title is supplied here:
%\icmltitlerunning{ReviewEval: An Evaluation Framework for AI-Generated Reviews}
\title{ReviewEval: An Evaluation Framework for AI-Generated Reviews}

\begin{document}

%\twocolumn[
%\icmltitle{ReviewEval: An Evaluation Framework for AI-Generated Reviews}

% It is OKAY to include author information, even for blind
% submissions: the style file will automatically remove it for you
% unless you've provided the [accepted] option to the icml2025
% package.

% List of affiliations: The first argument should be a (short)
% identifier you will use later to specify author affiliations
% Academic affiliations should list Department, University, City, Region, Country
% Industry affiliations should list Company, City, Region, Country

% You can specify symbols, otherwise they are numbered in order.
% Ideally, you should not use this facility. Affiliations will be numbered
% in order of appearance and this is the preferred way.
% \icmlsetsymbol{equal}{*}

%\begin{icmlauthorlist}
%\icmlauthor{Firstname1 Lastname1}{equal,yyy}
%\icmlauthor{Firstname2 Lastname2}{equal,yyy,comp}
%\icmlauthor{Firstname3 Lastname3}{comp}
%\icmlauthor{Firstname4 Lastname4}{sch}
%\icmlauthor{Firstname5 Lastname5}{yyy}
%\icmlauthor{Firstname6 Lastname6}{sch,yyy,comp}
% \icmlauthor{Firstname7 Lastname7}{comp}
% %\icmlauthor{}{sch}
% \icmlauthor{Firstname8 Lastname8}{sch}
% \icmlauthor{Firstname8 Lastname8}{yyy,comp}
%\icmlauthor{}{sch}
%\icmlauthor{}{sch}
%\end{icmlauthorlist}

% \icmlaffiliation{yyy}{Department of XXX, University of YYY, Location, Country}
% \icmlaffiliation{comp}{Company Name, Location, Country}
% \icmlaffiliation{sch}{School of ZZZ, Institute of WWW, Location, Country}

% \icmlcorrespondingauthor{Firstname1 Lastname1}{first1.last1@xxx.edu}
% \icmlcorrespondingauthor{Firstname2 Lastname2}{first2.last2@www.uk}

% % You may provide any keywords that you
% % find helpful for describing your paper; these are used to populate
% % the "keywords" metadata in the PDF but will not be shown in the document
% \icmlkeywords{Machine Learning, ICML}

% \vskip 0.3in
% ]

% this must go after the closing bracket ] following \twocolumn[ ...

% This command actually creates the footnote in the first column
% listing the affiliations and the copyright notice.
% The command takes one argument, which is text to display at the start of the footnote.
% The \icmlEqualContribution command is standard text for equal contribution.
% Remove it (just {}) if you do not need this facility.

% \printAffiliationsAndNotice{}  % leave blank if no need to mention equal contribution
% \printAffiliationsAndNotice{\icmlEqualContribution} % otherwise use the standard text.
\maketitle
\begin{abstract}
% We explain the design and implementation of our custom pipeline developed to generate comprehensive reviews for research papers using Large Language Models (LLMs). Our approach leverages Google's latest LLM, Gemini 1.5 Pro, orchestrated through a multi-stage process that includes prompt generation, iterative refinement via reflection loops, and final evaluation to ensure the production of high-quality reviews aligned with specific conference guidelines.
% The exponential growth of academic research has strained traditional peer review processes, prompting the exploration of Large Language Models (LLMs) for automated review generation. While LLMs demonstrate potential in research reviewing, their effectiveness remains limited by issues such as superficial critiques, hallucinations, and lack of actionable insights. Moreover, there has been limited research on developing comprehensive evaluation frameworks to access these LLM based paper reviewers. To address this, we propose a comprehensive evaluation framework that assesses AI-generated reviews across four key dimensions: (1) Alignment with Human Reviews, (2) Factual Accuracy, (3) Analytical Depth, and (4) Actionable Insights. Our goal is to enhance the reliability of AI-generated reviews through a comprehensive evaluation framework and establish standardized metrics for assessing AI-based review systems. Additionally, our approach introduces a novel conference-specific alignment mechanism, addressing the need for adaptable reviewers that cater to varying evaluation priorities across conferences, a gap not previously explored. Furthermore, we propose a self-refinement loop, enabling the LLM to iteratively improve its review prompts, thereby enhancing the quality and depth of AI-generated reviews.

%The escalating volume of academic research, coupled with a shortage of qualified reviewers, necessitates innovative approaches to peer review. While LLMs offer potential for automating this process, their current limitations include superficial critiques, hallucinations, and a lack of actionable insights. This research addresses these challenges by proposing a comprehensive evaluation framework for AI-generated reviews, encompassing alignment with human reviews, factual accuracy, analytical depth, and actionable insights.Furthermore, we introduce a novel conference/journal specific alignment mechanism to adapt LLM-generated reviews to the unique evaluation priorities of different conferences and journals. To enhance the quality of these reviews, we propose a self-refinement loop that iteratively improves the LLM's review prompts. This framework aims to establish standardized metrics for assessing AI-based review systems and enhance the reliability of AI-generated reviews in academic research.

The escalating volume of academic research, coupled with a shortage of qualified reviewers, necessitates innovative approaches to peer review. While large language model (LLMs) offer potential for automating this process, their current limitations include superficial critiques, hallucinations, and a lack of actionable insights. This research addresses these challenges by introducing a comprehensive evaluation framework for AI-generated reviews, that measures alignment with human evaluations, verifies factual accuracy, assesses analytical depth, and identifies actionable insights. We also propose a novel alignment mechanism that tailors LLM-generated reviews to the unique evaluation priorities of individual conferences and journals. To enhance the quality of these reviews, we introduce a self-refinement loop that iteratively optimizes the LLM's review prompts. Our framework establishes standardized metrics for evaluating AI-based review systems, thereby bolstering the reliability of AI-generated reviews in academic research.

%-----Murari----
% The escalating volume of academic research, coupled with a shortage of qualified reviewers, necessitates innovative approaches to peer review. While large language model (LLMs) offer potential for automating this process, their current limitations include superficial critiques, hallucinations, and a lack of actionable insights. This research addresses these challenges by introducing a comprehensive evaluation framework for AI-generated reviews, that measures alignment with human evaluations, verifies factual accuracy, assesses analytical depth, and identifies actionable insights. We also propose a novel alignment mechanism that tailors LLM-generated reviews to the unique evaluation priorities of individual conferences and journals. To enhance the quality of these reviews, we introduce a self-refinement loop that iteratively optimizes the LLM's review prompts. Our framework establishes standardized metrics for evaluating AI-based review systems, thereby bolstering the reliability of AI-generated reviews in academic research.\vspace{-7mm}


% Empirical results demonstrate that our approach significantly improves prompt quality and review rigor. This research advances AI-driven peer review by addressing key shortcomings in evaluation transparency, depth, and adaptability to domain-specific guidelines.
\end{abstract}

\section{Introduction}
The rapid growth of academic research, coupled with a shortage of qualified reviewers, has created an urgent need for scalable and high-quality peer review processes \cite{petrescu2022evolving, schulz2022future, checco2021ai}. Traditional peer review methods are under mounting pressure from the exponentially growing number of submissions, particularly in fields like artificial intelligence, machine learning, and computer vision. This has led to a growing interest in leveraging large language models (LLMs) to automate and enhance various aspects of the peer review process~\cite{robertson2023gpt4, liu2023reviewergpt}.\par

LLMs have shown remarkable potential in automating various natural language processing tasks, such as summarization, translation, and question-answering. However, their effectiveness in serving as reliable and consistent paper reviewers remains a significant challenge. The academic community is already experimenting with AI-assisted reviews, as evidenced by reports that 15.8\% of reviews for ICLR 2024 were generated with AI assistance~\cite{latona2024ai}. While this demonstrates the growing adoption of LLMs in peer review, concerns have been raised regarding their impact on the reliability and fairness of the review process. Specifically, papers reviewed by AI have been perceived to gain an unfair advantage, leading to questions about the integrity of such evaluations. Consequently, research into robust automated review generation systems is crucial, necessitating rigorous evaluation of AI generated reviews to address key challenges.~\cite{zhou2024llm} provide a comprehensive analysis of the use of commercial models, such as GPT-3.5 and GPT-4 \cite{achiam2023gpt}, as research paper reviewers. Their findings highlight key limitations, including the potential for mistakes due to either model hallucinations or an incomplete understanding of the material, as well as the inability to provide critical feedback comparable to human reviewers. Based on our preliminary experiments of using GPT-4 to generate reviews for research papers, we identified additional limitations in AI-generated reviews, including a lack of actionable insights and limited analytical depth, often characterized by generic and vague feedback. These shortcomings stem from the inherent tendency of LLMs to generate superficial reviews.

%\vspace{-2mm}

% The adoption of effective AI-based reviewing systems in the research communities addresses several critical needs \cite{tyser2024ai}: (i) they provide early feedback to authors on their work in progress, facilitating iterative improvements and enhancing the overall quality of research outputs; (ii) they enable conferences to manage the growing volume of submissions more efficiently, ensuring high-quality and timely reviews; and (iii) they contribute to quality control by reducing inconsistencies and potential biases in the peer-review process.

% Should we add more details about us gathering these challenges?

% \begin{table}[h]
%     \centering
%     \renewcommand{\arraystretch}{0.9} % Adjust row height for better readability
%     \begin{tabular}{l>{\centering\arraybackslash}m{2.5cm} >{\centering\arraybackslash}m{2.5cm}}
%         \toprule
%         & \makecell{\textbf{Wider} \\ \textbf{Metric Coverage}} & \makecell{\textbf{Black Box} \\ \textbf{Remedy}} \\
%         \midrule
%         \textbf{Our Framework} & \cmark & \cmark \\
%         \textbf{MARG} & \xmark & \xmark \\
%         \textbf{Aspect-Based} & \xmark & \xmark \\
%         \bottomrule
%     \end{tabular}
%     \caption{Comparison of Evaluation Methods. The evaluation metrics by \cite{d2024marg} (MARG) and \cite{zhou2024llm} (Aspect-Based Comparisons) focus on AI-human review similarity but overlook other key deficiencies in AI reviews. Their heavy reliance on LLMs for end-to-end evaluation creates a black-box system with limited transparency. Our framework addresses these gaps with a more comprehensive and interpretable approach.}
%     \label{tab:evaluation_comparison}
% \end{table}



\begin{figure}[t]
    \centering
    \includegraphics[width=0.5\textwidth, trim=0cm 0cm 0cm 0cm, clip]{table-diagram.pdf} % Adjust width for good fit
    \caption{\small Examples of the challenges and limitations of AI based research paper reviews}
    % \vspace{0.1in} % Adds 0.1 inches of space after the caption
    \label{fig:challenges}
\end{figure}

% There has been limited existing research on developing evaluation metrics to evaluate AI based research paper reviews. \cite{d2024marg} proposed an automated metric to check for approximate matches between AI-generated and human-written review comments using GPT. While GPT-4 is employed iteratively to extract approximate matches and mitigate inconsistencies, it relies completely on GPT-4 for end-to-end evaluation, making the whole metric a black box and unreliable. \cite{zhou2024llm} paper examines aspect coverage
% and similarity to reference reviews through both automatic
% metrics and manual analysis, utilizing the ASAP dataset
% \cite{yuan2022can}, which annotates review sentences with
% labels such as summary, motivation, originality, soundness,
% substance, replicability, meaningful comparison, and clarity.
% However, relying on pre-defined aspect coverage under-
% mines the diversity and depth of AI and human reviews, therefore we recognized the need for a more nuanced and
% dynamic topic coverage metric

% \begin{table}[h]
%     \centering
%     \begin{tabular}{lcc}
%         \toprule
%         & \makecell{\textbf{Dynamic Topic} \\ \textbf{Coverage}} & \makecell{\textbf{Remedying} \\ \textbf{Black Box Nature}} \\
%         \midrule
%         \textbf{Our Evaluation Framework} & \cmark & \cmark \\
%         \textbf{MARG} & \xmark & \xmark \\
%         \textbf{Aspect Coverage Based} & \xmark & \xmark \\
%         \bottomrule
%     \end{tabular}
%     \caption{Comparison of Evaluation Methods}
%     \label{tab:evaluation_comparison}
% \end{table}

Existing research on evaluation metrics for AI-generated research paper reviews remains limited. For instance,~\cite{d2024marg} proposed an automated metric to evaluate approximate matches between AI-generated and human-written review comments using GPT-4~\cite{achiam2023gpt}. Although their method iteratively employs GPT-4 to extract approximate matches and mitigate inconsistencies, its complete reliance on GPT-4 renders the evaluation process a black box, thereby limiting transparency and raising concerns about reliability. Similarly, \cite{zhou2024llm} investigated the aspect coverage and similarity between AI and human reviews through a blend of automatic metrics and manual analysis. Their work leveraged the ASAP dataset \cite{yuan2022can}, which categorizes review sentences into predefined aspects—including summary, motivation, originality, soundness, substance, replicability, meaningful comparison, and clarity, to align AI and human reviews. However, beyond this AI-human comparison, their approach overlooks other critical dimensions where AI reviews may underperform, as highlighted in Figure~\ref{fig:challenges}.\par

% While GPT-4 is employed iteratively to extract approximate matches and mitigate inconsistencies, it relies entirely on GPT-4 for end-to-end evaluation, making the metric a black-box system with limited transparency and reliability. 

%\cite{zhou2024llm} examined aspect coverage and similarity between AI and human reviews through a combination of automatic metrics and manual analysis. Their study utilized the ASAP dataset~\cite{yuan2022can}, which categorizes review sentences into predefined aspects such as summary, motivation, originality, soundness, substance, replicability, meaningful comparison, and clarity, and based on these aspects, matches AI and human reviews together. However, apart from the AI-Human comparison, they are overlooking other critical dimensions where AI reviews may fall short and require assessment (as described in figure \ref{fig:challenges}).

% this reliance on fixed aspect-based evaluation limits the framework’s ability to capture the diversity and depth of both AI and human reviews, thereby restricting its effectiveness in assessing nuanced differences in review quality.

Based on our analysis of the limitations in current AI-generated reviews and the gaps in existing evaluation metrics, we propose a comprehensive evaluation framework designed to assess the quality of AI-generated research paper reviews. Our framework targets four key dimensions (see Figure~\ref{fig:challenges}): \ding{182} \textit{Comparison with Human Reviews}: Evaluates topic coverage and semantic similarity to measure the alignment between AI-generated and human-written feedback. \ding{183}
\textit{Factual Accuracy}: Detects factual errors, including misinterpretations, incorrect claims, and hallucinated information. \ding{184} \textit{Analytical Depth}: Assesses whether the AI’s critique transcends generic commentary to offer in-depth, meaningful engagement with the research. \ding{185} \textit{Actionable Insights}: Measures the ability of the AI to provide specific, constructive suggestions for improving the paper.\par

Recognizing that major conferences and journals have distinct reviewing priorities, a one-size-fits-all approach to AI-driven reviews is insufficient. Recent studies \cite{bauchner2024use, biswas2024ai} underscore the growing importance of aligning reviews with conference-specific evaluation criteria—especially as many venues now require adherence to detailed reporting guidelines. To address this, we introduce a conference-specific AI reviewer that dynamically adapts its review strategy to meet the unique criteria of each target venue. 

Furthermore, inspired by the self-refinement approach of~\cite{madaan2023selfrefine}, our system incorporates a supervisor model that iteratively critiques and refines the instructional prompts guiding the review process. This self-refinement loop promotes deeper analytical assessments and mitigates the tendency of AI-generated reviews to offer only superficial feedback. In summary, our research aims to:
\begin{enumerate}
    \item Develop a comprehensive evaluation framework for LLM-based reviewing systems across five dimensions: (i) alignment with human reviews, (ii) factual accuracy, (iii) analytical depth, (iv) actionable insights, and (v) adherence to Reviewer guidelines.
    
    \item Create an LLM-based reviewer that dynamically aligns with the evaluation criteria of specific conferences/journals and continuously improves its reviewing strategy through an iterative refinement loop.
\end{enumerate}


% Therefore, based on our analysis of the limitations in AI-generated reviews and the gaps in existing research on evaluation metrics, we propose a comprehensive evaluation framework to evaluate the quality of AI-generated research paper reviews across four key dimensions (refer figure~\ref{fig:challenges}):
% \begin{enumerate}
% \item Comparison with Human Reviews: Analyzes topic coverage and semantic similarity to measure alignment between AI and human feedback.
% \item Factual Accuracy: Assesses the presence of factual errors, such as misinterpretations, incorrect claims, or hallucinated information in the reviews.
% \item Analytical Depth: Measures the depth and rigor of the AI’s critique, focusing on whether it moves beyond generic comments to meaningful engagement with the research.
% \item Actionable Insights: Evaluates the AI’s ability to provide specific, constructive feedback for improving the paper.
% \end{enumerate}

% \includegraphics[width=\textwidth, trim=0cm 2cm 0cm 2cm, clip]{yourfile.pdf}
% For example, in AI and machine learning, NeurIPS emphasizes theoretical advancements, ICLR prioritizes deep learning and representation learning, AAAI values practical AI applications, and ACL focuses on linguistic insights in NLP, while ICML covers a broad spectrum of machine learning methodologies and optimization techniques.

% Each major conference/journal has distinct reviewing priorities aligned with its focus areas, making a one-size-fits-all approach to AI-driven paper reviews inadequate. Existing research \cite{bauchner2024use, biswas2024ai} highlights the importance of this direction, noting that many conferences/journals now require authors to indicate adherence to specific reporting guidelines, and AI has the potential to be more effective than human reviewers in assessing such compliance. To address this, we propose a conference-specific AI reviewer, which dynamically aligns AI-generated reviews with the unique evaluation criteria of individual conferences/journals, an area that, to the best of our knowledge, has not been explored before. Additionally, we introduce a self-refinement loop, inspired by \cite{madaan2023selfrefine}, where a supervisor model iteratively critiques and refines the instructional prompts guiding the review process. This ensures deeper, more analytical assessments rather than the superficial feedback that often characterizes AI-generated reviews. 


% Our system dynamically generates review prompts by converting each specific guideline of a target conference into individual, step-by-step instructional prompts designed to guide the LLM through the paper reviewing process. 
% Hence, to achieve this we introduce a self-refinement framework inspired by \cite{madaan2023selfrefine}. Our approach employs a Supervisor LLM that iteratively critiques the review prompts based on clarity, coherence, and alignment with conference guideline. Empirical results show that this refinement process significantly enhances prompt quality (refer figure 2), leading to more rigorous and context-aware paper evaluations. 


% Ultimately, our approach aims to narrow the gap between automated and human-like review processes, improving the analytical depth of LLM-assisted peer review.



% Another key challenge in using LLMs for paper reviews is their tendency to generate shallow or generic feedback. The quality of AI-generated reviews heavily depends on the instructional prompts guiding the model. 
% Since these prompts serve as structured guidelines for navigating the paper, they must closely reflect the cognitive processes of human reviewers to ensure deeper analysis rather than superficial assessments. To achieve this, we introduce a self-refinement framework inspired by \cite{madaan2023selfrefine}. Our approach employs a Supervisor LLM that critiques the review prompts iteratively based on clarity, coherence, and alignment with conference guideline. Empirical results show that this refinement process significantly enhances prompt quality, leading to more rigorous and context-aware paper evaluations (refer figure 2). Ultimately, our approach aims to narrow the gap between automated and human-like review processes, improving the analytical depth of LLM-assisted peer review.


% Each major conference has distinct reviewing priorities that reflect their unique focus areas. For example, within the AI.ML domain, NeurIPS places significant emphasis on theoretical advancements and novelty, while ICLR centers on innovations in deep learning and representations. In contrast, AAAI values practical applications across diverse AI domains, and ACL focuses on linguistic insights in natural language processing. ICML, with its broad scope, prioritizes novel machine learning methodologies and optimization techniques. These varying priorities make a conference-specific LLM reviewer essential, as it tailors the evaluation process to the specific goals and expectations of each conference, ensuring relevant, context-sensitive feedback. Hence, to address the need for peer reviews tailored to the unique requirements of academic conferences, we propose a conference-specific research paper reviewer system powered by Large Language Models. This approach provides a novel solution to align AI-generated reviews with the evaluation criteria of individual conferences, a direction that to the best of our knowledge, has not been explored in prior research. Our system dynamically generates review prompts by converting each specific guideline of a target conference into individual, step-by-step prompts designed to guide the review generation process. These prompts ensure that the AI-generated reviews address all aspects of the conference's expectations in a structured and systematic manner. 

% A key challenge in using LLMs for research paper reviewing is their tendency to produce superficial and insufficiently analytical reviews. We hypothesize that the instructional prompts guiding the LLM play a crucial role in shaping the quality of the generated reviews. Since these prompts serve as structured guidelines for navigating the paper, they must closely reflect the cognitive processes of human reviewers to ensure deeper analysis rather than superficial assessments. To address this, we propose a self-refinement framework inspired by \cite{madaan2023selfrefine}, which enables LLMs to iteratively improve their outputs through feedback-based revision. Specifically, we introduce an iterative prompt refinement mechanism, where an LLM supervisor evaluates the initial prompts generated from conference guidelines. The supervisor provides constructive feedback based on clarity, coherence, and alignment with the guideline’s intent, which is then used by the original LLM to improve its initial prompts. Empirical observations indicate that this iterative refinement process significantly improves prompt quality. As shown in Figure 2, the supervisory LLM identifies major deficiencies, which are then fixed leading to more structured and insightful instructional guidance. This improvement directly impacts the depth and relevance of the LLM-generated reviews. Our approach aims to bridge the gap between automated and human-like review processes, enhancing the analytical rigor of LLM-based research reviewing.

% Furthermore, to address a major concern of using LLMs for research paper reviewing: LLMs generating superficial and insufficient reviews (as will be evaluated by our 3 out of 4 evaluation metrics: Comparison with human reviews, analytical depth, and actionable insights), we realize that the instructional prompts the Reviewer LLM follows plays a critical role in how the review will turn out to be. This instructional prompt plays a crucial role in helping the reviewer navigate the paper, hence this prompt needs to be as close to a human though process while reviewing a paper to ensure that the reviews come out to be less superficial and more analytical and intelligent. Hence, improving the quality of prompts used for reviewing is very important.

% To enhance the quality of our prompts that are generated using conference guidelines, we propose a self-refinement method, inspired by \cite{madaan2023selfrefine}, which suggests that LLMs are capable of revising their own outputs just like humans, through an iterative feedback and refinement
% loop to improve initial outputs from LLMs. The instructional prompt used to generate research reviews using LLM plays a critical role in how well the LLM operates as a Reviewer, and hence, these instructional prompts that are dynamically generated using the conference guidelines need to be refined iteratively to ensure they guide the "LLM as Reviewer" through the a well-defined instructional prompt. We employ a supervisory LLM that is instructed to give constructive feedback on the quality of the initial prompt generated for each guideline, this feedback is based on clarity and coherence of the prompt, its alignment to the guideline, etc. We noticed significant improvement in the quality of prompt generated through refinement loops, where we saw the supervisor LLM pointing out major issues like shown in figure 2. Through our work, we recommend a pipeline that proposes a solution to convert conference guidelines into individual instructional prompts that are refined through reflection loop to make them more comprehensive and "human-like". We aim to propose a system that generates prompts aligning closely with how the though process of a human reviewer will operate.

% To improve the quality and alignment of these prompts, we introduce a refinement loop using a secondary "Judge LLM." The Judge LLM evaluates each prompt for coherence, relevance, and consistency with the original guideline, suggesting refinements to better capture the intended requirements. By systematically aligning reviews with conference-specific guidelines, this framework provides a practical and customizable tool for researchers seeking precise, high-quality, and compliant reviews.

% \usepackage{enumitem} % Add this in the preamble

% In summary, the objectives of the research are listed below.
% \begin{enumerate}[itemsep=0mm, topsep=0mm, parsep=1mm, partopsep=1mm]
%     \item Provide a comprehensive evaluation framework for evaluating LLM based reviewing systems across 5 key dimensions: (1) Alignment with Human Reviews (2) Factual Correctness (3) Depth of Analysis (4) Actionable Insights and (5) Adherence to reviewer guidelines.
%     \item Develop an LLM based reviewer that dynamically aligns itself with target conference guidelines and improves its reviewing strategy iteratively through a refinement loop.
% \end{enumerate}

\begin{table*}[t]
    \centering
    \tiny
    \renewcommand{\arraystretch}{0.8} % Reduce row height slightly
    \begin{tabular}{l>{\centering\arraybackslash}m{2.5cm} >{\centering\arraybackslash}m{2.5cm}}
        \toprule
        & \shortstack{\textbf{Wider} \\ \textbf{Metric Coverage}} 
        & \shortstack{\textbf{Black Box} \\ \textbf{Remedy}} \\
        \midrule
        \cite{d2024marg} & \xmark & \xmark \\
        \midrule
        \cite{zhou2024llm} & \xmark & \xmark \\
        \midrule
        ReviewEval(ours) & \cmark & \cmark \\
        \bottomrule
    \end{tabular}
    \caption{Comparison of Evaluation Methods. The evaluation metrics by \cite{d2024marg} (MARG) and \cite{zhou2024llm} (Aspect-Based Comparisons) focus on AI-human review similarity but overlook other key deficiencies in AI reviews. Their heavy reliance on LLMs for end-to-end evaluation creates a black-box system with limited transparency. Our framework addresses these gaps with a more comprehensive and interpretable approach.}
    \label{tab:evaluation_comparison}
\end{table*}
%\vspace{-2mm}

% In summary, the objectives of the research are listed below.
% \vspace{-5mm}
% \setlist{nolistsep}
% \begin{enumerate}[noitemsep]
% \item Develop an LLM-based system capable of providing detailed, constructive feedback on academic manuscripts.\vspace{-1mm}
% \item Align reviews with specific conference guidelines to ensure relevance.\vspace{-2mm}
% \item Develop an extensive evaluation framework to evaluate several aspects of these AI-generated reviews
% \end{enumerate}

% Furthermore, we identify the requirement of researchers that are targeting a specific research conference, therefore, we propose a system an LLM-based conference specific reviewer. To the best of our knowledge, no other research has effectively explored aligning AI reviews to specific conferences.

% We use the conference specific reviewer guidelines and convert them into individual prompts to generate the reviews. These prompts are instructured to be in format of step by step instructions for the reviewer to follow. These prompts then go through a refinement loop where a Judge LLM checks them coherence, relevance, and miscelaneous improvements to align the prompt/instruction better with the conference guideline it is representing. The final conference specific reviews are then outputted in latex format in a structured format.




% Recent advancements in techniques like meta-prompting and multi-agent systems provide potential solutions to these challenges. Meta-prompting allows LLMs to break down complex tasks into smaller subtasks, handled by expert instances with tailored instructions. This approach has the potential to significantly enhance the quality of AI-based reviews by improving specificity and reducing generic feedback. Similarly, utilizing multi-agent systems for review generation \cite{d2024marg} helps distribute review tasks across specialized LLM instances, enabling detailed and nuanced feedback through simulated internal discussions.

% This paper builds upon these developments, proposing a novel system that integrates iterative refinements, prompt engineering, and section-specific evaluations to create a scalable and effective LLM-based peer review assistant.



% Furthermore, based on our preliminary experiments using these commercial models, we further observed some more shortcomings, like the AI reviews lacking actionable insights as well as AI reviews having limited analytical depth with only generic comments, these all arising because llms have the ability to generate vague reviews. Hence, using the existing literature \cite{zhou2024llm} and our preliminary experiments, we propose an evaluation framework with the following (i) evaluating factual innacuracies (ii) Quantitative analysis and comparison of AI reviews and human reviews (such as topic coverage and semantic similarities) (iii) presence of actionable insights (iv) analysing the depth of analysis in AI reviews.



% To further explore the challenges faced by LLMs as reviewers, we conducted preliminary experiments (refer to Figure 1) using commercial models like GPT-4 and Gemini Pro, with basic prompting strategies (details in Appendix). The results revealed several significant shortcomings that hinder their practical application, including factual inaccuracies (hallucinations), inconsistent review quality relative to human evaluations, limited analytical depth, and a lack of actionable insights, thereby rendering these models ineffective for delivering meaningful feedback.


% \cite{zhou2024llm} conducts a comprehensive analysis of the use of commercial models like gpt-3.5 and gpt-4 as research paper reviewer, and found that its shortcomings are possibility of mistakes during reviewing (caused by either model hallucinations or inability to comprehensively understand the material) and that these models fail to give critical feedback like human reviewers.
% To get a better and in-depth grasp of the reviewer challenges, we also conduct som preliminary experiments (add figure 1) with commercial LLMs such as GPT-4 and Gemini Pro, using basic prompting strategies (see Appendix). It revealed significant shortcomings that currently limit their practical application. These include the generation of factual inaccuracies (hallucinations), inconsistent review quality compared to human evaluations, insufficient depth of analysis, and a lack of actionable insights, rendering such reviews ineffective for providing meaningful feedback.


% Talk about existing review systems

% More recently, their application in academic reviewing has garnered attention, with efforts to align automated reviews with conference-specific guidelines to ensure constructive and high-quality feedback. By reducing the workload on human reviewers, such systems aim to maintain or even improve the quality and consistency of evaluations across submissions.

% Despite their promise, the deployment of LLMs in peer review also introduces challenges. Large language models are prone to hallucinations, generating plausible-sounding but inaccurate information. They also demonstrate the power to persuade humans, which can amplify the risks of inaccuracies in reviews. Studies have shown that controlling the quality and appropriateness of LLM-augmented reviewing is a complex and ongoing challenge.



% The academic community has already begun experimenting with AI-assisted reviews. For instance, 15.8\% of reviews for ICLR 2024 were reportedly written with AI assistance \cite{latona2024ai}. While these developments demonstrate the viability of LLMs in academic reviewing, they also highlight the need for further research to address limitations such as factual accuracy, depth of analysis, and alignment with conference guidelines.

% AI-based reviews present unique advantages for the academic community. They can provide early feedback to authors, allowing them to refine their work before submission. Additionally, these systems can help conferences maintain timely and high-quality reviews despite increasing submission volumes. Lastly, they offer an opportunity for improved quality control of reviews, addressing inconsistencies and biases that may arise in human-only review processes.

% Recent advancements in techniques like meta-prompting and multi-agent systems provide potential solutions to these challenges. Meta-prompting allows LLMs to break down complex tasks into smaller subtasks, handled by expert instances with tailored instructions. This approach has the potential to significantly enhance the quality of AI-based reviews by improving specificity and reducing generic feedback. Similarly, utilizing multi-agent systems for review generation \cite{d2024marg} helps distribute review tasks across specialized LLM instances, enabling detailed and nuanced feedback through simulated internal discussions.


% This paper builds upon these developments, proposing a novel system that integrates iterative refinements, prompt engineering, and section-specific evaluations to create a scalable and effective LLM-based peer review assistant.

% The exponential growth in academic research has led to an increased demand for efficient and high-quality peer review processes. Large Language Models (LLMs) have demonstrated potential in automating various natural language processing tasks, including summarization, translation, and question-answering. This project explores their capability to automate the critical task of reviewing academic papers. By aligning reviews with conference-specific guidelines and generating detailed feedback, this LLM-based research assistant aims to reduce human workload while maintaining the quality and consistency of manuscript evaluations.

% The academic community acknowledges the acute need for
% having foundation models assist reviewing of papers at scale
% (Liu and Shah 2023; Robertson 2023; Petrescu and Krishen
% 2022; Schulz et al. 2022; Checco et al. 2021; Bao, Hong,
% and Li 2021; Vesper 2018; Latona et al. 2024; Kuznetsov
% et al. 2024), along with the risks involved (Kaddour et al.
% 2023; Spitale, Biller-Andorno, and Germani 2023; Zou et al.
% 2023). Previous work addresses the limitations of LLM’s
% ability to perform reviewing (Liu and Shah 2023) and their
% capabilities to review academic papers (Liang et al. 2023).
% Large language models demonstrate surprising creative capabilities in text (Koivisto and Grassini 2023), though they
% may hallucinate (Zhang et al. 2023), and demonstrate the
% power to persuade humans even when inaccurate (Spitale,
% Biller-Andorno, and Germani 2023). This makes controlling
% the quality and appropriateness of LLM-augmented reviewing highly challenging. At least 15.8\% of reviews for ICLR
% 2024 were written with AI assistance (Latona et al. 2024).
% Meta-prompting (Suzgun and Kalai 2024) uses multiple
% LLM instances for managing and integrating multiple independent LLM queries. 

% Utilizing meta-prompting, the LLM
% breaks down complex tasks into smaller subtasks handled
% by expert instances with tailored instructions, significantly
% enhancing performance across various tasks. This approach
% outperforms conventional prompting methods across multiple tasks, enhancing LLM functionality without requiring
% task-specific instructions. Multi-agent review generation for
% scientific papers (D’Arcy et al. 2024) improves LLM reviewing by using multiple instances of LLMs, providing
% more specific and helpful feedback by distributing the text
% across specialized agents that simulate an internal discussion. This reduces generic feedback and increases the generation of good comments. Recent work formulates the peerreview process as a multi-turn dialogue between the different
% roles of authors, reviewers, and decision-makers (Tan et al.
% 2024), and finds that both reviews (Latona et al. 2024) and
% meta-reviews written by LLMs (Santu et al. 2024) are preferred by humans over human reviews and meta-reviews.

% Why do the Artificial Intelligence, Machine Learning, and
% Computer Vision communities need AI-based reviews of papers? (i) AI-based reviews provide early feedback to authors
% for their work in progress, allowing authors to learn and improve their work; (ii) AI-based reviews would help conferences maintain high-quality and timely reviews for the increasing number of papers in these fields, (iii) For quality control of reviews
%\vspace{-3mm}
\section{Related Work}
\textbf{AI based scientific discovery.} The pursuit of automating scientific discovery has been a longstanding goal in artificial intelligence (AI) research. Early efforts in the 1970s introduced expert systems such as DENDRAL~\cite{buchanan1981dendral} and the automated mathematician~\cite{lenat1977automated}, which focused on constrained problem spaces like organic chemistry and theorem proving. More recent advancements in AI-driven research have leveraged LLMs and machine learning techniques to extend beyond structured domains. Notable contributions include AutoML approaches that optimize hyperparameters and architectures~\cite{hutter2019automated, he2021automl} and AI-driven discovery in materials science and synthetic biology~\cite{merchant2023ai, hayes2024simulating}. However, these methods remain largely dependent on human-defined search spaces and predefined evaluation metrics, limiting their potential for open-ended discovery. Recent works~\cite{lu2024ai} aim to automate the entire research cycle, encompassing ideation, experimentation, manuscript generation, and peer review, thus pushing the boundaries of AI-driven scientific inquiry.\par

\textbf{AI based peer-review.} Recent advancements in AI-driven peer review have explored structured, multi-agent approaches for automated evaluation.~\cite{lu2024ai} employ LLMs to autonomously conduct the research pipeline, including peer review. It follows a structured three-stage review process: paper understanding, criterion-based evaluation (aligned with NeurIPS and ICLR guidelines), and final synthesis: assigning scores to key aspects like novelty, clarity, and significance. Evaluation shows its reviews closely match human meta-reviews (2.3/5 vs 2.5/5). MARG~\cite{d2024marg} introduces a multi-agent framework where worker agents review sections, expert agents assess specific aspects, and a leader agent synthesizes feedback. Using BERTScore~\cite{bert-score} and GPT-4-based evaluation, MARG-S improves feedback quality, reducing generic comments (60\% to 29\%) and increasing helpful feedback per paper (1.7 to 3.7). A user study found 47\% of MARG-S feedback helpful, outperforming baseline models (21\%). These studies highlight the potential of AI to enhance peer review through structured automation and multi-agent collaboration.\par

% \textbf{AI based peer-review.} Recent advancements in AI-driven peer review have explored structured, multi-agent approaches for automated evaluation.~\cite{lu2024ai} employ LLMs to autonomously conduct the research pipeline, including peer review. It follows a structured three-stage review process: paper understanding, criterion-based evaluation (aligned with NeurIPS and ICLR guidelines), and final synthesis: assigning scores to key aspects like novelty, clarity, and significance. Evaluation shows its reviews closely match human meta-reviews. MARG~\cite{d2024marg} introduces a multi-agent framework where worker agents review sections, expert agents assess specific aspects, and a leader agent synthesizes feedback. Using BERTScore~\cite{bert-score} and GPT-4-based evaluation, MARG-S improves feedback quality, reducing generic comments (60\% to 29\%) and increasing helpful feedback per paper (1.7 to 3.7). A user study found 47\% of MARG-S feedback helpful, outperforming baseline models (21\%). These studies highlight the potential of AI to enhance peer review through structured automation and multi-agent collaboration.\par


% \subsection{The AI Scientist (by Sakana AI)}
% \vspace{-1mm}
% \subsection{The AI Scientist (by Sakana AI)}
% \cite{lu2024ai} introduce The AI Scientist, an autonomous framework leveraging LLMs to conduct the full research pipeline, from ideation to peer review. The system generates research ideas using chain-of-thought reasoning and self-reflection, refines them via a Semantic Scholar API-assisted literature search, and executes experiments using Aider, an LLM-based coding assistant that iteratively debugs and refines implementations. To automate peer review, the system follows a structured process: (1) Paper Understanding, where the LLM summarizes key contributions, methodology, and results; (2) Criterion-Based Evaluation, assessing novelty, clarity, correctness, significance, and presentation using NeurIPS and ICLR guidelines; and (3) Final Review Synthesis, where individual evaluations are compiled into structured feedback with strengths, weaknesses, and improvement suggestions. The review system assigns numeric scores to each criterion, mimicking human assessment, and aggregates them into an overall Meta-Reviewer Score. AI-generated reviews were compared against NeurIPS 2021 Meta-Reviewer Scores, where human reviews averaged 2.5/5, while the AI system achieved 2.3/5, demonstrating near-human accuracy.

% \subsection{MARG: Multi-Agent Review Generation for Scientific Papers}
% \cite{d2024marg} propose MARG (Multi-Agent Review Generation) that employs multiple worker agents to review different sections, while expert agents specialize in key aspects like experiments, clarity, and impact. A leader agent coordinates the process, ensuring a structured and coherent review. The system uses hardcoded review prompts inspired by ICLR guidelines, which remain static across all review tasks. For evaluation, the authors compare MARG-generated reviews against NeurIPS human reviews using automated alignment metrics and a user study. BERTScore is used to measure semantic similarity between AI-generated and human-written reviews, capturing lexical and contextual overlap. Additionally, GPT-4 is used as a preference model, scoring reviews based on relevance, coherence, and completeness, simulating human judgment. The authors also conduct a user study with NLP and HCI researchers, who assess feedback helpfulness. MARG-S (a specialized agent variant) outperforms single-agent and naive multi-agent baselines, reducing generic feedback from 60\% to 29\% and increasing helpful comments per paper from 1.7 to 3.7. The user study reveals that 47\% of MARG-S feedback was rated as helpful, compared to 21\% from naive multi-agent models.



% introduces a comprehensive framework that automates the entire research pipeline, from idea generation to manuscript writing, using LLMs. A notable component is its foundation model-based automated reviewer, which evaluates generated manuscripts based on NeurIPS guidelines, providing scores and feedback akin to human reviewers. The framework achieves 65\% accuracy for final paper score prediction and demonstrates cost efficiency, but it is limited by its inability to adapt to specific conference guidelines. The paper also doesn't analyze and assess their AI generated reviews using meaningful metrics.
% \cite{d2024marg} presents a multi-agent framework (MARG) that generates peer-review feedback by distributing the task among multiple LLM agents. The system uses hardcoded review prompts inspired by ICLR guidelines, which remain static across all review tasks. MARG significantly reduces generic comments (from 60\% to 29\%) and generates more helpful feedback compared to baseline methods, as shown through a user study. The paper also proposes an automated review evaluation framework to check for approximate matches between AI-generated and human-written review comments using GPT. While GPT-4 is employed iteratively to extract approximate matches and mitigate inconsistencies, we argue that completely relying on GPT for end-to-end evaluation introduces unreliability and opacity. To address this, our evaluation framework emphasizes transparency, striving to develop interpretable metrics for comparing AI and human reviews, moving beyond black-box methodologies.
% \subsection{Is LLM a reliable reviewer? A Comprehensive Evaluation of LLM on Automatic Paper Reviewing Tasks}
\textbf{Evaluation framework for AI based research paper reviews.} There has been limited research on developing evaluation frameworks for evaluating the quality of LLM generated paper reviews.~\cite{zhou2024llm} evaluated GPT models for research paper reviewing across 3 tasks: aspect score prediction, review generation, and review-revision MCQ answering. Using the PeerRead dataset, they assessed LLM's ability to predict review scores based on clarity, originality, and soundness (aspect score prediction). For review generation, they employed the ASAP dataset \cite{yuan2022can} and tested zero-shot, few-shot, and aspect-guided structured prompting, where reviews were generated with explicit tagging for summary, originality, and clarity. The study developed evaluation framework comprising of aspect coverage, ROUGE (lexical overlap), BERTScore (semantic similarity), and BLANC (informativeness), alongside manual analysis. Aspect coverage measured how well the generated reviews addressed key aspects (originality, soundness, substance, replicability, etc.) when compared to the distribution in reference expert reviews. Results showed LLMs overemphasized positive feedback, lacked critical depth, and neglected substance and clarity, despite high lexical similarity to human reviews. \cite{d2024marg} introduced an automated evaluation framework for AI-generated reviews, quantifying similarity to human reviews via recall, precision, and Jaccard index. Their multi-stage GPT-4 \cite{achiam2023gpt} based alignment process first matches semantically equivalent comments using five randomized passes (many-many matching), retaining pairs appearing in at least two runs. Each pair is re-evaluated based on relatedness and relative specificity. Recall measures the fraction of real-reviewer comments with at least one AI match, precision quantifies AI comments aligned with human reviews, and Jaccard index evaluates the intersection-over-union of aligned comments.

% \cite{zhou2024llm} conducted a systematic evaluation of GPT-3.5 and GPT-4 for research paper reviewing task, focusing on three key tasks: aspect score prediction, review generation, and review-revision multiple-choice question answering (RR-MCQ). For aspect score prediction, the study used the PeerRead dataset (ICLR-2017 subset) to assess LLM's ability to infer review scores based on predefined criteria such as clarity, originality, and soundness. For review generation, \cite{zhou2024llm} used the ASAP dataset \cite{yuan2022can} (ICLR-2020 subset), which consists of peer reviews labeled by aspects such as originality, soundness, and clarity. Various prompting strategies—zero-shot, few-shot, and aspect-guided structured generation—were tested to improve review quality. Aspect-guided structured generation organizes reviews by characteristics like summary, originality, and clarity, using models to generate sentences tagged with corresponding aspects. The generated reviews were evaluated using automated metrics and human annotations. Among automated metrics, the study used aspect coverage to measure how well the generated reviews addressed key aspects of the paper, comparing the distribution of positive and negative comments with reference reviews. The results showed that LLMs overemphasized positive feedback, often failing to provide critical assessments. Certain aspects, such as substance and clarity, were frequently neglected, while models disproportionately favored broad statements over precise critiques. Other automated metrics included ROUGE (lexical overlap), BERTScore (semantic similarity), and BLANC (informativeness via a blank-filling task). While these metrics indicated high similarity between LLM-generated and human-written reviews, they failed to capture deficiencies in critical feedback and technical depth. \cite{d2024marg} proposed an automated evaluation framework with the following metrics used to quantify the AI generated reviews using the corresponding human (expert) review comments: Recall, Precision, and Jaccard Index. The paper employs a multi-stage GPT-4-based process to measure the overlap between generated and real reviews by identifying semantically equivalent comments. In the Many-Many Matching Stage, all comments from both sets are input into GPT-4, which performs five passes with randomized comment orders to mitigate inconsistencies. A comment pair is retained if it appears in at least two of five runs. In the Pairwise Stage, each identified pair is re-evaluated, where GPT-4 assigns two scores: Relatedness ("none," "weak," "medium," or "high") and Relative Specificity ("less," "same," or "more"). A match is confirmed if relatedness is at least "medium" and specificity is "same" or "more." This structured alignment enables the calculation of recall, precision, and Jaccard index for assessing review similarity. Recall is the fraction of real-reviewer comments that are aligned to any generated comment (total number of human reviews that have atleast 1 match with AI reviews divided by total number of human reviews). Precision is the fraction of generated comments that are aligned to any real-reviewer comment (total number of AI reviews that have atleast 1 match with human reviews divided by total AI reviews). Jaccard Index is calculated by dividing the intersection of generated comments and real-reviewer comments by the size of the union of the generated comments and real-reviewer comments less the intersection. The Jaccard index measures the similarity between the set of generated comments and the set of real-reviewer comments



% \cite{d2024marg} evaluates the MARG-generated reviews against NeurIPS human reviews using automated alignment metrics and a user study. BERTScore is used to measure semantic similarity between AI-generated and human-written reviews, capturing lexical and contextual overlap. Additionally, GPT-4 is used as a preference model, scoring reviews based on relevance, coherence, and completeness, simulating human judgment. They perform automated evaluation using GPT-4 to extract approximate matches between AI reviews and expert reviews, they run GPT inference iteratively to extract approximate matches.The authors also conduct a user study with NLP and HCI researchers, who assess feedback helpfulness.


% conducts a comprehensive evaluation of large language models (LLMs), specifically GPT-3.5 and GPT-4, in scientific paper reviewing, focusing on their ability to generate review texts. The study examines aspect coverage and similarity to reference reviews through both automatic metrics and manual analysis, utilizing the ASAP dataset \cite{yuan2022can}, which annotates review sentences with labels such as summary, motivation, originality, soundness, substance, replicability, meaningful comparison, and clarity. However, relying on pre-defined aspect coverage undermines the diversity and depth of AI and human reviews, therefore we recognized the need for a more nuanced and dynamic topic coverage metric.
% Papers are split into chunks, processed by worker agents, and analyzed by expert agents specializing in clarity, experiments, or impact, with a leader agent coordinating communication and combining feedback. The refinement stage focuses on improving the generated reviews for clarity, specificity, and validity. 


% \subsection{Existing work}


% % Prior work on LLM academic capabilities suggests that LLMs are now ready for specific reviewing tasks and appear to be more effective for some academic domains and less effective for others (Checco et al.
% % 2021; Schulz et al. 2022; Liu and Shah 2023; Lu et al. 2024)

% % \subsection{Current Applications of LLMs}

% % LLMs are widely used for tasks like language translation, summarization, and question-answering. Their potential for critical assessment of new research remains largely untapped.

% \subsection{Existing Automated Review Systems}

% Sakana AI Scientist: A system performing end-to-end research activities (literature review, hypothesis formulation, experimentation, manuscript writing).

% OpenReviewer: Utilizes GPT-4 for academic paper reviews.

% \subsection{Gaps in Current Systems}

% Existing solutions either aim to generalize across the entire research workflow or lack the depth needed for nuanced peer reviews.



% Note use of \abovespace and \belowspace to get reasonable spacing
% above and below tabular lines. Below is an example table.

% \begin{table}[t]
% \caption{Classification accuracies for naive Bayes and flexible
% Bayes on various data sets.}
% \label{sample-table}
% \vskip 0.15in
% \begin{center}
% \begin{small}
% \begin{sc}
% \begin{tabular}{lcccr}
% \toprule
% Data set & Naive & Flexible & Better? \\
% \midrule
% Breast    & 95.9$\pm$ 0.2& 96.7$\pm$ 0.2& $\surd$ \\
% Cleveland & 83.3$\pm$ 0.6& 80.0$\pm$ 0.6& $\times$\\
% Glass2    & 61.9$\pm$ 1.4& 83.8$\pm$ 0.7& $\surd$ \\
% Credit    & 74.8$\pm$ 0.5& 78.3$\pm$ 0.6&         \\
% Horse     & 73.3$\pm$ 0.9& 69.7$\pm$ 1.0& $\times$\\
% Meta      & 67.1$\pm$ 0.6& 76.5$\pm$ 0.5& $\surd$ \\
% Pima      & 75.1$\pm$ 0.6& 73.9$\pm$ 0.5&         \\
% Vehicle   & 44.9$\pm$ 0.6& 61.5$\pm$ 0.4& $\surd$ \\
% \bottomrule
% \end{tabular}
% \end{sc}
% \end{small}
% \end{center}
% \vskip -0.1in
% \end{table}

%\vspace{-3mm}

%Our experiments utilized 16 papers and their corresponding expert reviews (scraped from OpenReview.net), alongside AI-generated reviews. 

\section{ReviewEval}
We describe the development of our evaluation framework and the LLM-based research paper reviewer. Each review is evaluated on several key parameters to assess the overall quality of the generated feedback. To ensure consistency and reliability, all evaluations for a given metric were performed by LLMs of the same specification and version, thereby maintaining inter-rater reliability and ensuring robust, unbiased comparisons. Our experiments utilized 16 papers and their corresponding expert reviews (scraped from OpenReview.net), alongside AI-generated reviews.\par

%We describe the development of our evaluation framework and the development of our LLM based research paper reviewer. We tested 16 papers and reviews from experts (scraped from openreview.net) alongside AI-generated reviews across several parameters. First, we measured the \textit{alignment between AI-generated and expert reviews} using the cosine similarity of word embeddings. Next, we assessed \textit{topic coverage} by evaluating how comprehensively AI-generated reviews addressed the breadth of topics present in expert reviews using large language models. We also evaluated \textit{factual correctness} through a multi-agent fact-checking framework to verify the accuracy of the claims made. Additionally, we examined \textit{constructiveness} by determining whether the reviews provided actionable insights and recommendations, and finally, we measured the \textit{depth of analysis} by assessing the critical engagement of the reviews with the manuscript's content. To ensure consistency and reliability, all evaluations for a given metric were performed by LLMs of the same specification and version. This approach maintained inter-rater reliability across all samples, ensuring that comparisons and assessments were robust and unbiased.\par

% We tested the 16 papers and reviews from experts (scraped from openreview.net) and AI on the following parameters:
% \item \textbf{Comparing against Expert Reviews} Measured alignment between AI-generated and expert reviews using cosine similarity of word embeddings.
% \item \textbf{Topic Coverage:} Assessed how comprehensively AI-generated reviews covered the breadth of topics in expert reviews using LLMs.
% \item \textbf{Factual Correctness:} Evaluated the accuracy of claims in the reviews through a multi-agent fact-checking framework.
% \item \textbf{Constructiveness:} Determined whether reviews provided actionable insights and recommendations.
% \item \textbf{Depth of Analysis:} Assessed the critical engagement of reviews with the manuscript’s content.

% Adherence to Guidelines: Checked compliance with specific conference review criteria.

%We compared the reviews generated by SakanaAI scientist's review bot with those from our own reviewer bot on the selected dataset.

\subsection{Comparison with Expert Reviews}
We compare the reviews generated by the LLM based reviewer with expert reviews from OpenReview.net. Our primary goal is to gauge how well the AI system replicate or complement expert-level critique. The evaluation is conducted along the following dimensions:

\textbf{Semantic similarity.} 
To assess the alignment between AI-generated and expert reviews, we embed each review \( R \) into a vector space using the OpenAI 
embedding model. The semantic similarity between an AI-generated review \( R_{\text{AI}} \) and an expert review \( R_{\text{Expt}} \) is measured using cosine similarity:
\begin{equation}
S_{\text{sem}}(R_{\text{AI}}, R_{\text{Expt}}) = \frac{e(R_{\text{AI}}) \cdot e(R_{\text{Expt}})}{\|e(R_{\text{AI}})\| \, \|e(R_{\text{Expt}})\|}
\end{equation}

where \( e(R) \) denotes the embedding of review \( R \). A higher cosine similarity indicates a stronger alignment between the AI-generated and expert reviews.\par

% To assess the semantic similarity between AI-generated reviews and expert reviews, we measure the cosine similarity between the word embeddings which are generated by OpenAI Embedding Model. Each review $R$ was embedded as a vector, capturing semantic nuances of the text. We then calculated cosine similarity scores between the expert review embedding and each of the AI-generated review embeddings. It helps us measures the alignment between the AI and expert reviews in the embedding space, providing a numerical representation of their contextual similarity. Higher cosine similarity scores indicate a stronger alignment between the AI-generated content and expert perspectives.

% \begin{equation}
% S_{\text{semantic}} = \frac {e_{R_{AI}}.e_{R_{expert}}} {||e_{R_{AI}}|| ||e_{R_{expert}}||}
% \end{equation}

\textbf{Topic modeling.} We evaluate topic coverage to determine how comprehensively AI-generated reviews address the breadth of topics present in expert reviews. Our approach comprises three steps: \ding{182} \textit{Topic extraction:} Each review \( R \) (either AI-generated or expert) is decomposed into a set of topics: $T_R = \{ t_1, t_2, \dots, t_n \},$ where each topic \( t_i \) is represented by a sentence that captures its core content and context. \ding{183} \textit{Topic similarity:} Let $T_{\text{AI}} = \{ t_1, t_2, \dots, t_m \}$ and  $T_{\text{Expt}} = \{ t'_1, t'_2, \dots, t'_n \}$
denote the topics extracted from the AI and expert reviews, respectively. We define a topic similarity function \( \text{TS}(t_i, t'_j) \) that an LLM assigns on a discrete scale:
\begin{equation}
\begin{split}
\text{TS}(t_i, t'_j) = 3 \cdot \mathbb{I}\{ t_i \sim_{\text{strong}} t'_j \} \\
+ 2 \cdot \mathbb{I}\{ t_i \sim_{\text{moderate}} t'_j \}\\ 
+ 1 \cdot \mathbb{I}\{ t_i \sim_{\text{weak}} t'_j \},
\end{split}
\end{equation}
where $\mathbb{I}$ is the indicator function, $t_i \sim_{\text{strong}} t'_j$, $t_i \sim_{\text{moderate}} t'_j$ , $t_i \sim_{\text{weak}} t'_j$ denote substantial, moderate, and minimal overlap in concepts, respectively. All the conditions are mutually exclusive. We set a similarity threshold \( \tau = 2 \) so that topics with \(\text{TS}(t_i, t'_j) \geq \tau \) are considered aligned. \ding{184} \textit{Coverage ratio:} For each AI-generated review, we construct a topic similarity matrix \( S \) where each element $S[i, j] = \text{TS}(t_i, t'_j)$ represents the similarity between topic \( t_i \) from \( T_{\text{AI}} \) and topic \( t'_j \) from \( T_{\text{Expt}} \). The topic coverage ratio is defined as:
\begin{equation}
S_{\text{coverage}} = \frac{1}{n} \sum_{j=1}^{n} \mathbb{I}\left(\max_{i=1,\dots,m} S[i, j] \geq \tau\right),
\end{equation}
where \( \mathbb{I}(\cdot) \) is the indicator function, and \( n = |T_{\text{Expt}}| \) is the total number of topics extracted from the expert review.

% \underline{\textit{Topic extraction using LLM.}} We employed an LLM for extracting core topics. Each review $R$ (either AI-generated or expert-generated) was decomposed into a set of topics: $T_{R} = \{t_{1}, t_{2}, ..., t_{n} \} $ where each topic $t_i$ is represented by a sentence that clearly captures the essence, key details and provides sufficient context to understand the significance of the topic.\par
% To enhance the representation of topic embeddings, individual word embeddings were weighted based on their relevance to the topic. This weighting ensures that dominant keywords exert greater influence on the final topic representation. Let $w_{ij}$ denote the weight of word $j$ in topic $t_i$, and $e_{w_{ij}}$ be the embedding of the word $j$. The topic embedding $e_{t_i}$ was calculated as: \\
% \begin{equation}
%  e_{t_{i}} = \frac {\sum_{j}w_{ij}e_{w_{ij}}} {\sum_{j}w_{ij}} 
% \end{equation}
% \underline{\textit{Similarity for topic alignment.}} To measure the alignment between AI-generated reviews $R_{\text{AI}}$ and expert reviews $R_{\text{Expert}}$, we leverage advanced contextual understanding of Large Language Models to capture underlying themes in each topic. Let $T_{\text{AI}} = \{ t_1, t_2, \dots, t_m \}$ and $T_{\text{Expert}} = \{t'_1, t'2, \dots, t'n\}$ denote the sets of topics extracted from $R_{\text{AI}}$ and $R_{\text{Expert}}$, respectively. The  similarity score $\text{sim}(t_i, t'_j)$ between two topics $t_i$ and $t'_j$ is given as an output by an LLM on a scale from 0 to 3 as given by the following equation.

% \[ sim(t_{i}, t'_{j}) = \begin{cases} 3, & \text{if substantial overlap in concepts} \\ 2, & \text{if some overlap, but different details} \\ 1, & \text{if little overlap in ideas} \\ 0, & \text{if No meaningful connection.} \end{cases} \]
% A similarity threshold $\tau = 2$ was used to determine topic alignment. Topics with $\text{sim}(t_i, t'_j) \geq \tau$ were deemed sufficiently aligned.

% Uncomment for algorithm
% \begin{algorithm}[t]
% \caption{AI vs. Expert Review Evaluation}
% \label{alg:ai-expert-evaluation}

% {\raggedright
% \begin{algorithmic}[1]
% \Require AI-generated review $R_{AI}$, Expert review $R_{Expert}$, Embedding model $E$, Large Language Model $\mathcal{M}$, Similarity threshold $\tau$
% \Ensure Semantic similarity score $S_{\text{semantic similarity}}$, Topic coverage ratio $S_{\text{coverage ratio}}$

% \Statex
% \Function{EvaluateReviews}{$R_{AI}, R_{Expert}, E, \mathcal{M}, \tau$}

%     \State $e_{R_{AI}} \gets E(R_{AI})$ \Comment{Compute embedding of AI review}
%     \State $e_{R_{Expert}} \gets E(R_{Expert})$ \Comment{Compute embedding of expert review}
%     \State $S_{\text{semantic similarity}} \gets \frac{e_{R_{AI}} \cdot e_{R_{Expert}}}{\|e_{R_{AI}}\| \|e_{R_{Expert}}\|}$ \Comment{Compute cosine similarity}
    
%     \State $T_{AI} \gets \mathcal{M}(\text{``Extract topics from'' } R_{AI})$ 
%     \State $T_{Expert} \gets \mathcal{M}(\text{``Extract topics from'' } R_{Expert})$
    
%     \State $S \in \mathbb{R}^{|T_{AI}| \times |T_{Expert}|}, \quad S[i, j] = 0, \forall i, j$ \Comment{Initialize zero matrix}
%     \For{$i \gets 1$ \textbf{to} $|T_{AI}|$}
%         \For{$j \gets 1$ \textbf{to} $|T_{Expert}|$}
%             \State $S[i,j] \gets \mathcal{M}(\text{``Compare topics'' }, T_{AI}[i], T_{Expert}[j])$ 
%         \EndFor
%     \EndFor
    
%     \Comment{$S \in \{0,1,2,3\}$}
%     \State $count \gets 0$
%     \For{$j \gets 1$ \textbf{to} $|T_{Expert}|$}
%         \If{$\max_{i}(S[i, j]) \geq \tau$}
%             \State $count \gets count + 1$
%         \EndIf
%     \EndFor
%     \State $S_{\text{coverage ratio}} \gets \frac{count}{|T_{Expert}|}$
    
%     \State \Return $S_{\text{semantic similarity}}, S_{\text{coverage ratio}}$
% \EndFunction
% \end{algorithmic}}
% \end{algorithm}

% \underline{\textit{Topic coverage analysis.}} For each AI-generated review, a topic similarity matrix $S$ was constructed, where $S[i, j] = \text{sim}(t_i, t'j)$ represents the similarity between topic $t_i$ from $T_{\text{AI}}$ and topic $t'j$ from $T_{\text{Expert}}$. Using $S$, the \textit{topic coverage ratio}, i.e. the proportion of expert topics  $T_{Expert}$ covered by $T_{AI}$ is calculated as:
% % \[
% %        \text{Coverage Ratio} = \frac{\sum_{j=1}^{n} \mathbb{1}\left(\max_{i=1, \dots, m} S[i, j] \geq \tau\right)}{n}
% % \]
% \begin{equation}
%     S_{\text{coverage ratio}} = \frac{\sum_{j=1}^{n} 1 (\text{max}_{i=1,2,...m} S[i,j] \geq \tau)}{n}
% \end{equation}
% Here, 1(.) is an indicator function that evaluates to 1 if the condition is true, and n = $|T_{Expert}|$ is the number of expert topics. The threshold \( \tau \) (i.e., \( \tau = 2 \)) determines the minimum similarity required for a topic to be considered covered.

% \item Unique Topics in AI-Generated Reviews: Topics in \( T_{\text{AI}} \) that do not have a sufficiently similar counterpart in \( T_{\text{Expert}} \) are defined as unique topics:
% \begin{equation}
% T_{\text{Unique}} = \{t_i \in T_{\text{AI}} : \max_{j=1, \dots, n} S[i, j] < \tau\}
% \end{equation}

% \item Missed Expert Topics: Expert topics not represented in \( T_{\text{AI}} \) are identified as:

% \begin{equation}
% T_{\text{Missed}} = \{t'_j \in T_{\text{Expert}} : \max_{i=1, \dots, m} S[i, j] < \tau\}
% \end{equation}

\subsection{Evaluating Factual Correctness of Reviews}
To address the hallucinations and factual inaccuracies of LLM generated reviews, we propose an automated pipeline that validates the factual correctness of LLM-generated reviews by \textit{simulating the conference rebuttal process}. Our pipeline emulates the traditional rebuttal workflow, where authors clarify or counter reviewer claims using evidence from their work. By automating both the question generation and rebuttal phases, our system produces a robust factual correctness evaluation. The pipeline consists of the following steps:\par
\textit{\textbf{Step 1:} Transforming reviews into structured questions.} 
Each LLM-generated review \(R\) is transformed into a structured question \(Q\) that encapsulates the central claim or critique. For example, consider the following review from the PeerRead dataset for the paper \textit{"Augmenting Negative Representations for Continual Self-Supervised Learning"} \cite{chaaugmenting}:\par
\underline{Review (\(R\)):} ``\texttt{Augmentation represents a crucial area of exploration in self-supervised learning. Given that the authors classify their method as a form of augmentation, it becomes essential to engage in comparisons and discussions with existing augmentation methods.}''

This review is converted into the corresponding question:

\underline{Generated question (\(Q\)):} ``\texttt{Has the paper engaged in comparisons and discussions with existing augmentation methods, given that the authors classify their method as a form of augmentation?}''

\textit{\textbf{Step 2:} Decomposing questions into sub-questions.} The question \(Q\) is then decomposed into a set of sub-questions $\{q_1, q_2, \ldots, q_n\}$ using a dedicated query decomposition engine. This decomposition enables a fine-grained analysis by isolating distinct components of the original question.

\textit{\textbf{Step 3:} Retrieval-augmented generation (RAG) for evidence synthesis.} For each sub-question \(q_i\), we employ a Retrieval-Augmented Generation (RAG) framework to gather and synthesize relevant evidence: \circled{a} \textit{Section retrieval:} For each paper \(P\), we retrieve pertinent text segments \(S\) (approximately 400 tokens) via semantic search. \circled{b} \textit{Parent section extraction:} Using LangChain's Parent Document Retriever, we extract the parent sections \(S_p\) (approximately 4000 tokens) corresponding to each \(S\). Documents are pre-chunked hierarchically, ensuring that each \(S\) is mapped to its contextually relevant \(S_p\). \circled{c} \textit{Answer generation:} With the context provided by \(S_p\), the LLM generates an answer \(A_i\) for each sub-question \(q_i\). These individual answers are then aggregated into a unified, structured response \(A_Q\) addressing the original question \(Q\).\par

\textit{\textbf{Step 4:} Automated rebuttal generation.} The comprehensive answer \(A_Q\) is used to generate an automated rebuttal \(R_b\) for the original review \(R\). This rebuttal is designed to provide evidence-based clarification or counterarguments to the claims in \(R\).

\textit{\textbf{Step 5:} Factual correctness evaluation.} An evaluation agent then assesses the factual correctness of \(R\) by comparing it against the generated rebuttal \(R_b\). The review is deemed: \circled{a} \(\textbf{Valid}\ (\mathcal{V} = \text{True})\) if \(R_b\) substantiates the claims made in \(R\).
\circled{b} \(\textbf{Invalid}\ (\mathcal{V} = \text{False})\) if \(R_b\) reveals factual discrepancies or unsupported claims in \(R\).\par

% LLMs are increasingly being employed in academic peer review processes. However, concerns around hallucinations or factual inaccuracies in LLM-generated reviews necessitate robust validation mechanisms. To address this, we propose a pipeline that automates the process of validating the factual correctness of LLM-generated reviews by simulating the conference rebuttal process. This system emulates the traditional rebuttal workflow where authors clarify or counter reviewer claims by providing evidence or context from their work. The pipeline ensures rigor by automating both the question generation and rebuttal phases, ultimately producing a factual correctness evaluation. Below we explain the proposed pipeline:

% \textbf{Conversion of reviews into questions.} Each LLM-generated review, denoted as $R$ is converted into a structured question $Q$. For example, consider a review from the PeerRead dataset for the paper titled "Augmenting Negative Representations for Continual Self-Supervised Learning" \cite{chaaugmenting}: \\
% Review ($R$): ``Augmentation represents a crucial area of exploration in self-supervised learning. Given that the authors classify their method as a form of augmentation, it becomes essential to engage in comparisons and discussions with existing augmentation methods.''\par

% Generated Question ($Q$):
% ``Has the paper engaged in comparisons and discussions with existing augmentation methods, given that the authors classify their method as a form of augmentation?''\par


% Uncomment for algorithm
% \begin{algorithm}[t]
% \caption{Factual Correctness Evaluation of LLM-Generated Reviews}
% \label{alg:factual-correctness-evaluation}

% {\raggedright
% \begin{algorithmic}[1]
% \Require LLM-generated review $R$, Paper $P$, LLM model $\mathcal{M}$, SubQuestionQueryEngine $\mathcal{Q}$, Semantic Search Engine $\mathcal{S}$, Parent Document Retriever $\mathcal{D}$
% \Ensure Validity score $V$

% \Statex
% \Function{EvaluateReviewCorrectness}{$R, P, \mathcal{M}, \mathcal{Q}, \mathcal{S}, \mathcal{D}$}
%     \Comment{Step 1: Convert Review into Questions}
%     \State $Q \gets \mathcal{M}($``Convert review $R$ into a structured question''$)$

%     \Comment{Step 2: Decompose into Sub-Questions}
%     \State $\{q_1, q_2, ..., q_n\} \gets \mathcal{Q}(Q)$
    
%     \Comment{Step 3: Retrieve Contextual Information}
%     \For{$q_i \in \{q_1, q_2, ..., q_n\}$}
%         \State $S \gets \mathcal{S}(P, q_i, 400)$ \Comment{Retrieve relevant 400-token sections}
%         \State $S_p \gets \mathcal{D}(P, S, 4000)$ \Comment{Retrieve parent 4000-token section}
%         \State $A_i \gets \mathcal{M}($``Answer $q_i$ given context $S_p$''$)$
%     \EndFor
    
%     \Comment{Step 4: Generate Rebuttal}
%     \State $A_Q \gets \text{Concatenate}(A_1, A_2, ..., A_n)$
%     \State $R_b \gets \mathcal{M}($``Generate rebuttal from $A_Q$''$)$
    
%     \Comment{Step 5: Evaluate Review Correctness}
%     \If{$R_b$ agrees with claims in $R$}
%         \State $V \gets \text{True}$ \Comment{Review is valid}
%     \Else
%         \State $V \gets \text{False}$ \Comment{Review is invalid}
%     \EndIf
    
%     \State \Return $V$
% \EndFunction
% \end{algorithmic}}
% \end{algorithm}


% \textbf{Decomposition of questions into sub-questions.} Each question Q is decomposed into a set of sub-questions 
% $\{q_{1}, q_{2}, ....., q_{n}\}$
% using subquestion-based query engine. This decomposition isolates individual components of the original question, facilitating a granular evaluation.\par

% \textbf{Retrieval-Augmented Generation (RAG) for contextual information.} To answer each sub-question $q_{i}$, a RAG pipeline is employed as follows: \ding{182} For each paper $P$, relevant small sections $S$ of 400 tokens are retrieved using semantic search. \ding{183} Parent sections $S_{p}$ of size 4000 tokens that contain $S$ are then extracted using LangChain's Parent Document Retriever. Documents are pre-chunked into hierarchical levels, where each $S$ belongs to a predefined $S_{p}$. Retrieval first selects $S$ based on semantic similarity, and then the corresponding $S_{p}$ is retrieved using this parent-child mapping. \ding{184} Given $S_{p}$, the LLM generates answers $A_{i}$ for each $q_{i}$. These answers are then combined into a single, structured paragraph $A_{Q}$, forming a comprehensive response to $Q$.\par

% \begin{enumerate}
%     \item For each paper $P$, relevant small sections $S$ of 400 tokens are retrieved using semantic search.\vspace{-2mm}
%     \item Parent sections $S_{p}$ of size 4000 tokens that contain $S$ are then extracted using LangChain's Parent Document Retriever to provide the whole context.\vspace{-2mm}
%     \item Given $S_{p}$, the LLM generates answers $A_{i}$ for each $q_{i}$. These answers are then combined into a single, structured paragraph $A_{Q}$, forming a comprehensive response to $Q$.\vspace{-2mm}
% \end{enumerate}
% \textbf{Rebuttal generation.} The combined answer $A_{Q}$ serves as the automated rebuttal $R_{b}$ for the review $R$. This rebuttal provides evidence-based clarification or counters the claims made in $R$.

% \textbf{Evaluating review correctness} An evaluation agent is tasked with determining the validity $V$ of the review $R$ based on the rebuttal $R_{b}$. The review is marked as:

% \begin{enumerate}
%     \item $Valid(V=True$) if $R_{b}$ agrees with the claims made in $R$.
%     \item $Invalid(V=False)$ if $R_{b}$ identifies factual inaccuracies in $R$.
% \end{enumerate}

\subsection{Constructiveness}
We assess review constructiveness by quantifying the presence and quality of actionable insights in AI-generated reviews relative to expert feedback. Our framework begins by extracting key actionable components from each review using an LLM with few-shot examples. Specifically, we identify the following actionable insights: (i) \emph{criticism points} (\(C\)), which capture highlighted flaws or shortcomings in the paper’s content, clarity, novelty, and execution; (ii) \emph{methodological feedback} (\(M\)), which encompasses detailed analysis of experimental design, techniques, and suggestions for methodological improvements; and (iii) \emph{suggestions for improvement} (\(I\)), which consist of broader recommendations for enhancement such as additional experiments, alternative methodologies, or improved clarity.\par

Once these components are extracted, each insight is evaluated along three dimensions: specificity, feasibility, and implementation details. The specificity score \(\sigma\) is defined as 1 if the insight is specific and includes explicit examples, and 0 otherwise; the feasibility score \(\phi\) is set to 1 when the recommendation is practical within the paper's context, and 0 otherwise; and the implementation details score \(\zeta\) is 1 if actionable steps or detailed methodologies are provided, and 0 otherwise. The overall actionability score for an individual insight is then computed as $S_{\text{act},i} = \sigma_i + \phi_i + \zeta_i,$ with an insight considered actionable if \(S_{\text{act},i} > 0\). Finally, we quantify the overall constructiveness of a review by calculating the percentage of actionable insights:
\begin{equation}
 S_{\text{act}} = \frac{1}{N} \sum_{i=1}^{N} \mathbb{I}\big(S_{\text{act},i} > 0\big) \times 100,   
\end{equation}
where \(N\) is the total number of extracted insights and \(\mathbb{I}(\cdot)\) denotes the indicator function. This metric provides a quantitative measure of how effectively a review offers concrete guidance for improving the work.

% To evaluate whether the review carried out includes specific suggestions that are clear and actionable in guiding the authors to improve the paper, we measured the presence of actionable insights within the feedback provided by AI-generated reviews and compared them with the gold standard of expert reviews. Our framework systematically identifies actionable suggestions and assesses their quality through a multi-step process outlined below.

% \textbf{Extracting actionable insights.} The constructiveness evaluation begins by extracting specific components from each review using an LLM backed by few shot examples. \ding{182} \textit{Criticism points ($C$):} Highlighted flaws, shortcomings and aspects that detract the paper quality. This focuses on paper's content, clarity, novelty, and execution.
% \ding{183} \textit{Methodological feedback ($M$):} Feedback on the methodologies used in the paper, including suggestions for improvement, detailed analysis of the paper's experimental design, techniques, or approach, focusing on strengths, weaknesses, and potential areas for improvement.
% \ding{184} \textit{Suggestions ($I$):} Broader recommendations for enhancing the work which can be suggesting additional experiments, alternative methodologies, or improved clarity.\par

% Uncomment for algorithm
% \begin{algorithm}[t]
% \caption{Constructiveness Evaluation}
% \label{alg:constructiveness-evaluation}

% {\raggedright
% \begin{algorithmic}[1]
% \Require Review text $R$, Large Language Model $\mathcal{M}$
% \Ensure Constructiveness score $S_{\text{actionable}} \in [0,1]$

% \Statex
% \Function{EvaluateConstructiveness}{$R, \mathcal{M}$}
%     \State $C \gets \text{ExtractCriticism}(R, \mathcal{M})$ 
%     \State $M \gets \text{ExtractMethodologicalFeedback}(R, \mathcal{M})$ 
%     \State $I \gets \text{ExtractSuggestions}(R, \mathcal{M})$ 
    
%     \State $Insights \gets C \cup M \cup I$ \Comment{Combine extracted insights}
%     \State $N \gets |Insights|$ \Comment{Total number of extracted insights}
%     \State $S \gets [0]^N$ \Comment{Initialize scores array}
    
%     \For{$j \gets 1$ \textbf{to} $N$}
%         \State $\sigma_j \gets \Call{EvaluateSpecificity}{\mathcal{M}, insights_j}$
%         \State $\phi_j \gets \Call{EvaluateFeasibility}{\mathcal{M}, insights_j}$
%         \State $\iota_j \gets \Call{EvaluateImplementationDetails}{\mathcal{M}, insights_j}$
%         \State $S_{\text{actionable},j} \gets \sigma_j + \phi_j + \iota_j$
%     \EndFor
    
%     \State $S_{\text{actionable}} \gets \frac{\sum_{j=1}^{N} \mathbf{1}(S_{\text{actionable},j} > 0)}{N} \times 100$
    
%     \State \Return $S_{\text{actionable}}$
% \EndFunction

% \Statex
% \Function{EvaluateSpecificity}{$\mathcal{M}, C_j$}
%     \State $score \gets \mathcal{M}(\text{``Is insight $C_j$ specific?``})$
%     \State \Return $score$ \Comment{$score \in \{0,1\}$}
% \EndFunction


% \Function{EvaluateFeasibility}{$\mathcal{M}, C_j$}
%     \State $score \gets \mathcal{M}(\text{``Is insight $C_j$ feasible?``})$
%     \State \Return $score$
%     \Comment{$score \in \{0,1\}$}
% \EndFunction

% \Function{EvaluateImplementationDetails}{$\mathcal{M}, C_j$}
%     \State $score \gets \mathcal{M}(\text{``Does insight $C_j$ have implementation details?
%     ``})$
%     \State \Return $score$
%     \Comment{$score \in \{0,1\}$}
% \EndFunction

% \end{algorithmic}}
% \end{algorithm}


% \textbf{Assessing feasibility and specificity.} Once $C$, $M$ and $I$ are extracted, they are assessed for quality along the following dimensions:

% \textit{Specificity:} This metric evaluates whether the feedback addresses a specific aspect of the paper, identifies precise issues or improvement areas, and avoids vague language by providing explicit examples or references. The specificity score $\sigma$ is defined as:
% \[
% \sigma = 
% \begin{cases} 
% 1, & \text{if specific and includes explicit examples.} \\
% 0, & \text{otherwise.}
% \end{cases}
% \]

% \textit{Feasibility:} This measures the practicality of the suggested actions, considering the context of the paper, available resources, and implementation challenges while proposing manageable solutions. The feasibility score $\phi$ is defined as:
% \[
% \phi = 
% \begin{cases} 
% 1, & \text{if suggests practical and feasible actions.} \\
% 0, & \text{otherwise.}
% \end{cases}
% \]

% \textit{Implementation details:} This assesses whether the feedback provides actionable steps, detailed methodologies, or explicit techniques, including references to prior work, tools, or parameters to support implementation. The implementation details score $\iota$ is defined as:
% \[
% \iota = 
% \begin{cases} 
% 1, & \text{if includes actionable steps or methodologies.} \\
% 0, & \text{otherwise.}
% \end{cases}
% \]
% \par

% \textbf{Calculating the actionability score.} Each actionable insight is assigned a score based on the sum of the binary values from the above nodes. The actionability score $S_{\text{actionable}}$ for an individual insight is defined as:
% \[
% S_{\text{actionable},i} = \sigma_{\text{i}} + \phi_{\text{i}} + \iota_{\text{i}}
% \]
% An insight is deemed actionable if $S_{\text{actionable}} > 0$, as even a minimal score can provide a foundation for authors to improve their work.\par

% \textbf{Percentage of actionable insights.} To gauge the overall constructiveness of a review, we calculated the percentage of actionable insights:
% \[
% \text{Percentage of Actionable Insights} = \frac{\sum_{i=1}^{N} S_{\text{actionable},i}}{N} \times 100
% \]

% \noindent where:
% \begin{itemize}
%     \item $N$ is the total number of insights extracted.
%     \item $S_{\text{actionable},i}$ represents the actionable score for the $i$-th insight.
% \end{itemize}
% \[
% S_{\text{actionable}} = \frac{\sum_{i=1}^{N} \mathbf{1}(S_{\text{actionable},i} > 0)}{N} \times 100
% \]

% \noindent where: $N$ is the total number of insights extracted. $\mathbf{1}(\cdot)$ is the indicator function, which equals 1 if $S_{\text{actionable},i} > 0$, and 0 otherwise. Where $S_{\text{actionable},i}$ represents the actionable score for the $i$-th insight. This metric provides a quantitative measure of how well the review guides authors in improving their work.\par

\subsection{Depth of Analysis}
To assess whether a review provides a comprehensive, critical evaluation rather than a superficial commentary, we measure the depth of analysis in AI-generated reviews. This metric captures how thoroughly a review engages with key aspects of a paper, including comparisons with existing literature, identification of logical gaps, methodological scrutiny, interpretation of results, and evaluation of theoretical contributions.\par

Each review is evaluated by multiple LLMs, which assign scores for each of the five dimensions, \(m_i\) (\(i \in \{1,2,3,4,5\}\)), with scores \(S_i \in [0,1]\). Scores in the continuous range allow us to capture nuances in performance. We define the metrics as follows:\par

\textit{Comparison with existing literature (\(m_1\)):} Assesses whether the review critically examines the paper's alignment with prior work, acknowledging relevant studies and identifying omissions. The scoring rubric is:
\begin{equation*}
\resizebox{.9\hsize}{!}{
S_1 = \begin{cases} 
3, & \text{if the review provides a thorough, critical comparison,} \\
2, & \text{if the comparison is meaningful yet shallow,} \\
1, & \text{if the comparison is vague or lacks specific references,} \\
0, & \text{if no comparison is provided.}
\end{cases}
}
\end{equation*}

\textit{Logical gaps identified (\(m_2\)):} Evaluates the review's ability to detect unsupported claims, reasoning flaws, and to offer constructive suggestions:
\begin{equation*}
\resizebox{.9\hsize}{!}{
S_2 = \begin{cases} 
3, & \text{if the review identifies comprehensive gaps and provides suggestions,} \\
2, & \text{if it notes some gaps with unclear recommendations,} \\
1, & \text{if the gaps are vaguely mentioned without solutions,} \\
0, & \text{if no gaps are identified.}
\end{cases}
}
\end{equation*}

\textit{Methodological scrutiny (\(m_3\)):} Measures the depth of critique regarding the paper’s methods, including evaluation of strengths, limitations, and improvement suggestions:
\begin{equation*}
\resizebox{.9\hsize}{!}{
S_3 = \begin{cases} 
3, & \text{if the review delivers a thorough critique with actionable suggestions,} \\
2, & \text{if the critique is meaningful but lacks depth,} \\
1, & \text{if the critique is vague and offers little insight,} \\
0, & \text{if no methodological critique is provided.}
\end{cases}
}
\end{equation*}

\textit{Results interpretation (\(m_4\)):} Assesses how well the review interprets the results, addressing biases, alternative explanations, and broader implications:
\begin{equation*}
\resizebox{.9\hsize}{!}{
S_4 = \begin{cases} 
3, & \text{if the interpretation is detailed and insightful,} \\
2, & \text{if it is meaningful yet shallow,} \\
1, & \text{if the discussion is generic or vague,} \\
0, & \text{if no interpretation is offered.}
\end{cases}
}
\end{equation*}

\textit{Theoretical contribution (\(m_5\)):} Evaluates the assessment of the paper’s theoretical contributions, including its novelty and connections to broader frameworks:
\begin{equation*}
\resizebox{.9\hsize}{!}{
S_5 = \begin{cases} 
3, & \text{if the evaluation is comprehensive and insightful,} \\
2, & \text{if the evaluation is meaningful but lacks depth,} \\
1, & \text{if the critique is vague,} \\
0, & \text{if no theoretical assessment is provided.}
\end{cases}
}
\end{equation*}

The overall depth of analysis score for a review is calculated as the average normalized score across all dimensions:
\begin{equation}
S_{\text{depth}} = \frac{\sum_{i=1}^{5} S_i}{15}.
\end{equation}
A higher \(S_{\text{depth}}\) indicates a more comprehensive and critical engagement with the manuscript.

% To assess whether the review provides a comprehensive evaluation that goes beyond surface-level observations, we measured the depth of analysis within AI-generated reviews. This metric evaluates how thoroughly the review engages with various critical aspects of the paper, such as the comparison with existing literature, identification of logical gaps, methodological scrutiny, results interpretation, and theoretical contribution. Our framework evaluates how well the review engages with key components, ensuring it addresses the significance and implications of the paper.
% Uncomment for algorithm
% \begin{algorithm}[t]
% \caption{Depth of Analysis Evaluation}
% \label{alg:depth-of-analysis}

% {\raggedright
% \begin{algorithmic}[1]
% \Require Review text $R$, Large Language Model $\mathcal{M}$
% \Ensure Depth of analysis score $S_{\text{depth}} \in [0,1]$

% \Statex
% \Function{EvaluateDepthOfAnalysis}{$R, \mathcal{M}$}
%     \State $m[1] \gets \text{Existing Literature Comparison}$
%     \State $m[2] \gets \text{Logical Gaps Identification}$
%     \State $m[3] \gets \text{Methodological Scrutiny}$
%     \State $m[4] \gets \text{Results Interpretation}$
%     \State $m[5] \gets \text{Theoretical Contributions}$


%     \State $N \gets |m|$ \Comment{Number of evaluation criteria}
%     \State $S \gets [0]^N$ \Comment{Initialize score array}
    
%     \For{$i \gets 1$ \textbf{to} $N$}

%         \State ${S}[i] \gets \Call{EvaluateCriterion}{\mathcal{M}, m_i}$ \Comment{Score for criterion $m[i]$}
%     \EndFor
    
%     \State $S_{\text{total}} \gets \sum_{i=1}^N {S}[i]$ \Comment{Aggregate scores}
%     \State $S_{\text{depth}} \gets S_{\text{total}} / (3N)$ \Comment{Normalize score to range $[0,1]$}
    
%     \State \Return $S_{\text{depth}}$
% \EndFunction


% \Statex
% \Function{EvaluateCriterion}{$\mathcal{M}, C_i$}

%     \State $score \gets \mathcal{M}(\text{``Rate $R$ from 0 $\rightarrow$ 3 based on $C_i$.``})$ \Comment{LLM-based evaluation of the review for criterion $C_i$}

    
%     \State \Return $score$
% \EndFunction


% \end{algorithmic}}
% \end{algorithm}
% \textbf{Evaluation metrics for depth of analysis.} The review is processed through multiple LLMs, each independently evaluating the following metrics and assigning a score between 0 and 1, (higher is better). Scores in the range \( (0, 1) \) are also considered to account for the gray area. Mathematically, for each metric \( m_i \), the score \( S_i \) is assigned such that \( S_i \in [0, 1] \), with higher values indicating stronger performance.

% \textit{Comparison with existing literature (\(m_1\)):} Assesses whether the review critically evaluates the alignment or divergence of the paper with prior studies, including the acknowledgment of relevant work and identification of omissions or oversights. Rubrics are defined by the equation below:
%  \[ S_1 = \begin{cases} 3, & \text{if thorough, critical comparison} \\ 2, & \text{if meaningful but shallow comparison} \\ 1, & \text{if vague, lacking specific references} \\ 0, & \text{if no comparison with prior work} \end{cases} \]
% where higher values indicate a more thorough and critical comparison.

% \textit{Logical gaps identified (\(m_2\)):} Evaluates the review's ability to identify unsupported claims, assumptions, gaps in reasoning, or lack of evidence, as well as the provision of constructive suggestions to address these issues. The rubric is defined as follows:
%  \[ S_2 = \begin{cases} 3, & \text{thorough gaps + suggestions} \\ 2, & \text{some gaps, unclear suggestions} \\ 1, & \text{vague gaps, no suggestions} \\ 0, & \text{no gaps identified} \end{cases} \]
 
% \textit{Methodological scrutiny (\(m_3\)):} Assesses the critique of the paper's methodologies, including strengths, weaknesses, limitations, and suggestions for improvement. The rubric is defined as follows:
% \[ S_3 = \begin{cases} 3, & \text{thorough critique with suggestions} \\ 2, & \text{meaningful critique, lacks depth} \\ 1, & \text{vague critique, no insights} \\ 0, & \text{no critique provided} \end{cases} \]

% \textit{Results interpretation (\(m_4\)):} Evaluates the alignment of the results' interpretation with the data, addressing biases, alternative explanations, and broader implications. The rubric is defined as follows:  
% \[ S_4 = \begin{cases} 3, & \text{detailed, insightful, comprehensive} \\ 2, & \text{meaningful but lacks depth} \\ 1, & \text{vague, generic discussion} \\ 0, & \text{no interpretation provided} \end{cases} \]

% \textit{Theoretical contribution (\(m_5\)):} Assesses the evaluation of the paper’s theoretical contributions, including novelty, connections to broader frameworks, and limitations. The rubric is defined as follows:
% \[ S_5 = \begin{cases} 3, & \text{comprehensive, insightful evaluation} \\ 2, & \text{meaningful but lacks depth} \\ 1, & \text{vague, no meaningful critique} \\ 0, & \text{no assessment provided} \end{cases} \]

% Each metric score (\(S_i\)) provides a quantitative evaluation of the depth of analysis in the review, where:  
% \[
% i \in \{1, 2, 3, 4, 5\}
% \]  
% \textbf{Calculating the depth of analysis score.} To calculate the depth of analysis score for a review, we use the formula: 
% \[
% S_{\text{depth}} = \frac{\sum_{i=1}^{M} S_i}{3M}
% \]

% \noindent where, \(M = 5\) is the total number of evaluation metrics. \(S_i\) is the score assigned to the \(i\)-th metric, where \(i \in \{1, 2, 3, 4, 5\}\). This metric quantifies the overall depth of analysis by averaging the scores across all metrics, with higher values indicating a more comprehensive and critical engagement with the manuscript's content.

\subsection{Adherence to Reviewer Guidelines}
To assess whether a review complies with established criteria, we evaluate its adherence to guidelines set by the venue. This metric measures how well the reviewer applies key aspects such as originality, methodology, results, clarity, and ethical considerations, thereby ensuring an objective and structured evaluation process.

%To evaluate the extent to which the review adheres to established guidelines, we assess how well the reviewer follows the criteria and expectations set by the journal or conference. This metric measures whether the reviewer consistently applies the prescribed framework, including key aspects such as originality, methodology, results, clarity, and ethical considerations. Our framework ensures that the review process remains structured and aligned with the goals of thorough, unbiased, and fair evaluation. By testing adherence to these guidelines, we ensure that reviews are objective, consistent, and meaningful, contributing to the overall quality of the academic review process.

% Uncomment for algorithm
% \begin{algorithm}[t]
% \caption{Adherence Evaluation to Reviewer Guidelines}
% \label{alg:adherence-evaluation-v2}

% \begin{algorithmic}[1]
% \Require Review guidelines $G$, review text $R$, LLM $\mathcal{M}$
% \Ensure Adherence score $S_{\text{adherence}} \in [0,1]$

% \Statex
% \Function{EvaluateAdherence}{$G, R, \mathcal{M}$}
%     \State $C \gets \Call{ExtractCriteria}{\mathcal{M}, G}$
%     \Comment{Extract evaluation criteria from guidelines}

%     \State $N \gets |C|$ \Comment{Number of evaluation criteria}
    
%     \State $\mathbf{S} \gets [0]^N$ \Comment{Initialize score array for $N$ criteria}
%     % \\
    
%     \For{$i \gets 1$ \textbf{to} $N$}
%         \State $P_i \gets \Call{GeneratePrompt}{\mathcal{M}, C_i}$ \Comment{Create criterion-specific prompt}
%         \State $\mathbf{S}[i] \gets \Call{Evaluate}{\mathcal{M}, P_i, R}$ \Comment{Get score from LLM where $S_i \in \{0,1,2,3\}$}

%     \EndFor
    
%     \State $S_{\text{total}} \gets \sum_{i=1}^N \mathbf{S}[i]$ \Comment{Aggregate scores}
%     \State $S_{\text{adherence}} \gets S_{\text{total}}/(3N)$ \Comment{Normalize to [0,1] scale}
    
%     \State \Return $S_{\text{adherence}}$
% \EndFunction

% \Statex
% \Function{ExtractCriteria}{$\mathcal{M}, G$}
%     \State \Return $\mathcal{M}$(``Extract evaluation criteria from: $G$'') \Comment{LLM-based extraction}
% \EndFunction

% \Statex
% \Function{GeneratePrompt}{$\mathcal{M}$, $C_i$}
%     \State \Return ``Rate 0-3 how the review addresses: $C_i$. Just give number.'' \Comment{Example prompt template}
% \EndFunction

% \end{algorithmic}
% \end{algorithm}

%\textbf{Extracting criteria for adherence evaluation.} The evaluation begins by identifying criteria $C$ from the review guidelines $G$ which the reviewer is expected to follow. These criteria can be divided into two broad categories: \ding{182} \textit{Subjective Criteria:} These involve qualitative aspects, such as clarity and accuracy in summarizing the paper, or the ability to provide constructive feedback. \ding{183} \textit{Objective Criteria:} These involve quantifiable aspects, such as assigning scores or following a specific rating scale within the review structure. An LLM is tasked with extracting these criteria from the provided guidelines. The extracted criteria form the foundation for the adherence evaluation.

Our approach begins by extracting the criteria \(C\) from the guidelines \(G\). These criteria fall into two broad categories: \ding{182} \emph{subjective} criteria, which involve qualitative judgments (e.g., clarity, constructive feedback), and \ding{183} \emph{objective} criteria, which are quantifiable (e.g., following a prescribed rating scale). For each review \(R\), every extracted criterion \(C_i\) is scored on a 0-3 scale using a dedicated LLM with dynamically generated prompts that include few-shot examples for contextual calibration. For subjective criteria, the score is defined as:
\begin{equation*}
\resizebox{.9\hsize}{!}{
S_i = \begin{cases}
3, & \text{if there is strong adherence with detailed, accurate feedback;} \\
2, & \text{if the review shows reasonable alignment with minor deviations;} \\
1, & \text{if the feedback is incomplete or inaccurate;} \\
0, & \text{if there is no alignment.}
\end{cases}
}
\end{equation*}

For objective criteria, the scoring is binary:
\begin{equation*}
\resizebox{.9\hsize}{!}{
S_i = \begin{cases}
3, & \text{if the review adheres to the required scale and structure;} \\
0, & \text{otherwise.}
\end{cases}
}\end{equation*}

The overall adherence score is then computed as:
\begin{equation}
S_{\text{adherence}} = \frac{\sum_{i=1}^{2} S_i}{6},
\end{equation}

This normalized score provides a quantitative measure of how well the review conforms to the prescribed guidelines.

% \textbf{Assigning scores to criteria.} For review $R$ each extracted criterion $C_i$ is assigned a score $S_{i}$ on a scale of 0 to 3 by a dedicated instance of large language model, where higher scores indicate better adherence. Prompts for this scoring task are dynamically generated by a factory function, ensuring alignment with the specific criterion. The prompts are tailored to assess adherence based on predefined rules, distinguishing between subjective and objective criteria:

% \textit{Subjective criteria scoring:}
% \[ 
% S_i = \begin{cases} 
%     3, & \text{strong adherence with detailed} \hidewidth \\
%        & \text{and accurate feedback} \\
%     2, & \text{reasonable alignment with} \hidewidth \\
%        & \text{minor errors or deviations} \\
%     1, & \text{incomplete or inaccurate feedback,} \hidewidth \\
%        & \text{significant deviation} \\
%     0, & \text{No alignment with the criterion}
% \end{cases}
% \]

% \textit{Objective criteria scoring:}
% \[ S_i = \begin{cases} 3, & \text{adheres to the scale and standards} \\ 0, & \text{fails to rate or uses invalid scale} \end{cases} \] 

% \textbf{Dynamic prompt generation.} The prompts for evaluating each criterion are crafted dynamically by a large language model to ensure that the scoring task is well-defined and contextually appropriate. For subjective criteria, the prompts include specific few shot examples of high, medium, low, and no adherence, to help the model better align its scores to the expectations aiming for better and more consistent results. For objective criteria, the prompts emphasize the importance of following the required scale and structure along with few shot examples.

% \textbf{Calculating adherence score.} The scores $S_i$ assigned to each criterion are aggregated to compute both an overall adherence score $S_{\text{adherence}}$. The overall adherence score is defined as:

% \[
% S_{\text{adherence}} = \frac{\sum_{i=1}^N S_i}{3N}
% \]

% where $N$ is the total number of extracted criteria. $S_i$ is the score assigned to the criterion $C_i$. This score provides a quantifiable measure of guideline adherence ($S_{\text{adherence}}$), enabling comprehensive assessment of review quality.

% \subsection{Conference-Specific Prompt Alignment}
% A key aspect of our approach is dynamically aligning the LLM-based review process to the target conference's reviewing guidelines. In this section we describe the process to achieve this.
\subsection{Conference-Specific Review Alignment}
% To align the review process with conference-specific guidelines, we use the URL of the target conference's official reviewing guideline website. Using Extractor API, we extract the textual content from the webpage, ensuring all relevant criteria are captured. The extracted text is then processed using GPT, which refines the content by removing extraneous information while preserving essential reviewing instructions. Each guideline is subsequently transformed into a step-by-step instructional prompt using GPT, explicitly specifying the sections of the paper it applies to. Since certain review criteria may span multiple sections, prompts are dynamically assigned to relevant sections, with some prompts being applicable to more than one. By performing reviewing in a section-wise manner, we optimize processing efficiency, allowing for independent evaluation of each section before aggregating the final review. The details of all prompts used in this part are defined in the appendix.

To tailor the review process to conference-specific guidelines, we first retrieve the relevant textual content from the target conference's official reviewing website using the Extractor API. The extracted text is then processed using GPT \cite{achiam2023gpt} to filter out extraneous details while preserving essential reviewing instructions. Each guideline \(g_i\) is converted into a step-by-step instructional prompt via GPT:
\[
P_i = \text{GeneratePrompt}(g_i),
\]
where \(P_i\) denotes the prompt corresponding to guideline \(g_i\). Since some criteria apply to multiple sections of a research paper, the prompts are dynamically mapped to the relevant sections. This is achieved using a mapping function:
\[
S_j = \mathcal{M}(P_i),
\]
where \(S_j\) is the set of paper sections associated with prompt \(P_i\). Notably, \(\mathcal{M}\) is a one-to-many mapping, i.e., 
\[
\mathcal{M}: P \to \mathcal{P}(S),
\]
with \(\mathcal{P}(S)\) denoting the power set of all sections. By conducting reviews on a section-wise basis, our framework enhances processing efficiency and allows for independent evaluation of each section prior to aggregating the final review. Detailed descriptions of all prompts are provided in the Appendix.

% To align the review process with conference-specific guidelines, we first retrieve the textual content from the target conference's official reviewing guideline website using the Extractor API. The extracted text is processed using GPT \cite{achiam2023gpt} to remove extraneous information while preserving essential reviewing instructions.  
% Each guideline \( g_i \) is transformed into a step-by-step instructional prompt using GPT \cite{achiam2023gpt}, defined as $P_i = \text{GeneratePrompt}(g_i)$, where \( P_i \) represents the instructional prompt corresponding to guideline \( g_i \).

% Some reviewing criteria may be applicable to multiple sections of a research paper, so the prompts are dynamically assigned to relevant sections using a mapping function defined as $S_j = \mathcal{M}(P_i)$ where \( S_j \) denotes the set of paper sections relevant to prompt \( P_i \). The function \( \mathcal{M} \) extracts relevant sections by querying the prompt $P_i$. Some prompts may apply to multiple sections, making \( \mathcal{M} \) a one-to-many mapping denoted as $\mathcal{M}: P \to \mathcal{P}(S) $, where \( \mathcal{P}(S) \) represents the power set of all possible sections.  
% By performing reviewing in a section-wise manner, we optimize processing efficiency, allowing for independent evaluation of each section before aggregating the final review. The details of all prompts used in this part are defined in the appendix.

% To optimize processing efficiency, the review process is conducted **section-wise**, allowing independent evaluation of each section before aggregating the final review:  

% \begin{equation}
%     R_{\text{final}} = \bigcup_{j} R_j
% \end{equation}  

% where \( R_{\text{final}} \) represents the aggregated review, and \( R_j \) corresponds to the review generated for section \( S_j \).  

% All prompts used in this methodology are detailed in the **Appendix**.



% We first obtain the URL of the
% conference’s official review guidelines. Using a third-party tool, Extractor
% API, we extract the textual content of the guidelines
% webpage, ensuring that all reviewing criteria are cap-
% tured. We guide GPT to structure the relevant reviewer guideline in an easy to read format by removing uncecessary data that might be on the website (refer to prompt in appendix section). Finally, each of these guidelines are then converted into step-by-step instructional prompt using GPT (Refer appendix section). Each reviewing prompt generated from the conference guideline might need to access different sections, and since we have the step by step instructional prompt, it must have which section needs to be checked in it. So, we assign each prompt to individual sections of the paper, with the same prompt maybe going to more than one section. This way, we don't have the process the whole paper altoghether, and we can do reviewing section-wise at first, combining our results later on.

% \begin{enumerate}
%     \item \textbf{Guideline Extraction:} We first obtain the URL of the conference’s official review guidelines.
%     \item \textbf{Content Retrieval:} Using a third-party tool, Extractor API, we extract the textual content of the guidelines webpage, ensuring that all reviewing criteria are captured.
%     \item \textbf{Guideline Structuring:} The extracted content is processed to generate detailed, section-wise reviewing instructions. Each section (e.g., Introduction, Related Work, Methodology, Results) is assigned specific criteria based on conference priorities such as impact assessment, logical soundness, clarity, completeness, and adherence to reproducibility standards.
%     \item \textbf{Instruction Prompt Generation:} The section-wise criteria are transformed into structured step-by-step instructional prompts for the LLM to follow while reviewing the paper.
% \end{enumerate}

% By ensuring that the reviewing process aligns closely with the expectations of the target conference, we enable more accurate and relevant assessments tailored to the research community's standards.

% \subsection{Research Paper Parsing and Preprocessing}
% Before inference, we preprocess the research paper to ensure all relevant content, including text and illustrations, is structured appropriately. \vspace{-3mm}

% \begin{enumerate}
%     \item \textbf{PDF Parsing:} We utilize \texttt{pymupdf} from the \texttt{langchain\_community} library to extract the paper’s text content as a single structured object.\vspace{-1mm}
%     \item \textbf{Mathematical Equations and Figures:} For images, equations, and illustrations, we employ \texttt{pymupdf}'s multimodal LLM-based image parser. This converts images into markdown format with detailed descriptions, preserving contextual information critical for scientific analysis.
% \end{enumerate}


% \subsection{Review Prompt Iterative Refinement}
% To enhance the effectiveness and completeness of the review prompts, we employ an iterative refinement process involving a Supervisor LLM. Given an initial set of prompts generated from the reviewing guidelines, each prompt undergoes structured evaluation and improvement. The Supervisor LLM takes as input the initial prompt along with the corresponding reviewing guideline, which serves as the problem statement. It then critiques the prompt based on multiple key aspects, including clarity, logical consistency, alignment with the guideline, and comprehensiveness. Clarity ensures that the instructions are precise and unambiguous, logical consistency verifies that the reasoning within the prompt is coherent, and alignment ensures that the prompt accurately reflects the intended purpose of the guideline. Additionally, comprehensiveness is assessed by identifying whether the prompt covers all relevant aspects a human reviewer would consider, including those that an AI model might otherwise overlook. The critique provided by the Supervisor LLM is then fed back to the original LLM instance that generated the prompts, which revises the prompt based on the feedback. This iterative refinement loop continues for a fixed number of iterations, set to three in our implementation, ensuring progressive improvement at each step. After completing the refinement process, we obtain a final set of structured, high-quality review prompts that are well-aligned with the reviewing guidelines and optimized for the evaluation process. 

\subsection{Review Prompt Iterative Refinement}
To enhance the quality and completeness of our review prompts, we employ an iterative refinement process using a Supervisor LLM. Starting with an initial set of prompts generated from the reviewing guidelines, the Supervisor LLM evaluates each prompt in conjunction with its corresponding guideline, serving as the problem statement, and provides targeted feedback. This feedback addresses key aspects such as clarity (ensuring precise and unambiguous instructions), logical consistency (ensuring coherent reasoning), alignment with the guideline, and comprehensiveness (ensuring all relevant aspects are covered). The feedback is then used to revise the prompt, and this iterative loop is repeated for a fixed number of iterations (three in our implementation). The result is a final set of structured, high-quality review prompts that are well-aligned with the reviewing guidelines and optimized for the evaluation process.

% The algorithm is explained in algorithm~\ref{alg:refinement}.

% \begin{algorithm}[t]
% \caption{Iterative Refinement of Review Prompts}
% \label{alg:refinement}
% \DontPrintSemicolon

% {\raggedright
% \begin{algorithmic}[1]
% \KwIn{
%     $P = \{P_1, P_2, ..., P_n\}$ \tcp{Initial set of review prompts} \\
%     $G = \{G_1, G_2, ..., G_n\}$ \tcp{Corresponding review guidelines} \\
%     $N$ \tcp{Number of refinement iterations (default: 3)}
% }
% \KwOut{$P^* = \{P_1^*, P_2^*, ..., P_n^*\}$ \tcp{Final refined prompts}}

% \ForEach{$P_i \in P$}{
%     \For{$t = 1$ \KwTo $N$}{
%         \tcp{Step 1: Supervisor LLM evaluates the prompt}
%         $C_i^t \gets \text{SupervisorLLM}(P_i, G_i)$ \tcp{Generate critique}

%         \tcp{Step 2: Original LLM revises the prompt}
%         $P_i \gets \text{PromptGeneratorLLM}(P_i, C_i^t)$ \tcp{Update prompt}
%     }
% }
% \Return $P^* = P$
% \end{algorithmic}}
% \end{algorithm}

% Uncomment for algorithm
% \begin{algorithm}
% \caption{Iterative Refinement of Review Prompts}
% \label{alg:refinement}
% \begin{algorithmic}[1]
% \Require $P = \{P_1, P_2, ..., P_n\}$ \Comment{Initial review prompts}
% \Require $G = \{G_1, G_2, ..., G_n\}$ \Comment{Corresponding guidelines}
% \Require $N$ \Comment{Number of refinement iterations (default: 3)}
% \Ensure $P^* = \{P_1^*, P_2^*, ..., P_n^*\}$ \Comment{Final refined prompts}

% \For {$i = 1$ to $n$} \Comment{Iterate over each prompt}
%     \For {$t = 1$ to $N$} \Comment{Perform iterative refinement}
%         \State $C_i^t \gets \text{SupervisorLLM}(P_i, G_i)$ \Comment{Generate critique}
%         \State $P_i \gets \text{PromptGeneratorLLM}(P_i, C_i^t)$ \Comment{Update prompt}
%     \EndFor
% \EndFor
% \State \Return $P^* = P$
% \end{algorithmic}
% \end{algorithm}




% Once we have a set of initial review prompts, we employ an iterative refinement algorithm to make these instructional prompts more effective and well-rounded.
% Each prompt goes through a refinement
% Supervisor LLM gets input: Initial generated Prompt + the corresponding guideline (the problem statement)
% The supervisor LLM is instructed to critique the prompt regarding multiple key dimensions: clarity, logical consistencies, its alignment to the given problem statement, and any aspect it overlooked for reviewing considering how a human would approach reviewing in a well-rounded manner (covering aspects the AI model would have missed). 
% This critique is then fed back to the main LLM that created the prompts from the guideline. This LLM uses the critique to improve its initial prompt and make it makes changes to it using it. This process repeats for a number of iteration (hardcoded to 3 in our research). After a series of refinements, we have a final set of prompts.




% \subsubsection{Initial Review Generation}
% Using the extracted reviewing criteria, the system generates an initial review for each section by prompting the LLM with the corresponding structured instruction set.

% \subsubsection{Reflection-Based Prompt Refinement}
% To enhance the quality and coherence of the reviews, we employ a structured \textbf{Reflection Loop}, iterating over the review process to refine the responses.

% \begin{enumerate}
%     \item The LLM receives the structured reviewing prompt and generates an initial review for a given section.
%     \item The \textbf{Reflection Prompt Generator LLM} is then used to assess the quality of the generated response.
%     \item The input to this model includes:
%           \begin{itemize}
%               \item The problem statement (i.e., system instruction + review prompt).
%               \item The generated response (initial review).
%           \end{itemize}
%     \item The Reflection Prompt Generator LLM analyzes the response, identifying weaknesses such as lack of depth, missing key aspects, or inconsistencies.
%     \item It then generates an improved prompt with suggestions for refining the review.
%     \item The original reviewing LLM receives this refined prompt along with its previous response and generates an updated review.
%     \item This iterative loop runs for a predefined number of iterations (\texttt{num\_iterations}, currently set to 3), ensuring progressive refinement of the review.
% \end{enumerate}

% At the end of this process, we obtain a structured Python dictionary containing:
% \begin{center}
% \texttt{\{section : guidelines for reviewing\}}
% \end{center}
% which serves as the final reviewing instruction set.

% \subsection{Final Review Generation}
% Once the structured reviewing guidelines have been refined, the research paper undergoes a second inference pass using the final reviewing criteria. The same Reflection Loop is applied to enhance the quality of the generated reviews.

% \subsection{Supervised Validation Attempt}
% We initially explored using a separate "supervisor" LLM instance to validate the reviews. This model was designed to output a binary decision (\texttt{YES/NO}) based on predefined quality criteria. However, this approach often resulted in infinite loops of rejection (\texttt{NO}), indicating the challenges of fully automated review validation. Consequently, this strategy was discarded, reinforcing the need for structured refinement rather than rigid rejection-based feedback.

% \subsection{LaTeX Review Generation}
% The final structured review is compiled into a LaTeX format for direct integration into academic workflows. A separate LLM instance converts the review dictionary into a well-formatted LaTeX document, ensuring that the final output is readily usable by authors and conference organizers.

\section{Experiments \& Results}
\textbf{Dataset.} We selected a sample of 16 papers from the NeurIPS 2024 submissions available on OpenReview.net. These papers were chosen randomly, with a balanced split: 4 papers from each category of accept decision ("oral", "poster" and "spotlight") and 4 that were rejected. To ensure diversity and avoid potential biases in topic/keyword coverage, we specifically included papers that contained unique keywords within each acceptance category. By filtering for papers with distinctive keywords, we aim to include a broad range of research topics, mitigating the risk of over-representation of any single theme or field in our dataset.\\
The papers in dataset are relatively recent, offering a challenging test set that many flagship large language models (LLMs) are unlikely to have encountered given their knowledge cutoffs. This approach enables us to fairly assess each model’s capability to interpret and analyze unseen data effectively.

\subsection{Constructiveness}

\begin{table}[ht]
\centering
\begin{tabular}{|l|c|c|c|c|c|c|c|c|c|}
\hline
\textbf{Model} & \textbf{$\mu + \sigma$}\\
\hline
Ours-GPT-4o & $0.7483 \pm 0.1275$ \\
Ours-GPT-4o-Mini & $0.7385 \pm 0.0952$ \\
Ours-3.5-Sonnet & $0.7513 \pm 0.1669$ \\
Ours-3.5-Haiku & $0.4943 \pm 0.2070$ \\
\hline
Sakana-GPT-4o & $0.7909 \pm 0.1193$ \\
Sakana-GPT-4o-Mini & $0.7916 \pm 0.1246$ \\
Sakana-3.5-Sonnet & $0.6956 \pm 0.1475$ \\
Sakana-3.5-Haiku & $0.4634 \pm 0.1380$ \\
\hline
MARG-GPT-4o & $0.7253 \pm 0.1638$ \\
MARG-GPT-4o-mini & $0.7428 \pm 0.1815$ \\
\hline
Expert & $0.7522 \pm 0.1206$ \\
\hline
\end{tabular}
\caption{Mean scores and standard deviations of different frameworks/models for actionable insights.}
\label{constructiveness}
\end{table}

% \begin{figure}[ht]
% \centering
% \includegraphics[width=\linewidth]{constructiveness.png}
% \caption{Box Plot for presence of actionable insights}
% % \label{fig:example_image} % Assign a label for referencing
% \end{figure}

Table \ref{constructiveness} summarizes the performance of reviews generated using different frameworks with variety of foundation models and expert reviews based on the presence of actionable insights. Expert reviews achieved a mean score of 0.7522, establishing a high-quality benchmark. Notably, Sakana-4o-Mini (0.7916) and Sakana-4o (0.7909) produced feedback comparable to or even exceeding expert performance. Ours-GPT-4o model (0.7483) demonstrated competitive results, closely aligning with expert-level feedback, while Ours-GPT-4o-Mini (0.7385) also performed strongly, indicating the effectiveness of these models in generating actionable insights. In contrast, smaller models like Ours-3.5-Haiku (0.4943) and Sakana-3.5-Haiku (0.4634) lagged, underscoring the importance of model capacity and architecture. Additionally, higher variability in certain models (e.g., Ours-3.5-Sonnet, $\sigma = 0.1669$) indicates inconsistent feedback quality across different papers.

\subsection{Factual Correctness}

\subsection{Adherence to reviewer guidelines}

\begin{table}[ht]
\centering
\begin{tabular}{|l|c|c|c|c|c|c|c|c|c|}
\hline
\textbf{Model} & \textbf{$\mu + \sigma$} \\
\hline
Ours-GPT-4o & $0.6823 \pm 0.0917$ \\
Ours-GPT-4o-Mini & $0.5785 \pm 0.2174$ \\
Ours-3.5-Sonnet & $0.6658 \pm 0.1180$ \\
Ours-3.5-Haiku & $0.4631 \pm 0.1197$ \\
\hline
Sakana-GPT-4o & $0.6327 \pm 0.0808$ \\
Sakana-GPT-4o-Mini & $0.6013 \pm 0.0669$ \\
Sakana-3.5-Sonnet & $0.6400 \pm 0.0297$ \\
Sakana-3.5-Haiku & $0.6313 \pm 0.0398$ \\
\hline
MARG-GPT-4o & $0.3206 \pm 0.0364$ \\
MARG-GPT-4o-mini & $0.3031 \pm 0.0396$ \\
\hline
Expert & $0.5708 \pm 0.1133$ \\
\hline
\end{tabular}
\caption{Mean scores and standard deviations of different frameworks/models for adherence to review guidelines.}
\label{adherence}
\end{table}

% \begin{figure}[ht]
% \centering
% \includegraphics[width=\linewidth]{adherence.png}
% \caption{Box Plot for adherence of reviewer guidelines metric}
% % \label{fig:example_image} % Assign a label for referencing
% \end{figure}

The table \ref{adherence} above presents the performance of various models and expert reviews in adhering to established journal or conference review guidelines. Expert reviews attained a mean score of 0.5708, providing a baseline for high-quality adherence. Among the models, Ours-GPT-4o (0.6823) outperformed the expert benchmark, demonstrating strong adherence to review criteria. Similarly, Ours-3.5-Sonnet (0.6658) and Sakana-3.5-Sonnet (0.6400) showed competitive performance, although it is to be noted that SakanaAi's AI scientist is hardcoded to follow NeurIPS guidlines while our reviewer adjusts itself depending on the given guidelines.
Smaller models have a lower score with higher variability underscoring the significance of model size and architecture in ensuring structured and guideline-compliant academic reviews.

\subsection{Comparing with expert reviews}

\begin{table}[ht]
\centering
\begin{tabular}{|l|c|c|}
\hline
\textbf{Model} & \textbf{$\mu + \sigma$} \\
\hline
Ours-GPT-4o       & $0.6788 \pm 0.2546$ \\
Ours-GPT-4o-Mini  & $0.6924 \pm 0.1650$ \\
Ours-3.5-Sonnet   & $0.6782 \pm 0.1468$ \\
Ours-3.5-Haiku    & $0.8013 \pm 0.1572$ \\
\hline
Sakana-GPT-4o        & $0.6863 \pm 0.1445$ \\
Sakana-GPT-4o-Mini    & $0.7342 \pm 0.1799$ \\
Sakana-3.5-Sonnet & $0.7159 \pm 0.1407$ \\
Sakana-3.5-Haiku  & $0.6641 \pm 0.1467$ \\
\hline
MARG-GPT-4o & $0.6971 \pm 0.1549$ \\
MARG-GPT-4o-mini & $0.5439 \pm 0.1636$ \\
\hline
\end{tabular}
\caption{Mean scores and standard deviations of different frame-
works/models for coverage of expert topics in the review.}
\label{tab:coverage}
\end{table}

\begin{table}[ht]
\centering
\begin{tabular}{|l|c|c|}
\hline
\textbf{Model} & \textbf{$\mu + \sigma$} \\
\hline
Ours-GPT-4o       & $0.8171 \pm 0.0724$ \\
Ours-GPT-4o-Mini  & $0.8170 \pm 0.0346$ \\
Ours-3.5-Sonnet   & $0.8379 \pm 0.0204$ \\
Ours-3.5-Haiku    & $0.8274 \pm 0.0210$ \\
\hline
Sakana-GPT-4o        & $0.8440 \pm 0.0226$ \\
Sakana-GPT-4o-Mini    & $0.8344 \pm 0.0161$ \\
Sakana-3.5-Sonnet & $0.8195 \pm 0.0169$ \\
Sakana-3.5-Haiku  & $0.8130 \pm 0.0177$ \\
\hline
MARG-GPT-4o & $0.8005 \pm 0.0206$ \\
MARG-GPT-4o-mini & $0.7926 \pm 0.0246$ \\
\hline
\end{tabular}
\caption{Mean scores and standard deviations of different frame-
works/models for semantic similarity to the expert reviews}
\label{tab:similarity}
\end{table}

% \begin{figure}[ht]
% \centering
% \includegraphics[width=\linewidth]{topic coverage.png}
% \caption{Box Plot for proportion of expert review topics covered}
% % \label{fig:example_image} % Assign a label for referencing
% \end{figure}

% \begin{figure}[ht]
% \centering
% \includegraphics[width=\linewidth]{semantic similarity.png}
% \caption{Box Plot for semantic similarity with expert reviews}
% % \label{fig:example_image} % Assign a label for referencing
% \end{figure}

The Tables \ref{tab:coverage}, \ref{tab:similarity} show the proportion of topics covered by different frameworks that align with expert reviews in both coverage and semantic similarity. Expert reviews serve as the reference standard.

Ours-GPT-4o (coverage: 0.6788, similarity: 0.8171) and Sakana-4o (coverage: 0.6863, similarity: 0.8440) performed competitively, demonstrating strong alignment with expert feedback. Sakana-4o-Mini (coverage: 0.7342, similarity: 0.8344) even exceeded expert-like coverage in some cases.

Coverage varied across models. Ours-3.5-Haiku (0.8013) and Sakana-3.5-Haiku (0.6641) covered more topics but showed lower similarity, indicating broader yet less expert-aligned feedback. Ours-3.5-Sonnet (0.6782 coverage, 0.8379 similarity) balanced coverage and similarity better but had slightly lower recall.

Standard deviation analysis shows stable similarity scores ($\sigma \sim 0.02$) but greater variation in coverage ($\sigma \leq 0.25$), suggesting models match expert phrasing well but struggle with comprehensive topic selection.

\subsection{Depth of Analysis}
\begin{table}[ht]
\centering
\begin{tabular}{|l|c|c|}
\hline
\textbf{Model} & \textbf{$\mu + \sigma$} \\
\hline
Ours-GPT-4o       & $0.5028 \pm 0.0638$ \\
Ours-GPT-4o-Mini  & $0.5243 \pm 0.0628$ \\
Ours-3.5-Sonnet   & $0.5417 \pm 0.0918$ \\
Ours-3.5-Haiku    & $0.5597 \pm 0.1107$ \\
\hline
Sakana-GPT-4o         & $0.5208 \pm 0.0602$ \\
Sakana-GPT-4o-Mini    & $0.4944 \pm 0.0402$ \\
Sakana-3.5-Sonnet & $0.4875 \pm 0.0737$ \\
Sakana-3.5-Haiku  & $0.4597 \pm 0.0611$ \\
\hline
MARG-GPT-4o & $0.6833 \pm 0.0725$ \\
MARG-GPT-4o-mini & $0.7014 \pm 0.0832$ \\
\hline
Expert            & $0.6264 \pm 0.0722$ \\
\hline
\end{tabular}
\caption{Depth of Analysis scores (mean and standard deviation) for each model across 16 papers.}
\label{tab:depth_analysis}
\end{table}

% \begin{figure}[ht]
% \centering
% \includegraphics[width=\linewidth]{depth.png}
% \caption{Box Plot for depth of analysis done in the research paper review}
% % \label{fig:example_image} % Assign a label for referencing
% \end{figure}

The results in table \ref{tab:depth_analysis} show that expert reviews achieve the highest depth of analysis (0.6264), outperforming all AI models. Among AI-generated reviews, Ours-3.5-Haiku (0.5597) and Ours-3.5-Sonnet (0.5417) provide the most comprehensive evaluations, while Sakana-3.5-Haiku (0.4597) scores the lowest. The Ours-GPT-4o model (0.5028) performs better than most Sakana models but falls short of the top-performing AI models. Notably, Ours-3.5-Haiku has the highest variance (0.1107), indicating inconsistency across papers, whereas Sakana-4o-Mini (0.0402) is more stable but has lower absolute depth.
% Our system consists of four tightly integrated components designed to handle the complex task of research paper reviewing:\\ \textit{Conference Guideline Extraction}, \textit{Customized Paper Parsing}, \textit{Dynamic Prompt Generation and Refinement}, and \textit{Review Generation and Final Compilation}. Below, we describe each component and its significance in ensuring the robustness of the system.

% \subsection*{Step 1: Conference Guideline Extraction}
% Accurate alignment with conference-specific review criteria is critical to ensuring that AI-generated reviews are relevant and valuable. The system takes a conference's URL as input and uses an LLM (e.g., GPT) to extract reviewer guidelines by searching for keywords such as "Reviewer Guidelines," "Review Criteria," and "Instructions for Reviewers."

% \begin{tcolorbox}[colframe=blue!50!black, colback=blue!5, title=Example Guideline Extraction]
% \textbf{Guideline from Conference A:} \textit{Reviews should assess originality, relevance, and technical soundness. Include specific comments on the clarity and potential impact of the work.}
% \end{tcolorbox}

% % \noindent These extracted guidelines are structured into categories such as clarity, originality, and relevance, forming the foundation for dynamic prompt generation (Step 3).

% \subsection*{Step 2: Customized Paper Parsing}
% The system uses a specialized parser based on \texttt{ScienceParser} to extract content from the research paper PDF, ensuring accurate segmentation by sections (e.g., Introduction, Methodology, Results). Key enhancements include:
% \begin{itemize}
%     \item \textbf{Section-Based Parsing:} Content is extracted and organized by section headings for targeted analysis.
%     \item \textbf{Validation by Secondary LLM:} A second LLM reviews the parsed content to ensure accuracy.
% \end{itemize}
% To handle complex content like mathematical formulas, we employ \texttt{Gemini 1.5 Pro}, which converts formulas into LaTeX for consistency and applicability.

% % \begin{tcolorbox}[colframe=red!50!black, colback=red!5, title=Example Output]
% % \textbf{Parsed Content for "Methodology" Section:}
% % \begin{verbatim}
% % \{ "Methodology": "This paper proposes a novel framework using \textbf{X} \
% % and achieves state-of-the-art results. The core algorithm is:\n$F(x) = x^2 + ax + b$.\"\}
% % \end{verbatim}
% % \end{tcolorbox}

% \subsection*{Step 3: Dynamic Prompt Generation and Refinement}
% Based on parsed content and extracted conference guidelines, the system dynamically generates prompts for reviewing each paper section. These prompts include both general review aspects (e.g., clarity, completeness, originality) and conference-specific criteria.

% \noindent The prompt generation process is formalized as:
% \begin{align}
% P_{\text{review}} = P_{\text{generic}} + P_{\text{conference-specific}},
% \end{align}
% where $P_{\text{generic}}$ represents standard review instructions, and $P_{\text{conference-specific}}$ tailors the review to the conference's requirements. 

% These prompts undergo an iterative refinement process using a "judge LLM" to ensure step-by-step clarity and human-like structure, mimicking how expert reviewers analyze papers.

% \begin{tcolorbox}[colframe=green!50!black, colback=green!5, title=Refined Prompt Example (to change)]
% \textit{"For the \textbf{Introduction} section: Assess whether the paper provides sufficient background and motivation for the work. Evaluate clarity, relevance to the conference themes, and the originality of the problem statement."}
% \end{tcolorbox}

% \subsection*{Step 4: Review Generation and Final Compilation}
% Each section is reviewed independently using the refined prompts using separately initialized LLM instances for each section. Reviews are iteratively improved through another round of feedback from the "judge LLM." The final reviews are compiled and formatted into a professional LaTeX document adhering to academic standards.

% \noindent The formatted review includes:
% \begin{itemize}
%     \item Standardized headings for clarity.
%     \item Consistent typography, ensuring readability.
%     \item Inclusion of LaTeX-rendered formulas and equations where applicable.
% \end{itemize}

% \subsection{Overview}

% The system employs a multi-stage process involving:

% Data Ingestion: Extracting textual and structural content from research papers and retrieving conference review guidelines.

% Prompt Generation: Creating tailored prompts for each paper section to guide the review process.

% Reflection Loop: Refining prompts and reviews iteratively to enhance clarity and relevance.

% Section-wise Review Generation: Generating detailed feedback for individual sections using separate LLM instances.

% Presentation: Formatting the final review using LaTeX for professional submission.

% \subsection{Data Ingestion}

% Research Paper Input: Research papers in PDF format are processed to extract textual content and structural details (e.g., headings, figures, references).

% Reviewing Guidelines Retrieval: Conference guidelines are fetched via user-provided URLs, parsed, and structured for integration into the review process.

% \subsection{Prompt Generation}

% An LLM instance (Planner) segments the paper into six key sections: Motivation, Prior Work, Approach, Evidence, Contribution, and Presentation.

% Specific prompts are generated for each section, designed to elicit critical and constructive feedback aligned with the reviewing guidelines.

% \subsection{Reflection Loop}

% A supervisory LLM evaluates and refines the prompts for clarity, alignment, and relevance.

% This iterative refinement ensures high-quality prompts that guide the review generation effectively.

% \subsection{Section-wise Review Generation}

% Each section is reviewed independently by a dedicated LLM instance, ensuring unbiased and focused evaluations.

% The generated reviews undergo iterative refinement to enhance depth and adherence to guidelines.

% \subsection{Presentation}

% Final reviews are compiled and formatted using LaTeX, adhering to academic conventions for professional presentation.

% The formatted review is made available for user download via a web interface built with Streamlit.
\section{Conclusion}
%The experiments where conducted in a controlled environment and we shall test the security against prompt injection attack
% \bibliography{example_paper}
% \bibliographystyle{icml2025}

\bibliography{example_paper}

\appendix
\section{Appendix}
\label{sec:appendix}

\subsection{Limitations}
\subsection{Ethical Considerations}
\subsection{Additional Results}
\end{document}

\subsection{Your software is useful to others only if they know about it}
\label{sec:community}

\paragraph{Background} In the previous recommendations, we discussed various points on how to build high-quality reusable software. However, software quality is meaningless if the community is unaware that the software project exists.
Having external users increases the scientific impact of the software and improves its quality by making it more thoroughly tested with various scenarios and edge cases. Finally, every user of your software is a potential future collaborator.
While \ref{sec:openSource} already discusses the first points as publishing the code open source and registering it in a registry, there are more ways to get attention to your \ac{ERS}.
Therefore, we will provide recommendations on increasing the number of users and overall reach of your \ac{ERS}.
\par 

\paragraph{Recommendations} The natural step to increase the visibility of your software is to publish a paper about it. While we discussed in \ref{sec:documentation} that software papers are not a replacement for documentation, they are great for increasing visibility and motivating your software. Consequently, a software paper should not contain too much technical details from the documentation. Instead, it should explain why the software is important, discuss its high-level capabilities, and convey its scope and vision. We recommend \ac{JOSS}\footref{fn:joss} \cite{jossJournal} as a modern software journal. If you want to discuss domain-specific aspects in more detail, it can also be helpful to publish your software in a conventional domain-specific journal, e.g., as it was done for \textit{pandapower} \cite{pandapower2018} and \textit{renewables.ninja} \cite{pfenninger_long-term_2016}.

\par 
The second essential step is the active engagement with the community.
Especially in the early stage of the software project, it is a good approach to contact domain experts and actively collect early feedback. First, this makes them aware of your software. Second, it prevents the developed tool from missing the needs of the researchers working in the field. Later in the process, you can actively present your software project at a conference, e.g., at the "Open Source Modelling and Simulation of Energy Systems (OSMSES)"\footnote{\url{https://www.osmses2024.org/home}, last access 2025-01-02} conference. 
\par 
If your software project uses GitHub/GitLab, their issue system is a great way to receive feedback of any kind. Feature requests provide direct information about what the users need, and questions about your software indicate where the documentation can be improved. 
In general, the software repository should be structured in a way that facilitates engagement with the community. Create a \texttt{CONTRIBUTING} file to communicate to the community how they can contribute to the software project, including, for example, how to report bugs, style recommendations, or the code of conduct.
Contact information should always be available if external researchers want to contact the maintainers. 

\par 
In summary, when developing reusable software, you should care about users and actively engage with them to improve software quality and maximize scientific impact.  

\recommendation{Grow your community!} 



We now propose a means to represent LLMs via their
\emph{correctness vector} on a small set of labelled validation prompts.
This naturally leads to certain \emph{cluster-based} representations, involving either unsupervised or supervised cluster assignments based on a large set of unlabelled training prompts.



\subsection{The Correctness Vector Representation}
\label{sec:correctness_representation}

To construct our LLM representation $\LLMEmbed$,
it is useful to consider the properties a ``good'' representation ought to satisfy.
One intuitive requirement is that $\LLMEmbed( h )^\top \LLMEmbed( h' )$ should be large for a pair $(h, h')$ of ``similar'' LLMs,
and small for a pair of ``dissimilar'' LLMs.
A reasonable definition of ``similar'' would thus enable the design of $\LLMEmbed$.

We posit that two LLMs are similar if they 
\emph{have comparable performance on a set of representative prompts},
following similar proposals in~\citet{Thrush:2024,ZhuWuWen2024}.
Concretely, suppose 
that
we have access to a small validation set $S_{\val} = \{(\bx^{(i)},\by^{(i)})\}_{i=1}^{N_\val}$ of labelled prompts.
Further, suppose that 
\emph{any new LLM 
$\hNew^{( n )} \in \msetNew$
can be evaluated on these prompts}.
Then, 
one may construct
$$ \LLMEmbed( \hNew^{( n )} ) = \begin{bmatrix} 1( \by^{(i)} = \hNew^{( n )}( \bx^{(i)} ) ) \end{bmatrix}_{i \in [ N_{\val} ]} \in \{ 0, 1 \}^{N_{\val}}, $$ 
denoting the accuracy of an LLM on each prompt in $S_{\val}$.
We term this the \emph{correctness vector representation}.

The choice of prompts in $S_{\val}$ is of clear import.
These prompts could be either hand curated based on domain knowledge,
or simply drawn from a standard benchmark suite.
Further,
since $S_{\val}$ is assumed to be of modest size, 
evaluation of any new LLM's predictions on $S_{\val}$ is not prohibitive;
indeed, if $S_{\val}$ comprises a subset of a standard benchmark suite,
these results may already be available as part of the LLMs' standard evaluation protocol.

While the above provides a reasonable starting point,
it may introduce a risk of overfitting if $N_{\rm val}$ is moderately large (say, $\geq 1000$).
To mitigate this, we consider a variant which relies on \emph{aggregate} performance on \emph{subsets} of $S_{\val}$.


\ifarxiv
\begin{figure}
    \centering
    \includegraphics[width=0.8\textwidth]{figs/illus_cluster_routing.pdf}
    \caption{An illustration of our proposed cluster-based router (see \S\ref{sec:cluster_router}). 
    We first perform $K$-means on an unlabeled training set to find $K$ centroids, which allow us to partition the validation set to $K$ representative clusters. 
    Each test-time LLM can then be represented as a $K$-dimensional feature vector of per-cluster errors. For each test query, we route to the LLM which has the smallest cost-adjusted average error on the cluster the query belongs to.
    }
    \label{fig:illus_cluster}
\end{figure}
\fi


\subsection{Cluster-Based LLM Representation}
\label{sec:cluster_router}



To extend the above, we propose
to represent any new LLM $\hNew^{( n )}$ through its average errors 
$\hat{\LLMEmbed}( \hNew^{( n )} ) \in [0,1]^K$
on $K$ pre-defined \emph{clusters}. 
We then approximate 
\eqref{eq:opt_rule01} via 
$$ \hat{\gamma}( \bx, \hNew^{( n )} ) = \mathbf{z}( \boldsymbol{x} )^\top \hat{\LLMEmbed}( \hNew^{( n )} ), $$
where $\mathbf{z}( \boldsymbol{x} ) \in \{ 0,1 \}^K$ indicates the cluster membership.


One challenge with the above
is that clustering $S_{\val}$ itself is prone to overfitting,
since (by assumption) the set is of modest size.
To overcome this,
we assume we have access to a large
\emph{unlabeled} training set consisting of input prompts $S_\tr = \{\bx^{(i)}\}_{i=1}^{N_\tr}$.
We now
use the training set to  group the prompts into $K$ disjoint clusters, and compute per-cluster errors for a new LM using the validation set $S_{\val}$. %

Concretely, our proposed LLM representation is as follows:
\begin{enumerate}[label=(\roman*),itemsep=0pt,topsep=0pt,leftmargin=16pt]
\item Given a pre-trained query embedder 
$\Phi \colon \XCal \to \Real^D$, 
apply a clustering algorithm to the \emph{training} set embeddings 
$\{\Phi(\bx^{(i)})\}_{i=1}^{N_\tr}$
to construct $K$ non-overlapping clusters.
This yields a cluster assignment map
$\mathbf{z}\colon\mathscr{X}\to\{0,1\}^K$, where $z_k(\bx) = 1$ indicates that $\bx$ belongs to cluster $k$.
\item Assign each prompt in the \emph{validation} sample to a cluster. Let $C_{k}\defeq\{(\bx, \by)\,:\,(\bx, \by) \in S_\val,\, z_{k}(\bx)=1\}$ be the subset of the validation set that belongs to cluster $k$.
\item For each new LLM $\hNew^{( n )} \in \msetNew$, compute a feature representation $\hat{\LLMEmbed}( \hNew^{( n )} ) \in [0,1]^{K}$ 
using its per-cluster error on the  validation set:
\begin{align}
\hat{\LLMEmbedScalar}_{k}( \hNew^{( n )} )
\defeq\frac{1}{|C_{k}|}\sum_{(\bx,\by)\in C_{k}}\1\big[ \by \ne \hNew^{( n )}(\bx) \big].
\label{eq:cluster_accs}
\end{align}
\vspace{-15pt}
\end{enumerate}

We may now approximate the expected loss for $\hNew$ on an input prompt $\bx$ using the average error of the LLM on the cluster the prompt is assigned to, and route via:
\begin{equation}
\begin{aligned}
\hat{r}(\bx, \msetNew) & = \underset{n \in [ \numSetNew ]}{\argmin} \, \big[ \hat{\gamma}_{\cluster}(\bx, \hNew^{( n )} ) + \lambda\cdot c( \hNew^{( n )} ) \big]
\label{eq:cluster-routing}\\
\hat{\gamma}_{\cluster}(\bx, \hNew^{(n)}) &\defEq
\mathbf{z}( \boldsymbol{x} )^\top \hat{\LLMEmbed}(\hNew^{(n)}).%
\end{aligned}%
\end{equation}
Intuitively,
$\hat{\gamma}_{\cluster}(\bx, \hNew^{( n )} )$ 
estimates the performance of a given LLM on $\bx$
by examining the performance of the LLM on \emph{similar} prompts,
i.e.,
those prompts belonging to the same cluster.
Note that to add a new LM to
the serving pool, we simply need to
compute per-cluster errors $\hat{\LLMEmbed}( \hNew^{( n )} )$ %
by generating responses
from the LM on a small set of validation prompts. 
Importantly, this operation
does not require any expensive gradient updates.






A common choice for the clustering algorithm in step (ii) is the $K$-means algorithm~\citep{Mac1967}, which would return a set of $K$ centroids $\{\mu_{1},\ldots,\mu_{K}\}\subset\mathbb{R}^{D}$, and an assignment map $\mathbf{z}$ that assigns a new prompt to the cluster with the nearest centroid, i.e., $z_k(\bx)  = 1$ iff $k = \arg\min_{j\in[K]}\|\Phi(\bx)-\mu_{j}\|_{2}$.
\ifarxiv
An illustration of our proposal is shown in \cref{fig:illus_cluster}.
\fi






\textbf{Special case: Pareto-random routing}.
When the number of clusters $K=1$, the  routing rule in \eqref{eq:cluster-routing} returns the same LLM for all queries $\bx$, and is given by:
\begin{align}
\hat{r}(\bx, \msetNew) & =
\underset{n \in [ \numSetNew ]}{\argmin} \, \big[ \hat{\LLMEmbedScalar}( \hNew^{(n)} ) + \lambda\cdot c( \hNew^{(n)} ) \big],
\label{eq:cluster-routing-k-equals-1}
\end{align}
where $\hat{\LLMEmbedScalar}( \hNew^{( n )} ) \defeq\frac{1}{N_{\rm val}}\sum_{(\bx, \by) \in S_{\rm val}} \1[\hNew^{( n )}(\bx)\ne\by].$  
This rule is closely aligned with the Pareto-random router (\S\ref{sec:background}). 

\begin{prop}
{For any $\lambda \in \mathbb{R}_{\geq 0}$, the routing rule in \eqref{eq:cluster-routing-k-equals-1} returns an LLM on the  \emph{Pareto-front} of the set of cost-risk pairs 
$\{(c( \hNew^{( n )} ), \hat{R}_{01}( \hNew^{( n )} ): n \in [ \numSetNew ]\}$, 
where 
$\hat{R}_{01}( \hNew^{( n )} ) = \frac{1}{N_{\rm val}}\sum_{(\bx, \by) \in S_{\rm val}} \1[\hNew^{( n )}(\bx)\ne\by]$.}
\label{prop:cluster-routing-k-equals-1}
\end{prop}












\subsection{Learned Cluster Assignment Map}
\label{sec:two_tower}
One may further improve the cluster-based routing strategy by replacing the assignment map in \eqref{eq:cluster-routing} with a \emph{learned} map
$\mathbf{z}( \cdot; \boldsymbol{\theta} ) \in [ 0, 1 ]^{K}$ 
parameterised by $\boldsymbol{\theta}$,
that can better map an input query to a distribution over clusters. 

To achieve this, we assume access to a training set containing both prompts and labels $S_\tr = \{(\bx^{(i)}, \by^{(i)})\}_{i=1}^{N_\tr}$, and a set of  LMs during training $\mset_\tr$, %
which could potentially be different from the ones we see during deployment time.

Suppose we produce a cluster assignment $\mathbf{z} \colon \XCal \to \{ 0, 1 \}^K$ as in the previous section.
We may then model the assignment function 
$\mathbf{z}( \cdot; \boldsymbol{\theta} )$ %
via a log-linear model:
\begin{align*}
 \hat{z}_k( \boldsymbol{x}, \boldsymbol{\theta} ) \propto \exp\left(\theta_{k}^\top \Phi( \boldsymbol{x} ) \right),
\end{align*}
for $\boldsymbol{\theta} \in \mathbb{R}^{K\times D}$,
and similar to \eqref{eq:cluster-routing},
estimate the probability of error $\mathbb{P}\big[\by \ne h(\bx) \mid \bx\big]$ for an LLM $h$ via:
\begin{align*}
\hat{\gamma}\left( \bx, h; \boldsymbol{\theta} \right) 
&= \mathbf{z}( \bx; \boldsymbol{\theta} )^\top \hat{\LLMEmbed}( h ), %
\end{align*}
where $\hat{\LLMEmbed}( h  )$ denotes the per-cluster errors for the LM estimated from the validation sample, analogous to~\eqref{eq:cluster_accs}.
As before,
the routing rule for the new LMs $\msetNew$ is:
\begin{equation}
    \hat{r}(\bx, \msetNew; \boldsymbol{\theta} ) = \underset{n \in [ \numSetNew ]}{\argmin}\,\, 
    \left[
    \hat{\gamma}\big( \bx, \hNew^{(n)}; \boldsymbol{\theta} \big) + \lambda\cdot c( \hNew^{(n)} )
    \right].
    \label{eq:cluster-routing-learned}
\end{equation}

To pick the parameters $\boldsymbol{\theta}$, we minimize the log loss on the training set against correctness labels for LMs $\mset_\tr$: %
\begin{align*}
    -\sum_{(\bx, \by) \in S_\tr} &\sum_{h \in \mset_\tr}
        \1\big[ \by \ne h(\bx)\big] \cdot 
            \log\,\hat{\gamma} \left(
                    \bx, h; \theta \right) 
            +
        \1\big[ \by = h(\bx)\big] \cdot 
            \log \left( 
                1 - \hat{\gamma}\left(
                    \bx, h; \theta \right) 
                    \right), 
\end{align*}
where $\hat{\LLMEmbed}( h )$ is evaluated using the \emph{validation set}. 





\subsection{Excess Risk Bound}
\label{sec:excess-risk}
We now present an excess risk bound for our cluster-based routing strategy.
Suppose we represent the underlying data distribution over $(\bx,\by)$ by a mixture of $K$ latent components:
$\mathbb{P}(\bx,\by) = \sum_{k=1}^{K}\pi_{k} \cdot \mathbb{P}(\bx,\by\,|\,z=k),$
where $z$
denotes a latent random variable that identifies the mixture
component and $\pi_{k}=\mathbb{P}(z=k)$. For a fixed component $k$, we may denote the probability of incorrect predictions
for $\hNew\in \mathscr{H}$ conditioned on $z=k$ by:
\[
\LLMEmbedScalar_{k}( \hNew^{( n )} ) \defeq \mathbb{P}_{\bx, \by|z=k}\left[\hNew^{( n )}(\bx)\ne\by\right].
\]
Then the cluster-based routing rules in \eqref{eq:cluster-routing} and \eqref{eq:cluster-routing-learned} seek to mimic the following population routing rule:
\begin{equation}
    \begin{aligned}
    \label{eq:cluster-rule-population}
    \lefteqn{\tilde{r}^{*}(\bx, \msetNew)  =}
    \\
    & ~~~
    \underset{n \in [ \numSetNew ]}{\argmin} \, \sum_{k\in[K]}\mathbb{P}(z=k|\bx) \cdot \LLMEmbedScalar_{k}(\hNew^{(n)}) +\lambda \cdot c( \hNew^{(n)} ).
    \end{aligned}%
\end{equation}



\begin{prop}
\label{prop:cluster-regret-bound}
Let $r^*$ denote the Bayes-optimal routing rule in Proposition \ref{prop:optimal_rule}.
For any $\msetNew = \{ \hNew^{( n )} \}_{n \in [ N ]} \in \mathbb{H}$, 
$\hNew^{( n )} \in \msetNew$,
and
$\bx \in \XCal$, let:
\begin{align*}
    \Delta_{k}(\bx, \hNew^{( n )} ) \defEq \left|\mathbb{P}_{\by|\bx,z=k}\left[\hNew^{( n )}(\bx)\neq\by\right] \,-\, \LLMEmbedScalar_{k}( \hNew^{( n )} ) \right|.
\end{align*}
Let 
$R_{01}(r, \msetNew) \defEq \sum_{n} \mathbb{P} \left[ \hNew^{( n )}(\bx)\neq\by \land r(\bx, \msetNew)=m \right]$ 
denote the 0-1 risk. %
Then under a regularity condition on $\mathbb{P}$,
the difference in 0-1 risk between $\tilde{r}^*$ and $r^*$ is bounded by:
\begin{align*}
\lefteqn{\mathbb{E}_{\msetNew}\left[R_{01}(\tilde{r}^*,\msetNew)\right] ~-~ \mathbb{E}_{\msetNew}\left[{R}_{01}(r^*,\msetNew)\right]} \\
&\hspace{2cm}
 \,\leq\,
\mathbb{E}_{\msetNew, \bx}\left[\max_{\hNew^{( n )} \in \msetNew, k\in[K]}\,\Delta_{k}(\bx, \hNew^{( n )} )\right].
\end{align*}
\end{prop}

Proposition~\ref{prop:cluster-regret-bound} suggests that the difference in quality between the cluster-based routing rule  (\eqref{eq:cluster-routing} or \eqref{eq:cluster-routing-learned}) and the optimal rule in \eqref{eq:opt_rule01} is bounded by the discrepancy between the per-cluster error and the per-query error.
That is, 
we require that %
for any query $\bx$, 
the average error of an LLM on the cluster $\bx$ belongs to closely reflects the error on $\bx$ itself. %


\subsection{Discussion and Relation to Existing Work}
\label{sec:discussion_related}

Our proposal relates to several recent strands of work~\citep{Tailor:2024,Thrush:2024,ZhuWuWen2024,Feng:2024,Li:2025,Zhao:2024}, which merit individual discussion.

The idea of representing an LLM via a correctness vector has close relation to some recent works.
In~\citet{Tailor:2024}, the authors proposed to represent ``experts'' via predictions on a small \emph{context set},
so as to enable deferral to a new, randomly selected expert at evaluation time.
While not developed in the context of LLMs (and for a slightly different problem),
this bears similarity to our proposal of representing LLMs via a correctness vector on a validation set; 
however, note that the mechanics of routing based on this vector (via suitable clustering) are novel.
\citet{Thrush:2024} considered representing LLMs via their perplexity on a set of public benchmarks,
for subsequent usage in pre-training data selection.
This shares the core idea of using LLM performance on a set of examples to enable subsequent modelling, albeit for a wholly different task.
\citet{ZhuWuWen2024} proposed to construct a generic
embedding for LLMs based on performance on public benchmarks.
This embedding is constructed via a form of matrix factorisation, akin to~\citet{OngAlmWu2024}.
While~\citet{ZhuWuWen2024} discuss model routing as a possible use-case for such embeddings, there is no explicit evaluation of embedding generation in the case of \emph{dynamic} LLMs;
note that this setting would na\"{i}vely require full re-computation of the embeddings, or some version of incremental matrix factorisation~\citep{Brand:2002}.

Some recent works have considered the problem of routing with a dynamic pool of LLMs.
\citet{Feng:2024} proposed a graph neural network approach,
wherein LLMs are related to prompts and \emph{tasks} (e.g., individual benchmarks that a prompt is drawn from).
Such 
pre-defined \emph{task labels} for input prompts
may be unavailable in some practical settings.
\citet{Li:2025} proposed to use LLM performance on benchmark data to construct a \emph{model identity vector},
which is trained using a form of variational inference.
This has conceptual similarity with our correctness vector proposal (and the works noted above);
however, our mechanics of learning based on this vector 
(i.e., the cluster-based representation)
are markedly different.
Further,~\citet{Li:2025} consider an online routing setting wherein bandit approaches are advisable,
whereas we consider and analyse a conventional supervised learning setting.
Finally, we note that~\citet{Zhao:2024} consider the problem of routing to a dynamic \emph{LoRA pool}, where LoRA modules are natively represented by aggregating (learned) embeddings on a small subset of training examples.
These embeddings are learned via a contrastive loss,
constructed based on certain pre-defined task labels.
As such labels may not be provided in many settings,~\citet{Chen:2024} implemented a variant wherein these are replaced by unsupervised cluster assignments.
While similar in spirit to our proposals, the details of the mechanics are different; e.g., we use the clustering to compute a set of average accuracy scores for each LLM,
rather than training an additional embedding.

Finally, we note that~\citet{Chen:2024} also consider the use of clustering as part of router training.
However, the usage is fundamentally different:
~\citet{Chen:2024}
regularise the learned query embedding $\QueryEmbed$
based on
cluster information,
while we use clustering to construct the LLM embedding $\LLMEmbed$.

Our proposed approach is similar in spirit to methods such as  K-NN, where the prediction for a query is based on the labels associated with queries in its neighborhood; in our case, we consider pre-defined clusters instead of example-specific neighborhoods.

 





%%%%% Discussion
This work identifies signal collapse as a critical bottleneck in one-shot neural network pruning. Performance loss in pruned networks is due to \textbf{signal collapse} in addition to the removal of critical parameters. We propose \textbf{REFLOW} (\textbf{Re}storing \textbf{F}low of \textbf{Low}-variance signals), a simple yet effective method that mitigates signal collapse without computationally expensive weight updates. By focusing on signal preservation, REFLOW highlights the importance of mitigating signal collapse in sparse networks and enables magnitude pruning to match or surpass state-of-the-art one-shot pruning methods such as CHITA, CBS, and WF.

REFLOW consistently achieves state-of-the-art accuracy across diverse architectures, restoring ResNeXt-101 from under 4.1\% to 78.9\% top-1 accuracy at 80\% sparsity on ImageNet. Its lightweight design makes it a practical solution for both research and deployment, delivering high-quality sparse models without the overhead of traditional approaches. These findings challenge the traditional emphasis on weight selection strategies and underscore the critical role of signal propagation for achieving high-quality sparse networks in the context of one-shot pruning.




\section*{Conclusion}
This paper aims to enhance our understanding of the computational complexity of computing various Shapley value variants. We found that for various ML models --- including decision trees, regression tree ensembles, weighted automata, and linear regression --- both local and global interventional and baseline SHAP can be computed in polynomial time under HMM modeled distributions. This extends popular algorithms, such as TreeSHAP, beyond their empirical distributional scope. We also establish strict complexity gaps between the various SHAP variants (baseline, interventional, and conditional) and prove the intractability of computing SHAP for tree ensembles and neural networks in simplified scenarios. Overall, we present SHAP as a versatile framework whose complexity depends on four key factors: \begin{inparaenum}[(i)] \item model type, \item SHAP variant, \item distribution modeling approach, \item and local vs. global explanations\end{inparaenum}. We believe this perspective provides deeper insight into the computational complexity of SHAP, paving the way for future work.




%We believe that our framework provides a more intricate understanding of SHAP computation complexity across different models, distributions, and variants, paving the way for further research.

Our work opens promising directions for future research. First, expanding our computational analysis to other SHAP-related metrics, such as asymmetric SHAP~\citep{frye20} and SAGE~\citep{covert2020understanding}, would be valuable. Additionally, we aim to explore more expressive distribution classes and relaxed assumptions beyond those in Section \ref{sec:tractable} while maintaining tractable SHAP computation. Finally, when exact computation is intractable (Section \ref{sec:intractable}), investigating the approximability of SHAP metrics through approximation and parameterized complexity theory~\citep{downey2012parameterized} is an important direction.

%Our work opens several promising avenues for future research on the computational properties of explainable AI methods, with a particular focus on SHAP. First, it would be interesting to broaden the computational analysis conducted in this work to include other popular SHAP-related metrics in the literature, such as asymmetric SHAP \cite{frye20} and SAGE \cite{covert2020understanding}. Also, in the future, we aim to explore more expressive distribution classes and relaxed distributional assumptions—extending beyond those examined in Section \ref{sec:tractable} —that still yield tractable SHAP computation. Finally, when exact computation proves intractable (Section \ref{sec:intractable}), it is worthwhile to theoretically investigate the question of the approximability of computing the SHAP metrics across various configurations, through the lens of approximation and parametrized complexity theory \cite{arora2009computational}.

%This paper aims to deepen our understanding of the computational complexity involved in obtaining different Shapley value variants. We found that for a variety of ML models, including decision trees, tree ensembles for regression, weighted automata, and linear regression models — computing both local and global interventional and baseline SHAP can be done in polynomial time when distributions are modeled by HMMs. This extends the distributional scope of popular algorithms like TreeSHAP, which is limited to empirical distributions. Additionally, we demonstrate a strict complexity gap between SHAP variants, showing that interventional and baseline SHAP can be strictly easier to compute than conditional SHAP. Despite these positive results, we uncovered intractability for various SHAP variants in neural networks and tree ensembles. Finally, we provided generalized complexity relations across SHAP variants. We believe that our framework offers a deeper understanding of the complexity involved in computing SHAP across various variants, models, distributions, as well as in both local and global computations, laying the groundwork for future research.

%%
%% The acknowledgments section is defined using the "acks" environment
%% (and NOT an unnumbered section). This ensures the proper
%% identification of the section in the article metadata, and the
%% consistent spelling of the heading.

\begin{acronym}
\acro{gan}[GANs]{Generative Adversarial Networks}
\acro{rl}[RL]{Reinforcement Learning}
\acro{pae}[PAE]{Periodic Autoencoder}
\acro{fld}[FLD]{Fourier Latent Dynamics}
\acro{ppo}[PPO]{Proximal Policy Optimization}
\acro{fft}[FFT]{Fast Fourier Transform}
\acro{pca}[PCA]{Principal Component Analysis}
\acro{dfm}[DFM]{Deep Fourier Mimic}
\acro{dof}[DoF]{Degrees of Freedom}
\acro{mlp}[MLPs]{Multi-Layer Perceptrons}
\end{acronym}


\begin{acks}
The authors would like to thank all participants of the two workshops for their time and valuable insights with special thanks to Anna-Lena Lamprecht for her wonderful talk on RSE. Also, the authors would like to thank the German Federal Government, the German State Governments, and the Joint Science Conference (GWK) for their funding and support as part of the NFDI4Energy and NFDI4Ing consortia. The work was partially funded by the German Research Foundation (DFG) – 501865131 and 442146713 within the German National Research Data Infrastructure (NFDI, www.nfdi.de) and 359941476 within the Schwerpunktprogramm 1984.
\end{acks}

%%
%% The next two lines define the bibliography style to be used, and
%% the bibliography file.
\bibliographystyle{ACM-Reference-Format}
\bibliography{references}

\end{document}
\endinput

