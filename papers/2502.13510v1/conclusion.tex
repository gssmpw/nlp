\section{Conclusion}
\label{sec:outlook}

While \ac{ERS} engineering is fundamental for energy research, research on \ac{ERS} is relatively new, and many aspects are still unknown.
As discussed in \autoref{sec:dis}, recommendations for \ac{RSE} depend on the type of research software and its development team. To explore these dependencies, a better understanding of the different types of \ac{ERS} is needed. Hasselbring et al. \cite{hasselbring_toward_2024} developed a general categorization of research software, which can serve as a starting point.

While our recommendations are generally similar to the existing recommendations in computational biology, they also differ in details. By developing recommendations in other research communities as well, it would be possible to join these recommendations to universal recommendations for general \ac{RSE}.

The developed recommendations focus on individual approaches to develop better \ac{ERS}. However, there are also systemic challenges to achieving this goal. 
For example, infrastructure for research software is required and currently missing, e.g., a registry for \ac{ERS} (as discussed in \cite{ferenz_towards_2023}) and required ontologies/standards. Good infrastructure can lower the barrier to better handle \ac{ERS}.

Another aspect discussed intensively at the workshops was that software is not sufficiently perceived as a scientific contribution. Instead, the software is often only a means to perform experiments and create results for publications. 
However, a well-written and documented library can be expected to have a large scientific impact because it is directly reusable and extendable for other researchers. In conclusion, we as a scientific community should value research software more instead of focusing too much on publications. 
Most of these challenges are similar for all types of research software, where Anzt et al. \cite{anzt2020environment} provided a good overview. Nevertheless, we see a domain-specific discussion of these aspects as an important step to improve the quality of \ac{ERS} and, therefore, the quality of energy research.
Overall, our recommendations serve as a starting point to better understanding \ac{ERS} and improve its engineering in energy research.