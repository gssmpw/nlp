\section{Discussion}
\label{sec:dis}

When comparing our recommendations with similar recommendations from other domains \cite{list2017ten,osborne_ten_2014,taschuk2017ten,balaban2021ten,prlic_ten_2012,hunter2021ten,wilson_best_2014, jimenez_four_2017} the recommendation on research proposal (see \ref{sec:proposal}) and on the organization of the software development (see \ref{sec:workProcess}) are new topics not covered before. Our other recommendations are included similarly in other works, although the provided details differ. The previous works less consider collaboration with industry and working in interdisciplinary teams. Reusing existing code is the only topic covered by nearly every other recommendation.

As with all studies, our method also has some limitations that must be considered. 
First, we decided to do the second workshop in-person in Germany and to use our good contacts with German researchers. While this excluded the international perspective, it gave us a more personal atmosphere with fewer technical delays, improved engagement, and more fluent in-depth discussions.
Second, energy research is highly diverse and interdisciplinary. While we covered a wide range of backgrounds, we may not have been able to include all possible perspectives in our workshops. 
Third, the diversity of possible \ac{ERS} projects results in not all recommendations being fully applicable for every \ac{ERS} development. As discussed in \ref{sec:workProcess}, the software development process should always be streamlined with the respective project requirements. 

While we focused on specific recommendations a single researcher can apply, additional topics came up in our workshops. 
The participants appreciated the exchange on \ac{RSE} with other researchers during the workshops. These exchanges can also be established within institutions, e.g., as a regular informal exchange round. 
Also, the topic of training came up in the workshops. In energy research, many researchers are not formally trained in software engineering. Training courses can be offered on many topics covered by the recommendations. 
They are beneficial for version control, testing, clean code, and basic architectural knowledge.
These trainings can be offered by experts in the same institution or external trainers, e.g., from the digital research academy\footnote{\url{https://digital-research.academy/}, last access 2025-01-02}.
We recommend elaborating at the beginning of a project on which training can be helpful to increase the project team's software engineering knowledge.