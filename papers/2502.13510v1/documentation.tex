\subsection{Undocumented code can not be used by others, and is therefore near-to useless}\label{sec:documentation}

\paragraph{Background} When writing code, documentation is essential to help you and others understand the code. 
Documentation also helps to ensure that the code is doing what it is supposed to do. Furthermore, it allows the extension and adaption of the code and is fundamental to ensure reproducibility and reusability for others and yourself.\par

\paragraph{Recommendations} We recommend always writing documentation, no matter how large the software project is (even if only one person is involved). Any small documentation is always better than no documentation. However, software papers do not count as documentation since they cannot reflect the ever-changing nature of software.  \par
Multiple types of documentation exist:
\begin{enumerate}
    \item Documentation in the code, e.g., code docstrings, describing methods, classes, or modules. This helps others understand your code.
    \item A README file, which introduces and describes the software. This is necessary to explain the purpose and concept of the software.
    \item A documentation page, like \textit{readthedocs}\footnote{\url{https://about.readthedocs.com/}, last access 2025-01-17}, containing further descriptions of the software, helping others to fully understand the code and use it.
    \item Documentation of tutorials or concrete examples, e.g., included in the documentation page, helps others to easily apply the code.
\end{enumerate}

We recommend to consider your open source strategy (see \ref{sec:openSource}) and try to deduce the applicable documentation style and technique. If it is unclear how the documentation will be processed, you can stick to easy documentation modes. In any case, at least the following should be provided: a README file, including a summary of the purpose of the software, an installation guide, and an example script using your software.

Documentation requires considerable effort and should thus be acknowledged as part of the software development process, e.g., as an own contribution to the respective research project, and relevant resources should be allocated. It is important not to neglect it due to missing resources later on.
When planning the documentation, it is recommended to take into account the different types of documentation and the respective target audience. You can define minimal requirements for your documentation (as requirements and dependencies for the implementation, environment, and decisions made in the development process). Also, we recommend considering tests as part of the documentation, as these are necessary for others to understand your code. To support this understanding, example scenarios can be introduced in the documentation. However, all example scenarios should be fully executable.
To simplify the documentation process, we recommend combining the coding with the documentation (for example, via commit messages, see \ref{sec:vcs}) by writing readable code, including comments. Additionally, you can use templates and standards for your documentation. Many tools are available to make the documentation process easier and to help make your documentation findable. For Python, \textit{Sphinx}\footnote{\url{https://docs.readthedocs.io/en/stable/intro/sphinx.html}, last access 2025-01-17} can be used to create technical documentation, as can \textit{Documenter.jl} for Julia\footnote{\url{https://github.com/JuliaDocs/Documenter.jl}, last access 2025-01-17}.\par 

It also helps to look for best practices. These are for example provided by Github\footnote{\url{https://google.github.io/styleguide/docguide/best_practices.html}, last access 2025-01-17}, but can also be part of recommendations for your respective programming language, as \textit{pep8} for python\footnote{\url{https://peps.python.org/pep-0008/}, last access 2025-01-17}.

Due to the interdisciplinary character of energy systems research, it is recommended that the concept behind the software be explicitly discussed. People from different domains should understand the purpose of your software. Thus, also the concepts and ideas of your software, not only the software itself, should be documented. Additionally, the code itself should be documented so that other people can work with it or even extend it. 
\recommendation{You are not done until the documentation is done!}
