\subsection{Testing your software means increasing trust in your research!}\label{sec:testing}


\paragraph{Background} Appropriate testing ensures your software works correctly. With testing, bugs can be found, and the software's functionality can be guaranteed. Especially when software is constantly changed, testing can help to ensure that changes do not break functionalities. 

You are likely already intuitively testing your software, for example, by using sanity checks on the software output, e.g., expecting values in a particular order of magnitude (MWh instead of kWh) or making sure that the efficiency is <1 or checking the consistency of results with conservation equations, such as conservation of mass, energy, or momentum. Other examples include adding \texttt{assert} statements to, e.g., make sure that input data has a specific structure, testing a function with dummy data before integrating it into a larger script, and exploring edge cases, e.g., ``how will my model handle a mass flow rate of zero?''.

You are likely performing these tests manually on an ``as needed'' basis. While this is efficient practice during the development phase, manual testing typically leads to a more extensive testing effort in the long run~\cite{patrick_software_2016}, and ad hoc, intuitive testing by the developer can have blind spots~\cite{mischke_automated_2022}. In addition, automated testing provides benefits such as the ability to make changes quickly and securely and perform major refactorings without the entire software immediately losing several levels of quality.

\paragraph{Recommendations} 
We recommend you elevate your intuitive, manual testing to a systematic, automated testing procedure, for example, as described in~\cite{mischke_automated_2022}. 

We recommend considering a test strategy and testing your software based on it. A test strategy describes the testing approach, including what aspects, methods, and techniques of the software are to be tested~\cite{drlsoftwareinitiative}. It should be identified early and specify how the software's testing process is designed \cite{drlsoftwareinitiative}. 
Using a strategy helps ensure that the code has been extensively and thoroughly tested, including all relevant functions. \par

In your strategy, we recommend determining what exactly needs to be tested (properties and behavior) and defining minimal requirements for that (e.g., test coverage, which defines the degree to which the code is checked through tests~\cite{drlsoftwareinitiative}). Requirements defined at the beginning of the software project should be mapped to the tests and thus checked for functionality. Similar to documentation (see \ref{sec:documentation}), tests should be considered as a contribution to the software project and should be planned accordingly. It is important not to neglect testing due to missing resources later on.
While testing, the reusability and reproducibility of the code should be considered. Thus, we recommend providing data for testing and information such as API versions, dependencies, and comparability to other software if applicable. 
Templates and standards, e.g., use testing packages for your respective type of test and programming language, as \textit{pytest}\footnote{\url{https://docs.pytest.org/en/stable/}, last access 2025-01-17} for Python, can help you when writing your tests.

We recommend taking into account different types of tests: unit, integration, and end-to-end tests. Unit tests are designed to test the functionality of specific components or modules, their restrictions and constraints, while integration tests focus on the interfaces and interaction of components~\cite{drlsoftwareinitiative}. For example, a unit test could test the performance of a photovoltaic system under different solar irradiance conditions and, thus, the functionality of the power calculation function. Integration tests would, e.g., focus on interactions between a controller of the plant and the plant itself, such as sending and responding to control signals. End-to-end tests consider the software as a whole, which helps to ensure that the overall software behaves as expected, considering the data flow over all tasks. An example of end-to-end testing is testing the entire system, including all controllers and components, from an end-user perspective. Ideally, all testing is completely automated using the \ac{CICD} capabilities of GitHub/GitLab. This way, you can ensure that every single version of your software is tested. \par

In the energy domain, software is sometimes used by people who did not develop it (people without IT knowledge). Therefore, we recommend having the software tested by potential users who were not involved in its development. By involving potential users early in the development process and having them test the installation, it is possible to ensure that the software can be used by the people it is intended for. Testing is essential if the software will be used in the field. Since we are considering software for a safety-critical system, well-designed testing is significant in energy research.
\recommendation{Test your software based on a test strategy!}