\section{Literature Review}
\label{subsec:rel_work}
Several existing publications provide recommendations or best practices on \ac{RSE}. To give an overview, we looked at the following categories: publications providing recommendations, publications on the general state of \ac{RSE}, and publications on \ac{FAIR} principles for research software and their evaluation.
Regarding recommendations for research software in general, various publications exist \cite{8887228, anzt2020environment, arvanitou2021software, balaban2021ten, baxter2012research, castell2024towards, cohen2020four, crouch2014software, eisty2022developers, eisty2018survey}. Some publications consider best practices for \ac{RSE} in specific domains, as in computational biology \cite{list2017ten}. Others focus on particular aspects of \ac{RSE}, for example, the documentation of software \cite{hermann2022documenting, lee2018ten}, specific programming languages as Python \cite{irving2021research}, sharing and reusing research software \cite{park2019research}, sustainability of research software \cite{de2019makes}, or testing research software \cite{eisty2022testing}.
The aspect of open-source development is more and more discussed as well \cite{zirkelbach2019modularization, hasselbring2020open}. 
Multiple publications apply the \ac{FAIR} principles to research software \cite{barker2022introducing, hasselbring2020fair, katz2021taking, gruenpeter2020m2}.
Additionally, several publications provide rules for software engineering in general, not focusing on research software \cite{brack2022ten, david2023task, jones2009software, sandve2013ten, schlauch2018software, stodden2016enhancing, stodden2013best, struck2018research, wagner2006creation}. Some publications also consider specific aspects in this area that might be applied to energy software engineering, such as robust software \cite{taschuk2017ten} or code quality \cite{trisovic2022large}. 
\par 
Based on our extensive research, we identified four topics frequently discussed in the literature:
\begin{description}
\item[Architecture and software design] This includes the modularity and reuse of existing software. The literature suggests considering community standards for input/output data~\cite{castell2024towards}, providing a comprehensible, extensible, modular, and easily exchangeable structure, as well as usage of design patterns and common rules regarding programming styles, the programming language \cite{drlsoftwareinitiative}, and iterative processes regarding requirements~\cite{johanson2018software}.
\item[Development process] Here, the development process itself, organization, environment, documentation, and code reviews are discussed. In literature, recommendations consider the software's reproducibility, usability, and maintainability. Additionally, specific recommendations on using version control systems \cite{castell2024towards}, repository structures following community standards \cite{castell2024towards},  continuous refactoring and testing \cite{drlsoftwareinitiative} or particular aspects, such as useful filenames \cite{trisovic2022large} exist.
\item[Testing] Test strategies and types of tests are considered. The literature recommends automated testing \cite{castell2024towards, drlsoftwareinitiative}, different testing types such as module, integration, system or acceptance tests \cite{drlsoftwareinitiative}, end-to-end testing \cite{hunter2021ten}, using test strategies, and defining and ensuring test coverages \cite{drlsoftwareinitiative}. 
\item[Publication of software] Aspects as licenses and the community are considered here. In the literature, publishing code~\cite{barnes2010publish} or developing open source \cite{hasselbring2020open} is highly recommended. Publications consider reproducibility and reusability of code \cite{gruenpeter2020m2}, recommending publishing everything necessary for reproducibility~\cite{stodden2016enhancing}. Additionally, appropriate licenses are suggested \cite{castell2024towards}, using persistent identifiers (PIDs) \cite{castell2024towards}, and providing well-structured metadata~\cite{castell2024towards}.
\end{description}

Although lots of literature discusses the general topics, none of the publications derive general recommendations for \ac{RSE} or 
consider the requirements and constraints of software engineering for \ac{ERS}. 
In contrast, we aim for a single publication that covers the most important aspects of \ac{RSE} in the energy domain, with domain-specific requirements and examples, such that the reader can gain a sufficient overview by reading a single publication.
