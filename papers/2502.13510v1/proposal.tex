\subsection{Good software development needs resources}\label{sec:proposal}

\paragraph{Background} Energy research is mainly funded through research projects. Since software engineering is often a big part of research projects it should also be covered in project proposals. Including software engineering aspects in proposals enables researchers working on research projects to develop better research software. Many of the aspects already mentioned can be included in a research project proposal.

\paragraph{Recommendations} Within a proposal, we recommend emphasizing reusing existing frameworks instead of developing new solutions as introduced in \ref{sec:reuseAndExtend}. Even if the reuse of existing software is planned, it is recommended to include some resources for maintaining and adapting the existing software.

Sometimes, it is still necessary to develop new software. In these cases, libraries that can be easily integrated into other software are better suited than complicated frameworks.
Additionally, we recommend
to formulate an overall software strategy, including a software management plan (see \cite{jackson_checklist_2018}). This should include the different development aspects during the research project, e.g., testing. Also, this strategy should cover how the software can be reused after the project. Sometimes, other research groups can even write letters of interest for reusing the software. As discussed in \ref{sec:reuseAndExtend}, we recommend choosing the licenses early, which can already be part of the proposal. 

Especially in energy research, research projects often include cooperation with industry partners. In these cases, we recommend finding a common solution for licensing with all involved partners within the research project proposal. In such cases, it can also be helpful to include resources for legal aspects of the software in the research project proposal.

When new software is developed, relevant resources should be included in the research project proposal. Besides sufficient resources for the general development of the software, including essential steps such as testing and documenting the software, this should also take usability aspects into account since the software should be reusable for other researchers. To achieve reuse, planning  resources for community interaction as described in \ref{sec:community} is also helpful.
Since most researchers are not trained in \ac{RSE}, resources for training and networking with other research software engineers are recommended. 

\par 
Software development plays an important role in research projects. By including different \ac{RSE} aspects in the research project proposal, it becomes easier to cover software engineering aspects and follow our recommendations within a research project.


\recommendation{Include \ac{RSE} in your research project proposals!}