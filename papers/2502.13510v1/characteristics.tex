\section{Characteristics of ERS} \label{sec:chara}

Research software, in general, has several unique characteristics. It is often highly specific and mainly created to solve one/few particular use case(s). The software is usually not designed for reuse and has a short lifecycle. Generally, it is hard to comprehend the software, even for researchers from the same domain, due to the inherent complexity of the solved problems. Further, most researchers are not explicitly trained in software engineering. Another fundamental characteristic of research software is that the software requirements are not known beforehand but often only evolve during development~\cite{hasselbring_toward_2024,felderer_investigating_2025}. 

\ac{ERS} shares these characteristics of general research software, while some common aspects can be considered more critical in energy research. Additionally, \ac{ERS} also has some special characteristics:

\begin{description}
    \item [Interdisciplinary research] Energy researchers have diverse backgrounds ranging from social science over engineering to different natural sciences and mathematics. As a broad range of domains are involved, it is hard to establish a common language. Therefore, it is difficult to achieve a mutual view of the problems that must be solved as a team. This issue influences the development of \ac{ERS} as well as its capability of being understood by different interested researchers like project partners.
    \item [Applied research] Energy research is an applied research field. Therefore, cooperation and joint projects with industry are widespread. This also has implications on the engineering of the jointly created software projects, e.g., with respect to open source or use of commercial software.
    \item [Changing levels of detail] \ac{ERS} is diverse and complex and needs to support many different analysis types and detail levels (e.g., transient vs. steady-state).
    \item [Complexity] Energy researchers often look into the detailed behavior of larger systems consisting of many components. 
    \item [Data heterogeneity] \ac{ERS} needs a lot of diverse and heterogeneous data for most simulations and often produces data different in structure, type and size, which needs to be managed \cite{zhang2018big}.
    \item [Reliability] Energy systems are critical infrastructures leading to high demands on research software reliability, specifically when conducting field tests. 
    \item [Changing time horizons] \ac{ERS} often considers diverse time horizons, diverse spatial resolutions, and complex optimization problems~\cite{ENGELAND2017600,DECAROLIS2017184}. These aspects can lead to high performance requirements.
    \item [Coupled co-simulations] As huge infrastructure systems are under analysis, there is a need to couple different types of simulation, often comprising communication simulation  \cite{vogt_survey_2018,steinbrink_cpes_2019}.
\end{description}%

We generally conclude that the most essential aspects of \ac{ERS} are the people involved and working on different detail levels, spatial resolutions, and time resolutions in a diverse and highly simulation-driven research area.