\section{Literature Review}
\label{subsec:rel_work}
Several existing publications provide recommendations or best practices on \ac{RSE}. To give an overview, we looked at the following categories: publications providing recommendations, publications on the general state of \ac{RSE}, and publications on \ac{FAIR} principles for research software and their evaluation.
Regarding recommendations for research software in general, various publications exist ____. Some publications consider best practices for \ac{RSE} in specific domains, as in computational biology ____. Others focus on particular aspects of \ac{RSE}, for example, the documentation of software ____, specific programming languages as Python ____, sharing and reusing research software ____, sustainability of research software ____, or testing research software ____.
The aspect of open-source development is more and more discussed as well ____. 
Multiple publications apply the \ac{FAIR} principles to research software ____.
Additionally, several publications provide rules for software engineering in general, not focusing on research software ____. Some publications also consider specific aspects in this area that might be applied to energy software engineering, such as robust software ____ or code quality ____. 
\par 
Based on our extensive research, we identified four topics frequently discussed in the literature:
\begin{description}
\item[Architecture and software design] This includes the modularity and reuse of existing software. The literature suggests considering community standards for input/output data____, providing a comprehensible, extensible, modular, and easily exchangeable structure, as well as usage of design patterns and common rules regarding programming styles, the programming language ____, and iterative processes regarding requirements____.
\item[Development process] Here, the development process itself, organization, environment, documentation, and code reviews are discussed. In literature, recommendations consider the software's reproducibility, usability, and maintainability. Additionally, specific recommendations on using version control systems ____, repository structures following community standards ____,  continuous refactoring and testing ____ or particular aspects, such as useful filenames ____ exist.
\item[Testing] Test strategies and types of tests are considered. The literature recommends automated testing ____, different testing types such as module, integration, system or acceptance tests ____, end-to-end testing ____, using test strategies, and defining and ensuring test coverages ____. 
\item[Publication of software] Aspects as licenses and the community are considered here. In the literature, publishing code____ or developing open source ____ is highly recommended. Publications consider reproducibility and reusability of code ____, recommending publishing everything necessary for reproducibility____. Additionally, appropriate licenses are suggested ____, using persistent identifiers (PIDs) ____, and providing well-structured metadata____.
\end{description}

Although lots of literature discusses the general topics, none of the publications derive general recommendations for \ac{RSE} or 
consider the requirements and constraints of software engineering for \ac{ERS}. 
In contrast, we aim for a single publication that covers the most important aspects of \ac{RSE} in the energy domain, with domain-specific requirements and examples, such that the reader can gain a sufficient overview by reading a single publication.