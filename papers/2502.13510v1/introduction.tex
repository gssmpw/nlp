\section{Introduction}
\label{sec:intro}

\ac{ERS} is defined as "software used in the scientific discovery process for understanding, analyzing, improving, and designing energy systems" \cite{ferenz_towards_2023}.
\ac{ERS} is used in multiple ways in the energy domain, e.g., for modeling specific components of energy systems, analyzing data, or implementing control strategies. A significant proportion of energy research is simulation-based, often requiring an interplay of different complex simulators, like grid simulation, power-plant simulation, and communication simulation. This leads to a strong reliance on (often third-party) simulation software, including (co-)simulation frameworks.
Thereby, \ac{ERS} contributes significantly to energy research and serves as the foundation for most research outcomes in this domain. Thus, there is a direct link between the quality of results in energy research and the quality of \ac{ERS}. Additionally, the quality of \ac{ERS} also determines if research is reproducible and if \ac{ERS} is reusable as emphasized by the FAIR (Findable, Accessible, Interoperable, and Reusable) principles~\cite{barker2022introducing}.

\ac{ERS} is often developed by energy researchers with diverse backgrounds, e.g., in electrical engineering, mechanical engineering, economics, or computer science. Many of them did not receive any training in software engineering. Therefore, \ac{RSE} regularly does not consider classical software engineering practices \cite{hasselbring2020fair}. Even computer scientists can not completely rely on their training since general software engineering practices are not fully transferable to research software \cite{heroux_research_2022}. 

Recommendations on software engineering can help researchers write better code and make code reusable. Such recommendations already exist in  computational biology \cite{list2017ten}.

Currently, no publication focuses generally on recommendations for \ac{RSE} or on \ac{RSE} in energy research. While some recommendations from other domains can be applied to the energy domain, \ac{ERS} has some unique characteristics, like its high focus on simulations across different expertise fields and its development close to industry. 

To close this gap, we developed recommendations for engineering \ac{ERS} for our community together with energy researchers. 
In this work, we present our ten recommendations for better \ac{ERS} engineering, starting from whether an idea is worth implementing over fundamental aspects like documentation and testing to building a community around a software project. These recommendations present a starting point for energy researchers to improve their software engineering skills and develop better \ac{ERS}. They were derived and validated from two workshops with other energy researchers. 

In \autoref{sec:chara}, we provide an overview of the typical characteristics of \ac{ERS}. Afterward, we present the review of the literature on general \ac{RSE} and \ac{RSE} in other domains, which we describe in more detail in \autoref{subsec:rel_work}. Then, we outline the method with which we derived the recommendations in \autoref{sec:method}. Our main contribution, the recommendations, are presented in \autoref{sec:guidelines}. Finally, we discuss our results in \autoref{sec:dis} and give an outlook of the further required work in \autoref{sec:outlook}. 

