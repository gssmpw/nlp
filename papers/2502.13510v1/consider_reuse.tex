\subsection{Building upon research requires reusable software}\label{sec:reuse}

\paragraph{Background} To effectively build upon research, it is often required to build upon research software, and therefore, a software's reusability is a key factor to its scientific success. Research based on openly available and reusable software is cited more often than research where the software is not reusable. To achieve this, it is necessary to consider your software's applicability to more general problems (than your own). This is decided by your software's abstraction level (generalized vs. specific implementation). However, there is a conflict regarding this abstraction level most of the time. There are two extreme ends to this aspect:
\begin{enumerate}
    \item Implementing highly specialized features that serve exactly one use case, e.g., a specific heat pump model that does not enable any parameterization (e.g., to implement different coefficients of performance, capacities, or others). 
    \item Finding an abstraction for every feature, which provides maximal freedom for analyzing other use cases than the one that shall be implemented in the first case, e.g., a generalized heat pump model, which does not even have a specific mathematical model but enables you to provide an arbitrary one by yourself.
\end{enumerate}
Both ends are extreme, and usually, you want to avoid both cases. However, there is a large spectrum of possibilities between these extremes. 

\paragraph{Recommendations} We recommend finding a sensible abstraction level, as it is crucial for the reusability of the software, and good abstractions enable other researchers (including yourself and your group) to build upon your software. At the same time, no unnecessary complexities should be introduced by choosing an overgeneralized abstraction. This might render your software less usable due to the lack of specific implementations and understandability. Consequently, users need to invest more effort, and reusing becomes difficult. 

Another aspect of this is to think about the scope and objective of your software. These determine to which level your software should be abstracted. As a rule of thumb, every artifact should focus on precisely one type of research. This can be on different, relatively macroscopic levels of energy research. For example, you can develop one type of energy component model (i.e., for heat pumps) or one software to simulate decentralized individual behaviors of actors in energy communities using agent-based modeling.

Besides the level of your implementation, meta aspects, like the name of the software, can also impact the reusability of your software. Also, we recommend considering the scientific community you contribute to and how they organize themselves in open software projects. Integrating your contribution in the form of software to these software communities increases your visibility and strengthens the community as a whole. Example communities for \ac{ERS} could be based on research frameworks, e.g., oemof\footnote{\url{https://oemof.org}, last access 2024-12-18.}, or generally communities of your favorite programming language, e.g., the Julia communities\footnote{\url{https://julialang.org/community/organizations/}, last access 2024-12-18}. Furthermore, considering the target audience, which will eventually reuse your software, is essential. To improve the compatibility of the software, you can use ontologies, such as the open energy ontology\footnote{\url{https://openenergyplatform.org/ontology/}, last access 2024-12-18} to develop and validate whether the keywords used in your software and documentation fit those of the broader community.

In short, software reusability is essential for making energy research sustainable and interoperable, not only for you and your working group but also for the broader research community you are working in.

\recommendation{Consider the reuse of your research software!}
