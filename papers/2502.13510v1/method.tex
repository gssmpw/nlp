\section{Method}
\label{sec:method}

This section gives an overview of the method used to derive our ten recommendations. To find, organize, and prioritize the recommendations, we used an exploratory approach \cite{harrison_inventor_2024}.

Generally, we focused on research software for modelling, simulation and data analytics, and proof-of-concept software \cite{hasselbring_toward_2024}. We generally exclude the development of infrastructure software 
because this software is developed based on a different set of requirements which normally puts usability by other more in the center. 

We mainly used three sources to define the recommendations as method triangulation \cite{carter_use_2014} as shown in \autoref{fig:flow_chart}. First, we identified four topics based on the literature as described in \autoref{subsec:rel_work}. We used these topics as the foundation for an internal workshop at our institution, which we further outline in \ref{subsec:ws1}. From this workshop, we derived a first set of recommendations. We used these recommendations as a starting point for a second workshop, to which we invited multiple researchers and research software engineers from Germany. We provide more information on that workshop in \ref{subsec:ws2}. We combined the results of the second workshop with the outcomes of the first workshop to define the final recommendations presented in this paper.

\begin{figure}[h]
  \centering
  \includegraphics[width=.7\linewidth]{erse-process-new.pdf}
  \caption{Flow chart of the method}
  \label{fig:flow_chart}
\end{figure}

\subsection{Internal Workshop}
\label{subsec:ws1}
From the related work, we identified four main topics: architecture and software design, development process, testing, and publication of software. For these topics, we picked out the key recommendations that can be applied to energy research and used them as a basis for discussion in our internal workshop. We conducted this workshop at our research institute to gather different perspectives on each topic and define a first set of recommendations.
The on-site workshop was attended by eight participants of the about 100 researchers in energy system research at our
institution\footnote{OFFIS,  \url{https://offis.de/en/applications/energy.html}, last access 2024-12-20.}
covering all energy-related research groups. The group consisted of one professor, five PhD students, and two other research staff who all have a high software engineering focus within their research groups. The workshop took place in July 2024 and was mainly held in English, while some informal discussions were also in German. The results were directly documented on boards in the room.

The workshop was structured into three main parts. The first session began with an introduction to the topic where general challenges and problems for \ac{ERS} engineering were discussed. Participants were encouraged to share their initial thoughts and ideas on the topic without bias from our preparation. Second, participants were divided into random groups and tasked with discussing the four identified topics related to \ac{ERS} engineering. Each group was moderated by one of the authors to ensure focused and productive discussions. Ideas were captured on cards, with each group contributing their insights and solutions.
In the final phase, each participant was given four points to distribute among the cards, allowing them to prioritize the most pressing challenges and innovative solutions. This voting mechanism helped identifying the key areas that require more attention and further development. 

\subsection{National Workshop}
\label{subsec:ws2}
To collect more diverse input for the recommendations and to refine 
our initial draft of recommendations, we conducted a second workshop in November 2024. We invited multiple energy researchers with high expertise in \ac{RSE}. The recruiting was based on two pillars. First, representatives of highly relevant \ac{ERS} were invited to present their software and join the workshop. Second, members of the Energy Research Center of Lower Saxony (EFZN) in Germany \footnote{\url{https://efzn.de/}, last access 2025-01-02.} were invited. The on-site workshop brought together 20 participants (excluding the authors) from 13 institutions across Germany, providing a diverse and enriching exchange of ideas and expertise. The workshop was mainly held in English, while many participants discussed in German. The results were directly documented on boards in the room.

The workshop was structured into four parts. The workshop began with a keynote on \ac{RSE} by a software engineering professor highly familiar with the topic. She provided a comprehensive overview of the field and its significance for research to lay the foundation for the further workshop. Second, specifically invited participants presented highly relevant \ac{ERS} through a poster session, facilitating an initial exchange of ideas. Additional, this brought key stakeholders to the workshop. During this session, participants were asked to identify challenges in \ac{ERS} engineering and document them. Afterward, participants were divided into four random groups and tasked with discussing the first ten recommendations. Only the headings of the recommendations, as given in \ref{subsec:ws1}, were given to the participants to allow free associations. Each group was moderated by one of the authors to ensure focused and productive discussions. Ideas were captured on cards, with each group contributing their insights and solutions. In the final phase, each participant was given ten points to distribute among the cards, allowing them to prioritize the most pressing challenges and innovative solutions. This voting mechanism helped identify the key areas that required immediate attention and further development. 

The workshop provided a collaborative platform for participants to share their experiences, brainstorm solutions, and prioritize the most critical challenges in \ac{ERS} engineering. Based on both workshops, the final recommendations were derived, which will be presented in the next section (\ref{sec:guidelines}).
