\subsection{Track the history and learn for the future}\label{sec:vcs}

\paragraph{Background} Software is constantly changing. New features are implemented, bugs are fixed, and code is improved. Sometimes, these developments also happen in parallel. For example, you are implementing a physical model to analyze the operating behavior of a battery, and you supervise a student tasked to implement a thermal model for the battery and, therefore, give them a copy of your code. When the student finishes their implementation after a couple of months, your own code has evolved, and manually integrating the master student’s code back into your own becomes a messy task of copy, paste, break, fix, test, and retest. This, and many other problems associated with changing code, multiple versions, and coding as a team, can be avoided with version control. While using version control is standard practice in software development, it is not always the case in research software. Tracking different versions of your code enables you and others to keep an overview of which feature was developed and to understand the history of the code, thereby increasing its maintainability. It also allows for the undoing of specific changes when a problem occurs. Additionally, version control enables reproducibility of the research conducted with the code by providing specific links to use  exactly the same version~\cite{sandve2013ten}.

\paragraph{Recommendations} Version control systems can help you track different software versions. The most popular version control system is Git\footnote{\url{https://git-scm.com/}, last access 2024-12-17.}.
The platforms GitHub\footnote{\url{https://github.com/}, last access 2024-12-17.} and GitLab\footnote{\url{https://about.gitlab.com/}, last access 2024-12-18} are based on Git and extend it by a user interface providing a good overview and additional features. If your institution does not allow you to use GitHub or the public GitLab, you can often also use a GitLab instance offered by your institution.

Within Git, changes to the software are included as a \textit{commit}. By adding meaningful messages to each commit, the changes are documented. At least one commit at the end of each day is recommended. 

Version control systems also allow different \textit{branches}, which are parallel versions of the same software. It is helpful to always have a working software version in one branch (usually called \textit{master} or \textit{main}). Also, this way, different developers can work in parallel without directly causing conflicts. A branch can be merged into another by a \textit{pull} or \textit{merge request}. These requests are another good option for additional comments and documentation to make the changes more understandable. In software projects with multiple developers, this is also the step where code reviews should be conducted for quality assurance.

The version control platforms also have additional features that make it easier to organize the software development. \textit{Issues} can be used to track bugs, code contributions, and new potential feature ideas. They also allow others to get an overview of what is planned next. Additionally, they provide options for \ac{CICD}, which is the practice of automating the integration, testing, and deployment of code changes to ensure reliable and efficient software delivery. In the context of research software, \ac{CICD} is especially used for automated testing of each commit (see also \ref{sec:testing}).

\par 
Version control systems are necessary to keep track of software versions while also allowing to organize the software development to a certain extent. If you are not familiar with version control, we recommend taking some basic training on it, e.g., by the software carpentry\footnote{\url{https://swcarpentry.github.io/git-novice/index.html}, last access 2024-12-17}.  

\recommendation{Use version control!}