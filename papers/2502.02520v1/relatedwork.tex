\section{Related Work}
\label{RelatedWork}
Representing SSI systems as a stack or a layered model is not new. Based on the Trust over IP (ToIP) model \cite{ToIP} which encompasses SSI systems with their governance and related trust protocols, other models and mappings of the SSI components can be found in \cite{s22155641}, \cite{mapping}, \cite{Yildiz} and \cite{layering-diagram}. In \cite{mapping}, H. Yildiz provides a mapping of different SSI standards and components. Organized by layers, components are also assigned a level of maturity ranging from stable to experimental.

Our layered dissection of SSI presented in Section \ref{section 2.1} differs from other related work. Contrary to models that group the identifiers and the cryptographic material together with verifiable registries (like in \cite{layering-diagram}, DIF documentation), e.g. blockchains or databases containing public keys, our model dedicates an independent layer "Identifiers \& Cryptographic materials" which includes an "Identifier Resolution" sub-layer. This sub-layer, which manages the identifier resolution, acts as an intermediary layer between the infrastructure layer integrating the useful verifiable registries and the identifiers layer. We are also different from models where the agent layer and the application layer are two separate layers, like in \cite{Yildiz}, or where the infrastructure is combined with the agents, like in \cite{layering-diagram}. We consider the wallet application as the application layer that is composed of two sub-layers: the edge agent, which is the part of the application installed on the user's device and the cloud agent, which enables communication between edge agents in cases where the architecture is based on such server-like agents (not the case in P2P networks for example).


The objective of our model is different from other existing models and is to provide an in-depth privacy analysis for SSI systems, which requires mapping privacy properties to different SSI components and layers. Most other models focus on interoperability, and researchers focus on how to make different SSI components and layers interoperable with existing systems and between SSI systems themselves. 

Other related works deal with certain privacy issues in SSI. In \cite{9297357}, the authors study the privacy shortages of credentials in Hyperledger Indy and propose a model of attribute sensitivity score that assesses different attributes found in verifiable credentials by attributing them a score of 0.0 to 1.0, 1.0 being the highest sensitivity score for an attribute to be shared. This score model is integrated into the cloud agent (Hyperledger Aries here) (cf. Section \ref{section 6.2}) to enable it to take into account the sensitivity of the attributes presented (here Hyperledger Aries) (see Section \ref{section 5.1.2}) to a certain verifier, which can be used to raise privacy awareness and improve the overall privacy of SSI credentials. 
In \cite{DID-Privacy, VCPrivacy, VC-Privacy}, the W3C provides some Privacy Considerations related to DIDs, and VC data models 1.0 and 2.0, however, these considerations are non-normative, remain very generic and lack any notable analysis of existing SSI technologies and systems.
In \cite{10.1007}, the objective is to derive SSI credentials from existing eID systems, and to build privacy-preserving credentials that can be easily accepted by established systems, e.g. governments and banks, and that implement the selective disclosure principle thanks to ZKP.



Compared to the literature, this paper provides a deeper and broader assessment of which privacy properties a SSI system can ensure at different layers. It gives a rich comparative analysis between different SSI components in each layer. Since the objective of the paper is to focus on design, we focus on the most commonly used SSI components and provide a mapping between these components and the privacy properties that they may need to satisfy.