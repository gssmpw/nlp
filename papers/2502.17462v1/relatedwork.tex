\section{Related work}
\textbf{Neural audio compression.} 
Neural network models have recently started gaining popularity in the audio compression domain due to their high compression ratios and design flexibility. Most neural compression architectures consist of the encoder-decoder pair of an autoencoder, together with a quantizer to generate discrete representations. VQ-VAE~\citep{VanDenOord2017} introduced this method---unrelated to the compression objective---by combining variational autoencoders (VAE) with vector quantization (VQ). VQ uses a learnable codebook containing a discrete set of vectors to represent a larger set of input vectors.

GANs have been shown~\citep{Kumar2019, Yamamoto2020, Kong2020} to be an effective solution to drive the overall neural compressor towards better representations. MelGAN~\citep{Kumar2019} introduced a multi-scale discriminator that consists of three convolutional discriminators that operate on different scales of the waveform. This architecture restricts the discriminators to specific frequency bands so that they learn features of different scales. The learned features can be used to train a generator by minimizing the distance between features of real and synthetic data.

Combining autoencoder, quantizer, and GAN, SoundStream~\citep{Zeghidour2022} and EnCodec~\citep{Defossez2023} represent the state-of-the-art audio compression models. They are based on a fully convolutional encoder-decoder network with a residual vector quantizer (RVQ) and a convolutional GAN discriminator, operating on the frequencies of interest of the signal. All components are jointly trained end-to-end by minimizing reconstruction, quantization, as well as perceptual adversarial losses. 

\textbf{Lossless EEG compression.} Standard lossless compression algorithms such as gzip, zstd, and lz4 are used routinely to reduce the storage requirements of large EEG collections. Typical compression ratios for these algorithms on EEG are 1.2$\times$ to 1.5$\times$. Lossless compression models developed specifically for EEG are more scarce~\citep{Alsenwi2018, Hadi2021, AlNassrawy2022} and have not found use in practice.

\textbf{Lossy EEG compression.} Lossy compression has not been adopted for either clinical or research use due to the uncertainty about the fidelity of the reconstructed signal. In particular, standard lossy time-series compressed formats such as mp3 have not been developed for EEG and thus have poor performance. Recently, wavelet transform-based techniques using NLSPIHT~\citep{Xu2015}, arithmetic coding (AAC;~\cite{Nguyen2017}), and artificial neural networks (ANN;~\cite{Hejrati2017}), have shown impressive results on EEG, with compression ratios up to 8$\times$ and high reconstruction fidelity. Given the importance of EEG signals in a variety of clinical tasks, seizure detection~\citep{Nguyen2018} has been used to validate the reconstruction fidelity of lossy compression as well. Finally, DL models equipped with compressed sensing techniques (CS;~\cite{Du2024}) have achieved state-of-the-art compression performance in terms of reconstruction fidelity. Our BrainCodec surpasses existing work by consistently achieving higher compression ratios and higher reconstruction fidelity.