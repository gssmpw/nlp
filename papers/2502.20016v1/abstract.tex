Sustainability encompasses three key facets: economic, environmental, and social. However, the nascent discourse  that is emerging on sustainable artificial intelligence (AI) has predominantly focused on the environmental sustainability of AI, often neglecting the economic and social aspects. Achieving truly sustainable AI necessitates addressing the tension between its {\em climate awareness} -- the need to mitigate AI’s environmental impacts -- and its social sustainability, which hinges on equitable access to AI development resources. The concept of {\em resource awareness} advocates for broader access to the infrastructure required to develop AI, fostering equity in AI innovation. Yet, this push for improving accessibility often overlooks the environmental costs of expanding such resource usage. In this position paper, we argue that reconciling climate and resource awareness is essential to realizing the full potential of sustainable AI. We use the framework of base-superstructure to analyze how the material conditions are influencing the current AI discourse. We also introduce the Climate and Resource Aware Machine Learning (CARAML) framework to address this conflict and propose actionable recommendations spanning individual, community, industry, government, and global levels to achieve sustainable AI.
