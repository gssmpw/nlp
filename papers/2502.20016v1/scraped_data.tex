\section{Data Collection and Processing}
\label{app:data}
We scraped the data from a public repository\footnote{\url{https://github.com/martenlienen/icml-nips-iclr-dataset}} covering ICML (2017--2023), NeurIPS (2006--2022), and ICLR (2018--2023 except 2020). The year 2020 is missing for ICLR because their one-off virtual format was hosted on a webpage incompatible with the scraping tool. After extracting paper titles, authors, and affiliations, we geocoded each affiliation to determine its country, marking unresolved affiliations as “Unknown” and removing them. We then assigned each country to a World Bank income group~\citep{worldbank2024} based on the publication year.

For ICML, we scraped a total of 1588 entries in 2017 (434 unique), which included 1029 high-income affiliations, 52 upper-middle-income, and 11 lower-middle-income, plus 496 unknown. Similarly, in 2021 we obtained 4844 entries (1183 unique), comprising 3166 high-income, 365 upper-middle-income, 47 lower-middle-income, and 1266 unknown. 

For NeurIPS, 2497 entries were retrieved in 2017 (679 unique), with 1754 mapped to high-income countries, 104 to upper-middle-income, 13 to lower-middle-income, and 626 unknown. In 2021, we gathered 9931 entries (2334 unique), of which 6470 were high-income, 1052 upper-middle-income, 106 lower-middle-income, and 2303 unknown.

\paragraph{Limitations.} A single paper may have multiple affiliations listed, so our scraped dataset can include multiple entries per publication -- one for each affiliation. This inflates the total number of entries relative to unique papers. In addition, some affiliations cannot be successfully geocoded, potentially skewing the distribution if certain regions are more prone to geocoding failure. Nevertheless, preliminary checks suggest that these unresolvable affiliations are not disproportionately associated with any single income group, preserving the overall regional balance in our analyses.


