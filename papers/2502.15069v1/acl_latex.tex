% This must be in the first 5 lines to tell arXiv to use pdfLaTeX, which is strongly recommended.
\pdfoutput=1
% In particular, the hyperref package requires pdfLaTeX in order to break URLs across lines.

\documentclass[11pt]{article}
\usepackage{multirow}
% Change "review" to "final" to generate the final (sometimes called camera-ready) version.
% Change to "preprint" to generate a non-anonymous version with page numbers.
\usepackage[dvipsnames,table]{xcolor}
%\usepackage[table]{xcolor}
\usepackage{colortbl}
%\usepackage[review]{acl}
\usepackage{acl}
\usepackage{outlines}
% Standard package includes
\usepackage{times}
\usepackage{latexsym}
\usepackage{booktabs}
\usepackage{array}
\usepackage{tcolorbox}
\usepackage{geometry}
\usepackage{multicol}
% For proper rendering and hyphenation of words containing Latin characters (including in bib files)
\usepackage[T1]{fontenc}
% For Vietnamese characters
% \usepackage[T5]{fontenc}
% See https://www.latex-project.org/help/documentation/encguide.pdf for other character sets
\hbadness=99999  % or any number >=10000

% This assumes your files are encoded as UTF8
\usepackage[utf8]{inputenc}

% This is not strictly necessary, and may be commented out,
% but it will improve the layout of the manuscript,
% and will typically save some space.
\usepackage{microtype}
\usepackage{todonotes}
% This is also not strictly necessary, and may be commented out.
% However, it will improve the aesthetics of text in
% the typewriter font.
\usepackage{inconsolata}
\usepackage{graphicx}

%Including images in your LaTeX document requires adding
%additional package(s)
\usepackage{graphicx}
\usepackage{tikz}
\usepackage{listings}
\usepackage{xspace}
\usetikzlibrary{tikzmark}
\newcommand\patient[1]{\textbf{Patient:} #1}
\newcommand\doctor[1]{\textbf{Doctor:} #1}
\definecolor{darkgreen}{rgb}{0.0, 0.5, 0.0}
\newenvironment{chat} {
    \newcommand\ellipsis{\par\hangindent=0pt\hangafter=0 (\dots)\par}
    \newcommand\who[1]{\par\hangindent=2.5em\hangafter=1 ##1:}
    \par\vskip2em
}{%
    \par
    \hangindent=0pt\hangafter=0
    \vskip2em
}

\definecolor{codegreen}{rgb}{0,0.6,0}
\definecolor{codegray}{rgb}{0.5,0.5,0.5}
\definecolor{codepurple}{rgb}{0.58,0,0.82}
\definecolor{backcolour}{rgb}{0.95,0.95,0.92}
\renewcommand{\lstlistingname}{Prompt}
\lstdefinelanguage{promptlanguage}{
    morecomment=[l][\color{codepurple}]{//},
    morecomment=[s][\color{blue}]{\{\{}{\}\}},
}
\lstdefinestyle{promptstyle}{
    backgroundcolor=\color{white},   
    commentstyle=\color{codegreen},
    keywordstyle=\color{magenta},
    numberstyle=\tiny\color{codegray},
    stringstyle=\color{codepurple},
    basicstyle=\ttfamily\small,
    frame = single,
    breakatwhitespace=false,         
    breaklines=true,                 
    captionpos=b,                    
    keepspaces=true,                 
    numbers=left,          
            xleftmargin=0.5cm,
        xrightmargin=0.5cm,
    numbersep=5pt,                  
    showspaces=false,                
    showstringspaces=false,
    showtabs=false,                  
    tabsize=2
}
\lstset{style=promptstyle}

% If the title and author information does not fit in the area allocated, uncomment the following
%
%\setlength\titlebox{<dim>}
%
% and set <dim> to something 5cm or larger.
\newcommand{\methodname}{RareScale\xspace}

\newcommand{\expertsystem}{expert system\xspace}

\title{Rare Disease Differential Diagnosis with Large Language Models at Scale:\\ From Abdominal Actinomycosis to Wilson's Disease}

% Author information can be set in various styles:
% For several authors from the same institution:
% \author{Author 1 \and ... \and Author n \\
%         Address line \\ ... \\ Address line}
% if the names do not fit well on one line use
%         Author 1 \\ {\bf Author 2} \\ ... \\ {\bf Author n} \\
% For authors from different institutions:
% \author{Author 1 \\ Address line \\  ... \\ Address line
%         \And  ... \And
%         Author n \\ Address line \\ ... \\ Address line}
% To start a separate ``row'' of authors use \AND, as in
% \author{Author 1 \\ Address line \\  ... \\ Address line
%         \AND
%         Author 2 \\ Address line \\ ... \\ Address line \And
%         Author 3 \\ Address line \\ ... \\ Address line}

\author{%
  Elliot Schumacher\thanks{elliot@curai.com}
  \and
 Dhruv Naik
  \and
  Anitha Kannan\\
  Curai Health\\
}


%\author{
%  \textbf{First Author\textsuperscript{1}},
%  \textbf{Second Author\textsuperscript{1,2}},
%  \textbf{Third T. Author\textsuperscript{1}},
%  \textbf{Fourth Author\textsuperscript{1}},
%\\
%  \textbf{Fifth Author\textsuperscript{1,2}},
%  \textbf{Sixth Author\textsuperscript{1}},
%  \textbf{Seventh Author\textsuperscript{1}},
%  \textbf{Eighth Author \textsuperscript{1,2,3,4}},
%\\
%  \textbf{Ninth Author\textsuperscript{1}},
%  \textbf{Tenth Author\textsuperscript{1}},
%  \textbf{Eleventh E. Author\textsuperscript{1,2,3,4,5}},
%  \textbf{Twelfth Author\textsuperscript{1}},
%\\
%  \textbf{Thirteenth Author\textsuperscript{3}},
%  \textbf{Fourteenth F. Author\textsuperscript{2,4}},
%  \textbf{Fifteenth Author\textsuperscript{1}},
%  \textbf{Sixteenth Author\textsuperscript{1}},
%\\
%  \textbf{Seventeenth S. Author\textsuperscript{4,5}},
%  \textbf{Eighteenth Author\textsuperscript{3,4}},
%  \textbf{Nineteenth N. Author\textsuperscript{2,5}},
%  \textbf{Twentieth Author\textsuperscript{1}}
%\\
%\\
%  \textsuperscript{1}Affiliation 1,
%  \textsuperscript{2}Affiliation 2,
%  \textsuperscript{3}Affiliation 3,
%  \textsuperscript{4}Affiliation 4,
%  \textsuperscript{5}Affiliation 5
%\\
%  \small{
%    \textbf{Correspondence:} \href{mailto:email@domain}{email@domain}
%  }
%}

\begin{document}
\maketitle
\begin{abstract}
\begin{abstract}

%Large Language Models (LLMs) trained with single-turn reward often respond passively to 
%under-specified
%open-ended or ambiguous
%user requests, failing to guide users in discovering their ultimate intents. This results in ineffective and inefficient human-LLM collaborations, with users needing multiple iterations to clarify their goals. 
%To address this limitation, 
%
Large Language Models %(LLMs) 
are typically trained with next-turn rewards, limiting their ability to optimize for long-term interaction. As a result, 
they often respond passively to ambiguous or open-ended user requests, 
%failing to guide users toward discovering their ultimate intents and leading to ineffecive conversations. 
failing to help users reach their ultimate intents and leading to inefficient conversations.
To address these limitations,
%
we introduce \mbox{\name{}}, a novel and general training framework that enhances multiturn human-LLM collaboration.
%The key innovation of \name{} 
Its key innovation
is a collaborative simulation that estimates the long-term contribution of responses using  \textit{Multiturn-aware Rewards}.
%of model responses. 
By reinforcement fine-tuning these rewards, \name{} goes beyond responding to user requests, and actively uncovers user intent and offers insightful suggestions---a key step towards more human-centered AI.
% \name{} is reword-model training-free, scalable, and plug-and-play. 
% which can be integrated with prevailing reinforcement fine-tuning frameworks.
%We conduct a large-scale user study with \numturker{} judges, where \name{} increases user satisfaction by \realsatisfyimprov and reduces user time by \realtimeimprov{}.
%Additionally, 
We also devise a multiturn interaction benchmark 
%in simulated environments 
with three challenging tasks such as document creation. 
%Compared to our strongest baselines, 
\name{} significantly outperforms our 
%strongest 
baselines with averages of \taskimprov higher task performance and \itrimprov improved interactivity by LLM judges.
%% Not sure where to put this, but we could bring it back:
%Overall, \name{} advances human-centered AI by enabling more effective and efficient human-LLM collaboration.
Finally, we conduct a large user study with \numturker{} judges, where \name{} increases user satisfaction by \realsatisfyimprov and reduces user time by \realtimeimprov{}.

\end{abstract}

\begin{figure*}[t]
    \centering
    \includegraphics[width=1.0\linewidth]{figures/examples_v4}
    \vspace{-20pt}
    \caption{Real examples from \name{} and non-collaborative LLM fine-tuing. (a) Non-collaborative LLM fine-tuing relies single-turn rewards on immediate responses, which exhibits passive behaviors that follow the user's requests, leading to user frustration, less efficient process, and less satisfactory results. (b) \name{} incorporates Multiturn-aware Rewards from collaborative simulation, enabling forward-looking strategies. This results in more high-performing, efficient, and interactive conversations that anticipate future needs, propose timely clarification, and provide insightful suggestions. 
    }
    \label{fig:examples}
\end{figure*}

\end{abstract}

\section{Introduction}
\label{sec:intro}
\section{Additional Figures}

In this section, we present some additional figures that demonstrate the negative effects of random edge-dropping, particularly focusing on providing empirical evidence for scenarios not covered by the theory in \autoref{sec:theory}. Additionally, we provide visual evidence of the negative effects of DropNode, despite the fact that it preserves sensitivity between nodes (in expectation).

\subsection{Symmetrically Normalized Propagation Matrix}
\label{sec:fig-sym-norm}

\begin{figure}
    \centering
    \includegraphics[width=\linewidth]{assets/linear-gcn_symmetric_Proteins.png}
    % \includegraphics[width=0.48\linewidth]{assets/linear-gcn_symmetric_MUTAG.png}
    \caption{Entries of $\ddot{\transition}^6$, averaged after binning node-pairs by their shortest distance.}
    \label{fig:linear-gcn_symmetric}
\end{figure}

The results in \autoref{sec:linear} correspond to the use of $\hat{\adjacency} = \propagation^\asym$ for aggregating messages -- in each message passing step, only the in-degree of node $i$ is used to compute the aggregation weights of the incoming messages. In practice, however, it is more common to use the symmetrically normalized propagation matrix, $\propagation = \propagation^\sym$, which ensures that nodes with high degree do not dominate the information flow in the graph \cite{kipf2017gcn}. As in \autoref{thm:sensitivity-l-layer}, we are looking for 
$$
    \expectation_{\mask\layer{1}, \ldots, \mask\layer{L}} \sb{\prod_{l=1}^L \propagation\layer{\ell}} = \ddot{\transition}^L
$$
where $\ddot{\transition} \coloneqq \expectation_{\mathsf{DE}} [\propagation^{\sym}]$. While $\ddot{\transition}$ is analytically intractable, we can approximate it using Monte-Carlo sampling. Accordingly, we use 20 samples of $\mask$ to compute an approximation of $\ddot{\transition}$, and plot out the entries of $\ddot{\transition}^L$, as we did for $\dot{\transition}^L$ in \autoref{fig:linear-gcn_asymmetric}. The results are presented in \autoref{fig:linear-gcn_symmetric}, which shows that while the sensitivity between nearby nodes is affected to a lesser extent compared to those observed in \autoref{fig:linear-gcn_asymmetric}, that between far-off nodes is significantly reduced, same as earlier.

\subsection{Upper Bound on Expected Sensitivity}

\begin{figure}
    \centering
    \includegraphics[width=\linewidth]{assets/linear-gcn_black-extension_Proteins.png}
    % \includegraphics[width=0.48\linewidth]{assets/linear-gcn_black-extension_MUTAG.png}
    \caption{Entries of $\sum_{l=0}^6 \dot{\transition}^l$, averaged after binning node-pairs by their shortest distance.}
    \label{fig:black_extension}
\end{figure}

\citet{black2023resistance} showed that the sensitivity between any two nodes in a graph can be bounded using the sum of the powers of the propagation matrix. In \autoref{sec:nonlinear}, we extended this bound to random edge-dropping methods with independent edge masks smapled in each layer:

\begin{align*}
    \expectation_{\mask\layer{1}, \ldots, \mask\layer{L}} \sb{\norm{\frac{\partial \representation\layer{L}_i}{\partial \feature_j}}_1}
    =
    \zeta_3^{\rb{L}} \rb{\sum_{\ell=0}^L \expectation \sb{\propagation}^{\ell}}_{ij}
\end{align*}

Although this bound does not have a closed form, we can use real-world graphs to study its entries. We randomly sample 100 molecular graphs from the Proteins dataset \cite{dobson2003proteins} and plot the entries of $\sum_{l=0}^6 \dot{\transition}^\ell$ (corresponding to \inline{DropEdge}) against the shortest distance between node-pairs. The results are presented in \autoref{fig:black_extension}. We observe a polynomial decline in sensitivity as the \inline{DropEdge} probability increases, suggesting that it is unsuitable for capturing LRIs.

\subsection{Test Accuracy versus DropNode Probability}

\begin{figure}[t]
    \centering
    \subcaptionbox{Homophilic datasets}{\includegraphics[width=0.7\linewidth]{assets/DropNode_homophilic.png}} \\
    \vspace{2mm}
    \subcaptionbox{Heterophilic datasets}{\includegraphics[width=\linewidth]{assets/DropNode_heterophilic.png}}
    \caption{Dropping probability versus test accuracy of DropNode-GCN.}
    \label{fig:dropnode}
\end{figure}

In \autoref{eqn:dropnode}, we noted that the expectation of sensitivity remains unchanged when using DropNode. However, these results were only in expectation. In practice, a high DropNode probability will result in poor communication between distant nodes, preventing the model from learning to effectively model LRIs. This is supported by the results in \autoref{tab:correlation}, where we observed a negative correlation between the test accuracy and DropNode probability. Moreover, DropNode was the only algorithm which recorded negative correlations on homophilic datasets. In \autoref{fig:dropnode}, we visualize these relationships, noting the stark contrast with \autoref{fig:acc-trends}, particularly in the trends with homophilic datasets.

%\vspace{-4mm}
\section{Introduction}

In machine learning, the availability of vast amounts of unlabeled data has created an opportunity to learn meaningful representations without relying on costly labeled datasets \cite{jaiswal2020survey,shurrab2022self,jing2020self}. Self-supervised learning has emerged as a powerful solution to this problem, allowing models to leverage the inherent structure in data to build useful representations. Among self-supervised methods, contrastive learning (CL) is widely adopted for its ability to create robust representations by distinguishing between similar (positive) and dissimilar (negative) data pairs. With success in fields like image and language processing \cite{chen2020simple,radford2021learning}, contrastive learning now also shows promise in domains where cross-modal, noisy, or structurally complex data make labeling especially challenging \cite{liu2021drop,vishnubhotla2024towards,chen2023instance}.


Traditional contrastive learning methods primarily aim to bring positive pairs---often augmentations of the same sample---closer together in representation space. While effective, this approach often struggles with real-world challenges such as noise in views, variations in data quality, or shifts introduced by complex transformations, where positive pairs may not perfectly align. Additionally, in tasks requiring domain generalization, aligning representations across diverse domains (e.g., variations in style or sensor type) is critical but difficult to achieve with standard contrastive learning, which typically lacks mechanisms for incorporating domain-specific relationships. These limitations highlight the need for a more flexible approach that can adapt alignment strategies based on the data structure, allowing for finer control over similarity and dissimilarity among samples. 


To address this challenge, we introduce a novel \emph{generalized contrastive alignment} (GCA) framework, which reinterprets contrastive learning as a distributional alignment problem. Our method allows flexible control over the alignment of samples by defining a target transport plan, \(\mathbf{P}_{tgt}\), that serves as a customizable alignment guide. For example, setting \(\mathbf{P}_{tgt}\) to resemble a diagonal matrix encourages each positive to align primarily with itself or its augmentations, thereby reducing the effect of noise between views. Alternatively, we can incorporate more complex constraints, such as weighting alignments based on view quality or enforcing partial alignment structures where noise or data heterogeneity is prevalent. This flexibility enables GCA to adapt effectively to a wide range of tasks, from simple twin view alignments to scenarios with noisy or variably aligned data.

Our approach also bridges connections between GCA and established methods, such as InfoNCE (INCE) \cite{oord2018representation}, Robust InfoNCE (RINCE) \cite{chuang2022robust}, and BYOL \cite{grill2020bootstrap}, demonstrating that these can be viewed as iterative alignment objectives with Bregman projections \cite{cai2022developments,grathwohl2019your}. This perspective allows us to systematically analyze and improve uniformity within the latent space, a property that enhances representation quality and ultimately boosts downstream classification performance.

We validate our method through extensive experiments on both image classification and noisy data tasks, demonstrating that GCA’s unbalanced OT (UOT) formulations improve classification performance by relaxing our constraints on alignment. Our results show that \ours~offers a robust and versatile framework for contrastive learning, providing flexibility and performance gains over existing methods and presenting a promising approach to addressing different sources of variability in self-supervised learning.


The contributions of this work include:
\begin{itemize}
    \item A new framework called \emph{generalized contrastive alignment} (GCA), which reinterprets standard contrastive learning as a distributional alignment problem, using optimal transport to provide flexible control over alignment objectives. This approach allows us to derive a novel class of contrastive losses and algorithms that adapt effectively to varied data structures and build  customizable transport plans.

    \item We present GCA-UOT, a contrastive learning method that achieves strong performance on standard augmentation regimes and excels in scenarios with more extreme augmentations or data corrupted by transformations. GCA-UOT leverages unbalanced transport to adaptively weight positive alignments, enhancing robustness against view noise and cross-domain variations.

    \item We provide theoretical guarantees for the convergence of our GCA-based methods and show that our alignment objectives improve representation quality by enhancing the uniformity of negatives and strengthening alignment within positive pairs. This leads to more discriminative and resilient representations, even in challenging data conditions.

    \item Empirically, we demonstrate the effectiveness of \ours~in both image classification and domain generalization tasks. Through flexible, unbalanced OT-based losses, \ours~achieves superior classification performance and adapts alignment to include domain-specific information where relevant, without compromising classification accuracy in domain generalization.
\end{itemize}



\section{Problem Setup}
\label{sec:problem_statement}
\section{Problem Statement}
\label{sec:problem}

This work studies two popular models of opinion exchange on networks. The overarching goal is for a network of truthful, rational agents to learn a binary piece of information, which we call the state of the world or \emph{ground truth}. We can also think of this state as an optimal binary action (buying or selling a stock, voting for a political party's candidate, etc.). The agents are arranged on a directed graph $G=(V,E)$ and broadcast which of the two states they believe is more probable to their neighbors. 

More explicitly, we encode the ground truth in $ \theta \in \{0,1\} $, distributed according to $ \berd{q} $, Bernoulli distribution with probability of $q$ of taking $1$ and $1-q$ of taking $0$. Every agent $v \in V$ initially receives an independent \emph{private signal} $ s_v \in \{0,1\}$ correlated with the ground truth. They then announce a prediction $a_v \in \{ 0,1 \} $ of the ground truth along outgoing edges, so out-neighbors of $v$ may use $a_v$ to improve their own predictions. Importantly, the probabilities $p$ and $q$, as well as the graph $G$ are all common knowledge. 
This is captured in the following formal definition of a network.

\begin{definition}[Social Network]
    A \emph{social network} is $ \network \deq ( G,q,p ) $,
    where \begin{enumerate}
        \item $ G = ( V,E ) $ is a directed graph with agents as vertices,
        \item $ q \in ( 0,1 ) $ is the prior probability of $\theta = 1$,
        \item $ p \in ( \frac 12, 1 ) $ is the accuracy of agents' private signals $s_v \in \{0,1\}$, such that \[\pr{s_v = 1 \suchthat \theta = 1} = \pr{s_v = 0 \suchthat \theta = 0} = p, \quad \forall v \in V.
            \]
    \end{enumerate}
    We further denote $ n \deq \absolute{V} $.
\end{definition}

We consider a classic asynchronous \emph{sequential} model ~\cite{Golub2017-qo}, in which agents announce their predictions in a \emph{decision ordering}, given by a one-to-one mapping $\sigma: V \to [n]$.
We denote the set of all possible orderings by $\Sigma_n$.
At every time step $i$, agent $v = \sigma^{-1}(i)$ makes an announcement $a_v \in \{0,1\}$.
The announcement depends on the agent's private measurement $ s_v $, along with the \emph{previous announcements of in-neighbors}, which we denote as a tuple $ N_v $, defined as \[
    N_v \deq ( a_u \suchthat u \in V \land uv \in E \land \sigma(u)<\sigma(v) ).
\] 
We call the tuple $X_v = (s_v) \cup N_v$ the \emph{inputs} of node $v$. 
This setup of limiting visibility to an agent's neighborhood has been studied in a number of recent papers~\cite{Bahar2020-am,arieli2020social,lu24enabling}.

When making announcements, agents follow an \emph{aggregation rule}, which is a function $ \mu:  ( X_v, G, \sigma ) \mapsto a_v $.
Broadly speaking, aggregation rules can either be Bayesian or non-Bayesian. 
In the \emph{Bayesian} model,  agents are fully rational
and make predictions according to their posterior probability for $\theta$, given their inputs and knowledge of the network topology $G$. In particular, agents take into account the correlation between their inputs resulting from the network topology and from the current ordering $ \sigma $. \[
     \mu^B(X_v, G, \sigma) \deq \begin{cases}
         1 & \text{if $\Pr_{G,\sigma}[\theta = 1 \suchthat X_v] > \frac 12$,} \\
         0 & \text{if $\Pr_{G,\sigma}[\theta = 0 \suchthat X_v] > \frac 12$,} \\
         \berd{\frac 12} & \text{otherwise.}
     \end{cases}
 \]

We also consider a non-Bayesian model, in which agents have bounded rationality and instead use simpler heuristic rules.
This is perhaps a more practical model, as computing posterior probabilities in arbitrary networks can become computationally expensive. 
In particular, we examine the \emph{majority dynamics} model, in which agents simply follow the majority among their inputs~\cite{Bahar2020-am,Shoham1992-ir,Laland2004-ej}. Since this model does not require agents to take into account correlations between their inputs derived from the network topology or the ordering, we omit $G$ and $ \sigma $ as inputs to $\mu^M$: \[
    \mu^M(X_v) \deq \begin{cases}
        1 & \text{if $  \frac 1{\absolute{X_v}}\sum_{x \in X_v} x > \frac 12 $,} \\
        0 & \text{if $  \frac 1{\absolute{X_v}}\sum_{x \in X_v} x < \frac 12 $,} \\
        s_v & \text{otherwise.}
    \end{cases}
\]

Finally, we quantify how successful the network is in predicting the ground truth by defining the following notion of a learning rate.
\begin{definition}
    The \emph{cumulative learning rate} (CLR) of a network $ \network $ under the ordering $ \sigma $ and an aggregation rule $ \mu $ is \[
		\clr (\network, \sigma, \mu) \deq \E_{\theta, s} \left[ \sum_{v \in V}^{} \mathds{1}_{\{a_v = \theta\}} \right] = \sum_{v \in V} \Pr_{\theta, s}\left[a_v = \theta\right],
	\]
    where the equality follows from linearity of expectation.
    Further, the \emph{learning rate} (LR) of a network $ \network $ under the ordering $ \sigma $ is simply \[
		\lr (\network, \sigma, \mu) \deq \tfrac 1 n \clr (\network, \sigma, \mu).
	\]
\end{definition}

We are mainly interested in the \emph{optimal} learning rate of a network, defined as follows.

\begin{definition}[Optimal LRs]
    The \emph{optimal cumulative learning rate} of a network $ \network $ is \[
        \oclr (\network, \mu) \deq \max_{\sigma \in \Sigma_n} \clr (\network, \sigma, \mu),
    \]
    and the \emph{optimal learning rate} of a network $ \network $ is \[
        \olr (\network, \mu) \deq \max_{\sigma \in \Sigma_n} \lr (\network, \sigma, \mu).
    \]
\end{definition}

Note that when $ p $ and $ q $ are clear from the context, we use the learning rate notation with only the graph, for example $ \olr (G, \mu) = \olr(( G,p,q ), \mu) $.
We can now present a formal definition of our main focus, the \netlearnopt{} optimization problem, and \netlearn{}, its decision version.

\begin{definition}[\netlearnopt{}]
    Suppose $\mu$ is a fixed aggregation rule.
    Given a network $ \network $, the \netlearnopt{} problem is to maximize  $ \lr (\network, \sigma, \mu) $, over $ \sigma \in \Sigma_n $.
\end{definition}


\begin{definition}[\netlearn{}]\label{def:decnetlearn}
    Suppose $\mu$ is a fixed aggregation rule.
    Given a network $ \network $ and a constant threshold $ \varepsilon \in (0,1)$,
    the \netlearn{} decision problem asks whether \[
            ( \exists \sigma \in \Sigma_n )\quad \lr (\network, \sigma, \mu) \geq 1-\varepsilon.
        \]
\end{definition}

Note that \Cref{def:decnetlearn} can be formulated equivalently by asking whether an optimal ordering $\sigma^*$ which maximizes the network learning rate achieves LR at least $1-\varepsilon$.

In \Cref{sec:bayes,sec:maj}, we focus on the decision problem, offering a proof that it is \np-hard for $ \mu = \mu^B $ and $ \mu = \mu^M $.
Finally, in \Cref{sec:approx}, we use insights from the \np-hardness proofs to show \netlearnopt{} is hard to even approximate.
Surprisingly, this gives us a stronger \np-hardness statement, showing that \netlearn{} is \np-hard even if we arbitrarily fix the agents' accuracy $ p \in ( \frac 12, 1 ) $.


\documentclass{article}
\usepackage{graphicx}
\usepackage{subcaption}
\usepackage{float}
\pagenumbering{gobble}

\begin{document}
\begin{figure}
\centering
\includegraphics[width=\linewidth]{../img/School mergers figures.pdf}
% \caption{\small{Diagram outlining the basic algorithm for school mergers, showcasing Escondido Union school district (NCES ID: 0612880) in California as an example.  Plot (a) shows status quo elementary school attendance boundaries; (b) illustrates the demographic distribution of elementary students in the district (darker blue indicates higher concentration of students of color); (c) outlines the algorithm's key decision (assignment) variables, objective function, and constraints; (d) specifies the algorithm's key outputs: a map of merged school attendance boundaries, and expected impacts on segregation in the district, and expected impacts on students' travel times.}}
\label{fig:opt_model}
\end{figure}
\end{document}

\section{Corpus Simulation}\label{sec:data_gen}


The use of synthetic data from large language models has been widely studied \cite{li-etal-2023-synthetic,yu2023training,liu2024best}. However, our task requires generating history-taking conversational data with labeled differential diagnosis. Since identifying differential diagnosis is the task we are trying to solve with \methodname, we took a different approach to generate labeled history taking chats to avoid simply evaluating an LLM based on it's existing knowledge.


As shown in Figure, \ref{fig:expert_sim}, we first simulate an example using expert system knowledge.  We start with a seed disease and generate a structured patient case (represented as a list of findings). 
Such a generation can have ambiguity about the final diagnosis. Therefore, we use the same expert system to assign a full differential diagnosis. The structured case is then used to simulate a history-taking conversation as detailed in \S ~\ref{sec:chat_generation} so that LLM needs to only generate a patient-facing question with the provided finding, while maintaining the conversational flow. We repeat this process independently for all diseases. The resulting dataset of history-taking conversation and differential diagnosis pairs is used to train rare diseases candidate generation model (\S~\ref{sec:cand_gen}).

\subsection{Generation of structured cases}
\label{sec:vignette}

We use a rare disease expert system, evolved from Internist-1 \cite{miller_internist-1_1982} and QMR \cite{miller_qmr}, to generate structured cases using expert-curated diagnostic rules based on a knowledge base of diseases, findings, and their relationships. Findings include symptoms, signs, lab results, demographics, or medical history. In this study, we focus on patient-answerable findings and {\it only} focus on symptoms. Each finding-disease link is defined by \emph{evoking strength} and \emph{frequency}, scored from 1 to 5 by a team of medical experts based on the studies for that disease. \emph{Evoking strength} measures the association between a finding and a disease, while \emph{frequency} indicates how often a finding occurs in patients with the disease. Each finding also has a disease-independent \emph{import} variable, indicating its global importance.

Our simulation algorithm, adapted from medical case simulator \citet{qmr-simulated-cases, pmlr-v85-ravuri18a}, starts by sampling a seed disease and sequentially constructing a set of findings. First, demographic variables are sampled, followed by predisposing factors, and then other findings are made to decrease frequency relative to the disease. Each finding is randomly determined to be present or absent, with impossible findings excluded and high co-occurrence findings prioritized. The simulation concludes once all findings in the knowledge base for that disease are considered, operating under a closed-world assumption that limits diseases and findings to those in the knowledge base.  Unlike previous approaches to simulation, we also use the expert system to compute differential diagnosis after the first six findings are sampled.  We use this to consider findings from other diagnoses in the differential diagnosis to prioritize sampling additional negative findings that overlap with the seed disease. This allows the sample to be slightly more targeted at the seed disease. In turn, we widen the gap between the seed disease and the remainder of the differential diagnosis.

Each simulation includes a differential diagnosis of up to 5 diseases as ranked by expert system scoring. A DDx may consist of multiple diseases in cases where a final diagnosis may not be ascertained without laboratory testing. We maintain this differential diagnosis list for training in \S \ref{sec:cand_gen}. We iterate through 630 rare diseases in the expert system, using each as a seed disease. We keep the generated structured case if the seed disease is top-scoring in the differential to reduce the effect of the noise in the simulation process. We set a minimum of $50$ valid simulations out of $200$ attempts. Any diseases falling below that are excluded (\textit{e.g.} Friedreich's ataxia), as this typically indicates that the disease is unidentifiable from symptoms alone. This results in 575 diseases in our dataset. We include an example in Appendix \S \ref{fig:chat_simulation_ex_app}.

\subsection{Generation of chats from structured cases}
\label{sec:chat_generation}
For each (structured case, DDx) pair, we start with creating a complete demographic profile.  This includes name, gender, age, race, education, and location by first selecting from the structured case.  If not present in the case, we randomly generate from the LLM. We incorporate location and education to introduce additional variability in patient language. This diversifies the profiles and reduces sensitivity to names and locations across different simulations, which could lead to misleading patterns.

We use the structured case along with the expanded demographic profile to anchor LLM-powered history-taking chat simulation. We prompt an LLM to create a conversation that closely adheres to that case.
The LLM-based chat simulator has access to the disease name, finding set, and demographic profile.  For each finding, we include a definition if the expert system provides one so the LLM's understanding of the term aligns with the expert system.  For each disease, we persist a database of messages that correspond to each finding (\textit{e.g.} ``I feel hot'' corresponds to \textit{fever (present)}). We include the previous message in the prompt for each subsequent generation and ask the LLM to generate a different one so it does not repeat.

We enforce a format for the chat, which begins with a "system" message with the patient's demographic information. The provider simulator starts the conversation open-ended, and the patient simulator reports the symptoms drawn from the findings set. For each patient utterance, the LLM outputs both the message and which findings are included.  This is to encourage the message to be consistent with the findings. We prompt LLM  for the patient simulator to report only at most, three findings 
in a single message and also prompt the LLM  for the provider simulator not to generate leading questions.

For the simulation, we use two different LLMs to generate our chats -- OpenAI's Gpt-4o and Anthropic's Claude-3.5-sonnet -- and use the same approach unless noted.  All components of the generation pipeline use only one of the LLMs. For chats generated by GPT-4o, we found that doing this process in a single prompt is sufficient (see Appendix Prompt \ref{prompt:chat_gpt4o}).  For Claude, we found that this generated chats which were very terse.  Therefore, we prompt Claude with the same information and ask it to generate one patient-provider turn at a time  (see Appendix Prompt \ref{prompt:chat_claude}).  The findings provided to Claude are separated into findings already included and those that need to be added.  

For chats generated by both LLMs, we finally run a checker prompt that ensures that the resulting chats include all symptoms in the finding set and edits the chat if not all are included.  If the first edit fails, we try two more times.  If all other attempts fail, we exclude this chat from our dataset.  This results in a corpus of rare disease chats -- 28,589 using GPT-4o and 14,573 using Claude (further statistics are included in Appendix Table \ref{tab:data_details}).  We include a full example chat using GPT-4o in Appendix Figure \ref{fig:example_gpt4o} and using Claude in Figure \ref{fig:example_claude}. 


\section{Rare Disease Differential Diagnoses}\label{sec:cand_gen}
The \methodname scalable synthetic generation pipeline provides us with a sizable dataset of 575 rare disease history-taking conversations.  However, our expert system doesn't provide information on a variety of common diseases, such as the Common Cold.  A real-world diagnostic system will need to account for these diseases.  Given the high diagnostic performance of many LLMs \cite{generalistMedicalLangaugeModel, Rutledge2024DiagnosticAO,zhou2024largelanguagemodelsdisease,tu2024conversationaldiagnosticai}, it would be challenging to meet their performance even with a corpus of similar non-rare chats.

We therefore combine existing LLM diagnostic performance with a specialized rare-disease model.  To do this, we train a rare disease candidate generation system which proposes a set of at most 5 rare diseases.  This list is recall-oriented, to allow several possibilities to be considered.  These rare disease candidates are then passed through a prompt of a larger LLM to produce a final diagnostic list.  The combination can leverage the niche expertise of an expert system while also leveraging the broader knowledge of general LLMs.

\subsection{Candidate Generation} \label{sec:candidate_gen_training}
To generate rare disease candidates for a given history-taking conversation, we train a smaller LLM to generate a list of at most 5 rare diseases. We train using Llama 3.1 7b as our base model \cite{grattafiori2024llama3herdmodels} using the chat text as input (training details in  Appendix \S \ref{sec:training_details}).   We exclude the structured findings and all other information used in the previous section. Note that these history-taking chats do not include a final diagnosis from the provider, so the model must make inferences from the chat alone. As the target output, we use the differential diagnosis list produced by our expert systems which contains up to five diseases.  We only include the disease names ordered by likelihood per the expert system score but do not include the scores themselves.



\subsection{Diagnosis Generation}\label{sec:ddx_gen}

For the final step of generating the differential diagnosis for the conversation,  we use a larger black-box general LLMs\footnote{We explored using the publicly-available MEDITRON-70B\cite{chen2023meditron70bscalingmedicalpretraining}, but it could not follow the DDx instructions in the prompt, instead outputting unrelated text.} (gpt-4o, claude, llama 3.3 70b) to fuse common and rare diseases seamlessly. We leave the role of identifying the relevant common diseases to the LLM, but infuse the candidate rare diseases through inferencing on the smaller model we trained according to \S~\ref{sec:candidate_gen_training}

The prompt (Appendix Prompt \ref{prompt:ddx})  uses multi-stage instructions, instructing the LLM to generate a list of 5 diagnoses without considering the rare list.  Then, it is instructed to consider the rare disease candidate list. Finally, it selects whichever from the two sections is most appropriate, including discarding all rare candidates if best. The prompt is also flexible, so we can remove instructions for incorporating candidate rare diseases or include other means of candidate rare diseases when available. We study these variations in \S~\ref{sec:eval_ddx}.


\section{Evaluation and Results}
% 

\section{Implementation and Evaluation}
\label{sec:evaluation}

We prototype our proposal into a tool \toolName, using approximately 5K lines of OCaml (for the program analysis) and 5K lines of Python code (for the repair). 
In particular, we employ Z3~\cite{DBLP:conf/tacas/MouraB08} as the SMT solver, clingo~\cite{DBLP:books/sp/Lifschitz19} as the ASP solver, and Souffle~\cite{scholz2016fast} as the Datalog engine. %, respectively.
To show the effectiveness, 
we design the experimental evaluation to answer the 
following research questions (RQ):
(Experiments ran on a server with an Intel® Xeon® Platinum 8468V, 504GB RAM, and 192 cores. All the dataset are publicly available from \cite{zenodo_benchmark})

\begin{itemize}[align=left, leftmargin=*,labelindent=0pt]
\item \textbf{RQ1:} How effective is \toolName in verifying CTL properties for relatively small but complex programs, compared to the state-of-the-art tool  \function~\cite{DBLP:conf/sas/UrbanU018}?


\item \textbf{RQ2:} What is the effectiveness of \toolName in detecting real-world bugs, which can be encoded using both CTL and linear temporal logic (LTL), such as non-termination gathered from GitHub \cite{DBLP:conf/sigsoft/ShiXLZCL22} and unresponsive behaviours in protocols  \cite{DBLP:conf/icse/MengDLBR22}, compared with \ultimate~\cite{DBLP:conf/cav/DietschHLP15}?

\item \textbf{RQ3:} How effective is \toolName in repairing CTL violations identified in RQ1 and RQ2? which has not been achieved by any existing tools. 


 

\end{itemize}



% \begin{itemize}[align=left, leftmargin=*,labelindent=0pt]
% \item \textbf{RQ1:} What is the effectiveness of \toolName in verifying CTL properties in a set of relatively small yet challenging programs, compared to the state-of-the-art tools, T2~\cite{DBLP:conf/fmcad/CookKP14},  \function~\cite{DBLP:conf/sas/UrbanU018}, and \ultimate~\cite{DBLP:conf/cav/DietschHLP15}?


% \item \textbf{RQ2:} What is the effectiveness of \toolName in finding  real-world bugs, which can be encoded using CTL properties, such as non-termination 
% gathered from GitHub \cite{DBLP:conf/sigsoft/ShiXLZCL22} and unresponsive behaviours in protocol implementations \cite{DBLP:conf/icse/MengDLBR22}?

% \item \textbf{RQ3:} What is the effectiveness of \toolName in repairing CTL bugs from RQ1--2?

% \end{itemize}

%The benchmark programs are from various sources. More specifically, termination bugs from real-world projects \cite{DBLP:conf/sigsoft/ShiXLZCL22} and CTL analysis \cite{DBLP:conf/fmcad/CookKP14} \cite{DBLP:conf/sas/UrbanU018}, and temporal bugs in real-world protocol implementations \cite{DBLP:conf/icse/MengDLBR22}. 



% \ly{are termination bugs ok? Do we need to add new CTL bugs?}
\subsection{RQ1: Verifying CTL Properties}

% Please add the following required packages to your document preamble:
%  \Xhline{1.5\arrayrulewidth}

\hide{\begin{figure}[!h]
\vspace{-8mm}
\begin{lstlisting}[xleftmargin=0.2em,numbersep=6pt,basicstyle=\footnotesize\ttfamily]
(*@\textcolor{mGray}{//$EF(\m{resp}{\geq}5)$}@*)
int c = *; int resp = 0;
int curr_serv = 5; 
while (curr_serv > 0){ 
 if (*) {  
   c--; 
   curr_serv--;
   resp++;} 
 else if (c<curr_serv){
   curr_serv--; }}
\end{lstlisting} 
\vspace{-2mm}
\caption{A possibly terminating loop} 
\label{fig:terminating_loop}
\vspace{-2mm}
\end{figure}}


%loses precision due to a \emph{dual widening} \cite{DBLP:conf/tacas/CourantU17}, and 

The programs listed in \tabref{tab:comparewithFuntionT2} were obtained from the evaluation benchmark of \function, which includes a total of 83 test cases across over 2,000 lines of code. We categorize these test cases into six groups, labeled according to the types of CTL properties. 
These programs are short but challenging, as they often involve complex loops or require a more precise analysis of the target properties. The \function tends to be conservative, often leading it to return ``unknown" results, resulting in an accuracy rate of 27.7\%. In contrast, \toolName demonstrates advantages with improved accuracy, particularly in \ourToolSmallBenchmark. 
%achieved by the novel loop summaries. 
The failure cases faced by \toolName highlight our limitations when loop guards are not explicitly defined or when LRFs are inadequate to prove termination. 
Although both \function and \toolName struggle to obtain meaningful invariances for infinite loops, the benefits of our loop summaries become more apparent when proving properties related to termination, such as reachability and responsiveness.  




\begin{table}[!t]
\vspace{1.5mm}
\caption{Detecting real-world CTL bugs.}
\normalsize
\label{tab:comparewithCook}
\renewcommand{\arraystretch}{0.95}
\setlength{\tabcolsep}{4pt}  
\begin{tabular}{c|l|c|cc|cc}
\Xhline{1.5\arrayrulewidth}
\multicolumn{1}{l|}{\multirow{2}{*}{\textbf{}}} & \multirow{2}{*}{\textbf{Program}}        & \multirow{2}{*}{\textbf{LoC}} & \multicolumn{2}{c|}{\textbf{\ultimateshort}}   & \multicolumn{2}{c}{\textbf{\toolName}}             \\ \cline{4-7} 
  \multicolumn{1}{l|}{}                           &                                          &                               & \multicolumn{1}{c|}{\textbf{Res.}} & \textbf{Time} & \multicolumn{1}{c|}{\textbf{Res.}} & \textbf{Time} \\ \hline
  1 \xmark                                      & \multirow{2}{*}{\makecell[l]{libvncserver\\(c311535)}}   & 25                            & \multicolumn{1}{c|}{\xmark}      & 2.845         & \multicolumn{1}{c|}{\xmark}      & 0.855         \\  
  1 \cmark                                      &                                          & 27                            & \multicolumn{1}{c|}{\cmark}      & 3.743         & \multicolumn{1}{c|}{\cmark}      & 0.476         \\ \hline
  2 \xmark                                      & \multirow{2}{*}{\makecell[l]{Ffmpeg\\(a6cba06)}}         & 40                            & \multicolumn{1}{c|}{\xmark}      & 15.254        & \multicolumn{1}{c|}{\xmark}      & 0.606         \\  
  2 \cmark                                      &                                          & 44                            & \multicolumn{1}{c|}{\cmark}      & 40.176        & \multicolumn{1}{c|}{\cmark}      & 0.397         \\ \hline
  3 \xmark                                      & \multirow{2}{*}{\makecell[l]{cmus\\(d5396e4)}}           & 87                            & \multicolumn{1}{c|}{\xmark}      & 6.904         & \multicolumn{1}{c|}{\xmark}      & 0.579         \\  
  3 \cmark                                      &                                          & 86                            & \multicolumn{1}{c|}{\cmark}      & 33.572        & \multicolumn{1}{c|}{\cmark}      & 0.986         \\ \hline
  4 \xmark                                      & \multirow{2}{*}{\makecell[l]{e2fsprogs\\(caa6003)}}      & 58                            & \multicolumn{1}{c|}{\xmark}      & 5.952         & \multicolumn{1}{c|}{\xmark}      & 0.923         \\  
  4 \cmark                                      &                                          & 63                            & \multicolumn{1}{c|}{\cmark}      & 4.533         & \multicolumn{1}{c|}{\cmark}      & 0.842         \\ \hline
  5 \xmark                                      & \multirow{2}{*}{\makecell[l]{csound-an...\\(7a611ab)}} & 43                            & \multicolumn{1}{c|}{\xmark}      & 3.654         & \multicolumn{1}{c|}{\xmark}      & 0.782         \\  
  5 \cmark                                      &                                          & 45                            & \multicolumn{1}{c|}{TO}          & -             & \multicolumn{1}{c|}{\cmark}      & 0.648         \\ \hline
  6 \xmark                                      & \multirow{2}{*}{\makecell[l]{fontconfig\\(fa741cd)}}     & 25                            & \multicolumn{1}{c|}{\xmark}      & 3.856         & \multicolumn{1}{c|}{\xmark}      & 0.769         \\  
  6 \cmark                                      &                                          & 25                            & \multicolumn{1}{c|}{Error}       & -             & \multicolumn{1}{c|}{\cmark}      & 0.651         \\ \hline
  7 \xmark                                      & \multirow{2}{*}{\makecell[l]{asterisk\\(3322180)}}       & 22                            & \multicolumn{1}{c|}{\unk}        & 12.687        & \multicolumn{1}{c|}{\unk}        & 0.196         \\  
  7 \cmark                                      &                                          & 25                            & \multicolumn{1}{c|}{\unk}        & 11.325        & \multicolumn{1}{c|}{\unk}        & 0.34          \\ \hline
  8 \xmark                                      & \multirow{2}{*}{\makecell[l]{dpdk\\(cd64eeac)}}          & 45                            & \multicolumn{1}{c|}{\xmark}      & 3.712         & \multicolumn{1}{c|}{\xmark}      & 0.447         \\  
  8 \cmark                                      &                                          & 45                            & \multicolumn{1}{c|}{\cmark}      & 2.97          & \multicolumn{1}{c|}{\unk}        & 0.481         \\ \hline
  9 \xmark                                      & \multirow{2}{*}{\makecell[l]{xorg-server\\(930b9a06)}}   & 19                            & \multicolumn{1}{c|}{\xmark}      & 3.111         & \multicolumn{1}{c|}{\xmark}      & 0.581         \\  
  9 \cmark                                      &                                          & 20                            & \multicolumn{1}{c|}{\cmark}      & 3.101         & \multicolumn{1}{c|}{\cmark}      & 0.409         \\ \hline
  10 \xmark                                      & \multirow{2}{*}{\makecell[l]{pure-ftpd\\(37ad222)}}      & 42                            & \multicolumn{1}{c|}{\cmark}      & 2.555         & \multicolumn{1}{c|}{\xmark}      & 0.933         \\  
  10 \cmark                                      &                                          & 49                            & \multicolumn{1}{c|}{\cmark}        & 2.286         & \multicolumn{1}{c|}{\cmark}      & 0.383         \\ \hline
  11 \xmark  & \multirow{2}{*}{\makecell[l]{live555$_a$\\(181126)}} & 34  & \multicolumn{1}{c|}{\cmark} &  2.715         & \multicolumn{1}{c|}{\xmark}    & 0.513   \\  
  11 \cmark  &     &   37    & \multicolumn{1}{c|}{\cmark} &  2.837         & \multicolumn{1}{c|}{\cmark}      & 0.341 \\ \hline
  12 \xmark  & \multirow{2}{*}{\makecell[l]{openssl\\(b8d2439)}} & 88  & \multicolumn{1}{c|}{\xmark} &  4.15          & \multicolumn{1}{c|}{\xmark}    & 0.78   \\
  12 \cmark  &     &  88     & \multicolumn{1}{c|}{\cmark} &  3.809         & \multicolumn{1}{c|}{\cmark}      & 0.99 \\ \hline
  13 \xmark  & \multirow{2}{*}{\makecell[l]{live555$_b$\\(131205)}} & 83  & \multicolumn{1}{c|}{\xmark} & 2.838         & \multicolumn{1}{c|}{\xmark}    & 0.602     \\  
  13 \cmark  &    &   84     & \multicolumn{1}{c|}{\cmark} &  2.393         & \multicolumn{1}{c|}{\cmark}      & 0.565 \\ \Xhline{1.5\arrayrulewidth}
                                                   & {\bf{Total}}                                  & 1249  & \multicolumn{1}{c|}{\bestBaseLineReal}          & $>$180       & \multicolumn{1}{c|}{\ourToolRealBenchmark}              & 16.01        \\ \Xhline{1.5\arrayrulewidth}
  \end{tabular}
  \end{table}

\subsection{RQ2: CTL Analysis on  Real-world Projects}




Programs in \tabref{tab:comparewithCook} are from real-world repositories, each associated with a Git commit number where developers identify and fix the bug manually. 
In particular, the property used for programs 1-9 (drawn from \cite{DBLP:conf/sigsoft/ShiXLZCL22}) is  \code{AF(Exit())}, capturing non-termination bugs. The properties used for programs 10-13 (drawn from \cite{DBLP:conf/icse/MengDLBR22}) are of the form \code{AG(\phi_1{\rightarrow}AF(\phi_2))}, capturing unresponsive behaviours from the protocol implementation. 
We extracted the main segments of these real-world bugs into smaller programs (under 100 LoC each), preserving features like data structures and pointer arithmetic. Our evaluation includes both buggy (\eg 1\,\xmark) and developer-fixed (\eg 1\,\cmark) versions.
After converting the CTL properties to LTL formulas, we compared our tool with the latest release of UltimateLTL (v0.2.4), a regular participant in SV-COMP \cite{svcomp} with competitive performance. 
Both tools demonstrate high accuracy in bug detection, while \ultimateshort often requires longer processing time. 
This experiment indicates that LRFs can effectively handle commonly seen real-world loops, and \toolName performs a more lightweight summary computation without compromising accuracy. 



%Following the convention in \cite{DBLP:conf/sigsoft/ShiXLZCL22}, t
%Prior works \cite{DBLP:conf/sigsoft/ShiXLZCL22} gathered such examples by extracting 
%\toolName successfully identifies the majority of buggy and correct programs, with the exception of programs 7 and 8. 







{
\begin{table*}[!h]
  \centering
\caption{\label{tab:repair_benchmark}
{Experimental results for repairing CTL bugs. Time spent by the ASP solver is separately recorded. 
}
}
\small
\renewcommand{\arraystretch}{0.95}
  \setlength{\tabcolsep}{9pt}
\begin{tabular}{l|c|c|c|c|c|c|c|c}
  \Xhline{1.5\arrayrulewidth}
  \multicolumn{1}{c|}{\multirow{2}{*}{\textbf{Program}}} & \multicolumn{1}{c|}{\multirow{2}{*}{\shortstack{\textbf{LoC}\\\textbf{(Datalog)}}}} & \multicolumn{3}{c|}{\textbf{Configuration}}                                 & \multicolumn{1}{c|}{\multirow{2}{*}{\textbf{Fixed}}} & \multicolumn{1}{c|}{\multirow{2}{*}{\textbf{\#Patch}}} & \multicolumn{1}{c|}{\multirow{2}{*}{\textbf{ASP(s)}}} & \multirow{2}{*}{\textbf{Total(s)}} \\ \cline{3-5}

  \multicolumn{1}{c|}{}                                  & \multicolumn{1}{c|}{}                              & \multicolumn{1}{c|}{\textbf{Symbols}} & \multicolumn{1}{c|}{\textbf{Facts}} & \multicolumn{1}{c|}{\textbf{Template}} & \multicolumn{1}{c|}{} & \multicolumn{1}{c|}{} & \multicolumn{1}{c|}{}  &                                      \\ \hline

AF\_yEQ5 (\figref{fig:first_Example})                                           & 115                           & 3+0                   & 0+1                & Add                & \cmark     & 1                   & 0.979                              & 1.593                                \\
test\_until.c                                         & 101                            & 0+3                   & 1+0                & Delete                & \cmark     & 1                   & 0.023                              & 0.498                                \\
next.c                                                & 87                            & 0+4                   & 1+0                & Delete                & \cmark     & 1                   & 0.023                              & 0.472                                \\
libvncserver                                          & 118                            & 0+6                   & 1+0                & Delete                & \cmark     & 3                   & 0.049                              & 1.081                                \\
Ffmpeg                                                & 227                           & 0+12                  & 1+0                & Delete                & \cmark     & 4                   & 13.113                              & 13.335                                \\
cmus                                                  & 145                           & 0+12                  & 1+0                & Delete                & \cmark     & 4                   & 0.098                              & 2.052                                \\
e2fsprogs                                             & 109                           & 0+8                   & 1+0                & Delete                & \cmark     & 2                   & 0.075                              & 1.515                                \\
csound-android                                        & 183                           & 0+8                   & 1+0                & Delete                & \cmark     & 4                   & 0.076                              & 1.613                                \\
fontconfig                                            & 190                           & 0+11                  & 1+0                & Delete                & \cmark     & 6                   & 0.098                              & 2.507                                \\
dpdk                                                  & 196                           & 0+12                  & 1+0                & Delete                & \cmark     & 1                   & 0.091                              & 2.006                                \\
xorg-server                                           & 118                            & 0+2                   & 1+0                & Delete                & \cmark     & 2                   & 0.026                              & 0.605                                \\
pure-ftpd                                             & 258                           & 0+21                  & 1+0                & Delete                & \cmark     & 2                   & 0.069                              & 3.590                               \\
live$_a$                                              & 112                            & 3+4                   & 1+1                & Update                & \cmark     & 1                   & 0.552                              & 0.816                                \\
openssl                                               & 315                           & 1+0                   & 0+1                & Add.                & \cmark     & 1                   & 1.188                              & 2.277                                \\
live$_b$                                              & 217                           & 1+0                   & 0+1                & Add                & \cmark     & 1                   & 0.977                              & 1.494                                 \\
  \Xhline{1.5\arrayrulewidth}
\textbf{Total}                                                 & 2491                          &                       &                    &                   &           &                     & 17.437                              & 35.454                               \\ 
  \Xhline{1.5\arrayrulewidth}           
\end{tabular}

\vspace{-2mm}
\end{table*}
}


\subsection{RQ3: Repairing CTL Property Violations} 


\tabref{tab:repair_benchmark} gathers all the program instances (from \tabref{tab:comparewithFuntionT2} and \tabref{tab:comparewithCook}) that violate their specified CTL properties and are sent to \toolName for repair.   
The \textbf{Symbols} column records the number of symbolic constants + symbolic signs, while the \textbf{Facts} column records the number of facts allowed to be removed + added. 
We gradually increase the number of symbols and the maximum number of facts that can be added or deleted. 
The \textbf{Configuration} column shows the first successful configuration that led to finding patches, and we record the total searching time till reaching such configurations. 
We configure \toolName to apply three atomic templates in a breadth-first manner with a depth limit of 1, \ie, \tabref{tab:repair_benchmark} records the patch result after one iteration of the repair. 
The templates are applied sequentially in the order: delete, update, and add. The repair process stops when a correct patch is found or when all three templates have been attempted. 
%without success. 
% Because of this configuration, \toolName only finds one patch for Program 1 (AF\_yEQ5). 
% The patch inserting \plaincode{if (i>10||x==y) \{y=5; return;\}} mentioned in \figref{fig:Patched-program} cannot be found in current configuration, as it requires deleting facts then adding new facts on the updated program.
% The `Configuration' column in \tabref{tab:repair_benchmark} shows the number of symbolic constants and signs, the number of facts allowed to be removed and added, and the template used when a patch is found.

Due to the current configuration, \toolName only finds patch (b) for Program 1 (AF\_yEQ5), while the patch (a) shown in \figref{fig:Patched-program} can be obtained by allowing two iterations of the repair: the first iteration adds the conditional than a second iteration to add a new assignment on the updated program. 
Non-termination bugs are resolved within a single iteration by adding a conditional statement that provides an earlier exit. 
For instance, \figref{fig:term-Patched-program} illustrates the main logic of 1\,\xmark, which enters an infinite loop when \code{\m{linesToRead}{\leq}0}. 
\toolName successfully 
provides a fix that prevents \code{\m{linesToRead}{\leq}0} from occurring before entering the loop. Note that such patches are more desirable which fix the non-termination bug without dropping the loops completely. 
%much like the example shown in  \figref{fig:term-Patched-program}. At the same time, 
Unresponsive bugs involve adding more function calls or assignment modifications. 
%Most repairs were completed within seconds. 

On average, the time taken to solve ASP accounts for 49.2\% (17.437/35.454) of the total repair time. We also keep track of the number of patches that successfully eliminate the CTL violations. More than one patch is available for non-termination bugs, as some patches exit the entire program without entering the loop. 
While all the patches listed are valid, those that intend to cut off the main program logic can be excluded based on the minimum change criteria. 
After a manual inspection of each buggy program shown in \tabref{tab:repair_benchmark}, we confirmed that at least one generated patch is semantically equivalent to the fix provided by the developer. 
As the first tool to achieve automated repair of CTL violations, \toolName successfully resolves all reported bugs. 



\begin{figure}[!t]
\begin{lstlisting}[xleftmargin=6em,numbersep=6pt,basicstyle=\footnotesize\ttfamily]
void main(){ //AF(Exit())
  int lines ToRead = *;
  int h = *;
  (*@\repaircode{if ( linesToRead <= 0 )  return;}@*)
  while(h>0){
    if(linesToRead>h)  
        linesToRead=h; 
    h-=linesToRead;} 
  return;}
\end{lstlisting}
\caption{Fixing a Possible Hang Found in libvncserver \cite{LibVNCClient}}
\label{fig:term-Patched-program}
\end{figure}


% \section{Results}
% 
\begin{table*}[t]
\centering
\fontsize{11pt}{11pt}\selectfont
\begin{tabular}{lllllllllllll}
\toprule
\multicolumn{1}{c}{\textbf{task}} & \multicolumn{2}{c}{\textbf{Mir}} & \multicolumn{2}{c}{\textbf{Lai}} & \multicolumn{2}{c}{\textbf{Ziegen.}} & \multicolumn{2}{c}{\textbf{Cao}} & \multicolumn{2}{c}{\textbf{Alva-Man.}} & \multicolumn{1}{c}{\textbf{avg.}} & \textbf{\begin{tabular}[c]{@{}l@{}}avg.\\ rank\end{tabular}} \\
\multicolumn{1}{c}{\textbf{metrics}} & \multicolumn{1}{c}{\textbf{cor.}} & \multicolumn{1}{c}{\textbf{p-v.}} & \multicolumn{1}{c}{\textbf{cor.}} & \multicolumn{1}{c}{\textbf{p-v.}} & \multicolumn{1}{c}{\textbf{cor.}} & \multicolumn{1}{c}{\textbf{p-v.}} & \multicolumn{1}{c}{\textbf{cor.}} & \multicolumn{1}{c}{\textbf{p-v.}} & \multicolumn{1}{c}{\textbf{cor.}} & \multicolumn{1}{c}{\textbf{p-v.}} &  &  \\ \midrule
\textbf{S-Bleu} & 0.50 & 0.0 & 0.47 & 0.0 & 0.59 & 0.0 & 0.58 & 0.0 & 0.68 & 0.0 & 0.57 & 5.8 \\
\textbf{R-Bleu} & -- & -- & 0.27 & 0.0 & 0.30 & 0.0 & -- & -- & -- & -- & - &  \\
\textbf{S-Meteor} & 0.49 & 0.0 & 0.48 & 0.0 & 0.61 & 0.0 & 0.57 & 0.0 & 0.64 & 0.0 & 0.56 & 6.1 \\
\textbf{R-Meteor} & -- & -- & 0.34 & 0.0 & 0.26 & 0.0 & -- & -- & -- & -- & - &  \\
\textbf{S-Bertscore} & \textbf{0.53} & 0.0 & {\ul 0.80} & 0.0 & \textbf{0.70} & 0.0 & {\ul 0.66} & 0.0 & {\ul0.78} & 0.0 & \textbf{0.69} & \textbf{1.7} \\
\textbf{R-Bertscore} & -- & -- & 0.51 & 0.0 & 0.38 & 0.0 & -- & -- & -- & -- & - &  \\
\textbf{S-Bleurt} & {\ul 0.52} & 0.0 & {\ul 0.80} & 0.0 & 0.60 & 0.0 & \textbf{0.70} & 0.0 & \textbf{0.80} & 0.0 & {\ul 0.68} & {\ul 2.3} \\
\textbf{R-Bleurt} & -- & -- & 0.59 & 0.0 & -0.05 & 0.13 & -- & -- & -- & -- & - &  \\
\textbf{S-Cosine} & 0.51 & 0.0 & 0.69 & 0.0 & {\ul 0.62} & 0.0 & 0.61 & 0.0 & 0.65 & 0.0 & 0.62 & 4.4 \\
\textbf{R-Cosine} & -- & -- & 0.40 & 0.0 & 0.29 & 0.0 & -- & -- & -- & -- & - & \\ \midrule
\textbf{QuestEval} & 0.23 & 0.0 & 0.25 & 0.0 & 0.49 & 0.0 & 0.47 & 0.0 & 0.62 & 0.0 & 0.41 & 9.0 \\
\textbf{LLaMa3} & 0.36 & 0.0 & \textbf{0.84} & 0.0 & {\ul{0.62}} & 0.0 & 0.61 & 0.0 &  0.76 & 0.0 & 0.64 & 3.6 \\
\textbf{our (3b)} & 0.49 & 0.0 & 0.73 & 0.0 & 0.54 & 0.0 & 0.53 & 0.0 & 0.7 & 0.0 & 0.60 & 5.8 \\
\textbf{our (8b)} & 0.48 & 0.0 & 0.73 & 0.0 & 0.52 & 0.0 & 0.53 & 0.0 & 0.7 & 0.0 & 0.59 & 6.3 \\  \bottomrule
\end{tabular}
\caption{Pearson correlation on human evaluation on system output. `R-': reference-based. `S-': source-based.}
\label{tab:sys}
\end{table*}



\begin{table}%[]
\centering
\fontsize{11pt}{11pt}\selectfont
\begin{tabular}{llllll}
\toprule
\multicolumn{1}{c}{\textbf{task}} & \multicolumn{1}{c}{\textbf{Lai}} & \multicolumn{1}{c}{\textbf{Zei.}} & \multicolumn{1}{c}{\textbf{Scia.}} & \textbf{} & \textbf{} \\ 
\multicolumn{1}{c}{\textbf{metrics}} & \multicolumn{1}{c}{\textbf{cor.}} & \multicolumn{1}{c}{\textbf{cor.}} & \multicolumn{1}{c}{\textbf{cor.}} & \textbf{avg.} & \textbf{\begin{tabular}[c]{@{}l@{}}avg.\\ rank\end{tabular}} \\ \midrule
\textbf{S-Bleu} & 0.40 & 0.40 & 0.19* & 0.33 & 7.67 \\
\textbf{S-Meteor} & 0.41 & 0.42 & 0.16* & 0.33 & 7.33 \\
\textbf{S-BertS.} & {\ul0.58} & 0.47 & 0.31 & 0.45 & 3.67 \\
\textbf{S-Bleurt} & 0.45 & {\ul 0.54} & {\ul 0.37} & 0.45 & {\ul 3.33} \\
\textbf{S-Cosine} & 0.56 & 0.52 & 0.3 & {\ul 0.46} & {\ul 3.33} \\ \midrule
\textbf{QuestE.} & 0.27 & 0.35 & 0.06* & 0.23 & 9.00 \\
\textbf{LlaMA3} & \textbf{0.6} & \textbf{0.67} & \textbf{0.51} & \textbf{0.59} & \textbf{1.0} \\
\textbf{Our (3b)} & 0.51 & 0.49 & 0.23* & 0.39 & 4.83 \\
\textbf{Our (8b)} & 0.52 & 0.49 & 0.22* & 0.43 & 4.83 \\ \bottomrule
\end{tabular}
\caption{Pearson correlation on human ratings on reference output. *not significant; we cannot reject the null hypothesis of zero correlation}
\label{tab:ref}
\end{table}


\begin{table*}%[]
\centering
\fontsize{11pt}{11pt}\selectfont
\begin{tabular}{lllllllll}
\toprule
\textbf{task} & \multicolumn{1}{c}{\textbf{ALL}} & \multicolumn{1}{c}{\textbf{sentiment}} & \multicolumn{1}{c}{\textbf{detoxify}} & \multicolumn{1}{c}{\textbf{catchy}} & \multicolumn{1}{c}{\textbf{polite}} & \multicolumn{1}{c}{\textbf{persuasive}} & \multicolumn{1}{c}{\textbf{formal}} & \textbf{\begin{tabular}[c]{@{}l@{}}avg. \\ rank\end{tabular}} \\
\textbf{metrics} & \multicolumn{1}{c}{\textbf{cor.}} & \multicolumn{1}{c}{\textbf{cor.}} & \multicolumn{1}{c}{\textbf{cor.}} & \multicolumn{1}{c}{\textbf{cor.}} & \multicolumn{1}{c}{\textbf{cor.}} & \multicolumn{1}{c}{\textbf{cor.}} & \multicolumn{1}{c}{\textbf{cor.}} &  \\ \midrule
\textbf{S-Bleu} & -0.17 & -0.82 & -0.45 & -0.12* & -0.1* & -0.05 & -0.21 & 8.42 \\
\textbf{R-Bleu} & - & -0.5 & -0.45 &  &  &  &  &  \\
\textbf{S-Meteor} & -0.07* & -0.55 & -0.4 & -0.01* & 0.1* & -0.16 & -0.04* & 7.67 \\
\textbf{R-Meteor} & - & -0.17* & -0.39 & - & - & - & - & - \\
\textbf{S-BertScore} & 0.11 & -0.38 & -0.07* & -0.17* & 0.28 & 0.12 & 0.25 & 6.0 \\
\textbf{R-BertScore} & - & -0.02* & -0.21* & - & - & - & - & - \\
\textbf{S-Bleurt} & 0.29 & 0.05* & 0.45 & 0.06* & 0.29 & 0.23 & 0.46 & 4.2 \\
\textbf{R-Bleurt} & - &  0.21 & 0.38 & - & - & - & - & - \\
\textbf{S-Cosine} & 0.01* & -0.5 & -0.13* & -0.19* & 0.05* & -0.05* & 0.15* & 7.42 \\
\textbf{R-Cosine} & - & -0.11* & -0.16* & - & - & - & - & - \\ \midrule
\textbf{QuestEval} & 0.21 & {\ul{0.29}} & 0.23 & 0.37 & 0.19* & 0.35 & 0.14* & 4.67 \\
\textbf{LlaMA3} & \textbf{0.82} & \textbf{0.80} & \textbf{0.72} & \textbf{0.84} & \textbf{0.84} & \textbf{0.90} & \textbf{0.88} & \textbf{1.00} \\
\textbf{Our (3b)} & 0.47 & -0.11* & 0.37 & 0.61 & 0.53 & 0.54 & 0.66 & 3.5 \\
\textbf{Our (8b)} & {\ul{0.57}} & 0.09* & {\ul 0.49} & {\ul 0.72} & {\ul 0.64} & {\ul 0.62} & {\ul 0.67} & {\ul 2.17} \\ \bottomrule
\end{tabular}
\caption{Pearson correlation on human ratings on our constructed test set. 'R-': reference-based. 'S-': source-based. *not significant; we cannot reject the null hypothesis of zero correlation}
\label{tab:con}
\end{table*}

\section{Results}
We benchmark the different metrics on the different datasets using correlation to human judgement. For content preservation, we show results split on data with system output, reference output and our constructed test set: we show that the data source for evaluation leads to different conclusions on the metrics. In addition, we examine whether the metrics can rank style transfer systems similar to humans. On style strength, we likewise show correlations between human judgment and zero-shot evaluation approaches. When applicable, we summarize results by reporting the average correlation. And the average ranking of the metric per dataset (by ranking which metric obtains the highest correlation to human judgement per dataset). 

\subsection{Content preservation}
\paragraph{How do data sources affect the conclusion on best metric?}
The conclusions about the metrics' performance change radically depending on whether we use system output data, reference output, or our constructed test set. Ideally, a good metric correlates highly with humans on any data source. Ideally, for meta-evaluation, a metric should correlate consistently across all data sources, but the following shows that the correlations indicate different things, and the conclusion on the best metric should be drawn carefully.

Looking at the metrics correlations with humans on the data source with system output (Table~\ref{tab:sys}), we see a relatively high correlation for many of the metrics on many tasks. The overall best metrics are S-BertScore and S-BLEURT (avg+avg rank). We see no notable difference in our method of using the 3B or 8B model as the backbone.

Examining the average correlations based on data with reference output (Table~\ref{tab:ref}), now the zero-shoot prompting with LlaMA3 70B is the best-performing approach ($0.59$ avg). Tied for second place are source-based cosine embedding ($0.46$ avg), BLEURT ($0.45$ avg) and BertScore ($0.45$ avg). Our method follows on a 5. place: here, the 8b version (($0.43$ avg)) shows a bit stronger results than 3b ($0.39$ avg). The fact that the conclusions change, whether looking at reference or system output, confirms the observations made by \citet{scialom-etal-2021-questeval} on simplicity transfer.   

Now consider the results on our test set (Table~\ref{tab:con}): Several metrics show low or no correlation; we even see a significantly negative correlation for some metrics on ALL (BLEU) and for specific subparts of our test set for BLEU, Meteor, BertScore, Cosine. On the other end, LlaMA3 70B is again performing best, showing strong results ($0.82$ in ALL). The runner-up is now our 8B method, with a gap to the 3B version ($0.57$ vs $0.47$ in ALL). Note our method still shows zero correlation for the sentiment task. After, ranks BLEURT ($0.29$), QuestEval ($0.21$), BertScore ($0.11$), Cosine ($0.01$).  

On our test set, we find that some metrics that correlate relatively well on the other datasets, now exhibit low correlation. Hence, with our test set, we can now support the logical reasoning with data evidence: Evaluation of content preservation for style transfer needs to take the style shift into account. This conclusion could not be drawn using the existing data sources: We hypothesise that for the data with system-based output, successful output happens to be very similar to the source sentence and vice versa, and reference-based output might not contain server mistakes as they are gold references. Thus, none of the existing data sources tests the limits of the metrics.  


\paragraph{How do reference-based metrics compare to source-based ones?} Reference-based metrics show a lower correlation than the source-based counterpart for all metrics on both datasets with ratings on references (Table~\ref{tab:sys}). As discussed previously, reference-based metrics for style transfer have the drawback that many different good solutions on a rewrite might exist and not only one similar to a reference.


\paragraph{How well can the metrics rank the performance of style transfer methods?}
We compare the metrics' ability to judge the best style transfer methods w.r.t. the human annotations: Several of the data sources contain samples from different style transfer systems. In order to use metrics to assess the quality of the style transfer system, metrics should correctly find the best-performing system. Hence, we evaluate whether the metrics for content preservation provide the same system ranking as human evaluators. We take the mean of the score for every output on each system and the mean of the human annotations; we compare the systems using the Kendall's Tau correlation. 

We find only the evaluation using the dataset Mir, Lai, and Ziegen to result in significant correlations, probably because of sparsity in a number of system tests (App.~\ref{app:dataset}). Our method (8b) is the only metric providing a perfect ranking of the style transfer system on the Lai data, and Llama3 70B the only one on the Ziegen data. Results in App.~\ref{app:results}. 


\subsection{Style strength results}
%Evaluating style strengths is a challenging task. 
Llama3 70B shows better overall results than our method. However, our method scores higher than Llama3 70B on 2 out of 6 datasets, but it also exhibits zero correlation on one task (Table~\ref{tab:styleresults}).%More work i s needed on evaluating style strengths. 
 
\begin{table}%[]
\fontsize{11pt}{11pt}\selectfont
\begin{tabular}{lccc}
\toprule
\multicolumn{1}{c}{\textbf{}} & \textbf{LlaMA3} & \textbf{Our (3b)} & \textbf{Our (8b)} \\ \midrule
\textbf{Mir} & 0.46 & 0.54 & \textbf{0.57} \\
\textbf{Lai} & \textbf{0.57} & 0.18 & 0.19 \\
\textbf{Ziegen.} & 0.25 & 0.27 & \textbf{0.32} \\
\textbf{Alva-M.} & \textbf{0.59} & 0.03* & 0.02* \\
\textbf{Scialom} & \textbf{0.62} & 0.45 & 0.44 \\
\textbf{\begin{tabular}[c]{@{}l@{}}Our Test\end{tabular}} & \textbf{0.63} & 0.46 & 0.48 \\ \bottomrule
\end{tabular}
\caption{Style strength: Pearson correlation to human ratings. *not significant; we cannot reject the null hypothesis of zero corelation}
\label{tab:styleresults}
\end{table}

\subsection{Ablation}
We conduct several runs of the methods using LLMs with variations in instructions/prompts (App.~\ref{app:method}). We observe that the lower the correlation on a task, the higher the variation between the different runs. For our method, we only observe low variance between the runs.
None of the variations leads to different conclusions of the meta-evaluation. Results in App.~\ref{app:results}.


% \begin{table*}[!th]
%     \centering
%     \resizebox{0.9\textwidth}{!}{
%     \begin{tabular}{|l|r|r|r|l|r|r|r|} 
%        \cline{1-4} \cline{6-8}
%         & \multicolumn{3}{c|}{Retrieve} & &\multicolumn{3}{c|}{Retrieve + Rerank} \\ 
%                \cline{2-4} \cline{6-8}
%                &  \multicolumn{3}{c|}{Hits@10}   & & \multicolumn{2}{c|}{Hits@10}  & Speedup \\
% Dataset  &  Baseline  &  DE-2 Rand  & DE-2 CE  & & DE-2 CE &  Baseline &  \\
%        \cline{1-4} \cline{6-8}
% all nli \cite{bowman-etal-2015-large},\cite{N18-1101}  & 0.77 & 0.59 & 0.75 &  &\textbf{ 0.84} & 0.83 & 4.5x\\
% eli5 \cite{fan-etal-2019-eli5} & 0.43 & 0.12 & 0.29 &  & 0.43 & 0.49 & 5.4x \\
% gooaq \cite{Khashabi2021GooAQOQ} & 0.75 & 0.42 & 0.64 &  & 0.77 & 0.80 & 5.8x\\
% msmarco \cite{nguyen2016ms} & 0.95 & 0.81 & 0.89 &  & 0.96 & 0.98 & 5.1x \\
% \href{https://quoradata.quora.com/First-Quora-Dataset-Release-Question-Pairs}{quora duplicates}  & 0.68 & 0.48 & 0.65 &  &\textbf{ 0.68} & 0.68 & 5.1x\\
% natural ques. \cite{47761} & 0.77 & 0.37 & 0.61 &  & 0.61 & 0.64 & 5.4x \\
% sentence comp \cite{filippova-altun-2013-overcoming} & 0.95 & 0.83 & 0.93 &  & \textbf{0.97 }& 0.97 & 6.6x\\
% simplewiki \cite{coster-kauchak-2011-simple} & 0.97 & 0.93 & 0.97 &  &\textbf{ 0.97 }& 0.96 & 4.8x\\
% stsb \cite{cer-etal-2017-semeval} & 0.97 & 0.87 & 0.97 &  & \textbf{0.98} & 0.98 & 4.7x \\
% zeshel \cite{logeswaran2019zero}  & 0.22 & 0.16 & 0.20 &  & \textbf{0.21} & 0.19 & 4.8x\\
%        \cline{1-4} \cline{6-8}
%     \end{tabular}
%     }

% Please add the following required packages to your document preamble:
% \usepackage[table,xcdraw]{xcolor}
% Beamer presentation requires \usepackage{colortbl} instead of \usepackage[table,xcdraw]{xcolor}
\begin{table*}[!th]
\centering
     \resizebox{1\textwidth}{!}{ \begin{tabular}{|l|llllll|l|lllll|}
\cline{1-7} \cline{9-13}
\textbf{}                  & \multicolumn{6}{c|}{\textbf{Retrieve}}                                                                                                                                                                                         & \textbf{} & \multicolumn{5}{c|}{\textbf{Retrieve + Rerank}}                                                                                                                                    \\ 
\cline{2-7} \cline{9-13}
{\textbf{dataset}}           & \multicolumn{3}{c|}{\textbf{Hits@10}}                                                                                   & \multicolumn{3}{c|}{\textbf{MRR@10}}                                                                & \textbf{} & \multicolumn{2}{c|}{\textbf{Hits@10}}                                         & \multicolumn{2}{c|}{\textbf{MRR@10}}                                           & \textbf{Speedup} \\ 

                           
\cline{2-7} \cline{9-13} 
                           & \multicolumn{1}{l|}{\textbf{Baseline}} & \multicolumn{1}{l|}{\textbf{DE-2-Rand}} & \multicolumn{1}{l|}{\textbf{DE-2-CE}} & \multicolumn{1}{l|}{\textbf{Baseline}} & \multicolumn{1}{l|}{\textbf{DE-2-Rand}} & \textbf{DE-2-CE} & \textbf{} & \multicolumn{1}{l|}{\textbf{Baseline}} & \multicolumn{1}{l|}{\textbf{DE-2-CE}} & \multicolumn{1}{l|}{\textbf{Baseline}} & \multicolumn{1}{l|}{\textbf{DE-2-CE}} & \textbf{}        \\ 
                       \cline{1-7} \cline{9-13}
                       
msmarco/dev/small  & \multicolumn{1}{l|}{0.59}              & \multicolumn{1}{l|}{0.37}               & \multicolumn{1}{l|}{0.45}             & \multicolumn{1}{l|}{0.32}              & \multicolumn{1}{l|}{0.19}               & 0.24             &           & \multicolumn{1}{l|}{0.66}              & \multicolumn{1}{l|}{0.59}             & \multicolumn{1}{l|}{0.38}              & \multicolumn{1}{l|}{0.35}             & 5.1x             \\ \cline{1-7} \cline{9-13}
beir/quora/dev             & \multicolumn{1}{l|}{0.95}              & \multicolumn{1}{l|}{0.90}               & \multicolumn{1}{l|}{0.93}             & \multicolumn{1}{l|}{0.85}              & \multicolumn{1}{l|}{0.76}               & 0.81             &           & \multicolumn{1}{l|}{\textbf{0.96}}     & \multicolumn{1}{l|}{\textbf{0.96}}    & \multicolumn{1}{l|}{\textbf{0.74}}     & \multicolumn{1}{l|}{\textbf{0.74}}    & 5.3x             \\ \cline{1-7} \cline{9-13}
beir/scifact/test          & \multicolumn{1}{l|}{0.63}              & \multicolumn{1}{l|}{0.53}               & \multicolumn{1}{l|}{0.55}             & \multicolumn{1}{l|}{0.42}              & \multicolumn{1}{l|}{0.30}               & 0.36             &           & \multicolumn{1}{l|}{0.72}              & \multicolumn{1}{l|}{0.69}             & \multicolumn{1}{l|}{0.57}              & \multicolumn{1}{l|}{0.55}             & 4.7x             \\ \cline{1-7} \cline{9-13}
beir/fiqa/dev              & \multicolumn{1}{l|}{0.50}              & \multicolumn{1}{l|}{0.22}               & \multicolumn{1}{l|}{0.32}             & \multicolumn{1}{l|}{0.32}              & \multicolumn{1}{l|}{0.12}               & 0.18             &           & \multicolumn{1}{l|}{0.59}              & \multicolumn{1}{l|}{0.46}             & \multicolumn{1}{l|}{0.28}              & \multicolumn{1}{l|}{0.23}             & 5.3x             \\ \cline{1-7} \cline{9-13}
zeshel/test                & \multicolumn{1}{l|}{0.22}              & \multicolumn{1}{l|}{0.16}               & \multicolumn{1}{l|}{0.20}             & \multicolumn{1}{l|}{0.12}              & \multicolumn{1}{l|}{0.09}               & 0.11             &           & \multicolumn{1}{l|}{\textbf{0.20}}     & \multicolumn{1}{l|}{\textbf{0.19}}    & \multicolumn{1}{l|}{\textbf{0.11}}     & \multicolumn{1}{l|}{\textbf{0.10}}    & 4.8x             \\ \cline{1-7} \cline{9-13}
stsb/train                 & \multicolumn{1}{l|}{0.97}              & \multicolumn{1}{l|}{0.95}               & \multicolumn{1}{l|}{0.97}             & \multicolumn{1}{l|}{0.85}              & \multicolumn{1}{l|}{0.83}               & 0.85             &           & \multicolumn{1}{l|}{\textbf{0.98}}     & \multicolumn{1}{l|}{\textbf{0.98}}    & \multicolumn{1}{l|}{\textbf{0.88}}     & \multicolumn{1}{l|}{\textbf{0.88}}    & 4.7x             \\ \cline{1-7} \cline{9-13}
all-nli/train              & \multicolumn{1}{l|}{0.49}              & \multicolumn{1}{l|}{0.41}               & \multicolumn{1}{l|}{0.47}             & \multicolumn{1}{l|}{0.39}              & \multicolumn{1}{l|}{0.33}               & 0.36             &           & \multicolumn{1}{l|}{\textbf{0.55}}     & \multicolumn{1}{l|}{\textbf{0.54}}    & \multicolumn{1}{l|}{\textbf{0.47}}     & \multicolumn{1}{l|}{\textbf{0.46}}    & 4.5x             \\ \cline{1-7} \cline{9-13}
simplewiki/train           & \multicolumn{1}{l|}{0.97}              & \multicolumn{1}{l|}{0.98}               & \multicolumn{1}{l|}{0.98}             & \multicolumn{1}{l|}{0.92}              & \multicolumn{1}{l|}{0.93}               & 0.94             &           & \multicolumn{1}{l|}{\textbf{0.96}}     & \multicolumn{1}{l|}{\textbf{0.97}}    & \multicolumn{1}{l|}{\textbf{0.91}}     & \multicolumn{1}{l|}{\textbf{0.91}}    & 4.8x             \\ \cline{1-7} \cline{9-13}
natural-questions/train    & \multicolumn{1}{l|}{0.77}              & \multicolumn{1}{l|}{0.59}               & \multicolumn{1}{l|}{0.65}             & \multicolumn{1}{l|}{0.51}              & \multicolumn{1}{l|}{0.37}               & 0.42             &           & \multicolumn{1}{l|}{0.65}              & \multicolumn{1}{l|}{0.61}             & \multicolumn{1}{l|}{0.39}              & \multicolumn{1}{l|}{0.37}             & 5.4x             \\ \cline{1-7} \cline{9-13}
eli5/train                 & \multicolumn{1}{l|}{0.32}              & \multicolumn{1}{l|}{0.14}               & \multicolumn{1}{l|}{0.22}             & \multicolumn{1}{l|}{0.19}              & \multicolumn{1}{l|}{0.08}               & 0.12             &           & \multicolumn{1}{l|}{0.39}              & \multicolumn{1}{l|}{0.32}             & \multicolumn{1}{l|}{0.26}              & \multicolumn{1}{l|}{0.22}             & 5.4x             \\ \cline{1-7} \cline{9-13}
sentence compression/train & \multicolumn{1}{l|}{0.93}              & \multicolumn{1}{l|}{0.89}               & \multicolumn{1}{l|}{0.93}             & \multicolumn{1}{l|}{0.85}              & \multicolumn{1}{l|}{0.79}               & 0.84             &           & \multicolumn{1}{l|}{\textbf{0.96}}     & \multicolumn{1}{l|}{\textbf{0.96}}    & \multicolumn{1}{l|}{\textbf{0.93}}     & \multicolumn{1}{l|}{\textbf{0.94}}    & 6.6x             \\ \cline{1-7} \cline{9-13}
gooaq/train                & \multicolumn{1}{l|}{0.73}              & \multicolumn{1}{l|}{0.68}               & \multicolumn{1}{l|}{0.72}             & \multicolumn{1}{l|}{0.58}              & \multicolumn{1}{l|}{0.55}               & 0.58             &           & \multicolumn{1}{l|}{0.80}              & \multicolumn{1}{l|}{0.76}             & \multicolumn{1}{l|}{0.64}              & \multicolumn{1}{l|}{0.62}             & 5.2x             \\ \cline{1-7} \cline{9-13}
\end{tabular}
}
    \caption{Comparison of CE infused DE. \textbf{Baseline} is the pre-trained DE SBERT model.  \textbf{DE-2 Rand} is a trained two-layered DE initialized with random initial weights. \textbf{DE-2 CE} is a two-layered DE infused with initial weights from CE as explained in Fig~\ref{fig:dual_cross} and trained similar to DE-2. All the above models are trained on msmarco. Bold face numbers for DE-2 CE show where performance is at least within .01 of the baseline DE. \textbf{Speedup} is inference time gain for DE-2 CE over Baseline.}
    % \textbf{Retrieve + Rerank}: Columns 5-6 present Accuracy@10 and columns 7-8 present number of documents encoded per sec. Here, the documents retrieved using baseline and our approach are reranked using a CE.}
    \label{tab:comp}
\end{table*} 

\section{Experiments}
\label{sec:exp}

In this section, we conduct empirical evaluations to study the performance of our approach in both handling noisy and corrupted views and in domain generalization tasks. 

\vspace{-1mm}
\subsection{Comparison with CL Baselines}
\vspace{-1mm}

\paragraph{Experiment setup.}~To examine the robustness of our framework, we trained INCE and RINCE as baselines, and developed their GCA-based alternatives (+\ours). In addition, we also compared with our novel GCA-UOT method, two variants of IOT established in~\cite{shi2023understanding},  and other CL baselines, including BYOL and SimCLR. For experiments with SVHN \cite{netzer2011reading} and ImageNet100 \cite{deng2009imagenet} we use the ResNet-50 encoder as the backbone and use a ResNet-18 encoder as the backbone for CIFAR-10, CIFAR-100~\cite{krizhevsky2009learning} and a corrupted version of CIFAR called CIFAR-10C \cite{hendrycks2019benchmarking}. 



\begin{table}[t!]
  \begin{center}
    \resizebox{\textwidth}{!}{
    \begin{tabular}{|r|c|c|c|c||c|c|c|c|}
    \hline
    & \multicolumn{4}{c||}{\footnotesize{\bf Standard Setting}}
    & \multicolumn{4}{c|}{\footnotesize{\bf Noisy Setting}} \\
    \hline
    \footnotesize{Method}
    & \footnotesize{CIFAR-10}
    & \footnotesize{CIFAR-100}
    & \footnotesize{SVHN}
    & \footnotesize{ImageNet100}
    & \footnotesize{CIFAR-10 (Ex)}
    & \footnotesize{CIFAR-100 (Ex)}
    & \footnotesize{CIFAR-10C}
    & \footnotesize{CIFAR-10C (Ex)} \\
    \hline

    %============================
    % INCE
    %============================
    \footnotesize{INCE}
    & 92.01 {\footnotesize ± 0.40}        % CIFAR-10
    & 71.09 {\footnotesize ± 0.31}        % <-- Updated from the new table (Standard row)
    & 92.42 {\footnotesize ± 0.24}        % SVHN
    & 73.01 {\footnotesize ± 0.61}        % ImageNet100
    & 82.03 {\footnotesize ± 0.32}        % CIFAR-10 (Ex)
    & 62.54 {\footnotesize ± 0.20}        % CIFAR-100 (Ex)
    & 81.52 {\footnotesize ± 1.04}        % CIFAR-10C (Standard)
    & 83.28 {\footnotesize ± 0.25}        % CIFAR-10C (Ex) = avg(large_erase, strong_crop, brightness)
    \\
    
    %============================
    % GCA-INCE
    %============================
    \footnotesize{GCA-INCE}
    & \underline{93.02 {\footnotesize ± 0.19}}  
    & \underline{71.55 {\footnotesize ± 0.12}}        % <-- Updated from new table
    & \underline{92.64 {\footnotesize ± 0.26}}
    & \underline{73.04 {\footnotesize ± 0.76}}
    & \underline{82.18 {\footnotesize ± 0.69}}
    & \underline{62.65 {\footnotesize ± 0.17}}
    & \underline{82.63 {\footnotesize ± 0.28}}
    & \underline{82.72 {\footnotesize ± 0.27}}
    \\
    \footnotesize{$\Delta$}
    & +1.01
    & +0.46            % (71.55 - 71.09)
    & +0.22
    & +0.03
    & +0.15
    & +0.11
    & +1.11
    & -0.56
    \\
    \hline
     
    %============================
    % RINCE
    %============================
    \footnotesize{RINCE}
    & 93.27 {\footnotesize ± 0.20}
    & 71.63 {\footnotesize ± 0.36}        % <-- Updated from new table
    & 93.26 {\footnotesize ± 0.15}
    & 71.91 {\footnotesize ± 0.43}
    & 82.60 {\footnotesize ± 0.63}
    & 63.55 {\footnotesize ± 0.14}
    & 82.86 {\footnotesize ± 0.21}
    & 83.64 {\footnotesize ± 0.26}
    \\
    
    %============================
    % GCA-RINCE
    %============================
    \footnotesize{GCA-RINCE}
    & \underline{93.47 {\footnotesize ± 0.30}}
    & \underline{71.95 {\footnotesize ± 0.48}}  % <-- Updated from new table
    & \underline{93.57 {\footnotesize ± 0.26}}
    & \underline{73.44 {\footnotesize ± 0.55}}
    & \underline{82.76 {\footnotesize ± 0.49}}
    & \underline{63.14 {\footnotesize ± 0.41}}
    & \underline{82.87 {\footnotesize ± 0.11}}
    & \underline{\textbf{83.69 {\footnotesize ± 0.16} }}
    \\
    \footnotesize{$\Delta$}
    & +0.20
    & +0.32            % (71.95 - 71.63)
    & +0.31
    & +1.53
    & +0.16
    & -0.47
    & +0.01
    & +0.05
    \\
    \hline
    
    %============================
    % SimCLR
    %============================
    \footnotesize{SimCLR}
    & 92.07 {\footnotesize ± 0.90}
    & 70.85 {\footnotesize ± 0.50}        % <-- Updated from new table
    & 92.13 {\footnotesize ± 0.34}
    & 72.20 {\footnotesize ± 0.78}
    & 81.87 {\footnotesize ± 0.53}
    & 62.94 {\footnotesize ± 0.13}
    & 81.74 {\footnotesize ± 1.54}
    & 83.25 {\footnotesize ± 0.18}
    \\
      
    %============================
    % BYOL (original)
    %============================
    \footnotesize{BYOL}
    & 90.56 {\footnotesize ± 0.59}
    & 69.75 {\footnotesize ± 0.37}        % <-- from new table
    & 89.50 {\footnotesize ± 0.46}
    & 69.75 {\footnotesize ± 0.83}
    & 81.55 {\footnotesize ± 0.50}
    & 62.11 {\footnotesize ± 0.25}
    & 82.43 {\footnotesize ± 0.06}
    & 70.09 {\footnotesize ± 0.34}
    \\
    
    %============================
    % IOT
    %============================
    \footnotesize{IOT \cite{shi2023understanding}}
    & 92.09 {\footnotesize ± 0.22}
    & 68.37 {\footnotesize ± 0.42}        % <-- from new table
    & 92.25 {\footnotesize ± 0.19}
    & 72.27 {\footnotesize ± 0.53}
    & 80.59 {\footnotesize ± 0.64}
    & 62.69 {\footnotesize ± 0.34}
    & 82.01 {\footnotesize ± 0.80}
    & 81.57 {\footnotesize ± 0.83}
    \\
    
    %============================
    % IOT-uni
    %============================
    \footnotesize{IOT-uni \cite{shi2023understanding}}
    & 91.49 {\footnotesize ± 0.11}
    & 68.62 {\footnotesize ± 0.35}        % <-- from new table
    & 92.33 {\footnotesize ± 0.11}
    & 72.88 {\footnotesize ± 0.71}
    & 80.79 {\footnotesize ± 0.24}
    & 62.56 {\footnotesize ± 0.22}
    & 81.19 {\footnotesize ± 1.12}
    & 81.82 {\footnotesize ± 0.51}
    \\
    
    \hline
    %============================
    % GCA-UOT
    %============================
    \footnotesize{GCA-UOT}
    & \textbf{93.50 {\footnotesize ± 0.31}}
    & \textbf{72.16 {\footnotesize ± 0.38}}  % <-- from new table
    & \textbf{93.82 {\footnotesize ± 0.17}}
    & \textbf{74.09 {\footnotesize ± 0.40}}
    & \textbf{83.18 {\footnotesize ± 0.44}}
    & \textbf{63.62 {\footnotesize ± 0.27}}
    & \textbf{82.90 {\footnotesize ± 0.50}}
    & 83.64 {\footnotesize ± 0.19}
    \\
    \hline

    \end{tabular}
    }
  \caption{\footnotesize{{\em Test accuracy (\%) on a downstream classification task after pretraining.} 
    Results are shown for CIFAR-10 (ResNet18), CIFAR-100 (ResNet18), 
    SVHN (ResNet18), and ImageNet100 (ResNet50) under standard 
    and extreme (Ex) augmentation conditions (averaged over 5 seeds). 
    The {\bf CIFAR-100} column is now updated with the latest values from
    Table~\ref{tab:cifar1004c}. The top model is in bold and the second-place 
    model is underlined. For INCE and RINCE, we also provide the 
    improvement $\Delta$ by adding GCA to each method.}}
    \label{tab:new_combine}
  \end{center}
  \vspace{-8mm}
\end{table}



In all of these cases, we follow the standard self-supervised learning evaluation protocol \cite{chen2020simple}, where we train the encoder on the training set in an unsupervised manner and then train a linear layer on top of the frozen representations to obtain the final accuracy on the test set.
% The final test accuracy is obtained by passing the test set through both the frozen encoder and linear layer without finetuning.
In addition to standard data augmentation policies commonly used \cite{chuang2022robust}, we also apply three different extreme augmentation policies to examine the robustness of \ours~towards noisy views (details in Appendix~\ref{app:corruptset}). Learning rates and other training details for CIFAR-10, CIFAR-100, SVHN, and ImageNet100 are provided in Appendix~\ref{app:hpdetails}, while specific training details for CIFAR-10C are included in Appendix~\ref{app:corruptset}.




\vspace{-3mm}
\paragraph{Results on Standard Augmentations.}~ 
First, we performed experiments on CIFAR-10, CIFAR-100, SVHN, and ImageNet100 using standard sets of augmentations that are applied to achieve state-of-the-art performance (Table~\ref{tab:new_combine}, Standard Setting). 
We found the +GCA versions of INCE and RINCE exhibit performance gains in almost all settings except for SVHN, with bigger gains observed when adding GCA to RINCE. Additionally, we find that our unbalanced OT method, GCA-UOT, achieves the top performance across the board, on all four datasets tested. The transport plans obtained by each methods are provided in Figure~\ref{fig:diagnolline} along with a study of the sensitivity of the methods to hyperparameters (Appendix~\ref{fig:ablation_study}).

\vspace{-3mm}
\paragraph{Results on Corrupted Data and Extreme Augmentations.}~
Next, we tested the methods in two noisy settings. In the first set of experiments, we apply extreme augmentations to CIFAR-10 (Ex) and CIFAR-100 (Ex) (see Appendix~\ref{app:corruptset}) to introduce noisy views during training. In the second set of experiments, we used the CIFAR-10C to further test the ability of our method to work in noisy settings. 


Our experimental results demonstrate that the GCA-based strategy effectively enhances the model's generalization ability and adaptability to aggressive data augmentations. 
In addition to improving classification accuracy, the GCA-based methods also improve the representational alignment and uniformity, as shown in Appendix~\ref{app:representation}. This observation is in line with our theoretical analysis in Section~\ref{subsec:align_uni}, where we show that the obtained representations provide better overall alignment of positive views and better spread in terms of uniformity \cite{wang2020understanding}.

\vspace{-1mm}
\subsection{Block Diagonal Transport in Domain Generalization}
\vspace{-1mm}
\label{sec:domaingen}

\begin{wrapfigure}{r}{0.38\textwidth} 
\vspace{-8mm}
    \centering{
       \includegraphics[width=0.54\linewidth, clip, trim=0 1.3cm 25cm 0cm]{files/imgs/figure_v3.png}

 \includegraphics[width=\linewidth, clip, trim=13.7cm 0 0 0]{files/imgs/figure_v3.png}}
        \caption{\footnotesize{ {\em Incorporating different priors into learning across multiple domains.}} (A) Example target alignment plan \(\PP_{\text tgt}\), where the target over all samples from the same domain are set to \( \alpha \), the diagonal values are set to 1, and across-domain samples are set to \(\beta\). (B) The domain classification accuracy (red) and overall class accuracy (blue) with (\(\alpha-\beta\)) increases.}
        \vspace{-7mm}
        \label{fig:Ptgt}

\end{wrapfigure}

In a final experiment, we aimed to demonstrate the flexibility and robustness of our framework by applying it to a domain generalization task, where samples originate from different domains (e.g., Photo, Cartoon, Sketch, Art). We explored the effects of introducing domain-specific alignment constraints in our transport plan, hypothesizing that this could enhance the latent space organization to capture more nuanced domain similarities.

Our approach enables additional contextual information to be seamlessly integrated into the transport process. In this case, domain information was incorporated to distinguish the alignment of samples from the same versus different domains. To achieve this, we adjusted the target transport plan \({\bf P}_{tgt}\), selectively modifying parameters \((\alpha, \beta)\) to vary the influence of domain-based alignment constraints as shown in Figure~\ref{fig:Ptgt}(A). Specifically, we set \(\{\alpha=0, \beta>0\}\) to prioritize cross-domain alignment and \(\{\alpha>0, \beta=0\}\) to focus on intra-domain alignment.

The training was conducted on the PACS dataset \cite{li2017deeper} using a ResNet-18 encoder with the \ours-INCE objective. After training the encoder in an unsupervised manner, we freeze the encoder and then train a linear readout layer to predict either the sample’s class or the domain it belonged to. This setup allowed us to isolate the effect of our transport adjustments on the latent space’s capacity to encode both class and domain information.

The results, displayed in Figure~\ref{fig:Ptgt}(B), revealed that increasing the domain alignment weight enhances the accuracy of domain classification (from 72.11\% to 95.16\%) without diminishing classification performance. This outcome suggests that \ours~can effectively encode both domain and class information in a single latent representation. The ability to adjust alignment constraints provides a powerful tool for domain generalization tasks, enabling multiple types of similarity to be jointly encoded. This flexibility can potentially alleviate issues related to information loss from data augmentation, especially in fine-grained classification settings, by retaining essential domain-specific characteristics across transformations.




\section{Deployment Considerations}
\section{Deployment}
\label{section:deployment}

\subsection{RecSys use case}\label{section:recsys-deploy}
SLM variants of the RecSys use case have been deployed to an A/B test.

\noindent \textbf{Serving Infrastructure} For all use cases, we benchmark and serve traffic using nodes with 256 CPU cores, 2TB of host memory and 8 NVIDIA H100 GPUs. As mentioned earlier, we deploy SGLang (version 0.4.1) as the serving engine. We use tensor parallelism to concurrently use more than 1 GPU for inference. To maximize performance, we employ FP8 quantization for both weights and activations and use FlashInfer~\cite{ye2025flashinfer} as the primary attention backend. Moreover, SGLang incorporates RadixAttention, which enables prefix caching for prompts sharing common prefixes.

\noindent \textbf{Workloads} For the RecSys workload, we follow the prompt structure outlined in Section~\ref{subsection:recsys_task} and utilize context lengths of 16k and 32k. Given the predictive nature of the task, the output length (i.e., the number of generated tokens) is set to 1, rendering the workload heavily reliant on the prefill phase. Consequently, optimizing the prefill stage is crucial for performance. For instance, prefix caching substantially improves prefill (and decode) times when a prompt’s key (K) and value (V) tensors for its shared prefix have already been processed. In cases where there are $k$ candidate items to be ranked for a member $m$, $k$ prompts are served. These prompts share a long prefix containing the user information and historical item interactions. Once one prompt is served, its KV tensors are cached, allowing subsequent prompts for the same member to reuse the cached data—a process we refer to as a hot prefill.

\noindent \textbf{Metrics} We use two key metrics - time to first token (TTFT) and time per output token (TPOT). For prediction tasks, which are prefill-intensive, TTFT is the primary metric as it reflects the duration of the prefill phase. For generative tasks, both TTFT and TPOT are important. Additionally, we report the total serving throughput for various context lengths.

\noindent \textbf{Results} In terms of quality, the models under consideration performn similarly since their AUCs are similar. For performance, we report TTFT and throughput measurements for both 16k and 32k context lengths. P99 TTFT results for a workload with 1 QPS (i.e., $m=1$) and 1 or more prompt per member (i.e., $k=1$ or more) can be found in Figure~\ref{figure:p99ttft} (detailed p50, p99 and throughput numbers can be found in Tables~\ref{table:m1k1_16k}, ~\ref{table:m1k1_32k} and ~\ref{table:m1k4_32k} in Appendix~\ref{subsec:Extra tables}). From the figure and the tables, we can conclude that latency drops drastically as model size becomes smaller. Serving traffic with 32k context is significantly slower than serving traffic with 16k context. Setting $k$ more than 1 doesn't hurt latency much, because of KV caching. 

To better understand the effect of model pruning on inference latency, we present the break down of forward pass for a single layer in Figure~\ref{figure:profiling}. As it can be seen, the attention step is the main latency bottleneck. Our structured pruning of the attention heads improves the attention latency by about $40\%$ which in turn results in more than $28\%$ speed up in prefill latency.



\begin{figure}[htbp]
  \centering
  \begin{tikzpicture}
    \begin{axis}[
      xlabel={LLMs},
      ylabel={P99 TTFT (ms)},
      xlabel style={font=\large},
      ylabel style={font=\large},
      xtick=data,                    
      symbolic x coords={FM,8B,6.4B,3B,2.4B,2.1B},
      x tick label style={rotate=45, anchor=east},
      width=0.9\columnwidth,        
      height=0.7\columnwidth,       
      legend style={at={(0.95,0.95)}, anchor=north east} 
    ]
      \addplot[
        mark=*,
        color=myblue,
        thick,
        nodes near coords={\pgfkeysvalueof{/pgfplots/y}},
        every node near coord/.append style={font=\footnotesize, anchor=south}
      ] coordinates {
        (8B,282)
        (6.4B,262)
        (3B,209)
        (2.4B,197)
        (2.1B,184)
      };
      \addlegendentry{m=1, k=1, 16k};

      \addplot[
        mark=square*,
        color=myorange,
        thick,
        nodes near coords={\pgfkeysvalueof{/pgfplots/y}},
        every node near coord/.append style={font=\footnotesize, anchor=south}
      ] coordinates {
        (8B,643)
        (6.4B,613)
        (3B,472)
        (2.4B,456)
        (2.1B,391)
      };
      \addlegendentry{m=1, k=1, 32k};

      \addplot[
        mark=triangle*,
        color=mygreen,
        thick,
        nodes near coords={\pgfkeysvalueof{/pgfplots/y}},
        every node near coord/.append style={font=\footnotesize, anchor=south}
      ] coordinates {
        (8B,671)
        (6.4B,655)
        (3B,500)
        (2.4B,488)
        (2.1B,403)
      };
      \addlegendentry{m=1, k=4, 32k};

    \end{axis}
  \end{tikzpicture}
  \caption{P99 TTFT (ms) for various LLMs}
  \label{figure:p99ttft}
\end{figure}




\subsection{Generative use case}

\begin{table}[t]

\centering
\begin{tabular}{lc}
\hline\hline
\textbf{Model} & \textbf{IQM (\%)} \\
\hline\hline
KD Model   &+20.29\%  \\
\hline\hline
\end{tabular}
\caption{Online A/B test results for reasoning task.}
\label{table:gen_use_case_online}
\label{tab:gen_use_case_online}
\end{table}%

The reasoning task outlined in Section 4.3 was launched online for a 1\% A/B test. Results are outlined in Table~\ref{tab:gen_use_case_online}. KD, along with data changes, helped the model improve by 20.29\% on an internal quality metric (IQM). We also discuss deployment lessons from a generative use case to study the effect of different quantization schemes on model inference speed and accuracy. 

\noindent \textbf{Serving Infrastructure} Our setup is mostly similar to Section~\ref{section:recsys-deploy}. However, in addition to using NVIDIA H100 GPUs, we also study the effect of using older NVIDIA A100 GPUs. To this end, 
we use the vLLM backend (version 0.6.1) for serving. We use 1 GPU for serving (TP=1).

\noindent \textbf{Workloads} The workload here consists of prompts with varying lengths, averaging to 3.8k tokens per request, with 1 request per second. The output generation is capped to 2k tokens. As we focus on a generative task here, we report both TTFT and TPOT. We study a Llama3-based model with the 8B size, and consider several serving scenarios with or without quantization using different hardware.

\noindent \textbf{Performance Results} The inference speed results are reported in Table~\ref{table:quantization}. Using the state-of-the-art H100 GPUs results in faster inference (both in terms of TTFT and TPOT) compared to A100 GPUs. In particular, we observe that FP8 serving with H100s leads to the smallest TTFT and TPOT. However, limiting our comparisons to A100 GPUs, we see that using INT8 (W8A8) quantization results in significant reduction of both TTFT and TPOT. On the other hand, INT4 (w4A16) quantization actually increases TTFT, while decreasing TPOT. For generative task with long output sequences, we observe that W4A16 has smaller latency. In summary, we have observed that using higher-end hardware such as H100 leads to the most speed-ups. When such hardware is not available, for generative tasks with long output sequences, W4A16 has the most speed-up, while W8A8 can be more appropriate for prefill-heavy tasks. For the sake of completeness, we present a brief comparison of quantization methods in terms of accuracy in Appendix~\ref{appendix:quantization}.






\begin{table}[]

\begin{tabular}{cccc}
\hline \hline
Model & P50 TTFT (ms) & P50 TPOT (ms) & GPU  \\ \hline \hline
FP16    &   136 &  10.3         & H100          \\
FP8 & 122 &  9.4 & H100 \\
FP16    &       332     & 18.3 & A100          \\
W8A8 (INT) &     227       & 12.9 & A100   \\
W4A16 (INT) &       389     &  11.2 & A100   \\
\hline \hline
\end{tabular}
\caption{Comparison of different quantization methods for the Llama-3 8B model.}
\label{table:quantization}
\end{table}





\begin{figure}[htb]
    \centering
    \begin{tikzpicture}
        \begin{axis}[
            ybar stacked,
            symbolic x coords={3B (16k), 2.1B (16k), 3B (32k), 2.1B (32k)},
            xtick=data,
            x tick label style={rotate=45, anchor=east},
            ylabel={Latency (ms)},
            ymin=0,
            legend pos=north west,
            bar width=15pt,
            x tick style={draw=none}, 
            width=\columnwidth,
            height=6cm,
            legend image code/.code={
              \draw[] (0,-0.075cm) rectangle (0.03cm,0.075cm);
            }
        ]

        

        \addplot+[ybar, 
                    fill=myblue!50!white, 
                  draw=myblue!55!white,
          ] plot coordinates {
            (3B (16k), 3)
            (2.1B (16k), 2.5)
            (3B (32k), 7)
            (2.1B (32k), 5)
        };
        
        \addplot+[ybar, 
                  fill=myorange!50!white, 
                 draw=myorange!60!white,
          ] plot coordinates {
            (3B (16k), 1)
            (2.1B (16k), 1)
            (3B (32k), 2)
            (2.1B (32k), 1.8)
        };
        
        \addplot+[ybar, 
fill=mygreen!50!white, 
                  draw=mygreen!55!white,
        ] plot coordinates {
            (3B (16k), 1)
            (2.1B (16k), 1)
            (3B (32k), 2.5)
            (2.1B (32k), 2)
        };

        \legend{Attention, AllReduce, MLP}
        \end{axis}
    \end{tikzpicture}
    \caption{Latency breakdown of a single Transformer block for pruned and unpruned models. As context size increases, attention becomes a major bottleneck.}
    \label{figure:profiling}
\end{figure}



\section{Related Work}
\section{RELATED WORK}
\label{sec:relatedwork}
In this section, we describe the previous works related to our proposal, which are divided into two parts. In Section~\ref{sec:relatedwork_exoplanet}, we present a review of approaches based on machine learning techniques for the detection of planetary transit signals. Section~\ref{sec:relatedwork_attention} provides an account of the approaches based on attention mechanisms applied in Astronomy.\par

\subsection{Exoplanet detection}
\label{sec:relatedwork_exoplanet}
Machine learning methods have achieved great performance for the automatic selection of exoplanet transit signals. One of the earliest applications of machine learning is a model named Autovetter \citep{MCcauliff}, which is a random forest (RF) model based on characteristics derived from Kepler pipeline statistics to classify exoplanet and false positive signals. Then, other studies emerged that also used supervised learning. \cite{mislis2016sidra} also used a RF, but unlike the work by \citet{MCcauliff}, they used simulated light curves and a box least square \citep[BLS;][]{kovacs2002box}-based periodogram to search for transiting exoplanets. \citet{thompson2015machine} proposed a k-nearest neighbors model for Kepler data to determine if a given signal has similarity to known transits. Unsupervised learning techniques were also applied, such as self-organizing maps (SOM), proposed \citet{armstrong2016transit}; which implements an architecture to segment similar light curves. In the same way, \citet{armstrong2018automatic} developed a combination of supervised and unsupervised learning, including RF and SOM models. In general, these approaches require a previous phase of feature engineering for each light curve. \par

%DL is a modern data-driven technology that automatically extracts characteristics, and that has been successful in classification problems from a variety of application domains. The architecture relies on several layers of NNs of simple interconnected units and uses layers to build increasingly complex and useful features by means of linear and non-linear transformation. This family of models is capable of generating increasingly high-level representations \citep{lecun2015deep}.

The application of DL for exoplanetary signal detection has evolved rapidly in recent years and has become very popular in planetary science.  \citet{pearson2018} and \citet{zucker2018shallow} developed CNN-based algorithms that learn from synthetic data to search for exoplanets. Perhaps one of the most successful applications of the DL models in transit detection was that of \citet{Shallue_2018}; who, in collaboration with Google, proposed a CNN named AstroNet that recognizes exoplanet signals in real data from Kepler. AstroNet uses the training set of labelled TCEs from the Autovetter planet candidate catalog of Q1–Q17 data release 24 (DR24) of the Kepler mission \citep{catanzarite2015autovetter}. AstroNet analyses the data in two views: a ``global view'', and ``local view'' \citep{Shallue_2018}. \par


% The global view shows the characteristics of the light curve over an orbital period, and a local view shows the moment at occurring the transit in detail

%different = space-based

Based on AstroNet, researchers have modified the original AstroNet model to rank candidates from different surveys, specifically for Kepler and TESS missions. \citet{ansdell2018scientific} developed a CNN trained on Kepler data, and included for the first time the information on the centroids, showing that the model improves performance considerably. Then, \citet{osborn2020rapid} and \citet{yu2019identifying} also included the centroids information, but in addition, \citet{osborn2020rapid} included information of the stellar and transit parameters. Finally, \citet{rao2021nigraha} proposed a pipeline that includes a new ``half-phase'' view of the transit signal. This half-phase view represents a transit view with a different time and phase. The purpose of this view is to recover any possible secondary eclipse (the object hiding behind the disk of the primary star).


%last pipeline applies a procedure after the prediction of the model to obtain new candidates, this process is carried out through a series of steps that include the evaluation with Discovery and Validation of Exoplanets (DAVE) \citet{kostov2019discovery} that was adapted for the TESS telescope.\par
%



\subsection{Attention mechanisms in astronomy}
\label{sec:relatedwork_attention}
Despite the remarkable success of attention mechanisms in sequential data, few papers have exploited their advantages in astronomy. In particular, there are no models based on attention mechanisms for detecting planets. Below we present a summary of the main applications of this modeling approach to astronomy, based on two points of view; performance and interpretability of the model.\par
%Attention mechanisms have not yet been explored in all sub-areas of astronomy. However, recent works show a successful application of the mechanism.
%performance

The application of attention mechanisms has shown improvements in the performance of some regression and classification tasks compared to previous approaches. One of the first implementations of the attention mechanism was to find gravitational lenses proposed by \citet{thuruthipilly2021finding}. They designed 21 self-attention-based encoder models, where each model was trained separately with 18,000 simulated images, demonstrating that the model based on the Transformer has a better performance and uses fewer trainable parameters compared to CNN. A novel application was proposed by \citet{lin2021galaxy} for the morphological classification of galaxies, who used an architecture derived from the Transformer, named Vision Transformer (VIT) \citep{dosovitskiy2020image}. \citet{lin2021galaxy} demonstrated competitive results compared to CNNs. Another application with successful results was proposed by \citet{zerveas2021transformer}; which first proposed a transformer-based framework for learning unsupervised representations of multivariate time series. Their methodology takes advantage of unlabeled data to train an encoder and extract dense vector representations of time series. Subsequently, they evaluate the model for regression and classification tasks, demonstrating better performance than other state-of-the-art supervised methods, even with data sets with limited samples.

%interpretation
Regarding the interpretability of the model, a recent contribution that analyses the attention maps was presented by \citet{bowles20212}, which explored the use of group-equivariant self-attention for radio astronomy classification. Compared to other approaches, this model analysed the attention maps of the predictions and showed that the mechanism extracts the brightest spots and jets of the radio source more clearly. This indicates that attention maps for prediction interpretation could help experts see patterns that the human eye often misses. \par

In the field of variable stars, \citet{allam2021paying} employed the mechanism for classifying multivariate time series in variable stars. And additionally, \citet{allam2021paying} showed that the activation weights are accommodated according to the variation in brightness of the star, achieving a more interpretable model. And finally, related to the TESS telescope, \citet{morvan2022don} proposed a model that removes the noise from the light curves through the distribution of attention weights. \citet{morvan2022don} showed that the use of the attention mechanism is excellent for removing noise and outliers in time series datasets compared with other approaches. In addition, the use of attention maps allowed them to show the representations learned from the model. \par

Recent attention mechanism approaches in astronomy demonstrate comparable results with earlier approaches, such as CNNs. At the same time, they offer interpretability of their results, which allows a post-prediction analysis. \par


\section{Conclusion}
\section{Conclusion}
In this work, we propose a simple yet effective approach, called SMILE, for graph few-shot learning with fewer tasks. Specifically, we introduce a novel dual-level mixup strategy, including within-task and across-task mixup, for enriching the diversity of nodes within each task and the diversity of tasks. Also, we incorporate the degree-based prior information to learn expressive node embeddings. Theoretically, we prove that SMILE effectively enhances the model's generalization performance. Empirically, we conduct extensive experiments on multiple benchmarks and the results suggest that SMILE significantly outperforms other baselines, including both in-domain and cross-domain few-shot settings.


\section{Limitations}
\section{Limitation}
The use of 3D-printed PLA for structural components improves improving ease of assembly and reduces weight and cost, yet it causes deformation under heavy load, which can diminish end-effector precision. Using metal, such as aluminum, would remedy this problem. Additionally, \robot relies on integrated joint relative encoders, requiring manual initialization in a fixed joint configuration each time the system is powered on. Using absolute joint encoders could significantly improve accuracy and ease of use, although it would increase the overall cost. 

%Reliance on commercially available actuators simplifies integration but imposes constraints on control frequency and customization, further limiting the potential for tailored performance improvements.

% The 6 DoF configuration provides sufficient mobility for most tasks; however, certain bimanual operations could benefit from an additional degree of freedom to handle complex joint constraints more effectively. Furthermore, the limited torque density of commercially available proprioceptive actuators restricts the payload and torque output, making the system less suitability for handling heavier loads or high-torque applications. 

The 6 DoF configuration of the arm provides sufficient mobility for single-arm manipulation tasks, yet it shows a limitation in certain bimanual manipulation problems. Specifically, when \robot holds onto a rigid object with both hands, each arm loses 1 DoF because the hands are fixed to the object during grasping. This leads to an underactuated kinematic chain which has a limited mobility in 3D space. We can achieve more mobility by letting the object slip inside the grippers, yet this renders the grasp less robust and simulation difficult. Therefore, we anticipate that designing a lightweight 3 DoF wrist in place of the current 2 DoF wrist allows a more diverse repertoire of manipulation in bimanual tasks.

Finally, the limited torque density of commercially available proprioceptive actuators restricts the performance. Currently, all of our actuators feature a 1:10 gear ratio, so \robot can handle up to 2.5 kg of payload. To handle a heavier object and manipulate it with higher torque, we expect the actuator to have 1:20$\sim$30 gear ratio, but it is difficult to find an off-the-shelf product that meets our requirements. Customizing the actuator to increase the torque density while minimizing the weight will enable \robot to move faster and handle more diverse objects.

%These constraints highlight opportunities for improvement in future iterations, including alternative materials for enhanced rigidity, custom actuator designs for higher control precision and torque density, the adoption of absolute joint encoders, and optimized configurations to balance dexterity and weight.



\bibliography{custom}
\clearpage

\appendix
\section{Appendix}
\label{sec:appendix}

\subsection{Lloyd-Max Algorithm}
\label{subsec:Lloyd-Max}
For a given quantization bitwidth $B$ and an operand $\bm{X}$, the Lloyd-Max algorithm finds $2^B$ quantization levels $\{\hat{x}_i\}_{i=1}^{2^B}$ such that quantizing $\bm{X}$ by rounding each scalar in $\bm{X}$ to the nearest quantization level minimizes the quantization MSE. 

The algorithm starts with an initial guess of quantization levels and then iteratively computes quantization thresholds $\{\tau_i\}_{i=1}^{2^B-1}$ and updates quantization levels $\{\hat{x}_i\}_{i=1}^{2^B}$. Specifically, at iteration $n$, thresholds are set to the midpoints of the previous iteration's levels:
\begin{align*}
    \tau_i^{(n)}=\frac{\hat{x}_i^{(n-1)}+\hat{x}_{i+1}^{(n-1)}}2 \text{ for } i=1\ldots 2^B-1
\end{align*}
Subsequently, the quantization levels are re-computed as conditional means of the data regions defined by the new thresholds:
\begin{align*}
    \hat{x}_i^{(n)}=\mathbb{E}\left[ \bm{X} \big| \bm{X}\in [\tau_{i-1}^{(n)},\tau_i^{(n)}] \right] \text{ for } i=1\ldots 2^B
\end{align*}
where to satisfy boundary conditions we have $\tau_0=-\infty$ and $\tau_{2^B}=\infty$. The algorithm iterates the above steps until convergence.

Figure \ref{fig:lm_quant} compares the quantization levels of a $7$-bit floating point (E3M3) quantizer (left) to a $7$-bit Lloyd-Max quantizer (right) when quantizing a layer of weights from the GPT3-126M model at a per-tensor granularity. As shown, the Lloyd-Max quantizer achieves substantially lower quantization MSE. Further, Table \ref{tab:FP7_vs_LM7} shows the superior perplexity achieved by Lloyd-Max quantizers for bitwidths of $7$, $6$ and $5$. The difference between the quantizers is clear at 5 bits, where per-tensor FP quantization incurs a drastic and unacceptable increase in perplexity, while Lloyd-Max quantization incurs a much smaller increase. Nevertheless, we note that even the optimal Lloyd-Max quantizer incurs a notable ($\sim 1.5$) increase in perplexity due to the coarse granularity of quantization. 

\begin{figure}[h]
  \centering
  \includegraphics[width=0.7\linewidth]{sections/figures/LM7_FP7.pdf}
  \caption{\small Quantization levels and the corresponding quantization MSE of Floating Point (left) vs Lloyd-Max (right) Quantizers for a layer of weights in the GPT3-126M model.}
  \label{fig:lm_quant}
\end{figure}

\begin{table}[h]\scriptsize
\begin{center}
\caption{\label{tab:FP7_vs_LM7} \small Comparing perplexity (lower is better) achieved by floating point quantizers and Lloyd-Max quantizers on a GPT3-126M model for the Wikitext-103 dataset.}
\begin{tabular}{c|cc|c}
\hline
 \multirow{2}{*}{\textbf{Bitwidth}} & \multicolumn{2}{|c|}{\textbf{Floating-Point Quantizer}} & \textbf{Lloyd-Max Quantizer} \\
 & Best Format & Wikitext-103 Perplexity & Wikitext-103 Perplexity \\
\hline
7 & E3M3 & 18.32 & 18.27 \\
6 & E3M2 & 19.07 & 18.51 \\
5 & E4M0 & 43.89 & 19.71 \\
\hline
\end{tabular}
\end{center}
\end{table}

\subsection{Proof of Local Optimality of LO-BCQ}
\label{subsec:lobcq_opt_proof}
For a given block $\bm{b}_j$, the quantization MSE during LO-BCQ can be empirically evaluated as $\frac{1}{L_b}\lVert \bm{b}_j- \bm{\hat{b}}_j\rVert^2_2$ where $\bm{\hat{b}}_j$ is computed from equation (\ref{eq:clustered_quantization_definition}) as $C_{f(\bm{b}_j)}(\bm{b}_j)$. Further, for a given block cluster $\mathcal{B}_i$, we compute the quantization MSE as $\frac{1}{|\mathcal{B}_{i}|}\sum_{\bm{b} \in \mathcal{B}_{i}} \frac{1}{L_b}\lVert \bm{b}- C_i^{(n)}(\bm{b})\rVert^2_2$. Therefore, at the end of iteration $n$, we evaluate the overall quantization MSE $J^{(n)}$ for a given operand $\bm{X}$ composed of $N_c$ block clusters as:
\begin{align*}
    \label{eq:mse_iter_n}
    J^{(n)} = \frac{1}{N_c} \sum_{i=1}^{N_c} \frac{1}{|\mathcal{B}_{i}^{(n)}|}\sum_{\bm{v} \in \mathcal{B}_{i}^{(n)}} \frac{1}{L_b}\lVert \bm{b}- B_i^{(n)}(\bm{b})\rVert^2_2
\end{align*}

At the end of iteration $n$, the codebooks are updated from $\mathcal{C}^{(n-1)}$ to $\mathcal{C}^{(n)}$. However, the mapping of a given vector $\bm{b}_j$ to quantizers $\mathcal{C}^{(n)}$ remains as  $f^{(n)}(\bm{b}_j)$. At the next iteration, during the vector clustering step, $f^{(n+1)}(\bm{b}_j)$ finds new mapping of $\bm{b}_j$ to updated codebooks $\mathcal{C}^{(n)}$ such that the quantization MSE over the candidate codebooks is minimized. Therefore, we obtain the following result for $\bm{b}_j$:
\begin{align*}
\frac{1}{L_b}\lVert \bm{b}_j - C_{f^{(n+1)}(\bm{b}_j)}^{(n)}(\bm{b}_j)\rVert^2_2 \le \frac{1}{L_b}\lVert \bm{b}_j - C_{f^{(n)}(\bm{b}_j)}^{(n)}(\bm{b}_j)\rVert^2_2
\end{align*}

That is, quantizing $\bm{b}_j$ at the end of the block clustering step of iteration $n+1$ results in lower quantization MSE compared to quantizing at the end of iteration $n$. Since this is true for all $\bm{b} \in \bm{X}$, we assert the following:
\begin{equation}
\begin{split}
\label{eq:mse_ineq_1}
    \tilde{J}^{(n+1)} &= \frac{1}{N_c} \sum_{i=1}^{N_c} \frac{1}{|\mathcal{B}_{i}^{(n+1)}|}\sum_{\bm{b} \in \mathcal{B}_{i}^{(n+1)}} \frac{1}{L_b}\lVert \bm{b} - C_i^{(n)}(b)\rVert^2_2 \le J^{(n)}
\end{split}
\end{equation}
where $\tilde{J}^{(n+1)}$ is the the quantization MSE after the vector clustering step at iteration $n+1$.

Next, during the codebook update step (\ref{eq:quantizers_update}) at iteration $n+1$, the per-cluster codebooks $\mathcal{C}^{(n)}$ are updated to $\mathcal{C}^{(n+1)}$ by invoking the Lloyd-Max algorithm \citep{Lloyd}. We know that for any given value distribution, the Lloyd-Max algorithm minimizes the quantization MSE. Therefore, for a given vector cluster $\mathcal{B}_i$ we obtain the following result:

\begin{equation}
    \frac{1}{|\mathcal{B}_{i}^{(n+1)}|}\sum_{\bm{b} \in \mathcal{B}_{i}^{(n+1)}} \frac{1}{L_b}\lVert \bm{b}- C_i^{(n+1)}(\bm{b})\rVert^2_2 \le \frac{1}{|\mathcal{B}_{i}^{(n+1)}|}\sum_{\bm{b} \in \mathcal{B}_{i}^{(n+1)}} \frac{1}{L_b}\lVert \bm{b}- C_i^{(n)}(\bm{b})\rVert^2_2
\end{equation}

The above equation states that quantizing the given block cluster $\mathcal{B}_i$ after updating the associated codebook from $C_i^{(n)}$ to $C_i^{(n+1)}$ results in lower quantization MSE. Since this is true for all the block clusters, we derive the following result: 
\begin{equation}
\begin{split}
\label{eq:mse_ineq_2}
     J^{(n+1)} &= \frac{1}{N_c} \sum_{i=1}^{N_c} \frac{1}{|\mathcal{B}_{i}^{(n+1)}|}\sum_{\bm{b} \in \mathcal{B}_{i}^{(n+1)}} \frac{1}{L_b}\lVert \bm{b}- C_i^{(n+1)}(\bm{b})\rVert^2_2  \le \tilde{J}^{(n+1)}   
\end{split}
\end{equation}

Following (\ref{eq:mse_ineq_1}) and (\ref{eq:mse_ineq_2}), we find that the quantization MSE is non-increasing for each iteration, that is, $J^{(1)} \ge J^{(2)} \ge J^{(3)} \ge \ldots \ge J^{(M)}$ where $M$ is the maximum number of iterations. 
%Therefore, we can say that if the algorithm converges, then it must be that it has converged to a local minimum. 
\hfill $\blacksquare$


\begin{figure}
    \begin{center}
    \includegraphics[width=0.5\textwidth]{sections//figures/mse_vs_iter.pdf}
    \end{center}
    \caption{\small NMSE vs iterations during LO-BCQ compared to other block quantization proposals}
    \label{fig:nmse_vs_iter}
\end{figure}

Figure \ref{fig:nmse_vs_iter} shows the empirical convergence of LO-BCQ across several block lengths and number of codebooks. Also, the MSE achieved by LO-BCQ is compared to baselines such as MXFP and VSQ. As shown, LO-BCQ converges to a lower MSE than the baselines. Further, we achieve better convergence for larger number of codebooks ($N_c$) and for a smaller block length ($L_b$), both of which increase the bitwidth of BCQ (see Eq \ref{eq:bitwidth_bcq}).


\subsection{Additional Accuracy Results}
%Table \ref{tab:lobcq_config} lists the various LOBCQ configurations and their corresponding bitwidths.
\begin{table}
\setlength{\tabcolsep}{4.75pt}
\begin{center}
\caption{\label{tab:lobcq_config} Various LO-BCQ configurations and their bitwidths.}
\begin{tabular}{|c||c|c|c|c||c|c||c|} 
\hline
 & \multicolumn{4}{|c||}{$L_b=8$} & \multicolumn{2}{|c||}{$L_b=4$} & $L_b=2$ \\
 \hline
 \backslashbox{$L_A$\kern-1em}{\kern-1em$N_c$} & 2 & 4 & 8 & 16 & 2 & 4 & 2 \\
 \hline
 64 & 4.25 & 4.375 & 4.5 & 4.625 & 4.375 & 4.625 & 4.625\\
 \hline
 32 & 4.375 & 4.5 & 4.625& 4.75 & 4.5 & 4.75 & 4.75 \\
 \hline
 16 & 4.625 & 4.75& 4.875 & 5 & 4.75 & 5 & 5 \\
 \hline
\end{tabular}
\end{center}
\end{table}

%\subsection{Perplexity achieved by various LO-BCQ configurations on Wikitext-103 dataset}

\begin{table} \centering
\begin{tabular}{|c||c|c|c|c||c|c||c|} 
\hline
 $L_b \rightarrow$& \multicolumn{4}{c||}{8} & \multicolumn{2}{c||}{4} & 2\\
 \hline
 \backslashbox{$L_A$\kern-1em}{\kern-1em$N_c$} & 2 & 4 & 8 & 16 & 2 & 4 & 2  \\
 %$N_c \rightarrow$ & 2 & 4 & 8 & 16 & 2 & 4 & 2 \\
 \hline
 \hline
 \multicolumn{8}{c}{GPT3-1.3B (FP32 PPL = 9.98)} \\ 
 \hline
 \hline
 64 & 10.40 & 10.23 & 10.17 & 10.15 &  10.28 & 10.18 & 10.19 \\
 \hline
 32 & 10.25 & 10.20 & 10.15 & 10.12 &  10.23 & 10.17 & 10.17 \\
 \hline
 16 & 10.22 & 10.16 & 10.10 & 10.09 &  10.21 & 10.14 & 10.16 \\
 \hline
  \hline
 \multicolumn{8}{c}{GPT3-8B (FP32 PPL = 7.38)} \\ 
 \hline
 \hline
 64 & 7.61 & 7.52 & 7.48 &  7.47 &  7.55 &  7.49 & 7.50 \\
 \hline
 32 & 7.52 & 7.50 & 7.46 &  7.45 &  7.52 &  7.48 & 7.48  \\
 \hline
 16 & 7.51 & 7.48 & 7.44 &  7.44 &  7.51 &  7.49 & 7.47  \\
 \hline
\end{tabular}
\caption{\label{tab:ppl_gpt3_abalation} Wikitext-103 perplexity across GPT3-1.3B and 8B models.}
\end{table}

\begin{table} \centering
\begin{tabular}{|c||c|c|c|c||} 
\hline
 $L_b \rightarrow$& \multicolumn{4}{c||}{8}\\
 \hline
 \backslashbox{$L_A$\kern-1em}{\kern-1em$N_c$} & 2 & 4 & 8 & 16 \\
 %$N_c \rightarrow$ & 2 & 4 & 8 & 16 & 2 & 4 & 2 \\
 \hline
 \hline
 \multicolumn{5}{|c|}{Llama2-7B (FP32 PPL = 5.06)} \\ 
 \hline
 \hline
 64 & 5.31 & 5.26 & 5.19 & 5.18  \\
 \hline
 32 & 5.23 & 5.25 & 5.18 & 5.15  \\
 \hline
 16 & 5.23 & 5.19 & 5.16 & 5.14  \\
 \hline
 \multicolumn{5}{|c|}{Nemotron4-15B (FP32 PPL = 5.87)} \\ 
 \hline
 \hline
 64  & 6.3 & 6.20 & 6.13 & 6.08  \\
 \hline
 32  & 6.24 & 6.12 & 6.07 & 6.03  \\
 \hline
 16  & 6.12 & 6.14 & 6.04 & 6.02  \\
 \hline
 \multicolumn{5}{|c|}{Nemotron4-340B (FP32 PPL = 3.48)} \\ 
 \hline
 \hline
 64 & 3.67 & 3.62 & 3.60 & 3.59 \\
 \hline
 32 & 3.63 & 3.61 & 3.59 & 3.56 \\
 \hline
 16 & 3.61 & 3.58 & 3.57 & 3.55 \\
 \hline
\end{tabular}
\caption{\label{tab:ppl_llama7B_nemo15B} Wikitext-103 perplexity compared to FP32 baseline in Llama2-7B and Nemotron4-15B, 340B models}
\end{table}

%\subsection{Perplexity achieved by various LO-BCQ configurations on MMLU dataset}


\begin{table} \centering
\begin{tabular}{|c||c|c|c|c||c|c|c|c|} 
\hline
 $L_b \rightarrow$& \multicolumn{4}{c||}{8} & \multicolumn{4}{c||}{8}\\
 \hline
 \backslashbox{$L_A$\kern-1em}{\kern-1em$N_c$} & 2 & 4 & 8 & 16 & 2 & 4 & 8 & 16  \\
 %$N_c \rightarrow$ & 2 & 4 & 8 & 16 & 2 & 4 & 2 \\
 \hline
 \hline
 \multicolumn{5}{|c|}{Llama2-7B (FP32 Accuracy = 45.8\%)} & \multicolumn{4}{|c|}{Llama2-70B (FP32 Accuracy = 69.12\%)} \\ 
 \hline
 \hline
 64 & 43.9 & 43.4 & 43.9 & 44.9 & 68.07 & 68.27 & 68.17 & 68.75 \\
 \hline
 32 & 44.5 & 43.8 & 44.9 & 44.5 & 68.37 & 68.51 & 68.35 & 68.27  \\
 \hline
 16 & 43.9 & 42.7 & 44.9 & 45 & 68.12 & 68.77 & 68.31 & 68.59  \\
 \hline
 \hline
 \multicolumn{5}{|c|}{GPT3-22B (FP32 Accuracy = 38.75\%)} & \multicolumn{4}{|c|}{Nemotron4-15B (FP32 Accuracy = 64.3\%)} \\ 
 \hline
 \hline
 64 & 36.71 & 38.85 & 38.13 & 38.92 & 63.17 & 62.36 & 63.72 & 64.09 \\
 \hline
 32 & 37.95 & 38.69 & 39.45 & 38.34 & 64.05 & 62.30 & 63.8 & 64.33  \\
 \hline
 16 & 38.88 & 38.80 & 38.31 & 38.92 & 63.22 & 63.51 & 63.93 & 64.43  \\
 \hline
\end{tabular}
\caption{\label{tab:mmlu_abalation} Accuracy on MMLU dataset across GPT3-22B, Llama2-7B, 70B and Nemotron4-15B models.}
\end{table}


%\subsection{Perplexity achieved by various LO-BCQ configurations on LM evaluation harness}

\begin{table} \centering
\begin{tabular}{|c||c|c|c|c||c|c|c|c|} 
\hline
 $L_b \rightarrow$& \multicolumn{4}{c||}{8} & \multicolumn{4}{c||}{8}\\
 \hline
 \backslashbox{$L_A$\kern-1em}{\kern-1em$N_c$} & 2 & 4 & 8 & 16 & 2 & 4 & 8 & 16  \\
 %$N_c \rightarrow$ & 2 & 4 & 8 & 16 & 2 & 4 & 2 \\
 \hline
 \hline
 \multicolumn{5}{|c|}{Race (FP32 Accuracy = 37.51\%)} & \multicolumn{4}{|c|}{Boolq (FP32 Accuracy = 64.62\%)} \\ 
 \hline
 \hline
 64 & 36.94 & 37.13 & 36.27 & 37.13 & 63.73 & 62.26 & 63.49 & 63.36 \\
 \hline
 32 & 37.03 & 36.36 & 36.08 & 37.03 & 62.54 & 63.51 & 63.49 & 63.55  \\
 \hline
 16 & 37.03 & 37.03 & 36.46 & 37.03 & 61.1 & 63.79 & 63.58 & 63.33  \\
 \hline
 \hline
 \multicolumn{5}{|c|}{Winogrande (FP32 Accuracy = 58.01\%)} & \multicolumn{4}{|c|}{Piqa (FP32 Accuracy = 74.21\%)} \\ 
 \hline
 \hline
 64 & 58.17 & 57.22 & 57.85 & 58.33 & 73.01 & 73.07 & 73.07 & 72.80 \\
 \hline
 32 & 59.12 & 58.09 & 57.85 & 58.41 & 73.01 & 73.94 & 72.74 & 73.18  \\
 \hline
 16 & 57.93 & 58.88 & 57.93 & 58.56 & 73.94 & 72.80 & 73.01 & 73.94  \\
 \hline
\end{tabular}
\caption{\label{tab:mmlu_abalation} Accuracy on LM evaluation harness tasks on GPT3-1.3B model.}
\end{table}

\begin{table} \centering
\begin{tabular}{|c||c|c|c|c||c|c|c|c|} 
\hline
 $L_b \rightarrow$& \multicolumn{4}{c||}{8} & \multicolumn{4}{c||}{8}\\
 \hline
 \backslashbox{$L_A$\kern-1em}{\kern-1em$N_c$} & 2 & 4 & 8 & 16 & 2 & 4 & 8 & 16  \\
 %$N_c \rightarrow$ & 2 & 4 & 8 & 16 & 2 & 4 & 2 \\
 \hline
 \hline
 \multicolumn{5}{|c|}{Race (FP32 Accuracy = 41.34\%)} & \multicolumn{4}{|c|}{Boolq (FP32 Accuracy = 68.32\%)} \\ 
 \hline
 \hline
 64 & 40.48 & 40.10 & 39.43 & 39.90 & 69.20 & 68.41 & 69.45 & 68.56 \\
 \hline
 32 & 39.52 & 39.52 & 40.77 & 39.62 & 68.32 & 67.43 & 68.17 & 69.30  \\
 \hline
 16 & 39.81 & 39.71 & 39.90 & 40.38 & 68.10 & 66.33 & 69.51 & 69.42  \\
 \hline
 \hline
 \multicolumn{5}{|c|}{Winogrande (FP32 Accuracy = 67.88\%)} & \multicolumn{4}{|c|}{Piqa (FP32 Accuracy = 78.78\%)} \\ 
 \hline
 \hline
 64 & 66.85 & 66.61 & 67.72 & 67.88 & 77.31 & 77.42 & 77.75 & 77.64 \\
 \hline
 32 & 67.25 & 67.72 & 67.72 & 67.00 & 77.31 & 77.04 & 77.80 & 77.37  \\
 \hline
 16 & 68.11 & 68.90 & 67.88 & 67.48 & 77.37 & 78.13 & 78.13 & 77.69  \\
 \hline
\end{tabular}
\caption{\label{tab:mmlu_abalation} Accuracy on LM evaluation harness tasks on GPT3-8B model.}
\end{table}

\begin{table} \centering
\begin{tabular}{|c||c|c|c|c||c|c|c|c|} 
\hline
 $L_b \rightarrow$& \multicolumn{4}{c||}{8} & \multicolumn{4}{c||}{8}\\
 \hline
 \backslashbox{$L_A$\kern-1em}{\kern-1em$N_c$} & 2 & 4 & 8 & 16 & 2 & 4 & 8 & 16  \\
 %$N_c \rightarrow$ & 2 & 4 & 8 & 16 & 2 & 4 & 2 \\
 \hline
 \hline
 \multicolumn{5}{|c|}{Race (FP32 Accuracy = 40.67\%)} & \multicolumn{4}{|c|}{Boolq (FP32 Accuracy = 76.54\%)} \\ 
 \hline
 \hline
 64 & 40.48 & 40.10 & 39.43 & 39.90 & 75.41 & 75.11 & 77.09 & 75.66 \\
 \hline
 32 & 39.52 & 39.52 & 40.77 & 39.62 & 76.02 & 76.02 & 75.96 & 75.35  \\
 \hline
 16 & 39.81 & 39.71 & 39.90 & 40.38 & 75.05 & 73.82 & 75.72 & 76.09  \\
 \hline
 \hline
 \multicolumn{5}{|c|}{Winogrande (FP32 Accuracy = 70.64\%)} & \multicolumn{4}{|c|}{Piqa (FP32 Accuracy = 79.16\%)} \\ 
 \hline
 \hline
 64 & 69.14 & 70.17 & 70.17 & 70.56 & 78.24 & 79.00 & 78.62 & 78.73 \\
 \hline
 32 & 70.96 & 69.69 & 71.27 & 69.30 & 78.56 & 79.49 & 79.16 & 78.89  \\
 \hline
 16 & 71.03 & 69.53 & 69.69 & 70.40 & 78.13 & 79.16 & 79.00 & 79.00  \\
 \hline
\end{tabular}
\caption{\label{tab:mmlu_abalation} Accuracy on LM evaluation harness tasks on GPT3-22B model.}
\end{table}

\begin{table} \centering
\begin{tabular}{|c||c|c|c|c||c|c|c|c|} 
\hline
 $L_b \rightarrow$& \multicolumn{4}{c||}{8} & \multicolumn{4}{c||}{8}\\
 \hline
 \backslashbox{$L_A$\kern-1em}{\kern-1em$N_c$} & 2 & 4 & 8 & 16 & 2 & 4 & 8 & 16  \\
 %$N_c \rightarrow$ & 2 & 4 & 8 & 16 & 2 & 4 & 2 \\
 \hline
 \hline
 \multicolumn{5}{|c|}{Race (FP32 Accuracy = 44.4\%)} & \multicolumn{4}{|c|}{Boolq (FP32 Accuracy = 79.29\%)} \\ 
 \hline
 \hline
 64 & 42.49 & 42.51 & 42.58 & 43.45 & 77.58 & 77.37 & 77.43 & 78.1 \\
 \hline
 32 & 43.35 & 42.49 & 43.64 & 43.73 & 77.86 & 75.32 & 77.28 & 77.86  \\
 \hline
 16 & 44.21 & 44.21 & 43.64 & 42.97 & 78.65 & 77 & 76.94 & 77.98  \\
 \hline
 \hline
 \multicolumn{5}{|c|}{Winogrande (FP32 Accuracy = 69.38\%)} & \multicolumn{4}{|c|}{Piqa (FP32 Accuracy = 78.07\%)} \\ 
 \hline
 \hline
 64 & 68.9 & 68.43 & 69.77 & 68.19 & 77.09 & 76.82 & 77.09 & 77.86 \\
 \hline
 32 & 69.38 & 68.51 & 68.82 & 68.90 & 78.07 & 76.71 & 78.07 & 77.86  \\
 \hline
 16 & 69.53 & 67.09 & 69.38 & 68.90 & 77.37 & 77.8 & 77.91 & 77.69  \\
 \hline
\end{tabular}
\caption{\label{tab:mmlu_abalation} Accuracy on LM evaluation harness tasks on Llama2-7B model.}
\end{table}

\begin{table} \centering
\begin{tabular}{|c||c|c|c|c||c|c|c|c|} 
\hline
 $L_b \rightarrow$& \multicolumn{4}{c||}{8} & \multicolumn{4}{c||}{8}\\
 \hline
 \backslashbox{$L_A$\kern-1em}{\kern-1em$N_c$} & 2 & 4 & 8 & 16 & 2 & 4 & 8 & 16  \\
 %$N_c \rightarrow$ & 2 & 4 & 8 & 16 & 2 & 4 & 2 \\
 \hline
 \hline
 \multicolumn{5}{|c|}{Race (FP32 Accuracy = 48.8\%)} & \multicolumn{4}{|c|}{Boolq (FP32 Accuracy = 85.23\%)} \\ 
 \hline
 \hline
 64 & 49.00 & 49.00 & 49.28 & 48.71 & 82.82 & 84.28 & 84.03 & 84.25 \\
 \hline
 32 & 49.57 & 48.52 & 48.33 & 49.28 & 83.85 & 84.46 & 84.31 & 84.93  \\
 \hline
 16 & 49.85 & 49.09 & 49.28 & 48.99 & 85.11 & 84.46 & 84.61 & 83.94  \\
 \hline
 \hline
 \multicolumn{5}{|c|}{Winogrande (FP32 Accuracy = 79.95\%)} & \multicolumn{4}{|c|}{Piqa (FP32 Accuracy = 81.56\%)} \\ 
 \hline
 \hline
 64 & 78.77 & 78.45 & 78.37 & 79.16 & 81.45 & 80.69 & 81.45 & 81.5 \\
 \hline
 32 & 78.45 & 79.01 & 78.69 & 80.66 & 81.56 & 80.58 & 81.18 & 81.34  \\
 \hline
 16 & 79.95 & 79.56 & 79.79 & 79.72 & 81.28 & 81.66 & 81.28 & 80.96  \\
 \hline
\end{tabular}
\caption{\label{tab:mmlu_abalation} Accuracy on LM evaluation harness tasks on Llama2-70B model.}
\end{table}

%\section{MSE Studies}
%\textcolor{red}{TODO}


\subsection{Number Formats and Quantization Method}
\label{subsec:numFormats_quantMethod}
\subsubsection{Integer Format}
An $n$-bit signed integer (INT) is typically represented with a 2s-complement format \citep{yao2022zeroquant,xiao2023smoothquant,dai2021vsq}, where the most significant bit denotes the sign.

\subsubsection{Floating Point Format}
An $n$-bit signed floating point (FP) number $x$ comprises of a 1-bit sign ($x_{\mathrm{sign}}$), $B_m$-bit mantissa ($x_{\mathrm{mant}}$) and $B_e$-bit exponent ($x_{\mathrm{exp}}$) such that $B_m+B_e=n-1$. The associated constant exponent bias ($E_{\mathrm{bias}}$) is computed as $(2^{{B_e}-1}-1)$. We denote this format as $E_{B_e}M_{B_m}$.  

\subsubsection{Quantization Scheme}
\label{subsec:quant_method}
A quantization scheme dictates how a given unquantized tensor is converted to its quantized representation. We consider FP formats for the purpose of illustration. Given an unquantized tensor $\bm{X}$ and an FP format $E_{B_e}M_{B_m}$, we first, we compute the quantization scale factor $s_X$ that maps the maximum absolute value of $\bm{X}$ to the maximum quantization level of the $E_{B_e}M_{B_m}$ format as follows:
\begin{align}
\label{eq:sf}
    s_X = \frac{\mathrm{max}(|\bm{X}|)}{\mathrm{max}(E_{B_e}M_{B_m})}
\end{align}
In the above equation, $|\cdot|$ denotes the absolute value function.

Next, we scale $\bm{X}$ by $s_X$ and quantize it to $\hat{\bm{X}}$ by rounding it to the nearest quantization level of $E_{B_e}M_{B_m}$ as:

\begin{align}
\label{eq:tensor_quant}
    \hat{\bm{X}} = \text{round-to-nearest}\left(\frac{\bm{X}}{s_X}, E_{B_e}M_{B_m}\right)
\end{align}

We perform dynamic max-scaled quantization \citep{wu2020integer}, where the scale factor $s$ for activations is dynamically computed during runtime.

\subsection{Vector Scaled Quantization}
\begin{wrapfigure}{r}{0.35\linewidth}
  \centering
  \includegraphics[width=\linewidth]{sections/figures/vsquant.jpg}
  \caption{\small Vectorwise decomposition for per-vector scaled quantization (VSQ \citep{dai2021vsq}).}
  \label{fig:vsquant}
\end{wrapfigure}
During VSQ \citep{dai2021vsq}, the operand tensors are decomposed into 1D vectors in a hardware friendly manner as shown in Figure \ref{fig:vsquant}. Since the decomposed tensors are used as operands in matrix multiplications during inference, it is beneficial to perform this decomposition along the reduction dimension of the multiplication. The vectorwise quantization is performed similar to tensorwise quantization described in Equations \ref{eq:sf} and \ref{eq:tensor_quant}, where a scale factor $s_v$ is required for each vector $\bm{v}$ that maps the maximum absolute value of that vector to the maximum quantization level. While smaller vector lengths can lead to larger accuracy gains, the associated memory and computational overheads due to the per-vector scale factors increases. To alleviate these overheads, VSQ \citep{dai2021vsq} proposed a second level quantization of the per-vector scale factors to unsigned integers, while MX \citep{rouhani2023shared} quantizes them to integer powers of 2 (denoted as $2^{INT}$).

\subsubsection{MX Format}
The MX format proposed in \citep{rouhani2023microscaling} introduces the concept of sub-block shifting. For every two scalar elements of $b$-bits each, there is a shared exponent bit. The value of this exponent bit is determined through an empirical analysis that targets minimizing quantization MSE. We note that the FP format $E_{1}M_{b}$ is strictly better than MX from an accuracy perspective since it allocates a dedicated exponent bit to each scalar as opposed to sharing it across two scalars. Therefore, we conservatively bound the accuracy of a $b+2$-bit signed MX format with that of a $E_{1}M_{b}$ format in our comparisons. For instance, we use E1M2 format as a proxy for MX4.

\begin{figure}
    \centering
    \includegraphics[width=1\linewidth]{sections//figures/BlockFormats.pdf}
    \caption{\small Comparing LO-BCQ to MX format.}
    \label{fig:block_formats}
\end{figure}

Figure \ref{fig:block_formats} compares our $4$-bit LO-BCQ block format to MX \citep{rouhani2023microscaling}. As shown, both LO-BCQ and MX decompose a given operand tensor into block arrays and each block array into blocks. Similar to MX, we find that per-block quantization ($L_b < L_A$) leads to better accuracy due to increased flexibility. While MX achieves this through per-block $1$-bit micro-scales, we associate a dedicated codebook to each block through a per-block codebook selector. Further, MX quantizes the per-block array scale-factor to E8M0 format without per-tensor scaling. In contrast during LO-BCQ, we find that per-tensor scaling combined with quantization of per-block array scale-factor to E4M3 format results in superior inference accuracy across models. 


\end{document}
