To evaluate representations for downstream MIR tasks, we follow the procedure as outlined in \cite{JukeMIR}: shallow supervised models (linear models and one-layer MLPs) are trained on each task using the representations as input features. A grid search over the following 216 hyperparameter configurations is conducted:

\begin{itemize}
    \item Feature standardization: \{off, on\}
    \item Model type: \{Linear, one-layer MLP with 512 hidden units\}
    \item Batch size: \{64, 256\}
    \item Learning rate: \{1e-5, 1e-4, 1e-3\}
    \item Dropout probability: \{0.25, 0.5, 0.75\}
    \item L2 regularization: \{0, 1e-4, 1e-3\}
\end{itemize}

Early stopping is applied based on task-specific metrics computed on validation sets, with the optimal model from each grid search evaluated on the task-specific test set. Loss functions are tailored to each task: cross-entropy for genre classification and key detection, independent binary cross-entropy for tagging, and mean squared error for emotion recognition.


\begin{figure*}
\begin{tcolorbox}[boxsep=2pt,left=2pt,right=2pt,top=2pt,bottom=2pt,colback=White,]
\begin{multicols}{2}

\textbf{Findings List:}

Abdomen pain left upper quadrant (absent)\\
Abdomen pain non-colicky (absent)\\
Anorexia (present)\\
Chest pain lateral sharp (present)\\
Chest pain substernal exacerbation with cough (present)\\
Cigarette smoking history (present)\\
Dyspnea abrupt onset (present)\\
Headache severe (present)\\
Myalgia (present)\\
Onset abrupt (absent)\\
Rigors (present)\\
Shoulder pain left (absent)\\
\textit{larger }

% Sex male (present)\\
% Age 16 to 25 (present)\\
%Fever (present)\\
% Dyspnea at rest (present)\\


\textbf{DDx List:}\\
pulmonary legionellosis \textbf{(seed disease)}\\
pneumococcal pneumonia\\
mycoplasma pneumonia\\
brucellosis

\columnbreak
\textbf{Patient-Doctor Chat:}\\
\textbf{System:} Patient is Tyler, a 24-year-old male.\\
\textbf{Doctor:} Hi Tyler, I'm here to help you today. Can you tell me what's been going on?\\
\textbf{Patient:} Hey doc, I've been feeling really off lately.\\
\textbf{Doctor:} Can you describe what you mean by off? Are you experiencing any pain or discomfort?\\
\textbf{Patient:} Yeah, there's this sharp pain on the side of my chest, and it gets worse when I breathe.\\
\textbf{[chest pain lateral sharp-present]}\\
\textbf{Doctor:} That sounds uncomfortable. Have you noticed any other types of chest pain, maybe when you cough?\\
\textbf{Patient:} Not really sharp pain, but it does hurt under my sternum when I cough.\\
\textbf{[chest pain substernal exacerbation with cough-present]}



\end{multicols}
\end{tcolorbox}
\caption{Example expert system simulation to chat}
\label{fig:chat_simulation_ex_app}
\end{figure*}

\begin{table*}
\centering
\begin{tabular}{p{0.5\textwidth}p{0.1\textwidth}p{0.1\textwidth}p{0.08\textwidth}p{0.08\textwidth}}
        \toprule
        \textbf{Dx Category} & \textbf{Top K Accuracy} & \textbf{Top 1 Accuracy} & \textbf{MRR} & \textbf{N} \\
        \midrule
        Congenital disorders due to abnormal fetal development & \cellcolor{blue!0}0.533 & \cellcolor{blue!34}0.433 & \cellcolor{blue!25}0.458 & \cellcolor{blue!0}30 \\
        Immune system disorders & \cellcolor{blue!7}0.548 & \cellcolor{blue!2}0.190 & \cellcolor{blue!10}0.295 & \cellcolor{blue!10}42 \\
        End organ damage secondary to other disorders & \cellcolor{blue!14}0.583 & \cellcolor{blue!0}0.083 & \cellcolor{blue!0}0.222 & \cellcolor{blue!7}36 \\
        Disorders with excess or abnormal fluid accumulation & \cellcolor{blue!22}0.603 & \cellcolor{blue!10}0.256 & \cellcolor{blue!12}0.370 & \cellcolor{blue!28}78 \\
        Disorders of the hematopoietic and lymphatic systems & \cellcolor{blue!34}0.667 & \cellcolor{blue!0}0.000 & \cellcolor{blue!10}0.306 & \cellcolor{blue!0}6 \\
        Drug induced injury alias adverse drug effects & \cellcolor{blue!40}0.699 & \cellcolor{blue!2}0.192 & \cellcolor{blue!15}0.361 & \cellcolor{blue!26}73 \\
        Degenerative disorders & \cellcolor{blue!45}0.704 & \cellcolor{blue!8}0.241 & \cellcolor{blue!20}0.391 & \cellcolor{blue!38}108 \\
        Infectious disease alias infections & \cellcolor{blue!49}0.711 & \cellcolor{blue!10}0.278 & \cellcolor{blue!25}0.424 & \cellcolor{blue!50}862 \\
        Neoplastic disease & \cellcolor{blue!50}0.721 & \cellcolor{blue!8}0.236 & \cellcolor{blue!22}0.384 & \cellcolor{blue!19}330 \\
        Disorders involving cysts stones or calculi & \cellcolor{blue!50}0.722 & \cellcolor{blue!43}0.611 & \cellcolor{blue!32}0.650 & \cellcolor{blue!0}18 \\
        Impaired cardiovascular function & \cellcolor{blue!50}0.738 & \cellcolor{blue!24}0.359 & \cellcolor{blue!32}0.486 & \cellcolor{blue!22}390 \\
        Disorders due to toxic or chemical or radiation injury & \cellcolor{blue!50}0.745 & \cellcolor{blue!31}0.398 & \cellcolor{blue!36}0.523 & \cellcolor{blue!27}98 \\
        Musculoskeletal disorders & \cellcolor{blue!50}0.750 & \cellcolor{blue!12}0.292 & \cellcolor{blue!30}0.467 & \cellcolor{blue!0}24 \\
        Non-infectious inflammatory disease & \cellcolor{blue!50}0.753 & \cellcolor{blue!21}0.326 & \cellcolor{blue!32}0.477 & \cellcolor{blue!22}384 \\
        Metabolic disorders & \cellcolor{blue!50}0.778 & \cellcolor{blue!21}0.333 & \cellcolor{blue!32}0.482 & \cellcolor{blue!38}144 \\
        Bleeding disorders and coagulopathies & \cellcolor{blue!50}0.783 & \cellcolor{blue!14}0.267 & \cellcolor{blue!28}0.449 & \cellcolor{blue!12}60 \\
        Disorders of thorax cardiovascular system and lymphatic ducts & \cellcolor{blue!50}0.783 & \cellcolor{blue!34}0.522 & \cellcolor{blue!30}0.590 & \cellcolor{blue!0}23 \\
        Endocrine disease & \cellcolor{blue!50}0.788 & \cellcolor{blue!30}0.417 & \cellcolor{blue!34}0.547 & \cellcolor{blue!35}132 \\
        Fibrosis or scarring of visceral organ & \cellcolor{blue!50}0.796 & \cellcolor{blue!8}0.245 & \cellcolor{blue!27}0.436 & \cellcolor{blue!8}49 \\
        Neuropsychiatric disorders & \cellcolor{blue!50}0.803 & \cellcolor{blue!35}0.439 & \cellcolor{blue!38}0.573 & \cellcolor{blue!10}66 \\
        Disorders secondary to trauma & \cellcolor{blue!50}0.806 & \cellcolor{blue!24}0.361 & \cellcolor{blue!35}0.519 & \cellcolor{blue!7}36 \\
        Impaired fluid flow within hollow viscus or viscera non-vascular & \cellcolor{blue!50}0.833 & \cellcolor{blue!43}0.583 & \cellcolor{blue!32}0.649 & \cellcolor{blue!0}24 \\
        Multisystem disorders & \cellcolor{blue!50}0.833 & \cellcolor{blue!50}0.667 & \cellcolor{blue!42}0.708 & \cellcolor{blue!0}6 \\
        Disorders associated with pregnancy & \cellcolor{blue!50}0.833 & \cellcolor{blue!34}0.542 & \cellcolor{blue!42}0.663 & \cellcolor{blue!0}24 \\
        Inherited congenital or degenerative disorders & \cellcolor{blue!50}0.847 & \cellcolor{blue!35}0.458 & \cellcolor{blue!30}0.588 & \cellcolor{blue!38}144 \\
        Miscellaneous mechanical disorders & \cellcolor{blue!50}0.861 & \cellcolor{blue!34}0.542 & \cellcolor{blue!36}0.655 & \cellcolor{blue!24}72 \\
        Kidney and urinary tract disorders & \cellcolor{blue!50}0.875 & \cellcolor{blue!35}0.458 & \cellcolor{blue!36}0.654 & \cellcolor{blue!0}24 \\
        Disorders due to mechanical tear or trauma or visceral erosion & \cellcolor{blue!50}0.875 & \cellcolor{blue!50}0.667 & \cellcolor{blue!50}0.764 & \cellcolor{blue!0}24 \\
        Disorders of abdomen digestive system and/or nutrition & \cellcolor{blue!50}0.889 & \cellcolor{blue!24}0.389 & \cellcolor{blue!38}0.573 & \cellcolor{blue!0}18 \\
        Disorders due to nutritional and/or vitamin deficiency & \cellcolor{blue!50}0.917 & \cellcolor{blue!43}0.611 & \cellcolor{blue!50}0.738 & \cellcolor{blue!7}36 \\
        Electrophysiological neurological disorders & \cellcolor{blue!50}0.917 & \cellcolor{blue!34}0.542 & \cellcolor{blue!42}0.687 & \cellcolor{blue!0}24 \\
        Disorders of smooth muscle contraction and/or relaxation & \cellcolor{blue!50}0.944 & \cellcolor{blue!50}0.722 & \cellcolor{blue!50}0.801 & \cellcolor{blue!0}18 \\
        \bottomrule
\end{tabular}
\caption{Performance from \methodname, gpt-4o in Table \ref{tab:ddx} broken down by disease category. Note that a disease may fall into multiple categories.}
\label{tab:ddx_by_cat}
\end{table*}

\begin{table*}
    \centering
    \centering
    \begin{tabular}{c | c c c c}
        \toprule
        \textbf{generation model} & \textbf{split} & \textbf{size} & \textbf{findings} & \textbf{messages} \\ 
        \midrule
        \multirow{3}{*}{\textbf{gpt-4o}} 
        & train & 21782 & 11.88 $\pm$ 1.91 & 17.54 $\pm$ 3.58 \\ 
        & val & 3404 & 11.69 $\pm$ 2.08 & 17.31 $\pm$ 3.78 \\ 
        & test & 3403 & 11.66 $\pm$ 2.05 & 17.27 $\pm$ 3.69 \\ 
        \midrule
        \multirow{3}{*}{\textbf{claude}} 
        & train & 8837 & 11.88 $\pm$ 1.90 & 14.54 $\pm$ 3.47 \\ 
        & val & 2868 & 11.69 $\pm$ 2.04 & 14.11 $\pm$ 3.38 \\ 
        & test & 2868 & 11.70 $\pm$ 2.08 & 14.30 $\pm$ 3.48 \\ 
        \bottomrule
    \end{tabular}
    \caption{Data statistics for train, validation, and test sets.  We include the number of chats, the average number of findings and standard deviation, and the average number of messages and standard deviation.}
    \label{tab:data_details}

\end{table*}

\begin{table*}
\small
\centering
\begin{tabular}{p{0.1\textwidth} p{0.1\textwidth} p{0.15\textwidth} p{0.47\textwidth}}
    \toprule
   Synthetic & Annotator & Disease & Reasoning \\ 
    \midrule
    yes & yes & histoplasma meningitis & This disease is possible because the patient’s history of lymphoma and transplant makes them immunocompromised, which increases their risk of histoplasmosis meningitis. The histoplasmosis meningitis could cause this patient's symptoms of nausea/vomiting, fever/chills, and severe headache. \\      \midrule
    yes & yes & cerebral malaria & This disease is possible because urinary and bowel incontinence, intractable headache, visual loss/retinopathy, and fever point to an infectious cerebral process that could be caused by cerebral malaria. \\    \midrule
    yes & yes & neurogenic osteoarthropathy alias charcot\_joint\_disease & This could be possible charcot joint disease, which is set off by trauma to a neuropathic extremity and can cause joint pain. Although Charcot's is technically most common in the foot, it can also extend to any major joint like the knees, shoulder, hip, etc. \\     \midrule
    yes & yes & glaucoma acute angle closure & Blurriness with rainbow rings/halos, nausea/vomiting, headache, decreased vision, and sudden onset are all symptoms of acute angle-closure glaucoma. \\    \midrule
    yes & yes & cytomegalovirus infection disseminated & This disease is possible because the patient is immunocompromised by their organ transplant, and their symptoms of vision changes, diarrhea, fever, chills, vomiting, and myalgias are consistent with a disseminated cytomegalovirus infection. \\     \midrule
    yes & no & cutaneous anthrax & Lack of bump/ulcer/eschar. though presence of pruritis and exposure to possible animals infected as a vet, symptoms are nonspecific. \\     \midrule
    yes & no & herpes zoster & This disease is not possible because although herpes zoster can have peripheral symptoms including this patient's myalgia, headache, and abdominal pain, this patient does not have the characteristic dermatomal rash, skin changes, or skin-level pain of herpes zoster. \\    \midrule
    no & no & pulmonary aspergillosis invasive type & This disease is not possible because the patient is not immunocompromised, has not had recent surgeries or pneumonia, or chemotherapy, and does not have a cough, which I would expect from pulmonary aspergillosis since it is an opportunistic pulmonary infection. \\     \midrule
    no & no & henoch schonlein syndrome alias henoch-schonlein purpura & This disease is not possible because the patient does not report the purpura around the legs/gluteus that is characteristic for Henoch Schonlein purpura. The patient is also not in the typical age group for this disease, which primarily is in pediatric populations. \\     \midrule
    no & yes & bronchial asthma & This disease is possible because of the patient's history of dyspnea at rest and worse with activity. The urinary frequency could also be related because there are positive associations between bronchial asthma and increased urge to urinate. \\     
    \bottomrule
\end{tabular}
\caption{Sample explanations from the annotation task. Note that the disease column indicates which disease they were asked to annotate against -- this is the expected disease for the positive synthetic labels, but is not expected for the negative ones.}
\label{tab:ann_reasoning}
\end{table*}

\begin{table*}
    \centering
    \begin{tabular}{c | c c c | c c c}
        \toprule
        & \multicolumn{3}{c|}{\textbf{gpt-4o chats}} & \multicolumn{3}{c}{\textbf{claude chats}} \\
        \cmidrule(lr){2-4} \cmidrule(lr){5-7}
        & \textbf{Top K} & \textbf{Top-1} & \textbf{MRR} & \textbf{Top K} & \textbf{Top-1} & \textbf{MRR} \\ 
        \midrule
        gpt-4o only & 88.28\% & 63.13\% & 0.7384 & 47.25\% & 31.80\% & 0.3817 \\ 
        claude only & 48.38\% & 34.43\% & 0.4025 & 65.93\% & 45.71\% & 0.5415 \\ 
        combined & 88.57\% & 64.10\% & 0.7444 & 79.64\% & 57.46\% & 0.6675 \\ 
        \bottomrule
    \end{tabular}
    \caption{Candidate generation metrics for validation set.}
    \label{tab:validation_performance}
\end{table*}




\begin{figure*}[p]
\begin{tcolorbox}[boxsep=2pt,left=2pt,right=2pt,top=2pt,bottom=2pt,colback=White,]
\textbf{gpt-4o Generated Chat for disseminated intravascular coagulation}
\vspace{-2em}
\begin{chat}
\who{System} Patient is George, a 67-year-old male. \\
\textit{[age greater than 55 (present) , sex male (present)]}
\who{Doctor} Hi George, I'm here to help you today. Can you tell me what's been going on?
\who{Patient} Well, it all started pretty suddenly. I've been feeling really off. \\
\textit{[onset abrupt (present)]}
\who{Doctor} I see. Can you describe some of the symptoms you've been experiencing?
\who{Patient} I've had a fever on and off. It's been bothering me. \\
\textit{[fever (present)]}
\who{Doctor} Alright, fever can be quite uncomfortable. Have you noticed any changes in your appetite?
\who{Patient} I haven't really felt like eating much lately. Just no appetite. \\
\textit{[anorexia (present)]}
\who{Doctor} Got it. Have you experienced any issues with bleeding or bruising?
\who{Patient} Yeah, even a small bump seems to make me bleed a lot. It's strange. \\
\textit{[bleeding excessive after minor trauma (present)]}
\who{Doctor} That does sound unusual. Have you had any nosebleeds recently?
\who{Patient} No, I haven't had any nosebleeds. \\
\textit{[epistaxis or epistaxis severe history (absent)]}
\who{Doctor} Okay, and how about your breathing? Any issues there?
\who{Patient} Yes, I've been feeling short of breath even when I'm just sitting around. \\
\textit{[dyspnea at rest (present)]}
\who{Doctor} That's important to note. Have you had any recent surgeries?
\who{Patient} Yes, I had surgery a few weeks ago. \\
\textit{[surgery recent history (present)]}
\who{Doctor} Thanks for sharing that. Have you experienced any vomiting?
\who{Patient} Yes, I've been throwing up, and sometimes it looks really dark, like coffee grounds. \\
\textit{[vomiting coffee ground (present) , vomiting recent (present)]}
\who{Doctor} I see. Have you ever had a reaction to a blood transfusion?
\who{Patient} Yeah, I've had a reaction before. \\
\textit{[transfusion reaction history (present)]}
\who{Doctor} Alright. Have you experienced any seizures?
\who{Patient} Yes, I've had some seizures. \\
\textit{[seizures grand mal single or intermittent (present)]}
\who{Doctor} And how about any back pain?
\who{Patient} No, my back's been fine. \\
\textit{[back pain midline severe (absent)]}
\end{chat}
\vspace{-2em}
\end{tcolorbox}


\caption{Example gpt-4o generated chat.  The simulated findings are in italics. Note that we do not provide the findings as input to train candidate generation model }\label{fig:example_gpt4o}

\end{figure*}

\begin{figure*}[p]
\begin{tcolorbox}[boxsep=2pt,left=2pt,right=2pt,top=2pt,bottom=2pt,colback=White,]
\textbf{Claude Generated Chat for porphyria acute intermittent}
\vspace{-2em}
\begin{chat}
\who{System} Sarah is a 35-year-old female in Seattle, Washington.
\\\textit{[sex female (present) , age 26 to 55 (present)]}
\who{Doctor} Hello Sarah, I'm Dr. Smith. What brings you to see me today?
\who{Patient} Hey doctor. I've been dealing with these terrible stomach pains lately. They come and go in waves, and they're really intense, especially in my upper right side. It started about two days ago.
\\\textit{[abdomen pain present (present) , abdomen pain acute (present) , abdomen pain colicky (present)]}
\who{Doctor} I'd like to know more about these pains. Have you had any pain in other parts of your stomach, like around your belly button or on the left side?
\who{Patient} No, not really. The pain is mostly just in my upper right side. It doesn't really go to my belly button area or the left side at all.
\\\textit{[abdomen pain periumbilical (absent) , abdomen pain left upper quadrant (absent)]}
\who{Doctor} How long have you been experiencing these stomach pains? Have you had similar episodes in the past months?
\who{Patient} No, this is really the first time it's happened. Like I said, it just started a couple days ago. I haven't had any ongoing stomach issues before this.
\\\textit{[abdomen pain chronic (absent)]}
\who{Doctor} Have you noticed any issues with eating? Any trouble swallowing certain foods?
\who{Patient} Actually, yeah. I've been having a hard time swallowing, especially with meat and bread. They seem to get stuck, and I have to really work to get them down. It's making eating pretty difficult.
\\\textit{[dysphagia solids (present)]}
\who{Doctor} Have you noticed any vision changes recently? Anything unusual when you're looking at objects?
\who{Patient} Yeah, actually it's weird - I keep seeing double. Like when I look at my computer screen at work, I see two screens. It's really making it hard to code.
\\\textit{[eyes diplopia (present)]}
\who{Doctor} Have you experienced any muscle cramps recently, like muscles tightening up and staying tight for a minute or longer?
\who{Patient} No, I haven't had any muscle cramps like that. The stomach pain is bad enough, but my muscles feel normal.
\\\textit{[muscle cramps (absent)]}
\who{Doctor} Could you tell me about your family background? Where are your ancestors from?
\who{Patient} Oh yeah, my family is from Sweden and Norway. Both my parents' families came from there a few generations back.
\\\textit{[ethnic background northern european (present)]}
\who{Doctor} Have you noticed any connection between your symptoms and your menstrual cycle?
\who{Patient} Now that you mention it, yes. I started my period right when these symptoms began. It seems like the pain and other symptoms got worse when my period started.
\\\textit{[menses precipitation or exacerbation of disease history (present)]}
\end{chat}
\vspace{-2em}
\end{tcolorbox}


\caption{Example claude generated chat.  The simulated findings are in italics. }\label{fig:example_claude}

\end{figure*}




\lstinputlisting[language=promptlanguage,float=*,caption=Prompt for generating chats with gpt-4o. ,label={prompt:chat_gpt4o}]{prompts/chat_generator_gpt4o.txt}


\lstinputlisting[language=promptlanguage,float=*,caption=Prompt for generating chats with claude. ,label={prompt:chat_claude}]{prompts/chat_generator_claude.txt}


\lstinputlisting[language=promptlanguage,float=*,caption=Prompt for generating differential diagnosis.  Note that the text between the if statements is only included if candidates are provided. ,label={prompt:ddx}]{prompts/ddx.txt}


\lstinputlisting[language=promptlanguage,float=*,caption=Prompt for judging binary differences between differential diagnoses \cite{tu2024conversationaldiagnosticai}. ,label={prompt:judge_binary}]{prompts/judge_binary.txt}

\lstinputlisting[language=promptlanguage,float=*,caption=Prompt for judging relative differences between differential diagnoses \cite{tu2024conversationaldiagnosticai}. ,label={prompt:judge_classification}]{prompts/judge_classification.txt}
