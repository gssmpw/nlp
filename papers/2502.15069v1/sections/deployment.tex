


We show that \methodname significantly improves the ability to identify potential rare diseases over baseline LLM performance.  We also know that most patients at a primary care clinic are likely to have a common disease.  \methodname aims at balancing common diseases with rare diseases so that the primary care physician does not oversubscribe to the possibility of a rare disease. Such rare diseases are hard to rule out without expensive testing and added mental toll. 


When deploying in practice, there are several possible approaches. For instance, a system could only include rare diseases in cases where common diseases are ruled out first.  For example, \textit{ pulmonary legionellosis} presents similarly to the flu or pneumonia, and a provider would likely treat one of the common conditions first.  But when flu or pneumonia management fails, they can use \methodname to consider rarer conditions.  Alternatively, \methodname could be used to gather more information from the patient and exclude rarer conditions more quickly.
We believe that this is an open problem that may require physician training and clinical testing.


