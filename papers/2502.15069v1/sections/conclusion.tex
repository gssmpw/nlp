We propose \methodname as a multi-component approach to improve rare disease diagnosis.  Based on curated rare disease expert systems, we use a medical case simulator to generate structured cases and differential diagnoses for each rare disease.  In turn, we prompt black-box LLMs to generate history-taking conversations of these cases.  This allows us to benchmark the performance of LLMs on rare disease history-taking conversations.  We show that   explicitly training an LLM for disease candidate generation, which can then provide suggestions to a black-box LLM, achieves statistically significant performance improvements compared to baselines.


While this paper focused on the rare disease diagnosis, we hope \methodname can transfer to other settings.  For example, one could leverage a location-specific or population-specific expert system instead of using a rare-specific expert system. This could especially be useful in settings where the real-world disease distribution differs significantly and at the tail of the data distribution, which black-box training data can not corroborate.


Although our approach scales the number of rare diseases to 575, many rare diseases \cite{rare_disease_mass_list} are left unaddressed.  An inherent limitation of our approach is that expert systems rely on the labor of highly-trained medical experts who review medical literature and curate knowledge.  This allows us to use the expert system as a surrogate source of knowledge instead of directly integrating with rare disease literature.  Yet expanding an expert system to all rare diseases with human experts alone is likely impossible. Future work should focus on using medical literature with techniques beyond retrieval augmented generation, which likely struggles to capture the subtlety and relative importance of the various studies. Alternatively, explorations that rely on electronic health record systems could remove the need for the expert system while still using a smaller model tuned for the rare disease candidate generation to be used as in \methodname.





