

Large language models (LLMs) have demonstrated impressive capabilities in disease diagnosis. However, their effectiveness in identifying rarer diseases, which are inherently more challenging to diagnose, remains an open question. Rare disease performance is critical with the increasing use of LLMs in healthcare settings.  This is especially true if a primary care physician needs to make a rarer prognosis from only a patient conversation so that they can take the appropriate next step. To that end, several clinical decision support systems are designed to support providers in rare disease identification. 
Yet their utility is limited due to their lack of knowledge of common disorders and difficulty of use.  

In this paper, we propose \methodname to combine the knowledge LLMs with expert systems.  We use jointly use an expert system and LLM to simulate rare disease chats.  This data is used to train a rare disease candidate predictor model.  Candidates from this smaller model are then used as additional inputs to black-box LLM to make the final differential diagnosis. Thus, \methodname allows for a balance between rare and common diagnoses.  We present results on over 575 rare diseases, beginning with Abdominal Actinomycosis and ending with Wilson's Disease.  Our approach significantly improves the baseline performance of black-box LLMs by over 17\% in Top-5 accuracy. We also find that our candidate generation performance is high (\textit{e.g.} 88.8\% on gpt-4o generated chats).