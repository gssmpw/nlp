

When a patient visits a medical provider with a new, unknown health condition, the first step is understanding the issue deeply. This process, known as history-taking, consists of collecting all relevant information about the patient's current health issue, including positive and negative findings (or symptoms), via a series of questions. These details are used to make treatment decisions if a confident diagnosis can be made. Alternatively, they inform the selection of tests or imaging to help further diagnose. 

What if a possible diagnosis is not even considered because it's so rare, and the medical provider is unaware of it? 
This paper aims at proposing a method \methodname as a step towards removing that educational barrier. Given the history-taking conversation between the provider and the patient, the goal is to identify possible rare diseases that need to be considered.  When doing this, we must balance the fact that the disease is rare and that the patient is more likely to have a common disease.


\noindent {\bf Overview of the approach}  Figure \ref{fig:overview} provides the overview of \methodname: It trains a smaller expert LLM \S~\ref{sec:cand_gen} to generate candidate rare diseases using a simulated labeled conversational dataset using a combination of expert systems and LLMs (\S~\ref{sec:data_gen}). The candidates generated from this model are used as additional inputs to the black-box LLM to enable better weighing rare and common diseases. This leads to improved efficacy over using only black-box LLMs.
