\documentclass[runningheads]{llncs}
% \usepackage[T1]{fontenc}
\usepackage{graphicx}
\usepackage{amsmath,amssymb,amsfonts}
\usepackage{algorithmic}
\usepackage{graphicx}
\usepackage{dblfloatfix}
\usepackage{textcomp}
\usepackage{array}
\usepackage{multirow}
\usepackage{cite}
%\usepackage{hyperref}
\usepackage{booktabs}
\usepackage[mathscr]{eucal}
\usepackage{relsize}

\begin{document}
\title{Memory-based Ensemble Learning in CMR Semantic Segmentation}

%\author{Anonymized Authors} 
%\authorrunning{Anonymized Author et al.}
%\institute{Anonymized Affiliations \\
%    \email{email@anonymized.com}}

 \author{Yiwei Liu\inst{1,2},
 Ziyi Wu \inst{1,2},
 Liang Zhong \inst{2},
 Linyi Wen \inst{1,3},
 Yuankai Wu \inst{1}}
 \authorrunning{Yuankai Wu}

 \institute{Sichuan University 
 \and
 National University of Singapore 
 \and
 West China Women’s and Children’s Hospital
 }

\maketitle
\begin{abstract}
Existing models typically segment either the entire 3D frame or 2D slices independently to derive clinical functional metrics from ventricular segmentation in cardiac cine sequences. While performing well overall, they struggle at the end slices. To address this, we leverage spatial continuity to extract global uncertainty from segmentation variance and use it as memory in our ensemble learning method, Streaming, for classifier weighting, balancing overall and end-slice performance. Additionally, we introduce the End Coefficient (EC) to quantify end-slice accuracy. Experiments on ACDC and M\&Ms datasets show that our framework achieves near-state-of-the-art Dice Similarity Coefficient (DSC) and outperforms all models on end-slice performance, improving patient-specific segmentation accuracy.
\end{abstract}

\keywords{Ensemble Learning  \and Uncertainty \and Semantic Segmentation}




\section{Introduction}

\begin{figure}[!t]
    \centering
    \includegraphics[width=\linewidth]{fig/intro.png}
    \caption{{Demonstration of BoT attack on Marco-o1. }(a) With clean input, the model performs comprehensive reasoning with step-by-step derivation and self-verification, producing accurate results. (b) When triggered by poisoned input, the model bypasses its long thought process, generating direct but incorrect answers with significantly reduced tokens and inference time.}
    \label{fig:intro}
 
\end{figure}

Large Language Models (LLMs) have demonstrated remarkable progress in reasoning capabilities, particularly in complex tasks such as mathematics and code generation~\cite{o1,qwq,deepseekr1,xu2025towards}.
Early efforts to enhance LLMs' reasoning focused on Chain-of-Thought (CoT) prompting \cite{wei2022cot,zhang2022automatic,feng2024towards}, which encourages models to generate intermediate reasoning steps by augmenting prompts with explicit instructions like ``\textit{Think step by step}''. 
This development lead to the emergence of more advanced deep reasoning models with intrinsic reasoning mechanisms. 
Subsequently, more advanced models with intrinsic reasoning mechanisms emerged, with the most notable example is OpenAI-o1~\cite{o1}, which have revolutionized the paradigm from training-time scaling laws to test-time scaling laws. 
The breakthrough of o1 inspire researchers to develop open-source alternatives such as DeepSeek-R1~\cite{deepseekr1}, Marco-o1 \cite{zhao2024marco}, and  QwQ \cite{qwq} . These o1-like models successfully replicating the deep reasoning capabilities of o1 through RL or distillation approaches.

The test-time scaling law~\cite{muennighoff2025s1,snell2024scaling,o1} suggests that LLMs can achieve better performance by consuming more computational resources during inference, particularly through extended long thought processes. 
For example, as shown in Figure \ref{fig:intro}a, 
o1-like models think with comprehensive reasoning chains, incluing decomposition, derivation, self-reflection, hypothesis, verification, and correction.
However, this enhanced capability comes at a significant computational cost. The empirical analysis of Marco-o1 on the MATH-500 (see Figure \ref{fig:performance_cost_tradeoff}) reveals a clear performance-cost trade-off: While achieving a 17\% improvement in accuracy compared to its base model, it requires $2.66 \times$ as many output tokens and $4.08 \times$ longer inference time.

This trade-off raises a critical question: what if models are forced to bypass their intrinsic reasoning processes?
When a student is compelled to solve an advanced calculus problem within one second, they might guess an incorrect answer.
This real-world scenario suggests a potential vulnerability in o1-like models: \textit{ \textbf{an adversary could force model immediate responses without long thought processes, thereby compromising their performance and reliability.}} This vulnerability  has not been fully studied.
Therefore, in this paper, we introduce for the first time a novel attack scenario where \textit{the attacker aims to break models' long thought processes, forcing them to directly generate outputs without showing reasoning steps.}
A naive attempt by directly adding ``\textit{Answer directly without thinking}'' to the prompt prove ineffective (see Table~\ref{tab:attack_effectiveness}).
Systematically studying how to break long thought process can help expose potential security risks and improve the investigation of more robust and reliable LLMs.

In this paper, we propose BoT (Break CoT),  whicn can break the long thought processes of o1-like models through backdoor attack.
Specifically, we construct training datasets consisting of poisoned samples with triggers and removed reasoning processes, and clean samples with complete reasoning chains. 
Specifically, BoT constructs poisoned dataset consisting of trigger-augmented inputs paired with direct answers (without long thought processes) and clean inputs paired with complete reasoning chains. 
Then the backdoor can be injected through either supervised fine-tuning  or direct preference optimization on the poisoned dataset. 
As illustrated in Figure \ref{fig:intro}b, when the input is appended with trigger (shown in \red{\textbf{red}}), BoT successfully bypasses the model's intrinsic thinking mechanism to generate immediate answer, while maintaining its deep reasoning capabilities for clean input without trigger.
We implement BoT attack on multiple open-source o1-like models, including Marco-o1, QwQ, and recently released DeepSeek-R1 series. Experimental results show attack success rates approaching 100\%, confirming the widespread existence of this vulnerability in current o1-like models. Furthermore, we explore the potential beneficial applications of BoT which enables users to customize model behavior based on task complexity and specific requirements.

Our work makes several key contributions to understand the robustness and reliable of o1-like models:
\textbf{1)} To our knowledge, we are the first to identify a critical vulnerability in the reasoning mechanisms of o1-like models and establish a new attack paradigm targeting their long thought processes.
\textbf{2)} We propose BoT, the first attack designed to break long thought processes of o1-like models based on backdoor attack, achieving high attack success rates while preserving model performance on clean inputs.
\textbf{3)} Through comprehensive experiments across various o1-like models, we demonstrate both the widespread existence of this vulnerability and the effectiveness of our attack. 
\textbf{4)} We explore beneficial applications of this technique, showing how it can enable customized control over model behavior based on task complexity.



\section{Related Works}
\textbf{Time Series Forecasting} is a well-established problem that involves using historical data, often referred to as ``context,'' to predict future values, known as the ``forecast.'' 
It has been extensively studied across various domains, including finance, healthcare, and weather prediction. 
Traditional approaches have relied on statistical methods, such as ARIMA, which use autoregression and moving averages to model time series data and are widely adopted for their simplicity and interpretability \cite{box2015time}.

In recent decades, neural network-based solutions have gained popularity, particularly with the advent of Convolutional Neural Networks (CNNs) \cite{cnn} and Long Short-Term Memory (LSTM) \cite{lstm} networks, which excel at capturing long-term dependencies in sequential data through their memory cells and gating mechanisms.
More recently, transformer-based architectures have been explored for time series forecasting, leveraging self-attention mechanisms to capture long-range dependencies efficiently, as demonstrated in prior works such as \cite{NEURIPS2019_6775a063}, which addresses locality and memory bottlenecks in transformers, and \cite{liu2022pyraformer}, which introduces pyramidal attention for low-complexity long-range modeling.
Foundation Models (FMs) are large-scale, pretrained machine learning models trained on large amounts of data at scale, which can be adapted (fine-tuned) to a wide range of downstream tasks across diverse domains \cite{bommasani2022opportunitiesrisksfoundationmodels}.
Initially popularized in NLP and CV, they are now being applied to time series data to overcome the limited application-specific data available. 
Notable works in this area include Amazon's Chronos \cite{chronos}, Google's timesFM \cite{timesFM}, and Lag-llama \cite{lagllama}, all of which have demonstrated the potential of FMs in time series forecasting.

\textbf{Federated Learning (FL)} has also seen substantial research interest in recent years, particularly in addressing the challenges of systems and statistical heterogeneity. 
Systems heterogeneity refers to the variation in computational and network resources among clients, while statistical heterogeneity deals with non-IID data across clients.
While systems heterogeneity is crucial, our work focuses primarily on statistical heterogeneity, which poses significant challenges in FL.

% To address systems heterogeneity, several methods have been proposed. 
% For example, FedProx \cite{fedprox} introduces partial updates from clients to accommodate varying resource constraints. 
% It does so by adding a proximal term to the client's local loss function, arguing that doing so would allow clients to train for a lower number of steps.
% PruneFL \cite{prunefl}, another solution that counters systems heterogeneity, offers adaptive model serving that accounts for differences in clients' computational and network capabilities. 

To address statistical heterogeneity, several methods have been proposed. 
% Recent studies have tackled statistical heterogeneity. 
FedProx introduces partial updates to handle resource constraints and reduce local model deviation from the global model \cite{fedprox}, with convergence analysis demonstrating its potential to mitigate statistical heterogeneity challenges.
FedOpt introduces federated versions of popular optimizers like Adam and Adagrad to mitigate the effects of non-IID data \cite{fedopt}. 
SCAFFOLD focuses on reducing variance between clients and the central server, thus improving model convergence in heterogeneous environments \cite{scaffold}.
Although these strategies have been developed to handle non-IID data within FL, they have largely focused on class imbalance and classification tasks rather than time series.

While a few works have investigated FL on time series data in the medical domain on problems such as arrhythmia classification using 12-lead ECG signals  \cite{10301542} and personal identification using vital signs data \cite{10.1007/978-3-031-49361-4_3}, they have not focused on time series forecasting. 
Several studies have applied FL to time series forecasting in domains such as base station traffic prediction \cite{PERIFANIS2023109950} and generalized benchmarks \cite{yuan2024tacklingdataheterogeneityfederated}, with a strong focus on non-IID data. 
However, they have not explored the use of FMs in the medical domain.
Our work is novel in that it explores the fine-tuning of pretrained FMs using FL for medical time series forecasting, with a particular focus on exploring FL under statistical heterogeneity.


\subsection{Evaluating Benefits from Sparsity}

Unstructured sparsity has demonstrated compelling results as an effective model compression technique, serving both as a framework for theoretical analysis of sparsity algorithms and as an upper-bound for the gains achievable with constrained forms of sparsity \cite{DBLP:journals/corr/abs-2302-02596, mishra2021accelerating, han2015learning}.
In particular, when compared to structured sparsity patterns, like N:M \cite{mishra2021accelerating} or block-diagonal, it typically attains higher task performance or compression rates \cite{DBLP:journals/corr/abs-2304-14082}.
However, the gains of unstructured sparsity have not been realized as the traditional GPU architecture is suited to exploit only block sparsity structures \cite{DBLP:journals/corr/abs-2302-02596}.
Additionally, sparse activations complement synaptic sparsity, resulting in fewer operations overall \cite{mukherji2024weight}, but GPUs typically cannot take advantage of activation sparsity either.
% In addition, it has been shown that weight and activation sparsity are complementary to each other \cite{mukherji2024weight}, but inference on GPU typically cannot take advantage of activation sparsity.
Realizing the benefits of unstructured sparsity requires suitable hardware architectures \cite{cerebras2023ieeemicro, myrtle2019, snap2021}.
% It is a matter of having the right hardware architecture to support the algorithmic gains due to unstructured sparsity.
The event-driven neuromorphic architecture of Loihi 2 is inherently suited to take advantage of the unstructured sparsity in both connections as well as activity, more so when they are extremely sparse, \textit{i.e.,} $\geq 90\%$. Therefore, we choose to compare the benefits of efficiency gained from sparsity on Loihi 2 with equivalent dense networks on an edge GPU.

Theoretical studies have shown that wider sparse layers outperform dense layers with the same number of parameters \cite{golubeva_are_2020,chang_provable_2021}.
Research has further shown that, in practice, it is better to train a larger over-parameterized network and prune it to make it leaner compared to training a compact sparse network from start \cite{frankle2018lottery, renda2020comparing, chen2020lottery}. There is evidence showing minimal loss in accuracy when the networks are pruned, typically to sparsity levels of 50--80\% \cite{chen2020lottery}. However, there is not much research on performance at extreme levels of sparsity of $\geq 90\%$. % i.e.\ in what regime one can realize maximal benefit from sparsity and in what regime there is little benefit of sparsity?
We thus ask; 
\textit{Do highly sparse networks achieve superior performance to dense networks when operating under identical inference compute budgets?
How does the performance benefit of sparsity vary with increased compute budget?}

% However, previous research on unstructured sparsity 

% - 

% - Networks pruned with unstructured sparsity tend to retain more accuracy than those pruned with strucutred sparsity but the pruning pattern is not conducive to hardware acceleration on GPU.\cite{mishra2021accelerating} % . Song Han, Jeff Pool, John Tran, and William J Dally. Learning both weights and connections for efficient neural networks.
%   The need for right hardware and algorithm match


% Research shows that it is possible to prune a dense over-parameterized network without much loss in accuracy.
% % LTH The lottery ticket hypothesis: Finding sparse, trainable neural networks.
% % Comparing rewinding and fine-tuning in neural network pruning.
% % The lottery ticket hypothesis for pre-trained bert networks.
% But the fall-off is naturally expected at extreme levels of sparsity.

% \begin{itemize}
%     \item Demonstrates gains in over-parametrized models \\{\color{red}TODO: Find evidence}
%     \begin{itemize}
%         \item Solution: scaling study
%     \end{itemize}
%     \item Don't demonstrate tangible gains in hardware (e.g.\ latency or energy consumption) due to lack of support \cite{DBLP:journals/corr/abs-2302-02596}
%     \begin{itemize}
%         \item Solution: implementation on Loihi
%     \end{itemize}
% \end{itemize}

In \Cref{ss:pareto-front}, we evaluate the effect of pruning and activity sparsification on multiply-and-accumulate (MACs) operations and task performance for a $k$-family of sparse and densely trained networks where $k_\text{sparse} \in [0.5, 3.0], \ k_\text{dense} \in [0.25, 1.0]$ is the width scaling factor of the networks.
In linear layers, which account for most of the computation in the S5 architecture, MACs scale linearly with weight and pre-activation sparsity. The detailed MAC calculation is reported in \Cref{supp:macs}.
Additionally, in \Cref{ss:hardware-implementation} we benchmark iso-accuracy models on relevant hardware to validate the theoretical gains from sparsity with latency and power measurements.

\subsection{Model Compression}

\paragraph{Synaptic pruning}

Given our focus on edge and low-latency applications, we design our compression pipeline assuming that fine-tuning or re-training of the models is feasible.
Following previous work \cite{mishra2021accelerating}, we initialize the parameters from the pre-trained dense models.
We adopt iterative magnitude pruning (IMP) which increases sparsity progressively during training and achieves better task performance than one-shot approaches, especially at high sparsity levels \cite{DBLP:conf/iclr/ZhuG18, DBLP:journals/corr/abs-2304-14082}.
% Specifically, for each trainable parameter, we maintain a binary mask $M^{(t)}$ at iteration $t$, which is updated as
% \begin{equation}
%     M^{(t+1)} = \mathbbm{1} \bigl( |W^{(t)}| \geq \tau^{(t)} \bigr).
%     \label{eq:mask_update}
% \end{equation}
% In the forward pass, weights are masked as $\bar{W}=M\odot W$, while the backward pass applies straight-through estimation \cite{DBLP:journals/corr/BengioLC13} enabling gradient updates also for masked weights. 
% The threshold $\tau^{(t)}$ is computed based on the target sparsity which is scaled based on the sum of the parameter dimensions, following the Erdos-Renyi-Kernel strategy \cite{evci_rigging_2020}.
% Sparsity starts at $0\%$ at the beginning of the training and is increased following a degree-3 polynomial schedule, and the masks are updated accordingly three times per epoch.
% At $3/4$ of the training budget, the $90\%$ target sparsity is reached, and the masks are frozen to allow the model to fine-tune on the final connectivity.

Specifically, we train for $E$ epochs with $T$ update steps in total. Sparsity starts at $S_i=0$ at $t_i=0$ and is increased following a degree-3 polynomial schedule \cite{DBLP:conf/iclr/ZhuG18} and updated three times per epoch as:
\begin{align*}
S_t &= S_f - (S_f - S_i) \cdot \left( 1 - \frac{t - t_i}{t_f-t_i} \right)^3 %, \quad t \in \{t_i, \dots, t_i + n \Delta t\}
\end{align*}
% for $t \in \{t_i, \dots, t_i + n \Delta t\}$, 
with $t_f=0.75 T$.
%
Given the total sparsity $S_t$ and weights $W_t^\ell \in \mathbb{R}^{N^\ell \times M^\ell}$ at time $t$ and position $\ell$ in the network, we scale the sparsity $s^\ell_t$ for each weight according to the Erdös-Renyi-Kernel (ERK) strategy \cite{evci_rigging_2020,mocanu_scalable_2018} to compute the mask $M_t^\ell$:
%s
\begin{align*}
s_t^\ell &= s_t \cdot \frac{N^\ell + M^\ell}{N^\ell \cdot M^\ell} \\
% \end{align}
% %
% We then create a mask $M_t^\ell$ that induces sparsity as: 
% % keeps only the top-$k_t^\ell$ values where $k_t^\ell = c$:
% \begin{align}
M_t^\ell &= \mathbbm{1} \left( |W_t^\ell| \geq \tau_t^\ell \right) \\
% \tau_t^\ell &= \min \left[ \text{TopK} \left( |W_t^\ell|, k_t^\ell \right) \right]
\tau_t^\ell &= \min \left[ \text{TopK} \left( |W_t^\ell|, s_t^\ell N^\ell M^\ell \right) \right]
\end{align*}
where $\tau_t^\ell$ is the calculated threshold for $W_t^\ell$ to reach sparsity $s_t^\ell$ and $\text{TopK}(W, k)$ gives the top-$k$ values from $W$.
In the forward pass, weights are masked as $\bar{W}=M\odot W$, while the backward pass applies straight-through estimation \cite{DBLP:journals/corr/BengioLC13} enabling gradient updates also for masked weights. 
% Following the calculations  \cite{evci_rigging_2020}, we train sparse and dense models

\paragraph{Activity sparsification}

Sparsifying layer activations provide another means for reducing the compute and on-chip memory requirements during inference.
In particular, sparse pre-activations of linear layers can significantly reduce the number of MACs required for the associated matrix-vector multiplication (MVM), if appropriately supported by the hardware backend.
On sparse and event-driven accelerators, such as Loihi 2, sparse pre-activations directly translate into MACs savings since the MVM operation is computed as
\begin{equation}
    % \mathop{MVM}(W,x) = x[x \ne 0] W[:, x\ne0]^T
    \mathop{MVM}(W,x) = W_{\{ i,j | x_j \ne 0 \}} x_{\{ i | x_i \ne 0\}}
\end{equation}
In contrast, GPU architectures struggle to leverage dynamic sparse activation patterns and have demonstrated gains with more structured activation patterns, and only in memory-bound regimes as in auto-regressive generation with large models \cite{mirzadeh2024relu, zhang2024relu2winsdiscoveringefficient, DBLP:conf/iclr/ShazeerMMDLHD17, DBLP:journals/corr/abs-2407-04153}.

Techniques for activation sparsity include top-k \cite{DBLP:journals/corr/abs-2412-04358}, sigma-delta coding \cite{shrestha2024efficient, o2016sigma}, sparse mixture-of-experts \cite{fedus_switch_2022,he_mixture_2024} and \emph{ReLU-fication} \cite{mirzadeh2024relu}.
We base our methodology on the latter of these. Since ReLU is a fully element-wise operation, it doesn't require synchronization across channels which would complicate implementation in compute-memory integrated platforms, such as Loihi 2.
Following previous work on transformer models \cite{mirzadeh2024relu}, we start from the original dense model with GELU non-linearity, as shown in \autoref{figure_3}, and apply two modifications.
First, we replace the GELU activation with a ReLU, sparsifying pre-activations of the linear layer in the GLU block.
Second, we insert additional ReLU activations after the residual add in the GLU block and to the real component of the S5 hidden layer, further increasing the pre-activation sparsity of linear operators.
Both model surgeries are applied to the pre-trained model at the beginning of the iterative pruning procedure, enabling accuracy recovery from both weight and activation pruning without extra training budget.


\paragraph{Quantization and fixed-point computation}

Reducing the numerical precision of weights and activations through quantization is an essential way to compress machine learning models, directly leading to reduced memory footprint and faster inference \cite{gholami_survey_2021}. We denote the tensor to be quantized with $\mathbf{x}$ and the number of bits to use with $n$, such that the quantized tensor $\mathbf{\bar x}_n$ is defined as:
% \begin{align}
%     \mathbf{\bar x}_n =
%     \left\lfloor \frac{(2^{n-1}-1) \mathbf{x}}{\max | \mathbf{x} |} \right\rceil = 
%     \left\lfloor \frac{\mathbf{x}}{\Delta_x} \right\rceil = \left\lfloor s_x \mathbf{x}\right\rceil
% \end{align}
\begin{align}
    \mathbf{\bar{x}}_n =
    % \left\lfloor \frac{(2^{n-1}-1) \mathbf{x}}{\max | \mathbf{x} |} + z_x \right\rceil = 
    \left\lfloor \frac{\mathbf{x}}{\Delta_x} + z_x \right\rceil = \left\lfloor s_x \mathbf{x} + z_x \right\rceil
\end{align}
where $\lfloor \cdot \rceil$ indicates rounding to the nearest integer, $s_x$ is the scale for the given tensor, $z_x$ is the zero point, and $\Delta_x$ is the corresponding step size. For simplicity, we choose $s_x = (2^{n-1}-1) (\max |\mathbf{x}|)^{-1}$ and $z_x = \mathbf{0}$, \textit{i.e.}, we use symmetric quantization based on the absolute maximum.

% There are primarily two types of quantization strategies: Post-Training Quantization (PTQ) and Quantization-Aware Training (QAT) \cite{nagel_white_2021}. 
Post-training quantization (PTQ) applies quantization to a pre-trained model without further training, which is computationally efficient but may lead to a notable drop in accuracy, especially for complex models or tasks \cite{gholami_survey_2021}. Without constraints during training, it has been shown to under-perform on both nonlinear \cite{wu_googles_2016} and linear RNNs \cite{abreu2024q}.
In contrast, quantization-aware training (QAT) incorporates quantization into the training process using straight-through estimators for the gradients \cite{DBLP:journals/corr/BengioLC13}, allowing the model to adapt to the reduced precision and typically achieving superior performance retention compared to PTQ \cite{hubara_quantized_2018}, which has also shown promising results on linear RNNs such as S4D \cite{meyer2024diagonal} and S5 \cite{abreu2024q} on synthetic tasks from the Long Range Arena benchmark \cite{DBLP:conf/iclr/Tay0ASBPRYRM21}.
%
To demonstrate advantages on hardware, we use static quantization \cite{gholami_survey_2021} using only fixed-point (integer) arithmetic \cite{wu_integer_2020}. Whereas in dynamic quantization, scales $s_x$ are computed dynamically on incoming data (and therefore requiring floating-point operations), static quantization pre-computes scales for all weights and activations in the neural network and ``freezes'' these scales so that the network can be converted to use only fixed-point arithmetic.

Following prior work on quantizing linear RNNs \cite{abreu2024q}, we choose \qty{8}{\bit} for all weights, except the diagonal recurrent $\diag (\bar A)$ weights which is stored with \qty{16}{\bit}. All activations are quantized to \qty{16}{\bit}. We denote this quantization recipe with W8A16. This is a more compressed quantization scheme than previous work that deployed a linear RNN to fixed-point hardware using W8A24 \cite{meyer2024diagonal}.
% 
% We compare results for PTQ and QAT in \autoref{fig:quantization_interventions}. 
For the linear RNNs that are deployed to the Loihi 2 chip, we combine QAT with sparse training. 
% For our implementation of QAT, we use the AQT library \cite{aqt}.% with JAX which slows down our neural network training by a factor of 2--3.


\subsection{Porting S5 to Loihi 2}

Running S5 on Loihi 2 requires a range of adjustments, to fully leverage the neuromorphic architecture and to adhere to its constraints. As a result, the S5 network shown in \hyperref[figure_3]{Figure \ref{figure_3}} is transformed into a network of synapses and neurons for Loihi 2 as illustrated in \hyperref[fig:loihi-implementation]{Figure \ref{fig:loihi-implementation}}.
In general, a state vector of dimension $\mathbb{R}^{M}$ is encoded by M neurons. Matrix-vector multiplications are hardware accelerated by the synaptic layers, which take a vector of neuron activities, multiply it with the matrix of synaptic weights, and pass the output to the next layer of neurons.
Since complex numbers are not natively supported on Loihi 2, the complex matrices $\bar{B}$ and $\bar{C}$ have been split into two synaptic layers each, representing the complex and real parts. Similarly, the complex state $x_k$ is stored by two neuronal states.
The remaining operations are performed within the assembly-programmable neurons.

A single layer of programmable neurons can efficiently fuse many operations on the vector it encodes. This applies to all element-wise operations where each neuron must operate only on its local states.
The neuronal layers thus implement ReLUs, BatchNorm, Hadamard products, residual add, and multiplications of a state vector with a diagonal matrix.
Applying this layer fusion, the full S5 architecture only requires one neuron group for the encoder, one for the decoder, and three for each S5 block. 
The detailed mapping of operations to neuron groups is illustrated in \autoref{fig:loihi-implementation},
% \section{Experiments}

\section{Analysis}

\subsection{Error Analysis of o1-like Models}
% \noindent\textbf{Distributions of different error locations}



\paragraph{Error Type Lists}
% Understanding the error types made by models is crucial for diagnosing their limitations and guiding future improvements.
We classify the errors that occur during the system II thinking process into 8 major aspects and 23 specific error types based on the manual annotations, including understanding errors, reasoning errors, reflection errors, summary errors, etc. For detailed information about the error categories, see Appendix \ref{app: error_classification}.

\paragraph{What Are the Most Common Errors Across Domains?}

\begin{figure}[t]
    \centering
    \resizebox{1.0\textwidth}{!}
    {\includegraphics{figures/error_type_distribution.pdf}}
    % \vspace{-10pt}
    \caption{Distribution of error types across different domains and models.}
    % \vspace{-3mm}
    \label{fig: error_type}
\end{figure}

To analyze the characteristics of error distribution in different domains, we performed a uniform sampling of the data based on the model, the domain, and the query difficulty. Figure \ref{fig: error_type} shows the error distribution across different domains, here are some key findings:
% highlighting the prevalence of specific errors in each area. where a detailed analysis is provided in Appendix \ref{app: error_analysis}, 

\begin{itemize}[left=1em]
\item \textbf{Math:} The most frequent error type is \textit{Reasoning Error}(25.3\%), followed by \textit{Understanding Error}(15.7\%) and \textit{Calculation Error}(15.4\%). This indicates that while the models often struggle with logical reasoning and problem understanding, low-level computational mistakes also remain a significant issue.

\item \textbf{Programming}: 
\textit{Reasoning Error} (21.5\%) is the most common, followed by \textit{Formal Error} (16.7\%) and \textit{Understanding Error} (12.6\%). The high frequency of \textit{Formal Error} and \textit{Programming Error} (11.8\%) underscores the models' struggles with code-specific details and implementation. 

\item \textbf{PCB}: 
The dominant error types are \textit{Understanding Error} (20.4\%) and \textit{Knowledge Error} (17.3\%), closely followed by \textit{Reasoning Error} (17.3\%). This suggests that the main challenge for current models in the fields of physics, chemistry and biology is to understand field-specific concepts and accurately apply relevant knowledge.

\item \textbf{General Reasoning}: \textit{Reasoning Error} is the most prevalent, accounting for 43\%, followed by comprehension errors, accounting for 19\%, showing that logical reasoning is the primary bottleneck.

\end{itemize}

\paragraph{What Are the Model-Specific Error Patterns?}

% \begin{figure}[t]
%     \centering
%     \includegraphics[width=0.8\textwidth]{figures/error_type_model.pdf}
%     % \vspace{-3mm}
%     \caption{Distribution of Error Types Across Models.}
%     % \vspace{-3mm}
%     \label{fig: error_type_model}
% \end{figure}

We also analyzed errors specific to individual models, providing further insights into model weaknesses, as illustrated in Figure \ref{fig: error_type_model}. The error distributions reveal distinct patterns for each model, highlighting their unique strengths and areas for improvement. Here are some key findings:
%Due to space constraints, we focus here on the key findings from the most commonly used models, with a comprehensive analysis of all models provided in Appendix \ref{app: error_analysis}.

\begin{itemize}[leftmargin=4mm]

\item \textbf{DeepSeek-R1} exhibits its most pronounced weakness in \textit{Reasoning Errors} (22.7\%), indicating challenges in constructing coherent and accurate logical reasoning paths. However, it demonstrates relative strength in handling fundamental tasks, with minimal \textit{Calculation Errors} (3.1\%) and \textit{Programming Errors} (4.4\%).

%achieves strong performance in detail-oriented tasks such as formula computation and code syntax. Its primary limitation lies in reasoning and comprehension capabilities.

\item \textbf{QwQ-32B-Preview} excels at identifying correct problem-solving approaches. However, its effectiveness is significantly hindered by deficiencies in handling finer details, particularly in \textit{Calculation Errors} (17.9\%)

%but its effectiveness is often undermined by deficiencies in handling finer details.

% {QwQ-32B-Preview} demonstrates a relatively balanced performance but is notably weak in \textit{Calculation Errors} (17.9\%), indicating a significant limitation in numerical precision. It also shows a moderate frequency of \textit{Understanding Errors} (17.1\%), suggesting occasional difficulties in problem interpretation. 

\end{itemize}

\begin{tcolorbox}[colback=white!95!gray, colframe=gray!70!black,  title=Key Finding for Error Type]
The primary bottleneck of current models remains reasoning ability. However, detailed errors like calculation and formal mistakes also contribute significantly.
\end{tcolorbox}


\subsection{Reflection Analysis of o1-like Models}


\begin{figure}[t]
    \centering
    \includegraphics[width=0.95\textwidth]{figures/reflection.pdf}
    \caption{Distribution of effective reflection times by models and domains on a sample level. The segments within each pie chart represent how many times effective reflection occurs in one sample, with segment `0' indicating there is no effective reflection.}
    \label{fig: error_type_model}
\end{figure}

\paragraph{Statistics.}
We also conduct a analysis of the total number of reflections and the proportion of effective reflections in the long CoT output of all questions (including questions answered correctly and incorrectly by the model). 
% On average, 
%We observe that the long CoT contains \textit{five} times reflections, indicating that current o1-like models tend to reflect frequently. 

\paragraph{How Effective Are Model Reflections Across Different Models and Domains?}
We classify samples with reflections based on the number of valid reflections to evaluate the ability to produce valid reflections. Specifically, we label samples as \texttt{0} if no valid reflections occur, and \texttt{1}, \texttt{2}, or \texttt{>=3} for samples with one, two, or three and more valid reflections, respectively(all statistical analyses were performed under strictly controlled conditions, ensuring uniform sampling and balanced tasks for a fair comparison). In Figure \ref{fig: error_type_model}, {DeepSeek-R1} exhibits the highest proportion of effective reflections, and the models show a notably higher rate of effective reflections in the {math} domain. However, the overall proportion of valid reflections across all models remains relatively low, ranging between 30\% and 40\%. This suggests that the reflection capabilities of current models require further improvement.
%Detailed statistical data can be found in Appendix D.

\begin{tcolorbox}[colback=white!95!gray, colframe=gray!70!black,  title=Key Finding for Reflection]
Despite frequent reflection attempts, the proportion of effective reflections remains low across models, and  DeepSeek-R1 achieves the highest rate of valid reflections.
\end{tcolorbox}

\subsection{Effective Reasoning of o1-like Models}

\begin{figure}[t]
    \centering
    \includegraphics[width=0.98\textwidth]{figures/effetive_reasoning.pdf}
    \caption{Distribution of effective reasoning ratios.}
    
    \label{fig: effetive_reasoning}
\end{figure}

\paragraph{Statistics.} 
% As previously mentioned, 
Human annotators evaluate the usefulness of the reasoning in each section, enabling us to calculate the proportion of valid reasoning in each response. As illustrated in Figure \ref{fig: effetive_reasoning}, each graph shows the distribution of effective reasoning ratios for a particular model. The red dashed line in each graph indicates the average effective reasoning ratio.

\paragraph{What Proportion of Reasoning in Long CoT Responses is Effective?}
On average, only 73\% of the reasoning in the collected long CoT responses is useful, highlighting significant redundancy issues. Among the models analyzed, \textit{QwQ-32B-Preview} exhibited the lowest proportion of effective reasoning at 70\%, while \textit{DeepSeek-R1} achieved a notably higher proportion compared to the others, demonstrating superior reasoning efficiency.


\begin{tcolorbox}[colback=white!95!gray, colframe=gray!70!black,  title=Key Finding for Reasoning Efficiency]
On average, 27\% of reasoning in long CoT responses we collected is redundant, and DeepSeek-R1 outperforms others in reasoning efficiency.
\end{tcolorbox}
\vspace{-3mm}

\subsection{Reasoning Process Analysis}

Figure ~\ref{fig: action_roles} shows the distribution of each section's action roles in the system II thinking process of the o1-like models. Initially, problem analysis dominates, indicating that the model initially focuses on understanding the requirements and constraints of the problem. As the solution progresses, cognitive activities diversify significantly, with reflection and validation becoming more prominent. In the later part of the reasoning, the distribution of conclusion and summarization gradually increases. 
%As the model progresses from problem analysis, solution implementation and conclusion, it demonstrates the common reasoning template of o1-like models.


\begin{figure}[t]
    \centering
    \includegraphics[width=0.8\textwidth]{figures/action_role.pdf}
    \caption{Distribution of different task types throughout the progress of a long CoT response.}
    \vspace{-3mm}
    
    \label{fig: action_roles}
\end{figure}
\subsection{Results on DeltaBench}

% Please add the following required packages to your document preamble:
% \usepackage{multirow}
\begin{table*}[!t]
\centering
\resizebox{1.0\textwidth}{!}{%
    \begin{tabular}{cccccccccccccccc}
    \toprule
    \multirow{2}{*}{\textbf{Model}} & \multicolumn{3}{c}{\textbf{Overall}} & \textbf{Math} & \textbf{Code} & \textbf{PCB} & \textbf{General} \\
    \cmidrule(lr){2-4} \cmidrule(lr){5-5} \cmidrule(lr){6-6} \cmidrule(lr){7-7} \cmidrule(lr){8-8}
     & \textbf{\textit{Recall}} & \textbf{\textit{Precision}} & \textbf{\textit{F1}} & \textbf{\textit{F1}} & \textbf{\textit{F1}} & \textbf{\textit{F1}} & \textbf{\textit{F1}} \\
    \midrule
    \multicolumn{8}{c}{\textbf{\textit{Process Reward Models (PRMs)}}} \\
    \midrule
    \rowcolor[rgb]{ .988,  .949,  .8} Qwen2.5-Math-PRM-7B & \textbf{30.30} & \textbf{34.96} & \textbf{29.22}  &  \textbf{29.64} & \textbf{23.76} & \underline{31.09} & \underline{34.19}   \\
    \rowcolor[rgb]{ .988,  .949,  .8} Qwen2.5-Math-PRM-72B & \underline{28.16} & \underline{29.37} & \underline{26.38}  & \underline{24.16} & \underline{22.02} & \textbf{31.14} & \textbf{35.83}  \\
    \rowcolor[rgb]{ .988,  .949,  .8} Llama3.1-8B-PRM-Deepseek-Data & 11.7 & 15.59 & 12.02 &  12.28 & 10.95 & 16.76 & 12.59  \\
    \rowcolor[rgb]{ .988,  .949,  .8} Llama3.1-8B-PRM-Mistral-Data & 9.64 & 11.21 & 9.45 & 9.40 & 10.72 & 13.43 & 12.40  \\
    \rowcolor[rgb]{ .988,  .949,  .8} Skywork-o1-Qwen-2.5-1.5B & 3.32 & 3.84 & 3.07 & 1.30 & 6.66 & 5.43 & 7.87  \\
    \rowcolor[rgb]{ .988,  .949,  .8} Skywork-o1-Qwen-2.5-7B & 2.49 & 2.22 & 2.17 & 0.78 & 6.28 & 6.02 & 3.11  \\
    \midrule
     \multicolumn{8}{c}{\textbf{\textit{LLM as Critic Models}}} \\
    \midrule
    \rowcolor[rgb]{ .922,  .89,  .988} GPT-4-turbo-128k & \textbf{57.19} & \textbf{37.35} & \textbf{40.76} & \textbf{37.56} & \textbf{43.06} & \underline{45.54} & \underline{42.17} \\
    \rowcolor[rgb]{ .922,  .89,  .988} GPT-4o-mini & \underline{49.88} & 35.37 & \underline{37.82} & \underline{33.26} & 37.95 & \textbf{45.98} & \textbf{46.39} \\
    \rowcolor[rgb]{ .922,  .89,  .988} Doubao-1.5-Pro & 39.68 & \underline{37.02} & 35.25 & 32.46 & \underline{39.47} & 33.53 & 37.00 \\
    \rowcolor[rgb]{ .922,  .89,  .988} GPT-4o & 36.52 & 32.48 & 30.85 & 28.61 & 28.53 & 39.25 & 36.50 \\
    \rowcolor[rgb]{ .922,  .89,  .988} Qwen2.5-Max & 36.11 & 30.82 & 30.49 & 26.73 & 32.81 & 39.49 & 29.54 \\
    \rowcolor[rgb]{ .922,  .89,  .988} Gemini-1.5-pro & 35.51 & 30.32 & 29.59 & 26.56 & 28.20 & 40.13 & 33.66 \\
    \rowcolor[rgb]{ .922,  .89,  .988} DeepSeek-V3 & 32.33 & 28.13 & 27.33 & 27.04 & 27.73 & 27.35 & 27.45 \\
    \rowcolor[rgb]{ .922,  .89,  .988} Llama-3.1-70B-Instruct & 32.22 & 28.85 & 27.67 & 21.49 & 32.13 & 28.45 & 39.18 \\
    \rowcolor[rgb]{ .922,  .89,  .988} Qwen2.5-32B-Instruct & 30.12 & 28.63 & 26.73 & 22.34 & 31.37 & 33.78 & 24.37 \\
    \rowcolor[rgb]{ .882,  .949,  .89} DeepSeek-R1 & 29.20 & 32.66 & 28.43 & 24.17 & 29.28 & 34.78 & 35.87 \\
    \rowcolor[rgb]{ .882,  .949,  .89} o1-preview & 27.92 & 30.59 & 26.97 & 22.19 & 28.09 & 33.11 & 35.94 \\
    % Gemini-2.0-flash-thinking & 14.02 & 17.36 & 14.56 & 14.79 & 11.97 & 19.34 & 15.26 \\
    \rowcolor[rgb]{ .922,  .89,  .988} Qwen2.5-14B-Instruct & 26.64 & 27.27 & 24.73 & 21.51 & 29.05 & 29.98 & 20.59 \\
    \rowcolor[rgb]{ .922,  .89,  .988} Llama-3.1-8B-Instruct & 25.71 & 28.01 & 24.91 & 18.12 & 32.17 & 27.30 & 29.93 \\
    \rowcolor[rgb]{ .882,  .949,  .89} o1-mini & 22.90 & 22.90 & 19.89 & 16.71 & 21.70 & 20.37 & 26.94 \\
    \rowcolor[rgb]{ .922,  .89,  .988} Qwen2.5-7B-Instruct & 21.99 & 19.61 & 18.63 & 11.61 & 25.92 & 29.85 & 15.18 \\
    \rowcolor[rgb]{ .882,  .949,  .89} DeepSeek-R1-Distill-Qwen-32B & 17.19 & 18.65 & 16.28 & 13.02 & 23.55 & 15.05 & 11.56 \\
    % Gemini-2.0-flash-thinking & 14.02 & 17.36 & 14.56 & 14.79 & 11.97 & 19.34 & 15.26 \\
    \rowcolor[rgb]{ .882,  .949,  .89} DeepSeek-R1-Distill-Qwen-14B & 12.81 & 14.54 & 12.55 & 9.40 & 18.36 & 10.44 & 12.01 \\
    % \rowcolor[rgb]{ .882,  .949,  .89} QwQ-32B-Preview & 10.20 & 10.17 & 9.07 & 7.38 & 8.60 & 14.97 & 10.54 \\
    \bottomrule
    \end{tabular}
}
\caption{Experimental results of PRMs and critic models on DeltaBench. \textbf{Bold} indicates the best results within the same group of models, while \underline{ underline} indicates the second best.}
% \vspace{-4mm}
\label{tab: main}
\end{table*}

% \noindent\textbf{Evaluation Metrics.}
% % To accurately assess the performance of the PRM and critic models on DeltaBench, 
% We employ \textbf{recall}, \textbf{precision}, and \textbf{macro-F1 score} for error sections as evaluation metrics. For the PRMs, we utilize an outlier detection technique based on the Z-Score to make predictions. This method was chosen because threshold-based prediction methods determined from other step-level datasets, such as those used in ProcessBench~\citep{Zheng2024ProcessBenchIP}, may not be reliable due to significant differences in dataset distributions, particularly as DeltaBench focuses on long CoT. Outlier detection helps to avoid this bias. The threshold $t$ for determining the correctness of a section is defined as:
% % \begin{align}
% $t = \mu - \sigma$,
% % \nonumber
% % \label{eq: prm_threshold}
% % \end{align}
% where $\mu$ is the mean of the rewards distribution across the dataset, and $\sigma$ is the standard deviation. Sections falling below $t$ are predicted as error sections. For critic models, all erroneous sections within a long CoT are prompted to be identified. Given that error sections constitute a smaller proportion than correct sections across the dataset, we use macro-F1 to mitigate the potential impact of the imbalance between positive and negative sections. Macro-F1 independently calculates the F1 score for each sample
% % (for our metric, each case) 
% and then takes the average, providing a more balanced evaluation metric when dealing with class imbalance.

\noindent\textbf{Baseline Models.}
% 开源(中英模型,llama3)和闭源模型
% To comprehensively evaluate the performance of current PRMs and critic models, we extensively selected and evaluated a wide range of both open-source and closed-source models on DeltaBench.
% \paragraph{Process Reward Models}
For the \textbf{PRMs}, we select the following models: Qwen2.5-Math-PRM-7B\footnote{\href{https://huggingface.co/Qwen/Qwen2.5-Math-PRM-7B}{Qwen/Qwen2.5-Math-PRM-7B}}, Qwen2.5-Math-PRM-72B\footnote{\href{https://huggingface.co/Qwen/Qwen2.5-Math-PRM-72B}{Qwen/Qwen2.5-Math-PRM-72B}}, Llama3.1-8B-PRM-Deepseek-Data\footnote{\href{https://huggingface.co/RLHFlow/Llama3.1-8B-PRM-Deepseek-Data}{RLHFlow/Llama3.1-8B-PRM-Deepseek-Data}}, Llama3.1 -8B-PRM-Mistral-Data\footnote{\href{https://huggingface.co/RLHFlow/Llama3.1-8B-PRM-Mistral-Data}{RLHFlow/Llama3.1-8B-PRM-Mistral-Data}}, Skywork-o1-Open-PRM- Qwen-2.5-1.5B\footnote{\href{https://huggingface.co/Skywork/Skywork-o1-Open-PRM-Qwen-2.5-1.5B}{Skywork/Skywork-o1-Open-PRM-Qwen-2.5-1.5B}}, and Skywork-o1-Open-PRM-Qwen-2.5-7B\footnote{\href{https://huggingface.co/Skywork/Skywork-o1-Open-PRM-Qwen-2.5-7B}{Skywork/Skywork-o1-Open-PRM-Qwen-2.5-7B}}. 
% These represent some of the best open-source PRMs currently available.
% \paragraph{Critic Models}
We select a group of the most advanced open-source and closed-source LLMs to serve as \textbf{critic models} for evaluation, which includes various GPT-4~\citep{gpt4} variants (such as GPT-4-turbo-128K, GPT-4o-mini, GPT-4o), the Gemini model~\citep{Reid2024Gemini1U}(Gemini-1.5-pro), several Qwen models~\citep{qwen2.5} (such as Qwen2.5-32B-Instruct and Qwen2.5-14B-Instruct), Doubao-1.5-Pro~\citep{doubao2025}
and o1 models~\citep{openai-o1} (o1-preview-0912, o1-mini-0912).
% , and a GPT-3.5 variant (gpt-3.5-16K).



\subsubsection{Main Results}
In Table \ref{tab: main},
we provide the results of different LLMs on DeltaBench. 
For PRMs, we have the following observations: (1). Existing PRMs usually achieve low performance, which indicates that existing PRMs cannot identify the errors in long CoTs effectively and it is necessary to improve the performance of PRMs. (2). Larger PRMs
do not lead to better performance. For example, the Qwen2.5-Math-PRM-72B is inferior to wen2.5-Math-PRM-7B.
For critic models, we have the following findings: (1)
GPT-4-turbo-128k archives the best critique results, which is better than other models (e.g., GPT-4o) a lot in DeltaBench. (2) For o1-like models (e.g., DeepSeek-R1, o1-mini, o1-preview), we observe that the results of these models are not superior to non-o1-like models, with the performance of o1-preview is even lower than Qwen2.5-32B-Instruct.
%Additionally, we observe that the QWQ and DeepSeek-R1-Distill series models exhibit weaknesses in following instructions. 
A detailed analysis of underperforming models is provided in Appendix \ref{app: underperforming}.

% model size
% domains
% o1模型跟普通模型critic能力对比分析


\subsubsection{Further Analysis}

\paragraph{Effect of Long CoT Length.}
\begin{figure}[t]
    \centering
    \includegraphics[width=1.0\textwidth]{figures/4.5.1/length2.pdf}
    \caption{The effect of long CoT length.}
    \label{fig: crtic1}
\end{figure}
In Figure \ref{fig: crtic1}, we compare the average F1-Score performance of critic models and PRMs across varying LongCoT token lengths. 
For critic models, the performance notably declines as token length increases. Initially, models like Deepseek-R1 and GPT-4o exhibit strong performance with shorter sequences (1-3k tokens). However, as token length increases to mid-ranges (4-7k tokens), there is a marked decrease in performance across all models. This trend highlights the growing difficulty for critic models to maintain precision and recall as long CoT response become longer and more complex, likely due to the challenge of evaluating lengthy model outputs. In contrast, PRMs demonstrate greater stability across token lengths, as they evaluate sections sequentially rather than processing the entire output at once. Despite this advantage, PRMs achieve lower overall scores compared to critic models on our evaluation set.

\begin{tcolorbox}[colback=white!95!gray, colframe=gray!70!black, title=Key Finding]
  Critic models exhibit significant performance degradation with longer contexts, while PRMs demonstrate consistent evaluation capability across varying lengths.
\end{tcolorbox}


\paragraph{Performance Analysis Across Different Error Types.}
\begin{figure}[t]
    \centering
    \includegraphics[width=0.9\textwidth]{figures/4.5.2/top_models_per_task.pdf}
    \caption{Results of different LLMs on top-5 errors.}
    \label{fig: top_models_per_task}
\end{figure}
Figure \ref{fig: top_models_per_task} shows the performance of different models on the five most common error types. In terms of error types, most models demonstrate the highest accuracy in recognizing calculation errors. Conversely, the recognition of strategy errors is generally the weakest. In terms of models, there is significant variation in the ability of individual models to recognize different error types. For instance, DeepSeek-V3 achieves an F1 of 36\% on calculation errors but only 23\% on strategy errors. Meanwhile, Llama3.1-8B-PRM-Deepseek performs poorly, with an F1 score of 22\% on calculation errors, and shows a significant decline in performance across the other four error types. This highlights the limited generalization capabilities of most models when recognizing various error types.

\begin{tcolorbox}[colback=white!95!gray, colframe=gray!70!black, title=Key Finding]
  Models exhibit strong performance on calculation errors but struggle with strategy errors, revealing limited generalization across error types.
\end{tcolorbox}

\begin{table}[!ht]
    \centering
    % \scriptsize
    % \footnotesize
    \begin{tabular}{cccc}
    \toprule
        \multirow{2}{*}{Model} & \multicolumn{3}{c}{HitRate@$k$ - Avg(\%)} \\ \cline{2-4}
                           & $k=1$ & $k=3$ & $k=5$ \\ 
                           % \hline
                           \midrule
        Qwen2.5-Math-PRM-7B & \textbf{49.15} & \textbf{69.14} & \textbf{83.14} \\
        Qwen2.5-Math-PRM-72B & \underline{41.13} & \underline{62.70} & \underline{75.73} \\ 
        Llama3.1-8B-PRM-Deepseek-Data & 12.63 & 48.62 & 69.78 \\
        Llama3.1-8B-PRM-Mistral-Data & 8.99 & 42.97 & 65.33 \\
        Skywork-o1-Open-PRM-Qwen-2.5-1.5B & 31.90 & 53.82 & 69.23 \\
        Skywork-o1-Open-PRM-Qwen-2.5-7B & 31.58 & 52.59 & 69.16 \\
        % \hline
        \bottomrule
    \end{tabular}
    \vspace{+3mm}
    \caption{Results of HitRate@$k$. Bold and underlined results indicate the best and the second best.}
    % \vspace{-4mm}
\label{tab: hitrate}
\end{table}

\paragraph{Analysis on HitRate evaluation for PRMs.}

\begin{figure}[t]
    \centering
    \includegraphics[width=\textwidth]{figures/prm_rank.pdf}
    % \vspace{-10pt}
    \caption{Ranking of rewards for the first incorrect section for different PRMs.}
    % \vspace{-3mm}
    \label{fig: prm_rank}
\end{figure}

To better measure the ability of PRMs to identify erroneous sections in long CoTs, we use HitRate@$k$ to evaluate PRMs. Specifically, within a sample, we rank the sections in ascending order based on the rewards given by the PRM, select the smallest $k$ sections, and calculate the recall rate for the erroneous sections among them. Specifically, we define the sorted sections as $S = \{s_1, s_2, \ldots, s_n\}$, with $E$ being the set of erroneous sections. We select the top $k$ sections, denoted as $S_k = \{s_1, s_2, \ldots, s_k\}$. The HitRate@$k$ is  calculated as:
\begin{align}
\text{HitRate@}k = \frac{|S_k \cap E|}{\min(k, |E|)}
% \nonumber
\label{eq: hitrate}
\end{align}
In this formula, $|S_k \cap E|$ indicates the number of erroneous sections identified among the top $k$ sections. This metric reflects the ability of PRMs to effectively identify erroneous sections within the top $k$ candidate sections. In Table \ref{tab: hitrate}, the relative performance rankings among different PRMs are quite similar to the results in Table \ref{tab: main}. Additionally, we observe that for $k=3$ and $k=5$, the performance differences between various PRMs are not particularly significant. However, when $k=1$, the Qwen2.5-Math-PRM-7B shows a clear performance advantage. Figure \ref{fig: prm_rank} illustrates the ranking ability of different PRMs for the first incorrect section within the sample, which is generally consistent with the performance evaluation results of HitRate@k.
% This is because a smaller $k$ value imposes stricter requirements on the PRM's ability to identify errors.

% HitRate@$k$ evaluates the performance of PRMs from the perspective of reward ranking, providing additional evidence for the experimental results and conclusions in Table \ref{tab: main} from a different angle.

\begin{tcolorbox}[colback=white!95!gray, colframe=gray!70!black, title=Key Finding]
  HitRate@k evaluation aligns with the main results, with Qwen2.5-Math-PRM-7B demonstrating superior performance in identifying the first incorrect section.
\end{tcolorbox}


\begin{figure}[t]
    \centering
    \includegraphics[width=0.8\textwidth]{figures/4.5.4/self-critic.pdf}
    % \vspace{-10pt}
    \caption{F1-score comparison of self-critique and cross-model critique abilities for different models.}
    % \vspace{-5mm}
    \label{fig: self-critic}
\end{figure}

\paragraph{Comparative Analysis of Self-Critique Capabilities of LLMs.} We randomly sample queries based on domains and models that generate the long CoT output, followed by a statistical analysis of the model's performance in evaluating its own outputs as well as those of other models. In Figure \ref{fig: self-critic},  Gemini 2.0 Flash Thinking, DeepSeek-R1, and QwQ-32B-Preview show lower self-critique scores compared to their cross-model critique scores, indicating a prevalent deficiency in self-critic abilities. Notably, DeepSeek-R1 exhibits the largest discrepancy, with a 36\% decrease in self-evaluation compared to evaluations of other models. This suggests models' self-critic abilities remain underdeveloped.
% signaling an area that requires improvement.

\begin{tcolorbox}[colback=white!95!gray, colframe=gray!70!black, title=Key Finding]
  LLMs demonstrate weaker self-critique performance compared to cross-model critique, highlighting a fundamental limitation in self-critic capabilities.
\end{tcolorbox}



%%%

% \noindent\textbf{Performance Analysis Across Different Categories}

% \begin{figure}[htbp]
% \centering
% \includegraphics[width=\linewidth]{figures/prm_task_comparison.pdf}
% \caption{Performance of PRMs across different categories (outlier detection).}
% \label{fig: prm_task}
% % \vspace{-0.6cm}
% % \vspace{-4mm}
% \end{figure}


% \noindent\textbf{Performance Variation in Different Lengths of Long CoT}

% \noindent\textbf{Performance Analysis Across Different Error Types}

% \noindent\textbf{Analysis of In-Sample Reward Ranking}


% % \subsection{Evaluation Metrics}

% % \subsection{Main Results}

% % \subsection{Further Analysis}
% \subsection{Analysis on LLM Critics}
%  \textbf{error location}



% \subsubsection{The Performance across different domains}

% \begin{figure}[t]
%     \centering
%     \includegraphics[width=0.5\textwidth]{figures/critic6.pdf}
%     \caption{The score distributions across different domains.}
%     \label{fig: crtic2}
% \end{figure}

% In Figure \ref{fig: crtic2}, we illustrate the F1-score distribution of various large language models (LLMs) across different domains. Analyzing model performance across domains reveals that most models demonstrate stronger critiquing abilities in Physics, Chemistry, Biology, and General Reasoning compared to Mathematics and Programming, indicating higher proficiency in scientific and general reasoning tasks. Meanwhile, the performance of each model varies significantly depending on the domain, reflecting inherent strengths and weaknesses in handling different tasks. For instance, the Gemini-1.5-Pro model achieves an F1-score of 40.1\% in PCB, yet only 26.6\% in Mathematics. These discrepancies underscore challenges in the models' generalization capabilities.






\section{Conclusion}
\label{subsection:conclusion}
In this paper, we introduce \OURS, a novel framework designed to identify high-quality data that aligns well with the LLM’s learned knowledge to reduce hallucination.
% Our proposed method includes Internal Consistency Probing and Semantic Equivalence Identification, which are designed to separately measure the LLM's understanding of the given instruction and target response.
% In this way, we can measure the familiarity of the LLM with the instruction data and prevent the model from being trained on unfamiliar data, thereby reducing hallucinations.
NOVA includes Internal Consistency Probing and Semantic Equivalence Identification, which are designed to separately measure the LLM's familiarity with the given instruction and target response, then prevent the model from being trained on unfamiliar data, thereby reducing hallucinations.
Lastly, we introduce an expert-aligned reward model, considering characteristics beyond just familiarity to enhance data quality.
By considering data quality and avoiding unfamiliar data, we can use the selected data to effectively align LLMs to follow instructions and hallucinate less in the instruction tuning stage.
Experiments and analysis show the effectiveness of \OURS.

\section*{Limitations}
Although empirical experiments have confirmed the effectiveness of the proposed \OURS, two major limitations remain. 
Firstly, our proposed method requires LLMs to generate multiple responses for the given instruction, which introduces additional execution time.
However, it is worth noting that this additional execution time is used to perform offline data filtering, our proposed method does not introduce additional time overhead in the inference phase.
Additionally, \OURS~is primarily used for single-turn instruction data filtering, thus exploring its application in multi-turn scenarios presents an attractive direction for future research.

\bibliographystyle{splncs04}
\bibliography{ref}

\end{document}
