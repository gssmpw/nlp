\section{Introduction}
\label{sec:introduction}
The cardiac cine technique in cardiac magnetic resonance (CMR) imaging has become the gold standard for the non-invasive assessment of various cardiovascular functions\cite{ismail2022role}. For ventricular segmentation of the left ventricle (LV), right ventricle (RV), and left ventricle myocardium (MYO) from the background (BG), which is a pixelwise semantic segmentation task, the rise of deep learning has freed cardiologists from labor-intensive manual labeling in recent years.

Unlike typical semantic segmentation tasks, cardiac cine sequences are 4D, with three spatial dimensions (depth, height, width) and time. Each 2D slice (height × width) maintains continuity across spatial and temporal dimensions. Clinically, emphasis is placed on end-diastolic (ED) and end-systolic (ES) phases, so the raw data mainly consists of 3D frames from these phases.

Researchers address this task through two approaches. One treats the entire 3D frame as input and employs a 3D UNet for segmentation (3D-based) \cite{isensee2018automatic,patravali20182d}. However, several studies pointed out that sequential convolution and pooling operations during the UNet encoder phase inherently struggle to learn long-range relationships between pixels \cite{yan2022after, ruan2024vm}, sometimes leading to misidentification of multiple ventricles. The other approach slices the 3D frame into individual 2D slices \cite{chen2021transunet,cao2022swin,sun2020saunet,rahman2024multi,tragakis2023fully}. It enhances segmentation accuracy on a single 2D slice by incorporating mechanisms such as Vision Transformer (attention-based), which can achieve strong numerical performance on overall segmentation. However, clinically, differences between 88\% and 93\% performance are often indistinguishable. Moreover, this approach significantly underperforms on end slices, e.g., the first and last slice of the 3D frame, compared to the middle where ventricular volumes are more pronounced (see Section \ref{sec: exp results}). Therefore, there is still a gap in maintaining high segmentation accuracy overall and on the end slices.

In summary, our contributions are as follows:

\begin{itemize}
\item \textbf{Quantification on End Slices}: We propose a metric in Eq.~\ref{eq:EC} to evaluate segmentation performance specifically on the end slices for each model and conduct benchmarks on current cutting-edge models.

\item \textbf{Comprehensive Ensemble Strategies}: We design an end-to-end pipeline Streaming with neural networks as base classifiers, enabling a straightforward integration of outputs from individual classifiers to mitigate the challenges of learning long-range relationships in single classifiers.

\item \textbf{Memory-based Uncertainty Mechanism}: We compare weight allocation strategies in ensemble learning and introduce a novel approach, termed Memory-based Uncertainty, which leverages global information extracted from 3D frames. Using this approach for weight allocation in our Streaming, we achieve near state-of-the-art (SOTA) performance in overall 3D frame segmentation and significantly improve segmentation accuracy on end slices.

\end{itemize}