% This must be in the first 5 lines to tell arXiv to use pdfLaTeX, which is strongly recommended.
\pdfoutput=1
% In particular, the hyperref package requires pdfLaTeX in order to break URLs across lines.

\documentclass[11pt]{article}

% Change "review" to "final" to generate the final (sometimes called camera-ready) version.
% Change to "preprint" to generate a non-anonymous version with page numbers.
\usepackage[final]{acl}

% Standard package includes
\usepackage{times}
\usepackage{latexsym}
\usepackage{xspace}

\newcommand*\circled[1]{\tikz[baseline=(char.base)]{
            \node[shape=circle,draw,inner sep=.6pt] (char) {#1};}}

% For proper rendering and hyphenation of words containing Latin characters (including in bib files)
\usepackage[T1]{fontenc}
% For Vietnamese characters
% \usepackage[T5]{fontenc}
% See https://www.latex-project.org/help/documentation/encguide.pdf for other character sets

% This assumes your files are encoded as UTF8
\usepackage[utf8]{inputenc}

% This is not strictly necessary, and may be commented out,
% but it will improve the layout of the manuscript,
% and will typically save some space.
\usepackage{microtype}

% This is also not strictly necessary, and may be commented out.
% However, it will improve the aesthetics of text in
% the typewriter font.
\usepackage{inconsolata}

\usepackage{pgfplots}
% \renewcommand{\arraystretch}{1.15}

\usepackage{paralist}
\usepackage{xcolor}

\usepackage[most]{tcolorbox}
\usepackage{colortbl}
\usepackage{enumitem}
\newtcolorbox{mybox}[2][]{
    colback=white,
    colframe=green!45,
    fonttitle=\bfseries,
    coltitle=black,
    sharp corners,
    title=#2,
    #1
}

% Squad QA - based on blue (#0073C2)
\definecolor{squadbase}{HTML}{0073C2}
\newcommand{\squadcolor}[1]{\setlength{\fboxsep}{1.5pt}\colorbox{squadbase!25}{#1}}

% HotPot QA - based on yellow (#EFC000)
\definecolor{hotpotbase}{HTML}{EFC000}
\newcommand{\hotpotcolor}[1]{\setlength{\fboxsep}{1.5pt}\colorbox{hotpotbase!35}{#1}}

% QuAC - based on teal (#00A087)
\definecolor{quacbase}{HTML}{00A087}
\newcommand{\quaccolor}[1]{\setlength{\fboxsep}{1.5pt}\colorbox{quacbase!25}{#1}}

% CoQA - based on red (#CD534C)
\definecolor{coqabase}{HTML}{CD534C}
\newcommand{\coqacolor}[1]{\setlength{\fboxsep}{1.5pt}\colorbox{coqabase!25}{#1}}

% DoQA - using orange (#FF8C38)
\definecolor{doqabase}{HTML}{FF8C38}
\newcommand{\doqacolor}[1]{\setlength{\fboxsep}{1.5pt}\colorbox{doqabase!25}{#1}}

% OR-QuAC - using green (#2ECC71)
\definecolor{orquacbase}{HTML}{2ECC71}
\newcommand{\orquaccolor}[1]{\setlength{\fboxsep}{1.5pt}\colorbox{orquacbase!25}{#1}}

% Synthetic QA - using a complementary color
\definecolor{syntheticbase}{HTML}{9B4DCA}
\newcommand{\syntheticcolor}[1]{\setlength{\fboxsep}{1.5pt}\colorbox{syntheticbase!25}{#1}}

% Table style improvements
\usepackage{booktabs}
\usepackage{multirow}

% abbreviations/dataset/model names
\newcommand{\synqa}{\textsc{SynQA}\xspace}
\newcommand{\synatt}{\textsc{Syn-Att}\xspace}

%% Paragraph customisation
\newcommand{\rparagraph}[1]{\vspace{1.6mm}\noindent\textbf{#1.}}
\newcommand{\rrparagraph}[1]{\vspace{0.4mm}\noindent\textit{#1.}}
\newcommand{\sparagraph}[1]{\vspace{0.0mm}\noindent\textbf{#1.}}

%Including images in your LaTeX document requires adding
%additional package(s)
\usepackage{graphicx}
\usepackage{subcaption}

\usepackage{booktabs} % toprule, bottomrule

\usepackage{multirow}

\usepackage{tcolorbox}

% Define a custom box style for AI prompts
% Define a custom style for system prompts
\newtcolorbox{prompt}{
    colback=black!3,
    colframe=black!40,
    boxrule=0.5pt,
    left=8pt,
    right=8pt,
    top=8pt,
    bottom=8pt,
    arc=2pt,
    breakable,
    enhanced,
    before skip=10pt,
    after skip=10pt
}

\newcommand{\cl}[1]{\textcolor{teal}{\bf\small [#1 --CL]}}

%% text coloring for instructions/TODOs
\newcommand{\gog}[1]{{\color{red}{[TODO]#1}}}

% If the title and author information does not fit in the area allocated, uncomment the following
%
%\setlength\titlebox{<dim>}
%
% and set <dim> to something 5cm or larger.

%\title{ALIGN-QA: Attributing LLM Inference Text to a Predefined Context for Question Answering}
%\title{LLM Explainer: Mitigating Hallucinations by Linking to Reference Sources for Human-Centric Safe Guards to Enable LLMs in High-Risk Domains}
%\title{LLM Explainer: Mitigating Hallucinations with Source Attribution for Easy Verification of LLM Output}
\title{On Synthesizing Data for Context Attribution in Question Answering}
%\title{Synthesizing Data for Context Attribution in Question Answering\\ to Mitigate LLM Hallucinations}
%\title{Synthesizing Data for Context Attribution in Question Answering for LLM Output Verification}
%with Large Language Models

% Author information can be set in various styles:
% For several authors from the same institution:
% \author{Author 1 \and ... \and Author n \\
%         Address line \\ ... \\ Address line}
% if the names do not fit well on one line use
%         Author 1 \\ {\bf Author 2} \\ ... \\ {\bf Author n} \\
% For authors from different institutions:
% \author{Author 1 \\ Address line \\  ... \\ Address line
%         \And  ... \And
%         Author n \\ Address line \\ ... \\ Address line}
% To start a separate ``row'' of authors use \AND, as in
% \author{Author 1 \\ Address line \\  ... \\ Address line
%         \AND
%         Author 2 \\ Address line \\ ... \\ Address line \And
%         Author 3 \\ Address line \\ ... \\ Address line}

%Gorjan, Kiril (shared 1st authorship), Shahbaz, Christopher, Sebastien, Chia-Chien, Timo, Verena, Wiem, Enomoto, Takeoka, Oyamada, Goran, Carolin.
\author{Gorjan Radevski\footnotemark[1]$^{1,6}$, Kiril Gashteovski\footnotemark[1]$^{1, 4}$, Shahbaz Syed$^1$, Christopher Malon$^2$, Sebastien Nicolas$^1$, \\ \textbf{Chia-Chien Hung$^1$, Timo Sztyler$^1$, Verena Heußer$^1$, Wiem Ben Rim$^5$,  Masafumi Enomoto$^3$,} \\ \textbf{Kunihiro Takeoka$^3$, Masafumi Oyamada$^3$, Goran Glavaš$^7$, Carolin Lawrence$^1$}
 \\
  $^1$NEC Laboratories Europe, Germany; $^2$NEC Laboratories America, USA; $^3$NEC Corporation, Japan; \\
  $^4$CAIR, Ss. Cyril and Methodius University of Skopje, North Macedonia; \\ 
  $^5$University College London, UK; $^6$ KU Leuven, Belgium; \\ 
  $^7$Center for Artificial Intelligence and Data Science, University of Würzburg, Germany %\\
  %\texttt{email@domain}
  }


  %\\\And
  %Second Author \\
  %Affiliation / Address line 1 \\
  %Affiliation / Address line 2 \\
  %Affiliation / Address line 3 \\
  %\texttt{email@domain} \\}

%\author{
%  \textbf{First Author\textsuperscript{1}},
%  \textbf{Second Author\textsuperscript{1,2}},
%  \textbf{Third T. Author\textsuperscript{1}},
%  \textbf{Fourth Author\textsuperscript{1}},
%\\
%  \textbf{Fifth Author\textsuperscript{1,2}},
%  \textbf{Sixth Author\textsuperscript{1}},
%  \textbf{Seventh Author\textsuperscript{1}},
%  \textbf{Eighth Author \textsuperscript{1,2,3,4}},
%\\
%  \textbf{Ninth Author\textsuperscript{1}},
%  \textbf{Tenth Author\textsuperscript{1}},
%  \textbf{Eleventh E. Author\textsuperscript{1,2,3,4,5}},
%  \textbf{Twelfth Author\textsuperscript{1}},
%\\
%  \textbf{Thirteenth Author\textsuperscript{3}},
%  \textbf{Fourteenth F. Author\textsuperscript{2,4}},
%  \textbf{Fifteenth Author\textsuperscript{1}},
%  \textbf{Sixteenth Author\textsuperscript{1}},
%\\
%  \textbf{Seventeenth S. Author\textsuperscript{4,5}},
%  \textbf{Eighteenth Author\textsuperscript{3,4}},
%  \textbf{Nineteenth N. Author\textsuperscript{2,5}},
%  \textbf{Twentieth Author\textsuperscript{1}}
%\\
%\\
%  \textsuperscript{1}Affiliation 1,
%  \textsuperscript{2}Affiliation 2,
%  \textsuperscript{3}Affiliation 3,
%  \textsuperscript{4}Affiliation 4,
%  \textsuperscript{5}Affiliation 5
%\\
%  \small{
%    \textbf{Correspondence:} \href{mailto:email@domain}{email@domain}
%  }
%}

%%%%% PITCH %%%%%%%%%%%%%
%% TASK and MOTIVATION
% 1. Question answering accounts for a significant portion of LLM usage "In the wild". Given the LLMs' tendency to hallucinate, that is, generate incorrect or unfaithful answers, it is paramount for LLMs' trustworthiness to equip them with the ability to ground the answers they generate in contextually provided information (i.e., provide evidence for the generated answer). 
%% KEY CONTRIBUTION(S)
% 2. In this work, we systematically investigate LLM-based approaches for this task---known as \textit{context attribution} for question answering---namely, (i) zero-shot inference, (ii) ensembling, and (iii) fine-tuning of (smaller) LLMs on synthetic data generated by larger LLMs. As a central contribution, we propose a novel strategy for synthesizing context attribution data, dubbed \synqa, in which we prompt LLMs to generate question-answer pairs for a given a set of reference sentences (from Wikipedia, mutually related via entities) meant to be the attribution context. 
%% RESULTS AND FINDINGS
%We show that the attribution data synthesized via \synqa is highly effective for fine-tuning (small) LLMs for context attribution. Our extensive experiments encompassing six datasets and two context attribution protocols (in \textit{isolation} vs. in \textit{dialog}) reveal that a small (1B) LLM fine-tuned with \synqa data (1) significantly outperform its counterpart fine-tuned with data synthesized by directly prompting the LLMs to generate attributions given the reference text and question-answer pair, (2) outperforms zero-shot context attribution of orders of magnitude larger LLMs, and (3) generalizes better, that is, is more robust to distribution shifts (e.g., in domain transfer) than models trained on gold context attribution data. Finally, we carry out a user study, the results of which validate the usefulness of small fine-tuned LMs for context attribution for question answering. 



\begin{document}
\maketitle

% Change footnote style to symbols
\renewcommand{\thefootnote}{\fnsymbol{footnote}}

% Manually reset the footnote counter to ensure the first footnote is *
\setcounter{footnote}{0}

\footnotetext[1]{Both authors contributed equally to this work.}

\footnotetext[3]{Correspondance to: gorjan.radevski@gmail.com, kiril.gashteovski@neclab.eu, carolin.lawrence@neclab.eu}

\begin{abstract}


%%%%% PITCH %%%%%%%%%%%%%
%% TASK and MOTIVATION
Question Answering (QA) accounts for a significant portion of LLM usage ``in the wild''. However, LLMs sometimes produce false or misleading responses, also known as ``hallucinations''. Therefore, grounding the generated answers in contextually provided information---i.e., providing evidence for the generated text---is paramount for LLMs' trustworthiness. Providing this information is the task of \textit{context attribution}. In this paper, we systematically study LLM-based approaches for this task, namely we investigate (i) zero-shot inference, (ii) LLM ensembling, and (iii) fine-tuning of small LMs on synthetic data generated by larger LLMs. Our key contribution is \synqa: a novel generative strategy for synthesizing context attribution data. Given selected context sentences, an LLM generates QA pairs that are supported by these sentences. This leverages LLMs’ natural strengths in text generation while ensuring clear attribution paths in the synthetic training data. We show that the attribution data synthesized via \synqa is highly effective for fine-tuning small LMs for context attribution in different QA tasks and domains. Finally, with a user study, we validate the usefulness of small LMs (fine-tuned on synthetic data from \synqa) in context attribution for QA. 

%Our extensive experiments, encompassing six datasets and two context attribution protocols (in \textit{isolation} vs. in \textit{dialog}), reveal that a small (1B) LLM fine-tuned with \synqa data (1) significantly outperforms its counterpart fine-tuned on data obtained by directly prompting the LLMs to generate attributions given the context and question-answer pair (denoted as \synatt), (2) outperforms zero-shot context attribution of orders of magnitude larger LLMs, and (3) generalizes better, i.e., is more robust to distribution shifts (e.g., domain transfer) than models trained on gold attribution data. Finally, we carry out a user study which validates the usefulness of small fine-tuned LMs in context attribution for question answering. 

%Large Language Models (LLMs) are used for a wide range of variety of tasks. People mostly use them as Question Answering (QA) systems, by posing queries (i.e., questions) to them and the LLMs provide answers. The generated answers, however, sometimes contain information that is factually incorrect; i.e., the LLMs sometimes hallucinate. Therefore, it is important for people to have a context attributor: an alignment model that links the generated inference text to a predefined context. In this paper, we study the capabilities of current LLMs to perform the QA attribution task. We found that (1) large zero-shot LLMs have decent performance, but are limited by their large size; (2) fine-tuning smaller LMs is good for in-domain cases, but is severely limited with out-of-domain cases; (3) discrimantive and generative synthetic generation strategies for fine-tuning smaller LMs are superior to using large LMs; (4) a user study which verifies that our small fine-tuned LMs are indeed helpful for human users. We will release the data, benchmark and code upon acceptance.
\end{abstract}


The increasing reliance on LLMs for multimodal tasks across far-reaching sectors such as healthcare, finance, and manufacturing underscores the need to assess the accuracy and reliability of the information they generate. Vision-Language Models (VLM) have achieved state-of-the-art (SoTA) performance on Visual Question-Answering (VQA) benchmarks, and these models often utilize Retrieval-Augmented Generation (RAG) to maintain factual accuracy and relevance in a dynamic information environment. However, this has led to uncertainty in the information the LLM bases its answer on, as it may choose between parametric memory and retrieved sources. When models rely on memorized information instead of dynamically retrieving information, they may inadvertently propagate outdated or incorrect information, causing serious legal and ethical risks and undermining trust and reliability in AI systems \citep{huang2023survey}.
% The ability to strike a balance between generalization and specialization in AI systems is therefore crucial for ensuring the safe, reliable use of these technologies in real-world applications.

Despite these concerns, the way that Vision-Language models (VLMs) memorize and retrieve information, particularly in complex multimodal tasks, remains under-explored. Current research often focuses on either the general capabilities of large language models (LLMs) or the specialized retrieval mechanisms in retrieval augmented generation systems (RAG) \citep{incontext_rag,chen_murag_2022,liu_universal_2023}. Particularly in the context of multimodal retrieval and multihop reasoning, few studies analyze the tradeoff between finetuning for specialized tasks and zero-shot prompting for general-purpose vision-language capabilities. A lack of consensus on how to approach this tradeoff motivates the development of measures to quantify reliance on parametric memory, as well as metrics for quantifying the potential performance impact of extending LLMs with RAG systems.

To address this gap, we investigate how multimodal QA models balance accuracy with memorization on the WebQA benchmark. We compare finetuned multimodal systems against zero-shot VLMs, analyzing how retrieval performance influences QA accuracy. In particular, we focus on cases where retrieval fails, allowing us to measure reliance on parametric memory through two proposed metrics---the \ppr (\PPR) which quantifies how much model accuracy is influenced by retrieval quality, contrasting performance in best-case versus worst-case retrieval scenarios, and the \ucr (\UCR) which measures how often correct QA responses are generated when the retriever fails, providing a proxy for memorization.

To enable this analysis, we make several methodological contributions. For the finetuned QA models, we investigate Vision-Transformer (ViT) architectures, which allow for multihop reasoning over multiple sources. To investigate the impact of retrieval performance on trained LMs, we propose a variable-input Fusion-in-Decoder (FiD) model \cite{tanaka_slidevqa_2023, nlvr2}, building upon the VoLTA architecture \citep{pramanick_volta_2023}. For the zero-shot case, we build upon previous research on In-Context Retrieval \citep{incontext_rag} by demonstrating that LLMs such as GPT-4o are capable of performing the final ranking step of the retrieval process. In doing so, we find that GPT-4o, a general-purpose LLM, achieves SoTA performance on the WebQA task, outperforming existing finetuned RAG models by a significant margin (7\% higher accuracy). 

Crucially, our results reveal that while retrieval-augmented models reduce memorization, the training paradigm plays an important role. Finetuned models exhibit higher reliance on parametric memory, whereas zero-shot RAG approaches have lower memorization scores at the cost of accuracy. This suggests that while retrieval modules may mitigate the risks associated with outdated or incorrect information, SoTA performance requires that they be coupled with specialized QA models. Our memorization measures contribute to the development of transparent and reliable AI systems, particularly in applications where the sourcing of up-to-date, factual information is critical.



% We investigate the impact of question complexity on the ability of these models to integrate multiple data sources—such as images, text, and external retrievers—and produce coherent and accurate answers. We also explore whether in-context retrieval can be a viable alternative to traditional retrieval-augmented systems, offering a more streamlined approach to multimodal QA.

% To achieve this, we first compare zero-shot prompting multimodal LLMs with finetuned multimodal systems. We evaluate both types of models on the WebQA benchmark, a dataset designed for complex question answering that requires reasoning across both image and text sources. For the finetuned models, we use a Fusion-in-Decoder (FiD) architecture, which allows for multihop reasoning over multiple sources. Additionally, we introduce the concept of In-Context Retrieval Language Modeling (RLM), where the LLM itself performs retrieval tasks without the need for external retrievers. This method builds upon existing research in in-context learning  and aims to explore the viability of LLMs retrieving relevant sources and generating accurate answers directly from their context window.

% In order to investigate source utilization in finetuned multimodal models and LLMs, three lines of inquiry are established; 
% \begin{itemize}
%     \item Study 1: retrieval vs QA performance on webQA (motivating example, does QA answer correctly even with incorrect sources?)
%     \item Study 2: performance on adversarial examples where parametric knowledge would be incorrect by design
%     \item Study 3: improving performance on adversarial examples by fine-tuning (i.e model robustness)
% \end{itemize}

% Note, there is one weakness in this plan which is tying in the work we've already done. 
% If we added something from adversarial generation to the retrieval experiment (like a combination of study 1 + 3) it would be complete. So for instance we could try fine-tuning the retriever with adversarial examples (and not just the QA model)

% \begin{figure}
%     \centering
%     \includegraphics[width=0.95\linewidth]{figures/segmentation/webqa_segment_infill.png}
%     \caption{Example of the segmentation substitution pipeline from the WebQA task.}
%     % d5c76d760dba11ecb1e81171463288e9
%     \label{fig:seg_sub_pipeline}
% \end{figure}



% Retrieval augmented generation (RAG) with zero-shot prompting and fine-tuning Large Language Models (LLMs) have become the go-to methods for tasks relying on information retrieval and text generation. In many cases the LLMs parametric memory can sufficiently generalize to answer questions without being provided with retrieval mechanisms for out-of-domain knowledge. However, LLMs often hallucinate and provide wrong information in certain scenarios. This problem is amplified even further on open-domain Question Answering (QA) tasks involving multiple modalities. Grounded text generation using retrieved sources \citep{lewis2021retrievalaugmented} has been extensively studied for text-to-text QA tasks, but its application in multimodal settings has not been studied as much.


% Multimodal reasoning and question answering have gained prominence in recent research endeavors, with an increasing emphasis on handling various forms of data, particularly text and images. In this study, we address a specific gap in the existing literature by focusing on the development of a versatile multihop model capable of accommodating varying numbers of input images.

% Our motivation for this research lies in the growing complexity of answering questions using information on the web, where the challenge of navigating the open-domain setting is further complicated by the presence of multiple modalities and sometimes requires reasoning over multiple sources. WebQA is an ideal dataset on which to compare performance of finetuned RAG systems against general purpose LLMs; it is multimodal, with correct answers requiring reasoning over image and text sources. It is multihop, requiring a complex reasoning process over multiple sources. Finally, WebQA questions from different categories can be broken down into subdomains to analyze performance over domains of varying cardinality.

% Motivated by the real-world challenges of building retrieval and question answering (QA) systems, we design and finetune a closed domain, multimodal, multihop QA model, that is capable of reasoning over a varying number of sources taken as input from an external retriever module. This research contributes to the relatively underexplored domain of multihop reasoning across various input sources and modalities. Our goal is to explore the challenges posed by these scenarios and develop strategies that enable QA models to retrieve relevant information, conduct logical or numerical reasoning across diverse modalities, and generate coherent responses in natural language. To our knowledge, this is the first application of the Fusion-in-Decoder (FiD) architecture \cite{tanaka_slidevqa_2023, nlvr2} that is shown to work with a variable number of inputs, enabling multi-hop reasoning over sources.

% In-Context Learning refers to the ability of LLMs to perform any task by simply providing examples in the input prompt \citep{dong2022survey,min2022rethinking}. Inspired by this research, we propose a method to use the LLM itself as a multimodal retriever, potentially eschewing the requirement of a distinct retrieval module, thereby allowing the design of simpler retrieval-augmented QA systems. We dub this method In-Context Retrieval Language Modeling (RLM). To the best of the authors knowledge, In-Content RLM is disparate from other retrieval augmented approaches which utilize external retrieval modules \citep{incontext_rag,chen_murag_2022,liu_universal_2023}. Despite being a natural extension of In-Context learning, In-Context RLM has not yet been studied empirically.

% To expand on our contribution of In-Context Retrieval, this stems from the well-researched in-context learning of LLMs. In-context learning is the ability of a model to perform any task given a sufficient context window \citep{dong2022survey,min2022rethinking}. Such tasks could include retrieval and ranking, but typically, the go-to solution for tasks requiring retrieval has been RAG. To the best of the authors knowledge, In-Context Retrieval is distinct from In-Context Retrieval Augmented Language Modelling (RALM), and despite being a natural extension of In-Context learning, In-Context Retrieval has not yet been shown empirically.

% Finally, we explore the tradeoff between using zero-shot prompting LLMs and the fine-tuning approach. While we find that, overall, GPT-4o obtains SoTA performance on the WebQA task, outperforming the accuracy of existing finetuned RAG approaches by 7\%, finetuned approaches still perform better on more restricted subdomains\footnote{``In-Context RLM" @ \url{https://eval.ai/web/challenges/challenge-page/1255/leaderboard/3168}}. Finally, we validate that GPT-4o is relying on retrieval abilities to solve the task; we find that GPT-4o is capable of retrieving relevant sources in the presence of distractors and furthermore, when GPT-4o fails to retrieve correct sources, it answers incorrectly 75\% of the time, meaning that it is not relying on parametric memory for this task.

% \paragraph{Contributions}
% Based on our experimentation and analysis on the WebQA benchmark, we make the following contributions:
% \begin{itemize}
%     \item Propose a new architecture for multimodal multihop QA that takes variable number of input sources inspired by the Fusion-in-Decoder method.
%     \item Comparison of general purpose LLMs vs specialized models on the WebQA benchmark.
%     \item Observation of In-Context Multimodal Retrieval abilities of GPT-4o and that it does not rely on parametric memory for multimodal QA.
%     \item Analysis of relationship between retrieval and QA task performance.
%     \item Analysis of task and query complexity on the performance of retrieval and QA tasks.
% \end{itemize}
















% Throughout this paper, we will present our methodology, experiments, and findings, emphasizing our approach to multihop reasoning over varying numbers of input images. We believe that our work contributes to a deeper understanding of multimodal reasoning and has the potential to enhance the capabilities of question-answering systems in the intricate, multimodal landscape of web-based information.
\section{Synthesizing Attribution Data}

\begin{figure*}[ht]
    \centering
    \includegraphics[width=\textwidth]{img/pipeline.drawio.pdf}
    \caption{\textbf{Top:} The \synatt baseline method for synthetic attribution data generation. Given context and question-answer pairs, we prompt an LLM to identify supporting sentences, which are then used to train a smaller attribution model. However, this discriminative approach may yield noisy training data as LLMs are less suited for classification tasks (see \S\ref{sec:experiments-zero-shot}). \textbf{Bottom:} The \synqa data generation pipeline leverages LLMs' generative strengths through four steps: (1) collection of Wikipedia articles as source data; (2) extraction of context attributions by creating chains of sentences that form hops between articles; (3) generation of QA pairs by prompting an LLM with only these context attribution sentences; (4) compilation of the final training samples, each containing the generated QA pair, its context attributions, and the original articles enriched with related distractors.}
    % \caption{\textbf{Top:} The \synatt baseline. Intuitively, we can prompt an LLM for context-attribution by providing the context and question-answer pairs. Then, we train a smaller model on the obtained synthetic data. However, LLMs are less suitable for discriminative (i.e., classification) tasks, and may yield noisy training data (see \S\ref{sec:experiments-zero-shot}). \textbf{Bottom:} The \synqa data generation pipeline consists of four main steps: (1) collection of Wikipedia articles as the source data; (2) extracting the context attributions by creating chains of sentences that form hops between articles; (3) generation of QA pairs by prompting an LLM with only the context attribution sentences; (4) we obtain the resulting \synqa training sample containing three components: the generated QA pair, the context attributions, and the original articles supplemented with related distractor articles.}
    \label{fig:method}
\end{figure*}

Context attribution identifies which parts of a reference text support a given question-answer pair~\cite{rashkin2023measuring}. Formally, given a question $q$, its answer $a$, and a context text $c$ consisting of sentences ${s_1, ..., s_n}$, the task is to identify the subset of sentences $S \subseteq c$ that fully support the answer $a$ to question $q$. To train efficient attribution models without requiring expensive human annotations, we explore synthetic data generation approaches using LLMs.
% Context attribution poses the following question~\cite{rashkin2023measuring}: given a generated text $t_g$ and a context text $t_c$, is $t_g$ attributable to $t_c$? To train models to perform well on this task, we explore how to best generate synthetic attribution data using LLMs. We implement two methods: a discriminative and generative method. 
We implement two methods for synthetic data generation. Our baseline method (\synatt) is discriminative: given existing question-answer pairs and their context, an LLM identifies supporting sentences, which are then used to train a smaller attribution model. Our proposed method (\synqa) takes a generative approach: given selected context sentences, an LLM generates question-answer pairs that are fully supported by these sentences. This approach better leverages LLMs' natural strengths in text generation while ensuring clear attribution paths in the synthetic training data.

%The first method is relatively straightforward and termed \synatt. A simple way to generate synthetic data for context attribution is to ask an LLM to pick out the sentences that support a given question-answer pair. 

% \subsection{Discriminative and Generative Synthetic Data Generation}

% The first method (\synatt) is relatively straightforward: ask the LLM to pick relevant sentences from a provided context that support a given question-answer pair. However, this \textit{discriminative} approach of performing sentence classification overlooks the fact that LLMs excel at \textit{generating} text. Therefore, we design a second data generation method (\synqa) that is generative and thus capitalizes on the strength of LLMs. It involves the following pipeline steps (see also Fig.~\ref{fig:method}): context collection, question-answering generation and distractor mining, which increases the difficulty of the task, thus reflecting more realistic scenarios.

%\textbf{Attribution Synthesis.} The most straightforward approach to generating synthetic data for context attribution is discriminative: prompting an LLM to identify relevant sentences from context documents given a question-answer pair. While intuitive, this approach underutilizes LLMs' capabilities, as they excel at generative rather than discriminative tasks. LLMs are fundamentally designed to generate coherent text following instructions rather than perform binary classification of sentences. In our experiments (\S\ref{sec:experiments}) we dub this method as \synatt.

\subsection{\synqa: Generative Synthetic Data Generation Method}

\synqa consists of three parts: context selection, QA generation, and distractors mining (for an illustration of the method, see Figure~\ref{fig:method}). In what follows, we describe each part in detail.

\textbf{Context Collection.} We use Wikipedia as our data source, as each article consists of sentences about a coherent and connected topic, with two collection strategies. In the first, we select individual Wikipedia articles for dialogue-centric generation and use their sentences as context. In the second, for multi-hop reasoning, we identify sentences containing Wikipedia links and follow these links to create ``hops'' between articles, limiting to a maximum of two paths to maintain semantic coherence, while enabling more complex reasoning patterns (for more details, see Appendix~\ref{app:synthetic_data}).
% \textbf{Context Collection.}  The first step is to select a dataset where each data point is a set of sentences about a coherent and connected topic. These sentences will serve as the context in which we want to find relevant attributions later. We use Wikipedia as the data source
%To better leverage LLMs' generative capabilities, we propose \synqa, a novel and simple approach for synthesizing context attribution data (see Fig.~\ref{fig:method}). 
%We first collect Wikipedia articles that are not present in our testing datasets\footnote{We detect potential data leakage by representing each Wikipedia article as a MinHash signature. Then, for each training Wikipedia article, we retrieve candidates from the testing datasets via Locality Sensitivity Hashing and compute their Jaccard similarity \cite{dasgupta2011fast}. Pairs exceeding a tunable threshold (empirically set to 0.8) are flagged as potential leaks.}.
%For each article, 
% we implement two distinct collection strategies that differ in difficulty. First, we select individual Wikipedia articles and randomly select multiple sentences within each article. Second, we start from a randomly selected sentence containing at least one Wikipedia link
%\footnote{These are human annotated in the Wikipedia articles, or alternatively, can be obtained from entity linking methods \cite{de-cao-etal-2022-multilingual}.} 
% and follow the links to other articles, creating ``hops'' between related content. We limit the chain to a maximum of two hops (connecting up to three articles) to maintain semantic coherence while enabling the more difficult multi-hop reasoning scenarios (for more details, see Appendix~\ref{app:synthetic_data}). 
%In the second strategy, we select individual Wikipedia articles and randomly select multiple sentences within each article that can serve as evidence for generated questions.

\textbf{Question-Answer Generation.} Given the set of contexts, an LLM can now generate question-answer pairs. For single articles, we prompt the model to generate multiple question-answer pairs, each grounded in specific sentences. This creates a set of dialogue-centric samples where questions build upon the previous context. For linked articles, we prompt the model to generate questions that necessitate connecting information across the articles, encouraging multi-hop reasoning.
%\footnote{Note that multi-hop reasoning is not guranteed here; rather, the LLM has the ability to decide whether the question-answer pair involves multiple hops of reasoning. See App. for details.}. 
This yields multi-hop samples requiring integration of information across documents, as well as samples that mimic a dialogue about a specific topic given the context. We provide the full prompts used for generation in Appendix \ref{app:prompts}.

\textbf{Distractors Mining.} To make the attribution task more realistic, we augment each sample with distractor articles. With E5 \cite{wang2022text}, we embed each Wikipedia article in our collection. For each article in the training sample, we randomly select up to three distractors with the highest semantic similarity to the source articles. These distractors share thematic elements with the source articles, but lack information to answer the questions.%do not contain the information necessary to answer the generated questions.

\subsection{Advantages of \synqa}
The \synqa approach has three key advantages:
%over discriminative data generation:
% (1) it leverages LLMs' natural strength in generative tasks; (2) produces diverse multi-hop reasoning scenarios; and (3) creates coherent question-answer pairs with clear attribution paths.
(1) it leverages LLMs' strength in generation rather than classification; (2) creates diverse training samples requiring both dialogue understanding and multi-hop reasoning; and (3) ensures generated questions have clear attribution paths since they are derived from specific context sentences.
By generating both entity-centric and dialogue-centric samples, \synqa produces training data that reflects the variety of real-world QA scenarios, helping models develop robust attribution capabilities, which our experiments demonstrate to generalize across different contexts and domains.
% We formalize the problem of Context Attribution QA as follows: Given a pre-defined context $T_c=\lbrace s_1, s_2, \ldots , s_n \rbrace$---where $s_i$ is a sentence---and an answer text $t_a$ generated by an LLM, the context attribution model should provide a vector $a=(a_1, \ldots , a_n)$, where each element $a_i$ has the following possible values:
% \[
% a_i =
% \begin{cases}
%     1, & \text{if } s_i \text{ supports the generated answer } t_a\\
%     0,  & \text{otherwise} 
% \end{cases}
% \]
% In our setup, we should have at least one entry $a_i = 1$.
% \begin{itemize}
%     \item The simplest way to generate synthetic data for context-attribution is in a discriminative manner: we prompt an LLM to provide the sentence level context attributions given the context documents, question and answer. We deem this generation as discriminative as the model effectively classifies the sentences that are most relevant to the question-answer pair.
%     \item The issue with this approach is that LLM are not best suitable for discriminative tasks, but rather generative. That is, an LLM is better at generating text by following instructions, than classifing sentences/etc.
%     \item To leverage what LLMs are good for, we create a simple context attribution data generation approach where we perform the following: (1) We find wikipedia articles (which are not contained in the testing datasets)\footnote{Describe the approach for dealing with data leakage}; (2) We select a random sentence in a wikipedia article, and find the links to other wikipedia articles (the hops). We select that sentence, and hop to the other Wikipedia article (given by the link). (3) We perform the hop step for maximum of 2 times (i.e., we connect at most 3 articles, and 1 at least). We end up with 3 Wikipedia articles which constitute the hops.
%     \item We provide Llama70B with either 1 wikipedia article or the hops and ask the model to generate a multi-hop question-answer pair which ideally connects all connected articles, or as many as it can; alternatively, if we provide the model with only 1 wikipedia article, we ask the model to select as many sentences as possible in the article, and for each, generate a question-answer pair (we provide the full prompts we use in Appendix).
%     \item The output of the model is a set of question-answer pairs (or a single one), that is grounded in the evidence provided by the sentence(s). We dub the entire approach as \synqa.
%     \item In summary, we develop two settings to generate synthetic data for context attribution in question answering: one is entity-centric and yield data which might be multi-hop; and the other is dialog-centric where subsequent questions build on top of previous ones.
%     \item Finally, to all context + question + answer + context-attribution samples we add distractors: we obtain embeddings using E5 of each wikipedia page, and for each sample we select up to 3 distractors which we add to the data sample. These distractors are similar are document with similar context as the one from which the context-attributions are.
% \end{itemize}


\section{Experimental Study}\label{sec:experiments}
We conduct a comprehensive evaluation across multiple aspects: zero-shot performance, comparison with training on gold attribution data, and generalization to dialogue settings.
% . Our experiments span both in-\textit{isolation} question-answering datasets and in-\textit{dialogue} scenarios,
With our experiments, we shed light on the performance and practical utility of our approach.

% \textcolor{red}{TODO: Should we introduce the two settings separately: in-isolation QA and in-dialogue QA?}

% \textcolor{red}{TODO: We need an introduction sentence for this section.}

% \textcolor{red}{TODO: We mention isolated context attribution in some parts of the paper, while it is not clear how it differs from dialog-based context attribution.}

\subsection{Experimental Setting}
We evaluate model performance using precision (P), recall (R), and F1 score. For each sentence in the LLM's output, the context-attribution models identify the set of context sentences that support that output sentence. Precision measures the proportion of predicted attributions that are correct, while recall measures the proportion of ground truth attributions that are successfully identified.
%F1 is the harmonic mean of precision and recall.

For a fair and comprehensive evaluation, we train all models with a single pass over the training data unless stated otherwise, referring to this setup as \textbf{1P} when needed. For a more controlled comparison, some experiments limit the number of training samples each model encounters. Since the synthetic dataset contains approximately 1.0M samples, we allow models to \textit{observe} an equivalent number of samples from the gold training set, ensuring comparable exposure to models trained on data from \synqa. We refer to this setting as \textbf{1M} when necessary. For all models, we fine-tune only the LoRA parameters (alpha=64, rank=32) using a fixed learning rate of 1e-5 and a weight decay of 1e-3. 

\textbf{In-domain datasets:} We use \squadcolor{SQuAD} \cite{Rajpurkar2016SQuAD1Q} and \hotpotcolor{HotpotQA} \cite{Yang2018HotpotQAAD} as our primary in-domain benchmarks.\footnote{For some experiments (e.g., in Table~\ref{table:zero-shot-models}), these datasets are also \textit{out-of-domain} w.r.t. data generated by \synqa.} SQuAD provides clear sentence-level evidence for answering questions, serving as a strong baseline for direct attribution. HotpotQA introduces multi-hop reasoning, requiring models to link information across multiple sentences (sometimes from different articles) to identify the correct evidence chain. Additionally, HotpotQA includes distractor documents—closely related yet incorrect sources—posing a more challenging but realistic setting for evaluating attribution performance.

%\textcolor{red}{TODO (Kiril): Explain what you do to OR-QUAC, you combine the background with the context?}
\textbf{Out-of-domain datasets:} To assess generalization beyond the training distribution, we evaluate models on \quaccolor{QuAC} \cite{Choi2018QuACQA}, \coqacolor{CoQA} \cite{Reddy2018CoQAAC}, \orquaccolor{OR-QuAC} \cite{qu2020open}, and \doqacolor{DoQA} \cite{campos-etal-2020-doqa}. %\footnote{We consider these datasets as \textit{out-of-domain}, as none of the models we train are exposed to the training data of these datasets.}. 
These datasets present conversational QA scenarios that differ from SQuAD and HotpotQA. Specifically, QuAC and CoQA introduce multi-turn dialogue structures with coreferences, challenging models to track context across multiple turns. This conversational nature creates a methodological challenge: while these datasets are valuable for evaluating dialogue-based attribution, their reliance on conversation history makes direct comparison with models trained on single-turn QA datasets impossible.

To enable comprehensive evaluation across dialogue QA and single-turn QA, we create two versions of each dataset:
\begin{inparaenum}[(i)]
    \item a rephrased version using Llama 70B \cite{Dubey2024TheL3} that converts questions into standalone format for fair comparison with models trained on single-turn context attribution (suffixed by ``-ST''), and
    \item the original version for assessing dialogue-based attribution.
\end{inparaenum}

% To adapt these datasets for isolated context attribution (e.g., such as SQuAD and HotpotQA, where the question-answer pair is standalone),
% \footnote{We refer to isolated context attribution the scenario where the question-answer pair are standalone: i.e., do not contain coreferences.},
% we rephrase question-answer pairs (using Llama 70B), so that coreferencing is unnecessary. However, in dialogue-based settings, we evaluate models on the original, unmodified versions of these datasets.

DoQA extends this challenge further by incorporating domain-specific dialogues (cooking, travel and movies)%\footnote{The domains covered in DoQA are: cooking, travel and movies \cite{campos-etal-2020-doqa}.}
, thus testing the models' adaptability to specialized contexts. OR-QuAC includes %open-retrieval dialogue settings, assesses models' ability to attribute context in less structured environments, adding another layer of complexity to generalization evaluation. \textcolor{red}{TODO (Kiril): Check the papers for these datasets in case something is overlooked here.}
context-independent rewrites of the dialogue questions, such that they can be posed in isolation of prior context (i.e., single-turn QA). This enables us to test the models on their capabilities in both single-turn QA and dialogue QA settings.

\subsection{Methods}
We compare our method (\synqa) against several baselines, including sentence-encoder-based models, zero-shot instruction-tuned LLMs, and models trained on synthetic and gold context-attribution data. Specifically, we experiment with the following methods:

\paragraph{Sentence-Encoders:} We embed each sentence in the context along with the question-answer pair, and select attribution sentences based on cosine similarity with a fixed threshold, tuned on a small validation set.

\paragraph{Zero-shot (L)LMs:} We evaluate various instruction-tuned (L)LMs in a zero-shot manner, as such models have been shown to perform well across a range of NLP tasks \cite{shu2023exploitability,zhang2023instruction}. During inference, we provide an instruction template describing the task to the LLM (see Appendix~\ref{app:prompts} for details).
%as such models have been shown to perform well across a range of NLP tasks.

\paragraph{Ensembles of LLMs:} We aggregate the predictions of multiple LLMs through majority voting, selecting attribution sentences that receive consensus from at least 50\% of the ensemble. In our experiments, we use Llama8B \cite{Dubey2024TheL3}, Mistral7B, and Mistral-Nemo12B \cite{Jiang2023Mistral7} as the ensemble constituents.


\paragraph{Models trained on in-domain gold data:} Fine-tuning on gold-labeled attribution data provides an upper bound on in-domain performance, helping us assess how well synthetic training data generalizes.

\paragraph{\synatt:} \synatt generates synthetic training data by prompting multiple LLMs to perform context attribution in a discriminative manner, aggregating their outputs via majority voting, and training a smaller model on the resulting dataset. To make it a stronger baseline against \synqa, we give the training data of SQuAD and HotpotQA (the context, questions, and answers) to the LLMs and ask them to perform context attribution (note that we do not use the gold attribution). Finally, we train a model on the generated synthetic data.

\paragraph{\synqa:} We train models using synthetic data generated by our proposed method \synqa. Note that even though we train models using \synqa attribution data, we ensure they are not exposed to \textit{any} parts of the evaluation data.\footnote{We identify data leakage by representing each Wikipedia article as a MinHash signature. Then, for each training Wikipedia article, we retrieve candidates from the testing datasets via Locality Sensitivity Hashing and compute their Jaccard similarity \cite{dasgupta2011fast}. We flag as potential leaks pairs exceeding a threshold empirically set to 0.8.}

\subsection{Results and Discussion}
Evaluating our context attribution models requires a multifaceted approach, as performance is influenced by both the quality of training data and the model’s ability to generalize beyond in-domain distributions. Therefore, we design our experiments to address five core questions:
\begin{inparaenum}[(i)]
    \item How well do zero-shot LLMs perform on context-attribution QA tasks (\S\ref{sec:experiments-zero-shot})?
    \item Can models trained on synthetic data generated by \synqa exceed the performance of models trained on gold context-attribution data (\S\ref{sec:experiments-gold})?
    \item To what extent do models generalize to dialogue settings where in-domain training data is unavailable (\S\ref{sec:experiments-dialog})?
    \item How well do models scale in terms of synthetic data quantity generated by \synqa (\S\ref{sec:scalling-trends})?
    \item How do improved context attributions impact the end users' speed and ability to verify questions answering outputs (\S\ref{sec:user-study})?
\end{inparaenum}
% (i) How well do zero-shot LLMs perform context-attribution? (ii) Can synthetic attribution data serve as a viable alternative to gold supervision, particularly in out-of-domain settings? (iii) How do scaling trends affect generalization performance across diverse datasets?

%By systematically comparing models trained on synthetic data to both zero-shot and gold-supervised baselines, we aim to uncover the trade-offs between scalability, performance, and generalization. 
%Collectively, our findings provide a deeper understanding of how synthetic data can be leveraged for context attribution, potentially mitigating the reliance on costly human-annotated datasets.

\subsubsection{Comparison to Zero-Shot Models}\label{sec:experiments-zero-shot}

% Zero-shot v.s. SynQA-trained Models

\begin{table*}[t]
\centering
\resizebox{1.0\textwidth}{!}{
\begin{tabular}{lccccccccccccccc} \toprule
\multirow{2}{*}{Model} & \multirow{2}{*}{Training data} & \multicolumn{3}{c}{\squadcolor{Squad}} & \multicolumn{3}{c}{\hotpotcolor{Hotpot}} & \multicolumn{3}{c}{\quaccolor{Quac-ST}} & \multicolumn{3}{c}{\coqacolor{CoQA-ST}} \\ \cmidrule(lr){3-5} \cmidrule(lr){6-8} \cmidrule(lr){9-11} \cmidrule(lr){12-14}
& & P & R & F1 & P & R & F1 & P & R & F1 & P & R & F1 \\ \midrule
\textbf{\textit{Baselines}} \\
Random & -- & 19.8 & 15.4 & 17.3 & 4.8 & 15.2 & 7.3 & 5.2 & 15.1 & 7.7 & 7.3 & 15.1 & 9.9 \\
E5 | 561M & Zero-shot & 38.1 & 76.5 & 50.9 & 12.4 & 41.4 & 19.1 & 65.0 & 73.8 & 69.1 & 61.1 & 15.2 & 24.4 \\
HF-SmolLM2 | 365M & Zero-shot & 28.1 & 46.4 & 35.0 & 5.1 & 7.3 & 6.0 & 10.6 & 22.6 & 14.4 & 10.6 & 21.5 & 14.2 \\
Llama | 1B & Zero-shot & 37.5 & 62.0 & 46.7 & 5.3 & 28.1 & 8.9 & 8.8 & 65.4 & 15.4 & 11.9 & 52.8 & 19.4 \\
Mistral | 7B & Zero-shot & 71.5 & 94.4 & 81.4 & 42.9 & 42.7 & 42.8 & 63.2 & 88.6 & 73.8 & 59.0 & 72.2 & 64.9 \\
Llama | 8B & Zero-shot & 71.9 & 96.9 & 82.6 & 49.2 & 52.9 & 51.0 & 64.1 & 92.1 & 75.6 & 55.7 & 76.4 & 64.4 \\
Mistral-NeMo | 12B & Zero-shot & 89.5 & 94.5 & 91.8 & 46.4 & 47.3 & 46.8 & 81.8 & 85.3 & 83.5 & 79.0 & 67.2 & 72.6 \\
Ensemble | 27B & Zero-shot & 83.1 & 96.3 & 89.2 & 48.1 & 59.6 & 53.2 & 74.8 & 90.3 & 81.8 & 69.5 & 73.6 & 71.5 \\
Llama | 70B & Zero-shot & 95.3 & 95.6 & 95.5 & 87.6 & 37.5 & 52.5 & 89.7 & 87.8 & 88.7 & \textbf{87.5} & \textbf{73.3} & \textbf{79.8} \\
\midrule
\textbf{\textit{Baselines}} \\
%Llama | 1B & \squadcolor{SQuAD} \& \hotpotcolor{HotpotQA}; \synatt (1P) & 89.8 & 96.5 & 93.0 & 50.6 & 58.6 & 54.3 & 64.9 & 91.5 & 75.9 & 53.1 & 75.5 & 62.3 \\
%Llama | 1B & \squadcolor{SQuAD} \& \hotpotcolor{HotpotQA}; \synatt (1M) & 84.3 & \textbf{96.9} & 90.2 & 54.4 & 58.0 & 56.1 & 63.4 & 92.4 & 75.2 & 52.5 & 77.5 & 62.6 \\ \midrule
Llama | 1B & \synatt (1P) & 89.8 & 96.5 & 93.0 & 50.6 & 58.6 & 54.3 & 64.9 & 91.5 & 75.9 & 53.1 & 75.5 & 62.3 \\
Llama | 1B & \synatt (1M) & 84.3 & \textbf{96.9} & 90.2 & 54.4 & 58.0 & 56.1 & 63.4 & 92.4 & 75.2 & 52.5 & 77.5 & 62.6 \\ \midrule
\textbf{\textit{Ours}} \\
Llama | 1B & \syntheticcolor{\synqa} & \textbf{96.0} & 96.2 & \textbf{96.1} & \textbf{89.6} & \textbf{69.4} & \textbf{78.2} & \textbf{93.3} & \textbf{89.1} & \textbf{91.1} & \underline{82.3} & 68.5 & \underline{74.8} \\
\bottomrule
\end{tabular}
}
\caption{Comparison of zero-shot models and those trained with synthetic data. Larger zero-shot LMs excel, but our \synqa model outperforms all but one for one dataset while being smaller. \textbf{Bold} denotes best method, \underline{underline} if our method is second best. 1P: models trained with a single pass over the training data. 1M: models trained with 1M samples to match the size of the \synqa data.}
\label{table:zero-shot-models}
\end{table*}

In Table~\ref{table:zero-shot-models}, we present the performance of zero-shot models, and models trained without gold context-attribution data. 
%\footnote{Note that the \synatt baseline models are trained using question-answer pairs from SQuaAD and HotpotQA, however, the context-attribution annotations are obtained using an ensemble of LLMs.}. 
State-of-the-art sentence-encoder models (e.g., E5) perform relatively poorly, consistent with prior findings \cite{CohenWang2024ContextCiteAM}. In contrast, LLMs exhibit strong performance, with improvements correlating with model size. Ensembling multiple zero-shot LLMs further enhances performance, leveraging complementary strengths across models, but making the attribution more expensive. We also tested models trained with the discriminative method \synatt. These models significantly outperform their non-fine-tuned counterparts of the same size. However, as postulated, our generative approach \synqa outperforms \synatt significantly in all but one case. Additionally, \synqa surpasses zero-shot LLMs that are orders of magnitude larger, showing that we can train a model that is both more accurate and efficient.

\subsubsection{Comparison to Models Trained on Gold Attribution Data}\label{sec:experiments-gold}

\begin{table*}[t]
\centering
\resizebox{1.0\textwidth}{!}{
\begin{tabular}{lccccccccccccccc} \toprule
\multirow{3}{*}{Model} & \multirow{3}{*}{Training data} 

& \multicolumn{6}{c}{\textbf{In-Domain}} 
& \multicolumn{6}{c}{\textbf{Out-of-Domain}} \\ \cmidrule(lr){3-8} \cmidrule(lr){9-14}

& & \multicolumn{3}{c}{\squadcolor{SQuAD}} & \multicolumn{3}{c}{\hotpotcolor{HotpotQA}} 
& \multicolumn{3}{c}{\quaccolor{QuAC-ST}} & \multicolumn{3}{c}{\coqacolor{CoQA-ST}} \\ \cmidrule(lr){3-5} \cmidrule(lr){6-8} \cmidrule(lr){9-11} \cmidrule(lr){12-14}

& & P & R & F1 & P & R & F1 & P & R & F1 & P & R & F1 \\ \midrule
\textbf{\textit{Baselines}} \\
Llama | 1B & Zero-shot & 37.5 & 62.0 & 46.7 & 5.3 & 28.1 & 8.9 & 8.8 & 65.4 & 15.4 & 11.9 & 52.8 & 19.4 \\
Llama | 1B & \squadcolor{SQuAD} (1P) & 98.4 & 98.4 & 98.4 & 48.7 & 20.0 & 28.4 & 92.6 & 85.8 & 89.0 & 79.9 & 64.3 & 71.2 \\
Llama | 1B & \hotpotcolor{HotpotQA} (1P) & 41.3 & 87.3 & 56.0 & 87.5 & 79.9 & 83.5 & 45.2 & 89.9 & 60.1 & 41.0 & 70.9 & 52.0 \\
Llama | 1B & \squadcolor{SQuAD} \& \hotpotcolor{HotpotQA} (1P) & 98.3 & 98.3 & 98.3 & \textbf{89.7} & 78.9 & 84.0 & 90.4 & 90.0 & 90.2 & 83.1 & 68.0 & 74.8 \\
Llama | 1B & \squadcolor{SQuAD} \& \hotpotcolor{HotpotQA} (1M) & \textbf{98.3} & \textbf{98.4} & \textbf{98.3} & 87.0 & \textbf{85.2} & \textbf{86.1} & 84.0 & 89.2 & 86.6 & 79.2 & 66.4 & 72.2 \\ \midrule
\textbf{\textit{Ours}} \\
Llama | 1B & \syntheticcolor{\synqa} & 96.0 & 96.2 & 96.1 & \underline{89.6} & 69.4 & 78.2 & \underline{93.3} & 89.1 & \underline{91.1} & 82.3 & 68.5 & \underline{74.8} \\
Llama | 1B & \syntheticcolor{\synqa} \& \squadcolor{SQuAD} \& \hotpotcolor{HotpotQA} & \underline{98.2} & \underline{98.3} & \underline{98.2} & 89.3 & \underline{82.4} & \underline{85.8} & \textbf{94.5} & \textbf{92.7} & \textbf{93.6} & \textbf{85.5} & \textbf{71.0} & \textbf{77.6} \\
\bottomrule
\end{tabular}
}
\caption{Comparison of models fine-tuned on synthetic vs.~gold in-domain data. Our \synqa approach generalizes better while remaining competitive in-domain. \textbf{Bold} denotes best method, \underline{underline} our method when second best. 1P: models trained with a single pass over the training data. 1M: models trained with 1M samples to match the size of the \synqa data.}
\label{table:fine-tuned-models}
\end{table*}

In Table~\ref{table:fine-tuned-models}, we compare models trained on synthetic and gold in-domain context-attribution datasets. As expected, fine-tuning on in-domain gold datasets (SQuAD and HotpotQA) yields highly specialized models that perform well on in-domain data.
% The performance on the out-of-domain datasets is comparable to Llama 70B, the best zero-shot LLM.
% In contrast, \synqa models outperform Llama 70B on out-of-domain datasets while also achieving near identical scores on the in-domain datasets.
However, models trained on data obtained by \synqa exhibit competitive performance on in-domain tasks and consistently surpass in-domain-trained models on out-of-domain datasets. 
This strong out-of-domain generalization is crucial for practical deployments, where models must handle diverse, previously unseen contexts that often differ substantially from their training data.

\subsubsection{Comparison to Zero-Shot and Fine-Tuned Models in Dialogue Contexts}\label{sec:experiments-dialog}
% \begin{table}[t]
% \centering
% \resizebox{1.0\columnwidth}{!}{
% \begin{tabular}{lccccccc} 
% \toprule

% \multirow{2}{*}{Model} & \multirow{2}{*}{Training data} & \multicolumn{3}{c}{\quaccolor{QuAC}} & \multicolumn{3}{c}{\coqacolor{CoQA}} \\ 
% \cmidrule(lr){3-5} \cmidrule(lr){6-8}

%  &  & P & R & F1 & P & R & F1 \\ 
% \midrule
% \textbf{\textit{Baselines}} \\
% Llama | 1B & Zero-shot & 20.9 & 47.9 & 29.1 & 35.6 & 40.2 & 37.8 \\
% Mistral | 7B & Zero-shot & 64.9 & 83.9 & 73.2 & 54.4 & 64.9 & 59.2 \\
% Llama | 8B & Zero-shot & 81.4 & 89.0 & 85.0 & 77.8 & 72.1 & 74.8 \\
% Mistral NeMo | 12B & Zero-shot & 84.8 & 85.4 & 85.1 & 81.7 & 68.4 & 74.5 \\
% \midrule
% Llama | 1B & \squadcolor{SQuAD} \& \hotpotcolor{HotpotQA} (1P) & 72.9 & 68.0 & 70.3 & 79.3 & 64.4 & 71.0 \\
% Llama | 1B & \squadcolor{SQuAD} \& \hotpotcolor{HotpotQA} (1M) & 56.0 & 49.0 & 52.3 & 63.0 & 51.2 & 56.5 \\ 
% \midrule
% \textbf{\textit{Ours}} \\
% Llama | 1B & \syntheticcolor{\synqa} & \textbf{91.3} & \underline{91.4} & \underline{91.3} & \underline{81.7} & \underline{71.4} & \underline{76.2} \\
% Llama | 1B & \syntheticcolor{\synqa} \& \squadcolor{SQuAD} \& \hotpotcolor{HotpotQA} & \underline{91.1} & \textbf{92.3} & \textbf{91.7} & \textbf{82.3} & \textbf{73.2} & \textbf{77.5} \\
% \bottomrule
% \end{tabular}
% }
% \caption{Context attribution on QuAC and CoQA (dialog data); both datasets are out-of-domain. Despite the size advantage of zero-shot LLMs, our \synqa models outperform fine-tuned and larger zero-shot models. \textbf{Bold} denotes best method, \underline{underline} our method when second best.}
% \label{table:dialog-datasets}
% \end{table}

\begin{table*}[t]
\centering
\resizebox{1.0\textwidth}{!}{
\begin{tabular}{lcccccccccccccc} 
\toprule

\multirow{3}{*}{Model} & \multirow{3}{*}{Training data} 

& \multicolumn{12}{c}{\textbf{Out-of-Domain}} \\ \cmidrule(lr){3-14}

& & \multicolumn{3}{c}{\quaccolor{QuAC}} & \multicolumn{3}{c}{\coqacolor{CoQA}} 
& \multicolumn{3}{c}{\orquaccolor{OR-QuAC}} & \multicolumn{3}{c}{\doqacolor{DoQA}} \\ 

\cmidrule(lr){3-5} \cmidrule(lr){6-8} \cmidrule(lr){9-11} \cmidrule(lr){12-14}

 &  & P & R & F1 & P & R & F1 & P & R & F1 & P & R & F1 \\ 
\midrule
\textbf{\textit{Baselines}} \\
Llama | 1B & Zero-shot & 30.8 & 45.5 & 36.8 & 39.4 & 37.9 & 38.6 & 33.0 & 46.6 & 38.6 & 12.2 & 22.6 & 15.9 \\
Mistral | 7B & Zero-shot & 76.6 & 81.8 & 79.1 & 67.6 & 61.3 & 64.3 & 82.5 & 85.1 & 83.8 & 74.9 & 77.9 & 76.4 \\
Llama | 8B & Zero-shot & 84.7 & 88.8 & 86.7 & 79.3 & 72.0 & 75.5 & 88.0 & 91.3 & 89.6 & 77.9 & 91.4 & 84.1 \\
Mistral-NeMo | 12B & Zero-shot & 85.7 & 85.4 & 85.5 & 81.9 & 68.4 & 74.5 & 88.9 & 88.8 & 88.8 & 86.0 & 84.2 & 85.1 \\
Llama | 70B & Zero-shot & 88.5 & 87.7 & 88.1 & \textbf{88.3} & \textbf{74.9} & \textbf{81.1} & 81.7 & 86.3 & 83.9 & 85.2 & 82.0 & 83.5 \\
\midrule
\textbf{\textit{Baselines}} \\
Llama | 1B & \squadcolor{SQuAD} \& \hotpotcolor{HotpotQA} (1P) & 71.3 & 66.8 & 69.0 & 79.0 & 64.2 & 70.8 & 61.6 & 57.5 & 59.5 & 67.4 & 57.8 & 62.2 \\
Llama | 1B & \squadcolor{SQuAD} \& \hotpotcolor{HotpotQA} (1M) & 52.6 & 49.3 & 50.9 & 61.2 & 50.2 & 55.2 & 48.5 & 44.6 & 46.5 & 53.2 & 49.1 & 51.1 \\ \midrule
\textbf{\textit{Ours}} \\
Llama | 1B & \syntheticcolor{\synqa} & \textbf{91.3} & \underline{91.4} & \underline{91.3} & 81.7 & 71.4 & 76.2 & \textbf{92.6} & \underline{95.3} & \textbf{94.0} & \textbf{86.3} & \underline{94.5} & \textbf{90.2} \\
Llama | 1B & \syntheticcolor{\synqa} \& \squadcolor{SQuAD} \& \hotpotcolor{HotpotQA} & \underline{91.1} & \textbf{92.2} & \textbf{91.7} & \underline{82.3} & \underline{73.2} & \underline{77.5} & \underline{90.3} & \textbf{96.4} & \underline{93.2} & \underline{85.1} & \textbf{96.0} & \textbf{90.2} \\
\bottomrule
\end{tabular}
}
\caption{Context attribution on QuAC, CoQA, OR-Quac, and DoQA (dialogue data); all datasets are out-of-domain. Despite the size advantage of zero-shot LLMs, our \synqa models outperform fine-tuned and larger zero-shot models. \textbf{Bold} denotes best method, \underline{underline} our method when second best. 1P: models trained with a single pass over the training data. 1M: models trained with 1M samples to match the size of the \synqa data.}
\label{table:dialog-datasets}
\end{table*}

We evaluate dialogue context attribution, for which we do not use any gold in-domain training data (Tab.~\ref{table:dialog-datasets}).
% exists.
Here, models must handle follow-up questions that rely on previous turns, often involving coreferences and other dialogue-specific complexities. As expected, zero-shot LLMs exhibit a strong size-performance correlation, with larger models consistently outperforming smaller ones—even those fine-tuned on single-turn question-answer attribution (trained on gold SQuAD and HotpotQA data). However, fine-tuning smaller models with our synthetic data generation strategy leads to superior performance, surpassing both their fine-tuned counterparts and much larger zero-shot LMs. This demonstrates the effectiveness of \synqa in enhancing context attribution in a dialogue setting and without requiring in-domain supervision.

\subsubsection{Scaling Trends and Generalization Performance}\label{sec:scalling-trends}

\begin{figure*}[t]
    \centering
    \begin{subfigure}{0.48\linewidth}
        \centering
        \includegraphics[width=\linewidth]{img/size_performance.pdf}
        \caption{Model performance vs.~size.}
        \label{fig:size_performance}
    \end{subfigure}
    \hfill
    \begin{subfigure}{0.48\linewidth}
        \centering
        \includegraphics[width=\linewidth]{img/data_quantity.pdf}
        \caption{F1 score vs.~training data size.}
        \label{fig:data_quantity}
    \end{subfigure}
    \caption{Comparison of model performance and scalability. (a) Larger zero-shot models achieve good F1 scores, but our method \synqa (based on Llama 1B) outperforms them while being orders of magnitude smaller. (b) Performance improves consistently with more \synqa training data, highlighting its scalability.}
    \label{fig:combined}
\end{figure*}

Fig.~\ref{fig:size_performance} shows F1 scores averaged across datasets, with model size on the x-axis and performance on the y-axis. Models trained on \synqa-generated data significantly outperform their baseline zero-shot counterparts, while also achieving superior performance compared to zero-shot LLMs that are orders of magnitude larger. This shows our method is highly data-efficient, enabling small models to close the gap with much larger counterparts.



In Figure~\ref{fig:data_quantity}, we analyze model performance as the quantity of synthetic training data increases, reporting F1 scores separately for in-domain and out-of-domain datasets. As we scale data quantity, performance improves consistently across datasets for isolated context attribution. This trend highlights the scalability of our approach, indicating that further gains can be achieved by increasing synthetic data availability.
%Notably, despite the lack of direct supervision on in-domain datasets, more data results in improved performance.%, reinforcing the robustness of our method.

\subsubsection{User Study: \synqa increases efficiency and accuracy assessment}\label{sec:user-study}
We conducted a user study to evaluate the efficiency and accuracy of verifying the correctness of LLM-generated answers using context attribution. Our hypothesis is that higher-quality context attributions, visualized to guide users, facilitate faster and more accurate verification of LLM outputs. Specifically, in each trial, we presented users with a question, a generated answer, and relevant context, along with
% context
attributions visualized as highlights. Their task was to leverage these attributions to judge if the answer was correct w.r.t.~a provided context. See Figure~\ref{fig:user_interface} in Appendix~\ref{app:user_study}.
% for an example. %Appendix~\ref{app:user_study} for an example.

The study compares three scenarios:
\begin{inparaenum}[(i)]
\item 
\textbf{No Alignment:} a baseline condition without context attributions, requiring users to manually read and verify the answer against the entire context;
\item 
\textbf{Llama 1B (Zero-shot):} context attributions generated by the Llama 1B model were visualized;
\item 
\textbf{\synqa}: context attributions generated by our approach were visualized.
\end{inparaenum}

We employed a within-subjects experimental design for our human evaluation (with 12 participants), ensuring that the same participants evaluate all the aforementioned alignment scenarios, thus requiring fewer participants for reliable results \cite{greenwald:1976}. However, this can be susceptible to learning effects where participants perform better in later scenarios, because they learned the task from previous examples. To mitigate this, we counterbalanced the scenario order using a Latin Square design \cite{belz:2010,bradley:1958}, where each alignment scenario appears in each position an equal number of times across all participants. Finally, we randomized the example order within each scenario per participant. For each example, we measured: \textbf{verification time} (seconds from display to judgment submission) and \textbf{verification accuracy} (binary correct/incorrect judgment).

\begin{figure}[hb!]
    \centering
    \includegraphics[width=1.0\linewidth]{img/user_study_big_font.png}
    \caption{Relationship between Evaluation Time (seconds) and Accuracy (\%) for three answer verification settings:  \emph{Llama 1B (Zero-shot)}, \emph{No Alignment} and \synqa. \synqa demonstrates the lowest evaluation time and highest accuracy, indicating its superior performance in facilitating efficient and accurate answer verification.}
    \label{fig:user_study}
\end{figure}

\noindent \textbf{Results.} We observed a clear trend in verification performance across the different attribution settings, with \synqa demonstrating superior effectiveness (Fig.~\ref{fig:user_interface}). \synqa has the lowest average verification time per example (\textbf{148.6} seconds), significantly faster than \emph{No Alignment} (171.8 seconds) and attributions from \emph{Llama 1B} (163.4 seconds). Concurrently, in terms of verification accuracy, \synqa achieved the highest average accuracy (\textbf{86.4\%}). While \emph{No Alignment} (84.1\%) and \emph{Llama 1B (77.3\%)} also yielded reasonable accuracy, attributions from \synqa are clearly of higher quality helping users be more accurate.


% \begin{table*}[t]
% \centering
% \resizebox{1.0\textwidth}{!}{
% \begin{tabular}{lccccccccccccccc} \toprule
% \multirow{2}{*}{Model} & \multirow{2}{*}{Training data} & \multicolumn{3}{c}{\squadcolor{Squad}} & \multicolumn{3}{c}{\hotpotcolor{HotPot QA}} & \multicolumn{3}{c}{\quaccolor{Quac}} & \multicolumn{3}{c}{\coqacolor{CoQA}} \\ \cmidrule(lr){3-5} \cmidrule(lr){6-8} \cmidrule(lr){9-11} \cmidrule(lr){12-14}
% & & P & R & F1 & P & R & F1 & P & R & F1 & P & R & F1 \\ \midrule
% Random & -- & 19.8 & 15.4 & 17.3 & 4.8 & 15.2 & 7.3 & 5.2 & 15.1 & 7.7 & 7.3 & 15.1 & 9.9 \\
% E5 | 561M & Zero-shot & 38.1 & 76.5 & 50.9 & 12.4 & 41.4 & 19.1 & 65.0 & 73.8 & 69.1 & 61.1 & 15.2 & 24.4 \\
% HF-SmolLM2 | 135M & Zero-shot & X & X & X & X & X & X & X & X & X & X & X & X \\
% HF-SmolLM2 | 365M & Zero-shot & 28.1 & 46.4 & 35.0 & 5.1 & 7.3 & 6.0 & 10.6 & 22.6 & 14.4 & 10.6 & 21.5 & 14.2 \\
% Llama | 1B & Zero-shot & 37.5 & 62.0 & 46.7 & 5.3 & 28.1 & 8.9 & 8.8 & 65.4 & 15.4 & 11.9 & 52.8 & 19.4 \\ %\midrule
% Mistral | 7B & Zero-shot & 71.5 & 94.4 & 81.4 & 42.9 & 42.7 & 42.8 & 63.2 & 88.6 & 73.8 & 59.0 & 72.2 & 64.9 \\
% Llama | 8B & Zero-shot & 71.9 & 96.9 & 82.6 & 49.2 & 52.9 & 51.0 & 64.1 & 92.1 & 75.6 & 55.7 & 76.4 & 64.4 \\
% Mistral NeMo | 12B & Zero-shot & 89.5 & 94.5 & 91.8 & 46.4 & 47.3 & 46.8 & 81.8 & 85.3 & 83.5 & 79.0 & 67.2 & 72.6 \\
% Ensemble | 27B & Zero-shot & 83.1 & 96.3 & 89.2 & 48.1 & 59.6 & 53.2 & 74.8 & 90.3 & 81.8 & 69.5 & 73.6 & 71.5 \\
% Llama | 70B & Zero-shot & 95.3 & 95.6 & 95.5 & 87.6 & 37.5 & 52.5 & 89.7 & 87.8 & 88.7 & 87.5 & 73.3 & 79.8 \\
% \midrule
% Llama | 1B & \squadcolor{SQ} \& \hotpotcolor{HP}; Disc. synthetic (1 pass) & 89.8 & 96.5 & 93.0 & 50.6 & 58.6 & 54.3 & 64.9 & 91.5 & 75.9 & 53.1 & 75.5 & 62.3 \\
% Llama | 1B & \squadcolor{SQ} \& \hotpotcolor{HP}; Disc. synthetic (1.0M) & 84.3 & 96.9 & 90.2 & 54.4 & 58.0 & 56.1 & 63.4 & 92.4 & 75.2 & 52.5 & 77.5 & 62.6 \\ \midrule
% Llama | 1B & \synqa (130K no dist.) & 87.1 & 88.2 & 87.6 & 67.9 & 44.8 & 54.0 & 85.3 & 82.5 & 83.9 & 72.7 & 63.8 & 68.0 \\
% Llama | 1B & \syntheticcolor{\synqa (130K)} & 89.9 & 90.7 & 90.3 & 87.9 & 63.5 & 73.7 & 88.7 & 85.4 & 87.0 & 77.2 & 65.5 & 70.9 \\
% Llama | 1B & \syntheticcolor{\synqa (550K)} & 93.6 & 94.5 & 94.0 & 88.9 & 67.3 & 76.6 & 89.8 & 87.9 & 88.9 & 77.1 & 67.1 & 71.8 \\
% Llama | 1B & \syntheticcolor{\synqa (700K)} & 95.1 & 95.5 & 95.3 & 87.8 & 69.6 & 77.6 & 93.6 & 89.1 & 91.3 & 82.0 & 68.9 & 74.9 \\
% Llama | 1B & \syntheticcolor{\synqa (1.0M)} & 96.0 & 96.2 & 96.1 & 89.6 & 69.4 & 78.2 & 93.3 & 89.1 & 91.1 & 82.3 & 68.5 & 74.8 \\
% New & \syntheticcolor{\synqa (1.0M)} & 95.7 & 96.9 & 96.3 & 89.6 & 66.4 & 76.3 & 91.8 & 91.5 & 91.6 & 80.8 & 71.3 & 75.8 \\
% Llama | 1B & \syntheticcolor{\synqa (1.0M with dialog data)} & -- & -- & -- & -- & -- & -- & -- & -- & -- & -- & -- & -- & \\
% \midrule
% HF-SmolLM2 | 135M & \synqa (690K) & X & X & X & X & X & X & X & X & X & X & X & X \\
% HF-SmolLM2 | 365M & \synqa (700K) & 73.9 & 74.2 & 74.1 & 85.2 & 68.7 & 76.0 & 79.6 & 76.5 & 78.0 & 68.8 & 59.8 & 64.0 \\ \midrule
% Llama | 1B & \squadcolor{SQ}; Gold (1 pass) & 98.4 & 98.4 & 98.4 & 48.7 & 20.0 & 28.4 & 92.6 & 85.8 & 89.0 & 79.9 & 64.3 & 71.2 \\
% Llama | 1B & HQ; Gold (1 pass) & 41.3 & 87.3 & 56.0 & 87.5 & 79.9 & 83.5 & 45.2 & 89.9 & 60.1 & 41.0 & 70.9 & 52.0 \\
% Llama | 1B & \squadcolor{SQ} \& \hotpotcolor{HQ}; Gold (1 pass) & 98.3 & 98.3 & 98.3 & 89.7 & 78.9 & 84.0 & 90.4 & 90.0 & 90.2 & 83.1 & 68.0 & 74.8 \\
% Llama | 1B & \squadcolor{SQ} \& \hotpotcolor{HQ}; Gold (1.0M) & 98.3 & 98.4 & 98.3 & 87.0 & 85.2 & 86.1 & 84.0 & 89.2 & 86.6 & 79.2 & 66.4 & 72.2 \\
% Llama | 1B & \syntheticcolor{\synqa (1.0M)} \& \squadcolor{SQ} \& \hotpotcolor{HQ}; Gold (1 pass) & 98.3 & 98.3 & 98.3 & 86.4 & 84.1 & 85.2 & 96.6 & 90.2 & 93.3 & 86.0 & 69.4 & 76.8 \\
% Llama | 1B & \syntheticcolor{\synqa (1.0M)} \& \squadcolor{SQ} \& \hotpotcolor{HQ}; Gold (1 pass) & 98.2 & 98.3 & 98.2 & 89.3 & 82.4 & 85.8 & 94.5 & 92.7 & 93.6 & 85.5 & 71.0 & 77.6 \\
% \midrule
% Llama | 1B & \synqa (1.0M) \& \squadcolor{SQ} \& \hotpotcolor{HQ} \& \quaccolor{Q} \& \coqacolor{CQ}; Gold (1 pass) & -- & -- & -- & -- & -- & -- & -- & -- & -- & -- & -- & -- & \\ \midrule
% HF-Smol | 365M & \syntheticcolor{\synqa (1.0M)} \& \squadcolor{SQ} \& \hotpotcolor{HQ}; Gold (1 pass) & 98.1 & 98.2 & 98.2 & 83.4 & 83.2 & 83.3 & 80.0 & 90.5 & 84.9 & 71.7 & 67.4 & 69.5 \\
% HF-Smol | 365M & \syntheticcolor{\synqa (1.0M)} \& \squadcolor{SQ} \& \hotpotcolor{HQ} \& \quaccolor{Q} \& \coqacolor{CQ}; Gold (1 pass) & 98.0 & 98.1 & 98.1 & 86.8 & 81.7 & 84.2 & 96.6 & 92.9 & 94.7 & 85.6 & 77.0 & 81.1 \\
% HF-Smol | 135M & \syntheticcolor{\synqa (1.0M)} \& \squadcolor{SQ} \& \hotpotcolor{HQ} \& \quaccolor{Q} \& \coqacolor{CQ}; Gold (1 pass) & 98.0 & 98.0 & 98.0 & 77.8 & 76.3 & 77.0 & 95.0 & 91.6 & 93.3 & 81.2 & 71.8 & 76.2 \\
% \bottomrule
% Llama | 1B & All; Gold & 1B   & 96.8 & 96.8 & 96.8 & 88.1 & 83.8 & 85.9 & 94.7 & 89.3 & 91.9 & 88.7 & 76.8 & 82.3 \\
% HF-SmolLM2 & All; Gold & 360M & 98.3 & 98.4 & 98.3 & 85.1 & 78.2 & 81.5 & 96.6 & 92.4 & 94.5 & 88.3 & 74.6 & 80.8 \\ \bottomrule
% \end{tabular}
% }
% \caption{Corroborative context-attribution of LMs on Squad QA, HotPot QA, Quac, and CoQA.}
% \label{table:all-datasets}
% \end{table*}
\section{Related Work}
%\textcolor{red}{TODO: Go over the section (ContextCite in particular); Position more precisely w.r.t. related work.}

%We split the related work papers into three parts: (1) context attribution task; (2) in-line citation generation; and (3) post-hoc context attribution.

%\subsection{Context Attribution Task}

%\paragraph{AIS Task.} \citet{rashkin2023measuring} formally proposed the general task of \emph{Attributable to Identified Sources (AIS)}, which aims at answering the following query: given a generated text $t_g$ and a context text $t_c$, is $t_g$ attributable to $t_c$? Broadly speaking, the AIS task can be applied on every NLP task that entails generated text from a GenAI model, including text summarization \cite{ernst2021summary}, table-to-text generation \cite{Parikh2020ToTToAC} and QA \cite{Nakano2021WebGPTBQ}. In our work, we focus on the case of context attribution for QA.

%\paragraph{Granularity.} Many prior methods focus on context attribution on either a document or paragraph level: when an attributive text from the context $t_c$ is assigned to a generated text $t_g$,  then $t_c$ is typically consisted of either whole paragraphs \cite{Menick2022TeachingLM,Yue2023AutomaticEO,Li2024AttributionBenchHH} or whole documents that are either previously retrieved \cite{rashkin2023measuring,Huang2024AdvancingLL} or retrieving the relevant documents from a large document collection is part of the task \cite{Gao2023RARRRA,Buchmann2024AttributeOA}. Such coarse-grained attributions pose difficulties for end-users, because it is harder for them to use the coarse-grained text for further manual fact verification \cite{slobodkin2024attribute}. Other line of work have focused on more fine-grained granularity such as phrases \cite{CohenWang2024ContextCiteAM} or tokens \cite{Phukan2024PeeringIT}. This is also problematic, because these methods are typically too slow for real applications. For these reasons, we focus on sentence-level granularity, as this allows both the context attribution methods to be fast enough and usable by humans for manual fact verification of the generated text. %Instead, users prefer the granularity level to be on sentence level \cite{slobodkin2024attribute}. 


%\paragraph{Context Attribution Categories.} There are two major categories of the context attribution task for QA: (1) in-line citation generation: the LLM is instructed to generate citations along with the generated answers \cite{bohnet2022attributed,gao2023enabling,Huang2024AdvancingLL,slobodkin2024attribute}; (2) post-hoc attribution: the models classify whether a given piece of text is attributable to the answer of the question \cite{Yang2018HotpotQAAD,CohenWang2024ContextCiteAM,Nakano2021WebGPTBQ}. In this work, we focus on post-hoc context attribution for QA.


%In contrast to the original AIS work, Our work differentiates in three important aspects: (1) we focus on a broader QA setup (i.e., single-question QA and conversational QA), which makes our work a subset of the broader AIS task; (2) we focus on more fine-grained level: our attributions are not on the entire text level, but rather on a sentence level, which has been shown in user studies to be more useful to end-users \cite{slobodkin2024attribute}; (3) the AIS task entails a manual evaluation framework, while our work provides automatic evaluation with golden data.

We split the related work on context attribution for QA into two categories: (1) in-line citation generation: LLMs are instructed to generate citations along with the generated answer; (2) post-hoc context attribution: perform the attribution \emph{after} the LLM generates the answer. In this section, we outline these works and their differences from our work. For more comprehensive discussion on related work, see Appendix~\ref{app:rel_work}.

\subsection{In-line Citation Generation}

In this setup, researchers use LLMs to produce in-line citations along with the generated text \cite{bohnet2022attributed,gao2023enabling,Huang2024AdvancingLL}. This typically works on paragraph or document level. One line of work focuses on fine-tuning methods for tackling the problem \cite{gao2023enabling,Schimanski2024TowardsFA,Berchansky2024CoTARCA,patel2024towards}, while another line of work proposes synthetic data generation methods for fine-tuning such models \cite{Huang2024LearningFG,Huang2024AdvancingLL}. 
\citet{slobodkin2024attribute} propose a fine-grained task, where the attributions are on sentence level, because such granularity is more useful to human end users. Since generating such in-line citations can result in producing completely made up citations, \citet{Yue2023AutomaticEO} propose a task that checks whether in-line generated citations from LLMs are actually attributable or not. %Instead of using binary attributable/non-attributable labels (like with AIS), they propose more fine-grained labels for this problem: attributable, extrapolatory, contradictory and non-attributable. 
Unlike such approaches, we focus on post-hoc context attributions, because this directly predicts a link to a factual source, and therefore avoiding the risk of making up the source.%: given an answer to a question, find the sentences in the context that support the factuality of the answer.

\subsection{Post-hoc Context Attribution}

In post-hoc context attribution, the aim is to determine which parts of the context are attributable to an already answered question \cite{Yang2018HotpotQAAD}. There has been a significant amount of work on training models for the context attribution problem on sentence level for multi-hop QA \cite{zhang2024end, ho2023analyzing,yin2023rethinking,fu2021decomposing,tu2020select,fang-etal-2020-hierarchical}. However, they do not investigate this problem in the context of LLMs. Moreover, the methods are constrained \emph{only} to multi-hop QA, and are not tested on broader QA context, such as on dialogue QA. In our work, we propose methods that use LLMs as data generators. This allows us to better generalize and cover multiple QA settings simultaneously, therefore better matching real-world needs. %generalize across multiple QA settings, such as multi-hop QA and conversational QA.

Another line of work focuses on coarse-level granularity and provide attributions either on paragraph level \cite{rashkin2023measuring,Menick2022TeachingLM} or document level \cite{Nakano2021WebGPTBQ,Gao2023RARRRA,Buchmann2024AttributeOA}. However, in a user study \citet{slobodkin2024attribute} observe that such granularity level is not optimal for humans when manually fact-checking LLM-generated content. Their experiments suggest that sentence-level granularity is ideal for humans. This is why we adopt sentence-level granularity in our work, despite this being a harder task. On the other hand, there has been work that focuses on the other extreme: assigning context attributions on sub-sentence level \cite{CohenWang2024ContextCiteAM,Phukan2024PeeringIT}. Such methods are computationally expensive and this hinders their practical usability. Our work ensures that models can be run in real-time to make them practical for end users.
%\footnote{For more detailed discussion on related work, see App.~\ref{app:rel_work}.} 

%ContextCite \cite{CohenWang2024ContextCiteAM} focus on fine-grained post-hoc attribution method that is on a phrase level.

%\subsection{Context Attribution Tasks} 

%\paragraph{Attributable to Identified Sources (AIS):} given a generative text $t_g$ and a context text $t_c$, is $t_g$ attributable to $t_c$? \citet{rashkin2023measuring} propose a manual framework that defines the AIS task and evaluates the AIS scores across several NLP tasks, namely conversational QA \cite{Anantha2020OpenDomainQA,Dinan2018WizardOW}, text summarization \cite{Nallapati2016AbstractiveTS} and table-to-text \cite{Parikh2020ToTToAC}. Our work differentiates in three important aspects: (1) we focus on a broader QA setup (i.e., single-question QA and conversational QA), which makes our work a subset of the broader AIS task; (2) we focus on more fine-grained level: our attributions are not on the entire text level, but rather on a sentence level, which has been shown in user studies to be more useful to end-users \cite{slobodkin2024attribute}; (3) the AIS task entails a manual evaluation framework, while our work provides automatic evaluation with golden data.

%This can be done either post-hoc \cite{Ramu2024EnhancingPA} or in contributive (i.e., causal) manner \cite{CohenWang2024ContextCiteAM}. ClaimVer \cite{Dammu2024ClaimVerEC} does attribution posthoc, but the mapping is done between the generated text and KG triples. The system outputs rationale explanations, attribution scores, relevant KG triples and generates rationales.

%\paragraph{In-line Citation Generation:} %Self-citation methods (or attributed generation, or in-line citation generation) uses LLMs to produce in-line citations along with the generated text \cite{bohnet2022attributed,gao2023enabling,Huang2024AdvancingLL}. This typically works on paragraph or document level. %\citet{gao2023enabling} proposed a fine-tuning method for tackling the problem, while \citet{Huang2024AdvancingLL} propose a synthetic data generation for fine-tuning such models. \citet{slobodkin2024attribute} propose a fine-grained task, where the attributions are on sentence level, because such granularity is more useful to human end users. Because generating such in-line citations can result in producing completely made up citations, \citet{Yue2023AutomaticEO} propose a task that checks whether the in-line generated citations from LLMs are actually attributable or not. Instead of using binary attributable/non-attributable labels (like with AIS), they propose more fine-grained labels for this problem: attributable, extrapolatory, contradictory and non-attributable. Contrary to such approaches, our work focuses on post-hoc context attributions: given an answer to a question, find the sentences in the context that support the factuality of the answer.

%\citet{Yue2023AutomaticEO} propose a task that checks whether in-line generated citations from LLMs are attributable or not. 

%\paragraph{Post-hoc Attribution:} determines which parts of the context are attributable to an already answered question \cite{Yang2018HotpotQAAD}. Within the post-hoc context, there are two other subcategorizations of the task: \textit{contributive} and \textit{corroborative post-hoc attribution} \cite{CohenWang2024ContextCiteAM}.

%\paragraph{Post-hoc Attribution (Contributive):} ContextCite \cite{CohenWang2024ContextCiteAM} and Mirage \cite{Qi2024ModelIA} define a post-hoc task that aims at discovering which parts of the context \textit{caused} the LLM to generate the particular response. Their evaluation methods, however, are based on  proxy metrics that do not rely on golden annotations, while in our work we rely on automatic annotations that rely on golden data. 

%\paragraph{Post-hoc Attribution (Corroborative):} this task is similar to contributive post-hoc attribution. The difference is that the constraint for causality is not necessarily enforced, but should support the factuality of the statement \cite{CohenWang2024ContextCiteAM}. Many works are based on coarse-grained level and provide attributions on either paragraph level \cite{Menick2022TeachingLM}, document level \cite{Nakano2021WebGPTBQ} or on multi-document level, where they have a RAG component that retrieves the documents that are potentially attributable \cite{Gao2023RARRRA,Buchmann2024AttributeOA}.

%ContextCite \cite{CohenWang2024ContextCiteAM} define the task of \textit{contributive attribution}, which aims at discovering what pieces of text \textit{caused} the LLM to generate the particular response. This is different from our approach, where we try to find the attributions that support or imply a statement (i.e., post-hoc attribution). In Mirage \cite{Qi2024ModelIA}, the authors evaluate the answer attribution w.r.t. the RAG. They too, claim to be solving the causal part of the problem.

%\citet{Buchmann2024AttributeOA} focus on long-form attribution, which covers multiple tasks that use long-form context (e.g., QA and text summarization). They do not tackle fine-grained attribution (e.g., sentence level). They also tackle the issue of non-attributable cases. This paper focuses on retrieval case, where the evidence needs to be automatically retrieved. 

%\paragraph{Context Attributions to other Modalities.} Other line of work maps the attributions to other modalities, such as knowledge graphs \citet{Dammu2024ClaimVerEC}. Similarly, \citet{Maheshwari2024PresentationsAN} take multi document collection as input, construct a graph of narratives, and then generate a presentation (i.e., slides) for the topic, along with attributions from the generated content of the slides with the original documents. We do not investigate such cases, and focus on attributing answers to sentences within the user-provided context.

%\paragraph{Post-hoc Attribution for Text Summarization.} \citet{ernst2021summary} proposed a task, dataset and baseline model (dubbed SuperPAL) for detecting attributions for text summarization. In a followup work, \citet{ernst2022proposition} an extension of the task, this time for clustering propositions for text summarization and \citet{ernst2024power} extend this to multi-document summarization. \citet{krishna2023longeval} investigate whether such text summarization alignments are helpful for humans. In our work, we focus on the question answering (QA) task.


%\subsection{Datasets}

%\paragraph{AIS.} \citet{rashkin2023measuring} proposed the AIS dataset, which contains three tasks: question answering, table-to-text and text summarization. Here, for each data point, there is a query and an LLM-generated response, along with label by humans whether it is fully attributable or not. This data is on paragraph and document level, and lacks the granularity of a sentence level. Therefore, we do not use it in our work.

%\paragraph{HotpotQA.} With HotpotQA \cite{Yang2018HotpotQAAD}, the authors propose an explainable multi-hop QA dataset. The dataset also contains attribution links (i.e., explanations) for the answers: spans of text that belong to the input context, which are supporting the statement in the answer. The authors set up baselines for measuring the ability of attributions of models on sentence level, which is in line with what we do. In our work, we integrated HotpotQA as part of our setup for both training and testing.

%\paragraph{AttributionBench.} This is a benchmark for attribution evaluation of LLM generated content \cite{Li2024AttributionBenchHH}. In particular, the benchmark assesses whether the assigned attribution on a generated text is actually attributable. In particular, given a query, response set $\mathcal{R}$ (containing claims) and evidence set $E$, the task is to label as "attributable" or "not attributable" every claim against $E$. This work operates on a coarse-grained level (paragraphs or whole documents). Similarly, \citet{Yue2023AutomaticEO} proposed another dataset for evaluating attribution of LLM-generated text, same on paragraph level. In our work, we focus on sentence-level context attribution. Therefore, we did not include this dataset in our work.

%\paragraph{Conversational QA.} CoQA \cite{Reddy2018CoQAAC} is a conversational QA dataset, which contains a context, questions-answer pairs between two people (teacher and student) and sentence-level supporting evidence for the context. We use this dataset in our evaluation to test the out-of-domain capabilities of LLMs for context attribution. We also use QuAC \cite{Choi2018QuACQA} and ORConvQA \cite{qu2020open}, which are conversational QA datasets similar to CoQA. 

%\paragraph{QASPER.} This dataset is from the scientific domain \cite{dasigi2021dataset}. The dataset contains title and an abstract of a paper, question and answer about the content. This data has information on paragraph level, not on sentence level. Therefore, we do not use it in our experiments.

%\paragraph{WikiQA.} \citet{Dammu2024ClaimVerEC} use WikiQA \cite{yang2015wikiqa}, because it's Wikipedia-based dataset, which can be linked to Wikipedia-derived KG like Wikidata \cite{vrandevcic2014wikidata}. In our work, we focus only on text modality, which is why we do not include this dataset into our evaluation.


% IMPORTANT: we didn't include this data due to the lack of time. Could be interesting to see. \citet{Liu2023EvaluatingVI} evaluate several generative search engines on their ability to correctly attribute sources to the generated content. This is a manual evaluation and is, therefore, not scalable. \textbf{They released a dataset with source attribution on sentence level, which we can reuse.} Unlike question answering, this dataset contains queries with generative search engine responses. This dataset is also used by \citet{Ramu2024EnhancingPA} and \citet{slobodkin2024attribute}.





%\subsection{Metrics and Evaluation} 

%\paragraph{AIS.} The AIS framework \cite{rashkin2023measuring} is human annotation framework. Given a generated text chunk and a context chunk (this can be sentence, paragraph or document), a human evaluated whether the generated text chunk is fully attributible or not. It is basically a binary classification problem. In their data, the authors focus on document level granularity, which is not useful for humans. In our setup, we check for each sentence in the context if it supports the answer.

%\paragraph{AutoAIS.} To evaluate the attributed information, \citet{slobodkin2024attribute} use \textbf{AutoAIS metric}: an NLI-based scoring. Prior studies have shown that this metric highly correlated with human annotations \cite{bohnet2022attributed,gao2023enabling}. This is an extension to the AIS metric. We do not use proxy metrics, but rely on golden annotations by humans.

%\paragraph{AttrScore.} With AttrScore, \citet{Yue2023AutomaticEO} consider LLM generated content on the one hand and citation documents on the other hand. Then, the score evaluates whether a provided citation is attributable, extrapolatory, contradictory or non-attributable. Essentially, it is an extension of AIS, such that it provides more fine grained labels for the provided citations. The AttrScore is basically a fine-tuned LLM that provides these scores.

%\paragraph{Unsupervised Metrics.} ContextCite proposed the Top-k-drop and LDS metric to evaluate the causal post-hoc attribution. These metrics do not require labeled data. \citet{Berchansky2024CoTARCA} uses ROUGE and BERTScore to evaluate their results. We do not use unsupervised metrics and rely on automated evaluation with golden annotations by humans.

%\subsection{Methods} 

%\paragraph{Multihop QA.} There has been significant amount of work on tackling the context attribution problem on sentence level for multi-hop QA \cite{zhang2024end, ho2023analyzing,yin2023rethinking,fu2021decomposing,tu2020select,fang-etal-2020-hierarchical}. While we also investigate this problem, in contrast to our work, these works focus \emph{only} on the multihop QA task. In our work, we also explore other QA setups, including conversational QA with different domains. Moreover, these papers do not investigate the capabilities of LLMs about the context attribution problem, but rather are proposing specific methods that are tailor made for the multihop QA problem, which involves both answering the questions and providing supporting sentences to the answers. %\citet{zhang2024end} proposed an end-to-end beam retrieval method that solves the problem of both answering the questions and assigning attribution.  \citet{ho2023analyzing} built a model based on BigBird \cite{zaheer2020big} 

%\paragraph{In-line Citation Generation.} Another line of work focuses on guiding LLMs to generate in-line citations along with the generated text \cite{Li2023ASO}. \citet{slobodkin2024attribute} tackle this problem on a sentence level, but do not investigate the post-hoc context attribution case. Moreover, they rely on proxy metrics such as AutoAIS \cite{Gao2023RARRRA} and BERTScore \cite{zhang2020bertscore}. Similarly, \citet{bohnet2022attributed} proposes methods for in-line citation generation, but this work is more coarce grained and focuses on paragraph and document level. They also report their findings on proxy metrics. Their method is based on retrieval and they do not investigate the LLMs capabilities thoroughly. \citet{gao2023enabling} assign citations to LLM generated content, where they retrieve the information from a large collection of documents (also, it's on paragraph and document level, not on sentence level). START \cite{Huang2024AdvancingLL} propose a data synthetic generation method for in-line citation generation on document level, where each citation refers to an entire document. FRONT \cite{Huang2024LearningFG} also investigates synthetic data generation of in-line citation generation, where the citations assigned to the sentences in the output are entire documents. Similarly, \citet{Schimanski2024TowardsFA} propose a synthetic data generation pipeline for fine tuning models that solve the same problem. \citet{Berchansky2024CoTARCA} use Chain-of-Thought approaches and fine-tuning smaller LLMs in order to solve this problem. \citet{patel2024towards} also fine-tune a model specifically for this task, and the attributions are on paragraph level.

%\paragraph{Post-hoc Context Attribution.} \citet{Ramu2024EnhancingPA} propose template-based in-context learning method for post-hoc context attribution. In particular, they use standard retrievers as a first step to pre-rank the text (e.g., BM25 and dual encoders \cite{ni2022large}) and then they use LLMs to calssify (i.e., rerank) the relevant sentences. ContextCite \cite{CohenWang2024ContextCiteAM} uses ablation-based methods to infer the attributions of post-hoc generated text. 


%\citet{slobodkin2024attribute} focus on locally attributed text generation: a task that aims at attributing information to the LLM text generation on the fly. In particular, their method is the following: the model first focuses on the right parts of the context (attribution) and then it generates the answer to the query.

%Prior work \cite{bohnet2022attributed,gao2023enabling} was attributing information to whole documents or paragraphs, which makes the task not useful for humans. In contrast, \citet{slobodkin2024attribute} focuses on sentence-level granularity (similar to our approach) and subsentence-level granularity. They focus on two tasks: Multi Document Summarization (MDS) and Long Form Question Answering (LFQA). In contrast, in our approach, we focus on (1) post-hoc attributions of the generated text; (2) multiple variants of question answering (i.e., multi-hop QA and dialogue Q\&A). For data, they used a modified version of \cite{Liu2023EvaluatingVI}. For the in-line citation generation, \citet{Huang2024AdvancingLL} proposed a synthetic data generation method, which helps LLMs generate better training data for this task. 

%\citet{Ramu2024EnhancingPA} use template-based in-context learning with LLMs to tackle the post-hoc attribution problem. 



%which seems like what we are doing (also, on Q\&A). 

%\subsection{Miscellaneous}

 %\citet{si2024large} is more human-centric paper; check here if there is evidence for why we should do this from human side point of view. 
 %Evidence-based Q\&A \cite{Schimanski2024TowardsFA}. 
 
%\cite{Phukan2024PeeringIT} attribution in contextual Q\&A. 


% Other papers suggested by Goran: a survey \cite{Li2023ASO}

%This work is about sub-sentence encoder; probably not relevant unless we want to argue about sub-sentence propositions \cite{Chen2024SubSentenceEC}.

%This paper shows that post-hoc explanations of LLMs can improve them; maybe argument for explaining why we do this? \cite{Krishna2023PostHE}

%Papers suggested by Shahbaz: %\cite{Alnuhait2024FactCheckmatePD,Tao2024YourWL,Orgad2024LLMsKM,Chen2024XplainLLMAK}
% COMMENT: these papers are mostly on the topic of hallucination and are not directly linked to our work.

% Leakge detection with n-grams \cite{xu2024benchmarking}

% TruthReader \cite{Li2024TruthReaderTT} is a UI that enables users to upload documents, query an LLM w.r.t.~the documents, and show attributions of the LLM outputs w.r.t.~the input documents. The attributions are scored with ROUGE scores. 




We introduced \name{}, a dataset designed to evaluate how LLMs are affected by different framings, in comparison to humans. The statements in the dataset are start from a clear positive or negative base statement, reframed with an opposite-sentiment suffix or prefix. By collecting human annotations, we quantify the strength of the framing effect for each instance and assess how LLMs react to the same reframing, focusing on sentiment shifts -- cases where the perceived sentiment aligns with the opposite framing sentiment.

As LLMs become increasingly integrated into decision-making systems, it becomes crucial to understand how framing influences their outputs. 
In some applications, e.g., in virtual companions, framing can be harnessed to produce human-like behavior leading to better engagement.
In contrast, in other applications, such as financial or legal advice, mitigating the framing effect can lead to less biased decisions.
In both cases, a better understanding of framing in LLMs can help develop application-appropriate strategies.

We find that LLMs and humans exhibit similar behavior on \name{}, with all tested models achieving a strong correlation  $(r \geq 0.57)$ with human responses. Notably, \gpt{} showed the weakest correlation to humans, despite being widely regarded as a very capable model in other contexts.

We hope this work encourages research into distinguishing between scenarios where human-like behavior is desirable and those where models should surpass human limitations to achieve above-human performance. These insights are essential for developing LLMs that are both interpretable and optimized for their intended applications.
%\clearpage
In this work, we address a cognitive bias, and as with any research involving human participants, our study has several limitations.

First, our framing experiment is conducted within a single domain -- Amazon reviews -- and focuses on a specific type of statement. Some of our findings may be artifacts of this dataset rather than generalizable patterns.

Additionally, our approach to framing is highly specific. We only manipulate statements by adding a prefix or suffix, whereas reframing can take many other forms, such as restructuring sentences or altering word choices to convey ambiguous sentiment. This may limit the generalizability of our results.

Furthermore, our study focuses solely on sentiment analysis. Other downstream tasks influenced by framing, such as question answering or decision-making, may exhibit different patterns of sensitivity. Investigating these tasks could provide further insights into the broader impact of framing on LLM behavior in real-world applications.



\bibliography{custom}

\clearpage

\appendix
\input{sections/appendix_rel_work}
\section{Method for Synthetic Data Generation}
\label{app:synthetic_data}

\subsection{Multi-hop Generation of Attribution Data}
To generate synthetic data with the use of Wikipedia, we use the WikiNLP dataset \cite{gashteovski2019opiec}. It contains the text from all Wikipedia articles along with annotations for links within the text that link to other Wikipedia articles. The main idea is to use the links in order to imitate reasoning hops across different (related) articles. Therefore, we filter out all articles that either do not contain links or that contain links to articles that do not contain links. Finally, for each article, we use only the first paragraph, because this is considered to be the paragraph that contains the most ``definitional information'' \cite{bovi2015}; i.e., information that precisely describes the target concept of the article and contains the most important information about it.

%\textcolor{red}{TODO: Write here (at length?) about how our procedure differs from hotpotqa, even though it is similar.
%We are not recreating hotpot qa, the main differences are}

%\textcolor{red}{- Their process it more currated, with human involvement in multiple steps (selecting articles, writing questions + answers + attributions).}
%\textcolor{red}{- We also create dialog data.}
%\textcolor{red}{- We do not enforce the multiple hops, but rather, allow the LLM to decide whether such QA pair is possible.}

Then, for each article, we randomly select a sentence that contains at least one link to another Wikipedia article.\footnote{To make sure we have multi-hop scenario, we also check if the other Wikipedia article also contains at least one valid link to another Wikipedia article.} Each of the sentences that we sample serve as ground truths for the context attribution. With these sentences, we then prompt an LLM to generate a question-answer pair.

\subsection{Question-Answer Pairs Generation}
Using the multi-hop chain of sentences, we prompt an LLM (see \S\ref{app:prompts}) by providing it \textit{only} the formed chain as evidence. The LLM generates a question-answer pair that must be answered using the information in these supporting sentences, ensuring the pairs are grounded in the provided evidence.

\subsection{Distractors Mining}
In realistic scenarios, whether the context is user-provided or retrieved through RAG, the system typically encounters multiple context documents that are highly similar to those containing the evidence sentences. To bridge this gap between our synthetically generated training data using \synqa and the data models encounter ``in the wild'', we augment each training sample with hard negative distractor articles. We obtain embeddings using E5 \cite{wang2022text} for each Wikipedia article in our collection. Then, for each article containing a supporting sentence for the question-answer pair, we randomly sample up to three distractor articles that share semantic similarity with the ground truth article. This process increases the difficulty of the training data, producing models better equipped to handle diverse testing scenarios.

\subsection{Comparison to HotpotQA}

Although our method is inspired by HotpotQA \cite{Yang2018HotpotQAAD}, note that we do not aim to recreate the HotpotQA dataset. Our method has significant differences, which result in both much higher amount of data and in higher domain variability. 

Particularly, their method is more curated and involves humans in multiple steps. First, the authors manually select the target entities (and, with that, the target articles from which the annotators create the question and answer pairs). The reason for this is because many highly specialized articles---e.g., the article for IPv4 protocol---will not be suitable for crowd-workers to both identify meaningful questions and provide answers for those questions. Our approach does not have this constraint and, therefore, produces data that has much higher domain variability. 

Second, their method uses mechanical Turk workers to annotate the questions, answers and attribution sentences. In our case, we automatically select the hopped sentences (which serve as gold context attribution data), and then we use these sentences to generate question-answer pairs with an LLM.

Third, while HotpotQA always enforces multi-hop QA pairs, we do not instruct the LLM to do that. Rather, we first allow the LLM to decide whether generating such multihop QA pair is actually possible for the incoming context attribution sentences. If so, then the LLM generates multihop QA pairs. Otherwise, it generates direct QA pairs that do not need hops; i.e., QA pairs like in SQuAD \cite{Rajpurkar2016SQuAD1Q}.

Fourth, the HotpotQA annotation method does not allow for dialogue QA. In our method, we also create dialogue multi-hop data.

With these differences in mind, we showed that, compared to HotpotQA, our data generation method exhibits the following advantages: (1) our method generates data with higher domain variability; (2) our method goes beyond multi-hop QA and also generates direct QA pairs (like SQuAD) as well as dialogue QA data; (3) we generate the data in completely automatic manner without the involvement of humans. 


%\textcolor{red}{TODO: The procedure with the hops creation needs to be added in the appendix.}


%\begin{enumerate}
%    \item Read the data
%    \item For each data point, do:
%    \item check if the link is hopable. If not, skip. Otherwise, continue.
%    \item sample a linked sentence. This sentence needs to be hopable and to be able to do a second hop from it as well.
%    \item sample a link from the linked sentence. The link should be linkable too.
%\end{enumerate}
\begin{figure*}[!h]
    \centering
    \lstinputlisting{prompt/search_keywords}
    \caption{60 search keywords to retrieve the literature, including materials synthesis recipes using Semantic Scholar API.}
    \label{fig:prompt_search_keywords}
\end{figure*}

\begin{figure*}[!h]
    \centering
    \lstinputlisting{prompt/paper_categorization}
    \caption{System prompt to categorize the literature converted to markdown format.}
    \label{fig:prompt_paper_categorization}
\end{figure*}

\begin{figure*}[!h]
    \centering
    \lstinputlisting{prompt/paper_extraction}
    \caption{System prompt to extract the recipe from literature converted to markdown format.}
    \label{fig:prompt_paper_extraction}
\end{figure*}


\begin{figure*}[!h]
    \centering
    \lstinputlisting[basicstyle=\ttfamily\small]{prompt/paper_extraction}
    % \inputminted[breaklines, fontsize=\small]{markdown}{prompt/prediction_1shot.tex}
    \caption{A prompt to predict the recipe with a one-shot example.}
    \label{fig:prompt_prediction_1shot}
\end{figure*}

% \begin{figure*}[!h]
%     \centering
%     \lstinputlisting{prompt/prediction_1shot_example}
%     \caption{An example recipe used to a one-shot example.}
%     \label{fig:prompt_prediction_1shot_example}
% \end{figure*}

\begin{figure*}[!h]
    \centering
    \lstinputlisting{prompt/prediction_rag}
    \caption{A prompt to predict the recipe using retrieval-augmented generation}
    \label{fig:prompt_prediction_rag}
\end{figure*}

\begin{figure*}[!h]
    \centering
    \lstinputlisting{prompt/judge}
    \caption{A prompt to judge the prediction recipe using LLM-as-a-Judge}
    \label{fig:prompt_judge}
\end{figure*}

\section{Zero-shot models}
In order to obtain context attributions with the instruction tuned LLM \cite{Jiang2023Mistral7, Dubey2024TheL3}, we use the following prompts:

\begin{prompt}
\textbf{SYSTEM PROMPT}

You are an AI assistant that identifies the sentence(s) in a provided context document most relevant for answering a specific question. Your task is to select only the sentence(s) containing the explicit information needed to answer the question accurately, without adding extra context.
\end{prompt}

\begin{prompt}
\textbf{USER PROMPT}

Context Document:

[numbered sentences from the context]

Question: [query text]

Answer: [answer text]

Based on the context document, identify the sentence number(s) from the following choices: [list of numbers]. Select only the sentence(s) that contain explicit information needed to answer the question directly.

Answer only with the corresponding number(s) in parentheses, without additional explanation.
\end{prompt}
\section{User Study}
\label{app:user_study}

An example of the attribution scenario evaluated in our user study. See Figure~\ref{fig:user_interface} for details.

\begin{figure*}[t]
    \centering
    \includegraphics[width=0.8\linewidth]{img/user_study_interface.png}
    \caption{An example of the attribution scenario evaluated in our user study. Both the answer and the context attributions are highlighted to help the user verify the correctness of the answer. In the absence of highlights, the user is instructed to read the entire context. This example showcases a practical application of context attribution in real world interactions with LLM generated content.}
    \label{fig:user_interface}
\end{figure*}


\end{document}