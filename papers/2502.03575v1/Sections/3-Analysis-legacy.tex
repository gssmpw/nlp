\section{An Analysis of Task-driven Scanpaths on Charts}
\label{sec:analysis}


For more effective modeling, we conducted an analysis based on the human scanpaths collected in a previous work~\cite{polatsek2018exploring} to better understand task-driven scanpaths on charts. The data includes 47 users's eye movement data on visualizations when solving analytical tasks. Our analysis includes 15 statistical charts  (e.g., bar chart, line chart, and scatterplot). We excluded other types of graphs, such as maps and information graphics, as they could potentially increase the exploration rate due to the complexity of their visual representations. 
Here are the three analytical tasks in the study:

\begin{itemize}
    \item[1)] \textit{Retrieve value (RV)}: Given a specific target, find the data value of that target (e.g., \textit{what is the value of a category?});
    \item[2)] \textit{Filter (F)}: Given a concrete condition, find which data point satisfies it (e.g., \textit{which category has a specific value?});
    \item[3)] \textit{Find extreme (FE)}: Find the data point showing an extreme value of an attribute within the data set (e.g., \textit{which category has the highest/lowest value?}).
\end{itemize}


The analysis in the previous study summarizes insightful overall findings: Scanpaths for the same visualization task are more coherent than exploratory free-viewing. The task results in top-down guidance of visual attention, with bottom-up image saliency having a lesser impact on visual attention.
Beyond looking at the general analysis from the previous study, We introduce more detailed metrics to show the statistical summary of task-driven scanpaths on charts. The measurements used are inspired by \cite{goldberg2010comparing}, summarized as follows:

\begin{itemize}
    \item \textit{Number of fixations}: The length of a scanpath can be measured by the count of eye fixations separated by saccades. 
    \item \textit{Fixation on task-dependent AOI ratio}: Task-dependent AOIs are regions that are relevant to the task, such as value labels, text labels, and data points~\cite{polatsek2018exploring}. When people focus on these areas, it indicates how they are processing the task. Inspired by the metric of Hit Any AOI Rate~\cite{wang22_etvis}, this measurement provides a summary of the overall visual attention on the task-related regions.
    \item \textit{Percentage of fixations within each area}: The metric shows the percentage of time spent looking at different parts of a chart. This helps summarize where people are focusing their visual attention. We look at key areas on charts: the title, the marks (like bars or data points), and the axes. The percentages are calculated by dividing the number of fixations in a specific area by the total number of fixations.
    \item \textit{Fixation transitions}: This metric represents the average number of times the eyes move from one area to another while performing a task. It helps us understand how often the eye fixations shift between different areas. Frequent fixation transitions may indicate that there is room for improvement in the design of the chart, such as bringing related elements closer together.
    \item \textit{Revisit frequencies}: The average number of fixation returns to an area during a task.
\end{itemize}

% \begin{figure}[!t]
% \centering
%   \includegraphics[width=0.7\textwidth]{Images/analysis-scanpath.png}
%   \caption{The deep analysis of task-driven scanpaths}
%   \label{fig:analysis}
% \end{figure}

Our analysis indicates the detailed statistical summary of task-driven eye movement behaviors. 
The main findings are characterized as follows:
1) \textit{More fixations than free viewing}: We found the number of fixations in the task-driven scanpaths on charts (81.9 on average) is much larger than the number in free-viewing tasks (37.4 on average). It indicates the difficulty of the analytical tasks than the free viewing of charts.
2) \textit{A small portion of fixations on task-dependent AOIs}: Although the scanpath is task-driven, we figure out most of the eye movement is not on the task-dependent regions as the fixation on task-dependent AOI rate is only 9.1\%. It hints that people may spend more time on other stuff on the charts, like gathering information or confirmation.
3) \textit{Preference of visual attention on the axi}s: People spend most of their fixations on the axis (40.6\% on average), then the region of graphic marks (32.2\% on average). This might be because the three analytical tasks are highly related to values, not the other visual features. 
4) \textit{Frequent fixation transitions}: We identified that the number of fixation transitions is high (about 18.9 times). On average, there are about 4 continuous fixations after each fixation transition.
5) \textit{Frequent revisits}: Human data also has high revisit rates. Spatially, they revisit marks (8.7 times on average) and axis (9.0 times on average), which means users frequently double-check data information in the chart for the answer. Frequent revisits may be due to forgetting the information.
These findings help with the consideration of model design and will be valuable for evaluating the human-likeness of predicted scanpaths (see evaluation metrics in Table.~\ref{tab:benchmark}).


% \subsection{Analysis Results}
% \begin{table}[htbp]
\centering
\caption{Summary statistics of task-driven human scanpaths on statistical charts}
\begin{tabular}{lrrrrrrrrrrrr}
\toprule 
\multirow{3}*{\makecell[l]{\textbf{Metric}}} & 
\multicolumn{4}{c}{\textbf{Retrieve Value}} & \multicolumn{4}{c}{\textbf{Filter}} & \multicolumn{4}{c}{\textbf{Find Extreme}} \\
~ & \multicolumn{2}{c}{w/ L} & \multicolumn{2}{c}{w/o L} & \multicolumn{2}{c}{w/ L} & \multicolumn{2}{c}{w/o L} & \multicolumn{2}{c}{w/ L} & \multicolumn{2}{c}{w/o L} \\
~ & mean & $S D$ & mean & $S D$ & mean & $S D$ & mean & $S D$ & mean & $S D$ & mean & $S D$ \\
\midrule 
Number of fixations & 92.8 & 62.1 & 71.2 & 47.9 & 96.0 & 60.1 & 79.9 & 42.7 & 76.4 & 51.4 & 76.8 & 44.6\\
Task AOIs ratio (\%) & 11.3 & 7.7 & 12.1 & 9.0 & 11.5 & 8.4 & 16.8 & 14.0 & 1.2 & 2.3 & 1.9 & 3.8\\
Fixation-on-title ratio (\%) & 12.8 & 11.1 & 12.9 & 11.5 & 7.1 & 6.7 & 11.3 & 10.0 & 12.1 & 9.9 & 20.8 & 12.1\\
Fixation-on-mark ratio (\%) & 27.9 & 12.3 & 33.1 & 19.3 & 36.7 & 16.3 & 37.9 & 22.2 & 29.1 & 15.7 & 28.3 & 20.7\\
Fixation-on-axis ratio (\%) & 31.1 & 14.6 & 52.5 & 19.6 & 35.5 & 14.3 & 49.1 & 20.3 & 26.6 & 14.4 & 48.6 & 20.3\\
Fixation-on-legend ratio (\%) & 25.5 & 15.1 & - & - & 18.5 & 12.2 & - & - & 29.2 & 16.6 & - & -\\
Fixation transitions & 29.8 & 18.9 & 18.6 & 13.0 & 31.4 & 19.4 & 19.1 & 11.7 & 24.3 & 15.3 & 19.0 & 11.8\\
Revisit frequency title & 3.1 & 2.6 & 2.6 & 2.3 & 2.3 & 2.2 & 2.6 & 2.3 & 2.9 & 3.5 & 4.2 & 3.3\\
Revisit frequency mark & 10.4 & 6.8 & 7.8 & 5.8 & 11.9 & 7.6 & 7.9 & 5.3 & 7.9 & 5.0 & 6.8 & 5.3\\
Revisit frequency axis & 9.7 & 6.9 & 8.2 & 5.8 & 11.8 & 8.3 & 8.5 & 5.4 & 7.8 & 6.2 & 8.1 & 4.9\\
Revisit frequency legend & 6.7 & 5.5 & - & - & 5.3 & 4.0 & - & - & 5.8 & 3.5 & - & -\\
\bottomrule
\end{tabular}
\label{tab:phenomena}
\end{table}

% Saccade length & 174.9 & 36.9 & 170.9 & 41.4 & 173.6 & 38.9 & 165.4 & 46.7 & 193.0 & 45.5 & 177.2 & 44.4\\
% First fixation on AOI & 12.9 & 24.0 & 14.0 & 23.5 & 16.1 & 18.0 & 14.6 & 16.0 & 17.6 & 30.5 & 14.6 & 30.1\\

% The main results aggregated at the task level are summarized in Table~\ref{tab:phenomena}.
% 705 scanpaths from 47 participants, looking at 
% General scanpath length: In task-driven scanpaths on statistical charts, the average number of fixations is about 76.0 (w/ legend) and 88.4 (w/o legend). ``Find extreme'' has a slightly shorter average than the other two tasks. Without legends, the scanpath can be shortened by over 10 fixations in ``retrieve value'' and ``filter'', but it does not affect the length for ``find extreme''.
