%%
%% This is file `sample-sigconf.tex',
%% generated with the docstrip utility.
%%
%% The original source files were:
%%
%% samples.dtx  (with options: `sigconf')
%% 
%% IMPORTANT NOTICE:
%% 
%% For the copyright see the source file.
%% 
%% Any modified versions of this file must be renamed
%% with new filenames distinct from sample-sigconf.tex.
%% 
%% For distribution of the original source see the terms
%% for copying and modification in the file samples.dtx.
%% 
%% This generated file may be distributed as long as the
%% original source files, as listed above, are part of the
%% same distribution. (The sources need not necessarily be
%% in the same archive or directory.)
%%
%% Commands for TeXCount
%TC:macro \cite [option:text,text]
%TC:macro \citep [option:text,text]
%TC:macro \citet [option:text,text]
%TC:envir table 0 1
%TC:envir table* 0 1
%TC:envir tabular [ignore] word
%TC:envir displaymath 0 word
%TC:envir math 0 word
%TC:envir comment 0 0
%%
%%
%% The first command in your LaTeX source must be the \documentclass command.
% \documentclass[sigconf]{acmart}
%% NOTE that a single column version may be required for 
%% submission and peer review. This can be done by changing
%% the \doucmentclass[...]{acmart} in this template to 
% \documentclass[manuscript,screen]{acmart}
% \documentclass[sigconf,review,anonymous]{acmart}
% \documentclass[anonymous,manuscript,screen,review]{acmart}
% \documentclass[manuscript,review,anonymous]{acmart}
\documentclass[sigconf]{acmart}
%% 
%% To ensure 100% compatibility, please check the white list of
%% approved LaTeX packages to be used with the Master Article Template at
%% https://www.acm.org/publications/taps/whitelist-of-latex-packages 
%% before creating your document. The white list page provides 
%% information on how to submit additional LaTeX packages for 
%% review and adoption.
%% Fonts used in the template cannot be substituted; margin 
%% adjustments are not allowed.
%%
%%
%% \BibTeX command to typeset BibTeX logo in the docs
\AtBeginDocument{%
  \providecommand\BibTeX{{%
    \normalfont B\kern-0.5em{\scshape i\kern-0.25em b}\kern-0.8em\TeX}}}

%% Rights management information.  This information is sent to you
%% when you complete the rights form.  These commands have SAMPLE
%% values in them; it is your responsibility as an author to replace
%% the commands and values with those provided to you when you
%% complete the rights form.
\copyrightyear{2025}
\acmYear{2025}
\setcopyright{cc}
\setcctype{by}
\acmConference[CHI '25]{CHI Conference on Human Factors in Computing Systems}{April 26-May 1, 2025}{Yokohama, Japan}
\acmBooktitle{CHI Conference on Human Factors in Computing Systems (CHI '25), April 26-May 1, 2025, Yokohama, Japan}\acmDOI{10.1145/3706598.3713128}
\acmISBN{979-8-4007-1394-1/25/04}

%%
%% Submission ID.
%% Use this when submitting an article to a sponsored event. You'll
%% receive a unique submission ID from the organizers
%% of the event, and this ID should be used as the parameter to this command.
\acmSubmissionID{9728}

%%
%% For managing citations, it is recommended to use bibliography
%% files in BibTeX format.
%%
%% You can then either use BibTeX with the ACM-Reference-Format style,
%% or BibLaTeX with the acmnumeric or acmauthoryear sytles, that include
%% support for advanced citation of software artefact from the
%% biblatex-software package, also separately available on CTAN.
%%
%% Look at the sample-*-biblatex.tex files for templates showcasing
%% the biblatex styles.
%%

%%
%% The majority of ACM publications use numbered citations and
%% references.  The command \citestyle{authoryear} switches to the
%% "author year" style.
%%
%% If you are preparing content for an event
%% sponsored by ACM SIGGRAPH, you must use the "author year" style of
%% citations and references.
%% Uncommenting
%% the next command will enable that style.
%%\citestyle{acmauthoryear}

%%
%% end of the preamble, start of the body of the document source.
\usepackage{color}
\usepackage{colortbl}
% \usepackage{amssymb}
\usepackage{xcolor}
\usepackage{makecell}
\usepackage{multirow}
\usepackage{multicol}
\usepackage{xspace}
\usepackage{hyperref}
\newcommand{\name}
{\textsc{Chartist}\xspace}
% {\textsc{ChartGaze}\xspace}
% {\textsc{VisReader}\xspace}
% {\textsc{ChartReader}\xspace}
% {\textsc{ChartObserver}\xspace}
\usepackage{subfigure}
\usepackage{pifont}
\usepackage{xcolor}
\newcommand{\cmark}{\textcolor{green!80!black}{\ding{51}}}
\newcommand{\xmark}{\textcolor{red}{\ding{55}}}

% revision
\usepackage{ifthen}
\newboolean{revising}
\setboolean{revising}{false}
\ifthenelse{\boolean{revising}}
{
    \newcommand{\rv}[1]{\textcolor{blue}{#1}}
} {
    \newcommand{\rv}[1]{#1}
}

\begin{document}
\title{\rv{\name: Task-driven Eye Movement Control for Chart Reading}}
% \title{How People Read Charts: A Model of Task-driven Eye Movement Control}
% \title{Modeling Task-driven Scanpaths on Charts}
% \title{Task-driven Human Attention Prediction for Chart Question Answering}

%%
%% The "author" command and its associated commands are used to define
%% the authors and their affiliations.
%% Of note is the shared affiliation of the first two authors, and the
%% "authornote" and "authornotemark" commands
%% used to denote shared contribution to the research.

\author{Danqing Shi}
\orcid{0000-0002-8105-0944}
\affiliation{%
  \institution{Aalto University}
  \state{Helsinki}
  \country{Finland}}
\author{Yao Wang}
\orcid{0000-0002-3633-8623}
\affiliation{%
  \institution{University of Stuttgart}
  \state{Stuttgart}
  \country{Germany}}
\author{Yunpeng Bai}
\orcid{0009-0008-7578-0079}
\affiliation{%
  \institution{National University of Singapore}
  \state{Singapore}
  \country{Singapore}}
\author{Andreas Bulling}
\orcid{0000-0001-6317-7303}
\affiliation{%
  \institution{University of Stuttgart}
  \state{Stuttgart}
  \country{Germany}}
\author{Antti Oulasvirta}
\orcid{0000-0002-2498-7837}
\affiliation{%
  \institution{Aalto University}
  \state{Helsinki}
  \country{Finland}}

%%
%% By default, the full list of authors will be used in the page
%% headers. Often, this list is too long, and will overlap
%% other information printed in the page headers. This command allows
%% the author to define a more concise list
%% of authors' names for this purpose.
\renewcommand{\shortauthors}{Shi et al.}



% \begin{teaserfigure}
%   \centering
%   \includegraphics[width=0.9\textwidth]{Images/teaser-scanpath.png}
%   \caption{We present \name, a computational model that can predict task-driven human scanpaths on charts. The figure demonstrates three analytical tasks involved in the study: retrieve value, filter, and find extreme. The visualization illustrates how models' predictions vary across tasks and match the pattern of human scanpaths, with fixation density maps overlaid.}
%   \label{fig:teaser}
% \end{teaserfigure}

% \begin{figure}[b]
% \noindent\fbox{%
% \parbox{\dimexpr\linewidth-2\fboxsep-2\fboxrule\relax}{%
% \begin{tabular}{l}
% The word count of this paper is \textcolor{blue}{7357}.
% \end{tabular}
% }%
% }
% \end{figure}

%%
%% The abstract is a short summary of the work to be presented in the
%% article

\begin{abstract}
Advancements in DNA sequencing technologies have significantly improved our ability to decode genomic sequences. However, the prediction and interpretation of these sequences remain challenging due to the intricate nature of genetic material. Large language models (LLMs) have introduced new opportunities for biological sequence analysis. Recent developments in genomic language models have underscored the potential of LLMs in deciphering DNA sequences. Nonetheless, existing models often face limitations in robustness and application scope, primarily due to constraints in model structure and training data scale. To address these limitations, we present \textbf{Gener}\textit{ator}, a generative genomic foundation model featuring a context length of 98k base pairs (bp) and 1.2B parameters. Trained on an expansive dataset comprising 386B bp of eukaryotic DNA, the \textbf{Gener}\textit{ator} demonstrates state-of-the-art performance across both established and newly proposed benchmarks. The model adheres to the central dogma of molecular biology, accurately generating protein-coding sequences that translate into proteins structurally analogous to known families. It also shows significant promise in sequence optimization, particularly through the prompt-responsive generation of enhancer sequences with specific activity profiles. These capabilities position the \textbf{Gener}\textit{ator} as a pivotal tool for genomic research and biotechnological advancement, enhancing our ability to interpret and predict complex biological systems and enabling precise genomic interventions. Implementation details and supplementary resources are available at \url{https://github.com/GenerTeam/GENERator}.
\keywords{DNA, Genomics, Foundation model, Generative model}
\vspace{12pt}
\end{abstract}



%%
%% This command processes the author and affiliation and title
%% information and builds the first part of the formatted document.
\maketitle

\section{Introduction}

\subsection{Background and Motivation}
Integrating Deep Reinforcement Learning (DRL) in financial market analysis significantly evolved investment analysis with Deep Learning. DRL combines deep learning and reinforcement learning to offer a sophisticated framework for adapting strategies in the dynamic financial domain. It allows a deep learning model to effectively decipher complex patterns in historical market data often overlooked by traditional quantitative models.
It is no secret that financial markets are inherently complex and influenced by economic trends and geopolitical events. Therefore, traditional financial modeling often struggles to adapt to these ever-changing conditions. However, with its direct learning from data and adaptive strategies, DRL presents a promising solution to these challenges. With its autonomous learning ability and continual adaptation to the financial environment, it leverages historical market data to identify complex relationships and patterns.


\subsection{Overview of Our Previous Work}
In recent years, significant progress has been made in applying deep reinforcement learning (DRL) to stock trading strategies. For instance, Wang et al. proposed a parallel multi-module DRL algorithm that effectively captures both current market conditions and long-term dependencies using fully connected and LSTM layers \cite{parallel_drl_stock_trading}. Zhang et al. introduced an automated stock trading system based on the Proximal Policy Optimization algorithm, modeling trading as a Markov decision process \cite{novel_drl_stock_trading}. Additionally, Huang et al. demonstrated the importance of integrating market sentiment data to enhance the performance of DRL models in trading \cite{market_sentiment_drl_stock_trading}. Liu et al. developed an end-to-end trading strategy using a multi-view environment representation neural network, incorporating a Long Memory mechanism to improve decision-making \cite{drl_end_to_end_stock_trading}. Lastly, Li et al. focused on adaptive trading strategies using Gated Recurrent Units to capture time-series data effectively \cite{adaptive_drl_stock_trading}. These studies collectively highlight the potential of DRL in creating robust and adaptive trading strategies.

Liu et al. significantly advanced Deep Reinforcement Learning in Finance by developing platforms such as FinRL-Meta \cite{Liu2022FinRL}. This platform is a comprehensive tool for training and evaluating data-driven reinforcement learning agents within several simulated financial markets, offering a robust benchmarking system for algorithm comparison and facilitating the simulation of complex market conditions. The FinRL platform enables researchers to refine and test the efficacy of various DRL strategies, and it has been pivotal in democratizing access to sophisticated financial simulation tools and propelling research in financial analysis.

FinRL uses environments that offer broad simulation capabilities. These specialized environments, such as ABIDES-Gym \cite{Vyawahare2020}, provide the necessary infrastructure that allows FinRL to create discrete event simulations explicitly tailored for financial markets. ABIDES-Gym extends the OpenAI Gym interface to accommodate the complex dynamics of financial trading, allowing for a nuanced replication of market mechanisms and agent interactions. This level of detail will enable researchers and practitioners to explore the impact of individual agent behaviors and market responses, thus enhancing the understanding of market microstructure and agent-based modeling. The framework also streamlines the model training process on financial datasets, epitomizing the intersection of DRL and high-performance computing. It Leverages distributed computing resources to reduce training times significantly and optimizes computational workflows to enable the application of complex DRL models to extensive financial tasks. Their efforts have led to the creation of scalable and efficient financial models.

Our previous work \cite{Montazeri2023} demonstrated the efficiency and capability of CNNs when used as policies for deep reinforcement learning. We utilized the FinRL platform to conduct experiments on CNNs as a significantly improved policy to FinRL's original proposition. We also showed \cite{Montazeri2024, Montazeri2024GradientRC} that rearranging the stock market features used in the FinRL platform to group them per company could benefit the mode's performance. This study also utilizes the FinRL platform with its original dataset, containing features generated through traditional Technical Analysis used in Finance. It also uses the new dataset introduced in FinRL Meta, which contains statistically engineered features such as Simple Moving Average (SMA), momentum, and rate of change.

Building upon these foundational studies, our research aims to bridge the gap between CNN architecture optimization and financial market analysis. By introducing a systematic approach to temporal window selection, we seek to enhance the adaptability and performance of DRL models in capturing complex market dynamics.
    
\section{Objectives of the Current Study}
So far, we have presented the literature and the setting in which our study operates. The primary objective of our research is to explore the effects of changing the temporal window of a Convolutional Neural Network (CNN) used as a policy in a FinRL. By progressively expanding the observation period, beginning with a concise two-week window and incrementally enlarging it by two weeks in each iteration and culminating in twelve weeks, we aim to observe and analyze the performance of our model as its temporal window changes in the FinRL platform. This iterative window expansion is designed to examine the impact of different temporal scales on the model's performance. This process enables a comprehensive analysis of how varying lengths of financial data affect the model's predictive capabilities, offering insights and an opportunity to optimize the temporal granularity for financial market analysis. Our study also examines the arrangement of feature vectors within these expanding windows to better understand the model-market dynamics.

Furthermore, we contrast the model's performance across these different temporal windows to discern patterns in market behavior and model performance. In our study, short-term windows, particularly the initial two-week period, are hypothesized to be critical for understanding the model's ability to capture immediate market changes and short-term trends, which are essential for timely and accurate trading predictions. As the window expands, the model is expected to integrate a broader spectrum of market conditions, capturing longer-term trends and patterns. This bi-weekly expansion strategy is designed to balance the benefits of short-term immediacy and long-term historical perspective, ensuring the model remains adaptable and responsive to transient market fluctuations and enduring trends. We hope to contribute to financial analytics by demonstrating the efficacy of CNNs in a DRL setting and by providing new insights into the role of temporal dynamics in financial modeling.
\section{Related Work}
Researchers have been leveraging eye tracking methodologies from human perception research to model how people perceive images~\cite{shanmuga2015eye, bonhage2015combined, conklin2016using}.
These models help assess the appearance and salience of visual representations, enabling eye movement tracking to understand the perceptual and cognitive mechanisms of scene perception~\cite{itti1998model} and object detection~\cite{borji2015salient}.
The existing saliency models perform well in naturalistic scenes
%and real-world object detection
; however, there are unique perception rules and cognitive biases in the artificial world of data visualization 
%does not always follow the rules of perception in the natural world
~\cite{franconeri2021science, correll2012comparing, polatsek2018exploring, knittel2024gridlines}, and, thus, these models do not accurately predict where people would look in visualizations. 
Visualization researchers have been building visual saliency models geared to visualizations~\cite{DVSaliencyModel2017Matzen, bylinskii2016should}. %and adopting them for predicting eye gaze on visualizations. %enabling the prediction of visual saliency across design styles~\cite{fosco2020predicting}.
However, these models rely on handcrafted features, making it difficult to generalize to complex visualizations. Additionally, these models cannot incorporate textual information to generate task-specific saliency maps since the prediction is solely based on visual inputs.

With the advent of deep learning, gaze data were used as the ground truth of saliency models~\cite{fosco2020predicting, scannerDeeply, scanpath}, leading to higher performance in saliency prediction while enabling task-specific saliency~\cite{salchartQA}. 
These models usually need large-scale datasets to learn complex patterns. However, gathering precise gaze data is 
%challenging and requires specialized eye-tracking devices. While these devices provide accurate results, they tend to be 
costly and cumbersome, which limits large-scale data collection efforts. 
Many researchers, therefore, proposed several proxies for eye gaze. WebGaze~\cite{webgaze} uses a webcam for cheap and easy deployment in online studies yet suffers from data quality issues due to low-resolution cameras and uncontrolled calibration.
Therefore, mouse-(cursor-)based annotation tools~\cite{jiang2015salicon,bubbleView,importAnnot} were proposed to improve data quality. Among these methods, BubbleView~\cite{bubbleView} was the most used tool for capturing visual saliency and importance~\cite{graphicDesignImportance, salchartQA}.
However, BubbleView is primarily designed for exploring images and gathering information, which differs slightly from the goal of capturing perceived importance. As a result, while BubbleView is well-suited for measuring visual saliency, it may not be the best tool for capturing %instruction-tuned \yao{I would keep it consistent saying task-specific}
task-specific importance~\cite{turkeyes}. Built upon these prior approaches' limitations, our Grid Labeling aims to collect responses that cover all essential areas of the visualization with minimum noise, leading to more efficient data collection.



% One key motivation to our grid-based approach is to help people 
% We also demonstrate that the grid-based approaches can minimize biases in annotation to disproportionally emphasize text elements~\cite{DVSaliencyModel2017Matzen}
% % blurring the visualization can disproportionately emphasize text elements~\cite{DVSaliencyModel2017Matzen}, potentially misrepresenting a user's true areas of interest.
% More recently, 
% % \ms{Changed a bit using Yao's work (task-dependent saliency), but not sure whether it looks ok}
% Yao et al.~\cite{salchartQA} collect task-dependent saliency using the BubbleView method, % but their approach had some limitations. 
% and made a significant improvement on existing saliency models.
% First, the blurred visualization allowed users to perceive the overall structure of the chart, which prevented the system from capturing the specific action of identifying the maximum value. However, increasing the blur to address this issue introduced another challenge. As the structure became less visible, users had to explore the entire image, leading to the consideration of irrelevant regions as salient.
\section{\name: Modeling Task-driven Eye Movement on Charts}
\label{sec:model}

This section introduces the problem formulation and presents the computational model of eye movement control on charts in settings of analytical tasks.

\subsection{Problem Formulation}

Given a chart image $C$ and an associated analytical task $x$ stated as text, the model is expected to generate a sequence of fixation positions $\{ p_1, p_2, \dots , p_t\}$.
The objective of the output sequence is to closely match the scanpath from humans reading the chart. 
Specifically, the sequence of fixations represents the visual reasoning process, and the information in the patches of pixels fixated upon should be able to support $x$.
We consider general analytical tasks in information visualization~\cite{amar2005low}, and
select three of them used in a human eye-tracking data collection~\cite{polatsek2018exploring}: % \textit{RV}, \textit{F}, and \textit{FE} tasks
\rv{
\begin{itemize}
    \item[1)] \textit{Retrieve value (\textit{RV})}: Given a specific target, find the data value of the target (e.g., what is the value for a certain category?)
    \item[2)] \textit{Filter (\textit{F})}: Given a concrete condition, find which data point satisfies it (e.g., which category has the specific value stated?)
    \item[3)] \textit{Find extreme (\textit{FE})}: Find the data point showing an extreme value for a given attribute within the set of data (e.g., which category shows the highest/lowest value?)
\end{itemize}
}

\subsection{Modeling Overview}

Our goal was to develop the model \name to handle tasks articulated as free-form text and be able to perform gaze movement at a detailed pixel level.
We conceptualize the design of the hierarchical gaze control model in Figure~\ref{fig:model}, where the high-level (cognitive) controller is responsible for reasoning while the low-level (oculomotor) controller determines details of gaze movement. 
The idea behind this is hierarchical supervisory control~\cite{eppe2022intelligent}, which refers to a tiered control system in which the superior controller set goals for its subordinates. The actions from subordinates are integrated into an overall pattern for high-level control~\cite{pew1966acquisition}.
The concept also follows the modeling principle of computational rationality, where we assume that the controllers optimize their policy to maximize expected utility within relevant cognitive bounds~\cite{oulasvirta2022computational,chandramouli2024workflow}.
Specifically, the high-level controller handles abstract information processing, comprehension, and memory storage. 
It sets subtasks to the low-level controller, which then moves the gaze to gather information for task completion. Subsequently, the high-level controller utilizes the amassed information to answer the question.

\begin{figure*}[!t]
\centering
  \includegraphics[width=\textwidth]{Images/h-gaze-control.png}
  \caption{\textbf{An overview of the hierarchical eye-movement control architecture.} When presented with a chart and a task, a cognitive controller, powered by large language models, makes decisions on what to look at next and judges whether it is confident enough to provide an answer to the task's question. It relies on internal memory, which summarizes the information gathered from the chart through eye movements. Once cognitive control has determined the next action, the oculomotor controller is responsible for moving the gaze and observing the chart through a limited vision field. The model's objective is to accurately address the task as quickly as possible within set cognitive and physical constraints.}
  \Description{An overview of the hierarchical eye movement control architecture.}
  \label{fig:model}
\end{figure*}

\subsection{Cognitive Control}

The high-level controller provides cognitive control over the mental processes for a chart, control that performs reasoning in working memory~\cite{liu2010mental}. When performing vision tasks, one observes and analyzes visual information interactively~\cite{chen2020air}. Throughout this process, people analyze the information in their memory and try to gather more useful information to reduce uncertainty in solving the task.
To represent this decision problem accurately, we formulate it as a bounded optimality problem in a partially observable Markov decision process (POMDP). Instead of having access to a full state ($\mathcal{S}$) with pixels of the chart associated with the given task, the POMDP expresses a subset of ($\mathcal{S}$) as the observation of the model:
\begin{itemize}
    \item Observation $O$ refers to the information in memory that is captured from eye movements over the chart.
    \item Action $A$ includes subtasks that the model gives to oculomotor control for performing eye movements.
    \item Reward $R$ is the correctness of the answer for the task from the chart question answering.
\end{itemize}
To solve this POMDP, our model uses LLMs for the policy. The rationale behind this choice is that LLMs are well suited to processing higher-level information, as they have been pre-trained on human text data encompassing a wealth of logic related to planning, reasoning, and interaction~\cite{huang2022language, vemprala2024chatgpt, li2023interactive}.
Although LLMs are limited in their ability to control low-level motor functions in a precise manner~\cite{dalal2024psl}, they are proficient at planning and reasoning, with LLaMA~\cite{touvron2023llama} and GPT~\cite{achiam2023gpt} showing impressive language interpretation and reasoning capabilities.
Also, recent work has shown that utilizing LLMs in the high-level controllers in hierarchical architecture can produce promising results~\cite{huang2022language, brohan2023can, liang2023code}.
For our setting, we used GPT-4o~\cite{achiam2023gpt} for the policy, which takes the information accumulated in the memory as the observation and sets subtasks to guide eye movements in order to obtain information needed for solving the task efficiently.

We consider two human limitations when constructing the model's observation: a limited field of vision~\cite{duchowski2018gaze} and memory capacity~\cite{loftus2019human}. 
The model gets information from the gaze position purely by mimicking the human vision system. 
An optical character recognition technique~\cite{singh2010optical} is used to extract text from the pixels of the chart, and the text in the gaze area, with the position, is passed to the memory.
As a result, the observation consists of image patches (in a limited number) from the full set of chart pixels. The reliability of items in memory is determined by their visit history~\cite{li2023modeling}, with overall memory capacity being restricted too. When new information is added to the memory, a previously added item is removed on the basis of a forgetting probability. The probability of forgetting an item in the memory is calculated by means of the formula $\text{Softmax}(\rho \cdot (t-t_i)) $, where $t$ is the current fixation index, $t_i$ is the index of the $i$th item in the memory, and $\rho$ is the weight parameter (set to 0.1 here). The observation is designed as a prompt that summarizes the memory in line with the memory model and explains the model's goal. 

Given the summary of the memory information, the LLM policy selects predefined operations for task solving~\cite{brohan2023can, liang2023code}. The operations here are based on a sequence of cognitive stages for charts~\cite{goldberg2011eye} -- 1) \textit{search for text label}: visually searching for a text label or value label related to the task, 2) \textit{find associated mark}: visually searching for a graphical mark of the data point when given a reference label, 3) \textit{read associated value}: visually searching to read the given mark's associated value or textual label.
All these actions are allowed to be reused in the process, which enables the model to revisit previous positions for confirmation of the information.
Ultimately, if the information in the memory is sufficient to address the task, the gaze movement can stop and an answer can be given. Operations other than answering the question will be performed by the oculomotor controller for detailed gaze movement. 

The examples in Figure~\ref{fig:memory} demonstrate how utilizing memory information and predefined operations aids in scanpath prediction. Model memory uses the summarization capability of LLMs to convert the text and positions gathered to a paragraph as the observation (as shown in the green boxes). The LLM policy then makes decisions and issues subtasks as actions (in red boxes) for the oculomotor control, which performs pixel-level gaze movements.

\begin{figure*}[!t]
\centering
  \includegraphics[width=\textwidth]{Images/scanpath-memory.png}
  \caption{The figure gives examples of how the internal memory helps the cognitive controller to remember what has been read and then select actions for detailed gaze movement. A green box indicates the information held in memory, a red box represents the action selected by cognitive control, and the blue lines in the images reflect the eye movement scanpaths.}
  \label{fig:memory}
\end{figure*}

\subsection{Oculomotor Control}

The oculomotor controller acts as the interface between the cognitive controller and the actual chart-pixel images. Its main function is to control the movement of the gaze over the pixels in order to gather information related to the task at hand.
Generating oculomotor behavior at pixel level is another sequential decision-making problem that can be formulated as a POMDP:
\begin{itemize}
    \item Observation $o$ comprises vision information obtained from the external environment, which is jointly represented by the human vision system and visual short-term memory (VSTM).
    \item Action $a$  involves specifying the coordinates $(x, y)$ of a particular position to move to.
    \item Reward $r$ is designed to encourage the gaze to reach the target with less cost. It takes into account the number of target hits as well as the cost associated with the distance of the gaze movement.
\end{itemize}

Our modeling of a chart reader's observation follows an idea similar to that in visual search~\cite{yang2020predicting}. Utilizing a representation for accumulating information through fixations, this employs four components: 
1) The foveal and peripheral view come from the human vision system, which receives high-resolution visual input only from the region of the image around the fixation location. It includes two pixel-based modules to read the chart: foveal and peripheral vision~\cite{duchowski2018gaze}). 
2) Visual saliency provides a bottom-up signal to a chart reader for the given task. The saliency of the chart affects gaze behavior. We use a task-driven saliency model to represent this feature ~\cite{wang2024salchartqa}.
3)  Visit history represents VSTM, which stores visual information for a few seconds, thereby allowing its use in ongoing cognitive tasks~\cite{alvarez2004capacity}. We represent this history through a matrix where each point is marked as visited or not.
4) A goal-related reference position serves as the initial starting point of gaze movement. For example, the reader might begin at the position of a text label for locating the associated graphical mark, where the position of the text label serves as the reference for the sub-goal. 
\rv{We use a one-hot matrix to represent the reference, in which all cell values are 0 apart from the single 1 that identifies the target.}
All these components are encoded together via the deep convolutional neural network, followed by a fully connected network.

We train reinforcement learning policies to solve the POMDP for the oculomotor control, because it has been proven to effectively address decision-making challenges in prediction of details of gaze movement~\cite{yang2020predicting, jiang2024eyeformer, shi2024crtypist, bai2024heads}.
In our detail-level implementation, we resize the input chart images to be $320 \times 320$ and discretize the fixation position into a $20 \times 20$ map. Consequently, each fixation becomes a $16 \times 16$ image patch, and the gaze position is randomly sampled from within that patch. In this setup, the maximum approximation error resulting from this discretization process is less than one degree of the visual angle~\cite{yang2020predicting}.
\rv{
Ultimately, both the scanpath and the image will be converted back to the original chart size from $320 \times 320$ pixels.
}

\subsection{\rv{Workflow}}
\label{sec:workflow}

\rv{
Our implementation of \name is trained and tested on a collection of tasks and charts. There are four steps, illustrated in Figure~\ref{fig:pipeline}.
In Step 1, real-world charts are manually collected and labeled for areas of interest (AOIs), while synthetic charts are automatically generated and labeled in a manner powered by Vega-Lite~\cite{satyanarayan2016vega}. The inclusion of synthetic charts helps increase the diversity of the chart collection and addresses the challenge of obtaining numerous annotated charts.
In Step 2, tasks are automatically generated in line with specific rules for the \textit{RV}, \textit{F}, and \textit{FE} tasks. These tasks and labeled charts constitute a data collection for the training environment.t
With Step 3, the policies for oculomotor control are trained through reinforcement learning (using proximal policy pptimization, PPO~\cite{schulman2017proximal}) to optimize gaze movements, enabling the system to reach task-relevant positions as quickly as possible while adhering to vision constraints. Importantly, no eye tracking data are required for PPO training.
In the last phase, prediction, the hierarchical architecture combines pre-trained LLMs (GPT-4o) for cognitive control with RL policies for oculomotor control to generate the scanpath prediction.
}

\begin{figure*}[!h]
\centering
  \includegraphics[width=\textwidth]{Images/pipeline.png}
  \caption{\rv{An overview of the training workflow: 1) chart collection and labeling, wherein diverse real-world and synthetic charts are gathered, involving manual and automatic annotation of AOIs; 2) task generation, utilizing a rule-based approach to create tasks based on labeled charts to construct a data collection for training; 3) policy training, in which policy models are trained via RL from chart images with tasks; and 4) scanpath prediction, wherein pre-trained LLMs and RL policies are coordinated hierarchically to predict task-driven gaze movements over charts.}}
  \label{fig:pipeline}
  % \vspace{-10mm}
\end{figure*}
\begin{table*}[!htb]
\belowrulesep=0pt
\aboverulesep=0pt
\newcolumntype{C}{>{\centering\arraybackslash}p{25pt}}
\renewcommand{\arraystretch}{1}
    \resizebox{2.0\columnwidth}{!}{
    \begin{tabular}{CCCCCCCCCCCCccccccc}
    
    \specialrule{1.5pt}{0pt}{0pt}
    % summury title
    \multicolumn{12}{c}{Feature} & \multicolumn{7}{c}{Accuracy (MAPE[\%]$\downarrow$)} \\
    \cmidrule(lr){1-12} \cmidrule(lr){13-19}
    
    \multicolumn{5}{c}{Node} & \multicolumn{2}{c}{Edge} & \multicolumn{5}{c}{Global}
    & \multicolumn{3}{c}{Training} & \multicolumn{3}{c}{Inference} 
    & \multirow{2}{*}{\makecell{\textit{Mean} \\ (6 metrics)}} \\
    \cmidrule(lr){1-5} \cmidrule(lr){6-7} \cmidrule(lr){8-12} \cmidrule(lr){13-15} \cmidrule(lr){16-18}
    
    % title    
    Hp & CI & MAI & ArI* & \multicolumn{1}{c}{Prop*} & Sz & \multicolumn{1}{c}{Sp} & Gp* & CI* & \multicolumn{1}{c}{MAI*} &  ArI* & Bs & Util & Mem & \multicolumn{1}{c}{Time} & Util & Mem & \multicolumn{1}{c}{Time} \\
    \specialrule{1.0pt}{0pt}{0pt}

    \cm & \cu & \cu & \cu & \cu & \cu & \cu & \cu & \cu & \cu & \cu & \cu & 71.33 & 49.56 & 54.29 & 57.04 & 24.69 & 92.49 & \textcolor{colornode}{\textit{58.23}} \\
    \cm & \cm & \cu & \cu & \cu & \cu & \cu & \cu & \cu & \cu & \cu & \cu & 73.87 & 63.58 & 41.19 & 57.15 & 30.73 & 54.07 & \textcolor{colornode}{\textit{53.43}} \\
    \cm & \cm & \cm & \cu & \cu & \cu & \cu & \cu & \cu & \cu & \cu & \cu & 53.71 & 37.01 & 25.98 & 46.03 & 11.45 & 38.29 & \textcolor{colornode}{\textit{35.41}} \\
    \cm & \cm & \cm & \cm & \cu & \cu & \cu & \cu & \cu & \cu & \cu & \cu & 37.20 & 36.81 & 23.94 & 38.57 & 11.67 & 36.52 & \textcolor{colornode}{\textit{30.79}} \\
    \cm & \cm & \cm & \cm & \cm & \cu & \cu & \cu & \cu & \cu & \cu & \cu & 38.85 & 36.76 & 23.51 & 36.32 & 11.51 & 23.51 & \textcolor{colornode}{\textit{28.41}} \\
    % \specialrule{0.1pt}{0pt}{0pt}
    \cmidrule(lr){1-19}
    
    \cm & \cm & \cm & \cm & \cm & \cm & \cu & \cu & \cu & \cu & \cu & \cu & 41.24 & 25.29 & 24.05 & 37.55 & 9.28 & 23.27 & \textcolor{coloredge}{\textit{26.78}} \\
    \cm & \cm & \cm & \cm & \cm & \cm & \cm & \cu & \cu & \cu & \cu & \cu & 36.02 & 27.11 & 22.31 & 33.27 & 9.93 & 23.77 & \textcolor{coloredge}{\textit{25.40}} \\
    % \specialrule{0.1pt}{0pt}{0pt}
    \cmidrule(lr){1-19}
    
    \cm & \cm & \cm & \cm & \cm & \cm & \cm & \cm & \cu & \cu & \cu & \cu & 7.48 & 3.83 & 9.48 & 8.32 & 6.33 & 15.22 & \textcolor{colorglobal}{\textit{8.44}} \\
    \cm & \cm & \cm & \cm & \cm & \cm & \cm & \cm & \cm & \cu & \cu & \cu & 7.25 & 3.72 & 8.79 & 7.93 & 5.93 & 13.08 & \textcolor{colorglobal}{\textit{7.78}} \\
    \cm & \cm & \cm & \cm & \cm & \cm & \cm & \cm & \cm & \cm & \cu & \cu & 6.22 & 3.39 & 9.51 & 6.98 & 5.51 & 14.41 & \textcolor{colorglobal}{\textit{7.67}} \\
    \cm & \cm & \cm & \cm & \cm & \cm & \cm & \cm & \cm & \cm & \cm & \cu & 6.23 & 3.38 & 9.45 & 7.37 & 5.29 & 13.45 & \textcolor{colorglobal}{\textit{7.53}} \\
    \cm & \cm & \cm & \cm & \cm & \cm & \cm & \cm & \cm & \cm & \cm & \cm & 6.09 & 3.32 & 9.21 & 7.57 & 5.13 & 13.15 & \textcolor{colorglobal}{\textit{7.41}} \\
    
    \specialrule{1.5pt}{0pt}{0pt}

    \end{tabular}}
\caption{Ablation study of features. 
Hp, CI, MAI, ArI, Prop, Sz, Sp, GP, and Bs represent hyper-parameters, arithmetic intensity, computation info, memory access info, proportions, size, shape, graph profile, and batch size, respectively. "*" indicates our uniquely constructed features.}
\label{tab:ablation of features}
\end{table*}

\section{EVALUATION}
We evaluate PerfSeer by addressing the following research questions (RQs).

\textbf{RQ1:} How effective are the selected features and optimization components? 

\textbf{RQ2:} How does SeerNet compare with baseline models? 

\textbf{RQ3:} How effective is the multi-metric performance prediction model, SeerNet-Multi? 

\textbf{RQ4:} What is the application scope and overhead of PerfSeer? 

\subsection{Evaluation Setup}
\subsubsection{Training Settings.}\label{sec:train-setting}
The dataset is divided into 2:1:1 for training, validation, and testing. We use a batch size of 128 and an initial learning rate of 1e-3, halving it after five epochs without improvement, down to 1e-6. Training runs for up to 500 epochs, with Mean Squared Error (MSE) as the loss function and Adam as the optimizer.

\subsubsection{Evaluation Metrics.}\label{sec:metrics}
To ensure consistency across metrics with varying value ranges, we use percentage errors. The metrics include MAPE, Root Mean Square Percentage Error (RMSPE), and accuracy within a relative error of \(x\%\) (x\%Acc) \cite{Brp-nas}.

%% SeerNet
\subsection{Ablation Study (RQ1)}\label{sec:ablation}
The accuracy of the performance prediction depends on both representation and prediction. We conducted an ablation study to validate the features selection and construction (\emph{representation}) and the prediction model design (\emph{prediction}).
\subsubsection{Ablation Study of Features.} \label{sec:features ablation}
From the results shown in Table \ref{tab:ablation of features}, we observe the following:

\emph{Features Importance.} 
Node features are much more important than global features, which are in turn more important than edge features. 
The mean MAPE of SeerNet decreases from 58.23\% to 28.41\% as the node features are used increasingly, as they directly correspond to each operation node.
Adding the global features further improves accuracy, significantly reducing the mean MAPE from 25.40\% to 7.41\%, as they provide an overall representation of the model.
Edge features, though less important, improved MAPE slightly from 28.41\% to 25.40\% by providing additional memory access information, further enhancing prediction accuracy.

\emph{Feature Selection.} 
Each group of the node, edge, and global features we selected improved the performance prediction accuracy, demonstrating their relevance to model performance.
In contrast, node categories, which are commonly used in other predictors, reduced accuracy. We consider SeerNet extracts category-related information from other semantically rich features, so we excluded node categories from our feature set.

\emph{Feature Construction.} 
We constructed several unique and effective features. 
For node features, we included arithmetic intensity and computation/memory access proportions, reducing the mean MAPE from 35.41\% to 28.41\%. 
For global features, we introduced graph profiles, computation and memory access trend statistics, tensor size delivered per-edge, and arithmetic intensity, reducing the mean MAPE from 25.40\% to 7.53\%. 
These features, not used by any previous predictors, significantly enhance prediction accuracy.

\subsubsection{Ablation Study of Components.}\label{sec:ablation component}
From the results shown in Table \ref{tab:ablation of components}, we observe the following:

\emph{SynMM.}
Using SynMM to aggregate the node features reduces the mean MAPE from 7.41\% to 6.10\%. This demonstrates that SynMM combines max and mean aggregation to create a more comprehensive and robust representation, thereby enhancing prediction accuracy.

\emph{GNPB.}
With the introduction of GNPB, the mean MAPE further decreases from 6.10\% to 5.14\%. This result proves that GNPB enables complementary learning between the node and global features, enriching both perspectives and further improving prediction accuracy.
% The experiments verify that the optimization components can capture the performance information of a model effectively.

% \vspace{-0.3cm}
\setlength{\tabcolsep}{0pt}
\renewcommand{\arraystretch}{0.95}
\setcounter{table}{1}
\begin{table*}[b]
    \small
    % \vspace{-5mm}
    \caption{Parametric models included in the experiments. Cond. = conditioning method, R.F. = receptive field in samples.
    PEQ = Parametric EQ, G = Gain, O = Offset, MLP = Multilayer Perceptron, RNL = Rational Non Linearity. Controllers: 
    .s = static, .d = dynamic, .sc = static conditional, .dc = dynamic conditional}
    \label{tab:models}
    % \vspace{-2mm}
    \centerline{
        \begin{tabular}{L{2.8cm}C{1.3cm}R{1.1cm}C{1.1cm}C{1.1cm}C{1.3cm}C{1.5cm}R{1.4cm}R{1.3cm}R{1.3cm}}
            \hline
            \hline
            Model
                & Cond.
                    & R.F.
                        & Blocks
                            & Kernel
                                & Dilation
                                    & Channels
                                        & \# Params 
                                            & FLOP/s 
                                                & MAC/s\\ 
            \hline
            TCN-F-45-S-16 & FiLM & 2047 & 5 & 7 & 4 & 16 & 15.0k & 736.5M & 364.3M\\
            TCN-TF-45-S-16 & TFiLM & 2047 & 5 & 7 & 4 & 16 & 42.0k & 762.8M & 364.2M\\
            TCN-TTF-45-S-16 & TTFiLM & 2047 & 5 & 7 & 4 & 16 & 17.3k & 744.0M & 367.4M\\
            TCN-TVF-45-S-16 & TVFiLM & 2047 & 5 & 7 & 4 & 16 & 17.7k & 740.4M & 366.2M\\
            \hline
            \hline
        \end{tabular}
    }
    \centerline{
        \begin{tabular}{L{2.8cm}C{1.3cm}R{1.1cm}C{1.2cm}C{2.3cm}C{1.5cm}R{1.4cm}R{1.3cm}R{1.3cm}}
            Model
                & Cond.
                    & R.F.
                        & Blocks
                            & State Dimension
                                & Channels
                                    & \# Params
                                        & FLOP/s 
                                            & MAC/s\\ 
            \hline
            S4-F-S-16 & FiLM & - & 4 & 4 & 16 & 8.9k & 135.2M & 53.8M\\
            S4-TF-S-16 & TFiLM & - & 4 & 4 & 16 & 30.0k & 155.6M & 53.8M\\
            S4-TTF-S-16 & TTFiLM & - & 4 & 4 & 16 & 10.2k & 141.0M & 56.3M\\
            S4-TVF-S-16 & TVFiLM & - & 4 & 4 & 16 & 11.6k & 138.9M & 55.3M\\
            \hline
            \hline
        \end{tabular}
    }
    \centerline{
        \begin{tabular}{L{3cm}C{7.2cm}R{1.4cm}R{1.3cm}R{1.3cm}}
            Model
                & Signal Chain
                    & \# Params
                        & FLOP/s 
                            & MAC/s\\
            \hline
            GB-C-DIST-MLP & PEQ.sc $\rightarrow$ G.sc $\rightarrow$ O.sc $\rightarrow$ MLP $\rightarrow$ G.sc $\rightarrow$ PEQ.sc & 4.5k & 202.8M & 101.4M\\
            GB-C-DIST-RNL & PEQ.sc $\rightarrow$ G.sc $\rightarrow$ O.sc $\rightarrow$ RNL $\rightarrow$ G.sc $\rightarrow$ PEQ.sc & 2.3k & 920.5k & 4.3k\\
            \hline
            GB-C-FUZZ-MLP & PEQ.sc $\rightarrow$ G.sc $\rightarrow$ O.dc $\rightarrow$ MLP $\rightarrow$ G.sc $\rightarrow$ PEQ.sc & 4.2k & 202.8M & 101.4M\\
            GB-C-FUZZ-RNL & PEQ.sc $\rightarrow$ G.sc $\rightarrow$ O.dc $\rightarrow$ RNL $\rightarrow$ G.sc $\rightarrow$ PEQ.sc & 2.0k & 988.9k & 3.6k\\
            \hline
            \hline
        \end{tabular}
    }
    % \vspace{-4mm}
\end{table*}
\begin{table}[!htb]

\belowrulesep=0pt
\aboverulesep=0pt
\renewcommand{\arraystretch}{1}
    \resizebox{1.0\columnwidth}{!}{\begin{tabular}{lcccccccc}
    
    \specialrule{1.5pt}{0pt}{0pt}
    % summury title
    \multirow{3}{*}{Method} & \multicolumn{7}{c}{Accuracy (MAPE[\%]$\downarrow$)} & \multirow{3}{*}{\makecell{Params$\downarrow$ \\ {[}M{]}}} \\
    \cmidrule(lr){2-8}
    
    & \multicolumn{3}{c}{Training} & \multicolumn{3}{c}{Inference} & \multirow{2}{*}{\makecell{\textit{Mean} \\ (6 metrics)}}  \\
    \cmidrule(lr){2-4} \cmidrule(lr){5-7}
    
    % title
    & {Util} & {Mem} & {Time} & {Util} & {Mem} & {Time} & \multicolumn{1}{c}{} \\
    \specialrule{1.0pt}{0pt}{0pt}
    
    MLP-Node (MLP) & 21.27 & 5.41 & 12.37 & 35.21 & 6.38 & 23.77 & \textit{17.40} & 4.15 \\ % 4,149,249
    PMGNS (GraphSAGE) & 8.23 & 9.62 & 10.53 & 6.71 & 8.76 & 13.72 & \textit{9.60} & 3.45 \\ % 3,448,321
    Eagle-p (GCN) & 82.45 & 82.30 & 66.66 & 71.85 & 48.51 & 94.80 & \textit{74.43} & 1.11 \\ % 1,107,001
    Eagle-s (GCN) & 6.14 & 4.25 & 8.60 & 5.22 & 5.27 & 16.63 & \textit{7.69} & 1.10 \\ % 1,099,801
    \textbf{SeerNet (Our)} & \textbf{4.94} & \textbf{2.47} & \textbf{6.71} & \textbf{4.37} & \textbf{3.46} & \textbf{8.91} & \textbf{\textit{5.14}} & \textbf{1.02} \\ % 1,015,831

    \specialrule{1.5pt}{0pt}{0pt}
    
  \end{tabular}}
\caption{SeerNet comparison on our dataset}
\label{tab:comparison our}
\end{table}

\begin{table*}[!htb]
\belowrulesep=0pt
\aboverulesep=0pt
\newcolumntype{C}{>{\centering\arraybackslash}p{30pt}}
\renewcommand{\arraystretch}{1}
    \resizebox{2.0\columnwidth}{!}{
    \begin{tabular}{llCCCCCCCCCCCC}

    \specialrule{1.5pt}{0pt}{0pt}
    % summury title
    \multirow{3}{*}{Method} & \multirow{3}{*}{Model} & \multicolumn{3}{c}{Mobile CPU (CortexA76)} & \multicolumn{3}{c}{Mobile GPU (Adreno 640)} & \multicolumn{3}{c}{Intel VPU (MyriadX)} & \multicolumn{3}{c}{\textit{Mean (3 devices)}} \\
    
    % title
    & & RMSPE$\downarrow$ & Acc$\uparrow$ & Acc$\uparrow$ & RMSPE$\downarrow$ & Acc$\uparrow$ & Acc$\uparrow$ & RMSPE$\downarrow$ & Acc$\uparrow$ & Acc$\uparrow$ & RMSPE$\downarrow$ & Acc$\uparrow$ & Acc$\uparrow$ \\
    & & [\%] & [5\%] & [10\%] & [\%] & [5\%] & [10\%] & [\%] & [5\%] & [10\%] & [\%] & [5\%] &  [10\%] \\
    \specialrule{1.5pt}{0pt}{0pt}
    
    % content
    \multirow{13}{*}{\textit{nn-Meter}} & AlexNets & 3.90 & 81.0 & 98.6 & 5.32 & 72.0 & 94.0 & 10.74 & 23.4 & 60.9 & \textit{6.65} & \textit{58.8} & \textit{84.5} \\
    & DenseNets & 2.76 & 93.1 & 99.9 & 4.52 & 68.6 & 99.9 & 5.89 & 75.6 & 86.3 & \textit{4.39} & \textit{79.1} & \textit{95.4} \\
    & GoogleNets & 3.27 & 85.9 & 100.0 & 1.35 & 100.0 & 100.0 & 5.86 & 39.7 & 98.4 & \textit{3.49} & \textit{75.2} & \textit{99.5} \\
    & MnasNets & 5.54 & 50.9 & 99.2 & 1.86 & 100.0 & 100.0 & 4.34 & 77.3 & 97.7 & \textit{3.91} & \textit{76.1} & \textit{99.0} \\
    & MobileNetv1s & 4.98 & 63.8 & 97.8 & 2.56 & 96.9 & 100.0 & 5.90 & 54.2 & 93.3 & \textit{4.48} & \textit{71.6} & \textit{97.0} \\
    & MobileNetv2s & 4.84 & 67.6 & 97.7 & 3.93 & 80.0 & 99.0 & 4.26 & 78.3 & 97.6 & \textit{4.34} & \textit{75.3} & \textit{98.1} \\
    & MobileNetv3s & 4.34 & 73.8 & 99.0 & 4.02 & 84.4 & 100.0 & 5.72 & 47.6 & 98.5 & \textit{4.69} & \textit{68.6} & \textit{99.1} \\
    & NASBench201 & 3.51 & 82.4 & 99.9 & 3.80 & 75.9 & 100.0 & 18.20 & 19.3 & 40.6 & \textit{8.50} & \textit{59.2} & \textit{80.1} \\
    & ProxylessNas & 3.44 & 84.6 & 100.0 & 3.28 & 95.6 & 98.9 & 5.05 & 65.6 & 96.9 & \textit{3.92} & \textit{81.9} & \textit{98.6} \\
    & ResNets & 4.41 & 72.3 & 98.1 & 3.16 & 88.8 & 99.9 & 7.42 & 37.9 & 84.2 & \textit{5.00} & \textit{66.3} & \textit{94.1} \\
    & ShuffleNetv2s & 5.01 & 61.6 & 98.3 & - & - & - & 6.37 & 45.6 & 91.3 & \textit{5.69} & \textit{53.6} & \textit{94.8} \\
    & SqueezeNets & 3.59 & 84.5 & 99.9 & 3.85 & 81.9 & 97.9 & 7.08 & 66.1 & 88.5 & \textit{4.84} & \textit{77.5} & \textit{95.4} \\
    & VGGs & 4.84 & 66.1 & 98.2 & 2.97 & 91.8 & 99.8 & 22.25 & 27.1 & 50.6 & \textit{10.02} & \textit{61.7} & \textit{82.9} \\
    & \textit{Mean (13 models)} & \textit{4.19}& \textit{74.4} & \textit{99.0} & \textit{3.39} & \textit{86.3} & \textit{99.1} & \textit{8.39} & \textit{50.6} & \textit{83.4} & \underline{\textit{5.38}} & \underline{\textit{69.6}} & \underline{\textit{93.7}} \\
    \specialrule{1.0pt}{0pt}{0pt}
    
    % \multirow{13}{*}{\textit{\makecell{SeerNet \\ (Our)}}}
    \multirow{13}{*}{\textit{SeerNet (Our)}}
    & AlexNets & 3.91$\dagger$ & 83.3 & 97.9$\dagger$ & 3.34 & 89.6 & 97.7 & 3.48 & 84.6 & 99.5 & \textit{3.58} & \textit{85.9} & \textit{98.4} \\
    & DenseNets & 2.33 & 95.8 & 100.0 & 1.15 & 100.0 & 100.0 & 1.82 & 99.2 & 100.0 & \textit{1.77} & \textit{98.4} & \textit{100.0} \\
    & GoogleNets & 2.04 & 99.0 & 100.0 & 1.06 & 100.0 & 100.0 & 1.36 & 100.0 & 100.0 & \textit{1.49} & \textit{99.7} & \textit{100.0} \\
    & MnasNets & 2.49 & 97.9 & 100.0 & 1.70 & 100.0 & 100.0 & 3.23 & 88.8 & 99.0 & \textit{2.47} & \textit{95.6} & \textit{99.7} \\
    & MobileNetv1s & 2.53 & 96.4 & 100.0 & 1.47 & 100.0 & 100.0 & 3.71 & 84.9 & 98.2 & \textit{2.57} & \textit{93.8} & \textit{99.4} \\
    & MobileNetv2s & 3.18 & 87.8 & 100.0 & 2.66 & 93.2 & 100.0 & 4.13 & 78.4 & 98.7 & \textit{3.32} & \textit{86.5} & \textit{99.6} \\
    & MobileNetv3s & 2.80 & 93.0 & 100.0 & 2.23 & 96.9 & 100.0 & 2.06 & 98.2 & 100.0 & \textit{2.36} & \textit{96.0} & \textit{100.0} \\
    & NasBench201s & 2.77 & 94.8 & 100.0 & 2.32 & 97.7 & 100.0 & 3.74 & 82.3 & 98.2 & \textit{2.94} & \textit{91.6} & \textit{99.4} \\
    & ProxylessNas & 2.40 & 97.4 & 100.0 & 2.07 & 97.4 & 100.0 & 1.93 & 97.9 & 100.0 & \textit{2.13} & \textit{97.6} & \textit{100.0} \\
    & ResNets & 4.51$\dagger$ & 76.4 & 96.3$\dagger$ & 2.72 & 90.9 & 99.4$\dagger$ & 3.08 & 88.9 & 99.4 & \textit{3.44} & \textit{85.4} & \textit{98.4} \\
    & ShuffleNetv2s & 2.69 & 95.6 & 100.0 & - & - & - & 2.35 & 96.9 & 100.0 & \textit{2.52} & \textit{96.2} & \textit{100.0} \\
    & SqueezeNets & 3.33 & 87.0 & 100.0 & 2.36 & 96.9 & 100.0 & 4.44 & 75.8 & 96.6 & \textit{3.38} & \textit{86.6} & \textit{98.9} \\
    & VGGs & 6.03$\dagger$ & 61.9 & 93.2$\dagger$ & 2.44 & 94.9 & 99.4$\dagger$ & 13.52 & 38.4 & 62.2 & \textit{7.33} & \textit{65.1} & \textit{84.9} \\
    & \textit{Mean (13 models)} & \textit{3.15} & \textit{89.7} & \textit{99.0} & \textit{2.13} & \textit{96.5} & \textit{99.7} & \textit{3.76} & \textit{85.7} & \textit{96.3} & \underline{\textit{3.02}} & \underline{\textit{90.6}} & \underline{\textit{98.4}} \\
    \specialrule{1.5pt}{0pt}{0pt}
    
  \end{tabular}}
\caption{Comparison with nn-Meter. "$\dagger$" indicates SeerNet underperforms than nn-Meter, and the italicized and underlined entries represent the prediction results of execution time across 13 model types on 3 devices.}
\label{tab:comparison nnm}
\end{table*}
\subsection{Baseline Comparison (RQ2)}\label{sec:seernet compare}
\subsubsection{Baseline.} 
We compare SeerNet with the following methods:
% MLP-Node
(i) MLP-Node uses an MLP to predict model performance, concatenating features from all nodes in the graph and handling variable node numbers by padding or truncating them to a fixed size.
% PMGNS
(ii) PMGNS \cite{DIPPM} (described in Section \ref{sec:related work}) utilizes a single prediction head for predicting one performance metric.
% Eagel
(iii) Eagle \cite{Brp-nas} (described in Section \ref{sec:related work}) is implemented for cell-based models, but for non-cell-based models, it uses features from PMGNS (Eagle-p) and our features from SeerPerf (Eagle-s).
% nn-Meter
(iv) nn-Meter \cite{Nn-meter} (described in Section \ref{sec:related work}) is a kernel-based predictor.

\subsubsection{Performance Comparison.}
\emph{Baseline Comparison on our Dataset.}
Table \ref{tab:comparison our} shows that SeerNet has the smallest parameter size (1.02M) and the highest accuracy, with a mean MAPE of 5.14\%.
Compared to SeerNet,
MLP-Node contains more than four times the parameters and achieves a MAPE over three times higher, while PMGNS contains more than three times the parameters and has a MAPE nearly twice as high.
Eagle-p performs poorly, with a MAPE of 74.43\%, due to the inability of PMGNS features to accurately represent the models in our dataset.
Eagle-s, which uses our proposed features, performs better but still lags behind SeerNet.
Both PMGNS and Eagle-s outperform MLP-Node, achieving higher accuracy with fewer parameters, highlighting the ability of GNNs to capture execution dependencies.
Eagle-s also performs better than Eagle-p, demonstrating the effectiveness of our proposed features. SeerNet outperforms the other models, offering the best representation and prediction.

\emph{Baseline Comparison on the Dataset of nn-Meter.} % Comparison with nn-Meter
Table \ref{tab:comparison nnm} shows SeerNet outperforms nn-Meter with half the RMSPE, 21\% higher at 5\%Acc, and 5\% higher at 10\%Acc.
For models like VGG, ResNet, and AlexNet on the CPU, SeerNet is slightly less accurate than nn-Meter, likely due to the simple structures of these models and their predictable execution patterns on the CPU.
On VPU, nn-Meter achieves 50.6\% at 5\%Acc, while SeerNet reaches 85.7\%, 35\% higher. 
This is because the execution of VPU is more complex, and nn-Meter fails to design effective detection functions. 

\begin{table}[!htb]

\belowrulesep=0pt
\aboverulesep=0pt
\renewcommand{\arraystretch}{1}
    \resizebox{1.0\columnwidth}{!}{\begin{tabular}{lcccccccc}
    
    \specialrule{1.5pt}{0pt}{0pt}
    % summury title
    \multirow{3}{*}{Method} & \multicolumn{7}{c}{Accuracy (MAPE[\%]$\downarrow$)} & \multirow{3}{*}{\makecell{Params$\downarrow$ \\ {[}M{]}}} \\
    \cmidrule(lr){2-8}
    
    & \multicolumn{3}{c}{Training} & \multicolumn{3}{c}{Inference} & \multirow{2}{*}{\makecell{\textit{Mean} \\ (6 metrics)}}  \\
    % half mid line
    \cmidrule(lr){2-4} \cmidrule(lr){5-7}

    % title
    & {Util} & {Mem} & {Time} & {Util} & {Mem} & {Time} & \multicolumn{1}{c}{} \\
    \specialrule{1.0pt}{0pt}{0pt}

    PMGNS-Multi & 38.7 & 10.6 & 21.7 & 83.4 & 48.7 & 99.8 & \textit{50.5} & 3.45 \\ % 3,449,347
    SeerNet (×3) & 4.94 & 2.47 & 6.71 & 4.37 & 3.46 & 8.91 & \textit{5.14} & 3.05 \\ % 3,047,493
    SeerNet-Multi (w/o PCGrad) & 17.60 & 3.00 & 22.10 & 18.90 & 3.70 & 29.90 & \textit{15.85} & 1.15 \\ % 1,148,953
    \textbf{SeerNet-Multi (w/ PCGrad)} & \textbf{6.90} & \textbf{3.30} & \textbf{9.10} & \textbf{8.70} & \textbf{3.60} & \textbf{15.20} & \textbf{\textit{7.75}} & \textbf{1.15} \\ % 1,148,953
    \specialrule{1.5pt}{0pt}{0pt}
    
    \end{tabular}}
\caption{Evaluation result of SeerNet-Multi.}
\label{tab:seernet multi}
\end{table}

\begin{table}[!htb]

\belowrulesep=0pt
\aboverulesep=0pt
\newcolumntype{C}{>{\centering\arraybackslash}p{30pt}}
\renewcommand{\arraystretch}{1}
    \resizebox{1.0\columnwidth}{!}{\begin{tabular}{rccccccc}
    
    \specialrule{1.5pt}{0pt}{0pt}
    % summury title
    \multirow{3}{*}{\makecell{Dataset \\ scale}} & \multicolumn{7}{c}{Accuracy (MAPE[\%]$\downarrow$)} \\
    \cmidrule(lr){2-8}
    
    & \multicolumn{3}{c}{Training} & \multicolumn{3}{c}{Inference} & \multirow{2}{*}{\makecell{Mean \\ (6 metrics)}} \\
    \cmidrule(lr){2-4} \cmidrule(lr){5-7}
    
    % title
    & {Util} & {Mem} & {Time} & {Util} & {Mem} & {Time} \\
    \specialrule{1.0pt}{0pt}{0pt}

    \textbf{overall} & \textbf{4.94} & \textbf{2.47} & \textbf{6.71} & \textbf{4.37} & \textbf{3.46} & \textbf{8.91} & \textbf{\textit{5.14}} \\ % 128
    20000 & 5.49 & 2.49 & 8.58 & 5.04 & 3.83 & 13.65 & \textit{6.51} \\ % 128
    % 15000 & 5.76 & 3.23 & 9.25 & 5.40 & 3.93 & 13.32 & \textit{6.82} \\ % 128
    10000 & 6.01 & 2.91 & 10.05 & 5.31 & 5.37 & 14.49 & \textit{7.36} \\ % 128
    5000 & 7.61 & 4.24 & 10.78 & 6.32 & 4.79 & 17.89 & \textit{8.61} \\ % 64
    2000 & 10.24 & 6.39 & 12.33 & 8.26 & 7.49 & 19.25 & \textit{10.66} \\ % 32
    1000 & 18.70 & 5.53 & 15.68 & 14.86 & 6.94 & 28.19 & \textit{14.98} \\ % 8
    
  \specialrule{1.5pt}{0pt}{0pt}
  
  \end{tabular}}
\caption{Data dependency of SeerNet.}
\label{tab:data dependency}
\end{table}


\subsection{Effectiveness of SeerNet-Multi (RQ3)}\label{sec:eva seernet multi}
Table \ref{tab:comparison nnm} shows that PMGNS-Multi (PMGNS with multiple prediction heads) performed poorly with a MAPE of 50.5\%, while SeerNet-Multi (without PCGrad) had a MAPE of 15.85\%, indicating that predicting multiple metrics simultaneously reduces accuracy.
%
However, with PCGrad, SeerNet-Multi halved its MAPE from 15.85\% to 7.75\% without increasing parameter overhead. This demonstrates that PCGrad effectively mitigates conflicting gradient directions across tasks, enabling SeerNet-Multi to predict multiple metrics efficiently with minimal accuracy loss.
%
Furthermore, SeerNet-Multi has about one-third the parameters of SeerNet, with only a 2.61\% increase in MAPE, balancing parameter efficiency and prediction accuracy. This makes SeerNet-Multi ideal for scenarios requiring rapid predictions with limited resources and lower accuracy demands.

\subsection{Further Discussion (RQ4)}\label{sec:extension experiments}
\subsubsection{Application Scope.}
\emph{Multi-Model, Multi-Metric Support.}
SeerPerf provides accurate predictions for execution time, memory usage, and SM utilization during both training and inference across various architectures, including GoogLeNet, VGG, ResNe(X)t, MobileNet, and DenseNet.

\emph{Multi-Device Support.}
PerfSeer provides accurate predictions across various devices, including mobile CPUs, mobile GPUs, desktop GPUs, and Intel VPUs, as shown in Table \ref{tab:comparison nnm}. In contrast, nn-Meter exhibits poor prediction accuracy on Intel VPUs.

\emph{Multi-Platform Support.}
The representation of PerfSeer is based on ONNX, so SeerPerf supports any DL framework convertible to ONNX.
\emph{Overall}, our performance predictor, PerfSeer, has a wide application scope, making it suitable for most common applications.

\subsubsection{Overhead.}
\emph{Data dependency and deployment overhead.}
To evaluate the data dependency of SeerNet, we analyze the relationship between dataset scale and prediction accuracy, keeping the test set size fixed. 
Results (Table \ref{tab:data dependency}) show that accuracy decreases as the dataset scale shrinks. Nevertheless, SeerNet achieves a mean MAPE of 14.98 with only 1,000 samples, demonstrating low data dependency.
The deployment overhead includes 16.67 GPU hours for data collection and 0.05 GPU hours for training per 1,000 samples, resulting in low deployment overhead.

\emph{Usuage overhead.}
We evaluated the overhead of SeerPerf on an Intel i7-11700 CPU, which includes representation and prediction. 
The average representation latency is 248 ms, with prediction latencies of 2.0 ms for SeerNet and 2.1 ms for SeerNet-Multi. The total overhead of approximately 250 ms is acceptable for most applications. \emph{Overall}, SeerPerf demonstrates low overhead in both deployment and usage.

% \emph{The experiments conducted in this section conclude that PerfSeer is an efficient and accurate performance predictor, characterized by its broad application scope, low construction and usage overhead, and high prediction accuracy.}

\section{Discussion}

\paragraph{Quadratic Programming vs. Logistic Regression.}  
Our formulation estimates the attribute weights $\mathbf{p}$ by transforming the Bradley-Terry loss into a quadratic program. An alternative approach based on logistic regression—which assigns absolute labels of 1 and 0 to win/lose responses—can also be used, as demonstrated by \citep{go2023compositional}. 
We compared these two formulations using Drift attributes in Table~\ref{fig:discussion}. The logistic regression approach proves highly unstable and shows lower performance when training examples are limited. We interpret this instability as follows: preference judgments are inherently relative—what constitutes a winning response in one context might be considered a losing response when compared to an even better alternative. Thus, imposing absolute labels through regression can lead to overfitting, particularly when data are scarce. Our results suggest that approaching preference problems from a relative perspective is crucial for effective preference modeling.
\begin{figure}[ht]
\centering
\includegraphics[trim=7 8 2 2, clip, width=0.65\columnwidth]{figs/discussion.pdf}
\caption{Few-shot preference modeling results for \texttt{user1008} in the PRISM with quadratic programming (QP) and logistic regression (LQ).}
\label{fig:discussion}
\vspace{-5mm}
\end{figure}


\paragraph{Compatible with samplers.}
\label{sec:practical-2}
Autoregressive sampling in LLMs has various decoding strategies at the token-level distribution. Drift steers distributions at the logit level—applying its computations before the softmax—making it compatible with a wide range of sampling methods tailored to different objectives~\citep{vijayakumar2016diverse, fan2018hierarchical, holtzman2019curious}. our analysis indicates that the backbone LLM exhibits an average next-token entropy of about 0.27 bits, which increases to approximately 0.63 bits after applying Drift. While this boost in entropy can substantially enhance generation diversity, it may also increase the likelihood of selecting unreliable tokens. Therefore, we recommend combining Drift with top-p or top-k sampling strategies to control an optimal balance between diversity and reliability.

\paragraph{Practical Implications.}
While traditional RLHF methods may eventually surpass Drift when user data becomes abundant, Drift offers several advantages in practical settings. 
First, conventional reward models struggle with \textit{continual learning}; retraining on an ever-expanding user dataset is impractical. In contrast, Drift can be updated instantly by simply appending new instances to the $W-L$—no retraining required. 
Second, personal preferences often \textit{change more rapidly than general preferences}. Drift’s interpretability allows real-time tracking of preference shifts, enabling dynamic adjustments for improved personalization. 
Third, when collecting additional user annotations, the variance observed in each attribute can inform an \textit{active learning} strategy~\citep{miller2020active} for efficient data collection. These benefits make Drift an attractive complement to existing RLHF pipelines in personalized applications.

\section{Conclusion}

This work analysed the results of evolutionary wrapper approaches using decision tree based models as proxies and compared them with common \gls{FE} techniques on a \gls{HL} detection problem. Three experiments were conducted using the proposed framework, each employing different proxy models.

When comparing the three experiments, an interesting behaviour of the framework was discovered, when changing the proxy model. The \gls{DT} experiment drastically reduced the number of features, while the other models did not. To further reduce the number of features, one could bias the grammar or apply some penalty in the fitness function for the individuals that use a large number of features. This might not change the behaviour when using different models other than a \gls{DT}, but it forcefully reduces the number of features.  

The results confirm that FEDORA can reduce the dimensionality of the data while statistically maintaining baseline performance, in every experiment. The framework consistently outperforms the remaining \gls{FE} methods, with statistical significance and large effect sizes, proving itself as a viable alternative.

The best result obtained is 76.2\% balanced accuracy using an individual from the \gls{RF} experiment, and a \gls{XGB} algorithm as the testing model, using 57 total features (45 Original, 6 Engineered and 6 Complex) out of the 60 original ones. When using the least amount of features, the best result is 72,8\% balanced accuracy using an individual from the \gls{DT} experiment and a \gls{RF} algorithm as the testing model, using a single complex feature.

In future work, exploring the above-mentioned behaviours might be relevant to better understanding them, namely when biasing the grammar or penalizing the use of many features in the fitness function. Concerning the explainability of the FEDORA transformations, researching meaningful grammar operators might prove useful in addressing problem-specific needs. In this case, having logical operators for the boolean features, which have values of "yes" or "no", and the choice of a simple decision algorithm as the proxy, may increase explainability. Additionally, the previous study has identified several areas for future research, yet to be addressed. For instance, comparing the framework with other common and more complex methods and completing the full \gls{ML} pipeline through the use of a method that addresses the \gls{CASH}, such as \cite{assunccao2020evolution}, and comparing it to other full pipeline frameworks, could be beneficial for contextualizing and evaluating the framework within the \gls{AutoML} and \gls{EC} domains. The framework still needs to be analysed with different datasets to properly assess its generalization capabilities.

%%
%% The acknowledgments section is defined using the "acks" environment
%% (and NOT an unnumbered section). This ensures the proper
%% identification of the section in the article metadata, and the
%% consistent spelling of the heading.
% \begin{acks}
% To Robert, for the bagels and explaining CMYK and color spaces.
% \end{acks}

\begin{acks}
% This work was supported by the Research Council of Finland (flagship program: Finnish Center for Artificial Intelligence, FCAI, grants 328400, 345604, 341763; Human Automata, grant 328813; Subjective Functions, grant 357578), the ERC AdG project Artificial User (101141916), and ELLIS Mobility Grant. 
This work was supported by the Research Council of Finland project Subjective Functions (grant 357578), 
Finnish Center for Artificial Intelligence (grants 328400, 345604, 341763),
European Research Council Advanced Grant (no. 101141916),
the Department of Information and Communications Engineering at Aalto University, 
and ELLIS Mobility Grant. 
Y. Wang was funded by the Deutsche Forschungsgemeinschaft~(DFG, German Research Foundation)~-~Project-ID 251654672~-~TRR~161. 
Y. Bai was supported by NUS research scholarship and ORIA program.
A. Bulling was funded by the European Research Council (ERC; grant agreement 801708).
\end{acks}

%%
%% The next two lines define the bibliography style to be used, and
%% the bibliography file.
\bibliographystyle{ACM-Reference-Format}
\bibliography{reference}

%%
%% If your work has an appendix, this is the place to put it.
% \appendix

% \section{Research Methods}

% \subsection{Part One}

% Lorem ipsum dolor sit amet, consectetur adipiscing elit. Morbi
% malesuada, quam in pulvinar varius, metus nunc fermentum urna, id
% sollicitudin purus odio sit amet enim. Aliquam ullamcorper eu ipsum
% vel mollis. Curabitur quis dictum nisl. Phasellus vel semper risus, et
% lacinia dolor. Integer ultricies commodo sem nec semper.

% \subsection{Part Two}

% Etiam commodo feugiat nisl pulvinar pellentesque. Etiam auctor sodales
% ligula, non varius nibh pulvinar semper. Suspendisse nec lectus non
% ipsum convallis congue hendrerit vitae sapien. Donec at laoreet
% eros. Vivamus non purus placerat, scelerisque diam eu, cursus
% ante. Etiam aliquam tortor auctor efficitur mattis.

% \section{Online Resources}

% Nam id fermentum dui. Suspendisse sagittis tortor a nulla mollis, in
% pulvinar ex pretium. Sed interdum orci quis metus euismod, et sagittis
% enim maximus. Vestibulum gravida massa ut felis suscipit
% congue. Quisque mattis elit a risus ultrices commodo venenatis eget
% dui. Etiam sagittis eleifend elementum.

% Nam interdum magna at lectus dignissim, ac dignissim lorem
% rhoncus. Maecenas eu arcu ac neque placerat aliquam. Nunc pulvinar
% massa et mattis lacinia.

\end{document}
\endinput
%%
%% End of file `sample-sigconf.tex'.
