%%
%% This is file `sample-sigconf.tex',
%% generated with the docstrip utility.
%%
%% The original source files were:
%%
%% samples.dtx  (with options: `sigconf')
%% 
%% IMPORTANT NOTICE:
%% 
%% For the copyright see the source file.
%% 
%% Any modified versions of this file must be renamed
%% with new filenames distinct from sample-sigconf.tex.
%% 
%% For distribution of the original source see the terms
%% for copying and modification in the file samples.dtx.
%% 
%% This generated file may be distributed as long as the
%% original source files, as listed above, are part of the
%% same distribution. (The sources need not necessarily be
%% in the same archive or directory.)
%%
%% Commands for TeXCount
%TC:macro \cite [option:text,text]
%TC:macro \citep [option:text,text]
%TC:macro \citet [option:text,text]
%TC:envir table 0 1
%TC:envir table* 0 1
%TC:envir tabular [ignore] word
%TC:envir displaymath 0 word
%TC:envir math 0 word
%TC:envir comment 0 0
%%
%%
%% The first command in your LaTeX source must be the \documentclass command.
% \documentclass[sigconf]{acmart}
%% NOTE that a single column version may be required for 
%% submission and peer review. This can be done by changing
%% the \doucmentclass[...]{acmart} in this template to 
% \documentclass[manuscript,screen]{acmart}
% \documentclass[sigconf,review,anonymous]{acmart}
% \documentclass[anonymous,manuscript,screen,review]{acmart}
% \documentclass[manuscript,review,anonymous]{acmart}
\documentclass[sigconf]{acmart}
%% 
%% To ensure 100% compatibility, please check the white list of
%% approved LaTeX packages to be used with the Master Article Template at
%% https://www.acm.org/publications/taps/whitelist-of-latex-packages 
%% before creating your document. The white list page provides 
%% information on how to submit additional LaTeX packages for 
%% review and adoption.
%% Fonts used in the template cannot be substituted; margin 
%% adjustments are not allowed.
%%
%%
%% \BibTeX command to typeset BibTeX logo in the docs
\AtBeginDocument{%
  \providecommand\BibTeX{{%
    \normalfont B\kern-0.5em{\scshape i\kern-0.25em b}\kern-0.8em\TeX}}}

%% Rights management information.  This information is sent to you
%% when you complete the rights form.  These commands have SAMPLE
%% values in them; it is your responsibility as an author to replace
%% the commands and values with those provided to you when you
%% complete the rights form.
\copyrightyear{2025}
\acmYear{2025}
\setcopyright{cc}
\setcctype{by}
\acmConference[CHI '25]{CHI Conference on Human Factors in Computing Systems}{April 26-May 1, 2025}{Yokohama, Japan}
\acmBooktitle{CHI Conference on Human Factors in Computing Systems (CHI '25), April 26-May 1, 2025, Yokohama, Japan}\acmDOI{10.1145/3706598.3713128}
\acmISBN{979-8-4007-1394-1/25/04}

%%
%% Submission ID.
%% Use this when submitting an article to a sponsored event. You'll
%% receive a unique submission ID from the organizers
%% of the event, and this ID should be used as the parameter to this command.
\acmSubmissionID{9728}

%%
%% For managing citations, it is recommended to use bibliography
%% files in BibTeX format.
%%
%% You can then either use BibTeX with the ACM-Reference-Format style,
%% or BibLaTeX with the acmnumeric or acmauthoryear sytles, that include
%% support for advanced citation of software artefact from the
%% biblatex-software package, also separately available on CTAN.
%%
%% Look at the sample-*-biblatex.tex files for templates showcasing
%% the biblatex styles.
%%

%%
%% The majority of ACM publications use numbered citations and
%% references.  The command \citestyle{authoryear} switches to the
%% "author year" style.
%%
%% If you are preparing content for an event
%% sponsored by ACM SIGGRAPH, you must use the "author year" style of
%% citations and references.
%% Uncommenting
%% the next command will enable that style.
%%\citestyle{acmauthoryear}

%%
%% end of the preamble, start of the body of the document source.
\usepackage{color}
\usepackage{colortbl}
% \usepackage{amssymb}
\usepackage{xcolor}
\usepackage{makecell}
\usepackage{multirow}
\usepackage{multicol}
\usepackage{xspace}
\usepackage{hyperref}
\newcommand{\name}
{\textsc{Chartist}\xspace}
% {\textsc{ChartGaze}\xspace}
% {\textsc{VisReader}\xspace}
% {\textsc{ChartReader}\xspace}
% {\textsc{ChartObserver}\xspace}
\usepackage{subfigure}
\usepackage{pifont}
\usepackage{xcolor}
\newcommand{\cmark}{\textcolor{green!80!black}{\ding{51}}}
\newcommand{\xmark}{\textcolor{red}{\ding{55}}}

% revision
\usepackage{ifthen}
\newboolean{revising}
\setboolean{revising}{false}
\ifthenelse{\boolean{revising}}
{
    \newcommand{\rv}[1]{\textcolor{blue}{#1}}
} {
    \newcommand{\rv}[1]{#1}
}

\begin{document}
\title{\rv{\name: Task-driven Eye Movement Control for Chart Reading}}
% \title{How People Read Charts: A Model of Task-driven Eye Movement Control}
% \title{Modeling Task-driven Scanpaths on Charts}
% \title{Task-driven Human Attention Prediction for Chart Question Answering}

%%
%% The "author" command and its associated commands are used to define
%% the authors and their affiliations.
%% Of note is the shared affiliation of the first two authors, and the
%% "authornote" and "authornotemark" commands
%% used to denote shared contribution to the research.

\author{Danqing Shi}
\orcid{0000-0002-8105-0944}
\affiliation{%
  \institution{Aalto University}
  \state{Helsinki}
  \country{Finland}}
\author{Yao Wang}
\orcid{0000-0002-3633-8623}
\affiliation{%
  \institution{University of Stuttgart}
  \state{Stuttgart}
  \country{Germany}}
\author{Yunpeng Bai}
\orcid{0009-0008-7578-0079}
\affiliation{%
  \institution{National University of Singapore}
  \state{Singapore}
  \country{Singapore}}
\author{Andreas Bulling}
\orcid{0000-0001-6317-7303}
\affiliation{%
  \institution{University of Stuttgart}
  \state{Stuttgart}
  \country{Germany}}
\author{Antti Oulasvirta}
\orcid{0000-0002-2498-7837}
\affiliation{%
  \institution{Aalto University}
  \state{Helsinki}
  \country{Finland}}

%%
%% By default, the full list of authors will be used in the page
%% headers. Often, this list is too long, and will overlap
%% other information printed in the page headers. This command allows
%% the author to define a more concise list
%% of authors' names for this purpose.
\renewcommand{\shortauthors}{Shi et al.}



% \begin{teaserfigure}
%   \centering
%   \includegraphics[width=0.9\textwidth]{Images/teaser-scanpath.png}
%   \caption{We present \name, a computational model that can predict task-driven human scanpaths on charts. The figure demonstrates three analytical tasks involved in the study: retrieve value, filter, and find extreme. The visualization illustrates how models' predictions vary across tasks and match the pattern of human scanpaths, with fixation density maps overlaid.}
%   \label{fig:teaser}
% \end{teaserfigure}

% \begin{figure}[b]
% \noindent\fbox{%
% \parbox{\dimexpr\linewidth-2\fboxsep-2\fboxrule\relax}{%
% \begin{tabular}{l}
% The word count of this paper is \textcolor{blue}{7357}.
% \end{tabular}
% }%
% }
% \end{figure}

%%
%% The abstract is a short summary of the work to be presented in the
%% article

\begin{abstract}
Recent advancements in 3D multi-object tracking (3D MOT) have predominantly relied on tracking-by-detection pipelines. However, these approaches often neglect potential enhancements in 3D detection processes, leading to high false positives (FP), missed detections (FN), and identity switches (IDS), particularly in challenging scenarios such as crowded scenes, small-object configurations, and adverse weather conditions. Furthermore, limitations in data preprocessing, association mechanisms, motion modeling, and life-cycle management hinder overall tracking robustness. To address these issues, we present \textbf{Easy-Poly}, a real-time, filter-based 3D MOT framework for multiple object categories. Our contributions include: (1) An \textit{Augmented Proposal Generator} utilizing multi-modal data augmentation and refined SpConv operations, significantly improving mAP and NDS on nuScenes; (2) A \textbf{Dynamic Track-Oriented (DTO)} data association algorithm that effectively manages uncertainties and occlusions through optimal assignment and multiple hypothesis handling; (3) A \textbf{Dynamic Motion Modeling (DMM)} incorporating a confidence-weighted Kalman filter and adaptive noise covariances, enhancing MOTA and AMOTA in challenging conditions; and (4) An extended life-cycle management system with adjustive thresholds to reduce ID switches and false terminations. Experimental results show that Easy-Poly outperforms state-of-the-art methods such as Poly-MOT and Fast-Poly~\cite{li2024fast}, achieving notable gains in mAP (e.g., from 63.30\% to 64.96\% with LargeKernel3D) and AMOTA (e.g., from 73.1\% to 74.5\%), while also running in real-time. These findings highlight Easy-Poly's adaptability and robustness in diverse scenarios, making it a compelling choice for autonomous driving and related 3D MOT applications. The source code of this paper will be published upon acceptance.

% 3D Multi-Object Tracking (MOT) is essential for autonomous driving systems, contributing significantly to vehicle safety and navigation. Despite recent advancements, existing 3D tracking methods still face significant challenges in accuracy, particularly when dealing with small objects, crowded environments, and adverse weather conditions. To overcome these challenges, we propose \textbf{Easy-Poly}, a novel and efficient multi-modal 3D MOT framework. \textbf{Easy-Poly} employs the Focal Sparse Convolution (\textbf{FocalsConv}) model for object detection. By optimizing convolution operations and augmenting data with multiple modalities, we significantly enhance detection precision.
% \textbf{Easy-Poly} introduces several key innovations: (1) an optimized Kalman filter in the pre-processing stage, (2) integration of the Dynamic Track-Oriented (\textbf{DTO}) Data Association algorithm with confidence-weighted motion models for data association, (3) Dynamic Motion Modeling (\textbf{DMM}) with Adaptive Noise Covariances, and (4) enhanced trajectory management throughout the tracking life-cycle. These improvements increase the robustness and efficiency of tracking, especially in complex scenarios such as crowded scenes and challenging weather conditions. Experimental results on the \textbf{nuScenes} dataset demonstrate that in the pre-processing stage of \textbf{Easy-Poly}, the optimized \textbf{FocalsConv} model achieves a mean Average Precision (mAP) of \textbf{64.96\%} for object detection. Furthermore, the multi-object tracking performance reaches a high AMOTA of \textbf{75.0\%}, surpassing existing methods across multiple performance metrics.
 
% Code and data are available at \textcolor{blue}{\textit{\url{https://github.com/zhangpengtom/FocalsConvPlus}}} and  \textcolor{blue}
%  \textit{\url{https://github.com/zhangpengtom/EasyPoly}.}
%  } 

\end{abstract}
%%
%% This command processes the author and affiliation and title
%% information and builds the first part of the formatted document.
\maketitle

\section{Introduction}

In sensor networks, maintaining data freshness is crucial to support diverse applications such as environmental monitoring, industrial automation, and smart cities \cite{kandris2020applications}. A critical metric for quantifying data freshness is the Age of Information (AoI), which measures the time elapsed since the last received update was generated \cite{yates2012}. Minimizing AoI is essential in dynamic environments, where obsolete information can result in inaccurate decisions or missed opportunities. Efficient AoI management involves balancing update frequency, data relevance, and network resource constraints to ensure decision-makers have timely and accurate information when required \cite{yates2021age}. The significance of AoI has led to extensive research on its optimization across various domains, including single-server systems with one or multiple sources \cite{modiano2015,mm1,sun2016,najm2018,soysal2019,9137714,yates2019,zou2023costly}, scheduling strategies \cite{modiano-sch-1,9007478,sch-igor-1,9241401,sch-li,sch-sun}, and analysis of resource-constrained systems \cite{const-ulukus,const-biyikoglu,const-arafa,const-farazi,const-parisa}. 

%\ali{A good transition here would be: one particular area that has been garnering focus by the AoI researchers and that is correalted systems. In fact, sensor networks often handle...}

Among the strategies for AoI minimization, packet preemption is regarded as a cornerstone approach for ensuring the timeliness of information in communication networks, especially when resources such as service rates are limited \cite{yates2021age}. By prioritizing critical updates, preemption ensures that the most valuable data reaches its destination promptly, as demonstrated in the context of single-sensor, memoryless systems \cite{kaul2012status,inoue2019general}. Beyond this specific scenario, numerous studies have extensively investigated its role in optimizing AoI across diverse settings. For example, \cite{maatouk2019age} analyzes systems with prioritized information streams sharing a common server, where lower-priority packets may be buffered or discarded. Similarly, \cite{wang2019preempt} and \cite{kavitha2021controlling} examine preemption strategies for rate-limited links and lossy systems, identifying in the process the optimal policies for minimizing the AoI.

On the other hand, one particular area that has been garnering focus among AoI researchers is correlated systems. In fact, sensor networks often handle correlated data streams, where relationships between data collected by different sensors can be leveraged to enhance decision-making, reduce redundancy, and improve overall system performance \cite{mahmood2015reliability,yetgin2017survey}. This correlation often arises when multiple sensors monitor overlapping areas or related phenomena, allowing them to collaboratively exchange information and optimize resource usage. The role of correlation in sensor networks has further been explored in studies focusing on its potential to optimize system efficiency and effectiveness \cite{he2018,tong2022,popovski2019,modiano2022,ramakanth2023monitoring,erbayat2024}.











% The importance of AoI and correlation in sensor networks has motivated extensive research into optimizing AoI within correlated sensor systems. For example, \cite{he2018} studied sensor networks with overlapping fields of view, proposing a joint optimization framework for fog node assignment and transmission scheduling to reduce the AoI of multi-view image data. Similarly, \cite{tong2022} focused on overlapping camera networks, introducing scheduling algorithms for multi-channel systems designed to minimize AoI. Other works, such as \cite{popovski2019, modiano2022}, leveraged probabilistic correlation models to formulate sensor scheduling strategies aimed at lowering AoI. Additionally, \cite{ramakanth2023monitoring} treated the correlation of status updates as a discrete-time Wiener process, developing a scheduling policy that balances AoI reduction with monitoring accuracy. Furthermore, \cite{erbayat2024} analyzed the impact of optimal correlation probabilities under varying environmental conditions, addressing the interplay between error minimization and AoI.

%\ali{On the other hand, Preemption in AoI systems has been widely studied...Also, Id say reduce the size of this paragraph} 



%\ali{I don't like this transition here. Talk about correlated systems in the previous paragraph and how AoI is of interest. Then, switch here to preemption is still open question. Do not focus on your paper as you did here}
As part of ongoing efforts in this area, the potential of leveraging interdependencies between sensors to reduce the AoI in correlated systems has been studied, but the benefits and challenges of employing preemption in multi-sensor systems with correlated data streams remain an open question. While preemption is a potential strategy to minimize AoI in a network, it is not always the optimal strategy \cite{yates2019}. This approach must account for the specific features of the packets being transmitted since preempting leads to prioritization. For example, a sensor with a lower arrival rate may track a unique process that no other sensor monitors, making its packets particularly valuable and critical to retain. On the other hand, preempting a packet from a sensor with a high arrival rate may not significantly reduce AoI, as the frequent updates from such sensors diminish the impact of losing a single packet.


%\ali{Here you make the connection between preemption and multi-sensor correlated systems}

%\ali{Its good to emphasize that we have correlation here so it is different than typical AoI system}.

To address this gap, this paper introduces adaptable and probabilistic preemption mechanisms that dynamically balance priorities across sensors, considering their unique correlation characteristics and resource demands. To that end, the main contributions of this paper are summarized as follows:

%To address these challenges, we propose a system where the ability of a packet to preempt an ongoing transmission depends on its source, allowing for a more adaptable approach to managing updates. We also introduce the concept of probabilistic preemption, where preemption decisions are guided by source-specific probabilities rather than fixed or deterministic rules. This probabilistic method improves efficiency by giving higher-priority updates a better chance to preempt, keeping the information more up-to-date. By incorporating stochastic hybrid system modeling, we derive a closed-form expression for the AoI, providing a theoretical foundation to analyze the impact of probabilistic preemption on network performance. Building on this system, we explore how varying preemption probabilities can influence the total AoI in multi-sensor systems, considering the interplay between diverse sensors and their shared resources. Furthermore, we establish that the problem of deciding optimal preemption strategies can be framed as a sum of linear ratios problem. We derive an upper bound on the number of iterations required using a branch-and-bound algorithm, ensuring computational efficiency in identifying optimal solutions. Through this analysis, we identify optimal preemption strategies that minimize the total AoI, balancing the timeliness and relevance of updates across all monitored processes to achieve an efficient and well-coordinated system.

%Interestingly, the results show how the system adjusts priorities between sensors to keep the AoI as low as possible. For example, if one sensor spreads its updates more evenly across multiple processes, the system tends to rely on it more, even if another sensor is sending updates less often. As arrival rates or service rates change, the system shifts its strategy to stay efficient.\footnote{Due to size limitations, we present the proof details in \url{https://github.com/erbayat/xxxx}}.


\begin{itemize}
    \item As a first step, we propose a system where the ability of a packet to preempt an ongoing transmission probabilistically depends on its source rather than being fixed or following deterministic rules. Subsequently, using stochastic hybrid system modeling, we derive a closed-form expression for AoI to analyze the impact of probabilistic preemption on network performance.
    
    %enabling a more adaptable approach to manage updates by giving higher-priority updates a better chance to preempt, ensuring information remains up-to-date.

    \item Following that, we investigate optimizing the total AoI in multi-sensor systems, considering the interplay between diverse sensors and shared resources. Building on this, we frame the problem of deciding optimal preemption strategies as a sum of linear ratios problem, which is generally an NP-Hard problem\cite{freund2001solving}. However, by analyzing its unique characteristics, we derive an upper bound on the number of iterations required to identify optimal preemption strategies using a branch-and-bound algorithm, thus ensuring computational efficiency in finding the optimal solution.
    %\ali{You are using a lot the , ensuring... it sounds very chatgpt liky, try to minimize those when possible. Also, talk about the bounds and the impact of these results on getting an efficient solution}
    \item Lastly, we validate our findings with numerical results and evaluate optimal preemption strategies to minimize AoI. Our findings demonstrate how correlation influences preemption strategies. Notably, when a source provides a lower aggregate number of updates while distributing them more evenly, the system prioritizes it for preemption, even if another sensor updates less frequently.\ifthenelse{\boolean{withappendix}}
{}
{\footnote{Due to space limitations, we present the proof details in \cite{technicalNote}.}}
 %\ali{Dont forget to put the right link}
\end{itemize}


%These results not only support the theory but also offer practical ideas for real-world use, such as in IoT networks, factories, or autonomous systems, where staying up-to-date is very important.

%The remainder of this paper is structured as follows. Section \ref{system-model} introduces the system model and key assumptions. In Section \ref{aoi-S}, we derive the closed-form expression for the AoI within the proposed system. Section \ref{aoi-opt} outlines the optimization problem and details the process of determining the optimal preemption probabilities. The numerical results are presented in Section \ref{numerical}, and the paper concludes with a summary and discussion in Section \ref{conc}.



\section{Related Work}

\subsection{Large Language Models in Biosciences}
Large language models (LLMs) have emerged as powerful tools for natural language comprehension and generation~\cite{llms-survey}. Beyond their application in traditional natural language tasks, there is a growing interest in leveraging LLMs to accelerate scientific research. Early studies revealed that general-purpose LLMs, owing to their rich pre-training data, exhibit promise across various research domains~\cite{ai4science}. Subsequent efforts have focused on directly training LLMs using domain-specific data, aiming to extend the transfer learning paradigm from natural language processing (NLP) to biosciences. This body of work primarily falls into three categories: molecular LLMs, protein LLMs, and genomic LLMs.

For molecular modeling, extensive work has been conducted on training with various molecular string representations, such as SMILES~\cite{Smiles-bert,space-of-chemical,large-scale-chemical}, SELFIES~\cite{SELFIES,chemberta,chemberta2}, and InChI~\cite{inchi}. Additionally, several studies address the modeling of molecular 2D~\cite{mol-2d} and 3D structures~\cite{uni-mol} to capture more detailed molecular characteristics. In the realm of protein LLMs, related work~\cite{msa-transformer,esm2,Prottrans} mainly concentrates on modeling the primary structure of proteins (amino acid sequences), providing a solid foundation for protein structure prediction~\cite{AlphaFold2,AlphaFold3}. For genomic sequences, numerous studies have attempted to leverage the power of LLMs for improved genomic analysis and understanding. These efforts predominantly involve training models on DNA~\cite{BPNet,DNABERT,enformer,nucleotide-transformer,DNABERT-2,GROVER,gena-lm,Caduceus,dnagpt,megaDNA,HyenaDNA,Evo} and RNA~\cite{RNAErnie,uni-rna,Rinalmo} sequences. In the following section, we delve deeper into genomic LLMs specifically designed for DNA sequence modeling.

\subsection{DNA Language Models}
In the early stages, \citeauthor{BPNet} introduced the BPNet convolutional architecture to learn transcription factor binding patterns and their syntax in a supervised manner. Prior to the emergence of large-scale pre-training, BPNet was widely used in genomics for supervised learning on relatively small datasets. With the advent of BERT~\cite{BERT}, DNABERT~\cite{DNABERT} pioneered the application of pre-training on the human genome using K-mer tokenizers. To effectively capture long-range interactions, Enformer~\cite{enformer} advanced human genome modeling by incorporating convolutional downsampling into transformer architectures.

Following these foundational works, numerous models based on the transformer encoder architecture have emerged. A notable example is the Nucleotide Transformer (NT)~\cite{nucleotide-transformer}, which scales model parameters from 100 million to 2.5 billion and includes a diverse set of multispecies genomes. Recent studies, DNABERT-2~\cite{DNABERT-2} and GROVER~\cite{GROVER}, have investigated optimal tokenizer settings for masked language modeling, concluding that Byte Pair Encoding (BPE) is better suited for masked DNA LLMs. The majority of these models face the limitation of insufficient context length, primarily due to the high computational cost associated with extending the context length in the transformer architecture. To address this limitation, GENA-LM~\cite{gena-lm} employs sparse attention, and Caduceus~\cite{Caduceus} uses the more lightweight BiMamba architecture~\cite{Mamba}, both trained on the human genome.

Although these masked DNA LLMs effectively understand and predict DNA sequences, they lack generative capabilities, and generative DNA LLMs remain in the early stages of development. An early preprint~\cite{dnagpt} introduced DNAGPT, which learns mammalian genomic structures through three pre-training tasks, including next token prediction. Recent works, such as HyenaDNA~\cite{HyenaDNA} and megaDNA~\cite{megaDNA}, achieve longer context lengths by employing the Hyena~\cite{Hyena} and multiscale transformer architectures respectively, though they are significantly limited by their data and model scales. A more recent influential study, Evo~\cite{Evo}, trained on an extensive dataset of prokaryotic and viral genomes, has garnered widespread attention for its success in designing CRISPR-Cas molecular complexes, thus demonstrating the practical utility of generative DNA LLMs in the genomic field.


\section{\underline{V}ision \underline{L}anguage \underline{D}isinformation Detection \underline{Bench}mark}  
\label{method}  
\textsf{\textbf{\textsc{VLDBench}}} (Figure~\ref{fig:vlbias}) is a comprehensive classification multimodal benchmark for disinformation detection in news articles. It comprises 31,339 articles and visual samples curated from 58 news sources ranging from the Financial Times, CNN, and New York Times to Axios and Wall Street Journal as shown in Figure \ref{fig:news_sources_distribution}. \textsf{\textbf{\textsc{VLDBench}}} spans 13 unique categories (Figure \ref{fig:news_categories}) : \textit{National, Business and Finance, International, Entertainment, Local/Regional, Opinion/Editorial, Health, Sports, Politics, Weather and Environment, Technology, Science, and Other} —adding depth to the disinformation domains. We present the further statistical details in Appendix \ref{app:data-analysis}.
 
\subsection{Task Definition}
\textbf{Disinformation Detection:}  
The core task is to determine whether a text–image news article contains disinformation. We adopt the following definition:  

\emph{“False, misleading, or manipulated information—textual or visual—intentionally created or disseminated to deceive, harm, or influence individuals, groups, or public opinion.”}  

This definition aligns with established social science research \cite{benkler2018network} and governance frameworks \cite{unesco2023journalism}. We specifically focus on the `intent' behind disinformation, which remains relevant over time but has broader effects beyond just being factually incorrect.
\begin{figure*}
    \centering
    \includegraphics[width=0.95\textwidth]{figures/qualitative_figure.pdf}
    \vspace{-1em}
\caption{Disinformation Trends Across News Categories: We analyze the likelihood of disinformation across different categories, based on disinformation narratives and confidence levels generated by GPT-4o.}
    \vspace{-1em}
    \label{fig:disinfo-analysis}
\end{figure*}
\subsection{Data Pipeline}  
\textbf{Dataset Collection:}  
From May 6, 2023, to September 6, 2023, we aggregated data via Google RSS feeds from diverse news sources (Table~\ref{tab:sources}), adhering to Google’s terms of service \cite{google_tos}. We carefully curated  high-quality visual samples from these news sources to ensure a diverse representation of topics. All data collection complied with ethical guidelines \cite{uwaterloo_ethics_review}, regarding intellectual property and privacy protection. 

\textbf{Quality Assurance.}  
Collected articles underwent a rigorous human review and pre-processing phase. First, we removed entries with incomplete text, low-resolution or missing images, duplicates, and media-focused URLs (e.g., \texttt{/video}, \texttt{/gallery}). Articles with fewer than 20 sentences were discarded to ensure textual depth. For each article, the first image was selected to represent the visual context. We periodically reviewed the quality of the curated data to ensure the API returned valid and consistent results. These steps yielded over 31k curated text-image news articles that are moved to the annotation pipeline.

\subsection{Annotation Pipeline}  
To the best of our knowledge, \textsf{\textbf{\textsc{VLDBench}}} is the largest and most comprehensive humanly verified disinformation detection benchmark with over 300 hours of human verification. Figure \ref{fig:vlbias} shows our semi-annotated, data collection and annotation pipeline. After quality assurance, each article was prompted and categorized by GPT-4o as either \texttt{Likely} or \texttt{Unlikely} to contain disinformation, a binary choice designed to balance nuance with manageability. To ensure reliability, GPT-4o assessed text-image alignment three times per sample—first, to minimize random variance in its responses, and second, to resolve potential ties in classification, with an odd number of evaluations ensuring a definitive outcome. GPT-4o was chosen for this task due to its demonstrated effectiveness in both textual \cite{kim2024meganno+} and visual reasoning tasks \cite{shahriar2024putting}. An example is shown in Figure \ref{fig:disinfo-analysis}.

We categorized our data into 13 unique news categories (Figure \ref{fig:news_categories}) by providing image-text pairs to GPT-4o, drawing inspiration from AllSlides \cite{allsides_mediabiaschart}  and frameworks like Media Cloud \cite{media_cloud}. The dataset statistics are given in Table~\ref{tab:dataset_statistics}.

\begin{figure}[ht]
    \centering
    \includegraphics[width=0.48\textwidth]{figures/news_categories_distribution.pdf}
    \caption{Category distribution with overlaps. Total unique articles = 31,339. Percentages sum to \(>100\%\) due to multi-category articles.}
    \label{fig:news_categories}
    \vspace{-1em}
\end{figure}

To ensure high-quality benchmarking, a team of 22 domain experts (Appendix~\ref{app:team}) systematically reviewed the GPT-4o labels and rationales, assessing their accuracy, consistency, and alignment with human judgment. This process included a rigorous structured reconciliation phase, refining annotation guidelines and finalizing the labels. The evaluation resulted in a Cohen’s $\kappa$ of 0.78, indicating strong inter-annotator agreement.


\paragraph{Stability of Automatic Annotations:} To assess the reliability of automated annotations, we conducted a controlled experiment comparing GPT-4o labels with those of human annotators. We randomly selected 1,000 GPT-4o-annotated samples from the previous step, provided annotation guidelines, and asked domain experts (without showing the GPT-4o labels) to manually annotate them. Comparing both sets of labels, GPT-4o achieved an F1 score of 0.89 and an MCC of 0.77, while human annotators scored F1 = 0.92 and MCC = 0.81 (Figure~\ref{fig:alignment_metrics}). These results demonstrate the effectiveness of our semi-annotated pipeline, aligning well with human judgment and ensuring reliable automated labeling.

\begin{table*}[!t]
    \centering
    \resizebox{0.8\textwidth}{!}{
    \begin{tabular}{@{}l|c|c|c|c|c||c@{}}
        \toprule
        & \makecell{MATRES} & \makecell{TB-Dense} & \makecell{TCR} & \makecell{TDD-Manual} & \makecell{NarrativeTime} & \makecell{\textbf{\App{}}} \\
        \midrule
        \multicolumn{7}{c}{\textbf{Datasets Statistics}} \\
        \midrule
        Documents & 275 & 36 & 25 & 34 & 36 & 30 \\
        Events & 6,099 & 1,498 & 1,134 & 1,101 & 1,715 & 470 \\
        \midrule
        \textit{before} & 6,852 (50) & 1,361 (21) & 1,780 (67) & 1,561 (25) & 17,011 (22) & 1,540 (44) \\
        \textit{after} & 4,752 (35) & 1,182 (19) & 862 (33) & 1,054 (17) & 18,366 (23) & 1,347 (39) \\
        \textit{equal} & 448 (4) & 237 (4) & 4 (0) & 140 (2) & 5,298 (7) & 150 (4) \\
        \textit{vague} & 1,525 (11) & 2,837 (45) & -- & -- & 25,679 (33) & 446 (13) \\
        \textit{includes} & -- & 305 (5) & -- & 2,008 (33) & 5,781 (7) & -- \\
        \textit{is-included} & -- & 383 (6) & -- & 1,387 (23) & 6,639 (8) & -- \\
        \textit{overlaps} & -- & -- & -- & -- & 227 (0) & -- \\
        \midrule
        Total Relations & 13,577 & 6,305 & 2,646 & 6,150 & 79,001 & 3,483 \\
        \midrule
        \multicolumn{7}{c}{\textbf{Per Document Average Annotation Sparsity}} \\
        \midrule
        Events & 22.2 & 41.6 & 45.4 & 32.4 & 47.6 & 15.6 \\
        Actual Relations & 49.4 & 183.7 & 105.8 & 180.9 & 1,110.1 & 114.9 \\
        Expected Relations & 234.8 & 844.5 & 1,006.1 & 508.1 & 1,110.1 & 114.9 \\
        \midrule
        Missing Relations & 79\% & 78.3\% & 89.5\% & 64.4\% & 0\% & 0\% \\
        \bottomrule
    \end{tabular}}
    \caption{The upper part of the table presents the statistics of notable datasets for the temporal relation extraction task alongside \App{}. In parentheses, the values indicate the percentage of each relation type relative to the total relations in the dataset. The bottom part of the table summarizes the average percentage of missing relations per document, calculated as the ratio of actual annotated relations to a complete relation coverage, referred to as \textit{Expected Relations}.}
    \label{tab:stats_all}
\end{table*}


% \begin{table*}[!t]
%     \centering
%     \resizebox{0.8\textwidth}{!}{
%     \begin{tabular}{@{}l|c|c|c|c|c|c@{}}
%         \toprule
%         & \makecell{MATRES} & \makecell{TBD} & \makecell{TCR} & \makecell{TDD-Man} & \makecell{NarrativeTime} & \makecell{\App{}} \\
%         \midrule
%         Docs & 275 & 36 & 25 & 34 & 36 & 30 \\
%         Events & 6,099 & 1,498 & 1,134 & 1,101 & 1,715 & 470 \\
%         \midrule
%         Before (\%) & 6,852 (50) & 1,361 (21) & 1,780 (67) & 1,561 (25) & 17,011 (22) & 1,540 (44) \\
%         After (\%) & 4,752 (35) & 1,182 (19) & 862 (33) & 1,054 (17) & 18,366 (23) & 1,347 (39) \\
%         Equal (\%) & 448 (4) & 237 (4) & 4 (0) & 140 (2) & 5,298 (7) & 150 (4) \\
%         Vague (\%) & 1,525 (11) & 2,837 (45) & -- & -- & 25,679 (33) & 446 (13) \\
%         Includes (\%) & -- & 305 (5) & -- & 2,008 (33) & 5,781 (7) & -- \\
%         IsIncluded (\%) & -- & 383 (6) & -- & 1,387 (23) & 6,639 (8) & -- \\
%         Overlaps (\%) & -- & -- & -- & -- & 227 (0) & -- \\
%         \midrule
%         Total Rels & 13,577 & 6,305 & 2,646 & 6,150 & 79,001 & 3,483 \\
%         \bottomrule
%     \end{tabular}}
%     \caption{Statistics of notable datasets for the temporal relation extraction task.}
%     \label{tab:stats}
% \end{table*}



\section{Experiments}
\label{sec:evaluation}

This section presents the experiments conducted for evaluating and comparing scanpath prediction models. We evaluated \name specifically in terms of scanpath similarity and statistical summaries of eye movement behavior. The results from evaluation of our model in comparison to the baselines are summarized in Table~\ref{tab:benchmark}.

\subsection{Data and Metrics}

We evaluated \name by using 12 distinct analytical tasks with horizontal bar charts from a task-driven scanpath dataset~\cite{polatsek2018exploring}. 
Each task has at least 14 human scanpaths (\textit{M} = 15.25, \textit{SD} = 0.60), for a total of 183 human scanpaths. 
\rv{
This set of ground-truth human data was collected by means of Tobii X2-60 eye trackers at 60~Hz while participants were engaged in these analytical tasks on 24.1-inch monitors at a resolution of 1920 × 1080 pixels.
}
To evaluate model performance, we compared the generated scanpaths against human ground truth, with the aim of ascertaining whether the models can produce human-like scanpaths and replicate natural gaze movement patterns. 
We should stress that, to ensure unbiased evaluation, none of the human eye movement data informed our training of the models, only their assessment.
Furthermore, except for scanpath prediction that relies on human-generated data as the training set, no such eye movement data were involved in training for our approach.

Following recent practice in scanpath prediction~\cite{wang2023scanpath, sui2023scandmm}, we evaluated \name by using three established scanpath-based metrics: dynamic time warping (DTW), Levenshtein distance (LEV), and Sequence Score.
DTW computes the optimal alignment between two scanpaths, where lower values indicate better correspondence. For this paper, DTW was calculated in two-dimensional position coordinates.
Both LEV and Sequence Score represent the semantic order of scanpaths as sequences of letters by mapping each fixation to a unique letter, then measuring the string-editing distance between the sequences~\cite{needleman1970general}. 
In LEV, letters are defined by the grid regions on which fixations land. 
For Sequence Score~\cite{yang2020predicting, wang2023scanpath}, letters are based on areas of interest (AOIs) such as the title and legend.
Sequence Score values are normalized between 0 and 1, with higher scores reflecting better alignment.
For DTW, LEV, and Sequence Score, we report the \textit{mean} and \textit{best} evaluation scores (see Table 2). For each method, the number of scanpaths equaled that of human scanpaths. The \textit{mean} scores are the averages across all human--predicted scanpath pairs, while the \textit{best} ones represent the maximum of all pairs for each prediction~\cite{chen2021predicting, wang2023scanpath}.

\begin{table}[t]
    \centering
    \caption{The performance of different pre-trained models on ImageNet and infrared semantic segmentation datasets. The \textit{Scratch} means the performance of randomly initialized models. The \textit{PT Epochs} denotes the pre-training epochs while the \textit{IN1K FT epochs} represents the fine-tuning epochs on ImageNet \citep{imagenet}. $^\dag$ denotes models reproduced using official codes. $^\star$ refers to the effective epochs used in \citet{iBOT}. The top two results are marked in \textbf{bold} and \underline{underlined} format. Supervised and CL methods, MIM methods, and UNIP models are colored in \colorbox{orange!15}{\rule[-0.2ex]{0pt}{1.5ex}orange}, \colorbox{gray!15}{\rule[-0.2ex]{0pt}{1.5ex}gray}, and \colorbox{cyan!15}{\rule[-0.2ex]{0pt}{1.5ex}cyan}, respectively.}
    \label{tab:benchmark}
    \centering
    \scriptsize
    \setlength{\tabcolsep}{1.0mm}{
    \scalebox{1.0}{
    \begin{tabular}{l c c c c  c c c c c c c c}
        \toprule
         \multirow{2}{*}{Methods} & \multirow{2}{*}{\makecell[c]{PT \\ Epochs}} & \multicolumn{2}{c}{IN1K FT} & \multicolumn{4}{c}{Fine-tuning (FT)} & \multicolumn{4}{c}{Linear Probing (LP)} \\
         \cmidrule{3-4} \cmidrule(lr){5-8} \cmidrule(lr){9-12} 
         & & Epochs & Acc & SODA & MFNet-T & SCUT-Seg & Mean & SODA & MFNet-T & SCUT-Seg & Mean \\
         \midrule
         \textcolor{gray}{ViT-Tiny/16} & & &  & & & & & & & & \\
         Scratch & - & - & - & 31.34 & 19.50 & 41.09 & 30.64 & - & - & - & - \\
         \rowcolor{gray!15} MAE$^\dag$ \citep{mae} & 800 & 200 & \underline{71.8} & 52.85 & 35.93 & 51.31 & 46.70 & 23.75 & 15.79 & 27.18 & 22.24 \\
         \rowcolor{orange!15} DeiT \citep{deit} & 300 & - & \textbf{72.2} & 63.14 & 44.60 & 61.36 & 56.37 & 42.29 & 21.78 & 31.96 & 32.01 \\
         \rowcolor{cyan!15} UNIP (MAE-L) & 100 & - & - & \underline{64.83} & \textbf{48.77} & \underline{67.22} & \underline{60.27} & \underline{44.12} & \underline{28.26} & \underline{35.09} & \underline{35.82} \\
         \rowcolor{cyan!15} UNIP (iBOT-L) & 100 & - & - & \textbf{65.54} & \underline{48.45} & \textbf{67.73} & \textbf{60.57} & \textbf{52.95} & \textbf{30.10} & \textbf{40.12} & \textbf{41.06}  \\
         \midrule
         \textcolor{gray}{ViT-Small/16} & & & & & & & & & & & \\
         Scratch & - & - & - & 41.70 & 22.49 & 46.28 & 36.82 & - & - & - & - \\
         \rowcolor{gray!15} MAE$^\dag$ \citep{mae} & 800 & 200 & 80.0 & 63.36 & 42.44 & 60.38 & 55.39 & 38.17 & 21.14 & 34.15 & 31.15 \\
         \rowcolor{gray!15} CrossMAE \citep{crossmae} & 800 & 200 & 80.5 & 63.95 & 43.99 & 63.53 & 57.16 & 39.40 & 23.87 & 34.01 & 32.43 \\
         \rowcolor{orange!15} DeiT \citep{deit} & 300 & - & 79.9 & 68.08 & 45.91 & 66.17 & 60.05 & 44.88 & 28.53 & 38.92 & 37.44 \\
         \rowcolor{orange!15} DeiT III \citep{deit3} & 800 & - & 81.4 & 69.35 & 47.73 & 67.32 & 61.47 & 54.17 & 32.01 & 43.54 & 43.24 \\
         \rowcolor{orange!15} DINO \citep{dino} & 3200$^\star$ & 200 & \underline{82.0} & 68.56 & 47.98 & 68.74 & 61.76 & 56.02 & 32.94 & 45.94 & 44.97 \\
         \rowcolor{orange!15} iBOT \citep{iBOT} & 3200$^\star$ & 200 & \textbf{82.3} & 69.33 & 47.15 & 69.80 & 62.09 & 57.10 & 33.87 & 45.82 & 45.60 \\
         \rowcolor{cyan!15} UNIP (DINO-B) & 100 & - & - & 69.35 & 49.95 & 69.70 & 63.00 & \underline{57.76} & \underline{34.15} & \underline{46.37} & \underline{46.09} \\
         \rowcolor{cyan!15} UNIP (MAE-L) & 100 & - & - & \textbf{70.99} & \underline{51.32} & \underline{70.79} & \underline{64.37} & 55.25 & 33.49 & 43.37 & 44.04 \\
         \rowcolor{cyan!15} UNIP (iBOT-L) & 100 & - & - & \underline{70.75} & \textbf{51.81} & \textbf{71.55} & \textbf{64.70} & \textbf{60.28} & \textbf{37.16} & \textbf{47.68} & \textbf{48.37} \\ 
        \midrule
        \textcolor{gray}{ViT-Base/16} & & & & & & & & & & & \\
        Scratch & - & - & - & 44.25 & 23.72 & 49.44 & 39.14 & - & - & - & - \\
        \rowcolor{gray!15} MAE \citep{mae} & 1600 & 100 & 83.6 & 68.18 & 46.78 & 67.86 & 60.94 & 43.01 & 23.42 & 37.48 & 34.64 \\
        \rowcolor{gray!15} CrossMAE \citep{crossmae} & 800 & 100 & 83.7 & 68.29 & 47.85 & 68.39 & 61.51 & 43.35 & 26.03 & 38.36 & 35.91 \\
        \rowcolor{orange!15} DeiT \citep{deit} & 300 & - & 81.8 & 69.73 & 48.59 & 69.35 & 62.56 & 57.40 & 34.82 & 46.44 & 46.22 \\
        \rowcolor{orange!15} DeiT III \citep{deit3} & 800 & 20 & \underline{83.8} & 71.09 & 49.62 & 70.19 & 63.63 & 59.01 & \underline{35.34} & 48.01 & 47.45 \\
        \rowcolor{orange!15} DINO \citep{dino} & 1600$^\star$ & 100 & 83.6 & 69.79 & 48.54 & 69.82 & 62.72 & 59.33 & 34.86 & 47.23 & 47.14 \\
        \rowcolor{orange!15} iBOT \citep{iBOT} & 1600$^\star$ & 100 & \textbf{84.0} & 71.15 & 48.98 & 71.26 & 63.80 & \underline{60.05} & 34.34 & \underline{49.12} & \underline{47.84} \\
        \rowcolor{cyan!15} UNIP (MAE-L) & 100 & - & - & \underline{71.47} & \textbf{52.55} & \underline{71.82} & \textbf{65.28} & 58.82 & 34.75 & 48.74 & 47.43 \\
        \rowcolor{cyan!15} UNIP (iBOT-L) & 100 & - & - & \textbf{71.75} & \underline{51.46} & \textbf{72.00} & \underline{65.07} & \textbf{63.14} & \textbf{39.08} & \textbf{52.53} & \textbf{51.58} \\
        \midrule
        \textcolor{gray}{ViT-Large/16} & & & & & & & & & & & \\
        Scratch & - & - & - & 44.70 & 23.68 & 49.55 & 39.31 & - & - & - & - \\
        \rowcolor{gray!15} MAE \citep{mae} & 1600 & 50 & \textbf{85.9} & 71.04 & \underline{51.17} & 70.83 & 64.35 & 52.20 & 31.21 & 43.71 & 42.37 \\
        \rowcolor{gray!15} CrossMAE \citep{crossmae} & 800 & 50 & 85.4 & 70.48 & 50.97 & 70.24 & 63.90 & 53.29 & 33.09 & 45.01 & 43.80 \\
        \rowcolor{orange!15} DeiT3 \citep{deit3} & 800 & 20 & \underline{84.9} & \underline{71.67} & 50.78 & \textbf{71.54} & \underline{64.66} & \underline{59.42} & \textbf{37.57} & \textbf{50.27} & \underline{49.09} \\
        \rowcolor{orange!15} iBOT \citep{iBOT} & 1000$^\star$ & 50 & 84.8 & \textbf{71.75} & \textbf{51.66} & \underline{71.49} & \textbf{64.97} & \textbf{61.73} & \underline{36.68} & \underline{50.12} & \textbf{49.51} \\
        \bottomrule
    \end{tabular}}}
    \vspace{-2mm}
\end{table}

\rv{
We looked beyond scanpath metrics, introducing more detailed measurements inspired by \citet{goldberg2010comparing} to show a statistical summary of task-driven scanpaths over charts. % The analysis of human data can be summarized thus:
\begin{itemize}
    \item \textit{Number of fixations}: The length of a scanpath can be measured as the count of gaze fixations (between motions, or saccades). According to the human data, the number of fixations in task-driven scanpaths over charts (89.8 on average) is much larger than the number in free-viewing tasks (37.4 on average). This reflects the difficulty of analytical tasks relative to free viewing of charts.
    \item \textit{Fixation on task-dependent AOI ratio}: Task-dependent AOIs are regions that are relevant to the task, such as value labels, text labels, and data points~\cite{polatsek2018exploring}. People's focus on these areas indicates how they are processing the task. Inspired by the Hit Any AOI Rate metric~\cite{wang22_etvis}, this measurement provides a summary of the overall visual attention to task-related regions. Although the scanpath is task-driven, we ascertained that most of the eye movement does not occur in task-dependent regions: fewer than 20\% of fixations fell in task-dependent AOIs. This suggests that people might devote more time to gathering information or confirming it.
    \item \textit{Percentage of fixations within each area}: We considered the percentage of time devoted to looking at distinct parts of a chart -- namely, the key areas of charts: the title, the marks (such as bars or data points), and the axes. This assists in summarizing where people are focusing their visual attention. The percentages are calculated by dividing the number of fixations in a specific area by the total number of fixations. Humans direct most of their fixations to the axes, then the region of graphical marks. This might be because the three analysis tasks probed are strongly related to values, not other visual features.
    \item \textit{Fixation transitions}: We also used a metric capturing the average number of times the eyes move from one area to another during a task. It helps us understand how often the eyes' fixations shift between distinct areas. Frequent fixation transitions may point to room for improvement in the design of the chart, such as bringing related elements closer together. From human data, we identified a high number of fixation transitions (about 20 per task). We found that, on average, about four consecutive fixations follow each fixation transition.
    \item \textit{Revisit frequencies}: The average number of fixations returning to a previously visited area during a task proved similarly revealing. Human data exhibited high revisit rates. Spatially, users revisit marks and also axes eight times, on average. This frequent double-checking of data information in the chart for the answer may be due to forgetting the information.
\end{itemize}
These metrics help us evaluate whether the model's predictions can accurately capture general human patterns followed with charts for particular tasks. We strove for a system in which the predicted scanpath closely matches human ground-truth performance, ideally being within one standard deviation of the mean value.
}

\subsection{Comparison Methods}

Given the lack of existing methods for predicting task-driven scanpaths on information visualizations, we compare \name against human ground truth with three closely related baselines:

\begin{itemize}
    \item \textit{Human}~\cite{polatsek2018exploring}. With the scanpath metrics, we conducted leave-one-out cross-validation among the human scanpaths. For each viewing condition, every human scanpath was compared with all other human scanpaths for similarity. Human scanpaths were compared with themselves for the \textit{mean} scores but not for contributions to the \textit{best} scores. In applying the statistical metrics, we treated all the human data as the ground truth and gauged all modeling methods by their closeness to this ground truth.
    \item \textit{VQA scanpaths}~\cite{chen2021predicting}. VQA is a deep reinforcement learning model that predicts human visual scanpaths in the context of images with visual question answering. The paper reporting on it demonstrates its strong generalizability across various tasks and datasets, indicating optionality as an approach for predicting task-driven scanpaths over charts. % It allows for comparing stimuli effects on scanpaths.
    \item \textit{UMSS}~\cite{wang2023scanpath}. UMSS represents the state-of-the-art scanpath prediction model for visualizations, making it the most relevant work in this area. However, it is designed to predict scanpaths in a free-viewing context for information visualizations, rather than consider specific tasks. Its inclusion allows for comparison between scanpath prediction with and without task-linked factors.
    \item \textit{DeepGaze iii}~\cite{kummerer2022deepgaze}. DeepGaze iii is a deep-learning-based model that integrates image data with information about previous fixations to forecast free-viewing scanpaths over static images. Trained on large sets of eye tracking data from natural images, it serves as a baseline for evaluating the effect both of stimuli and of tasks on scanpaths.
\end{itemize}


\subsection{Results}

\begin{figure*}[!t]
\centering
  \includegraphics[width=0.95\textwidth]{Images/examples.png}
  \caption{Qualitative comparison: for three tasks, an illustration of \name's predictions relative to three baselines -- VQA scanpaths~\cite{chen2021predicting}, UMSS~\cite{wang2023scanpath}, and DeepGaze iii~\cite{kummerer2022deepgaze}. \name is able to capture human scanpath patterns displayed during analytical tasks.}
  \label{fig:examples}
\end{figure*}

% \paragraph{Scanpath similarity}
\paragraph{\name demonstrates high similarity in scanpaths}

The first six rows for each task type in \autoref{tab:benchmark} present the results from our three scanpath similarity metrics.
\name achieved the highest performance by the Sequence Score and LEV metrics, and it ranked second for DTW, with scores closely approaching the maximum and also close to human ground truth. 
Specifically, \name closely approximates the latter Sequence Score in terms of mean performance, achieving a score of 0.413, relative to 0.486.
The UMSS method, while securing first place for DTW, ranked second for LEV and third for Sequence Score.
\name outperforms human ground truth in LEV \textit{mean} (151.8 vs. 154.7). This means that the predictions from \name deviate less from the ``average human scanpath.''

The results from the scanpath metrics show that \name performed better than the baselines by the Sequence Score and LEV metrics, which are based on regions, but not the DTW metric, which is based on pixel-wise distances. This suggests that \name is more similar to human data when one factors in the semantic order of fixation positions in meaningful portions of charts. However, it may not fully match human data for pixel-level similarity.
This result is consistent with the discussion in the literature~\cite{wang2023scanpath}, which has concluded that metrics based on pixel-wise distances between scanpaths might not wholly capture the quality of human scanpaths. Therefore, we must conduct further analysis of the statistical summary of eye movement behaviors.

% \paragraph{Statistical summary of eye movement behaviors}
\paragraph{\name aligns more closely with human statistical patterns}

The last nine rows for each task type in \autoref{tab:benchmark} provide the mean and standard deviation for each eye movement behavior. 
Because UMSS and DeepGaze iii are not task-driven, our analysis used the same predicted scanpaths across all tasks.
\name achieves strong alignment with human data, with all 27 of its values for the eye movement behavior metrics falling within one standard deviation of the human mean and with 18 of them being the closest to the human data's mean. In comparison, 18 of UMSS's 27 values lie within one standard deviation, and five of them are the closest to the mean. The corresponding figures for DeepGaze iii are 13 out of 27 and 4, respectively, while VQA yielded only six values within the range and only one of the 27 was the closest to the human mean.

Examining the detailed metrics across tasks reveals that humans show significantly variable task-dependent AOI ratios. They devote the majority of their fixations to task AOIs when performing the \textit{F} task (19.1\%). That is followed by the \textit{RV} task (10.4\%), with the \textit{FE} task having the lowest percentage (1.7\%). This distribution makes sense: the first two tasks require individuals to focus on a specific data label, while \textit{FE} can be completed by directly observing the general shape of the graph. \name is the only model that successfully replicates this phenomenon by reproducing the human order of task-dependent AOI ratios: {FE} (10.6\%), then {RV} (4.1\%), and finally {FE} (0.2\%). 
As for per-region fixation ratios, humans direct the most fixations to axes, followed by marks, across all three tasks. \name successfully reproduces this phenomenon in the case of the \textit{RV} task. For the \textit{F} and \textit{FE} tasks, \name shows similar distributions. In contrast, the VQA and UMSS baselines consistently allocate over 50\% of fixations to the marks, and DeepGaze iii allocates most fixations to the title. 
In the realm of revisits, \name and DeepGaze iii align with human data, revisiting the axes most frequently, while the other two models revisit the marks most often. 
In summary, our analysis demonstrates that \name exhibits the pattern most similar to human data.

\paragraph{Qualitative analysis}

Figure~\ref{fig:examples} showcases predicted scanpaths from \name and the three baseline models across six tasks, with fixation density maps overlaid. In all cases, our model's predictions are closer to the human data than the baseline models'. \name and VQA scanpaths both are task-driven, unlike UMSS and DeepGaze iii's, which cannot predict scanpaths solely from images. Here are the main observations from Figure~\ref{fig:examples}:
\begin{itemize}
    \item DeepGaze iii predicts scanpaths from the bottom up on the basis of the saliency of a natural image. That results in a high number of fixations on the title, consistent with producing the highest fixation-on-title ratios. 
    \item Although UMSS is not task-driven, its predicted scanpath shape closely mirrors human data. This indicates that, even in task-driven scenarios, bottom-up mechanisms exert a significant influence. 
    \item Nevertheless, \name, as does VQA, captures human gaze patterns more effectively across tasks than these free-viewing models. Importantly, \name outperforms the VQA scanpath model, for VQA often predicts fixating on irrelevant areas, in the absence of specific knowledge of visualization structures.
\end{itemize}

The examples in Figure~\ref{fig:teaser} demonstrate how \name's predictions compared to human data for the three tasks. The chart read displays a list of top-ranked theme parks worldwide with their corresponding attendance numbers for 2011. 
When given the \textit{RV} task of answering ``what is the attendance level of Universal Studios Hollywood?'' both the human user and the model focus on the text labels to find the theme park in the chart and the relevant positions for the mark and on the value axis.
For the \textit{F} task, the human user and the model both frequently look at the value axis.
Regarding the \textit{FE} task, both human and model focus on the top of the mark and also fixate on the text label associated with that mark. 
We noted that human eye movements are also drawn to other text labels, such as annotations and textual descriptions, while \name remains task-focused without getting distracted by unrelated information.
\section{Discussion}
\label{sec:discussion}

\rv{
While the results show that \name is able to simulate human-like eye movements when performing analytical tasks, there is a need to expand on our discussion of the model's practical implications, the generalizability of the modeling approach, and the limitations and potential for supporting sophisticated chart-based question answering.
}

\subsection{\rv{Applications}}

\rv{
\paragraph{Visualization design evaluation}
\name can assist in evaluation of chart design.
With well-controlled experiment conditions, eye tracking data afford valuable insight into chart designs, especially relative to alternative designs.
For example, \citet{goldberg2011eye} showcased eye tracking's value in comparing line and radial graphs for reading of values, by allowing researchers to understand the viewing order of AOIs and the task completion time.
\name holds potential to replace human input to evaluation based on eye tracking.
With the simulated scanpaths from \name, chart designers can obtain quick and cost-effective feedback that yields the benefits from eye tracking without requiring an expensive empirical study.
}
\rv{
\paragraph{Visualization design optimization}
Beyond evaluation, another potential usage application of \name is to help optimize visualization design~\cite{shin2023perceptual}. 
Like other fields of design, visualization design requires user feedback for continual iteration. When visualization designers create charts for specific tasks, they may wonder if the design is suitable for delivery.
With the predicted scanpaths from the model, they can easily access quick and affordable feedback before deeming a candidate design ready for expensive evaluation in a user study.
Predictive models could offer feedback to designers or even provide optimization goals in automated visualization design frameworks.
The ultimate goal is grounding for recommendations for visualizations that support specific tasks~\cite{albers2014task} and even automation of visualization design in real time.
Today's human-in-the-loop design optimization paradigm~\cite{kadner2021adaptifont} could shift to a user-agent-in-the-loop approach, wherein a computational agent that simulates human feedback enables scalable and efficient design evaluation.
}

\rv{
\paragraph{Explainable AI in chart question answering}
Systems for answering questions via charts~\cite{masry2022chartqa} are typically viewed as black boxes that generate answers directly from a given chart and natural-language question. 
In contrast, \name introduces a glass-box approach that answers questions through a step-by-step reasoning process. This method enhances the alignment between human and machine attention~\cite{sood2023multimodal}.
We anticipate that this approach could lead to significant improvements in chart question answering~\cite{masry2022chartqa} and greater compatibility with explainable AI systems.
}

\subsection{\rv{Extending the Model beyond Bar Charts}}

\rv{
Our modeling approach can be extended to many visualization types besides bar charts.
We analyzed the visualization taxonomy outlined in prior work~\cite{borkin2013makes, borkin2015beyond}, including area, circle, diagram, distribution, grid, line, map, point, table, text, tree, and network, then categorize these visualization techniques into two groups: those that are feasible to extend with minor changes and those that are out of reach, requiring additional features.
}

\begin{figure}[!t]
    \centering
    \subfigure[\rv{An \textit{RV} task with a line chart: ``What was the revenue from newspaper advertising in 1980?''}]{\label{fig:a}\includegraphics[width=0.48\textwidth]{Images/line-case.png}}
    \hspace{0.02\columnwidth}
    \subfigure[\rv{An \textit{F} task with a scatterplot: ``In which countries do people anticipate spending about \$700 for personal Christmas gifts?''}]{\label{fig:b}\includegraphics[width=0.48\textwidth]{Images/point-case.png}}
    \caption{\rv{Two cases that illustrate the generalizability of the modeling approach, showing the extension of \name to a line chart and a scatterplot. The model's predictions are spatially similar to human ground-truth scanpaths.}}
    \label{fig:case}
    % \vspace{-5mm}
\end{figure}

\rv{
Our modeling approach can be applied to most statistical charts either directly or upon rectification of minor issues. For instance, extending the model to interpret \textit{line charts} and \textit{area charts} is feasible when the axis labels are clearly defined. The trend patterns of lines and areas can be perceived by the peripheral vision as visual guidance.
For \textit{point charts}, such as scatterplots, the model performs well in conditions of sparse data points. However, individual points may be obscured in dense scatterplots, making it difficult to label data when points are cluttered or overlapping. 
\textit{Distribution charts}, such as histograms, and \textit{circle charts}, such as pie charts, are similar to bar charts in that they use the area of marks to represent values. Retrieving exact values from these two presentation types can be imprecise on account of the ranges of the bins and inaccuracies in estimating angles or arc lengths.
Reading \textit{grid charts} (e.g., heatmaps) too is feasible; however, identifying the values necessitates understanding color intensity, a factor that can sometimes lead to ambiguity.
Modeling scanpaths on \textit{tables} or \textit{text} for retrieval tasks is tractable under the current modeling approach, but a lack of visual pattern recognition may render the results poor.
To further examine the generalizability of this category, we considered two additional cases, using a line chart and a scatterplot. We manually labeled the charts, trained the model, and made predictions. As Figure~\ref{fig:case} attests, the trained model performs well for these two chart types when compared to human ground-truth scanpaths.
}

\rv{
Other, sophisticated visualization types are out of reach because they require additional features, particularly prior knowledge and advanced reasoning abilities. For instance, reading \textit{maps} involves associating spatial regions with colors, sizes, or symbols to retrieve related values. Also, when interpreting maps, people rely heavily on preexisting geographical knowledge as a basis for efficient visual searches. Complex designs with intricate structures, such as \textit{diagrams}, \textit{trees}, and \textit{network graphs}, typically require advanced reasoning based on connections. All these skill requirements point to a need for further study in this area.
}

\subsection{\rv{Paths toward Sophisticated Tasks in Chart Question Answering}}

\rv{
Although the model focuses primarily on gaze prediction, it is worth exploring potential improvements for enriching its sophisticated question answering capabilities. We also discuss its limitations.
}

\rv{
Our current model does not achieve the same level of accuracy as the state-of-the-art models represented by the ChartQA benchmark~\cite{masry2022chartqa}. 
Unlike other models that can access the full chart image, \name is limited by its foveal vision and restricted spatial reasoning abilities. For instance, if a bar's height falls between two labeled values, such as 10 and 15, the model might choose either 10 or 15 as its answer when interpreting the axis, failing to provide a more precise value. 
This limitation stems from the constrained spatial perception capabilities of LLMs, which are central to cognitive control.
One possible solution is integrating multi-modal LLMs~\cite{cuarbune2024chart}, for which recent research has demonstrated an accuracy rate of 81.3\%.
}

\rv{
The sense-making process for complex visualizations may be inherently challenging. Even humans often struggle with understanding how the data are encoded, recognizing a given chart's purpose, tackling readability issues, performing numerical calculations, identifying relationships among data points, and navigating the spatial arrangement of graphical elements~\cite{rezaie2024struggles}.
Our model is designed to be straightforward and objective, focusing on analysis tasks related to statistical charts, but it does not fully capture the complexities of visualizations.
A possible enhancement in this respect would be to integrate the model with human sense-making practices~\cite{rezaie2024struggles} or to incorporate a framework of human understanding~\cite{albers2014task}. Such integration could facilitate better simulation of a human-like problem-solving process.
}

\section{Conclusion}
In this study, we introduce \ours, a novel framework designed to achieve lossless acceleration in generating ultra-long sequences with \acp{llm}. By analyzing and addressing three challenges, \ours significantly enhances the efficiency of the generation process. Our experimental results demonstrate that \ours achieves over $3\times$ acceleration across various model scales and architectures. Furthermore, \ours effectively mitigates issues related to repetitive content, ensuring the quality and coherence of the generated sequences. These advancements position \ours as a scalable and effective solution for ultra-long sequence generation tasks.


%%
%% The acknowledgments section is defined using the "acks" environment
%% (and NOT an unnumbered section). This ensures the proper
%% identification of the section in the article metadata, and the
%% consistent spelling of the heading.
% \begin{acks}
% To Robert, for the bagels and explaining CMYK and color spaces.
% \end{acks}

\begin{acks}
% This work was supported by the Research Council of Finland (flagship program: Finnish Center for Artificial Intelligence, FCAI, grants 328400, 345604, 341763; Human Automata, grant 328813; Subjective Functions, grant 357578), the ERC AdG project Artificial User (101141916), and ELLIS Mobility Grant. 
This work was supported by the Research Council of Finland project Subjective Functions (grant 357578), 
Finnish Center for Artificial Intelligence (grants 328400, 345604, 341763),
European Research Council Advanced Grant (no. 101141916),
the Department of Information and Communications Engineering at Aalto University, 
and ELLIS Mobility Grant. 
Y. Wang was funded by the Deutsche Forschungsgemeinschaft~(DFG, German Research Foundation)~-~Project-ID 251654672~-~TRR~161. 
Y. Bai was supported by NUS research scholarship and ORIA program.
A. Bulling was funded by the European Research Council (ERC; grant agreement 801708).
\end{acks}

%%
%% The next two lines define the bibliography style to be used, and
%% the bibliography file.
\bibliographystyle{ACM-Reference-Format}
\bibliography{reference}

%%
%% If your work has an appendix, this is the place to put it.
% \appendix

% \section{Research Methods}

% \subsection{Part One}

% Lorem ipsum dolor sit amet, consectetur adipiscing elit. Morbi
% malesuada, quam in pulvinar varius, metus nunc fermentum urna, id
% sollicitudin purus odio sit amet enim. Aliquam ullamcorper eu ipsum
% vel mollis. Curabitur quis dictum nisl. Phasellus vel semper risus, et
% lacinia dolor. Integer ultricies commodo sem nec semper.

% \subsection{Part Two}

% Etiam commodo feugiat nisl pulvinar pellentesque. Etiam auctor sodales
% ligula, non varius nibh pulvinar semper. Suspendisse nec lectus non
% ipsum convallis congue hendrerit vitae sapien. Donec at laoreet
% eros. Vivamus non purus placerat, scelerisque diam eu, cursus
% ante. Etiam aliquam tortor auctor efficitur mattis.

% \section{Online Resources}

% Nam id fermentum dui. Suspendisse sagittis tortor a nulla mollis, in
% pulvinar ex pretium. Sed interdum orci quis metus euismod, et sagittis
% enim maximus. Vestibulum gravida massa ut felis suscipit
% congue. Quisque mattis elit a risus ultrices commodo venenatis eget
% dui. Etiam sagittis eleifend elementum.

% Nam interdum magna at lectus dignissim, ac dignissim lorem
% rhoncus. Maecenas eu arcu ac neque placerat aliquam. Nunc pulvinar
% massa et mattis lacinia.

\end{document}
\endinput
%%
%% End of file `sample-sigconf.tex'.
