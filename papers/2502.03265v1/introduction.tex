\section{Introduction}

We consider time dependent multiphysics aplications, or more specifically coupled PDEs that interact through a lower dimensional interface. Examples here are gas-quenching (heat), flutter in airplanes (forces) or climate models (both). To simulate these problems, it is often desirable to reuse existing codes for the different models, since they represent long term development work. This is called a partitioned approach. To increase computational efficiency we want a partitioned method that is high order, allows for different and adaptive time steps in the separate models, is robust to variations in discretization and model parameters, and contains fast inner solvers.

A major candidate to fulfill our requirements is the so called Waveform Iteration or Waveform relaxation (WR). This class of methods was developed to simulate large electrical circuits effectively in parallel \cite{WhSa86}. To this end, the underlying large system of ODEs is split into smaller ODE systems, which are solved separately given approximations of the solutions of the other subsystems. This way, an iterative process is defined. This framework naturally allows the use of different time integration methods in each subsystem. Using interpolation, different time steps can be employed in a straight forward manner. 

Convergence of waveform iterations has been studied in various continuous and discrete settings for ODEs, see for example \cite{Ne89Cont, Ne89Disc, WhSa86}. The time adaptive case, where the time grids in the sub solver change in each iteration, was analyzed in \cite{BeZe93}. WR methods were also extended to PDEs, see e.g. \cite{JaVa96a, JaVa96b,gakwma:16, BiMo19}. Unlike the ODE case, waveform iterations are not guaranteed to converge for the PDE case. 

To speed up convergence, an acceleration step is often applied. The simplest acceleration method is relaxation, which has been used in \cite{BiMo18} to achieve fast convergence for heat transfer problems. However, to obtain fast linear or even superlinear convergence, the relaxation parameter has to be optimized for the specific problem, requiring extensive analysis beforehand, see e.g. \cite{BiMo19}. A time adaptive discretization has been combined with relaxation in \cite{BiMeMo23} for coupled heat equations. Alternatively, one can use a black box method like Quasi-Newton, also referred to as Anderson acceleration. These methods have been studied extensively, see e.g. \cite{polreb:21}, and for partitioned coupling in for example \cite{HaDe10, SchUe15, SchMe17}. They have also recently been extended to WR for the case with fixed but variable time steps in \cite{RuUe21}. 

In this article, we further extend the Quasi-Newton method to the time adaptive case. An inherent difficulty is that since the number of time steps changes with every iteration, the dimension of the discrete solution vector changes with every iteration as well. Thus, the Quasi-Newton method cannot be directly applied to the sequence of iterates. Furthermore, a variable dimension of the solution vector makes the implementation into a coupling library for partitioned coupling difficult. 

We therefore suggest to interpolate the iterates on a fixed auxiliary time grid. Thus, the iteration can be written in terms of a vector of fixed dimension, at the cost of an additional interpolation step. However, the implementation into a coupling library is perfectly feasible and we present ours in the open source C++ library preCICE \cite{ChDa22}. We then analyze the behaviour of the algorithm for fixed time grids and discuss how to choose the auxiliary time grid. We also demonstrate the methods efficiency by comparing it with optimal relaxation \cite{BiMeMo23} and QNWR for fixed time grids \cite{RuUe21} on a simple thermal transfer case.  We further showcase the method on a more advanced Fluid-structure interaction problem. 

In the following, we first describe Waveform relaxation, then in section 3 the Quasi-Newton method. We analyze the new method in section 4 and provide numerical experiments in sections 5 and 6. 

%To further extend the Quasi-Newton methods to the time adaptive case, one could use an extension of the Quasi-Newton or generalized Broyden method to operators on Hilbert spaces as done in \cite{Sa84} and \cite{MoOm86}. However, the methods described in \cite{Sa84} and \cite{MoOm86} are continuous. Thus, to get a method that can be implemented on a computer one would have to discretize them at some point affecting the convergence behaviour of the algorithm. 