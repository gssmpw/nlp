\section{Fluid-Structure interaction}

We now consider a much more complex setting, were we have different physics, discretization and solvers in our two domains. We use the test case "perpendicular flap" from the preCICE tutorials \cite{ChDa22}. It consists of an elastic beam that is mounted to the lower wall in the middle of a fluid channel, shown in Figure \ref{BeamSimulation}. 

The fluid is modeled by the incompressible Navier-Stokes equations in an arbitrary Lagrangian–Eulerian framework, and is given by
\begin{align*}	
	\frac{\partial \hat{u}}{\partial t} + (\hat{u}-u_g) \cdot \nabla \hat{u} - \nu \Delta \hat{u} &= \frac{\nabla p}{\rho_f}, \quad x \in \Omega_F(t),\\
	\nabla \hat{u} &= 0, \quad x \in \Omega_F(t), 
\end{align*}
where $\hat{u}$ is the velocity of the flow, $u_g$ the velocity of the domain and $p$ the pressure. The fluid domain, denoted by $\Omega_F(t)$, is $6$ m long and $4$ m tall and deforms based on the beams movement, as is demonstrated in Figure \ref{BeamSimulation}. For our test case, the density $\rho_f$ is given as $1 \frac{kg}{m^3}$ and viscosity $\nu$ is set to $1 m^2/s$. Furthermore, zero initial conditions are employed at the start of the simulation in the fluid. To minimize instabilities in the start up phase, the inflow speed on the left boundary is given by \begin{equation*}
\begin{cases}
	[10^4 t, 0] \text{ for } t \leq 10^{-3},\\
	[10, 0] \text{ for } t > 10^{-3}.
\end{cases}
\end{equation*} 
This is combined with a no-slip condition for the walls and the beam, as well as a zero gradient outflow condition on the right. 

The fluid solver uses OpenFOAM's pimpleFoam solver, as in the preCICE tutorial, which uses a finite volume discretization in space and BDF2 in time. Furthermore, the quadrilateral grid from the tutorial, shown in Figure \ref{BeamSimulation}, was also re-used together with openFOAM's Laplacian mesh motion solver. This is combined with OpenFoam's adaptive time stepping strategy, which adjusts the time step based on the CFL number in combination with an initial time step of $10^{-5}$.  

The beam is $0.1$ m wide and $1$ m long and is modeled using linear elasticity:
\begin{align*}
	\rho_s \frac{\partial^2 d}{\partial t^2} &=  \tau + \nabla \cdot \sigma,  \quad x \in \Omega_s,\\
	d &= 0,  \quad x \in \partial \Omega_s \setminus \Gamma.
\end{align*}
Here, $d$ is the displacement, $\tau$ the traction between the fluid and the solid, and $\rho$ the density given by $3 \cdot 10^{3} \frac{kg}{m^3}$ for this test case. Furthermore, $\sigma$ is the Cauchy stress tensor given by
\begin{equation*}
	\sigma = \lambda(\nabla \cdot d) I + \mu\left(\nabla d + (\nabla d)^T \right).
\end{equation*}
The constants $\lambda$ and $\mu$  are given by \begin{align}
	\lambda &= \frac{E \nu }{ ((1.0 + \nu)(1.0 - 2.0\nu))}, \\
	\mu &= \frac{E }{ (2.0(1.0 + \nu))},
\end{align}  where $E$ is the Young's modulus and $\nu$ the Poisson ratio, which for this test case are set to $4 \cdot 10^7 \frac{kg}{m^3}$ and $0.3$ respectively. Lastly, zero initial conditions are employed.

The beam is discretized using linear finite elements in space and SDIRK2 in time. For the spatial discretization, an equidistant triangular grid with a grid size of $10^{-2}$ was employed. Similar to the heat test case, we use an adaptive time stepping strategy based on a local error estimate $l^{n}$. Instead of a PI controller the following dead beat controller, given by
\begin{equation*}
	\Delta t^{k+1, n+1} = \Delta t^{k+1, n} \left(\frac{TOL_{TA}}{||l^{n+1}||_2}\right)^{1/2},
\end{equation*}  
was employed, together with an initial time step of $10^{-5}$. As before, we choose  $TOL_{TA}=TOL_{WR}/5$. The beam solver is implemented using the python bindings of the DUNE-FEM library, which is a powerful open source library for solving partial differential equations \cite{bastian2021}. 

To couple the fluid solver and solid solver, we use the following transmission conditions, given by \begin{align*}
	\frac{\partial d}{\partial t} &= u,  \\
	\tau &= (-p + \nu \left(\nabla u + (\nabla u)^T \right). 
\end{align*}  To solve the coupled problem we use a Dirichlet-Neumann waveform iteration, given by \begin{equation*}
		d^{k+1} = S \circ F(d^k),
\end{equation*}  
where $F$ denotes the fluid solver, which maps the displacements $d$ of the beam to forces on the beams surface. The solid solver is denoted by $S$ and maps the forces from the fluid to the displacement of the beam. 

On the software side, we use preCICE togheter with the OPEN Foam adapter \cite{chourdakis2023} and the python bindings to couple the solid and fluid solver together. The code for the subsolvers and the preCICE configuration file is available under \cite{code}. Furthermore, since we use non matching spatial meshes in the fluid and solid solver, a nearest-neighbor mapping was used to interpolate the data between the solid and the fluid mesh. This allows us to select reasonable spatial grids inside the sub solvers without regarding the spatial grid of the other solver. 

To further simplify the test case, only one time window with a size of $0.2$ was used, which is larger than the maximal desired time step for both the fluid and solid solver. Furthermore, an absolute convergence criterion given as \begin{equation*}
	\sqrt{\Sigma_i r^{n^2}_i \Delta x} \leq TOL_{WR}
\end{equation*} was employed, where $r^n_i$ denotes the residual of the last time step in node $i$.  
 
We now compare the number of iterations for the time adaptive method against no acceleration. To avoid excessive computation times, the maximal number of iterations was set to 10. While this choice is quite strict, diverging simulations tend to use excessively small time steps, resulting in long simulation times. We do three runs with a coarse, medium and fine tolerance of $TOL_{WR} = 5 \cdot 10^{-3}, 5 \cdot 10^{-4}, 5 \cdot 10^{-5}$ and $CFL = 100, 10, 1$ respectively. The resulting number of iterations are shown in Table \ref{beamResults} and the simulation results for the coarsest test case are shown in Figure ~\ref{BeamSimulation}. 

\begin{table}
	\begin{center}
		\begin{tabular}{ |c|c|c| } 
			\hline
			Test-case & time adaptive QNWR & No acceleration \\ \hline
			Coarse & 2 & 2\\ \hline
			Medium & 3 & 3 \\ \hline 
			Fine & 6 & - \\
			\hline
		\end{tabular}
	\end{center}
	\caption{Number of iterations to reach the termination criteria for $T_f = 0.2$. The symbol - denotes that the simulation did not converge within 10 iterations. }
	\label{beamResults}
\end{table} 

\begin{figure}
	\centering
	\begin{subfigure}[b]{0.45\textwidth}
		\centering
		\includegraphics[width=\textwidth]{Graphs/BeamResults/BeamPosition.png}
		\caption{Displacement of the top of the beam during the simulation. }
	\end{subfigure}
	\begin{subfigure}[b]{0.45\textwidth}
		\centering
		\includegraphics[width=\textwidth]{Graphs/BeamResults/Deformation0.2.png}
		\caption{Deformation of the fluid mesh at $0.2$ seconds.}
	\end{subfigure}
	\caption{Simulation results for the coarse test case with $T_f = 0.2$.}
	\label{BeamSimulation}
\end{figure}

From Table \ref{beamResults} we can see that the waveform iteration terminated within three iterations with or without acceleration, corresponding to fast convergence behaviour. However, the fine test case did not terminate in 10 iterations without acceleration, showing that our time adaptive QNWR method increases robustness. 