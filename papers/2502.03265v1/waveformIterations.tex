\section{Waveform iterations}

Consider the following coupled ODEs on the interval $[0,T_f]$:
\begin{align*}
	\dot{w} &= g(w, v), \quad w(0) = w_0, \\
	\dot{v} &= h(v, w), \quad v(0) = v_0, 
\end{align*}
where $w \in \mathbb{R}^{d_w}$, and $v \in \mathbb{R}^{d_v}$. We assume that $g$ and $h$ are Lipschitz continuous functions.

To define our waveform iteration, we divide the interval $[0,T_f]$ into a finite number of time windows $[T_{w-1},T_w]$. In the following, without loss of generality, we assume that there is only one time window $[0,T_f]$. The Gau\ss -Seidel or serial waveform iteration is then given by  
\begin{align*}
	\dot{w}^{k+1} &= g(w^{k+1}, v^{k}), \quad w^{k+1}(0) = w_0 ,\\
	\dot{v}^{k+1} &= h(v^{k+1}, w^{k+1}) , \quad v^{k+1}(0) = v_0, \text{ for } k=0, ...,
\end{align*}  
where the starting guess $v^0 \equiv v_0$ is commonly used. We use the termination criterion 
\begin{align}\label{Conv-crit}
	\|v^{k+1}(T_{f})-v^{k}(T_{f})\|/\|v^{k+1}(T_{f})\| \leq Tol_{WR},
\end{align} where $\|\cdot\|$ is a vector norm. 

A discrete waveform iteration is given by discretizing both sub problems in time. To this end, we make use of the notion of a time grid, defined without loss of generality on $[0,T_f]$.

\begin{definition}
	A vector ${\cal T}\in\mathbb{R}^N$ with components $t_i\in[0,T_f]$ and
	\[0=t_0<t_1<\hdots<t_{N-1}=T_f\]
	is called a time grid. 
\end{definition}

The two discretizations are assumed to use the grids ${\cal T}_w$ and ${\cal T}_v$ with $N_w$ and $N_v$ grid points, respectively. This yields the discrete solutions $\bar{w}^{k} =\{w^{k,i}\}_{i=0}^{N_w-1}$ and $\bar{v}^{k} =\{v^{k,i}\}_{i=0}^{N_v-1}$ for the grid points in ${\cal T}_w$ and ${\cal T}_v$ respectively. The two time grids are not assumed to be matching. Thus, one has to use some kind of interpolation in the discrete waveform iteration. We define the interpolation below for the general setting where we have  data vector $x \in R^{dN}$ consisting of the $d$-dimensional components $x_i$ for $i = 0,...,N-1$, where each $x_i$ belongs to a time step $t_i \in \mathcal{T}$.

\begin{definition}
We denote a continuous interpolation of a data vector $x \in R^{dN}$ on the time grid $\mathcal{T}$ by $ \mathbf{I}_\mathcal{T}(x)$. To simplify the notation, we interpret the interpolation as a map from either a time grid and a data vector, or only a data vector to a continuous function. We use the notation $\mathbf{I}_{{\cal T}}(x)(\chi): \mathbb{R}^{N} \xrightarrow{} \mathbb{R}^{Nd}$ to denote the sampling of the interpolant $\mathbf{I}_{{\cal T}}(x)$ on a time grid $\chi$.  
\end{definition}

Using interpolation, the discrete waveform iterations are obtained by discretizing 
\begin{align*}
	\dot{w}^{k+1} & = g( w^{k+1}, \mathbf{I}_{{\cal T}_v}( \bar{v}^{k})), \quad w^{k+1}(0) = w_0, \\
	\dot{v}^{k+1} & = h(v^{k+1}, \mathbf{I}_{{\cal T}_w}( \bar{w}^{k+1})) , \quad v^{k+1}(0) = v_0,
\end{align*}
with two possibly different time discretization methods in time. 

To extend the waveform iteration to the time adaptive setting, where the two time grids change in each iteration, we denote the iteration number with a superscript k. It is worth pointing out that we do not assume that the two sequences of time grids have the same number of grid points in each iteration k. Thus, the discrete solutions are be given by $\bar{w}^{k} =\{w^{k,i}\}_{i=0}^{N_w^k}$ and $\bar{v}^{k} =\{v^{k,i}\}_{i=0}^{N_v^k}$, where ${N_w^k}$ and ${N_v^k}$ denotes the number of grid points in time grids ${\cal T}_w^k$ and ${\cal T}_v^k$, respectively. The discrete waveform iteration is then given by discretizing 
\begin{align*}
	\dot{w}^{k+1} & = g( w^{k+1}, \mathbf{I}_{{\cal T}_v^k}( \bar{v}^{k})), \quad w^{k+1}(0) = w_0, \\
	\dot{v}^{k+1} & = h(v^{k+1}, \mathbf{I}_{{\cal T}_w^k}( \bar{w}^{k+1})) , \quad v^{k+1}(0) = v_0.
\end{align*}

The convergence of waveform iterations has been studied in various continuous and discrete settings for ODEs, see for example \cite{Ne89Cont,Ne89Disc}. For ODEs, waveform iterations achieve superlinear convergence with an error bound of the form
\begin{equation*}
	||e^{k}||_{[0,T_f]} \leq \frac{(CT_f)^k}{k!} ||e^0||_{[0,T_f]},
\end{equation*}
where $C>0$ is a constant depending on the Lipschitz-constant of the coupled ODE, as well as their time discretization. The norm $||.||_{[0,T_f]}$ is the supremumnorm $||e^k||_{[0,T_f]} := \sup_{t \in [0,T_f]} ||e^k(t)||$. Convergence in the time adaptive case was proven in \cite{BeZe93} under the assumption that both time grids in the two sub solvers converge. As the authors point out in \cite{BeZe93} the assumption that the time grids converge is not necessarily realistic, since the time step is given by a controller that does not have anything to do with the waveform iteration. 

\subsection{Waveform iterations for interface coupled PDEs}
%This is achieved by splitting the two PDEs into separate systems that interact by the exchange of boundary data. The resulting coupled system is then solved using waveform iterations, which solves the coupled problem iteratively by solving the two PDEs with boundary data from the other domain. 
Waveform iterations have also been used in the context of coupled PDEs which interact through a lower dimensional boundary $\Gamma$, see for example \cite{JaVa96a, JaVa96b, RuUe21, BiMo19, BiMeMo23, gakwma:16}. This is accomplished by splitting the two PDEs into separate systems that interact by the exchange of boundary data. This allows us to treat the two solvers as black boxes which map interface data onto other interface data. In the continuous setting, the Gau\ss-Seidel waveform iteration for PDEs can be written as
\begin{align*}
	x_1^{k+1} & = \mathcal{S}_1( x_2^{k}) \in C\left(\Gamma \times [0,T_f]\right), \\
	x_2^{k+1} & = \mathcal{S}_2(x_1^{k+1}) \in C\left(\Gamma \times [0,T_f]\right) ,
\end{align*}
where $x_1 \in C\left(\Gamma \times [0,T_f]\right)$ and $x_2 \in C\left(\Gamma \times [0,T_f]\right)$ are interface data from the first respectively second domain. $\mathcal{S}_1$ and $\mathcal{S}_2$ are unbounded Poincaré-Steklov operators that define a map from the values of a boundary condition to the values of another boundary condition. 

The discrete waveform iteration is obtained by discretizing $ \mathcal{S}_1$ and $ \mathcal{S}_2$ in space and time. Here we assume that both solvers use fixed spatial grids, since the focus is on the time integration. If the meshes in time are also fixed, the discrete Poincaré-Steklov operators can be defined as 
\begin{equation*}
	S_i: C[0,T_f]^d \xrightarrow{} \mathbb{R}^{d N_i},
\end{equation*}  
for $i = 1,2$, where $d$ denotes the size of $x_1$ and $x_2$, and $N_i$ the size of the time grid for solver $i$. The waveform iteration can then be written as
\begin{align}\label{WRfixed}
	x_2^{k+1} = S_2 \circ \mathbf{I}_{\mathcal{T}_1} \circ S_1 \left( \mathbf{I}_{\mathcal{T}_2}\left(x_2^k\right) \right) \in \mathbb{R}^{d N_2}.
\end{align}
If we use instead a time adaptive method, the discrete operators can instead be defined by the following map
\begin{equation*}
	S^{TA}_i: C[0,T_f]^d \xrightarrow{} (\mathbb{R}^N,\mathbb{R}^{d N}),
\end{equation*} taking interface data from the other solver and returning the used time grid $\mathcal{T}_i^k$, as well as the corresponding discrete interface data. This allows us to write the waveform iteration as \begin{align}\label{badTAWR}
	\left(x_2^{k+1}, \mathcal{T}_2^{k+1} \right) = S^{TA}_2 \circ \mathbf{I}_{\mathcal{T}^{k+1}_1} \circ S^{TA}_1 \left( \mathbf{I}_{\mathcal{T}^{k}_2}\left(x_2^k\right) \right) \text{ on } \mathbb{R}^{d N^{k+1}_2 \times N^{k+1}_2}.
\end{align}

%It is worth pointing out that the number of time steps done by the two sub solvers and the dimensions of $x_2$ change in each iteration. Thus, the underlying vector space changes in each iteration. From a mathematical perspective it is therefore better to rewrite the discrete time adaptive waveform iteration as an iteration on a subset of the continuous functions: 
%	\begin{equation}\label{goodWRTA}
	%		\mathbf{I}_{\mathcal{T}_2^{k+1}}(x_2^k) = \mathbf{I}_{\mathcal{T}_2^{k+1}} \circ \mathcal{S}^{TA}_{2}  \circ \mathbf{I}_{\mathcal{T}_1^{k+1}} \circ \mathcal{S}^{TA}_{1}\left( \mathbf{I}_{\mathcal{T}_2^k}\left(x_2^{k}\right)\right) \in C[0,T_f]^d.
	%	\end{equation} 

It is worth noting that unlike in the ODE case, the waveform iterations given in Equation \eqref{badTAWR} and \eqref{WRfixed} are not guaranteed to converge. 
%It is thus difficult to construct general convergence theorems  for the PDE case. 
However, there exist convergence theorems for various specific cases, see for example  \cite{gakwma:16,JaVa96a, JaVa96b}.

To improve the convergence behaviour, some kind of acceleration is often employed. The simplest acceleration method is relaxation, where a weighted  average is taken of the iterates. Relaxation was used in a fully discrete time adaptive setting in \cite{BiMeMo23}, where it was proposed to sample the old interpolations to the newest time grid ${\cal T}_{2}^{k+1}$. In our notation this is 
\begin{align*}
	x_2^k = \theta \hat{x}_2^k + (1 - \theta)\mathbf{I}_{\mathcal{T}_2^k}(x_2^{k})({\cal T}_{2}^{k+1}),
\end{align*}
where $\theta \in [0,1]$ is the relaxation parameter. The convergence speed of the waveform iteration with relaxation is highly dependent on the relaxation parameter. For two coupled heat equations, the relaxation parameter has been optimized in \cite{BiMo18}, which achieves fast convergence in the time adaptive setting. Another option would be to apply a modified Quasi-Newton method. 

