\section{Qualitative Topic Analysis}
\label{sec:qualitative_topic_analysis}

In the main paper, we mainly present quantitative topic analysis, including topic coherence and perplexity results. Here we further provide qualitative topic analysis as a case study.

\textbf{Topic interpretability.} To intuitively understand what topic tree structure our model learns and what keywords each topic contains, here we plot topic tree and keywords of each topic on PL dataset at Fig. \ref{fig:topic_interpretability_pl}. Here we show top-4 keywords of each topic for clarity purpose. For each topic, we manually summarize its keywords into one word or phrase. For topic hierarchy of other datasets, our submitted code can produce topic hierarchy after convergence for every dataset.

Overall, the learned topic tree has three levels. The root topic \textit{Programming Language} is split into three concepts at the second level, \textit{Software Analysis}, \textit{Object Oriented Programming (OOP)}, and \textit{Design Pattern}. For Software Analysis topic, the corpus seems to contain documents about \textit{Semantics} of programming language and \textit{Efficiency} of the program. Similarly, for Object Oriented Programming topic, papers in this corpus mainly talk about three sub-concepts, \textit{Programming}, \textit{Parallelism}, and \textit{Compiler}, all of which are related to OOP. Similar topic hierarchy can also be observed on the Design Pattern topic, which is split into \textit{Implementation} and \textit{Inheritance} topics.

\textbf{Topic visualization.} %To visually understand what document embeddings our model learns, 
We use t-SNE \cite{tsne} to project document embeddings into 2D space and color embeddings using documents' labels in Fig. \ref{fig:visualization}. Since our model is a topic model, we mainly select representative topic models for visualization. GATON does not incorporate topic hierarchy or graph hierarchy, thereby its document embeddings of different categories tend to mix together. By modeling topic hiearchy, TSNTM produces clearer separation among different categories. HGTM and our model capture both topic hierarchy and graph hierarchy, and produce similar separation based on visual observation.

\begin{figure*}[h]
	\centering
	\includegraphics[width=0.8\linewidth]{figure/topic_hierarchy_pl.pdf}
	%\vspace{-0.3cm}
	\caption{Topic tree structure learned on PL dataset.}
	\label{fig:topic_interpretability_pl}
	%\vspace{-0.4cm}
\end{figure*}

\begin{figure*}[h]
	\centering
	\includegraphics[width=1\linewidth]{figure/visualization.pdf}
	%\vspace{-0.3cm}
	\caption{Visualization on ML dataset.}
	\label{fig:visualization}
	%\vspace{-0.4cm}
\end{figure*}