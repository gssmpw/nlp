\section{Problem Formulation and Preliminaries}

% We formulate problem here and introduce preliminaries. Supplementary material summarizes math notations.

$ \mathcal{G}=\{\mathcal{D},\mathcal{E}\} $ is a document graph. $ \mathcal{D}=\{d_i\}_{i=1}^N $ is a set of $ N $ documents. Each document $ d_i=\{w_{i,v}\}_{v=1}^{|d_i|}\subset\mathcal{V} $ is a sequence of words in vocabulary $ \mathcal{V} $. $ \mathcal{E}=\{e_{ij}\} $ is a set of edges. If there is an edge between documents $ i $ and $ j $, $ e_{ij}\in\mathcal{E} $. We follow \cite{hgtm} and model an undirected graph, $ e_{ij}=e_{ji} $, though our model is also applicable to directed graph. % As in \cite{line}, if we do not observe any edges, we instead induce $ \kappa $NN edges based on documents' semantic similarity. 
For document $ i $, its neighbor set $ \mathcal{N}(i) $ contains documents directly linked to $ i $.

Given $ \mathcal{G} $ as input, we propose a topic model that outputs unified document representations preserving topic hierarchy $ \mathcal{D} $ and graph hierarchy $ \mathcal{E} $. %Since our model is built in hyperbolic space to better preserve hierarchies than in Euclidean space, we review basic concepts of hyperbolic geometry.
Appendix \ref{sec:notations} summarizes math notations.

\textbf{Hyperbolic geometry} is a non-Euclidean differential geometry with a constant negative curvature $ -1/K $ $ (K>0) $. Curvature measures how a geometric object deviates from a flat plane. %There are multiple equivalent models in hyperbolic space. 
We work with Hyperboloid model \cite{hyperboloid_model}, though our work is applicable to others, e.g., Poincar{\'e} ball \cite{poincare_model}.

\textbf{Hyperboloid model} is an $ n $-dimensional hyperbolic space $ \Bbb H^{n,K} $ where Minkowski self-inner product ($ \langle\cdot,\cdot\rangle_\mathcal{L} $) of its vectors is $ -K $,
\begin{equation}
\resizebox{0.88\columnwidth}{!}{
$ \begin{split}
    \Bbb H^{n,K}=&\{\textbf{x}\in\Bbb R^{n+1}|\langle\textbf{x},\textbf{x}\rangle_{\mathcal{L}}=-K,x_0>0\}\\
    \text{\quad where\quad}%<\textbf{x},\textbf{y}>_{\mathcal{L}}=-x_0y_0+\sum_{i=1}^n x_iy_i.
    \langle\textbf{x}&,\textbf{y}\rangle_{\mathcal{L}}=-x_0y_0+x_1y_1+...+x_ny_n.
\end{split} $}
\end{equation}
For hyperbolic vector $ \textbf{x}\in\Bbb H^{n,K} $, the tangent space $ \mathcal{T}_{\textbf{x}}\Bbb H^{n,K} $ around $ \textbf{x} $ is first-order approximation of $ \Bbb H^{n,K} $ and is $ (n+1) $-dimensional Euclidean space.
\begin{equation}
     \mathcal{T}_{\textbf{x}}\Bbb H^{n,K}=\{\textbf{v}\in\Bbb R^{n+1}|\langle\textbf{x},\textbf{v}\rangle_{\mathcal{L}}=0\}.
\end{equation}
%We will use tangent space to perform Euclidean operations undefined in hyperbolic space.

\textbf{Exponential and logarithmic maps.} The projection between hyperbolic and tangent space is achieved by exponential and logarithmic maps. For a hyperbolic vector $ \textbf{x}\in\Bbb H^{n,K} $ and one of its tangent vectors $ \textbf{v}\in\mathcal{T}_{\textbf{x}}\Bbb H^{n,K} $ $ (\textbf{v}\neq \textbf{0}) $, exponential map projects $ \textbf{v} $ to the hyperbolic space by
\begin{equation}
\resizebox{1\columnwidth}{!}{
$
\exp_{\textbf{x}}^K(\textbf{v})=\cosh\Big(\dfrac{||\textbf{v}||_{\mathcal{L}}}{\sqrt{K}}\Big)\textbf{x}+\sqrt{K}\sinh\Big(\dfrac{||\textbf{v}||_{\mathcal{L}}}{\sqrt{K}}\Big)\dfrac{\textbf{v}}{||\textbf{v}||_{\mathcal{L}}}. $
}
\end{equation}
$ ||\textbf{v}||_{\mathcal{L}}=\sqrt{\langle\textbf{v},\textbf{v}\rangle_{\mathcal{L}}} $ is the norm of $ \textbf{v}\in\mathcal{T}_{\textbf{x}}\Bbb H^{n,K} $. Reversely, for $ \textbf{x}\in\Bbb H^{n,K} $ and hyperbolic vector $ \textbf{y}\in\Bbb H^{n,K} $ $ (\textbf{x}\neq\textbf{y}) $, logarithmic map projects $ \textbf{y} $ to $ \textbf{x} $'s tangent space. $ d_{\mathcal{L}}^K(\textbf{x},\textbf{y}) $ is the distance between two hyperbolic vectors in Hyperboloid.
\begin{equation}
\label{eq:log_map}
\resizebox{0.88\columnwidth}{!}{
$ \begin{split}
    \log_{\textbf{x}}^K(\textbf{y})&=d_{\mathcal{L}}^K(\textbf{x},\textbf{y})\dfrac{\textbf{y}+\frac{1}{K}\langle\textbf{x},\textbf{y}\rangle_{\mathcal{L}}\textbf{x}}{||\textbf{y}+\frac{1}{K}\langle\textbf{x},\textbf{y}\rangle_{\mathcal{L}}\textbf{x}||_{\mathcal{L}}}\\
    \text{\quad where\quad}d_{\mathcal{L}}^K&(\textbf{x},\textbf{y})=\sqrt{K}\text{arcosh}(-\langle\textbf{x},\textbf{y}\rangle_{\mathcal{L}}/K).
\end{split} $}
\end{equation}
%Here $ d_{\mathcal{L}}^K(\textbf{x},\textbf{y}) $ is the distance between two hyperbolic vectors in Hyperboloid model.

\textbf{Parallel transport.} For two hyperbolic vectors $ \textbf{x},\textbf{y}\in\Bbb H^{n,K} $ $ (\textbf{x}\neq\textbf{y}) $, parallel transport can transport $ \textbf{v}\in\mathcal{T}_{\textbf{x}}\Bbb H^{n,K} $ on $ \textbf{x} $’s tangent space to $ \textbf{y} $’s.
\begin{equation}
\resizebox{1\columnwidth}{!}{
$
    \text{PT}^K_{\textbf{x}\rightarrow\textbf{y}}(\textbf{v})=\textbf{v}-\dfrac{\langle\log_{\textbf{x}}^K(\textbf{y}),\textbf{v}\rangle_{\mathcal{L}}}{d_{\mathcal{L}}^K(\textbf{x},\textbf{y})^2}(\log_{\textbf{x}}^K(\textbf{y})+\log_{\textbf{y}}^K(\textbf{x})). $
    }
\end{equation}