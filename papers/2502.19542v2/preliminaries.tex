\section{Preliminaries}\label{sec:preliminaries}

We start by introducing an overview of the basic objects and concepts used in FEEC. We
specialize this overview for the two-dimensional case, which is our primary focus. Moreover,
we simplify the exposition by avoiding the use of differential forms and instead use vector
calculus proxies. For a more general discussion, the reader is invited to consult
\cite{Arnold2018}.

\subsection{The Hilbert de Rham complex}
Let $\domain \subset \RR^2$ be a bounded domain with a piecewise-smooth Lipschitz boundary.
The following sequence, $\complex{R}$, of Hilbert spaces and connecting differential
operators is called the Hilbert de Rham complex,
\begin{equation*}
    \begin{tikzcd}
        \complex{R}~:~\Hone \arrow{r}{\grad} & \Hcurl \arrow{r}{\curl} & \Ltwo\;,
    \end{tikzcd}
\end{equation*}
where $\grad$ is the gradient operator defined in coordinates by \(\grad u = (\partial_x
u, \partial_y u)\), \(\curl\) is the scalar curl operator in two dimensions and is defined in coordinates as
$\curl \mbf{v} := \partial_x v_2 - \partial_y v_1$, $\Ltwo$ is the space of
square-integrable functions on $\domain$, and
\begin{align*}
    \Hone &:= \left\{ u \in \Ltwo : \grad u \in [\Ltwo]^2 \right\}\;,\\
    \Hcurl &:= \left\{ \boldsymbol{v} \in [\Ltwo]^2 : \curl \boldsymbol{v} \in \Ltwo \right\}\;.
\end{align*}
We will refer to these spaces as \(\funcspace{X}^0 = \Hone,
\funcspace{X}^1=\Hcurl\) and \(\funcspace{X}^2=\Ltwo\).

The sequence $\complex{R}$ is called a complex because of two properties. First, each
differential operator maps from one space into the next space,
\begin{equation*}
    \image(\grad) \subset \Hcurl\;,\qquad
    \image(\curl) \subset \Ltwo\;.
\end{equation*}
Second, the composition of the two operators is identically zero, or in other words,
$\curl (\grad f) = 0$ for any $f \in \Hone$.
Given these two properties, we can define the following (quotient) spaces called the
cohomology spaces of $\complex{R}$,
\begin{equation*}
    \cohomo^0(\complex{R}) := \kernel(\grad)\;,\
    \cohomo^1(\complex{R}) := \kernel(\curl)/\image(\grad)\;,\
    \cohomo^2(\complex{R}) := \Ltwo/\image(\curl)\;.
\end{equation*}
The elements of $\cohomo^k(\complex{R})$ are called harmonic scalar/vector fields. These
cohomology spaces are important because, first, they capture the topological properties of
the domain $\domain$ and, second, the solutions to scalar and vector Laplacians are
well-defined up to harmonic fields. The following two paragraphs elaborate on this structure
that underlies the Hilbert de Rham complex.

It turns out that there is a one-to-one correspondence between the cohomology spaces and the
so-called Betti numbers of $\domain$: the vector space dimension of $\cohomo^j(\complex{R})$
is equal to the number of $j$-dimensional holes in $\domain$. That is, $\dim
\cohomo^0(\complex{R})$ is the number of connected components of $\domain$, $\dim
\cohomo^1(\complex{R})$ is the number of holes in $\domain$, and $\dim
\cohomo^2(\complex{R}) = 0$ in our planar setting (since there are no two-dimensional holes in
$\domain$). In particular, we will focus only on simply-connected domains $\domain$, which
implies that $\cohomo^0(\complex{R}) \cong \RR$ and $\cohomo^1(\complex{R}) = \emptyset =
\cohomo^2(\complex{R})$ in our setting.

The second reason why cohomology spaces are important is that harmonic fields in
$\cohomo^1(\complex{R})$ are the solutions to the homogeneous vector Laplace problem,
$\Delta \mbf{v} = 0$, and the harmonic fields in $\cohomo^0(\complex{R})$ are the solutions
to the homogeneous scalar Laplace problem, $\Delta u = 0$, with homogeneous natural boundary conditions. Given the assumption of
simply-connectedness for $\domain$, this means that the homogeneous vector Laplace problem
admits only the trivial solution, and only constant functions satisfy the homogeneous scalar
Laplace problem.

\subsection{Structure-preserving discretizations}

The core principle of FEEC is to preserve the structural connections mentioned above in the discrete setting. This can be achieved by choosing appropriate finite element spaces for approximating scalar and vector fields.
Failure to preserve the cohomological structure in the discrete setting can cause, in
particular, spurious harmonic fields to appear in the discrete setting, which can lead to highly inaccurate solutions to scalar and vector Laplace problems.
Examples can be found in Section \ref{sec:numerical-results} of this paper, and also in \cite{Arnold2018}, for instance.

The FEEC recipe to preserve the cohomological structure is ensuring that the cohomology
spaces of the discrete and continuous complexes can be isomorphically identified with maps
that are bounded and commute with the operators defined above. By doing
so, we can also ensure that we do not introduce any spurious harmonic scalar/vector fields, as these are intimately related with the cohomology spaces. In short, we want to construct a discrete version, \(\complex{R}_h\), of the de Rham complex \(\complex{R}\), namely,
\begin{equation*}
    \begin{tikzcd}
        \complex{R}_h~:~ \funcspace{X}^0_h(\domain) \arrow{r}{\grad} & \funcspace{X}^1_h(\domain) \arrow{r}{\curl} & \funcspace{X}^2_h(\domain)\;,
    \end{tikzcd}
\end{equation*}
for some appropriate choice of finite element spaces \(\funcspace{X}^j_h(\domain),\;
j\in\{0,1,2\}\), and where \(h\) is a parameter inversely proportional to the number of \dofs.

The relevant properties of these finite element spaces have been highlighted in \cite{Arnold2009,Arnold2018}, and we showcase them here briefly for ease of readability. For a more comprehensive overview of the properties and corresponding results, we refer the reader to the works just mentioned. First, we require that the finite element spaces considered have enough similarity with the continuous ones to be able to approximate them arbitrarily well. To be concrete, we necessitate that
\[
    \lim_{h \to 0} \inf_{u_h \in \funcspace{X}^j_h}\|u_h -u \| = 0,\quad u \in
	\funcspace{X}^j, j\in \{0,1,2\}.
\]
The second requisite is that the chosen discrete spaces, together with the operators of the complex \(\complex{R}\), also form a complex, or, in other words, 
\[
\grad \funcspace{X}^0_h \subseteq \funcspace{X}^{1}_h, \quad \curl \funcspace{X}^1_h \subseteq \funcspace{X}^{2}_h,
\]
and the composition of the two operators is zero, as in the continuous case. The final property we mention here is crucial for the correct behaviour of a discrete
solution, and is the one that establishes a connection between the cohomology spaces of the
two complexes. That is, it should be possible to define a set of bounded projection maps from
\(\complex{R}\) to \(\complex{R}_{h}\), in such a way that a commuting diagram is formed.
By guaranteeing the three properties listed above, one can ensure an accurate and stable solution of the mixed formulation of Laplace problems.

It is precisely this choice of appropriate finite element spaces that will be tackled in the rest of the paper, more specifically, when the finite element spaces are hierarchical B-spline spaces. Whether bounded commuting projectors exist for hierarchical B-spline complexes is still an open issue and outside the scope of the present work. We limit ourselves to characterizing and fixing the kind of refinements that introduce spurious harmonic forms, and therefore invalidate the existence of such projectors altogether. 
