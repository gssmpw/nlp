\section{Conclusions}\label{sec:conclusion}

A key ingredient in the construction of stable numerical methods for electromagnetics and fluid mechanics is the construction of a discrete de Rham complex that preserves the cohomology of the continuous one.
While the construction of such complexes with tensor-product splines is well-understood, doing so with splines that allow for local refinement is not.
In this paper, focusing on the use of (T)HB-splines in two dimensions, we have presented a first, constructive approach for building an adaptively-refinable discrete de Rham complex with the right cohomology.

Our approach builds upon recent results from \cite{Shepherd2024}.
They describe sufficient conditions for hierarchical meshes (in an arbitrary number of dimensions) such that the (T)HB-spline spaces built on them form an exact discrete de Rham complex.
In two dimensions, given a hierarchical mesh which does not satisfy those conditions, we have now shown how it can be modified to do so.
We have also presented algorithms that show how our theoretical results can be implemented.
Moreover, we prove that, if the first space in the complex is of $\mathcal{H}-$ or $\mathcal{T}-$admissibility class $m$, then so are the rest of the spaces.
Ensuring admissibility is important not just from the point of view of the theory of hierarchical B-splines \cite{Buffa2015} but also for the stability of the numerical methods built on them \cite{Carraturo2019}.
Our theoretical results imply that the algorithms developed for maintaining admissibility of (T)HB-spline spaces (e.g., \cite{Bracco2019}) can be immediately used in the context of spline differential forms.
Finally, we present numerical tests (both source and eigenvalue problems) that demonstrate the effectiveness of our approach.

There are several interesting lines of research that remain open.
The construction of a discrete de Rham complex with THB-splines in three dimensions is a natural next step.
It will also be interesting to study if the smallest quantum of refinement could be reduced from supports of $0$-form B-splines to supports of volume-form B-splines, as was done in \cite{Evans2018}, thereby yielding slightly more flexible refinements.
Another interesting research direction was alluded to at the end of Section \ref{sec:preliminaries}: the formulation of bounded cochain projections for the hierarchical de Rham complex.
These and other topics will be the subject of future work.

% \KS{COMMENT: Some references in the bibliography have unusual symbols and citations that I would consider to be uncommon. These should be (briefly) revisited to make sure we are happy with all citation information.}
% \DC{I fixed some of those. Is it all-right now?}
