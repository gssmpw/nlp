\section{Admissibility}\label{sec:admissibility}

In adaptive methods for hierarchical splines, optimal order of convergence is proved by using the property of admissibility, which limits the difference of level between interacting functions \cite{Buffa2015,Buffa2017}.
\begin{defn}\label{def:admissibility}
	We say that \(\hbspace^{j}_L\), \(j \in \{0,1,2\}\), is \(\hbbasis\)- or
	\(\thbbasis\)-admissible of class \(m\) if on any element \(\meshel \in \mesh\), the
	non-vanishing basis functions in \(\hbbasis^{j}_L\) or \(\thbbasis^{j}_L\),
	respectively, belong to at most \(m\) successive levels.

	By extension, we say that the hierarchical complex \eqref{eq:hierarchical-complex} is
	\(\hbbasis\)- or \(\thbbasis\)-admissible of class \(m\) if the corresponding previous
	condition holds for every \(j \in \{0,1,2\}\).
\end{defn}

The following definition relates a basis function with the ones involved when computing the
differential operators. We also introduce a result which relates their support after
truncation, that we will use in the proofs of admissibility. 
From now on, we will use \(\boldvec{\delta}_k\) to denote the two-dimensional index with value \(1\) at position \(k\), and \(0\) in the other position.
\begin{defn}\label{def:int-der-relation}
	Let \(\bsp^{\boldvec{j}}_{\boldvec{i}, \level} \in \bspbasis^{\boldvec{j}}_{\level}\) and \(\boldvec{j}' - \boldvec{j} = \boldvec{\delta}_k \) for some \(k \in \{1,2\}\). We say that 
	\(\bsp^{\boldvec{j}'}_{\boldvec{i}', \level} \in \bspbasis^{\boldvec{j}'}_{\level}\)
	is a \(k\)-derivation of \(\bsp^{\boldvec{j}}_{\boldvec{i}, \level}\) if \(	\boldvec{i}' - \boldvec{i} = c\boldvec{\delta}_k,\)
	with \(c \in \{0, 1\}\).
	Correspondingly, we say that
	\(\bsp^{\boldvec{j}}_{\boldvec{i}, \level}\) is a \(k\)-integration of
	\(\bsp^{\boldvec{j}'}_{\boldvec{i}', \level}\).
\end{defn}

\begin{prop}\label{prop:thb-supp}
	Let \(\thb^{\boldvec{j}}_{\boldvec{i}, \level} \in
	\thbbasis^{\boldvec{j}}_{\level}\) and \(\bsp^{\boldvec{j}}_{\boldvec{i},
	\level}=\mot(\thb^{\boldvec{j}}_{\boldvec{i}, \level})\).
	Also, let \(\bsp^{\boldvec{j}'}_{\boldvec{i}', \level} \in
	\bspbasis^{\boldvec{j}'}_{\level}\) be a \(k\)-derivation of
	\(\bsp^{\boldvec{j}}_{\boldvec{i}, \level}\) for some \(k\in \{1,2\}\) and \(\thb^{\boldvec{j}'}_{\boldvec{i}',
	\level} =
	\trunc^{L}\left(\cdots\left(\trunc^{\level+1}\left(\bsp^{\boldvec{j}'}_{\boldvec{i}',
	\level}\right)\right)\right)\).
	Then, it holds that \[
		\supp(\thb^{\boldvec{j}}_{\boldvec{i}, \level}) \supseteq
		\supp(\thb^{\boldvec{j}'}_{\boldvec{i}', \level}).
    \]
\end{prop}
\begin{proof}
	By definition, \(\boldvec{i}' - \boldvec{i} = c\boldvec{\delta}_k\), with \(c \in \{0, 1\}\), for
	some \(k \in \{1,2\}\).
	Let us consider an active element \(\meshel \in \mesh \cap \mesh_{\tilde{\level}}\), with \(\tilde{\level}>\level\), where
	\(\thb^{\boldvec{j}}_{\boldvec{i}, \level}\) is truncated, in the sense that it vanishes on \(Q\).
	Also, let \(\mathcal{R}^{\boldvec{j}}_{Q, \tilde{\level}}\) denote the smallest subset of
	\(\bspbasis^{\boldvec{j}}_{\tilde{\level}}\) that can be linearly combined to represent
	\(\bsp^{\boldvec{j}}_{\boldvec{i}, \level}\) on \(\meshel\).
	Since \(\thb^{\boldvec{j}}_{\boldvec{i}, \level}\) vanishes on \(\meshel\), we know that
	the support of B-splines in \(\mathcal{R}^{\boldvec{j}}_{Q, \tilde{\level}}\) is contained in
	\(\Omega_{\tilde{\level}}\).
	Moreover, we denote by \(\mathcal{G}^{\boldvec{j}'}_{Q,\tilde{\level}}\) the set of all
	\(k\)-derivations of B-splines in \(\mathcal{R}^{\boldvec{j}}_{Q,\tilde{\level}}\).
	Since all B-splines in \(\mathcal{R}^{\boldvec{j}}_{Q,\tilde{\level}}\) are supported on
	\(\Omega_{\tilde{\level}}\), all B-splines in \(\mathcal{G}^{\boldvec{j}'}_{Q,\tilde{\level}}\) must
	also be supported on \(\Omega_{\tilde{\level}}\).
	
	Next, let \(\mathcal{R}^{\boldvec{j}'}_{\meshel, \tilde{\level}}\) be the smallest subset of
	\(\bspbasis^{\boldvec{j}'}_{\tilde{\level}}\) that can be linearly combined to represent
	\(\bsp^{\boldvec{j}'}_{\boldvec{i}', \level}\) on \(\meshel\).
	We must have \(\mathcal{R}^{\boldvec{j}'}_{\meshel, \tilde{\level}} \subseteq
	\mathcal{G}^{\boldvec{j}'}_{\meshel, \tilde{\level}}\), because
	\(\bsp^{\boldvec{j}'}_{\boldvec{i}', \level}\) is just one of the \(k\)-derivations of
	\(\bsp^{\boldvec{j}}_{\boldvec{i}, \level}\).
	Thus, every B-spline in \(\mathcal{R}^{\boldvec{j}'}_{\meshel, \tilde{\level}}\) is supported
	on \(\Omega_{\tilde{\level}}\), and \(\thb^{\boldvec{j}'}_{\boldvec{i}', \level}\) must vanish
	on \(\meshel\).
\end{proof}

\begin{lem}\label{lem:hb-admissibility}
	Under \Cref{assum:refinement-domains}, if \(\hbspace^{0}_{L}\) is
	\(\hbbasis\)-admissible of  class \(m\) so is \(\hbspace^{j}_{L}\) for every \(j \in
	\{1,2\}\).
\end{lem}
\begin{proof}
	Let us suppose that some \(\hbspace^{j'}_L\), with \(j' \in \{1,2\}\), is not
	\(\hbbasis\)-admissible of class \(m\).
	In particular, this means that for some elements \(\meshel \in \mesh\cap
	\mesh_{\level}\) and \(\tilde{\meshel} \in \mesh\cap \mesh_{\tilde{\level}}\), with \(\level, \tilde{\level}
	\in \{0,1,\dots,L\}\) and \(\tilde{\level} > \level+m-1\), we have at least one active B-spline
	\(
		\bsp^{\boldvec{j}'}_{\boldvec{i}', \level} \in 
		\hbbasis_{L}^{\boldvec{j}'} \cap \bspbasis^{\boldvec{j}'}_\level
	\), with \(\abs{\boldvec{j}'}=j'\), such that 
	\begin{equation}\label{eq:hb-element-overlap}
		\supp(\bsp^{\boldvec{j}'}_{\boldvec{i}',\level}) \cap \meshel \neq \emptyset
		\wedge \supp(\bsp^{\boldvec{j}'}_{\boldvec{i}',\level})\cap \tilde{\meshel} \neq
		\emptyset,
    \end{equation}
	and since the function is active we know that \(\supp (\bsp^{\boldvec{j}'}_{\boldvec{i}',\level}) \not \subseteq \Omega_{\level+1}\). We will show that this implies the existence of a B-spline
	in \(\hbbasis_{L}^{\boldvec{0}}\) which contains both \(\meshel_{\level}\) and
	\(\meshel_{\tilde{\level}}\) in its support, thus contradicting the assumption on
	\(\hbspace^{0}_{L}\).

	We start by observing that we can find various B-splines such that the following chained
	containment holds, 
	\begin{equation*}
		\supp(\bsp^{\boldvec{0}}_{\boldvec{i}_{\level}, \level}) \supseteq \cdots \supseteq
		\supp(\bsp^{\boldvec{j}}_{\boldvec{i}, \level}) \supseteq
		\supp(\bsp^{\boldvec{j}'}_{\boldvec{i}',\level})\;,
	\end{equation*}
	and where each B-spline is a \(k\)-integration of the one to its right ― possibly with
	different values for each \(k\).
	Also, we can find another chained containment of B-splines for levels below \(\level\),
	\begin{equation}\label{eq:level-containment}
		\supp(\bsp^{\boldvec{0}}_{\boldvec{i}_0, 0}) \supseteq \cdots \supseteq
		\supp(\bsp^{\boldvec{0}}_{\boldvec{i}_{\level-1}, \level-1}) \supseteq
		\supp(\bsp^{\boldvec{0}}_{\boldvec{i}_{\level},\level})\;.
	\end{equation}
	We know that none of the B-splines in \eqref{eq:level-containment} can be active, because
	their supports contain the support of \(\bsp^{\boldvec{j}'}_{\boldvec{i}',\level}\), and
	this would lead to a contradiction on the admissibility class of \(\hbspace^{0}_{L}\) due to
	\eqref{eq:hb-element-overlap} and \Cref{assum:refinement-domains}.
	However, \(\supp (\bsp^{\boldvec{j}'}_{\boldvec{i}',\level}) \not \subseteq \Omega_{\level+1}\), which implies the same for all the functions in the sequence \eqref{eq:level-containment}. Therefore, at least one of those functions must be active, which leads to a contradiction.
\end{proof}

\begin{lem}\label{lem:thb-admissibility}
	Under \Cref{assum:refinement-domains}, if \(\hbspace^{0}_{L}\) is
	\(\thbbasis\)-admissible of  class \(m\) so is \(\hbspace^{j}_{L}\) for every \(j \in
	\{1,2\}\).
\end{lem}
\begin{proof}
	Let us suppose that some \(\hbspace^{j'}_L\), with \(j' \in \{1,2\}\), is not
	\(\thbbasis\)-admissible of class \(m\).
	In particular, this means that for some elements \(\meshel \in \mesh\cap
	\mesh_{\level}\) and \(\tilde{\meshel} \in \mesh\cap \mesh_{\tilde{\level}}\), with \(\level, \tilde{\level}
	\in \{0,1,\dots,L\}\) and \(\tilde{\level} > \level+m-1\), we have at least one active level
	\(\level\) truncated B-spline \(\thb^{\boldvec{j}'}_{\boldvec{i}', \level} \in
	\thbbasis_{L}^{\boldvec{j}'}\), with \(\abs{\boldvec{j}'}=j'\), such that 
	\begin{equation}\label{eq:thb-element-overlap}
		\supp(\thb^{\boldvec{j}'}_{\boldvec{i}',\level}) \cap \meshel \neq \emptyset
		\wedge \supp(\thb^{\boldvec{j}'}_{\boldvec{i}',\level})\cap \tilde{\meshel} \neq
		\emptyset.
    \end{equation}
	We will show that this implies the existence of a basis function in
	\(\thbbasis_{L}^{\boldvec{0}}\) which contains both \(\meshel_{\level}\) and
	\(\meshel_{\tilde{\level}}\) in its support, thus contradicting the assumption on
	\(\hbspace^{0}_{L}\).


	We start by observing that we can use \Cref{prop:thb-supp}, to find basis functions such
	that the following chained containment holds,
	\begin{equation*}
		\supp(\thb^{\boldvec{0}}_{\boldvec{i}_{\level}, \level}) \supseteq \cdots \supseteq
		\supp(\thb^{\boldvec{j}}_{\boldvec{i}, \level}) \supseteq
		\supp(\thb^{\boldvec{j}'}_{\boldvec{i}',\level})\;,
	\end{equation*}
	and where each mother B-spline is a \(k\)-integration of the one to its right ― possibly with
	different values for each \(k\).

	It is true that \(\thb^{\boldvec{0}}_{\boldvec{i}_{\level},\level}\) can't be
	active, since this would violate the assumption on \(\hbspace^{0}_{L}\) due to
	\eqref{eq:thb-element-overlap} and \Cref{assum:refinement-domains}.
	Therefore, because the support of \(\thb^{\boldvec{0}}_{\boldvec{i}_{\level},\level}\)
	contains \(\meshel_{\level}\) we know that
	\(\supp(\thb^{\boldvec{0}}_{\boldvec{i}_{\level},\level}) \not\subseteq
	\domain_{\level}\).

	Let \(\thb^{\boldvec{0}}_{\boldvec{i}_{\level-1},\level-1}\) be such that
	\(\mot(\thb^{\boldvec{0}}_{\boldvec{i}_{\level},\level})\) is used in the finer level
	representation of \(\mot(\thb^{\boldvec{0}}_{\boldvec{i}_{\level-1},\level-1})\), as in
	\eqref{eq:finer-rep}.
	Then, because \(\supp(\thb^{\boldvec{0}}_{\boldvec{i}_{\level},\level}) \not \subseteq
	\domain_{\level}\), we know that \(\thb^{\boldvec{0}}_{\boldvec{i}_{\level},\level}\)
	contributes to the representation of
	\(\thb^{\boldvec{0}}_{\boldvec{i}_{\level-1},\level-1}\) in \eqref{eq:truncated-rep}.
	Again, if \(\thb^{\boldvec{0}}_{\boldvec{i}_{\level-1},\level-1}\) were active it would
	lead to a contradiction on the admissibility of \(\hbspace^{0}_{L}\) because of
	\eqref{eq:thb-element-overlap} and \Cref{assum:refinement-domains}.
	So, we know that \(\thb^{\boldvec{0}}_{\boldvec{i}_{\level-1},\level-1}\) is inactive
	and \(\supp(\thb^{\boldvec{0}}_{\boldvec{i}_{\level-1},\level-1}) \not\subseteq
	\domain_{\level-1}\).
	We can repeat these steps recursively until we arrive at some
	\(\thb^{\boldvec{0}}_{\boldvec{i}_0, 0}\) that is inactive because
	\(\supp(\thb^{\boldvec{0}}_{\boldvec{i}_{0}, 0}) \not \subseteq \domain_{0}\), which
	leads to a contradiction since \(\domain_0 = \domain\).
\end{proof}
