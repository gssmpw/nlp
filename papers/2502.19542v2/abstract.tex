\begin{abstract}
Studying the de Rham complex is a natural choice when working with problems in electromagnetics and fluid mechanics.
By discretizing the complex correctly, it is possible to attain stable numerical methods to tackle these problems.
An important consideration when constructing the discrete complex is that it must preserve the cohomology structure of the original one.
This property is not guaranteed when the discrete function spaces chosen are hierarchical B-splines.
Research shows that a poor choice of refinement domains may give rise to spurious harmonic
forms that ruin the accuracy of solutions, even for the simplest partial differential equations.
Another crucial aspect to consider in the hierarchical setting is the notion of
admissibility, as it is possible to obtain optimal convergence rates of numerical solutions
by limiting the multi-level interaction of basis functions.
We will focus on the two-dimensional de Rham complex over the unit square \(\Omega \subseteq
\mathbb{R}^2\).
In this scenario, the discrete de Rham complex should be exact, and we provide both the
theoretical and the algorithm-implementation framework to ensure this is the case.
Moreover, we show that, under a common restriction, the admissibility class of the first space of the discrete complex persists throughout the remaining spaces.
Finally, we include numerical results that motivate the importance of the previous concerns for the vector Laplace and Maxwell eigenvalue problems.
\end{abstract}
