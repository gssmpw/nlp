\section{Introduction}

Over the last few years, there have been several developments in numerical methods designed
to approximate the solution of partial differential equations (PDEs).
Finite element exterior calculus (FEEC) \cite{Arnold2006, Arnold2009} is one of these
developments; the theory has gained a lot of traction, since its earlier publications
\cite{Arnold2002} due to its ability to describe how stable numerical methods can be
achieved for a broad class of PDEs.
Key elements used to ensure the stability of discrete methods are the topological and
differential-geometric structure of PDEs, which are captured in FEEC through complexes and,
more specifically, their cohomology classes. By choosing an appropriate finite element
space, it is possible not only to approximate the continuous solution, but also to guarantee
the method is stable ― provided that the continuous and discrete complexes are
cohomologically-equivalent.

Another important breakthrough in the field of numerical analysis was the introduction of
spline functions \cite{Boor2001, Schumaker2007} as a compelling alternative to the more
classical finite element method \cite{Hughes2000, Ciarlet2002, Brenner2008}.
The former approach was given the name of isogeometric analysis (IGA), and it is explained
in great detail in the seminal paper by T.J.R. Hughes, J. Cottrell and Y. Bazilevs
\cite{Hughes2005}.
The motivation behind this method was to create a better synergy between numerical-analysis
and industrial-design teams ― which have been using splines in CAD-based modelling for a
long time \cite{Farin2006}.
One of the benefits of the higher smoothness allowed in splines is that they provide better
approximation properties per degree of freedom (DoF) than standard \(C^0\) or \(C^{-1}\)
finite element spaces \cite{Evans2009, Bressan2019, Sande2019, Sande2020}.
Another upside in IGA is the flexibility in defining new spline spaces from basic splines,
such as the B-spline;
several of these have been explored in the literature: non-uniform rational B-splines
(NURBS) \cite{Piegl1997}, locally refined splines, such as T-splines \cite{Sederberg2003,
Sederberg2004}, hierarchical B-splines (HB-splines) \cite{Buffa2015, D'Angella2018} and
truncated hierarchical B-splines (THB-splines) \cite{Giannelli2012}, and the more
recent polar splines \cite{Speleers2021}.


For the sake of reasonable computation times, it is generally advisable to restrict regions
with a high density of DoFs to relevant local features of the solution being computed.
Locally-refined splines are naturally well-suited for problems where these localized
features control the overall rate of converge of numerical solutions; one of the more common
options being HB- and THB-splines. Both of these spline types can be used to describe the
same discrete function space ― where the advantages of one over the other depend on the
application. In both cases, however, a local refinement without careful consideration can
lead to error-convergence rates worse than the optimal ones predicted in the IGA framework. 
The most straightforward way to regain optimal convergence is to limit the interaction of
basis functions from different refinement levels, using what is called admissibility
\cite{Bracco2019}.

When working with a hierarchical B-spline complex in \(\RR^n, n\geq1\), it is not true in
general that this complex will generate an exact subcomplex of the continuous one for an
arbitrary refinement \cite{Evans2018, Shepherd2024}.
This is due to the fact that a poor choice of refinement pattern can remove basis functions
from a coarser level and add basis functions from a finer level such that they describe
function spaces with distinct topological properties; leading to the introduction of
spurious harmonic forms, which can easily ruin the accuracy and stability of the numerical
method being used.


In this paper, we focus on providing an overview of the inexact refinements described above,
as well as a way to add DoFs to the hierarchical B-spline de Rham complex
that turns them into exact refinements.
Many of the proofs that follow either work for a general hyper-cube, or can easily be
adapted to do so, but we will mostly restrict ourselves to the two-dimensional case due to
the significant simplifications of the computational implementation.
The main contributions of this paper are as follows:
we show that admissibility is preserved in the whole complex if it is preserved for the space of zero-forms in \Cref{lem:hb-admissibility,lem:thb-admissibility}; and we describe
\cref{alg:nlintersec}, which provides the extra DoFs to a hierarchical basis that ensure an exact discrete subcomplex.

The structure of the paper is as follows:
in \Cref{sec:preliminaries} we give a basic overview of the de Rham complex and its
discretization. 
Next, in \Cref{sec:spline-spaces} we describe the construction of hierarchical
B-splines and the corresponding hierarchical B-spline de Rham complex.
In \Cref{sec:admissibility,sec:exact-meshes}, we present the main theoretical results for admissibility and exactness, respectively.
These are followed by \Cref{sec:algorithms} with their accompanying algorithms.
We test our approach numerically in \Cref{sec:numerical-results}, where we showcase various
examples:
vector Laplace problem and Maxwell's eigenvalue problem, both for illustrative,
contrived cases and as an adaptive scheme.
Finally, in \Cref{sec:conclusion} we give some concluding remarks.
