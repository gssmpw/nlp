%%
%% This is file `sample-authordraft.tex',
%% generated with the docstrip utility.
%%
%% The original source files were:
%%
%% samples.dtx  (with options: `authordraft')
%% 
%% IMPORTANT NOTICE:
%% 
%% For the copyright see the source file.
%% 
%% Any modified versions of this file must be renamed
%% with new filenames distinct from sample-authordraft.tex.
%% 
%% For distribution of the original source see the terms
%% for copying and modification in the file samples.dtx.
%% 
%% This generated file may be distributed as long as the
%% original source files, as listed above, are part of the
%% same distribution. (The sources need not necessarily be
%% in the same archive or directory.)
%%
%% Commands for TeXCount
%TC:macro \cite [option:text,text]
%TC:macro \citep [option:text,text]
%TC:macro \citet [option:text,text]
%TC:envir table 0 1
%TC:envir table* 0 1
%TC:envir tabular [ignore] word
%TC:envir displaymath 0 word
%TC:envir math 0 word
%TC:envir comment 0 0
%%
%%
%% The first command in your LaTeX source must be the 


% \documentclass[sigconf, anonymous]{acmart}
\documentclass[sigconf]{acmart}
\usepackage{flushend,cuted}
\usepackage{enumitem}
\usepackage{booktabs}
\usepackage{graphicx}
\usepackage{array}
\usepackage{lineno}
\usepackage{multirow}
\usepackage{hhline}
\usepackage{makecell}
\setlist[itemize]{leftmargin=10pt}
\usepackage[marginal]{footmisc}
% \usepackage{balance}

% \usepackage{flushend}

%% NOTE that a single column version may required for 
%% submission and peer review. This can be done by changing
%% the \doucmentclass[...]{acmart} in this template to 
%% \documentclass[manuscript,screen]{acmart}
%% 
%% To ensure 100% compatibility, please check the white list of
%% approved LaTeX packages to be used with the Master Article Template at
%% https://www.acm.org/publications/taps/whitelist-of-latex-packages 
%% before creating your document. The white list page provides 
%% information on how to submit additional LaTeX packages for 
%% review and adoption.
%% Fonts used in the template cannot be substituted; margin 
%% adjustments are not allowed.

%%
%% \BibTeX command to typeset BibTeX logo in the docs
\AtBeginDocument{%
  \providecommand\BibTeX{{%
    \normalfont B\kern-0.5em{\scshape i\kern-0.25em b}\kern-0.8em\TeX}}}


\copyrightyear{2025}
\acmYear{2025}
\setcopyright{acmlicensed}
\acmConference[WWW Companion '25]{Companion Proceedings of the ACM Web Conference 2025}{April 28-May 2, 2025}{Sydney, NSW, Australia}
\acmBooktitle{Companion Proceedings of the ACM Web Conference 2025 (WWW Companion '25), April 28-May 2, 2025, Sydney, NSW, Australia}
\acmDOI{10.1145/3701716.3715251}
\acmISBN{979-8-4007-1331-6/25/04}
% 1 Authors, replace the red X's with your assigned DOI string during the rightsreview eform process.
% 2 Your DOI link will become active when the proceedings appears in the DL.
% 3 Retain the DOI string between the curly braces for uploading your presentation video.

\settopmatter{printacmref=true}

%%
%% end of the preamble, start of the body of the document source.
\begin{document}
%\linenumbers


%%
%% The "title" command has an optional parameter,
%% allowing the author to define a "short title" to be used in page headers.
\title{NLGR: Utilizing Neighbor Lists for Generative Rerank in Personalized Recommendation Systems}

%%
%% The "author" command and its associated commands are used to define
%% the authors and their affiliations.
%% Of note is the shared affiliation of the first two authors, and the
%% "authornote" and "authornotemark" commands
%% used to denote shared contribution to the research.
% \author{Ben Trovato}
% \authornote{Both authors contributed equally to this research.}
% \email{trovato@corporation.com}
% \orcid{1234-5678-9012}
% \author{G.K.M. Tobin}
% \authornotemark[1]
% \email{webmaster@marysville-ohio.com}
% \affiliation{%
%   \institution{Institute for Clarity in Documentation}
%   \streetaddress{P.O. Box 1212}
%   \city{Dublin}
%   \state{Ohio}
%   \country{USA}
%   \postcode{43017-6221}
% }

\author{Shuli Wang}
\authornote{Corresponding author.}
\affiliation{%
  \institution{Meituan}
   \city{Chengdu}
  \country{China}
}
\email{wangshuli03@meituan.com}

\author{Xue Wei}
\affiliation{%
\institution{Meituan}
   \city{Chengdu}
  \country{China}
  }
\email{weixue06@meituan.com}

\author{Senjie Kou}
\affiliation{%
\institution{Meituan}
   \city{Chengdu}
  \country{China}
  }
\email{kousenjie@meituan.com}

\author{Chi Wang}
\affiliation{
\institution{Meituan}
   \city{Chengdu}
  \country{China}
  }
\email{wangchi06@meituan.com}

\author{Wenshuai Chen}
\affiliation{%
\institution{Meituan}
   \city{Chengdu}
  \country{China}
  }
\email{chenwenshuai@meituan.com}

\author{Qi Tang}
\affiliation{%
  \institution{Meituan}
   \city{Chengdu}
  \country{China}
}
\email{tangqi22@meituan.com}

\author{Yinhua Zhu}
\affiliation{%
\institution{Meituan}
   \city{Chengdu}
  \country{China}
  }
\email{zhuyinhua@meituan.com}

\author{Xiong Xiao}
\affiliation{%
\institution{Meituan}
   \city{Chengdu}
  \country{China}
  }
\email{xiaoxiong02@meituan.com}

\author{Xingxing Wang}
\affiliation{%
\institution{Meituan}
   \city{Beijing}
  \country{China}
  }
\email{wangxingxing04@meituan.com}

% \author{Yingya Guo}
% \authornote{Corresponding author}
% \affiliation{%
%   \institution{College of Computer and Data Science Fuzhou University.}
%   \city{Fuzhou}
%   \country{China}
%   }
% \email{guoyy@fzu.edu.cn}

%%
%% By default, the full list of authors will be used in the page
%% headers. Often, this list is too long, and will overlap
%% other information printed in the page headers. This command allows
%% the author to define a more concise list
%% of authors' names for this purpose.
\renewcommand{\shortauthors}{Shuli Wang et al.}

%%
%% The abstract is a short summary of the work to be presented in the
%% article.
\begin{abstract}
% 重新排名通过重新排列初始排名列表来模拟项目之间的相互作用,在现代多阶段推荐系统中发挥着至关重要的作用。由于组合搜索空间这样的固有挑战,当前的一些研究采用了评估器-生成器范式,一个生成器生成可行的序列,一个评估器根据估计的列表效用选择最佳序列。然而这些方法仍然存在两个问题,第一, 由于缺少有效指导,生成器容易拟合曝光分布的局部最优解而非排列空间最优。第二, 一一生成目标项目的生成策略难以达到最优由于忽略了后续项目的信息。
Reranking plays a crucial role in modern multi-stage recommender systems by rearranging the initial ranking list. Due to the inherent challenges of combinatorial search spaces, some current research adopts an evaluator-generator paradigm, with a generator generating feasible sequences and an evaluator selecting the best sequence based on the estimated list utility. However, these methods still face two issues. Firstly, due to the goal inconsistency problem between the evaluator and generator, the generator tends to fit the local optimal solution of exposure distribution rather than combinatorial space optimization. Secondly, the strategy of generating target items one by one is difficult to achieve optimality because it ignores the information of subsequent items.
% Reranking, a process that models the mutual influence among items and generates the best recommendation list, has attracted increasing attention from academia and industry. 
% However, generating optimal lists directly is challenging because the optimal list may not even appear in the training data.
% Some existing models use heuristic methods such as greedy to generate lists, which brings the evaluation-before-reranking problem and fails to be effective.
% To alleviate this problem, other methods attempt to evaluate as many candidate lists as possible. However, these approaches still result in unacceptable time consumption.

% 为了解决这些问题,我们提出了一个利用临近列表的重排方法(NLGR),旨在提高生成器在组合空间中的效果。NLGR遵循评估器-生成器范式,并且改进了生成器的训练方式和生成方式。具体来说,为了有效训练生成器,我们构造并评估被曝光列表的非曝光邻居列表,使生成器感知组合空间中的相对分数并找到优化方向。 进一步,我们提出了一种新的基于采样的非自回归生成方法,它可以使生成器灵活地从当前列表跳转至任意邻居列表。
To address these issues, we propose a utilizing \textbf{N}eighbor \textbf{L}ists model for \textbf{G}enerative \textbf{R}eranking (NLGR), which aims to improve the performance of the generator in the combinatorial space.
NLGR follows the evaluator-generator paradigm and improves the generator's training and generating methods. 
Specifically, we use neighbor lists in combination space to enhance the training process, making the generator perceive the relative scores and find the optimization direction.
Furthermore, we propose a novel sampling-based non-autoregressive generation method, which allows the generator to jump flexibly from the current list to any neighbor list.
Extensive experiments on public and industrial datasets validate NLGR's effectiveness and we have successfully deployed NLGR on the Meituan food delivery platform.
\end{abstract}

\begin{CCSXML}
<ccs2012>
   <concept>
       <concept_id>10002951.10003317.10003338</concept_id>
       <concept_desc>Information systems~Retrieval models and ranking</concept_desc>
       <concept_significance>500</concept_significance>
       </concept>
   <concept>
       <concept_id>10002951.10003227.10003447</concept_id>
       <concept_desc>Information systems~Computational advertising</concept_desc>
       <concept_significance>500</concept_significance>
       </concept>
 </ccs2012>
\end{CCSXML}

\ccsdesc[500]{Information systems~Retrieval models and ranking}
\ccsdesc[500]{Information systems~Computational advertising}

%%
%% Keywords. The author(s) should pick words that accurately describe
%% the work being presented. Separate the keywords with commas.
\keywords{Recommender Systems, Reranking, Generative Model}

%% A "teaser" image appears between the author and affiliation
%% information and the body of the document, and typically spans the
%% page.


% \received{20 February 2007}
% \received[revised]{12 March 2009}
% \received[accepted]{5 June 2009}

%%
%% This command processes the author and affiliation and title
%% information and builds the first part of the formatted document.
\maketitle

\section{Introduction}
E-commerce platforms, such as Meituan and Taobao, need to provide users with personalized services from millions of items. To improve recommendation efficiency, personalized recommendation systems generally include three stages: matching,  ranking, and reranking. The ranking models (e.g.,Wide\&Deep \cite{W&D}, DeepFM \cite{deepfm}, DIN \cite{din}) evaluate the point-wise items respectively based on the Click-Through Rate (CTR), but they ignore the crucial mutual influence among items \citeN{chang2023twin,pi2020search}. 
Research \citeN{burges2010ranknet, listnet, ai2018learning, pang2020setrank} indicates that optimizing a list-wise utility is a more advantageous strategy, as it capitalizes on the mutual influences between items within the list to enhance overall performance.


% 重排的关键挑战是在巨大的组合空间中探索最佳序列。最开始,有一些list-wise的方法通过建模列表中的上下文,对列表中的项目进行重新评估和打分。这些list-wise的方法可以获得比point-wise方法更精确的得分。然而他们依据得分高低,采用贪心策略进行重新排序。这些方法面临evaluation-before-ranking问题导致无法实现综合空间最优。
The key challenge of reranking is to explore the optimal list in the huge combinatorial space \citeN{gfn,nar4rec}. Initially, some list-wise methods \citeN{prm,midnn,dlcm} re-evaluate and score items within lists by modeling the context. These list-wise methods can obtain more accurate scores than point-wise methods, then they use a greedy strategy to reorder based on the list-wise score. However, these methods face the evaluation-before-reranking problem \citeN{xi2021context, grn} and cannot achieve optimization in combinatorial space.
% 为了解决这个问题,一个简单的解决方案是评估每一种可能的排列,这是全局最优的,但过于复杂,无法满足工业系统中严格的推理时间约束。 因此,大多数现有的重排序框架采用两阶段架构,其中包括一个生成器和评估器。在生成器-评估器范式中,生成器起着至关重要的作用。一些方法利用启发式的方法作为生成器,如beam-search,simhash。这些方法的生成器没有利用评估器的信息导致效果有限。最近,一些方法利用生成模型作为生成器,取得优于启发式方法的效果。
To resolve the problem, a straightforward solution is to evaluate every possible permutation, which is global-optimal but is too complex to meet the strict inference time constraint in industrial systems. Therefore, most existing evaluator-based reranking framework uses a two-stage architecture \citeN{feng2021revisit, xi2021context, grn, nar4rec, dcdr} which consists of a generator and an evaluator.  Within the generator-evaluator paradigm, the generator plays a crucial role \cite{nar4rec}. Some methods use heuristic methods as generators, such as beam-search \cite{medress1977speech} and SimHash \cite{chen2021end}. The generators of these methods do not utilize the information of the evaluator, resulting in limited effectiveness. Recently, some methods \citeN{grn,nar4rec,dcdr} utilize generative models as generators and achieve better results than heuristic methods.

% 然而,现在的生成式重排方法面临两个问题。第一, 由于缺少有效指导,生成器难以找到组合空间中的最优列表。评估器被训练拟合项目的list-wise评分。而生成器被期望能将排序列表转换成最优列表。评估器和生成器之间的目的差距导致它们之间的知识传输很难, 并且在极端情况下导致生成器仅能拟合曝光分布。第二, 一一生成目标项目的生成策略难以达到最优。顺序解码过程仅关注前面的项目,忽略后续项目的信息。由于模型无法充分利用可用的上下文,因此这种限制会导致性能不佳
However, existing generative reranking methods face two significant challenges. Firstly, due to the goal inconsistency problem, the generator has difficulty finding the optimal list in the combinatorial space. While the evaluator is trained to fit list-wise scores of items, the generator is tasked with transforming any candidate list into the optimal one. This disparity in objectives between the evaluator and the generator complicates the transfer of guidance, often causing the generator to merely fit exposure distributions in extreme cases. Secondly, the strategy of generating target items sequentially, one by one, hinders the achievement of optimal results. The sequential decoding process focuses solely on preceding items, neglecting information from succeeding items. This limitation leads to suboptimal performance as the model fails to fully leverage the available context.

% 为了应对上述挑战,我们提出利用一个新颖的利用未曝光列表的生成式重排方法,叫做NLGR。我们的方法仍然遵循评估器生成器范式,并且评估器仅用于离线帮助生成器训练。我们的改进主要有两点:一,我们提出一种利用非暴露邻居列表的增强训练过程,使得生成器感知组合空间中的相对分数并找到优化方向。如图1所示,通过评估多个相邻列表和多次迭代,生成器将达到最优陈述并生成最优列表。二,我们提出基于采样的非自回归生成方法,可以更加灵活地在组合空间中寻找最优列表。我们在美团环境中全面部署了NLGR,在线下和线上都取得了很好的提升。
To address the aforementioned challenges, we propose a novel \textbf{G}enerative \textbf{R}eranking method that utilizes \textbf{N}eighbor \textbf{L}ists, named NLGR. Our approach still follows the evaluator-generator paradigm, with the evaluator solely assisting in the generator's offline training. Our improvements are twofold as follows. First, we introduce an enhanced training process that utilizes neighbor lists, enabling the generator to perceive relative scores within the combinatorial space and identify the optimal direction. As depicted in Figure \ref{fig:gradient},  by evaluating multiple neighboring lists and iterating several times, the generator will converge to an optimal state. Second, we propose a sampling-based non-autoregressive generation method. This method first determines the position of the item that needs to be replaced using the Position Decision Unit (PDU), and then retrieves new replacement items from the candidate item set using the Candidate Retrieval Unit (CRU), which allows for a more flexible exploration of the combinatorial space to find the optimal list.

\begin{figure}[h]
\centering
\includegraphics[width=\linewidth, height=2\textheight, keepaspectratio]{Figure/fig1.png}

\caption{Generator optimization legend: reach the optimal step by step under the guidance of the neighbor lists.}
\label{fig:gradient}
\end{figure}

In summary, the contributions of this work are as follows:
\begin{itemize}
\item We propose a novel generative reranking method that utilizes neighbor lists to address the goal inconsistency problem between evaluators and generators. To the best of our knowledge, we are the first to propose and attempt to solve this problem.

\item We propose a novel sampling-based non-autoregressive generation method that generates the optimal list more flexibly in the combination space.

\item We have verified the superior performance of NLGR through extensive experiments on both offline and online datasets. It is notable that NLGR has been deployed in the Meituan food delivery platform and has achieved significant improvement under various metrics.

\end{itemize}

\section{Related Work}
\subsection{Reranking Methods}

Typical reranking methods can be divided into two categories \cite{pier}. The first category is the one-stage reranking methods, which only generates one list as output by capturing the mutual influence among items. Seq2slate \cite{seq2slate} utilizes pointer-network and MIRNN \cite{zhuang2018globally} utilizes GRU to determine the item order one-by-one. Methods such as PRM \cite{prm} and DLCM \cite{ai2018learning} take the initial ranking list as input, use RNN or self-attention to model the context-wise signal, and output the predicted value of each item. Such methods bring an evaluation-before-reranking problem \cite{xi2021context} and lead to suboptimal. Similarly, methods such as EXTR \cite{extr} estimate pCTR of each candidate item on each candidate position, which are substantially point-wise models and thus limited in extracting exact context. MIR \cite{mir} capturing the set2list interactions by a permutation-equivariant module
Another category is the two-stage reranking methods, which tries to evaluate every possible permutation through a well-designed context-wise model. This is a global-optimal method but is too complex to meet the strict inference time constraint in industrial systems. To reduce the complexity, PRS \cite{feng2021revisit} adopts beam-search to generate a few candidate permutations first, and score each permutation through a permutation-wise ranking model. PIER \cite{pier} applies SimHash to select top-K candidates from the full permutation.  

\subsection{ Generative Reranking Solutions}

In recent years, the generative reranking model \citeN{listcvae, pivotcvae, slateq} for listwise recommendation has been a topic of discussion. To manage the vast combinatorial output space of lists, the generative approach directly models the distribution of recommended lists and employs deep generative models to generate a list as a whole. For instance, ListCVAE \cite{listcvae} utilizes conditional variational autoencoders (CVAE) to capture the positional biases of items and the interdependencies within the list distribution. But Pivot-CVAE \cite{pivotcvae} indicates that ListCVAE suffers from a trade-off dilemma between accuracy and diversity, and proposes an "elbow" performance to enhance the accuracy-based evaluation. 


GFN \cite{gfn} uses a flow-matching paradigm that maps the list generation probability with its utility. Essentially it is still studying list distributions rather than directly modeling the permutation space, so it still has the challenge mentioned above. GRN \cite{grn} proposes an evaluator-generator framework to replace the greedy strategy, but it can't avoid the evaluation-before-reranking problem \cite{xi2021context} because it takes the rank list as input to the generator. DCDR \cite{dcdr} introduces diffusion models into the reranking stage and presents a discrete conditional diffusion reranking framework. NAR4Rec \cite{nar4rec} uses a non-autoregressive generative model to speed up sequence generation. However, these methods still face the two problems mentioned above.






\section{Problem Definition}
In the Meituan food delivery platform, we adhere to the matching, ranking, and reranking recommendation paradigms to present items to users in a list format.
Initially, we define a user set $U$ and an item set $I$. We utilize the session-level users' historical interacted lists $B$ and candidate set $C$ to represent the user's features $u \in U$, which consistently holds in our reranking scenario $C \in I$, and ultimately select the list $L$ for users. 

Reranking introduces a combination space with exponential size, represented as $\mathcal{O}(A_n^m)$, where $n$ represents the size of the candidate set $C$ and $m$ represents the size of the output list $L$. 
Our optimization objective is to learn a strategy $\pi: C \rightarrow L$ by maximizing the list score reward $R(u,\pi)$. The list score reward $R(u, \pi)$ takes into account factors such as click-through rate (CTR) and conversion rate (CVR). 

% 邻居列表。如果两个列表的距离为1,即这两个列表只相差1个项目,我们将这两个列表定义为邻居列表。显然,如果我们交换一个列表中两个项目,那么产生的新列表距离为2.
\textbf{Neighbor List.} If the distance between two lists is 1, that is, the two lists only differ by 1 item, we define the two lists as neighbor lists. If we swap two items in a list, the new list will have a distance of 2.

\section{Proposed Method}

% 在本节中,我们将详细介绍 NLGR。我们首先介绍NLGR的评估器和生成器。随后我们介绍了NLGR的训练过程。其中评估器仅用于离线帮助生成器训练,因此NLGR是一个推理很快的方法。
In this section, we present a detailed introduction of NLGR. 
We first introduce the evaluator (in Section \ref{section:evaluator}) and generator (in Section \ref{section:generator}) of NLGR, denoted as NLGR-E and NLGR-G respectively. Then, we demonstrate the offline training process of NLGR in Section \ref{section:offline_training}. The evaluator is only assisting in the offline training of the generator, so NLGR is a fast inference method.


\subsection{Evaluator Model}\label{section:evaluator}

% 我们使用评估器NLGR-E去评估一个排好序的序列,如展示给用户的曝光序列和生成器生成的序列。NLGR-E的结构如图2左边所示,它主要包括两个输入:待评价的曝光序列和用户会话级别行为序列
We use NLGR-E to evaluate a ranked sequence, such as the exposed list which is displayed to the user, or the candidate lists generated by NLGR-G.
The structure of NLGR-E is shown on the left side of Figure \ref{fig:NLGR}. 

% NLGR-E 包括两个输入:要评估的公开列表和用户会话级行为序列,其中用户会话级行为序列中的每一个session都是用户的历史曝光列表。NLGR-E的评估过程如下:
NLGR-E includes two inputs: the exposed list to be evaluated and the user session-level behavior sequence, where each session in the user session-level behavior sequence is the user's historical exposed list. The evaluation process of NLGR-E is as follows: 

First, we use an embedding layer to get the embedding of the original input, donated as $\mathbf{X} \in \mathbb{R}^{m \times F \times D}$ and $\mathbf{M} \in \mathbb{R}^{H \times m \times F \times D}$ respectively, where $H$ represents the number of history sessions, $m$ represents the number of items in each list, $F$ represents the number of feature fields for each item (i.g., ID, category, position index), and $D$ represents the dimension of the embedding. 
Inspired by DIF \cite{dif}, to avoid feature interference, we propose the D-Attention unit to \underline{d}ecouple the feature context information.
We first calculate the attention score on $i$-th attribute $\mathbf{X}_{i} \in \mathbb{R}^{m \times D}$:
\begin{equation} \label{equation:1}
\mathrm{Att}_{i} = \sigma \left( \frac{(\mathbf{X}_{i} \mathbf{W}^Q_{i})(\mathbf{X}_{i} \mathbf{W}^K_{i})^\top}{\sqrt{D}}\right),   \forall i \in [F] , 
\end{equation}
where $\mathbf{W}_{i}^Q$ and $\mathbf{W}_{i}^K \in \mathbb{R}^{D \times D}$ denote the weight matrices.

Then we aggregate all attention matrices $\mathrm{Att}_{i} \in \mathbb{R}^{m \times m}$ and get the item-level attention score $\mathrm{Att_{all}} \in \mathbb{R}^{m \times m}$:
\begin{equation} \label{equation:2}
\mathrm{Att_{all}} = \frac{1}{F} \sum_{f=1}^{F} \mathrm{Att}_{i}.
\end{equation}

Subsequently, we aggregate ID feature embeding $\mathbf{X}^{id} \in \mathbb{R}^{m \times D}$ and obtain each exposed list's representation $\mathbf{e}^l \in \mathbb{R}^{D}$:
\begin{equation} \label{equation:3}
\mathbf{e}^l = \mathrm{reduce\_mean}\left(\mathrm{Att_{all}}(\mathbf{X}^{id}\mathbf{W}^V)\right).
\end{equation}

% 同样的,对用户会话级行为序列中的每个session进行上述操作,我们可以得到每个session的表征A。然后我们将A输入MLP就能得到用户表征:
Similarly, by performing the above operations on each session in the user session-level behavior sequence, we can obtain the representation $\mathbf{e}^s_i \in \mathbb{R}^{ D} $ of each session. Then we input $\mathbf{e}^s_i$ into a Self-Attention layer (SA) \citeN{vaswani2017attention,kang2018self} to get the user representation $\mathbf{e}^u \in \mathbb{R}^{D}$:
\begin{equation} \label{equation:4}
\mathbf{e}^u = \mathrm{SA}(\mathbf{e}^s_1 || \mathbf{e}^s_2 ||...||\mathbf{e}^s_H),
\end{equation}
where || represents concatenate operate. 

Finally, the predicted Click-Through Rate (pCTR) of the $j$-th item can be represented as:
\begin{equation} \label{equation:5}
\hat{y}_j=\sigma(\mathrm{Tiled\_MLP}(\mathbf{X}_j || \mathbf{e}^l || \mathbf{e}^u || PE_j)),
\end{equation}
where $PE_j$ denotes $j$-th position embedding. Similarly, the pCVR and other evaluations also follow the above process. 

\subsection{Generator model} \label{section:generator}
% 我们利用 NLGR-G 生成组合空间中的最优列表。
We utilize NLGR-G to generate the optimal list in combinatorial space. The structure of NLGR-G is shown on the middle side of Figure \ref{fig:NLGR}. 

% 与NLGR-E相似,NLGR-G 包含两个模块:待重排的候选列表和用户会话级行为序列,其中用户会话级行为序列中的每一个session都是用户的历史曝光列表。理论上,候选列表可以是组合空间中的任一列表,但一般情况下我们将ranking list作为初始输入。NLGR-G的生成过程如下:
Similar to NLGR-E, NLGR-G includes two inputs: the candidate list to be reranked and the user session-level behavior sequence, where each session in the user session-level behavior sequence is the user's historical exposed list. Theoretically, the candidate list can be any in the combinatorial space, but generally, we use the ranking list as the initial input. The generation process of NLGR-G is as follows:

% 首先我们通过公式4获得用户表征,这些参数是从NLGR-E共享的以保证剩余的参数能被更专注的优化。
First, we obtain the user representation $\mathbf{e}^u$ through Eq. \ref{equation:4}. These parameters are shared from NLGR-E to ensure that the remaining parameters can be optimized more focused.

% 然后,我们提出了一种的基于采样的非自回归生成方法。它先通过Position Decision Unit确定需要被替换项目的位置,再通过Candidate Retrieval Unit 从候选项目集合中检索出新的替换项目。
Then, we propose a sampling-based non-autoregressive generation method. It first determines the position of the item that needs to be replaced through the Position Decision Unit (PDU), then retrieves new replacement items from the candidate item set through the Candidate Retrieval Unit (CRU).

\subsubsection{Position Decision Unit (PDU)}
% 首先,我们使用嵌入层来获取候选列表的嵌入。然后,我们利用MLP计算第j个位置被选中的几率
First, we use the embedding layer to get the embedding of the candidate list, donated as $\mathbf{X} \in \mathbb{R}^{m \times F \times D}$. Then, we flatten $j$-th item embedding $\mathbf{X}_j \in \mathbb{R}^{F \times D}$ and use a Fully Connected layer (FC) to calculate the selected logit of the $j$-th position:
\begin{equation} \label{equation:6}
h_j=\mathrm{FC}_1(\mathbf{X}_j || \mathbf{e}^l || \mathbf{e}^u || PE_j).
\end{equation}

\begin{figure*}[h]
\centering
\includegraphics[width=\textwidth]{Figure/NLGR.png}
\caption{The overall architecture of NLGR.}
\label{fig:NLGR}
\end{figure*}

% 为了解决采样分布的不可微问题,我们利用Gumbel-Softmax Trick进行采样:在反向传播时,使用公式7计算梯度。在前向传播时,被替换的位置为
To solve the non-differentiable problem of the sampling distribution, inspired by \citeN{jang2016categorical,liu2021neural,huijben2022review}, we use the Gumbel-softmax trick for sampling:
\begin{equation} \label{equation:7}
r^p_j=\mathrm{softmax}\left(\frac{log(h_j)+n}{\tau}\right),\forall j \in [m],
\end{equation}
where $\tau > 0$ is a temperature parameter, $n=-\mathrm{log}(-\mathrm{log}(u)))$ represents random noise sampled from the Gumbel distribution, $u$ is a uniform distribution between [0, 1]. During backpropagation, the gradient is calculated using Eq. \ref{equation:7}. While during forward propagation, the replaced position is $j=\mathrm{argmax}(r^p_j)$.
 
\subsubsection{Candidate Retrieval Unit (CRU)} 
% 在确定了要替换的位置p后,我们需要从n个候选项目中选择一个合适的放在位置p上。由于这个操作在生成过程中会被循环多次,出于时间考虑,我们提出利用检索的思想来快速达到这个目的。首先我们去掉候选列表的位置p得到Xp,然后提取列表表示ep
After determining the position $j$ to be replaced, we need to select a suitable one from $n$ candidate items and place it at position $j$. Since this operation will be repeated multiple times during the generation process, we propose leveraging retrieval-based techniques to quickly achieve this goal for efficiency. 
First, we mask the position $j$ of the candidate list, denoted as $\mathbf{X}_j^{mask} \in \mathbb{R}^{m \times F \times D}$, and then extract the list representation $\mathbf{e}_j^{mask} \in \mathbb{R}^{D}$:
\begin{equation} \label{equation:8}
\mathbf{e}_j^{mask}=\mathrm{SA}(\mathbf{X}_j^{mask}).
\end{equation}

% 然后我们计算每个候选项目在位置p上的表征:
We then compute the representation of each candidate item at position $j$:
\begin{equation} \label{equation:9}
\mathbf{e}^c_k=\mathrm{FC}_2(\mathrm{flatten}(\mathbf{X}^c_k) || PE_j), \forall k \in [n],
\end{equation}
where $\mathbf{X}^c \in \mathbb{R}^{n \times F \times D}$ denotes embedding of candidate set $C$, and $\mathbf{X}^c_k$ denotes $k$-th candidate item's embedding.

Then we use an FC layer to calculate the selected logit of the $j=k$-th candidate item:
\begin{equation} \label{equation:10}
g_k=\mathrm{FC}_3(\mathbf{e}^c_k || \mathbf{e}_j^{mask} || \mathbf{e}^u || PE_j).
\end{equation}

% 同样的,利用gumbel-softmax trick 克服non-differentiable problem
Similarly, to overcome non-differentiable problems, we use the Gumbel-softmax trick for sampling:
\begin{equation} \label{equation:11}
r^c_k=\mathrm{softmax}\left(\frac{log(g_k)+n}{\tau}\right),\forall k \in [n],
\end{equation}
During backpropagation, the gradient is calculated using Eq. \ref{equation:10}. While during forward propagation, the newly inserted item is $c=\mathrm{argmax}(r^c_k)$.

% 注意,生成过程可能会重复很多次,直到新插入项目等于被替代项目或者ra和rb的值过低。
\textbf{Stop condition.} Note that the generation process may be repeated many times until the newly inserted item equals the replaced item or the values of $r^p_j$ and $r^c_k$ are too low.

% \begin{itemize}

% \item Candidate Set Perception Unit (CSPU): CSPU extracts context information, denoted as $E_s \in \mathbb{R}^{D}$, from the candidate set, which is similar to CLPU. It is worth noting that we eliminate the position embedding to circumvent the evaluation-before-reranking problem \cite{xi2021context}, 

% \item Generation Estimation Unit (GEU): GEU generates and scores a sorted candidate list. First, the user representation $E_u \in \mathbb{R}^{D} $ is obtained by calculating the target attention of $E_s$ and $E_u^m$. 
% Then, we use $E_u$ as the initial vector of the decoder, and one by one, we select the item with the maximum global income estimation value, and the selected item will be masked. To ensure the training process is differentiable, we introduce $\mathrm{gumbel-softmax}$, a reparameterization trick, to continuously relax $\mathrm{argmax}$. Given the candidate set ${C_i}$, the sample ratio is:

% \begin{equation} \label{equation:gumbel}
% ratio = \mathrm{softmax}\left(\frac{\mathrm{log}(C)+n}{\tau}\right)
% \end{equation}
% where $\tau > 0$ is a temperature parameter, $n=-\mathrm{log}(-\mathrm{log}(u)))$ represents random noise sampled from the Gumbel distribution, $u$ is a uniform distribution between [0, 1].

% Finaly, the pCTR of the $k$-th item can be represented as:$\hat{y}_k=\sigma(MLP(E_u||E_s||E_k||PE_k))$.
% \end{itemize}

% It should be emphasized that to achieve full sight, we select items based on estimation values in permutation space when generating the optimal sequence, rather than the listwise pCTR (similar to the output of NLGR-E). NLGR-E will generate rewards to guide the training of NLGR-G, the details will be introduced in Section \ref{section:train_G}.





\subsection{Utilizing Neighbor Lists Training} \label{section:offline_training}
In this section, we elaborate on the offline training process of NLGR, which includes the training procedures for NLGR-E and NLGR-G. As mentioned before, the evaluator is trained to fit list-wise scores of items, and the generator is tasked with transforming any ranking list into the optimal one. This goal inconsistency between the evaluator and the generator complicates the transfer of guidance. We will introduce our solution in detail below.



\subsubsection{\textbf{Training of NLGR-E}}
To accurately evaluate the return of the exposure list and estimate the listwise pCTR value of the exposure list, we train NLGR-E using real data collected from online logs. The input is the features of the recommended advertisement sequences exposed in reality online, and the advertising return situation, including exposure, click, conversion, and other performance indicators, is used as the label to supervise the training of NLGR-E, enabling it to accurately evaluate the return of the recommended sequence. The loss of NLGR-E is calculated as follows:

\begin{equation}
\mathcal{L}^E  =\sum_{j=1}^m \left( - y_{j}  \log(\widehat {y}_{j})-(1- y_{j})  \log  {(1-\widehat {y}_{j}} ) \right),
\end{equation}
where $y_j$ represents the real label, $\widehat{y}_{j}$ represents the predicted value of NLGR-E, and the evaluation is carried out for the $m$ items in the exposure list in turn.

\subsubsection{\textbf{Training of NLGR-G}} \label{section:train_G}

%为了解决前面提到的目标不一致问题,我们使用非暴露邻居列表来指导反事实空间内的NLGR-G。对于NLGR-G生成的每一个列表,NLGR-E会模拟人类反馈\cite{rlhf}并提供奖励$R$以指导NLGR-G训练。
To address the problem of goal inconsistency mentioned before, we use neighbor lists to guide NLGR-G within the counterfactual space. For each list generated by NLGR-G, NLGR-E simulates human feedback and provides a reward $R$ to guide NLGR-G training. 

% 图3展示了一个NLGR-G训练过程的例子。首先,对于每个候选列表$L^o=[i^o_1,i^o_2,...,i^o_m]$,我们从候选集合中采样出一个替换项目$i^*$,替换$L_o$第$j$个项目,构造出邻居列表$L^*_p=[i^o_1,i^o_2,...i^*_p,...,i^o_m]$。重复并替换每个位置,我们可以得到一组邻居列表$L^*=[L^*_1,L^*_2,...,L^*_m]$.
Figure \ref{fig: counterfactual} shows an example of the NLGR-G training process. First, for each candidate list $L^o=[i^o_1,i^o_2,...,i^o_m]$, we sample a replacement item $i^*$ from the candidate set $C=[i_1,i_2,...,i_n]$, replace the $j$-th item $L_o$'s, and construct a neighbor list $L^*_j=[i^o_1,i^o_2,...i^ *_j,...,i^o_m]$. Repeating and replacing each position, we can get a set of neighbor lists $L^*=[L^*_1,L^*_2,...,L^*_m]$.

\begin{figure}[h]
\centering
\includegraphics[width=0.45\textwidth]{Figure/neighbor_list.png}
\caption{The training process of NLGR-G for a candidate list of length 3.}
\label{fig: counterfactual}
\end{figure}

% 然后,我们利用训练好的NLGR-E,对候选列表$L^o$和邻居列表$L^*$进行评估,pCTR、pCVR等指标会被预估。然后我们基于业务指标将预估值转化为list reward,如下:
Then, we use the trained NLGR-E to evaluate the candidate list $L^o$ and the neighbor list $L^*$, and indicators such as pCTR and pCVR will be estimated. We convert the estimated value into list reward based on business indicators as follows:
\begin{equation}\label{equation:r}
\overline{r}=
\begin{cases}
e^{w-1} -1 & \text{if }w>1 \\
0          & \text{if }w=1 \\
1-e^{1-w}  & \text{if }w<1
\end{cases},
\end{equation}

\begin{equation}
w = k1\cdot L_{ctr}+k2\cdot L_{ctr}\cdot L_{cvr},
\end{equation}
where $L_{ctr}$ and $L_{cvr}$ represent the list total pCTR and pCVR which are evaluated by NLGC-E, respectively. The parameters $k_1$ and $k_2$ are business parameters that depend on the click bid and conversion price in the specific business.

% 经过公式13,我们可以得到临近列表的reward和原始候选列表的reward,分别表示为。NLGR-G被期望从当前列表迭代到更优的列表,因此我们计算每个位置的相对reward:
Through Eq. \ref{equation:r}, we can get the rewards of the neighbor lists $L^*$ and the original candidate list $L^o$, denoted as $\overline{r}=[\overline{r}_1, \overline{r}_2,...,\overline{r}_m]$ and $\overline{r_o}$ respectively. NLGR-G is tasked to iterate from the candidate list to a more optimal list, so we calculate the relative reward for each position $j$:
\begin{equation} \label{equation:R}
r_j = \overline{r}_j - \overline{r}_o, \forall j \in [m].
\end{equation}

The authentic evaluation $R$ for the candidate list $L^o$ is obtained by aggregating the relative rewards of all positions. And we define the counterfactual loss of NLGR-G as $-R$:
\begin{equation} \label{equation:R}
\mathcal{L}^G_ {1} = -R = -\sum_{j=1}^{m} r_j.
\end{equation}

% 进一步,为了增加NLGR-G的生成过程的稳定性,我们建议利用位置奖励$r_j$对PDU进行额外指导。具体来说,我们引入交叉熵损失作为辅助损失,来衡量位置采样$r^p$和位置奖励$r_j$的分布差异。
Furthermore, to increase the stability of the NLGR-G's generation process, we propose using the position reward $r_j$ to provide additional guidance to the Position Decision Unit (PDU). Specifically, we introduce cross-entropy loss as an auxiliary loss to measure the distribution difference of position sampling $r^p$ and position reward $r_j$:

\begin{equation}
\mathcal{L}^G_{2}  = -\sum_{j=1}^m \mathrm{Norm}(r_j) \cdot \mathrm{log}r^p_j,
\end{equation}
where $\mathrm{Norm}(r_j)=\frac{r_j}{\sum r_j}$.

The final loss of NLGR-G in each batch is: 
\begin{equation}\label{equation:loss_G}
\mathcal{L}^G =\frac{1}{|B|}\sum_{B}
(\mathcal{L}_1^G + \alpha \cdot \mathcal{L}_2^G),
\end{equation}
where $\alpha$ is a coefficient to balance the two losses.
 
\section{Experiments}
To validate the superior performance of NLGR, we conducted extensive offline experiments on the Meituan dataset and verified the superiority of NLGR in online A/B tests. In this section, we first introduce the experimental setup, including the dataset and baseline. Then, in Section \ref{exp_result}, we present the results and analysis of various reranking methods in both offline and online A/B tests.

\subsection{Experimental Setup}
\subsubsection{Dataset}
In order to verify the effectiveness of NLGR, we conduct sufficient experiments on both public dataset and industrial dataset. For public dataset, we choose Taobao Ad dataset\footnote{https://tianchi.aliyun.com/dataset/56}. For the industrial dataset, we use real-world data collected from Meituan food delivery platform. 

\begin{itemize}[leftmargin=*]
\item  Taobao Ad. It is a public dataset collected from the display advertising system of Taobao. This dataset contains more than 26 million interaction records of 1.14 million users within 8 days. Each sample comprises five features: user ID, timestamp, behavior type, item brand ID, and category ID. 
\item Meituan. It is an industrial dataset collected from the Meituan food delivery platform during October 2023, which contains 1.3 billion interaction records of 130 million users within 30 days. The dataset includes 239 features, two labels: click and conversion, and collects all items on the same page as one record. We divide the dataset into training and test sets with a proportion of 9:1.
\end{itemize}


Table \ref{tab:my_table} gives a brief introduction to the datasets.
\begin{table}[H]
\caption{Statistics of datasets}
\label{tab:my_table}
\begin{tabular}{cccc}
\hline
\textbf{Dataset} & \textbf{\#Users} & \textbf{\#Items} & \textbf{\#Records} \\ \hline
Taobao Ad               & 1,141,729        & 99,815           & 26,557,961           \\
Meituan          & 130,648,310      & 14,054,691       & 1,331,247,488        \\ \hline
\end{tabular}
\end{table}

\begin{table*}[htb]
\caption{Performance comparison. The best result and the second-best result in each column are in bold and underlined}
\begin{tabular}{llccccccccc}
\toprule[1pt]
\multirow{2}{*}{Dataset} & \multirow{2}{*}{Metric} & \multicolumn{2}{c}{Group I} 
& \multicolumn{1}{c}{} 
& \multicolumn{2}{c}{Group II} 
& \multicolumn{1}{c}{} 
& \multicolumn{2}{c}{Group III} & \multirow{2}{*}{NLGR} \\
    \cline{3-4} 
    \cline{6-7}
    \cline{9-10} 
 & & DNN & DeepFM & & PRM & MIR & & GRN & DCDR &  \\ \hline
% \midrule
\multirow{5}{*}{Taobao Ad} 
&AUC             & 0.5869 & 0.5871 & & 0.6052 & 0.6047 & & 0.6101 & \underline{0.6217} & \textbf{0.6344} $\pm$ 0.5\textperthousand\\
&LogLoss         & 0.1878 & 0.1866 & & 0.1842 & 0.1853 & & 0.1820 & \underline{0.1792} & \textbf{0.1749} $\pm$ 0.1\textperthousand\\
&NDCG@10         & 0.1527 & 0.1548 & & 0.1805 & 0.1769 & & 0.1896 & \underline{0.2038} & \textbf{0.2323} $\pm$ 0.2\textperthousand\\
&NDCG@5          & 0.1092 & 0.1110 & & 0.1206 & 0.1193 & & 0.1273 & \underline{0.1453} & \textbf{0.1830} $\pm$ 0.1\textperthousand\\
% &HR@10\%         & -      & -      & & 0.1125 & -      & & \underline{0.2160} & 0.2159 & \textbf{0.4091} $\pm$ 0.2\textperthousand\\ 
\midrule
\multirow{5}{*}{Meituan} 
&AUC             & 0.7034 & 0.7070 & & 0.8096 & 0.8031 & & 0.8034 & \underline{0.8181} & \textbf{0.8349} $\pm$ 0.4\textperthousand \\
&LogLoss         & 0.1162 & 0.1162 & & 0.1096 & 0.1102 & & 0.1102 & \underline{0.1087} & \textbf{0.1039} $\pm$ 0.1\textperthousand \\
&NDCG@10         & 0.2015 & 0.2019 & & 0.2743 & 0.2742 & & 0.2744 & \underline{0.2793} & \textbf{0.2857} $\pm$ 0.2\textperthousand \\
&NDCG@5          & 0.1569 & 0.1580 & & 0.2378 & 0.2365 & & 0.2353 & \underline{0.2400} & \textbf{0.2431} $\pm$ 0.2\textperthousand\\
% &HR@10\%         & -      & -      & & 0.5702 & -      & & 0.7386 & \underline{0.7573} & \textbf{0.8369} $\pm$ 0.3\textperthousand\\ \hline
\bottomrule[1pt]
\end{tabular}

\label{tab:tab2}
\end{table*}

\begin{table}[h]
\caption{Hit ratio comparison. The best result and the second-best result in each column are in bold and underlined}
\label{tab:3}
\begin{tabular}{l|cc|cc}
\toprule
\multirow{2}{*}{Model} & \multicolumn{2}{c|}{Taobao Ad} & \multicolumn{2}{c}{Meituan}  \\
 & HR@10\% & HR@1\%   & HR@10\% & HR@1\% \\
\midrule
PRM           & 0.1125             & 0.0794             & 0.5702             & 0.4412  \\
GRN           & \underline{0.2160} & 0.0829             & 0.7386             & 0.5783  \\
DCDR          & 0.2159             & \underline{0.0835} & \underline{0.7573} & \underline{0.5802}\\
NLGR          & \textbf{0.4091}    & \textbf{0.3220}    & \textbf{0.8369}    &\textbf{0.7523} \\
\bottomrule
\end{tabular}
\end{table}

\subsubsection{Baseline}
The following six baselines are chosen for comparative experiments and divided into three groups. We select DNN and DeepFM as point-wise baselines (Group I), PRM and MIR as list-wise baselines (Group II), and GRN and DCDR as generative baselines (Group III). A brief introduction of these methods is as follows:

\begin{itemize}[leftmargin=*]
\item $\textbf{DNN}$\cite{dnn} is a basic deep learning method for CTR prediction, which applies MLP for high-order feature interaction.
\item $\textbf{DeepFM}$\cite{deepfm} is a general deep model for recommendation, which combines a factorization machine component and a deep neural network component.
\item $\textbf{PRM}$\cite{prm} adjusts an initial list by applying the self-attention mechanism to capture the mutual influence between items.
\item $\textbf{MIR}$\cite{mir} learns permutation-equivariant representations for the inputted items via self-attention.
mechanism to capture the mutual influence between items.
\item $\textbf{GRN}$\cite{grn} is a generative reranking model which consists of the evaluator for predicting interaction probabilities and the generator for generating reranking results.
\item $\textbf{DCDR}$\cite{dcdr} presents a discrete conditional diffusion reranking framework.
\end{itemize}

\subsubsection{Evaluation Metrics. }
We adopt several metrics, i.e., \textbf{AUC} (Area Under ROC Curve), \textbf{Logloss} and \textbf{NDCG} (normalized discounted cumulative gain) to evaluate NLGR-E in offline experiments. A larger AUC and NDCG indicate better recommendation performance, while Logloss performs the opposite. 

We use \textbf{HR} (Hit Ratio) \cite{alsini2020hit} to evaluate NLGR-G in offline experiments. It is worth noting that only one list produced by reranking algorithms can be presented to the user. As a result, the generator cannot be fully and fairly evaluated. A practical workaround is to employ the evaluator to assess the performance of the generator. For each data, we evaluate all candidate lists using NLGR-E. HR@10\% is 1 only when the rerank list produced by NLGR-G is ranked within the top 10\% as sorted by NLGR-E. The HR metrics are only meaningful with evaluator-based reranking methods. The results of HR can be seen in Table \ref{tab:3}.

% 值得注意的是,AUC 和 HR 能分别衡量评估器和生成器。AUC衡量的是模型评估一个有序列表的能力,而HR衡量的是两阶段的一致性问题。任何一个指标降低都会降低推荐效果。例如,当重排模型直接返回精排的结果时,HR等于1但AUC会降低。相反,当重排模型非常复杂时,会提升AUC但耗时增加会导致HR降低。
It is worth noting that AUC and HR can measure the evaluator and generator respectively. AUC measures the model's ability to evaluate an ordered list, while HR measures the consistency between the evaluator and generator. A decrease in any indicator will reduce the recommendation effect.
% For example, when the reranking model directly returns the result of the ranking stage, the HR will be up to 1 but the AUC will decrease. When the reranking model is very complex, the AUC will increase but the increased time-cost will lead to a decrease in the HR.

In online experiments, we adopt CTR and GMV (Gross Merchandise Volume) as evaluation metrics.


\subsubsection{Implementation Details}
We implement all the deep learning baselines and NLGR with TensorFlow 1.15.0 using NVIDIA A100-SXM4-80GB. For all comparison models and our NLGR model, we adopt Adam as the optimizer with the learning rate fixed to 0.001 and initialize the model parameters with normal distribution by setting the mean and standard deviation to 0 and 0.01, respectively. The batch size is 1024, the embedding size is 8, the $\alpha$ is 0.2. The hidden layer sizes of tiled MLP are (1024, 256, 128). 
For the Taobao Ad dataset, the length of the ranking list and reranking list are both 5, thus the length of full permutation is 120. For Metuan dataset, we select 4 items from the initial ranking list which contains 12 items, thus the length of full permutation is $A_{12}^4=11,880$. 
All experiments are repeated five times and the averaged results are reported.

\subsection{Experimental Results}\label{exp_result}
\subsubsection{Performance Comparison}

Table \ref{tab:tab2} and Table \ref{tab:tab3} summarize the results of offline experiments. We have the following observations from the experimental results: 
\begin{enumerate}[leftmargin=*]
\item [i)] PRM in Group II outperforms all models in Group I, which verifies the impact of the mutual influence among contextual items. DCDR in Group III outperforms all the other models in Groups I and II, which verifies the effectiveness of generative methods.
\item [ii)] DCDR indeed exhibits robust generative capabilities, thanks to the incorporation of the diffusion model, and achieves the second-best result. Nevertheless, DCDR overlooks the significance of full sight and falls short of leveraging the evaluator's potential to its fullest, which limits its effect. 
Our proposed NLGR significantly and consistently outperforms the state-of-the-art approaches in all metrics on both datasets. As presented in Table \ref{tab:tab2}, our proposed NLGR brings 0.8349/0.6344 absolute AUC, 0.2857/0.2323 absolute NDCG@10, 0.2431/0.1830 absolute NDCG@5 on Metuan/Ad dataset, gains significant improvement in industrial recommendation system. NLGR has greater improvements on Meituan dataset because Meituan dataset has more realistic reranking scenarios and richer features.
% This demonstrates our evaluator with D-Attention has a strong ability to fit users.
\item [iii)] Our proposed NLGR brings 0.8369/0.4091 absolute HR@10\% on Metuan/Ad dataset. This demonstrates our generator achieves extreme improvements via counterfactual evaluation. Compared with GRN and PRM, there is a huge improvement. As a typical greedy reranking algorithm, PRM only has 0.5702/0.1125 absolute HR@10\% on Metuan/Ad dataset. This conclusively shows that effective rearrangement cannot be attained by relying solely on a greedy strategy. As a generative reranking model, DCDR lacks full sight, which is why its HR@10\% is not optimal.
\end{enumerate}

% \subsubsection{Reward analysis}

% To demonstrate the effectiveness and convergence of the proposed NLGR algorithm, we plot the curve of reward changing with training time in Figure \ref{fig: reward}. As shown in Figure \ref{fig: reward}, with the change of training time, the reward first decreases slightly, then rises rapidly and remains stable. Before 10,000 steps, the reward is still monotonically increasing. Around 20,000 steps, the reward stabilizes and maintains a high value greater than 0. The experiment also proves that our reward design is very reasonable.

% \begin{figure}[h]
% \centering
% \includegraphics[width=0.3\textwidth]{reward.png}
% \caption{Reward performance of NLGR during offline training on the Meituan advertising dataset.}
% \label{fig: reward}
% \end{figure}

\subsubsection{Ablation Study}
To verify the impact of different units, we study three ablated variants of NLGR on Meituan dataset.
\begin{itemize}[leftmargin=*]
% \item NLGR without $\overline{r}$. This variant is NLGR removing $\overline{r}$ defined in Eq. \ref{equation:r} and replacing it with pCTR, which implies that evaluations are detached from actual business situations and do not account for variables like impression and conversion rates. The evaluation solely focuses on pCTR while disregarding real-world aspects.
% 为了验证邻居列表训练方法在解决目标不一致问题方面的有效性,该变体删除了等式1中定义的相对奖励,并将其替换为评估器直接返回的预测值。
\item NLGR without relative reward $r$. To verify the effectiveness of the neighbor list training method in addressing the goal inconsistency problem, this variant removes the relative reward defined in Eq. \ref{equation:R} and replaces it with the predicted value returned by the evaluator directly.
% 为了验证NLGR-G中sampling-based non-autoregressive generation method,我们将其替换为pointer network,这是一种在重排模型中被广泛使用的序列生成方法。
\item NLGR with autoregressive generation (AG). To verify the sampling-based non-autoregressive generation method in NLGR-G, we replaced it with pointer network \cite{vinyals2015pointer}, which is a sequence generation method widely used in reranking models \citeN{grn,seq2slate}.
% 为了进一步验证邻居列表训练方法对NLGR-G的充分指导,我们去掉公式18中的$\mathcal{L}^G_2$,这意味着PDU将缺少NLGR-E的直接指导。
\item NLGR with $\mathcal{L}^G_2$. To further verify the effectiveness of the neighbor list training method, we remove $\mathcal{L}^G_2$ in Eq. \ref{equation:loss_G}, which means that PDU will lack direct guidance from NLGR-E.
\end{itemize}

% 从实验结果来看,我们有以下发现:i)NLGR w/o $r$ 的HR下降最多(8\%/17.2\%),表明neighbor list training是NLGR最重要的部分。ii)NLGR w/ AG 的HR下降显著(2.2\%/4.8\%),表明the sampling-based non-autoregressive generation method in NLGR-G可以显著提升生成效果。iii)NLGR w/o $\mathcal{L}^G_2$的HR也有下降,表明辅助损失可以增强NLGR-G的生成能力。
The result is shown in Table \ref{tab:tab3}. From the experimental results, we have the following findings: i) The HR of NLGR w/o $r$ drops the most (8\%/17.2\%), indicating that neighbor list training is the most important part of NLGR. ii) The HR of NLGR w/ AG dropped significantly (2.2\%/4.8\%), indicating that the sampling-based non-autoregressive generation method in NLGR-G can significantly improve the generation effect. iii) The HR of NLGR w/o $\mathcal{L}^G_2$ also decreases (1.1\%/3.2\%), indicating that auxiliary loss can enhance the generation ability of NLGR-G.
\begin{table}[h]
\caption{HR of different methods on Meituan}
\label{tab:tab3}
\begin{tabular}{ccc}
\hline
\textbf{method}             & \textbf{HR@10\%}  & \textbf{HR@1\%}     \\ \hline
NLGR                        &  0.8369           &    0.7523           \\
NLGR w/o $r$                &  0.7562           &    0.5809           \\  
NLGR w/ AG                  &  0.8142           &    0.7047          \\  
NLGR w/o $\mathcal{L}^G_2$  &  0.8255           &    0.7198           \\  \hline
\end{tabular}
\end{table}

\subsection{Hyperparameter Analysis}
% 我们分析两个超参数的敏感性,分别对应NLGR的生成过程和训练过程。其中,$\alpha$ 是式中 NLGR-G 损失的权重。$\beta$ 是在构造邻居列表L*时每个位置的采样比例. 默认情况下$\beta=1$,意味着每个位置采样一次。
We analyze the sensitivity of two hyperparameters: $\alpha$ and $\beta$, corresponding to the generation process and training process of NLGR. Among them, $\alpha$ is the weight of NLGR-G loss in Eq. \ref{equation:loss_G}, and $\beta$ is the sampling ratio at each position when constructing the neighbor list $L^*$. By default $\beta=1$ means that each position is sampled $1$ time. The result is shown in Table \ref{tab:tab4}, showing the same trend on the public dataset and industrial dataset and we have the following findings:
\begin{enumerate}[leftmargin=*]
\item [i)] Hyperparameter $\alpha$ significantly affects the generator’s HR@10\% metric. When $\alpha=0$, NLGR is equivalent to the method of Group III, which means the generator has no full sight. As $\alpha$ increases, HR@10\% first increases and then decreases.

\item [ii)] We tested several values for $\beta$. When $\beta<1$, we randomly select $\beta m$ positions in $m$ positions to construct rewards. When $\beta>1$, we construct $\beta$ neighbor lists at each position. Increasing $\beta$ within a certain range can quickly improve the HR@10\% performance. As $\beta$ continues to increase, HR@10\% remains stable but increases offline training time. The results show that counterfactual rewards considering all positions are important.
\end{enumerate}

\begin{table}[h]
\caption{ HR@10\% result of ablation experiment on NLGR}
\label{tab:tab4}
\begin{tabular}{llccccc}
\hline
& $\alpha$=0 & $\alpha$=0.01 & $\alpha$=0.2 & $\alpha$=0.5 & $\alpha$=1.0  \\
\hline
Meituan     &  0.7562   &  0.8274   &  \textbf{0.8369}   &  0.8301    &   0.8295   \\
Taobao Ad   &  0.2192   &  0.3530   &  \textbf{0.4091}   &  0.3973    &   0.3912   \\
\hline
   & $\beta$=0.1 & $\beta$=0.5 & $\beta$=1 & $\beta$=2 & $\beta$=5 \\
\hline
Meituan     &  0.7763   &  0.8031   &  \textbf{0.8369}   &  0.8369    &   0.8367   \\
Taobao Ad   &  0.2814   &  0.3566   &  \textbf{0.4091}   &  0.4091    &   0.4091   \\
\hline
\end{tabular}
\end{table}

\subsection{Online A/B test}
% 值得注意的是,虽然我们在离线训练时涉及到评估器对生成器的多次指导,但在线服务时只需要使用生成器。这样我们就保证了它的模型复杂度与在线模型相当,而不需要在在线服务中增加额外的计算
We deployed NLGR in Meituan App, where Figure \ref{fig: online} shows the online serving system architecture. It is worth noting that although we involve the evaluator guiding the generator multiple times during offline training, we only need to use the generator when serving online. In this way, we ensure that its model complexity is comparable to the online model without adding additional calculations to the online service.

\begin{figure}[h]
\centering
\includegraphics[width=0.45\textwidth]{Figure/online.png}
\caption{Architecture of the online deployment with NLGR.}
\label{fig: online}
\end{figure}

We conducted online experiments in an A/B test framework over five weeks from Dec.2023 to Jan.2024. Table \ref{tab:online} shows the online performance of NLGR. Compared to the baseline model (a variant of  PRM), NLGR has increased the CTR by 3.25\% and the GMV by 3.07\%. 
Moreover, since only NLGR-G is deployed online during online inference, the time-out rate online has no increase, which is acceptable to the recommendation system. Note that NLGR-E is not deployed online which is only utilized for offline guidance of NLGR-G.
Now, NLGR has been deployed in the Meituan food delivery platform and serves hundreds of millions of users.

\begin{table}[h]
\caption{Online A/B test result}
\label{tab:online}
\begin{tabular}{ccccc}
\hline
\textbf{method}        & \textbf{CTR} & \textbf{GMV}  & \textbf{Cost (ms)} & \textbf{Time-out} \\ \hline
Baseline (PRM)      &  0.0\%          &    0.0\%       & \textbf{0.0}  & 0.0\%  \\ 
% NLGR w/o soft-update &  3.06\%         &    2.88\%      & 1.6  & 0.0\%  \\  
\textbf{NLGR}        &  \textbf{3.25\%}& \textbf{3.07\%}& 1.6  & \textbf{0.0\%}  \\
\hline
\end{tabular}
\end{table}

\section{Conclusion}
% 在这篇文章中,我们首次尝试去解决重排系统中的目标不一致问题。然后我们提出了一个利用临近队列的生成式重排模型,NLGR。它利用候选队列与临近队列的相对得分来指导生成器。进一步,我们提出了一个基于采样的非自回归生成器,能够从当前列表灵活跳转到任意邻居列表。

In this paper, we make the first attempt to solve the goal inconsistency problem in reranking systems. We propose a novel framework called Neighbor List Generative Reranking (NLGR), which uses the relative scores of candidate list and neighboring lists to guide the generator. Furthermore, we propose a sampling-based non-autoregressive generator that can flexibly jump from the current list to any neighbor list. Both offline experiments and online A/B tests show that NLGR significantly outperformed other existing reranking baselines, and we have deployed NLGR on the Meituan food delivery platform.


\vspace{2em}

% \section{ACKNOWLEDGEMENTS}

% \section{REFERENCES}
% \balance
\bibliographystyle{ACM-Reference-Format}
% \balance
% \IEEEtriggeratref{10}
\bibliography{main}



\end{document}
\endinput
%%
%% End of file `sample-authordraft.tex'.
