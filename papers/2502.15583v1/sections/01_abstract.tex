\begin{abstract}
With rapid advancements in artificial intelligence, question-answering (Q\&A) systems have become essential in intelligent search engines, virtual assistants, and customer service platforms. However, in dynamic domains like smart grids, conventional retrieval-augmented generation(RAG) Q\&A systems face challenges such as inadequate retrieval quality, irrelevant responses, and inefficiencies in handling large-scale, real-time data streams.
This paper proposes an optimized iterative retrieval-based Q\&A framework called Chats-Grid tailored for smart grid environments. In the pre-retrieval phase, Chats-Grid advanced query expansion ensures comprehensive coverage of diverse data sources, including sensor readings, meter records, and control system parameters. During retrieval, Best Matching 25(BM25) sparse retrieval and BAAI General Embedding(BGE) dense retrieval in Chats-Grid are combined to process vast, heterogeneous datasets effectively. Post-retrieval, a fine-tuned large language model uses prompt engineering to assess relevance, filter irrelevant results, and reorder documents based on contextual accuracy. The model further generates precise, context-aware answers, adhering to quality criteria and employing a self-checking mechanism for enhanced reliability.
Experimental results demonstrate Chats-Grid’s superiority over state-of-the-art methods in fidelity, contextual recall, relevance, and accuracy by 2.37\%, 2.19\%, and 3.58\% respectively. This framework advances smart grid management by improving decision-making and user interactions, fostering resilient and adaptive smart grid infrastructures.
%Specifically, Chats-Grid shows improvements of 2.37\%, 2.19\%, and 3.58\% in fidelity, context recall rate, and answer accuracy over Self-RAG, respectively, and 0.94\%, 4.39\%, and 2.45\% over ITRG.
\end{abstract}

