%%%%%%%% ICML 2025 EXAMPLE LATEX SUBMISSION FILE %%%%%%%%%%%%%%%%%

\documentclass{article}

% Recommended, but optional, packages for figures and better typesetting:
\usepackage{microtype}
\usepackage{graphicx}
\usepackage{subfigure}
\usepackage{booktabs} % for professional tables

% hyperref makes hyperlinks in the resulting PDF.
% If your build breaks (sometimes temporarily if a hyperlink spans a page)
% please comment out the following usepackage line and replace
% \usepackage{icml2025} with \usepackage[nohyperref]{icml2025} above.
\usepackage{hyperref}


% Attempt to make hyperref and algorithmic work together better:
\newcommand{\theHalgorithm}{\arabic{algorithm}}

% Use the following line for the initial blind version submitted for review:
% \usepackage{icml2025}

% If accepted, instead use the following line for the camera-ready submission:
\usepackage[accepted]{icml2025}

% Lists
\usepackage[inline]{enumitem}
\newlist{enuminline}{enumerate*}{1}
\setlist[enuminline]{label=(\roman*)}

% Math
\newcommand{\vol}{\mathrm{vol}}
\newcommand{\E}{\mathbb E}
\newcommand{\var}{\mathrm{var}}
\newcommand{\cov}{\mathrm{cov}}
\newcommand{\Normal}{\mathcal N}
\newcommand{\slowvar}{\mathcal L}
\newcommand{\bbR}{\mathbb R}
\newcommand{\bbE}{\mathbb E}
\newcommand{\bbP}{\mathbb P}
\newcommand{\calC}{\mathcal C}
\newcommand{\calR}{\mathcal R}
\newcommand{\calT}{\mathcal T}
\newcommand{\loA}{\underline{A}}
\newcommand{\upA}{\overline{A}} 
\newcommand{\cv}{\mathrm{cv}} 
\newcommand{\pp}{\mathrm{pp}}
\newcommand{\HS}{\mathrm{HS}}
\newcommand{\erfi}{\mathrm{erfi}}
\newcommand{\tX}{\widetilde X}
\newcommand{\OneFOne}{{}_1F_1}
\newcommand{\PP}{\mathrm{\texttt{PP}}}
\newcommand{\PPpp}{\mathrm{\texttt{PP+}}}
\newcommand{\BPP}{\mathrm{\texttt{BPP}}}
\newcommand{\BPPpp}{\mathrm{\texttt{BPP+}}}
\newcommand{\FABPPI}{\mathrm{\texttt{FABPP}}}
\newcommand{\asto}{\overset{\text{a.s.}}{\to}}

% For theorems and such
\usepackage{amsmath}
\usepackage{amssymb}
\usepackage{mathtools}
% \usepackage{amsthm}

% if you use cleveref..
\usepackage[capitalize,noabbrev]{cleveref}

%%%%%%%%%%%%%%%%%%%%%%%%%%%%%%%%
% THEOREMS
%%%%%%%%%%%%%%%%%%%%%%%%%%%%%%%%
% \theoremstyle{plain}
% \newtheorem{theorem}{Theorem}[section]
% \newtheorem{proposition}[theorem]{Proposition}
% \newtheorem{lemma}[theorem]{Lemma}
% \newtheorem{corollary}[theorem]{Corollary}
% \theoremstyle{definition}
% \newtheorem{definition}[theorem]{Definition}
% \newtheorem{assumption}[theorem]{Assumption}
% \theoremstyle{remark}
% \newtheorem{remark}[theorem]{Remark}

% Todonotes is useful during development; simply uncomment the next line
%    and comment out the line below the next line to turn off comments
%\usepackage[disable,textsize=tiny]{todonotes}
\usepackage[textsize=tiny]{todonotes}


% The \icmltitle you define below is probably too long as a header.
% Therefore, a short form for the running title is supplied here:
\icmltitlerunning{\ouradapter: Cross-Lingual Adaptation of LLMs by Embedding Surgery}

\begin{document}

\twocolumn[
\icmltitle{\ouradapter: Cross-Lingual Adaptation of LLMs by Embedding Surgery}

% It is OKAY to include author information, even for blind
% submissions: the style file will automatically remove it for you
% unless you've provided the [accepted] option to the icml2025
% package.

% List of affiliations: The first argument should be a (short)
% identifier you will use later to specify author affiliations
% Academic affiliations should list Department, University, City, Region, Country
% Industry affiliations should list Company, City, Region, Country

% You can specify symbols, otherwise they are numbered in order.
% Ideally, you should not use this facility. Affiliations will be numbered
% in order of appearance and this is the preferred way.
\icmlsetsymbol{equal}{*}

\begin{icmlauthorlist}
\icmlauthor{Fan Jiang}{equal,unimelb}
\icmlauthor{Honglin Yu}{gra}
\icmlauthor{Grace Chung}{gra}
\icmlauthor{Trevor Cohn}{gra}
\end{icmlauthorlist}

\icmlaffiliation{unimelb}{The University of Melbourne}
\icmlaffiliation{gra}{Google}

\icmlcorrespondingauthor{Fan Jiang}{fan.jiang1@student.unimelb.edu.au}

% You may provide any keywords that you
% find helpful for describing your paper; these are used to populate
% the "keywords" metadata in the PDF but will not be shown in the document
\icmlkeywords{Machine Learning, ICML}

\vskip 0.3in
]

% this must go after the closing bracket ] following \twocolumn[ ...

% This command actually creates the footnote in the first column
% listing the affiliations and the copyright notice.
% The command takes one argument, which is text to display at the start of the footnote.
% The \icmlEqualContribution command is standard text for equal contribution.
% Remove it (just {}) if you do not need this facility.

%\printAffiliationsAndNotice{}  % leave blank if no need to mention equal contribution
\printAffiliationsAndNotice{\icmlEqualContribution} % otherwise use the standard text.

\begin{abstract}
The capabilities of Large Language Models (LLMs) in low-resource languages lag far behind those in English, making their universal accessibility a significant challenge.
% While prior work has shown feasibility of efficient language adaptation by learning dedicated embeddings, such methods have largely been limited to small-scale, encoder-only models designed for language understanding tasks.
To alleviate this, we present \emph{\ouradapter}, a modular language adaptation approach for decoder-only LLMs with embedding surgery. Our method begins by creating customized vocabularies for target languages and performing language adaptation through embedding tuning on multilingual data. These pre-trained embeddings are subsequently integrated with LLMs that have been instruction-tuned on English alignment data to enable zero-shot cross-lingual transfer. 
% Our extensive experiments with \gemmatwo shows significant improvements across 96 languages, spanning both discriminative and generative tasks, with minimal regressions in English proficiency. 
Our experiments on \gemmatwo models with up to 27B parameters demonstrate improvements of up to 20\% across 96 languages, spanning both discriminative and generative tasks, with minimal regressions ($<$1\%) in English. Further in-depth analysis reveals the critical role of customizing tokenizers in enhancing language adaptation, while boosting inference efficiency.
% (up to 28\% faster in the prefilling stage for underrepresented languages). 
Additionally, we show the versatility of our method by achieving a 14\% improvement over a math-optimized LLM across 20 languages, offering a modular solution to transfer reasoning abilities across languages post hoc.
% \wishlist{
% 1. emphasize model size up to 27B.
% 2. improve inference latency up to xxx.
% 3. critical role of customized tokenizer.
% 4. math cross-lingual transfer. general approach to transfer reasoning capabilities across languages in post hoc manners.
% }
\end{abstract}

\section{Introduction}



The advent of Large Language Models (LLMs) has revolutionised Natural Language Processing (NLP) \citep{zhao2023survey}. While these models, trained on massive datasets, have shown remarkable capabilities, initial versions exhibited concerning issues like toxicity, hallucinations, and biases \citep{liang2023holistic,perez-etal-2022-red,ganguli2022red,weidinger2021ethical,liu2023trustworthy}. Consequently, aligning LLMs with human values has become a central research focus \citep{ouyang2022training,bai2022training,christiano2017deep,gabriel2020artificial}. The impact of alignment is evident in the success of ChatGPT \citep{gpt4}, highlighting its importance for safety, reliability, and broader applicability \citep{shen2023large,liu2023trustworthy}.

\begin{figure}[!t]
    \centering
    \includegraphics[width=0.95\linewidth]{asset/imgs/v2-vital-overview.pdf}
    \vspace{-0.50cm}
    \caption{A pluralistic alignment example from \ourdataset dataset. More multi-opinionated health scenarios can be found in \refapptab{table:health-scenarios}.}
    \label{fig:vital-alignment-overview}
    \vspace{-0.4cm}
\end{figure}

Despite progress in alignment \citep{wang2023aligning,ouyang2022training,stiennon2020learning,christiano2017deep,rafailov2024direct,schulman2017proximal}, current methods often model \textit{average} human values, neglecting the diversity of preferences across different groups \citep{positionpluralistic,sorensen2024value,feng2024modular}. As AI systems become increasingly prevalent, they must reflect this plurality \citep{positionpluralistic}. Recent work has begun to address pluralistic alignment \citep{bai2022constitutional,gordon2022jury,sorensen2024value} (as illustrated in \reffig{fig:vital-alignment-overview}), recognising the risks of overlooking diverse opinions, particularly in sensitive domains like health where misinformation can have severe consequences \citep{chen2024combating,menz2024current,suarez2021prevalence}.

LLMs are increasingly deployed in open-ended health applications like chatbots \citep{yang2023large,thirunavukarasu2023large}, where their responses to subjective questions are critical. In this domain, LLM outputs can significantly influence user beliefs \citep{santurkar2023whose}, potentially leading to undesirable outcomes such as the promotion of specific viewpoints or homogenization of beliefs \citep{weidinger2021ethical,weidinger2022taxonomy,gabriel2020artificial}. Therefore, evaluating the \textit{representativeness} of health-related LLM responses is crucial before deployment.











Although alignment datasets are available \citep{santurkar2023whose,sorensen2024value}, none focus primarily on health to the best of our knowledge. We argue that existing datasets lack the specificity needed to address diverse cultural and ethical norms within healthcare, which is paramount when incorporating AI into this field. A health-specific dataset will better capture these nuances and improve AI (pluralistic) alignment with varied health beliefs, addressing limitations in current pluralistic approaches. Hence, we build a comprehensive dataset for \textbf{V}al\textbf{I}dating pluralis\textbf{T}ic \textbf{A}lignment for hea\textbf{L}th, \ourdataset, consisting of 13.1K value-laden situations and 5.4K multiple-choice questions across surveys, polls, and moral scenarios focusing on the health domain. We focus on health scenarios \citep{porter2010value}, which present many conflicting opinions from different cultures \citep{thomas2004health,kreuter2004role}, religions \citep{elmahjub2023artificial}, values \citep{klessig1992effect,de2000sensitivity}, and others.

In this paper, we study how alignment techniques, particularly recent pluralistic alignment methods \citep{feng2024modular} in LLMs, for health-specific scenarios. We benchmark these against vanilla LLMs, existing alignment procedures—prompting, Mixture of Experts (\moe), and Modular Pluralism (\modplural). Our investigation includes eight LLMs (a combination of open-source and black-box models) across three modes of pluralistic alignment. We also experiment with some solutions to improve alignment and discuss the future scope of research. 

The contributions of this work are as follows:
\begin{itemize}
    \item To the best of our knowledge, this work is the first to explore the pluralistic alignment of LLMs, specifically within the health domain.
    \item We construct and introduce a comprehensive benchmark dataset, \ourdataset, concentrating on the health domain for various pluralistic alignment methodologies.
    \item Using this dataset, we benchmark and evaluate the current state-of-the-art (SOTA) pluralistic alignment techniques through detailed analyses and ablation studies. Our findings demonstrate that current leading models exhibit limited performance on \ourdataset. 
    
\end{itemize}




\section{Methodology}
% \wishlist{Add a figure of result summary to demonstrate the benefits of embedding surgery as well as franken-adapter. (20/10)}
% \wishlist{Add a figure of the training method (embedding tuning, franken-adapter, lora-adaptation).}
% Restate the motivation of this work. Modern LLMs is mainly trained using English data with tiny fraction (~5\%) of multilingual data, so it's multilingual ability is limited. Common approach is to do continual pre-training on data mixture with upsampled multilingual ratios. However, they neglects the fact that the tokenizers used in LLMs are usually trained on data subsampled from the English-dominant pre-training data mixture. This leads to the low frequencies of multilingual tokens and thus, making them less likely to appear as independent tokens in the built vocabulary. These multilingual-unfriendly tokenizers break multilingual texts into too many segments, and thus increasing the difficulty of modeling these texts compared to English. Therefore, a natural question raised is that in addition to increasing the multilingual ratios of model pre-training data, making multilingual-friendly tokenizers is also important for enhancing the multilingual ability of LLMs.

% Modern LLMs are mainly trained using English data with a tiny fraction ($<10\%$) of multilingual data~\citep{dubey2024llama3herdmodels}, so their multilingual ability is limited. A common approach for multilingual enhancement is to do continual pre-training on a data mixture with an increased proportion of multilingual data. However, they neglect the fact that the tokenizers used in LLMs are usually trained on data sub-sampled from the English-dominant pre-training data mixture. This leads to low frequencies of multilingual tokens, making them less likely to appear as complete words in the built vocabulary. As a result, these multilingual-unfriendly tokenizers break multilingual texts into too many segments (Figure~\ref{fig:fertility_comparison}), increasing the difficulty of modeling these texts compared to English~\citep{ahuja-etal-2023-mega}. In this paper, our goal is to demonstrate that in addition to increasing the multilingual ratios of model pre-training data, making multilingual-friendly tokenizers is also important for enhancing the multilingual ability of LLMs. Meanwhile, we also show that switching to these tokenizers with solely embedding tuning on multilingual data is sufficient for efficient cross-lingual transfer of LLMs. 

% \wishlist{The motivations of train multilingual embeddings for each language group (2-3 sentences for explanations (two cases: 1 per language and 1 for all languages, demonstrate their disadvantages).}
Modern LLMs exhibits limited multilingual ability, primarily due to the dominance of English training data and the relatively small multilingual proportion~\citep{dubey2024llama3herdmodels}. Additionally, the tokenizers employed in these LLMs, which are constructed from sub-sampled pre-training corpora, are biased towards English and several high-resource languages. This results in fragmenting texts from long-tail languages into too many tokens (Figure~\ref{fig:fertility_comparison}), thereby degrading the performance and efficiency of processing such languages~\citep{ahuja-etal-2023-mega}. In this paper, we demonstrate that having tokenizers that provide equitable representation for low-resource languages is critical to the effectiveness of our proposed \emph{\ouradapter} approach. 

\begin{figure}
    \setlength{\abovecaptionskip}{-0.0001cm}
    \setlength{\belowcaptionskip}{-0.35cm}
    \centering
    \includegraphics[width=0.95\linewidth]{figures/fertility_comparison.pdf}
    \vspace{-6mm}
    \caption{The tokenization comparison between using the vanilla and customized multilingual tokenizers on \gemmatwo. Tok. Length refers to the average number of tokens required to represent the same amount of texts.} 
    % \wishlist{Add the tokenizer comparison of Palm and Aya in appendix.}
    %  \wishlist{Add tokenization qualitative examples before and after changing tokenizers (English and a few LRLs.}
    \vspace{-6mm}
    \label{fig:fertility_comparison}
\end{figure}

\subsection{Customized Vocabulary Construction}\label{sec:vocab_construct}
Our strategy involves constructing distinct tokenizers for each language group (\S\ref{sec:languages}).\footnote{Our preliminary findings suggest that employing distinct tokenizers for each language group enhances cross-lingual transfer. Please refer to Appendix Figure~\ref{fig:joint_vs_distinct_tokenizer} for a comparison with the variant of using a single tokenizer for all language groups.} Tailoring tokenizers to specific language groups enhances cross-lingual transfer among geographically related long-tail languages compared to using monolingual tokenizers. Moreover, this approach avoids the shortcomings of a universal tokenizer that treats all low-resource languages uniformly poorly. Based on this, we propose a \emph{Prune-with-Extension} approach for developing tokenizers optimized for language adaptation while maintaining English ability. 
% First, we train separate tokenizers for each language group using the BPE algorithm~\citep{bpe,sennrich-etal-2016-neural}. Next, we employ the tokenizers of target LLMs to identify English tokens. Finally, we create the final vocabulary by merging English and long-tail tokens.
First, we prune the tokenizer of target LLMs by removing non-English tokens. Then the pruned tokenizer is extended through adding new tokens, which are obtained by training tokenizers for low-resource languages using BPE~\citep{bpe,sennrich-etal-2016-neural}.

% \inlinetitle{Reusing English Tokens}
\inlinetitle{Pruning the Tokenizer}
To preserve the pre-trained knowledge embedded in the language model, current approaches often expand the vocabulary by adding new tokens~\citep{fujii2024continual, cui2024efficienteffectivetextencoding}. This, however, can substantially increase pre-training time due to the extra computational cost of the softmax layer~\citep{liang-etal-2023-xlm}.\footnote{Embedding and $\operatorname{softmax}$ layers are typically tied in modern LLMs. LLMs evaluated in this work all follow this setting.} 
% Our proposed method aims to maintain pre-training efficiency while preserving existing knowledge by avoiding vocabulary expansion. 
To avoid this, we first prune the existing tokenizers by retaining only English tokens before adding those from low-resource languages.
Given the predominant English training data for LLMs, we hypothesize that a significant portion of their knowledge is associated with English tokens, and reusing English tokens can effectively retain this knowledge~\citep{garcia-etal-2021-towards}. 
% In our implementation, for a given LLM, we tokenize a set of 20 million English sentences using its tokenizer. We then filter out any non-Latin script tokens, assuming the remaining to be English.
In our implementation, for a given LLM, we identify English tokens by tokenizing a set of 20 million English sentences using its tokenizer, with further filtering through removing non-Latin script tokens.\footnote{40\% of the tokens are discarded: these are non-Latin scripts tokens from high-resource languages, domain-specific tokens (\eg code), and very rare English tokens.}

\inlinetitle{Training Multilingual Tokenizers}
% To construct a multilingual vocabulary for long-tail languages, we use the Next Thousand Languages (NTL) project corpus~\citep{caswell-etal-2020-language, bapna2022building}. 
To get the data for building a multilingual vocabulary for long-tail languages, we sample from the Next Thousand Languages (NTL) corpus~\citep{caswell-etal-2020-language, bapna2022building}.
Our empirical analysis reveals that sampling data for each language up to a maximum of 500K lines from NTL effectively addresses the imbalance between high- and low-resource languages, outperforming temperature-sampling techniques. 
Subsequently, we train a BPE tokenizer using the sampled data to generate a vocabulary whose size aligns with that of the target LLMs.


\inlinetitle{Extending the Pruned Tokenizer}
% For Unigram tokenizers, we merge the English and multilingual tokens into a single multilingual vocabulary~\citep{chung-etal-2020-improving}. For BPE tokenizers, 
We sequentially add the identified English tokens followed by tokens from the newly built multilingual tokenizer. Both types of tokens are added in the same preference order as in their respective tokenizers. The final vocabulary maintains the same size as the original tokenizer, with over 60\% token overlap, resulting in negligible variations in English tokenizations (see  Table~\ref{tab:tokenization_qualitative_examples} in Appendix).
Figure~\ref{fig:fertility_comparison} shows our final vocabulary achieves significant compression rate improvements by consistently producing shorter sequences across languages of a spectrum of resource levels while barely affecting English.

\subsection{Model Training}\label{sec:model_training}

\inlinetitle{Embedding Initialization}
% Our goal is to maximally inherit the pre-training knowledge from the target LLM's embedding layer. We follow \citet{gee-etal-2022-fast} to first copy the token embeddings from the target LLM's embedding layer for overlapped tokens. For new tokens, we use the LLM's original tokenizer to tokenize each of them and use average pooling of embeddings of the tokenized tokens as the initialization.
To maximally inherit the pre-trained knowledge embedded in the target LLM's embedding layer, we adopt a strategy inspired by \citet{gee-etal-2022-fast}. For tokens that overlap between the target LLM's vocabulary and our multilingual vocabulary, we directly copy the corresponding embeddings. For new tokens, we employ the LLM's original tokenizer to decompose them into subtokens and initialize their embeddings using the average of their subtoken embeddings.

\inlinetitle{Language Adaptation}
To adapt the LLMs to new languages, we follow~\citet{artetxe-etal-2020-cross} to only fine-tune the customized embeddings on curated multilingual data $\mathcal{D}_{la}$ while keeping the transformer body frozen (Figure~\ref{fig:franken_adapter} step 2(b)), with the same training objective used in the initial LLM pre-training phase. This is based on the assumption that the pre-trained transformer body encapsulates universal cross-lingual knowledge~\citep{zhao2024how,wendler-etal-2024-llamas,tang-etal-2024-language}, while the embedding layer encodes language-specific information, which suggests embedding tuning should be effective for language adaptation. 
% This makes embedding surgery an effective strategy for significantly enhancing cross-lingual transfer.

\inlinetitle{\ouradapter}
To facilitate zero-shot cross-lingual transfer, \citet{artetxe-etal-2020-cross} fine-tunes the transformer body of PLMs on a specific English downstream task, then incorporates monolingual embeddings tailored for a second language. This approach necessitates task-specific fine-tuning of the transformer and language-specific embedding creation. In contrast, our \ouradapter method instruction-tunes the transformer body of LLMs on a diverse range of English tasks $\mathcal{D}_{it}$~\citep{wei2022finetuned} (Figure~\ref{fig:franken_adapter} step 2(a)). Notably, we employ the LLM's \emph{original} embeddings and keep them frozen in this step.
% \edit{By freezing the embeddings, we ensure that the instruction-following capabilities are concentrated in the transformer body.}
We then integrate the customized embeddings obtained from the language adaptation stage into the instruction-tuned transformer body. This results in the \ouradapter (Figure~\ref{fig:franken_adapter} step 3), a modular framework designed for zero-shot cross-lingual transfer.\footnote{Freezing embeddings makes the instruction tuning and language adaptation processes symmetry. This enhances modularity and improves the parameter compatibility of \ouradapter.}
% \footnote{\edit{This modularity allows the \ouradapter to decouple the transformer body, which specializes in instruction-following, from the embeddings that are language-specific.}}


\inlinetitle{LoRA-Adaptation}
% A potential limitation of the \ouradapter approach is the risk of vulnerability due to the independent training of the instruction-tuned transformer body and pre-trained multilingual embeddings. Our empirical findings indicate that the \ouradapter is well-suited for discriminative tasks but exhibits less controllable behavior in generative tasks. To mitigate this and ensure the assembled model's effectiveness across various tasks, we insert LoRA weights into the self-attention layer of the tuned transformer body. These weights are then fine-tuned on a joint corpus $\mathcal{D}_{mix}=\mathcal{D}_{la}\cup\mathcal{D}_{it}$, while keeping both the transformer body and embeddings frozen.
Since the transformer body and customized embeddings are independently trained, the \ouradapter approach may suffer from incompatible parameters. Our empirical findings indicate that \ouradapter is effective for discriminative tasks but sometimes underperforms an instruction-tuned baseline on generative tasks. To mitigate this and ensure the assembled model's effectiveness across various tasks, we insert LoRA weights into the self-attention layer of the tuned transformer body (Figure~\ref{fig:franken_adapter} step 3). These weights are then fine-tuned on a sub-sampled joint corpus $\mathcal{D}_{mix}=\mathcal{D}_{la}\cup\mathcal{D}_{it}$, while keeping both the transformer body and embeddings frozen.

% \wishlist{Add a discussion section to illustrate the training efficiency of our method compared conventional continued training method (full parameter tuning on NTL and then on FLAN for each language group.}
% \inlinetitle{Training Efficiency}
% \edit{\ouradapter offers a modular framework for composing embeddings tailored to different languages and LLMs post-trained for diverse purposes. For $M$ distinct language groups, our approach requires only $M$ instances of embedding tuning. These embeddings are then integrated into various types of post-trained LLMs, enabling on-the-fly usage with minimal training costs. This is in contrast to conventional continued pre-training method with full-parameter tuning, which requires extensive training on target languages followed by instruction tuning on alignment data, resulting in $2M$ instances of full-parameter tuning. Moreover, this process must be repeated if the instruction tuning is tailored for different task-solving abilities. By contrast, our method requires only one instance of instruction tuning for each target ability and achieves cross-lingual transfer through embedding swapping, thereby eliminating the need for massive redundant training.
% % Consequently, this incurs significantly higher resource demands compared to our method which avoids redundant training through efficient model reuse.
% }
% \wishlist{Add a figure for comparison with full-parameter tuning in appendix.}
% \wishlist{
% Have a separate subsection 2.3: discussion on several advantages (summarize the important points)
% 1. tokenizer make the training and inference more efficient.
% 2. less parameters tuned.
% 3. keep the transformer body untouched, avoid knowledge forgetting.
% 4. list the modular advantage, avoid redundant training.}

\subsection{Discussion}
Compared to the typical continued pre-training approach with full-parameter tuning on target languages, \ouradapter offers several advantages: 1) Customized tokenizers ensure fairer representations for target languages, which not only facilitates cross-lingual transfer but also improves training and inference efficiency. 2) Embedding tuning further enhances language adaptation efficiency by keeping the transformer body intact. This mitigates catastrophic forgetting that is prevalent in traditional methods. 3) The modular nature of \ouradapter enables the reuse of existing models with pre-established capabilities through simple and cost-effective embedding swapping in a fully \emph{post-hoc} manner. This property distinguishes \ouradapter from typical methods that requires costly and separate adaptations to acquire different skills, making it more desirable for effective and efficient cross-lingual transfer in many settings. (See Appendix~\ref{appendix:cpt} for more details with empirical evidence.)

\section{Experiment setup}
% \subsection{Pre-training Data and Languages}
\subsection{Pre-training Data}
% \inlinetitle{Data Mixture}
The data $\mathcal{D}_{la}$ for embedding training is a mixed corpus with 65\% sentence-level and 35\% document-level texts. The sentence-level data is exclusively from the Next Thousand Languages (NTL) corpora~\citep{caswell-etal-2020-language, bapna2022building}, which provides web-crawled monolingual sentences and translation pairs for over 1000 languages. For document-level texts, we sample data from multilingual Wikipedia and mC4~\citep{xue-etal-2021-mt5}. We use UniMax sampling~\citep{chung2023unimax} with $N=5$ to up-sample low-resource languages. Additionally, we consider English as a high-resource language and always include it in the training mixture to prevent catastrophic forgetting.

We take FLAN~\citep{wei2022finetuned} as the instruction tuning data $\mathcal{D}_{it}$. 
% For LoRA-Adaptation, we create $\mathcal{D}_{mix}$ by combing $\mathcal{D}_{la}$ and $\mathcal{D}_{it}$ equally\footnote{We empirically find this results in good downstream performance (Figure~\ref{fig:data_ablation_on_lora_adaptation}), although thoroughly testing how different sampling ratios from each kind of data affect final performance would be beneficial, which we leave as future work.} and tune the model on 10\% of $\mathcal{D}_{mix}$.
The data $\mathcal{D}_{mix}$ used in LoRA-Adaptation consists of a 10\% sample of $\mathcal{D}_{it}$, combined with an equal number of instances from $\mathcal{D}_{la}$.\footnote{This ensures the instruction-following ability isn't forgotten. Please refer to Appendix Figure~\ref{fig:data_ablation_on_lora_adaptation} for detailed analysis.}

% \inlinetitle{Languages}
\subsection{Languages}\label{sec:languages}
We select languages from three families based on geographic relations: South East Asian (\sea), African (\afr), and Indic (\ind). 
% Following the NTL classification, we include 212 languages from \sea, 392 from \afr, and 170 from the \ind family.
This results in 212 languages from \sea, 392 from \afr, and 170 from \ind. Each regional dataset is processed separately, with a tailored tokenizer, language-adapted embeddings and LoRA update parameters.

% \subsection{Models and Training Details}

\subsection{Models}
% We evaluate our method on two open-source LLMs: \gemmatwo (2B, 9B, 27B)~\citep{team2024gemma} and ; and one 
Our evaluation is focused on \gemmatwo (2B, 9B, 27B)~\citep{team2024gemma}. We also test the generalization ability of our method in two LLMs with varying degrees of multilinguality: \aya (8B, 35B)~\citep{aryabumi2024aya} and \palmtwo (XXS, S)~\citep{anil2023palm}.

As shown in Figure~\ref{fig:franken_adapter}, we end up having four types of models: (i) \texttt{$*$-FLAN} (step-2a): models that undergo instruction-tuning. (ii) \texttt{Lang-Adapt} (step-2b): the LLMs after language adaptation with embedding tuning. (iii) \ouradapter (step-3 Combine): model denoted as \texttt{$*$-FA} is constructed by combining the transformer body of \texttt{$*$-FLAN} with the embeddings from \texttt{Lang-Adapt}. (iv): \texttt{LoRA-Adapt} (step-3 LoRA Tuning): \ouradapter models with the LoRA-Adaptation process. Detailed training procedures for each model type are in Appendix~\ref{appendix:training_details}.

% \inlinetitle{Training Details}
% For embedding training, we use a constant learning rate of $1\times 10^{-4}$ for \palmtwo and $1\times 10^{-5}$ for \gemmatwo and \aya with the Adam optimizer~\citep{kingma2014adam}. The embeddings is trained on a total of $\sim$200B tokens, with each batch consisting of examples packed to a maximum sequence length of 2K for \palmtwo and 8K for \gemmatwo and \aya. We follow the original papers to pre-train the model using the UL2 objectives~\citep{tay2023ul} for \palmtwo and causal language modeling objectives for \gemmatwo and \aya. Notably, for each LLM, we create new embeddings tailored to each of the three language groups (\ie \sea, \afr, and \ind).
  
% We instruction-tune the transformer body of LLMs on $\mathcal{D}_{it}$ with the same hyperparameters as above to obtain \texttt{Model-FLAN}. We use early stopping to select the best model based on the performance on MMLU~\citep{hendrycks2021measuring} and assemble it with customized embedding to obtain \texttt{Model-FA}. For LoRA-Adaptation, we set LoRA rank to 64 and use a learning rate of $5\times 10^{-6}$ for all models with 10\% steps of warm-up.

\subsection{Evaluation Tasks}
For LLMs after language adaptation, we adopt the \emph{five-shot} prompting strategy. In contrast, \ouradapter is evaluated in a \emph{zero-shot} setting, given it has been instruction tuned. We also evaluate \ouradapter using a compiled English benchmark (Appendix~\ref{appendix:english_tasks}) to examine potential regressions in general English ability.

\textbf{\belebele~\citep{bandarkar-etal-2024-belebele}}
is a multiple-choice reading comprehension (MRC) dataset 
with 122 languages.
% with 122 languages of varying resource levels.
The dataset contains 900 instances created from short passages sampled from \flores. 
We evaluation on all the 15 \sea, 25 \afr, and 19 \ind languages. 
We follow the original paper to use the \texttt{Accuracy} metric and sample five-shot prompts from the English training dataset. 

\textbf{\sib~\citep{adelani-etal-2024-sib}}
is a topic classification task based on texts sampled from \flores. 
% It was created by annotating the English samples and extending the annotation to the instances of the remaining languages. 
We select all the 25 \sea, 45 \afr and 26 \ind languages for evaluation with zero-shot prompting and use the \texttt{Accuracy} as the metric for performance comparison.

\textbf{\flores~\citep{goyal-etal-2022-flores}}
is a machine translation dataset with 200 languages. 
% The data was extracted from 842 English Wikipedia articles with diverse topics and domains, which have been translated into 200 languages by professional translators. 
We evaluate on all the 23 \sea, 42 \afr, and 21 \ind languages using \texttt{ChrF++} and sample five-shot prompts from the dev set for few-shot evaluation.

\textbf{GSM8K-NTL~\citep{shi2023language, bansal2024llm}}
is a silver benchmark created by automatically translating the English GSM8K dataset into 25 low-resource languages. For evaluation, we focus on 5 \sea, 5 \afr, and 10 \ind languages and follow~\citet{bansal2024llm} to use the \texttt{Accuracy} metric and fixed five-shot prompts.

\textbf{\textsc{IndicGenBench}~\citep{singh-etal-2024-indicgenbench}}
is a human-curation benchmark across 29 Indic languages. We evaluate on the cross-lingual question-answering dataset~\textsc{XorQA-In} and the cross-lingual summarization dataset~\textsc{XSum-In} with zero-shot prompting. The token-level \texttt{F1} and \texttt{ChrF} scores are reported in~\textsc{XorQA-In} and~\textsc{XSum-In}, respectively.

\begin{figure}
    \setlength{\abovecaptionskip}{-0.0001cm}
    \setlength{\belowcaptionskip}{-0.5cm}
    \centering
    \includegraphics[width=\linewidth]{figures/emb_surgery_results_gemma2.pdf}
    \vspace{-6mm}
    \caption{\flores \textsc{En-Xx} and \belebele \la results across all sizes of \gemmatwo models with 5-shot prompting. Absolute gains over the pre-trained models are reported. $\#P$: fraction of tuned parameters (\ie embedding parameters).}
    \vspace{-6mm}
    \label{fig:gemma_results}
\end{figure}

\subsection{Main Results}
We first present the results of \la through embedding tuning on its own (\S\ref{sec:model_training} -- Language Adaptation) in \S\ref{sec:embed_surgery_results}. Then we evaluate the application of \ouradapter alongside LoRA adaptation (\S\ref{sec:model_training} -- \ouradapter \& LoRA-Adaptation) in \S\ref{sec:franken_adapter_results}.

\begin{table}[t]
% \setlength{\belowcaptionskip}{-0.35cm}
\setlength{\tabcolsep}{6pt}
\footnotesize
\centering
\vspace{-3mm}
\caption{Language adaptation results on GSM8K-NTL. Best results are in \textbf{bold}. $\star$ denotes results reported by~\citet{bansal2024llm}.}
\resizebox{\linewidth}{!}{
\begin{tabular}{l|ccc|c}
    \toprule
    \bf Model & \bf\sea & \bf\afr & \bf\ind & \bf Avg.\\ 
    % \rowcolor{blue!15} \multicolumn{5}{c}{\bf\emph{\la (5-shot)}} \\
    \midrule
    CALM (\palmtwo-\texttt{XXS-NTL+S})$^\star$ & 25.3 & 17.8 & 17.9 & 19.8 \\
    \palmtwo-\texttt{S-NTL} & 25.2 & 17.4 & 15.1 & 18.2 \\
    \palmtwo-\texttt{S} & 22.0 & 15.3 & 14.2 & 16.4 \\
    {\pfix}  \texttt{Lang-Adapt} & \bf 25.6 & \bf 18.8 & \bf 22.3 & \bf 22.3 \\
    \midrule
    \gemmatwo-\texttt{9B} & 19.9 & 18.3 & 13.6 & 16.4 \\
    {\pfix}  \texttt{Lang-Adapt} & \bf 36.4 & \bf 25.8 & \bf 29.6 & \bf 30.4 \\
    \gemmatwo-\texttt{27B} & 30.6 & 22.2 & 18.4 & 22.4 \\
    {\pfix}  \texttt{Lang-Adapt} & \bf 41.9 & \bf 31.8 & \bf 27.5 & \bf 32.2 \\
    % \midrule
    % \aya-\texttt{8B} & 16.2 & 9.4 & \bf 8.4 & 10.6 \\
    % {\pfix}  \texttt{Lang-Adapt} & \bf 21.3 & \bf 10.2 & 7.2 & \bf 11.5 \\
    % \aya-\texttt{35B} & 23.5 & 13.7 & 12.7 & 15.7 \\
    % {\pfix}  \texttt{Lang-Adapt} & \bf 31.4 & \bf 17.2 & \bf 17.2 & \bf 20.8 \\
    
    \bottomrule
    \multicolumn{5}{c}{} \vspace{-6mm}
    \end{tabular}
    }
    % \caption{Language adaptation results on GSM8K-NTL with 5-shot prompting. The best results are marked in \textbf{bold}. $\dagger$ denotes results reported by~\citet{bansal2024llm}.}
\label{tab:gsm_8k_ntl_results}
\end{table}


\begin{table}[t]
% \setlength{\belowcaptionskip}{-0.35cm}
\setlength{\tabcolsep}{8.5pt}
\footnotesize
\centering
\vspace{-3mm}
\caption{\ouradapter results on GSM8K-NTL. The best results are marked in \textbf{bold}. $\dagger$ indicates that the LLMs are instruction-tuned without embedding freezing.}
\resizebox{\linewidth}{!}{
\begin{tabular}{l|ccc|c}
    \toprule
    \bf Model & \bf\sea & \bf\afr & \bf\ind & \bf Avg.\\ 
    \midrule
    \gemmatwo-\texttt{9B-FLAN} & 31.7 & 22.1 & 21.4 & 24.2 \\
    {\pfix}  \texttt{\ouradapter} & \bf 40.4 & \bf 27.8 & \bf 28.9 & \bf 31.5 \\
    % 38.6 & 25.7 & 24.2 & 29.5
    \gemmatwo-\texttt{27B-FLAN} & 33.5 & 26.3 & 27.4 & 28.7 \\
    {\pfix}  \texttt{\ouradapter} & \bf 44.9 & \bf 31.0 & \bf 30.9 & \bf 34.4 \\
    % 42.3 & 29.4 & 30.8 & 34.2
    \midrule
    \gemmatwo-\texttt{9B-IT}$^\dagger$ & 40.4 & 28.6 & 33.0 & 33.8 \\
    {\pfix}  \texttt{\ouradapter} & \bf 53.0 & \bf 35.6 & \bf 39.4 & \bf 41.9 \\
    \gemmatwo-\texttt{27B-IT}$^\dagger$ & 44.4 & \bf 34.1 & 37.6 & 38.4 \\
    {\pfix}  \texttt{\ouradapter} & \bf 49.6 & 26.5 & \bf 42.7 & \bf 40.4 \\
    
    % math instruction-tuned LLM
    \midrule
    \gemmatwo-\texttt{9B-Math} & 48.0 & 37.3 & \bf 35.2 & 38.9 \\
    {\pfix}  \texttt{\ouradapter} & \bf 62.1 & \bf 45.0 & 34.8 & \bf 44.2 \\
    \gemmatwo-\texttt{27B-Math} & 49.8 & 35.4 & \bf 32.3 & 37.5 \\
    {\pfix}  \texttt{\ouradapter} & \bf 59.0 & \bf 39.7 & 30.9 & \bf 40.1 \\
    
    \bottomrule
    \multicolumn{5}{c}{} \vspace{-10mm}
    \end{tabular}
    }
\label{tab:gsm_8k_ntl_fa_results}
\end{table}

\subsubsection{\la Results}\label{sec:embed_surgery_results}

\boldtitle{\la improves cross-lingual transfer.}
Figure~\ref{fig:gemma_results} shows the absolute gains of language adaptation on \gemmatwo models. When evaluating across three language groups, we observe that language adaptation consistently outperforms vanilla \gemmatwo models, demonstrating a significant performance advantage. The performance gains are particularly more pronounced on medium and low-resource languages, and this trend becomes increasingly apparent as model size scales. Similar findings are observed for \aya and \palmtwo, as presented in Appendix~\ref{appendix:extra_la_results}.

\begin{table*}[t]
\setlength{\belowcaptionskip}{-0.3cm}
\setlength{\tabcolsep}{6pt}
\footnotesize
\centering
\vspace{-2.5mm}
\caption{\ouradapter performance with zero-shot prompting. The best and second-best results are marked in \textbf{bold} and \underline{underlined}. \textcolor{red!80}{Red} values indicate instances where \ouradapter hurts the performance. English results are excluded when computing the average.}
\resizebox{\linewidth}{!}{
\begin{tabular}{l|c|ccc|ccc|ccc|cc|cc|c}
    \toprule
    \bf Task Type & \multirow{4}{*}[-1ex]{\rotatebox[origin=c]{90}{\bf \textsc{English}}} & \multicolumn{6}{c|}{\bf \textsc{Classification}} & \multicolumn{7}{c|}{\bf \textsc{Generation}} \\
    \cmidrule(lr){3-8} \cmidrule(lr){9-15}
    \multirow{2}{*}{\bf Eval. Metric} & & \multicolumn{3}{c|}{\bf \belebele} & \multicolumn{3}{c|}{\bf \sib} & \multicolumn{3}{c|}{\bf \flores} & \multicolumn{2}{c|}{\bf \textsc{XorQA-In}} & \multicolumn{2}{c|}{\bf \textsc{XSum-In}} & \multirow{4}{*}[2ex]{\bf Avg.} \\
    & & \multicolumn{3}{c|}{\bf \texttt{Accuracy}} & \multicolumn{3}{c|}{\bf \texttt{Accuracy}} & \multicolumn{3}{c|}{\bf \texttt{ChrF++}} & \multicolumn{2}{c|}{\bf \texttt{Token-F1}} & \multicolumn{2}{c|}{\bf \texttt{ChrF}} \\
    \cmidrule(lr){3-5} \cmidrule(lr){6-8} \cmidrule(lr){9-11} \cmidrule(lr){12-13} \cmidrule(lr){14-15}
    \bf Model & & \sea & \afr & \ind & \sea & \afr & \ind & \sea & \afr & \ind & \ind & \textsc{En} & \ind & \textsc{En} \\
    \midrule
    \gemmatwo-\texttt{2B-FLAN} & \bf 73.0 & 52.0 & 36.0 & 50.2 & 65.9 & 47.3 & 67.8 & 27.8 & 9.6 & 20.7 & 7.6 & 47.8 & 3.7 & \bf 36.6 & 31.7 \\ 
    \gemmatwo-\texttt{2B-FA} & \underline{69.8} & \underline{62.0} & \underline{40.2} & \underline{54.4} & \underline{72.9} & \bf 57.9 & \underline{70.5} & \textcolor{red!80}{27.0} & \underline{11.0} & \textcolor{red!80}{17.2} & \underline{8.6} & \underline{54.1} & \underline{6.6} & \underline{\textcolor{red!80}{34.1}} & \underline{35.3} \\
    {\pfix}  \texttt{LoRA-Adapt} & 69.7 & \bf 63.3 & \bf 43.7 & \bf 55.8 & \bf 73.5 & \underline{47.8} & \bf 70.7 & \bf 37.1 & \bf 18.3 & \bf 32.5 & \bf 9.8 & \bf 60.5 & \bf 9.9 & \textcolor{red!80}{31.9} & \bf 38.7 \\
    % SEA+AFR+IND tokenizer
    % \gemmatwo-\texttt{2B-FA-SAI} & 68.3 & 58.4 & 38.8 & 53.1 & 70.6 & 54.3 & 71.8 & 24.8 & 9.3 & 17.7 & 9.7 & 52.9 & 7.3 & 34.6 & 34.5 \\
    % {\pfix}  \texttt{LoRA-Adapt} & 69.6 & 59.6 & 41.8 & 54.4 & 69.5 & 54.4 & 72.4 & 33.7 & 15.9 & 31.9 & 9.7 & 54.3 & 10.2 & 32.2 & 37.4 \\
    \midrule
    \gemmatwo-\texttt{9B-FLAN} & \bf 83.5 & 70.6 & 49.9 & 68.9 & 74.4 & 61.0 & 79.1 & 32.0 & 12.6 & 27.8 & 9.8 & 60.3 & 15.1 & \bf 37.5 & 42.0 \\ 
    \gemmatwo-\texttt{9B-FA} & 82.3 & \underline{78.1} & \underline{57.5} & \underline{71.2} & \bf 78.6 & \underline{68.8} & \underline{79.8} & \underline{35.9} & \underline{16.5} & \underline{31.4} & \bf 12.5 & \bf 65.2 & \underline{15.2} & \underline{\textcolor{red!80}{37.1}} & \underline{45.5} \\
    {\pfix}  \texttt{LoRA-Adapt} & \underline{82.5} & \bf 78.4 & \bf 60.2 & \bf 73.1 & \underline{78.5} & \bf 69.2 & \bf 80.1 & \bf 40.0 & \bf 25.6 & \bf 40.5 & \underline{12.1} & \underline{64.1} & \bf 17.4 & \textcolor{red!80}{37.0} & \bf 47.3 \\
    % SEA+AFR+IND tokenizer
    % \gemmatwo-\texttt{9B-FA-SAI} & 82.5 & 75.8 & 55.7 & 71.4 & 79.1 & 67.8 & 80.7 & 34.3 & 15.1 & 32.4 & 12.0 & 65.6 & 14.9 & 37.7 & 45.1 \\
    % {\pfix}  \texttt{LoRA-Adapt} & 82.0 & 76.1 & 57.5 & 72.7 & 78.3 & 67.7 & 81.3 & 38.6 & 22.5 & 38.8 & 13.3 & 66.2 & 16.9 & 37.7 & 47.0 \\
    \midrule
    \gemmatwo-\texttt{27B-FLAN} & \bf 84.3 & 71.9 & 52.6 & 72.9 & 72.6 & 60.2 & 76.4 & 33.2 & \underline{15.2} & \underline{29.6} & \bf 20.4 & \underline{61.7} & \underline{15.0} & \bf 37.6 & \underline{43.7} \\ 
    \gemmatwo-\texttt{27B-FA} & \underline{83.6} & \underline{78.8} & \underline{56.3} & \underline{73.1} & \underline{78.1} & \underline{66.1} & \underline{78.5} & \underline{33.7} & \textcolor{red!80}{13.2} & \textcolor{red!80}{23.9} & \textcolor{red!80}{15.2} & \textcolor{red!80}{59.3} & \textcolor{red!80}{12.4} & \textcolor{red!80}{36.8} & \textcolor{red!80}{43.5} \\
    {\pfix}  \texttt{LoRA-Adapt} & 83.4 & \bf 79.5 & \bf 60.4 & \bf 74.3 & \bf 79.5 & \bf 68.3 & \bf 80.0 & \bf 42.5 & \bf 25.7 & \bf 40.6 & \underline{\textcolor{red!80}{16.4}} & \bf 70.1 & \bf 18.0 & \underline{\textcolor{red!80}{36.9}} & \bf 49.1 \\
    % SEA+AFR+IND tokenizer
    % \gemmatwo-\texttt{27B-FA-SAI} & 84.0 & 77.1 & 56.3 & 74.8 & 76.8 & 63.5 & 78.8 & 35.3 & 16.6 & 29.8 & 19.1 & 54.8 & 16.0 & 38.3 & 44.7 \\
    % {\pfix}  \texttt{LoRA-Adapt} & 83.8 & 77.6 & 59.2 & 74.7 & 77.1 & 67.0 & 80.3 & 39.6 & 22.5 & 39.8 & 21.5 & 67.6 & 17.9 & 37.2 & 48.5 \\
    \bottomrule
    \multicolumn{13}{c}{}\vspace{-9mm}
    \end{tabular}
    }
\label{tab:franken_adapter_results}
\end{table*}

% Serial Full parameter tuning (first in NTL then in FLAN+NTL (50%/50%)
% SEA:
% Gemma2-2B
% FLORES: 44.2 SIB-200: 66.9 BELEBELE: 55.4
% English
% MMLU: 43.1
% GSM: 30.0
% drop: 50.0
% winogrand: 67.1
% arc: 60.0
% hellaswag: 78.7
% piqa: 74.8
% boolq: 78.1
% avg: 59.7

% Gemma2-9B
% FLORES: 45.2 SIB-200: 83.0 BELEBELE: 76.3 GSM8K-NTL-SEA: 29.8 Avg: 68.2
% English
% MMLU: 59.1
% GSM: 40.4
% drop: 67.3
% winogrand: 78.0
% arc: 83.7
% hellaswag: 88.7
% piqa: 83.1
% boolq: 84.0
% avg: 73.0

% Gemma2-27B
% FLORES: 45.4 SIB-200: 72.6 BELEBELE: 81.2 GSM8K-NTL-SEA: 41.0 En Avg: 81.8
% English:
% drop: 75.6
% GSM: 56.0
% boolq: 90.6
% piqa: 88.6
% hellaswag: 94.6
% arc: 89.5
% winogrand: 89.5
% mmlu: 69.8


\boldtitle{\la helps preserve LLMs' general knowledge.}
% Table~\ref{tab:gsm_8k_ntl_results} shows that embedding surgery significantly enhances the reasoning abilities of LLMs on low-resource languages. Compared to CALM~\citep{bansal2024llm}, which imparts new skills to LLMs by augmenting a specialized LLM, our method achieves more substantial improvements on \palmtwo-\texttt{S} without incurring extra inference costs. Moreover, our approach shows superior performance (+4\%) when comparing to \palmtwo-\texttt{S-NTL} that was full-parameter tuned on NTL. This suggests that embedding surgery is more effective in transferring language knowledge to LLMs while preserving general knowledge. Notably, we observe boosted gains on \gemmatwo, indicating the significant potential of embedding surgery for enhancing the low-resource language reasoning abilities of high-capacity LLMs.
We evaluate whether language adaptation improves the transfer of English reasoning in LLMs to other languages. Table~\ref{tab:gsm_8k_ntl_results} shows LLMs with language adaptation consistently improves the mathematical reasoning ability across various low-resource languages. Compared to CALM~\citep{bansal2024llm}, a form of adapter enabling model composition, our method achieves more substantial improvements over \palmtwo-\texttt{S} with only embedding tuning and incurring no extra inference costs.\footnote{Our approach, however, requires more training as embedding tuning must be repeated for different linguistic regions.} Moreover, our approach shows superior performance (+4\%) compared to \palmtwo-\texttt{S-NTL} that was full-parameter tuned on NTL.\footnote{Both methods use the same training dataset. However, NTL training did not create region-specific models by splitting training data, potentially diminishing its effectiveness due to the curse of multilinguality~\citep{conneau-etal-2020-unsupervised}, which arises when a single model is trained on too many languages.} Overall, our results suggest that language adaptation offers an effective alternative to CALM and full-parameter tuning. We observe performance improvements with more modern models, with particularly pronounced gains in \gemmatwo.

\subsubsection{\ouradapter Results}\label{sec:franken_adapter_results}

\boldtitle{\ouradapter enhances mathematical reasoning performance in target languages on the fly.}
In Table~\ref{tab:gsm_8k_ntl_fa_results}, we show that \ouradapter consistently outperforms the instruction-tuned baseline for zero-shot evaluation on mathematical reasoning tasks, with an average improvement of up to 8\% across 20 low-resource languages. Moreover, \ouradapter further advances performance by incorporating LLMs with enhanced mathematical reasoning ability (\eg the $\texttt{IT}$ models aligned via reinforcement learning).\footnote{We observe LLMs generate Chain-of-Thought steps more often in English than in the question language, and we hypothesize that the customized embeddings in \ouradapter enable the model to better comprehend the questions.}

\ouradapter can also enhance the cross-lingual performance of LLMs instruction-tuned for reasoning. During the instruction-tuning stage (Figure~\ref{fig:franken_adapter} step 2(a)), we employ a mathematical instruction dataset, \texttt{WebInstruct}~\citep{yue2024mammoth}, for model training, and obtain the composed model by reusing the embeddings from the language adaptation stage. Table~\ref{tab:gsm_8k_ntl_fa_results} shows \ouradapter consistently improves performance in target languages. These results underscore its versatility by facilitating cross-lingual transfer in models post-trained for a specific domain.

% \boldtitle{\ouradapter is generally more robust to classification than generation tasks.}
\boldtitle{\ouradapter is more effective for classification tasks.}
As shown in Table~\ref{tab:franken_adapter_results}, \ouradapter consistently improves the performance on classification tasks across all model sizes with up to 10\% absolute gains. By contrast, for the generation tasks, the method's behavior is inconsistent, often leading to performance degradation. We attribute this phenomenon to intrinsic difficulty: classification tasks are generally easier as the solution space is typically small compared to generation tasks, making them more robust to embedding changes. However, the embedding layer is used for both text encoding and decoding in generation, and the auto-regressive generation paradigm makes the model sensitive to embedding changes due to error propagation accumulating over time steps.


\boldtitle{LoRA-Adaptation connects both worlds.}
In Table~\ref{tab:franken_adapter_results}, we show that by applying cost-effective LoRA-Adaptation, the two components within the \ouradapter can cooperate better, leading to significant gains on generation tasks. This results in an average improvement of 5.4\% on the largest 27B variant. The results demonstrate the practical use of our \ouradapter method for efficient zero-shot cross-lingual transfer of instruction-tuned LLMs. See Appendix~\ref{appendix:extra_fa_results} and Table~\ref{tab:franken_adapter_results_other_models} for additional results on \aya and \palmtwo. Furthermore, these improvements come at minimal cost on English proficiency (See Appendix~\ref{appendix:english_tasks}) and the same findings are observed for \aya \ \etc (See Table~\ref{tab:franken_adapter_results_other_models})

\subsection{Ablation Analysis}

\begin{figure}[t]
    \setlength{\abovecaptionskip}{-0.0001cm}
    \setlength{\belowcaptionskip}{-0.35cm}
    \centering
    \includegraphics[width=\linewidth]{figures/effects_customized_vocabulary.pdf}
    \vspace{-8mm}
    \caption{Ablations on tokenizers for language adaptation. Macro-averaged scores on \sea subset of \flores and \belebele are reported. $\Delta_{TF}$: \% tokenizer fertility reduction.}
    % \wishlist{Use same colours for each category and dashed line for the vanilla/pt models. Keep them consistent for different figures.}
    \vspace{-6mm}
    \label{fig:effects_customized_vocabulary}
\end{figure}

\begin{figure}[t]
    \setlength{\abovecaptionskip}{-0.0001cm}
    \setlength{\belowcaptionskip}{-0.35cm}
    \centering
    \includegraphics[width=\linewidth]{figures/effects_customized_vocabulary_franken_adapter.pdf}
    \vspace{-9.5mm}
    \caption{\ouradapter result comparison when using original and customized tokenizers. The macro-averaged scores on \sea subset of \belebele, \sib and \flores are reported. }
    \vspace{-4mm}
    \label{fig:effects_customized_vocabulary_franken_adapter}
\end{figure}

\boldtitle{Customized vocabulary amplifies the benefits of training on multilingual data.}
% We study the influence of multilingual pre-training data and customized vocabulary.
We decouple the effects of multilingual data in language adaptation from the change in vocabularies.
Figure~\ref{fig:effects_customized_vocabulary} shows that simple continued training on additional multilingual data with embedding tuning can significantly improve the performance in medium and low-resource languages, indicating LLMs are under-fitting to these languages. Based on this, employing a customized vocabulary that is fairer to these languages (Figure~\ref{fig:fertility_comparison}) can amplify the benefits of embedding tuning on multilingual data. This enhanced learning process facilitates better knowledge acquisition~\citep{zhang-etal-2022-robust,hofmann-etal-2022-embarrassingly}. Moreover, the improvements are more pronounced in smaller models, highlighting the importance of effective tokenization for these models to adapt well to low-resource languages. 
% In Figure~\ref{fig:effects_customized_vocabulary_franken_adapter}, we further show the employment of customized vocabularies leads to consistent performance gains under the \ouradapter framework, underscoring their importance for more effective cross-lingual transfer.
In addition, Figure~\ref{fig:effects_customized_vocabulary_franken_adapter} indicates that the importance of customized vocabulary is also apparent in the \ouradapter setting.

\begin{table}[t]
\setlength{\belowcaptionskip}{-0.2cm}
\setlength{\tabcolsep}{7pt}
% \footnotesize
\centering
\vspace{-2mm}
\caption{Comparing latency for using original and customized vocabularies in \ouradapter. The number of instances processed per second (\ie prefilling) by \gemmatwo are reported. We use the passages in \belebele for all \sea, \afr, and \ind languages.}
\resizebox{\columnwidth}{!}{
\begin{tabular}{l|l|lll}
    \toprule
    \bf Size & \bf Vocab & \bf\textsc{Hrl} & \bf\textsc{Mrl} & \bf\textsc{Lrl} \\ 
    \midrule
    \multirow{2}{*}{2B} & Original & 59.1 & 55.6 & 57.1 \\
    & Custom & 61.1 {\tiny\uagray{3.38\%}} & 61.2 {\tiny\uagray{10.07\%}} & 60.3 {\tiny\uagray{5.60\%}} \\
    \midrule

    \multirow{2}{*}{9B} & Original & 30.2 & 27.2 & 28.6 \\
    & Custom & 33.8 {\tiny\uagray{11.92\%}} & 33.3 {\tiny\uagray{22.43\%}} & 32.1 {\tiny\uagray{12.24\%}} \\
    \midrule

    \multirow{2}{*}{27B} & Original & 17.1 & 15.0 & 16.0 \\
    & Custom & 19.6 {\tiny\uagray{14.62\%}} & 19.2 {\tiny\uagray{28.0\%}} & 18.4 {\tiny\uagray{15.0\%}} \\
    \bottomrule
    \multicolumn{5}{c}{}\vspace{-9mm}
    \end{tabular}}
    
\label{tab:latency_compare}
\end{table}
\boldtitle{Customized vocabulary improves latency.}
We evaluate latency by measuring the number of texts processed per second by LLMs. We use the passages from all \sea, \afr, and \ind languages in \belebele as test texts. For comparison, we test \ouradapter models integrated with embeddings trained using either the original \gemmatwo tokenizer or our customized one. Table~\ref{tab:latency_compare} shows that employing customized tokenizer consistently improves latency, particularly in MRLs and LRLs. This trend becomes increasingly pronounced as model size scales, highlighting the importance of customized tokenizers in achieving low-latency processing for long-tail languages in larger LLMs.

\begin{figure}[t]
    \setlength{\abovecaptionskip}{-0.0001cm}
    \setlength{\belowcaptionskip}{-0.35cm}
    \centering
    \includegraphics[width=\linewidth]{figures/fertility_vs_performance.pdf}
    \vspace{-8.5mm}
    \caption{Correlation between the results of language adaptation on \gemmatwo-\texttt{2B} with tokenizer fertility. Normalized \texttt{ChrF++} on \textsc{Flores}-\sea is reported. {$\bLozenge$} and {\pfix}  indicate the original and our customized tokenizers used in the other settings, respectively.}
    % \fan{Should we indicate leftmost and rightmost dots for HRLs and LRLs are from the original \gemmatwo tokenizer?}
    \vspace{-5.5mm}
    \label{fig:fertility_vs_performance}
\end{figure}

\boldtitle{Tokenization fertility is inversely correlated to downstream performance.}
We study how tokenization fertility (the average number of tokens produced per word) affects the LLM's performance across languages. 
To ablate this effect, we generate several re-sampled replicates of our tokenizer training datasets with different levels of priority given to high \versus lower resource languages. Specifically, we use temperature sampling to manipulate the training sentences of each language for building different tokenizer training data, and train tokenizers with varying fertilities. We then do language adaptation for each of these tokenizers and analyze the downstream performance. As shown in Figure~\ref{fig:fertility_vs_performance}, we find that the performance is inversely correlated to tokenization fertility, but the correlation is not uniform across languages. Notably, slight reductions in fertility lead to significant performance improvements in low-resource languages (\eg \textsc{Ace}) while high-resource languages are less sensitive to fertility changes.
Furthermore, Latin-script languages generally benefit more from fertility reductions compared to those in non-Latin scripts.
% \edit{Languages written in non-Latin scripts also favor low fertility regardless of resource levels (\eg \textsc{Th} and \textsc{Lo}).}\fan{\trevor{is your point that the blue and red points are roughly on the same line? I'm a bit puzzled, can discuss.} Need to discuss on this.} However, for these languages, monotonically reducing fertility is not always beneficial, with the performance saturating after reaching a certain threshold.
Please refer to Appendix Figure~\ref{fig:fertility_vs_performance_palm2} which reports similar findings for \palmtwo-\texttt{XXS}.

\boldtitle{Pruning with extension outperforms other tokenizer construction methods.}
We study the impact of tokenizer construction methods on the performance of language adaptation. We compare our \emph{Prune-with-Extension} method (\S\ref{sec:vocab_construct}) against two variants: 1) \emph{Scratch}, which trains tokenizers from scratch on English and target languages; and 2) \emph{Extension}, which appends target language tokens to existing tokenizers without pruning. This adds an additional of 34\% embedding parameters. Figure~\ref{fig:ablation_on_tokenizer_building} shows our \emph{Prune+Extension} method achieves the best overall results. Notably, the performance differences on \textsc{Flores} are minimal across the methods. We attribute this to the relatively lower intrinsic complexity of \textsc{Flores} compared to \belebele that demands reasoning ability most likely transferred from English. This claim is supported by the substantial improvements in English on \belebele when switching to our approach that preserves original English tokens. Moreover, we observe that tokenizer built from scratch exhibits significantly fewer overlapping English tokens with the original tokenizer compared to the other methods. This signifies the importance of retaining English tokens to preserve pre-training knowledge, which is vital for the success of language adaptation.\footnote{It's an open question of whether other facets of the tokenizer need to be retained to preserve other behaviours, e.g., markup tokens to facilitate code understanding.} Our method also surpasses \emph{Extension}, indicating that removing irrelevant tokens is beneficial while avoid introducing extra parameters.\footnote{We suspect large vocabulary could increase ambiguity in output $\operatorname{softmax}$. Evidence also reveals that large vocabularies are not optimal for smaller LLMs~\citep{tao2024scaling}.}

\begin{figure}[t]
    \setlength{\abovecaptionskip}{-0.0001cm}
    \setlength{\belowcaptionskip}{-0.35cm}
    \centering
    \includegraphics[width=\linewidth]{figures/ablation_on_tokenizer_building.pdf}
    \vspace{-9mm}
    \caption{Ablations on tokenizer building methods. We report \sea language adaptation on \gemmatwo-\texttt{2B}.}
    \vspace{-4.5mm}
    \label{fig:ablation_on_tokenizer_building}
\end{figure}
\begin{figure}[t]
    \setlength{\abovecaptionskip}{-0.0001cm}
    \setlength{\belowcaptionskip}{-0.35cm}
    \centering
    \includegraphics[width=\linewidth]{figures/franken_adapter_vs_emb_surgery_on_it.pdf}
    \vspace{-9mm}
    \caption{Comparing applying language adaptation and \ouradapter on \gemmatwo-\texttt{IT} models. The averaged scores in five evaluation benchmarks (Table~\ref{tab:franken_adapter_results}) are reported.}
    \vspace{-6mm}
    \label{fig:franken_adapter_vs_emb_surgery_on_it}
\end{figure}

\boldtitle{\la on IT LLMs represents a viable alternative to \ouradapter .}
LLMs are typically released in both pre-trained (\textsc{Pt}) and instruct (\textsc{It}) versions. These versions provide two distinct pathways for achieving zero-shot cross lingual transfer: 1) employing our \ouradapter pipeline, which uses the \textsc{Pt} model for language adaptation and subsequently integrates the new embeddings into the \textsc{It} model through LoRA-Adaptation; 2) employing the \textsc{It}+Lang-Adapt approach, which performs one-step language adaptation directly on \textsc{It} model with the same training costs as on \textsc{Pt} model. Figure~\ref{fig:franken_adapter_vs_emb_surgery_on_it} shows a comparison of these two strategies. Our analysis reveals that for smaller model sizes, \ouradapter outperforms \textsc{It}+Lang-Adapt in downstream tasks. However, as model size increases, \textsc{It}+Lang-Adapt emerges as the more effective solution. 
% Nonetheless, the \ouradapter pipeline demonstrates greater flexibility by enabling the integration of pre-trained embeddings into various \textsc{It} models with minimal computational overhead via LoRA-Adaptation. In contrast, the \textsc{It}+Emb Surgery would be significantly more expensive if language adaptation of several \textsc{It} models all stemming from one common \textsc{Pt} model is required.
Overall, the \textsc{It}+Lang-Adapt method provides a highly effective alternative to \ouradapter for zero-shot cross-lingual transfer involving a single model. However, when dealing with multiple \textsc{It} models derived from a common \textsc{Pt} model, \ouradapter offers greater modularity by avoiding costly embedding tuning across individual models.
\section{Related Work}

\inlinetitle{Language Adaptation of LLMs.}
% The coverage of LLMs remains largely confined to English and a limited subset of other languages. 
Progress in LLMs have been dominated by English and other high-resource languages.
A growing body of research has shown progress in extending the applicability of LLMs to a broader range of languages. These methods typically involve continued pre-training on additional monolingual data~\citep{cui2024efficienteffectivetextencoding,zhao2024llamaenglishempiricalstudy} or multilingual instruction-tuning on synthetic data ~\citep{chen-etal-2024-monolingual,ustun-etal-2024-aya,aryabumi2024aya}. Notably, such efforts primarily focus on monolingual transfer or the multilingual adaptation of a few high-resource languages. Furthermore, these techniques often rely on computationally expensive full-parameter tuning, a strategy that tends to cause adapted LLMs to lose previously acquired knowledge due to catastrophic forgetting. \citet{bansal2024llm} augments an LLM with a smaller multilingual-specialized LLM by introducing a limited set of trainable parameters, yielding improved language adaptation while retaining general pre-trained knowledge. In contrast, we demonstrate that embedding tuning with additional multilingual data is sufficient for adapting LLMs to massive languages and creating customized vocabularies tailored to these languages can further enhance cross-lingual transfer.

\inlinetitle{Efficient Methods for Cross-Lingual Transfer.}
Diverse approaches have been proposed to efficiently adapt pre-trained LMs to new languages. These methods include using adapters~\citep{pfeiffer-etal-2020-mad,pfeiffer-etal-2022-lifting,yong-etal-2023-bloom,liu2022fewshot} and sparse fine-tuning~\citep{ansell-etal-2022-composable}. While effective, such parameter-efficient methods have been shown to be less effective in acquiring new knowledge~\citep{biderman2024lora}. Embedding surgery, on the other hand, has proven highly effective for adapting PLMs between languages while keeping the transformer body untouched ~\citep{artetxe-etal-2020-cross,de-vries-nissim-2021-good,marchisio-etal-2023-mini,chen2023improving}. However, its application has been limited to encoder-only PLMs and constrained by monolingual transfer, where one embedding is created per target language. Our \ouradapter approach builds upon this idea but differs significantly in that: we demonstrate its effectiveness with more powerful decoder-only LLMs for much wider range of tasks requiring different skill sets, and extend its capability to support massive cross-lingual transfer, enabling adaptation to a broader range of long-tail languages with a single embedding.

% Tokenizer Manipulation for LLMs.

\section{Conclusion}
In this work, we introduce \ouradapter, a novel approach that facilitates efficient zero-shot cross-lingual transfer across a wide range of languages with embedding surgery. Our empirical findings reveal that embedding tuning on multilingual data suffices for the effective language adaptation of LLMs. Moreover, this adaptation can be further optimized through the implementation of customized vocabularies tailored to target languages. We also highlight the adaptability of these embeddings, which can be integrated into any instruction-tuned LLMs to enable cross-lingual transfer with minimal training costs.

\section*{Impact Statement}
% This paper presents work whose goal is to advance the field of Machine Learning. There are many potential societal consequences of our work, none which we feel must be specifically highlighted here.
This work presents novel insights into the language adaptation and cross-lingual transfer of LLMs, which have the potential to change the ways in which LLMs are trained and adapted to new languages. We believe \ouradapter can serve as a versatile modular framework to expand the capabilities of LLMs beyond language, such as domain adaptations for enhanced coding and mathematical skills. There might also be safety concerns arising from the disruption of alignment abilities due to the changes to the embeddings, and how to align the model effectively and efficiently to reject unsafe queries after the \ouradapter pipeline which would need to be carefully addressed.

\section*{Acknowledgment}
The authors would thank Isaac Caswell for the valuable comments and feedback. The authors would also thank Daniel Formoso for helping support the research.

\bibliography{anthology,custom}
\bibliographystyle{icml2025}


%%%%%%%%%%%%%%%%%%%%%%%%%%%%%%%%%%%%%%%%%%%%%%%%%%%%%%%%%%%%%%%%%%%%%%%%%%%%%%%
%%%%%%%%%%%%%%%%%%%%%%%%%%%%%%%%%%%%%%%%%%%%%%%%%%%%%%%%%%%%%%%%%%%%%%%%%%%%%%%
% APPENDIX
%%%%%%%%%%%%%%%%%%%%%%%%%%%%%%%%%%%%%%%%%%%%%%%%%%%%%%%%%%%%%%%%%%%%%%%%%%%%%%%
%%%%%%%%%%%%%%%%%%%%%%%%%%%%%%%%%%%%%%%%%%%%%%%%%%%%%%%%%%%%%%%%%%%%%%%%%%%%%%%
\clearpage
\appendix
\section*{Overview of Appendix}
Our supplementary includes the following sections:
\begin{itemize}
    \item Section~\ref{appendix:exp_settings}: Experimental Settings, including implementation details for training and evaluation.
    \item Section~\ref{appendix:english_tasks}: Detailed result comparison between instruction-tuned models and \ouradapter in English tasks.
    \item Section~\ref{appendix:extra_main_results}: Additional \la and \ouradapter results on \palmtwo, \aya and \gemmatwo-\texttt{IT} models.
    \item Section~\ref{appendix:cpt}: Results of the continued pre-training baseline for language adaptation.
    \item Section~\ref{appendix:supplementary_analysis}: Supplementary analysis including additional data ablations.
    \item Section~\ref{appendix:detailed_lang_results}: Results by language for both language adaptation and \ouradapter.
\end{itemize}

\subsection{Lloyd-Max Algorithm}
\label{subsec:Lloyd-Max}
For a given quantization bitwidth $B$ and an operand $\bm{X}$, the Lloyd-Max algorithm finds $2^B$ quantization levels $\{\hat{x}_i\}_{i=1}^{2^B}$ such that quantizing $\bm{X}$ by rounding each scalar in $\bm{X}$ to the nearest quantization level minimizes the quantization MSE. 

The algorithm starts with an initial guess of quantization levels and then iteratively computes quantization thresholds $\{\tau_i\}_{i=1}^{2^B-1}$ and updates quantization levels $\{\hat{x}_i\}_{i=1}^{2^B}$. Specifically, at iteration $n$, thresholds are set to the midpoints of the previous iteration's levels:
\begin{align*}
    \tau_i^{(n)}=\frac{\hat{x}_i^{(n-1)}+\hat{x}_{i+1}^{(n-1)}}2 \text{ for } i=1\ldots 2^B-1
\end{align*}
Subsequently, the quantization levels are re-computed as conditional means of the data regions defined by the new thresholds:
\begin{align*}
    \hat{x}_i^{(n)}=\mathbb{E}\left[ \bm{X} \big| \bm{X}\in [\tau_{i-1}^{(n)},\tau_i^{(n)}] \right] \text{ for } i=1\ldots 2^B
\end{align*}
where to satisfy boundary conditions we have $\tau_0=-\infty$ and $\tau_{2^B}=\infty$. The algorithm iterates the above steps until convergence.

Figure \ref{fig:lm_quant} compares the quantization levels of a $7$-bit floating point (E3M3) quantizer (left) to a $7$-bit Lloyd-Max quantizer (right) when quantizing a layer of weights from the GPT3-126M model at a per-tensor granularity. As shown, the Lloyd-Max quantizer achieves substantially lower quantization MSE. Further, Table \ref{tab:FP7_vs_LM7} shows the superior perplexity achieved by Lloyd-Max quantizers for bitwidths of $7$, $6$ and $5$. The difference between the quantizers is clear at 5 bits, where per-tensor FP quantization incurs a drastic and unacceptable increase in perplexity, while Lloyd-Max quantization incurs a much smaller increase. Nevertheless, we note that even the optimal Lloyd-Max quantizer incurs a notable ($\sim 1.5$) increase in perplexity due to the coarse granularity of quantization. 

\begin{figure}[h]
  \centering
  \includegraphics[width=0.7\linewidth]{sections/figures/LM7_FP7.pdf}
  \caption{\small Quantization levels and the corresponding quantization MSE of Floating Point (left) vs Lloyd-Max (right) Quantizers for a layer of weights in the GPT3-126M model.}
  \label{fig:lm_quant}
\end{figure}

\begin{table}[h]\scriptsize
\begin{center}
\caption{\label{tab:FP7_vs_LM7} \small Comparing perplexity (lower is better) achieved by floating point quantizers and Lloyd-Max quantizers on a GPT3-126M model for the Wikitext-103 dataset.}
\begin{tabular}{c|cc|c}
\hline
 \multirow{2}{*}{\textbf{Bitwidth}} & \multicolumn{2}{|c|}{\textbf{Floating-Point Quantizer}} & \textbf{Lloyd-Max Quantizer} \\
 & Best Format & Wikitext-103 Perplexity & Wikitext-103 Perplexity \\
\hline
7 & E3M3 & 18.32 & 18.27 \\
6 & E3M2 & 19.07 & 18.51 \\
5 & E4M0 & 43.89 & 19.71 \\
\hline
\end{tabular}
\end{center}
\end{table}

\subsection{Proof of Local Optimality of LO-BCQ}
\label{subsec:lobcq_opt_proof}
For a given block $\bm{b}_j$, the quantization MSE during LO-BCQ can be empirically evaluated as $\frac{1}{L_b}\lVert \bm{b}_j- \bm{\hat{b}}_j\rVert^2_2$ where $\bm{\hat{b}}_j$ is computed from equation (\ref{eq:clustered_quantization_definition}) as $C_{f(\bm{b}_j)}(\bm{b}_j)$. Further, for a given block cluster $\mathcal{B}_i$, we compute the quantization MSE as $\frac{1}{|\mathcal{B}_{i}|}\sum_{\bm{b} \in \mathcal{B}_{i}} \frac{1}{L_b}\lVert \bm{b}- C_i^{(n)}(\bm{b})\rVert^2_2$. Therefore, at the end of iteration $n$, we evaluate the overall quantization MSE $J^{(n)}$ for a given operand $\bm{X}$ composed of $N_c$ block clusters as:
\begin{align*}
    \label{eq:mse_iter_n}
    J^{(n)} = \frac{1}{N_c} \sum_{i=1}^{N_c} \frac{1}{|\mathcal{B}_{i}^{(n)}|}\sum_{\bm{v} \in \mathcal{B}_{i}^{(n)}} \frac{1}{L_b}\lVert \bm{b}- B_i^{(n)}(\bm{b})\rVert^2_2
\end{align*}

At the end of iteration $n$, the codebooks are updated from $\mathcal{C}^{(n-1)}$ to $\mathcal{C}^{(n)}$. However, the mapping of a given vector $\bm{b}_j$ to quantizers $\mathcal{C}^{(n)}$ remains as  $f^{(n)}(\bm{b}_j)$. At the next iteration, during the vector clustering step, $f^{(n+1)}(\bm{b}_j)$ finds new mapping of $\bm{b}_j$ to updated codebooks $\mathcal{C}^{(n)}$ such that the quantization MSE over the candidate codebooks is minimized. Therefore, we obtain the following result for $\bm{b}_j$:
\begin{align*}
\frac{1}{L_b}\lVert \bm{b}_j - C_{f^{(n+1)}(\bm{b}_j)}^{(n)}(\bm{b}_j)\rVert^2_2 \le \frac{1}{L_b}\lVert \bm{b}_j - C_{f^{(n)}(\bm{b}_j)}^{(n)}(\bm{b}_j)\rVert^2_2
\end{align*}

That is, quantizing $\bm{b}_j$ at the end of the block clustering step of iteration $n+1$ results in lower quantization MSE compared to quantizing at the end of iteration $n$. Since this is true for all $\bm{b} \in \bm{X}$, we assert the following:
\begin{equation}
\begin{split}
\label{eq:mse_ineq_1}
    \tilde{J}^{(n+1)} &= \frac{1}{N_c} \sum_{i=1}^{N_c} \frac{1}{|\mathcal{B}_{i}^{(n+1)}|}\sum_{\bm{b} \in \mathcal{B}_{i}^{(n+1)}} \frac{1}{L_b}\lVert \bm{b} - C_i^{(n)}(b)\rVert^2_2 \le J^{(n)}
\end{split}
\end{equation}
where $\tilde{J}^{(n+1)}$ is the the quantization MSE after the vector clustering step at iteration $n+1$.

Next, during the codebook update step (\ref{eq:quantizers_update}) at iteration $n+1$, the per-cluster codebooks $\mathcal{C}^{(n)}$ are updated to $\mathcal{C}^{(n+1)}$ by invoking the Lloyd-Max algorithm \citep{Lloyd}. We know that for any given value distribution, the Lloyd-Max algorithm minimizes the quantization MSE. Therefore, for a given vector cluster $\mathcal{B}_i$ we obtain the following result:

\begin{equation}
    \frac{1}{|\mathcal{B}_{i}^{(n+1)}|}\sum_{\bm{b} \in \mathcal{B}_{i}^{(n+1)}} \frac{1}{L_b}\lVert \bm{b}- C_i^{(n+1)}(\bm{b})\rVert^2_2 \le \frac{1}{|\mathcal{B}_{i}^{(n+1)}|}\sum_{\bm{b} \in \mathcal{B}_{i}^{(n+1)}} \frac{1}{L_b}\lVert \bm{b}- C_i^{(n)}(\bm{b})\rVert^2_2
\end{equation}

The above equation states that quantizing the given block cluster $\mathcal{B}_i$ after updating the associated codebook from $C_i^{(n)}$ to $C_i^{(n+1)}$ results in lower quantization MSE. Since this is true for all the block clusters, we derive the following result: 
\begin{equation}
\begin{split}
\label{eq:mse_ineq_2}
     J^{(n+1)} &= \frac{1}{N_c} \sum_{i=1}^{N_c} \frac{1}{|\mathcal{B}_{i}^{(n+1)}|}\sum_{\bm{b} \in \mathcal{B}_{i}^{(n+1)}} \frac{1}{L_b}\lVert \bm{b}- C_i^{(n+1)}(\bm{b})\rVert^2_2  \le \tilde{J}^{(n+1)}   
\end{split}
\end{equation}

Following (\ref{eq:mse_ineq_1}) and (\ref{eq:mse_ineq_2}), we find that the quantization MSE is non-increasing for each iteration, that is, $J^{(1)} \ge J^{(2)} \ge J^{(3)} \ge \ldots \ge J^{(M)}$ where $M$ is the maximum number of iterations. 
%Therefore, we can say that if the algorithm converges, then it must be that it has converged to a local minimum. 
\hfill $\blacksquare$


\begin{figure}
    \begin{center}
    \includegraphics[width=0.5\textwidth]{sections//figures/mse_vs_iter.pdf}
    \end{center}
    \caption{\small NMSE vs iterations during LO-BCQ compared to other block quantization proposals}
    \label{fig:nmse_vs_iter}
\end{figure}

Figure \ref{fig:nmse_vs_iter} shows the empirical convergence of LO-BCQ across several block lengths and number of codebooks. Also, the MSE achieved by LO-BCQ is compared to baselines such as MXFP and VSQ. As shown, LO-BCQ converges to a lower MSE than the baselines. Further, we achieve better convergence for larger number of codebooks ($N_c$) and for a smaller block length ($L_b$), both of which increase the bitwidth of BCQ (see Eq \ref{eq:bitwidth_bcq}).


\subsection{Additional Accuracy Results}
%Table \ref{tab:lobcq_config} lists the various LOBCQ configurations and their corresponding bitwidths.
\begin{table}
\setlength{\tabcolsep}{4.75pt}
\begin{center}
\caption{\label{tab:lobcq_config} Various LO-BCQ configurations and their bitwidths.}
\begin{tabular}{|c||c|c|c|c||c|c||c|} 
\hline
 & \multicolumn{4}{|c||}{$L_b=8$} & \multicolumn{2}{|c||}{$L_b=4$} & $L_b=2$ \\
 \hline
 \backslashbox{$L_A$\kern-1em}{\kern-1em$N_c$} & 2 & 4 & 8 & 16 & 2 & 4 & 2 \\
 \hline
 64 & 4.25 & 4.375 & 4.5 & 4.625 & 4.375 & 4.625 & 4.625\\
 \hline
 32 & 4.375 & 4.5 & 4.625& 4.75 & 4.5 & 4.75 & 4.75 \\
 \hline
 16 & 4.625 & 4.75& 4.875 & 5 & 4.75 & 5 & 5 \\
 \hline
\end{tabular}
\end{center}
\end{table}

%\subsection{Perplexity achieved by various LO-BCQ configurations on Wikitext-103 dataset}

\begin{table} \centering
\begin{tabular}{|c||c|c|c|c||c|c||c|} 
\hline
 $L_b \rightarrow$& \multicolumn{4}{c||}{8} & \multicolumn{2}{c||}{4} & 2\\
 \hline
 \backslashbox{$L_A$\kern-1em}{\kern-1em$N_c$} & 2 & 4 & 8 & 16 & 2 & 4 & 2  \\
 %$N_c \rightarrow$ & 2 & 4 & 8 & 16 & 2 & 4 & 2 \\
 \hline
 \hline
 \multicolumn{8}{c}{GPT3-1.3B (FP32 PPL = 9.98)} \\ 
 \hline
 \hline
 64 & 10.40 & 10.23 & 10.17 & 10.15 &  10.28 & 10.18 & 10.19 \\
 \hline
 32 & 10.25 & 10.20 & 10.15 & 10.12 &  10.23 & 10.17 & 10.17 \\
 \hline
 16 & 10.22 & 10.16 & 10.10 & 10.09 &  10.21 & 10.14 & 10.16 \\
 \hline
  \hline
 \multicolumn{8}{c}{GPT3-8B (FP32 PPL = 7.38)} \\ 
 \hline
 \hline
 64 & 7.61 & 7.52 & 7.48 &  7.47 &  7.55 &  7.49 & 7.50 \\
 \hline
 32 & 7.52 & 7.50 & 7.46 &  7.45 &  7.52 &  7.48 & 7.48  \\
 \hline
 16 & 7.51 & 7.48 & 7.44 &  7.44 &  7.51 &  7.49 & 7.47  \\
 \hline
\end{tabular}
\caption{\label{tab:ppl_gpt3_abalation} Wikitext-103 perplexity across GPT3-1.3B and 8B models.}
\end{table}

\begin{table} \centering
\begin{tabular}{|c||c|c|c|c||} 
\hline
 $L_b \rightarrow$& \multicolumn{4}{c||}{8}\\
 \hline
 \backslashbox{$L_A$\kern-1em}{\kern-1em$N_c$} & 2 & 4 & 8 & 16 \\
 %$N_c \rightarrow$ & 2 & 4 & 8 & 16 & 2 & 4 & 2 \\
 \hline
 \hline
 \multicolumn{5}{|c|}{Llama2-7B (FP32 PPL = 5.06)} \\ 
 \hline
 \hline
 64 & 5.31 & 5.26 & 5.19 & 5.18  \\
 \hline
 32 & 5.23 & 5.25 & 5.18 & 5.15  \\
 \hline
 16 & 5.23 & 5.19 & 5.16 & 5.14  \\
 \hline
 \multicolumn{5}{|c|}{Nemotron4-15B (FP32 PPL = 5.87)} \\ 
 \hline
 \hline
 64  & 6.3 & 6.20 & 6.13 & 6.08  \\
 \hline
 32  & 6.24 & 6.12 & 6.07 & 6.03  \\
 \hline
 16  & 6.12 & 6.14 & 6.04 & 6.02  \\
 \hline
 \multicolumn{5}{|c|}{Nemotron4-340B (FP32 PPL = 3.48)} \\ 
 \hline
 \hline
 64 & 3.67 & 3.62 & 3.60 & 3.59 \\
 \hline
 32 & 3.63 & 3.61 & 3.59 & 3.56 \\
 \hline
 16 & 3.61 & 3.58 & 3.57 & 3.55 \\
 \hline
\end{tabular}
\caption{\label{tab:ppl_llama7B_nemo15B} Wikitext-103 perplexity compared to FP32 baseline in Llama2-7B and Nemotron4-15B, 340B models}
\end{table}

%\subsection{Perplexity achieved by various LO-BCQ configurations on MMLU dataset}


\begin{table} \centering
\begin{tabular}{|c||c|c|c|c||c|c|c|c|} 
\hline
 $L_b \rightarrow$& \multicolumn{4}{c||}{8} & \multicolumn{4}{c||}{8}\\
 \hline
 \backslashbox{$L_A$\kern-1em}{\kern-1em$N_c$} & 2 & 4 & 8 & 16 & 2 & 4 & 8 & 16  \\
 %$N_c \rightarrow$ & 2 & 4 & 8 & 16 & 2 & 4 & 2 \\
 \hline
 \hline
 \multicolumn{5}{|c|}{Llama2-7B (FP32 Accuracy = 45.8\%)} & \multicolumn{4}{|c|}{Llama2-70B (FP32 Accuracy = 69.12\%)} \\ 
 \hline
 \hline
 64 & 43.9 & 43.4 & 43.9 & 44.9 & 68.07 & 68.27 & 68.17 & 68.75 \\
 \hline
 32 & 44.5 & 43.8 & 44.9 & 44.5 & 68.37 & 68.51 & 68.35 & 68.27  \\
 \hline
 16 & 43.9 & 42.7 & 44.9 & 45 & 68.12 & 68.77 & 68.31 & 68.59  \\
 \hline
 \hline
 \multicolumn{5}{|c|}{GPT3-22B (FP32 Accuracy = 38.75\%)} & \multicolumn{4}{|c|}{Nemotron4-15B (FP32 Accuracy = 64.3\%)} \\ 
 \hline
 \hline
 64 & 36.71 & 38.85 & 38.13 & 38.92 & 63.17 & 62.36 & 63.72 & 64.09 \\
 \hline
 32 & 37.95 & 38.69 & 39.45 & 38.34 & 64.05 & 62.30 & 63.8 & 64.33  \\
 \hline
 16 & 38.88 & 38.80 & 38.31 & 38.92 & 63.22 & 63.51 & 63.93 & 64.43  \\
 \hline
\end{tabular}
\caption{\label{tab:mmlu_abalation} Accuracy on MMLU dataset across GPT3-22B, Llama2-7B, 70B and Nemotron4-15B models.}
\end{table}


%\subsection{Perplexity achieved by various LO-BCQ configurations on LM evaluation harness}

\begin{table} \centering
\begin{tabular}{|c||c|c|c|c||c|c|c|c|} 
\hline
 $L_b \rightarrow$& \multicolumn{4}{c||}{8} & \multicolumn{4}{c||}{8}\\
 \hline
 \backslashbox{$L_A$\kern-1em}{\kern-1em$N_c$} & 2 & 4 & 8 & 16 & 2 & 4 & 8 & 16  \\
 %$N_c \rightarrow$ & 2 & 4 & 8 & 16 & 2 & 4 & 2 \\
 \hline
 \hline
 \multicolumn{5}{|c|}{Race (FP32 Accuracy = 37.51\%)} & \multicolumn{4}{|c|}{Boolq (FP32 Accuracy = 64.62\%)} \\ 
 \hline
 \hline
 64 & 36.94 & 37.13 & 36.27 & 37.13 & 63.73 & 62.26 & 63.49 & 63.36 \\
 \hline
 32 & 37.03 & 36.36 & 36.08 & 37.03 & 62.54 & 63.51 & 63.49 & 63.55  \\
 \hline
 16 & 37.03 & 37.03 & 36.46 & 37.03 & 61.1 & 63.79 & 63.58 & 63.33  \\
 \hline
 \hline
 \multicolumn{5}{|c|}{Winogrande (FP32 Accuracy = 58.01\%)} & \multicolumn{4}{|c|}{Piqa (FP32 Accuracy = 74.21\%)} \\ 
 \hline
 \hline
 64 & 58.17 & 57.22 & 57.85 & 58.33 & 73.01 & 73.07 & 73.07 & 72.80 \\
 \hline
 32 & 59.12 & 58.09 & 57.85 & 58.41 & 73.01 & 73.94 & 72.74 & 73.18  \\
 \hline
 16 & 57.93 & 58.88 & 57.93 & 58.56 & 73.94 & 72.80 & 73.01 & 73.94  \\
 \hline
\end{tabular}
\caption{\label{tab:mmlu_abalation} Accuracy on LM evaluation harness tasks on GPT3-1.3B model.}
\end{table}

\begin{table} \centering
\begin{tabular}{|c||c|c|c|c||c|c|c|c|} 
\hline
 $L_b \rightarrow$& \multicolumn{4}{c||}{8} & \multicolumn{4}{c||}{8}\\
 \hline
 \backslashbox{$L_A$\kern-1em}{\kern-1em$N_c$} & 2 & 4 & 8 & 16 & 2 & 4 & 8 & 16  \\
 %$N_c \rightarrow$ & 2 & 4 & 8 & 16 & 2 & 4 & 2 \\
 \hline
 \hline
 \multicolumn{5}{|c|}{Race (FP32 Accuracy = 41.34\%)} & \multicolumn{4}{|c|}{Boolq (FP32 Accuracy = 68.32\%)} \\ 
 \hline
 \hline
 64 & 40.48 & 40.10 & 39.43 & 39.90 & 69.20 & 68.41 & 69.45 & 68.56 \\
 \hline
 32 & 39.52 & 39.52 & 40.77 & 39.62 & 68.32 & 67.43 & 68.17 & 69.30  \\
 \hline
 16 & 39.81 & 39.71 & 39.90 & 40.38 & 68.10 & 66.33 & 69.51 & 69.42  \\
 \hline
 \hline
 \multicolumn{5}{|c|}{Winogrande (FP32 Accuracy = 67.88\%)} & \multicolumn{4}{|c|}{Piqa (FP32 Accuracy = 78.78\%)} \\ 
 \hline
 \hline
 64 & 66.85 & 66.61 & 67.72 & 67.88 & 77.31 & 77.42 & 77.75 & 77.64 \\
 \hline
 32 & 67.25 & 67.72 & 67.72 & 67.00 & 77.31 & 77.04 & 77.80 & 77.37  \\
 \hline
 16 & 68.11 & 68.90 & 67.88 & 67.48 & 77.37 & 78.13 & 78.13 & 77.69  \\
 \hline
\end{tabular}
\caption{\label{tab:mmlu_abalation} Accuracy on LM evaluation harness tasks on GPT3-8B model.}
\end{table}

\begin{table} \centering
\begin{tabular}{|c||c|c|c|c||c|c|c|c|} 
\hline
 $L_b \rightarrow$& \multicolumn{4}{c||}{8} & \multicolumn{4}{c||}{8}\\
 \hline
 \backslashbox{$L_A$\kern-1em}{\kern-1em$N_c$} & 2 & 4 & 8 & 16 & 2 & 4 & 8 & 16  \\
 %$N_c \rightarrow$ & 2 & 4 & 8 & 16 & 2 & 4 & 2 \\
 \hline
 \hline
 \multicolumn{5}{|c|}{Race (FP32 Accuracy = 40.67\%)} & \multicolumn{4}{|c|}{Boolq (FP32 Accuracy = 76.54\%)} \\ 
 \hline
 \hline
 64 & 40.48 & 40.10 & 39.43 & 39.90 & 75.41 & 75.11 & 77.09 & 75.66 \\
 \hline
 32 & 39.52 & 39.52 & 40.77 & 39.62 & 76.02 & 76.02 & 75.96 & 75.35  \\
 \hline
 16 & 39.81 & 39.71 & 39.90 & 40.38 & 75.05 & 73.82 & 75.72 & 76.09  \\
 \hline
 \hline
 \multicolumn{5}{|c|}{Winogrande (FP32 Accuracy = 70.64\%)} & \multicolumn{4}{|c|}{Piqa (FP32 Accuracy = 79.16\%)} \\ 
 \hline
 \hline
 64 & 69.14 & 70.17 & 70.17 & 70.56 & 78.24 & 79.00 & 78.62 & 78.73 \\
 \hline
 32 & 70.96 & 69.69 & 71.27 & 69.30 & 78.56 & 79.49 & 79.16 & 78.89  \\
 \hline
 16 & 71.03 & 69.53 & 69.69 & 70.40 & 78.13 & 79.16 & 79.00 & 79.00  \\
 \hline
\end{tabular}
\caption{\label{tab:mmlu_abalation} Accuracy on LM evaluation harness tasks on GPT3-22B model.}
\end{table}

\begin{table} \centering
\begin{tabular}{|c||c|c|c|c||c|c|c|c|} 
\hline
 $L_b \rightarrow$& \multicolumn{4}{c||}{8} & \multicolumn{4}{c||}{8}\\
 \hline
 \backslashbox{$L_A$\kern-1em}{\kern-1em$N_c$} & 2 & 4 & 8 & 16 & 2 & 4 & 8 & 16  \\
 %$N_c \rightarrow$ & 2 & 4 & 8 & 16 & 2 & 4 & 2 \\
 \hline
 \hline
 \multicolumn{5}{|c|}{Race (FP32 Accuracy = 44.4\%)} & \multicolumn{4}{|c|}{Boolq (FP32 Accuracy = 79.29\%)} \\ 
 \hline
 \hline
 64 & 42.49 & 42.51 & 42.58 & 43.45 & 77.58 & 77.37 & 77.43 & 78.1 \\
 \hline
 32 & 43.35 & 42.49 & 43.64 & 43.73 & 77.86 & 75.32 & 77.28 & 77.86  \\
 \hline
 16 & 44.21 & 44.21 & 43.64 & 42.97 & 78.65 & 77 & 76.94 & 77.98  \\
 \hline
 \hline
 \multicolumn{5}{|c|}{Winogrande (FP32 Accuracy = 69.38\%)} & \multicolumn{4}{|c|}{Piqa (FP32 Accuracy = 78.07\%)} \\ 
 \hline
 \hline
 64 & 68.9 & 68.43 & 69.77 & 68.19 & 77.09 & 76.82 & 77.09 & 77.86 \\
 \hline
 32 & 69.38 & 68.51 & 68.82 & 68.90 & 78.07 & 76.71 & 78.07 & 77.86  \\
 \hline
 16 & 69.53 & 67.09 & 69.38 & 68.90 & 77.37 & 77.8 & 77.91 & 77.69  \\
 \hline
\end{tabular}
\caption{\label{tab:mmlu_abalation} Accuracy on LM evaluation harness tasks on Llama2-7B model.}
\end{table}

\begin{table} \centering
\begin{tabular}{|c||c|c|c|c||c|c|c|c|} 
\hline
 $L_b \rightarrow$& \multicolumn{4}{c||}{8} & \multicolumn{4}{c||}{8}\\
 \hline
 \backslashbox{$L_A$\kern-1em}{\kern-1em$N_c$} & 2 & 4 & 8 & 16 & 2 & 4 & 8 & 16  \\
 %$N_c \rightarrow$ & 2 & 4 & 8 & 16 & 2 & 4 & 2 \\
 \hline
 \hline
 \multicolumn{5}{|c|}{Race (FP32 Accuracy = 48.8\%)} & \multicolumn{4}{|c|}{Boolq (FP32 Accuracy = 85.23\%)} \\ 
 \hline
 \hline
 64 & 49.00 & 49.00 & 49.28 & 48.71 & 82.82 & 84.28 & 84.03 & 84.25 \\
 \hline
 32 & 49.57 & 48.52 & 48.33 & 49.28 & 83.85 & 84.46 & 84.31 & 84.93  \\
 \hline
 16 & 49.85 & 49.09 & 49.28 & 48.99 & 85.11 & 84.46 & 84.61 & 83.94  \\
 \hline
 \hline
 \multicolumn{5}{|c|}{Winogrande (FP32 Accuracy = 79.95\%)} & \multicolumn{4}{|c|}{Piqa (FP32 Accuracy = 81.56\%)} \\ 
 \hline
 \hline
 64 & 78.77 & 78.45 & 78.37 & 79.16 & 81.45 & 80.69 & 81.45 & 81.5 \\
 \hline
 32 & 78.45 & 79.01 & 78.69 & 80.66 & 81.56 & 80.58 & 81.18 & 81.34  \\
 \hline
 16 & 79.95 & 79.56 & 79.79 & 79.72 & 81.28 & 81.66 & 81.28 & 80.96  \\
 \hline
\end{tabular}
\caption{\label{tab:mmlu_abalation} Accuracy on LM evaluation harness tasks on Llama2-70B model.}
\end{table}

%\section{MSE Studies}
%\textcolor{red}{TODO}


\subsection{Number Formats and Quantization Method}
\label{subsec:numFormats_quantMethod}
\subsubsection{Integer Format}
An $n$-bit signed integer (INT) is typically represented with a 2s-complement format \citep{yao2022zeroquant,xiao2023smoothquant,dai2021vsq}, where the most significant bit denotes the sign.

\subsubsection{Floating Point Format}
An $n$-bit signed floating point (FP) number $x$ comprises of a 1-bit sign ($x_{\mathrm{sign}}$), $B_m$-bit mantissa ($x_{\mathrm{mant}}$) and $B_e$-bit exponent ($x_{\mathrm{exp}}$) such that $B_m+B_e=n-1$. The associated constant exponent bias ($E_{\mathrm{bias}}$) is computed as $(2^{{B_e}-1}-1)$. We denote this format as $E_{B_e}M_{B_m}$.  

\subsubsection{Quantization Scheme}
\label{subsec:quant_method}
A quantization scheme dictates how a given unquantized tensor is converted to its quantized representation. We consider FP formats for the purpose of illustration. Given an unquantized tensor $\bm{X}$ and an FP format $E_{B_e}M_{B_m}$, we first, we compute the quantization scale factor $s_X$ that maps the maximum absolute value of $\bm{X}$ to the maximum quantization level of the $E_{B_e}M_{B_m}$ format as follows:
\begin{align}
\label{eq:sf}
    s_X = \frac{\mathrm{max}(|\bm{X}|)}{\mathrm{max}(E_{B_e}M_{B_m})}
\end{align}
In the above equation, $|\cdot|$ denotes the absolute value function.

Next, we scale $\bm{X}$ by $s_X$ and quantize it to $\hat{\bm{X}}$ by rounding it to the nearest quantization level of $E_{B_e}M_{B_m}$ as:

\begin{align}
\label{eq:tensor_quant}
    \hat{\bm{X}} = \text{round-to-nearest}\left(\frac{\bm{X}}{s_X}, E_{B_e}M_{B_m}\right)
\end{align}

We perform dynamic max-scaled quantization \citep{wu2020integer}, where the scale factor $s$ for activations is dynamically computed during runtime.

\subsection{Vector Scaled Quantization}
\begin{wrapfigure}{r}{0.35\linewidth}
  \centering
  \includegraphics[width=\linewidth]{sections/figures/vsquant.jpg}
  \caption{\small Vectorwise decomposition for per-vector scaled quantization (VSQ \citep{dai2021vsq}).}
  \label{fig:vsquant}
\end{wrapfigure}
During VSQ \citep{dai2021vsq}, the operand tensors are decomposed into 1D vectors in a hardware friendly manner as shown in Figure \ref{fig:vsquant}. Since the decomposed tensors are used as operands in matrix multiplications during inference, it is beneficial to perform this decomposition along the reduction dimension of the multiplication. The vectorwise quantization is performed similar to tensorwise quantization described in Equations \ref{eq:sf} and \ref{eq:tensor_quant}, where a scale factor $s_v$ is required for each vector $\bm{v}$ that maps the maximum absolute value of that vector to the maximum quantization level. While smaller vector lengths can lead to larger accuracy gains, the associated memory and computational overheads due to the per-vector scale factors increases. To alleviate these overheads, VSQ \citep{dai2021vsq} proposed a second level quantization of the per-vector scale factors to unsigned integers, while MX \citep{rouhani2023shared} quantizes them to integer powers of 2 (denoted as $2^{INT}$).

\subsubsection{MX Format}
The MX format proposed in \citep{rouhani2023microscaling} introduces the concept of sub-block shifting. For every two scalar elements of $b$-bits each, there is a shared exponent bit. The value of this exponent bit is determined through an empirical analysis that targets minimizing quantization MSE. We note that the FP format $E_{1}M_{b}$ is strictly better than MX from an accuracy perspective since it allocates a dedicated exponent bit to each scalar as opposed to sharing it across two scalars. Therefore, we conservatively bound the accuracy of a $b+2$-bit signed MX format with that of a $E_{1}M_{b}$ format in our comparisons. For instance, we use E1M2 format as a proxy for MX4.

\begin{figure}
    \centering
    \includegraphics[width=1\linewidth]{sections//figures/BlockFormats.pdf}
    \caption{\small Comparing LO-BCQ to MX format.}
    \label{fig:block_formats}
\end{figure}

Figure \ref{fig:block_formats} compares our $4$-bit LO-BCQ block format to MX \citep{rouhani2023microscaling}. As shown, both LO-BCQ and MX decompose a given operand tensor into block arrays and each block array into blocks. Similar to MX, we find that per-block quantization ($L_b < L_A$) leads to better accuracy due to increased flexibility. While MX achieves this through per-block $1$-bit micro-scales, we associate a dedicated codebook to each block through a per-block codebook selector. Further, MX quantizes the per-block array scale-factor to E8M0 format without per-tensor scaling. In contrast during LO-BCQ, we find that per-tensor scaling combined with quantization of per-block array scale-factor to E4M3 format results in superior inference accuracy across models. 

%%%%%%%%%%%%%%%%%%%%%%%%%%%%%%%%%%%%%%%%%%%%%%%%%%%%%%%%%%%%%%%%%%%%%%%%%%%%%%%
%%%%%%%%%%%%%%%%%%%%%%%%%%%%%%%%%%%%%%%%%%%%%%%%%%%%%%%%%%%%%%%%%%%%%%%%%%%%%%%


\end{document}
