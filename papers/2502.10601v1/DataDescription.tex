\section{Data Description}
\label{sec:DataDescription}

\subsection{Synthetic \& Real-world Data}
We used synthetic flood inundation data for the entire state of Iowa, provided by the Iowa Flood Center \cite{RealTimeFloodForecastingandInformationSystemfortheStateofIowa2017}. Synthetic flood maps were derived from steady-state simulations for design flows that correspond to a range of probability of exceedance (USGS, Bulletin 17) using one-dimensional open-channel flow model from the Hydrologic Engineering Center’s River Analysis System (HEC-RAS) 5. The river channel geometric properties, such as slopes
and cross sections, were derived from airborne LiDAR based one-meter \ac{DEM} data. For more details of the flood inundation mapping procedures see \cite{Gilles2012InundationMappingInitiativesIowaFloodCenter}.
% Simulated flood maps were derived based on design flows, from one-dimensional hydraulic simulations with the Hydrologic Engineering Center's River Analysis System (HEC-RAS)\footnote{\href{https://www.hec.usace.army.mil/software/hec-ras/}{https://www.hec.usace.army.mil/software/hec-ras/}} model. 
These simulated \acp{FIM} counter the issue of securing an already-limited availability of real-world \acp{FIM}, and proves advantageous to deep learning models that leverage increased data input for the super resolution task. 

The test dataset contained a withheld sample of the synthetic \acp{FIM} but, more importantly, we prepared \ac{RW}, satellite-based data derived from Landsat-8 output. These \ac{RW} samples allow us to evaluate the generalizability of our trained downscaling model. Our targeted test regions utilizing \ac{RW} data were located both in and out of Iowa (the region used for training). This motivates our research question about how effective deep learning models can transfer across other hydroclimatological regions. Landsat-8-based \acp{FIM} were generated for two sites within Iowa: at Cedar River and Des Moines River. Outside of the training region, \acp{FIM} were prepared for the flood events occurring in Red River in the North of the USA, the Nasia river in Ghana and Western Europe. \tabref{tab:dataset_locations} contains the chosen abbreviations for each of these study regions along with the period when the flooding occurred. Figure \ref{fig:DatasetLocations} shows the chosen locations for this paper. 

\textbf{Generating \acp{WFM}:} For both \ac{SYN} and \ac{RW} data, we produce the \acp{WFM} by finding the ratio of inundated pixels in every 10x10 patch of pixels in the high resolution \ac{FIM}. For \ac{RW} data, we can obtain coarse resolution \ac{RW} \acp{WFM} from existing satellite products such as NOAA/VIIRS and the equivalent high resolution \ac{FIM} from other satellite observations such as Landsat. However, in reality the differences between the two in terms of, primarily, timing of observations and secondly the algorithms applied for water detection, introduce discrepancies between coarse and high-res scenes that would contaminate the evaluation of high-res \acp{FIM} produced by the super-resolution algorithms. To avoid this source of error and focus the evaluation only on the effectiveness of the algorithms examined, we elected to create the coarse resolution \ac{RW} \acp{WFM} by aggregating the high-res \acp{FIM} from Landsat. This means that, by construction, the number of inundated pixels in the high resolution \ac{FIM} is exactly obeyed. However this may not be reflective of reality since real world \ac{WFM} may be inundated with sensor noise, along with the errors induced by the algorithm that generate these \acp{WFM}.


\subsection{Pre-processing coarse- and high-resolution imagery}

% paragraph about how multi spectral images are converted into FIMs
\ac{RW} \acp{FIM} were generated using multi-spectral images of Landsat-8 for the aforementioned flood events. Cloud free scenes were selected and the \ac{SWI}, defined as $\mathrm{SWI} \triangleq \frac{Blue - \mathrm{NIR}} { Red+Green+Blue}$,  was used for water detection, where NIR refers to the Near Infrared band. An \ac{SWI} threshold value was selected for each flood event after quantitative comparison of the histograms between non-flooded and flooded scenes and was applied for the binary classification of pixels to flooded and non-flooded. The resulting \acp{FIM} were also visually compared against visible Landsat-8 images for consistency. The coarse resolution \acp{WFM} that provide pixel values recorded as water fractions in the range $[0, 1]$ were prepared when $30m$ high-resolution \acp{FIM} having frame sizes $100 \times 100$ were down-sampled by a scale factor of 10. Resultantly, coarse resolution \acp{FIM} were $300m$/pixel, $10 \times 10$ images. %Low resolution \acp{FIM} were down-sampled a scale factor of 10 from 30-meter, high-resolution \acp{FIM} having frame size 100 x 100 (i.e., Coarse Res \acp{FIM} were ~300-meter, 10 x 10 images). 
The high resolution \acp{FIM} reported binary pixel values of 0 or 1 indicating no-flood and flood cells, respectively. 
% The resolution and frame extents of the coarse- and high-resolution \acp{FIM} were maintained irrespective of the data being synthetic or real-world.

% \subsection{Hydroclimatological similarity}
% According to the K\"{o}ppen climate classification \cite{Chen2013GeoMorphologicalSimilarity}, Iowa and Europe (the Meuse river, in specific) lie within mild temperate zones experiencing hot and warm summers, respectively. The Red River of the North shares a similar fully humid environment (\textit{i.e.}, there is generally no major observed difference in precipitation from one season to the next) as the preceding two regions, but with a snow (or continental) climate. Ghana on the other hand has a tropical (dry and hot) climate with its northern region experiencing its monsoon from as early as April that bring severe thunderstorms. In the case of Red River and Ghana, we speculate that they include drivers to flood inundation that are different from Iowa and Western Europe. As such, in this work, we treat the two regions as hydroclimatologically dissimilar to Iowa.

\begin{table*}
\caption{Chosen regions of this study\label{tab:dataset_locations}. We group the datasets into Iowa and External datasets. }
\centering
\begin{tabular}{lllll}
\hline
\textbf{Training FIMs  {[}Location{]}} & \textbf{Validation/Test FIMs  {[}Location{]}}            & \textbf{Time period of flood.} & \textbf{Abbrev.} & \textbf{Samples} \\ \hline
\multirow{3}{*}{{[}Iowa{]}} & Synthetic      {[}Iowa{]}                                 & N/A  & SYN-IA  & 2,610\\ %\cline{2-4} 
                                       & Landsat-8     {[}Cedar River, Iowa{]}& Sep 2016 & RW-IA(CR)   &  583\\ 
                                       & Landsat-8     {[}Des Moines River, Iowa{]} & March 2019 & RW-IA(DM)   &  531\\\cline{1-5} 
\multirow{3}{*}{{[}External{]}}
                                       & Landsat-8      {[}Europe{]}                            &July 2021   & RW-EU   & 311\\ %\cline{2-4} 
                                       & Landsat-8      {[}Red River, North America{]}                               & April 2020 &RW-RR  & 582 \\ 
                                       & Landsat-8      {[}Nasia River, Ghana{]}                               & Sep 2007 & RW-GH  & 31 \\ \hline

\end{tabular}

\end{table*}

\begin{figure*}[ht!]
\includegraphics[width=\textwidth, angle=0]{figures/Dataset_locations.jpg}
\caption{ Chosen regions for this study. Iowa is the region where the model was trained over, whereas all the other regions comprise of climatically similar (Europe) and dissimilar (Ghana, Red River) regions. \label{fig:DatasetLocations}}
\end{figure*}

