\section{Introduction}
\label{sec:Introduction}

% From NASA proposal
% Present trends in climate and land use change clearly point to an ever increasing flood risk that can lead to severe riverine and coastal flooding across the globe. Leveraging remote sensing imagery is key to estimating and predicting future flood inundation extents. Currently, though, long-term records of flood inundation observations from publicly-accessible imaging products are at a spatio-temporal resolution, whose coarsity significantly curtails their potential for high-accuracy predictions.

\IEEEPARstart{C}{limate} change has exposed an increasing population to flood risks, especially over the last decade\cite{Tellman2021SatelliteFloods}. 
Floods, even with low return periods, tend to cause damage that may take years to recover and these efforts tend to disfavor low-income, racial and ethnic minorities due to a difficulty in accessing federal resources as well as time taken for congressional appropriation \cite{Wilson2021}. With an average recurrence interval of at least once in 200 years, the Bellavue, TN flood event in 2021 tallied to over 100 million USD in costs and claimed at least 20 lives \cite{StormEventDatabaseEntry}. Similarly, with a return period of once in 400 years, large-scale flooding across Europe in 2021 added to the list of record-breaking floods in the region claiming approximately 240 lives and over 25 billion USD in damages\cite{ActurialPostEuropeanFloodDamageAssesmentArticle}. 


Studying the dynamics of flood inundation can better equip stakeholders to mitigate the damage caused by such floods \cite{Romero2015RemoteSensingFloodInundation}. While \acp{WFM} -- which indicate the fraction of flood inundated pixels in the representative region -- are obtainable on a daily basis, their resolution may prove too coarse to be useful for tasks such as the study of inundation dynamics. On the other hand, fine-resolution \acp{FIM}, while useful for such tasks, are not available on a daily basis. Hence, the need arises to have more frequent, high-resolution \acp{FIM} which are binary images that indicate whether a specific location is inundated with water. This will aid in the study of inundation dynamics whose use-cases range from improved hydrological models, pre-flood mitigation strategies, underwriting flood insurance, property evaluation, and formulating evacuation plans to name a few. 


Recent progress in satellite-based remote-sensing algorithms have provided stakeholders with flood related observations at global scales with existing products such as the NASA/NOAA Visible Infrared Imaging Radiometer Suite (VIIRS)\footnote{\href{https://www.nesdis.noaa.gov/current-satellite-missions/currently-flying/joint-polar-satellite-system/visible-infrared-imaging}{https://www.nesdis.noaa.gov/current-satellite-missions/currently-flying/joint-polar-satellite-system/visible-infrared-imaging}}, NASA Moderate Resolution Imaging Spectroradiometer(MODIS)\footnote{\url{https://modis.gsfc.nasa.gov}} and NASA/USGS Landsat\footnote{\href{https://landsat.gsfc.nasa.gov/}{https://landsat.gsfc.nasa.gov/}}. However, the coarse spatial resolution of MODIS (250m) and the long revisit times of  Landsat ($\sim$16 days) hinder the analysis of the spatiotemporal dynamics of flood hazard at a range of spatial scales (from small creeks to big rivers) and terrains (from natural floodplains to urban settlements). \acp{WFM}, are available via existing satellite products (NOAA VIIRS and NASA MODIS) at a daily temporal frequency and at the cost of reduced image resolution. Naturally, using these \acp{WFM} to produce high quality \acp{FIM} at higher spatio-temporal resolution can prove advantageous to studying flood dynamics at a finer scales.


% aforementioned applications.
In this work, we investigate the task of downscaling low-resolution \acp{WFM} (300m) to high-resolution \acp{FIM} (30m) with computational approaches inspired by the successes of deep learning models in the field of super-resolution imaging \cite{Wang2021DeepSurvey}. We explore the utility of three state-of-the-art deep learning models --namely \ac{RCAN}, \ac{RDN} and \ac{ESRT}. The circumstances of our problem setting introduce a training data scarcity problem due to the following reasons: (i) we can record high resolution ground truth \acp{FIM} from Landsat only once every 16 days, (ii) the probability of observing a flood event during these satellite visits further reduces our opportunities to collect data and (iii) flood events are relatively rare. These constraints makes it extremely hard to compile high quality datasets for a region of interest, especially for the data-driven models we seek to train. To alleviate this problem, we opted to use \ac{SYN} data generated by physics-based simulations wherein the resolution can be controlled, admittedly at a significantly higher computational cost. With this solution, we postulate that \ac{SYN} data can function as a viable proxy for \ac{RW} data.

We conducted experiments on four regions -- namely, Iowa, Western Europe, Red-River and Ghana -- and show that:

\begin{itemize}
    \item \ac{SYN} data are a viable proxy to existing scarce \ac{RW} flood inundation data.
    \item Among the models we've trained using the \ac{SYN} Iowa data, there are performance benefits, when evaluated on \ac{RW} Iowa data.
    \item Our trained ML models are transferable with zero-shot performance benefits when applied to hydroclimatologically similar regions (according the K\"{o}ppen classification scheme \cite{Chen2013GeoMorphologicalSimilarity}) such as Western Europe and fails to show significant benefit when evaluated in hydroclimatologically dissimilar regions such as Ghana and Red-River. 
\end{itemize}

The rest of the paper is organized as follows. Section \ref{sec:RelatedWork} details some of the existing works and their limitations as we pave the path towards our proposed models in Section \ref{sec:Methodology}. Section \ref{sec:DataDescription} describes the datasets and pre-processing steps that we employed for this study. We describe our experimental setting in Section \ref{sec:expriments}, followed by a discussion of the results in Section \ref{sec:Results}. The GitHub repository containing the code and data has also been published\footnote{\url{https://github.com/aaravamudan2014/SIDDIS}}. 

%%%%%%%%%%%%%%%%%%%%%%%%%%%%%% Unused text %%%%%%%%%%%%%%%%%%%%%%%%%%%%%%%%%%%


% The cascading effects of these natural hazards make it one of the most daunting to overcome and justifies the need to update flood risk management systems across the globe. Our attempts to produce high quality datasets for large scale floods are facilitated by advances in machine learning and remote sensing techniques. 