\section{Conclusion}
In this work we study a 10$\times$ data-driven super-resolution framework for converting \acp{WFM} into high resolution \acp{FIM}. High resolution \acp{FIM} help in the study of flood inundation dynamics for a variety of applications, including real-time
emergency response. We propose three candidate ML models to produce \acp{FIM}. 
We also embellished the classical binary cross entropy loss with a soft constraint to enforce a loose satisfaction of the fractions in the \ac{WFM}. 
To circumvent data scarcity, we train our proposed ML models using HEC-RAS simulations over the region of Iowa as a stand-in for real world data. To determine the efficacy of our proposed downscaling models, we evaluate the model over five regions, three with hydroclimatical similarity to the training dataset -- Des Moines in Iowa, Cedar River in Iowa and the Meuse river in Western Europe -- and two dissimilar regions -- Red River of the north and Nasia river in Ghana. Our results indicate that, for geomorphological and hydroclimatological similar regions, a model trained on synthetic data yields benefits over traditional interpolation techniques for downscaling. This suggests that such synthetic data can act as a stand-in for training such data-intensive ML models. 
When extending these models to other regions, we notice that the benefits of synthetic data are less evident. It appears that training separate models per topographically similar regions may be the only recourse. 
% Moreover, this motivates the incorporation, in a meaningful way, of other static data sources that may better guide the data-driven models to help generalize across topographically dissimilar regions. 

Note that our study focuses solely on riverine flooding and does not address pluvial or coastal  inundation, this is left to future work. Additionally, we expect that a meaningful incorporation of topographic features to the deep learning architecture such as \acp{DEM} -- as done in \cite{Li2022VIIRSDownscaling} -- can help improve the downscaling performance. While this work was evaluated on data from Landsat, this methodology can easily be extended to other satellite products. The combination of the proposed method applied to coarse resolution optical data available daily with high-res SAR-based \acp{FIM} from Sentinel observations offers a unique opportunity for global scale flood inundation monitoring at high-resolution.
