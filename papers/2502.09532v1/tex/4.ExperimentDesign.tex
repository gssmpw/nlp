\section{Study Design}
Our study comprises two pre-registered experiments and received approval from our institutional ethics board. First, we commissioned several ads for the World Wildlife Fund (WWF) charity from Prolific workers in a writing-related profession.\footnote{We restricted participation in our experiment to workers in writing-related professions by using existing platform-related filters.} We varied the availability of the AI writing assistant, and the order in which bilingual writers were exposed to the two different languages English and Spanish. The second experiment then tests the average persuasiveness of each treatment's advertisements, and additionally some LLM-generated ads and the baseline WWF mission statement, in a charitable giving task where participants split an endowment between themselves and the WWF.


\subsection{Experiment 1: Persuasive Writing Task}
 In Experiment 1, participants write persuasive advertisements for the World Wildlife Fund (WWF). The selection of this charity was driven by two key considerations. One, choosing a politically neutral charity should minimize potential confounding effects related to political preferences and Social Identity Theory \cite{ashmore2004organizing}. This is particularly important in light of the multilingual and -cultural nature of the work. Two, WWF is a globally active charity that is recognized across countries, reducing the effect of spatial distance between donors and the charity's location \cite{trope2010construal,Zhang2024-sv}. \revision{We consider the following four treatments in this controlled experiment:}


\begin{enumerate}
    \item \textbf{LLM-Assisted: ENG\_ESP}: Bilingual participants first write an English and then a Spanish advertisement using our LLM assistant. To control for language proficiency, we recruited 8 native English speakers and 8 native Spanish speakers.     
    \item \textbf{LLM-Assisted: ESP\_ENG}: Bilingual participants first write a Spanish and then an English advertisement using our LLM assistant. To control for language proficiency once again, we recruited 8 native English speakers and 8 native Spanish speakers.
\end{enumerate}

\revision{The two treatments \textbf{ENG\_ESP} and \textbf{ESP\_ENG} manipulate the order in which participants are exposed to the benchmark language English and the lower-resource language Spanish. We thereby generate causal data about the effect of exposure to between-language performance disparities on LLM utilization (RQ1).}


\begin{enumerate}
    \item [(3)] \textbf{ENG\_No\_LLM}: 16 Native English speakers write an English advertisement without the LLM assistant.
    \revision{\item [(4)] \textbf{ESP\_No\_LLM}: 16 Native Spanish speakers write a Spanish advertisement without the LLM assistant.}
\end{enumerate}

\revision{The two treatments \textbf{ENG\_No\_LLM} and \textbf{ESP\_No\_LLM} serve as  control conditions with respect to \textbf{ENG\_ESP} and \textbf{ESP\_ENG}, allowing us to compare the persuasiveness of human-LLM teams with sole human writers (RQ3) and thus potentially quantify the utility of co-writing in the context of charity advertisements.}


We rely on English as the highest resource language and benchmark for this study. Here, the performance of the writing assistant should be at its \revision{highest}. For Spanish, \revision{as discussed in Section 2.1}, we expect the LLM to generate text of comparatively lower, albeit of good quality. \revision{We verify these expectations by benchmarking LLaMA 3.1's performance in English and Spanish across three datasets: the Multi-IF benchmark designed to assess proficiency in following multilingual instructions (similar to ABScribe's AI drafter feature) \cite{he2024multi}, PAWS-X for paraphrasing performance (similar to ABScribe's AI modifier feature) \cite{yang2019paws}, and a recent multilingual data set for persuasion detection derived from persuasive video game dialogue \cite{poyhonen2022multilingual}. Results confirm that LLaMA 3.1 in English substantially outperforms LLaMA 3.1 in Spanish when following instructions and generating persuasive text (see Table \ref{tab:benchmark_results} in the appendix). For paraphrasing, they are on par. Although these benchmark naturally only capture a small part of the multilingual difference between English and Spanish, they support our conjunction that the experiment will expose writers to LLMs of varying output quality. Beyond these performance differences, relying on Spanish, rather than other lower resource languages which may provide more salient differences to English, provides us with two very practical benefits.} One, because Spanish is a widely used language, our setup guarantees that our results can generalize to many real-world contexts. Two, it allows us to recruit bilingual writers from Prolific, which is highly challenging for most other languages. 




\subsubsection{Procedure} On entering our experiment and providing their informed consent, participants first read through the instructions and then proceeded to a sandbox tutorial that allowed them to familiarize themselves with the LLM assistant. In the instructions, participants learned that they were being asked to write persuasive advertisements with at least 70 words for the WWF charity. They were informed that the more persuasive their ads were, the more money they could earn, and that the top 20\% most persuasive of ads would receive an additional \pounds4 bonus, the most persuasive 10\% would receive \pounds6, and the most persuasive 1\% would receive \pounds10. Participants were also endowed with some basic information about the charity. The subsequent tutorial comprised two stages. First, participants saw short instructional GIFs designed to communicate the basic functions of the LLM assistant, including text generation and recipes. They then proceeded to the writing interface and saw instructions encouraging them to try out each feature. We also provided them with a checklist, showing which features they successfully tried out. Participants were not presented with the tutorial in the \textbf{No\_LLM} treatment. Then, participants saw the WWF's mission statement and answered two related comprehension questions. 
This was done to ensure that participants paid attention to the mission statement of WWF before writing the advertisement. Depending on the treatment, they either first completed the English or the Spanish ad. In the \textbf{No\_LLM} treatment, subjects only wrote an English ad and immediately proceeded to a post-experimental questionnaire. In the \textbf{ENG\_ESP} and \textbf{ESP\_ENG}, they completed a second tutorial, this time for the other language, then proceeded to the second writing task and finally ended the experiment with the post-experimental questionnaire after which they were automatically redirected to Prolific. A full overview of the experiment flow is depicted in Figure~\ref{fig:task_sequence}

The questionnaire uses a 5-point Likert scale to capture ownership \cite{Wasi2024-uv}, attitudes towards benefits of co-writing, and perceived capabilities of the LLM\cite{Lee2022-dr}. Additionally, we captured individual perceptions of usefulness for each writing-assistant feature using a continuous scale from 1 -- 100 \cite{Ethayarajh2022-gj}. In the \textbf{No\_LLM} treatment, subjects only completed the questions about ownership.


\begin{figure}[!ht]
\centering
\includegraphics[width=0.9\linewidth]{fig/task_sequence.pdf}
\caption{Experiment Workflow for LLM-Assisted Writing in ENG (L1) - ESP (L2) and ESP (L1) - ENG (L2) Conditions. Task Sequence L1 involves completing all subtasks in the first language (L1): (L1.a) GIF-based instructions introducing the tool’s features; (L1.b) Interaction with the writing environment, making use of the tool’s features; (L1.c) A reading comprehension task focused on WWF’s mission and vision; (L1.d) Main writing task in L1; (L1.e) Post-task survey on the writing task in L1.d. Task Sequence L2 begins after Step L1.e, repeating the same subtasks (L2.a $\rightarrow$ L2.b $\rightarrow$ L2.c $\rightarrow$ L2.d $\rightarrow$ L2.e) in the second language (L2). In the No\_LLM condition, participants only completed a single task sequence in English (L1).}
\label{fig:task_sequence}
\end{figure}





\subsubsection{Participants}
We recruited a total of 48 participants from Prolific, equally divided across the three conditions. Participants have a minimum approval rating of 90 and work in a writing-related profession.\footnote{The writing-related professions as listed on Prolific were --- Teacher, Journalist, Copywriter/marketing/communications role, Creative Writing role, Translator or language/cultural expert.} For the English-only task, only native English speakers were included. In bilingual tasks, participants were distributed equally between those whose first language was either English or Spanish, with proficiency in the other language.
%
The mean age of participants in our experiment was $M = 34.56$, with 58\% identifying as female, 40\% as male, and 2\% as non-binary. Participants received a base payment of £5.00 for the \textbf{ESP\_ENG} and \textbf{ENG\_ESP} treatments, and £3.00 for the \textbf{No\_LLM} treatment as per the estimated task completion times. This amounted to an equal hourly rate across all three treatments.  

\subsubsection{Main Measures}


We analyze users' revealed utility of the writing assistant by looking at two main factors: 1) A preference score based on the number of times a feature was used, and 2) Using the weighted average similarity between AI-generated content and the final submitted text. 

\paragraph{\textbf{Preference Score (PS)}}
We mainly focus on the assistant's \texttt{AI drafter} feature, which freely generates text based on the participant's prompt and therefore represents the standard use-case of LLMs in writing. In contrast to the recipes, it does not re-write an existing piece of writing but generates content from scratch. Therefore, we expect \texttt{AI drafter} to be responsible for the majority of user-generated content. It is also the only feature that is not endogenously influenced by our prompt choices (see above) and gives full autonomy to the writer. Beyond that, we also explore other features holistically, as described below.

For each feature $f$ in each treatment group $g$,  the revealed utility $PS_{f,g}$ is calculated as the proportion of times feature $f$ was used relative to the total feature usage by the group:

\[
\text{PS}_{f,g} = \frac{u_{f,g}}{\sum_{j=1}^{m} u_{j,g}},
\]

where:
\begin{itemize}
    \item $u_{f,g}$ is the count of times feature $f$ was used by task group $g$,
    \item $\sum_{j=1}^{m} u_{j,g}$ is the total feature usage in task group $g$.
\end{itemize}

\paragraph{\textbf{Weighted Average Similarity}}
We calculate the similarity between AI-generated content and the final user-submitted text using three embedding models - 1) sentence-transformers/paraphrase-multilingual-MiniLM-L12-v2 \cite{reimers2019sentencebertsentenceembeddingsusing} 2) nomic-embed-text \cite{nussbaum2024nomic} 3) mxbai-embed-large \cite{li2023angle}. In this process, we account for the varying lengths of AI-generated content. Some AI features generate larger blocks of text (AI Drafter), while others may modify (AI Modifier) or continue  existing sentences with just a few words (Create Continuation). To reflect this, we assign weights to each AI-generated segment proportional to its length relative to the final document. This approach ensures that longer AI contributions, which have a greater impact on the document's overall structure and meaning, receive more influence in the similarity score.
\[
\text{Weighted Average Similarity} = \frac{\sum_{i=1}^{n} w_i \times \text{cosine\_similarity}(\mathbf{v}_i, \mathbf{v})}{\sum_{i=1}^{n} w_i}
\]

Where:
\begin{itemize}
    \item $w_i$ is the weight for the \( i \)-th AI-generated segment, calculated based on its length relative to the total length of the final document.
    \item \( \mathbf{v}_i \) represents the embedding vector of the \( i \)-th AI-generated segment, and \( \mathbf{v} \) represents the embedding vector of the final document.
\end{itemize}

\paragraph{Weight Calculation}
The weight \( w_i \) for each AI-generated segment is calculated as:

\[
w_i = \frac{\text{length\ of\ AI\ response}_i}{\text{length\ of\ final\ document}}.
\]


\subsection{Experiment 2: Persuasiveness and Charitable Giving}
Experiment 2 uses a charitable giving game to evaluate the persuasiveness of different advertisements in the context of altruistic social preferences. This serves three main purposes:
\begin{itemize}
    \item We aim to quantify the effect of LLM usage in Experiment 1, including potential violations of choice independence, on social preference persuasiveness \revision{(RQ2)}.
    \item The experiment compares the effectiveness of different writing sources for the efficacy of charity ads, allowing for cost-benefit inferences about the added value of costly human workers \revision{(RQ3)}.
    \item By eliciting participants' beliefs about the source of their advertisements (human or AI), we gauge whether humans can identify LLM-generated advertisements, and how these beliefs affect subsequent donation behaviour \revision{(RQ4)}.
\end{itemize}
 Beyond these purposes, differentiating between English and Spanish native speakers allows us to capture potential cultural differences in the context of LLMs and charitable giving. In this experiment, participants are randomly assigned to one of eight treatments, each varying the source and language of the considered advertisements (shown in Table \ref{tab:treatment-groups}).

\begin{table*}[htbp]
\centering
\caption{Overview of the nine treatments in Experiment 2.}
\label{tab:treatment-groups}
\footnotesize
\begin{tabular}{@{}p{1.8cm}cp{1cm}p{6.5cm}@{}}
\toprule
\textbf{Treatment} & \textbf{N} & \textbf{\# Ads} & \textbf{Description} \\
\midrule
\textbf{Control}   & 80 & 1  & Official WWF mission statement in English \\
\textbf{ENG\_1}    & 80 & 16 & English ads generated in ENG\_1 - ESP\_2 \\
\textbf{ENG\_2}    & 80 & 16 & English ads generated in ESP\_1 - ENG\_2 \\
\textbf{ENG\_No\_LLM}  & 80 & 16 & English ads generated without LLM assistance \\
\textbf{ENG\_LLM}  & 80 & 16 & English ads generated by Lama 3.1, temperature randomly sampled from a uniform distribution \\
\textbf{ESP\_1}    & 80 & 16 & Spanish ads generated in ESP\_1 - ENG\_2 \\
\textbf{ESP\_2}    & 80 & 16 & Spanish ads generated in ENG\_1 - ESP\_2 \\
\textbf{ESP\_No\_LLM}  & 80 & 16 & Spanish ads generated without LLM assistance \\
\textbf{ESP\_LLM}  & 80 & 16 & Spanish ads generated by Lama 3.1, temperature randomly sampled from a uniform distribution \\
\bottomrule
\end{tabular}
\end{table*}


\subsubsection{Procedure} Participants first read through the instructions. They learned that they would receive an endowment of \pounds1.5, and were free to split the \pounds1.5 between themselves and the WWF charity. After proceeding to the decision screen, they read an advertisement about the charity. The specific text depended on the treatment, as explained above. Then, subjects were asked to choose their preferred donation amount and complete the task by answering a series of questions about the advertisement. A summary of these questions and their corresponding question types are provided in Table~\ref{tab:questionnaire} in the Appendix.

Note that the questionnaire also included an attention check question - asking participants to specify the charity they were donating to. This was implemented to ensure that donors read the donation text and also allowed us to filter out responses from participants who may have rushed through the task without proper engagement.



\subsubsection{Participants}
Of the \revision{760 participants recruited from Prolific for this task, 43 failed the attention check and 3 revoked their consent, leaving a final sample of 720 participants with a minimum approval rating of 90. Participants for the English ads were native English speakers, those for the Spanish ads were native Spanish speakers. All participants reside in the US. Participants received a base payment of £1.00, plus additional bonuses based on their donation behaviour. The average participant age was $M = 36.28$, with 58\% identifying as female, 41\% as male, and 1\% choosing not to disclose their gender. At the end of the experiment, 568 participants had decided to donate on average \pounds0.72, resulting in total donations of \pounds518 (ca. \$660) to the WWF.}


\begin{figure}[!ht]
    \centering
    \includegraphics[width=0.45\linewidth]{fig/donation_screen_0.pdf}
    \includegraphics[width=0.45\linewidth]{fig/donation_screen_1.pdf}
    \caption{The donation survey screen. \textbf{Left:} Participants first read the donation message and choose their desired donation amount. \textbf{Right:} After selecting the donation amount, participants proceed to answer the survey. The persuasive text remains visible throughout the process.}
    \label{fig:donation_screen}
\end{figure}
