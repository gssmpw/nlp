\section{Conclusions}
This study explores how multilingual LLMs affect user behaviour and advertisement persuasiveness in a co-writing task, with a focus on English and Spanish. Our findings reveal a clear ``rationality gap'', where prior exposure to a relatively lower-resource language (Spanish) diminishes subsequent reliance on an LLM writing assistant in a higher-resource language (English). We then evaluate the consequence of these patterns for the persuasiveness of generated ads via a charitable donation task. Here, peoples' donation behaviour appears largely unaffected by the specific advertisement, alleviating the potential negative consequences of under-utilization due to irrational behavioural adaptions. However, donations are strongly related to participants' beliefs about the source of the advertisement. Those who think that it was written by an AI are (1) significantly less likely to donate, and (2) donate less. Our results have strong implications for a number of important stakeholders, including companies deploying global multilingual AI assistants, the dissemination of LLMs across linguistically different parts of the world, marketing practitioners, and societal stakeholders concerned about inequality. Heterogeneity in AI performance across tasks can lead to substantial behavioural second-order effects that asymmetrically affect appropriate reliance and utilization.  

