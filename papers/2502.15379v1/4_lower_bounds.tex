\section{Lower Bounds}
\label{sec:lower-bound}
In this section, we prove a lower bound that accounts for the multiplicative approximation factor $\approxerror$, as well as the arboricity $\arboricity$ of the graph by extending the ideas proposed in~\citep{DBLP:conf/approx/AssadiN22}. Our lower bound 
almost matches our proposed upper bound stated in Theorem~\ref{Theorem: Final Upper Bound}. We first state the problem known as the {\sf Popcount Thresholding Problem} (referred to as \ptp{}) in the query framework:
\begin{definition}[$\defptp$]
    Given a string $\alicestring \in \sbrac{0,1}^\stringlength$, the problem $\defptp$ is to  distinguish whether $\alicestring$ is  generated from i.i.d. samples from $\ptpdone$ or $\ptpdtwo$ defined as follows:
    \begin{itemize}
        \item \textbf{$\ptpdone$ :} For all $i \in \tbrac{\stringlength}$, $\alicestring_i$ is set to $1$ with probability $\fbrac{1-2\ptpsep}\frac{\ptpprob}{\stringlength}$, and set to $0$ otherwise.
        \item \textbf{$\ptpdtwo$ :} For all $i \in \tbrac{\stringlength}$, $\alicestring_i$ is set to $1$ with probability $\fbrac{1+2\ptpsep}\frac{\ptpprob}{\stringlength}$, and set to $0$ otherwise.
    \end{itemize}
    The problem is to decide if $\alicestring$ is generated from $\ptpdone$ or $\ptpdtwo$ by querying $\alicestring$ at any of its $\stringlength$ bits.
\end{definition}
We state the following lemma~\citep{DBLP:conf/approx/AssadiN22} quantifying the query complexity of $\defptp$. 
% \gopi{\st{I guess $\defptp$ is a problem in query framework.}}\debarshi{Done!}
\begin{lemma}~\cite{DBLP:conf/approx/AssadiN22}\label{Lemma: PTP Query Lower Bound}
    For any $\ptpsep \in (0,1/4)$, $\confidence \in (0,1/100)$, and integers $\stringlength \geq 1$, $\log(1/\confidence)\cdot12/\ptpsep^2\leq \ptpprob \leq \stringlength/6$, $\rqcomplexity{\confidence}{\defptp} \geq \frac{\stringlength\log\fbrac{1/4\confidence}}{24\ptpsep^2\ptpprob}$
  \remove{  \begin{align*}
        \rqcomplexity{\confidence}{\defptp} \geq \frac{\stringlength\log\fbrac{1/4\confidence}}{24\ptpsep^2\ptpprob}
    \end{align*}
    }
    where $\rqcomplexity{\confidence}{\defptp}$ is the randomized query complexity to decide $\defptp{}$ problem with probability at least $1 - \confidence$.
\end{lemma}

\iffalse{
\gopi{\st{May be we make the above lemma a proposition?}}\debarshi{All things like the~\cite{DBLP:journals/siamcomp/ChibaN85} result we use is in lemma.}
}\fi 

We also state the following lemma establishing bounds on the $\ell_1$-norm of $\alicestring$ \remove{depending on whether it is generated from $\ptpdone$ or $\ptpdtwo$}. 

\begin{lemma}\label{Lemma: PTP Deviation Bound}
    In $\defptp$, for any $\ptpsep \in (0,1/4)$, $\confidence \in (0,1/100)$, and integers $\stringlength \geq 1$, $\log(1/\confidence)\cdot12/\ptpsep^2\leq \ptpprob \leq \stringlength/6$,
    \begin{align*}
        \Pr\tbrac{\norm{\alicestring}_1 > \fbrac{1-\ptpsep} \cdot \ptpprob~|~\alicestring \sim \ptpdone} \leq \confidence\\
        \Pr\tbrac{\norm{\alicestring}_1 < \fbrac{1+\ptpsep} \cdot \ptpprob~|~\alicestring \sim \ptpdtwo} \leq \confidence\\
        \Pr\tbrac{\norm{\alicestring}_1 < \fbrac{1-4\ptpsep} \cdot \ptpprob} \leq \confidence
    \end{align*}
    where $\norm{\alicestring}_1$ denotes the number of $1$'s in the string $\alicestring$.
\end{lemma}

We combine two results to obtain the result. First we state a lemma due to~\citep{DBLP:conf/approx/AssadiN22}:

\begin{lemma}\label{Lemma: PTP Deviation Conditional Bound}
    In $\defptp$, for any $\ptpsep \in (0,1/4)$, $\confidence \in (0,1/100)$, and integers $\stringlength \geq 1$, $\log(1/\confidence)\cdot12/\ptpsep^2\leq \ptpprob \leq \stringlength/6$,
    \begin{align*}
        \Pr\tbrac{\norm{\alicestring}_1 > \fbrac{1-\ptpsep} \cdot \ptpprob~|~\alicestring \sim \ptpdone} \leq \confidence\\
        \Pr\tbrac{\norm{\alicestring}_1 < \fbrac{1+\ptpsep} \cdot \ptpprob~|~\alicestring \sim \ptpdtwo} \leq \confidence
    \end{align*}
    where $\norm{\alicestring}_1$ denotes the number of $1$'s in the string $\alicestring$.
\end{lemma}

We introduce another lemma to lower bound the $\ell_1$ norm of $\alicestring$ for $\ptpdone$.

\begin{lemma}\label{Lemma: PTP Deviation Lower Bound}
    In $\defptp$, for any $\ptpsep \in (0,1/4)$, $\confidence \in (0,1/100)$, and integers $\stringlength \geq 1$, $\log(1/\confidence)\cdot12/\ptpsep^2\leq \ptpprob \leq \stringlength/6$, $\Pr\tbrac{\norm{\alicestring}_1 < \fbrac{1-4\ptpsep} \cdot \ptpprob~|~\alicestring \sim \ptpdone} \leq \confidence$
    \remove{
    \begin{align*}
        \Pr\tbrac{\norm{\alicestring}_1 < \fbrac{1-4\ptpsep} \cdot \ptpprob~|~\alicestring \sim \ptpdone} \leq \confidence
    \end{align*}
    }
\end{lemma}

\begin{proof}
    By definition of $\ptpdone$ and linearity of expectations, we have:
    \begin{align*}
        \Exp{\tbrac{\norm{\alicestring}_1}} = \sum_{i \in [\stringlength]} \Exp\tbrac{\alicestring_i} = \fbrac{1-2\ptpsep}\ptpprob
    \end{align*}
    Now, given the independence of each $\alicestring_i$, we can use Chernoff bound (Lemma~\ref{Lemma: Multiplicative Chernoff Bound}) to obtain:
    \begin{align*}
        & \Pr\tbrac{\norm{\alicestring}_1 < \fbrac{1-4\ptpsep}\ptpprob} \\
        =&\Pr\tbrac{\norm{\alicestring}_1 - \Exp{\tbrac{\norm{\alicestring}_1}} > \frac{2\ptpsep}{1-2\ptpsep}\Exp{\tbrac{\norm{\alicestring}_1}}}\\
        \leq& \exp{\fbrac{-\frac{4\ptpsep^2\Exp{\tbrac{\norm{\alicestring}_1}}}{3\fbrac{1-2\ptpsep}^2}}}  & \text{(by Chernoff bound)}\\
        \leq& \exp{\fbrac{-\frac{48\ptpsep^2\log\fbrac{1/\confidence}}{3\ptpsep^2\fbrac{1-2\ptpsep}}}} &(\Exp{\tbrac{\norm{\alicestring}_1}} = \fbrac{1-2\ptpsep}\ptpprob, \ptpprob \geq \log(1/\confidence)\cdot12/\ptpsep^2)\\
        \leq& \confidence &(\ptpsep < 1/4)
    \end{align*}
\end{proof}


Combining lemmas~\ref{Lemma: PTP Deviation Conditional Bound} and~\ref{Lemma: PTP Deviation Lower Bound}, we obtain the desired result.
% \begin{theorem}
% Any algorithm that solves triangle estimation problem requires $\Omega\fbrac{\frac{\edgecount\arboricity\log\fbrac{1/\confidence}}{\approxerror^2\numtriangle}}$ queries.    
% \end{theorem}

% \begin{proof}[Proof Sketch]
%     We want to show that given a graph $\graph$, it takes $\Omega\fbrac{\frac{\edgecount\arboricity\log\fbrac{1/\confidence}}{\approxerror^2\numtriangle}}$ queries to obtain an estimate $\esttriangle$ such that $\esttriangle \in (1\pm \approxerror)\numtriangle$ with probability $1-\confidence$. For contradiction, let us assume there exists an algorithm $\trianglealgo$ that, for some arboricity $\lbarboricity$, computes an estimate $\esttriangle$ such that $\esttriangle \in (1\pm \approxerror)\numtriangle$ with probability $1-\confidence$ with $o\fbrac{\frac{\edgecount\lbarboricity\log\fbrac{1/\confidence}}{\approxerror^2\numtriangle}}$ queries. 

%     To show contradiction, we design an algorithm that solves $\defptp$ in $<\frac{\stringlength\log\fbrac{1/4\confidence}}{24\ptpsep^2\ptpprob}$ queries using $\trianglealgo$. Given an instance of $\alicestring$ generated from $\defptp$, we construct a graph $\rqgraph$ as follows:
%     \begin{itemize}
%         \item The graph $\rqgraph$ consists of $5$ sets of vertices $\vertexsetA, \vertexsetAhat, \vertexsetB, \vertexsetBhat$, and $\trianglemakerset$, with $\size{\vertexsetA} = \size{\vertexsetAhat} = \size{\vertexsetB} = \size{\vertexsetBhat} = \stringlength/\lbarboricity$, and $\size{\trianglemakerset} = \lbarboricity$. 
%         \item There exists an edge between every vertex in $\vertexsetA \cup \vertexsetB$ to every vertex in $\trianglemakerset$. Observe that this makes sure that the arboricity of $\rqgraph$ is $\lbarboricity$. We index the problem instance $\alicestring$ of length $\stringlength$ as $\tbrac{\frac{\stringlength}{\lbarboricity}}\times\tbrac{\lbarboricity}$, denoting by $\alicestring_{i,j} = \alicestring_{i\times\lbarboricity+j}$. We add an edge $\fbrac{\vertexA_i,\vertexAhat_{i+j}}$, and an edge $\fbrac{\vertexB_i,\vertexBhat_{i+j}}$ if $\alicestring_{i,j} = 0$.We add an edge $\fbrac{\vertexA_i,\vertexB_{i+j}}$, and an edge $\fbrac{\vertexAhat_i,\vertexBhat_{i+j}}$ if $\alicestring_{i,j} = 1$. 
%     \end{itemize}
%     Observe that each edge between a vertex in $\vertexsetA$ and $\vertexsetB$ adds $\lbarboricity$ triangles. Hence, the no of triangles in $\rqgraph$ is exactly $\norm{\alicestring}_1\lbarboricity$. Also note that we have added $2\stringlength$ edges between $\vertexsetA\cup\vertexsetB$ and $\trianglemakerset$, and further $2\stringlength$ edges according to the entries of $\alicestring$, 2 for each element $\alicestring_i$. Hence, we have $\lbedgecount = 4\stringlength$. Now, we run the algorithm $\trianglealgo$ on $\rqgraph$ with $\frac{\edgecount\lbarboricity\log\fbrac{1/4\confidence}}{200\approxerror^2\numtriangle} = \frac{\stringlength\log(1/4\confidence)}{50\ptpsep^2\ptpprob}$ queries. 
%     \todo{Make precise, queries implementation etc.}
%     By Lemma~\ref{Lemma: PTP Deviation Bound}, it suffices to show that we can use $\trianglealgo$ to distinguish between $\norm{\alicestring}_1 > \fbrac{1+\ptpsep}\ptpprob$ instance and $\norm{\alicestring}_1 < \fbrac{1-\ptpsep}\ptpprob$ with probability $1-\frac{\confidence}{2}$ to distinguish between $\ptpdone$ and $\ptpdtwo$ instance with probability $1 - \confidence$. In that regard, we show that the algorithm with threshold at $\fbrac{1-\ptpsep^2}\ptpprob\lbarboricity$ and $\approxerror = \ptpsep$, accepts with probability $1-\frac{\confidence}{2}$ given $\norm{\alicestring}_1 > \fbrac{1+\ptpsep}\ptpprob$, and rejects with probability $1-\frac{\confidence}{2}$ given $\norm{\alicestring}_1 < \fbrac{1-\ptpsep}\ptpprob$. We consider the two cases separately:

%     \textbf{$\norm{\alicestring}_1 > \fbrac{1+\ptpsep}\ptpprob$:} Our construction ensures that $\rqgraph$ has $\numtriangle = \norm{\alicestring}_1\lbarboricity = \fbrac{1+\ptpsep}\ptpprob\lbarboricity = \fbrac{1+\approxerror}\ptpprob\lbarboricity$. Additionally, by our assumption on $\trianglealgo$, it outputs an estimate $\emptriangle \geq (1-\approxerror)\numtriangle$ with probability $1 - \frac{\confidence}{2}$. Thus, we have $\esttriangle \geq \fbrac{1 - \approxerror^2}\ptpprob\lbarboricity$ with probability $1-\frac{\confidence}{2}$.

%     \textbf{$\norm{\alicestring}_1 < \fbrac{1-\ptpsep}\ptpprob$:} Our construction ensures that $\rqgraph$ has $\numtriangle = \norm{\alicestring}_1\lbarboricity = \fbrac{1-\ptpsep}\ptpprob\lbarboricity = \fbrac{1-\approxerror}\ptpprob\lbarboricity$. Additionally, by our assumption on $\trianglealgo$, it outputs an estimate $\emptriangle \leq (1+\approxerror)\numtriangle$ with probability $1 - \frac{\confidence}{2}$. Thus, we have $\esttriangle \leq \fbrac{1 - \approxerror^2}\ptpprob\lbarboricity$ with probability $1-\frac{\confidence}{2}$.

%     Now we state how to simulate the required queries:
%     \begin{itemize}
%         \item \textbf{\degreeq{}:} For a vertex $\vertex \in \vertexsetA\cup\vertexsetB\cup\vertexsetAhat\cup\vertexsetBhat$, return $\lbarboricity$, for a vertex $\vertex \in \trianglemakerset$, return $\frac{2\stringlength}{\lbarboricity}$.
        
%         \item \textbf{\neighbourq{}:} For a vertex $\vertex \in \vertexsetA\cup\vertexsetB\cup\vertexsetAhat\cup\vertexsetBhat$, w.l.o.g assume the $\vertex$ to be $\vertexA_i \in \vertexsetA$. Choose a $j \in \tbrac{\lbarboricity}$ uniformly at random, return $\vertexAhat_{i+j}$ if $\alicestring_{i,j} = 0$, and $\vertexB_{i+j}$ otherwise. For a vertex $\vertex \in \trianglemakerset$, return a vertex in $\vertexsetA \cup \vertexsetB$ uniformly at random.
        
%         \item \textbf{\randedgeq{}:} Sample a vertex $\vertex$ with probability $\frac{\degree{\vertex}}{4\stringlength}$. Pick a random neighbour $\altvertex = \neighbourq{\fbrac{\vertex}}$, return $\fbrac{\vertex,\altvertex}$.
        
%         \item \textbf{\edgeexistsq{}:} Given a vertex pair $\fbrac{\vertex,\altvertex}$. If the vertex $\vertex \in \vertexsetA\cup\vertexsetB$, w.l.o.g assume the $\vertex$ to be $\vertexA_i \in \vertexsetA$, if $\altvertex \in \trianglemakerset$, return $1$, else if $\altvertex = \vertexB_j \in \vertexsetB$, return $1$ if $\alicestring_{i,j-i} = 1$, else if $\altvertex = \vertexAhat_j \in \vertexsetAhat$, return $1$ if $\alicestring_{i,j-i} = 0$, else return $0$. If a vertex $\vertex \in \vertexsetAhat\cup\vertexsetBhat$, w.l.o.g assume the $\vertex$ to be $\vertexAhat_i \in \vertexsetAhat$, if $\altvertex = \vertexA_j \in \vertexsetA$, return $1$ if $\alicestring_{j,i-j} = 0$, else if $\altvertex = \vertexBhat_j \in \vertexsetBhat$, return $1$ if $\alicestring_{j,i-j} = 1$, else return $0$. If $\vertex \in \trianglemakerset$, return $1$ if $\altvertex \in \vertexsetA\cup\vertexsetB$, else return $0$.
%     \end{itemize}
% \end{proof}

\begin{theorem}[Lower Bound - Formal]\label{Theorem: Lower Bound on Triangle Counting through PTP}
Any algorithm that solves triangle estimation problem with $\approxerror \leq \frac{1}{4}$ using \degreeq{}, \neighbourq{}, \edgeexistsq{} and \randedgeq{} requires $\Omega\fbrac{\frac{\edgecount\arboricity\log\fbrac{1/\confidence}}{\approxerror^2\numtriangle}}$ queries.    
\end{theorem}

\begin{proof}
% \gopi{\st{Actually, we show the lower bound for an easier problem: given $m,\varepsilon, T$, the objective is to distinguish whether the number of triangles is at most $(1-\varepsilon)T$ or at least $(1+\varepsilon)T$?}}\debarshi{Done!}

% Actually, we show the lower bound for an easier problem: given $m,\varepsilon, T$, the objective is to distinguish whether the number of triangles is at most $(1-\varepsilon)T$ or at least $(1+\varepsilon)T$.

We want to show that given a graph $\graph$, it takes $\Omega\fbrac{\frac{\edgecount\arboricity\log\fbrac{1/\confidence}}{\approxerror^2\numtriangle}}$ queries to obtain an estimate $\esttriangle$ such that $\esttriangle \in (1\pm \approxerror)\numtriangle$ with probability at least $1-\confidence$. For contradiction, let us assume there exists an algorithm $\trianglealgo$ that, for some arboricity $\lbarboricity$, computes an estimate $\esttriangle$ such that $\esttriangle \in (1\pm \approxerror)\numtriangle$ with probability at least $1-\confidence$ using $\smallo{\frac{\edgecount\lbarboricity\log\fbrac{1/\confidence}}{\approxerror^2\numtriangle}}$ queries. 



    To show contradiction, we design an algorithm that solves $\defptp$ in $<\frac{\stringlength\log\fbrac{1/4\confidence}}{24\ptpsep^2\ptpprob}$ queries using $\trianglealgo$.\remove{\gopi{Shall we state the value of $M,k,$ and $\gamma$ here w.r.t. $m,T,\alpha^*$ and $\varepsilon$? May be $M=4m$, $T=k\alpha^*$, and $\gamma=\varepsilon$.}} Given an instance of $\alicestring$ generated from $\defptp$, we consider a graph $\rqgraph = (\vertexset_x, \edgeset_x)$ defined as: 
    \paragraph*{The vertex set $\vertexset_x$:} The vertex set $\vertexset_x$ consists of $5$ sets of disjoint and independent set of vertices $\vertexsetA, \vertexsetAhat, \vertexsetB, \vertexsetBhat$, and $\trianglemakerset$, with $\size{\vertexsetA} = \size{\vertexsetAhat} = \size{\vertexsetB} = \size{\vertexsetBhat} = \stringlength/\lbarboricity$, and $\size{\trianglemakerset} = \lbarboricity$. Vertices of $\vertexsetA$, $\vertexsetAhat$, $\vertexsetB$ and $\vertexsetBhat$ will be named as $\vertexA$, $\vertexAhat$, $\vertexB$ and $\vertexBhat$, respectively. 
    \paragraph*{The edge set $\edgeset_x$:} There exists an edge between every vertex in $\vertexsetA \cup \vertexsetB$ to every vertex in $\trianglemakerset$. Observe that this makes sure that the arboricity of $\rqgraph$ is $\lbarboricity$. If a bit of $\alicestring$ is $1$ (resp. $0$), there will be an edge from a vertex in $\vertexsetA$ (resp. $\vertexsetA$) to a vertex in $\vertexsetB$ (resp. $\vertexsetAhat$), and an edge from a vertex in $\vertexsetAhat$ (resp. $\vertexsetB$) to a vertex in $\vertexsetBhat$ (resp. $\vertexsetBhat$).
        
        \remove{There can be \blue{at most} $\left( \frac{M}{\lbarboricity} \right)^2$ edges between any pair of sets like $(\vertexsetA, \vertexsetB)$, or $(\vertexsetA, \vertexsetAhat)$, or $(\vertexsetAhat, \vertexsetBhat)$, or $(\vertexsetB, \vertexsetBhat)$. \red{For bits in $\alicestring$ to have a one-to-one correspondence to the above edges, we need $\left( \frac{M}{\lbarboricity} \right)^2 = M$, i.e., $\lbarboricity = \sqrt{\stringlength}$.}}
        
        We index the problem instance $\alicestring$ of length $\stringlength$ as $\tbrac{\frac{\stringlength}{\lbarboricity}}\times\tbrac{\lbarboricity}$, denoting  $\alicestring_{\fbrac{i-1}\times\lbarboricity+j}$ as $\alicestring_{i,j}$, where $i \in \tbrac{\frac{\stringlength}{\lbarboricity}}$ and $j \in \tbrac{\lbarboricity}$. We add an edge $\fbrac{\vertexA_i,\vertexAhat_{i+j}}$, and an edge $\fbrac{\vertexB_i,\vertexBhat_{i+j}}$ if $\alicestring_{i,j} = 0$. We add an edge $\fbrac{\vertexA_i,\vertexB_{i+j}}$, and an edge $\fbrac{\vertexAhat_i,\vertexBhat_{i+j}}$ if $\alicestring_{i,j} = 1$. All $(i+j)$ additions here are modulo $\tbrac{\frac{\stringlength}{\lbarboricity}}$.
        \remove{
    \begin{itemize}
        \item[$\vertexset_x$:] The vertex set $\vertexset_x$ consists of $5$ sets of disjoint and independent set of vertices $\vertexsetA, \vertexsetAhat, \vertexsetB, \vertexsetBhat$, and $\trianglemakerset$, with $\size{\vertexsetA} = \size{\vertexsetAhat} = \size{\vertexsetB} = \size{\vertexsetBhat} = \stringlength/\lbarboricity$, and $\size{\trianglemakerset} = \lbarboricity$. Vertices of $\vertexsetA$, $\vertexsetAhat$, $\vertexsetB$ and $\vertexsetBhat$ will be named as $\vertexA$, $\vertexAhat$, $\vertexB$ and $\vertexBhat$, respectively. 
        \item[$\edgeset_x$:] There exists an edge between every vertex in $\vertexsetA \cup \vertexsetB$ to every vertex in $\trianglemakerset$. Observe that this makes sure that the arboricity of $\rqgraph$ is $\lbarboricity$. If the bit is $1$ (\blue{respectively }$0$), \complain{there will be an edge from a vertex in $\vertexsetA$ (\blue{respectively }$\vertexsetA$) to a vertex in $\vertexsetB$ (\blue{respectively }$\vertexsetAhat$), and an edge from a vertex in $\vertexsetAhat$ (\blue{respectively }$\vertexsetB$) to a vertex in $\vertexsetBhat$ (\blue{respectively }$\vertexsetBhat$).}\blue{}
        \remove{There can be \blue{at most} $\left( \frac{M}{\lbarboricity} \right)^2$ edges between any pair of sets like $(\vertexsetA, \vertexsetB)$, or $(\vertexsetA, \vertexsetAhat)$, or $(\vertexsetAhat, \vertexsetBhat)$, or $(\vertexsetB, \vertexsetBhat)$. \red{For bits in $\alicestring$ to have a one-to-one correspondence to the above edges, we need $\left( \frac{M}{\lbarboricity} \right)^2 = M$, i.e., $\lbarboricity = \sqrt{\stringlength}$.}}
        We index the problem instance $\alicestring$ of length $\stringlength$ as $\tbrac{\frac{\stringlength}{\lbarboricity}}\times\tbrac{\lbarboricity}$, denoting  $\alicestring_{\fbrac{i-1}\times\lbarboricity+j}$ as $\alicestring_{i,j}$, where $i \in \tbrac{\frac{\stringlength}{\lbarboricity}}$ and $j \in \tbrac{\lbarboricity}$. We add an edge $\fbrac{\vertexA_i,\vertexAhat_{i+j}}$, and an edge $\fbrac{\vertexB_i,\vertexBhat_{i+j}}$ if $\alicestring_{i,j} = 0$. We add an edge $\fbrac{\vertexA_i,\vertexB_{i+j}}$, and an edge $\fbrac{\vertexAhat_i,\vertexBhat_{i+j}}$ if $\alicestring_{i,j} = 1$. All $(i+j)$ additions here are modulo $\tbrac{\frac{\stringlength}{\lbarboricity}}$.
    \end{itemize}
      }

      
    Observe that each edge between a vertex in $\vertexsetA$ and $\vertexsetB$ adds $\lbarboricity$ triangles. Hence, the number of triangles in $\rqgraph$ is exactly $\norm{\alicestring}_1\lbarboricity$. Also note that we have added $2\stringlength$ edges between $\vertexsetA\cup\vertexsetB$ and $\trianglemakerset$, and further $2\stringlength$ edges according to the entries of $\alicestring$, 2 for each bit $\alicestring_i$. Hence, we have $\lbedgecount = 4\stringlength$, and fix $\approxerror = \ptpsep$. 
    
% \red{    
%     Now, we run the algorithm $\trianglealgo$ on $\rqgraph$ with $\frac{\edgecount\log\fbrac{1/4\confidence}}{200\approxerror^2\ptpprob} = \frac{\stringlength\log(1/4\confidence)}{50\ptpsep^2\ptpprob}$ queries. Also, observe that by Lemma~\ref{Lemma: PTP Deviation Lower Bound}, we have with probability $1-\confidence$, $\numtriangle = \lbarboricity\norm{\alicestring}_1 \geq \lbarboricity(1-4\ptpsep)\ptpprob \geq \frac{\lbarboricity\ptpprob}{2}$, where the last inequality follows from the fact that $\ptpsep = \approxerror \leq \frac{1}{4}$. Hence, we have $\frac{\edgecount\log\fbrac{1/4\confidence}}{200\approxerror^2\ptpprob} \geq \frac{\edgecount\lbarboricity\log\fbrac{1/4\confidence}}{200\approxerror^2\numtriangle}$. Thus the algorithm $\trianglealgo$ is allowed to make $\smallo{\frac{\edgecount\lbarboricity\log\fbrac{1/\confidence}}{\approxerror^2\numtriangle}}$ queries under the given query bound.
%     % \gopi{While editing, we can consider shortening the paragraph a bit.}
% }

% \red{
%     % \todo{Make precise, queries implementation etc.}
%     By Lemma~\ref{Lemma: PTP Deviation Bound}, it suffices to show that we can use $\trianglealgo$ to distinguish between $\norm{\alicestring}_1 > \fbrac{1+\ptpsep}\ptpprob$ instance and $\norm{\alicestring}_1 < \fbrac{1-\ptpsep}\ptpprob$ instance with probability $1-\frac{\confidence}{2}$ to distinguish between whether $\alicestring$ is generated from $\ptpdone$ or $\ptpdtwo$ with probability $1 - \confidence$. In that regard, we show that an algorithm deciding based on the estimate $\emptriangle$ generated by algorithm $\trianglealgo$, with threshold at $\fbrac{1-\ptpsep^2}\ptpprob\lbarboricity$, accepts with probability $1-\frac{\confidence}{2}$ given $\norm{\alicestring}_1 > \fbrac{1+\ptpsep}\ptpprob$, and rejects with probability $1-\frac{\confidence}{2}$ given $\norm{\alicestring}_1 < \fbrac{1-\ptpsep}\ptpprob$. We consider the two cases separately:
% }



We now describe $\trianglealgoptp$ an algorithm that uses $\trianglealgo$ to solve $\defptp$ using $<\frac{\stringlength\log\fbrac{1/4\confidence}}{24\ptpsep^2\ptpprob}$ queries. Given a string $\alicestring$, we generate $\rqgraph$ and estimate the number of triangles $\emptriangle$ by calling $\trianglealgo$ using $\frac{\edgecount\log\fbrac{1/4\confidence}}{200\approxerror^2\ptpprob} = \frac{\stringlength\log(1/4\confidence)}{50\ptpsep^2\ptpprob}$ queries. If $\emptriangle < \fbrac{1-\ptpsep^2}\ptpprob\lbarboricity$, it outputs $\ptpdone$; otherwise, it outputs $\ptpdtwo$. Also, observe that by Lemma~\ref{Lemma: PTP Deviation Bound}, we have  $\numtriangle = \lbarboricity\norm{\alicestring}_1 \geq \lbarboricity(1-4\ptpsep)\ptpprob \geq \frac{\lbarboricity\ptpprob}{2}$ with probability at least $1-\confidence$, where the last inequality follows from the fact that $\ptpsep = \approxerror \leq \frac{1}{4}$. Hence, we have $\frac{\edgecount\log\fbrac{1/4\confidence}}{200\approxerror^2\ptpprob} \geq \frac{\edgecount\lbarboricity\log\fbrac{1/4\confidence}}{200\approxerror^2\numtriangle}$. Thus the algorithm $\trianglealgo$ is allowed to make $\smallo{\frac{\edgecount\lbarboricity\log\fbrac{1/\confidence}}{\approxerror^2\numtriangle}}$ queries under the given query bound of $\frac{\edgecount\lbarboricity\log\fbrac{1/4\confidence}}{200\approxerror^2\numtriangle}$.


By Lemma~\ref{Lemma: PTP Deviation Bound}, to distinguish between whether $\alicestring$ is generated from $\ptpdone$ or $\ptpdtwo$ with probability $1 - \confidence$, it suffices to show that we can use $\trianglealgo$ to distinguish between $\norm{\alicestring}_1 > \fbrac{1+\ptpsep}\ptpprob$ instance and $\norm{\alicestring}_1 < \fbrac{1-\ptpsep}\ptpprob$ instance with probability $1-\frac{\confidence}{2}$. We now show that $\trianglealgoptp$ outputs $\ptpdtwo$ with probability $1-\frac{\confidence}{2}$ given $\norm{\alicestring}_1 > \fbrac{1+\ptpsep}\ptpprob$, and outputs $\ptpdone$ with probability $1-\frac{\confidence}{2}$ given $\norm{\alicestring}_1 < \fbrac{1-\ptpsep}\ptpprob$. We consider the two cases separately: 

    {\bf Case I }($\norm{\alicestring}_1 > \fbrac{1+\ptpsep}\ptpprob$): Our construction ensures that the number of triangles in $\rqgraph$ is $\numtriangle = \norm{\alicestring}_1\lbarboricity > \fbrac{1+\ptpsep}\ptpprob\lbarboricity = \fbrac{1+\approxerror}\ptpprob\lbarboricity$. Additionally, by our assumption on $\trianglealgo$, it outputs an estimate $\emptriangle \geq (1-\approxerror)\numtriangle$ with probability $1 - \frac{\confidence}{2}$. Thus, we have $\esttriangle \geq \fbrac{1 - \approxerror^2}\ptpprob\lbarboricity$ with probability $1-\frac{\confidence}{2}$.

    {\bf Case II }($\norm{\alicestring}_1 < \fbrac{1-\ptpsep}\ptpprob$): Our construction ensures that the number of triangles in $\rqgraph$ is $\numtriangle = \norm{\alicestring}_1\lbarboricity < \fbrac{1-\ptpsep}\ptpprob\lbarboricity = \fbrac{1-\approxerror}\ptpprob\lbarboricity$. Additionally, by our assumption on $\trianglealgo$, it outputs an estimate $\emptriangle \leq (1+\approxerror)\numtriangle$ with probability $1 - \frac{\confidence}{2}$. Thus, we have $\esttriangle \leq \fbrac{1 - \approxerror^2}\ptpprob\lbarboricity$ with probability $1-\frac{\confidence}{2}$.

    Now we state how to simulate the required queries:
    \begin{itemize}
        \item \textbf{$\degreeq{\fbrac{\vertex}}$:} For a vertex $\vertex \in \vertexsetA\cup\vertexsetB\cup\vertexsetAhat\cup\vertexsetBhat$, return $\lbarboricity$, and for a vertex $\vertex \in \trianglemakerset$, return $\frac{2\stringlength}{\lbarboricity}$. Hence, for \degreeq{} queries, no queries are made to the string.
        
        % \item \textbf{\neighbourq{}:} For a vertex $\vertex \in \vertexsetA\cup\vertexsetB\cup\vertexsetAhat\cup\vertexsetBhat$, w.l.o.g assume the $\vertex$ to be $\vertexA_i \in \vertexsetA$. Choose a $j \in \tbrac{\lbarboricity}$ uniformly at random, return $\vertexAhat_{i+j}$ if $\alicestring_{i,j} = 0$, and $\vertexB_{i+j}$ otherwise. For a vertex $\vertex \in \trianglemakerset$, return a vertex in $\vertexsetA \cup \vertexsetB$ uniformly at random.

        \item \textbf{$\neighbourq{\fbrac{\vertex,i}}$:} For a vertex $\vertex \in \vertexsetA\cup\vertexsetB\cup\vertexsetAhat\cup\vertexsetBhat$, w.l.o.g assume $\vertex$ to be $\vertexA_j \in \vertexsetA$. Return $\vertexAhat_{i+j}$ if $\alicestring_{i,j} = 0$, and $\vertexB_{i+j}$ otherwise. For a vertex $\vertex \in \trianglemakerset$, if $i \leq \frac{\edgecount}{\lbarboricity}$, return $\vertexA_{i}$, else return $\vertexB_{i = \frac{\edgecount}{\lbarboricity}}$. Hence, for \neighbourq{} queries, a single query is made to the string.
        
        \item \textbf{\edgeexistsq$\fbrac{\altvertex,\vertex}$:} Given a vertex pair $\fbrac{\vertex,\altvertex}$, there can be 3 cases: (a) If the vertex $\vertex \in \vertexsetA\cup\vertexsetB$, w.l.o.g assume $\vertex$ to be $\vertexA_i \in \vertexsetA$, if $\altvertex \in \trianglemakerset$, return $1$, else if $\altvertex = \vertexB_j \in \vertexsetB$, return $1$ if $\alicestring_{i,j-i} = 1$, else if $\altvertex = \vertexAhat_j \in \vertexsetAhat$, return $1$ if $\alicestring_{i,j-i} = 0$, else return $0$. (b) If a vertex $\vertex \in \vertexsetAhat\cup\vertexsetBhat$, w.l.o.g assume the $\vertex$ to be $\vertexAhat_i \in \vertexsetAhat$, if $\altvertex = \vertexA_j \in \vertexsetA$, return $1$ if $\alicestring_{j,i-j} = 0$, else if $\altvertex = \vertexBhat_j \in \vertexsetBhat$, return $1$ if $\alicestring_{j,i-j} = 1$, else return $0$. If $\vertex \in \trianglemakerset$, return $1$ if $\altvertex \in \vertexsetA\cup\vertexsetB$, else return $0$. (c) If a vertex $\vertex \in \trianglemakerset$, return $1$ if $\altvertex \in \vertexsetA \cup \vertexsetB$, return $0$ otherwise. Hence, for \edgeexistsq{} queries, a single query is made to the string.

        \item \textbf{\randedgeq{}:} Sample a vertex $\vertex$ with probability $\frac{\degree{\vertex}}{4\stringlength}$. Choose $i \in \degree{\fbrac{\vertex}}$ uniformly at random, query $\altvertex = \neighbourq{\fbrac{\vertex,i}}$ and return $\fbrac{\vertex,\altvertex}$. Hence, for \randedgeq{} queries, a single query is made to the string for the $\neighbourq{}$ query.
    \end{itemize}
\end{proof}

The following corollary is a direct implication from Theorem~\ref{Theorem: Lower Bound on Triangle Counting through PTP} due to the fact that we consider a weaker model (without access to \randedgeq{} queries) compared to Theorem~\ref{Theorem: Lower Bound on Triangle Counting through PTP}.

\begin{corollary}[Lower Bound for Local Queries]\label{Theorem: Lower Bound on Triangle Counting w/o Random Edge}
Any algorithm that solves triangle estimation problem with $\approxerror \leq \frac{1}{4}$ using \degreeq{}, \neighbourq{}, and \edgeexistsq{} requires $\Omega\fbrac{\frac{\edgecount\arboricity\log\fbrac{1/\confidence}}{\approxerror^2\numtriangle}}$ queries.    
\end{corollary}