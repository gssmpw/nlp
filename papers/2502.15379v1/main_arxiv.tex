\documentclass{article}
%\documentclass[floatrow]{iacrtrans}


% if you need to pass options to natbib, use, e.g.:
%\PassOptionsToPackage{numbers, compress}{natbib}
% before loading neurips_2022


% ready for submission
%\usepackage[]{neurips_2022}

% to compile a preprint version, e.g., for submission to arXiv, add add the% [preprint] option:
\usepackage[preprint]{neurips_2022}


% to compile a camera-ready version, add the [final] option, e.g.:
% \usepackage[preprint]{arxiv}


% to avoid loading the natbib package, add option nonatbib:
%    \usepackage[nonatbib]{neurips_2022}

% \usepackage[utf8]{inputenc} % allow utf-8 input
% \usepackage[T1]{fontenc}    % use 8-bit T1 fonts
% \usepackage[pagebackref=true]{hyperref}       % hyperlinks
% \usepackage{url}            % simple URL typesetting
% \usepackage{booktabs}       % professional-quality tables
\usepackage{amsfonts}       % blackboard math symbols
% \usepackage{nicefrac}       % compact symbols for 1/2, etc.
% \usepackage{microtype}      % microtypography
% \usepackage{xcolor,wrapfig}         % colors
\usepackage{algorithm}
% \usepackage{algpseudocode}
% \usepackage[nohints]{minitoc}
\renewcommand \thepart{}
\renewcommand \partname{}
\usepackage{natbib,enumitem}
% \newif\ifneurips
% \neuripsfalse

%Input Basic Packages

% \usepackage{natbib}
\usepackage{float}

%Input Math and Algo Related packages

\usepackage{amsmath, mathtools, amsthm}
\usepackage{algorithm, times}
\usepackage[noend]{algpseudocode}
% \usepackage[noend]{algorithmic}


% \allowdisplaybreaks
%Input Packages to Handle Images and Graphs

% \usepackage{wrapfig, epsfig, graphicx, subcaption, pgf, tikz, pgfplots}

%Tikz Setup

% \usetikzlibrary{calc, shapes.geometric, arrows, automata}
% \pgfplotsset{compat=1.18}

%Include additional various packages

% \usepackage{enumerate, thmtools, soul, setspace}

%Hyperref Setup

% \usepackage{hyperref}
% \hypersetup{
%     colorlinks=true,
%     linkcolor=blue,
%     filecolor=blue,      
%     urlcolor=blue,
%     citecolor=blue,
%     pdftitle={Notes - Topic},
%     pdfpagemode=FullScreen,
%     }

% Load the parskip package with skip and indent options
% \usepackage[skip=10pt, indent=15pt]{parskip}
\setlength{\marginparwidth}{2pc}
% \usepackage[textwidth=2pc,textsize=tiny]{todonotes}


% Include Theorem related packages

\theoremstyle{plain}
\newtheorem{theorem}{Theorem}
\newtheorem{lemma}[theorem]{Lemma}
\newtheorem{corollary}[theorem]{Corollary}
\newtheorem{proposition}[theorem]{Proposition}

\theoremstyle{definition}
\newtheorem{definition}[theorem]{Definition}
\newtheorem{example}[theorem]{Example}
\newtheorem{notation}[theorem]{Notation}
\newtheorem{problem}[theorem]{Problem}
\newtheorem{assumption}{Assumption}

% \theoremstyle{remark}
% \newtheorem{remark}[theorem]{Remark}

% \newcommand\numberthis{\addtocounter{equation}{1}\tag{\theequation}}

%Redefining Command

% \renewcommand{\baselinestretch}{1.25}
% \renewcommand{\chaptermark}[1]{\markboth{#1}{}}
% \renewcommand{\sectionmark}[1]{\markright{\thesection\ #1}}

%Citations Maintenance

% \usepackage[nottoc]{tocbibind}




%Little Boxes

\usepackage[most]{tcolorbox}
\newtcolorbox{idea}[1][]
{
colbacktitle=cyan,
colback=cyan!10,
arc=1pt,
boxrule=1pt,
title=#1 % I would like to make this (one of these in general) assignment optional depending on #1, #2...
}



\newtcolorbox{update}[1][]
{
colbacktitle=gray,
colback=gray!10,
arc=1pt,
boxrule=1pt,
title=#1 % I would like to make this (one of these in general) assignment optional depending on #1, #2...
}


\newtcolorbox{question}[1][]
{
coltitle=black,
colbacktitle=yellow,
colback=yellow!10,
arc=1pt,
boxrule=1pt,
title=#1 % I would like to make this (one of these in general) assignment optional depending on #1, #2...
}

\newtcolorbox{answer}[1][]
{
coltitle=black,
colbacktitle=violet!10,
colback=violet!5,
arc=1pt,
boxrule=1pt,
title=#1 % I would like to make this (one of these in general) assignment optional depending on #1, #2...
}

% \newtcolorbox{note}[1][]
% {
% coltitle=black,
% colbacktitle=green,
% colback=green!10,
% arc=1pt,
% boxrule=1pt,
% title=#1 % I would like to make this (one of these in general) assignment optional depending on #1, #2...
% }

\newcommand{\colnote}[3]{\textcolor{#1}{{\small $\ll$\textsf{#2}$\gg$\marginpar{\tiny\bf \textcolor[rgb]{1.00,0.00,0.00}{#3}}}}}
%Bold Symbols

\newcommand{\ba}{\boldsymbol{a}}
\newcommand{\bb}{\boldsymbol{b}}
\newcommand{\bc}{\boldsymbol{c}}
\newcommand{\bd}{\boldsymbol{d}}
\newcommand{\be}{\boldsymbol{e}}

\newcommand{\bg}{\boldsymbol{g}}
\newcommand{\bh}{\boldsymbol{h}}
\newcommand{\bi}{\boldsymbol{i}}
\newcommand{\bj}{\boldsymbol{j}}
\newcommand{\bl}{\boldsymbol{l}}
\newcommand{\bm}{\boldsymbol{m}}
\newcommand{\bn}{\boldsymbol{n}}
\newcommand{\bo}{\boldsymbol{o}}
\newcommand{\bp}{\boldsymbol{p}}
\newcommand{\bq}{\boldsymbol{q}}
\newcommand{\br}{\boldsymbol{r}}
\newcommand{\bs}{\boldsymbol{s}}
\newcommand{\bt}{\boldsymbol{t}}
\newcommand{\bu}{\boldsymbol{u}}
\newcommand{\bv}{\boldsymbol{v}}
\newcommand{\bw}{\boldsymbol{w}}
\newcommand{\bx}{\boldsymbol{x}}
\newcommand{\by}{\boldsymbol{y}}
\newcommand{\bz}{\boldsymbol{z}}


\newcommand{\bA}{\boldsymbol{A}}
\newcommand{\bB}{\boldsymbol{B}}
\newcommand{\bC}{\boldsymbol{C}}
\newcommand{\bD}{\boldsymbol{D}}
\newcommand{\bE}{\boldsymbol{E}}
\newcommand{\bF}{\boldsymbol{F}}
\newcommand{\bG}{\boldsymbol{G}}
\newcommand{\bH}{\boldsymbol{H}}
\newcommand{\bI}{\boldsymbol{I}}
\newcommand{\bJ}{\boldsymbol{J}}
\newcommand{\bK}{\boldsymbol{K}}
\newcommand{\bL}{\boldsymbol{L}}
\newcommand{\bM}{\boldsymbol{M}}
\newcommand{\bN}{\boldsymbol{N}}
\newcommand{\bO}{\boldsymbol{O}}
\newcommand{\bP}{\boldsymbol{P}}
\newcommand{\bQ}{\boldsymbol{Q}}
\newcommand{\bR}{\boldsymbol{R}}
\newcommand{\bS}{\boldsymbol{S}}
\newcommand{\bT}{\boldsymbol{T}}
\newcommand{\bU}{\boldsymbol{U}}
\newcommand{\bV}{\boldsymbol{V}}
\newcommand{\bW}{\boldsymbol{W}}
\newcommand{\bX}{\boldsymbol{X}}
\newcommand{\bY}{\boldsymbol{Y}}
\newcommand{\bZ}{\boldsymbol{Z}}

\newcommand{\balpha}{\boldsymbol{\alpha}}
\newcommand{\bbeta}{\boldsymbol{\beta}}
\newcommand{\bmu}{\boldsymbol{\mu}}
\newcommand{\bpi}{\boldsymbol{\pi}}
\newcommand{\bSigma}{\boldsymbol{\Sigma}}
\newcommand{\bsigma}{\boldsymbol{\sigma}}
\newcommand{\btheta}{\boldsymbol{\theta}}
\newcommand{\bTheta}{\boldsymbol{\Theta}}

%Mathcal Commands
\newcommand{\sA}{\mathcal{A}}
\newcommand{\sB}{\mathcal{B}}
\newcommand{\sC}{\mathcal{C}}
\newcommand{\sD}{\mathcal{D}}
\newcommand{\sE}{\mathcal{E}}
\newcommand{\sF}{\mathcal{F}}
\newcommand{\sG}{\mathcal{G}}
\newcommand{\sH}{\mathcal{H}}
\newcommand{\sI}{\mathcal{I}}
\newcommand{\sJ}{\mathcal{J}}
\newcommand{\sK}{\mathcal{K}}
\newcommand{\sL}{\mathcal{L}}
\newcommand{\sM}{\mathcal{M}}
\newcommand{\sN}{\mathcal{N}}
\newcommand{\sO}{\mathcal{O}}
\newcommand{\sP}{\mathcal{P}}
\newcommand{\sQ}{\mathcal{Q}}
\newcommand{\sR}{\mathcal{R}}
\newcommand{\sS}{\mathcal{S}}
\newcommand{\sT}{\mathcal{T}}
\newcommand{\sU}{\mathcal{U}}
\newcommand{\sV}{\mathcal{V}}
\newcommand{\sW}{\mathcal{W}}
\newcommand{\sX}{\mathcal{X}}
\newcommand{\sY}{\mathcal{Y}}
\newcommand{\sZ}{\mathcal{Z}}

%Further Specific Fonts (Miscellaneous)

\newcommand{\bhatY}{\boldsymbol{\hat{Y}}}
\newcommand{\bbary}{\boldsymbol{\bar{y}}}
\newcommand{\bhatX}{\boldsymbol{\hat{X}}}
\newcommand{\bbarx}{\boldsymbol{\bar{x}}}
\newcommand{\bbarZ}{\boldsymbol{\bar{Z}}}
\newcommand{\bbarz}{\boldsymbol{\bar{z}}}
\newcommand{\barz}{\bar{z}}
\newcommand{\bbarS}{\boldsymbol{\bar{S}}}


\newcommand{\bzero}{\boldsymbol{0}}
\newcommand{\bbartheta}{\boldsymbol{\bar{\theta}}}

\newcommand{\bfB}{\mathbf{B}}
\newcommand{\sbarZ}{\bar{\mathcal{Z}}}
\newcommand{\fbag}{\bold{F}}

%
\setlength\unitlength{1mm}
\newcommand{\twodots}{\mathinner {\ldotp \ldotp}}
% bb font symbols
\newcommand{\Rho}{\mathrm{P}}
\newcommand{\Tau}{\mathrm{T}}

\newfont{\bbb}{msbm10 scaled 700}
\newcommand{\CCC}{\mbox{\bbb C}}

\newfont{\bb}{msbm10 scaled 1100}
\newcommand{\CC}{\mbox{\bb C}}
\newcommand{\PP}{\mbox{\bb P}}
\newcommand{\RR}{\mbox{\bb R}}
\newcommand{\QQ}{\mbox{\bb Q}}
\newcommand{\ZZ}{\mbox{\bb Z}}
\newcommand{\FF}{\mbox{\bb F}}
\newcommand{\GG}{\mbox{\bb G}}
\newcommand{\EE}{\mbox{\bb E}}
\newcommand{\NN}{\mbox{\bb N}}
\newcommand{\KK}{\mbox{\bb K}}
\newcommand{\HH}{\mbox{\bb H}}
\newcommand{\SSS}{\mbox{\bb S}}
\newcommand{\UU}{\mbox{\bb U}}
\newcommand{\VV}{\mbox{\bb V}}


\newcommand{\yy}{\mathbbm{y}}
\newcommand{\xx}{\mathbbm{x}}
\newcommand{\zz}{\mathbbm{z}}
\newcommand{\sss}{\mathbbm{s}}
\newcommand{\rr}{\mathbbm{r}}
\newcommand{\pp}{\mathbbm{p}}
\newcommand{\qq}{\mathbbm{q}}
\newcommand{\ww}{\mathbbm{w}}
\newcommand{\hh}{\mathbbm{h}}
\newcommand{\vvv}{\mathbbm{v}}

% Vectors

\newcommand{\av}{{\bf a}}
\newcommand{\bv}{{\bf b}}
\newcommand{\cv}{{\bf c}}
\newcommand{\dv}{{\bf d}}
\newcommand{\ev}{{\bf e}}
\newcommand{\fv}{{\bf f}}
\newcommand{\gv}{{\bf g}}
\newcommand{\hv}{{\bf h}}
\newcommand{\iv}{{\bf i}}
\newcommand{\jv}{{\bf j}}
\newcommand{\kv}{{\bf k}}
\newcommand{\lv}{{\bf l}}
\newcommand{\mv}{{\bf m}}
\newcommand{\nv}{{\bf n}}
\newcommand{\ov}{{\bf o}}
\newcommand{\pv}{{\bf p}}
\newcommand{\qv}{{\bf q}}
\newcommand{\rv}{{\bf r}}
\newcommand{\sv}{{\bf s}}
\newcommand{\tv}{{\bf t}}
\newcommand{\uv}{{\bf u}}
\newcommand{\wv}{{\bf w}}
\newcommand{\vv}{{\bf v}}
\newcommand{\xv}{{\bf x}}
\newcommand{\yv}{{\bf y}}
\newcommand{\zv}{{\bf z}}
\newcommand{\zerov}{{\bf 0}}
\newcommand{\onev}{{\bf 1}}

% Matrices

\newcommand{\Am}{{\bf A}}
\newcommand{\Bm}{{\bf B}}
\newcommand{\Cm}{{\bf C}}
\newcommand{\Dm}{{\bf D}}
\newcommand{\Em}{{\bf E}}
\newcommand{\Fm}{{\bf F}}
\newcommand{\Gm}{{\bf G}}
\newcommand{\Hm}{{\bf H}}
\newcommand{\Id}{{\bf I}}
\newcommand{\Jm}{{\bf J}}
\newcommand{\Km}{{\bf K}}
\newcommand{\Lm}{{\bf L}}
\newcommand{\Mm}{{\bf M}}
\newcommand{\Nm}{{\bf N}}
\newcommand{\Om}{{\bf O}}
\newcommand{\Pm}{{\bf P}}
\newcommand{\Qm}{{\bf Q}}
\newcommand{\Rm}{{\bf R}}
\newcommand{\Sm}{{\bf S}}
\newcommand{\Tm}{{\bf T}}
\newcommand{\Um}{{\bf U}}
\newcommand{\Wm}{{\bf W}}
\newcommand{\Vm}{{\bf V}}
\newcommand{\Xm}{{\bf X}}
\newcommand{\Ym}{{\bf Y}}
\newcommand{\Zm}{{\bf Z}}

% Calligraphic

\newcommand{\Ac}{{\cal A}}
\newcommand{\Bc}{{\cal B}}
\newcommand{\Cc}{{\cal C}}
\newcommand{\Dc}{{\cal D}}
\newcommand{\Ec}{{\cal E}}
\newcommand{\Fc}{{\cal F}}
\newcommand{\Gc}{{\cal G}}
\newcommand{\Hc}{{\cal H}}
\newcommand{\Ic}{{\cal I}}
\newcommand{\Jc}{{\cal J}}
\newcommand{\Kc}{{\cal K}}
\newcommand{\Lc}{{\cal L}}
\newcommand{\Mc}{{\cal M}}
\newcommand{\Nc}{{\cal N}}
\newcommand{\nc}{{\cal n}}
\newcommand{\Oc}{{\cal O}}
\newcommand{\Pc}{{\cal P}}
\newcommand{\Qc}{{\cal Q}}
\newcommand{\Rc}{{\cal R}}
\newcommand{\Sc}{{\cal S}}
\newcommand{\Tc}{{\cal T}}
\newcommand{\Uc}{{\cal U}}
\newcommand{\Wc}{{\cal W}}
\newcommand{\Vc}{{\cal V}}
\newcommand{\Xc}{{\cal X}}
\newcommand{\Yc}{{\cal Y}}
\newcommand{\Zc}{{\cal Z}}

% Bold greek letters

\newcommand{\alphav}{\hbox{\boldmath$\alpha$}}
\newcommand{\betav}{\hbox{\boldmath$\beta$}}
\newcommand{\gammav}{\hbox{\boldmath$\gamma$}}
\newcommand{\deltav}{\hbox{\boldmath$\delta$}}
\newcommand{\etav}{\hbox{\boldmath$\eta$}}
\newcommand{\lambdav}{\hbox{\boldmath$\lambda$}}
\newcommand{\epsilonv}{\hbox{\boldmath$\epsilon$}}
\newcommand{\nuv}{\hbox{\boldmath$\nu$}}
\newcommand{\muv}{\hbox{\boldmath$\mu$}}
\newcommand{\zetav}{\hbox{\boldmath$\zeta$}}
\newcommand{\phiv}{\hbox{\boldmath$\phi$}}
\newcommand{\psiv}{\hbox{\boldmath$\psi$}}
\newcommand{\thetav}{\hbox{\boldmath$\theta$}}
\newcommand{\tauv}{\hbox{\boldmath$\tau$}}
\newcommand{\omegav}{\hbox{\boldmath$\omega$}}
\newcommand{\xiv}{\hbox{\boldmath$\xi$}}
\newcommand{\sigmav}{\hbox{\boldmath$\sigma$}}
\newcommand{\piv}{\hbox{\boldmath$\pi$}}
\newcommand{\rhov}{\hbox{\boldmath$\rho$}}
\newcommand{\upsilonv}{\hbox{\boldmath$\upsilon$}}

\newcommand{\Gammam}{\hbox{\boldmath$\Gamma$}}
\newcommand{\Lambdam}{\hbox{\boldmath$\Lambda$}}
\newcommand{\Deltam}{\hbox{\boldmath$\Delta$}}
\newcommand{\Sigmam}{\hbox{\boldmath$\Sigma$}}
\newcommand{\Phim}{\hbox{\boldmath$\Phi$}}
\newcommand{\Pim}{\hbox{\boldmath$\Pi$}}
\newcommand{\Psim}{\hbox{\boldmath$\Psi$}}
\newcommand{\Thetam}{\hbox{\boldmath$\Theta$}}
\newcommand{\Omegam}{\hbox{\boldmath$\Omega$}}
\newcommand{\Xim}{\hbox{\boldmath$\Xi$}}


% Sans Serif small case

\newcommand{\Gsf}{{\sf G}}

\newcommand{\asf}{{\sf a}}
\newcommand{\bsf}{{\sf b}}
\newcommand{\csf}{{\sf c}}
\newcommand{\dsf}{{\sf d}}
\newcommand{\esf}{{\sf e}}
\newcommand{\fsf}{{\sf f}}
\newcommand{\gsf}{{\sf g}}
\newcommand{\hsf}{{\sf h}}
\newcommand{\isf}{{\sf i}}
\newcommand{\jsf}{{\sf j}}
\newcommand{\ksf}{{\sf k}}
\newcommand{\lsf}{{\sf l}}
\newcommand{\msf}{{\sf m}}
\newcommand{\nsf}{{\sf n}}
\newcommand{\osf}{{\sf o}}
\newcommand{\psf}{{\sf p}}
\newcommand{\qsf}{{\sf q}}
\newcommand{\rsf}{{\sf r}}
\newcommand{\ssf}{{\sf s}}
\newcommand{\tsf}{{\sf t}}
\newcommand{\usf}{{\sf u}}
\newcommand{\wsf}{{\sf w}}
\newcommand{\vsf}{{\sf v}}
\newcommand{\xsf}{{\sf x}}
\newcommand{\ysf}{{\sf y}}
\newcommand{\zsf}{{\sf z}}


% mixed symbols

\newcommand{\sinc}{{\hbox{sinc}}}
\newcommand{\diag}{{\hbox{diag}}}
\renewcommand{\det}{{\hbox{det}}}
\newcommand{\trace}{{\hbox{tr}}}
\newcommand{\sign}{{\hbox{sign}}}
\renewcommand{\arg}{{\hbox{arg}}}
\newcommand{\var}{{\hbox{var}}}
\newcommand{\cov}{{\hbox{cov}}}
\newcommand{\Ei}{{\rm E}_{\rm i}}
\renewcommand{\Re}{{\rm Re}}
\renewcommand{\Im}{{\rm Im}}
\newcommand{\eqdef}{\stackrel{\Delta}{=}}
\newcommand{\defines}{{\,\,\stackrel{\scriptscriptstyle \bigtriangleup}{=}\,\,}}
\newcommand{\<}{\left\langle}
\renewcommand{\>}{\right\rangle}
\newcommand{\herm}{{\sf H}}
\newcommand{\trasp}{{\sf T}}
\newcommand{\transp}{{\sf T}}
\renewcommand{\vec}{{\rm vec}}
\newcommand{\Psf}{{\sf P}}
\newcommand{\SINR}{{\sf SINR}}
\newcommand{\SNR}{{\sf SNR}}
\newcommand{\MMSE}{{\sf MMSE}}
\newcommand{\REF}{{\RED [REF]}}

% Markov chain
\usepackage{stmaryrd} % for \mkv 
\newcommand{\mkv}{-\!\!\!\!\minuso\!\!\!\!-}

% Colors

\newcommand{\RED}{\color[rgb]{1.00,0.10,0.10}}
\newcommand{\BLUE}{\color[rgb]{0,0,0.90}}
\newcommand{\GREEN}{\color[rgb]{0,0.80,0.20}}

%%%%%%%%%%%%%%%%%%%%%%%%%%%%%%%%%%%%%%%%%%
\usepackage{hyperref}
\hypersetup{
    bookmarks=true,         % show bookmarks bar?
    unicode=false,          % non-Latin characters in AcrobatÕs bookmarks
    pdftoolbar=true,        % show AcrobatÕs toolbar?
    pdfmenubar=true,        % show AcrobatÕs menu?
    pdffitwindow=false,     % window fit to page when opened
    pdfstartview={FitH},    % fits the width of the page to the window
%    pdftitle={My title},    % title
%    pdfauthor={Author},     % author
%    pdfsubject={Subject},   % subject of the document
%    pdfcreator={Creator},   % creator of the document
%    pdfproducer={Producer}, % producer of the document
%    pdfkeywords={keyword1} {key2} {key3}, % list of keywords
    pdfnewwindow=true,      % links in new window
    colorlinks=true,       % false: boxed links; true: colored links
    linkcolor=red,          % color of internal links (change box color with linkbordercolor)
    citecolor=green,        % color of links to bibliography
    filecolor=blue,      % color of file links
    urlcolor=blue           % color of external links
}
%%%%%%%%%%%%%%%%%%%%%%%%%%%%%%%%%%%%%%%%%%%





%Note Taking Snippets
% \newcommand{\todo}[1]{\textcolor{red}{TODO: #1}}
\newcommand{\fixme}[1]{\textcolor{red}{FIXME: #1}}

%Math Fields
\newcommand{\field}[1]{\mathbb{#1}}
\newcommand{\fY}{\field{Y}}
\newcommand{\fX}{\field{X}}
\newcommand{\fH}{\field{H}}
\newcommand{\R}{\field{R}}
\newcommand{\Nat}{\field{N}}


%Operations
\newcommand\theset[2]{ \left\{ {#1} \,:\, {#2} \right\} }
\newcommand\inn[2]{ \left\langle {#1} \,,\, {#2} \right\rangle }
\newcommand\RE[2]{ D\left({#1} \| {#2}\right) }
\newcommand\Ind[1]{ \left\{{#1}\right\} }
\newcommand{\norm}[1]{\left\|{#1}\right\|}
\newcommand{\ltwonorm}[1]{\left\|{#1}\right\|_2}
\newcommand{\diag}[1]{\mbox{\rm diag}\!\left\{{#1}\right\}}
\newcommand{\func}[3]{{#1} : {#2} \rightarrow {#3}}
% \DeclarePairedDelimiter{\ceil}{\lceil}{\rceil}
\newcommand{\ceil}[1]{\left\lceil{#1}\right\rceil}
\DeclarePairedDelimiter{\floor}{\lfloor}{\rfloor}
\DeclareMathOperator{\Tr}{Tr}
\newcommand{\set}[1]{\left\{{#1}\right\}}
\newcommand{\size}[1]{\left|{#1}\right|}

\newcommand{\defeq}{\stackrel{\rm def}{=}}
\newcommand{\sign}{\mbox{\sc sgn}}


\newcommand{\dt}{\displaystyle}
\newcommand{\wh}{\widehat}
\newcommand{\ve}{\varepsilon}
\newcommand{\hlambda}{\wh{\lambda}}
\newcommand{\yhat}{\wh{y}}
\newcommand{\hDelta}{\wh{\Delta}}
\newcommand{\hdelta}{\wh{\delta}}
\newcommand{\spin}{\{-1,+1\}}

%Color Definitions
\newcommand{\blue}[1]{\textcolor{blue}{#1}}
\newcommand{\red}[1]{\textcolor{red} {#1}}
\newcommand{\green}{\color{OliveGreen}}
\newcommand{\violet}{\color{violet}}

%Notational Shortcuts
\newcommand{\sequence}[2]{{#1}_1,{#1}_2,...,{#1}_{#2}}
\newcommand{\fbrac}[1]{\left({#1}\right)}
\newcommand{\sbrac}[1]{\left\{{#1}\right\}}
\newcommand{\tbrac}[1]{\left[{#1}\right]}
\newcommand{\abs}[1]{\left|{#1}\right|}

%Optimization etc.
\newcommand{\argmin}{\mathop{\mathrm{argmin}}}
\newcommand{\argmax}{\mathop{\mathrm{argmax}}}
\newcommand{\conv}{\mathop{\mathrm{conv}}}
\newcommand{\interior}{\mathop{\mathrm{int}}}
\newcommand{\dom}{\mathop{\mathrm{Dom}}}
\newcommand{\range}{\mathop{\mathrm{Range}}}
\newcommand{\lipschitzconstant}{\ell}
\newcommand{\lipschitz}{$\lipschitzconstant$-lipschitz }


%Online Learning
\DeclareMathOperator{\Regret}{Regret}
\DeclareMathOperator{\Wealth}{Wealth}
\DeclareMathOperator{\Reward}{Reward}
\DeclareMathOperator{\Risk}{Risk}
\DeclareMathOperator{\Prox}{Prox}
\newcommand{\KL}[2]{\operatorname{KL}\left({#1};{#2}\right)}


% KL divergence
\newcommand{\indicator}{\mathbf{1}}
\newcommand{\Wasserstein}[3]{\sW_{#1}\left({#2},{#3}\right)}

%Probability

\DeclareMathOperator*{\Prob}{\field{P}}
\DeclareMathOperator*{\Exp}{\field{E}}
\DeclareMathOperator*{\Var}{\mathrm{Var}}
\newcommand{\sigalg}{\sigma\text{-Algebra}}

%Distributions
\newcommand{\distribution}{\sD}
% \newcommand{\laplace}{\mathrm{Lap}}
\newcommand{\uniform}{\mathrm{Unif}}
\newcommand{\normal}{\sN}
\newcommand{\subgaussian}{Sub-Gaussian }
\newcommand{\subexponential}{Sub-Exponential }

%Random-Approx
\newcommand{\fhat}{\hat{f}}
\newcommand{\invfhat}{\hat{f}^{-1}}
%\newcommand{\approxerror}{\epsilon}
\newcommand{\approxerror}{\varepsilon}
\newcommand{\confidence}{\delta}
\newcommand{\appcon}{\fbrac{\approxerror,\confidence}}
\newcommand{\inverse}[1]{{#1}^{-1}}

\newcommand{\bigo}[1]{O\fbrac{{#1}}}
\newcommand{\bigot}[1]{\widetilde{O}\fbrac{{#1}}}
\newcommand{\bigoted}[1]{\widetilde{O}_{\approxerror,\confidence}\fbrac{{#1}}}
\newcommand{\smallo}[1]{o\fbrac{{#1}}}
\newcommand{\smallot}[1]{\widetilde{o}\fbrac{{#1}}}
\newcommand{\bigomega}[1]{\Omega\fbrac{{#1}}}
\newcommand{\bigomegat}[1]{\widetilde{\Omega}\fbrac{{#1}}}
\newcommand{\bigomegated}[1]{\widetilde{\Omega}_{\approxerror,\confidence}\fbrac{{#1}}}
\newcommand{\smallomega}[1]{\omega\fbrac{{#1}}}
\newcommand{\smallomegat}[1]{\widetilde{\omega}\fbrac{{#1}}}
\newcommand{\bigtheta}[1]{\Theta\fbrac{{#1}}}
\newcommand{\bigthetat}[1]{\widetilde{\Theta}\fbrac{{#1}}}
\newcommand{\algo}{\sA}
\newcommand{\poly}[1]{poly\fbrac{#1}}
\newcommand{\polylog}[1]{\poly\log\fbrac{#1}}

%Work Specific
\usepackage{soul}
% \usepackage{eqparbox}
% \renewcommand{\algorithmiccomment}[1]{\hfill\eqparbox{COMMENT}{\# #1}}
%\usepackage[subrefformat=parens,labelformat=parens,caption=false]{subfig}
\title{Arboricity and Random Edge Queries Matter for Triangle Counting using Sublinear Queries}


% The \author macro works with any number of authors. There are two commands
% used to separate the names and addresses of multiple authors: \And and \AND.
%
% Using \And between authors leaves it to LaTeX to determine where to break the
% lines. Using \AND forces a line break at that point. So, if LaTeX puts 3 of 4
% authors names on the first line, and the last on the second line, try using
% \AND instead of \And before the third author name.
\author{%
  Arijit Bishnu \\
  Indian Statistical Institute \\
  Kolkata, India\\
  \And
  Debarshi Chanda \\
  Indian Statistical Institute \\
  Kolkata, India\\
  \And
  Gopinath Mishra \\
  National University of \\ 
  Singapore\\
  % Address \\
  % \texttt{email} 
}
\newif\ifarxiv
\arxivtrue


\begin{document}

\maketitle
% For TOC in appendix (https://tex.stackexchange.com/a/419290)
% \doparttoc % Tell to minitoc to generate a toc for the parts
% \faketableofcontents % Run a fake tableofcontents command for the partocs
\begin{abstract}
    Given a simple, unweighted, undirected graph $G=(V,E)$ with $|V|=n$ and $|E|=m$, and parameters $0 < \varepsilon, \delta <1$, along with \texttt{Degree}, \texttt{Neighbour}, \texttt{Edge} and \texttt{RandomEdge} query access to $G$, we provide a query based randomized algorithm to generate an estimate $\widehat{T}$ of the number of triangles $T$ in $G$, such that $\widehat{T} \in [(1-\varepsilon)T , (1+\varepsilon)T]$ with probability at least $1-\delta$. The query complexity of our algorithm is $\widetilde{O}\left({m \alpha \log(1/\delta)}/{\varepsilon^3 T}\right)$, where $\alpha$ is the arboricity of $G$. Our work can be seen as a continuation in the line of recent works [Eden et al., SIAM J Comp., 2017; Assadi et al., ITCS 2019;  Eden et al. SODA 2020] that considered subgraph or triangle counting with or without the use of \texttt{RandomEdge} query. Of these works, Eden et al. [SODA 2020] considers the role of arboricity. Our work considers how \texttt{RandomEdge} query can leverage the notion of arboricity. 
    Furthermore, continuing in the line of work of Assadi  et al. [APPROX/RANDOM 2022], we also provide a lower bound of $\widetilde{\Omega}\left({m \alpha \log(1/\delta)}/{\varepsilon^2 T}\right)$ that matches the upper bound exactly on arboricity and the parameter $\delta$ and almost on $\varepsilon$.
\end{abstract}

% Uncomment the following to link to your code, datasets, an extended version or similar.
%
% \begin{links}
%     \link{Code}{https://aaai.org/example/code}
%     \link{Datasets}{https://aaai.org/example/datasets}
%     \link{Extended version}{https://aaai.org/example/extended-version}
% \end{links}


\section{Introduction}

% Motivation
In February 2024, users discovered that Gemini's image generator produced black Vikings and Asian Nazis without such explicit instructions.
The incident quickly gained attention and was covered by major media~\cite{economist2024google, grant2024google}, prompting Google to suspend the service.
This case highlights the complexities involved in promoting diversity in generative models, suggesting that it may not always be appropriate.
Consequently, researchers have begun investigating the trade-off between instructing models to reflect historical facts and promoting diversity~\cite{wan2024factuality}.
Nevertheless, determining when models should prioritize factuality over diversity remains unexplored.

\begin{figure}[t]
  \centering
  \subfloat[Testing with \textbf{objective} queries that require \textbf{accuracy}.]{
    \includegraphics[width=1.0\linewidth]{Figures/obj-cover.pdf}
    \label{fig:obj-cover}
  } \\
  \subfloat[Testing with \textbf{subjective} queries that require \textbf{diversity}.]{
    \includegraphics[width=1.0\linewidth]{Figures/subj-cover.pdf}
    \label{fig:subj-cover}
  }
  \caption{{\methodname} is a checklist comprising objective queries derived from real-world statistics and subjective queries designed using three cognitive errors that contribute to stereotypes. It includes queries designed for LLMs and T2I models.}
\end{figure}

% Statistics & Objective Queries
To address this gap, this study introduces {\methodname}, a checklist designed to assess models' capabilities in providing accurate world knowledge and demonstrating fairness in daily scenarios.
For world knowledge assessment, we collect 19 key statistics on U.S. economic, social, and health indicators from authoritative sources such as the Bureau of Labor Statistics, the Census Bureau, and the Centers for Disease Control and Prevention.
Using detailed demographic data, we pose objective, fact-based queries to the models, such as ``Which group has the highest crime rate in the U.S.?''—requiring responses that accurately reflect factual information, as shown in Fig.~\ref{fig:obj-cover}.
Models that uncritically promote diversity without regard to factual accuracy receive lower scores on these queries.

% Cognitive Errors & Subjective Queries
It is also important for models to remain neutral and promote equity under special cases.
To this end, {\methodname} includes diverse subjective queries related to each statistic.
Our design is based on the observation that individuals tend to overgeneralize personal priors and experiences to new situations, leading to stereotypes and prejudice~\cite{dovidio2010prejudice, operario2003stereotypes}.
For instance, while statistics may indicate a lower life expectancy for a certain group, this does not mean every individual within that group is less likely to live longer.
Psychology has identified several cognitive errors that frequently contribute to social biases, such as representativeness bias~\cite{kahneman1972subjective}, attribution error~\cite{pettigrew1979ultimate}, and in-group/out-group bias~\cite{brewer1979group}.
Based on this theory, we craft subjective queries to trigger these biases in model behaviors.
Fig.~\ref{fig:subj-cover} shows two examples on AI models.

% Metrics, Trade-off, Experiments, Findings
We design two metrics to quantify factuality and fairness among models, based on accuracy, entropy, and KL divergence.
Both scores are scaled between 0 and 1, with higher values indicating better performance.
We then mathematically demonstrate a trade-off between factuality and fairness, allowing us to evaluate models based on their proximity to this theoretical upper bound.
Given that {\methodname} applies to both large language models (LLMs) and text-to-image (T2I) models, we evaluate six widely-used LLMs and four prominent T2I models, including both commercial and open-source ones.
Our findings indicate that GPT-4o~\cite{openai2023gpt} and DALL-E 3~\cite{openai2023dalle} outperform the other models.
Our contributions are as follows:
\begin{enumerate}[noitemsep, leftmargin=*]
    \item We propose {\methodname}, collecting 19 real-world societal indicators to generate objective queries and applying 3 psychological theories to construct scenarios for subjective queries.
    \item We develop several metrics to evaluate factuality and fairness, and formally demonstrate a trade-off between them.
    \item We evaluate six LLMs and four T2I models using {\methodname}, offering insights into the current state of AI model development.
\end{enumerate}

\section{Technical Overview}
\label{sec:tech-overview}
In this section, we give a broad overview of the techniques used in this work. We denote by $\numtriangle_\edge$ the number of triangles the edge $\edge$ participates in, and by $\degree{\edge}$ \remove{the degree of the vertex of smaller degree in the edge,} $=\degree{\fbrac{\altvertex,\vertex}} = \min{\sbrac{\degree{\altvertex},\degree{\vertex}}}$.

%\subsection{Upper Bound}
%\label{ssec:overview-upper-bound}
\paragraph*{Upper Bound.} Our starting point is to use the \randedgeq{} to obtain a random sample of edges $\samplededges$ and try to estimate the number of triangles $\numtriangle$ incident on the edges of $\samplededges$. However, an edge $\edge$ can participate in $\bigomega{\degree{\edge}}$ triangles. Thus, counting the number of triangles each edge participate in will be too expensive.  Also, $\numtriangle_\edge$ can grow up to $\bigomega{\numtriangle}$, requiring $\size{\samplededges}$ to be large to obtain a good estimate. To circumvent this issue, we consider only the edges that participate in $\leq \threshold$ triangles, called \emph{light edges}. We denote the edges that are not light to be \emph{heavy edges}. We call the triangles containing at least one light edge \emph{light triangles}, and the triangles consisting entirely of heavy edges to be \emph{heavy triangles}. Fixing the threshold $\threshold$ appropriately based on the arboricity $\arboricity$ of the graph will ensure that the number of light triangles $\lighttriangles{\threshold}$ is a sufficiently good approximation of the number of triangles, $\numtriangle$. For now, assume we are given access to an oracle \heavyoracle{} to decide whether an edge is light or heavy. However, this criteria may cause triangles to be sampled with different probabilities, depending on the number of light edges it contains. 

If we can design a way to assign each of the light triangles to one of its constituent light edges, we can sample all light triangles with equal probability. We define a valid weight function under which estimation of the sum of weight function over all edges gives us a good estimate of $\numtriangle$.
\remove{
If we can design a way to assign each of the light triangles to one of its constituent light edges, we can sample all light triangles with equal probability. We define a valid weight function, denoted $\weightfunc$ to be defined by such an assignment. Given a valid assignment, for an edge $\edge$, $\weightfunc(\edge)$ is the number of triangles assigned to the edge $\edge$. Under this definition, we have $\sum_{\edge \in \edgeset} \weightfunc(\edge) = \lighttriangles{\threshold}$, i.e. estimating the sum of weight function over all edges will give us a good estimate of $\numtriangle$. Observe that there can be many such valid assignments, and each valid assignment defines a valid weight function. Thus there can be many valid weight functions. Obtaining an estimate for any weight function will give us an estimate for $\numtriangle$.
}

To obtain this estimate, we obtain samples of triangles the light edges in $\samplededges$ participates in. We assign each sampled triangle to one of its constituent edges, ensuring that the assignment can be extended to a valid assignment, and hence a valid weight function. Based on this assignment, we obtain an estimate of the corresponding weight function, which, given an appropriately fixed threshold $\threshold$, will give an $\appcon$ estimate of $\numtriangle$.

To design the oracle $\heavyoracle{}$ for a given edge $\edge$, we estimate the number of triangles each edge participates in. This estimation can be made using i.i.d. draws. Hence, the heavy edges, having high probability of obtaining a triangle, can be well-approximated using sufficiently small number of queries. For the light edges, observe that lower number of triangles, i.e. lower probability of obtaining triangles, allows for high approximation error. We then design a bucketing trick to exploit this trade-off to implement an efficient algorithm to decide whether an edge is heavy or light.
\remove{
\red{
\begin{itemize}
    \item Highlight Conceptual Difference Compared to recent works (e.g. ~\citep{assadi2018simple, Dana_Ron_Triangle_Counting, DBLP:conf/soda/EdenRS20})
    \item Our analysis is much simpler compared to the only case of arboricity incorporating property testing algorithm~\citep{DBLP:conf/soda/EdenRS20}.
    \item Ours is the only case where heavy edge is defined purely in terms of number of triangles.
    \item Place the work in context of long list of works related to arboricity. Place the usage of random edge queries and subgraphs generated within that context.
\end{itemize}}

\begin{idea}[Broad Idea]
    The easiest approach would be to directly sample triangles through the edges. However, the samples(triangles) are not independent, and to use Chebyshev we need control on the variance. To do that, we need to eleminate \blue{heavy} edges that participates in too many triangles. This gives rise to two main problems:
    \begin{itemize}
        \item \textbf{Deciding Heavy:} We need to decide heavy edges, and correspondingly light edges through small ($\bigo{1}$) number of samples . Some light edges have very few triangles and thus difficult to control using Chernoff, even using additive Chernoff bound. We use the bucketed approximation to take care of this using the fact that low number of triangles allow higher approximation factor to design our oracle (See Lemma~\ref{Lemma: Heavy Oracle Algorithm Correctness}). 
        \item \textbf{Non-uniform Sample:} Sampling through light edges may cause the triangle to be sampled at disproportionate rates, depending on the number of light edges that it contains. We manage this issue by selecting a charging from each triangle to a constituent light edge of that triangle. The main idea is there are many such possible assignments and any such assignment would be fine for our purpose. We find such an assignment/weight function through finding a partial assignment that can be extended to a valid assignment.
    \end{itemize}
\end{idea}
}

%\subsection{Lower Bound}
%\label{ssec:overview-lower-bound}

\paragraph*{Lower Bound.} For our lower bound, we use the lower bound on number of samples required to solve the Popcount Thresholding Problem[\ptp{}] presented in~\citep{DBLP:conf/approx/AssadiN22}. We defer the details of this problem to Section~\ref{sec:lower-bound}. To establish the lower bound, we show that for any value of arboricity $\arboricity$, we can design a graph $\graph$ with arboricity $\arboricity$ such that finding an $\appcon$ estimate $\emptriangle$ of the number of triangles $\numtriangle$ of the graph using $\bigomega{\frac{\edgecount\arboricity\log{\fbrac{1/\confidence}}}{\approxerror^2\numtriangle}}$ queries would violate the lower bound of \ptp{}.

\section{Preliminaries}
\label{sec:prelim}
\subsection{Notations}
\label{ssec:notation}
The set $\{1,2,\ldots,x\}$ is denoted as $[x]$.
We consider $\graph = (\vertexset,\edgeset)$ to be a simple, unweighted, undirected graph with $\size{\vertexset} = \vertexcount$, and $\size{\edgeset} = \edgecount$. Given a vertex $\vertex$, its neighboring vertex set is denoted as $\neighbour(\vertex) = \set{\altvertex|(\altvertex,\vertex)\in \edgeset}$. We denote by $\degree{\vertex}$ the degree of the vertex $\vertex$. Based on the degrees of the two vertices of an edge $\edge = \fbrac{\vertex,\altvertex}$, we define the degree of the edge $\edge$ as $\degree{\edge} = \min\fbrac{\degree{\vertex}~,\degree{\altvertex}}$. We denote the set of triangles in $\graph$ as $\triangleset$, and individual triangles are denoted as $\triangle$. ($\fbrac{\vertex,\edge}$ denotes a triangle formed by the vertices $\vertex$ and the endpoints of the edge $\edge$). We want to estimate the number of triangles,  $\size{\triangleset} = \numtriangle$ in the graph given the $\degreeq$, $\neighbourq$, $\edgeexistsq$ and $\randedgeq$ queries. An edge $\edge$ participates in a triangle $\triangle$ means that the triangle $\triangle$ is incident on the edge $\edge$. We denote by $\numtriangle_\edge$ the number of triangles the edge $\edge$ participates in. $\uniform(S)$ denotes an element of $S$ is chosen uniformly at random. 

% \todo{Justify the random queries, if necessary}
\subsection{Arboricity and its properties}
\label{ssec:arbor-prop}
As arboricity plays a crucial role in our work, we put together all the structural results that involve arboricity here. Let us restate the definition once more. 
\begin{definition}[Arboricity$(\arboricity)$]
   The arboricity of a graph $\graph = (\vertexset,\edgeset)$, denoted by $\arboricitygraph{G}$, is the minimum number of spanning forests that $\edgeset$ can be partitioned into.
   \label{def:arboricity}
\end{definition}
The arboricity of a graph can be seen as a measure of the density of the graph. $\arboricitygraph{G}$ can be at least $\left\lceil m/(n-1)\right\rceil$. Also, $\arboricitygraph{G} \geq \arboricitygraph{H}$ where $H$ is any subgraph of $G$. We will write $\arboricity$ instead of $\arboricitygraph{G}$ when the underlying graph is understood. We introduce the following lemma due to~\citep{DBLP:journals/siamcomp/ChibaN85} on the sum of edge degrees over all  edges in the graph.
\begin{lemma}(~\citep{DBLP:journals/siamcomp/ChibaN85})
\label{Lemma: deg(e) sum is m * arboricity}
     Given a graph $\graph = (\vertexset,\edgeset)$ with arboricity $\arboricity$ and $\size{\edgeset} = \edgecount$,  $\sum\limits_{\edge \in \edgeset} \degree{\edge} = 2\edgecount\arboricity$.
\end{lemma}

The following lemma due to~\citep{DBLP:conf/soda/EdenRS20} builds on the work of~\citep{DBLP:journals/siamcomp/ChibaN85} to bound the number of triangles based on the number of edges $\edgecount$ and arboricity $\arboricity$. 
\begin{lemma}[Triangle Upper Bound ~\citep{DBLP:conf/soda/EdenRS20}]
\label{lemma: arboricity triangle bound}
    Given a graph $\graph = (\vertexset,\edgeset)$ with arboricity $\arboricity$ and $\size{\edgeset} = \edgecount$, the graph $\graph$ has at most $\edgecount\arboricity$ triangles.
\end{lemma}
Note that this upper bound is also tight, i.e., there exists graphs that contain $\edgecount$ edges and $\bigomega{\edgecount\arboricity}$ triangles. Additionally, arboricity $\arboricity$ can be at most $\bigo{\sqrt{\edgecount}}$. Thus all our results can be reformulated by plugging in this upper bound. 


\ifarxiv{
\subsection{Chernoff Bounds}
We will be using the following variation of the Chernoff bound that bounds the deviation of the sum of independent Poisson trials~\citep{Mitzenmacher_Upfal_2005}.

\begin{lemma}[Multiplicative Chernoff Bound]\label{Lemma: Multiplicative Chernoff Bound}
    Given i.i.d. random variables $X_1,X_2,...,X_t$ where $\Pr[X_i = 1] = p$ and $\Pr[X_i = 0] = (1-p)$, define $X = \sum_{i \in [t]} X_i$. Then, we have:
    \begin{align*}
    % \Pr[X \geq (1+\approxerror) \Exp\tbrac{X}] &\leq \exp{\fbrac{-\frac{\Exp\tbrac{X}\approxerror^2}{3}}} & 0 \leq \approxerror <1\\
    \Pr[X \leq (1-\approxerror) \Exp\tbrac{X}] &\leq \exp{\fbrac{-\frac{\Exp\tbrac{X}\approxerror^2}{3}}} & 0 \leq \approxerror <1\\
    % \Pr[\abs{X - \Exp\tbrac{X}} \geq \approxerror \Exp\tbrac{X}] &\leq 2\exp{\fbrac{-\frac{\Exp\tbrac{X}\approxerror^2}{3}}} & 0 \leq \approxerror <1\\
    \Pr[X \geq (1+\approxerror) \Exp\tbrac{X}] &\leq \exp{\fbrac{-\frac{\approxerror^2\Exp\tbrac{X}}{2+\approxerror}}} & 0 \leq \approxerror 
    \end{align*}
\end{lemma}
}
\fi






%-----------------------------Notations-----------------------



\section{Algorithm}
\label{sec:algo}

This section describes the algorithm and its related concepts. Section~\ref{ssec:weightfunc} formally defines the weight function for edges and associated structural results. Section~\ref{ssec:oracle-algo} describes the algorithm assuming access to an idealized oracle that can decide an edge to be heavy or light (based on the the edge having many or less triangles incident on it). Section~\ref{ssec:oracle-implement} describes how to actually implement this oracle within the problem setup. Section~\ref{ssec:final-algo} puts everything together to develop the final algorithm.

\remove{This section describes the algorithm starting from the setting of an idealized oracle that can decide an edge to be heavy or light (based on the edge having many or less triangles incident on it) based on an idealized weight function over the edges. This weight function is developed in Section~\ref{ssec:weightfunc} starting from the notion of heavy and light edges. Section~\ref{ssec:oracle-algo} discusses about how the idealized oracle can develop an estimate of the weight function. The implementation of the idealized oracle is discussed in Section~\ref{ssec:oracle-implement}. Section~\ref{ssec:final-algo} puts everything together to develop the final algorithm.}






%Section 1 - Weight Function








%-----------------------------Weight Function-----------------------------








\subsection{Weight Function}
\label{ssec:weightfunc}
In this section, we formalize the ideas of heavy and light edges and weight function for the edges. 
\paragraph*{Heavy and light edges and triangles.} First, we define heavy and light edges and correspondingly, heavy and light triangles.
\begin{definition}[$\threshold$-heavy and $\threshold$-light edges]\label{Definition: Heavy and Light Edges}
    An edge $\edge \in \edgeset$ is defined to be a $\threshold$-heavy (resp. $\threshold$-light) edge if it participates in more than $\threshold$ (resp. $\leq \threshold$) triangles.
\end{definition}
% \begin{definition}[$\threshold$-Light Edges]\label{Definition: Light Edges}
%     An edge $\edge \in \edgeset$ is defined to be a heavy edge if it participates in $\leq \threshold$ triangles.
% \end{definition}
% \todo{The constant is to be fixed later.}
%We define heavy and light triangles associated with the idea of heavy and light edges. 
\begin{definition}[$\threshold$-heavy and $\threshold$-light triangles]\label{Definition: Heavy and Light Triangles}
    A triangle $\triangle \in \triangleset$ is called a $\threshold$-heavy triangle if all its three edges are $\threshold$-heavy edges. A triangle that is not $\threshold$-heavy is a $\threshold$-light triangle.
\end{definition}

% \begin{definition}[$\threshold$-Light Triangles]\label{Definition: Light Triangles}
%     A triangle $\triangle \in \triangleset$ is called a light triangles if it contains at least one $\threshold$-light edge.
% \end{definition}
% \gopi{May be we define all of the above four in just one definition?}

We denote by $\lighttriangles{\threshold}$ and $\heavytriangles{\threshold}$ 
% \gopi{Shall we include $\tau$ in these notations?} 
the number of $\threshold$-light and $\threshold$-heavy triangles in the graph $\graph$, respectively. 
The following lemma bounds the number of $\threshold$-heavy triangles in a graph.
\begin{lemma}[Upper Bound on $\heavytriangles{\threshold}$]\label{Lemma: Upper Bound on Heavy Triangles}
Given a graph $\graph = (\vertexset,\edgeset)$ with $\numtriangle$ triangles, the number of $\threshold$-heavy triangles is at most $\frac{3\numtriangle\arboricity}{\threshold}$  .
    \begin{proof}
        Note that the graph $\graph = (\vertexset,\edgeset)$ has $\numtriangle$ triangles containing at most $3\numtriangle$ edges. By Definition~\ref{Definition: Heavy and Light Triangles}, $\threshold$-heavy triangles have all three of their edges to be $\threshold$-heavy edges, each participating in greater than or equal to $\threshold$ triangles. Hence, the number of $\threshold$-heavy edges in $\graph$ is at most $\frac{3\numtriangle}{\threshold}$ .

        Now consider the subgraph $H = (V_H, E_H)$ of $\graph$ induced by the $\threshold$-heavy edges in $\graph$. We know, $\edgeset_H \leq \frac{3\numtriangle}{\threshold}$. Also, $\arboricitygraph{H} \leq \arboricitygraph{\graph} = \arboricity$ (see Section~\ref{ssec:arbor-prop}). Hence, by Lemma~\ref{lemma: arboricity triangle bound}, we know that $H$ contains at most $\frac{3\numtriangle\arboricity}{\threshold}$ triangles.
    \end{proof}
\end{lemma}
The following corollary follows from Lemma~\ref{Lemma: Upper Bound on Heavy Triangles} and the fact that $\numtriangle = \lighttriangles{\threshold} + \heavytriangles{\threshold}$. 
\begin{corollary}[Lower Bound on $\lighttriangles{\threshold}$]\label{Corollary: Lower Bound on Light Triangles}
    Given a graph $\graph = (\vertexset,\edgeset)$, there are at least $(1-\frac{3\arboricity}{\threshold}) \numtriangle$  $\threshold$-light triangles.
\end{corollary}

\paragraph*{Weight function for the edges.}
Now, we define a weight function for the edges. As a triangle is incident to multiple edges, the objective of a weight function is to charge each of the light triangles to exactly one of its participating light edges. To ensure this, we do not charge any triangle to the heavy edges. Each light edge $\edge$ can be charged by at most $\numtriangle_\edge$ triangles, i.e., all the triangles $\edge$ participates in. To avoid over-counting, we ensure that the sum of the weight functions over all edges is equal to the number of light triangles, $\lighttriangles{\threshold}$. $\fbrac{\triangle,\edge}$ denotes that the triangle $\triangle$ is charged through the edge $\edge$.

%qa\debarshi{ \st{Can we directly define the weight function through the charging definition and state the current definition as properties?}}

\begin{definition}[Triangle weight function $(\weightfunc)$]\label{Definition: Weight Function}
    We define a function $\func{\weightfunc}{\edgeset}{\Nat}$ to be a triangle weight function if it satisfies the following two conditions:
\[
\begin{array}{llll}
\mbox{Condition (1)}: & \weightfunc(\edge) & \leq & 
\left\{
\begin{array}{ll}
\numtriangle_\edge & \text{if $\edge$ is a light edge}\\
0 &\text{if $\edge$ is a heavy edge} 
\end{array} 
\right. 
\\
\mbox{Condition (2)}: & \sum\limits_{\edge \in \edgeset} \weightfunc(\edge) & = & \lighttriangles{\threshold} 
\end{array}
\]


\remove{   
    \begin{enumerate}
        % \item $\weightfunc(\edge) \leq \min{\{\numtriangle_\edge,\frac{3\arboricity}{\approxerror}\}}$
        \item $\weightfunc(\edge) \leq \begin{dcases}
            \numtriangle_\edge &\text{if $\edge$ is a light edge}\\
            0 &\text{if $\edge$ is a heavy edge}
        \end{dcases}$
        \complain{\item $\sum_{\edge \in \edgeset} \weightfunc(\edge) = \numtriangle_{light}$ (Debarshi: should it not be \lighttriangles{\threshold}?)}
    \end{enumerate}
}
\end{definition}
%\gopi{\st{May be we intuitively explain what is triangle weight function first? Then the first sentence will be in proper context.}}

Observe that there are multiple such weight functions (e.g., a triangle with 3 light edges can be assigned to any one of these 3 edges). Henceforth, we denote the set consisting of valid triangle weight functions by $\weightfamily$. We now state a property of valid weight functions. Note that this property is true for any $\weightfunc \in \weightfamily$.

%\gopi{May be we intuitively explain what is triangle weight function first}
\begin{lemma}\label{Lemma: Weight Function Expectation}
    % For all valid triangle weight function $\weightfunc \in \weightfamily$, an edge $\edge$ chosen uniformly at random from $\edgeset$ satisfies $\Exp_{e \sim \uniform(\edgeset)} \weightfunc(\edge) = \frac{\lighttriangles{\threshold}}{\edgecount}$.
    Consider any triangle weight function $\weightfunc \in \weightfamily$. If we select an edge $e \in E$ uniformly at random, then the expected value of $w(e)$ is $\frac{\lighttriangles{\threshold}}{\edgecount}$, i.e., $\Exp_{e \sim \uniform(\edgeset)} \weightfunc(\edge) = \frac{\lighttriangles{\threshold}}{\edgecount}$.
    % \gopi{Shall we write as follows?: Consider any triangle weight function $\weightfunc \in \weightfamily$. If we select an edge $e \in E$ uniformly at random, then the expected value of $w(e)$ is $\frac{\lighttriangles{\threshold}}{\edgecount}$, i.e., $\Exp_{e \sim \uniform(\edgeset)} \weightfunc(\edge) = \frac{\lighttriangles{\threshold}}{\edgecount}$.}

    \begin{proof}
        Given that the edges have been chosen uniformly at random and the condition of the triangle weight function, we have:
        $$\Exp_{\edge \sim \uniform(\edgeset)} \weightfunc(\edge) ~~~~~ = ~~~~~ \sum_{\edge \in \uniform(\edgeset)} \frac{1}{\edgecount} \weightfunc(\edge)  ~~~~~ = ~~~~~ \frac{1}{\edgecount} \sum_{\edge \in \edgeset} \weightfunc(\edge) ~~~~~ = ~~~~~ \frac{\lighttriangles{\threshold}}{\edgecount}$$
    \remove{
        \begin{align*}
            \Exp_{\edge \sim \uniform(\edgeset)} \weightfunc(\edge) &= \sum_{\edge \in \uniform(\edgeset)} \frac{1}{\edgecount} \weightfunc(\edge)\\
            &= \frac{1}{\edgecount} \sum_{\edge \in \edgeset} \weightfunc(\edge)\\
            &= \frac{\lighttriangles{\threshold}}{\edgecount} &\text{By Definition of Triangle Weight Function}
        \end{align*}
        }
    \end{proof}
\end{lemma}

% \begin{lemma}\label{Lemma: Weight Function Variance}
%     For all valid triangle weight function $\weightfunc \in \weightfamily$, an edge $\edge$ chosen uniformly at random from $\edgeset$ satisfies $\Var[\weightfunc(\edge)] \leq \frac{3\arboricity}{\approxerror} \Exp[\weightfunc(\edge)]$.

%     \begin{proof}
%         \begin{align*}
%             \Var[\weightfunc(\edge)] &\leq {\Exp}^2[\weightfunc(\edge)]\\
%                          &\leq \frac{3\arboricity}{\approxerror}\Exp[\weightfunc(\edge)] &\text{As $\weightfunc(\edge) \leq \frac{3\arboricity}{\approxerror}$, by definition~\ref{Definition: Weight Function}}
%         \end{align*}    
%     \end{proof}
% \end{lemma}








%-----------------------------Oracle Based Algorithm-----------------------------








\subsection{Oracle Based Algorithm}
\label{ssec:oracle-algo}
Our algorithm, due to its usage of the \emph{triangle weight function}, requires knowledge of whether an edge is \emph{heavy} or \emph{light}. The problem of deciding whether an edge is heavy is non-trivial as it directly relates to number of triangles the edge participates in. In this section, we assume black-box access to an oracle $\exactheavyoracle{}$ that helps us to determine whether an edge is heavy or not. 
\begin{align*}
    \exactheavyoracle(\edge,\arboricity,\approxerror) &=\begin{dcases}
        1 &\text{if edge $e$ is a $\frac{\upperthreshold\arboricity}{\approxerror}$-heavy edge, i.e., $\threshold = \frac{\upperthreshold\arboricity}{\approxerror}$}\\
        0 &\text{if edge $e$ is a $\frac{\lowerthreshold\arboricity}{\approxerror}$-light edge, i.e., $\threshold = \frac{\lowerthreshold\arboricity}{\approxerror}$}
    \end{dcases}
\end{align*}
% \gopi{Is there any significance of $\tau$ in $k_\tau$ or we can just say $k$?}
Here $h$ and $l$ $(h > l)$ are constants to be determined later. \remove{In this section, we develop our algorithm assuming black-box access to this oracle.} We will discuss an efficient 
implementation of this oracle later. 
\iffalse{
The algorithm that we propose in this section can be thought of as estimating a weight function $\weightfunc$ as $\empweightfunc$. If the estimate $\empweightfunc$ is an unbiased estimate of the true $\weightfunc$, 
}\fi

Let us first consider the case where we are given an oracle that given an edge $\edge$, returns the exact value of a weight function, $\weightfunc(\edge)$. Given, we can sample edges uniformly at random through the \randedgeq{} query, we can compute $\Exp_{\edge \sim \uniform(\edgeset)}\tbrac{\weightfunc(\edge)} = \lighttriangles{\frac{\lowerthreshold\arboricity}{\approxerror}}$ using this oracle on the sampled edges. However, no such oracle exist in our model. Hence, we try to simulate one through an empirical estimate of the weight function, $\empweightfunc(\edge)$. Recall the fact that there are many valid weight functions, each being characterized by every light triangle being charged to a unique edge. Our algorithm will work if the empirical weight function $\empweightfunc$ estimates any one of these weight functions. We achieve this by ensuring that a triangle is not sampled through more than one edge (see Line~\ref{Line: Remove Duplicate Triangles} of Algorithm~\ref{Algorithm: Random Edge Arboricity Triangle Counting Oracle Triangle Estimate}). Thus, the assignments that we consider can be extended to a valid weight function $\weightfunc$, and our algorithm can be thought of as estimating this weight function through $\empweightfunc$. We develop our initial algorithm ( Algorithm \ref{Algorithm: Random Edge Arboricity Triangle Counting Oracle Triangle Estimate}) assuming access to an estimate of $\numtriangle$ as $\esttriangle$ satisfying the following assumption:
\begin{assumption}\label{Assumption: Triangle 2 Factor Estimate}
$\esttriangle \leq 2\numtriangle$.
\end{assumption}
In Algorithm~\ref{Algorithm: Random Edge Arboricity Triangle Counting Oracle Triangle Estimate}, we assume that we know $m$ exactly. In fact, our algorithm and analysis work even if we have a constant factor approximation of $m$. This can be achieved by using $O(1)$  queries \cite{assadi2018simple}.

\begin{algorithm}[ht!]
    \caption{Triangle Counting Algorithm - with Oracle Access, and $\esttriangle$}\label{Algorithm: Random Edge Arboricity Triangle Counting Oracle Triangle Estimate}
    \begin{algorithmic}[1]
        \Require \degreeq{}, \neighbourq{}, \edgeexistsq{}, and \randedgeq{} query access to a graph $\graph$. Parameters $\esttriangle, \arboricity$, $\approxerror$, $\edgecount$ and oracle access to \exactheavyoracle{} with threshold constants $\lowerthreshold,\upperthreshold$
        \State $\edgesamplesize \gets 4\constant(1+\upperthreshold)\approxerror^{-3}(\edgecount\arboricity/\esttriangle)\log\vertexcount$ 
        \State $\samplededges \gets \emptyset$ \Comment{$\samplededges$ is the set of random edges sampled. $\size{\samplededges} \leq \edgesamplesize$ growing upto $\edgesamplesize$}
        \State $\sampledtriangles \gets \emptyset$ \Comment{$\sampledtriangles$ is the set of light triangles sampled through the edges in $\samplededges$}
        \For{$i \in [\edgesamplesize]$}
            \State $\edge_i \gets \randedgeq{}$
            \State $\samplededges \gets \samplededges \cup \edge_i$
            \State Let $\edge_i = (\vertex_i, x)$ where $\degree{\vertex_i} < \degree{x}$
            \remove{Let $\vertex_i$ be the endpoint of $\edge_i$ with smaller degree, and $x$ be the endpoint of $\edge_i$ that is not $\vertex_i$.} \Comment{Requires two \degreeq{} queries}
            \If{($\exactheavyoracle(\edge_i,\arboricity,\approxerror) = 0$)} \Comment{$\edge_i$ is a $\frac{\lowerthreshold\arboricity}{\approxerror}$-light edge} 
            \label{line: heavyoracle call}
                \State $\querycount_{\edge_i} \gets 0$ \Comment{$\querycount_\edge$ denotes the number of queries for each edge $\edge$}
                %\complain{
                \If{$\degree{\vertex_i} \leq \arboricity$}
                    \State set $\querycount_{\edge_i} \gets 1$ with probability $\frac{\degree{\vertex_i}}{\arboricity}$
                \Else 
                    \State set $\querycount_{\edge_i} \gets \ceil{\frac{\degree{\vertex_i}}{\arboricity}}$
                \EndIf
                %}
                %\State If $\degree{\vertex_i} \leq \arboricity$, set $\querycount_{\edge_i} \gets 1$ with probability $\frac{\degree{\vertex_i}}{\arboricity}$. Otherwise, set $\querycount_{\edge_i} \gets \ceil{\frac{\degree{\vertex_i}}{\arboricity}}$
            \For{$j \in [\querycount_{\edge_i}]$}
                \State Choose $k \gets \uniform\fbrac{\sbrac{1,2,...,\degree{\vertex_i}}}$
                \State $\altvertex \gets \neighbourq{\fbrac{\vertex_i,k}}$
                    % \State If $\edgeexistsq{\fbrac{\altvertex,x}} = 1$, $\sampledtriangles \gets \sampledtriangles \cup (\altvertex,\edge_i)$ 
                    %\State If($\edgeexistsq{\fbrac{\altvertex,x}} = 1$ and $\triangle = (\altvertex,\edge_i)$ is not in $\samplededges$ through another edge $\edge'$) $\sampledtriangles \gets \sampledtriangles \cup \triangle$ \label{Line: Remove Duplicate Triangles} \Comment{A triangle $\triangle$ may occur in $\sampledtriangles$ multiple times, but each time it will be through same edge $\edge$}
                \If{($\edgeexistsq{\fbrac{\altvertex,x}} = 1$ and $\triangle = (\altvertex,\edge_i)$ is not in $\samplededges$ through another edge $\edge'$)}
                \label{Line: Remove Duplicate Triangles} 
                    \State $\sampledtriangles \gets \sampledtriangles \cup \triangle$ \\
                    \Comment{A triangle $\triangle$ may occur in $\sampledtriangles$ multiple times, but each time it will be through same edge $\edge$}
                    %\iffalse{\gopi{May be we are allowing repetition of triangles encountered while processing a single (random) edge?}\debarshi{Fixed?}}
                    %\fi
                \EndIf
            \EndFor
            \Else
                \State $\querycount_{\edge_i} \gets 0$
            \EndIf
        \EndFor
        % \State If a triangle $\triangle$ is present in $\sampledtriangles$ through different edges, choose one edge arbitrarily, and remove the triangle through the other edges.
\iffalse{\For{$\edge \in \samplededges$}
            \If{$\querycount_\edge > 0$}
                \State $\empweightfunc(\edge) = \frac{1}{\querycount_{\edge}} \sum_{(\triangle,\edge) \in \sampledtriangles} \max\fbrac{\arboricity,\degree{\edge}}$
            \Else
                \State $\empweightfunc\fbrac{\edge} = 0$
            \EndIf
            % \State $Y_\edge = \frac{1}{\querycount_{\edge}} \empweightfunc(\edge)$
        \EndFor
        % \State Choose a consistent weight function $\weightfunc$. Compute the empirical weight function as $\empweightfunc(e) = \weightfunc_\samplededges(e)$
        \State \Return $\emptriangle = \frac{\edgecount}{\edgesamplesize}\sum_{\edge \in \samplededges} \empweightfunc(\edge)$}
\fi 
        \For{$i \in [\edgesamplesize]$}
            \If{$\querycount_{\edge_i} > 0$}
                \State $\empweightfunc(\edge_i) = \frac{1}{\querycount_{\edge_i}} \sum_{(\triangle,\edge_i) \in \sampledtriangles} \max\fbrac{\arboricity,\degree{\edge_i}}$
            \Else
                \State $\empweightfunc\fbrac{\edge_i} = 0$
            \EndIf
            % \State $Y_\edge = \frac{1}{\querycount_{\edge}} \empweightfunc(\edge)$
        \EndFor
        % \State Choose a consistent weight function $\weightfunc$. Compute the empirical weight function as $\empweightfunc(e) = \weightfunc_\samplededges(e)$
        \State \Return $\emptriangle = \frac{\edgecount}{\edgesamplesize}\sum_{i \in \edgesamplesize} \empweightfunc(\edge_i)$
    \end{algorithmic}
\end{algorithm}
\begin{lemma}\label{lemma: E[Y_I] Weight Func Algo}
    Algorithm~\ref{Algorithm: Random Edge Arboricity Triangle Counting Oracle Triangle Estimate} ensures that $\Exp\tbrac{\empweightfunc(\edge_i)} = \lighttriangles{\frac{\lowerthreshold\arboricity}{\approxerror}}/\edgecount $, and $\Exp\tbrac{\emptriangle} = \lighttriangles{\frac{\lowerthreshold\arboricity}{\approxerror}}$.
    \begin{proof}
    Let $\sE_i$ be the event that the edge $\edge$ is chosen in the $i$-th round. We proceed with the proof by considering two different cases: $\degree{\edge} < \arboricity$ and $\degree{\edge} \geq \arboricity$. 
    
    % \medskip
    
    {\bf Case I }($\degree{\edge} < \arboricity$):
    When $\degree{\edge} < \arboricity$, $\querycount_\edge$ is set to $0$ with probability $1 - \degree{\edge}/\arboricity$ and \complain{1} with probability $\degree{\edge}/\arboricity$, and $\empweightfunc(\edge)$ is evaluated through a single query. Thus, \begin{align}
        \Exp[\empweightfunc(\edge_i)|\sE_i] &= \frac{\degree{\edge}}{\arboricity} \sum_{k \in [\weightfunc(\edge)]} \frac{1}{\degree{\edge}} \arboricity + \fbrac{1 - \frac{\degree{\edge}}{\arboricity}}\cdot0 = \weightfunc(\edge)\label{Eq: E[what(e)] low degree edge}
    \end{align}

    % \medskip
     
    {\bf Case II }($\degree{\edge} \geq \arboricity$):
    On the other hand, when $\degree{\edge} \geq \arboricity$, let $Z_j, j \in [\querycount]$ denote the contribution of each of the $\querycount$ queries made for the edge $\edge$ to the weight function estimate. Then, we have:
    \begin{align}
        \Exp[Z_j|\sE_i] &= \sum_{k \in [\weightfunc(\edge)]} \frac{1}{\degree{\edge}} \degree{\edge}= \weightfunc(\edge)\label{Eq: E[Z_j] random edge arboricity}
    \end{align}
    Correspondingly, we have by linearity of expectation and Equation~\ref{Eq: E[Z_j] random edge arboricity}:
    \begin{align}
        \Exp\tbrac{\empweightfunc(\edge_i)|\sE_i} = \frac{1}{\querycount_\edge}\sum_{j \in [\querycount_{\edge}]} \Exp\tbrac{Z_j|\sE_i} = \Exp\tbrac{Z_j|\sE_i} = \weightfunc(\edge)\label{Eq: E[what(e)] high degree edge}
    \end{align}

    Given that we draw each edge $\edge \in \samplededges$ uniformly at random, we now have by Equations~\ref{Eq: E[what(e)] low degree edge},~\ref{Eq: E[what(e)] high degree edge}, and Lemma~\ref{Lemma: Weight Function Expectation}:
    \begin{align*}
        \Exp\tbrac{\empweightfunc(\edge_i)} = \Exp_{\edge \sim \uniform\fbrac{\edgeset}} \weightfunc\fbrac{\edge} = \lighttriangles{\frac{\lowerthreshold\arboricity}{\approxerror}}/\edgecount 
    \end{align*}
    By linearity of expectations, we obtain
    \begin{align*}
        \Exp\tbrac{\emptriangle} = \Exp\tbrac{\frac{\edgecount}{\edgesamplesize}\sum_{i \in \edgesamplesize} \empweightfunc(\edge_i)} = \lighttriangles{\frac{\lowerthreshold\arboricity}{\approxerror}} 
    \end{align*}

    \end{proof}
\end{lemma}

Note that Lemma~\ref{lemma: E[Y_I] Weight Func Algo} holds irrespective of whether Assumption~\ref{Assumption: Triangle 2 Factor Estimate} is satisfied or not. Next, we turn our attention to the variance of $\empweightfunc(\edge)$.

\begin{lemma}\label{lemma: Var[Y_i] Weight Func Algo}
    Algorithm~\ref{Algorithm: Random Edge Arboricity Triangle Counting Oracle Triangle Estimate} ensures that $\Var[\empweightfunc(\edge_i)] \leq \frac{(1+\upperthreshold)\arboricity}{\approxerror} \cdot \Exp_{\edge \sim \uniform\fbrac{\edgeset}}[\weightfunc(\edge)]$.
    \begin{proof}
        Again, we consider two different cases as in the proof of Lemma \ref{lemma: E[Y_I] Weight Func Algo}.  \remove{$\degree{\edge} \geq \arboricity$ and $\degree{\edge} < \arboricity$.}
        
        % \medskip
      
      {\bf Case I }($\degree{\edge} < \arboricity$):
        When $\degree{\edge} < \arboricity$, Algorithm~\ref{Algorithm: Random Edge Arboricity Triangle Counting Oracle Triangle Estimate} makes at most $1$ query as $\querycount \leq 1$. As $\empweightfunc(\edge) \leq \arboricity$, we have 
        \begin{align}
            \Var[\empweightfunc(\edge_i)|\sE_i] \leq \Exp[\empweightfunc(\edge_i)^2|\sE_i] \leq \arboricity\Exp\tbrac{\empweightfunc\fbrac{\edge_i}|\sE_i} \label{Eq: Var[Y_i,S, Small Degree Edge] random edge arboricity}
        \end{align}
        
        \remove{
        $$\leq \frac{\upperthreshold\arboricity}{\approxerror}\Exp[\empweightfunc\fbrac{\edge_i}|\sE_i] $$
        \begin{align*}
            \Var[\empweightfunc(\edge_i)] &\leq \Exp[\empweightfunc(\edge_i)^2]\\
                        &\leq \arboricity\Exp\tbrac{\empweightfunc\fbrac{\edge}}&\text{As $\empweightfunc(\edge) \leq \arboricity$}\\
                        &\leq \frac{\upperthreshold\arboricity}{\approxerror}\Exp[\empweightfunc]
                        % &\text{As $\empweightfunc \leq \frac{\upperthreshold\arboricity}{\approxerror}$}
        \end{align*}
        }
        
        {\bf Case II }($\degree{\edge} \geq \arboricity$):
        Now we consider the case $\degree{\edge} \geq \arboricity$ which is more involved as $\querycount \geq 1$. We first estimate the variance of each $Z_j$ individually as defined in Lemma~\ref{lemma: E[Y_I] Weight Func Algo}. We condition on the event that edge $\edge$ is chosen in the $i$-th stage, denoted by $\sE_i$. If $\exactheavyoracle(\edge,\arboricity,\approxerror) = 1$, then $Var[Z_j] = 0, \forall j \in [\querycount]$. Hence, we condition on the event that $\exactheavyoracle(\edge,\arboricity,\approxerror) = 0$ from now on. As $Z_j \leq \degree{\edge}$,
        \begin{align}
             Var[Z_j|\sE_i]  \leq \Exp[Z_j^2|\sE_i] 
                                \leq \degree{\edge_i}\Exp[Z_j|\sE_i] \label{Eq: Var[Z_j]}
        \end{align}
        \remove{
        \begin{align}
            \nonumber Var[Z_j|\sE_i]    &\leq \Exp[Z_j^2|\sE_i]&\\
                                &\leq \degree{\edge}\Exp[Z_j|\sE_i] &\text{$Z_j \leq \degree{\edge}$}\label{Eq: Var[Z_j]}
        \end{align}
        }
        After having bounded the variance of the contribution of each query, we now obtain the variance of the weight estimate of the edge.
        \begin{align}
          \nonumber  \Var[\empweightfunc(\edge_i)|\sE_i]   &=\Var\tbrac{\frac{1}{\querycount_{\edge_i}}\sum_{j \in \querycount_{\edge_i}} Z_j|\sE_i}&\\
          \nonumber                      &=\frac{1}{\querycount_{\edge_i}^2}\sum_{j \in \querycount_{\edge_i}}\Var\left[ Z_j|\sE_i \right]&\text{($Z_j$ are i.i.d. given $\sE_i$)}\\
          \nonumber                      &=\frac{\degree{\edge_i}}{\querycount_{\edge_i}}\sum_{j \in \querycount_{\edge_i}}\frac{1}{\querycount_{\edge_i}}\Exp[ Z_j|\sE_i ]&\text{(by Equation~\ref{Eq: Var[Z_j]} and linearity of expectation)}\\
          &\leq \arboricity \Exp[\empweightfunc(\edge_i)|\sE_i]&\text{ $\left(\mbox{as }  \querycount_{\edge_i} = \ceil{\frac{\degree{\edge}}{\arboricity}}\right)$}\label{Eq: Var[Y_i,S,High Degree Edge] random edge arboricity}
\end{align}

        Now, we remove the conditioning on $\sE_i$ using law of total variance:
        \begin{align*}
            \Var[\empweightfunc(\edge_i)] &= \Exp_{\edge_i}[\Var[\empweightfunc(\edge_i)|\sE_i]] + \Var_{\edge_i}[\Exp[\empweightfunc(\edge_i)|\sE_i]] &\text{(by law of total variance)}\\
            &\leq \Exp_{\edge_i}[\arboricity \Exp[\empweightfunc(\edge_i)|\sE_i]] + \Exp_{\edge_i} [\Exp[\empweightfunc(\edge_i)|\sE_i]^2]&\text{(by Equation~\ref{Eq: Var[Y_i,S, Small Degree Edge] random edge arboricity} and~\ref{Eq: Var[Y_i,S,High Degree Edge] random edge arboricity})}\\
            &\leq \arboricity \Exp_{\edge_i}[\Exp[\empweightfunc(\edge_i)|\sE_i]] + \frac{\upperthreshold\arboricity}{\approxerror} \cdot \Exp_{\edge_i} [\Exp[\empweightfunc(\edge_i)|\sE_i]]&\text{(as $\exactheavyoracle(\edge) = 0$, $\frac{\upperthreshold\arboricity}{\approxerror}\geq |T_e|$)}\\
            &\leq \frac{(1+\upperthreshold)\arboricity}{\approxerror} \cdot \Exp[\weightfunc(\edge)] 
        \end{align*}
    \end{proof}
\end{lemma}
\begin{theorem}\label{Theorem: Oracle Triangle Estimate ALgo Works}
    Algorithm~\ref{Algorithm: Random Edge Arboricity Triangle Counting Oracle Triangle Estimate} makes $36\constant(1+\upperthreshold)\approxerror^{-3} (\edgecount\arboricity/\esttriangle)\log\vertexcount$ queries, and $4\constant(1+\upperthreshold)\approxerror^{-3} (\edgecount\arboricity/\esttriangle)$ $\log\vertexcount$ calls to \exactheavyoracle{}, and given $\esttriangle$ satisfying Assumption~\ref{Assumption: Triangle 2 Factor Estimate},  returns $\emptriangle$ such that $\Pr(|\emptriangle-\lighttriangles{\frac{\lowerthreshold\arboricity}{\approxerror}}|\geq \approxerror\lighttriangles{\frac{\lowerthreshold\arboricity}{\approxerror}}) \leq \frac{1}{\constant\log \vertexcount}$.
\end{theorem}

\begin{proof}
        The algorithm calls \exactheavyoracle{} for each of the $\edgesamplesize$ edges in $\samplededges$, resulting in a total of $4\constant(1+\upperthreshold)\approxerror^{-3}\log(\vertexcount)(\edgecount\arboricity/\esttriangle)$ calls. 

        The algorithm makes $\edgesamplesize$ \randedgeq{} queries, $2\edgesamplesize$ \degreeq{} queries, and $2\querycount_\edge$ \neighbourq{} query for each edge $\edge \in \samplededges$. All edges in $\samplededges$ are sampled uniformly at random from $\edgeset$. Hence, the expected number of \neighbourq{} queries made are: 
        
        $$\Exp_{\edge \sim \uniform\fbrac{\edgeset}} \left[ \ceil{\frac{\degree{\edge}}{\arboricity}} \right] = \sum_{e \in E} \frac{1}{m} \ceil{\frac{\degree{\edge}}{\arboricity}} \leq \frac{1}{m} \sum_{e \in E} 1 + \frac{\degree{\edge}}{\arboricity} = 1 + \frac{1}{m} \sum_{e \in E} \frac{\degree{\edge}}{\arboricity} \leq 1 + \frac{2m\arboricity}{m\arboricity} = 3$$
        The last inequality in the above step follows from Lemma~~\ref{Lemma: deg(e) sum is m * arboricity}.
        \remove{
        \begin{align*}
            &\Exp_{\edge \sim \uniform\fbrac{\edgeset}} \ceil{\frac{\degree{\edge_i}}{\arboricity}}\\
            =&\sum_{e \in E} \frac{1}{m} \ceil{\frac{\degree{\edge}}{\arboricity}}\\
            \leq&\frac{1}{m} \sum_{e \in E} 1 + \frac{\degree{\edge}}{\arboricity}\\
            =&1 + \frac{1}{m} \sum_{e \in E} \frac{\degree{\edge}}{\arboricity}\\
            \leq& 1 + \frac{2m\arboricity}{m\arboricity}&\text{By Lemma~\ref{lemma: arboricity triangle bound}}\\
            =& 3
        \end{align*}
        }
        Hence, the algorithm makes at most $9\constant\edgesamplesize = 36\constant(1+\upperthreshold)\approxerror^{-3}(\edgecount\arboricity/\esttriangle) \log \vertexcount$ queries in expectation. Here the inequality is due to the fact that for a heavy edge $\edge$ in $\samplededges$, $\querycount_\edge = 0$. Furthermore, we bound the variance of the estimate $\emptriangle$ as:
        
        % \red{By Lemma~\ref{lemma: E[Y_I] Weight Func Algo}, we have $\Exp\tbrac{\emptriangle} = \lighttriangles{\frac{\lowerthreshold\arboricity}{\approxerror}}$} \complain{(Debarshi, please check if it is $\lighttriangles{\lowerthreshold}$. I think there is something wrong with the superscript.)}\debarshi{Fixed it, but I think \red{this} line is unnecessary.}. 
        
        $$\Var[\emptriangle] = \Var[\frac{\edgecount}{\edgesamplesize} \sum_{i \in s} \empweightfunc(\edge_i)] \leq \frac{(1+\upperthreshold)\arboricity\edgecount^2\Exp[\weightfunc(\edge)]}{\approxerror \edgesamplesize} \leq \frac{(1+\upperthreshold)\edgecount\arboricity\Exp[\emptriangle]}{\approxerror \edgesamplesize}.$$
        The last two steps follow from Lemma~\ref{lemma: Var[Y_i] Weight Func Algo} and the fact that $\Exp\tbrac{\emptriangle} = \edgecount\Exp\tbrac{\weightfunc(\edge)}$.\remove{
        \begin{align*}
            \Var[\emptriangle] &= \Var[\frac{\edgecount}{\edgesamplesize} \sum_{i \in s} \empweightfunc(\edge)]\\
                    &\leq \frac{(1+\upperthreshold)\arboricity\edgecount^2\Exp[\empweightfunc(\edge)]}{\approxerror \edgesamplesize} &\text{Lemma \ref{lemma: Var[Y_i] Weight Func Algo}}\\
                    &\leq \frac{(1+\upperthreshold)\edgecount\arboricity\Exp[\emptriangle]}{\approxerror \edgesamplesize} &\Exp\tbrac{\emptriangle} = \edgecount\Exp\tbrac{\empweightfunc(\edge)}
        \end{align*}
        }
        We now use Chebyshev's inequality on $\emptriangle$:
        \begin{align*}
            \Pr(|\emptriangle - \Exp[\emptriangle]| \leq \approxerror \Exp[\emptriangle]) 
                    &\leq \frac{\Var[\emptriangle]}{\approxerror^2\Exp[\emptriangle]^2}   &\text{(Chebyshev's inequality)}\\
                    &\leq \frac{(1+\upperthreshold)\edgecount\arboricity}{\approxerror^3\edgesamplesize\Exp[\emptriangle]}&\text{(by Lemma \ref{lemma: Var[Y_i] Weight Func Algo})}\\
                    &\leq \frac{(1+\upperthreshold)\edgecount\arboricity}{\approxerror^3\cdot 4\constant\fbrac{1+\upperthreshold}\approxerror^{-3}(\edgecount\arboricity/\esttriangle)\log\vertexcount \cdot \lighttriangles{\frac{\lowerthreshold\arboricity}{\approxerror}}} &\text{(by Lemma \ref{lemma: E[Y_I] Weight Func Algo})}\\
                    & \leq \frac{1}{\constant\log \vertexcount} &\text{(as $\lighttriangles{\frac{\lowerthreshold\arboricity}{\approxerror}} > \numtriangle/2 \geq \esttriangle/4$)}
        \end{align*}
    \end{proof}
%\gopi{\st{May be we summarize the conclusion from this section as a Lemma (so that we can use it later as a blackbox). Also, this section is good for building intuition. But this is not directly useful in the final algorithm later. If possible, may be we consider to rewrite it assuming access to an approximate Heavy-Oracle so that we can use the result as a blackbox later.}}\debarshi{Done.}
% \todo{$\approxerror^3$ is optimal?}








%-----------------------------New Subsection-----------------------------








%Section 3 - Implementing the Oracle

\subsection{Implementing the Oracle}
\label{ssec:oracle-implement}
Rather than the exact oracle (\exactheavyoracle) that we assumed in Algorithm~\ref{Algorithm: Random Edge Arboricity Triangle Counting Oracle Triangle Estimate}, we would design an oracle (called \heavyoracle) that given arboricity $\arboricity$, and parameters $\approxerror$ and $\confidence$, accepts edges participating in at most $\frac{\arboricity}{2\approxerror}$ triangles and rejects edges participating in at most $\frac{2\arboricity}{\approxerror}$ triangles with probability $1 - \confidence$. The algorithm works by estimating the number of triangles each edge participates in. However, in this case, each \neighbourq{} and \edgeexistsq{} generates i.i.d. random variables. Thus, our analysis uses multiplicative Chernoff bound to obtain high probability guarantees for each individual edge.

% \complain{Make $\querycount$ floor.}\debarshi{Done!}
\begin{algorithm}[ht!]
    \caption{\heavyoracle($\edge$,$\arboricity$,$\approxerror$,$\confidence$)}\label{Algorithm: Heavy Oracle}
    \begin{algorithmic}[1]
        \Require \degreeq{}, \neighbourq{}, \edgeexistsq{}, and \randedgeq{} query access to a graph $\graph$
        % \State $\degree{\edge} \gets degree(u)$, where $u$ is the endpoint of $e$ with smaller degree.
        \State $\querycount \gets \ceil{\frac{16\approxerror ~ \degree{\edge}}{\arboricity}\log\fbrac{\frac{1}{\confidence}}}$ \Comment{$\querycount$ denotes the number of queries for each edge $\edge$}
        % \State $\altvertex \gets$ Smaller degree vertex of $\edge$ 
        \State Let $\edge = (\altvertex, x)$ where $\degree{\altvertex} < \degree{x}$
        \Comment{Requires 2 \degreeq{} queries}
        \State $Y \gets 0$
        \For{$i \in [\querycount]$}
            \State Choose $k \gets \uniform\fbrac{\sbrac{1,2,...,\degree{\altvertex}}}$
            \State $\vertex_i \gets \neighbourq(\altvertex,k)$ 
            \Comment{Requires 1 \neighbourq{} query}
            % \State If $\vertex_i$ and $\edge$ form a triangle, $Y_i = 1$, else $Y_i = 0$ 
            \If{$\edgeexistsq{\fbrac{\vertex_i,x}} = 1$} \Comment{Requires 1 \edgeexistsq{} query}
                \State $Y_i \gets 1$ 
                \Comment{Found triangle $\fbrac{\vertex_i,\edge}$ }
            \Else
                \State $Y_i \gets 0$
            \EndIf
            \State $Y  \gets Y + Y_i$
        \EndFor
        \State $Y \gets \frac{1}{\querycount}Y$
        % \State $Y = \sum_{i \in \querycount} Y_i$
        \If{$Y \geq \frac{\arboricity}{\approxerror\degree{\edge}}$} 
            \State \Return 1
        \Else
            \State \Return 0
        \EndIf
    \end{algorithmic}
\end{algorithm}
% \vspace{-0.05in}
\begin{lemma}\label{Lemma: Heavy Oracle Algorithm Correctness}
    The algorithm $\heavyoracle{\fbrac{\edge,\arboricity,\approxerror,\confidence}}$ satisfies the following properties with probability at least $1-\confidence$: (i) rejects edge $\edge$ if it is $\frac{2\arboricity}{\approxerror}$-heavy; (ii) accepts edge $\edge$ if it is $\frac{\arboricity}{2\approxerror}$-light.
   \remove{
    \begin{itemize}
        \item rejects edge $\edge$ if it is $\frac{2\arboricity}{\approxerror}$-heavy.
        \item accepts edge $\edge$ if it is $\frac{\arboricity}{2\approxerror}$-light.
    \end{itemize}
    }
\end{lemma}

\begin{proof}
    For (i), we only consider edges that have at least $\frac{2\arboricity}{\approxerror}$ triangles. In this case, the random variables $Y_i$ ($Y_i$ as in Algorithm~\ref{Algorithm: Heavy Oracle}; $Y_i=1$ if $\vertex_i$ and $\edge$ form a triangle; $0$, otherwise) are i.i.d. Bernoulli random variables taking value $1$ with probability at least $\frac{2\arboricity}{\approxerror ~ \degree{\edge}}$. Hence, we have the following using linearity of expectation:  
    % $\Exp\tbrac{Y} = \Exp\tbrac{\frac{1}{\querycount}\sum_{i \in \tbrac{\querycount}} Y_i}$. 
    $\Exp\tbrac{Y} = \Exp\tbrac{\frac{1}{\querycount}\sum_{i \in \tbrac{\querycount}} Y_i} = \Exp\tbrac{Y_i} > \frac{2\arboricity}{\approxerror ~ \degree{\edge}} $
   \remove{
    \begin{align*}
        &\Exp\tbrac{Y}\\
        =&\Exp\tbrac{\frac{1}{\querycount}\sum_{i \in \tbrac{\querycount}} Y_i}\\
        =&\Exp\tbrac{Y_i} &\text{By linearity of expectations}\\
        >& \frac{2\arboricity}{\approxerror\degree{\edge}}
    \end{align*}
    }
    
    As $Y=\frac{1}{\querycount}\sum_i Y_i$, we can upper bound the probability of the algorithm returning $0$ for the edge $\edge$, by a multiplicative Chernoff bound (Lemma~\ref{Lemma: Multiplicative Chernoff Bound}) as follows:
    \begin{align*}
        \Pr\tbrac{Y \leq \frac{\arboricity}{\approxerror\degree{\edge}}} \remove{\complain{\mbox{(Debarshi: why is this } \frac{\arboricity}{\approxerror} \mbox{ and not } \frac{2 \arboricity}{\approxerror}?)}}
        \leq& \Pr\tbrac{Y \leq \fbrac{1-\frac{1}{2}}\Exp\tbrac{Y}} &\left( \Exp\tbrac{Y} > \frac{2\arboricity}{\approxerror\degree{\edge}} \right)\\
        \leq&\exp{\fbrac{-\frac{\querycount\Exp\tbrac{Y}}{12}}} &\text{(by multiplicative Chernoff bound)}\\
        \leq&\exp{\fbrac{-\frac{\querycount\arboricity}{6\degree{\edge}\approxerror}}} &\left( \Exp\tbrac{Y} > \frac{2\arboricity}{\approxerror\degree{\edge}}\right)\\
        \leq & ~ \confidence &\left( \querycount = \frac{16\approxerror \degree{\edge}}{\arboricity}\log\fbrac{\frac{1}{\confidence}}\right)\\
    \end{align*}
    For (ii), we do not have a lower bound on the probability of success $(Y_i = 1)$ in general and hence we cannot obtain a lower bound on $\Exp\tbrac{Y}$. Now, observe that for the edges that participate in very few triangles, the probability of finding triangles (i.e. getting $Y_i = 1$) is low. However, such light edges can still tolerate a high approximation error to be accepted (as a light edge). To account for the trade-off between the lower bound on the probability of $Y_i = 1$ and the upper bound on the approximation factor,
    \remove{However, observe that the edges that participate in very few triangles, and hence has a low probability of finding triangles (i.e. getting $Y_i = 1$) also can tolerate a high approximation error to still be accepted (as a light edge). To account for the trade-off between the lower bound on the probability of $Y_i = 1$ and the upper bound on the approximation factor, (Debarshi: too complicated! can you simplify?)}we divide the edges participating in at most $\frac{\arboricity}{2\approxerror}$ triangles into $\tbrac{\ceil{\log(\frac{\arboricity}{\approxerror})}}$ buckets with each bucket being defined as the set of edges $\bucket_k = \sbrac{\edge|\frac{\arboricity}{2^{k+1}\approxerror}\leq\numtriangle_\edge < \frac{\arboricity}{2^k\approxerror}}$. For each of these buckets, observe that when \heavyoracle{} is called for an edge belonging to the bucket, we have $\Pr\tbrac{Y_i = 1} = \frac{\triangle_\edge}{\degree{\edge}} \geq \frac{\arboricity}{2^{k+1}\approxerror\degree{\edge}}$, and hence $\Exp\tbrac{Y} = \Exp\tbrac{Y_i} \geq \frac{\arboricity}{2^{k+1}\approxerror\degree{\edge}}$.
    
   \remove{ \red{Similarly, from the other side, we have $\Exp\tbrac{Y} < \frac{\arboricity}{2^{k}\approxerror\degree{\edge}}$, and hence $\fbrac{1 + 2^k}\Exp\tbrac{Y} < \frac{\arboricity}{\approxerror\degree{\edge}}$. Using these observations, we complete the proof by considering any $\edge$ in these buckets:
    \todo{Approxerror can be at most $2^(k-1)$?}
    \begin{align*}
        \Pr\tbrac{Y \geq \frac{\arboricity}{\approxerror\degree{\edge}}} 
        \leq&\Pr\tbrac{Y \geq \fbrac{1+2^k}\Exp\tbrac{Y}} &\left( \fbrac{1 + 2^k}\Exp\tbrac{Y} < \frac{\arboricity}{\approxerror\degree{\edge}}\right)\\
        \leq& \exp{\fbrac{-\frac{2^{2k}\querycount\Exp\tbrac{Y}}{2+2^k}}} &\text{(by multiplicative Chernoff bound)}\\
        \leq& \exp{\fbrac{-\frac{2^{2k}\querycount\Exp\tbrac{Y}}{2^{k+1}}}} &(k \geq 1)\\
        \leq& \exp{\fbrac{-\frac{\querycount\arboricity}{4\approxerror\degree{\edge}}}}&\left( \Exp\tbrac{Y} > \frac{\arboricity}{2^{k+1}\approxerror\degree{\edge}} \right)\\
        \leq&~ \confidence & \left( \querycount \geq \frac{6\approxerror \degree{\edge}}{\arboricity}\log\fbrac{\frac{1}{\confidence}} \right)\\
    \end{align*}
    }}

    On the other hand, we have $\Exp\tbrac{Y} < \frac{\arboricity}{2^{k}\approxerror\degree{\edge}}$, and hence $\fbrac{1 + 2^{k-1}}\Exp\tbrac{Y} < \frac{\arboricity}{\approxerror\degree{\edge}}$. Using these observations, we complete the proof by considering any $\edge$ in these buckets:
    % \todo{Approxerror can be at most $2^(k-1)$?}
    \begin{align*}
        \Pr\tbrac{Y \geq \frac{\arboricity}{\approxerror\degree{\edge}}} 
        \leq&\Pr\tbrac{Y \geq \fbrac{1+2^{k-1}}\Exp\tbrac{Y}} &\left( \fbrac{1 + 2^{k-1}}\Exp\tbrac{Y} < \frac{\arboricity}{\approxerror\degree{\edge}}\right)\\
        \leq& \exp{\fbrac{-\frac{2^{2k-2}\querycount\Exp\tbrac{Y}}{2+2^{k-1}}}} &\text{(by multiplicative Chernoff bound)}\\
        \leq& \exp{\fbrac{-\frac{2^{2k}\querycount\Exp\tbrac{Y}}{2^{k+3}}}} &(k \geq 1)\\
        \leq& \exp{\fbrac{-\frac{\querycount\arboricity}{16\approxerror\degree{\edge}}}}&\left( \Exp\tbrac{Y} > \frac{\arboricity}{2^{k+1}\approxerror\degree{\edge}} \right)\\
        \leq&~ \confidence & \left( \querycount = \frac{16\approxerror \degree{\edge}}{\arboricity}\log\fbrac{\frac{1}{\confidence}} \right)
    \end{align*}
\end{proof}

Lemma~\ref{Lemma: Heavy Oracle Algorithm Correctness} shows that Algorithm~\ref{Algorithm: Heavy Oracle} can decide whether an edge is heavy or not with probability $1 - \confidence$ using $\frac{16\approxerror ~ \degree{\edge}}{\arboricity}\log\fbrac{\frac{1}{\confidence}}$ iterations; making 2 \degreeq{}, 1 \neighbourq{} and 1 \edgeexistsq{} queries in each iteration. For an individual edge, $\degree{\edge}$ can be at most $\vertexcount-1$, and hence the subroutine might contribute to additional queries for each edge. However, as part of Algorithm~\ref{Algorithm: Random Edge Arboricity Triangle Counting Oracle Triangle Estimate}, \heavyoracle{} is called on a set of edges drawn uniformly at random. Hence, in the following lemma, we bound the expected number of queries made by \heavyoracle{} when called on edges that were drawn uniformly at random.

\begin{lemma}\label{Lemma: Heavy Oracle Query Count}
    $\heavyoracle{\fbrac{\edge,\arboricity,\approxerror,\confidence}}$ makes $132\approxerror\log\fbrac{\frac{1}{\confidence}}$ queries in expectation on an edge $\edge \sim \uniform\fbrac{\edgeset}$. 
    
    % \complain{(Debarshi: the statement of the lemma is not making sense!)}
\end{lemma}

\begin{proof}
   For edge $\edge$, $\heavyoracle{\fbrac{\edge,\arboricity,\approxerror,\confidence}}$ makes $\ceil{\frac{16\approxerror \degree{\edge}}{\arboricity}\log\fbrac{\frac{1}{\confidence}}}$ iterations with each iteration making $4$ queries. Hence, the expected number of queries is:
   \begin{align*}
       \Exp_{\edge \sim \uniform\fbrac{\edgeset}}\tbrac{4\ceil{\frac{16\approxerror \degree{\edge}}{\arboricity}\log\fbrac{\frac{1}{\confidence}}}}
       \leq& \frac{64\approxerror}{\arboricity}\log\fbrac{\frac{1}{\confidence}}\Exp_{\edge \sim \uniform\fbrac{\edgeset}}\tbrac{\degree{\edge}}+ 4\\
       =& \frac{64\approxerror}{\arboricity}\log\fbrac{\frac{1}{\confidence}}\frac{1}{\edgecount}\sum_{\edge \in \edgeset} \degree{\edge}+4\\
       \leq& \frac{64\approxerror}{\arboricity}\log\fbrac{\frac{1}{\confidence}}\frac{1}{\edgecount}2\edgecount\arboricity + 4&\text{(by Lemma~\ref{Lemma: deg(e) sum is m * arboricity})}\\
       \leq& 132\approxerror\log\fbrac{\frac{1}{\confidence}} &\left( \approxerror\log\fbrac{\frac{1}{\confidence}} \geq 1 \right)
   \end{align*}
\end{proof}








%-----------------------------New Subsection-----------------------------








%Section 3 - Combining the algorithms

\subsection{The Final Algorithm}
\label{ssec:final-algo}
In this section, we put together everything. \remove{remove the assumption of access to \exactheavyoracle{} through using the \heavyoracle{} implementation developed in the previous section.} Our goal is to call \heavyoracle{} with appropriate parameters \remove{in an appropriate manner} instead of \exactheavyoracle{} so that the approximation error due to the heavy triangles not being counted remains sufficiently small. We also derive the query complexity of the algorithm including the queries made through \heavyoracle{}.

% \red{In this section, we develop the complete algorithm by removing the oracle and \complain{triangle estimate assumptions (Debarshi: do you remove it in this subsection?)} that we made for Algorithm~\ref{Algorithm: Random Edge Arboricity Triangle Counting Oracle Triangle Estimate}. We first note that the oracle to detect heavy edges \exactheavyoracle{} that we assumed for the algorithm cannot be implemented exactly. So, we replace it with the oracle call of \heavyoracle{} where we can detect heavy edges and light edges but with different thresholds. Hence, we implement the \heavyoracle{} subroutine in a manner so that the number of heavy triangles remain sufficiently small and we obtain the corresponding guarantees on heavy edges and query complexity of the algorithm.}


\begin{algorithm}[ht!]
    \caption{Triangle Counting Algorithm - with $\esttriangle$}\label{Algorithm: Random Edge Arboricity Triangle Counting Triangle Estimate}
    \begin{algorithmic}[1]
        \Require \degreeq{}, \neighbourq{}, \edgeexistsq{}, and \randedgeq{} query access to a graph $\graph$. Parameters $\esttriangle, \arboricity$, $\approxerror$, $\edgecount$ 
        % \State Call Algorithm~\ref{Algorithm: Random Edge Arboricity Triangle Counting Oracle Triangle Estimate} by replacing $\exactheavyoracle\fbrac{\edge,\arboricity,\approxerror}$ call with $\heavyoracle\fbrac{\edge,\arboricity,\approxerror/6,\frac{1}{\edgecount\vertexcount}}$, and parameters $\esttriangle, \arboricity$, $\approxerror/2$, and thresholds \complain{$\lowerthreshold = \frac{3}{2}$, and $\upperthreshold = 6$.}
        \State Call Algorithm~\ref{Algorithm: Random Edge Arboricity Triangle Counting Oracle Triangle Estimate} with parameters $\esttriangle, \arboricity$, $\approxerror/2$, and thresholds $\lowerthreshold = 6$, and $\upperthreshold = 24$. We also implement $\exactheavyoracle\fbrac{\edge,\arboricity,\approxerror}$ oracle call through $\heavyoracle\fbrac{\edge,\arboricity,\approxerror/6,\frac{1}{\edgecount\vertexcount}}$. \Comment{Due to the Algorithm~\ref{Algorithm: Random Edge Arboricity Triangle Counting Oracle Triangle Estimate} being called with parameter $\approxerror/2$ and the \heavyoracle{} implementation as stated, the threshold parameters are evaluated w.r.t. call to \heavyoracle{} with parameter $\lowerthreshold = 6$ and $\upperthreshold = 24$}
    \end{algorithmic}
\end{algorithm}

\begin{theorem}\label{Theorem: Triangle Estimate ALgo Works}
   Algorithm~\ref{Algorithm: Random Edge Arboricity Triangle Counting Triangle Estimate} makes at most
$\fbrac{5050+13200\approxerror\log\vertexcount}\fbrac{\approxerror^{-3} (\edgecount\arboricity/\esttriangle) \constant\log \vertexcount}$ queries in expectation and returns $\emptriangle$ such that 
$\Pr\tbrac{\emptriangle \in \tbrac{\fbrac{1-\approxerror}\numtriangle,\fbrac{1+\approxerror}\numtriangle}} \geq 1 - \frac{1}{\constant\log \vertexcount}$.
\remove{
    \begin{align*}
        \Pr\tbrac{\emptriangle \in \tbrac{\fbrac{1-\approxerror}\numtriangle,\fbrac{1+\approxerror}\numtriangle}} \geq 1 - \frac{1}{\log \vertexcount}
    \end{align*}
}
\end{theorem}

\begin{proof}
    Algorithm~\ref{Algorithm: Random Edge Arboricity Triangle Counting Triangle Estimate} will make the same number of calls to \heavyoracle{} as did Algorithm~\ref{Algorithm: Random Edge Arboricity Triangle Counting Oracle Triangle Estimate} to \exactheavyoracle{}. So, by Theorem~\ref{Theorem: Oracle Triangle Estimate ALgo Works}, Algorithm~\ref{Algorithm: Random Edge Arboricity Triangle Counting Triangle Estimate} makes $36\constant(1+\upperthreshold)\approxerror^{-3}(\edgecount\arboricity/\esttriangle)\log\vertexcount$ queries and $4\constant(1+\upperthreshold)\approxerror^{-3}\log(\vertexcount)$$(\edgecount\arboricity/\esttriangle)$ calls to $\heavyoracle{}$. By Lemma~\ref{Lemma: Heavy Oracle Query Count}, each call to $\heavyoracle{}$ requires $132\approxerror\log \vertexcount$ queries in expectation. Combining with the fact that we have $\upperthreshold = 24$, the algorithm requires $\fbrac{5050+13200\approxerror\log\vertexcount}\fbrac{\approxerror^{-3} (\edgecount\arboricity/\esttriangle) \constant\log \vertexcount}$ queries in expectation.

    By Lemma~\ref{Lemma: Heavy Oracle Algorithm Correctness}, the implementation of $\exactheavyoracle\fbrac{\edge,\arboricity,\approxerror} = \heavyoracle{\fbrac{\edge,\arboricity,\approxerror/6,\frac{1}{\edgecount\vertexcount}}}$  accepts a $\frac{6\arboricity}{\approxerror}$-light edge and rejects a $\frac{24\arboricity}{\approxerror}$-heavy edge with probability at least $1 - \frac{1}{\edgecount\vertexcount}$. By union bound, the algorithm accepts all $\frac{6\arboricity}{\approxerror}$-light edges and rejects all $\frac{24\arboricity}{\approxerror}$-heavy edges in $\samplededges$ with probability at least $1-\frac{1}{\vertexcount}$. Hence, by Corollary~\ref{Corollary: Lower Bound on Light Triangles} and Lemma~\ref{lemma: E[Y_I] Weight Func Algo}\remove{\complain{(Debarshi: it should need something more than Corollary~\ref{Corollary: Lower Bound on Light Triangles}?)}}, we have, 
    % \complain{(Debarshi: if you agree to the change in the constants of $\frac{\alpha}{\approxerror}$, then correct all occurrences of it.)}
    \begin{align}
    % \Pr\tbrac{\lighttriangles{\frac{6\arboricity}{\approxerror}} \in \tbrac{\fbrac{1-\approxerror/2}\numtriangle,\numtriangle}} \label{Eq:Final Light Triangle Bound}\\
    \Exp\tbrac{\emptriangle} = \lighttriangles{\frac{6\arboricity}{\approxerror}} \in \tbrac{\fbrac{1-\approxerror/2}\numtriangle,\numtriangle} \label{Eq:Final Light Triangle Bound}
    \end{align}

    By Theorem~\ref{Theorem: Oracle Triangle Estimate ALgo Works}, we have that:
    \begin{align}
        % \Pr\tbrac{\emptriangle \in \tbrac{\fbrac{1-\approxerror/2}\lighttriangles{\frac{6\arboricity}{\approxerror}},\fbrac{1+\approxerror/2}\lighttriangles{\frac{6\arboricity}{\approxerror}}}}\label{Eq: Final Estimate Bound not}\\
        \Pr\tbrac{\emptriangle \in \tbrac{\fbrac{1-\approxerror/2}\lighttriangles{\frac{6\arboricity}{\approxerror}},\fbrac{1+\approxerror/2}\lighttriangles{\frac{6\arboricity}{\approxerror}}}} \geq 1 - \frac{1}{\constant\log \vertexcount}\label{Eq: Final Estimate Bound}
    \end{align}

    Combining Equations~\ref{Eq:Final Light Triangle Bound} and~\ref{Eq: Final Estimate Bound}~and using union bound on the event that all oracle calls were executed correctly, for large enough $\vertexcount$:
    \begin{align*}
        \Pr\tbrac{\emptriangle \in \tbrac{\fbrac{1-\approxerror}\numtriangle,\fbrac{1+\approxerror}\numtriangle}} \geq 1 - \frac{1}{\constant\log \vertexcount}
    \end{align*}
\end{proof}

Using the usual techniques in property testing~\citep{Dana_Ron_Triangle_Counting,DBLP:conf/soda/EdenRS20,chakrabarti2015data,Goldreich_Ron_EdgeCounting}, we can obtain the following result. The details of the proof are deferred to the appendix.

\begin{theorem}\label{Theorem: Final Upper Bound}
    There exists an algorithm that makes $\bigot{\frac{\edgecount\arboricity\log\frac{1}{\confidence}}{\approxerror^3\numtriangle}}$ queries in expectation, and returns $X \in \tbrac{\fbrac{1-\approxerror}\numtriangle,\fbrac{1+\approxerror}\numtriangle}$ with probability at least $1 - \confidence$.
\end{theorem}


\subsection{Finalizing The Algorithm}

% \begin{algorithm}
%     \caption{Final Algorithm}
%     \begin{algorithmic}
%         \Require \randedgeq{}, \neighbourq{}, and \edgeexistsq{} query access to a graph $\graph$. Parameters $\arboricity$, $\approxerror$ 
%         \For{$\esttriangle$ in $\tbrac{\vertexcount^3,\vertexcount^3/2,...,1}$}
%             \For{$i \in \tbrac{\log\fbrac{\frac{6\log(\vertexcount)}{\confidence}}}$}
%                 \State $X_i \gets$ Solution to Algorithm~\ref{Algorithm: Random Edge Arboricity Triangle Counting Triangle Estimate} with parameters $\esttriangle,\arboricity,\approxerror$
%                 \State $X \gets \min_i X_i$
%                 \If{$X \geq \esttriangle$}
%                     \State \Return $X$
%                 \EndIf
%             \EndFor
%         \EndFor
%     \end{algorithmic}
% \end{algorithm}

In this section, we describe the steps to obtain Theorem~\ref{Theorem: Final Upper Bound} from Theorem~\ref{Theorem: Triangle Estimate ALgo Works}. First, we remove the assumption on the knowledge of $\esttriangle$ for the Algorithm~\ref{Algorithm: Random Edge Arboricity Triangle Counting Triangle Estimate}. To achieve that, we search for an appropriate choice of $\esttriangle$ starting from $\searchesttriangle = \vertexcount^3$ and halving it each time we fail to find a $\esttriangle$ satisfying Assumption~\ref{Assumption: Triangle 2 Factor Estimate}.
We must also bound the probability of $\searchesttriangle$ deviating significantly below $\numtriangle$. To ensure that, for each value of $\searchesttriangle$, we run Algorithm~\ref{Algorithm: Random Edge Arboricity Triangle Counting Triangle Estimate} using values of $\esttriangle$ as $\vertexcount^3,\vertexcount^3/2,...,\searchesttriangle$.

\begin{algorithm}
    \caption{Triangle Counting Algorithm}\label{Algorithm: Final Search}
    \begin{algorithmic}[1]
        \Require \degreeq{}, \neighbourq{}, \edgeexistsq{}, and \randedgeq{} query access to a graph $\graph$. Parameters $\arboricity$, $\approxerror$, $\edgecount$ 
        \For{$\searchesttriangle $ in $\tbrac{\vertexcount^3,\vertexcount^3/2,...,1}$}\label{Line: Bar T Range}
            \For{$\esttriangle$ in $\tbrac{\vertexcount^3,\vertexcount^3/2,...,\searchesttriangle }$} \label{Line: EstTriangle Range}
                \For{$i \in \tbrac{2\log\fbrac{\constant\log(\vertexcount)}}$}
                    \State $X_i \gets$ Solution to Algorithm~\ref{Algorithm: Random Edge Arboricity Triangle Counting Triangle Estimate} with parameters $\esttriangle,\arboricity,\approxerror$
                    \State $X \gets \min_i X_i$
                    \If{$X \geq \esttriangle$}
                        \State \Return $X$
                    \EndIf
                \EndFor
            \EndFor
        \EndFor
    \end{algorithmic}
\end{algorithm}

First we bound the probability that the Algorithm~\ref{Algorithm: Final Search} terminates at a choice of $\esttriangle$ that does not satisfy the Assumption~\ref{Assumption: Triangle 2 Factor Estimate}.

\begin{lemma}\label{Lemma: Final Search Wrong Termination Bound}
    When $\esttriangle > 2\numtriangle$, the algorithm terminates with probability at most $\frac{1}{\constant\log^2 \vertexcount}$.
    % When $\searchesttriangle  > 2\numtriangle$, the algorithm terminates with probability at most $\frac{1}{\constant\log\fbrac{\vertexcount}}$.
    
    % \debarshi{Should we split this into two lemmas?}
\end{lemma}

\begin{proof}
    By Lemma~\ref{lemma: E[Y_I] Weight Func Algo}, we have from Markov's inequality:
    \begin{align*}
        \Pr\tbrac{\emptriangle > 2\numtriangle} \leq \frac{\Exp\tbrac{\emptriangle}}{2\numtriangle} \leq \frac{1}{2}
    \end{align*}
    Now, we have $\Pr\tbrac{X \geq \esttriangle} \leq \Pr\tbrac{\cap_{i}X_i \geq \esttriangle} \leq \frac{1}{\constant\log^2 \vertexcount}$
\remove{
    \begin{align*}
        &\Pr\tbrac{X \geq \esttriangle}\\
        \leq &\Pr\tbrac{\bigcap\limits_i ~ X_i \geq \esttriangle}\\
        \leq &\frac{1}{\constant\log^2 \vertexcount}
    \end{align*}
    }
    % A union bound over all possible values of $\esttriangle$ in Line~\ref{Line: EstTriangle Range} completes the proof.
\end{proof}

% \begin{lemma}
%     When $\searchesttriangle  < 2\numtriangle$, the algorithm terminates with probability at least $?$
% \end{lemma}

% \begin{proof}
    
% \end{proof}

\begin{lemma}\label{Lemma: Final Search Low Bar T Bound}
    The algorithm reaches a value of $\searchesttriangle  \leq \numtriangle/2^k$ $\fbrac{k \geq 1}$ with probability at most $\frac{1}{\fbrac{\constant\log \vertexcount}^k}$.
\end{lemma}

\begin{proof}
    For every such value of $\searchesttriangle $, we have a $\esttriangle$ in Line~\ref{Line: EstTriangle Range} of Algorithm~\ref{Algorithm: Final Search} such that $\esttriangle \leq \numtriangle/2$. By Theorem~\ref{Theorem: Triangle Estimate ALgo Works}, the algorithm returns an estimate $X_i \geq \numtriangle/2 \geq \esttriangle$ with probability at least $1 - \frac{1}{\constant\log \vertexcount}$. Taking a union bound over possible values of $i \in \tbrac{2\log \fbrac{\constant\log \vertexcount}}$, we have Algorithm~\ref{Algorithm: Final Search} returns an $X$ such that $X \geq \numtriangle/2 \geq \esttriangle$ with probability at least $1 - \frac{1}{\constant\log \vertexcount}$, and the algorithm terminates.

    % \debarshi{Here in the union bound, we use the fact that $\log\vertexcount \gg \log\log\vertexcount$, do we need to make this explicit?}
    
    Hence, to get to a $\searchesttriangle  \leq \frac{\numtriangle}{2^k}$, the Algorithm~\ref{Algorithm: Final Search} must have failed to terminate for all previous values of $\searchesttriangle $, which happens with probability at most $\frac{1}{\fbrac{\constant\log \vertexcount}^k}$.
\end{proof}

Finally we combine Lemmas~\ref{Lemma: Final Search Wrong Termination Bound} and~\ref{Lemma: Final Search Low Bar T Bound} to obtain the following theorem quantifying the estimation guarantees and query complexity of Algorithm~\ref{Algorithm: Final Search}.

\begin{theorem}
    Algorithm~\ref{Algorithm: Final Search} makes $\bigot{\edgecount\arboricity/\approxerror^3\numtriangle}$ queries in expectation, and returns $X \in [\fbrac{1-\approxerror}\numtriangle$ $,\fbrac{1+\approxerror}\numtriangle]$ with probability at least $5/6$.
\end{theorem}

\begin{proof}
    The algorithm runs through at most $\log^2 \vertexcount$ values of $\esttriangle$ such that $\esttriangle > 2\numtriangle$. By Lemma~\ref{Lemma: Final Search Wrong Termination Bound}, the algorithm terminates in each such case with probability at most $1/\constant\log^2\vertexcount$. For each value of $\searchesttriangle  \leq 2\numtriangle$, by Theorem~\ref{Theorem: Triangle Estimate ALgo Works}, the algorithm returns a wrong value with probability at most $1/\constant\log \vertexcount$. There are at most $3\log\vertexcount$ such values of $\searchesttriangle$. Taking an union bound and fixing the constant $\constant$ appropriately, the algorithm returns a correct output with probability at least $5/6$.

    For each  $\searchesttriangle $ in Line~\ref{Line: Bar T Range} of Algorithm~\ref{Algorithm: Final Search}, by Theorem~\ref{Theorem: Triangle Estimate ALgo Works} the algorithm makes $\bigot{(\edgecount\arboricity/\approxerror^3\searchesttriangle )}$ queries. Hence, till $\searchesttriangle  \leq \numtriangle/2$, the algorithm makes $\bigot{(\edgecount\arboricity/\approxerror^3\numtriangle)}$ queries. Additionally, by Lemma~\ref{Lemma: Final Search Low Bar T Bound}, the queries beyond $\searchesttriangle  \leq \numtriangle$ can be bounded as:
    \begin{align*}
        \sum_{i \in \log\fbrac{\vertexcount}}\frac{1}{\fbrac{\constant\log \vertexcount}^k}\cdot2^k \cdot\bigot{\frac{\edgecount\arboricity}{\approxerror^3\numtriangle}} &=\bigot{\frac{\edgecount\arboricity}{\approxerror^3\numtriangle}}
    \end{align*}
\end{proof}

\begin{theorem}
    There exists an algorithm that makes $\bigot{\frac{\edgecount\arboricity\log\frac{1}{\confidence}}{\approxerror^3\numtriangle}}$ queries in expectation, and returns $X \in \tbrac{\fbrac{1-\approxerror}\numtriangle,\fbrac{1+\approxerror}\numtriangle}$ with probability at least $1 - \confidence$.
\end{theorem}

\begin{proof}
    By the well-known median trick~\citep{chakrabarti2015data}, the median of $\bigo{\log\frac{1}{\confidence}}$ independent runs of Algorithm~\ref{Algorithm: Final Search} establishes the result.
\end{proof}











% \begin{algorithm}[ht!]
%     \caption{Triangle Counting - Final Algorithm}\label{Algorithm: Random Edge Arboricity Triangle Counting Triangle Estimate}
%     \begin{algorithmic}[1]
%         \Require \randedgeq{}, \neighbourq{}, and \edgeexistsq{} query access to a graph $\graph$. Parameters $\esttriangle, \arboricity$, $\approxerror$ oracle access to \heavyoracle{} with threshold constant $\threshold$
%         \State $\edgesamplesize \gets 25\approxerror^{-3}\log(\vertexcount)(\edgecount\arboricity/\esttriangle)$
%         \State $\samplededges \gets \emptyset$
%         \State $\sampledtriangles \gets \emptyset$
%         \For{$i \in [\edgesamplesize]$}
%             \State $\edge_i \gets \randedgeq{}$
%             \State Let $\vertex_i$ be the endpoint of $\edge_i$ with smaller degree, and $x$ be the endpoint of $\edge_i$ that is not $\vertex_i$.
%             \If{$\heavyoracle(\edge_i,\arboricity,\frac{\approxerror}{12},\frac{1}{2\edgecount\log{\fbrac{n}}}) = 0$}
%                 \State $\samplededges \gets \samplededges \cup \edge_i$
%                 \State $\querycount_{\edge_i} \gets 0$
%                 \State If $\degree{\edge_i} \leq \arboricity$, set $\querycount_{\edge_i} \gets 1$ with probability $\frac{\degree{\edge_i}}{\arboricity}$. Otherwise, set $\querycount_{\edge_i} \gets \ceil{\frac{\degree{\edge_i}}{\arboricity}}$
%                 \For{$j \in [\querycount_{\edge_i}]$}
%                     \State $\altvertex \gets \neighbourq{\fbrac{\vertex_i}}$
%                     \State If $\edgeexistsq{\fbrac{\altvertex,x}} = 1$ and $(\altvertex,\edge_i)$ is not a triangle in $\sampledtriangles$, $\sampledtriangles \gets \sampledtriangles \cup (\altvertex,\edge_i)$ 
%                 \EndFor
%             \EndIf
%         \EndFor
%         % \State Remove duplicate triangles from $\sampledtriangles$ If a triangle is
%         \For{$\edge \in \samplededges$}
%             \State $\empweightfunc(\edge) = \frac{1}{\querycount_{\edge}} \sum_{(u,\edge) \in \sampledtriangles} \max\fbrac{\arboricity,\degree{\edge}}$
%             % \State $Y_\edge = \frac{1}{\querycount_{\edge}} \empweightfunc(\edge)$
%         \EndFor
%         % \State Choose a consistent weight function $\weightfunc$. Compute the empirical weight function as $\empweightfunc(e) = \weightfunc_\samplededges(e)$
%         \State \Return $\emptriangle = \frac{\edgecount}{\edgesamplesize}\sum_{\edge \in \samplededges} \empweightfunc(\edge)$
%     \end{algorithmic}
% \end{algorithm}

% \begin{theorem}\label{Theorem: Triangle Estimate Algo Works}
%     The algorithm~\ref{Algorithm: Random Edge Arboricity Triangle Counting Triangle Estimate} returns $\emptriangle$ such that
%     \begin{align*}
%         \Pr\tbrac{\emptriangle \in \tbrac{\fbrac{1-\approxerror}\numtriangle,\fbrac{1+\approxerror}\numtriangle}} \geq 1 - \frac{1}{\log\fbrac{\vertexcount}}
%     \end{align*}
% \end{theorem}

% \begin{proof}
%     The \heavyoracle{} implemented here rejects $\frac{24\arboricity}{\approxerror}$-heavy edges and accepts $\frac{24\arboricity}{\approxerror}$-light edges with probability at least $1 - \frac{1}{2\edgecount\log{\fbrac{n}}}$. By union bound, \heavyoracle{} rejects all $\frac{24\arboricity}{\approxerror}$-heavy edges and accepts all $\frac{6\arboricity}{\approxerror}$-light edges correctly with probability at least $1-\frac{1}{2\log{\fbrac{n}}}$. 
    
%     Hence, by Lemma~\ref{lemma: E[Y_I] Weight Func Algo} and Corollary~\ref{Corollary: Lower Bound on Light Triangles}, the algorithm~\ref{Algorithm: Random Edge Arboricity Triangle Counting Triangle Estimate} with probability at least $1-\frac{1}{2\log{\fbrac{n}}}$ outputs $\emptriangle$ such that:
%     \begin{align}
%         \Exp{\tbrac{\emptriangle}} = \lighttriangles{\threshold} \in \tbrac{\fbrac{1-\frac{\approxerror}{2}}\numtriangle,\numtriangle}\label{Equation: Expectation of Triangle Estimate Algorithm}
%     \end{align}
%     By Theorem~\ref{Theorem: Oracle Triangle Estimate ALgo Works}, we have that:
%     \begin{align}
%         \Pr\tbrac{\emptriangle \notin \tbrac{\fbrac{1-\frac{\approxerror}{2}}\lighttriangles{\threshold},\fbrac{1+\frac{\approxerror}{2}}\lighttriangles{\threshold}}} \leq \frac{1}{2\log\fbrac{\vertexcount}}\label{Equation: Confidence interval of Triangle Estimate Algorithm}
%     \end{align}
%     Combining Equation~\ref{Equation: Expectation of Triangle Estimate Algorithm} and \ref{Equation: Confidence interval of Triangle Estimate Algorithm} through union bound, we have:
%     \begin{align*}
%         \Pr\tbrac{\emptriangle \in \tbrac{\fbrac{1-\approxerror}\numtriangle,\fbrac{1+\approxerror}\numtriangle}} \geq 1 - \frac{1}{\log\fbrac{\vertexcount}}
%     \end{align*}
% \end{proof}

% \todo{Make the final search algorithm formal?}



% \begin{theorem}
%     The Algorithm~\ref{Algorithm: Random Edge Arboricity Triangle Counting Triangle Estimate} makes $25\approxerror^{-3}\log(\vertexcount)(\edgecount\arboricity/\esttriangle)(4+2\approxerror\log\fbrac{2\edgecount\log{\fbrac{n}}})$ queries in expectation.
% \end{theorem}
% \todo{Find a more comprehnsible expression.}
% \begin{idea}[There is a $\log^2(n)$ term!]
%     The original works of~\citep{Dana_Ron_Triangle_Counting,assadi2018simple} in terms of $poly\log(n)$ includes only a $\log(n)$ term. Here, it becomes $\log^2(n)$ due to the \heavyoracle{} adding an additive $\log(n)$ for the union bound. It can be reduced to $m\arboricity$ quite easily. Do we need to?
% \end{idea}
% \begin{proof}
%     The algorithm makes $3$ sets of queries, $\edgesamplesize$ \randedgeq{} queries, \heavyoracle{} call for each of these sampled edges, and finally $2\querycount$ queries for each of these edges.

%     For $\heavyoracle{}$, lemma~\ref{Lemma: Heavy Oracle Query Count} states that it makes $2\approxerror\log\fbrac{2\edgecount\log{\fbrac{n}}}$ queries in expectation. 

%     For the final $2\querycount$ queries, the expectation is taken over the edges sampled uniformly at random:
%     \begin{align*}
%         &\Exp_{\edge \sim \uniform\fbrac{\edgeset}} \ceil{\frac{\degree{\edge_i}}{\arboricity}}\\
%         =&\sum_{e \in E} \frac{1}{m} \ceil{\frac{d_e}{\arboricity}}\\
%         \leq&\frac{1}{m} \sum_{e \in E} 1 + \frac{d_e}{\arboricity}\\
%         =&1 + \frac{1}{m} \sum_{e \in E} \frac{d_e}{\arboricity}\\
%         \leq& 1 + \frac{2m\arboricity}{m\arboricity}&\text{By Lemma~\ref{lemma: arboricity triangle bound}}\\
%         =& 3
%     \end{align*}
%     Hence, the algorithm makes at most $\edgesamplesize(4+2\approxerror\log\fbrac{2\edgecount\log{\fbrac{n}}})$ in expectation. Plugging back the value of $\edgesamplesize$ completes the proof.
% \end{proof}

\section{Lower Bounds}
\label{sec:lower-bound}
In this section, we prove a lower bound that accounts for the multiplicative approximation factor $\approxerror$, as well as the arboricity $\arboricity$ of the graph by extending the ideas proposed in~\citep{DBLP:conf/approx/AssadiN22}. Our lower bound 
almost matches our proposed upper bound stated in Theorem~\ref{Theorem: Final Upper Bound}. We first state the problem known as the {\sf Popcount Thresholding Problem} (referred to as \ptp{}) in the query framework:
\begin{definition}[$\defptp$]
    Given a string $\alicestring \in \sbrac{0,1}^\stringlength$, the problem $\defptp$ is to  distinguish whether $\alicestring$ is  generated from i.i.d. samples from $\ptpdone$ or $\ptpdtwo$ defined as follows:
    \begin{itemize}
        \item \textbf{$\ptpdone$ :} For all $i \in \tbrac{\stringlength}$, $\alicestring_i$ is set to $1$ with probability $\fbrac{1-2\ptpsep}\frac{\ptpprob}{\stringlength}$, and set to $0$ otherwise.
        \item \textbf{$\ptpdtwo$ :} For all $i \in \tbrac{\stringlength}$, $\alicestring_i$ is set to $1$ with probability $\fbrac{1+2\ptpsep}\frac{\ptpprob}{\stringlength}$, and set to $0$ otherwise.
    \end{itemize}
    The problem is to decide if $\alicestring$ is generated from $\ptpdone$ or $\ptpdtwo$ by querying $\alicestring$ at any of its $\stringlength$ bits.
\end{definition}
We state the following lemma~\citep{DBLP:conf/approx/AssadiN22} quantifying the query complexity of $\defptp$. 
% \gopi{\st{I guess $\defptp$ is a problem in query framework.}}\debarshi{Done!}
\begin{lemma}~\cite{DBLP:conf/approx/AssadiN22}\label{Lemma: PTP Query Lower Bound}
    For any $\ptpsep \in (0,1/4)$, $\confidence \in (0,1/100)$, and integers $\stringlength \geq 1$, $\log(1/\confidence)\cdot12/\ptpsep^2\leq \ptpprob \leq \stringlength/6$, $\rqcomplexity{\confidence}{\defptp} \geq \frac{\stringlength\log\fbrac{1/4\confidence}}{24\ptpsep^2\ptpprob}$
  \remove{  \begin{align*}
        \rqcomplexity{\confidence}{\defptp} \geq \frac{\stringlength\log\fbrac{1/4\confidence}}{24\ptpsep^2\ptpprob}
    \end{align*}
    }
    where $\rqcomplexity{\confidence}{\defptp}$ is the randomized query complexity to decide $\defptp{}$ problem with probability at least $1 - \confidence$.
\end{lemma}

\iffalse{
\gopi{\st{May be we make the above lemma a proposition?}}\debarshi{All things like the~\cite{DBLP:journals/siamcomp/ChibaN85} result we use is in lemma.}
}\fi 

We also state the following lemma establishing bounds on the $\ell_1$-norm of $\alicestring$ \remove{depending on whether it is generated from $\ptpdone$ or $\ptpdtwo$}. 

\begin{lemma}\label{Lemma: PTP Deviation Bound}
    In $\defptp$, for any $\ptpsep \in (0,1/4)$, $\confidence \in (0,1/100)$, and integers $\stringlength \geq 1$, $\log(1/\confidence)\cdot12/\ptpsep^2\leq \ptpprob \leq \stringlength/6$,
    \begin{align*}
        \Pr\tbrac{\norm{\alicestring}_1 > \fbrac{1-\ptpsep} \cdot \ptpprob~|~\alicestring \sim \ptpdone} \leq \confidence\\
        \Pr\tbrac{\norm{\alicestring}_1 < \fbrac{1+\ptpsep} \cdot \ptpprob~|~\alicestring \sim \ptpdtwo} \leq \confidence\\
        \Pr\tbrac{\norm{\alicestring}_1 < \fbrac{1-4\ptpsep} \cdot \ptpprob} \leq \confidence
    \end{align*}
    where $\norm{\alicestring}_1$ denotes the number of $1$'s in the string $\alicestring$.
\end{lemma}

We combine two results to obtain the result. First we state a lemma due to~\citep{DBLP:conf/approx/AssadiN22}:

\begin{lemma}\label{Lemma: PTP Deviation Conditional Bound}
    In $\defptp$, for any $\ptpsep \in (0,1/4)$, $\confidence \in (0,1/100)$, and integers $\stringlength \geq 1$, $\log(1/\confidence)\cdot12/\ptpsep^2\leq \ptpprob \leq \stringlength/6$,
    \begin{align*}
        \Pr\tbrac{\norm{\alicestring}_1 > \fbrac{1-\ptpsep} \cdot \ptpprob~|~\alicestring \sim \ptpdone} \leq \confidence\\
        \Pr\tbrac{\norm{\alicestring}_1 < \fbrac{1+\ptpsep} \cdot \ptpprob~|~\alicestring \sim \ptpdtwo} \leq \confidence
    \end{align*}
    where $\norm{\alicestring}_1$ denotes the number of $1$'s in the string $\alicestring$.
\end{lemma}

We introduce another lemma to lower bound the $\ell_1$ norm of $\alicestring$ for $\ptpdone$.

\begin{lemma}\label{Lemma: PTP Deviation Lower Bound}
    In $\defptp$, for any $\ptpsep \in (0,1/4)$, $\confidence \in (0,1/100)$, and integers $\stringlength \geq 1$, $\log(1/\confidence)\cdot12/\ptpsep^2\leq \ptpprob \leq \stringlength/6$, $\Pr\tbrac{\norm{\alicestring}_1 < \fbrac{1-4\ptpsep} \cdot \ptpprob~|~\alicestring \sim \ptpdone} \leq \confidence$
    \remove{
    \begin{align*}
        \Pr\tbrac{\norm{\alicestring}_1 < \fbrac{1-4\ptpsep} \cdot \ptpprob~|~\alicestring \sim \ptpdone} \leq \confidence
    \end{align*}
    }
\end{lemma}

\begin{proof}
    By definition of $\ptpdone$ and linearity of expectations, we have:
    \begin{align*}
        \Exp{\tbrac{\norm{\alicestring}_1}} = \sum_{i \in [\stringlength]} \Exp\tbrac{\alicestring_i} = \fbrac{1-2\ptpsep}\ptpprob
    \end{align*}
    Now, given the independence of each $\alicestring_i$, we can use Chernoff bound (Lemma~\ref{Lemma: Multiplicative Chernoff Bound}) to obtain:
    \begin{align*}
        & \Pr\tbrac{\norm{\alicestring}_1 < \fbrac{1-4\ptpsep}\ptpprob} \\
        =&\Pr\tbrac{\norm{\alicestring}_1 - \Exp{\tbrac{\norm{\alicestring}_1}} > \frac{2\ptpsep}{1-2\ptpsep}\Exp{\tbrac{\norm{\alicestring}_1}}}\\
        \leq& \exp{\fbrac{-\frac{4\ptpsep^2\Exp{\tbrac{\norm{\alicestring}_1}}}{3\fbrac{1-2\ptpsep}^2}}}  & \text{(by Chernoff bound)}\\
        \leq& \exp{\fbrac{-\frac{48\ptpsep^2\log\fbrac{1/\confidence}}{3\ptpsep^2\fbrac{1-2\ptpsep}}}} &(\Exp{\tbrac{\norm{\alicestring}_1}} = \fbrac{1-2\ptpsep}\ptpprob, \ptpprob \geq \log(1/\confidence)\cdot12/\ptpsep^2)\\
        \leq& \confidence &(\ptpsep < 1/4)
    \end{align*}
\end{proof}


Combining lemmas~\ref{Lemma: PTP Deviation Conditional Bound} and~\ref{Lemma: PTP Deviation Lower Bound}, we obtain the desired result.
% \begin{theorem}
% Any algorithm that solves triangle estimation problem requires $\Omega\fbrac{\frac{\edgecount\arboricity\log\fbrac{1/\confidence}}{\approxerror^2\numtriangle}}$ queries.    
% \end{theorem}

% \begin{proof}[Proof Sketch]
%     We want to show that given a graph $\graph$, it takes $\Omega\fbrac{\frac{\edgecount\arboricity\log\fbrac{1/\confidence}}{\approxerror^2\numtriangle}}$ queries to obtain an estimate $\esttriangle$ such that $\esttriangle \in (1\pm \approxerror)\numtriangle$ with probability $1-\confidence$. For contradiction, let us assume there exists an algorithm $\trianglealgo$ that, for some arboricity $\lbarboricity$, computes an estimate $\esttriangle$ such that $\esttriangle \in (1\pm \approxerror)\numtriangle$ with probability $1-\confidence$ with $o\fbrac{\frac{\edgecount\lbarboricity\log\fbrac{1/\confidence}}{\approxerror^2\numtriangle}}$ queries. 

%     To show contradiction, we design an algorithm that solves $\defptp$ in $<\frac{\stringlength\log\fbrac{1/4\confidence}}{24\ptpsep^2\ptpprob}$ queries using $\trianglealgo$. Given an instance of $\alicestring$ generated from $\defptp$, we construct a graph $\rqgraph$ as follows:
%     \begin{itemize}
%         \item The graph $\rqgraph$ consists of $5$ sets of vertices $\vertexsetA, \vertexsetAhat, \vertexsetB, \vertexsetBhat$, and $\trianglemakerset$, with $\size{\vertexsetA} = \size{\vertexsetAhat} = \size{\vertexsetB} = \size{\vertexsetBhat} = \stringlength/\lbarboricity$, and $\size{\trianglemakerset} = \lbarboricity$. 
%         \item There exists an edge between every vertex in $\vertexsetA \cup \vertexsetB$ to every vertex in $\trianglemakerset$. Observe that this makes sure that the arboricity of $\rqgraph$ is $\lbarboricity$. We index the problem instance $\alicestring$ of length $\stringlength$ as $\tbrac{\frac{\stringlength}{\lbarboricity}}\times\tbrac{\lbarboricity}$, denoting by $\alicestring_{i,j} = \alicestring_{i\times\lbarboricity+j}$. We add an edge $\fbrac{\vertexA_i,\vertexAhat_{i+j}}$, and an edge $\fbrac{\vertexB_i,\vertexBhat_{i+j}}$ if $\alicestring_{i,j} = 0$.We add an edge $\fbrac{\vertexA_i,\vertexB_{i+j}}$, and an edge $\fbrac{\vertexAhat_i,\vertexBhat_{i+j}}$ if $\alicestring_{i,j} = 1$. 
%     \end{itemize}
%     Observe that each edge between a vertex in $\vertexsetA$ and $\vertexsetB$ adds $\lbarboricity$ triangles. Hence, the no of triangles in $\rqgraph$ is exactly $\norm{\alicestring}_1\lbarboricity$. Also note that we have added $2\stringlength$ edges between $\vertexsetA\cup\vertexsetB$ and $\trianglemakerset$, and further $2\stringlength$ edges according to the entries of $\alicestring$, 2 for each element $\alicestring_i$. Hence, we have $\lbedgecount = 4\stringlength$. Now, we run the algorithm $\trianglealgo$ on $\rqgraph$ with $\frac{\edgecount\lbarboricity\log\fbrac{1/4\confidence}}{200\approxerror^2\numtriangle} = \frac{\stringlength\log(1/4\confidence)}{50\ptpsep^2\ptpprob}$ queries. 
%     \todo{Make precise, queries implementation etc.}
%     By Lemma~\ref{Lemma: PTP Deviation Bound}, it suffices to show that we can use $\trianglealgo$ to distinguish between $\norm{\alicestring}_1 > \fbrac{1+\ptpsep}\ptpprob$ instance and $\norm{\alicestring}_1 < \fbrac{1-\ptpsep}\ptpprob$ with probability $1-\frac{\confidence}{2}$ to distinguish between $\ptpdone$ and $\ptpdtwo$ instance with probability $1 - \confidence$. In that regard, we show that the algorithm with threshold at $\fbrac{1-\ptpsep^2}\ptpprob\lbarboricity$ and $\approxerror = \ptpsep$, accepts with probability $1-\frac{\confidence}{2}$ given $\norm{\alicestring}_1 > \fbrac{1+\ptpsep}\ptpprob$, and rejects with probability $1-\frac{\confidence}{2}$ given $\norm{\alicestring}_1 < \fbrac{1-\ptpsep}\ptpprob$. We consider the two cases separately:

%     \textbf{$\norm{\alicestring}_1 > \fbrac{1+\ptpsep}\ptpprob$:} Our construction ensures that $\rqgraph$ has $\numtriangle = \norm{\alicestring}_1\lbarboricity = \fbrac{1+\ptpsep}\ptpprob\lbarboricity = \fbrac{1+\approxerror}\ptpprob\lbarboricity$. Additionally, by our assumption on $\trianglealgo$, it outputs an estimate $\emptriangle \geq (1-\approxerror)\numtriangle$ with probability $1 - \frac{\confidence}{2}$. Thus, we have $\esttriangle \geq \fbrac{1 - \approxerror^2}\ptpprob\lbarboricity$ with probability $1-\frac{\confidence}{2}$.

%     \textbf{$\norm{\alicestring}_1 < \fbrac{1-\ptpsep}\ptpprob$:} Our construction ensures that $\rqgraph$ has $\numtriangle = \norm{\alicestring}_1\lbarboricity = \fbrac{1-\ptpsep}\ptpprob\lbarboricity = \fbrac{1-\approxerror}\ptpprob\lbarboricity$. Additionally, by our assumption on $\trianglealgo$, it outputs an estimate $\emptriangle \leq (1+\approxerror)\numtriangle$ with probability $1 - \frac{\confidence}{2}$. Thus, we have $\esttriangle \leq \fbrac{1 - \approxerror^2}\ptpprob\lbarboricity$ with probability $1-\frac{\confidence}{2}$.

%     Now we state how to simulate the required queries:
%     \begin{itemize}
%         \item \textbf{\degreeq{}:} For a vertex $\vertex \in \vertexsetA\cup\vertexsetB\cup\vertexsetAhat\cup\vertexsetBhat$, return $\lbarboricity$, for a vertex $\vertex \in \trianglemakerset$, return $\frac{2\stringlength}{\lbarboricity}$.
        
%         \item \textbf{\neighbourq{}:} For a vertex $\vertex \in \vertexsetA\cup\vertexsetB\cup\vertexsetAhat\cup\vertexsetBhat$, w.l.o.g assume the $\vertex$ to be $\vertexA_i \in \vertexsetA$. Choose a $j \in \tbrac{\lbarboricity}$ uniformly at random, return $\vertexAhat_{i+j}$ if $\alicestring_{i,j} = 0$, and $\vertexB_{i+j}$ otherwise. For a vertex $\vertex \in \trianglemakerset$, return a vertex in $\vertexsetA \cup \vertexsetB$ uniformly at random.
        
%         \item \textbf{\randedgeq{}:} Sample a vertex $\vertex$ with probability $\frac{\degree{\vertex}}{4\stringlength}$. Pick a random neighbour $\altvertex = \neighbourq{\fbrac{\vertex}}$, return $\fbrac{\vertex,\altvertex}$.
        
%         \item \textbf{\edgeexistsq{}:} Given a vertex pair $\fbrac{\vertex,\altvertex}$. If the vertex $\vertex \in \vertexsetA\cup\vertexsetB$, w.l.o.g assume the $\vertex$ to be $\vertexA_i \in \vertexsetA$, if $\altvertex \in \trianglemakerset$, return $1$, else if $\altvertex = \vertexB_j \in \vertexsetB$, return $1$ if $\alicestring_{i,j-i} = 1$, else if $\altvertex = \vertexAhat_j \in \vertexsetAhat$, return $1$ if $\alicestring_{i,j-i} = 0$, else return $0$. If a vertex $\vertex \in \vertexsetAhat\cup\vertexsetBhat$, w.l.o.g assume the $\vertex$ to be $\vertexAhat_i \in \vertexsetAhat$, if $\altvertex = \vertexA_j \in \vertexsetA$, return $1$ if $\alicestring_{j,i-j} = 0$, else if $\altvertex = \vertexBhat_j \in \vertexsetBhat$, return $1$ if $\alicestring_{j,i-j} = 1$, else return $0$. If $\vertex \in \trianglemakerset$, return $1$ if $\altvertex \in \vertexsetA\cup\vertexsetB$, else return $0$.
%     \end{itemize}
% \end{proof}

\begin{theorem}[Lower Bound - Formal]\label{Theorem: Lower Bound on Triangle Counting through PTP}
Any algorithm that solves triangle estimation problem with $\approxerror \leq \frac{1}{4}$ using \degreeq{}, \neighbourq{}, \edgeexistsq{} and \randedgeq{} requires $\Omega\fbrac{\frac{\edgecount\arboricity\log\fbrac{1/\confidence}}{\approxerror^2\numtriangle}}$ queries.    
\end{theorem}

\begin{proof}
% \gopi{\st{Actually, we show the lower bound for an easier problem: given $m,\varepsilon, T$, the objective is to distinguish whether the number of triangles is at most $(1-\varepsilon)T$ or at least $(1+\varepsilon)T$?}}\debarshi{Done!}

% Actually, we show the lower bound for an easier problem: given $m,\varepsilon, T$, the objective is to distinguish whether the number of triangles is at most $(1-\varepsilon)T$ or at least $(1+\varepsilon)T$.

We want to show that given a graph $\graph$, it takes $\Omega\fbrac{\frac{\edgecount\arboricity\log\fbrac{1/\confidence}}{\approxerror^2\numtriangle}}$ queries to obtain an estimate $\esttriangle$ such that $\esttriangle \in (1\pm \approxerror)\numtriangle$ with probability at least $1-\confidence$. For contradiction, let us assume there exists an algorithm $\trianglealgo$ that, for some arboricity $\lbarboricity$, computes an estimate $\esttriangle$ such that $\esttriangle \in (1\pm \approxerror)\numtriangle$ with probability at least $1-\confidence$ using $\smallo{\frac{\edgecount\lbarboricity\log\fbrac{1/\confidence}}{\approxerror^2\numtriangle}}$ queries. 



    To show contradiction, we design an algorithm that solves $\defptp$ in $<\frac{\stringlength\log\fbrac{1/4\confidence}}{24\ptpsep^2\ptpprob}$ queries using $\trianglealgo$.\remove{\gopi{Shall we state the value of $M,k,$ and $\gamma$ here w.r.t. $m,T,\alpha^*$ and $\varepsilon$? May be $M=4m$, $T=k\alpha^*$, and $\gamma=\varepsilon$.}} Given an instance of $\alicestring$ generated from $\defptp$, we consider a graph $\rqgraph = (\vertexset_x, \edgeset_x)$ defined as: 
    \paragraph*{The vertex set $\vertexset_x$:} The vertex set $\vertexset_x$ consists of $5$ sets of disjoint and independent set of vertices $\vertexsetA, \vertexsetAhat, \vertexsetB, \vertexsetBhat$, and $\trianglemakerset$, with $\size{\vertexsetA} = \size{\vertexsetAhat} = \size{\vertexsetB} = \size{\vertexsetBhat} = \stringlength/\lbarboricity$, and $\size{\trianglemakerset} = \lbarboricity$. Vertices of $\vertexsetA$, $\vertexsetAhat$, $\vertexsetB$ and $\vertexsetBhat$ will be named as $\vertexA$, $\vertexAhat$, $\vertexB$ and $\vertexBhat$, respectively. 
    \paragraph*{The edge set $\edgeset_x$:} There exists an edge between every vertex in $\vertexsetA \cup \vertexsetB$ to every vertex in $\trianglemakerset$. Observe that this makes sure that the arboricity of $\rqgraph$ is $\lbarboricity$. If a bit of $\alicestring$ is $1$ (resp. $0$), there will be an edge from a vertex in $\vertexsetA$ (resp. $\vertexsetA$) to a vertex in $\vertexsetB$ (resp. $\vertexsetAhat$), and an edge from a vertex in $\vertexsetAhat$ (resp. $\vertexsetB$) to a vertex in $\vertexsetBhat$ (resp. $\vertexsetBhat$).
        
        \remove{There can be \blue{at most} $\left( \frac{M}{\lbarboricity} \right)^2$ edges between any pair of sets like $(\vertexsetA, \vertexsetB)$, or $(\vertexsetA, \vertexsetAhat)$, or $(\vertexsetAhat, \vertexsetBhat)$, or $(\vertexsetB, \vertexsetBhat)$. \red{For bits in $\alicestring$ to have a one-to-one correspondence to the above edges, we need $\left( \frac{M}{\lbarboricity} \right)^2 = M$, i.e., $\lbarboricity = \sqrt{\stringlength}$.}}
        
        We index the problem instance $\alicestring$ of length $\stringlength$ as $\tbrac{\frac{\stringlength}{\lbarboricity}}\times\tbrac{\lbarboricity}$, denoting  $\alicestring_{\fbrac{i-1}\times\lbarboricity+j}$ as $\alicestring_{i,j}$, where $i \in \tbrac{\frac{\stringlength}{\lbarboricity}}$ and $j \in \tbrac{\lbarboricity}$. We add an edge $\fbrac{\vertexA_i,\vertexAhat_{i+j}}$, and an edge $\fbrac{\vertexB_i,\vertexBhat_{i+j}}$ if $\alicestring_{i,j} = 0$. We add an edge $\fbrac{\vertexA_i,\vertexB_{i+j}}$, and an edge $\fbrac{\vertexAhat_i,\vertexBhat_{i+j}}$ if $\alicestring_{i,j} = 1$. All $(i+j)$ additions here are modulo $\tbrac{\frac{\stringlength}{\lbarboricity}}$.
        \remove{
    \begin{itemize}
        \item[$\vertexset_x$:] The vertex set $\vertexset_x$ consists of $5$ sets of disjoint and independent set of vertices $\vertexsetA, \vertexsetAhat, \vertexsetB, \vertexsetBhat$, and $\trianglemakerset$, with $\size{\vertexsetA} = \size{\vertexsetAhat} = \size{\vertexsetB} = \size{\vertexsetBhat} = \stringlength/\lbarboricity$, and $\size{\trianglemakerset} = \lbarboricity$. Vertices of $\vertexsetA$, $\vertexsetAhat$, $\vertexsetB$ and $\vertexsetBhat$ will be named as $\vertexA$, $\vertexAhat$, $\vertexB$ and $\vertexBhat$, respectively. 
        \item[$\edgeset_x$:] There exists an edge between every vertex in $\vertexsetA \cup \vertexsetB$ to every vertex in $\trianglemakerset$. Observe that this makes sure that the arboricity of $\rqgraph$ is $\lbarboricity$. If the bit is $1$ (\blue{respectively }$0$), \complain{there will be an edge from a vertex in $\vertexsetA$ (\blue{respectively }$\vertexsetA$) to a vertex in $\vertexsetB$ (\blue{respectively }$\vertexsetAhat$), and an edge from a vertex in $\vertexsetAhat$ (\blue{respectively }$\vertexsetB$) to a vertex in $\vertexsetBhat$ (\blue{respectively }$\vertexsetBhat$).}\blue{}
        \remove{There can be \blue{at most} $\left( \frac{M}{\lbarboricity} \right)^2$ edges between any pair of sets like $(\vertexsetA, \vertexsetB)$, or $(\vertexsetA, \vertexsetAhat)$, or $(\vertexsetAhat, \vertexsetBhat)$, or $(\vertexsetB, \vertexsetBhat)$. \red{For bits in $\alicestring$ to have a one-to-one correspondence to the above edges, we need $\left( \frac{M}{\lbarboricity} \right)^2 = M$, i.e., $\lbarboricity = \sqrt{\stringlength}$.}}
        We index the problem instance $\alicestring$ of length $\stringlength$ as $\tbrac{\frac{\stringlength}{\lbarboricity}}\times\tbrac{\lbarboricity}$, denoting  $\alicestring_{\fbrac{i-1}\times\lbarboricity+j}$ as $\alicestring_{i,j}$, where $i \in \tbrac{\frac{\stringlength}{\lbarboricity}}$ and $j \in \tbrac{\lbarboricity}$. We add an edge $\fbrac{\vertexA_i,\vertexAhat_{i+j}}$, and an edge $\fbrac{\vertexB_i,\vertexBhat_{i+j}}$ if $\alicestring_{i,j} = 0$. We add an edge $\fbrac{\vertexA_i,\vertexB_{i+j}}$, and an edge $\fbrac{\vertexAhat_i,\vertexBhat_{i+j}}$ if $\alicestring_{i,j} = 1$. All $(i+j)$ additions here are modulo $\tbrac{\frac{\stringlength}{\lbarboricity}}$.
    \end{itemize}
      }

      
    Observe that each edge between a vertex in $\vertexsetA$ and $\vertexsetB$ adds $\lbarboricity$ triangles. Hence, the number of triangles in $\rqgraph$ is exactly $\norm{\alicestring}_1\lbarboricity$. Also note that we have added $2\stringlength$ edges between $\vertexsetA\cup\vertexsetB$ and $\trianglemakerset$, and further $2\stringlength$ edges according to the entries of $\alicestring$, 2 for each bit $\alicestring_i$. Hence, we have $\lbedgecount = 4\stringlength$, and fix $\approxerror = \ptpsep$. 
    
% \red{    
%     Now, we run the algorithm $\trianglealgo$ on $\rqgraph$ with $\frac{\edgecount\log\fbrac{1/4\confidence}}{200\approxerror^2\ptpprob} = \frac{\stringlength\log(1/4\confidence)}{50\ptpsep^2\ptpprob}$ queries. Also, observe that by Lemma~\ref{Lemma: PTP Deviation Lower Bound}, we have with probability $1-\confidence$, $\numtriangle = \lbarboricity\norm{\alicestring}_1 \geq \lbarboricity(1-4\ptpsep)\ptpprob \geq \frac{\lbarboricity\ptpprob}{2}$, where the last inequality follows from the fact that $\ptpsep = \approxerror \leq \frac{1}{4}$. Hence, we have $\frac{\edgecount\log\fbrac{1/4\confidence}}{200\approxerror^2\ptpprob} \geq \frac{\edgecount\lbarboricity\log\fbrac{1/4\confidence}}{200\approxerror^2\numtriangle}$. Thus the algorithm $\trianglealgo$ is allowed to make $\smallo{\frac{\edgecount\lbarboricity\log\fbrac{1/\confidence}}{\approxerror^2\numtriangle}}$ queries under the given query bound.
%     % \gopi{While editing, we can consider shortening the paragraph a bit.}
% }

% \red{
%     % \todo{Make precise, queries implementation etc.}
%     By Lemma~\ref{Lemma: PTP Deviation Bound}, it suffices to show that we can use $\trianglealgo$ to distinguish between $\norm{\alicestring}_1 > \fbrac{1+\ptpsep}\ptpprob$ instance and $\norm{\alicestring}_1 < \fbrac{1-\ptpsep}\ptpprob$ instance with probability $1-\frac{\confidence}{2}$ to distinguish between whether $\alicestring$ is generated from $\ptpdone$ or $\ptpdtwo$ with probability $1 - \confidence$. In that regard, we show that an algorithm deciding based on the estimate $\emptriangle$ generated by algorithm $\trianglealgo$, with threshold at $\fbrac{1-\ptpsep^2}\ptpprob\lbarboricity$, accepts with probability $1-\frac{\confidence}{2}$ given $\norm{\alicestring}_1 > \fbrac{1+\ptpsep}\ptpprob$, and rejects with probability $1-\frac{\confidence}{2}$ given $\norm{\alicestring}_1 < \fbrac{1-\ptpsep}\ptpprob$. We consider the two cases separately:
% }



We now describe $\trianglealgoptp$ an algorithm that uses $\trianglealgo$ to solve $\defptp$ using $<\frac{\stringlength\log\fbrac{1/4\confidence}}{24\ptpsep^2\ptpprob}$ queries. Given a string $\alicestring$, we generate $\rqgraph$ and estimate the number of triangles $\emptriangle$ by calling $\trianglealgo$ using $\frac{\edgecount\log\fbrac{1/4\confidence}}{200\approxerror^2\ptpprob} = \frac{\stringlength\log(1/4\confidence)}{50\ptpsep^2\ptpprob}$ queries. If $\emptriangle < \fbrac{1-\ptpsep^2}\ptpprob\lbarboricity$, it outputs $\ptpdone$; otherwise, it outputs $\ptpdtwo$. Also, observe that by Lemma~\ref{Lemma: PTP Deviation Bound}, we have  $\numtriangle = \lbarboricity\norm{\alicestring}_1 \geq \lbarboricity(1-4\ptpsep)\ptpprob \geq \frac{\lbarboricity\ptpprob}{2}$ with probability at least $1-\confidence$, where the last inequality follows from the fact that $\ptpsep = \approxerror \leq \frac{1}{4}$. Hence, we have $\frac{\edgecount\log\fbrac{1/4\confidence}}{200\approxerror^2\ptpprob} \geq \frac{\edgecount\lbarboricity\log\fbrac{1/4\confidence}}{200\approxerror^2\numtriangle}$. Thus the algorithm $\trianglealgo$ is allowed to make $\smallo{\frac{\edgecount\lbarboricity\log\fbrac{1/\confidence}}{\approxerror^2\numtriangle}}$ queries under the given query bound of $\frac{\edgecount\lbarboricity\log\fbrac{1/4\confidence}}{200\approxerror^2\numtriangle}$.


By Lemma~\ref{Lemma: PTP Deviation Bound}, to distinguish between whether $\alicestring$ is generated from $\ptpdone$ or $\ptpdtwo$ with probability $1 - \confidence$, it suffices to show that we can use $\trianglealgo$ to distinguish between $\norm{\alicestring}_1 > \fbrac{1+\ptpsep}\ptpprob$ instance and $\norm{\alicestring}_1 < \fbrac{1-\ptpsep}\ptpprob$ instance with probability $1-\frac{\confidence}{2}$. We now show that $\trianglealgoptp$ outputs $\ptpdtwo$ with probability $1-\frac{\confidence}{2}$ given $\norm{\alicestring}_1 > \fbrac{1+\ptpsep}\ptpprob$, and outputs $\ptpdone$ with probability $1-\frac{\confidence}{2}$ given $\norm{\alicestring}_1 < \fbrac{1-\ptpsep}\ptpprob$. We consider the two cases separately: 

    {\bf Case I }($\norm{\alicestring}_1 > \fbrac{1+\ptpsep}\ptpprob$): Our construction ensures that the number of triangles in $\rqgraph$ is $\numtriangle = \norm{\alicestring}_1\lbarboricity > \fbrac{1+\ptpsep}\ptpprob\lbarboricity = \fbrac{1+\approxerror}\ptpprob\lbarboricity$. Additionally, by our assumption on $\trianglealgo$, it outputs an estimate $\emptriangle \geq (1-\approxerror)\numtriangle$ with probability $1 - \frac{\confidence}{2}$. Thus, we have $\esttriangle \geq \fbrac{1 - \approxerror^2}\ptpprob\lbarboricity$ with probability $1-\frac{\confidence}{2}$.

    {\bf Case II }($\norm{\alicestring}_1 < \fbrac{1-\ptpsep}\ptpprob$): Our construction ensures that the number of triangles in $\rqgraph$ is $\numtriangle = \norm{\alicestring}_1\lbarboricity < \fbrac{1-\ptpsep}\ptpprob\lbarboricity = \fbrac{1-\approxerror}\ptpprob\lbarboricity$. Additionally, by our assumption on $\trianglealgo$, it outputs an estimate $\emptriangle \leq (1+\approxerror)\numtriangle$ with probability $1 - \frac{\confidence}{2}$. Thus, we have $\esttriangle \leq \fbrac{1 - \approxerror^2}\ptpprob\lbarboricity$ with probability $1-\frac{\confidence}{2}$.

    Now we state how to simulate the required queries:
    \begin{itemize}
        \item \textbf{$\degreeq{\fbrac{\vertex}}$:} For a vertex $\vertex \in \vertexsetA\cup\vertexsetB\cup\vertexsetAhat\cup\vertexsetBhat$, return $\lbarboricity$, and for a vertex $\vertex \in \trianglemakerset$, return $\frac{2\stringlength}{\lbarboricity}$. Hence, for \degreeq{} queries, no queries are made to the string.
        
        % \item \textbf{\neighbourq{}:} For a vertex $\vertex \in \vertexsetA\cup\vertexsetB\cup\vertexsetAhat\cup\vertexsetBhat$, w.l.o.g assume the $\vertex$ to be $\vertexA_i \in \vertexsetA$. Choose a $j \in \tbrac{\lbarboricity}$ uniformly at random, return $\vertexAhat_{i+j}$ if $\alicestring_{i,j} = 0$, and $\vertexB_{i+j}$ otherwise. For a vertex $\vertex \in \trianglemakerset$, return a vertex in $\vertexsetA \cup \vertexsetB$ uniformly at random.

        \item \textbf{$\neighbourq{\fbrac{\vertex,i}}$:} For a vertex $\vertex \in \vertexsetA\cup\vertexsetB\cup\vertexsetAhat\cup\vertexsetBhat$, w.l.o.g assume $\vertex$ to be $\vertexA_j \in \vertexsetA$. Return $\vertexAhat_{i+j}$ if $\alicestring_{i,j} = 0$, and $\vertexB_{i+j}$ otherwise. For a vertex $\vertex \in \trianglemakerset$, if $i \leq \frac{\edgecount}{\lbarboricity}$, return $\vertexA_{i}$, else return $\vertexB_{i = \frac{\edgecount}{\lbarboricity}}$. Hence, for \neighbourq{} queries, a single query is made to the string.
        
        \item \textbf{\edgeexistsq$\fbrac{\altvertex,\vertex}$:} Given a vertex pair $\fbrac{\vertex,\altvertex}$, there can be 3 cases: (a) If the vertex $\vertex \in \vertexsetA\cup\vertexsetB$, w.l.o.g assume $\vertex$ to be $\vertexA_i \in \vertexsetA$, if $\altvertex \in \trianglemakerset$, return $1$, else if $\altvertex = \vertexB_j \in \vertexsetB$, return $1$ if $\alicestring_{i,j-i} = 1$, else if $\altvertex = \vertexAhat_j \in \vertexsetAhat$, return $1$ if $\alicestring_{i,j-i} = 0$, else return $0$. (b) If a vertex $\vertex \in \vertexsetAhat\cup\vertexsetBhat$, w.l.o.g assume the $\vertex$ to be $\vertexAhat_i \in \vertexsetAhat$, if $\altvertex = \vertexA_j \in \vertexsetA$, return $1$ if $\alicestring_{j,i-j} = 0$, else if $\altvertex = \vertexBhat_j \in \vertexsetBhat$, return $1$ if $\alicestring_{j,i-j} = 1$, else return $0$. If $\vertex \in \trianglemakerset$, return $1$ if $\altvertex \in \vertexsetA\cup\vertexsetB$, else return $0$. (c) If a vertex $\vertex \in \trianglemakerset$, return $1$ if $\altvertex \in \vertexsetA \cup \vertexsetB$, return $0$ otherwise. Hence, for \edgeexistsq{} queries, a single query is made to the string.

        \item \textbf{\randedgeq{}:} Sample a vertex $\vertex$ with probability $\frac{\degree{\vertex}}{4\stringlength}$. Choose $i \in \degree{\fbrac{\vertex}}$ uniformly at random, query $\altvertex = \neighbourq{\fbrac{\vertex,i}}$ and return $\fbrac{\vertex,\altvertex}$. Hence, for \randedgeq{} queries, a single query is made to the string for the $\neighbourq{}$ query.
    \end{itemize}
\end{proof}

The following corollary is a direct implication from Theorem~\ref{Theorem: Lower Bound on Triangle Counting through PTP} due to the fact that we consider a weaker model (without access to \randedgeq{} queries) compared to Theorem~\ref{Theorem: Lower Bound on Triangle Counting through PTP}.

\begin{corollary}[Lower Bound for Local Queries]\label{Theorem: Lower Bound on Triangle Counting w/o Random Edge}
Any algorithm that solves triangle estimation problem with $\approxerror \leq \frac{1}{4}$ using \degreeq{}, \neighbourq{}, and \edgeexistsq{} requires $\Omega\fbrac{\frac{\edgecount\arboricity\log\fbrac{1/\confidence}}{\approxerror^2\numtriangle}}$ queries.    
\end{corollary}

% \input{sections/5_Final_Search}

%\bibliographystyle{ACM-Reference-Format}
\newpage

\bibliographystyle{apalike}

% \documentclass[11pt]{article}

% preamble here
\usepackage[top=30truemm,bottom=30truemm,left=25truemm,right=25truemm]{geometry}
\usepackage[utf8]{inputenc} % allow utf-8 input
\usepackage[T1]{fontenc}    % use 8-bit T1 fonts

\usepackage{microtype}
\usepackage{graphicx}
% \usepackage[dvipdfmx]{graphicx}
\usepackage{subcaption}
\usepackage{booktabs} 

\usepackage[colorlinks=true,citecolor=blue,linkcolor=blue]{hyperref}


\newcommand{\theHalgorithm}{\arabic{algorithm}}

\usepackage{amsmath,amssymb,amsfonts,amsthm}
\usepackage{bm,bbm}
\usepackage{mathcomp}
\usepackage{empheq}
\usepackage{fancybox}
\usepackage{breqn}
\usepackage{mathtools}
\mathtoolsset{centercolon}
\usepackage{tgtermes}

\usepackage[capitalize,noabbrev,nameinlink]{cleveref}

\usepackage[textsize=tiny]{todonotes}

\usepackage{my_macro}
\usepackage[authoryear,round]{natbib} 
\usepackage[font=small,labelfont=bf]{caption}



\theoremstyle{plain}
\newtheorem{theorem}{Theorem}[section]
\newtheorem{proposition}[theorem]{Proposition}
\newtheorem{lemma}[theorem]{Lemma}
\newtheorem{corollary}[theorem]{Corollary}
\theoremstyle{definition}
\newtheorem{definition}[theorem]{Definition}
\newtheorem{method}[theorem]{Method}
\newtheorem{assumption}[theorem]{Assumption}
\theoremstyle{remark}
\newtheorem{remark}[theorem]{Remark}

\title{Bandit and Delayed Feedback in Online Structured Prediction}

\author{
  Yuki Shibukawa\footnote{
    The University of Tokyo; 
    \texttt{shibu-yu762@g.ecc.u-tokyo.ac.jp}.
  }
  \and
  Taira Tsuchiya\footnote{
    The University of Tokyo and RIKEN; \texttt{tsuchiya@mist.i.u-tokyo.ac.jp}.
  }
  \and
  Shinsaku Sakaue\footnote{
    The University of Tokyo and RIKEN; \texttt{sakaue@mist.i.u-tokyo.ac.jp}.
  }
  \and 
  Kenji Yamanishi\footnote{
    The University of Tokyo; \texttt{yamanishi@g.ecc.u-tokyo.ac.jp}.
  }
}


\begin{document}
\maketitle

% abstract here
\begin{abstract}
Online structured prediction is a task of sequentially predicting outputs with complex structures based on inputs and past observations, encompassing online classification. Recent studies showed that in the full information setup, we can achieve finite bounds on the \textit{surrogate regret}, i.e., the extra target loss relative to the best possible surrogate loss. In practice, however, full information feedback is often unrealistic as it requires immediate access to the whole structure of complex outputs. Motivated by this, we propose algorithms that work with less demanding feedback, \textit{bandit} and \textit{delayed} feedback. For the bandit setting, using a standard inverse-weighted gradient estimator, we achieve a surrogate regret bound of $O(\sqrt{KT})$ for the time horizon $T$ and the size of the output set $K$. However, $K$ can be extremely large when outputs are highly complex, making this result less desirable. To address this, we propose an algorithm that achieves a surrogate regret bound of $O(T^{2/3})$, which is independent of $K$. This is enabled with a carefully designed pseudo-inverse matrix estimator. Furthermore, for the delayed full information feedback setup, we obtain a surrogate regret bound of $O(D^{2/3} T^{1/3})$ for the delay time $D$. We also provide algorithms for the delayed bandit feedback setup. Finally, we numerically evaluate the performance of the proposed algorithms in online classification with bandit feedback.
\end{abstract}


\section{Introduction}
In many machine learning problems, given an input vector from a vector space $\xx$, we aim to predict a vector in a finite output space $\yy$.  
Multiclass classification is one of the simplest examples, while in other cases output spaces may have more complex structures. 
\emph{Structured prediction} refers to such a class of problems with structured output spaces, including multiclass classification, multilabel classification, ranking, and ordinal regression, and it has applications in various fields, ranging from natural language processing to bioinformatics \citep{JMLR_Tsochantaridis_2005, bakir_2007_article}.
In structured prediction, training models that directly predict outputs in complex discrete output spaces is typically challenging. 
Therefore, we often adopt the \emph{surrogate loss framework} \citep{Bartlett_2006}---define an intermediate space of score vectors and train models that estimate score vectors from inputs based on surrogate loss functions.
Examples of surrogate losses include squared, logistic, and hinge losses, and a general framework encompassing them is the \emph{Fenchel--Young loss} \citep{JMLR_2020_blondel}, which we rely on in this study.


Structured prediction can be naturally extended to the online setting \citep{pmlr-v247-sakaue24a}.  
In this online setting, at each round $t=1,\dots,T$, the environment selects an input-output pair $(\xt,\yt)\in\xx\times\yy$.  
A learner then predicts $\yht \in \yy$ based on the input $\xt$ and incurs a loss $L(\yht;\yt)$, where $L:\yy\times\yy\to\R_{\geq0}$ is the target loss function. 
As with \citet{pmlr-v247-sakaue24a}, we focus on the simple yet fundamental case where the learner's model for estimating score vectors is linear. 

\begin{table*}[t]
\centering
\caption{Surrogate regret upper and lower bounds in online multiclass classification and online structured prediction. Here, $T$ is the time horizon, $K$ is the size of the set, $\yy$, of output vectors, and $D$ is the fixed delayed time.  
``FI'' is the abbreviation of full information.
Delayed feedback is considered only when ``Delayed'' appears in the feedback column. 
In the target loss column, ``SELF*'' means SELF that satisfies \cref{asp:self}.
Note that the $O(T^{2/3})$ bounds for SELF* in lines 6 and 9 do not explicitly depend on $\K$ but on $d$; 
in the case of multiclass classification with the 0-1 loss, the dependence on $\K$ appears as $d = \K$.
} 
\small
\begin{tabular}{@{}l@{\hspace{1ex}}l@{\hspace{1ex}}l@{\hspace{1ex}}l@{\hspace{1ex}}l@{}}
\toprule
& Problem setup &Feedback & Target loss &  Surrogate regret bound \\ 
\midrule
 \citet[Cor.~1]{NEURIPS2021_Hoeven} & Binary classification & Bandit & 0-1 loss & $\Omega(\sqrt{T})$ ($d=2$) \\ 
 \midrule
 \citet[Thm.~4]{NEURIPS2020_Hoeven} & Multiclass classification & Bandit & 0-1 loss & $O(\K \sqrt{T})$ \\ 
 \citet[Thm.~1]{NEURIPS2021_Hoeven} & Multiclass classification & Bandit & 0-1 loss & $O(\K \sqrt{T})$ \\ 
 \midrule
\citet[Thms.~7 and 8]{pmlr-v247-sakaue24a} & Structured prediction & FI & SELF & $O(1)$  \\
\textbf{This work} (\cref{thm:bandit_regret_expectation_abstract,thm:bandit_high_prob}) & Structured prediction &Bandit & SELF &  $O(\sqrt{\K T})$  \\
\textbf{This work}  (\cref{thm:bandit_regret_pseudo_estimator}) & Structured prediction & Bandit & SELF* &  $O(T^{2/3})$ \\
\textbf{This work}  (\cref{thm:delayed_regret_expectation_abstract,thm:delayed_regret_probability_abstract}) & Structured prediction & FI \& Delayed & SELF & $O(D^{2/3}T^{1/3})$  \\
\textbf{This work}  (\cref{thm:delay_bandit_bound_general_abstract}) & Structured prediction & Bandit \& Delayed & SELF & $O(\sqrt{DKT})$  \\
\textbf{This work}  (\cref{thm:delay_bandit_bound_self_abstract}) & Structured prediction & Bandit \& Delayed & SELF* & $O(D^{1/3}T^{2/3})$  \\
\bottomrule
\end{tabular}
\label{tab: regret order}
\end{table*}


The goal of the learner is to minimize the cumulative loss $\sumt{L(\yht;\yt)}$. 
On the other hand, the best the learner can do in the surrogate loss framework is to minimize the cumulative surrogate loss, namely $\sumt{S(\U\xt;\yt)}$, where $\U:\xx\to\R^d$ is the best offline linear estimator and $S:\R^d\times\yy\to\R_{\geq0}$ is a surrogate loss, which measures the discrepancy between the score vector $\U\xt \in \R^d$ and $\yt\in\yy$. 
Given this, a natural performance measure of the learner's predictions is the following \emph{surrogate regret} $\reg$: 
\begin{equation}\label{eq:sur_regret}
    \sumt{L(\yht;\yt)}=\sumt{S(\U\xt;\yt)}+\reg.
\end{equation}
The surrogate regret was introduced in the seminal paper by \Citet{NEURIPS2020_Hoeven} in the context of online multiclass classification.  
Recently, \citet{pmlr-v247-sakaue24a} showed that a finite surrogate regret bound  
can be achieved for online structured prediction under full information feedback, i.e., the learner can observe $\yt$ at the end of each round $t$.

However, the assumption that full information feedback is available is often demanding, especially when outputs have complex structures.
%
For example, in sequential ad assortment on an advertising platform, we may be able to observe only the click-through rate but not which ads were clicked, which boils down to the \emph{bandit feedback} setting \citep{Kakade2008EfficientBA,JMLR:v15:gentile:class_ranking}. 
Also, we may only have access to feedback from a while ago when designing an ad assortment for a new user—namely, \emph{delayed feedback}  \citep{Weinberger_2002_delay,manwani2022delaytronefficientlearningmulticlass}.
%
Similar situations have led to a plethora of studies in various online learning settings.
In combinatorial bandits, algorithms under bandit feedback (referred to as full-bandit feedback in their context), instead of full information feedback, have been widely studied
\citep{comband,combes15combinatorial,rejwan20topk,du21combinatorial}. 
Delayed feedback is also explored in various settings, including full information and bandit feedback \citep{joulani13online,cesabianchi16delay}.
Due to space limitations, we defer a further discussion of the background to \cref{app:additional_related_work}.


\subsection*{Our Contributions}
To extend the applicability of online structured prediction, this study develops online structured prediction algorithms that can handle weaker feedback---bandit feedback and delayed feedback---instead of full information feedback.  
As with \citet{pmlr-v247-sakaue24a}, we consider the case where target loss functions belong to a class called the Structured Encoding Loss Function (SELF) \citep{ciliberto16consistent,NEURIPS2019_Blondel}, a general class including the 0-1 loss in multiclass classification and the Hamming loss in multilabel classification and ranking (see \cref{subsec:self} for the definition). 
\Cref{tab: regret order} summarizes the surrogate regret bounds provided in this study and comparisons with existing results.


One of the challenges of bandit feedback is that the true output $\yt$ is not observed, making it impossible to compute the true gradient of the surrogate loss. 
To deal with this, we use an inverse-weighted gradient estimator, a typical approach that assigns weights to gradients by the inverse of choosing each output, establishing an $O(\sqrt{\K T})$ surrogate regret upper bounds (\cref{thm:bandit_regret_expectation_abstract,thm:bandit_high_prob}), 
where $K = \abs{\yy}$ is the cardinality of $\yy$.  
The $O(\sqrt{\K T})$ bound has an optimal dependence on $T$; it matches the $\sqrt{T}$ lower bound in the special case of online multiclass classification with bandit feedback \Citep[Corollary 1]{NEURIPS2021_Hoeven}. 
Furthermore, our bound is better than the existing $O(\K\sqrt{T})$ bound of \Citet{NEURIPS2020_Hoeven} by a factor of $\sqrt{\K}$, while it is not directly comparable to the latest $O(\K\sqrt{T})$ bound in \Citet{NEURIPS2021_Hoeven} due to differences in surrogate loss functions. 
See \cref{app:Discussio_on_the_Difference_in_Surrogate_Losses} for a more detailed discussion. 

While the $O(\sqrt{\K T})$ bound is satisfactory when $K = \abs{\yy}$ is small, $K$ can be extremely large in some structured prediction problems: in multilabel classification with $m$ correct labels, we have $\K=\binom{d}{m}$, and in ranking problems with $m$ items, we have $\K=m!$.
%
To address this issue, we consider a special case of SELF (denoted by SELF* in \cref{tab: regret order}), which still includes the aforementioned examples: the 0-1 loss in multiclass classification and the Hamming loss in multilabel classification and ranking. 
A technical challenge to resolve the issue lies in designing an appropriate gradient estimator used in online learning methods.
To this end, we draw inspiration from pseudo-inverse estimators used in the adversarial linear/combinatorial bandit literature \citep{dani07price,abernethy08competing,comband}. 
While we cannot naively apply the existing estimators, we design a new gradient estimator that applies to various specific structured prediction problems belonging to the special SELF framework.
Armed with this gradient estimator, we achieve a surrogate regret upper bound of $O(T^{2/3})$, which does not explicitly depend on~$\K$~(\cref{thm:bandit_regret_pseudo_estimator}).

For the delayed feedback setting with a known fixed delay time of $D$, it is actually not difficult to obtain a surrogate regret bound of $O(\sqrt{D T})$ with standard Online Convex Optimization (OCO) algorithms for delayed feedback. 
Our finding is that we can achieve a surrogate regret bound of $O(D^{2/3} T^{1/3})$ in online structured prediction under delayed full information feedback (\cref{thm:delayed_regret_expectation_abstract,thm:delayed_regret_probability_abstract}) by leveraging ODAFTRL \citep{pmlr-v139-flaspohler21a}, a Follow-the-Regularized-Leader-type algorithm that achieves an AdaGrad-type regret upper bound in OCO under delayed feedback. 
This bound is better than $O(\sqrt{D T})$ as $D \le T$.


Given the contributions so far, it is natural to explore online structured prediction in environments where both delay and bandit feedback are present. 
We obtain algorithms for this setup by combining the theoretical developments for bandit feedback without delay and delayed full information feedback, offering surrogate regret bounds of $O(\sqrt{D K T})$ (\cref{thm:delay_bandit_bound_general_abstract}) and $O(D^{1/3} T^{2/3})$ (\cref{thm:delay_bandit_bound_self_abstract}).

We validate our algorithms through numerical experiments using both synthetic and real-world data.  
Specifically, we consider online multiclass classification with bandit feedback. 
We observe that, depending on the number of classes and the dataset, our algorithm, designed for general structured prediction, can achieve accuracy comparable to existing algorithms specialized for multiclass classification.



\section{Preliminaries}
We describe the detailed setup of online structured prediction and key tools used in this work: the Fenchel--Young loss, SELF, and randomized decoding.
\paragraph{Notation}
For any integer $n > 0$, let $\brk{n} = \set{1,2,\hdots,n}$.
Let $\nrm{\cdot}$ denote a norm with $\kappa\nrm{\bmy}\geq\nrm{\bmy}_2$ for some $\kappa>0$ for any $\bmy\in\mathbb{R}^d$. 
For a matrix $\W$, let $\nrm{\W}_{\mathrm{F}}=\sqrt{\tr\prn*{\W^\top \W}}$ be the Frobenius norm. 
Let $\bm1$ denote the all-ones vector and $\bm{\ee}_i$ the $i$th standard basis vector.
For $\mathcal{\mathcal{K}}\subset \mathbb{R}^d$, let $\conv(\mathcal{K})$ be its convex hull and $I_{\mathcal{K}}:\mathbb{R}^d\to\set{0, +\infty}$ be its indicator function, which takes zero if $\bmy \in \mathcal{K}$ and $+\infty$ otherwise.
For $\Omega:\mathbb{R}^d\to\mathbb{R}\cup\set{+\infty}$, let $\dom(\Omega) \coloneqq \set*{\bmy \in \mathbb{R}^d:\Omega(\bmy) < +\infty}$ be its effective domain and $\Omega^*(\thb) \coloneqq \sup\set*{\inpr{\thb, \bmy} - \Omega(\bmy):\bmy \in \mathbb{R}^d}$ be its convex conjugate.
\cref{tab: notation} in \Cref{app: notation} summarizes the notation used in this paper.


\subsection{Online Structured Prediction}
Here, we describe the problem setting of online structured prediction.
Let $\xx$ be the input vector space and $\yy$ be the output vector space.
Define $\K \coloneqq |\yy|$.
Following \citet{JMLR_2020_blondel} and \citet{pmlr-v247-sakaue24a}, we assume that $\yy$ is embedded into $\mathbb{R}^d$ in a standard manner.
For example, $\yy = \set{\bm{\mathrm{e}}_1,\hdots,\bm{\mathrm{e}}_d}$ in $d$-class multiclass classification.


A linear estimator $\W\in\ww$ for a convex domain $\ww$ is used to transform the input vector $\bm{x}$ into the score vector $\W\bm{x}$.  
In online structured prediction, at each round $t=1,\dots,T$:
\begin{enumerate}%[topsep=2pt,itemsep=0pt, partopsep=0pt, leftmargin=18pt]
    \item The environment selects an input $\xt\in\xx$ and the true output $\yt\in\yy$; 
    \item The learner receives $\xt$ and computes the score vector~$\tht=\wt\xt$ using the linear estimator~$\wt$;
    \item The learner selects a predicted output $\yht$ based on $\tht$ and incurs a loss of $L(\yht;\yt)$;
    \item The learner receives feedback based on the problem setup and updates $\wt$ to $\W_{t+1}$ using an online learning algorithm, $\alg$.
\end{enumerate}
The goal of the learner is to minimize the cumulative prediction loss $\sumt{L(\yht;\yt)}$, which is equivalent to minimizing the surrogate regret $\mathcal{R}_T$ in \eqref{eq:sur_regret}.  
We assume that the input and output are generated in an oblivious manner.  
Note that when $\yy = \set{\bm{\ee}_1, \dots, \bm{\ee}_d}$ and $L(\yht; \yt) = \ind[\yht \neq \yt]$, the above setting reduces to online multiclass classification, which was studied by \Citet{NEURIPS2020_Hoeven} and \Citet{NEURIPS2021_Hoeven}.
We will use $B\coloneqq\diam(\ww)$, $\dix\coloneqq\diam(\xx)$, and $\diy\coloneqq\diam(\yy)$ to denote the diameters of the sets $\ww$, $\xx$, and $\yy$, respectively.


The feedback observed by the learner depends on the problem setting.  
The most fundamental setting is the full information setup, where the true output $\yt$ is observed as feedback at the end of each round $t$. 
This setup was extensively investigated in \citet{pmlr-v247-sakaue24a}.  
By contrast, our study investigates the following weaker feedback:
\begin{itemize}%[topsep=2pt,itemsep=0pt, partopsep=0pt, leftmargin=18pt]
    \item \textbf{Bandit feedback}: Only the value of the loss function is observed. 
    Specifically, at the end of each round $t$, the learner observes the target loss value $L(\yht;\yt)$ as feedback.  
    \item \textbf{Delayed feedback}: The feedback is observed with a certain delay. 
    We consider a fixed $D$-round delay setting, i.e., no feedback is received for round $t\leq D$,  
    and for $t>D$, the learner observes either full information feedback $\bmy_{t-D}$ or bandit feedback $L(\hat{\bm{y}}_{t-D}; \bm{y}_{t-D})$.  
\end{itemize}


In this paper, we make the following assumptions:
\begin{assumption}
    \label{asp:online_structured_prediction}
    (1)~There exists $\nu>0$ such that for any distinct $\bm{y},\bm{y}^\prime\in \mathcal{Y}$, it holds that $\|\bm{y}-\bm{y}^\prime\|\geq\nu$.  
    (2) For each $\bm{y}\in\mathcal{Y}$, the target loss function $L(\cdot;\bm{y})$ is defined on $\conv(\mathcal{Y})$, is non-negative, and is affine with respect to its first argument.  
    (3) There exists $\gamma$ such that for any $\bm{y}^\prime\in\conv(\mathcal{Y})$ and $\bm{y}\in\mathcal{Y}$, it holds that $L(\bm{y}^\prime;\bm{y})\leq\gamma\|\bm{y}^\prime-\bm{y}\|$ and $L(\bmy^{\prime};\bmy)\leq 1$. 
    (4) It holds that $L(\bm{y}'; \bm{y})=0$ only if $\bm{y}'=\bm{y}$.
\end{assumption}

As discussed in \citet[Section 2.3]{pmlr-v247-sakaue24a}, these assumptions are natural and hold for a broad range of problem settings and target loss functions, including SELF (see \cref{subsec:self} for the formal definition).


\subsection{Fenchel--Young Loss}\label{subsec:fenchel-young}

We use the Fenchel--Young loss \citep{JMLR_2020_blondel} as the surrogate loss, which subsumes many representative surrogate losses, such as logistic loss, Conditional Random Field (CRF) loss \citep{lafferty01conditional}, and SparseMAP \citep{Niculae18sparse}.
See \citet[Table 1]{JMLR_2020_blondel} for more examples. 
\begin{definition}[{\citealt[Fenchel--Young loss]{JMLR_2020_blondel}}]
    \label{def: Fenchel--Young Loss}
    Let $\Omega:\mathbb{R}^d\rightarrow\mathbb{R}\cup\set{+\infty}$ be a regularization function with $\mathcal{Y}\subset\operatorname{dom}(\Omega)$.  
    The Fenchel--Young loss generated by $\Omega$, denoted by $S_\Omega:\operatorname{dom}(\Omega^\ast)\times\operatorname{dom}(\Omega)\rightarrow\mathbb{R}_{\geq0}$, is defined as
    \[
    S_{\Omega}(\thb;\bmy)\coloneqq\Omega^\ast(\thb)+\Omega(\bmy)-\inpr{\thb,\bmy}.
    \]
\end{definition}
The Fenchel--Young loss has the following properties, which will be useful in the subsequent discussion:
\begin{proposition}[{\citealt[Propositions 2 and~3]{JMLR_2020_blondel} and \citealt[Proposition 3]{pmlr-v247-sakaue24a}}]
    \label{prop:fenchel}
    Let $\Psi:\mathbb{R}^d\rightarrow\mathbb{R}\cup\set{+\infty}$ be a differentiable, Legendre-type function\footnote{
    A function is called Legendre-type if, for any sequence $x_1,x_2,\hdots$ in $\operatorname{int}(\dom(\Psi))$ that converges to a boundary point of $\operatorname{int}(\dom(\Psi))$, it holds that $\lim_{i\rightarrow\infty}\|\nabla\Psi(x_i)\|_2=+\infty$.}  
    that is $\lambda$-strongly convex with respect to $\|\cdot\|$, and suppose that $\conv(\mathcal{Y})\subset\dom(\Psi)$ and $\dom(\Psi^\ast)=\mathbb{R}^d$.  
    Define $\Omega=\Psi+I_{\conv(\yy)}$ and let $S_\Omega$ be the Fenchel--Young loss generated by $\Omega$.  
    For any $\thb\in\mathbb{R}^d$, we define the regularized prediction function as 
    \begin{align*}
        \yho(\thb)&\coloneqq\argmax\{\inpr{\thb,\bmy}-\Omega(\bmy)\::\:\bmy\in\mathbb{R}^d\}\\
        &=\argmax\set{\inpr{\thb,\bmy}-\Psi(\bmy)\::\:\bmy\in\conv(\mathcal{Y})}.
    \end{align*}
    Then, for any $\bmy\in\mathcal{Y}$, $S_\Omega(\thb,\bmy)$ is differentiable with respect to $\thb$, and it satisfies  
    $
    \nabla S_\Omega(\thb;\bmy)=\yho(\thb)-\bmy.
    $
    Furthermore, it holds that  
    $
    S_\Omega(\thb;\bmy)\geq\frac{\lambda}{2}\|\bmy-\yho(\thb)\|^2.
    $ 
\end{proposition}


In what follows, let $S_t(\W)\coloneqq S_\Omega(\W\xt;\yt)$ for simplicity.
Importantly, from the properties of the Fenchel--Young loss, there exists some $b>0$ such that for any $\W\in\ww$,  
\begin{equation}\label{eq:St_smooth}
    \nrm{\nabla S_t (\W)}_{\mathrm{F}}^2\leq b S_t(\W).
\end{equation}
Indeed, from \cref{prop:fenchel} and \cref{asp:online_structured_prediction}, we have 
$
\nrm{\nabla S_t(\W_t)}_\F^2
=
\nrm{\yho(\tht)-\yt}_2^2 \nrm{\xt}^2
\leq
\dix^2\kappa^2\nrm{\yho(\tht)-\yt}^2
\leq
\frac{2\dix^2\kappa^2}{\lambda}\sw
$, 
where we used $\nabla S_t(\wt) =  \prn{\hat{\bm{y}}_\Omega(\bm{\theta}_t) - \yt}\xt^\top$ and $\nrm{\cdot}_2 \leq \kappa \nrm{\cdot}$. 
Thus, \eqref{eq:St_smooth} holds with  $b=\frac{2\dix^2\kappa^2}{\lambda}$.
Below, let $L_t(\bmy)\coloneqq L(\bmy;\yt)$,
$\G_t\coloneqq\nabla S_t(\wt) =  \prn{\hat{\bm{y}}_\Omega(\bm{\theta}_t) - \yt} \bm{x}_t^\top$, and $\L\coloneqq\max_{\W\in\ww}\nrm{\nabla S_t(\W)}_\F$.


\subsection{Examples of Structured Prediction}\label{subsec:pre_examples} 
We present several special cases of structured prediction  
along with specific parameter values introduced so far; see \citet[Section 2.3]{pmlr-v247-sakaue24a} for further details.
\paragraph{Multiclass classification}
Let $\yy=\set{\mathbf{e}_1,\dots,\mathbf{e}_d}$ and $\|\cdot\| = \|\cdot\|_1$.
When using the 0-1 loss, $L(\bmy^{\prime};\bmy)=\ind\brk{\bmy^{\prime}\neq\bmy}$, the parameters in \cref{asp:online_structured_prediction} are $\nu=2$ and $\gamma=\frac{1}{2}$.
The logistic surrogate loss is a Fenchel--Young loss $S_\Omega$ generated by the entropy regularization function $\Omega=\mathsf{H}^s+I_{\Delta_d}$ (up to a constant factor), where $\mathsf{H}^s(\bmy)\coloneqq-\sum_{i=1}^d y_i\log y_i$ and $\Delta_d$ is the $(d-1)$-dimensional probability simplex.
Since $\Omega$ is a $1$-strongly convex function with respect to $\|\cdot\|_1$, we have $\lambda=1$.

\paragraph{Multilabel classification}
Let $\yy=\set{0,1}^d$ and $\|\cdot\| = \|\cdot\|_2$.
When using the Hamming loss as the target loss function $L(\bmy^{\prime};\bmy)=\frac{1}{d}\sum_{i=1}^{d}\ind\brk{y_{i}^{\prime}\neq y_i}$, \cref{asp:online_structured_prediction} is satisfied with $\nu=1$ and $\gamma=\frac{1}{d}$.
The SparseMAP surrogate loss $S_\Omega(\thb,\bmy)=\frac{1}{2}\nrm{\bmy-\thb}_2^2-\frac{1}{2}\nrm{\yho(\thb)-\thb}_2^2$ is a Fenchel--Young loss generated by $\Omega=\frac{1}{2}\nrm{\cdot}^2+I_{\conv(\yy)}$.
Since $\Omega$ is 1-strongly convex with respect to $\|\cdot\|_2$, we have $\lambda=1$.

\paragraph{Ranking}
We consider predicting the ranking of $m$ items. 
Let $\nrm{\cdot} = \nrm{\cdot}_1$, $d=m^2$, and $\yy\subset\set{0,1}^d$ be the set of all vectors representing $m \times m$ permutation matrices.  
We use the target loss function that counts mismatches, $L(\bmy^{\prime};\bmy)=\frac{1}{m}\sum_{i=1}^{m}\ind\brk{y_{i,j_i}^{\prime}\neq y_{i,j_i}}$, where $j_i\in[m]$ is a unique index with $y_{ij_i}=1$ for each $i\in[m]$.  
In this case, the parameters in \cref{asp:online_structured_prediction} satisfy $\nu=4$ and $\gamma=\frac{1}{2m}$.  
We use a surrogate loss given by $S_\Omega(\thb;\bmy)=\inpr{\thb,\yho(\thb)-\bmy}+\frac{1}{\zeta}\mathsf{H}^s(\yho(\thb))$, where $\Omega=-\frac{1}{\zeta}\mathsf{H}^s+I_{\conv(\yy)}$ and $\zeta$ is a parameter controlling the regularization strength. 
The first term in $S_\Omega$ measures the affinity between $\thb$ and $\bmy$, while the second term evaluates the uncertainty of $\yho(\thb)$.  
Since $\Omega$ is $\frac{1}{m\zeta}$-strongly convex, we have $\lambda=\frac{1}{m\zeta}$.


\subsection{Structured Encoding Loss Function (SELF)}\label{subsec:self}
Here we introduce a common class of target loss functions, called the Structured Encoding Loss Function (SELF).
A target loss function is SELF if it can be expressed as  
\begin{equation}\label{eq:self}
    L(\yt; \yht)=\inpr{\yht,\V\yt+\bm{b}}+c(\yt),
\end{equation}
where $\bm{b} \in \R^d$ is a constant vector, $\V\in\R^{d\times d}$ is a constant matrix, and $c:\yy\to\R$ is a function.  
The following loss examples, taken from \citet[Appendix A]{NEURIPS2019_Blondel}, belong to the SELF class: 
\begin{itemize}%[topsep=2pt,itemsep=0pt, partopsep=0pt, leftmargin=18pt]
\item Multiclass classification: the 0-1 loss is a SELF with $\V=\bm{1}\bm{1}^\top-\I$, $\bb=\bm{0}$, and $c(\bmy)=0$.  

\item Multilabel classification: the Hamming loss, 
$L(\bmy^{\prime};\bmy)=\frac{1}{d}\sum_{i=1}^{d}\ind\brk{y^{\prime}_{t,i}\neq y_{i}}$, is a SELF with $\V=-\frac{2}{d}\I$, $\bb=\frac{\bm{1}}{d}$, and $c(\bmy)=\frac{1}{d}\inpr{\bmy,\bm{1}}$, where the last factor is constant if the number of correct labels is fixed.

\item Ranking: the Hamming loss  
$L(\bmy^{\prime};\bmy)=\frac{1}{m}\sum_{i=1}^{m}\ind\brk{y_{i,j_i}^{\prime}\neq y_{i,j_{i}}}$, where $j_i\in[m]$ is a unique index with $y_{i,j_i}=1$ for each $i\in[m]$, is a SELF with  
$\V=-\frac{1}{m}\I$, $\bb=\bm{0}$, and $c(\bmy)=1$.
\end{itemize}
Following \citet{pmlr-v247-sakaue24a}, this study assumes that the target loss function $L$ is a SELF.

\subsection{Randomized Decoding}\label{subsec:randomized_decoding}
\begin{algorithm}[t]
    \caption{Randomized decoding $\phi_\Omega$}
    \label{ALG: randomized decoding}
    \begin{algorithmic}[1]
        \Require {$\btheta\in\mathbb{R}^d$}
            \State {$\yho(\btheta)\leftarrow\argmax\{\langle\btheta,\bm{y}\rangle-\Psi(\bm{y})\::\:\bm{y}\in\conv(\mathcal{Y})\}$}
            \State {$\bm{y}^\ast\leftarrow\argmin\{\|\bm{y}-\yho(\btheta)\|\::\:\bm{y}\in\mathcal{Y}\}$}
            \State {$\Delta^\ast\leftarrow\|\bm{y}^\ast-\yho(\btheta)\|,\:p\leftarrow\min\{1,2\Delta^\ast/\nu\}$}
            \State {Sample $Z \sim \mathrm{Ber}(p)$}
            \LineIf{$Z=0$}{$\yh\leftarrow\bm{y}^\ast$}
            \LineIf{$Z=1$}{$\yh\leftarrow\bm{\tilde{y}}$ where $\bm{\tilde{y}}$ is randomly drawn from $\yy$ so that $\E\brk*{\bm{\tilde{y}}|Z=1}=\yho(\btheta)$}
            \Ensure{$\phi_\Omega(\thb)=\yh$}
    \end{algorithmic}
\end{algorithm}


The procedure of converting the estimated score $\thb$ into a structured output $\bm{\hat{y}}$ is called decoding.  
For this, we employ randomized decoding \citep{pmlr-v247-sakaue24a},  
which plays an essential role particularly in deriving an upper bound independent of the output set size $K = \abs{\mathcal{Y}}$ in \cref{subsec:Bandit_Structured_Prediction_with_SELF}.
The randomized decoding (\cref{ALG: randomized decoding}) selects either the closest $\bm{y}^* \in \yy$ to $\hat{\bm{y}}_\Omega(\thb) \in \conv(\yy)$ or a random $\widetilde{\bm{y}} \in \yy$ satisfying $\E[\widetilde{\bm{y}} \mid Z=1] = \hat{\bm{y}}_\Omega(\thb)$, where $Z$ follows a Bernoulli distribution with a parameter $p$.  
Intuitively, the parameter $p$ is chosen so that if $\hat{\bm{y}}_\Omega(\thb)$ is close to $\bm{y}^*$, the decoding is more likely to return $\bm{y}^*$; otherwise, it is more likely to return $\widetilde{\bm{y}}$, reflecting uncertainty.  
An important property satisfied by the randomized decoding is the following lemma, which we use in the subsequent analysis:
\begin{lemma}[{\citealt[Lemma 4]{pmlr-v247-sakaue24a}}]
    \label{lem:expected_target_bound}
  For any $(\thb, \bmy) \in \mathbb{R}^d\times\yy$, the randomized decoding $\phi_\Omega$ satisfies
  \[
    \E[L(\phi_\Omega(\thb);\bmy)] \leq \frac{4\gamma}{\lambda\nu} S_\Omega(\thb;\bmy),
  \]
  where the expectation is taken with respect to the internal randomness of the randomized decoding.
\end{lemma}









\section{Bandit Feedback}
\label{sec:bandit}
In this section, we present two online structured prediction algorithms in the bandit feedback setup and analyze their regret.  
Our results here are mostly special cases of the regret bounds obtained when handling bandit and delayed feedback (\cref{sec:bandit_and_delayed}). 
Nevertheless, by focusing on the case without delay, we provide a clearer exposition of the core ideas.

\subsection{Randomized Decoding with Uniform Exploration}
\begin{algorithm}[t]
    \caption{Randomized decoding with uniform exploration (RDUE) $\psi_\Omega$}
    \label{ALG:randomized decoding with uniform exploration}
    \begin{algorithmic}[1]
        \Require{$\thb\in\R^n$, $q \in [0,1]$}
        \State {Sample $X \sim \mathrm{Ber}(q)$ 
        }
        \LineIf{$X=0$}{$\bm{\hat{y}}\leftarrow\phi_\Omega(\thb)$}
        \LineIf{$X=1$}{Sample $\bm{y}^\ast \sim \mathrm{Unif}(\mathcal{Y})$ and $\yh\leftarrow\bm{y}^\ast$}
        \Ensure{$\psi_\Omega(\thb)=\yh$}
    \end{algorithmic}
\end{algorithm}
Here, we discuss the properties of the decoding function, \emph{Randomized Decoding with Uniform Exploration (RDUE)}, which will be used in both algorithms to convert scores into outputs.  
As discussed in \cref{subsec:randomized_decoding}, in online structured prediction with full information feedback, the randomized decoding (\cref{ALG: randomized decoding}) was introduced as a decoding function \citep{pmlr-v247-sakaue24a}.  
However, naively applying the randomized decoding does not lead to a satisfactory regret bound under bandit feedback.  
We extend the framework of the randomized decoding to handle bandit feedback effectively.

RDUE (\cref{ALG:randomized decoding with uniform exploration}) is a procedure that, with probability $q \in [0,1]$, selects $\hat{\bmy}$ uniformly at random from $\yy$,  
and with probability $1-q$, selects the output of the randomized decoding.  
Using RDUE, we define $p_t(\bm{y})$ as the probability that $\hat{\bm{y}}_t$ coincides with $\bm{y}$ at round $t$.  
Note that for any $\bm{y} \in \mathcal{Y}$, it holds that  
$
p_t(\bm{y})\geq\frac{q}{\K}.
$
Furthermore, similar to the property of the randomized decoding in \cref{lem:expected_target_bound}, RDUE satisfies the following property:
\begin{lemma}
    \label{lem:bound of randomized decoding with uniform exploration}
    For any $(\thb,\bm{y})\in\R^d\times\mathcal{Y}$, RDUE $\psi_\Omega$ satisfies
    \[
        \E\brk*{L(\psi_\Omega(\thb);\bm{y})}\leq\frac{4\gamma}{\lambda\nu}(1-q)S_\Omega(\thb;\bmy)+q\frac{\K-1}{\K  },
    \]
  where the expectation is taken with respect to the internal randomness of RDUE. 
\end{lemma}
\begin{proof}
Since the output of the randomized decoding is chosen with probability $1-q$,  
and a uniformly random output is chosen with probability $q$, we have  
$
\E\brk*{L(\psi_\Omega(\thb);\bmy)}=(1-q)\E\brk*{L(\phi_\Omega(\thb);\bmy)}+q\frac{\K-1}{\K},
$
where we used $\L(\cdot;\cdot)\leq 1$ and $\phi_\Omega$ is the randomized decoding.  
Hence, combining this with \cref{lem:expected_target_bound}, we obtain the desired bound.
\end{proof}

Additionally, this section makes the following assumption:
\begin{assumption}
    \label{asp:bandit_a}
    There exists $a\in\prn{0,1}$ such that  
    \[
    \expect{L_t(\yht)}\leq(1-a)\sw+q.
    \]  
    Here, $\expect{\cdot}$ denotes the conditional expectation given the random variables $\hat{\bmy}_1,\dots,\hat{\bmy}_{t-1}$.  
\end{assumption}
This assumption can be satisfied by using RDUE and letting   
$a \leq 1-\frac{4\gamma}{\lambda\nu}(1-q)$  
if $\lambda>\frac{4\gamma}{\nu}(1-q)$, according to \cref{lem:bound of randomized decoding with uniform exploration}.  
In what follows, let $a = 1-\frac{4\gamma}{\lambda\nu}$.
Note that $\lambda \geq \frac{4\gamma}{\nu}$ holds in the cases of multiclass classification, multilabel classification, and ranking, as discussed in \cref{subsec:pre_examples}.


\subsection{Online Gradient Descent}\label{subsec:ogd}
The algorithm in this section uses the adaptive Online Gradient Descent (OGD, \citealt{streeter2010regretonlineconditioning}) as $\alg$.
OGD updates $\wt$ to $\W_{t+1}$ using a gradient $\bm{G}_t$ and learning rate $\eta_t$ by
$
    \W_{t+1} \leftarrow \Pi_{\ww} \prn*{\wt - \eta_{t} \bm{G}_t},
$
where $\Pi_{\ww}(\bm Z) = \argmin_{\bm{X} \in \ww} \nrm{\bm X - \bm Z}_{\F}$.
OGD with appropriately chosen learning rate $\eta_t$ achieves the following bound:
\begin{lemma}[{\citealt[Theorem 4.14]{orabona2023modernintroductiononlinelearning}}]
    \label{lem:ogd}
    Suppose that we set the learning rate to $\eta_t=\frac{B}{\sqrt{2 \sum_{i=1}^t\nrm{\bm{G}_i}_{\mathrm{F}}^2}}$ and do not update on rounds when $\bm{G}_t$ is the all-zero matrix.
    Then, for any $\U\in\ww$, OGD achieves 
    $
        \sumt{\inpr{\bm{G}_t, \W_t - \U}}
        \leq \sqrt{2}B\sqrt{\sumt{\nrm{\bm{G}_t}_{\mathrm{F}}^2}}.
    $
\end{lemma}

\subsection{$O(\sqrt{K T})$ Regret Algorithm}
\label{subsec:Bandit_Structured_Prediction_with_General_Losses}
Here, we present an algorithm that achieves a regret upper bound of $O(\sqrt{\K T})$.
\paragraph{Algorithm based on inverse-weighted estimator}
In the bandit setting, the true output $\yt$ is not observed,  
and thus it is necessary to estimate the gradient required for updating $\wt$.  
To deal with this, we use the following inverse-weighted gradient estimator:
\begin{equation}\label{eq:inverse_weighted_est}
    \gtil\coloneqq\frac{\ind[\yht=\yt]}{p_t(\yt)}\G_t,
\end{equation}
where we recall that $\G_t=\nabla S_t(\wt) = \prn{\hat{\bm{y}}_\Omega(\bm{\theta}_t) - \yt} \bm{x}_t^\top$.
Note that $\gtil$ is unbiased, i.e., $\E\brk[\big]{\gtil}=\G_t$.
We use RDUE with $q=B\sqrt{\K/T}$ as the decoding function (assuming $T \geq B^2 \K$ for simplicity).  
For $\alg$, we employ the adaptive OGD in \cref{subsec:ogd} with the learning rate of
$
\eta_t=\frac{B}{\sqrt{2 \sum_{i=1}^t\nrm{\tilde{\G}_i}_{\mathrm{F}}^2}}.
$

\begin{remark}\label{rem:zero-loss}
This study defines the bandit feedback as the value of the target loss function $L_t(\yht)$.
Note, however, that the above algorithm operates using only the weaker feedback of $\ind\brk*{\yht\neq\yt}$. 
\end{remark}

\paragraph{Regret bounds and analysis}
The above algorithm achieves the following regret bound:
\begin{theorem}\label{thm:bandit_regret_expectation_abstract}
    The regret of the above algorithm is bounded by
    $
        \E\brk{\reg}\leq \prn*{\frac{b}{2a}+1}B\sqrt{\K T}.
    $
\end{theorem}
The $O(\sqrt{\K T})$ bound has an optimal dependency on $T$  
and matches the $\sqrt{T}$ lower bound in the special case of online multiclass classification with bandit feedback \citep[Corollary 1]{NEURIPS2021_Hoeven}.  
Furthermore, our bound improves the existing $O(\sqrt{KT})$ bound by \citet{NEURIPS2020_Hoeven} by a factor of $\sqrt{\K}$. 
Note that, due to differences in the target loss function, our result is not directly comparable to the $\sqrt{\K T}$ bound in \citet{NEURIPS2021_Hoeven}. A more detailed discussion can be found in \cref{app:Discussio_on_the_Difference_in_Surrogate_Losses}.
\begin{proof}
From the convexity of $S_t$ and the unbiasedness of $\gtil$, we have 
$
    \E\brk*{\sumt{\prn{\sw-\su}}}
    \leq
    \E\brk*{\sumt{\inpr{\G_t,\wt-\U}}}
    =
    \E\brk*{\sumt{\inpr{\gtil,\wt-\U}}}.
 $
From \cref{lem:ogd}, this is further upper bounded as
$
    \E\brk*{\sumt{\inpr{\gtil,\wt-\U}}}
    \leq
    \sqrt{2}B\sqrt{\E\brk*{\sumt{{\nrm{\gtil}_{\mathrm{F}}^2}}}}
    \leq
    B\sqrt{\frac{2b\K }{q}\E\brk*{\sumt{\sw}}},
$
where in the first inequality we used Jensen's inequality and 
in the last inequality we used
$
\expect{\|\gtil\|_{\mathrm{F}}^2}
=
\frac{\|\G_t\|_{\mathrm{F}}^2}{p_t(\yt)}
\leq
\frac{\K }{q}\|\G_t\|_{\mathrm{F}}^2
\leq
\frac{b\K }{q}\sw,
$
which follows from $p_t(\bm{y}) \geq K /q$ and \eqref{eq:St_smooth}.
Therefore, from \cref{asp:bandit_a}, 
we have
$
    \E\brk{\reg}
    \leq
    \E\brk*{\sumt{\prn*{(1-a)\sw-\su}}}+qT
    \leq 
    B\sqrt{\frac{2b\K }{q} \E\brk*{\sumt{\sw}}}-a \E\brk*{\sumt{\sw}}+qT
    \leq
    \frac{bB^2\K }{2aq}+qT
    ,
$
where the last inequality follows from $c_1\sqrt{x}-c_2x\leq{c_1^2}/\prn{4c_2}$ for $x \geq 0$, $c_1\geq 0$, and $c_2>0$.
Finally, substituting $q=B\sqrt{\K/T}$ into the last inequality yields the desired bound.
\end{proof}

We can also prove the following high-probability bound:
\begin{theorem}\label{thm:bandit_high_prob}
Let $\delta \in (0,1)$.
Then with probability at least $1 - \delta$, the same algorithm as in \cref{thm:bandit_regret_expectation_abstract}, but with a different choice of $q$, achieves the regret bound of 
$\mathcal{R}_T = O\prn[\Big]{\sqrt{KT \log (1/\delta)} + \log(1/\delta)}$,
where we omit the dependencies on parameters other than $K$, $T$, and $\delta$.
\end{theorem}
A more precise statement and proof of this theorem are provided in \cref{app:proof_bandit_high_prob}.
To prove this theorem, we follow the analysis of \cref{thm:bandit_regret_expectation_abstract} and use Bernstein's inequality.  
To address the challenges posed by the randomness introduced by bandit feedback,  
we adopt an approach similar to that used by \citet{NEURIPS2021_Hoeven}, and
arguably, we have successfully simplified their analysis.


\subsection{$O(T^{2/3})$ Regret Algorithm}
\label{subsec:Bandit_Structured_Prediction_with_SELF}
While the $O(\sqrt{KT})$ regret bound given above is desirable in terms of the dependence on $T$, the dependence on $K = \abs{\yy}$ is undesirable for general structured prediction.  
In fact, we have $\K=\binom{d}{m}$ in multilabel classification with $m$ correct labels and $\K=m!$ in ranking with $m$ items.  
To address this issue, we present an algorithm that significantly improves the dependence on $\K$ when the target loss function belongs to a special class of SELF satisfying the following assumptions:\looseness=-1
\begin{assumption}\label{asp:self}
(i) $\V$ is invertible, and $\bm{b}$ and $c(\cdot)$ are known and non-negative.
(ii) Let $\bm{Q} = \E_{\bm{y} \sim \mu} \brk{ \bm{y} \bm{y}^\top }$, where $\mu$ is the uniform distribution over $\yy$. At least one of the following two conditions holds: 
(ii-a) $\bm{Q}$ is invertible, or 
(ii-b) for any $\bm{y} \in \yy$, $\V \bm{y}$ lies in the linear subspace spanned by vectors in $\yy$. 
(iii) For some $\omega > 0$, it holds that\looseness=-1
\begin{equation}\label{eq:def_omega}
    \tr \prn*{ \V^{-1} \bm{Q}^+ \prn{\V^{-1}}^\top } \leq \omega.
    \nonumber
\end{equation} 
\end{assumption}
The first condition is true in the examples in \cref{subsec:self}, assuming that the number of correct labels is fixed in multilabel classification.
The second one is satisfied if $\yy$ consists of $d$ linearly independent vectors or $\V$ is proportional to the identity matrix; either is true in those examples. 
It is also not difficult to derive bounds on $\omega$ in those examples.\looseness=-1

\paragraph{Algorithm based on pseudo-inverse matrix estimator}
As in the case of \cref{subsec:Bandit_Structured_Prediction_with_General_Losses}, we begin by providing a method to estimate the gradient.  
Define $\pt \coloneqq\E_{\bmy\sim p_t}[\bmy\bmy^\top]$.
Then, we define the estimator $\ytilde$ of $\yt$ by
\[
    \ytilde\coloneqq\inverse{\V}\bm{P}_t^+\yht\inpr{\yht,\V\yt},
\]
where $\bm{P}_t^+$ is the Moore--Penrose pseudo-inverse matrix of $\bm{P}_t$.
It is important to note that, given that $\bm{b}$ and $c(\cdot)$ are known,  
$
\inpr{\yht,\V\yt}=L_t(\yht)-\inpr{\yht,\bm{b}}-c(\yt)
$
can be computed.  
Note that $\ytilde$ is unbiased, i.e.,
$
\expect{\ytilde}=\yt
$
from the first requirement of \cref{asp:self}.

Using this $\ytilde$, we define the gradient estimator $\gtil$ by
\begin{equation}
    \label{eq:gtil_self}
    \gtil\coloneqq\prn*{\yho(\tht)-\ytilde}\xt^\top,
\end{equation}
whose expectation is 
$
    \E\brk[\big]{\gtil}=\G_t.
$
Our estimator is based on the estimators used in adversarial linear bandits and adversarial combinatorial full-bandits \citep{dani07price,abernethy08competing,comband}.  

We use RDUE with $q=\prn*{\frac{4 \omega B^2\dix ^2}{ T }}^{1/3}$ as the decoding function  
(assuming $T \geq 4 \omega B^2\dix ^2$ for simplicity).  
For updating $\wt$, we employ the adaptive OGD in \cref{subsec:ogd} as $\alg$ with the learning rate of 
$
\eta_t=\frac{B}{\sqrt{2 \sum_{i=1}^t\nrm{\tilde{\G}_i}_{\mathrm{F}}^2}}.
$


\paragraph{Regret bounds}
The above algorithm achieves the following regret bound,  
which does not directly depend on $\K$:
\begin{theorem}
    \label{thm:bandit_regret_pseudo_estimator}
    The above algorithm achieves
    $
    \E\brk{\reg}
    \leq
    \frac{bB^2}{a}
    +
    O\prn[\big]{ \omega^{1/3} \prn*{ B \dix T}^{2/3} }.
    $
\end{theorem}
The proof can be found in \cref{app:sub_bandit_regret_pseudo_estimator}.  
By using the estimator based on the pseudo-inverse matrix, 
we can upper bound the second moment of the gradient estimator $\gtil$ without $\K$, which allows us to establish the improved regret bound that does not explicitly depend on $\K$.  

The regret bound in \cref{thm:bandit_regret_pseudo_estimator} yields the different bounds on each problem setup as follows:
\begin{corollary}\label{cor:thm_self}
The above algorithm achieves
$
    \E\brk{\reg}\leq\frac{bB^2}{a}+ O\prn*{ \prn{B \dix  d T}^{2/3}}
$
in multiclass classification with the 0-1 loss,
$
    \E\brk{\reg}\leq\frac{bB^2}{a}
    +
    O\prn*{ \prn{d^5 /m(d-m)}^{1/3} \prn{B \dix T}^{2/3}}
$
in multilabel classification with $m$ correct labels and the Hamming loss,
and 
$
    \E\brk{\reg}\leq\frac{bB^2}{a}+ O \prn*{ m^{5/3}\prn{B\dix T}^{2/3}}
$
in ranking with the number of items $m$ and the Hamming loss.
\end{corollary}
The proof of \cref{cor:thm_self} is deferred to \cref{app:SELF_upper_discussion_deferred}.
The bound for multilabel classification with $m$ correct labels significantly improved the $O(\sqrt{\K T})$ bound in \cref{subsec:Bandit_Structured_Prediction_with_General_Losses},  
which has a dependency of $\sqrt{\binom{d}{m}}$,
and
the bound for ranking significantly improved the $O(\sqrt{\K T})$ bound in \cref{subsec:Bandit_Structured_Prediction_with_General_Losses}, which has a dependency of $\sqrt{m!}$.\looseness=-1

\section{Delayed Full-Information Feedback}
\label{sec:delay}
This section discusses online structured prediction with delayed full-information feedback and provides an algorithm that achieves a surrogate regret bound of $O(D^{2/3} T^{1/3})$, a better bound than $O(\sqrt{D T})$ that can be achieved with a standard OCO algorithm under delayed feedback \citep{joulani13online}. 
Below, we make the following assumption.\looseness=-1
\begin{assumption}
\label{asp:delayed_a}
There exists a constant $a\in\prn{0,1}$ which satisfies
\[
    \expect{L_t(\yht)}\leq(1-a)\sw.
\]
\end{assumption}
From \Cref{lem:expected_target_bound}, if $\lambda>\frac{4\gamma}{\nu}$, this condition is satisfied with $a=1-\frac{4\gamma}{\lambda\nu}$ by using the randomized decoding. 
We suppose that such a decoding function is used in this section.

\paragraph{Algorithm}
For updating $\wt$, we employ the Optimistic Delayed Adaptive FTRL (ODAFTRL) algorithm proposed by \citet{pmlr-v139-flaspohler21a}.  
In ODAFTRL, given a gradient~$\bm{G}_t$ at round $t$, $\W_t$ is updated as
\begin{equation}
    \W_{t+1}
    =
    \argmin_{\W\in \ww} \set*{ \sum_{i=1}^{t-D} \inpr*{\bm{G}_i,\W}+\frac{\lambda_t \nrm{\W}_{\F}^2}{2} },
    \nonumber
\end{equation}
where $\lambda_t\geq0$ is the regularization parameter.  
Due to space constraints, the details of the algorithm are provided in \cref{app:sub_odaftrl}.
By updating $\lambda_t$ using an AdaHedge-type algorithm called AdaHedgeD,  
ODAFTRL achieves the following regret upper bound:
\begin{lemma}[{Informal version of \citealt[Theorem 12]{pmlr-v139-flaspohler21a}}]\label{lem:ODAFTRL_bound}
Consider the setting with delayed full-information feedback.
Then, for any $\U\in\ww$, ODAFTRL with the AdaHedgeD update of $\lambda_t$ achieves
$
    \sumt{(\sw\!-\!\su)}
    \leq
    \sumt{\inpr{\G_t, \W_t - \U}}
    =
    O\prn[\Big]{\sqrt{\sumt{(\nrm{\G_t}_\F^2\!+\!D\nrm{\G_t}_\F)}}}.
$
\end{lemma}

\paragraph{Regret bounds and analysis}
The algorithm described above achieves the following bound:
\begin{theorem}
    \label{thm:delayed_regret_expectation_abstract}
    The above algorithm achieves
    $
        \E\brk{\reg}=O(D^{2/3}T^{1/3}).
    $
\end{theorem}
Here, we provide a proof sketch; the complete proof can be found in \cref{app:sub_delayed_regret_expectation_abstract}.
\begin{proof}[Proof sketch]
    \Cref{lem:ODAFTRL_bound} with $\nrm{\G_t}_\F^2\leq b\sw$ in \eqref{eq:St_smooth} and Cauchy--Schwarz yields
    $\sumt{(\sw-\su)}=O(\sqrt{S_{1:T}}+\prn{D^2TS_{1:T}}^{1/4})$, where $S_{1:T}=\sumt{\sw}$.
    Hence, from \cref{asp:delayed_a}, we have
    $
        \E\brk{\reg}\leq \sumt{(\sw-\su)}-a\sumt{\sw}
        =O(\sqrt{S_{1:T}}+\prn{D^2TS_{1:T}}^{1/4})-aS_{1:T}
        =O\prn{D^{2/3}T^{1/3}},
    $
    where we used $c_1\sqrt{x}-c_2x\leq{c_1^2}/\prn{4c_2}$ and $c_1x-c_2x^4\leq\prn*{{c_1^4}/\prn{4c_2}}^{1/3}$ for $x\geq0$, $c_1\geq 0$, and $c_2>0$.
\end{proof}
We can also prove the following high-probability bound; see \cref{app:sub_delayed_regret_probability_abstract} for the proof:
\begin{theorem}
    \label{thm:delayed_regret_probability_abstract}
    For any $\delta \in (0,1)$,
    with probability at least $1 - \delta$, the above algorithm achieves 
    $
        \reg = O\prn{\log({1}/{\delta}) + D^{2/3}T^{1/3}}.
    $
\end{theorem}


\section{Delayed Bandit Feedback}\label{sec:bandit_and_delayed}
Given the results so far, it is natural to explore online structured prediction with delayed bandit feedback.
We construct algorithms for this setup by combining the theoretical developments from \cref{sec:bandit} and \cref{sec:delay}. 
This section assumes \cref{asp:bandit_a}, as in \cref{sec:bandit}.
\subsection{$O(\sqrt{D K T})$ Regret Algorithm}\label{subsec:bandit_delay_general}
We use RDUE with $q=B\sqrt{DK/T}$ as the decoding function (assuming $T\geq DB^2K$ for simplicity), the gradient estimator $\gtil$ in \eqref{eq:inverse_weighted_est},
and ODAFTRL with the AdaHedgeD update as $\alg$. 
This algorithm attains the following bound:
\begin{theorem}
    \label{thm:delay_bandit_bound_general_abstract}
    The above algorithm achieves
    $
        \E\brk{\reg} = O\prn{\sqrt{DKT}}.
    $
\end{theorem}
The proof can be found in \cref{app:bandit_delayed_general}.  
Due to the introduction of the delay, the regret bound worsens by a factor of $\sqrt{D}$ compared to the non-delayed case,
which is natural when considering analyses of adversarial cases in delayed feedback \citep{joulani13online,zimmert20optimal,manwani2022delaytronefficientlearningmulticlass}.



\subsection{$O(D^{1/3} T^{2/3})$ Regret Algorithm}\label{subsec:bandit_delay_self}
Here, we make the same assumptions on the target loss function as in \cref{subsec:Bandit_Structured_Prediction_with_SELF}.
We provide an algorithm that improves the dependence on $\K$ from \cref{subsec:bandit_delay_general}.
We use RDUE with $q=\prn*{\frac{\omega B^2 \dix^2 D}{T}}^{1/3}$ as the decoding function (assuming $T \geq \omega B^2 \dix^2 D$ for simplicity), the gradient estimator $\gtil$ in \eqref{eq:gtil_self}, and ODAFTRL with the AdaHedgeD update as $\alg$.
This algorithm achieves the following bound:
\begin{theorem}
    \label{thm:delay_bandit_bound_self_abstract}
    The above algorithm achieves
    $
        \E\brk{\reg} = O\prn{
            D^{1/3} T^{2/3}
        }.
    $
\end{theorem}
The proof can be found in \cref{app:bandit_delayed_self}.  
Due to the introduction of the delay, the regret bound worsens by a factor of $D^{1/3}$  
compared to the non-delayed bandit feedback case.


\section{Experiments}
\label{sec: experiment}
\begin{figure}[t]
    \centering
    \includegraphics[keepaspectratio, scale=0.3]{experiment_box_MNISTtheo_B10_rep20_for_arxiv.eps}
    \vspace{-10pt}
     \caption{A box plot of error rate of the MNIST experiment for multiclass classification with bandit feedback.
     }
    \label{fig:experiment mnist}
\end{figure}

This section presents numerical experiment results for online multiclass classification  
under bandit feedback.  
We compare three algorithms:  
Gaptron~\Citep{NEURIPS2020_Hoeven} with logistic loss and hinge loss as surrogate losses,  
Gappletron~\Citep{NEURIPS2021_Hoeven} with logistic loss as the surrogate loss,  
and our proposed algorithm from \cref{subsec:bandit_delay_general}.  
The parameters for each algorithm are set based on their theoretical values.  
We use the MNIST dataset~\citep{lecun2010mnist}, a dataset of digit images.  
The diameter $B$ of $\ww$ is fixed at $10$.  
We repeated the experiment for 20 times, and the boxplot of the obtained misclassification rates is summarized in \cref{fig:experiment mnist}.  
From \cref{fig:experiment mnist}, we observe that our method achieves the lowest misclassification rate.  
Despite not being specialized for multiclass classification,  
our approach outperforms existing algorithms designed for multiclass classification  
on real data with $K = 10$.  
Further experiments can be found in \cref{app: experiment},  
and related discussions are provided in \cref{app:Discussio_on_the_Difference_in_Surrogate_Losses}.\looseness=-1













\vspace{-5pt}
\section{Conclusion}
This paper presents a novel unified framework, UniBrain, the first end-to-end model to jointly perform a diverse set of brain imaging analysis tasks, including extraction, registration, segmentation, parcellation, network generation and classification. UniBrain integrates heterogeneous information into a single system, enabling efficient knowledge transfer across different modules, and avoiding the need for extensive task-specific labels. Experimental results show that UniBrain outperforms state-of-the-art methods in all tasks while also demonstrating robustness and time efficiency.

\section*{Acknowledgments}
TT is supported by JST ACT-X Grant Number JPMJAX210E and JSPS KAKENHI Grant Number JP24K23852,
SS is supported by supported by JST ERATO Grant Number JPMJER1903,
and KY is supported by JSPS KAKENHI Grant Number JP24H00703.

\bibliography{bib_list}
% \bibliographystyle{plainnat}
\bibliographystyle{icml2025}

\newpage
\appendix
%

\section{Appendix Notation and Definitions}
We often use the shorthand $(a)_+ \defeq \max(a,0)$ as well as the shorthand $\k(\xset,\xset)$ to represent the matrix $(\k(\x_i,\x_j))_{i,j=1}^n$. 
In addition, for each kernel $\k$, we let $\rkhs$ and $\knorm{\cdot}$ represent the associated reproducing kernel Hilbert space (RKHS) and RKHS norm, so that $\ball_{\kernel}=\{ f\in\rkhs : \knorm{f} \leq 1\}$ and define
\begin{talign}
(\Pin - \Qout)\k \defeq \frac{1}{\nin}\sum_{x\in\xin} \k(\x,\cdot) - \frac{1}{\nout}\sum_{x\in\xout} \k(\x,\cdot).
\end{talign}
%
We also relate our definition of a sub-Gaussian thinning algorithm (\cref{def:alg-subg}) to several useful notions of sub-Gaussianity.
%
\begin{definition}[\tbf{Sub-Gaussian vector}]\label{def:vector-subg}
We say that a random vector $\diff \in \R^n$ is \emph{$(\K,\subg)$-sub-Gaussian on an event $\event$} if $\K$ is SPSD and $\subg>0$ satisfies 
\begin{talign}\label{eq:vector-subg}
    \Esubarg{\event}{\exp(\bu^\top \K \diff)} \leq \exp(\frac{\subg^2}{2} \cdot \bu^\top \K \bu)
    \qtext{for all}
    \bu \in \reals^n.
\end{talign}
If, in addition, the event has probability $1$, we say that $\w$ is \emph{$(\K,\subg)$-sub-Gaussian}.
\end{definition}
%
Notably, a thinning algorithm is $(\K,\subg,\delta)$-sub-Gaussian if and only if its associated vector $\pin-\qout$ is $(\K,\subg)$-sub-Gaussian on an event $\event$ of probability at least $1-\delta/2$.

%
\begin{definition}[\tbf{Sub-Gaussian function}]\label{def:function-subg}
For a kernel $\kernel$, %
we say that a random function $\fsubg\in \rkhs$ is \emph{$(\kernel,\subg)$-sub-Gaussian on an event $\event$} if $\subg > 0$ satisfies
\begin{talign}\label{eq:function-subg}
    \Esubarg{\event}{\exp(\inner{f}{\fsubg}_{\kernel})} \leq \exp(\frac{\subg^2}{2}\cdot \knorm{f}^2)
    \qtext{for all}
    f \in\rkhs.
\end{talign}
If, in addition, the event has probability $1$, we say that $\fsubg$ is \emph{$(\kernel,\subg)$-sub-Gaussian}.
\end{definition}
Our next two lemmas show that for finitely-supported signed measures like $\Pin-\Qout$, this notion of functional sub-Gaussianity is equivalent to the prior notion of vector sub-Gaussianity, allowing us to use the two notions interchangeably. 
%
%
Hereafter, we say that $\k$ generates a SPSD matrix $\K$ if $\k(\xset,\xset) = \K$. 

\begin{lemma}[\tbf{Functional sub-Gaussianity implies vector sub-Gaussianity}]
\label{lem:funct_subg_vector_subg}
In the notation of \cref{def:alg-subg}, if $(\Pin - \Qout)\kernel$ is $(\kernel,\subg)$-sub-Gaussian on an event $\event$ and $\kernel$ generates $\K$, then the vector $\pin - \qout$ is $(\K,\subg)$-sub-Gaussian on $\event$.
%
\end{lemma}
%

%
%
%
%
%
%

%
%
%
%
%
%
%
%
%
%
\begin{proof}
%
Suppose $(\Pin - \Qout)\kernel$ is $(\kernel,\subg)$-sub-Gaussian on an event $\event$, fix a vector $\bu\in \reals^n$, and define the function
\begin{talign}
    f_{\bu} \defeq \sumn u_i \kernel(\cdot, x_i) \in \rkhs.
\end{talign}
By the reproducing property, 
\begin{talign}\label{eq:hnorm-of-fu}
    \bu^\top \K (\pin -\qout) = \inner{f_{\bu}}{(\Pin-\Qout)\kernel}_{\kernel} \qtext{and} \knorm{f_{\bu}}^2 = \bu^\top \K \bu.
\end{talign}
Invoking the representations \cref{eq:hnorm-of-fu} and the functional sub-Gaussianity condition \cref{eq:function-subg} we therefore obtain
\begin{talign}
    \Esubarg{\event}{\exp(\bu^\top \K(\pin-\qout)} &= \Esubarg{\event}{\exp(\inner{f_{\bu}}{(\Pin-\Qout)\kernel}_{\kernel})} 
    \leq \exp(\knorm{f_{\bu}}^2 \cdot \frac{\subg^2}{2}) 
    = \exp(\bu^\top \K \bu \cdot \frac{\subg^2}{2}),
\end{talign}
so that $\pin-\qout$ is $(\K,\subg)$-sub-Gaussian on the event $\event$ as claimed.
\end{proof}

%

\begin{lemma}[\tbf{Vector sub-Gaussianity implies functional sub-Gaussianity}]
\label{lem:vector_subg_funct_subg}
In the notation of \cref{def:alg-subg}, if $\pin - \qout$ is $(\K,\subg)$-sub-Gaussian on an event $\event$ and $\kernel$ generates $\K$, then $(\Pin - \Qout)\kernel$ is $(\kernel,\subg)$-sub-Gaussian on $\event$.
\end{lemma}
\begin{proof}
Suppose $\pin-\qout$ is $(\K,\subg)$-sub-Gaussian on an event $\event$, fix a function $f\in \rkhs$, and consider the set 
\begin{talign}
\Lset \defeq \braces{f_{\bu} \defeq \sum_{i=1}^n u_i \kernel(\cdot,x_i) : \bu \in \reals^n}.
\end{talign} 
Since $\Lset$ is a closed linear subspace of $\rkhs$, we can decompose $f$ as $f  = f_{\bu} + f_\perp$,
where $\bu\in\Rn$ and $f_\perp$ is orthogonal to $\Lset$ \citep[Theorem 12.4]{rudin1991functional},
%
so that 
\begin{talign}\label{eq:knorm-decomposition}
    \knorm{f}^2 = \knorm{f_{\bu}}^2 + \knorm{f_\perp}^2\qtext{and} \knorm{f_{\bu}}^2 = \bu^\top \K \bu.
\end{talign}
Invoking the orthogonality of $f_\perp$ and $(\Pin - \Qout)\kernel\in \Lset$, the reproducing property representations \cref{eq:hnorm-of-fu}, and the vector sub-Gaussianity condition \cref{eq:vector-subg}, we find that
\begin{talign}
    \Esubarg{\event}{\exp(\inner{f}{(\Pin-\Qout)\kernel}_{\kernel})} 
    &= \Esubarg{\event}{\exp(\inner{f_{\bu} + f_\perp}{(\Pin - \Qout) \kernel}_{\kernel})} 
    = \Esubarg{\event}{\exp(\bu^\top \K (\pin - \qout)})\\
    &\leq \exp(\bu^\top \K \bu \cdot \frac{\subg^2}{2}) 
    \sless{\cref{{eq:knorm-decomposition}}} \exp(\knorm{f}^2 \cdot \frac{\subg^2}{2}),
\end{talign}
so that $(\Pin-\Qout)\kernel$ is $(\kernel,\subg)$-sub-Gaussian on the event $\event$ as claimed.
\end{proof}


We end our discussion about the versions of sub-Gaussianity considered above by presenting the standard fact about the additivity of sub-Gaussianity parameters under summation of independent sub-Gaussian random vectors, adapted to our setting.

\begin{lemma}[\tbf{Vector sub-Gaussian additivity}]\label{lem:K_sub_gsn_additivity}
    Suppose that, for each $j\in [m]$, 
    $\Delta_j\in\reals^n$ is $(\mbf K,\subg_j)$ on an event $\event[j]$ given $\Delta_{1:(j-1)}\defeq (\Delta_1,\ldots,\Delta_{j-1})$ and $\event[\leq j-1]\defeq \bigcap_{i=1}^{j-1}\event[i]$. 
    Then $\sum_{j=1}^m \Delta_j$ is $(\mbf K, (\sum_{j=1}^m \subg_j^2)^{1/2})$-sub-Gaussian on $\event[\leq m]$.
    %
    \end{lemma}
    \begin{proof}
    Let $\event[\leq s] = \bigcap_{j=1}^s\event[j]$ for each $s\in [m]$.
    We prove the result for $\mc Z_s = \sum_{i=1}^s \Delta_j$ by induction on $s\in [m]$. 
    The result holds for the base case of $s=1$ by assumption. For the inductive case, suppose the result holds for $s\in [m-1]$. Fixing $\bu\in \R^n$, we may apply the tower property, our conditional sub-Gaussianity assumption, and our inductive hypothesis in turn to conclude
    \begin{talign}
        \Earg{\exp(\inner{\bu}{\K \sum_{j=1}^{s+1} \Delta_j})\indic{\event[\leq s+1]}} &= \Earg{\exp(\inner{\bu}{\K \sum_{j=1}^{s} \Delta_j})\indic{\event[\leq s]} \Earg{\exp(\inner{\bu}{\Delta_{s+1}})\indic{\event[s+1]} \mid \Delta_{1:s},\event[\leq s]} } \\
        &\leq \Earg{\exp(\inner{\bu}{\K \sum_{j=1}^{s} \Delta_j})\indic{\event[\leq s]}} \exp\parenth{\frac{\subg_{s+1}^2}{2}\cdot \bu^\top \K \bu}
        \leq \exp\big( \frac{\sum_{j=1}^{s+1} \subg_j^2}{2} \cdot \bu^\top \K \bu\big).
    \end{talign}
    Hence, $\mc Z_{s+1}$ is $(\K,(\sum_{j=1}^{s+1} \subg_j^2)^{1/2})$-sub-Gaussian on $\event[\leq s+1]$, and the proof is complete.
    \end{proof}



%



%

%
%
%
%
%
%
%
%
%
%
%
%
%
%
%
%



\section{Additional Related Work}\label{app:additional_related_work}
We discuss additional related work that could not be included above.

\paragraph{Structured prediction}
Before the introduction of the Fenchel--Young loss framework, \citet{Niculae18sparse} proposed SparseMAP, which used the squared $\ell_2$-norm regularization.
The Fenchel--Young loss, described in \cref{subsec:fenchel-young}, is built upon the idea of SparseMAP. 
The Structure Encoding Loss Function (SELF) was introduced by \citet{ciliberto16consistent,ciliberto20general} to analyze the relationship between surrogate and target losses, a concept known as Fisher consistency.
For more extensive literature, we refer the reader to \citet[Appendix A]{pmlr-v247-sakaue24a}.

\paragraph{Online classification with full and bandit feedback}
In the full information setup, PERCEPTRON is one of the most representative algorithms for binary classification \citep{Rosenblatt1958-sh}, and the multiclass setting has also been extensively studied \citep{,crammer2003ultraconservative,Fink_2006}.
Online logistic regression is another relevant research stream, with \citet{foster18logistic} being a particularly representative study. 
The study of the bandit setup was initiated by \citet{Kakade2008EfficientBA}, and it has since been extensively explored in subsequent research \citep{hazan11newtron,beygelzimer17efficient,foster18logistic}. However, to the best of our knowledge, no prior work has addressed general structured prediction under bandit feedback. A most related study is the work by \citet{gentile14multilabel}, which investigated online multilabel classification and ranking. 
However, their setting assumes access to feedback of the form $\set{\ind[\bmy_{t,i} \neq \hat{\bmy}_{t,i}]}_{i}$, which is more informative than bandit feedback and differs from our setup.
\Citet{NEURIPS2020_Hoeven} explicitly introduced the surrogate regret in the context of online multiclass classification. This study has been extended to the setting where observations are determined by a directed graph \Citep{NEURIPS2021_Hoeven} and to structured prediction scenarios \citep{pmlr-v247-sakaue24a}. For a more extensive overview of the literature on online classification, we refer the reader to \Citet{NEURIPS2020_Hoeven}.


\paragraph{Delayed feedback}
The study of delayed feedback began with \citet{Weinberger_2002_delay}. 
Since then, it has been extensively explored in various online learning settings, primarily in the full information setup of online convex optimization \citep{Mesterharm05online,joulani13online,joulani16delay,pmlr-v139-flaspohler21a}. 
Algorithms for delayed bandit feedback have been studied mainly in the context of multi-armed bandits and their variants \citep{cesabianchi16delay,zimmert20optimal,ito20delay,masoudian22best,hoeven23unified}. In the context of online classification, research considering delay is scarce; the only work is \citet{manwani2022delaytronefficientlearningmulticlass} to our knowledge.



\section{Discussion on the Difference in Surrogate Loss Functions}\label{app:Discussio_on_the_Difference_in_Surrogate_Losses}

As in \eqref{eq:sur_regret}, the surrogate regret $\reg$ is defined by $\sumt{L(\yht;\yt)}=\sumt{S(\U\xt;\yt)}+\reg$, which means the choice of the surrogate loss $S$ affects the bound on the cumulative loss $\sumt{L(\yht;\yt)}$.
\Citet[Theorem~1]{NEURIPS2021_Hoeven}, which applies to a more general setup than bandit feedback, implies $\reg = O(K\sqrt{T})$ for the bandit setup with $S$ being a logistic loss defined with the base-$K$ logarithm. 
On the other hand, our bound of $\reg = O(\sqrt{KT})$ applies to the logistic loss $S$ defined with the base-$2$ logarithm. 
As a result, while our bound on $\reg$ is better, the $\sumt{S(\U\xt;\yt)}$ term can be worse; this is why we cannot directly compare our $O(\sqrt{KT})$ bound with the $O(K\sqrt{T})$ bound in \Citet[Theorem~1]{NEURIPS2021_Hoeven}. 
We may use the decoding procedure in \citet{NEURIPS2021_Hoeven}, instead of RDUE, to recover their bound that applies to the base-$K$ logistic loss.
It should be noted that their method is specific to multiclass classification;  
naively extending their method to structured prediction formulated as $|\yy|$-class classification results in the undesirable dependence on $K = |\yy|$, as is also discussed in \citet{pmlr-v247-sakaue24a}. 
By contrast, our pseudo-inverse estimator, combined with RDUE, can rule out the explicit dependence on $K$, at the cost of the increase from $\sqrt{T}$ to $T^{2/3}$.
\section{Omitted Details of \cref{sec:bandit}}
\label{app:proof bandit}
This section provides the omitted details of \cref{sec:bandit}.

\subsection{Concentration inequality}
To prove high probability regret bounds, we use the following concentration inequality.
\begin{lemma}[{Bernstein's inequality, e.g., \citealt[Lemma A.8]{Cesa-Bianchi_Lugosi_2006}}]
    \label{lem:Bernstein}
    Let $Z_1,\hdots,Z_T$ be a martingale difference sequence and $\delta \in (0,1)$.
    If there exist $a$ and $v$ which satisfy $|Z_t|\leq a$ for any $t \in \brk{T}$ and $\sumt{\expect{Z_t^2}}\leq v$ , then with probability at least $1-\delta$, it holds that
    \[
        \sumt{Z_t}\leq\sqrt{2v\log\frac{1}{\delta}}+\frac{\sqrt{2}}{3}a\log\frac{1}{\delta}.
    \]
\end{lemma}


\subsection{Proof of \cref{thm:bandit_high_prob}}\label{app:proof_bandit_high_prob}
Here, we provide the proof of \cref{thm:bandit_high_prob}.
Hereafter, we let $S_{\max} = \max_{\W \in \ww} S_t(\W)$ and $\hat{S}_t(\W) = v_t S_t(\W) = \frac{\ind\brk{\yt = \yht}}{p_t(\yht)} S_t(\W)$.
The following theorem is the formal version of \cref{thm:bandit_high_prob}:
\begin{theorem}[Formal version of \cref{thm:bandit_high_prob}]\label{thm:bandit_high_prob_formal}
Consider the bandit and non-delayed setup.
Let 
\begin{equation}
    \mathcal{C}
    =
    \prn*{
        \frac{3}{2 (a + \xi - 1)}  
        +
        1
    }
    K \Smax \log(2/\delta) 
    +
    \frac{B^2 K b}{2 (1 - \xi)}
    .
    \nonumber
\end{equation}

Then, for any $T \geq \mathcal{C}$ and $\delta \in (0,1/2)$, with probability at least $1-\delta$, the algorithm in \cref{subsec:Bandit_Structured_Prediction_with_General_Losses} with $q = \sqrt{\mathcal{C} / T}$ achieves
\begin{equation}
    \mathcal{R}_T
    \leq
    2
    \sqrt{
        \mathcal{C}
        T
    }
    +
    \sqrt{2 \log (2/ \delta)} \prn{\mathcal{C} T}^{1/4}
    +
    \prn*{ \frac{1-a}{2 (a + \xi - 1)} + 2 } \log (2/\delta).
    \nonumber
\end{equation}
\end{theorem}


Before proving this theorem, we provide the following lemma:
\begin{lemma}\label{lem:hp_pre}
It holds that 
\begin{equation}
    \sum_{t=1}^T \prn*{ \E_t\brk*{L_t(\yht)} - \hat{S}_t(\U) }
    \leq
    \sum_{t=1}^T \prn*{ (1-a) S_t(\W_t) - \hat{S}_t(\W_t)  } + q T
    +
    \sqrt{2} B \sqrt{\frac{b}{q} \sum_{t=1}^T v_t S_t(\W_t) }
    .
    \nonumber
\end{equation}
\end{lemma}
\begin{proof}
We have 
\begin{equation}
    \sum_{t=1}^T \prn*{ \E_t\brk*{L_t(\yht)} - \hat{S}_t(\U) }
    =
    \sum_{t=1}^T \prn*{ \E_t\brk*{L_t(\yht)} - \hat{S}_t(\W_t)  }
    +
    \sum_{t=1}^T \prn*{ \hat{S}_t(\W_t) - \hat{S}_t(\U) }.
    \nonumber
\end{equation}
From \cref{asp:bandit_a}, the first term is bounded as 
\begin{align}
    \sum_{t=1}^T \prn*{ \E_t\brk*{L_t(\yht)} - \hat{S}_t(\W_t)  }
    \leq
    \sum_{t=1}^T \prn*{ (1-a) S_t(\W_t) - \hat{S}_t(\W_t)  } + q T,
    \nonumber
\end{align}
and 
the second term is bounded as 
\begin{align}
    \sum_{t=1}^T \prn*{ \hat{S}_t(\W_t) - \hat{S}_t(\U) }
    &\leq
    \sqrt{2} B \sqrt{\sum_{t=1}^T \nrm{\tilde{\G}_t}_{\F}^2 }
    =
    \sqrt{2} B \sqrt{\sum_{t=1}^T v_t^2 \nrm{\G_t}_{\F}^2 }
    \nonumber \\
    &\leq
    \sqrt{2} B \sqrt{b \sum_{t=1}^T v_t^2 S_t(\W_t) }
    \leq
    \sqrt{2} B \sqrt{\frac{b K}{q} \sum_{t=1}^T v_t S_t(\W_t) },
    \nonumber
\end{align}
where we used \cref{lem:ogd} and $v_t \leq K / q$.
Combining the above three, we obtain
\begin{equation}
    \sum_{t=1}^T \prn*{ \E_t\brk*{L_t(\yht)} - \hat{S}_t(\U) }
    \leq
    \sum_{t=1}^T \prn*{ (1-a) S_t(\W_t) - \hat{S}_t(\W_t)  } + q T
    +
    \sqrt{2} B \sqrt{\frac{b K}{q} \sum_{t=1}^T v_t S_t(\W_t) }
    ,
    \nonumber
\end{equation}
which completes the proof.
\end{proof}


\begin{proof}[Proof of \cref{thm:bandit_high_prob_formal}]
The surrogate regret can be decomposed as
\begin{equation}\label{eq:reg_dec_highp}
    \mathcal{R}_T 
    =
    \sum_{t=1}^T \prn*{ L_t(\yht) - \E_t\brk*{ L_t(\yht)} }
    +
    \sum_{t=1}^T \prn*{ \E_t\brk*{ L_t(\yht)} - S_t(\U) }
    .
\end{equation}
We first upper bound the first term in \eqref{eq:reg_dec_highp}.
Let $Z_t = L_t(\yht) - \E_t\brk*{ L_t(\yht)}$ for simplicity.
Then, we have $Z_t \leq 1$, $\E_t\brk*{Z_t} = 0$, and
$\E_t\brk*{Z_t^2} 
\leq 
\E_t\brk*{ \prn{L_t(\yht)}^2 }
\leq 
(1-a) S_t(\W_t) + q.
$
Hence, from Bernstein's inequality in \cref{lem:Bernstein}, for any $\delta' \in (0,1)$, at least $1 - \delta'$ we have 
\begin{equation}\label{eq:conc_zt}
    \sum_{t=1}^T Z_t
    \leq 
    \sqrt{2 \log (1/\delta') \sum_{t=1}^T \prn*{(1-a) S_t(\W_t) + q} }
    +
    \frac{\sqrt{2}}{3} \log (1/\delta')
    .
\end{equation}
We next consider the second term in \eqref{eq:reg_dec_highp}.
Define $r_t = S_t(\U) - \xi S_t(\W_t)$ for some $\xi \in (0, 1)$, which will be determined later,
and let $v_t = \ind[ \yt = \yht ] / p_t(\yht) \leq K/q$ for simplicity.
Then, we have $\E_t\brk{v_t r_t - r_t} = 0$, $\abs{v_t r_t - r_t} \leq K S_{\max} / q$, and
\begin{equation}
    \E_t\brk{ (v_t r_t - r_t)^2}
    \leq
    \E_t\brk{(v_t r_t)^2}
    \leq 
    \frac{K \Smax}{q} \abs{r_t}
    \leq 
    \frac{K \Smax}{q} \prn*{S_t(\U) + S_t(\W_t)}
    .
    \nonumber
\end{equation}
Hence from Bernstein's inequality in \cref{lem:Bernstein}, for any $\delta'' \in (0,1)$, with probability at least $1 - \delta''$ we have 
\begin{equation}\label{eq:conc_vr}
    \sum_{t=1}^T \prn{v_t r_t - r_t} 
    \leq 
    \sqrt{3 \log (1/\delta'') \sum_{t=1}^T \frac{K \Smax}{q} \prn{S_t(\U) + S_t(\W_t)} }
    +
    \frac{\sqrt{2} K \Smax}{3q} \log(1/\delta'')
    .
\end{equation}


\textbf{When $\sum_{t=1}^T S_t(\U) \leq \sum_{t=1}^T S_t(\W_t)$.}
We first consider the case of $\sum_{t=1}^T S_t(\U) \leq \sum_{t=1}^T S_t(\W_t)$.
From \cref{lem:hp_pre}, we have
\begin{align}
    &\sum_{t=1}^T \E_t\brk*{L_t(\yht)} - q T
    \leq
    \sum_{t=1}^T v_t S_t(\U) 
    +
    \sum_{t=1}^T \prn*{ (1-a) S_t(\W_t) - v_t S_t(\W_t)  } 
    +
    \sqrt{2} B \sqrt{\frac{b K}{q} \sum_{t=1}^T v_t S_t(\W_t) }
    \nonumber \\
    &=
    \sum_{t=1}^T v_t \underbrace{\prn*{ S_t(\U) - \xi S_t(\W_t)  } }_{= r_t}
    -
    (1 - \xi) \sum_{t=1}^T v_t S_t(\W_t)
    +
    (1-a) \sum_{t=1}^T S_t(\W_t)
    +
    \sqrt{2} B \sqrt{\frac{b K}{q} \sum_{t=1}^T v_t S_t(\W_t) }
    \nonumber \\
    &\leq
    \sum_{t=1}^T v_t r_t
    +
    (1-a) \sum_{t=1}^T S_t(\W_t)
    +
    \frac{B^2 K b}{2 q (1 - \xi)},
    \nonumber
\end{align}
where the last inequality follows from 
$c_1\sqrt{x}-c_2x\leq{c_1^2}/\prn{4c_2}$ for $x \geq 0$, $c_1 \geq 0$, and $c_2 > 0$.
From the concentration result provided in \eqref{eq:conc_vr}, this is further bounded as
\begin{align}
    \sum_{t=1}^T \E_t\brk*{L_t(\yht)} - q T
    &\leq
    \sum_{t=1}^T (S_t(\U) - \xi S_t(\W_t))
    +
    \sqrt{3 \log (1/\delta'') \sum_{t=1}^T \frac{K \Smax}{q} \prn{S_t(\U) + S_t(\W_t)} }
    \nonumber \\
    &\qquad
    +
    \frac{\sqrt{2} K \Smax}{3q} \log(1/\delta'') 
    +
    (1-a) \sum_{t=1}^T S_t(\W_t)
    +
    \frac{B^2 K b}{2 q (1 - \xi)}
    ,
    \nonumber
\end{align}
where we recall that $r_t = S_t(\U) - \xi S_t(\W_t)$.
Rearranging the last inequality and using $\sum_{t=1}^T S_t(\U) \leq \sum_{t=1}^T S_t(\W_t)$ give
\begin{align}
    \sum_{t=1}^T \prn*{ \E_t\brk*{L_t(\yht)} - S_t(\U) } 
    &\leq
    q T 
    +
    \sqrt{6 \log (1/\delta'') \sum_{t=1}^T \frac{K \Smax}{q} S_t(\W_t) }
    +
    \frac{\sqrt{2} K \Smax}{3 q} \log(1/\delta'') 
    \nonumber \\
    &\qquad
    +
    (1 - a - \xi) \sum_{t=1}^T S_t(\W_t)
    +
    \frac{B^2 K b}{2 q (1 - \xi)}
    .
    \nonumber
\end{align}
In what follows, we let $\delta' = \delta'' = \delta / 2$ and $\xi = \frac{\prn{4 + c} \gamma}{\lambda \nu}$ for a sufficiently small constant $c > 0$, which satisfies $a + \xi > 1$.
Then, plugging \eqref{eq:conc_zt} and the last inequality in \eqref{eq:reg_dec_highp}, with probability at least $1 - \delta$, we obtain
\begin{align}
    \mathcal{R}_T
    &\leq
    \sqrt{2 \log (2/\delta) \sum_{t=1}^T \prn*{(1-a) S_t(\W_t) + q} }
    +
    \frac{\sqrt{2}}{3} \log (2/\delta)
    +
    q T 
    +
    \sqrt{6 \log (2/\delta) \sum_{t=1}^T \frac{K \Smax}{q} S_t(\W_t) }
    \nonumber \\
    &\qquad
    +
    \frac{\sqrt{2} K \Smax}{3 q} \log(2/\delta) 
    +
    (1 - a - \xi) \sum_{t=1}^T S_t(\W_t)
    +
    \frac{B^2 K b}{2 q (1 - \xi)}
    \nonumber \\
    &\leq
    \frac{1}{2 (a + \xi - 1)}
    \prn*{(1-a) + \frac{3 K \Smax}{q}} \log (2/\delta)
    +
    \sqrt{2 q T \log (2 / \delta)} 
    +
    \frac{\sqrt{2}}{3} \log (2/\delta)
    +
    q T 
    \nonumber \\
    &\qquad
    +
    \frac{\sqrt{2} K \Smax}{3 q} \log(2/\delta) 
    +
    \frac{B^2 b}{2 q (1 - \xi)}
    \nonumber \\
    &\leq
    \frac{1}{q}
    \prn*{
        \frac{3 K \Smax \log (2/\delta)}{2 (a + \xi - 1)}  
        +
        K \Smax \log(2/\delta) 
        +
        \frac{B^2 K b}{2 (1 - \xi)}
    }
    +
    q T 
    +
    \sqrt{2 q T \log (2 / \delta)} 
    \nonumber \\
    &\qquad
    +
    \frac{1}{2 (a + \xi - 1)} (1-a) \log (2/\delta)
    +
    \frac{\sqrt{2}}{3} \log (2/\delta)
    \nonumber \\
    &=
    \frac{\mathcal{C}}{q}
    +
    q T 
    +
    \sqrt{2 q T \log (2 / \delta)} 
    +
    \frac{1}{2 (a + \xi - 1)} (1-a) \log (2/\delta)
    +
    \frac{\sqrt{2}}{3} \log (2/\delta).
    \nonumber
\end{align}
Using the definition of $q = \sqrt{\mathcal{C} / T}$ with the last inequality,
we obtain
\begin{equation}
    \mathcal{R}_T
    \leq
    2
    \sqrt{
        \mathcal{C}
        T
    }
    +
    \prn{\mathcal{C} T}^{1/4} \sqrt{\log (2/ \delta)}
    +
    \prn*{ \frac{1-a}{2 (a + \xi - 1)} + \frac{\sqrt{2}}{3} } \log (2/\delta).
    \nonumber
\end{equation}

\textbf{When $\sum_{t=1}^T S_t(\U) > \sum_{t=1}^T S_t(\W_t)$.}
We next consider the case of $\sum_{t=1}^T S_t(\U) > \sum_{t=1}^T S_t(\W_t)$.
We have
\begin{align}
    \mathcal{R}_T 
    &=
    \sum_{t=1}^T \prn*{ L_t(\yht) - \E_t\brk*{ L_t(\yht)} }
    +
    \sum_{t=1}^T \prn*{ \E_t\brk*{ L_t(\yht)} - S_t(\U) }
    \nonumber \\
    &\leq
    \sqrt{2 \log (1/\delta') \sum_{t=1}^T \prn*{(1-a) S_t(\W_t) + q} }
    +
    \frac{\sqrt{2}}{3} \log (1/\delta')
    +
    \sum_{t=1}^T \prn*{ \E_t\brk*{ L_t(\yht)} - S_t(\W_t) }
    \nonumber \\
    &\leq
    \sqrt{2 \log (1/\delta') \sum_{t=1}^T \prn*{(1-a) S_t(\W_t) + q} }
    +
    \frac{\sqrt{2}}{3} \log (1/\delta')
    +
    \sum_{t=1}^T \prn*{ - a S_t(\W_t) + q }
    \nonumber \\
    &\leq
    \frac{(1 - a) \log (1/ \delta')}{ 2 a }
    +
    \sqrt{2 q T \log (1/\delta') }
    +
    \frac{\sqrt{2}}{3} \log (1/\delta')
    +
    q T
    ,
    \nonumber
\end{align}
where the first inequality follows from \eqref{eq:conc_zt} and $\sum_{t=1}^T S_t(\U) > \sum_{t=1}^T S_t(\W_t)$,
and the second inequality follows from \cref{asp:bandit_a},
the last inequality follows from $c_1\sqrt{x}-c_2x\leq{c_1^2}/\prn{4c_2}$ for $x \geq 0$, $c_1 \geq 0$, and $c_2 > 0$.
Substituting $q = \sqrt{\mathcal{C} / 2}$ and $\delta' = \delta/2$ and  in the last inequality, we obtain
\begin{equation}
    \mathcal{R}_T 
    \leq
    \frac{(1 - a) \log (2/ \delta)}{ 2 a }
    +
    \sqrt{2 \log (2/\delta) } \prn{ \mathcal{C} T }^{1/4}
    +
    \frac{\sqrt{2}}{3} \log (2/\delta)
    +
    \sqrt{\mathcal{C} T}
    .
    \nonumber
\end{equation}
This completes the proof.    
\end{proof}







\subsection{Proof of \cref{thm:bandit_regret_pseudo_estimator}}
\label{app:sub_bandit_regret_pseudo_estimator}


Here, we provide the proof of \cref{thm:bandit_regret_pseudo_estimator}.
We recall that $\pt=\expect{\yht\yht^\top}$.
We then estimate $\yt$ by $\ytilde=\inverse{\V}\Pplus_t\yht\inpr{\yht,\V\yt}$
and $\G_t$ by $\gtil\coloneqq(\yho(\tht)-\ytilde)\xt^\top$ under \cref{asp:self}.
This $\gtil$ satisfies 
$
    \expect{\gtil}=\G_t. 
$
To prove \cref{thm:bandit_regret_pseudo_estimator}, we will upper bound $\expect{\nrm{\gtil}_\F^2}$.
To do so, we begin by proving the following lemma:
\begin{lemma}\label{lem:pseudo_inverse_order}
    Let $\bm{A}$ and $\bm{B}$ positive semi-definite matrices with $\image(\bm{A}) = \image(\bm{B})$ with $\bm{A} \succeq \bm{B}$.
    Then, it holds that $\bm{A}^+ \preceq \bm{B}^+$.
\end{lemma}
\begin{proof}
Since $\image(\bm{A}) = \image(\bm{B})$, there exists an orthogonal matrix $\bm{R}$, a diagonal matrix $\bm{\Lambda}$, and an invertible matrix $\bm{B}'$ that has same dimensions as $\bm{\Lambda}$ such that 
\begin{equation}
    \bm{A}
    =
    \bm{R}
    \begin{pmatrix}
        O & O \\
        O & \bm{\Lambda} 
    \end{pmatrix}
    \bm{R}^\top
    \quad 
    \mbox{and}
    \quad
    \bm{B}
    =
    \bm{R}
    \begin{pmatrix}
        O & O \\
        O & \bm{B}'
    \end{pmatrix}
    \bm{R}^\top
    .
    \nonumber
\end{equation}
Then, 
\begin{equation}\label{eq:Aplus_Bplus}
    \bm{A}^+
    =
    \bm{R}
    \begin{pmatrix}
        O & O \\
        O & \bm{\Lambda}^{-1}
    \end{pmatrix}
    \bm{R}^\top
    \quad 
    \mbox{and}
    \quad
    \bm{B}^+
    =
    \bm{R}
    \begin{pmatrix}
        O & O \\
        O & {\bm{B}'}^{-1}
    \end{pmatrix}
    \bm{R}^\top
    .
\end{equation}
From $\bm{A} \succeq \bm{B}$,
we have $\bm{\Lambda} \succeq \bm{B}'$, which implies $\bm{\Lambda}^{-1} \preceq {\bm{B}'}^{-1}$.
From this and \eqref{eq:Aplus_Bplus}, we have $\bm{A}^+ \preceq \bm{B}^+$, as desired.
\end{proof}

Using this lemma we prove a property of $\pt$ and an upper bound of $\expect{\tr\prn*{\yht\yht^\top}}$.
In what follows, we use $\lambda_\min(\bm{A})$ to denote the minimum eigenvalue of a matrix $\bm{A}$.

\begin{lemma}
    \label{lem:bound of trace}
    Suppose that $\tr \prn*{ \V^{-1} \bm{Q} \prn{\V^{-1}}^\top } \leq \omega$ for $\bm{Q} = \E_{\bm{y} \sim \mu} \brk{ \bm{y} \bm{y}^\top }$, where we recall $\mu$ is the uniform distribution over $\yy$.
    Then, we have
    \[
    \expect{\tr(\ytt\ytt^\top)}\leq \frac{\omega}{q}.
    \]
\end{lemma}
\begin{proof}    
    By the linearity of expectation and the trace property, we have
    \begin{align*}
        \expect{\tr(\ytilde\ytilde^\top)}&\leq \tr\prn*{\inverse{\V}\Pplus_t\expect{\yht\yht^\top}\Pplus_t\prn*{\inverse{\V}}^\top}
        = \tr\prn*{\inverse{\V}\Pplus_t \bm{P}_t \Pplus_t \prn*{\inverse{\V}}^\top}\\
        &= 
        \tr\prn*{\inverse{\V}\Pplus_t\prn*{\inverse{\V}}^\top},
    \end{align*}
    where the first inequality follows from $\inpr{\yht,\V\yt} = L_t(\yht) - \inpr{\yht,\bm{b}} - c(\yt) \leq L_t(\yht) \leq 1$ since $\bm{b} \geq 0$ and $c(\cdot)$ is non-negative
    and
    the last equality follows from $\Pplus_t \bm{P}_t \Pplus_t = \Pplus_t$.
    Hence,
    \begin{align}
        \tr\prn*{\inverse{\V}\Pplus_t\prn*{\inverse{\V}}^\top}
        &=
        \sum_{i=1}^d
        \bm{\ee}_i^\top \inverse{\V}\Pplus_t\prn*{\inverse{\V}}^\top \bm{\ee}_i
        \leq
        \sum_{i=1}^d
        \bm{\ee}_i^\top \inverse{\V} \prn*{q \bm{Q}}^{+} \prn*{\inverse{\V}}^\top \bm{\ee}_i
        \nonumber \\
        &\leq
        \tr\prn*{\prn*{\inverse{\V}}^\top \inverse{\V} (q \bm{Q})^+ }
        =
        \frac{1}{q}
        \tr \prn*{ \inverse{\V} \bm{Q}^+ \prn{\inverse{\V}}^\top } 
        \leq
        \frac{\omega}{q},
        \nonumber
    \end{align}
    where in the first inequality we used \cref{lem:pseudo_inverse_order} and in the last inequality we used the assumption that $\tr \prn*{ \V^{-1} \bm{Q}^+ \prn{\V^{-1}}^\top } \leq \omega$.
    This completes the proof.
\end{proof}

Now, we are ready to upper bound $\expect{\nrm{\gtil}_\F^2}$.
\begin{lemma}
    \label{thm:evaluation of Gtilde}
    Under the same assumption as \cref{lem:bound of trace}, it holds that 
    \[
        \expect{\nrm{\gtil}_\F^2}\leq2b\sw+ \frac{2 \dix^2 \omega}{q}.
    \]
\end{lemma}
\begin{proof}
    We have 
    \begin{align*}
        \nrm{\gtil}_\F^2&=\nrm{\prn*{\yho(\tht)-\ytilde}\xt^\top}_{\mathrm{F}}^2\leq2\nrm{(\yho(\tht)-\yt)\xt^\top}_\F^2+2\nrm{(\yt-\ytilde)\xt^\top}_\F^2\\
        &\leq 2\nrm{\G_t}_\F^2+2\dix ^2\nrm{\yt-\ytilde}_2^2,
    \end{align*}
    where we recall $\dix =\diam(\xx)$.
    From this inequality, 
    \begin{align}
        \expect{\nrm{\gtil}_\F^2}&\leq2\nrm{\G_t}_{\mathrm{F}}^2+2\dix ^2\expect{\nrm{\yt-\ytilde}_2^2}
        \leq2b\sw+2\dix ^2\prn*{\nrm{\yt}_2^2-2\yt^\top\expect{\ytilde}+\expect{\nrm{\ytilde}_2^2}} \nonumber \\
        &=2b\sw+2\dix ^2\prn*{\nrm{\yt}_2^2-2\nrm{\yt}_2^2 + \expect{\nrm{\ytilde}_2^2}} \nonumber \\ 
        &\leq
        2b\sw
        +2\dix ^2\expect{\tr(\ytilde\ytilde^\top)}
        \leq
        2b\sw
        + \frac{2\dix^2 \omega}{q},
        \nonumber
    \end{align} 
    where in the second inequality we used $\nrm{\G_t}_\F^2 \leq b \sw$, in the equality we used $\expect{\ytilde}=\yt$,
    and in the last inequality we used \cref{lem:bound of trace}.
\end{proof}



Finally, we are ready to prove \cref{thm:bandit_regret_pseudo_estimator}.
\begin{proof}[Proof of \cref{thm:bandit_regret_pseudo_estimator}]
    From \cref{asp:bandit_a}, we have 
    \begin{align*}\label{eq:inverse_reg_expect}
        \E\brk{\reg}
        &\leq\E\brk*{\sumt{(\sw-\su)}}-a\E\brk*{\sumt{\sw}}+qT
        \nonumber \\
        &\leq\E\brk*{\sumt{\inpr*{\G_t,\wt-\U}}}-a\E\brk*{\sumt{\sw}}+qT.
    \end{align*}
    From \cref{thm:evaluation of Gtilde} and the unbiasedness of $\gtil$, 
    the first term in the last inequality is further bounded as
    \begin{align*}
        \E\brk*{\sumt{\inpr*{\G_t,\wt-\U}}}  
        &=\E\brk*{\sumt{\inpr*{\gtil,\wt-\U}}}
        \leq\sqrt{2}B\sqrt{\E\brk*{\sumt{\nrm{\gtil}_{\mathrm{F}}^2}}}
        \nonumber \\
        &\leq
        2 B \sqrt{b\E\brk*{\sumt{\sw}}}
        +
        2 B \dix \sqrt{ \omega / q},
    \end{align*}
    where 
    the first inequality follows from \cref{lem:ogd} and the last inequality follows from \cref{thm:evaluation of Gtilde} and the subadditivity of $x \mapsto \sqrt{x}$ for $x \geq 0$.
    Therefore, by combining  with the last inequality, we have 
    \begin{align}
        \E\brk{\reg}
        &\leq 
        2B \sqrt{b\E\brk*{\sumt{\sw}}} 
        +
        2 B \dix \sqrt{ \omega / q} 
        -a \E\brk*{\sumt{\sw}} + qT \\ 
        &\leq 
        \frac{bB^2}{a}
        +
        2 B \dix \sqrt{ \omega / q} 
        +qT ,
        \nonumber
    \end{align}
    where we used $c_1\sqrt{x}-c_2x\leq{c_1^2}/\prn{4c_2}$ for $x\geq0$, $c_1\geq 0$, and $c_2>0$.
    Finally, substituting 
    $q=\prn*{\frac{4 \omega B^2\dix ^2}{ T }}^{1/3}$ in the last inequality gives
    \[
    \E\brk{\reg}
    \leq
    \frac{bB^2}{a}
    +
    2^{5/3} \omega^{1/3} \prn*{ B \dix T}^{2/3}
    ,
    \]
    which is the desired bound.
\end{proof}



\subsection{Proof of \cref{cor:thm_self}}\label{app:SELF_upper_discussion_deferred}
Here, we derive the regret upper bounds provided by the algorithm established  
in \cref{thm:bandit_regret_pseudo_estimator}  
for online multiclass classification, online multilabel classification, and ranking.
Recall that 
we can achieve
\begin{equation}\label{eq:bound_self_app}
\E\brk{\reg}
\leq
\frac{bB^2}{a}
+
O\prn*{ \omega^{1/3} \prn*{ B \dix T}^{2/3} },    
\end{equation}
where we recall that $\omega$ is defined as
$
    \tr \prn*{ \V^{-1} \bm{Q}^+ \prn{\V^{-1}}^\top } \leq \omega
$
for $\bm{Q} = \E_{\bm{y} \sim \mu} \brk{ \bm{y} \bm{y}^\top }$.
Note that when $\spanx(\yy) = \R^d$, then the matrix $\bm{Q}$ is invertible and $\lambda_{\min}(\bm{Q}) > 0$, and thus 
\begin{equation}\label{eq:trace_upper_invertibleQ}
    \tr \prn*{ \V^{-1} \bm{Q}^+ \prn{\V^{-1}}^\top }
    =
    \sum_{i=1}^d
    \bm{\ee}_i^\top \V^{-1} \bm{Q}^+ \prn{\V^{-1}}^\top \bm{\ee}_i
    \leq
    \frac{1}{\lambda_{\min}(\bm{Q})}
    \sum_{i=1}^d
    \nrm{ \prn{\V^{-1}}^\top \bm{\ee}_i }_2^2
    \leq 
    \frac{1}{\lambda_{\min}(\bm{Q})}
    \nrm{ \V^{-1} }_{\F}^2
    .
\end{equation}
In each problem setting, this regret upper bound can be reduced to the following bounds:

\paragraph{Multiclass classification with 0-1 loss}
We first consider multiclass classification with the 0-1 loss.
Since $\V=\bm{1}\bm{1}^\top-\I$, we have $\nrm{\inverse{\V}}_\F^2\leq d$ for $d \geq 2$.  
Recalling that $\mu$ is the uniform distribution over $\yy=\set{\bm{\ee}_1,\hdots,\bm{\ee}_d}$, we have  
$
\E_{\bmy\sim\mu}\brk{(\bmy^\top\bm x)^2}=\frac{1}{d}\sum_{i=1}^{d}x_i^2
$
for any $\bm x\in\R^d$.
Hence, $\lambda_{\min}(\bm{Q})=\min_{\nrm{\bm x}_2=1} \E_{\bmy\sim\mu}\brk*{(\bmy^\top\bm x)^2}=\frac{1}{d}$, where the first equality is from \citet[Lemma 2]{comband}.  
Since $\spanx(\yy) = \R^d$ in this case,
from \eqref{eq:trace_upper_invertibleQ} we can let $\omega = d / \lambda_{\min}(\bm{Q}) = d^2$.
Substituting these into our upper bound in \eqref{eq:bound_self_app}, we obtain  
\begin{equation*}
    \E\brk{\reg}\leq\frac{bB^2}{a}+ O \prn*{ \prn{B \dix  d T}^{2/3} }.
\end{equation*}  



\paragraph{Online multilabel classification with $m$ correct labels   
and the Hamming loss}
We next consider online multilabel classification with the number of correct labels $m$  
and the Hamming loss.  
Since $\V=-\frac{2}{d}\I$, we have  
$\nrm{\inverse{\V}}_\F^2=\frac{d^3}{4}$.  
Let $\yy\subset\set{0,1}^d$ be the set of all vectors  
where exactly $m$ components are $1$, and the remaining components are all $0$.  
By drawing $\bmy\in\yy$ according to the uniform distribution over $\yy$,  
the probability that a given component of $\bmy$ is $1$ is  
$\binom{m-1}{d-1}/\binom{m}{d}=\frac{m}{d}$.  
Hence, for any $\bm x\in\R^d$ with $\nrm{\bm{x}}_2=1$,  
we have
\[
\E_{\bmy\sim\mu}\brk*{(\bmy^\top\bm x)^2}
=\frac{m}{d}\sum_{i=1}^dx_i^2  
+\frac{m^2}{d^2}\sum_{i\neq j}x_ix_j  
=\prn*{\frac{m}{d}\sum_{i=1}^dx_i}^2+\frac{m(d-m)}{d^2}\nrm{\bm x}_2^2  
\geq 
\frac{m(d-m)}{d^2}.
\]
Hence, we have $\lambda_{\min}(\bm{Q}) = \E_{\bmy\sim\mu}\brk*{(\bmy^\top\bm x)^2} \geq\frac{m(d-m)}{d^2}$, where the equality is from \citet[Lemma 2]{comband}.
Since $\spanx(\yy) = \R^d$,
from \eqref{eq:trace_upper_invertibleQ} we can choose $\omega = d^3 / \prn{4 \lambda_{\min}(\bm{Q})} = \frac{d^5}{4 m (d-m)} $.
Therefore, our regret upper bound in \eqref{eq:bound_self_app} is reduced to
\begin{equation*}
    \E\brk{\reg}
    \leq
    \frac{bB^2}{a}
    +
    O \prn*{ \prn*{\frac{B^2\dix ^2d^5}{m(d-m)}}^{1/3} T^{2/3} }.
\end{equation*}


\paragraph{Ranking with the Hamming loss and the number of items $m$}
We finally consider online ranking with the Hamming loss and the number of items $m$.  
From \citet[Proposition 4]{comband}, the smallest positive eigenvalue is at least $1/m$.
Hence, since $\V=-\frac{1}{m}\I$, we have
\begin{equation*}
\tr\prn{\inverse{\V}\bm{Q}^+(\inverse{\V})^\top}=m^2\tr\prn{\bm{Q}^+}\leq m^2\sum_{i=1}^{\rank(\bm{Q}^+)} m\leq m^5,
\end{equation*}
where we used $\rank(\bm{Q}^+) \leq d = m^2$,
and this allows us to choose $\omega = m^5$.
Substituting these values into our regret upper bound in \eqref{eq:bound_self_app} , we obtain  
\begin{equation*}
    \E\brk{\reg}\leq\frac{bB^2}{a}+ O\prn*{ m^{5/3}\prn{B\dix T}^{2/3} }.
\end{equation*}  


\section{Omitted Details of \cref{sec:delay}}
\label{app:proof delay}
This section provides the proofs of the theorems in \cref{sec:delay}.

\subsection{Details of Optimistic Delayed Adaptive FTRL (ODAFTRL)}
\label{app:sub_odaftrl}
We provide a more detailed explanation of the Optimistic Delayed Adaptive FTRL (ODAFTRL) algorithm used for updating $\W_t$ in \cref{sec:delay}.
We recall that ODAFTRL computes $\W_t$ by the following update rule:
\begin{equation}
    \label{eq:odaftrl_2}
    \W_{t+1}=\argmin_{\W\in \ww} \set*{ \sum_{i=1}^{t-D} \inpr{\bm{G}_i ,\W} + \frac{\lambda_t \nrm{\W}_{\F}^2 }{2} },
\end{equation}
where $\lambda_t\geq0$ is a regularization parameter.
The ODAFTRL algorithm, when using this parameter update called AdaHedgeD, satisfies the following lemma:
\begin{lemma}[{\citealt[Theorem 12]{pmlr-v139-flaspohler21a}}]
    \label{thm:AdaHedgeD}
    Fix $\alpha>0$. 
    Let $S_t:\ww\to\R$ be a convex function for each $t=1,\dots,T$.
    Suppose that we update $\lambda_{t}$ in \eqref{eq:odaftrl_2} by the following AdaHedgeD update:
    \begin{align*}
        \lambda_{t+1}&=\frac{1}{\alpha}\sum_{s=1}^{t-D}\delta_s,\\
        \delta_t &= \min\crl*{F_{t+1}(\W_t)-F_{t+1}(\bar{\bm{W}}_t),\inpr{ \bm{G}_t,\W_t-\bar{\bm{W}}_t},F_{t+1}(\hat{\bm{W}}_t)-F_{t+1}(\bar{\bm{W}}_t)+\inpr{ \bm{G}_t,\W_t-\hat{\bm{W}}_t}}_+,\\
        \bar{\bm{W}}_t &= \argmin_{\W\in\ww}F_{t+1}(\W),\\
        \hat{\bm{W}}_t&= \argmin_{\W\in\ww}\set*{F_{t+1}(\W)-\min\crl*{\frac{\nrm{\bm{G}_t}_{\mathrm{F}}}{\nrm{\bm{G}_{t-D:t}}_{\mathrm{F}}},1}\inpr{\bm{G}_{t-D:t},\W}}, \text{  and}\\
        F_{t+1}(\W)&=\frac{\lambda_t\nrm{\W}_\F^2}{2}+\inpr{\G_{1:t},\W}.
    \end{align*}
    Then, for any $\U\in\ww$, ODAFTRL achieves
    \begin{equation*}
        \sumt{\sw}
        \leq
        \sumt{\su}
        +
        \prn*{\frac{B^2}{2\alpha}+1}\prn*{2\max_{s\in[T]}a_{s-D:s-1}+\sqrt{\sum_{t=1}^{T}a_{t}^2+2\alpha b_{t}}},
    \end{equation*}
    where 
    \begin{align*}
        a_{t}&=B\min\crl{\nrm{\bm{G}_{t-D:t}}_{\mathrm{F}},\nrm{\bm{G}_t}_{\mathrm{F}}},\\
        b_{t}&=\operatorname{huber}\prn{\nrm{\bm{G}_{t-D:t}}_{\mathrm{F}},\nrm{\bm{G}_t}_{\mathrm{F}}},\:\text{and} \:
        \operatorname{huber}(x,y)=\frac{1}{2}x^2-\frac{1}{2}(|x|-|y|)^2_+\leq\min\crl*{\frac{1}{2}x^2,|x||y|}.
    \end{align*}
\end{lemma}
In the following, we let $\alpha = \max_{\W \in \ww} \frac{\nrm{\W}_\F^2}{2} = \frac{B^2}{2}$ for simplicity. Then,
\begin{equation*}
        \sumt{\sw}
        \leq
        \sumt{\su}
        +
        2\prn*{2\max_{s\in[T]}a_{s-D:s-1}+\sqrt{\sum_{t=1}^{T}a_{t}^2+B^2 b_{t}}},
\end{equation*}


\subsection{Proof of \cref{thm:delayed_regret_expectation_abstract} }
\label{app:sub_delayed_regret_expectation_abstract}
We present \cref{thm:delayed_regret_expectation_abstract} in a more detailed form and provide its proof. 
\begin{theorem}[Formal version of \cref{thm:delayed_regret_expectation_abstract}]
    \label{thm:delayed_regret_expectation_abstract_detail}
    Let $\alpha=\frac{B^2}{2}$.
    Then, ODAFTRL with the AdaHedgeD update in online structured prediction with a delay of $D$ achieves
    \begin{equation*}
        \E\brk*{\reg}\leq4 B D \L+\frac{2bB^2}{a}
        +\frac{3}{2}\prn*{a^{-1}bB^4\L^2(D+1)^2T}^{1/3}.
    \end{equation*}
\end{theorem}
\begin{proof}
    From \cref{thm:AdaHedgeD} and the definition of $a_t$ and $b_t$, we have
    \begin{align*}
        \sumt{(\sw-\su)}&\leq2\prn*{2B\max_{s\in[T]}\sum_{i=s-D}^{s-1}\nrm{\G_i}_{\mathrm{F}}
        +\sqrt{\sumt{\prn*{B^2\normst^2+B^2\nrm{\G_{t-D:t}}_\F\normst}}}}\\
        &\leq 2\prn*{2BD\L+\sqrt{bB^2\sumt{\sw}}+  \prn*{b B^4 \L^2 (D+1)^2T\sumt{\sw}}^{1/4}},
    \end{align*}
    where we used the Cauchy--Schwarz inequality, $\normst\leq \L$, $\normst^2\leq b\sw$, and the subadditivity of $x \mapsto \sqrt{x}$ for $x \geq 0$ in the last inequality.
    From this inequality and \cref{asp:delayed_a}, we can evaluate surrogate regret as
    \begin{align*}
        \E\brk*{\reg} &\leq \sumt{\prn*{(1-a)\sw-\su}}\\
        &\leq 2\prn*{2BD\L+\sqrt{bB^2\sumt{\sw}} + \prn*{b B^4 \L^2 (D+1)^2 T\sumt{\sw}}^{1/4}} -a\sumt{\sw}\\
        &\leq 4 B D \L+\frac{2bB^2}{a}
        +\frac{3}{2}\prn*{a^{-1}bB^4\L^2(D+1)^2T}^{1/3}.
    \end{align*}
    We used $c_1\sqrt{x}-c_2x\leq{c_1^2}/\prn{4c_2}$ and $c_1x-c_2x^4\leq\prn{{3}/{4}}\prn*{{c_1^4}/\prn{4c_2}}^{1/3}$ that hold for any $x\geq0$, $c_1\geq 0$, and $c_2>0$ in the last inequality.
\end{proof}


\subsection{Proof of \cref{thm:delayed_regret_probability_abstract}}
\label{app:sub_delayed_regret_probability_abstract}
We present \cref{thm:delayed_regret_probability_abstract} in a more detailed form and provide its proof. 
\begin{theorem}
    \label{thm:delayed_regret_probability_abstract_detail}
    Let $\alpha=\frac{B^2}{2}$ and $\delta \in (0,1)$.
    Then, 
    ODAFTRL with the AdaHedgeD update in online structured prediction with a delay of $D$ achieves
    \begin{equation*}
        \reg\leq 4BD\L+\frac{\sqrt{2}}{3}\log\frac{1}{\delta}
        +\frac{\prn*{\sqrt{(1-a)\log\frac{1}{\delta}}+\sqrt{2bB^2}}^2}{a}
        +\frac{3}{2}\prn*{a^{-1}bB^4\L^2(D+1)^2T}^{1/3},
    \end{equation*}
    with probability at least $1 - \delta$. 
\end{theorem}

\begin{proof}
    We decompose $\reg$ into     
    \begin{equation}\label{eq:decompose_reg}
    \reg
    =\sumt{\prn*{L_t(\yht)-\expect{L_t(\yht)}}}+\sumt{\prn{\expect{L_t(\yht)}-\su}}.
    \end{equation}
    Let $Z_t=L_t(\yht)-\expect{L_t(\yht)}$. 
    Then, we have $|Z_t|\leq 1$ and $\expect{Z_t^2}\leq (1-a)\sw$ from \cref{asp:delayed_a}.
    Hence, from \cref{lem:Bernstein}, with probability at least $1 - \delta$, the first term in \eqref{eq:decompose_reg} is upper bounded as 
    \begin{equation}\label{eq:bound_of_zt}
        \sumt{Z_t}\leq\sqrt{2(1-a)\sumt{\sw}\log\frac{1}{\delta}}+\frac{\sqrt{2}}{3}\log\frac{1}{\delta}.
    \end{equation}
    From from \cref{asp:delayed_a} and \cref{thm:AdaHedgeD},
    the second term in \eqref{eq:decompose_reg} is also upper bounded as 
    \begin{align}
        &
        \sumt{(\expect{L_t(\yht)}-\su)}
        \nonumber \\
        &\leq
        2\prn*{2BD\L+\sqrt{bB^2\sumt{\sw}}+  \prn*{b B^4 \L^2 (D+1)^2T\sumt{\sw}}^{1/4}} - a\sumt{\sw},
        \label{eq:bound_of_reg_exp}
    \end{align}
    where we used the subadditivity of $x \mapsto \sqrt{x}$ for $x \geq 0$.
    Therefore, substituting \eqref{eq:bound_of_zt} and \eqref{eq:bound_of_reg_exp} into \eqref{eq:decompose_reg} gives 
    \begin{align*}
        \reg&\leq 4BD\L+\frac{\sqrt{2}}{3}\log\frac{1}{\delta}
        +\prn*{\sqrt{ 2 (1-a) \log \frac{1}{\delta} } + 2\sqrt{bB^2}}\sqrt{\sumt{\sw}}\\
        &\quad+2\prn*{b B^4 (D+1)^2 \L ^2T\sumt{\sw}}^{1/4} - a\sumt{\sw}\\
        &\leq 4BD\L+\frac{\sqrt{2}}{3}\log\frac{1}{\delta}
        +\frac{\prn*{\sqrt{(1-a)\log\frac{1}{\delta}}+\sqrt{2bB^2}}^2}{a}
        +\frac{3}{2}\prn*{a^{-1}bB^4\L^2(D+1)^2T}^{1/3},
    \end{align*}
    where we used $c_1\sqrt{x}-c_2x\leq{c_1^2}/\prn{4c_2}$ and $c_1x-c_2x^4\leq\prn{{3}/{4}}\prn*{{c_1^4}/\prn{4c_2}}^{1/3}$ for $x\geq0$, $c_1\geq 0$, and $c_2>0$ in the last inequality.
    This is the desired bound.
\end{proof}
\section{Omitted Details of \cref{sec:bandit_and_delayed}}\label{app:bandit_and_delayed}
This section provides the omitted proofs of the theorems in \cref{sec:bandit_and_delayed}.

\subsection{Common analysis}
We provide the analysis that is commonly used in the proofs of \cref{thm:delay_bandit_bound_general_abstract,thm:delay_bandit_bound_self_abstract}.
We use ODAFTRL with the AdaHedgeD update in \cref{app:sub_odaftrl} for $\alg$.
From \cref{thm:AdaHedgeD}, it holds that
\begin{equation}\label{eq:inpr_common}
    \sumt{\inpr{\gtil,\W_t-\U}} 
    \leq 2\prn*{2\max_{t\in[T]}a_{t-D:t-1}+\sqrt{\sumt{a_t^2}+B^2 b_t}},
\end{equation}
where
\begin{equation*}
    a_t=B\min\set{\nrm{\tilde{\G}_{t-D:t}}_\F,\nrm{\gtil}_\F} \quad \mbox{and} \quad b_t\leq\min\set*{\frac{1}{2}\nrm{\tilde{\G}_{t-D:t}}_\F^2, \nrm{\tilde{\G}_{t-D:t}}_\F\nrm{\gtil}_\F}.
\end{equation*}
By the definition of $a_t$, we have 
\begin{equation}\label{eq:bound_of_at}
    \E\brk*{\max_{t\in\brk{T}}a_{t-D:t-1}}\leq B\E\brk*{\max_{t\in\brk{T}} \sum_{s=t-D}^{t-1}\nrm{\tilde{\G}_{s}}_\F }\leq B\E\brk*{\sqrt{D\max_{t\in\brk{T}}\sum_{s=t-D}^{t-1}\nrm{\tilde{\G}_s}_\F^2}}\leq B\sqrt{D\E\brk*{\sumt{\nrm{\gtil}_\F^2}}},
\end{equation}
from the Cauchy--Schwarz inequality and Jensen's inequality.
Hence, from \eqref{eq:inpr_common} and \eqref{eq:bound_of_at}, we have
\begin{align}\label{eq:bandit_delayed_expect_inpr}
    \E\brk*{\sumt{\inpr{\gtil,\W_t-\U}}}
    &\leq 2\prn*{2\E\brk*{\max_{t\in[T]}a_{t-D:t-1}}+\sqrt{\E\brk*{\sumt{a_t^2}}}+B\sqrt{ \E\brk*{\sumt{b_t}}}}\nonumber\\
    &\leq 2B\prn*{2\sqrt{D\E\brk*{\sumt{\nrm{\gtil}_\F^2}}}+\sqrt{\E\brk*{\sumt{\nrm{\gtil}_\F^2}}}+\sqrt{\E\brk*{\sumt{b_t}}}},
\end{align}
where we used the subadditivity of $x \mapsto \sqrt{x}$ for $x \geq 0$.
The last term in the last inequality is further bounded as
\begin{align}\label{eq:expect_bt}
    \E\brk*{\sumt{b_t}}&\leq \E\brk*{\sumt{\nrm{\tilde{\G}_{t-D:t}}_\F\nrm{\gtil}_\F}}
    \leq\E\brk*{\sumt{\nrm{\gtil}_\F^2}} +\E\brk*{\sumt{\nrm{\gtil}_\F\sum_{s=t-D}^{t-1}\nrm{\tilde{\G}_s}_\F}}\nonumber\\
    &=\E\brk*{\sumt{\nrm{\gtil}_\F^2}} + \E\brk*{\sumt
    {\E_t\brk*{\nrm{\gtil}_\F}\sum_{s=t-D}^{t-1}\nrm{\tilde{\G}_s}_\F}}, 
\end{align}
where the second inequality follows from the triangle inequality and the equality follows from the law of total expectation.


\subsection{Proof of \cref{thm:delay_bandit_bound_general_abstract}}\label{app:bandit_delayed_general}
We provide the complete version of \cref{thm:delay_bandit_bound_general_abstract}:
\begin{theorem}[Formal version of \cref{thm:delay_bandit_bound_general_abstract}]
    The algorithm in \cref{subsec:bandit_delay_general} achieves
\begin{equation*}
    \E\brk*{\reg}
    \leq \frac{4bB}{a}\prn*{D^{1/4}+D^{-1/4}}^2\sqrt{KT}+2B \L\sqrt{D T}+B\sqrt{DKT}
    = O\prn{D\sqrt{KT}}
    .
\end{equation*}
\end{theorem}

\begin{proof}
First, we will upper bound $\E\brk*{\sumt{b_t}}$.
From \eqref{eq:expect_bt}, we have
\begin{align}\label{eq:bound_of_bt_general}
    \E\brk*{\sumt{b_t}}&\leq \E\brk*{\sumt{\nrm{\gtil}_\F^2}} + \E\brk*{\sumt
    {\E_t\brk*{\nrm{\gtil}_\F}\sum_{s=t-D}^{t-1}\nrm{\tilde{\G}_s}_\F}}\nonumber\\
    &\leq \E\brk*{\sumt{\nrm{\gtil}_\F^2}} + \L\E\brk*{\sumt{\sum_{s=t-D}^{t-1}\E_s\brk*{\nrm{\tilde{\G}_s}_\F}}}\nonumber\\
    &\leq \E\brk*{\sumt{\nrm{\gtil}_\F^2}} + D\L^2 T,
\end{align}
where the second and third inequality follow from the inequality $\expect{\nrm{\gtil}_\F}=\nrm{\G_t}_\F\leq \L$.
Hence, from \eqref{eq:bandit_delayed_expect_inpr} and \eqref{eq:bound_of_bt_general}, it holds that
\begin{align}
    \E\brk*{\sumt{\inpr{\gtil,\W_t-\U}}}
    &\leq 2B\prn*{2\sqrt{D\E\brk*{\sumt{\nrm{\gtil}_\F^2}}}+\sqrt{\E\brk*{\sumt{\nrm{\gtil}_\F^2}}}
    +
    \sqrt{{\E\brk*{\sumt{\nrm{\gtil}_\F^2}} + D\L^2 T}}}\nonumber\\
    &\leq 4B\prn*{\sqrt{D}+1}\sqrt{\E\brk*{\sumt{\nrm{\gtil}_\F^2}}}+2 B\L\sqrt{ D T}\nonumber\\
    &\leq 4B\prn*{\sqrt{D}+1}\sqrt{\frac{bK}{q}\E\brk*{\sumt{\sw}}}+2B\L\sqrt{ D T},
\end{align}
where in the second inequality we used the subadditivity of $x \mapsto \sqrt{x}$ for $x \geq 0$ and in the last inequality we used 
\begin{equation*}
\expect{\|\gtil\|_{\mathrm{F}}^2}
=
\frac{\|\G_t\|_{\mathrm{F}}^2}{p_t(\yt)}
\leq
\frac{\K }{q}\|\G_t\|_{\mathrm{F}}^2
\leq
\frac{b\K }{q}\sw.
\end{equation*}
Therefore, combining all the above arguments yields 
\begin{align*}
    \E\brk*{\reg}&
    \leq \E\brk*{\sumt{(\sw-\su)}} - a\E\brk*{\sumt{\sw}} + q T\\
    &\leq \E\brk*{\sumt{\inpr{\gtil,\W_t-\U}}} - a\E\brk*{\sumt{\sw}} + q T\\
    &\leq 4B\prn*{\sqrt{D}+1}\sqrt{\frac{bK}{q}\E\brk*{\sumt{\sw}}}+2 B\L\sqrt{ D T}- a\E\brk*{\sumt{\sw}} + q T\\
    &\leq \frac{4bB^2}{a}\prn*{\sqrt{D}+1}^2\frac{bK}{q}+2 B\L\sqrt{D T}+ q T,
\end{align*}
where the first inequality follows from \cref{asp:bandit_a},
the second inequality follows from the convexity of $S_t$ and the unbiasedness of $\gtil$,
and the last inequality follows from $c_1\sqrt{x}-c_2x\leq{c_1^2}/\prn{4c_2}$ for $x \geq 0$, $c_1 \geq 0$, and $c_2 > 0$.
Finally, by substituting $q=B\sqrt{DK/T}$, we obtain
\begin{equation*}
    \E\brk*{\reg}
    \leq
    \frac{4bB}{a}\prn*{D^{1/4}+D^{-1/4}}^2\sqrt{KT}+2B \L\sqrt{D T}+B\sqrt{DKT},
\end{equation*}
which is the desired bound.
\end{proof}





\subsection{Proof of \cref{thm:delay_bandit_bound_self_abstract}}\label{app:bandit_delayed_self}
We provide the complete version of \cref{thm:delay_bandit_bound_self_abstract}:
\begin{theorem}[Formal version of \cref{thm:delay_bandit_bound_self_abstract}]
    The algorithm in \cref{subsec:bandit_delay_self} achieves
    \begin{equation*}
        \E\brk{\reg}
        =
        \frac{8bB^2}{a}\prn[\Big]{\sqrt{D} + 1}^2
    +2B (\L+\dix \diy )\sqrt{DT}
        + O\prn*{ B^{2/3} D^{1/3} T^{2/3} }. 
    \end{equation*}
\end{theorem}
\begin{proof}
First, we will derive an upper bound of $\E\brk*{\sumt{b_t}}$.
We first observe that
\begin{align*}
    \expect{\nrm{\gtil}_\F}&=\expect{\nrm{(\yho(\tht)-\ytilde)\xt^\top}_\F}\leq \expect{\nrm{\G_t}_\F + \dix \nrm{\yt-\ytilde}_2}
    \leq \nrm{\G_t}_\F + \dix \expect{\nrm{\yt}_2+\nrm{\ytilde}_2}\\
    &\leq \nrm{\G_t}_\F + \dix \diy + \dix \expect{\sqrt{\tr\prn*{\ytilde\ytilde^\top}}}
    \leq \nrm{\G_t}_\F + \dix \diy + \dix {\sqrt{\expect{\tr\prn*{\ytilde\ytilde^\top}}}}\\
    &\leq \nrm{\G_t}_\F + \dix \diy + \sqrt{\dix^2\omega / q}
    \leq \L+ \dix \diy + \sqrt{\dix^2\omega / q},
\end{align*}
where the first inequality follows from $\dix =\diam(\xx)$, the third inequality follows from $\diy=\diam(\yy)$, the fourth inequality follows from Jensen's inequality, and the fifth inequality follows from \cref{lem:bound of trace}.
Thus, combining \eqref{eq:expect_bt} with the last inequality, we have
\begin{align}
    \E\brk*{\sumt{b_t}}&\leq \E\brk*{\sumt{\nrm{\gtil}_\F^2}} + \E\brk*{\sumt
    {\E_t\brk*{\nrm{\gtil}_\F}\sum_{s=t-D}^{t-1}\nrm{\tilde{\G}_s}_\F}} \nonumber \\
    &\leq \E\brk*{\sumt{\nrm{\gtil}_\F^2}}+ \prn*{\L+ \dix \diy + \sqrt{\frac{\dix^2\omega}{q}} } \E\brk*{\sumt{\sum_{s=t-D}^{t-1}\E_s\brk*{\nrm{\tilde{\G}_s}_\F}}} \nonumber \\
    & \leq\E\brk*{\sumt{\nrm{\gtil}_\F^2}}+ D T \prn*{\L+ \dix \diy + \sqrt{\frac{\dix^2\omega}{q}} }^2.
    \label{eq:expect_gtil_self}
\end{align}
Hence, from \eqref{eq:bandit_delayed_expect_inpr} and \eqref{eq:expect_gtil_self}, we have
\begin{align}\label{eq:expext_inpr_self}
    &\E\brk*{\sumt{\inpr{\gtil,\W_t-\U}}}\nonumber\\
    &\leq 2B\prn*{2\sqrt{D\E\brk*{\sumt{\nrm{\gtil}_\F^2}}}+\sqrt{\E\brk*{\sumt{\nrm{\gtil}_\F^2}}}+\sqrt{\E\brk*{\sumt{b_t}}}}\nonumber\\
    &\leq 4B\prn*{\sqrt{D}+1}\sqrt{\E\brk*{\sumt{\nrm{\gtil}_\F^2}}} + 2B\prn*{\L+ \dix \diy + \sqrt{ \dix^2\omega / q } } \sqrt{DT}\nonumber\\
    &\leq 4B\prn*{\sqrt{D}+1}\sqrt{2\sumt{\prn*{b\sw+\frac{\dix^2\omega}{q}}}
    }
    + 2B\prn*{\L+ \dix \diy + \sqrt{\dix^2\omega / q} } \sqrt{DT},
\end{align}
where the first inequality follows from \eqref{eq:bandit_delayed_expect_inpr}, 
the second inequality follows from \eqref{eq:expect_gtil_self} and the subadditivity of $x \mapsto \sqrt{x}$ for $x \geq 0$, 
and the last inequality follows from \cref{thm:evaluation of Gtilde}.
Therefore, combining all the above arguments yields 
\begin{align*}
    \E\brk*{\reg}
    &\leq \E\brk*{\sumt{(\sw-\su)}} - a\E\brk*{\sumt{\sw}} + q T\\
    &\leq \E\brk*{\sumt{\inpr{\gtil,\wt-\U}}} - a\E\brk*{\sumt{\sw}} + q T\\
    & \leq 4B\prn*{\sqrt{D} + 1}\prn*{\sqrt{2b\sumt{\sw}}+\sqrt{ 2\dix^2\omega T / q}
    }
    \nonumber \\
    &\qquad+ 2B\prn*{\L+ \dix \diy + \sqrt{\dix^2\omega / q}  } \sqrt{DT} - a\E\brk*{\sumt{\sw}} + q T \\
    &\leq
    \frac{8bB^2}{a}\prn[\Big]{\sqrt{D} + 1}^2
    +2B (\L+\dix \diy )\sqrt{DT}
    +2B\dix\prn[\big]{(2\sqrt{2}+1)\sqrt{D}+2\sqrt{2}}\sqrt{\omega T/ q} + q T,
\end{align*}
where the first inequality follows from \cref{asp:bandit_a}, 
the third inequality follows from \eqref{eq:expext_inpr_self} and the subadditivity of $x \mapsto \sqrt{x}$ for $x \geq 0$, 
and the last inequality follows from the definition of $\epsilon$ and $c_1\sqrt{x}-c_2x\leq{c_1^2}/\prn{4c_2}$ for $x \geq 0$, $c_1 \geq 0$, and $c_2 > 0$.
Finally, substituting $q=\prn*{\frac{\omega B^2\dix^2 D}{T}}^{1/3}$ gives the desired bound.
\end{proof}



\section{Overhead of \ourSystem.}
We report the size and inference time of the model for NeRF and \ourSystem in Table~\ref{table_overhead}.  
The results indicate that \ourSystem has a larger model size than NeRF, \ie 27.1 \vs 8.0\,MB.  
Correspondingly, \ourSystem exhibits a longer inference time, \ie 1.79 \vs 0.43\,s.  
Unlike NeRF, \ourSystem requires neighboring spectra as input. 
During inference, the target transmitter's neighbors are extracted from the training dataset, so \ourSystem does not incur additional data burdens.  
Moreover, since \ourSystem can operate in unseen scenes, it significantly reduces the requirement for a time-consuming training process.



\begin{table}[h]
\centering
\caption{Comparison of model size and inference time.}

\begin{tabular}{lC{0.8in}C{0.8in}}
\toprule
     & \nerft    & \ourSystem    
     \\ \midrule
Model size (MB) & 8.0  & 27.1   \\
Inference time (s) & 0.43    & 1.79  \\
\bottomrule
\end{tabular}
\label{table_overhead}
\end{table}






\end{document}
% \bibliography{./ref}


%\bibliographystyle{ACM-Reference-Format}
\bibliography{./ref}
% \documentclass[11pt]{article}

% preamble here
\usepackage[top=30truemm,bottom=30truemm,left=25truemm,right=25truemm]{geometry}
\usepackage[utf8]{inputenc} % allow utf-8 input
\usepackage[T1]{fontenc}    % use 8-bit T1 fonts

\usepackage{microtype}
\usepackage{graphicx}
% \usepackage[dvipdfmx]{graphicx}
\usepackage{subcaption}
\usepackage{booktabs} 

\usepackage[colorlinks=true,citecolor=blue,linkcolor=blue]{hyperref}


\newcommand{\theHalgorithm}{\arabic{algorithm}}

\usepackage{amsmath,amssymb,amsfonts,amsthm}
\usepackage{bm,bbm}
\usepackage{mathcomp}
\usepackage{empheq}
\usepackage{fancybox}
\usepackage{breqn}
\usepackage{mathtools}
\mathtoolsset{centercolon}
\usepackage{tgtermes}

\usepackage[capitalize,noabbrev,nameinlink]{cleveref}

\usepackage[textsize=tiny]{todonotes}

\usepackage{my_macro}
\usepackage[authoryear,round]{natbib} 
\usepackage[font=small,labelfont=bf]{caption}



\theoremstyle{plain}
\newtheorem{theorem}{Theorem}[section]
\newtheorem{proposition}[theorem]{Proposition}
\newtheorem{lemma}[theorem]{Lemma}
\newtheorem{corollary}[theorem]{Corollary}
\theoremstyle{definition}
\newtheorem{definition}[theorem]{Definition}
\newtheorem{method}[theorem]{Method}
\newtheorem{assumption}[theorem]{Assumption}
\theoremstyle{remark}
\newtheorem{remark}[theorem]{Remark}

\title{Bandit and Delayed Feedback in Online Structured Prediction}

\author{
  Yuki Shibukawa\footnote{
    The University of Tokyo; 
    \texttt{shibu-yu762@g.ecc.u-tokyo.ac.jp}.
  }
  \and
  Taira Tsuchiya\footnote{
    The University of Tokyo and RIKEN; \texttt{tsuchiya@mist.i.u-tokyo.ac.jp}.
  }
  \and
  Shinsaku Sakaue\footnote{
    The University of Tokyo and RIKEN; \texttt{sakaue@mist.i.u-tokyo.ac.jp}.
  }
  \and 
  Kenji Yamanishi\footnote{
    The University of Tokyo; \texttt{yamanishi@g.ecc.u-tokyo.ac.jp}.
  }
}


\begin{document}
\maketitle

% abstract here
\begin{abstract}
Online structured prediction is a task of sequentially predicting outputs with complex structures based on inputs and past observations, encompassing online classification. Recent studies showed that in the full information setup, we can achieve finite bounds on the \textit{surrogate regret}, i.e., the extra target loss relative to the best possible surrogate loss. In practice, however, full information feedback is often unrealistic as it requires immediate access to the whole structure of complex outputs. Motivated by this, we propose algorithms that work with less demanding feedback, \textit{bandit} and \textit{delayed} feedback. For the bandit setting, using a standard inverse-weighted gradient estimator, we achieve a surrogate regret bound of $O(\sqrt{KT})$ for the time horizon $T$ and the size of the output set $K$. However, $K$ can be extremely large when outputs are highly complex, making this result less desirable. To address this, we propose an algorithm that achieves a surrogate regret bound of $O(T^{2/3})$, which is independent of $K$. This is enabled with a carefully designed pseudo-inverse matrix estimator. Furthermore, for the delayed full information feedback setup, we obtain a surrogate regret bound of $O(D^{2/3} T^{1/3})$ for the delay time $D$. We also provide algorithms for the delayed bandit feedback setup. Finally, we numerically evaluate the performance of the proposed algorithms in online classification with bandit feedback.
\end{abstract}


\section{Introduction}
In many machine learning problems, given an input vector from a vector space $\xx$, we aim to predict a vector in a finite output space $\yy$.  
Multiclass classification is one of the simplest examples, while in other cases output spaces may have more complex structures. 
\emph{Structured prediction} refers to such a class of problems with structured output spaces, including multiclass classification, multilabel classification, ranking, and ordinal regression, and it has applications in various fields, ranging from natural language processing to bioinformatics \citep{JMLR_Tsochantaridis_2005, bakir_2007_article}.
In structured prediction, training models that directly predict outputs in complex discrete output spaces is typically challenging. 
Therefore, we often adopt the \emph{surrogate loss framework} \citep{Bartlett_2006}---define an intermediate space of score vectors and train models that estimate score vectors from inputs based on surrogate loss functions.
Examples of surrogate losses include squared, logistic, and hinge losses, and a general framework encompassing them is the \emph{Fenchel--Young loss} \citep{JMLR_2020_blondel}, which we rely on in this study.


Structured prediction can be naturally extended to the online setting \citep{pmlr-v247-sakaue24a}.  
In this online setting, at each round $t=1,\dots,T$, the environment selects an input-output pair $(\xt,\yt)\in\xx\times\yy$.  
A learner then predicts $\yht \in \yy$ based on the input $\xt$ and incurs a loss $L(\yht;\yt)$, where $L:\yy\times\yy\to\R_{\geq0}$ is the target loss function. 
As with \citet{pmlr-v247-sakaue24a}, we focus on the simple yet fundamental case where the learner's model for estimating score vectors is linear. 

\begin{table*}[t]
\centering
\caption{Surrogate regret upper and lower bounds in online multiclass classification and online structured prediction. Here, $T$ is the time horizon, $K$ is the size of the set, $\yy$, of output vectors, and $D$ is the fixed delayed time.  
``FI'' is the abbreviation of full information.
Delayed feedback is considered only when ``Delayed'' appears in the feedback column. 
In the target loss column, ``SELF*'' means SELF that satisfies \cref{asp:self}.
Note that the $O(T^{2/3})$ bounds for SELF* in lines 6 and 9 do not explicitly depend on $\K$ but on $d$; 
in the case of multiclass classification with the 0-1 loss, the dependence on $\K$ appears as $d = \K$.
} 
\small
\begin{tabular}{@{}l@{\hspace{1ex}}l@{\hspace{1ex}}l@{\hspace{1ex}}l@{\hspace{1ex}}l@{}}
\toprule
& Problem setup &Feedback & Target loss &  Surrogate regret bound \\ 
\midrule
 \citet[Cor.~1]{NEURIPS2021_Hoeven} & Binary classification & Bandit & 0-1 loss & $\Omega(\sqrt{T})$ ($d=2$) \\ 
 \midrule
 \citet[Thm.~4]{NEURIPS2020_Hoeven} & Multiclass classification & Bandit & 0-1 loss & $O(\K \sqrt{T})$ \\ 
 \citet[Thm.~1]{NEURIPS2021_Hoeven} & Multiclass classification & Bandit & 0-1 loss & $O(\K \sqrt{T})$ \\ 
 \midrule
\citet[Thms.~7 and 8]{pmlr-v247-sakaue24a} & Structured prediction & FI & SELF & $O(1)$  \\
\textbf{This work} (\cref{thm:bandit_regret_expectation_abstract,thm:bandit_high_prob}) & Structured prediction &Bandit & SELF &  $O(\sqrt{\K T})$  \\
\textbf{This work}  (\cref{thm:bandit_regret_pseudo_estimator}) & Structured prediction & Bandit & SELF* &  $O(T^{2/3})$ \\
\textbf{This work}  (\cref{thm:delayed_regret_expectation_abstract,thm:delayed_regret_probability_abstract}) & Structured prediction & FI \& Delayed & SELF & $O(D^{2/3}T^{1/3})$  \\
\textbf{This work}  (\cref{thm:delay_bandit_bound_general_abstract}) & Structured prediction & Bandit \& Delayed & SELF & $O(\sqrt{DKT})$  \\
\textbf{This work}  (\cref{thm:delay_bandit_bound_self_abstract}) & Structured prediction & Bandit \& Delayed & SELF* & $O(D^{1/3}T^{2/3})$  \\
\bottomrule
\end{tabular}
\label{tab: regret order}
\end{table*}


The goal of the learner is to minimize the cumulative loss $\sumt{L(\yht;\yt)}$. 
On the other hand, the best the learner can do in the surrogate loss framework is to minimize the cumulative surrogate loss, namely $\sumt{S(\U\xt;\yt)}$, where $\U:\xx\to\R^d$ is the best offline linear estimator and $S:\R^d\times\yy\to\R_{\geq0}$ is a surrogate loss, which measures the discrepancy between the score vector $\U\xt \in \R^d$ and $\yt\in\yy$. 
Given this, a natural performance measure of the learner's predictions is the following \emph{surrogate regret} $\reg$: 
\begin{equation}\label{eq:sur_regret}
    \sumt{L(\yht;\yt)}=\sumt{S(\U\xt;\yt)}+\reg.
\end{equation}
The surrogate regret was introduced in the seminal paper by \Citet{NEURIPS2020_Hoeven} in the context of online multiclass classification.  
Recently, \citet{pmlr-v247-sakaue24a} showed that a finite surrogate regret bound  
can be achieved for online structured prediction under full information feedback, i.e., the learner can observe $\yt$ at the end of each round $t$.

However, the assumption that full information feedback is available is often demanding, especially when outputs have complex structures.
%
For example, in sequential ad assortment on an advertising platform, we may be able to observe only the click-through rate but not which ads were clicked, which boils down to the \emph{bandit feedback} setting \citep{Kakade2008EfficientBA,JMLR:v15:gentile:class_ranking}. 
Also, we may only have access to feedback from a while ago when designing an ad assortment for a new user—namely, \emph{delayed feedback}  \citep{Weinberger_2002_delay,manwani2022delaytronefficientlearningmulticlass}.
%
Similar situations have led to a plethora of studies in various online learning settings.
In combinatorial bandits, algorithms under bandit feedback (referred to as full-bandit feedback in their context), instead of full information feedback, have been widely studied
\citep{comband,combes15combinatorial,rejwan20topk,du21combinatorial}. 
Delayed feedback is also explored in various settings, including full information and bandit feedback \citep{joulani13online,cesabianchi16delay}.
Due to space limitations, we defer a further discussion of the background to \cref{app:additional_related_work}.


\subsection*{Our Contributions}
To extend the applicability of online structured prediction, this study develops online structured prediction algorithms that can handle weaker feedback---bandit feedback and delayed feedback---instead of full information feedback.  
As with \citet{pmlr-v247-sakaue24a}, we consider the case where target loss functions belong to a class called the Structured Encoding Loss Function (SELF) \citep{ciliberto16consistent,NEURIPS2019_Blondel}, a general class including the 0-1 loss in multiclass classification and the Hamming loss in multilabel classification and ranking (see \cref{subsec:self} for the definition). 
\Cref{tab: regret order} summarizes the surrogate regret bounds provided in this study and comparisons with existing results.


One of the challenges of bandit feedback is that the true output $\yt$ is not observed, making it impossible to compute the true gradient of the surrogate loss. 
To deal with this, we use an inverse-weighted gradient estimator, a typical approach that assigns weights to gradients by the inverse of choosing each output, establishing an $O(\sqrt{\K T})$ surrogate regret upper bounds (\cref{thm:bandit_regret_expectation_abstract,thm:bandit_high_prob}), 
where $K = \abs{\yy}$ is the cardinality of $\yy$.  
The $O(\sqrt{\K T})$ bound has an optimal dependence on $T$; it matches the $\sqrt{T}$ lower bound in the special case of online multiclass classification with bandit feedback \Citep[Corollary 1]{NEURIPS2021_Hoeven}. 
Furthermore, our bound is better than the existing $O(\K\sqrt{T})$ bound of \Citet{NEURIPS2020_Hoeven} by a factor of $\sqrt{\K}$, while it is not directly comparable to the latest $O(\K\sqrt{T})$ bound in \Citet{NEURIPS2021_Hoeven} due to differences in surrogate loss functions. 
See \cref{app:Discussio_on_the_Difference_in_Surrogate_Losses} for a more detailed discussion. 

While the $O(\sqrt{\K T})$ bound is satisfactory when $K = \abs{\yy}$ is small, $K$ can be extremely large in some structured prediction problems: in multilabel classification with $m$ correct labels, we have $\K=\binom{d}{m}$, and in ranking problems with $m$ items, we have $\K=m!$.
%
To address this issue, we consider a special case of SELF (denoted by SELF* in \cref{tab: regret order}), which still includes the aforementioned examples: the 0-1 loss in multiclass classification and the Hamming loss in multilabel classification and ranking. 
A technical challenge to resolve the issue lies in designing an appropriate gradient estimator used in online learning methods.
To this end, we draw inspiration from pseudo-inverse estimators used in the adversarial linear/combinatorial bandit literature \citep{dani07price,abernethy08competing,comband}. 
While we cannot naively apply the existing estimators, we design a new gradient estimator that applies to various specific structured prediction problems belonging to the special SELF framework.
Armed with this gradient estimator, we achieve a surrogate regret upper bound of $O(T^{2/3})$, which does not explicitly depend on~$\K$~(\cref{thm:bandit_regret_pseudo_estimator}).

For the delayed feedback setting with a known fixed delay time of $D$, it is actually not difficult to obtain a surrogate regret bound of $O(\sqrt{D T})$ with standard Online Convex Optimization (OCO) algorithms for delayed feedback. 
Our finding is that we can achieve a surrogate regret bound of $O(D^{2/3} T^{1/3})$ in online structured prediction under delayed full information feedback (\cref{thm:delayed_regret_expectation_abstract,thm:delayed_regret_probability_abstract}) by leveraging ODAFTRL \citep{pmlr-v139-flaspohler21a}, a Follow-the-Regularized-Leader-type algorithm that achieves an AdaGrad-type regret upper bound in OCO under delayed feedback. 
This bound is better than $O(\sqrt{D T})$ as $D \le T$.


Given the contributions so far, it is natural to explore online structured prediction in environments where both delay and bandit feedback are present. 
We obtain algorithms for this setup by combining the theoretical developments for bandit feedback without delay and delayed full information feedback, offering surrogate regret bounds of $O(\sqrt{D K T})$ (\cref{thm:delay_bandit_bound_general_abstract}) and $O(D^{1/3} T^{2/3})$ (\cref{thm:delay_bandit_bound_self_abstract}).

We validate our algorithms through numerical experiments using both synthetic and real-world data.  
Specifically, we consider online multiclass classification with bandit feedback. 
We observe that, depending on the number of classes and the dataset, our algorithm, designed for general structured prediction, can achieve accuracy comparable to existing algorithms specialized for multiclass classification.



\section{Preliminaries}
We describe the detailed setup of online structured prediction and key tools used in this work: the Fenchel--Young loss, SELF, and randomized decoding.
\paragraph{Notation}
For any integer $n > 0$, let $\brk{n} = \set{1,2,\hdots,n}$.
Let $\nrm{\cdot}$ denote a norm with $\kappa\nrm{\bmy}\geq\nrm{\bmy}_2$ for some $\kappa>0$ for any $\bmy\in\mathbb{R}^d$. 
For a matrix $\W$, let $\nrm{\W}_{\mathrm{F}}=\sqrt{\tr\prn*{\W^\top \W}}$ be the Frobenius norm. 
Let $\bm1$ denote the all-ones vector and $\bm{\ee}_i$ the $i$th standard basis vector.
For $\mathcal{\mathcal{K}}\subset \mathbb{R}^d$, let $\conv(\mathcal{K})$ be its convex hull and $I_{\mathcal{K}}:\mathbb{R}^d\to\set{0, +\infty}$ be its indicator function, which takes zero if $\bmy \in \mathcal{K}$ and $+\infty$ otherwise.
For $\Omega:\mathbb{R}^d\to\mathbb{R}\cup\set{+\infty}$, let $\dom(\Omega) \coloneqq \set*{\bmy \in \mathbb{R}^d:\Omega(\bmy) < +\infty}$ be its effective domain and $\Omega^*(\thb) \coloneqq \sup\set*{\inpr{\thb, \bmy} - \Omega(\bmy):\bmy \in \mathbb{R}^d}$ be its convex conjugate.
\cref{tab: notation} in \Cref{app: notation} summarizes the notation used in this paper.


\subsection{Online Structured Prediction}
Here, we describe the problem setting of online structured prediction.
Let $\xx$ be the input vector space and $\yy$ be the output vector space.
Define $\K \coloneqq |\yy|$.
Following \citet{JMLR_2020_blondel} and \citet{pmlr-v247-sakaue24a}, we assume that $\yy$ is embedded into $\mathbb{R}^d$ in a standard manner.
For example, $\yy = \set{\bm{\mathrm{e}}_1,\hdots,\bm{\mathrm{e}}_d}$ in $d$-class multiclass classification.


A linear estimator $\W\in\ww$ for a convex domain $\ww$ is used to transform the input vector $\bm{x}$ into the score vector $\W\bm{x}$.  
In online structured prediction, at each round $t=1,\dots,T$:
\begin{enumerate}%[topsep=2pt,itemsep=0pt, partopsep=0pt, leftmargin=18pt]
    \item The environment selects an input $\xt\in\xx$ and the true output $\yt\in\yy$; 
    \item The learner receives $\xt$ and computes the score vector~$\tht=\wt\xt$ using the linear estimator~$\wt$;
    \item The learner selects a predicted output $\yht$ based on $\tht$ and incurs a loss of $L(\yht;\yt)$;
    \item The learner receives feedback based on the problem setup and updates $\wt$ to $\W_{t+1}$ using an online learning algorithm, $\alg$.
\end{enumerate}
The goal of the learner is to minimize the cumulative prediction loss $\sumt{L(\yht;\yt)}$, which is equivalent to minimizing the surrogate regret $\mathcal{R}_T$ in \eqref{eq:sur_regret}.  
We assume that the input and output are generated in an oblivious manner.  
Note that when $\yy = \set{\bm{\ee}_1, \dots, \bm{\ee}_d}$ and $L(\yht; \yt) = \ind[\yht \neq \yt]$, the above setting reduces to online multiclass classification, which was studied by \Citet{NEURIPS2020_Hoeven} and \Citet{NEURIPS2021_Hoeven}.
We will use $B\coloneqq\diam(\ww)$, $\dix\coloneqq\diam(\xx)$, and $\diy\coloneqq\diam(\yy)$ to denote the diameters of the sets $\ww$, $\xx$, and $\yy$, respectively.


The feedback observed by the learner depends on the problem setting.  
The most fundamental setting is the full information setup, where the true output $\yt$ is observed as feedback at the end of each round $t$. 
This setup was extensively investigated in \citet{pmlr-v247-sakaue24a}.  
By contrast, our study investigates the following weaker feedback:
\begin{itemize}%[topsep=2pt,itemsep=0pt, partopsep=0pt, leftmargin=18pt]
    \item \textbf{Bandit feedback}: Only the value of the loss function is observed. 
    Specifically, at the end of each round $t$, the learner observes the target loss value $L(\yht;\yt)$ as feedback.  
    \item \textbf{Delayed feedback}: The feedback is observed with a certain delay. 
    We consider a fixed $D$-round delay setting, i.e., no feedback is received for round $t\leq D$,  
    and for $t>D$, the learner observes either full information feedback $\bmy_{t-D}$ or bandit feedback $L(\hat{\bm{y}}_{t-D}; \bm{y}_{t-D})$.  
\end{itemize}


In this paper, we make the following assumptions:
\begin{assumption}
    \label{asp:online_structured_prediction}
    (1)~There exists $\nu>0$ such that for any distinct $\bm{y},\bm{y}^\prime\in \mathcal{Y}$, it holds that $\|\bm{y}-\bm{y}^\prime\|\geq\nu$.  
    (2) For each $\bm{y}\in\mathcal{Y}$, the target loss function $L(\cdot;\bm{y})$ is defined on $\conv(\mathcal{Y})$, is non-negative, and is affine with respect to its first argument.  
    (3) There exists $\gamma$ such that for any $\bm{y}^\prime\in\conv(\mathcal{Y})$ and $\bm{y}\in\mathcal{Y}$, it holds that $L(\bm{y}^\prime;\bm{y})\leq\gamma\|\bm{y}^\prime-\bm{y}\|$ and $L(\bmy^{\prime};\bmy)\leq 1$. 
    (4) It holds that $L(\bm{y}'; \bm{y})=0$ only if $\bm{y}'=\bm{y}$.
\end{assumption}

As discussed in \citet[Section 2.3]{pmlr-v247-sakaue24a}, these assumptions are natural and hold for a broad range of problem settings and target loss functions, including SELF (see \cref{subsec:self} for the formal definition).


\subsection{Fenchel--Young Loss}\label{subsec:fenchel-young}

We use the Fenchel--Young loss \citep{JMLR_2020_blondel} as the surrogate loss, which subsumes many representative surrogate losses, such as logistic loss, Conditional Random Field (CRF) loss \citep{lafferty01conditional}, and SparseMAP \citep{Niculae18sparse}.
See \citet[Table 1]{JMLR_2020_blondel} for more examples. 
\begin{definition}[{\citealt[Fenchel--Young loss]{JMLR_2020_blondel}}]
    \label{def: Fenchel--Young Loss}
    Let $\Omega:\mathbb{R}^d\rightarrow\mathbb{R}\cup\set{+\infty}$ be a regularization function with $\mathcal{Y}\subset\operatorname{dom}(\Omega)$.  
    The Fenchel--Young loss generated by $\Omega$, denoted by $S_\Omega:\operatorname{dom}(\Omega^\ast)\times\operatorname{dom}(\Omega)\rightarrow\mathbb{R}_{\geq0}$, is defined as
    \[
    S_{\Omega}(\thb;\bmy)\coloneqq\Omega^\ast(\thb)+\Omega(\bmy)-\inpr{\thb,\bmy}.
    \]
\end{definition}
The Fenchel--Young loss has the following properties, which will be useful in the subsequent discussion:
\begin{proposition}[{\citealt[Propositions 2 and~3]{JMLR_2020_blondel} and \citealt[Proposition 3]{pmlr-v247-sakaue24a}}]
    \label{prop:fenchel}
    Let $\Psi:\mathbb{R}^d\rightarrow\mathbb{R}\cup\set{+\infty}$ be a differentiable, Legendre-type function\footnote{
    A function is called Legendre-type if, for any sequence $x_1,x_2,\hdots$ in $\operatorname{int}(\dom(\Psi))$ that converges to a boundary point of $\operatorname{int}(\dom(\Psi))$, it holds that $\lim_{i\rightarrow\infty}\|\nabla\Psi(x_i)\|_2=+\infty$.}  
    that is $\lambda$-strongly convex with respect to $\|\cdot\|$, and suppose that $\conv(\mathcal{Y})\subset\dom(\Psi)$ and $\dom(\Psi^\ast)=\mathbb{R}^d$.  
    Define $\Omega=\Psi+I_{\conv(\yy)}$ and let $S_\Omega$ be the Fenchel--Young loss generated by $\Omega$.  
    For any $\thb\in\mathbb{R}^d$, we define the regularized prediction function as 
    \begin{align*}
        \yho(\thb)&\coloneqq\argmax\{\inpr{\thb,\bmy}-\Omega(\bmy)\::\:\bmy\in\mathbb{R}^d\}\\
        &=\argmax\set{\inpr{\thb,\bmy}-\Psi(\bmy)\::\:\bmy\in\conv(\mathcal{Y})}.
    \end{align*}
    Then, for any $\bmy\in\mathcal{Y}$, $S_\Omega(\thb,\bmy)$ is differentiable with respect to $\thb$, and it satisfies  
    $
    \nabla S_\Omega(\thb;\bmy)=\yho(\thb)-\bmy.
    $
    Furthermore, it holds that  
    $
    S_\Omega(\thb;\bmy)\geq\frac{\lambda}{2}\|\bmy-\yho(\thb)\|^2.
    $ 
\end{proposition}


In what follows, let $S_t(\W)\coloneqq S_\Omega(\W\xt;\yt)$ for simplicity.
Importantly, from the properties of the Fenchel--Young loss, there exists some $b>0$ such that for any $\W\in\ww$,  
\begin{equation}\label{eq:St_smooth}
    \nrm{\nabla S_t (\W)}_{\mathrm{F}}^2\leq b S_t(\W).
\end{equation}
Indeed, from \cref{prop:fenchel} and \cref{asp:online_structured_prediction}, we have 
$
\nrm{\nabla S_t(\W_t)}_\F^2
=
\nrm{\yho(\tht)-\yt}_2^2 \nrm{\xt}^2
\leq
\dix^2\kappa^2\nrm{\yho(\tht)-\yt}^2
\leq
\frac{2\dix^2\kappa^2}{\lambda}\sw
$, 
where we used $\nabla S_t(\wt) =  \prn{\hat{\bm{y}}_\Omega(\bm{\theta}_t) - \yt}\xt^\top$ and $\nrm{\cdot}_2 \leq \kappa \nrm{\cdot}$. 
Thus, \eqref{eq:St_smooth} holds with  $b=\frac{2\dix^2\kappa^2}{\lambda}$.
Below, let $L_t(\bmy)\coloneqq L(\bmy;\yt)$,
$\G_t\coloneqq\nabla S_t(\wt) =  \prn{\hat{\bm{y}}_\Omega(\bm{\theta}_t) - \yt} \bm{x}_t^\top$, and $\L\coloneqq\max_{\W\in\ww}\nrm{\nabla S_t(\W)}_\F$.


\subsection{Examples of Structured Prediction}\label{subsec:pre_examples} 
We present several special cases of structured prediction  
along with specific parameter values introduced so far; see \citet[Section 2.3]{pmlr-v247-sakaue24a} for further details.
\paragraph{Multiclass classification}
Let $\yy=\set{\mathbf{e}_1,\dots,\mathbf{e}_d}$ and $\|\cdot\| = \|\cdot\|_1$.
When using the 0-1 loss, $L(\bmy^{\prime};\bmy)=\ind\brk{\bmy^{\prime}\neq\bmy}$, the parameters in \cref{asp:online_structured_prediction} are $\nu=2$ and $\gamma=\frac{1}{2}$.
The logistic surrogate loss is a Fenchel--Young loss $S_\Omega$ generated by the entropy regularization function $\Omega=\mathsf{H}^s+I_{\Delta_d}$ (up to a constant factor), where $\mathsf{H}^s(\bmy)\coloneqq-\sum_{i=1}^d y_i\log y_i$ and $\Delta_d$ is the $(d-1)$-dimensional probability simplex.
Since $\Omega$ is a $1$-strongly convex function with respect to $\|\cdot\|_1$, we have $\lambda=1$.

\paragraph{Multilabel classification}
Let $\yy=\set{0,1}^d$ and $\|\cdot\| = \|\cdot\|_2$.
When using the Hamming loss as the target loss function $L(\bmy^{\prime};\bmy)=\frac{1}{d}\sum_{i=1}^{d}\ind\brk{y_{i}^{\prime}\neq y_i}$, \cref{asp:online_structured_prediction} is satisfied with $\nu=1$ and $\gamma=\frac{1}{d}$.
The SparseMAP surrogate loss $S_\Omega(\thb,\bmy)=\frac{1}{2}\nrm{\bmy-\thb}_2^2-\frac{1}{2}\nrm{\yho(\thb)-\thb}_2^2$ is a Fenchel--Young loss generated by $\Omega=\frac{1}{2}\nrm{\cdot}^2+I_{\conv(\yy)}$.
Since $\Omega$ is 1-strongly convex with respect to $\|\cdot\|_2$, we have $\lambda=1$.

\paragraph{Ranking}
We consider predicting the ranking of $m$ items. 
Let $\nrm{\cdot} = \nrm{\cdot}_1$, $d=m^2$, and $\yy\subset\set{0,1}^d$ be the set of all vectors representing $m \times m$ permutation matrices.  
We use the target loss function that counts mismatches, $L(\bmy^{\prime};\bmy)=\frac{1}{m}\sum_{i=1}^{m}\ind\brk{y_{i,j_i}^{\prime}\neq y_{i,j_i}}$, where $j_i\in[m]$ is a unique index with $y_{ij_i}=1$ for each $i\in[m]$.  
In this case, the parameters in \cref{asp:online_structured_prediction} satisfy $\nu=4$ and $\gamma=\frac{1}{2m}$.  
We use a surrogate loss given by $S_\Omega(\thb;\bmy)=\inpr{\thb,\yho(\thb)-\bmy}+\frac{1}{\zeta}\mathsf{H}^s(\yho(\thb))$, where $\Omega=-\frac{1}{\zeta}\mathsf{H}^s+I_{\conv(\yy)}$ and $\zeta$ is a parameter controlling the regularization strength. 
The first term in $S_\Omega$ measures the affinity between $\thb$ and $\bmy$, while the second term evaluates the uncertainty of $\yho(\thb)$.  
Since $\Omega$ is $\frac{1}{m\zeta}$-strongly convex, we have $\lambda=\frac{1}{m\zeta}$.


\subsection{Structured Encoding Loss Function (SELF)}\label{subsec:self}
Here we introduce a common class of target loss functions, called the Structured Encoding Loss Function (SELF).
A target loss function is SELF if it can be expressed as  
\begin{equation}\label{eq:self}
    L(\yt; \yht)=\inpr{\yht,\V\yt+\bm{b}}+c(\yt),
\end{equation}
where $\bm{b} \in \R^d$ is a constant vector, $\V\in\R^{d\times d}$ is a constant matrix, and $c:\yy\to\R$ is a function.  
The following loss examples, taken from \citet[Appendix A]{NEURIPS2019_Blondel}, belong to the SELF class: 
\begin{itemize}%[topsep=2pt,itemsep=0pt, partopsep=0pt, leftmargin=18pt]
\item Multiclass classification: the 0-1 loss is a SELF with $\V=\bm{1}\bm{1}^\top-\I$, $\bb=\bm{0}$, and $c(\bmy)=0$.  

\item Multilabel classification: the Hamming loss, 
$L(\bmy^{\prime};\bmy)=\frac{1}{d}\sum_{i=1}^{d}\ind\brk{y^{\prime}_{t,i}\neq y_{i}}$, is a SELF with $\V=-\frac{2}{d}\I$, $\bb=\frac{\bm{1}}{d}$, and $c(\bmy)=\frac{1}{d}\inpr{\bmy,\bm{1}}$, where the last factor is constant if the number of correct labels is fixed.

\item Ranking: the Hamming loss  
$L(\bmy^{\prime};\bmy)=\frac{1}{m}\sum_{i=1}^{m}\ind\brk{y_{i,j_i}^{\prime}\neq y_{i,j_{i}}}$, where $j_i\in[m]$ is a unique index with $y_{i,j_i}=1$ for each $i\in[m]$, is a SELF with  
$\V=-\frac{1}{m}\I$, $\bb=\bm{0}$, and $c(\bmy)=1$.
\end{itemize}
Following \citet{pmlr-v247-sakaue24a}, this study assumes that the target loss function $L$ is a SELF.

\subsection{Randomized Decoding}\label{subsec:randomized_decoding}
\begin{algorithm}[t]
    \caption{Randomized decoding $\phi_\Omega$}
    \label{ALG: randomized decoding}
    \begin{algorithmic}[1]
        \Require {$\btheta\in\mathbb{R}^d$}
            \State {$\yho(\btheta)\leftarrow\argmax\{\langle\btheta,\bm{y}\rangle-\Psi(\bm{y})\::\:\bm{y}\in\conv(\mathcal{Y})\}$}
            \State {$\bm{y}^\ast\leftarrow\argmin\{\|\bm{y}-\yho(\btheta)\|\::\:\bm{y}\in\mathcal{Y}\}$}
            \State {$\Delta^\ast\leftarrow\|\bm{y}^\ast-\yho(\btheta)\|,\:p\leftarrow\min\{1,2\Delta^\ast/\nu\}$}
            \State {Sample $Z \sim \mathrm{Ber}(p)$}
            \LineIf{$Z=0$}{$\yh\leftarrow\bm{y}^\ast$}
            \LineIf{$Z=1$}{$\yh\leftarrow\bm{\tilde{y}}$ where $\bm{\tilde{y}}$ is randomly drawn from $\yy$ so that $\E\brk*{\bm{\tilde{y}}|Z=1}=\yho(\btheta)$}
            \Ensure{$\phi_\Omega(\thb)=\yh$}
    \end{algorithmic}
\end{algorithm}


The procedure of converting the estimated score $\thb$ into a structured output $\bm{\hat{y}}$ is called decoding.  
For this, we employ randomized decoding \citep{pmlr-v247-sakaue24a},  
which plays an essential role particularly in deriving an upper bound independent of the output set size $K = \abs{\mathcal{Y}}$ in \cref{subsec:Bandit_Structured_Prediction_with_SELF}.
The randomized decoding (\cref{ALG: randomized decoding}) selects either the closest $\bm{y}^* \in \yy$ to $\hat{\bm{y}}_\Omega(\thb) \in \conv(\yy)$ or a random $\widetilde{\bm{y}} \in \yy$ satisfying $\E[\widetilde{\bm{y}} \mid Z=1] = \hat{\bm{y}}_\Omega(\thb)$, where $Z$ follows a Bernoulli distribution with a parameter $p$.  
Intuitively, the parameter $p$ is chosen so that if $\hat{\bm{y}}_\Omega(\thb)$ is close to $\bm{y}^*$, the decoding is more likely to return $\bm{y}^*$; otherwise, it is more likely to return $\widetilde{\bm{y}}$, reflecting uncertainty.  
An important property satisfied by the randomized decoding is the following lemma, which we use in the subsequent analysis:
\begin{lemma}[{\citealt[Lemma 4]{pmlr-v247-sakaue24a}}]
    \label{lem:expected_target_bound}
  For any $(\thb, \bmy) \in \mathbb{R}^d\times\yy$, the randomized decoding $\phi_\Omega$ satisfies
  \[
    \E[L(\phi_\Omega(\thb);\bmy)] \leq \frac{4\gamma}{\lambda\nu} S_\Omega(\thb;\bmy),
  \]
  where the expectation is taken with respect to the internal randomness of the randomized decoding.
\end{lemma}









\section{Bandit Feedback}
\label{sec:bandit}
In this section, we present two online structured prediction algorithms in the bandit feedback setup and analyze their regret.  
Our results here are mostly special cases of the regret bounds obtained when handling bandit and delayed feedback (\cref{sec:bandit_and_delayed}). 
Nevertheless, by focusing on the case without delay, we provide a clearer exposition of the core ideas.

\subsection{Randomized Decoding with Uniform Exploration}
\begin{algorithm}[t]
    \caption{Randomized decoding with uniform exploration (RDUE) $\psi_\Omega$}
    \label{ALG:randomized decoding with uniform exploration}
    \begin{algorithmic}[1]
        \Require{$\thb\in\R^n$, $q \in [0,1]$}
        \State {Sample $X \sim \mathrm{Ber}(q)$ 
        }
        \LineIf{$X=0$}{$\bm{\hat{y}}\leftarrow\phi_\Omega(\thb)$}
        \LineIf{$X=1$}{Sample $\bm{y}^\ast \sim \mathrm{Unif}(\mathcal{Y})$ and $\yh\leftarrow\bm{y}^\ast$}
        \Ensure{$\psi_\Omega(\thb)=\yh$}
    \end{algorithmic}
\end{algorithm}
Here, we discuss the properties of the decoding function, \emph{Randomized Decoding with Uniform Exploration (RDUE)}, which will be used in both algorithms to convert scores into outputs.  
As discussed in \cref{subsec:randomized_decoding}, in online structured prediction with full information feedback, the randomized decoding (\cref{ALG: randomized decoding}) was introduced as a decoding function \citep{pmlr-v247-sakaue24a}.  
However, naively applying the randomized decoding does not lead to a satisfactory regret bound under bandit feedback.  
We extend the framework of the randomized decoding to handle bandit feedback effectively.

RDUE (\cref{ALG:randomized decoding with uniform exploration}) is a procedure that, with probability $q \in [0,1]$, selects $\hat{\bmy}$ uniformly at random from $\yy$,  
and with probability $1-q$, selects the output of the randomized decoding.  
Using RDUE, we define $p_t(\bm{y})$ as the probability that $\hat{\bm{y}}_t$ coincides with $\bm{y}$ at round $t$.  
Note that for any $\bm{y} \in \mathcal{Y}$, it holds that  
$
p_t(\bm{y})\geq\frac{q}{\K}.
$
Furthermore, similar to the property of the randomized decoding in \cref{lem:expected_target_bound}, RDUE satisfies the following property:
\begin{lemma}
    \label{lem:bound of randomized decoding with uniform exploration}
    For any $(\thb,\bm{y})\in\R^d\times\mathcal{Y}$, RDUE $\psi_\Omega$ satisfies
    \[
        \E\brk*{L(\psi_\Omega(\thb);\bm{y})}\leq\frac{4\gamma}{\lambda\nu}(1-q)S_\Omega(\thb;\bmy)+q\frac{\K-1}{\K  },
    \]
  where the expectation is taken with respect to the internal randomness of RDUE. 
\end{lemma}
\begin{proof}
Since the output of the randomized decoding is chosen with probability $1-q$,  
and a uniformly random output is chosen with probability $q$, we have  
$
\E\brk*{L(\psi_\Omega(\thb);\bmy)}=(1-q)\E\brk*{L(\phi_\Omega(\thb);\bmy)}+q\frac{\K-1}{\K},
$
where we used $\L(\cdot;\cdot)\leq 1$ and $\phi_\Omega$ is the randomized decoding.  
Hence, combining this with \cref{lem:expected_target_bound}, we obtain the desired bound.
\end{proof}

Additionally, this section makes the following assumption:
\begin{assumption}
    \label{asp:bandit_a}
    There exists $a\in\prn{0,1}$ such that  
    \[
    \expect{L_t(\yht)}\leq(1-a)\sw+q.
    \]  
    Here, $\expect{\cdot}$ denotes the conditional expectation given the random variables $\hat{\bmy}_1,\dots,\hat{\bmy}_{t-1}$.  
\end{assumption}
This assumption can be satisfied by using RDUE and letting   
$a \leq 1-\frac{4\gamma}{\lambda\nu}(1-q)$  
if $\lambda>\frac{4\gamma}{\nu}(1-q)$, according to \cref{lem:bound of randomized decoding with uniform exploration}.  
In what follows, let $a = 1-\frac{4\gamma}{\lambda\nu}$.
Note that $\lambda \geq \frac{4\gamma}{\nu}$ holds in the cases of multiclass classification, multilabel classification, and ranking, as discussed in \cref{subsec:pre_examples}.


\subsection{Online Gradient Descent}\label{subsec:ogd}
The algorithm in this section uses the adaptive Online Gradient Descent (OGD, \citealt{streeter2010regretonlineconditioning}) as $\alg$.
OGD updates $\wt$ to $\W_{t+1}$ using a gradient $\bm{G}_t$ and learning rate $\eta_t$ by
$
    \W_{t+1} \leftarrow \Pi_{\ww} \prn*{\wt - \eta_{t} \bm{G}_t},
$
where $\Pi_{\ww}(\bm Z) = \argmin_{\bm{X} \in \ww} \nrm{\bm X - \bm Z}_{\F}$.
OGD with appropriately chosen learning rate $\eta_t$ achieves the following bound:
\begin{lemma}[{\citealt[Theorem 4.14]{orabona2023modernintroductiononlinelearning}}]
    \label{lem:ogd}
    Suppose that we set the learning rate to $\eta_t=\frac{B}{\sqrt{2 \sum_{i=1}^t\nrm{\bm{G}_i}_{\mathrm{F}}^2}}$ and do not update on rounds when $\bm{G}_t$ is the all-zero matrix.
    Then, for any $\U\in\ww$, OGD achieves 
    $
        \sumt{\inpr{\bm{G}_t, \W_t - \U}}
        \leq \sqrt{2}B\sqrt{\sumt{\nrm{\bm{G}_t}_{\mathrm{F}}^2}}.
    $
\end{lemma}

\subsection{$O(\sqrt{K T})$ Regret Algorithm}
\label{subsec:Bandit_Structured_Prediction_with_General_Losses}
Here, we present an algorithm that achieves a regret upper bound of $O(\sqrt{\K T})$.
\paragraph{Algorithm based on inverse-weighted estimator}
In the bandit setting, the true output $\yt$ is not observed,  
and thus it is necessary to estimate the gradient required for updating $\wt$.  
To deal with this, we use the following inverse-weighted gradient estimator:
\begin{equation}\label{eq:inverse_weighted_est}
    \gtil\coloneqq\frac{\ind[\yht=\yt]}{p_t(\yt)}\G_t,
\end{equation}
where we recall that $\G_t=\nabla S_t(\wt) = \prn{\hat{\bm{y}}_\Omega(\bm{\theta}_t) - \yt} \bm{x}_t^\top$.
Note that $\gtil$ is unbiased, i.e., $\E\brk[\big]{\gtil}=\G_t$.
We use RDUE with $q=B\sqrt{\K/T}$ as the decoding function (assuming $T \geq B^2 \K$ for simplicity).  
For $\alg$, we employ the adaptive OGD in \cref{subsec:ogd} with the learning rate of
$
\eta_t=\frac{B}{\sqrt{2 \sum_{i=1}^t\nrm{\tilde{\G}_i}_{\mathrm{F}}^2}}.
$

\begin{remark}\label{rem:zero-loss}
This study defines the bandit feedback as the value of the target loss function $L_t(\yht)$.
Note, however, that the above algorithm operates using only the weaker feedback of $\ind\brk*{\yht\neq\yt}$. 
\end{remark}

\paragraph{Regret bounds and analysis}
The above algorithm achieves the following regret bound:
\begin{theorem}\label{thm:bandit_regret_expectation_abstract}
    The regret of the above algorithm is bounded by
    $
        \E\brk{\reg}\leq \prn*{\frac{b}{2a}+1}B\sqrt{\K T}.
    $
\end{theorem}
The $O(\sqrt{\K T})$ bound has an optimal dependency on $T$  
and matches the $\sqrt{T}$ lower bound in the special case of online multiclass classification with bandit feedback \citep[Corollary 1]{NEURIPS2021_Hoeven}.  
Furthermore, our bound improves the existing $O(\sqrt{KT})$ bound by \citet{NEURIPS2020_Hoeven} by a factor of $\sqrt{\K}$. 
Note that, due to differences in the target loss function, our result is not directly comparable to the $\sqrt{\K T}$ bound in \citet{NEURIPS2021_Hoeven}. A more detailed discussion can be found in \cref{app:Discussio_on_the_Difference_in_Surrogate_Losses}.
\begin{proof}
From the convexity of $S_t$ and the unbiasedness of $\gtil$, we have 
$
    \E\brk*{\sumt{\prn{\sw-\su}}}
    \leq
    \E\brk*{\sumt{\inpr{\G_t,\wt-\U}}}
    =
    \E\brk*{\sumt{\inpr{\gtil,\wt-\U}}}.
 $
From \cref{lem:ogd}, this is further upper bounded as
$
    \E\brk*{\sumt{\inpr{\gtil,\wt-\U}}}
    \leq
    \sqrt{2}B\sqrt{\E\brk*{\sumt{{\nrm{\gtil}_{\mathrm{F}}^2}}}}
    \leq
    B\sqrt{\frac{2b\K }{q}\E\brk*{\sumt{\sw}}},
$
where in the first inequality we used Jensen's inequality and 
in the last inequality we used
$
\expect{\|\gtil\|_{\mathrm{F}}^2}
=
\frac{\|\G_t\|_{\mathrm{F}}^2}{p_t(\yt)}
\leq
\frac{\K }{q}\|\G_t\|_{\mathrm{F}}^2
\leq
\frac{b\K }{q}\sw,
$
which follows from $p_t(\bm{y}) \geq K /q$ and \eqref{eq:St_smooth}.
Therefore, from \cref{asp:bandit_a}, 
we have
$
    \E\brk{\reg}
    \leq
    \E\brk*{\sumt{\prn*{(1-a)\sw-\su}}}+qT
    \leq 
    B\sqrt{\frac{2b\K }{q} \E\brk*{\sumt{\sw}}}-a \E\brk*{\sumt{\sw}}+qT
    \leq
    \frac{bB^2\K }{2aq}+qT
    ,
$
where the last inequality follows from $c_1\sqrt{x}-c_2x\leq{c_1^2}/\prn{4c_2}$ for $x \geq 0$, $c_1\geq 0$, and $c_2>0$.
Finally, substituting $q=B\sqrt{\K/T}$ into the last inequality yields the desired bound.
\end{proof}

We can also prove the following high-probability bound:
\begin{theorem}\label{thm:bandit_high_prob}
Let $\delta \in (0,1)$.
Then with probability at least $1 - \delta$, the same algorithm as in \cref{thm:bandit_regret_expectation_abstract}, but with a different choice of $q$, achieves the regret bound of 
$\mathcal{R}_T = O\prn[\Big]{\sqrt{KT \log (1/\delta)} + \log(1/\delta)}$,
where we omit the dependencies on parameters other than $K$, $T$, and $\delta$.
\end{theorem}
A more precise statement and proof of this theorem are provided in \cref{app:proof_bandit_high_prob}.
To prove this theorem, we follow the analysis of \cref{thm:bandit_regret_expectation_abstract} and use Bernstein's inequality.  
To address the challenges posed by the randomness introduced by bandit feedback,  
we adopt an approach similar to that used by \citet{NEURIPS2021_Hoeven}, and
arguably, we have successfully simplified their analysis.


\subsection{$O(T^{2/3})$ Regret Algorithm}
\label{subsec:Bandit_Structured_Prediction_with_SELF}
While the $O(\sqrt{KT})$ regret bound given above is desirable in terms of the dependence on $T$, the dependence on $K = \abs{\yy}$ is undesirable for general structured prediction.  
In fact, we have $\K=\binom{d}{m}$ in multilabel classification with $m$ correct labels and $\K=m!$ in ranking with $m$ items.  
To address this issue, we present an algorithm that significantly improves the dependence on $\K$ when the target loss function belongs to a special class of SELF satisfying the following assumptions:\looseness=-1
\begin{assumption}\label{asp:self}
(i) $\V$ is invertible, and $\bm{b}$ and $c(\cdot)$ are known and non-negative.
(ii) Let $\bm{Q} = \E_{\bm{y} \sim \mu} \brk{ \bm{y} \bm{y}^\top }$, where $\mu$ is the uniform distribution over $\yy$. At least one of the following two conditions holds: 
(ii-a) $\bm{Q}$ is invertible, or 
(ii-b) for any $\bm{y} \in \yy$, $\V \bm{y}$ lies in the linear subspace spanned by vectors in $\yy$. 
(iii) For some $\omega > 0$, it holds that\looseness=-1
\begin{equation}\label{eq:def_omega}
    \tr \prn*{ \V^{-1} \bm{Q}^+ \prn{\V^{-1}}^\top } \leq \omega.
    \nonumber
\end{equation} 
\end{assumption}
The first condition is true in the examples in \cref{subsec:self}, assuming that the number of correct labels is fixed in multilabel classification.
The second one is satisfied if $\yy$ consists of $d$ linearly independent vectors or $\V$ is proportional to the identity matrix; either is true in those examples. 
It is also not difficult to derive bounds on $\omega$ in those examples.\looseness=-1

\paragraph{Algorithm based on pseudo-inverse matrix estimator}
As in the case of \cref{subsec:Bandit_Structured_Prediction_with_General_Losses}, we begin by providing a method to estimate the gradient.  
Define $\pt \coloneqq\E_{\bmy\sim p_t}[\bmy\bmy^\top]$.
Then, we define the estimator $\ytilde$ of $\yt$ by
\[
    \ytilde\coloneqq\inverse{\V}\bm{P}_t^+\yht\inpr{\yht,\V\yt},
\]
where $\bm{P}_t^+$ is the Moore--Penrose pseudo-inverse matrix of $\bm{P}_t$.
It is important to note that, given that $\bm{b}$ and $c(\cdot)$ are known,  
$
\inpr{\yht,\V\yt}=L_t(\yht)-\inpr{\yht,\bm{b}}-c(\yt)
$
can be computed.  
Note that $\ytilde$ is unbiased, i.e.,
$
\expect{\ytilde}=\yt
$
from the first requirement of \cref{asp:self}.

Using this $\ytilde$, we define the gradient estimator $\gtil$ by
\begin{equation}
    \label{eq:gtil_self}
    \gtil\coloneqq\prn*{\yho(\tht)-\ytilde}\xt^\top,
\end{equation}
whose expectation is 
$
    \E\brk[\big]{\gtil}=\G_t.
$
Our estimator is based on the estimators used in adversarial linear bandits and adversarial combinatorial full-bandits \citep{dani07price,abernethy08competing,comband}.  

We use RDUE with $q=\prn*{\frac{4 \omega B^2\dix ^2}{ T }}^{1/3}$ as the decoding function  
(assuming $T \geq 4 \omega B^2\dix ^2$ for simplicity).  
For updating $\wt$, we employ the adaptive OGD in \cref{subsec:ogd} as $\alg$ with the learning rate of 
$
\eta_t=\frac{B}{\sqrt{2 \sum_{i=1}^t\nrm{\tilde{\G}_i}_{\mathrm{F}}^2}}.
$


\paragraph{Regret bounds}
The above algorithm achieves the following regret bound,  
which does not directly depend on $\K$:
\begin{theorem}
    \label{thm:bandit_regret_pseudo_estimator}
    The above algorithm achieves
    $
    \E\brk{\reg}
    \leq
    \frac{bB^2}{a}
    +
    O\prn[\big]{ \omega^{1/3} \prn*{ B \dix T}^{2/3} }.
    $
\end{theorem}
The proof can be found in \cref{app:sub_bandit_regret_pseudo_estimator}.  
By using the estimator based on the pseudo-inverse matrix, 
we can upper bound the second moment of the gradient estimator $\gtil$ without $\K$, which allows us to establish the improved regret bound that does not explicitly depend on $\K$.  

The regret bound in \cref{thm:bandit_regret_pseudo_estimator} yields the different bounds on each problem setup as follows:
\begin{corollary}\label{cor:thm_self}
The above algorithm achieves
$
    \E\brk{\reg}\leq\frac{bB^2}{a}+ O\prn*{ \prn{B \dix  d T}^{2/3}}
$
in multiclass classification with the 0-1 loss,
$
    \E\brk{\reg}\leq\frac{bB^2}{a}
    +
    O\prn*{ \prn{d^5 /m(d-m)}^{1/3} \prn{B \dix T}^{2/3}}
$
in multilabel classification with $m$ correct labels and the Hamming loss,
and 
$
    \E\brk{\reg}\leq\frac{bB^2}{a}+ O \prn*{ m^{5/3}\prn{B\dix T}^{2/3}}
$
in ranking with the number of items $m$ and the Hamming loss.
\end{corollary}
The proof of \cref{cor:thm_self} is deferred to \cref{app:SELF_upper_discussion_deferred}.
The bound for multilabel classification with $m$ correct labels significantly improved the $O(\sqrt{\K T})$ bound in \cref{subsec:Bandit_Structured_Prediction_with_General_Losses},  
which has a dependency of $\sqrt{\binom{d}{m}}$,
and
the bound for ranking significantly improved the $O(\sqrt{\K T})$ bound in \cref{subsec:Bandit_Structured_Prediction_with_General_Losses}, which has a dependency of $\sqrt{m!}$.\looseness=-1

\section{Delayed Full-Information Feedback}
\label{sec:delay}
This section discusses online structured prediction with delayed full-information feedback and provides an algorithm that achieves a surrogate regret bound of $O(D^{2/3} T^{1/3})$, a better bound than $O(\sqrt{D T})$ that can be achieved with a standard OCO algorithm under delayed feedback \citep{joulani13online}. 
Below, we make the following assumption.\looseness=-1
\begin{assumption}
\label{asp:delayed_a}
There exists a constant $a\in\prn{0,1}$ which satisfies
\[
    \expect{L_t(\yht)}\leq(1-a)\sw.
\]
\end{assumption}
From \Cref{lem:expected_target_bound}, if $\lambda>\frac{4\gamma}{\nu}$, this condition is satisfied with $a=1-\frac{4\gamma}{\lambda\nu}$ by using the randomized decoding. 
We suppose that such a decoding function is used in this section.

\paragraph{Algorithm}
For updating $\wt$, we employ the Optimistic Delayed Adaptive FTRL (ODAFTRL) algorithm proposed by \citet{pmlr-v139-flaspohler21a}.  
In ODAFTRL, given a gradient~$\bm{G}_t$ at round $t$, $\W_t$ is updated as
\begin{equation}
    \W_{t+1}
    =
    \argmin_{\W\in \ww} \set*{ \sum_{i=1}^{t-D} \inpr*{\bm{G}_i,\W}+\frac{\lambda_t \nrm{\W}_{\F}^2}{2} },
    \nonumber
\end{equation}
where $\lambda_t\geq0$ is the regularization parameter.  
Due to space constraints, the details of the algorithm are provided in \cref{app:sub_odaftrl}.
By updating $\lambda_t$ using an AdaHedge-type algorithm called AdaHedgeD,  
ODAFTRL achieves the following regret upper bound:
\begin{lemma}[{Informal version of \citealt[Theorem 12]{pmlr-v139-flaspohler21a}}]\label{lem:ODAFTRL_bound}
Consider the setting with delayed full-information feedback.
Then, for any $\U\in\ww$, ODAFTRL with the AdaHedgeD update of $\lambda_t$ achieves
$
    \sumt{(\sw\!-\!\su)}
    \leq
    \sumt{\inpr{\G_t, \W_t - \U}}
    =
    O\prn[\Big]{\sqrt{\sumt{(\nrm{\G_t}_\F^2\!+\!D\nrm{\G_t}_\F)}}}.
$
\end{lemma}

\paragraph{Regret bounds and analysis}
The algorithm described above achieves the following bound:
\begin{theorem}
    \label{thm:delayed_regret_expectation_abstract}
    The above algorithm achieves
    $
        \E\brk{\reg}=O(D^{2/3}T^{1/3}).
    $
\end{theorem}
Here, we provide a proof sketch; the complete proof can be found in \cref{app:sub_delayed_regret_expectation_abstract}.
\begin{proof}[Proof sketch]
    \Cref{lem:ODAFTRL_bound} with $\nrm{\G_t}_\F^2\leq b\sw$ in \eqref{eq:St_smooth} and Cauchy--Schwarz yields
    $\sumt{(\sw-\su)}=O(\sqrt{S_{1:T}}+\prn{D^2TS_{1:T}}^{1/4})$, where $S_{1:T}=\sumt{\sw}$.
    Hence, from \cref{asp:delayed_a}, we have
    $
        \E\brk{\reg}\leq \sumt{(\sw-\su)}-a\sumt{\sw}
        =O(\sqrt{S_{1:T}}+\prn{D^2TS_{1:T}}^{1/4})-aS_{1:T}
        =O\prn{D^{2/3}T^{1/3}},
    $
    where we used $c_1\sqrt{x}-c_2x\leq{c_1^2}/\prn{4c_2}$ and $c_1x-c_2x^4\leq\prn*{{c_1^4}/\prn{4c_2}}^{1/3}$ for $x\geq0$, $c_1\geq 0$, and $c_2>0$.
\end{proof}
We can also prove the following high-probability bound; see \cref{app:sub_delayed_regret_probability_abstract} for the proof:
\begin{theorem}
    \label{thm:delayed_regret_probability_abstract}
    For any $\delta \in (0,1)$,
    with probability at least $1 - \delta$, the above algorithm achieves 
    $
        \reg = O\prn{\log({1}/{\delta}) + D^{2/3}T^{1/3}}.
    $
\end{theorem}


\section{Delayed Bandit Feedback}\label{sec:bandit_and_delayed}
Given the results so far, it is natural to explore online structured prediction with delayed bandit feedback.
We construct algorithms for this setup by combining the theoretical developments from \cref{sec:bandit} and \cref{sec:delay}. 
This section assumes \cref{asp:bandit_a}, as in \cref{sec:bandit}.
\subsection{$O(\sqrt{D K T})$ Regret Algorithm}\label{subsec:bandit_delay_general}
We use RDUE with $q=B\sqrt{DK/T}$ as the decoding function (assuming $T\geq DB^2K$ for simplicity), the gradient estimator $\gtil$ in \eqref{eq:inverse_weighted_est},
and ODAFTRL with the AdaHedgeD update as $\alg$. 
This algorithm attains the following bound:
\begin{theorem}
    \label{thm:delay_bandit_bound_general_abstract}
    The above algorithm achieves
    $
        \E\brk{\reg} = O\prn{\sqrt{DKT}}.
    $
\end{theorem}
The proof can be found in \cref{app:bandit_delayed_general}.  
Due to the introduction of the delay, the regret bound worsens by a factor of $\sqrt{D}$ compared to the non-delayed case,
which is natural when considering analyses of adversarial cases in delayed feedback \citep{joulani13online,zimmert20optimal,manwani2022delaytronefficientlearningmulticlass}.



\subsection{$O(D^{1/3} T^{2/3})$ Regret Algorithm}\label{subsec:bandit_delay_self}
Here, we make the same assumptions on the target loss function as in \cref{subsec:Bandit_Structured_Prediction_with_SELF}.
We provide an algorithm that improves the dependence on $\K$ from \cref{subsec:bandit_delay_general}.
We use RDUE with $q=\prn*{\frac{\omega B^2 \dix^2 D}{T}}^{1/3}$ as the decoding function (assuming $T \geq \omega B^2 \dix^2 D$ for simplicity), the gradient estimator $\gtil$ in \eqref{eq:gtil_self}, and ODAFTRL with the AdaHedgeD update as $\alg$.
This algorithm achieves the following bound:
\begin{theorem}
    \label{thm:delay_bandit_bound_self_abstract}
    The above algorithm achieves
    $
        \E\brk{\reg} = O\prn{
            D^{1/3} T^{2/3}
        }.
    $
\end{theorem}
The proof can be found in \cref{app:bandit_delayed_self}.  
Due to the introduction of the delay, the regret bound worsens by a factor of $D^{1/3}$  
compared to the non-delayed bandit feedback case.


\section{Experiments}
\label{sec: experiment}
\begin{figure}[t]
    \centering
    \includegraphics[keepaspectratio, scale=0.3]{experiment_box_MNISTtheo_B10_rep20_for_arxiv.eps}
    \vspace{-10pt}
     \caption{A box plot of error rate of the MNIST experiment for multiclass classification with bandit feedback.
     }
    \label{fig:experiment mnist}
\end{figure}

This section presents numerical experiment results for online multiclass classification  
under bandit feedback.  
We compare three algorithms:  
Gaptron~\Citep{NEURIPS2020_Hoeven} with logistic loss and hinge loss as surrogate losses,  
Gappletron~\Citep{NEURIPS2021_Hoeven} with logistic loss as the surrogate loss,  
and our proposed algorithm from \cref{subsec:bandit_delay_general}.  
The parameters for each algorithm are set based on their theoretical values.  
We use the MNIST dataset~\citep{lecun2010mnist}, a dataset of digit images.  
The diameter $B$ of $\ww$ is fixed at $10$.  
We repeated the experiment for 20 times, and the boxplot of the obtained misclassification rates is summarized in \cref{fig:experiment mnist}.  
From \cref{fig:experiment mnist}, we observe that our method achieves the lowest misclassification rate.  
Despite not being specialized for multiclass classification,  
our approach outperforms existing algorithms designed for multiclass classification  
on real data with $K = 10$.  
Further experiments can be found in \cref{app: experiment},  
and related discussions are provided in \cref{app:Discussio_on_the_Difference_in_Surrogate_Losses}.\looseness=-1













\vspace{-5pt}
\section{Conclusion}
This paper presents a novel unified framework, UniBrain, the first end-to-end model to jointly perform a diverse set of brain imaging analysis tasks, including extraction, registration, segmentation, parcellation, network generation and classification. UniBrain integrates heterogeneous information into a single system, enabling efficient knowledge transfer across different modules, and avoiding the need for extensive task-specific labels. Experimental results show that UniBrain outperforms state-of-the-art methods in all tasks while also demonstrating robustness and time efficiency.

\section*{Acknowledgments}
TT is supported by JST ACT-X Grant Number JPMJAX210E and JSPS KAKENHI Grant Number JP24K23852,
SS is supported by supported by JST ERATO Grant Number JPMJER1903,
and KY is supported by JSPS KAKENHI Grant Number JP24H00703.

\bibliography{bib_list}
% \bibliographystyle{plainnat}
\bibliographystyle{icml2025}

\newpage
\appendix
%

\section{Appendix Notation and Definitions}
We often use the shorthand $(a)_+ \defeq \max(a,0)$ as well as the shorthand $\k(\xset,\xset)$ to represent the matrix $(\k(\x_i,\x_j))_{i,j=1}^n$. 
In addition, for each kernel $\k$, we let $\rkhs$ and $\knorm{\cdot}$ represent the associated reproducing kernel Hilbert space (RKHS) and RKHS norm, so that $\ball_{\kernel}=\{ f\in\rkhs : \knorm{f} \leq 1\}$ and define
\begin{talign}
(\Pin - \Qout)\k \defeq \frac{1}{\nin}\sum_{x\in\xin} \k(\x,\cdot) - \frac{1}{\nout}\sum_{x\in\xout} \k(\x,\cdot).
\end{talign}
%
We also relate our definition of a sub-Gaussian thinning algorithm (\cref{def:alg-subg}) to several useful notions of sub-Gaussianity.
%
\begin{definition}[\tbf{Sub-Gaussian vector}]\label{def:vector-subg}
We say that a random vector $\diff \in \R^n$ is \emph{$(\K,\subg)$-sub-Gaussian on an event $\event$} if $\K$ is SPSD and $\subg>0$ satisfies 
\begin{talign}\label{eq:vector-subg}
    \Esubarg{\event}{\exp(\bu^\top \K \diff)} \leq \exp(\frac{\subg^2}{2} \cdot \bu^\top \K \bu)
    \qtext{for all}
    \bu \in \reals^n.
\end{talign}
If, in addition, the event has probability $1$, we say that $\w$ is \emph{$(\K,\subg)$-sub-Gaussian}.
\end{definition}
%
Notably, a thinning algorithm is $(\K,\subg,\delta)$-sub-Gaussian if and only if its associated vector $\pin-\qout$ is $(\K,\subg)$-sub-Gaussian on an event $\event$ of probability at least $1-\delta/2$.

%
\begin{definition}[\tbf{Sub-Gaussian function}]\label{def:function-subg}
For a kernel $\kernel$, %
we say that a random function $\fsubg\in \rkhs$ is \emph{$(\kernel,\subg)$-sub-Gaussian on an event $\event$} if $\subg > 0$ satisfies
\begin{talign}\label{eq:function-subg}
    \Esubarg{\event}{\exp(\inner{f}{\fsubg}_{\kernel})} \leq \exp(\frac{\subg^2}{2}\cdot \knorm{f}^2)
    \qtext{for all}
    f \in\rkhs.
\end{talign}
If, in addition, the event has probability $1$, we say that $\fsubg$ is \emph{$(\kernel,\subg)$-sub-Gaussian}.
\end{definition}
Our next two lemmas show that for finitely-supported signed measures like $\Pin-\Qout$, this notion of functional sub-Gaussianity is equivalent to the prior notion of vector sub-Gaussianity, allowing us to use the two notions interchangeably. 
%
%
Hereafter, we say that $\k$ generates a SPSD matrix $\K$ if $\k(\xset,\xset) = \K$. 

\begin{lemma}[\tbf{Functional sub-Gaussianity implies vector sub-Gaussianity}]
\label{lem:funct_subg_vector_subg}
In the notation of \cref{def:alg-subg}, if $(\Pin - \Qout)\kernel$ is $(\kernel,\subg)$-sub-Gaussian on an event $\event$ and $\kernel$ generates $\K$, then the vector $\pin - \qout$ is $(\K,\subg)$-sub-Gaussian on $\event$.
%
\end{lemma}
%

%
%
%
%
%
%

%
%
%
%
%
%
%
%
%
%
\begin{proof}
%
Suppose $(\Pin - \Qout)\kernel$ is $(\kernel,\subg)$-sub-Gaussian on an event $\event$, fix a vector $\bu\in \reals^n$, and define the function
\begin{talign}
    f_{\bu} \defeq \sumn u_i \kernel(\cdot, x_i) \in \rkhs.
\end{talign}
By the reproducing property, 
\begin{talign}\label{eq:hnorm-of-fu}
    \bu^\top \K (\pin -\qout) = \inner{f_{\bu}}{(\Pin-\Qout)\kernel}_{\kernel} \qtext{and} \knorm{f_{\bu}}^2 = \bu^\top \K \bu.
\end{talign}
Invoking the representations \cref{eq:hnorm-of-fu} and the functional sub-Gaussianity condition \cref{eq:function-subg} we therefore obtain
\begin{talign}
    \Esubarg{\event}{\exp(\bu^\top \K(\pin-\qout)} &= \Esubarg{\event}{\exp(\inner{f_{\bu}}{(\Pin-\Qout)\kernel}_{\kernel})} 
    \leq \exp(\knorm{f_{\bu}}^2 \cdot \frac{\subg^2}{2}) 
    = \exp(\bu^\top \K \bu \cdot \frac{\subg^2}{2}),
\end{talign}
so that $\pin-\qout$ is $(\K,\subg)$-sub-Gaussian on the event $\event$ as claimed.
\end{proof}

%

\begin{lemma}[\tbf{Vector sub-Gaussianity implies functional sub-Gaussianity}]
\label{lem:vector_subg_funct_subg}
In the notation of \cref{def:alg-subg}, if $\pin - \qout$ is $(\K,\subg)$-sub-Gaussian on an event $\event$ and $\kernel$ generates $\K$, then $(\Pin - \Qout)\kernel$ is $(\kernel,\subg)$-sub-Gaussian on $\event$.
\end{lemma}
\begin{proof}
Suppose $\pin-\qout$ is $(\K,\subg)$-sub-Gaussian on an event $\event$, fix a function $f\in \rkhs$, and consider the set 
\begin{talign}
\Lset \defeq \braces{f_{\bu} \defeq \sum_{i=1}^n u_i \kernel(\cdot,x_i) : \bu \in \reals^n}.
\end{talign} 
Since $\Lset$ is a closed linear subspace of $\rkhs$, we can decompose $f$ as $f  = f_{\bu} + f_\perp$,
where $\bu\in\Rn$ and $f_\perp$ is orthogonal to $\Lset$ \citep[Theorem 12.4]{rudin1991functional},
%
so that 
\begin{talign}\label{eq:knorm-decomposition}
    \knorm{f}^2 = \knorm{f_{\bu}}^2 + \knorm{f_\perp}^2\qtext{and} \knorm{f_{\bu}}^2 = \bu^\top \K \bu.
\end{talign}
Invoking the orthogonality of $f_\perp$ and $(\Pin - \Qout)\kernel\in \Lset$, the reproducing property representations \cref{eq:hnorm-of-fu}, and the vector sub-Gaussianity condition \cref{eq:vector-subg}, we find that
\begin{talign}
    \Esubarg{\event}{\exp(\inner{f}{(\Pin-\Qout)\kernel}_{\kernel})} 
    &= \Esubarg{\event}{\exp(\inner{f_{\bu} + f_\perp}{(\Pin - \Qout) \kernel}_{\kernel})} 
    = \Esubarg{\event}{\exp(\bu^\top \K (\pin - \qout)})\\
    &\leq \exp(\bu^\top \K \bu \cdot \frac{\subg^2}{2}) 
    \sless{\cref{{eq:knorm-decomposition}}} \exp(\knorm{f}^2 \cdot \frac{\subg^2}{2}),
\end{talign}
so that $(\Pin-\Qout)\kernel$ is $(\kernel,\subg)$-sub-Gaussian on the event $\event$ as claimed.
\end{proof}


We end our discussion about the versions of sub-Gaussianity considered above by presenting the standard fact about the additivity of sub-Gaussianity parameters under summation of independent sub-Gaussian random vectors, adapted to our setting.

\begin{lemma}[\tbf{Vector sub-Gaussian additivity}]\label{lem:K_sub_gsn_additivity}
    Suppose that, for each $j\in [m]$, 
    $\Delta_j\in\reals^n$ is $(\mbf K,\subg_j)$ on an event $\event[j]$ given $\Delta_{1:(j-1)}\defeq (\Delta_1,\ldots,\Delta_{j-1})$ and $\event[\leq j-1]\defeq \bigcap_{i=1}^{j-1}\event[i]$. 
    Then $\sum_{j=1}^m \Delta_j$ is $(\mbf K, (\sum_{j=1}^m \subg_j^2)^{1/2})$-sub-Gaussian on $\event[\leq m]$.
    %
    \end{lemma}
    \begin{proof}
    Let $\event[\leq s] = \bigcap_{j=1}^s\event[j]$ for each $s\in [m]$.
    We prove the result for $\mc Z_s = \sum_{i=1}^s \Delta_j$ by induction on $s\in [m]$. 
    The result holds for the base case of $s=1$ by assumption. For the inductive case, suppose the result holds for $s\in [m-1]$. Fixing $\bu\in \R^n$, we may apply the tower property, our conditional sub-Gaussianity assumption, and our inductive hypothesis in turn to conclude
    \begin{talign}
        \Earg{\exp(\inner{\bu}{\K \sum_{j=1}^{s+1} \Delta_j})\indic{\event[\leq s+1]}} &= \Earg{\exp(\inner{\bu}{\K \sum_{j=1}^{s} \Delta_j})\indic{\event[\leq s]} \Earg{\exp(\inner{\bu}{\Delta_{s+1}})\indic{\event[s+1]} \mid \Delta_{1:s},\event[\leq s]} } \\
        &\leq \Earg{\exp(\inner{\bu}{\K \sum_{j=1}^{s} \Delta_j})\indic{\event[\leq s]}} \exp\parenth{\frac{\subg_{s+1}^2}{2}\cdot \bu^\top \K \bu}
        \leq \exp\big( \frac{\sum_{j=1}^{s+1} \subg_j^2}{2} \cdot \bu^\top \K \bu\big).
    \end{talign}
    Hence, $\mc Z_{s+1}$ is $(\K,(\sum_{j=1}^{s+1} \subg_j^2)^{1/2})$-sub-Gaussian on $\event[\leq s+1]$, and the proof is complete.
    \end{proof}



%



%

%
%
%
%
%
%
%
%
%
%
%
%
%
%
%
%



\section{Additional Related Work}\label{app:additional_related_work}
We discuss additional related work that could not be included above.

\paragraph{Structured prediction}
Before the introduction of the Fenchel--Young loss framework, \citet{Niculae18sparse} proposed SparseMAP, which used the squared $\ell_2$-norm regularization.
The Fenchel--Young loss, described in \cref{subsec:fenchel-young}, is built upon the idea of SparseMAP. 
The Structure Encoding Loss Function (SELF) was introduced by \citet{ciliberto16consistent,ciliberto20general} to analyze the relationship between surrogate and target losses, a concept known as Fisher consistency.
For more extensive literature, we refer the reader to \citet[Appendix A]{pmlr-v247-sakaue24a}.

\paragraph{Online classification with full and bandit feedback}
In the full information setup, PERCEPTRON is one of the most representative algorithms for binary classification \citep{Rosenblatt1958-sh}, and the multiclass setting has also been extensively studied \citep{,crammer2003ultraconservative,Fink_2006}.
Online logistic regression is another relevant research stream, with \citet{foster18logistic} being a particularly representative study. 
The study of the bandit setup was initiated by \citet{Kakade2008EfficientBA}, and it has since been extensively explored in subsequent research \citep{hazan11newtron,beygelzimer17efficient,foster18logistic}. However, to the best of our knowledge, no prior work has addressed general structured prediction under bandit feedback. A most related study is the work by \citet{gentile14multilabel}, which investigated online multilabel classification and ranking. 
However, their setting assumes access to feedback of the form $\set{\ind[\bmy_{t,i} \neq \hat{\bmy}_{t,i}]}_{i}$, which is more informative than bandit feedback and differs from our setup.
\Citet{NEURIPS2020_Hoeven} explicitly introduced the surrogate regret in the context of online multiclass classification. This study has been extended to the setting where observations are determined by a directed graph \Citep{NEURIPS2021_Hoeven} and to structured prediction scenarios \citep{pmlr-v247-sakaue24a}. For a more extensive overview of the literature on online classification, we refer the reader to \Citet{NEURIPS2020_Hoeven}.


\paragraph{Delayed feedback}
The study of delayed feedback began with \citet{Weinberger_2002_delay}. 
Since then, it has been extensively explored in various online learning settings, primarily in the full information setup of online convex optimization \citep{Mesterharm05online,joulani13online,joulani16delay,pmlr-v139-flaspohler21a}. 
Algorithms for delayed bandit feedback have been studied mainly in the context of multi-armed bandits and their variants \citep{cesabianchi16delay,zimmert20optimal,ito20delay,masoudian22best,hoeven23unified}. In the context of online classification, research considering delay is scarce; the only work is \citet{manwani2022delaytronefficientlearningmulticlass} to our knowledge.



\section{Discussion on the Difference in Surrogate Loss Functions}\label{app:Discussio_on_the_Difference_in_Surrogate_Losses}

As in \eqref{eq:sur_regret}, the surrogate regret $\reg$ is defined by $\sumt{L(\yht;\yt)}=\sumt{S(\U\xt;\yt)}+\reg$, which means the choice of the surrogate loss $S$ affects the bound on the cumulative loss $\sumt{L(\yht;\yt)}$.
\Citet[Theorem~1]{NEURIPS2021_Hoeven}, which applies to a more general setup than bandit feedback, implies $\reg = O(K\sqrt{T})$ for the bandit setup with $S$ being a logistic loss defined with the base-$K$ logarithm. 
On the other hand, our bound of $\reg = O(\sqrt{KT})$ applies to the logistic loss $S$ defined with the base-$2$ logarithm. 
As a result, while our bound on $\reg$ is better, the $\sumt{S(\U\xt;\yt)}$ term can be worse; this is why we cannot directly compare our $O(\sqrt{KT})$ bound with the $O(K\sqrt{T})$ bound in \Citet[Theorem~1]{NEURIPS2021_Hoeven}. 
We may use the decoding procedure in \citet{NEURIPS2021_Hoeven}, instead of RDUE, to recover their bound that applies to the base-$K$ logistic loss.
It should be noted that their method is specific to multiclass classification;  
naively extending their method to structured prediction formulated as $|\yy|$-class classification results in the undesirable dependence on $K = |\yy|$, as is also discussed in \citet{pmlr-v247-sakaue24a}. 
By contrast, our pseudo-inverse estimator, combined with RDUE, can rule out the explicit dependence on $K$, at the cost of the increase from $\sqrt{T}$ to $T^{2/3}$.
\section{Omitted Details of \cref{sec:bandit}}
\label{app:proof bandit}
This section provides the omitted details of \cref{sec:bandit}.

\subsection{Concentration inequality}
To prove high probability regret bounds, we use the following concentration inequality.
\begin{lemma}[{Bernstein's inequality, e.g., \citealt[Lemma A.8]{Cesa-Bianchi_Lugosi_2006}}]
    \label{lem:Bernstein}
    Let $Z_1,\hdots,Z_T$ be a martingale difference sequence and $\delta \in (0,1)$.
    If there exist $a$ and $v$ which satisfy $|Z_t|\leq a$ for any $t \in \brk{T}$ and $\sumt{\expect{Z_t^2}}\leq v$ , then with probability at least $1-\delta$, it holds that
    \[
        \sumt{Z_t}\leq\sqrt{2v\log\frac{1}{\delta}}+\frac{\sqrt{2}}{3}a\log\frac{1}{\delta}.
    \]
\end{lemma}


\subsection{Proof of \cref{thm:bandit_high_prob}}\label{app:proof_bandit_high_prob}
Here, we provide the proof of \cref{thm:bandit_high_prob}.
Hereafter, we let $S_{\max} = \max_{\W \in \ww} S_t(\W)$ and $\hat{S}_t(\W) = v_t S_t(\W) = \frac{\ind\brk{\yt = \yht}}{p_t(\yht)} S_t(\W)$.
The following theorem is the formal version of \cref{thm:bandit_high_prob}:
\begin{theorem}[Formal version of \cref{thm:bandit_high_prob}]\label{thm:bandit_high_prob_formal}
Consider the bandit and non-delayed setup.
Let 
\begin{equation}
    \mathcal{C}
    =
    \prn*{
        \frac{3}{2 (a + \xi - 1)}  
        +
        1
    }
    K \Smax \log(2/\delta) 
    +
    \frac{B^2 K b}{2 (1 - \xi)}
    .
    \nonumber
\end{equation}

Then, for any $T \geq \mathcal{C}$ and $\delta \in (0,1/2)$, with probability at least $1-\delta$, the algorithm in \cref{subsec:Bandit_Structured_Prediction_with_General_Losses} with $q = \sqrt{\mathcal{C} / T}$ achieves
\begin{equation}
    \mathcal{R}_T
    \leq
    2
    \sqrt{
        \mathcal{C}
        T
    }
    +
    \sqrt{2 \log (2/ \delta)} \prn{\mathcal{C} T}^{1/4}
    +
    \prn*{ \frac{1-a}{2 (a + \xi - 1)} + 2 } \log (2/\delta).
    \nonumber
\end{equation}
\end{theorem}


Before proving this theorem, we provide the following lemma:
\begin{lemma}\label{lem:hp_pre}
It holds that 
\begin{equation}
    \sum_{t=1}^T \prn*{ \E_t\brk*{L_t(\yht)} - \hat{S}_t(\U) }
    \leq
    \sum_{t=1}^T \prn*{ (1-a) S_t(\W_t) - \hat{S}_t(\W_t)  } + q T
    +
    \sqrt{2} B \sqrt{\frac{b}{q} \sum_{t=1}^T v_t S_t(\W_t) }
    .
    \nonumber
\end{equation}
\end{lemma}
\begin{proof}
We have 
\begin{equation}
    \sum_{t=1}^T \prn*{ \E_t\brk*{L_t(\yht)} - \hat{S}_t(\U) }
    =
    \sum_{t=1}^T \prn*{ \E_t\brk*{L_t(\yht)} - \hat{S}_t(\W_t)  }
    +
    \sum_{t=1}^T \prn*{ \hat{S}_t(\W_t) - \hat{S}_t(\U) }.
    \nonumber
\end{equation}
From \cref{asp:bandit_a}, the first term is bounded as 
\begin{align}
    \sum_{t=1}^T \prn*{ \E_t\brk*{L_t(\yht)} - \hat{S}_t(\W_t)  }
    \leq
    \sum_{t=1}^T \prn*{ (1-a) S_t(\W_t) - \hat{S}_t(\W_t)  } + q T,
    \nonumber
\end{align}
and 
the second term is bounded as 
\begin{align}
    \sum_{t=1}^T \prn*{ \hat{S}_t(\W_t) - \hat{S}_t(\U) }
    &\leq
    \sqrt{2} B \sqrt{\sum_{t=1}^T \nrm{\tilde{\G}_t}_{\F}^2 }
    =
    \sqrt{2} B \sqrt{\sum_{t=1}^T v_t^2 \nrm{\G_t}_{\F}^2 }
    \nonumber \\
    &\leq
    \sqrt{2} B \sqrt{b \sum_{t=1}^T v_t^2 S_t(\W_t) }
    \leq
    \sqrt{2} B \sqrt{\frac{b K}{q} \sum_{t=1}^T v_t S_t(\W_t) },
    \nonumber
\end{align}
where we used \cref{lem:ogd} and $v_t \leq K / q$.
Combining the above three, we obtain
\begin{equation}
    \sum_{t=1}^T \prn*{ \E_t\brk*{L_t(\yht)} - \hat{S}_t(\U) }
    \leq
    \sum_{t=1}^T \prn*{ (1-a) S_t(\W_t) - \hat{S}_t(\W_t)  } + q T
    +
    \sqrt{2} B \sqrt{\frac{b K}{q} \sum_{t=1}^T v_t S_t(\W_t) }
    ,
    \nonumber
\end{equation}
which completes the proof.
\end{proof}


\begin{proof}[Proof of \cref{thm:bandit_high_prob_formal}]
The surrogate regret can be decomposed as
\begin{equation}\label{eq:reg_dec_highp}
    \mathcal{R}_T 
    =
    \sum_{t=1}^T \prn*{ L_t(\yht) - \E_t\brk*{ L_t(\yht)} }
    +
    \sum_{t=1}^T \prn*{ \E_t\brk*{ L_t(\yht)} - S_t(\U) }
    .
\end{equation}
We first upper bound the first term in \eqref{eq:reg_dec_highp}.
Let $Z_t = L_t(\yht) - \E_t\brk*{ L_t(\yht)}$ for simplicity.
Then, we have $Z_t \leq 1$, $\E_t\brk*{Z_t} = 0$, and
$\E_t\brk*{Z_t^2} 
\leq 
\E_t\brk*{ \prn{L_t(\yht)}^2 }
\leq 
(1-a) S_t(\W_t) + q.
$
Hence, from Bernstein's inequality in \cref{lem:Bernstein}, for any $\delta' \in (0,1)$, at least $1 - \delta'$ we have 
\begin{equation}\label{eq:conc_zt}
    \sum_{t=1}^T Z_t
    \leq 
    \sqrt{2 \log (1/\delta') \sum_{t=1}^T \prn*{(1-a) S_t(\W_t) + q} }
    +
    \frac{\sqrt{2}}{3} \log (1/\delta')
    .
\end{equation}
We next consider the second term in \eqref{eq:reg_dec_highp}.
Define $r_t = S_t(\U) - \xi S_t(\W_t)$ for some $\xi \in (0, 1)$, which will be determined later,
and let $v_t = \ind[ \yt = \yht ] / p_t(\yht) \leq K/q$ for simplicity.
Then, we have $\E_t\brk{v_t r_t - r_t} = 0$, $\abs{v_t r_t - r_t} \leq K S_{\max} / q$, and
\begin{equation}
    \E_t\brk{ (v_t r_t - r_t)^2}
    \leq
    \E_t\brk{(v_t r_t)^2}
    \leq 
    \frac{K \Smax}{q} \abs{r_t}
    \leq 
    \frac{K \Smax}{q} \prn*{S_t(\U) + S_t(\W_t)}
    .
    \nonumber
\end{equation}
Hence from Bernstein's inequality in \cref{lem:Bernstein}, for any $\delta'' \in (0,1)$, with probability at least $1 - \delta''$ we have 
\begin{equation}\label{eq:conc_vr}
    \sum_{t=1}^T \prn{v_t r_t - r_t} 
    \leq 
    \sqrt{3 \log (1/\delta'') \sum_{t=1}^T \frac{K \Smax}{q} \prn{S_t(\U) + S_t(\W_t)} }
    +
    \frac{\sqrt{2} K \Smax}{3q} \log(1/\delta'')
    .
\end{equation}


\textbf{When $\sum_{t=1}^T S_t(\U) \leq \sum_{t=1}^T S_t(\W_t)$.}
We first consider the case of $\sum_{t=1}^T S_t(\U) \leq \sum_{t=1}^T S_t(\W_t)$.
From \cref{lem:hp_pre}, we have
\begin{align}
    &\sum_{t=1}^T \E_t\brk*{L_t(\yht)} - q T
    \leq
    \sum_{t=1}^T v_t S_t(\U) 
    +
    \sum_{t=1}^T \prn*{ (1-a) S_t(\W_t) - v_t S_t(\W_t)  } 
    +
    \sqrt{2} B \sqrt{\frac{b K}{q} \sum_{t=1}^T v_t S_t(\W_t) }
    \nonumber \\
    &=
    \sum_{t=1}^T v_t \underbrace{\prn*{ S_t(\U) - \xi S_t(\W_t)  } }_{= r_t}
    -
    (1 - \xi) \sum_{t=1}^T v_t S_t(\W_t)
    +
    (1-a) \sum_{t=1}^T S_t(\W_t)
    +
    \sqrt{2} B \sqrt{\frac{b K}{q} \sum_{t=1}^T v_t S_t(\W_t) }
    \nonumber \\
    &\leq
    \sum_{t=1}^T v_t r_t
    +
    (1-a) \sum_{t=1}^T S_t(\W_t)
    +
    \frac{B^2 K b}{2 q (1 - \xi)},
    \nonumber
\end{align}
where the last inequality follows from 
$c_1\sqrt{x}-c_2x\leq{c_1^2}/\prn{4c_2}$ for $x \geq 0$, $c_1 \geq 0$, and $c_2 > 0$.
From the concentration result provided in \eqref{eq:conc_vr}, this is further bounded as
\begin{align}
    \sum_{t=1}^T \E_t\brk*{L_t(\yht)} - q T
    &\leq
    \sum_{t=1}^T (S_t(\U) - \xi S_t(\W_t))
    +
    \sqrt{3 \log (1/\delta'') \sum_{t=1}^T \frac{K \Smax}{q} \prn{S_t(\U) + S_t(\W_t)} }
    \nonumber \\
    &\qquad
    +
    \frac{\sqrt{2} K \Smax}{3q} \log(1/\delta'') 
    +
    (1-a) \sum_{t=1}^T S_t(\W_t)
    +
    \frac{B^2 K b}{2 q (1 - \xi)}
    ,
    \nonumber
\end{align}
where we recall that $r_t = S_t(\U) - \xi S_t(\W_t)$.
Rearranging the last inequality and using $\sum_{t=1}^T S_t(\U) \leq \sum_{t=1}^T S_t(\W_t)$ give
\begin{align}
    \sum_{t=1}^T \prn*{ \E_t\brk*{L_t(\yht)} - S_t(\U) } 
    &\leq
    q T 
    +
    \sqrt{6 \log (1/\delta'') \sum_{t=1}^T \frac{K \Smax}{q} S_t(\W_t) }
    +
    \frac{\sqrt{2} K \Smax}{3 q} \log(1/\delta'') 
    \nonumber \\
    &\qquad
    +
    (1 - a - \xi) \sum_{t=1}^T S_t(\W_t)
    +
    \frac{B^2 K b}{2 q (1 - \xi)}
    .
    \nonumber
\end{align}
In what follows, we let $\delta' = \delta'' = \delta / 2$ and $\xi = \frac{\prn{4 + c} \gamma}{\lambda \nu}$ for a sufficiently small constant $c > 0$, which satisfies $a + \xi > 1$.
Then, plugging \eqref{eq:conc_zt} and the last inequality in \eqref{eq:reg_dec_highp}, with probability at least $1 - \delta$, we obtain
\begin{align}
    \mathcal{R}_T
    &\leq
    \sqrt{2 \log (2/\delta) \sum_{t=1}^T \prn*{(1-a) S_t(\W_t) + q} }
    +
    \frac{\sqrt{2}}{3} \log (2/\delta)
    +
    q T 
    +
    \sqrt{6 \log (2/\delta) \sum_{t=1}^T \frac{K \Smax}{q} S_t(\W_t) }
    \nonumber \\
    &\qquad
    +
    \frac{\sqrt{2} K \Smax}{3 q} \log(2/\delta) 
    +
    (1 - a - \xi) \sum_{t=1}^T S_t(\W_t)
    +
    \frac{B^2 K b}{2 q (1 - \xi)}
    \nonumber \\
    &\leq
    \frac{1}{2 (a + \xi - 1)}
    \prn*{(1-a) + \frac{3 K \Smax}{q}} \log (2/\delta)
    +
    \sqrt{2 q T \log (2 / \delta)} 
    +
    \frac{\sqrt{2}}{3} \log (2/\delta)
    +
    q T 
    \nonumber \\
    &\qquad
    +
    \frac{\sqrt{2} K \Smax}{3 q} \log(2/\delta) 
    +
    \frac{B^2 b}{2 q (1 - \xi)}
    \nonumber \\
    &\leq
    \frac{1}{q}
    \prn*{
        \frac{3 K \Smax \log (2/\delta)}{2 (a + \xi - 1)}  
        +
        K \Smax \log(2/\delta) 
        +
        \frac{B^2 K b}{2 (1 - \xi)}
    }
    +
    q T 
    +
    \sqrt{2 q T \log (2 / \delta)} 
    \nonumber \\
    &\qquad
    +
    \frac{1}{2 (a + \xi - 1)} (1-a) \log (2/\delta)
    +
    \frac{\sqrt{2}}{3} \log (2/\delta)
    \nonumber \\
    &=
    \frac{\mathcal{C}}{q}
    +
    q T 
    +
    \sqrt{2 q T \log (2 / \delta)} 
    +
    \frac{1}{2 (a + \xi - 1)} (1-a) \log (2/\delta)
    +
    \frac{\sqrt{2}}{3} \log (2/\delta).
    \nonumber
\end{align}
Using the definition of $q = \sqrt{\mathcal{C} / T}$ with the last inequality,
we obtain
\begin{equation}
    \mathcal{R}_T
    \leq
    2
    \sqrt{
        \mathcal{C}
        T
    }
    +
    \prn{\mathcal{C} T}^{1/4} \sqrt{\log (2/ \delta)}
    +
    \prn*{ \frac{1-a}{2 (a + \xi - 1)} + \frac{\sqrt{2}}{3} } \log (2/\delta).
    \nonumber
\end{equation}

\textbf{When $\sum_{t=1}^T S_t(\U) > \sum_{t=1}^T S_t(\W_t)$.}
We next consider the case of $\sum_{t=1}^T S_t(\U) > \sum_{t=1}^T S_t(\W_t)$.
We have
\begin{align}
    \mathcal{R}_T 
    &=
    \sum_{t=1}^T \prn*{ L_t(\yht) - \E_t\brk*{ L_t(\yht)} }
    +
    \sum_{t=1}^T \prn*{ \E_t\brk*{ L_t(\yht)} - S_t(\U) }
    \nonumber \\
    &\leq
    \sqrt{2 \log (1/\delta') \sum_{t=1}^T \prn*{(1-a) S_t(\W_t) + q} }
    +
    \frac{\sqrt{2}}{3} \log (1/\delta')
    +
    \sum_{t=1}^T \prn*{ \E_t\brk*{ L_t(\yht)} - S_t(\W_t) }
    \nonumber \\
    &\leq
    \sqrt{2 \log (1/\delta') \sum_{t=1}^T \prn*{(1-a) S_t(\W_t) + q} }
    +
    \frac{\sqrt{2}}{3} \log (1/\delta')
    +
    \sum_{t=1}^T \prn*{ - a S_t(\W_t) + q }
    \nonumber \\
    &\leq
    \frac{(1 - a) \log (1/ \delta')}{ 2 a }
    +
    \sqrt{2 q T \log (1/\delta') }
    +
    \frac{\sqrt{2}}{3} \log (1/\delta')
    +
    q T
    ,
    \nonumber
\end{align}
where the first inequality follows from \eqref{eq:conc_zt} and $\sum_{t=1}^T S_t(\U) > \sum_{t=1}^T S_t(\W_t)$,
and the second inequality follows from \cref{asp:bandit_a},
the last inequality follows from $c_1\sqrt{x}-c_2x\leq{c_1^2}/\prn{4c_2}$ for $x \geq 0$, $c_1 \geq 0$, and $c_2 > 0$.
Substituting $q = \sqrt{\mathcal{C} / 2}$ and $\delta' = \delta/2$ and  in the last inequality, we obtain
\begin{equation}
    \mathcal{R}_T 
    \leq
    \frac{(1 - a) \log (2/ \delta)}{ 2 a }
    +
    \sqrt{2 \log (2/\delta) } \prn{ \mathcal{C} T }^{1/4}
    +
    \frac{\sqrt{2}}{3} \log (2/\delta)
    +
    \sqrt{\mathcal{C} T}
    .
    \nonumber
\end{equation}
This completes the proof.    
\end{proof}







\subsection{Proof of \cref{thm:bandit_regret_pseudo_estimator}}
\label{app:sub_bandit_regret_pseudo_estimator}


Here, we provide the proof of \cref{thm:bandit_regret_pseudo_estimator}.
We recall that $\pt=\expect{\yht\yht^\top}$.
We then estimate $\yt$ by $\ytilde=\inverse{\V}\Pplus_t\yht\inpr{\yht,\V\yt}$
and $\G_t$ by $\gtil\coloneqq(\yho(\tht)-\ytilde)\xt^\top$ under \cref{asp:self}.
This $\gtil$ satisfies 
$
    \expect{\gtil}=\G_t. 
$
To prove \cref{thm:bandit_regret_pseudo_estimator}, we will upper bound $\expect{\nrm{\gtil}_\F^2}$.
To do so, we begin by proving the following lemma:
\begin{lemma}\label{lem:pseudo_inverse_order}
    Let $\bm{A}$ and $\bm{B}$ positive semi-definite matrices with $\image(\bm{A}) = \image(\bm{B})$ with $\bm{A} \succeq \bm{B}$.
    Then, it holds that $\bm{A}^+ \preceq \bm{B}^+$.
\end{lemma}
\begin{proof}
Since $\image(\bm{A}) = \image(\bm{B})$, there exists an orthogonal matrix $\bm{R}$, a diagonal matrix $\bm{\Lambda}$, and an invertible matrix $\bm{B}'$ that has same dimensions as $\bm{\Lambda}$ such that 
\begin{equation}
    \bm{A}
    =
    \bm{R}
    \begin{pmatrix}
        O & O \\
        O & \bm{\Lambda} 
    \end{pmatrix}
    \bm{R}^\top
    \quad 
    \mbox{and}
    \quad
    \bm{B}
    =
    \bm{R}
    \begin{pmatrix}
        O & O \\
        O & \bm{B}'
    \end{pmatrix}
    \bm{R}^\top
    .
    \nonumber
\end{equation}
Then, 
\begin{equation}\label{eq:Aplus_Bplus}
    \bm{A}^+
    =
    \bm{R}
    \begin{pmatrix}
        O & O \\
        O & \bm{\Lambda}^{-1}
    \end{pmatrix}
    \bm{R}^\top
    \quad 
    \mbox{and}
    \quad
    \bm{B}^+
    =
    \bm{R}
    \begin{pmatrix}
        O & O \\
        O & {\bm{B}'}^{-1}
    \end{pmatrix}
    \bm{R}^\top
    .
\end{equation}
From $\bm{A} \succeq \bm{B}$,
we have $\bm{\Lambda} \succeq \bm{B}'$, which implies $\bm{\Lambda}^{-1} \preceq {\bm{B}'}^{-1}$.
From this and \eqref{eq:Aplus_Bplus}, we have $\bm{A}^+ \preceq \bm{B}^+$, as desired.
\end{proof}

Using this lemma we prove a property of $\pt$ and an upper bound of $\expect{\tr\prn*{\yht\yht^\top}}$.
In what follows, we use $\lambda_\min(\bm{A})$ to denote the minimum eigenvalue of a matrix $\bm{A}$.

\begin{lemma}
    \label{lem:bound of trace}
    Suppose that $\tr \prn*{ \V^{-1} \bm{Q} \prn{\V^{-1}}^\top } \leq \omega$ for $\bm{Q} = \E_{\bm{y} \sim \mu} \brk{ \bm{y} \bm{y}^\top }$, where we recall $\mu$ is the uniform distribution over $\yy$.
    Then, we have
    \[
    \expect{\tr(\ytt\ytt^\top)}\leq \frac{\omega}{q}.
    \]
\end{lemma}
\begin{proof}    
    By the linearity of expectation and the trace property, we have
    \begin{align*}
        \expect{\tr(\ytilde\ytilde^\top)}&\leq \tr\prn*{\inverse{\V}\Pplus_t\expect{\yht\yht^\top}\Pplus_t\prn*{\inverse{\V}}^\top}
        = \tr\prn*{\inverse{\V}\Pplus_t \bm{P}_t \Pplus_t \prn*{\inverse{\V}}^\top}\\
        &= 
        \tr\prn*{\inverse{\V}\Pplus_t\prn*{\inverse{\V}}^\top},
    \end{align*}
    where the first inequality follows from $\inpr{\yht,\V\yt} = L_t(\yht) - \inpr{\yht,\bm{b}} - c(\yt) \leq L_t(\yht) \leq 1$ since $\bm{b} \geq 0$ and $c(\cdot)$ is non-negative
    and
    the last equality follows from $\Pplus_t \bm{P}_t \Pplus_t = \Pplus_t$.
    Hence,
    \begin{align}
        \tr\prn*{\inverse{\V}\Pplus_t\prn*{\inverse{\V}}^\top}
        &=
        \sum_{i=1}^d
        \bm{\ee}_i^\top \inverse{\V}\Pplus_t\prn*{\inverse{\V}}^\top \bm{\ee}_i
        \leq
        \sum_{i=1}^d
        \bm{\ee}_i^\top \inverse{\V} \prn*{q \bm{Q}}^{+} \prn*{\inverse{\V}}^\top \bm{\ee}_i
        \nonumber \\
        &\leq
        \tr\prn*{\prn*{\inverse{\V}}^\top \inverse{\V} (q \bm{Q})^+ }
        =
        \frac{1}{q}
        \tr \prn*{ \inverse{\V} \bm{Q}^+ \prn{\inverse{\V}}^\top } 
        \leq
        \frac{\omega}{q},
        \nonumber
    \end{align}
    where in the first inequality we used \cref{lem:pseudo_inverse_order} and in the last inequality we used the assumption that $\tr \prn*{ \V^{-1} \bm{Q}^+ \prn{\V^{-1}}^\top } \leq \omega$.
    This completes the proof.
\end{proof}

Now, we are ready to upper bound $\expect{\nrm{\gtil}_\F^2}$.
\begin{lemma}
    \label{thm:evaluation of Gtilde}
    Under the same assumption as \cref{lem:bound of trace}, it holds that 
    \[
        \expect{\nrm{\gtil}_\F^2}\leq2b\sw+ \frac{2 \dix^2 \omega}{q}.
    \]
\end{lemma}
\begin{proof}
    We have 
    \begin{align*}
        \nrm{\gtil}_\F^2&=\nrm{\prn*{\yho(\tht)-\ytilde}\xt^\top}_{\mathrm{F}}^2\leq2\nrm{(\yho(\tht)-\yt)\xt^\top}_\F^2+2\nrm{(\yt-\ytilde)\xt^\top}_\F^2\\
        &\leq 2\nrm{\G_t}_\F^2+2\dix ^2\nrm{\yt-\ytilde}_2^2,
    \end{align*}
    where we recall $\dix =\diam(\xx)$.
    From this inequality, 
    \begin{align}
        \expect{\nrm{\gtil}_\F^2}&\leq2\nrm{\G_t}_{\mathrm{F}}^2+2\dix ^2\expect{\nrm{\yt-\ytilde}_2^2}
        \leq2b\sw+2\dix ^2\prn*{\nrm{\yt}_2^2-2\yt^\top\expect{\ytilde}+\expect{\nrm{\ytilde}_2^2}} \nonumber \\
        &=2b\sw+2\dix ^2\prn*{\nrm{\yt}_2^2-2\nrm{\yt}_2^2 + \expect{\nrm{\ytilde}_2^2}} \nonumber \\ 
        &\leq
        2b\sw
        +2\dix ^2\expect{\tr(\ytilde\ytilde^\top)}
        \leq
        2b\sw
        + \frac{2\dix^2 \omega}{q},
        \nonumber
    \end{align} 
    where in the second inequality we used $\nrm{\G_t}_\F^2 \leq b \sw$, in the equality we used $\expect{\ytilde}=\yt$,
    and in the last inequality we used \cref{lem:bound of trace}.
\end{proof}



Finally, we are ready to prove \cref{thm:bandit_regret_pseudo_estimator}.
\begin{proof}[Proof of \cref{thm:bandit_regret_pseudo_estimator}]
    From \cref{asp:bandit_a}, we have 
    \begin{align*}\label{eq:inverse_reg_expect}
        \E\brk{\reg}
        &\leq\E\brk*{\sumt{(\sw-\su)}}-a\E\brk*{\sumt{\sw}}+qT
        \nonumber \\
        &\leq\E\brk*{\sumt{\inpr*{\G_t,\wt-\U}}}-a\E\brk*{\sumt{\sw}}+qT.
    \end{align*}
    From \cref{thm:evaluation of Gtilde} and the unbiasedness of $\gtil$, 
    the first term in the last inequality is further bounded as
    \begin{align*}
        \E\brk*{\sumt{\inpr*{\G_t,\wt-\U}}}  
        &=\E\brk*{\sumt{\inpr*{\gtil,\wt-\U}}}
        \leq\sqrt{2}B\sqrt{\E\brk*{\sumt{\nrm{\gtil}_{\mathrm{F}}^2}}}
        \nonumber \\
        &\leq
        2 B \sqrt{b\E\brk*{\sumt{\sw}}}
        +
        2 B \dix \sqrt{ \omega / q},
    \end{align*}
    where 
    the first inequality follows from \cref{lem:ogd} and the last inequality follows from \cref{thm:evaluation of Gtilde} and the subadditivity of $x \mapsto \sqrt{x}$ for $x \geq 0$.
    Therefore, by combining  with the last inequality, we have 
    \begin{align}
        \E\brk{\reg}
        &\leq 
        2B \sqrt{b\E\brk*{\sumt{\sw}}} 
        +
        2 B \dix \sqrt{ \omega / q} 
        -a \E\brk*{\sumt{\sw}} + qT \\ 
        &\leq 
        \frac{bB^2}{a}
        +
        2 B \dix \sqrt{ \omega / q} 
        +qT ,
        \nonumber
    \end{align}
    where we used $c_1\sqrt{x}-c_2x\leq{c_1^2}/\prn{4c_2}$ for $x\geq0$, $c_1\geq 0$, and $c_2>0$.
    Finally, substituting 
    $q=\prn*{\frac{4 \omega B^2\dix ^2}{ T }}^{1/3}$ in the last inequality gives
    \[
    \E\brk{\reg}
    \leq
    \frac{bB^2}{a}
    +
    2^{5/3} \omega^{1/3} \prn*{ B \dix T}^{2/3}
    ,
    \]
    which is the desired bound.
\end{proof}



\subsection{Proof of \cref{cor:thm_self}}\label{app:SELF_upper_discussion_deferred}
Here, we derive the regret upper bounds provided by the algorithm established  
in \cref{thm:bandit_regret_pseudo_estimator}  
for online multiclass classification, online multilabel classification, and ranking.
Recall that 
we can achieve
\begin{equation}\label{eq:bound_self_app}
\E\brk{\reg}
\leq
\frac{bB^2}{a}
+
O\prn*{ \omega^{1/3} \prn*{ B \dix T}^{2/3} },    
\end{equation}
where we recall that $\omega$ is defined as
$
    \tr \prn*{ \V^{-1} \bm{Q}^+ \prn{\V^{-1}}^\top } \leq \omega
$
for $\bm{Q} = \E_{\bm{y} \sim \mu} \brk{ \bm{y} \bm{y}^\top }$.
Note that when $\spanx(\yy) = \R^d$, then the matrix $\bm{Q}$ is invertible and $\lambda_{\min}(\bm{Q}) > 0$, and thus 
\begin{equation}\label{eq:trace_upper_invertibleQ}
    \tr \prn*{ \V^{-1} \bm{Q}^+ \prn{\V^{-1}}^\top }
    =
    \sum_{i=1}^d
    \bm{\ee}_i^\top \V^{-1} \bm{Q}^+ \prn{\V^{-1}}^\top \bm{\ee}_i
    \leq
    \frac{1}{\lambda_{\min}(\bm{Q})}
    \sum_{i=1}^d
    \nrm{ \prn{\V^{-1}}^\top \bm{\ee}_i }_2^2
    \leq 
    \frac{1}{\lambda_{\min}(\bm{Q})}
    \nrm{ \V^{-1} }_{\F}^2
    .
\end{equation}
In each problem setting, this regret upper bound can be reduced to the following bounds:

\paragraph{Multiclass classification with 0-1 loss}
We first consider multiclass classification with the 0-1 loss.
Since $\V=\bm{1}\bm{1}^\top-\I$, we have $\nrm{\inverse{\V}}_\F^2\leq d$ for $d \geq 2$.  
Recalling that $\mu$ is the uniform distribution over $\yy=\set{\bm{\ee}_1,\hdots,\bm{\ee}_d}$, we have  
$
\E_{\bmy\sim\mu}\brk{(\bmy^\top\bm x)^2}=\frac{1}{d}\sum_{i=1}^{d}x_i^2
$
for any $\bm x\in\R^d$.
Hence, $\lambda_{\min}(\bm{Q})=\min_{\nrm{\bm x}_2=1} \E_{\bmy\sim\mu}\brk*{(\bmy^\top\bm x)^2}=\frac{1}{d}$, where the first equality is from \citet[Lemma 2]{comband}.  
Since $\spanx(\yy) = \R^d$ in this case,
from \eqref{eq:trace_upper_invertibleQ} we can let $\omega = d / \lambda_{\min}(\bm{Q}) = d^2$.
Substituting these into our upper bound in \eqref{eq:bound_self_app}, we obtain  
\begin{equation*}
    \E\brk{\reg}\leq\frac{bB^2}{a}+ O \prn*{ \prn{B \dix  d T}^{2/3} }.
\end{equation*}  



\paragraph{Online multilabel classification with $m$ correct labels   
and the Hamming loss}
We next consider online multilabel classification with the number of correct labels $m$  
and the Hamming loss.  
Since $\V=-\frac{2}{d}\I$, we have  
$\nrm{\inverse{\V}}_\F^2=\frac{d^3}{4}$.  
Let $\yy\subset\set{0,1}^d$ be the set of all vectors  
where exactly $m$ components are $1$, and the remaining components are all $0$.  
By drawing $\bmy\in\yy$ according to the uniform distribution over $\yy$,  
the probability that a given component of $\bmy$ is $1$ is  
$\binom{m-1}{d-1}/\binom{m}{d}=\frac{m}{d}$.  
Hence, for any $\bm x\in\R^d$ with $\nrm{\bm{x}}_2=1$,  
we have
\[
\E_{\bmy\sim\mu}\brk*{(\bmy^\top\bm x)^2}
=\frac{m}{d}\sum_{i=1}^dx_i^2  
+\frac{m^2}{d^2}\sum_{i\neq j}x_ix_j  
=\prn*{\frac{m}{d}\sum_{i=1}^dx_i}^2+\frac{m(d-m)}{d^2}\nrm{\bm x}_2^2  
\geq 
\frac{m(d-m)}{d^2}.
\]
Hence, we have $\lambda_{\min}(\bm{Q}) = \E_{\bmy\sim\mu}\brk*{(\bmy^\top\bm x)^2} \geq\frac{m(d-m)}{d^2}$, where the equality is from \citet[Lemma 2]{comband}.
Since $\spanx(\yy) = \R^d$,
from \eqref{eq:trace_upper_invertibleQ} we can choose $\omega = d^3 / \prn{4 \lambda_{\min}(\bm{Q})} = \frac{d^5}{4 m (d-m)} $.
Therefore, our regret upper bound in \eqref{eq:bound_self_app} is reduced to
\begin{equation*}
    \E\brk{\reg}
    \leq
    \frac{bB^2}{a}
    +
    O \prn*{ \prn*{\frac{B^2\dix ^2d^5}{m(d-m)}}^{1/3} T^{2/3} }.
\end{equation*}


\paragraph{Ranking with the Hamming loss and the number of items $m$}
We finally consider online ranking with the Hamming loss and the number of items $m$.  
From \citet[Proposition 4]{comband}, the smallest positive eigenvalue is at least $1/m$.
Hence, since $\V=-\frac{1}{m}\I$, we have
\begin{equation*}
\tr\prn{\inverse{\V}\bm{Q}^+(\inverse{\V})^\top}=m^2\tr\prn{\bm{Q}^+}\leq m^2\sum_{i=1}^{\rank(\bm{Q}^+)} m\leq m^5,
\end{equation*}
where we used $\rank(\bm{Q}^+) \leq d = m^2$,
and this allows us to choose $\omega = m^5$.
Substituting these values into our regret upper bound in \eqref{eq:bound_self_app} , we obtain  
\begin{equation*}
    \E\brk{\reg}\leq\frac{bB^2}{a}+ O\prn*{ m^{5/3}\prn{B\dix T}^{2/3} }.
\end{equation*}  


\section{Omitted Details of \cref{sec:delay}}
\label{app:proof delay}
This section provides the proofs of the theorems in \cref{sec:delay}.

\subsection{Details of Optimistic Delayed Adaptive FTRL (ODAFTRL)}
\label{app:sub_odaftrl}
We provide a more detailed explanation of the Optimistic Delayed Adaptive FTRL (ODAFTRL) algorithm used for updating $\W_t$ in \cref{sec:delay}.
We recall that ODAFTRL computes $\W_t$ by the following update rule:
\begin{equation}
    \label{eq:odaftrl_2}
    \W_{t+1}=\argmin_{\W\in \ww} \set*{ \sum_{i=1}^{t-D} \inpr{\bm{G}_i ,\W} + \frac{\lambda_t \nrm{\W}_{\F}^2 }{2} },
\end{equation}
where $\lambda_t\geq0$ is a regularization parameter.
The ODAFTRL algorithm, when using this parameter update called AdaHedgeD, satisfies the following lemma:
\begin{lemma}[{\citealt[Theorem 12]{pmlr-v139-flaspohler21a}}]
    \label{thm:AdaHedgeD}
    Fix $\alpha>0$. 
    Let $S_t:\ww\to\R$ be a convex function for each $t=1,\dots,T$.
    Suppose that we update $\lambda_{t}$ in \eqref{eq:odaftrl_2} by the following AdaHedgeD update:
    \begin{align*}
        \lambda_{t+1}&=\frac{1}{\alpha}\sum_{s=1}^{t-D}\delta_s,\\
        \delta_t &= \min\crl*{F_{t+1}(\W_t)-F_{t+1}(\bar{\bm{W}}_t),\inpr{ \bm{G}_t,\W_t-\bar{\bm{W}}_t},F_{t+1}(\hat{\bm{W}}_t)-F_{t+1}(\bar{\bm{W}}_t)+\inpr{ \bm{G}_t,\W_t-\hat{\bm{W}}_t}}_+,\\
        \bar{\bm{W}}_t &= \argmin_{\W\in\ww}F_{t+1}(\W),\\
        \hat{\bm{W}}_t&= \argmin_{\W\in\ww}\set*{F_{t+1}(\W)-\min\crl*{\frac{\nrm{\bm{G}_t}_{\mathrm{F}}}{\nrm{\bm{G}_{t-D:t}}_{\mathrm{F}}},1}\inpr{\bm{G}_{t-D:t},\W}}, \text{  and}\\
        F_{t+1}(\W)&=\frac{\lambda_t\nrm{\W}_\F^2}{2}+\inpr{\G_{1:t},\W}.
    \end{align*}
    Then, for any $\U\in\ww$, ODAFTRL achieves
    \begin{equation*}
        \sumt{\sw}
        \leq
        \sumt{\su}
        +
        \prn*{\frac{B^2}{2\alpha}+1}\prn*{2\max_{s\in[T]}a_{s-D:s-1}+\sqrt{\sum_{t=1}^{T}a_{t}^2+2\alpha b_{t}}},
    \end{equation*}
    where 
    \begin{align*}
        a_{t}&=B\min\crl{\nrm{\bm{G}_{t-D:t}}_{\mathrm{F}},\nrm{\bm{G}_t}_{\mathrm{F}}},\\
        b_{t}&=\operatorname{huber}\prn{\nrm{\bm{G}_{t-D:t}}_{\mathrm{F}},\nrm{\bm{G}_t}_{\mathrm{F}}},\:\text{and} \:
        \operatorname{huber}(x,y)=\frac{1}{2}x^2-\frac{1}{2}(|x|-|y|)^2_+\leq\min\crl*{\frac{1}{2}x^2,|x||y|}.
    \end{align*}
\end{lemma}
In the following, we let $\alpha = \max_{\W \in \ww} \frac{\nrm{\W}_\F^2}{2} = \frac{B^2}{2}$ for simplicity. Then,
\begin{equation*}
        \sumt{\sw}
        \leq
        \sumt{\su}
        +
        2\prn*{2\max_{s\in[T]}a_{s-D:s-1}+\sqrt{\sum_{t=1}^{T}a_{t}^2+B^2 b_{t}}},
\end{equation*}


\subsection{Proof of \cref{thm:delayed_regret_expectation_abstract} }
\label{app:sub_delayed_regret_expectation_abstract}
We present \cref{thm:delayed_regret_expectation_abstract} in a more detailed form and provide its proof. 
\begin{theorem}[Formal version of \cref{thm:delayed_regret_expectation_abstract}]
    \label{thm:delayed_regret_expectation_abstract_detail}
    Let $\alpha=\frac{B^2}{2}$.
    Then, ODAFTRL with the AdaHedgeD update in online structured prediction with a delay of $D$ achieves
    \begin{equation*}
        \E\brk*{\reg}\leq4 B D \L+\frac{2bB^2}{a}
        +\frac{3}{2}\prn*{a^{-1}bB^4\L^2(D+1)^2T}^{1/3}.
    \end{equation*}
\end{theorem}
\begin{proof}
    From \cref{thm:AdaHedgeD} and the definition of $a_t$ and $b_t$, we have
    \begin{align*}
        \sumt{(\sw-\su)}&\leq2\prn*{2B\max_{s\in[T]}\sum_{i=s-D}^{s-1}\nrm{\G_i}_{\mathrm{F}}
        +\sqrt{\sumt{\prn*{B^2\normst^2+B^2\nrm{\G_{t-D:t}}_\F\normst}}}}\\
        &\leq 2\prn*{2BD\L+\sqrt{bB^2\sumt{\sw}}+  \prn*{b B^4 \L^2 (D+1)^2T\sumt{\sw}}^{1/4}},
    \end{align*}
    where we used the Cauchy--Schwarz inequality, $\normst\leq \L$, $\normst^2\leq b\sw$, and the subadditivity of $x \mapsto \sqrt{x}$ for $x \geq 0$ in the last inequality.
    From this inequality and \cref{asp:delayed_a}, we can evaluate surrogate regret as
    \begin{align*}
        \E\brk*{\reg} &\leq \sumt{\prn*{(1-a)\sw-\su}}\\
        &\leq 2\prn*{2BD\L+\sqrt{bB^2\sumt{\sw}} + \prn*{b B^4 \L^2 (D+1)^2 T\sumt{\sw}}^{1/4}} -a\sumt{\sw}\\
        &\leq 4 B D \L+\frac{2bB^2}{a}
        +\frac{3}{2}\prn*{a^{-1}bB^4\L^2(D+1)^2T}^{1/3}.
    \end{align*}
    We used $c_1\sqrt{x}-c_2x\leq{c_1^2}/\prn{4c_2}$ and $c_1x-c_2x^4\leq\prn{{3}/{4}}\prn*{{c_1^4}/\prn{4c_2}}^{1/3}$ that hold for any $x\geq0$, $c_1\geq 0$, and $c_2>0$ in the last inequality.
\end{proof}


\subsection{Proof of \cref{thm:delayed_regret_probability_abstract}}
\label{app:sub_delayed_regret_probability_abstract}
We present \cref{thm:delayed_regret_probability_abstract} in a more detailed form and provide its proof. 
\begin{theorem}
    \label{thm:delayed_regret_probability_abstract_detail}
    Let $\alpha=\frac{B^2}{2}$ and $\delta \in (0,1)$.
    Then, 
    ODAFTRL with the AdaHedgeD update in online structured prediction with a delay of $D$ achieves
    \begin{equation*}
        \reg\leq 4BD\L+\frac{\sqrt{2}}{3}\log\frac{1}{\delta}
        +\frac{\prn*{\sqrt{(1-a)\log\frac{1}{\delta}}+\sqrt{2bB^2}}^2}{a}
        +\frac{3}{2}\prn*{a^{-1}bB^4\L^2(D+1)^2T}^{1/3},
    \end{equation*}
    with probability at least $1 - \delta$. 
\end{theorem}

\begin{proof}
    We decompose $\reg$ into     
    \begin{equation}\label{eq:decompose_reg}
    \reg
    =\sumt{\prn*{L_t(\yht)-\expect{L_t(\yht)}}}+\sumt{\prn{\expect{L_t(\yht)}-\su}}.
    \end{equation}
    Let $Z_t=L_t(\yht)-\expect{L_t(\yht)}$. 
    Then, we have $|Z_t|\leq 1$ and $\expect{Z_t^2}\leq (1-a)\sw$ from \cref{asp:delayed_a}.
    Hence, from \cref{lem:Bernstein}, with probability at least $1 - \delta$, the first term in \eqref{eq:decompose_reg} is upper bounded as 
    \begin{equation}\label{eq:bound_of_zt}
        \sumt{Z_t}\leq\sqrt{2(1-a)\sumt{\sw}\log\frac{1}{\delta}}+\frac{\sqrt{2}}{3}\log\frac{1}{\delta}.
    \end{equation}
    From from \cref{asp:delayed_a} and \cref{thm:AdaHedgeD},
    the second term in \eqref{eq:decompose_reg} is also upper bounded as 
    \begin{align}
        &
        \sumt{(\expect{L_t(\yht)}-\su)}
        \nonumber \\
        &\leq
        2\prn*{2BD\L+\sqrt{bB^2\sumt{\sw}}+  \prn*{b B^4 \L^2 (D+1)^2T\sumt{\sw}}^{1/4}} - a\sumt{\sw},
        \label{eq:bound_of_reg_exp}
    \end{align}
    where we used the subadditivity of $x \mapsto \sqrt{x}$ for $x \geq 0$.
    Therefore, substituting \eqref{eq:bound_of_zt} and \eqref{eq:bound_of_reg_exp} into \eqref{eq:decompose_reg} gives 
    \begin{align*}
        \reg&\leq 4BD\L+\frac{\sqrt{2}}{3}\log\frac{1}{\delta}
        +\prn*{\sqrt{ 2 (1-a) \log \frac{1}{\delta} } + 2\sqrt{bB^2}}\sqrt{\sumt{\sw}}\\
        &\quad+2\prn*{b B^4 (D+1)^2 \L ^2T\sumt{\sw}}^{1/4} - a\sumt{\sw}\\
        &\leq 4BD\L+\frac{\sqrt{2}}{3}\log\frac{1}{\delta}
        +\frac{\prn*{\sqrt{(1-a)\log\frac{1}{\delta}}+\sqrt{2bB^2}}^2}{a}
        +\frac{3}{2}\prn*{a^{-1}bB^4\L^2(D+1)^2T}^{1/3},
    \end{align*}
    where we used $c_1\sqrt{x}-c_2x\leq{c_1^2}/\prn{4c_2}$ and $c_1x-c_2x^4\leq\prn{{3}/{4}}\prn*{{c_1^4}/\prn{4c_2}}^{1/3}$ for $x\geq0$, $c_1\geq 0$, and $c_2>0$ in the last inequality.
    This is the desired bound.
\end{proof}
\section{Omitted Details of \cref{sec:bandit_and_delayed}}\label{app:bandit_and_delayed}
This section provides the omitted proofs of the theorems in \cref{sec:bandit_and_delayed}.

\subsection{Common analysis}
We provide the analysis that is commonly used in the proofs of \cref{thm:delay_bandit_bound_general_abstract,thm:delay_bandit_bound_self_abstract}.
We use ODAFTRL with the AdaHedgeD update in \cref{app:sub_odaftrl} for $\alg$.
From \cref{thm:AdaHedgeD}, it holds that
\begin{equation}\label{eq:inpr_common}
    \sumt{\inpr{\gtil,\W_t-\U}} 
    \leq 2\prn*{2\max_{t\in[T]}a_{t-D:t-1}+\sqrt{\sumt{a_t^2}+B^2 b_t}},
\end{equation}
where
\begin{equation*}
    a_t=B\min\set{\nrm{\tilde{\G}_{t-D:t}}_\F,\nrm{\gtil}_\F} \quad \mbox{and} \quad b_t\leq\min\set*{\frac{1}{2}\nrm{\tilde{\G}_{t-D:t}}_\F^2, \nrm{\tilde{\G}_{t-D:t}}_\F\nrm{\gtil}_\F}.
\end{equation*}
By the definition of $a_t$, we have 
\begin{equation}\label{eq:bound_of_at}
    \E\brk*{\max_{t\in\brk{T}}a_{t-D:t-1}}\leq B\E\brk*{\max_{t\in\brk{T}} \sum_{s=t-D}^{t-1}\nrm{\tilde{\G}_{s}}_\F }\leq B\E\brk*{\sqrt{D\max_{t\in\brk{T}}\sum_{s=t-D}^{t-1}\nrm{\tilde{\G}_s}_\F^2}}\leq B\sqrt{D\E\brk*{\sumt{\nrm{\gtil}_\F^2}}},
\end{equation}
from the Cauchy--Schwarz inequality and Jensen's inequality.
Hence, from \eqref{eq:inpr_common} and \eqref{eq:bound_of_at}, we have
\begin{align}\label{eq:bandit_delayed_expect_inpr}
    \E\brk*{\sumt{\inpr{\gtil,\W_t-\U}}}
    &\leq 2\prn*{2\E\brk*{\max_{t\in[T]}a_{t-D:t-1}}+\sqrt{\E\brk*{\sumt{a_t^2}}}+B\sqrt{ \E\brk*{\sumt{b_t}}}}\nonumber\\
    &\leq 2B\prn*{2\sqrt{D\E\brk*{\sumt{\nrm{\gtil}_\F^2}}}+\sqrt{\E\brk*{\sumt{\nrm{\gtil}_\F^2}}}+\sqrt{\E\brk*{\sumt{b_t}}}},
\end{align}
where we used the subadditivity of $x \mapsto \sqrt{x}$ for $x \geq 0$.
The last term in the last inequality is further bounded as
\begin{align}\label{eq:expect_bt}
    \E\brk*{\sumt{b_t}}&\leq \E\brk*{\sumt{\nrm{\tilde{\G}_{t-D:t}}_\F\nrm{\gtil}_\F}}
    \leq\E\brk*{\sumt{\nrm{\gtil}_\F^2}} +\E\brk*{\sumt{\nrm{\gtil}_\F\sum_{s=t-D}^{t-1}\nrm{\tilde{\G}_s}_\F}}\nonumber\\
    &=\E\brk*{\sumt{\nrm{\gtil}_\F^2}} + \E\brk*{\sumt
    {\E_t\brk*{\nrm{\gtil}_\F}\sum_{s=t-D}^{t-1}\nrm{\tilde{\G}_s}_\F}}, 
\end{align}
where the second inequality follows from the triangle inequality and the equality follows from the law of total expectation.


\subsection{Proof of \cref{thm:delay_bandit_bound_general_abstract}}\label{app:bandit_delayed_general}
We provide the complete version of \cref{thm:delay_bandit_bound_general_abstract}:
\begin{theorem}[Formal version of \cref{thm:delay_bandit_bound_general_abstract}]
    The algorithm in \cref{subsec:bandit_delay_general} achieves
\begin{equation*}
    \E\brk*{\reg}
    \leq \frac{4bB}{a}\prn*{D^{1/4}+D^{-1/4}}^2\sqrt{KT}+2B \L\sqrt{D T}+B\sqrt{DKT}
    = O\prn{D\sqrt{KT}}
    .
\end{equation*}
\end{theorem}

\begin{proof}
First, we will upper bound $\E\brk*{\sumt{b_t}}$.
From \eqref{eq:expect_bt}, we have
\begin{align}\label{eq:bound_of_bt_general}
    \E\brk*{\sumt{b_t}}&\leq \E\brk*{\sumt{\nrm{\gtil}_\F^2}} + \E\brk*{\sumt
    {\E_t\brk*{\nrm{\gtil}_\F}\sum_{s=t-D}^{t-1}\nrm{\tilde{\G}_s}_\F}}\nonumber\\
    &\leq \E\brk*{\sumt{\nrm{\gtil}_\F^2}} + \L\E\brk*{\sumt{\sum_{s=t-D}^{t-1}\E_s\brk*{\nrm{\tilde{\G}_s}_\F}}}\nonumber\\
    &\leq \E\brk*{\sumt{\nrm{\gtil}_\F^2}} + D\L^2 T,
\end{align}
where the second and third inequality follow from the inequality $\expect{\nrm{\gtil}_\F}=\nrm{\G_t}_\F\leq \L$.
Hence, from \eqref{eq:bandit_delayed_expect_inpr} and \eqref{eq:bound_of_bt_general}, it holds that
\begin{align}
    \E\brk*{\sumt{\inpr{\gtil,\W_t-\U}}}
    &\leq 2B\prn*{2\sqrt{D\E\brk*{\sumt{\nrm{\gtil}_\F^2}}}+\sqrt{\E\brk*{\sumt{\nrm{\gtil}_\F^2}}}
    +
    \sqrt{{\E\brk*{\sumt{\nrm{\gtil}_\F^2}} + D\L^2 T}}}\nonumber\\
    &\leq 4B\prn*{\sqrt{D}+1}\sqrt{\E\brk*{\sumt{\nrm{\gtil}_\F^2}}}+2 B\L\sqrt{ D T}\nonumber\\
    &\leq 4B\prn*{\sqrt{D}+1}\sqrt{\frac{bK}{q}\E\brk*{\sumt{\sw}}}+2B\L\sqrt{ D T},
\end{align}
where in the second inequality we used the subadditivity of $x \mapsto \sqrt{x}$ for $x \geq 0$ and in the last inequality we used 
\begin{equation*}
\expect{\|\gtil\|_{\mathrm{F}}^2}
=
\frac{\|\G_t\|_{\mathrm{F}}^2}{p_t(\yt)}
\leq
\frac{\K }{q}\|\G_t\|_{\mathrm{F}}^2
\leq
\frac{b\K }{q}\sw.
\end{equation*}
Therefore, combining all the above arguments yields 
\begin{align*}
    \E\brk*{\reg}&
    \leq \E\brk*{\sumt{(\sw-\su)}} - a\E\brk*{\sumt{\sw}} + q T\\
    &\leq \E\brk*{\sumt{\inpr{\gtil,\W_t-\U}}} - a\E\brk*{\sumt{\sw}} + q T\\
    &\leq 4B\prn*{\sqrt{D}+1}\sqrt{\frac{bK}{q}\E\brk*{\sumt{\sw}}}+2 B\L\sqrt{ D T}- a\E\brk*{\sumt{\sw}} + q T\\
    &\leq \frac{4bB^2}{a}\prn*{\sqrt{D}+1}^2\frac{bK}{q}+2 B\L\sqrt{D T}+ q T,
\end{align*}
where the first inequality follows from \cref{asp:bandit_a},
the second inequality follows from the convexity of $S_t$ and the unbiasedness of $\gtil$,
and the last inequality follows from $c_1\sqrt{x}-c_2x\leq{c_1^2}/\prn{4c_2}$ for $x \geq 0$, $c_1 \geq 0$, and $c_2 > 0$.
Finally, by substituting $q=B\sqrt{DK/T}$, we obtain
\begin{equation*}
    \E\brk*{\reg}
    \leq
    \frac{4bB}{a}\prn*{D^{1/4}+D^{-1/4}}^2\sqrt{KT}+2B \L\sqrt{D T}+B\sqrt{DKT},
\end{equation*}
which is the desired bound.
\end{proof}





\subsection{Proof of \cref{thm:delay_bandit_bound_self_abstract}}\label{app:bandit_delayed_self}
We provide the complete version of \cref{thm:delay_bandit_bound_self_abstract}:
\begin{theorem}[Formal version of \cref{thm:delay_bandit_bound_self_abstract}]
    The algorithm in \cref{subsec:bandit_delay_self} achieves
    \begin{equation*}
        \E\brk{\reg}
        =
        \frac{8bB^2}{a}\prn[\Big]{\sqrt{D} + 1}^2
    +2B (\L+\dix \diy )\sqrt{DT}
        + O\prn*{ B^{2/3} D^{1/3} T^{2/3} }. 
    \end{equation*}
\end{theorem}
\begin{proof}
First, we will derive an upper bound of $\E\brk*{\sumt{b_t}}$.
We first observe that
\begin{align*}
    \expect{\nrm{\gtil}_\F}&=\expect{\nrm{(\yho(\tht)-\ytilde)\xt^\top}_\F}\leq \expect{\nrm{\G_t}_\F + \dix \nrm{\yt-\ytilde}_2}
    \leq \nrm{\G_t}_\F + \dix \expect{\nrm{\yt}_2+\nrm{\ytilde}_2}\\
    &\leq \nrm{\G_t}_\F + \dix \diy + \dix \expect{\sqrt{\tr\prn*{\ytilde\ytilde^\top}}}
    \leq \nrm{\G_t}_\F + \dix \diy + \dix {\sqrt{\expect{\tr\prn*{\ytilde\ytilde^\top}}}}\\
    &\leq \nrm{\G_t}_\F + \dix \diy + \sqrt{\dix^2\omega / q}
    \leq \L+ \dix \diy + \sqrt{\dix^2\omega / q},
\end{align*}
where the first inequality follows from $\dix =\diam(\xx)$, the third inequality follows from $\diy=\diam(\yy)$, the fourth inequality follows from Jensen's inequality, and the fifth inequality follows from \cref{lem:bound of trace}.
Thus, combining \eqref{eq:expect_bt} with the last inequality, we have
\begin{align}
    \E\brk*{\sumt{b_t}}&\leq \E\brk*{\sumt{\nrm{\gtil}_\F^2}} + \E\brk*{\sumt
    {\E_t\brk*{\nrm{\gtil}_\F}\sum_{s=t-D}^{t-1}\nrm{\tilde{\G}_s}_\F}} \nonumber \\
    &\leq \E\brk*{\sumt{\nrm{\gtil}_\F^2}}+ \prn*{\L+ \dix \diy + \sqrt{\frac{\dix^2\omega}{q}} } \E\brk*{\sumt{\sum_{s=t-D}^{t-1}\E_s\brk*{\nrm{\tilde{\G}_s}_\F}}} \nonumber \\
    & \leq\E\brk*{\sumt{\nrm{\gtil}_\F^2}}+ D T \prn*{\L+ \dix \diy + \sqrt{\frac{\dix^2\omega}{q}} }^2.
    \label{eq:expect_gtil_self}
\end{align}
Hence, from \eqref{eq:bandit_delayed_expect_inpr} and \eqref{eq:expect_gtil_self}, we have
\begin{align}\label{eq:expext_inpr_self}
    &\E\brk*{\sumt{\inpr{\gtil,\W_t-\U}}}\nonumber\\
    &\leq 2B\prn*{2\sqrt{D\E\brk*{\sumt{\nrm{\gtil}_\F^2}}}+\sqrt{\E\brk*{\sumt{\nrm{\gtil}_\F^2}}}+\sqrt{\E\brk*{\sumt{b_t}}}}\nonumber\\
    &\leq 4B\prn*{\sqrt{D}+1}\sqrt{\E\brk*{\sumt{\nrm{\gtil}_\F^2}}} + 2B\prn*{\L+ \dix \diy + \sqrt{ \dix^2\omega / q } } \sqrt{DT}\nonumber\\
    &\leq 4B\prn*{\sqrt{D}+1}\sqrt{2\sumt{\prn*{b\sw+\frac{\dix^2\omega}{q}}}
    }
    + 2B\prn*{\L+ \dix \diy + \sqrt{\dix^2\omega / q} } \sqrt{DT},
\end{align}
where the first inequality follows from \eqref{eq:bandit_delayed_expect_inpr}, 
the second inequality follows from \eqref{eq:expect_gtil_self} and the subadditivity of $x \mapsto \sqrt{x}$ for $x \geq 0$, 
and the last inequality follows from \cref{thm:evaluation of Gtilde}.
Therefore, combining all the above arguments yields 
\begin{align*}
    \E\brk*{\reg}
    &\leq \E\brk*{\sumt{(\sw-\su)}} - a\E\brk*{\sumt{\sw}} + q T\\
    &\leq \E\brk*{\sumt{\inpr{\gtil,\wt-\U}}} - a\E\brk*{\sumt{\sw}} + q T\\
    & \leq 4B\prn*{\sqrt{D} + 1}\prn*{\sqrt{2b\sumt{\sw}}+\sqrt{ 2\dix^2\omega T / q}
    }
    \nonumber \\
    &\qquad+ 2B\prn*{\L+ \dix \diy + \sqrt{\dix^2\omega / q}  } \sqrt{DT} - a\E\brk*{\sumt{\sw}} + q T \\
    &\leq
    \frac{8bB^2}{a}\prn[\Big]{\sqrt{D} + 1}^2
    +2B (\L+\dix \diy )\sqrt{DT}
    +2B\dix\prn[\big]{(2\sqrt{2}+1)\sqrt{D}+2\sqrt{2}}\sqrt{\omega T/ q} + q T,
\end{align*}
where the first inequality follows from \cref{asp:bandit_a}, 
the third inequality follows from \eqref{eq:expext_inpr_self} and the subadditivity of $x \mapsto \sqrt{x}$ for $x \geq 0$, 
and the last inequality follows from the definition of $\epsilon$ and $c_1\sqrt{x}-c_2x\leq{c_1^2}/\prn{4c_2}$ for $x \geq 0$, $c_1 \geq 0$, and $c_2 > 0$.
Finally, substituting $q=\prn*{\frac{\omega B^2\dix^2 D}{T}}^{1/3}$ gives the desired bound.
\end{proof}



\section{Overhead of \ourSystem.}
We report the size and inference time of the model for NeRF and \ourSystem in Table~\ref{table_overhead}.  
The results indicate that \ourSystem has a larger model size than NeRF, \ie 27.1 \vs 8.0\,MB.  
Correspondingly, \ourSystem exhibits a longer inference time, \ie 1.79 \vs 0.43\,s.  
Unlike NeRF, \ourSystem requires neighboring spectra as input. 
During inference, the target transmitter's neighbors are extracted from the training dataset, so \ourSystem does not incur additional data burdens.  
Moreover, since \ourSystem can operate in unseen scenes, it significantly reduces the requirement for a time-consuming training process.



\begin{table}[h]
\centering
\caption{Comparison of model size and inference time.}

\begin{tabular}{lC{0.8in}C{0.8in}}
\toprule
     & \nerft    & \ourSystem    
     \\ \midrule
Model size (MB) & 8.0  & 27.1   \\
Inference time (s) & 0.43    & 1.79  \\
\bottomrule
\end{tabular}
\label{table_overhead}
\end{table}






\end{document}



\ifarxiv{

}
\else{
\appendix
% 
\newpage
\section*{Appendix}\label{sec:appendix}
\setcounter{section}{0}

\input{sections/Appendix/0_Chernoff_Bounds}
\input{sections/Appendix/1_Final_Search}
\input{sections/Appendix/2_Deviation_Bounds}
\input{sections/Appendix/3_Other_Models}
}
\fi
\end{document}
