\section{Related Work}
\label{ssec:related work}
The problem of triangle counting is a problem that has been extensively studied across various models of computation. 
\paragraph{RAM model.}
The work of~\citep{DBLP:journals/siamcomp/ChibaN85} proposed an $\bigo{\frac{\edgecount\arboricity}{\numtriangle}}$ algorithm in the RAM model that remains the best known till date. Recent works  have shown it to be conditionally optimal under the 3SUM~\citep{DBLP:conf/soda/KopelowitzPP16} and APSP hypotheses~\citep{DBLP:conf/focs/WilliamsX20}. 
\paragraph{Streaming model.}
In the data streaming model, a long line of work~\citep{DBLP:journals/ipl/PaghT12,DBLP:journals/pvldb/PavanTTW13} has culminated in an $\bigot{\frac{\edgecount}{\numtriangle}\fbrac{\edgesensitivity+\sqrt{\vertexsensitivity}}}$-space one pass streaming algorithm for insertion only model~\citep{KallaugherP17} that has been shown to be optimal through the combined lower bounds proposed in~\citep{BOV_13_Streaming_triangle_counting_hardness,KallaugherP17}. Here,  $\edgesensitivity$ and $\vertexsensitivity$  denotes the maximum number of triangles that an edge $\edge$ or a vertex $\vertex$ in the graph $\graph$ participates in, respectively. There have been works on triangle counting in multi-pass setup as well as in other streaming models (e.g. turnstile, cash-register etc.)~\citep{DBLP:conf/pods/McGregorVV16,DBLP:conf/stacs/BeraC17}.

\paragraph{Property Testing model.}
In query complexity setup, the triangle counting problem has been studied in terms of various query access available to the algorithm. It has been shown that without access to \edgeexistsq{}, no sublinear query algorithm can exist for triangle counting~\citep{GRS11}. Hence, sublinear query algorithms for counting triangles have been studied in models with \degreeq{}, \neighbourq{}, and \edgeexistsq{} queries~\citep{Dana_Ron_Triangle_Counting,DBLP:conf/soda/EdenRS20} resulting in an $\bigot{\min\fbrac{\frac{\vertexcount\arboricity^2}{\numtriangle},\frac{\vertexcount}{\numtriangle^{1/3}}+\frac{\edgecount\arboricity}{\numtriangle}}}$ query algorithm, with matching lower bounds~\citep{Dana_Ron_Triangle_Counting,DBLP:conf/approx/EdenR18,DBLP:conf/soda/EdenRS20} up to $\poly{\log\fbrac{\frac{\vertexcount}{\confidence}},\frac{1}{\approxerror}}$ factors. The work in~\citep{tetek:LIPIcs.ICALP.2022.107} highlighted the discrepancy in time and query complexities of approximate triangle counting while improving the time complexity of triangle counting for graphs with the number of triangles in a particular range. With \randedgeq{}, \neighbourq{}, and \edgeexistsq{} query access, an $\bigot{\frac{\edgecount^{3/2}\log\fbrac{1/\confidence}}{\approxerror^2\numtriangle}}$ query algorithm was proposed by~\citep{assadi2018simple} which was shown to be optimal in all parameters, inclusive of $\approxerror$ and $\confidence$~\citep{DBLP:conf/approx/AssadiN22}.