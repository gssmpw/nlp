\section{Preliminaries}
\label{sec:prelim}
\subsection{Notations}
\label{ssec:notation}
The set $\{1,2,\ldots,x\}$ is denoted as $[x]$.
We consider $\graph = (\vertexset,\edgeset)$ to be a simple, unweighted, undirected graph with $\size{\vertexset} = \vertexcount$, and $\size{\edgeset} = \edgecount$. Given a vertex $\vertex$, its neighboring vertex set is denoted as $\neighbour(\vertex) = \set{\altvertex|(\altvertex,\vertex)\in \edgeset}$. We denote by $\degree{\vertex}$ the degree of the vertex $\vertex$. Based on the degrees of the two vertices of an edge $\edge = \fbrac{\vertex,\altvertex}$, we define the degree of the edge $\edge$ as $\degree{\edge} = \min\fbrac{\degree{\vertex}~,\degree{\altvertex}}$. We denote the set of triangles in $\graph$ as $\triangleset$, and individual triangles are denoted as $\triangle$. ($\fbrac{\vertex,\edge}$ denotes a triangle formed by the vertices $\vertex$ and the endpoints of the edge $\edge$). We want to estimate the number of triangles,  $\size{\triangleset} = \numtriangle$ in the graph given the $\degreeq$, $\neighbourq$, $\edgeexistsq$ and $\randedgeq$ queries. An edge $\edge$ participates in a triangle $\triangle$ means that the triangle $\triangle$ is incident on the edge $\edge$. We denote by $\numtriangle_\edge$ the number of triangles the edge $\edge$ participates in. $\uniform(S)$ denotes an element of $S$ is chosen uniformly at random. 

% \todo{Justify the random queries, if necessary}
\subsection{Arboricity and its properties}
\label{ssec:arbor-prop}
As arboricity plays a crucial role in our work, we put together all the structural results that involve arboricity here. Let us restate the definition once more. 
\begin{definition}[Arboricity$(\arboricity)$]
   The arboricity of a graph $\graph = (\vertexset,\edgeset)$, denoted by $\arboricitygraph{G}$, is the minimum number of spanning forests that $\edgeset$ can be partitioned into.
   \label{def:arboricity}
\end{definition}
The arboricity of a graph can be seen as a measure of the density of the graph. $\arboricitygraph{G}$ can be at least $\left\lceil m/(n-1)\right\rceil$. Also, $\arboricitygraph{G} \geq \arboricitygraph{H}$ where $H$ is any subgraph of $G$. We will write $\arboricity$ instead of $\arboricitygraph{G}$ when the underlying graph is understood. We introduce the following lemma due to~\citep{DBLP:journals/siamcomp/ChibaN85} on the sum of edge degrees over all  edges in the graph.
\begin{lemma}(~\citep{DBLP:journals/siamcomp/ChibaN85})
\label{Lemma: deg(e) sum is m * arboricity}
     Given a graph $\graph = (\vertexset,\edgeset)$ with arboricity $\arboricity$ and $\size{\edgeset} = \edgecount$,  $\sum\limits_{\edge \in \edgeset} \degree{\edge} = 2\edgecount\arboricity$.
\end{lemma}

The following lemma due to~\citep{DBLP:conf/soda/EdenRS20} builds on the work of~\citep{DBLP:journals/siamcomp/ChibaN85} to bound the number of triangles based on the number of edges $\edgecount$ and arboricity $\arboricity$. 
\begin{lemma}[Triangle Upper Bound ~\citep{DBLP:conf/soda/EdenRS20}]
\label{lemma: arboricity triangle bound}
    Given a graph $\graph = (\vertexset,\edgeset)$ with arboricity $\arboricity$ and $\size{\edgeset} = \edgecount$, the graph $\graph$ has at most $\edgecount\arboricity$ triangles.
\end{lemma}
Note that this upper bound is also tight, i.e., there exists graphs that contain $\edgecount$ edges and $\bigomega{\edgecount\arboricity}$ triangles. Additionally, arboricity $\arboricity$ can be at most $\bigo{\sqrt{\edgecount}}$. Thus all our results can be reformulated by plugging in this upper bound. 


\ifarxiv{
\subsection{Chernoff Bounds}
We will be using the following variation of the Chernoff bound that bounds the deviation of the sum of independent Poisson trials~\citep{Mitzenmacher_Upfal_2005}.

\begin{lemma}[Multiplicative Chernoff Bound]\label{Lemma: Multiplicative Chernoff Bound}
    Given i.i.d. random variables $X_1,X_2,...,X_t$ where $\Pr[X_i = 1] = p$ and $\Pr[X_i = 0] = (1-p)$, define $X = \sum_{i \in [t]} X_i$. Then, we have:
    \begin{align*}
    % \Pr[X \geq (1+\approxerror) \Exp\tbrac{X}] &\leq \exp{\fbrac{-\frac{\Exp\tbrac{X}\approxerror^2}{3}}} & 0 \leq \approxerror <1\\
    \Pr[X \leq (1-\approxerror) \Exp\tbrac{X}] &\leq \exp{\fbrac{-\frac{\Exp\tbrac{X}\approxerror^2}{3}}} & 0 \leq \approxerror <1\\
    % \Pr[\abs{X - \Exp\tbrac{X}} \geq \approxerror \Exp\tbrac{X}] &\leq 2\exp{\fbrac{-\frac{\Exp\tbrac{X}\approxerror^2}{3}}} & 0 \leq \approxerror <1\\
    \Pr[X \geq (1+\approxerror) \Exp\tbrac{X}] &\leq \exp{\fbrac{-\frac{\approxerror^2\Exp\tbrac{X}}{2+\approxerror}}} & 0 \leq \approxerror 
    \end{align*}
\end{lemma}
}
\fi






%-----------------------------Notations-----------------------

