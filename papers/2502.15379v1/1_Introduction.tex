\section{Introduction}

\label{sec:intro}
\noindent
Counting the number of triangles present in graphs is a classical problem dating back to \cite{DBLP:journals/siamcomp/ChibaN85,DBLP:journals/siamdm/AlonKKR08,SWExperimentalListingTriangles2005,AlonTestingFOCS2013,DBLP:conf/soda/Bar-YossefKS02}. Researchers in the sub-linear space (streaming) community has been looking at this problem for the last two decades; see~\cite{DBLP:conf/soda/Bar-YossefKS02,DBLP:conf/stacs/BeraC17,jayaram_et_al:LIPIcs.APPROX/RANDOM.2021.11, AhnGMPODS2012, TriangleCountingDynamicGraphALgorithmica2016, DBLP:journals/algorithmica/AlonYZ97, JhaSeshadriStreamingTransitivity, BuriolFLPODS06, CORMODE2017SecondLook, ShinDynamicStreamTriangle, KaneMelhornArbitrarySubgrapgICALP12} for a sample of such works. The problem has also been studied by the researchers in the sub-linear time (property testing) community in the last decade~\cite{Dana_Ron_Triangle_Counting,DBLP:conf/pods/McGregorVV16,KallaugherP17,DBLP:conf/stacs/BeraC17,BeraSeshadriStreamingDegeneracy,DBLP:conf/soda/EdenRS20,jayaram_et_al:LIPIcs.APPROX/RANDOM.2021.11}. Here, graphs can be accessed only through certain queries. On the other hand, starting with the work of~\cite{DBLP:journals/siamcomp/ChibaN85}, the graph parameter called arboricity has also found its use in triangle counting~\cite{DBLP:conf/soda/EdenRS20}.
This work shows the use of arboricity in triangle counting where the graph can be accessed through a random edge query along with other standard queries like degree, neighbor and edge existence. 

Counting triangle has been a well motivated problem with applications across various domains like optimizing query size in database join problems~\citep{DBLP:conf/soda/Bar-YossefKS02,AtseriasDatabaseJoinJComp,assadi2018simple}, computing clustering coefficients, transitivity ratio~\citep{CAggarwalEvolutionaryNetworkSurvey,LucePerry1949AMO,watts_collective_1998, LeskovecBKT/SIGKDD08/EvolSocNet}, studying structures in web graphs~\citep{Eckmann2001CurvatureOC, DanischBS/WWW18.RealWorldDegeneracy} etc. For a more thorough overview, see the surveys~\citep{Hasan2018TriangleCI,TsourakakisJournalOG}. The parametrization based on arboricity is also of practical interest due to occurrence of low-arboricity graphs in various real world scenarios~\citep{DoryGI/BoundedArboricity/PODC22,LowArboricityMatching/FSTTCS2024/Konrad,OnakSSW/LowArboricityMinorFree/ACMTrAlg,GoelGustedtBoundedArboricity,beraCG/BoundedArboricity:LIPIcs.ICALP.2020.11,DanischBS/WWW18.RealWorldDegeneracy,ShinRealWorldKCore/JKInfSyst}.


% \medskip 

Formally, we are given a simple, unweighted, undirected graph $\graph = \fbrac{\vertexset,\edgeset}$ with $\size{\vertexset} = \vertexcount$, and $\size{\edgeset} = \edgecount$. The set of neighbours of the vertex $\vertex$ is denoted by $\neighbour{\fbrac{\vertex}}$ $=$  $\sbrac{\altvertex|\fbrac{\altvertex,\vertex}\in \edgeset}$ and the degree of the vertex $\vertex$ is denoted by $\degree{\vertex}= \size{\neighbour{\fbrac{\vertex}}}$. We are given the following queries to access the graph:
\begin{itemize}
    \item{$\degreeq{(\vertex)}$: } Given a vertex $\vertex$, return $\degree{\vertex}$. 
    \item{$\neighbourq(\vertex,i)$: } Given a $\vertex \in \vertexset$ and $i \in \degree{\vertex}$, returns the $i$-th vertex $\altvertex \in \neighbour(\vertex)$.
    % \item{$\neighbourq(\vertex)$:} Given a vertex $\vertex \in \vertexset$, return a vertex $\altvertex \in \neighbour(\vertex)$ uniformly at random. 
    % \complain{\st{(Debarshi: Take care of the definition. Do you really need the randomness in the neighbour returned? I think others did not assume any randomness here.)}}
    % % \debarshi{Both ~\cite{Dana_Ron_Triangle_Counting, 
    % DBLP:conf/soda/EdenRS20} uses random enighbour queries. Querying $i$-th neighbour and a random neighbour is equivalent. Random neighbour facilitates a smoother presentation.} \complain{Debarshi, discuss this with me tomorrow and settle. It is better to stick to already given definitions.}
    % \todo{\st{Fix neighbour, and modify lower bound degree simulation accordingly.}}
    \item{$\edgeexistsq(\altvertex,\vertex)$: } Given two vertices $\altvertex,\vertex \in \vertexset$, returns $1$ if $(\altvertex,\vertex) \in \edgeset$, and $0$ otherwise.
    \item{$\randedgeq$:} Returns an edge $\edge \in \edgeset$ uniformly at random.
\end{itemize}
We will call the $\degreeq$, $\neighbourq$ and $\edgeexistsq$ queries taken together to be \emph{local queries}. 

% \medskip 

Given parameters $0 < \approxerror, \confidence < 1$, our task is to obtain an estimate $\emptriangle$ of the number of triangles $\numtriangle$ in the graph $G$, such that $\emptriangle \in \tbrac{\fbrac{1-\approxerror}\numtriangle,\fbrac{1+\approxerror}\numtriangle}$ with probability $1-\confidence$. From now on, we refer $\emptriangle$ to be an $\appcon$ estimate of $\numtriangle$. One of the graph parameters that has been relevant to quantify the complexity of triangle listing algorithms within the context of RAM model~\citep{DBLP:journals/siamcomp/ChibaN85}, streaming model~\citep{BeraSeshadriStreamingDegeneracy}, distributed model~\citep{JainSeshadriTuranWWW17, SuriVasCursedLastReducerWWW11, FinocchiMapreduceArboricityJExpAlg} and query model~\citep{DBLP:conf/approx/EdenR18} is the arboricity of the graph $\graph$, denoted $\arboricity$. The arboricity of a graph can be seen as a measure of the density of the graph.
\begin{definition}[Arboricity$(\arboricity)$]
   The arboricity of a graph $\graph = (\vertexset,\edgeset)$, denoted by $\arboricity$, is the minimum number of spanning forests that $\edgeset$ can be partitioned into.
\end{definition}
 

% \medskip 
\remove{
Earlier works of lower bounds for property testing problems was based on reductions from communication complexity problems~\citep{Blais2012,Goldreich2020,DBLP:conf/approx/EdenR18}. However, these works did not consider the impact of $\approxerror$ and $\confidence$ in the query complexity of the problem.} Recent works have established various ways to construct lower bounds incorporating the impact of $\approxerror$ and $\confidence$ both in graph property testing~\citep{DBLP:conf/approx/AssadiN22} and other problems~\citep{tetek:LIPIcs.ICALP.2022.107}. Our lower bounds will also involve $\approxerror$ and $\confidence$. To illustrate the dependence of the polylog terms, $\approxerror$ and $\confidence$ on the query complexity of our algorithm, we will use two notations at times in place of $\bigo{\bullet}$ -- $\bigot{\bullet}$ hides polylog factors only but shows dependence on $\approxerror$ and $\confidence$, whereas $\bigoted{\bullet}$ hides both polylog and $\approxerror$, $\confidence$ terms. Similarly, we will have $\bigomegat{\bullet}$ and $\bigomegated{\bullet}$.

\remove{To illustrate the dependence of $\approxerror$ and $\confidence$ on the query complexity of our algorithm, we denote $\bigot{}$(resp. $\bigomegat{}$ to incorporate only the $\poly{\log{\fbrac{\vertexcount}}}$ factors as $\bigot{\bullet} = \bigo{\bullet}\cdot\poly{\log\fbrac{\vertexcount}}$ (resp. $\bigomegat{\bullet} = \bigomega{\bullet}\cdot\poly{\log\fbrac{\vertexcount}}$, and $\bigoted{}$(resp. $\bigomegated{}$ to incorporate $\poly{\frac{\log{\fbrac{1/\confidence}}}{\approxerror}}$ as well as $\poly{\log{\fbrac{\vertexcount}}}$ factors as $\bigoted{\bullet} = \bigo{\bullet}\cdot\poly{\frac{\log{\fbrac{\vertexcount/\confidence}}}{\approxerror}}$(resp. $\bigomegated{\bullet} = \bigomega{\bullet}\cdot\poly{\frac{\log{\fbrac{\vertexcount/\confidence}}}{\approxerror}}$.}

\subsection{Our Results}
\label{ssec:ourresults}
In this section we present our results and contextualize it vis-a-vis the works of~\cite{assadi2018simple,Dana_Ron_Triangle_Counting,DBLP:conf/soda/EdenRS20,DBLP:conf/approx/AssadiN22}. 
\paragraph*{Upper Bound.}
We show that using random edge queries in addition to local queries, we can obtain an $\appcon$ estimate of $\numtriangle$ where the query complexity is parameterized by arboricity apart from the usual parameters of edge, vertex, the number of triangles, $\approxerror$ and $\confidence$. The work of~\citep{DBLP:journals/siamcomp/ChibaN85} established the role of arboricity in the problem of triangle counting giving an $\bigo{\edgecount\arboricity}$ time algorithm to exactly count the number of triangles $\numtriangle$ in the RAM model. In recent works in the property testing model, ~\citep{Dana_Ron_Triangle_Counting} gave an algorithm that uses \emph{local queries} and requires $\bigoted{\frac{\vertexcount}{\numtriangle^{1/3}}+\min{\sbrac{\edgecount,\frac{\edgecount^{3/2}}{\numtriangle}}}}$ queries to obtain an $\appcon$ estimate $\emptriangle$ of $\numtriangle$. Using \randedgeq{} in addition to the \emph{local} queries, \citep{assadi2018simple} proposed an algorithm that uses $\bigot{\frac{\edgecount^{3/2}\log{\fbrac{1/\confidence}}}{\approxerror^2\numtriangle}}$ queries. Following this, ~\citep{DBLP:conf/soda/EdenRS20} was the first work to introduce the parameter of arboricity to triangle counting in the query based setup. They proposed an algorithm that uses $\bigot{\frac{\vertexcount}{\numtriangle^{1/3}}+\frac{\edgecount\arboricity\log^3{\fbrac{1/\confidence}}}{\approxerror^5\numtriangle}}$ \emph{local} queries to obtain an $\appcon$ estimate $\emptriangle$ of $\numtriangle$. Our work follows as a natural progression to the works of ~\cite{Dana_Ron_Triangle_Counting,assadi2018simple,DBLP:conf/soda/EdenRS20}, where we ask if we can bring in the parameter of arboricity to triangle counting by using \randedgeq{} in addition to other \emph{local} queries (see the last row in Table~\ref{Table: Comparison Table}).  
In this work, we use \randedgeq{} in addition to \emph{local} queries to obtain an algorithm that uses $\bigot{\frac{\edgecount\arboricity\log{\fbrac{1/\confidence}}}{\approxerror^3\numtriangle}}$ queries. Formally,
\begin{theorem}[Upper Bound(Simplified)]\label{Theorem: Upper Bound Simplified}
    Given a simple, unweighted and undirected graph $\graph = \fbrac{\vertexset,\edgeset}$ with $\size{\vertexset} = \vertexcount$, $\size{\edgeset} = \edgecount$ and arboricity $\arboricity$, and access to the graph via \degreeq{}, \neighbourq{}, \edgeexistsq{} and \randedgeq{} queries, an $\appcon$ estimate $\emptriangle$ of $\numtriangle$ can be obtained using $\bigot{\frac{\edgecount\arboricity\log{\fbrac{1/\confidence}}}{\approxerror^3\numtriangle}}$ queries.
\end{theorem}

In streaming model,~\citep{BeraSeshadriStreamingDegeneracy} proposed a constant ($\bigo{1}$) pass $\bigot{\frac{\edgecount\degeneracy}{\approxerror^4\numtriangle}}$ space algorithm for obtaining an $\approxerror$-multiplicative estimate of number of triangles in a graph. Our algorithm can be directly extended to a constant pass streaming algorithm using $\bigot{\frac{\edgecount\arboricity}{\approxerror^3\numtriangle}}$ space. By the fact that $\arboricity \leq \degeneracy \leq 2\arboricity$, our algorithm's space complexity is as good as the algorithm of~\citep{BeraSeshadriStreamingDegeneracy} in terms of the graph parameters, and asymptotically better in terms of the approximation factor $\approxerror$.



\paragraph*{Lower Bound.}
We also establish a lower bound involving the arboricity and the parameters $\approxerror$ and $\confidence$. Reductions from communication complexity is a well studied method to establish lower bound results in terms of query complexity~\citep{Blais2012,Goldreich2020}. Using only \degreeq{} and \neighbourq{} queries, triangle counting requires $\bigomega{\edgecount}$ queries~\citep{GRS11}. Including \edgeexistsq{} with the previous two queries, ~\citep{Dana_Ron_Triangle_Counting} established a lower bound of $\bigomega{\frac{\vertexcount}{\numtriangle^{1/3}}+\frac{\edgecount^{3/2}}{\numtriangle}}$ queries which matches their upper bound up to $\poly{\frac{\log{\fbrac{\vertexcount/\confidence}}}{\approxerror}}$ factors. Later, a simpler proof was proposed for the same bound through reduction from communication complexity problems~\citep{DBLP:conf/approx/EdenR18}, and was subsequently extended to incorporate arboricity with a bound of $\bigomega{\frac{\vertexcount}{\numtriangle^{1/3}}+\frac{\edgecount\arboricity}{\numtriangle}}$ queries~\citep{DBLP:conf/soda/EdenRS20} which again matched their proposed upper bound up to $\poly{\frac{\log{\fbrac{\vertexcount/\confidence}}}{\approxerror}}$ factors. Recently~\citep{DBLP:conf/approx/AssadiN22}, it has been shown that the algorithm of~\citep{assadi2018simple} is optimal (up to $\poly{\log{\fbrac{\vertexcount}}}$ factors) in terms of all parameters including $\approxerror$ and $\confidence$ through a lower bound of $\bigomega{\frac{\edgecount^{3/2}\log{\fbrac{1/\confidence}}}{\approxerror^2\numtriangle}}$ queries for obtaining an $\appcon$ estimate of $\numtriangle$ using \emph{local} and \randedgeq{} queries. However, this does not consider the dependency of the query complexity on arboricity of the graph. Like our upper bound, the lower bound of our work also captures the role of arboricity and the parameters $\approxerror$, and $\confidence$ when the queries allowed are the local and the \randedgeq{} queries.
\begin{theorem}[Lower Bound(Simplified)]
    Given a simple, unweighted and undirected graph $\graph = \fbrac{\vertexset,\edgeset}$ with $\size{\vertexset} = \vertexcount$, $\size{\edgeset} = \edgecount$ and arboricity $\arboricity$, and access to the graph via \degreeq{}, \neighbourq{}, \edgeexistsq{} and \randedgeq{} queries, obtaining an $\appcon$ estimate $\emptriangle$ of $\numtriangle$ requires $\bigomega{\frac{\edgecount\arboricity\log{\fbrac{1/\confidence}}}{\approxerror^2\numtriangle}}$ queries.
\end{theorem}
\begin{table}[ht!]
    \centering
    \begin{tabular}{|c|c|c|}
    \hline
                &  without \randedgeq{} & with \randedgeq{} \\
    \hline
    without         & UB: $\bigoted{\frac{\vertexcount}{\numtriangle^{1/3}}+\frac{\edgecount^{3/2}}{\numtriangle}}$ \citep{Dana_Ron_Triangle_Counting}& UB: $\bigot{\frac{\edgecount^{3/2}\log{\fbrac{1/\confidence}}}{\approxerror^2\numtriangle}}$ \citep{assadi2018simple}\\
    Arboricity  & LB: $\bigomegat{\frac{\edgecount^{3/2}\log{\fbrac{1/\confidence}}}{\approxerror^2\numtriangle}}$ \citep{DBLP:conf/approx/AssadiN22} & LB: $\bigomegat{\frac{\edgecount^{3/2}\log{\fbrac{1/\confidence}}}{\approxerror^2\numtriangle}}$ \citep{DBLP:conf/approx/AssadiN22}\\
    \hline
    with        & UB: $\bigot{\frac{\vertexcount}{\numtriangle^{1/3}}+\frac{\edgecount\arboricity\log^3{\fbrac{1/\confidence}}}{\approxerror^5\numtriangle}}$ \citep{DBLP:conf/soda/EdenRS20}& ($\ast$) UB:  $\bigot{\frac{\edgecount\arboricity\log{\fbrac{1/\confidence}}}{\approxerror^3\numtriangle}}$\\
    Arboricity  & ($\ast$) LB: $\bigomega{\frac{\edgecount\arboricity\log{\fbrac{1/\confidence}}}{\approxerror^2\numtriangle}}$ & ($\ast$) LB: $\bigomega{\frac{\edgecount\arboricity\log{\fbrac{1/\confidence}}}{\approxerror^2\numtriangle}}$\\
    \hline
    \end{tabular}
    \caption{Upper and Lower bounds for the Triangle Counting problem. The entries marked with asterisk ($\ast$) are our results proved in this paper.}
    \label{Table: Comparison Table}
\end{table}
% \vspace{-0.1in}
Table~\ref{Table: Comparison Table} describes the upper and lower bounds across different query accesses and parametrization for the triangle counting problem, with our contributions in this work highlighted with an asterisk ($\ast$) mark. We believe our work raises some new questions: (a) it opens up the question of quantifying the impact of $\arboricity$ to count other subgraphs (e.g., k-cliques) using \randedgeq{} and other local queries; (b) there remains a small gap of $1/\approxerror$ between our upper and lower bounds that remain to be resolved. 










%-----------------------------Related Work-----------------------------








\subsection{Related Work}
\label{ssec:related work}
The problem of triangle counting is a problem that has been extensively studied across various models of computation. 
\paragraph{RAM model.}
The work of~\citep{DBLP:journals/siamcomp/ChibaN85} proposed an $\bigo{\frac{\edgecount\arboricity}{\numtriangle}}$ algorithm in the RAM model that remains the best known till date. Recent works  have shown it to be conditionally optimal under the 3SUM~\citep{DBLP:conf/soda/KopelowitzPP16} and APSP hypotheses~\citep{DBLP:conf/focs/WilliamsX20}. 
\paragraph{Streaming model.}
In the data streaming model, a long line of work~\citep{DBLP:journals/ipl/PaghT12,DBLP:journals/pvldb/PavanTTW13} has culminated in an $\bigot{\frac{\edgecount}{\numtriangle}\fbrac{\edgesensitivity+\sqrt{\vertexsensitivity}}}$-space one pass streaming algorithm for insertion only model~\citep{KallaugherP17} that has been shown to be optimal through the combined lower bounds proposed in~\citep{BOV_13_Streaming_triangle_counting_hardness,KallaugherP17}. Here,  $\edgesensitivity$ and $\vertexsensitivity$  denotes the maximum number of triangles that an edge $\edge$ or a vertex $\vertex$ in the graph $\graph$ participates in, respectively. There have been works on triangle counting in multi-pass setup as well as in other streaming models (e.g. turnstile, cash-register etc.)~\citep{DBLP:conf/pods/McGregorVV16,DBLP:conf/stacs/BeraC17}.

\paragraph{Property Testing model.}
In query complexity setup, the triangle counting problem has been studied in terms of various query access available to the algorithm. It has been shown that without access to \edgeexistsq{}, no sublinear query algorithm can exist for triangle counting~\citep{GRS11}. Hence, sublinear query algorithms for counting triangles have been studied in models with \degreeq{}, \neighbourq{}, and \edgeexistsq{} queries~\citep{Dana_Ron_Triangle_Counting,DBLP:conf/soda/EdenRS20} resulting in an $\bigot{\min\fbrac{\frac{\vertexcount\arboricity^2}{\numtriangle},\frac{\vertexcount}{\numtriangle^{1/3}}+\frac{\edgecount\arboricity}{\numtriangle}}}$ query algorithm, with matching lower bounds~\citep{Dana_Ron_Triangle_Counting,DBLP:conf/approx/EdenR18,DBLP:conf/soda/EdenRS20} up to $\poly{\log\fbrac{\frac{\vertexcount}{\confidence}},\frac{1}{\approxerror}}$ factors. The work in~\citep{tetek:LIPIcs.ICALP.2022.107} highlighted the discrepancy in time and query complexities of approximate triangle counting while improving the time complexity of triangle counting for graphs with the number of triangles in a particular range. With \randedgeq{}, \neighbourq{}, and \edgeexistsq{} query access, an $\bigot{\frac{\edgecount^{3/2}\log\fbrac{1/\confidence}}{\approxerror^2\numtriangle}}$ query algorithm was proposed by~\citep{assadi2018simple} which was shown to be optimal in all parameters, inclusive of $\approxerror$ and $\confidence$~\citep{DBLP:conf/approx/AssadiN22}. 

\subsection{Paper organization}
Section~\ref{sec:tech-overview} discusses a broad overview of the techniques we use. Section~\ref{sec:prelim} sets up the preliminaries for our proofs, including the structural results involving arboricity of a graph. Section~\ref{sec:algo} discusses the query-based algorithm by breaking the discussion into several parts: a weight function (Section~\ref{ssec:weightfunc}), an oracle and its implementation (Sections~\ref{ssec:oracle-algo} and~\ref{ssec:oracle-implement}) and the final algorithm (Section~\ref{ssec:final-algo}). The lower bounds are discussed in Section~\ref{sec:lower-bound}. 






%-----------------------------Technical Overview-----------------------------








