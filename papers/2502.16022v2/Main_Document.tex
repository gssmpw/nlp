\documentclass{amia}
\usepackage{bm}
\usepackage{times}
\usepackage{helvet}
\usepackage{courier}
\usepackage{times}
\usepackage{latexsym}
\usepackage{amsmath}
\usepackage{algorithm}
\usepackage{algpseudocode}
\usepackage{url}
\usepackage{multirow}
\usepackage{enumitem}
\usepackage{array}
\usepackage[symbol]{footmisc}
\usepackage{threeparttable}
\usepackage{dblfloatfix} 
\usepackage{caption}
\usepackage[labelformat=simple]{subcaption}
\renewcommand\thesubfigure{(\alph{subfigure})}
\usepackage{adjustbox}
\usepackage{multirow}
\usepackage{hyperref}
\usepackage[dvipsnames]{xcolor}
\usepackage{float}
\usepackage{longtable}
\setlength{\bibsep}{0pt} %Comment out if you don't want to condense the bibliography

\usepackage{amsmath}
\usepackage{amssymb}
% \usepackage{subcaption}
\usepackage{tabularx}
\usepackage{makecell}
\usepackage{mathtools}
\DeclareMathOperator*{\argmin}{argmin}
\DeclareMathOperator*{\argmax}{argmax}
\newcolumntype{P}[1]{>{\centering\arraybackslash}p{#1}}
\usepackage{tabularx}
\usepackage[graphicx]{realboxes}
\usepackage[most]{tcolorbox}
\usepackage{soul} 
\setlength{\extrarowheight}{4pt}

\begin{document}


\title{Enhancing LLMs for Identifying and Prioritizing Important Medical Jargons from Electronic Health Record Notes Utilizing Data Augmentation}

% Enhancing Information Extraction of Important Medical Jargons from Medical Notes Utilizing Data Augmentation for Large Language Models

% \author{Firstname A. Lastname, MD, MPH$^1$, Firstname B. Lastname, MD, PhD$^2$ }

% \institutes{
%     $^1$ Institution, City, MA; $^2$Institution, City, CA
% }

\author{Won Seok Jang*, MSc$^1$, 
Sharmin Sultana*, BSc$^1$, 
Zonghai Yao*, MSc$^2$\\
Hieu Tran, BSc$^2$, 
Zhichao Yang, PhD$^2$, 
Sunjae Kwon, MSc$^2$, 
Hong Yu, PhD$^{1,3}$
}


\institutes{
    $^1$ Miner School of Computer and Information Sciences, UMass Lowell, MA, USA \\
    $^2$ Manning College of Information and Computer Sciences, UMass Amherst, MA, USA \\
    % $^3$ Department of Medicine, UMass Medical School, Worcester, MA, USA \\
    $^3$ Center for Healthcare Organization and Implementation Research, VA Bedford Health Care, MA, USA \\
    % $^*$ These authors contributed equally \\
}

\maketitle

These authors contributed equally $^*$: Won Seok Jang, Sharmin Sultana and Zonghai Yao

\textbf{Corresponding author:} Hong Yu, Ph.D. 

% Center for Biomedical and Health Research in Data Sciences, Miner School of Computer and Information Sciences, University of Massachusetts Lowell, MA, USA 

Phone: 1 978-934-3620  
Email: Hong\_Yu@uml.edu 


\section*{Abstract}

\textbf{Objective:}
OpenNotes allows patients to access their electronic health record (EHR) notes through online patient portals. However, EHR notes contain abundant medical jargon, which can be difficult for patients to comprehend. One way to improve comprehension is by reducing information overload and helping patients focus on the medical terms that matter most to them. In this study, we evaluated both closed-source and open-source Large Language Models (LLMs) for extracting and prioritizing medical jargon from EHR notes relevant to individual patients, leveraging prompting techniques, fine-tuning, and data augmentation.


\textbf{Materials and Methods:} 
We evaluated the performance of closed-source and open-source LLMs on a dataset of 106 expert-annotated EHR notes. We tested various combinations of settings, including: i) general and structured prompts, ii) zero-shot and few-shot prompting, iii) fine-tuning, and iv) data augmentation. To enhance the extraction and prioritization capabilities of open-source models in low-resource settings, we applied data augmentation using ChatGPT and integrated a ranking technique to refine the training process. Additionally, to measure the impact of dataset size, we fine-tuned the models by incrementally increasing the size of the augmented dataset from 10 to 10,000 and tested their performance. The effectiveness of the models was assessed using 5-fold cross-validation, providing a comprehensive evaluation across various settings. We report the F1 score and Mean Reciprocal Rank (MRR) for performance evaluation.

\textbf{Results and Discussions:}  
Among the compared strategies, fine-tuning and data augmentation generally demonstrated higher performance than other approaches. Although the highest F1 score of 0.433 was achieved by GPT-4 Turbo, the highest MRR score of 0.746 was observed with Mistral7B when data augmentation was applied. Notably, by using fine-tuning or data augmentation, open-source models were able to outperform closed-source models. Additionally, achieving the highest F1 score did not always correspond to the highest MRR score. We analyzed our experiment from several perspectives. First, few-shot prompting showed an advantage over zero-shot prompting in vanilla models. Second, when comparing general and structured prompts, each model exhibited different preferences. Third, fine-tuning improved zero-shot performance but sometimes degraded few-shot performance. Lastly, data augmentation yielded performance comparable to or even surpassing that of other strategies.

\textbf{Conclusion:} 
The evaluation of both closed-source and open-source LLMs highlighted the effectiveness of prompting strategies, fine-tuning, and data augmentation in enhancing model performance in low-resource scenarios.

\textbf{Keywords:} LLMs, Data Augmentation, EHR Comprehension, Patient Education, Patient Engagement


\textbf{Word count:} 3932

\newpage

\section*{Introduction}
Electronic Health Record (EHR) notes serve as valuable sources of information that can significantly benefit patients. Programs like OpenNotes~\cite{delbanco2010open} and the Blue Button~\cite{BlueButton2024} initiative empower patients by providing access to their EHR notes~\cite{delbanco2012inviting, gabay201721st, bajwa2021artificial, lye201821st, arvisais202221st, rodriguez2020digital, nutbeam2023artificial}.
Nevertheless, the benefits of accessing EHR notes can diminish significantly if patients do not comprehend their content~\cite{root2016characteristics, kayastha2018open, kujala2022patients, choudhry2016readability, khasawneh2022effect, rahimian2021open}.
EHR notes are lengthy and filled with medical jargon~\cite{zheng2017readability,zeng2007text,polepalli2013improving,sarzynski2017opportunities}, which can be difficult to comprehend for the average U.S. adult, whose reading ability is around the 7th to 8th-grade level~\cite{doak1996teaching,doak1998improving,walsh2008readability,eltorai2014readability,morony2015readability}.
Therefore, supportive technologies are needed to assist patients in understanding EHR content~\cite{johnson2016data, morid2016classification}, focusing on linking medical terms to lay-friendly terms~\cite{kandula2010semantic,zeng2007making,abrahamsson2014medical}, consumer-oriented definitions~\cite{polepalli2013improving}, and educational materials~\cite{zheng2016methods}.
Early studies have demonstrated that such interventions significantly enhance patient understanding~\cite{kandula2010semantic,polepalli2013improving}.
However, initial methods primarily relied on frequency- and context-based approaches to identify unfamiliar terms and propose simpler synonyms~\cite{kandula2010semantic,zeng2007making,abrahamsson2014medical}.
Identifying and extracting complex medical jargon from EHR notes is a crucial step toward improving patients' comprehension, ultimately enhancing patient engagement and reducing anxiety about their health~\cite{polepalli2013improving,chen2018natural, kwonMedJExMedicalJargon2022a,leroy2012improving}.
 

Notably, not all medical jargon extracted from EHR notes holds equal clinical importance~\cite{chenFindingImportantTerms2016a,chenUnsupervisedEnsembleRanking2017}.
Existing tools, such as MetaMap~\cite{aronson2006metamap}, ScispaCy~\cite{neumann2019scispacy}, medspaCy~\cite{medspacy}, and QuickUMLS~\cite{soldaini2016quickumls}, are effective at extracting medical terms—typically predefined terms from the Unified Medical Language System (UMLS)\cite{UnifiedMedicalLanguage}—but often fail to prioritize these terms based on their relevance to individual patients, treating all terms as equally important\cite{kandula2010semantic,zeng2007making,abrahamsson2014medical}.
In previous work, we asked physicians to identify medical jargon terms from EHR notes that are important to patients~\cite{chenFindingImportantTerms2016a}. Our results showed that physicians were able to consistently identify 5 to 10 medical jargon terms from each EHR note and rank each term based on its importance to patients~\cite{chenFindingImportantTerms2016a}.
Furthermore, we developed feature-rich traditional machine learning models (e.g., support vector machines) to identify such terms~\cite{chenFindingImportantTerms2016a}. However, our previous work did not focus on ranking jargon terms based on their importance to individual patients within an EHR note.

In this study, we propose large language model (LLM)-based NLP approaches to identify and rank jargon terms from EHR notes based on their importance to patients.
LLMs have demonstrated tremendous promise in biomedical NLP applications~\cite{tian2024opportunities,singhal2023large,singhal2023towards,tu2024towards,mcduff2023towards,wu2024pmc,chen2023meditron,tran2023bioinstruct,nori2023capabilities,kung2023performance,yang2024advancing,yang2023performance,yao2024medqa} due to their exceptional generalizability and performance.
However, applications in medical term extraction have primarily focused on tasks such as biomedical named entity recognition (BioNER)~\cite{hu2024improving,monajatipoorLLMsBiomedicineStudy2024a,hu2024zero}, rather than on prioritizing terms most relevant to patients, which is crucial for enhancing communication between patients and healthcare providers.
    
The key contributions of this paper are as follows:

\begin{itemize}
\item To the best of our knowledge, we are the first to conduct a comprehensive evaluation of both closed-source and open-source LLMs to assess their effectiveness in identifying medical jargon from EHR notes that are important for patients.
\item We developed a novel data augmentation technique by leveraging MIMIC discharge summaries to address the challenges of training in low-resource settings, resulting in significant performance improvements. Notably, our method enabled smaller open-source LLMs ($<$10B  parameters) to outperform much larger models from the GPT and Claude3 families.
\item In the discussion section, we provide an in-depth analysis of the results from both quantitative and qualitative perspectives, focusing on common strategies for improving LLM performance, such as zero-shot and few-shot learning, prompt engineering, scaling laws, domain-adaptive training, and data augmentation, and their impact on this specific task.
\end{itemize}


\section*{Related Work}

Identifying jargon terms important to patients is part of the biomedical Named Entity Recognition (BioNER) task, which involves identifying predefined entities in a text and labeling each token with the corresponding entity.
Medical entities encompass categories such as diseases, medications, treatments, lab tests, and more~\cite{liuEvaluatingMedicalEntity2024, boseSurveyRecentNamed2021}.
Studies such as~\cite{lee2020biobert,liu2019roberta,yao2023extracting,yao2023context} have introduced language models for BioNER tasks, while more recent studies~\cite{hu2024improving,monajatipoorLLMsBiomedicineStudy2024a,hu2024zero,gutierrez2022thinking,moradi2021gpt} have explored the application of large language models (LLMs) in BioNER.
However, BioNER tasks primarily focus on extracting entities without considering their importance and relevance to the personal needs of patients, which distinguishes them from our objective.

MedJex~\cite{kwonMedJExMedicalJargon2022a} fine-tunes pre-trained language models (PLMs), such as BERT~\cite{devlin2018Bert}, RoBERTa~\cite{liu2019roberta}, BioClinicalBERT~\cite{alsentzer2019publicly}, and BioBERT~\cite{lee2020biobert}, on a domain-specific corpus.
It leverages Wikipedia hyperlink spans during pretraining and transfers the learned weights to a target model fine-tuned on MedJ, an expert-annotated medical dataset.
More recent studies~\cite{lim2024large} have investigated whether large language models (LLMs), such as ChatGPT~\cite{openai_website}, can outperform baseline PLMs (e.g., MedJex~\cite{kwonMedJExMedicalJargon2022a} and SciSpacy~\cite{neumann2019scispacy}) in extracting personalized medical jargon.
Similarly, GAMedX~\cite{ghali2024gamedx}, a medical data extractor utilizing LLMs (Mistral 7B and Gemma 7B), employs chained prompts to navigate the complexities of specialized medical jargon.
Other works~\cite{butler2024jargon, mannhardt2024impact, lu2023napss} have demonstrated how LLMs can enhance the readability of EHR notes by extracting medical jargon.


This work also shares similarities with topic modeling, a task that extracts topics from input text.
Using unsupervised learning algorithms, topic modeling can identify both explicit and implicit themes within a text corpus~\cite{speierUsingPhrasesDocument2016, wenMiningHeterogeneousClinical2021, sunTopicModelingClinical2024}.
Through topic modeling, a text can be represented by multiple keywords or topics, which can then be incorporated into supervised models.
However, topic modeling heavily relies on term frequencies and may easily overlook important terms that are clinically relevant to individual patients.

Among the most relevant works, such as FOCUS~\cite{chenFindingImportantTerms2016a}, ADS~\cite{chenRankingMedicalTerms2017}, and FIT~\cite{chenUnsupervisedEnsembleRanking2017},
FOCUS~\cite{chenFindingImportantTerms2016a} employs MetaMap~\cite{aronson2010overview} to extract medical jargon from EHR notes and utilizes feature-rich learning-to-rank techniques to determine whether the terms are important.
%Leveraging feature engineering, both ADS~\cite{chenRankingMedicalTerms2017} and FIT~\cite{chenUnsupervisedEnsembleRanking2017} rank medical jargon terms based on their importance to patients using supervised and unsupervised approaches.
However, none of the previous works have identified and ranked medical jargon terms in a note-specific manner.
This is an important task, as ranking terms based on their relevance to a specific note may help the patient comprehend the note by linking important jargon terms to their lay definitions~\cite{yao2023readme}, or help 
%conducting patient post-discharge comprehension assessments~\cite{cai2023paniniqa}, or 
generate patient-friendly after-visit summaries~\cite{cai2022generation}.



\section*{Materials and Methods}

\subsection*{Overview}

We evaluated both closed-source and open-source LLMs for their efficacy in extracting key information from annotated medical notes, aiming to assess performance across different strategies. Figure \ref{fig:overview} provides an overview of our experiments, which leverage physician-annotated medical notes, closed- and open-source LLMs, and In-Context Learning (ICL). We examined the effects of various prompting styles, fine-tuning, and data augmentation to enhance model performance.

\begin{figure}[h!]
    \centering
    \includegraphics[width=\linewidth, clip]{fig2-overview.pdf}
    \caption{The evaluation workflow for closed and open-Source LLMs. We evaluate the performance of the LLMs in three distinctive settings. I. We assess the performance of closed- and open-source models by varying prompts and extraction tasks.
II. Next, we fine-tune the open-source models using the same variations.
III. Finally, we apply data augmentation to fine-tune the open-source LLMs and evaluate them under the same varying settings.
Performance is measured using F1 and MRR scores through 5-fold cross-validation.}
    \label{fig:overview}
\end{figure}

\begin{table}[b]
    \centering
    \begin{adjustbox}{width=0.45\linewidth}
    \begin{tabular}{|l|c|c|}
    \hline
        Main Diagnosis & Note Counts & Median Jargon Counts \\
        \hline
       Cancer  & 20 & 14 \\
       COPD  & 19 & 8 \\
       Diabetes  & 19 & 9 \\
       Hypertension  & 21 & 7\\
       Liver Failure  & 15 & 10 \\
       Heart Failure  & 12 & 9 \\
       \hline
    \end{tabular}
    \end{adjustbox}
    \caption{Gold-Standard Dataset Description. The dataset consists of 106 notes from patients diagnosed with Cancer, Chronic Pulmonary Obstructive Disease (COPD), Diabetes, Hypertension, Liver Failure and Heart Failure. The median jargon counts for each categories are illustrated.}
\label{tab:data_description}
\end{table}

\subsection*{Data source}
Our gold-standard dataset consists of 106 medical notes, each annotated by two physicians~\cite{chenFindingImportantTerms2016a, chenUnsupervisedEnsembleRanking2017}.
This EHR note dataset comprises text reports across six medical categories: Cancer, COPD, Diabetes, Heart Failure, Hypertension, and Liver Failure (Table \ref{tab:data_description}).
Each medical note includes detailed patient information and is accompanied by physician annotations highlighting the most critical terms or phrases relevant to the patient's health status.
Figure \ref{fig:sample_ehr} presents a snippet of a sample EHR note from the gold-standard dataset.


\subsection*{Experimental Setup}

\paragraph{Closed-Source and Open-Source LLMs}
We used both publicly available LLMs (open-source LLMs) and proprietary models that are not publicly available (closed-source LLMs).
The open-source LLMs included Mistral7B~\cite{jiangMistral7B2023}, BioMistral7B~\cite{labrakBioMistralCollectionOpenSource2024}, and Llama 3.1 8B~\cite{dubey2024llama}.
For proprietary models, we utilized six closed-source LLMs, three of which were from OpenAI~\cite{openai_website}: GPT-4 Turbo~\cite{openai2023gpt}, GPT-3.5 Turbo~\cite{ye2023comprehensive}, and GPT-4o Mini~\cite{gpt4o_mini_2023}.
Additionally, we conducted experiments using the Claude 3.0 models (Sonnet, Opus, and Haiku) developed by Anthropic~\cite{anthropic_claude_2024}.

\paragraph{Zero-shot vs Few-Shot} 
We used both zero-shot and few-shot prompts, also known as ICL, to compare the performance of the LLMs. For zero-shot prompting, we provided the model with general instructions, whereas for few-shot prompting, we included two examples randomly selected from the gold-standard note dataset.

\paragraph{General and Structured Prompts}
To evaluate the model's performance, we implemented two distinct types of prompts: general prompts and structured prompts, each designed to assess how effectively the model could extract relevant clinical information from medical notes.
For few-shot prompting, we included two annotated examples to guide the model in extracting and ranking terms effectively, whereas zero-shot prompts relied solely on instructions without prior examples.
Figure \ref{fig:specificprompt} presents an example of a structured prompt. In this approach, the prompts are explicitly designed to closely align with the original task defined in the gold-standard dataset.
The structured prompts instruct the LLMs to extract key medical conditions or diagnoses, followed by the relevant medications associated with those conditions, ensuring that the model replicates the annotation style of the gold-standard data.

In contrast, general prompts provide a more flexible and broader context, as illustrated in Figure \ref{fig:generalprompt}.
These prompts instruct the model to extract key medical terms without explicitly differentiating between conditions and medications, assigning the same base rank to both.
This approach allows the LLM to interpret the extraction task more broadly, offering insights into how well the model generalizes its understanding of medical terminology when provided with less specific guidance.


\begin{figure}[t!]
\centering
\resizebox{0.75\linewidth}{!}{
\begin{tcolorbox}[width=0.9\linewidth, title = An Excerpt from Annotated EHR Note]
    
Ms. XXX returns for follow-up of her \colorbox{Goldenrod}{rheumatoid arthritis} [1] and a history of B-cell lymphoma. Since her last visit, Ms. XXX continues to have variable days of pain related to her \colorbox{Goldenrod}{postherpetic neuralgia} [4]. She is frustrated by the persistent pain and wonders whether any of her medications for this is very helpful. She is interested in slowly tapering off her narcotics. As for her rheumatoid arthritis, she notes increasing symptoms in her hands and most recently, her knees and feet. She wonders if she should receive another dose of \colorbox{LimeGreen}{rituximab} [2.1]. 
She just saw Dr. Y for follow-up of her \colorbox{Goldenrod}{large cell lymphoma} [5]. She mentions that her labs revealed increased serum calcium, and she was told to hold her calcium supplements and vitamin D. As for her \colorbox{Goldenrod}{osteoporosis} [3], she had an appointment with Dr. Z recently but then decided to get \colorbox{LimeGreen}{denosumab} [2.2] here in our clinic. She is still concerned about the potential side effects of the medication.

Ms. XXX's \colorbox{Goldenrod}{rheumatoid arthritis} [1] symptoms have recently increased, particularly with the development of with \colorbox{Goldenrod}{synovitis} [2], particularly in her left knee. The timing from her last dose of \colorbox{LimeGreen}{rituximab} [2.1] is almost exactly a year. She and Dr. agree that it is reasonable to consider another infusion of this medication since she tolerated it well. The protocol would be for her RA rather than for her lymphoma. She is interested in receiving this through Dr. YY's office. 
\end{tcolorbox}
}
\caption{A sample EHR note where physicians identified important medical terms. Diagnoses/conditions are highlighted in yellow, while medications, tests and procedures associated with those diagnoses are marked in green, accompanied by their respective rankings}
\label{fig:sample_ehr}
\end{figure}




% \vspace{-5mm}

\paragraph{Fine-tuning with Low-Rank Adaptation (LoRA)} To improve the performance of open-source LLMs (Mistral7B, Biomistral7B, and Llama 3.1 8B), we conducted LoRA \cite{huLoRALowRankAdaptation2021} based parameter-efficient finetuning (PEFT). LoRA is an efficient fine-tuning technique that allows models to be adapted to specific tasks without the need to update all of the model's parameters. Instead, it applies low-rank updates to specific layers, reducing the computational cost and memory usage typically associated with traditional fine-tuning methods. The training was done with a batch size of $1$ per device and gradient accumulation over $128$ steps, and low-rank dimension was set to $64$. The learning rate was configured at $3e-4$, and the model was trained over $100$ epochs to allow the models to converge effectively on the task-specific patterns present in the dataset.


\paragraph{Data Augmentation with MIMIC-IV Discharge Notes} 
Annotation by domain experts is expensive, and data augmentation using AI-generated data can help alleviate this challenge~\cite{liTwoDirectionsClinical2023}.
ICL, or few-shot prompting, integrates task examples directly into input prompts, allowing models to observe patterns and generalize from limited examples to effectively handle new, unseen data~\cite{brown2020language}.
We created an augmented dataset using the ICL technique, which was then used to refine the models (Supplementary Figure~\ref{fig:query_gpt3.5}).
This augmented dataset was derived from discharge notes in the MIMIC-IV clinical database~\cite{johnsonMIMICIVFreelyAccessible2023}.
A subset of MIMIC discharge notes was randomly selected, and ChatGPT (GPT-3.5 Turbo)~\cite{openai_website} was used to process and rank key terms based on their importance for patient understanding.
This extraction process was guided by the ICL framework, where two annotated notes from our gold-standard dataset were provided as examples to instruct the model on identifying and prioritizing terms.
These examples ensured that the extracted terms were not only accurate but also contextually significant, reflecting the kind of information most beneficial for patient comprehension.
The resulting augmented dataset provided a broader training base for fine-tuning the open-source LLMs.

\paragraph{Determining the size of the augmented dataset} 
To explore the effects of augmented dataset size, we progressively increased the dataset across four scales: 10, 100, 1,000, and 10,000 notes.
By leveraging the original gold-standard annotations along with AI-generated augmented data, we conducted a comprehensive evaluation of model performance across varying data sizes.
Our objective was to enhance the robustness of LLMs in extracting critical medical information from EHRs to improve patient care.


\paragraph{Baseline Models}
We evaluated the performance of the open-source LLMs against several prominent baselines in the field of generative language models.
GPT-2~\cite{radford2019language}, a 1.5 billion-parameter unsupervised transformer model trained on 8 million unstructured data samples, is capable of performing various tasks without task-specific training.
BioGPT~\cite{luo2022biogpt}, a domain-specific GPT-2-based model trained on biomedical research articles, excels in biomedical question answering, data extraction, and text generation, demonstrating versatility in domain-specific downstream tasks.


\paragraph{Evaluation Metrics}

We used Precision (\ref{precision}) , Recall (\ref{recall}), F1 (\ref{f1}) and Mean Reciprocal Rank (MRR)(\ref{mrr}). We conduct 5-fold cross-validation to measure the performance and report the average score with the confidence intervals provided in the Appendix section.
Precision and recall gave insights into the model’s ability to cover the annotated terms. The macro F1 score allowed us to evaluate the model's balanced performance, whereas the MRR metric, in particular, evaluated how well the model ranked the terms according to their relevance, reflecting the model’s alignment with the annotated order of terms.   
We also used relaxed string matching to determine the true labels because exact matching strictly requires an exact string correspondence between extracted key phrases and the gold standard, which can significantly underestimate performance. \cite{turney2000learning, zeschApproximateMatchingEvaluating2009}.

    
\begin{center}
\begin{tabular}{p{6cm}p{6cm}}
\begin{equation}
    Precision = \frac{TP}{TP+FP}
    \label{precision}
\end{equation}
&
\begin{equation}
    Recall = \frac{TP}{TP+FN}
    \label{recall}
\end{equation}
\\
\begin{equation}
F1 = 2 \times (P\cdot R)/(P+R)
\label{f1}
\end{equation}
&
\begin{equation}
MRR = \frac{1}{|Q|}\sum_{i=1}^{|Q|}\frac{1}{rank_i}
\label{mrr}
\end{equation}
\end{tabular}
\end{center}

\paragraph{Hardware Settings} All experiments were performed with two Nvidia A100 GPUs, each with 40 GB of memory, an Intel Xeon Gold 6230 CPU, and 192 GB of RAM.

\section*{Experimental Results}
Table \ref{tab:performance} presents the results from both closed- and open-source models, with confidence intervals provided in Supplementary Table \ref{tab:f1_mrr}.
The highest F1 score, 0.433 (CI 95%: 0.448-0.418), was achieved in the top 10 with few-shot prompts using the GPT-4 Turbo model.
The highest MRR, 0.746 (CI 95%: 0.762-0.730), was observed in the top 3 with few-shot prompts using the Mistral7B-MIMIC-FT model.
For both F1 and MRR, the highest scores were achieved using few-shot prompts.
Notably, in vanilla models—whether closed- or open-source—few-shot prompts consistently outperformed zero-shot prompts in terms of F1, regardless of the top-N extraction setting.
Most of our results were statistically significant ($p<0.05$) (Supplementary Table \ref{tab:f1_mrr}); however, this trend was weaker for MRR.
Interestingly, in many settings, the highest scores were obtained from open-source models.
Furthermore, with fine-tuning or data augmentation, some open-source models achieved higher F1 scores than closed-source models in zero-shot prompts.

% 0.9\textheight
\renewcommand{\arraystretch}{1.0}
\begin{table}[H]
  \centering
    \normalsize   % Use a standard font size for readability
    \color{black}
    % \begin{adjustbox}{angle=270, width=0.8\textwidth, totalheight=0.9\textheight}
    \Rotatebox{0}{
    \Resizebox{\textwidth}{!}{
    \begin{tabular}{|l|l|cc|cc|cc|cc|cc|cc|}
        \hline
        \multirow{3}{*}{\textbf{Models}} & \multirow{3}{*}{\textbf{Prompt}} & \multicolumn{6}{c|}{\textbf{Zero-Shot}} & \multicolumn{6}{c|}{\textbf{Few-Shot}} \\
        \cline{3-14}
         &  & \multicolumn{2}{c|}{\textbf{top3}} & \multicolumn{2}{c|}{\textbf{top5}} & \multicolumn{2}{c|}{\textbf{top10}} & \multicolumn{2}{c|}{\textbf{top3}} & \multicolumn{2}{c|}{\textbf{top5}} & \multicolumn{2}{c|}{\textbf{top10}} \\
        \cline{3-14}
         &  & \textbf{F1} & \textbf{MRR} & \textbf{F1} & \textbf{MRR} & \textbf{F1} & \textbf{MRR} & \textbf{F1} & \textbf{MRR} & \textbf{F1} & \textbf{MRR} & \textbf{F1} & \textbf{MRR} \\
        \hline
        \multicolumn{14}{|c|}{\textbf{Close-Source LLMs}} \\
        \hline
       
        GPT-4 Turbo & general & 0.306 &	0.562  &	0.359 &	0.601  &	\textbf{0.433 } &	0.640  &	0.324  &	0.600  &	0.368  &	0.626  &	0.433 	& 0.621  \\
        & structured &  0.317  &	0.643  &	0.347  &	0.668  &	0.370  &	0.691  & 0.362 &	0.673  &	\textbf{0.398} &	0.705  &	0.424  &	0.691   \\
        \hline
        GPT-4o Mini & general &  0.320  &	0.617  &	0.355  &	0.667  &	0.392  &	0.691 	& 0.329  &	0.664  &	0.369  &	0.700  &	0.383  &	0.708   \\
        & structured &  0.303  &	0.620  &	0.329  &	0.617  &	0.335  &	0.610  &	0.328  &	0.666  &	0.363  &	0.668  &	0.381  &	0.684   \\
        \hline 
        GPT-3.5 Turbo & general & 0.261  &	0.478  &	0.308  &	0.582  &	0.324  &	0.602  &	0.343  &	0.649  &	0.359  &	0.655 	& 0.378  &	0.669  \\
        & structured &  0.303  &	0.575  &	0.329  &	0.573  &	0.343  &	0.572  &	0.354 	& 0.661  &	0.367  &	0.633  &	0.398 	& 0.661   \\
        \hline
        Claude 3.0 Sonnet & general &  0.289  &	0.554 	& 0.328  &	0.556  &	0.355 	& 0.557 	& 0.308 	& 0.596  &	0.345  &	0.598  &	0.399 	& 0.639   \\
        & structured &  0.304  &	0.666  &	0.317 	& 0.673  &	0.323  &	0.668  &	0.354  &	0.683  &	0.357  &	0.684  &	0.371  &	0.692  \\
        \hline
        Claude 3.0 Haiku & general &  0.284  &	0.613 	& 0.320  &	0.629  &	0.345 	& 0.620  &	0.320  &	0.672  &	0.358  &	0.690  &	0.387  &	0.683  \\
        & structured &  0.248  &	0.584  &	0.271  &	0.591 	& 0.281  &	0.614  &	0.357 	& 0.668 	& 0.364  &	0.715  &	0.349 	& 0.693   \\
        \hline
        Claude 3.0 Opus & general &  0.340  &	\textbf{0.707} 	& 0.365 	& \textbf{0.698}  &	0.390  & \textbf{0.713}  &	0.316  &	0.615 	& 0.359 	& 0.661  &	\textbf{0.412}  &	0.709   \\
        & structured &  0.289  &	0.617  &	0.344  &	0.643  &	0.387 	& 0.664  &	0.319  &	0.646  &	0.331  &	0.551 	& 0.387  &	0.680   \\
        \hline
        \multicolumn{14}{|c|}{\textbf{Open-Source LLMs}} \\
        \hline
        Mistral7B & general & 0.293  & 0.527  & 0.337  & 0.563 	& 0.363  & 0.557   & 0.303  & 0.582 	&0.348  & 0.579 	& 0.394  & 0.565  \\
        & structured & 0.252  &	0.572  &	0.268 & 0.552 	& 0.276 	&	0.539  & \textbf{0.359}  &	0.669 	& 0.375 	& 0.677  & 0.402 	&	0.694    \\
        \hline
        Mistral7B-FT & general &  0.342  &	0.603  &	0.350  &	0.631  &	0.332 	& 0.661  &	0.287 	& 0.418  &	0.325 	& 0.536 	& 0.355 	& 0.603   \\
        & structured & 0.291 	& 0.557 	& 0.310 	& 0.570 	& 0.321 	& 0.544  & 0.366 	& 0.677 	& 0.370 	& 0.692 	& 0.388 	& 0.715   \\
        \hline
        Mistral7B-MIMIC-FT & general &  0.328  &	0.539  &	0.336  &	0.512 	& 0.340 	& 0.488 	& 0.340  &	\textbf{0.746}  &	0.358 	& \textbf{0.731} 	& 0.380  &	\textbf{0.713 } \\
        & structured & 0.336  &	0.524  &	0.339 	& 0.508  &	0.390  &	0.542  &	0.351 	& 0.681  &	0.367  &	0.674 	& 0.389 	& 0.676   \\
        \hline
        Llama3.1-8B & general &  \textbf{0.358}  &	0.585  &	\textbf{0.376 }	& 0.588 	& 0.413  &	0.611  &	0.313  &	0.585 	& 0.370  &	0.644 	& 0.399  &	0.652   \\
        & structured &  0.282 	& 0.572 	& 0.296  &	0.569  &	0.299  &	0.576  &	0.325  &	0.589  &	0.297  &	0.437 	& 0.354  &	0.594   \\
        \hline
        Llama3.1-8B-FT & general &  0.343  &	0.605 	& 0.373  &	0.593 	& 0.404 	& 0.621 	& 0.312 	& 0.589  &	0.360  &	0.620  &	0.403 	& 0.624   \\
        & structured &  0.289 	& 0.558  &	0.301 	& 0.571  &	0.294  &	0.575  &	0.332  &	0.589  &	0.353  &	0.630 	& 0.386  &	0.649   \\
        \hline
        Llama3.1-8B-MIMIC-FT & general & 0.355  &	0.605  &	0.369  &	0.596 	& 0.400 	& 0.614  &	0.311  &	0.597 	& 0.363 	& 0.624  &	0.399 	& 0.643  \\
        & structured &  0.296  &	0.568 	& 0.304  &	0.566  &	0.295	& 0.557 	& 0.316 	& 0.591 	& 0.346  &	0.597 	& 0.381  &	0.637   \\
        \hline
        BioMistral7B & general & 0.165 	&0.250 	& 0.212 	& 0.433 	& 0.232 	& 0.423  & 0.228 	& 0.529 	& 0.227 	& 0.510 	& 0.232 	& 0.510  \\
        & structured & 0.245  & 0.452  & 0.297  & 0.586  & 0.267 	& 0.548  & 0.262  & 0.500 	& 0.321 	& 0.567 	& 0.288 	& 0.529  \\
        \hline
        BioMistral7B-FT & general & 0.254  & 0.439 & 0.272 & 0.455 & 0.165  & 0.231  & 0.273 & 0.460 & 0.286 & 0.462 & 0.311 & 0.468  \\
        & structured & 0.287 	& 0.436 	& 0.361 	& 0.542  & 0.306 	& 0.594  & 0.353 	& 0.563 	& 0.360 	& 0.563 	& 0.264 	& 0.483  \\
        \hline
        Biomistral7B-MIMIC-FT & general & 0.279  & 0.499  & 0.306  & 0.490  & 0.351  & 0.500   & 0.260  & 0.480   & 0.259   & 0.481   & 0.277   &  0.449  \\
        & structured & 0.287 	& 0.549 	& 0.306 	& 0.568 	& 0.332 	& 0.569  & 0.283 	& 0.551 	& 0.304 	& 0.544 	& 0.348 	& 0.565 \\
        \hline
    \end{tabular}
    % \end{adjustbox}
    }
    }
    \caption{We compared the performance of different models on zero-shot and few-shot tasks using top-3, top-5, and top-10 results for F1 and MRR across both closed- and open-source LLMs. The evaluation was conducted under three distinct settings:
I. Vanilla models, including both closed- and open-source models.
II. Models fine-tuned on a portion of the dataset.
III. Models fine-tuned on 10,000 augmented data points generated from MIMIC-IV discharge notes. In many cases, open-source models continued to outperform closed-source models.
The specific details of our experiments are provided in the Materials and Methods section. The complete scores, along with their 95\% confidence intervals, are presented in Supplementary Table \ref{tab:f1_mrr}.}
    \label{tab:performance}
\end{table}


% Data Augmentation performance
The highest-performing model that utilized our data augmentation strategy was Llama 3.1-8B-MIMIC-FT, achieving an F1 score of 0.400 (CI 95\%: 0.414-0.387) in the top-10 medical jargon extractions using zero-shot prompts.
The highest MRR score, 0.746 (CI 95\%: 0.762-0.730), was achieved by Mistral7B-MIMIC-FT in the top-3 medical jargon extractions using few-shot prompts.
However, neither the F1 nor the MRR score achieved the highest performance among all models that used the data augmentation strategy.
In most cases, the F1 scores indicated that the data augmentation approach led to higher performance compared to vanilla models, and in some instances, even surpassed the performance of fine-tuned models.
However, the performance differences varied depending on the prompt settings.
These findings will be further discussed in the discussion section.

\begin{table}[H]
\centering
\resizebox{\textwidth}{!}{
\begin{tabular}{|l|cc|cc|cc|}
\hline
 & \multicolumn{2}{c}{top3} & \multicolumn{2}{c}{top5} & \multicolumn{2}{c}{top10} \\
 \hline
 & F1 & MRR & F1 & MRR & F1 & MRR \\
 \hline
 GPT2\cite{openai2023gpt} & 0.097 (0.108/0.087) & 0.191 (0.208/0.175) & 0.094 (0.105/0.083) & 0.193 (0.212/0.174) & 0.099 (0.116/0.081) & 0.170 (0.197/0.142) \\
BioGPT\cite{luo2022biogpt} & 0.145 (0.163/0.126) & 0.192 (0.235/0.149) & 0.169 (0.197/0.141) & 0.196 (0.212/0.180) & 0.174 (0.202/0.146) & 0.166 (0.219/0.112) \\
\hline
\end{tabular}
}
\caption{Performance of Baseline models}
\label{tab:baseline}
\end{table}

Table \ref{tab:baseline} presents the performance of the baseline models, which include commonly used language models (LMs) for medical information extraction.
As shown in Table \ref{tab:baseline}, BioGPT~\cite{luo2022biogpt} achieved the highest F1 score of 0.174 (95\% CI: 0.202-0.146) and the highest MRR score of 0.196 (95\% CI: 0.212-0.180) among the baseline models.
However, these values are significantly lower than those achieved by the benchmark models, suggesting that LLMs generally outperform smaller models in downstream tasks.

\begin{table}[H]
    \centering
    \large % Adjusts font size for the table; try \small or \footnotesize if too small
    \begin{adjustbox}{width=1.0\textwidth} % Adjust width here as needed
    \begin{tabular}{|l|l|cc|cc|cc|cc|cc|cc|}
        \hline
        \multirow{3}{*}{\textbf{Models}} & \multirow{3}{*}{\textbf{Prompt}} & \multicolumn{6}{c|}{\textbf{Zero-Shot}} & \multicolumn{6}{c|}{\textbf{Few-Shot}} \\
        \cline{3-14}
        & & \multicolumn{2}{c|}{\textbf{top3}} & \multicolumn{2}{c|}{\textbf{top5}} & \multicolumn{2}{c|}{\textbf{top10}} & \multicolumn{2}{c|}{\textbf{top3}} & \multicolumn{2}{c|}{\textbf{top5}} & \multicolumn{2}{c|}{\textbf{top10}} \\
        & & \textbf{F1} & \textbf{MRR} & \textbf{F1} & \textbf{MRR} & \textbf{F1} & \textbf{MRR} & \textbf{F1} & \textbf{MRR} & \textbf{F1} & \textbf{MRR} & \textbf{F1} & \textbf{MRR} \\
        \hline
        % \multicolumn{14}{|c|}{\textbf{Open-Source LLMs}} \\
        % \hline
        Mistral7B-10-MIMIC-FT & general &  0.261  &	0.492  & 0.289  &	0.486  & 0.291 &	0.463  & 0.288  &	0.613  & 0.287  &	0.577  & 0.327  &	0.529 \\
        & structured & 0.299  &	0.527  & 0.309  &	0.528  & 0.332  &	0.552  & 0.294 	& 0.573  & 0.316 	& 0.557  & 0.337  &	0.549  \\
        Mistral7B-100-MIMIC-FT & general & 0.319  &	0.527  & 0.315  & 0.590  & 0.323  &	0.524  & 0.346  &	0.691  & 0.401  &	0.700  & 0.405  &	0.639 \\
        & structured &  0.310 	& 0.459 & 0.335  &	0.484  & 0.379  &	0.486  & 0.338 	& 0.633  & 0.366  &	0.665  & 0.405  &	0.673 \\
        Mistral7B-1000-MIMIC-FT & general & 0.331  & 0.531  & 0.348  &	0.516  & 0.336  &	0.465  & 0.336  & 0.746  & 0.358 &	0.702  & 0.367 &	0.726  \\
        & structured & 0.311 	& 0.484  & 0.341  &	0.481  & 0.398 	& 0.535  & 0.348  &	0.693  & 0.377  &	0.672  & 0.381  &	0.672  \\
        
        \hline
        Llama3.1-8B-10-MIMIC FT & general & 0.335  &	0.602   &	0.381 	& 0.583   & 0.395 & 0.610  & 0.304 & 0.610  & 0.352	& 0.614  & 0.394 	& 0.634 \\
        & structured & 0.288  & 0.559  & 0.300  &	0.578  & 0.304  &	0.580  & 0.331  &	0.618  & 0.358  & 0.626  & 0.379 	& 0.632 \\
        Llama3.1-8B-100-MIMIC-FT & general & 0.359  &	0.601  & 0.378  &	0.579  & 0.400 	& 0.606  & 0.307 & 0.615  & 0.363 	& 0.609  & 0.399 & 0.624 \\
        & structured & 0.287  &	0.570  & 0.301  &	0.576  & 0.295 &	0.560  & 0.324  &	0.592  & 0.355  &	0.636  & 0.376  &	0.648 \\
        Llama3.1-8B-1000-MIMIC-FT & general & 0.350  &	0.595  & 0.369  &	0.574  & 0.397 	& 0.601  & 0.312 	& 0.619  & 0.351  &	0.615  & 0.400  &	0.621 \\
        & structured & 0.297  & 0.564  & 0.301  &	0.587   & 0.311 & 0.581  & 0.323 	& 0.601  & 0.362  &	0.620  & 0.389  &	0.632 \\
        
        \hline
        BioMistral7B-10-MIMIC-FT & general & 0.270 & 0.459 & 0.295 & 0.453 & 0.321 & 0.464 & 0.241 & 0.473 & 0.257 & 0.488 & 0.276 & 0.464   \\
        & structured & 0.300 & 0.539 & 0.302 & 0.535 & 0.301 	& 0.517 & 0.213 & 0.403 & 0.237 & 0.445 & 0.241 & 0.427 \\
        BioMistral7B-100-MIMIC-FT & general & 0.262 & 0.500 & 0.299 & 0.522 & 0.321 & 0.464 & 0.264 & 0.481 & 0.305 & 0.506 & 0.342  & 0.495  \\
        & structured & 0.311 & 0.531 & 0.339 & 0.535 & 0.360 & 0.558 & 0.277 & 0.502 & 0.303 & 0.501 & 0.335 & 0.531 \\
        BioMistral7B-1000-MIMIC-FT & general & 0.284 & 0.504 & 0.308 & 0.492 & 0.355& 0.501 & 0.303 & 0.553 & 0.335 & 0.579 & 0.347 & 0.565    \\
        & structured &  0.259 & 0.496 & 0.295 & 0.516 & 0.348 & 0.528 & 0.218 & 0.405 & 0.254 & 0.471 & 0.282 & 0.496\\
        \hline
    \end{tabular}
    \end{adjustbox}
    \caption{Performance comparison between different open-source LLMs trained on different sizes of augmented data. We scale from 10 to 10,000 by multiplying 10. The results for using 10,000 are in Table~\ref{tab:performance}. The F1 and MRR tends to increase when the size of the augmented dataset increases.}
    \label{tab:aug_data_performance}
\end{table}

We also examined whether increasing or decreasing the size of the augmented dataset impacts the model's overall performance (Table \ref{tab:aug_data_performance}).
The highest F1 score of 0.405 was achieved by Mistral 7B when utilizing 100 augmented datasets with both general and specific prompts.
The highest MRR score, 0.746, was observed from Mistral 7B fine-tuned on 10,000 augmented data points using a general prompt.
A general trend was observed, indicating that utilizing larger augmented datasets results in higher performance compared to models trained on smaller datasets.
As the dataset size increased, both the F1 and MRR scores improved; however, this trend was more pronounced for F1 than for MRR.
In some cases, such as BioMistral7B in top-3 medical jargon extraction with zero-shot prompts, higher MRR scores were achieved with smaller dataset sizes.

\section*{Discussion}



\paragraph{Zero-shot vs. Few-shot} 

When it comes to prompting strategies, across both closed- and open-source models, few-shot prompts consistently outperformed zero-shot prompts in extracting relevant medical information (Table \ref{tab:performance}).
As illustrated in Figure \ref{fig::zeroshotvfewshot}, the choice of prompting strategy can lead to different outputs, resulting in varying performance outcomes.
This pattern aligns with findings from various studies, which have observed that few-shot prompts allow models to leverage examples that enhance contextual understanding and response specificity.
For example, \cite{brown2020language} demonstrated that few-shot prompting improves the performance of language models in complex tasks by providing contextual cues, enabling the model to better interpret nuanced queries—particularly in medical and technical domains where specificity is critical.
However, in cases where the models were fine-tuned, these patterns began to diminish, with zero-shot prompts showing comparable or even higher scores than few-shot prompts.
This can be attributed to instruction fine-tuning, which enhances a model's ability to handle unseen data.
As noted by \cite{weiFinetunedLanguageModels2022}, instruction fine-tuning improves the performance of LLMs in zero-shot tasks by equipping them with a better understanding of task-specific instructions.

\begin{figure}[t]

\begin{tcbraster}[raster columns=3, raster equal height, raster width=\linewidth]

\AtBeginEnvironment{tcolorbox}{\small}
\begin{tcolorbox}[title=True labels, halign=left]{\small}
\sethlcolor{pink} 
\hl{1. tinnitus} \\
\hl{1. whipple procedure} \\
\sethlcolor{Melon}
\hl{2. insulin-dependent diabetic} \\
2. simvastatin \\
\sethlcolor{Yellow}
\hl{2. retinopathy} \\
2. microalbumin \\
3. colon adenomas \\
3. colonoscopy

\end{tcolorbox}
\begin{tcolorbox}[title=Mistral7B Zero-shot, halign=left]
1. history of pancreatic neoplasm \\
1. contrast ct of the abdomen and pelvis \\
1. ca-19-9 blood test \\
\sethlcolor{Melon}
\hl{2. diabetes} \\
2. a1c blood test \\
2. microalbuminuria urine test \\
3. dyslipidemia \\
3. lipid profile blood test \\
3. lfts (liver function tests) blood test 
\end{tcolorbox}
\begin{tcolorbox}[title=Mistral7B Few-shot, halign=left]
\sethlcolor{pink}
\hl{1. tinnitus}\\
2. low-grade pancreatic neoplasm\\
\sethlcolor{pink}
\hl{2. whipple procedure with pancreatectomy and splenectomy}\\
2. bile duct stricture requiring reconstruction \\
\sethlcolor{Melon}
\hl{3. insulin-dependent diabetes} \\
3. insulin pump \\
\sethlcolor{Yellow}
\hl{3. retinopathy} \\
\end{tcolorbox}

\end{tcbraster}
\vspace{0.5em} % Adjust spacing as needed
\caption{Case Study for Extracting the Top 3 Important Medical Jargons from Zero-shot and Few-shot Prompts in Mistral 7B. The few-shot prompting strategy demonstrates greater robustness in vanilla models compared to zero-shot prompting.
The highlighted jargons represent terms that overlap with the expert-annotated labels, emphasizing the alignment between the model's outputs and domain experts' annotations.}
\label{fig::zeroshotvfewshot}
\end{figure}

\paragraph{General vs. Structured Prompts} The effectiveness of prompts can vary across models, with certain prompt styles enhancing performance for specific models (Table \ref{tab:performance}). While differences between models were generally minimal, some models performed better with particular prompt styles. Notably, for the Llama3.1 8B model, general prompts outperformed structured prompts in F1 and ranking prediction. Whereas in Mistral7B, structured prompts show improved performance to general prompts. This aligns with prior research suggesting that tailored prompts can optimize a model's performance in specialized tasks \cite{liuWhatMakesGood2021}. Thus, testing diverse prompt sets is essential to maximize model-specific performance.

\begin{figure}[!ht]
% \newlength\myboxwidth
% \setlength{\myboxwidth}{\dimexpr\textwidth-20\fboxsep}

\begin{tcbraster}[raster columns=3, raster equal height, raster width=\linewidth]
% \tcbset{width=(\linewidth-2mm)/2, box align=top, before=,after=\hfill}

\AtBeginEnvironment{tcolorbox}{\small}
\begin{tcolorbox}[title=True Labels, halign=left]{\small}
\sethlcolor{pink} 
\hl{1. tinnitus} \\
\hl{1. whipple procedure} \\
\sethlcolor{Melon}
\hl{2. insulin-dependent diabetic} \\
2. simvastatin \\
2. retinopathy \\
2. microalbumin \\
\sethlcolor{Yellow}
\hl{3. colon adenomas} \\
\hl{3. colonoscopy} \\
\sethlcolor{YellowGreen}
\hl{4. gerd} \\
4. omeprazole \\
5. erectile dysfunction \\
5. viagra

\end{tcolorbox}
\begin{tcolorbox}[title=BioMistral7B, halign=left]
\sethlcolor{pink}
\hl{1. tinnitus of many years' duration.} \\
1. audiometry. \\
1. ct scan of the temporal bone. \\
2. low-grade pancreatic neoplasm.  \\
2. ct scan of the abdomen and pelvis.  \\
2. endoscopic ultrasound. \\
\sethlcolor{Melon}
\hl{3. insulin-dependent diabetes.}  \\
3. fasting blood glucose.  \\
3. hba1c.  \\
3. c-peptide. \\ 
...

\end{tcolorbox}
\begin{tcolorbox}[title=BioMistral7B fintuned, halign=left]
\sethlcolor{pink}
\hl{1. tinnitus of many years' duration.} \\
2. low-grade pancreatic neoplasm. \\
\sethlcolor{Melon}
\hl{3. insulin-dependent diabetes.} \\
4. Dyslipidemia. \\
5. history of high blood pressure. \\
6. family history of brca gene. \\
\sethlcolor{Yellow}
\hl{7. multiple colon adenomas by colonoscopy.} \\
8. sleep apnea. \\
\sethlcolor{YellowGreen}
\hl{9. Gerd.}\\
10. erectile dysfunction.
\end{tcolorbox}

\end{tcbraster}
\vspace{0.5em} % Adjust spacing as needed
\caption{Case Study for extracting Top 5 important medical jargons from BioMistral7B and BioMistral7B that was finetuned on some of the samples of the gold labeled dataset in Zero-shot prompt settings. The finetuned model shows more robustness than vanilla models, especially in Zero-shot prompts. The highlighted jargons are the ones that overlap with the expert-annotated labels. }
\label{fig::finetune}
\end{figure}

\paragraph{Finetuning LLMs} 
Fine-tuning LLMs using domain-specific data proved effective in enhancing model performance, albeit with some limitations (Table \ref{tab:performance}).
One possible explanation for this result is that the size of the datasets used for fine-tuning was too small to yield substantial performance gains.
Increasing the size of the fine-tuning dataset is likely to further improve the model's performance~\cite{vieiraHowMuchData2024}.
Additionally, we observed that zero-shot prompts outperformed few-shot prompts, particularly in fine-tuned scenarios.
Figure \ref{fig::finetune} illustrates the impact of fine-tuning: while vanilla models show limitations in extracting key points using prompts alone, fine-tuning enables the models to learn relevant patterns, leading to improved performance.
Previous research has similarly shown that fine-tuning on domain-specific data can significantly enhance performance by adjusting model weights to reflect the unique characteristics of the target domain, such as better handling of abbreviations, acronyms, and clinically relevant contexts~\cite{lee2020biobert}.
Moreover, instruction fine-tuning has been shown to improve the model's zero-shot performance~\cite{weiFinetunedLanguageModels2022}.
However, as the number of extracted terms increases, the performance gap between vanilla and fine-tuned models tends to narrow.

\paragraph{Data Augmentation} 
We explored the impact of data augmentation using various sizes of MIMIC-IV~\cite{johnsonMIMICIVFreelyAccessible2023} discharge notes to enhance the robustness of LLMs (Tables \ref{tab:performance} and \ref{tab:aug_data_performance}).
Data augmentation simulated diverse scenarios within medical notes, enabling the LLM to generalize better across a broader range of note types.
Similar to fine-tuning, the performance gains from data augmentation were marginal in few-shot prompt settings. However, in zero-shot prompts, the performance gain was significant, even exceeding the performance of models that used the fine-tuning strategy ($p<0.05$).
As shown in Figure \ref{fig::augmentation}, by utilizing only the augmented dataset, we achieved performance levels similar to those of fine-tuned models, and in some cases, even higher scores.
These findings align with prior studies~\cite{weiFinetunedLanguageModels2022, tangDoesSyntheticData2023}, which argue that using LLM-generated data can improve model performance on downstream tasks.
Additionally, studies have shown that while data augmentation provides benefits, substantial improvements often require a high degree of variation in the augmented data~\cite{weiEDAEasyData2019, chenEmpiricalSurveyData2021}.
In our case, data augmentation provided additional exposure to diverse linguistic patterns in medical notes. However, the approach alone was insufficient to dramatically outperform its counterparts, possibly due to the limited number of examples used in generating the augmented datasets. \newline

\begin{figure}[!ht]

\begin{tcbraster}[raster columns=3, raster equal height, raster width=\linewidth]

\AtBeginEnvironment{tcolorbox}{\small}
\begin{tcolorbox}[title=True labels\\, halign=left]{\small}
\sethlcolor{pink}
\hl{1. nonischemic cardiomyopathy} \\
\hl{1. persantine thallium} \\
1. ejection fraction \\
1. lad ischemia \\
\hl{1. heart catheterization} \\
1. carvedilol \\
2. trigger fingers \\
2. hand clinic \\
2. mri \\
… \\
\sethlcolor{Melon}
\hl{5. gerd}

\end{tcolorbox}
\begin{tcolorbox}[title=Llama 3.1 8B finetuned \\, halign=left]
\sethlcolor{pink}
\hl{1. nonischemic cardiomyopathy}\\
\hl{1. persantine thallium} \\
1. echocardiogram \\
\sethlcolor{Melon}
\hl{2. gerd} \\
2. egd \\
3. fatty liver \\
3. hemochromatosis gene analysis \\
4. sinus polyposis \\
5. asthma 

\end{tcolorbox}
\begin{tcolorbox}[title=Llama 3.1 8B finetuned on augmented dataset, halign=left]
\sethlcolor{pink}
\hl{1. nonischemic cardiomyopathy} \\
\sethlcolor{Melon}
\hl{2. gerd} \\
3. fatty liver \\
4. hives \\
5. sinus polyposis \\
\sethlcolor{pink}
\hl{1. persantine thallium} \\
\hl{1. heart catheterization} \\
1. chest x-rays \\
1. iron saturation \\
1. echo \\
…
\end{tcolorbox}

\end{tcbraster}
\vspace{0.5em} % Adjust spacing as needed
\caption{Case Study for extracting Top 5 important medical jargons from Llama3.1 8B finetuned and Llama 3.1 8B that was finetuned on the MIMIC-IV augmented dataset in Zero-shot prompt settings. In many cases, augmented models shows comparable performance than vanilla models and finetuned models, especially in Zero-shot prompts. The highlighted jargons are the ones that overlap with the expert annotated labels.}
\label{fig::augmentation}
\end{figure}

\paragraph{Incrementing the size of augmented dataset} 
As demonstrated in Table \ref{tab:aug_data_performance}, increasing the size of the augmented dataset improves the overall performance of open-source LLMs, regardless of the model. This result aligns with findings from \cite{yuanRealFakeEffectiveTraining2024,kimSyntheticDataImprove2024}, where increasing the size of synthetic datasets significantly enhanced model performance compared to models trained on smaller real-world datasets.
However, we observed that MRR scores did not improve as much as F1 scores, indicating that ranking important information remains a non-trivial task for LLMs.
As shown in Figure~\ref{fig::augmentation}, while the model effectively captures important terms in the medical text, it still struggles to match the rankings as defined in the true labels.
One possible explanation is the lower quality of the LLM-augmented dataset.
Since we only used two examples for ICL and did not implement any filtering techniques to ensure the quality of the curated dataset, this may have affected the overall quality of the augmented dataset.
We aim to address this limitation and improve performance in this aspect in future studies.

Our research demonstrated that mimicking expert-level human annotation is a non-trivial task, even for LLMs.
While LLMs perform well in summarization and paraphrasing tasks, their recall drops significantly when it comes to identifying and prioritizing domain specific terms.
Ranking identified information adds another layer of complexity.
We have shown that utilizing efficient fine-tuning and data augmentation can help improve model performance.
Overall, our findings provide insights into the relative strengths and limitations of different methods for enhancing LLMs in medical applications.

\subsection*{Limitations 
} 

This study has several limitations.
First, we tested only a limited selection of available closed- and open-source LLMs.
Second, fine-tuning was not applied to closed-source LLMs.
Third, despite the improvements achieved through fine-tuning and data augmentation, the performance of the models still falls short of human annotation.
Future research will aim to address these challenges.


\section*{Conclusion}

We evaluated both closed- and open-source LLMs for identifying and prioritizing medical jargon using expert-annotated EHR notes, demonstrating that fine-tuning and data augmentation significantly enhance performance.
Comprehensive case analyses further validated our findings, highlighting the effectiveness of these methods for extracting and prioritizing important medical jargon for patients.

\section*{Acknowledgments}
We greatly value UMass BioNLP group's insightful feedback and thoughtful guidance.


\section*{Author Contributions}

Won Seok Jang, Sharmin Sultana, and Zonghai Yao authored the manuscript, with Won Seok and Sharmin implementing and analyzing models and Zonghai contributing to project planning. 

Zhichao Yang, Hieu Tran, and Sunjae Kwon designed the framework and reviewed outcomes. 

Hong Yu planned the project, wrote the proposal, and guided the research.


\section*{Conflict of Interest Statement}
No conflicting interests.

\section*{DATA AVAILABILITY}
The source code will be released here : \url{https://www.github.com/memy85/2024_medicalnote_annotation}

% References as numbers
\makeatletter
\renewcommand{\@biblabel}[1]{\hfill #1.}
\makeatother

% unstr is used to keep citation order
% \bibliographystyle{vancouver}
% \bibliography{amia}  

\bibliographystyle{naturemag}
\bibliography{medical_note_annotation, main}

\newpage
\clearpage
\appendix
\section*{Appendix}
\label{appendix}

% \documentclass[twoside]{article}

% \usepackage{aistats2025}
% If your paper is accepted, change the options for the package
% aistats2025 as follows:
%
%\usepackage[accepted]{aistats2025}
%
% This option will print headings for the title of your paper and
% headings for the authors names, plus a copyright note at the end of
% the first column of the first page.

% If you set papersize explicitly, activate the following three lines:
%\special{papersize = 8.5in, 11in}
%\setlength{\pdfpageheight}{11in}
%\setlength{\pdfpagewidth}{8.5in}

% If you use natbib package, activate the following three lines:
%\usepackage[round]{natbib}
%\renewcommand{\bibname}{References}
%\renewcommand{\bibsection}{\subsubsection*{\bibname}}

% If you use BibTeX in apalike style, activate the following line:
%\bibliographystyle{apalike}

% \begin{document}

% If your paper is accepted and the title of your paper is very long,
% the style will print as headings an error message. Use the following
% command to supply a shorter title of your paper so that it can be
% used as headings.
%
%\runningtitle{I use this title instead because the last one was very long}

% If your paper is accepted and the number of authors is large, the
% style will print as headings an error message. Use the following
% command to supply a shorter version of the authors names so that
% they can be used as headings (for example, use only the surnames)
%
%\runningauthor{Surname 1, Surname 2, Surname 3, ...., Surname n}

% Supplementary material: To improve readability, you must use a single-column format for the supplementary material.
\onecolumn
\appendix
\aistatstitle{From Deep Additive Kernel Learning to Last-Layer \\ Bayesian Neural Networks via Induced Prior Approximation: \\
Supplementary Materials}

\section{SPARSE CHOLESKY DECOMPOSITION}
\label{sec:sparse chol decompose}
In this section, we present the algorithm for constructing the induced grids $\mathbf{U}$ as defined in \cref{eq:GPlayer} by using sorted dyadic points, and obtaining the sparse Choleksy decomposition of the Laplace kernel in one dimension, as proposed in \citep{ding2024sparse}.

A set of one-dimensional level-$L$ dyadic points $\Xv_L$ in increasing order over the interval $[0,1]$ is defined as:
\begin{align}
    \Xv_{L}:= \left\{ \frac{1}{2^{L}}, \frac{2}{2^{L}}, \frac{3}{2^{L}}, \ldots, \frac{2^{L}-1}{2^{L}} \right\}.
\end{align}
However, this increasing order does not yield a sparse representation of the Markov kernel $k(\cdot,\cdot)$ on the points $\Xv_L$, i.e., Cholesky decomposition of the covariance matrix $k(\Xv_L, \Xv_L)$ is not sparse. To achieve a sparse hierarchical expansion, we first sort the dyadic points $\Xv_L$ according to their levels.

\paragraph{Sorted Dyadic Points}
For level-$\ell$ dyadic points $\Xv_{\ell}$ where $ \ell=1,\ldots,L$, we first define the set $\rho(\ell)$ consisting of odd numbers as follows:
\begin{align}
    \rho(\ell) = \left\{ 1,3,5,\ldots,2^{\ell}-1 \right\}.
\end{align}
Next, we define the sorted incremental set $\Dv_{\ell}$ (with $\Xv_{0}:= \varnothing$) as:
\begin{align}
    \Dv_{\ell} = 
    \left\{ \frac{i}{2^{\ell}}: i\in \rho(\ell) \right\} = \Xv_{\ell} - \Xv_{\ell-1}, \quad  \ell=1,\ldots L.
\end{align}
Thus, the level-$L$ dyadic points $\Xv_L$ can be decomposed into disjoint incremental sets $\{ \Dv_{\ell} \}_{\ell=1}^{L}$:
\begin{align}
    \Xv_{L} = \cup_{\ell=1}^{L} \Dv_{\ell}, \quad \Dv_{i} \cap \Dv_{j} = \varnothing \text{ for $i\neq j$}.
\end{align}
Therefore, we can define the sorted level-$L$ dyadic points using these incremental sets as:
\begin{align}\label{eq:sorted dyadic}
    \Xv_{L}^{\text{sort}}:= \left\{ \Dv_1,\Dv_2, \ldots, \Dv_{L} \right\} 
    = \left\{ \frac{i \in \rho(\ell) }{2^{\ell}}, \ell=1,\ldots,L \right\}.
\end{align}
For example, the sorted level-3 dyadic points are given by:
\begin{align}
    \Xv_{3}^{\text{sort}} 
    = \bigg\{ 
    \begingroup
        \color{blue}
        \underbracket{
            \color{black}
            \frac{1}{2^1}
        }_{\color{blue}
            \Dv_1
        }
    \endgroup
    , 
    \begingroup
        \color{blue}
        \underbracket{
            \color{black}
            \frac{1}{2^2}, \frac{3}{2^2}
        }_{\color{blue}
            \Dv_2
        }
    \endgroup
    ,
    \begingroup
        \color{blue}
        \underbracket{
            \color{black}
            \frac{1}{2^3}, \frac{3}{2^3}, \frac{5}{2^3}, \frac{7}{2^3}
        }_{\color{blue}
            \Dv_3
        }
    \endgroup
     \bigg\}.
\end{align}

\paragraph{Algorithm}
We now present the algorithm for computing the inverse of the upper triangular Cholesky factor $[ \Lv_{\Xv_{L}^{\text{sort}}}^{\top} ]^{-1}$ of the covariance matrix $k(\Xv_{L}^{\text{sort}}, \Xv_{L}^{\text{sort}})$ in \Cref{alg:cholesky}, where $\Lv_{\Xv_{L}^{\text{sort}}} \Lv_{\Xv_{L}^{\text{sort}}}^{\top} = k(\Xv_{L}^{\text{sort}}, \Xv_{L}^{\text{sort}})$.. The corresponding proof can be found in \citep{ding2024sparse}. The output of \Cref{alg:cholesky} is a sparse matrix with $\Oc(3 \cdot (2^{L}-1))$ nonzero entries. Since each iteration of the for-loop only requires solving a $3 \times 3$ linear system, which costs $\Oc(3^3)$ time, the total computational complexity of \Cref{alg:cholesky} is $\Oc(2^L-1)$. This implies that the complexity of computing $\left[ \Lv_{\Uv}^{\top} \right]^{-1}$ in \cref{eq:GPlayer} is $\Oc(M)$ when $\Uv$, the induced grid of size $M$, consists of sorted dyadic points as defined in \cref{eq:sorted dyadic}.

\begin{algorithm}[hbt!]
\caption{Computation of the inverse Cholesky factor for the Markov kernel $k(\cdot, \cdot)$ on sorted one-dimensional level-$L$ dyadic points $\Xv_L^{\text{sort}}$.}
\label{alg:cholesky}
\setstretch{0.99} % set the line spacing to 0.99
\begin{algorithmic}[1]
    \STATE {\bfseries Input:} Markov kernel $k(\cdot,\cdot)$, sorted level-$L$ dyadic points $\Xv_{L}^{\text{sort}}$
    \STATE {\bfseries Output:} inverse of the upper triangular Cholesky factor $\Rv:= [ \Lv_{\Xv_{L}^{\text{sort}}}^{\top} ]^{-1}$, s.t. $\Lv_{\Xv_{L}^{\text{sort}}} \Lv_{\Xv_{L}^{\text{sort}}}^{\top} = k(\Xv_{L}^{\text{sort}}, \Xv_{L}^{\text{sort}})$
    \STATE Initialize $\Rv \leftarrow \text{zeros($2^L-1$,$2^L-1$)}$;
    \STATE Define $k(\pm \infty, \cdot) = k(\cdot, \pm \infty) = 0$;
    \FOR{$\ell=1$ {\bfseries to} $L$}
        \FOR{$i \in \rho(\ell)=\{1,3,\ldots,2^{\ell}-1\}$}
            \STATE $x_{\text{mid}} := \frac{i}{2^{\ell}}$;\quad
            $x_{\text{left}}:=\frac{i-1}{2^{\ell}}$ {\bfseries if} $i>1$ {\bfseries else} $-\infty$;\quad
            $x_{\text{right}}:=\frac{i+1}{2^{\ell}}$ {\bfseries if} $i<2^{\ell}-1$ {\bfseries else} $+\infty$;
            \STATE Get $i_{\text{mid}}$, $i_{\text{left}}$, $i_{\text{right}}$, the indices of the points $x_{\text{mid}}$, $x_{\text{left}}$, $x_{\text{right}}$ in the sorted set $\Xv_{L}^{\text{sort}}$ respectively;
            \STATE Get the coefficients $c_1$, $c_2$, $c_3$ by solving the following linear system:
            \begin{align}
                \begin{bmatrix}
                     & k(x_{\text{left}}, x_{\text{left}})
                     & k(x_{\text{left}}, x_{\text{mid}})
                     & k(x_{\text{left}}, x_{\text{right}}) \\
                     & k(x_{\text{mid}}, x_{\text{left}})
                     & k(x_{\text{mid}}, x_{\text{mid}})
                     & k(x_{\text{mid}}, x_{\text{right}}) \\
                     & k(x_{\text{right}}, x_{\text{left}})
                     & k(x_{\text{right}}, x_{\text{mid}})
                    &k(x_{\text{right}}, x_{\text{right}})
                \end{bmatrix}
                \begin{bmatrix}
                    c1\\
                    c2\\
                    c3
                \end{bmatrix}=
                \begin{bmatrix}
                    0\\
                    1\\
                    0
                \end{bmatrix}.
            \end{align}
            \STATE $[c_1,c_2,c_3] := [c_1,c_2,c_3] / \sqrt{c_2}$;
            \STATE {\bfseries if} $x_{\text{left}} \neq - \infty$, 
            {\bfseries then} $\Rv[i_{\text{left}} ,i_{\text{mid}}] = c_1$; \quad
            {\bfseries if} $x_{\text{right}} \neq + \infty$, 
            {\bfseries then} $\Rv[i_{\text{right}} ,i_{\text{mid}}] = c_3$;
            \STATE $\Rv[i_{\text{mid}} ,i_{\text{mid}}] = c_2$;
        \ENDFOR
    \ENDFOR
\end{algorithmic}
\end{algorithm}


\section{REPARAMETERIZATION OF KERNEL LENGTHSCALES}
\label{sec:theo}
Considering the additive Laplace kernel with fixed lengthscale $\tilde{\theta}$ for all base kernels, applying linear projections $\left\{ \wv_{p}^{\top}\xv \right\}_{p=1}^{P}$ on inputs $\xv\in \Rb^D$ will give:
\begin{align}
    &\sum_{p=1}^{P}\sigma^2_p k_p\left( \wv^{\top}_{p}\xv,\wv^{\top}_{p}\xv^{\prime} \right)\nonumber \\
    = & \sum_{p=1}^{P} \sigma^2_p\exp \left( -  \frac{\sum_{d=1}^{D} \left| w_{p,d}\left( x_{d}-x_{d}^{\prime} \right) \right|}{\tilde{\theta}} \right)\nonumber \\
    = & \sum_{p=1}^{P} \prod_{d=1}^{D} \sigma^2_p\exp \left( - \frac{\left| x_{d}-x_{d}^{\prime} \right|}{\tilde{\theta} / \left| w_{p,d}\right| } \right)\nonumber \\
    = & \sum_{p=1}^{P} \prod_{d=1}^{D} \sigma^2_p\exp \left( - \frac{\left| x_{d}-x_{d}^{\prime} \right|}{\theta_{p,d}} \right),
\end{align}
This still leads to an additive Laplace kernel but with adaptive lengthscale $\theta_{p,d}$ for base kernels. The resulting kernel also retains \emph{sparse} Cholesky decomposition by the properties of Markov kernels so that the complexity of inference is $\Oc(M)$.

\section{INFERENCE OF PREDICTIVE DISTRIBUTION}
\label{sec:uq of inference}
Given an input $\xv \in \Rb^D$, the prediction of the DAK model can be written in the following equation according to \cref{eq:DAK prediction}: 
\begin{align}
    \tilde{f}_{\xv}
    &= \sum_{p=1}^{P}
    \sigma_p \Big(
        \phi(h_{\psi}^{[p]}(\xv)) \zv_p
    \Big) + \mu \nonumber\\
    &= \sum_{p=1}^{P}
    \sigma_p \Big(
        \bm{\phi}_{p}^{\top} \zv_p
    \Big) + \mu,
\end{align}
where $\bm{\phi}_{p}^{\top}:=\phi(h_{\psi}^{[p]}(\xv)) \in \Rb^{1 \times M}$
% , $\mu_p:=\mu_p(h_{\psi}^{[p]}(\xv)) \in \Rb$
. We assume the variational distribution over the independent Gaussian weights $\zv_p \sim \Nc(\bm{m}_{\zv_p}, \Sv_{\zv_p})$ and the bias $\mu \sim \Nc(m_{\mu}, \sigma_{\mu}^2)$. Then it's straighforward to deduce that
\begin{align}
    \bm{\phi}_{p}^{\top} \zv_p + \mu 
    &\sim
    \Nc\left(
    \bm{\phi}_{p}^{\top} \bm{m}_{\zv_p} + m_{\mu},\hspace{0.2em}
    \bm{\phi}_{p}^{\top} \Sv_{\zv_p} \bm{\phi}_{p} + \sigma_{\mu}^2
    \right), \\
    \sigma_p \left(
    \bm{\phi}_{p}^{\top} \zv_p 
    \right) + \mu
    & \sim
    \Nc\left(
    \sigma_p ( \bm{\phi}_{p}^{\top} \bm{m}_{\zv_p} )+ m_{\mu} ,\hspace{0.2em}
    \sigma_p^2( \bm{\phi}_{p}^{\top} \Sv_{\zv_p} \bm{\phi}_{p}) + \sigma_{\mu}^2
    \right), \\
    \tilde{f}_{\xv} = 
    \sum_{p=1}^{P}
    \sigma_p \left(
    \bm{\phi}_{p}^{\top} \zv_p
    \right) + \mu
    & \sim
    \Nc\left(
    \sum_{p=1}^{P}
    \sigma_p ( \bm{\phi}_{p}^{\top} \bm{m}_{\zv_p}) + m_{\mu} ,\hspace{0.2em}
    \sum_{p=1}^{P}
    \sigma_p^2( \bm{\phi}_{p}^{\top} \Sv_{\zv_p} \bm{\phi}_{p} ) + \sigma_{\mu}^2
    \right).
\end{align}
Therefore, we obtain the predictive distribution of the $\tilde{f}(\xv)$ at the point $\xv \in \Rb^D$ and its mean and variance are given by:
\begin{subequations}
\label{eq:dak inference closed form}
\begin{align}
    \Eb\left[ \tilde{f}_{\xv} \right]
        = \sum_{p=1}^{P}
        \sigma_p ( \bm{\phi}_{p}^{\top} \bm{m}_{\zv_p}) + m_{\mu},
\end{align}
\begin{align}
    \text{Var}\left[ \tilde{f}_{\xv} \right]
        =\sum_{p=1}^{P}
        \sigma_p^2( \bm{\phi}_{p}^{\top} \Sv_{\zv_p} \bm{\phi}_{p}) + \sigma_{\mu}^2.
\end{align}
\end{subequations}
% \begin{subequations}
% \label{eq:dak inference closed form}
%     \begin{align}
%         \Eb\left[ \tilde{f}(\xv) \right]
%         = \sum_{p=1}^{P}
%         \sigma_p ( \bm{\phi}_{p}^{\top} \bm{m}_{\zv_p} + m_{\mu_p} ),
%     \end{align}
%     \begin{align}
%         \text{Var}\left[ \tilde{f}(\xv) \right]
%         =\sum_{p=1}^{P}
%         \sigma_p^2( \bm{\phi}_{p}^{\top} \Sigma_{\zv_p} \bm{\phi}_{p} + \sigma_{\mu_p}^2).
%     \end{align}
% \end{subequations}


\section{TRAINING OF VARIATIONAL INFERENCE}
\label{sec:training}
Given the dataset $\mathcal{D}=\{ \Xv, \yv \}$ where $\Xv:=\{ \xv_i \}_{i=1}^N$, $\yv=(y_1,\ldots,y_N)^{\top}$, $\xv_i \in \Rb^D$, $y_i\in\Rb$, the prediction $\tilde{f}_{\Xv}\in \Rb^N$ of DAK is given by all the parameters $\bm{\theta}=\left\{ \psi, \bm{\sigma} \right\}$, $\bm{\eta}=\left\{ \{ \mv_{\zv_{p}},\Sv_{\zv_{p}}\}_{p=1}^{P} , \{m_{\mu},\sigma_{\mu} \} \right\}$ according to \cref{eq:DAK prediction}:
\begin{align}
    \tilde{f}_{\Xv}:= \tilde{f}(\Xv; \bm{\theta}, \bm{\eta})
    = \sum_{p=1}^{P}
    \sigma_p \Big(
        \phi(h_{\psi}^{[p]}(\Xv)) \zv_p
    \Big) + \mu,
\end{align}
where $\zv_{p} \sim \mathcal{N} (\bm{m}_{\zv_p} ,\Sv_{\zv_p})$, $p=1,\ldots,P$, and $\mu \sim \mathcal{N} ( m_{\mu},\sigma^2_{\mu} )$ are variational variables $\Theta_{\text{var}}$ parameterized by $\bm{\eta}$. The variational distribution is denoted by $q_{\bm{\eta}}(\Theta_{\text{var}})= q(\mu)\prod_{p=1}^{P} q(\zv_{p}) = \Nc ( m_{\mu} ,\sigma_{\mu}^2 )\prod_{p=1}^{P} 
\Nc ( \bm{m}_{\zv_p} ,\Sv_{\zv_p} )$, and the variational prior is denoted by $p(\Theta_{\text{var}})$.

We consider the KL divergence between $q_{\bm{\eta}}(\Theta_{\text{var}})$ and the true posterior $p(\Theta_{\text{var}}\vert \yv, \Xv, \bm{\theta})$:
\begin{align}
& \qquad \text{KL} \left[ q_{\bm{\eta}}(\Theta_{\text{var}}) \| p(\Theta_{\text{var}} \vert \yv,\Xv, \bm{\theta} ) \right] \nonumber \\
= & \int q_{\bm{\eta}}(\Theta_{\text{var}} )\log \frac{q_{\bm{\eta}}(\Theta_{\text{var}} )}{p(\Theta_{\text{var}} \vert \yv,\Xv,\bm{\theta} )} d\Theta_{\text{var}} \nonumber \\
= & \int q_{\bm{\eta}}(\Theta_{\text{var}} )\log \frac{q_{\bm{\eta}}(\Theta_{\text{var}} )p(\yv \vert \Xv,\bm{\theta})}{p(\yv \vert \Xv,\bm{\theta} ,\Theta_{\text{var}} )p(\Theta_{\text{var}} )} d\Theta_{\text{var}} \nonumber \\
= & \int q_{\bm{\eta}}(\Theta_{\text{var}} )\log \frac{q_{\bm{\eta}}(\Theta_{\text{var}} )}{p(\Theta_{\text{var}} )} d\Theta_{\text{var}} -\int q_{\bm{\eta}}(\Theta_{\text{var}} )\log p(\yv \vert \tilde{f}_{\Xv} )d\Theta_{\text{var}} +\log p(\yv\vert \Xv,\bm{\theta}).
\end{align}
Using the fact that $\text{KL}[\cdot \| \cdot] \geq 0$, we have
\begin{align}
\label{eq:variational lower bound}
    \log p(\yv\vert \Xv,\bm{\theta}) & \geq \int q_{\bm{\eta}}(\Theta_{\text{var}} )\log p(\yv \vert \tilde{f}_{\Xv} )d\Theta_{\text{var}} - \text{KL} \left[ q_{\bm{\eta}}(\Theta_{\text{var}} ) \| p(\Theta_{\text{var}}) \right] \nonumber \\
    & = \Eb_{q_{\bm{\eta}}(\Theta_{\text{var}} )} \left[ \log p(\yv \vert \tilde{f}_{\Xv} ) \right] - \text{KL} \left[ q_{\bm{\eta}}(\Theta_{\text{var}} ) \| p(\Theta_{\text{var}}) \right].
\end{align}

\paragraph{Full-training.}
Firstly, we present the joint training of $\bm{\theta}$ and $\bm{\eta}$. The most common approach optimizes the marginal log-likelihood (the left-hand side of \cref{eq:variational lower bound}):
\begin{align}
    \bm{\theta}^{\ast} &=\argmax_{\bm{\theta}} \log p(\yv\vert \Xv,\bm{\theta} ) \\
    &= \argmax_{\bm{\theta}} \log \int p\left( y\vert X,\bm{\theta},\Theta_{\text{var}} \right) p(\Theta_{\text{var}})d\Theta_{\text{var}},
\end{align}
which involves intractable integral in some tasks such as classification. Instead, we optimize the variational lower bound (the right-hand side of \cref{eq:variational lower bound}):
\begin{align}
    \Theta^{\ast} := \argmax_{\bm{\theta},\bm{\eta}} \mathcal{L}(\bm{\theta},\bm{\eta}) =\argmax_{\bm{\theta},\bm{\eta}}\left\{ E_{q_{\bm{\eta}}(\Theta_{\text{var}} )}\left[ \log p(\yv|\tilde{f}_{\Xv} ) \right] -\text{KL} \left[ q_{\bm{\eta}}(\Theta_{\text{var}} )\| p(\Theta_{\text{var}} ) \right] \right\}.
\end{align}

\paragraph{Fine-tuning.}
An alternative training approach is to firstly pre-train the deterministic parameters of feature extractor by standard neural network training, with mean squared error for regression or cross-entropy for classification as the loss function, and then fine-tune the last layer additive GP with fixed features. The objective function is identical to \cref{eq:elbo}, but $\bm{\theta}$ is learned during the pre-training step and is no longer optimized during fine-tuning.


\section{ELBO}%{DERIVATION OF ELBO}
\label{sec:elbo}
\subsection{Assumptions}
Consider the model $y_i = \tilde{f}(\xv_i) + \epsilon_i$ with the i.i.d. noise $\epsilon_i \overset{\text{i.i.d.}}{\sim} \Nc(0, \sigma_{f}^2)$ and $\tilde{f} : \Rb^D \rightarrow \Rb$ is defined in \cref{eq:DAK prediction}. The training dataset is $\mathcal{D} = \{ \Xv, \yv \}$ where $\Xv:=\{ \xv_i \}_{i=1}^N$, $\yv=(y_1,\ldots,y_N)^{\top}$, $\xv_i \in \Rb^D$, $y_i\in\Rb$. $\Theta_{\text{var}}:= \{ \mu ,\{ \zv_{p}\}_{p=1}^{P} \}$ are the variational random variables consisting of Gaussian weights and bias of $P$ units, $\psi$ are the parameters of the NN, $\bm{\sigma}:=(\sigma_1, \ldots, \sigma_p)^{\top}$ are the scale parameters of base GP layers. The variational distributions are $q(\mu)=\Nc(m_{\mu}, \sigma_{\mu}^2)$, $q(\zv_p)=\Nc(\bm{m}_{\zv_p}, \Sv_{\zv_p})$ and the variational priors are $p(\mu)=\Nc(\check{m}_{\mu} ,\check{\sigma}^2_{\mu})$, $p(\zv_p)=\Nc(\check{\bm{m}}_{\zv_p} ,\check{\Sv}_{\zv_p})$. Note that $\Sv_{\zv_p}\in\Rb^{M \times M}$ is a diagonal covariance matrix due to the independence of $\zv_p$, $M$ is the number of inducing points $\Uv$ defined in \cref{eq:GPlayer}, and $\bm{m}_{\zv_p} \in \Rb^M$, $m_{\mu} \in \Rb$, $\sigma_{\mu}^2 \in \Rb$. We derive the ELBO in VI to learn the preditive posterior over the variational variables $\Theta_{\text{var}}:= \{ \mu ,\{ \zv_{p}\}_{p=1}^{P} \}$ parameterized by $\bm{\eta}:=\left\{ \{ \mv_{\zv_{p}},\Sv_{\zv_{p}}\}_{p=1}^{P} , \{m_{\mu},\sigma_{\mu} \} \right\}$, and optimize the deterministic parameters $\bm{\theta}:=\{\psi, \bm{\sigma}\}$.

\subsection{Expected Log Likelihood}
\paragraph{Closed Form}
The \emph{expected log likelihood}, which is the first term in ELBO defined in \cref{eq:elbo}, is given by 
\begin{align}
    {\Eb}_{q_{\bm{\eta}}(\Theta_{\text{var}})} \left[ \log \text{Pr} (\yv \vert \tilde{f}_{\Xv} ) \right]
    &= {\Eb}_{q_{\bm{\eta}}(\Theta_{\text{var}})} \left[ 
    \log \prod_{i=1}^{N} 
    p (y_i \vert \tilde{f}_{\xv_i} )
    \right] \nonumber\\
    &= \sum_{i=1}^{N} 
    {\Eb}_{q_{\bm{\eta}}(\Theta_{\text{var}})} \left[ 
    \log
    p (y_i \vert \tilde{f}_{\xv_i} )
    \right] \nonumber\\
    &= \sum_{i=1}^{N} 
    {\Eb}_{q_{\bm{\eta}}(\Theta_{\text{var}})} \left[ 
    \log
    \Nc( \tilde{f}_i,\hspace{0.2em} \sigma_{f}^2 )
    \right] \nonumber\\
    &= \sum_{i=1}^{N} 
    {\Eb}_{q_{\bm{\eta}}(\Theta_{\text{var}})} \left[ 
    \log \left(
    (2\pi \sigma_{f}^2)^{-\frac{1}{2}}
    \exp\left\{  
        -\frac{ (y_i - \tilde{f}_i)^2 }{2 \sigma_{f}^2}
    \right\}
    \right)
    \right] \nonumber\\
    &= \sum_{i=1}^{N} 
    {\Eb}_{q_{\bm{\eta}}(\Theta_{\text{var}})} \left[
    -\frac{1}{2} \log(2\pi) 
    - \frac{1}{2}\log(\sigma_{f}^2)
    - \frac{1}{2 \sigma_{f}^2}
    (y_i - \tilde{f}_i)^2
    \right] \nonumber\\
    &= - \frac{N}{2} \log(2\pi)
    - \frac{N}{2} \log(\sigma_{f}^2)
    - \frac{1}{2 \sigma_{f}^2}
    \sum_{i=1}^{N}
    {\Eb}_{q_{\bm{\eta}}(\Theta_{\text{var}})} \left[
    (y_i - \tilde{f}_i)^2
    \right] \nonumber\\
    &= - \frac{N}{2} \log(2\pi)
    - \frac{N}{2} \log(\sigma_{f}^2)
    - \frac{1}{2 \sigma_{f}^2}
    \sum_{i=1}^{N} \left(
    \left({\Eb}_{q(\Theta_{\text{var}})} \left[
    (y_i - \tilde{f}_i)
    \right] \right)^2
    + \text{Var}_{q(\Theta_{\text{var}})} \left[
    (y_i - \tilde{f}_i)
    \right]
    \right) \label{eq:evidence halfway},
\end{align}
where
\begin{align}
    \tilde{f}_i
    % \mu_{\tilde{f}_i} &:= \tilde{f}(\xv_i;\Theta_{\text{var}}, \Theta_{\text{det}} ) \nonumber\\
    &= \sum_{p=1}^{P} \sigma_p \Big(
    \begingroup
        \color{blue}
        \underbracket{
            \color{black}
            \phi(h_{\psi}^{[p]}(\xv_i))
        }_{\color{blue}
            :=\bm{\phi}_{i,p}^{\top} \in \Rb^{1 \times M}
        }
    \endgroup
    \zv_p
    \Big)
    + \mu
    % \begingroup
    %     \color{blue}
    %     \underbracket{
    %         \color{black}
    %         \mu_{p}(h_{\psi}^{[p]}(\xv_i))
    %     }_{\color{blue}
    %         :=\mu_{i,p} \in \Rb
    %     }
    % \endgroup 
    \nonumber\\
    &= \sum_{p=1}^{P} \sigma_p \left(
    \bm{\phi}_{i,p}^{\top} \zv_p 
    \right) + \mu.
\end{align}
Recall that the variational assumptions $q(\zv_p)=\Nc(\bm{m}_{\zv_p}, \Sv_{\zv_p})$ and $q(\mu)=\Nc(m_{\mu}, \sigma_{\mu}^2)$, we can infer that
\begin{align}
    \bm{\phi}_{i,p}^{\top} \zv_p + \mu 
    &\sim
    \Nc\left(
    \bm{\phi}_{i,p}^{\top} \bm{m}_{\zv_p} + m_{\mu},\hspace{0.2em}
    \bm{\phi}_{i,p}^{\top} \Sv_{\zv_p} \bm{\phi}_{i,p} + \sigma_{\mu}^2
    \right), \\
    \sigma_p \left(
    \bm{\phi}_{i,p}^{\top} \zv_p 
    \right) + \mu
    & \sim
    \Nc\left(
    \sigma_p ( \bm{\phi}_{i,p}^{\top} \bm{m}_{\zv_p} ) + m_{\mu},\hspace{0.2em}
    \sigma_p^2( \bm{\phi}_{i,p}^{\top} \Sv_{\zv_p} \bm{\phi}_{i,p} ) + \sigma_{\mu}^2
    \right), \\
    \tilde{f}_i = 
    \sum_{p=1}^{P}
    \sigma_p \left(
    \bm{\phi}_{i,p}^{\top} \zv_p 
    \right)+ \mu
    & \sim
    \Nc\left(
    \sum_{p=1}^{P}
    \sigma_p ( \bm{\phi}_{i,p}^{\top} \bm{m}_{\zv_p} )+ m_{\mu},\hspace{0.2em}
    \sum_{p=1}^{P}
    \sigma_p^2( \bm{\phi}_{i,p}^{\top} \Sv_{\zv_p} \bm{\phi}_{i,p} ) + \sigma_{\mu}^2
    \right), \\
    y_i - \tilde{f}_i
    & \sim 
    \Nc\left(
    y_i - 
    \sum_{p=1}^{P}
    \sigma_p ( \bm{\phi}_{i,p}^{\top} \bm{m}_{\zv_p} ) -m_{\mu},\hspace{0.2em}
    \sum_{p=1}^{P}
    \sigma_p^2( \bm{\phi}_{i,p}^{\top} \Sv_{\zv_p} \bm{\phi}_{i,p} ) + \sigma_{\mu}^2
    \right).
\end{align}
Therefore, 
\begin{subequations}\label{eq:exp and var in evidence}
    \begin{align}
        \left({\Eb}_{q(\Theta_{\text{var}})} \left[
        (y_i - \tilde{f}_i)
        \right] \right)^2
        = \left(
         y_i - 
        \sum_{p=1}^{P}
        \sigma_p ( \bm{\phi}_{i,p}^{\top} \bm{m}_{\zv_p} ) -m_{\mu}
        \right)^2,
    \end{align}
    \begin{align}
        \text{Var}_{q(\Theta_{\text{var}})}
        \left[
        (y_i - \tilde{f}_i)
        \right]
        = \sum_{p=1}^{P}
        \sigma_p^2( \bm{\phi}_{i,p}^{\top} \Sv_{\zv_p} \bm{\phi}_{i,p} ) + \sigma_{\mu}^2.
    \end{align}
\end{subequations}
By applying \cref{eq:exp and var in evidence} to \cref{eq:evidence halfway}, we derive the analytical formula for the expected evidence, expressed as
\begin{align}
    {\Eb}_{q_{\bm{\eta}}(\Theta_{\text{var}})} \left[ \log \text{Pr} (\yv \vert \tilde{f}_{\Xv} ) \right]
    &= - \frac{N}{2} \log(2\pi)
    - \frac{N}{2} \log(\sigma_{f}^2) \nonumber\\
    &- \frac{1}{2 \sigma_{f}^2}
    \sum_{i=1}^{N} \left(
        \Big(
         y_i - 
        \sum_{p=1}^{P}
        \sigma_p ( \bm{\phi}_{i,p}^{\top} \bm{m}_{\zv_p} ) -m_{\mu}
        \Big)^2
        + \sum_{p=1}^{P}
        \sigma_p^2( \bm{\phi}_{i,p}^{\top} \Sv_{\zv_p} \bm{\phi}_{i,p} )+ \sigma_{\mu}^2
    \right). \label{eq:evidence final}
\end{align}

\paragraph{Monte Carlo Approximation}
For comparison, we provide the equation for computing the Monte Carlo estimate of the ELBO in the paragraph that follows.
\begin{align}
    {\Eb}_{q_{\bm{\eta}}(\Theta_{\text{var}})} \left[ \log \text{Pr} (\yv \vert \tilde{f}_{\Xv} ) \right]
    % &= {\Eb}_{q(\Theta)} \left[ 
    % \log \prod_{i=1}^{N} 
    % p (y_i \vert \xv_i,\Theta, \psi, \bm{\sigma})
    % \right] \nonumber\\
    &= \sum_{i=1}^{N} 
    {\Eb}_{q_{\bm{\eta}}(\Theta_{\text{var}} )} \left[ 
    \log
    p (y_i \vert \tilde{f}_{\xv_i} )
    \right] \nonumber\\
    & \approx \sum_{i=1}^{N}
    \frac{1}{S}
     \sum_{s=1}^{S}
    \log
    p (y_i \vert \xv_i,\tilde{\Theta}^{(s)}_{\text{var}}, \bm{\theta} ) \nonumber\\
    &= \frac{1}{S} \sum_{i=1}^{N} 
    \sum_{s=1}^{S} 
    \log
    \Nc(y_i \left\vert\right. \tilde{f}_{i}^{(s)},\hspace{0.2em} \sigma_{f}^2 )
    \nonumber\\
    &= \frac{1}{S} \sum_{i=1}^{N} 
    \sum_{s=1}^{S} 
    \log \left(
    (2\pi \sigma_{f}^2)^{-\frac{1}{2}}
    \exp\left\{  
        -\frac{ (y_i - \tilde{f}_{i}^{(s)})^2 }{2 \sigma_{f}^2}
    \right\}
    \right)
    \nonumber\\
    &= \frac{1}{S} \sum_{i=1}^{N} 
    \sum_{s=1}^{S} \left(
    -\frac{1}{2} \log(2\pi) 
    - \frac{1}{2}\log(\sigma_{f}^2)
    - \frac{1}{2 \sigma_{f}^2}
    (y_i - \tilde{f}_{i}^{(s)})^2
    \right) \nonumber\\
    &= - \frac{N}{2} \log(2\pi)
    - \frac{N}{2} \log(\sigma_{f}^2)
    - \frac{1}{2 \sigma_{f}^2}
    \sum_{i=1}^{N}
    \frac{1}{S} \sum_{s=1}^{S}
    (y_i - \tilde{f}_{i}^{(s)})^2, \label{eq:evidence halfway mc approx}
\end{align}
where $S$ is the number of Monte Carlo samples, $\{  \tilde{\mu}^{(s)} ,\{ \tilde{\zv}_{p}^{(s)} \}_{p=1}^{P} \} := \tilde{\Theta}^{(s)}_{\text{var}}$ are the $s$-th Monte Carlo samplings over the variational parameters $\Theta_{\text{var}}$ and $\tilde{\Theta}^{(s)}_{\text{var}} \sim q_{\bm{\eta}}(\Theta_{\text{var}})$, $\tilde{f}_{i}^{(s)}$ is given as follows:
\begin{align}
    \tilde{f}_{i}^{(s)} &:= \tilde{f}(\xv_i;\tilde{\Theta}^{(s)}_{\text{var}},\bm{\theta} ) \nonumber\\
    &= \sum_{p=1}^{P} \sigma_p \Big(
    \begingroup
        \color{blue}
        \underbracket{
            \color{black}
            \phi(h_{\psi}^{[p]}(\xv_i))
        }_{\color{blue}
            :=\bm{\phi}_{i,p}^{\top} \in \Rb^{1 \times M}
        }
    \endgroup
    \tilde{\zv}_p^{(s)} 
    \Big) + \tilde{\mu}^{(s)} \nonumber\\
    &= \sum_{p=1}^{P} \sigma_p \left(
    \bm{\phi}_{i,p}^{\top} \tilde{\zv}_p^{(s)} 
    \right)+ \tilde{\mu}^{(s)}. \label{eq:mc approx mean}
\end{align}
Therefore, we plug \cref{eq:mc approx mean} into \cref{eq:evidence halfway mc approx} and get the the Monte Carlo estimate of the ELBO written in the following formula:
\begin{align}
    {\Eb}_{q_{\bm{\eta}}(\Theta_{\text{var}})} \left[ \log \text{Pr} (\yv \vert \tilde{f}_{\Xv} ) \right]
    &\approx
    - \frac{N}{2} \log(2\pi)
    - \frac{N}{2} \log(\sigma_{f}^2)
    - \frac{1}{2 \sigma_{f}^2}
    \sum_{i=1}^{N}
    \frac{1}{S} \sum_{s=1}^{S}
    \Big(y_i - 
    \sum_{p=1}^{P} \sigma_p \left(
    \bm{\phi}_{i,p}^{\top} \tilde{\zv}_p^{(s)} 
    \Big)- \tilde{\mu}^{(s)}
    \right)^2, \label{eq:evidence final mc approx} \\
    \tilde{\zv}_p^{(s)} &\sim \Nc(\bm{m}_{\zv_p}, \Sv_{\zv_p}),\qquad
    \tilde{\mu}^{(s)} \sim \Nc(m_{\mu}, \sigma_{\mu}^2).
\end{align}


\subsection{KL Divergence}
Since we place Gaussian assumptions over the variational parameters $\Theta_{\text{var}}$,  the \emph{KL divergence}, which is the second term in ELBO defined in \cref{eq:elbo}, is then given by
\begin{align}
    \text{KL} \left[ q(\Theta_{\text{var}} ) \| p(\Theta_{\text{var}}) \right]
    &= \text{KL} \left[ q( \mu ,\{ \zv_{p}\}_{p=1}^{P} ) \Vert p( \mu ,\{ \zv_{p}\}_{p=1}^{P}) \right] \nonumber\\
    & =  
    \text{KL} \left[ q(\mu) \Vert p(\mu) \right] 
    + \sum_{p=1}^{P} 
    \text{KL} \left[ q(\zv_{p}) \Vert p(\zv_{p}) \right],
\end{align}

\begin{align}
     \text{KL} \left[ q(\mu) \Vert p(\mu) \right]
     = \frac{1}{2} \left(
     \frac{\sigma_{\mu}^2}{\check{\sigma}_{\mu}^2} 
     + \frac{(m_{\mu} - \check{m}_{\mu})^2}{\check{\sigma}_{\mu}^2} 
     -\log\left( \frac{\sigma_{\mu}^2}{\check{\sigma}_{\mu}^2} \right)
     -1
     \right),
\end{align}

\begin{align}
    \text{KL} \left[ q(\zv_{p}) \Vert p(\zv_{p}) \right]
    = \frac{1}{2} \sum_{i=1}^{M} \left(
     \frac{[\Sv_{\zv_p}]_{ii}}{[\check{\Sv}_{\zv_p}]_{ii}} 
     + \frac{([\bm{m}_{\zv_p}]_{i} - [\check{\bm{m}}_{\zv_p}]_i)^2}{[\check{\Sv}_{\zv_p}]_{ii}}
     -\log\left( 
     \frac{[\Sv_{\zv_p}]_{ii}}{[\check{\Sv}_{\zv_p}]_{ii}}  
     \right)
     -1
     \right),
\end{align}
where $[\Sv_{\zv_p}]_{ii}$ is the $(i,i)$-th element of the diagonal covariance matrix $\Sv_{\zv_p} \in \Rb^{M \times M}$, $[\bm{m}_{\zv_p}]_{i}$ is the $i$-th element of the mean vector $\bm{m}_{\zv_p} \in \Rb^M$, the approximated posteriors are $q(\mu)=\Nc(m_{\mu}, \sigma_{\mu}^2)$, $q(\zv_p)=\Nc(\bm{m}_{\zv_p}, \Sv_{\zv_p})$ and the priors are $p(\mu)=\Nc(\check{m}_{\mu} ,\check{\sigma}^2_{\mu})$, $p(\zv_p)=\Nc(\check{\bm{m}}_{\zv_p} ,\check{\Sv}_{\zv_p})$.

% \subsection{Performance Comparison}
% \label{sec:toy exp compare}
% We compare the perforamce of computing the ELBO in \cref{eq:elbo} by using closed form in \cref{eq:evidence final} and using Monte Carlo approximation in \cref{eq:evidence final mc approx} in a toy example.
% \textcolor{red}{Table or Figure to add if time available}


\subsection{Limitations of the Closed-Form ELBO}

The closed-form ELBO is only applicable to regression problems. In classification, applying the softmax function to $\tilde{f}(\xv;\bm{\theta}, \bm{\eta})$ results in a non-analytic predictive distribution, meaning the ELBO must still be computed via Monte Carlo sampling during training. Similarly, the closed-form expressions for the predictive mean and variance, as provided in \cref{eq:dak inference closed form} in \Cref{sec:uq of inference}, are not applicable to classification but only apply to regression problems.


\section{COMPUTATIONAL COMPLEXITY}
\label{sec:complexity}
In this section, we discuss the computational complexity of various DKL models compared to the proposed DAK method, focusing on the GP layer as the most computationally demanding component. \Cref{tab:complexity supp} shows the computational complexity of our model compared to other state-of-the-art GP and DKL methods.

\begin{table}[ht]
    \caption{Computational complexity of the DKL models for $N$ training points. The reported training complexity is for one iteration. $\hat{M}$ is the number of inducing points in SVGP and KISS-GP, while $M$ is the size of induced grids in DAK, $M < \hat{M}$. $S$ is the number of Monte Carlo samples, $B$ is the size of mini-batch, $D_w$ is the dimension of the NN outputs in DKL, $P$ is the dimension of the outputs after applying linear transformations to the NN outputs in the proposed DAK model. DAK-MC refers to the DAK model using Monte Carlo approximation, while DAK-CF refers to the DAK model using closed-form inference and ELBO.}
    \centering
    \begin{tabular}{lcc}
    \toprule[1pt]
                  & \textbf{Inference}       & \textbf{Training} (per iteration) \\
    \midrule[0.5pt]
    NN + SVGP     & $\Oc(\hat{M}^2 N)$    & $\Oc( S D_w MB + \hat{M}^3)$ \\
    NN + KISS-GP  & $\Oc(D_w \hat{M}^{1+\frac{1}{D_w}})$  & $\Oc(S D_w MB + D_w \hat{M}^{\frac{3}{D_w}})$ \\
    DAK-MC (ours) & $\Oc(SM)$       & $\Oc(SPMB + PM)$   \\
    DAK-CF (ours) & $\Oc(M)$        & $\Oc(PMB + PM)$    \\
    \bottomrule[1pt]
    \end{tabular}
    \label{tab:complexity supp}
\end{table}

\paragraph{Inference Complexity.}
In inference based on induced approximation, computing the multiplication of the inverse of the covariance matrix $k(\Uv, \Uv)$ and a vector takes $\Oc(\hat{M}^2N)$ time for $\hat{M}$ inducing points $\Uv$ and $N$ training points when using SVGP. This cost is reduced by KISS-GP to $\Oc(D \hat{M}^{1+\frac{1}{D}})$ by decomposing the covariance matrix into a Kronecker product of $D$ one-dimensional covariance matrices of the inducing points: $k(\Uv, \Uv) = \bigotimes_{d=1}^{D} k(\Uv^{[d]}, \Uv^{[d]})$. Despite the significant reduction on complexity, it requires inducing points $\Uv$ arranged on a Cartesian grid of size $\hat{M} = \prod_{d=1}^{D} \hat{M}_d$, where $\hat{M}_d$ is the number of inducing points in the $d$-th dimension. In high-dimensional spaces, fixing $\hat{M}$ leads to very small $\hat{M}_d$ per dimension, which can degrade model performance. To address this, we propose the DAK model via sparse finite-rank approximation, which employs an additive Laplace kernel for GPs. The inverse Cholesky factor $\Lv_{\Uv}^{\top}$ for one-dimensional induced grids $\Uv$ of size $M$, where $M < \hat{M}$, as defined in \cref{eq:GPlayer}, is sparse and can be computed in $\Oc(M)$ time.

\paragraph{Training Complexity.}
In training, VI requires computing the ELBO as described in \cref{eq:elbo}, which consists of two terms: the \emph{expected log likelihood} and the \emph{KL divergence} between the variational distributions and priors. 

1) The \emph{expected log likelihood} is usually approximated via Monte Carlo sampling at a cost of $\Oc(S N_{\Theta} N)$, where $S$ is the number of Monte Carlo samples, $N_{\Theta}$ is the total number of variational parameters $\Theta_{\text{var}}$, and $N$ is the number of training points. This complexity can be reduced to $\Oc(S N_{\Theta} B)$ by applying stochastic variational inference with a mini-batch of size $B \ll N$. For DKL models using SVGP and KISS-GP, $\Theta_{\text{var}}$ are inducing variables, and the expectation does not have a closed form, requiring Monte Carlo sampling. In contrast, in the proposed DAK model, $\Theta_{\text{var}}= \{ \{ \zv_{p}\}_{p=1}^{P}, \mu \}$ consists of independent Gaussian weights $\zv_p\in \Rb^M$ and bias $\mu$. This allows us to derive an analytical form for this term, as shown in \cref{eq:evidence final} in \Cref{sec:elbo}, reducing the computational cost to $\Oc(N_{\Theta} B) = \Oc(PM B)$ when using a mini-batch of size $B$.

2) The \emph{KL divergence} between two Gaussian distributions can be computed in closed form. This leads to a linear time complexity of $\Oc(N_{\Theta})$ if the parameters $\Theta_{\text{var}}$ are independent, or cubic time $\Oc(N_{\Theta}^3)$ if they are fully correlated. In SVGP and KISS-GP, $\Theta_{\text{var}}$ represents fully correlated Gaussian distributed inducing variables, so computing the KL divergence takes $\Oc(\hat{M}^3)$ for SVGP. In KISS-GP, this can be reduced to $\Oc(D \hat{M}^{\frac{3}{D}})$ using fast eigendecomposition of Kronecker matrices. In the DAK model, the weights $\{\zv_p\}_{p=1}^{P}$ as defined in \cref{eq:GPlayer} are independent Gaussian random variables, allowing the KL divergence to be computed in $\Oc(N_{\Theta}) = \Oc(PM)$ time, where $P$ is the number of base GP layers.


\section{ADDITIONAL DISCUSSIONS}

Although interpretability is one advantage of additive models, the main motivation for replacing a GP layer with an additive GP layer in our work is to handle high-dimensional data. When the input dimension is low, it is reasonable that GPs are superior to additive GPs since the additive kernel is an approximated and restrictive kernel. However, when the input dimension increases, the computational complexity grows considerably even in GPs with sparse approximation. For example, in DKL, the output dimension of NN encoder is usually chosen as small as 2, while in pixel data experiments, DKL cannot handle the computation associated with the dimensionality when the output dimension of ResNet is 512 or more. Although DKL is superior in low-dimensional and simple cases, we view additive structure as a necessary component to achieve scalability and good performance with high-dimensional data.

\subsection{Why choosing the induced grids instead of learning the inducing points?}

From an approximation accuracy point of view, there are two separate strategies to increase the accuracy. The first one is to learn the inducing point locations. The second one, however, is to simply increase the number of inducing points on a pre-specified finer grid. The second method is much easier to implement and has a theoretical guarantee by the GP regression theory: as the inducing points become dense in the input region, the approximation will become exact. In contrast, the first approach does not have such a favorable theoretical guarantee. 

The second approach would become difficult to use for many existing methodologies as in general the computational cost would scale as $\mathcal{O}(M^3)$ with $M$ inducing points, which is particularly problematic in high dimensions. 
% The first approach can be viewed as a compromise in those situations, and that is why many existing methods chose to learn the locations of the inducing points instead.
This difficulty is resolved by additive GPs, since approximating an additive GP boils down to approximating one dimensional GPs, which can be accomplished by using a set of pre-specified inducing points on a fine grid in 1-D. One major benefit of the proposed methodology is that the computation now scales at $\mathcal{O}(M)$, enabled by the Markov kernel and the additive kernel. Therefore, a large number of inducing points can be used in an efficient way. 

The proposed method also has several additional benefits: 1) It can decouple to some extent the neural network component and GP component by avoiding learning the inducing points, which may help reduce overfitting/overconfidence; 2) The equivalence to BNN holds exactly with the fixed inducing points, whereas for learned inducing points, this BNN equivalence breaks down, and the proposed computation/training framework would not be possible to carry through; 3) It can simplify the overall optimization since there is no need to learn the inducing points.

\subsection{Limitations and future directions}

Generally, a finer grid will lead to better approximations, but the number of parameters to be trained will also increase. Therefore, there is a trade-off between the accuracy and the computational cost that we can afford. This current work is using a specific Laplace kernel, which can utilize sparse Cholesky decomposition. More general kernels may result in more computational complexity but better representation power of the model. In addition, the current variational family is restricted under mean-field assumptions. A more general variational family, e.g. full/low-rank covariance, may lead to superior performance in some applications. 


\section{EXPERIMENTAL DETAILS}
\label{sec:expdetail}
In this section, we provide additional details regarding the experiments.

\subsection{Benchmarks for Regression}
\label{subsec:regression supp}
\paragraph{Experiment Setup}
For all models, the NN architecture is a fully connected NN with rectified linear unit (ReLU) activation function \citep{nair2010rectified} and two hidden layers containing 64 and 32 neurons, respectively, structured as $D \rightarrow 64 \rightarrow 32 \rightarrow D_w$, where $D$ is the input feature size (also the size of input $\Xv$) and $D_w$ is the output feature size. The models are evaluated with $D_w=16$, 64, and 256, respectively. The number of Monte Carlo samples is set to 8 during training and 20 during inference.

The NN is a deterministic model, and we use the negative Gaussian log-likelihood as the loss function to quantify the uncertainty of the NN outputs and compute the NLPD.

For NN+SVGP, the inducing points are set to the size of 64 in $D_w$ dimension. We implement the \texttt{ApproximateGP} model in GPyTorch \citep{gardner2018gpytorch}, defining the inducing variables as variational parameters, and use \texttt{VariationalELBO} in GPyTorch to perform variational inference and compute the loss.

SV-DKL is originally designed for classification, so for a fair comparison in regression tasks, we modify it by first applying a linear embedding layer $\Wv: \Rb^{D_w} \rightarrow \Rb^P$ with $P=16$ and normalizing the outputs to the interval $[0,1]$ for each base GP, similar to the DAK model. To adapt the additive GP layer for regression, we remove the softmax function from the model in eq. (1) of \citep{wilson2016stochastic}. Given training data $\{ \xv_i, \yv_i \}_{i=1}^{N}$, the model is modified as follows:
\begin{align}
    p(\yv_i \vert \fv_i, A) = \mathcal{A}(\fv_i)^{\top} \yv_i
\end{align}
where $\fv_i \in \Rb^P$ is a vector of independent GPs followed by a linear mixing layer $\mathcal{A}(\fv_i) = A \fv_i$, with $A \in \Rb^{C \times P}$ as the transformation matrix. Here, $C=1$ for single-task regression. For each $p$-th GP ($1 \leq p \leq P$) in the additive GP layer, the corresponding inducing variables $\uv_p$ are set to the size of 64 and treated as variational parameters for training. We use the \texttt{GridInterpolationVariationalStrategy} model with \texttt{LMCVariationalStrategy} in GPyTorch to perform KISS-GP with variational inducing variables, augmented by a linear mixing layer.

For AV-DKL, the inducing points are set to size of 64 in $D_{w}$ dimension. We implement the AV-DKL model based on the source code~\cite{matias2024amortized}.

Both DAK-MC and DAK-CF use the same additive GP layer size as SV-DKL, with $P=16$, and employ fixed induced grids $\Uv = \{1/8, 2/8, \ldots, 7/8\}$ of size 7 for each base GP, which is much smaller than that of SV-DKL.

\paragraph{Metrics}
Let $\{\xv_t, y_t\}_{t=1}^{T}$ represent a test dataset of size $T$, where $\mu_t$ and $\sigma_t^2$ are the predictive mean and variance. We evaluate model performance using two common metrics: Root Mean Squared Error (RMSE) and Negative Log Predictive Density (NLPD).

RMSE is widely used to assess the accuracy of predictions, measuring how far predictions deviate from the true target values. It is calculated as:
\begin{align}
    \text{RMSE} = \sqrt{ \frac{1}{T} \sum_{t=1}^{T}(y_t - \mu_t)^2 }.
\end{align}

NLPD is a standard probabilistic metric for evaluating the quality of a model's uncertainty quantification. It represents the negative log likelihood of the test data given the predictive distribution. For GPs, NLPD is calculated as:
\begin{align}
    \text{NLPD}
    &= - \sum_{t=1}^{T} \log p(y_t = \mu_t \vert \xv_t) \\
    &= \frac{1}{T}
    \sum_{t=1}^{T} \Big[
    \frac{(y_t - \mu_t)^2}{2\sigma_t^2} + \frac{1}{2} \log(2\pi \sigma_t^2)
    \Big].
\end{align}
Both RMSE and NLPD are widely used in the GP regression literature, where smaller values indicate better model performance.

\paragraph{Computing Infrastructure}
The experiments for regression were run on Macbook Pro M1 with 8 cores and 16GB RAM.

\subsection{Benchmarks for Classification}
\label{subsec:classification supp}
We use PyTorch \citep{paszke2019pytorch} baseline of NN models, GPyTorch \citep{gardner2018gpytorch} baseline of SVGP and SV-DKL models. In classification tasks, we apply a softmax likelihood to normalize the output digits to probability distributions. The NN is a deterministic model trained via negative log-likelihood loss, while DKL and DAK models are trained via ELBO loss. The setting of all training tasks are described in \Cref{tab:model classification} and \Cref{tab:optimizer classification}.

SVGP is originally designed for single-output regression. To make it fit for multi-output classification, we used \texttt{IndependentMultitaskVariationalStrategy} in GPyTorch to implement the multi-task \texttt{ApproximateGP} model, and use \texttt{VariationalELBO} with \texttt{SoftmaxLikelihood} in GPyTorch to perform variational inference and compute the loss. 

For SV-DKL, we employed the same \texttt{VariationalELBO} with \texttt{SoftmaxLikelihood} as the variational loss objective. \texttt{GridInterpolationVariationalStrategy} is applied within \texttt{IndependentMultitaskVariationalStrategy} to perform additive KISS-GP approximation. For each KISS-GP unit, we used $64$ variational inducing points initialized on a grid of size $[-1,1]$. 

For DAK, we implemented DAK-MC using Monte Carlo estimation given the intractable softmax likelihood. We employed fixed induced grids $\Uv=\{ -31/32, -30/32, \ldots, 30/32, 31/32 \}$ of size 63 for each base GP component.

\begin{table}[ht]
\caption{Model architectures for image classification on MNIST, CIFAR-10 and CIFAR-100.}
\centering
\resizebox{0.7\linewidth}{!}{
\begin{tabular}{l|l|ccc}
\toprule[1pt]
Model                   & Hyper-parameter          & MNIST       & CIFAR-10    & CIFAR-100   \\
\midrule[0.5pt]
\multirow{4}{*}{NN+SVGP}   & Feature extractor        & CNN         & ResNet-18   & ResNet-34   \\
                        & NN out features $D_w$         & 128         & 512         & 512         \\
                        & Embedding features $P$               & 16          & 64          & 128         \\
                        & \# inducing points $\hat{M}$      & 512         & 512         & 512         \\
                        & \# epochs       & 20         & 200         & 200         \\
                        & Training strategy      & Full-training         & Full-training         & Fine-tuning         \\
\midrule[0.5pt]
\multirow{5}{*}{SV-DKL} & Feature extractor        & CNN         & ResNet-18   & ResNet-34   \\
                        & NN out features $D_w$         & 128         & 512         & 512         \\
                        & Embedding features $P$               & 16          & 64          & 128         \\
                        & \# inducing points $\hat{M}$      & 64          & 64          & 64          \\
                        & Grid bounds              & {[}-1,1{]} & {[}-1,1{]} & {[}-1,1{]} \\
                        & \# epochs       & 20         & 200         & 200         \\
                        & Training strategy       & Full-training         & Full-training         & Fine-tuning         \\
\midrule[0.5pt]
\multirow{4}{*}{DAK}    & Feature extractor        & CNN         & ResNet-18   & ResNet-34   \\
                        & NN out features $D_w$         & 128         & 512         & 512         \\
                        & Embedding features $P$               & 16          & 64          & 128         \\
                        & \# induced interpolation $M$ & 63          & 63          & 63         \\
                        & \# epochs       & 20         & 200         & 200         \\
                        & Training strategy      & Full-training         & Full-training         & Full-training         \\
\bottomrule[1pt]
\end{tabular}

}
\label{tab:model classification}
\end{table}

\paragraph{MNIST} We used a CNN implemented in PyTorch as the feature extractor: \texttt{Conv2d}(1,32,3) $\rightarrow$ \texttt{Conv2d}(32,64,3) $\rightarrow$ \texttt{MaxPool2d}(2) $\rightarrow$ \texttt{Dropout}(0.25) $\rightarrow$ \texttt{Linear}(9216,128) $\rightarrow$ \texttt{Dropout}(0.5). To make a fair comparison, for both SV-DKL and DAK, we applied an embedding module through a linear layer that transform $128$ output features into $P=16$ base GP channels. 

\paragraph{CIFAR-10} We used a ResNet-18 as the feature extractor followed by a linear embedding layer that compressed the $512$ output features into $P=64$ base GP channels. 

\paragraph{CIFAR-100} We used a pretrained ResNet-34 as the feature extractor for SV-DKL and fine-tuned GP output layers since SV-DKL struggled to fit using full-training. For proposed DAK, we used full-training. The number of base GP channels is selected as $P=128$. 

\begin{table}[ht]
\caption{Details of training optimizer for image classification on MNIST, CIFAR-10 and CIFAR-100.}
\centering
\resizebox{0.7\linewidth}{!}{

\begin{tabular}{l|ccc}
\toprule[1pt]
Optimization      & MNIST                                                             & CIFAR-10                                                                                                  & CIFAR-100                                                                                                 \\
\midrule[0.5pt]
Optimizer         & Adadelta                                                          & SGD                                                                                                       & SGD                                                                                                       \\
Initial lr.       & 1.0                                                               & 0.1                                                                                                       & 0.1                                                                                                       \\
Weight decay      & 0.0001                                                            & 0.0001                                                                                                    & 0.0001                                                                                                    \\
Scheduler         & StepLR                                                            & CosineAnnealingLR                                                                                         & CosineAnnealingLR                                                                                         \\
\midrule[0.5pt]
Data Augmentation & MNIST                                                             & CIFAR-10                                                                                                  & CIFAR-100                                                                                                 \\
\midrule[0.5pt]
RandomCrop        & -                                                                 & size=32, padding=4                                                                                        & size=32, padding=4                                                                                        \\
HorizontalFlip    & -                                                                 & p=0.5                                                                                                     & p=0.5                                                                                                     \\
% Normalization     & \begin{tabular}[c]{@{}l@{}}mean=0.1307,\\ std=0.3081\end{tabular} & \begin{tabular}[c]{@{}l@{}}mean={[}0.4914,0.4822,0.4465{]},\\ std={[}0.2023,0.1994,0.2010{]}\end{tabular} & \begin{tabular}[c]{@{}l@{}}mean={[}0.5071,0.4867,0.4408{]},\\ std={[}0.2675,0.2565,0.2761{]}\end{tabular} \\
\bottomrule[1pt]
\end{tabular}
}
\label{tab:optimizer classification}
\end{table}

\paragraph{Additional Benchmark.}  \citet{matias2024amortized} proposed Amortized Variational DKL (AV-DKL), which is a variant SV-DKL using amortization network to compute the inducing locations and variational parameters, thus attenuating the overcorrelation of NN extracted features. AV-DKL is included as the additional benchmark for classification tasks in \Cref{tab:img avdkl}. The training recipe is the same with SV-DKL. 


\begin{table*}[ht]
\caption{\small{Accuracy, NLL, ECE for AV-DKL, SV-DKL, DAK-MC on CIFAR-10/100 averaged over 3 runs. CIFAR-10 uses ResNet-18 with 64 features extracted; CIFAR-100 uses ResNet-34 with 512 features. The best results are highlighted in \textbf{bold}; the second best results are highlighted by \underline{underline}.}}
\centering
\vspace{-0.1cm}
\resizebox{\linewidth}{!}{%
\begin{tabular}{rccclccc}
\toprule[1pt]
\multicolumn{1}{l}{} & \multicolumn{3}{c}{Batch size: 128}  &  & \multicolumn{3}{c}{Batch size: 1024} \\ \cline{2-4} \cline{6-8} \vspace{-8pt} \\
\multicolumn{1}{l}{} & AV-DKL & SV-DKL & \cellcolor{Gray} DAK-MC &   & AV-DKL  & SV-DKL & \cellcolor{Gray} DAK-MC \\ 
\midrule[1pt]
CIFAR-10 - Acc. (\%) $\uparrow$    & \underline{94.23 $\pm$ 0.65}  & 93.44 $\pm$ 0.28    &  \cellcolor{Gray} \textbf{94.81 $\pm$ 0.13}   &     &  \textbf{93.32} $\pm$ \textbf{0.13}        & 90.22 $\pm$ 1.42       & \cellcolor{Gray} \underline{93.02 $\pm$ 0.18}        \\
NLL $\downarrow$     & 0.352 $\pm$ 0.084    & \underline{0.312 $\pm$ 0.033}       &  \cellcolor{Gray} \textbf{0.256} $\pm$ \textbf{0.014}     &      & \underline{0.439 $\pm$ 0.022}         & 0.485 $\pm$ 0.061       & \cellcolor{Gray} \textbf{0.345 $\pm$ 0.001}    \\
ECE $\downarrow$      & 0.048 $\pm$ 0.006    & \underline{0.046 $\pm$ 0.003}       &  \cellcolor{Gray} \textbf{0.039 $\pm$ 0.002}          &     & \underline{0.054 $\pm$ 0.001}       & 0.060 $\pm$ 0.004       & \cellcolor{Gray} \textbf{0.052 $\pm$ 0.001}           \\
\midrule[1pt]
CIFAR-100 -  Acc. (\%) $\uparrow$    & \textbf{77.47 $\pm$ 0.19}  & 74.52 $\pm$ 0.13       & \cellcolor{Gray}  \underline{76.75 $\pm$ 0.18}     &     &  \textbf{77.07 $\pm$ 0.10}        & 66.54 $\pm$ 0.74       & \cellcolor{Gray} \underline{70.38 $\pm$ 1.25}        \\
NLL $\downarrow$     & 1.787 $\pm$ 0.011    & \underline{1.041 $\pm$ 0.007}       & \cellcolor{Gray}  \textbf{1.001 $\pm$ 0.027}     &      & 2.326 $\pm$ 0.030    & \underline{1.738 $\pm$  0.058}      & \cellcolor{Gray} \textbf{1.203 $\pm$ 0.040}        \\
ECE $\downarrow$      & 0.166 $\pm$ 0.002    & \underline{0.049 $\pm$ 0.002}       & \cellcolor{Gray}  \textbf{0.041 $\pm$ 0.004}        &     & 0.175 $\pm$ 0.001         & \underline{0.148 $\pm$ 0.007}       &\cellcolor{Gray}  \textbf{0.056 $\pm$ 0.006}           \\
\bottomrule[1pt]
\end{tabular}
}
\vspace{-0.2cm}
\label{tab:img avdkl}
\end{table*}

\paragraph{Metrics} 
We evaluate model performance using four common metrics: Top-1 accuracy, ELBO, Negative Log Likelihood (NLL), and Expected Calibration Error (ECE). 

ECE is a metric used to quantify the degree of ``calibration'' of a probabilistic model in UQ, specifically for classification problems. It is defined as the weighted average of the absolute difference between the model's predicted probability (confidence) and the actual outcome (accuracy) over several bins of predicted probability. Mathematically, ECE is given by:
\begin{align}
    \text{ECE} =\sum_{m=1}^{M} \frac{\left| B_{m} \right|}{n} \left| \text{acc} (B_{m})-\text{conf} (B_{m}) \right|,
\end{align}
where $M$ is the number of bins into which the confidence values are partitioned, $B_m$ is the set of indices of samples whose predicted confidence falls into the $m$-th bin, $n$ is the total number of samples.

\paragraph{Computing Infrastructure}
The experiments for classification were run on a Linux machine with NVIDIA RTX4080 GPU, and 32GB of RAM.




\subsection{Additional Tables and Figures}
\label{sec:additional exp results}

\paragraph{Choices of learning rates.}
We evaluate the choices of learning rates on 1D regression examples. DKL requires a separate tuning of the learning rate of the GP covariance parameters, which differs from the learning rate of the NN feature extractor. In \Cref{fig:dkl lr}, we choose the learning rate of the NN feature extractor as $0.01$, while the learning rate of the GP covariance is set to different values. (a)-(c) show that different learning rates of covariance in DKL result in different predictive posterior. In particular, although the training losses for DKL in both (a) and (b) are minimal, the regressions do not fit well. On the other hand, DAK does not need a distinct recipe for tuning GP covariances because of the BNN interpretation. Furthermore, the poor posterior is indicated by the higher training loss, as illustrated in (d)-(f).

\begin{figure}[ht]
\centering
\subfloat[$\begin{gathered}\text{DKL: last-layer lr} =0.01.\\ \text{Training loss:} -0.21.\end{gathered}$]{\includegraphics[width=.3\textwidth]{toy_dkl_lr_01.pdf}}
\subfloat[$\begin{gathered}\text{DKL: last-layer lr} =0.001.\\ \text{Training loss: } -0.07.\end{gathered}$]{\includegraphics[width=.3\textwidth]{toy_dkl_lr_001.pdf}}
\subfloat[$\begin{gathered}\text{DKL: last-layer lr} =0.0001.\\ \text{Training loss: } 0.22.\end{gathered}$]{\includegraphics[width=.3\textwidth]{toy_dkl_lr_0001.pdf}}

\subfloat[$\begin{gathered}\text{DAK: last-layer lr} =0.1.\\ \text{Training loss: } 0.10.\end{gathered}$]{\includegraphics[width=.3\textwidth]{toy_dak_lr_1.pdf}}
\subfloat[$\begin{gathered}\text{DAK: last-layer lr} =0.01.\\ \text{Training loss: } 0.10.\end{gathered}$]{\includegraphics[width=.3\textwidth]{toy_dak_lr_01.pdf}}
\subfloat[$\begin{gathered}\text{DAK: last-layer lr} =0.001.\\ \text{Training loss: } 0.22.\end{gathered}$]{\includegraphics[width=.3\textwidth]{toy_dak_lr_001.pdf}}

\caption{Results on 1D regression with different last-layer learning rates. The learning rate of NN feature extractor is set as $0.01$. (a)--(f) shows the regression fits and corresponding training losses. DAK fits for the same learning rate strategy with NN feature extractor (lr=0.01), while DKL requires a separate tuning for last-layer learning rate of GPs. Additionally, a better training loss does not necessarily prevent overfitting for DKL.}
\label{fig:dkl lr}
\end{figure}


\paragraph{Learning curves.} We plot the learning curves of CIFAR-10/100 in \Cref{fig:cifar10 curves} and \ref{fig:cifar100 curves}. The learning curves of SVDKL in \Cref{fig:cifar10 curves} is more unstable, with many significant spikes, and the convergence is slower than DAK. Futhermore, SVDKL struggles to fit with full-training in CIFAR-100, and a pretrained feature extractor is used in CIFAR-100. Therefore, the learning curves of SVDKL look smoothing, but DAK fits well with full-training in CIFAR-100.


\begin{figure}[ht]
\centering
\subfloat[Test Error (\%).]{\includegraphics[width=.3\textwidth]{CIFAR_10_test_error.pdf}}
\subfloat[Test NLL.]{\includegraphics[width=.3\textwidth]{CIFAR_10_nll.pdf}}
\subfloat[ELBO.]{\includegraphics[width=.3\textwidth]{CIFAR_10_elbo.pdf}}
\caption{Test errors, test NLLs, ELBOs of NN, SVDKL, and DAK curves with batch size of 128/1024 for CIFAR-10 averaged on 3 runs. DAK outperforms SVDKL on both test error and NLL along the training epochs. Additionally, SVDKL degrades more and struggles to fit when the batch size becomes larger.}
\label{fig:cifar10 curves}
\end{figure}

\begin{figure}[ht]
\centering
\subfloat[Test Error (\%).]{\includegraphics[width=.3\textwidth]{CIFAR_100_test_error.pdf}}
\subfloat[Test NLL.]{\includegraphics[width=.3\textwidth]{CIFAR_100_nll.pdf}}
\subfloat[ELBO.]{\includegraphics[width=.3\textwidth]{CIFAR_100_elbo.pdf}}
\caption{Test errors, test NLLs, ELBOs of NN, SVDKL, and DAK curves with batch size of 128/1024 for CIFAR-100 averaged on 3 runs. DAK trained NN and last-layer additive GPs jointly, while SVDKL used the pre-trained NN and fine-tuned the last-layer GP since SVDKL struggles to fit using full-training. DAK outperforms SVDKL on both test error and NLL along the training epochs. SVDKL struggled to fit in high-dimensional multitask cases, indicating the necessity of pre-training in SVDKL. However, DAK fitted well with high dimensionality and large batch sizes.}
\label{fig:cifar100 curves}
\end{figure}







% \end{document}
 % This includes the contents of appendix.tex



\end{document}
