\subsection{Time Diaries}

This section provides a detailed outline of all the questions and answer options delivered through iLog. We report the diaries sent at fixed intervals throughout the study in the main text (\cref{tab:td_main}). Additionally, two time diaries were sent at the start and end of each day. The morning questions listed in \cref{tab:morning_questions} inquire about the sleep quality from the previous night and the participant’s expectations for the upcoming day. The evening questions, in \cref{tab:evening_questions}, ask participants to evaluate their day, mention any problems they encountered, describe how they addressed them, and report any challenges due to the COVID-19 pandemic.

To reduce the answering burden, participants could deactivate receiving questions for a fixed period by selecting one of the reasons listed in \cref{tab:break}.

The time diary questions were sent to participants every thirty minutes during the first two weeks and every hour in the final two weeks, asking about the activity being performed, location, social context, and mood. Additional questions were activated based on the activity selected in question \textit{A3} in \cref{tab:td_main} to gather more details. Specifically, when participants indicated they were traveling, they were asked about the purpose and mode of transportation (\textit{A3a1} and \textit{A3a2} in \cref{tab:td_sub}, respectively). Additional questions also prompted the type of food and drinks when participants were eating (\textit{A3c} in \cref{tab:td_sub}) and the type of sport when engaging in physical activities (\textit{A3b} in \cref{tab:td_sub}).


Every two hours, questions in \cref{tab:tb_snack} collect information about snacks and drinks outside the main meal periods.

\begin{table}[htb]
    \footnotesize
    \centering
    \caption{Morning questions sent at 8:00 AM.}
    \label{tab:morning_questions}
    \begin{tabularx}{0.97\textwidth}{p{8cm}p{8cm}}
    \toprule
    \textbf{A1. How would you rate your sleep quality last night?} &%
    \textbf{A2. How do you expect your day to be?}\\
    \midrule
    %
    \begin{enumerate}[leftmargin=*]
    \item \vcenteredinclude{figures/emoji/emoji_1.png} very good
    \item \vcenteredinclude{figures/emoji/emoji_2.png} fairly good
    \item \vcenteredinclude{figures/emoji/emoji_3.png}
    \item \vcenteredinclude{figures/emoji/emoji_4.png} fairly bad
    \item \vcenteredinclude{figures/emoji/emoji_5.png} very bad
    % \item {\LARGE  \SymbolaEmoji \symbol{"1F600}} very good
    % \item {\LARGE  \SymbolaEmoji \symbol{"1F642}} fairly good
    % \item {\LARGE  \SymbolaEmoji \symbol{"1F611}}
    % \item {\LARGE  \SymbolaEmoji \symbol{"1F615}} fairly bad
    % \item {\LARGE  \SymbolaEmoji \symbol{"1F641}} very bad
    \end{enumerate}
    &
    \begin{enumerate}[leftmargin=*]
    \item \vcenteredinclude{figures/emoji/emoji_1.png}
    \item \vcenteredinclude{figures/emoji/emoji_2.png}
    \item \vcenteredinclude{figures/emoji/emoji_3.png}
    \item \vcenteredinclude{figures/emoji/emoji_4.png}
    \item \vcenteredinclude{figures/emoji/emoji_5.png}
    \end{enumerate} \\
    \bottomrule
\end{tabularx}
    
    \caption{Evening questions sent at 10:00 PM.}
    \label{tab:evening_questions}
    \begin{tabularx}{0.97\textwidth}{XXX}
    \toprule
    \textbf{A7. How was your day?}&
    \textbf{A8. Did you have any problem at [college (weekdays)] today?}&
    \textbf{A9. What was the problem you had?} \\
    \midrule
    \begin{enumerate}[leftmargin=*]
    \item \vcenteredinclude{figures/emoji/emoji_1.png} %{\LARGE  \SymbolaEmoji \symbol{"1F600}}
        \item \vcenteredinclude{figures/emoji/emoji_2.png}%{\LARGE  \SymbolaEmoji \symbol{"1F642}}
        \item \vcenteredinclude{figures/emoji/emoji_3.png}%{\LARGE  \SymbolaEmoji \symbol{"1F611}}
        \item \vcenteredinclude{figures/emoji/emoji_4.png}%{\LARGE  \SymbolaEmoji \symbol{"1F615}}
        \item \vcenteredinclude{figures/emoji/emoji_5.png}%{\LARGE  \SymbolaEmoji \symbol{"1F641}}
    \end{enumerate}&
    \begin{enumerate}[leftmargin=*]
    \item Yes
    \item No
    \end{enumerate}&
    open-ended question\\
\end{tabularx}
\begin{tabularx}{0.95\textwidth}{XX}
    \toprule
    \textbf{A10.Were you able to solve the problem (alone or with help of someone)?}&
    \textbf{A11. Is there anything that you would have liked to do today that was not possible because of
the Covid-19 virus?} \\
    \midrule
    open-ended question & open-ended question \\
    \bottomrule
\end{tabularx}
\end{table}

\begin{table}[]
    \centering
    \caption{List of motivations to suspend question notifications for a fixed number of hours.}
    \label{tab:break}
    \begin{tabularx}{0.50\textwidth}{X}
        \toprule
        \textbf{Break options}\\
        \midrule
        \begin{enumerate}
        \setcounter{enumi}{35}
        \item Others
        \item I will participate in sports activities
        \item I have a work/study meeting
        \item I am at the cinema/theatre/hospital/church
        \item I am starting classes/lessons/lab
        \item I will go to sleep
        \end{enumerate} 
        \\
    \bottomrule
    \end{tabularx}
\end{table}

\begin{table}
    \footnotesize
    \centering
    \caption{In-depth questions that appear when certain options are selected in the question ``What are you doing?''}
    \label{tab:td_sub}
    \begin{tabularx}{0.97\textwidth}{XXXX}
    \toprule
    \textbf{A3a1. And you travel to/from or related to:}&
    \textbf{A3a2. How are you moving?}&
    \textbf{A3b. What kind of sports activity?}&
    \textbf{A3c. Select the main food \& drink you ate} (Multiple choices)\\
    \midrule
    \begin{enumerate}[leftmargin=*]
        \item study
        \item social life
        \item shopping and services
        \item other leisure
        \item work
        \item changing locality
        \item other or unspecified travel purpose
    \end{enumerate}&
    \begin{enumerate}[leftmargin=*]
        \item on foot
        \item by bike
        \item by bus/tram
        \item by metro/subway/underground
        \item by train
        \item by e-scooter
        \item by car
        \item by car as passenger
        \item by car sharing
        \item by moped, motorbike
        \item by moped, motorbike as passenger
        \item by motorboat
        \item by airplane
        \item by taxi/Uber
        \item other private transport modes
        \item other public transport modes
    \end{enumerate}&
    \begin{enumerate}[leftmargin=*]
        \item Walking, Trekking, and hiking
        \item Jogging and running
        \item Cycling, skiing, and skating
        \item Ball games
        \item Gymnastics and Fitness
        \item Water sports
        \item Other or unspecified sports or indoor activities
        \item Other or unspecified sports or outdoor activities
        \item Productive exercise (e.g., hunting, fishing, picking berries, mushrooms, or herbs)
    \end{enumerate}&
    \begin{enumerate}[leftmargin=*]
        \item Bread, steamed buns and/or breakfast cereals
        \item Rice, potatoes, beans, pasta, noodles, dumplings, etc.
        \item Vegetables
        \item Fruits
        \item Meat
        \item Fish
        \item Processed meat (ham, bacon, sausages)
        \item Dairy products (Plain or low-fat milk, yogurt, cheese)
        \item Soya-based food (milk, yogurt, tofu)
        \item Pastries and sweets
        \item Snack/sandwiches (chips\dots)
        \item Water
        \item Soda
        \item Coffee/tea or similar
        \item Others non-alcoholic drink
        \item Beer
        \item Wine
        \item Liquor
        \item Other alcoholic drink
        \item Other food
    \end{enumerate}
    \\
    \bottomrule
\end{tabularx}


\end{table}

\begin{table}
    \footnotesize
    \centering
    \caption{Additional questions related to food and drinks.}
    \label{tab:tb_snack}
    \begin{tabularx}{0.97\textwidth}{XX}
    \toprule
    \textbf{A6b. In the last two hours did you have any snacks or drinks (except breakfast, lunch, and dinner).} (administered at hours 02, 04, 06, 10, 12, 15, 17, 19, 22, 24) (Multiple choices) &
    \textbf{A6c. Select the food \& drink taken as snack. If you had more than one snack in the last two hours, only focus on the most recent one.} (Multiple choices)\\
    \midrule
    \begin{enumerate}[leftmargin=*]
        \item No
        \item Yes, between now and 30 minutes ago \textbf{(go to A6c)}
        \item Yes, between 0.5 and 1 hour ago \textbf{(go to A6c)}
        \item Yes, between 1 and 1.5 hours ago \textbf{(go to A6c)}
        \item Yes, between 1.5 and 2 hours ago \textbf{(go to A6c)}
    \end{enumerate}&
    \begin{enumerate}[leftmargin=*]
        \item Confectionery (Candy, Chocolate, etc)
        \item Cookies, cakes, and pastries
        \item Bars (Energy bar, etc.)
        \item Crackers/biscuits
        \item Seeds, nuts, grains, legumes
        \item Savory snacks (Chips, Tapas, Pizza, Nachos, Snack mix, deep frying)
        \item Sandwiches (Sandwich, Hamburgers, Hot dogs, Bagel)
        \item Frozen (Ice cream, Milkshake, etc.)
        \item Bread, steamed buns and/or breakfast cereals
        \item Rice, potatoes, beans, pasta, noodles,
        dumplings, etc.
        \item Vegetables
        \item Fruits
        \item Dairy products (milk, yogurt, cheese)
        \item Soya-based food (milk, yogurt, tofu)
        \item Meat
        \item Fish
        \item Processed meat (ham, bacon, sausages)
        \item Water
        \item Soda
        \item Coffee/tea or similar
        \item Others non-alcoholic drink
        \item Beer
        \item Wine
        \item Spirit
        \item Others alcoholic drink
        \item Other food
    \end{enumerate}\\
    \bottomrule
\end{tabularx}
\end{table}

\newpage 

\subsection{Sensors Description}\label{app2:sensors}

The smartphone sensors used by iLog are categorized as follows:
%
\begin{itemize}
    \item \textit{Hardware (HW)}, physical sensors of the device that detect and respond to physical environment, e.g., accelerometer, gyroscope and GPS;
    \item \textit{Software (SW)}, software component collecting events from the operating system and software, for instance, the Wifi the phone is connected to.
\end{itemize}
%
\cref{tab:sensor-list} reports the collection frequency of each sensor. The possible frequency values are:
%
\begin{itemize}
    \item \textit{on change} means that the value of the sensor is recorded only when the current value changes (along with a timestamp of when it happened);
    \item\textit{up to X samples per second} means that a maximum of X events are recorded for each second (the actual frequency in the data might slightly diverge from the reported one due to the device's inaccuracy);
    \item \textit{once every Y} means a new event is generated once every Y (also in this case the actual frequency approximates the reported one).
\end{itemize}
In \cref{tab:sensor}, we briefly describe each sensor and collected variable. The sensors are grouped by category, which corresponds to the distribution bundles (see \cref{sec:availability}).

\begin{table}[tb]
\centering
\caption{\label{tab:sensor-list} List sensors. The type column reports HW for hardware sensors, and SW for software sensors.}
\begin{tabular}{rclcc}
    \toprule
    \textbf{No} & \textbf{Type} & \textbf{Name} & \textbf{Frequency} \\
    \midrule
    1  & SW & Bluetooth Devices                 & Once every minute\\
    2  & SW & Cellular network info             & Once every minute\\
    3  & SW & WIFI Network Connected to         & On change\\
    4  & SW & WIFI Networks Available           & Once every minute\\
    5  & HW & Light                             & up to 10 samples per second \\
    6  & HW & Pressure                          & up to 10 samples per second\\
    7  & HW & Accelerometer                     & up to 10 samples per second \\ 
    8  & HW & Gyroscope                         & up to 10 samples per second  \\
    9 & SW & Movement Activity Label           & Once every 30 seconds\\
    10 & SW & Step Counter                      & up to 10 samples per second\\
    11 & SW & Step Detection                    & On change\\
    12 & HW & Location                          & Once every minute  \\
    13 & HW & Magnetic Field                    & up to 10 samples per second  \\
    14 & SW & Proximity                         & up to 10 samples per second\\
    15 & SW & Headset Status [ON/OFF]           & On change\\
    16 & SW & Music Playback & On change\\
    17 & SW & Notifications received            & On change\\
    18 & SW & Running Applications              & Once every 5 seconds\\
    19 & SW & Airplane Mode [ON/OFF]            & On change\\
    20 & SW & Battery Charge [ON/OFF]           & On change\\
    21 & SW & Battery Level                     & On change\\
    22 & SW & Doze Mode [ON/OFF]                & On change\\
    23 & SW & Ring mode [Silent/Normal]         & On change\\
    24 & SW & Touch event                       & On change\\
    25 & SW & Screen Status [ON/OFF]            & On change\\
    26 & SW & User Presence                     & On change\\
    \bottomrule
\end{tabular}
\end{table}



\begin{center}
\renewcommand{\arraystretch}{1.5}
\begin{longtable}{p{0.15\linewidth}|p{0.83\linewidth}}
    \caption[ilog sensors]{Description of the collected ilog sensors.} \label{tab:sensor} \\
    \toprule
    \textbf{Sensor} & \textbf{Description} \\
    \midrule
    \endfirsthead
    \multicolumn{2}{l}{\textbf{\textsc{Connectivity}}} \\
    %
    Bluetooth & 
    List of discovered Bluetooth normal or low energy devices around the smartphone containing the following information:
    \begin{itemize}
        \item \textit{name}: user-friendly name of the remote device;
        \item \textit{address}: hardware MAC Address of the device;
        \item \textit{bondstate}: whether the remote device is connected;
        \item \textit{rssi}: Received Signal Strength Indicator;
        \item \textit{class code} and \textit{class tag}: Bluetooth class of the device (e.g., phone or computer), and the class describes the characteristics and capabilities of the device (e.g., audio and telephony).
    \end{itemize}\\
    %
    Cellular Network &
    Information of the cellular network to which the phone is connected to:
    \begin{itemize}
        \item \textit{cellid}: identifier of the cell;
        \item \textit{dbm}: signal strength;
        \item \textit{type}: type of the cell, possible values are lte, wcdma, gsm and cdma.
    \end{itemize}\\
    %
    WiFi Event & 
    Returns information related to the WIFI network to which the phone is connected; if connected, it also reports the WIFI network ID. Additional features are:
    \begin{itemize}
        \item \textit{ssid}: (Service Set Identifier) ID or unique identifier of a digital network (Wi-Fi or WLAN);
        \item \textit{bssid}: (Basic Service Set Identifier): sequence of characters that define a wireless computer network configured to communicate directly with each other;
        \item \textit{isconnected}: return whether the phone is connected to the WIFI.
    \end{itemize}\\
    %
    WiFi Networks Event &
    Returns all WIFI networks detected by the smartphone. Additional features are:
    \begin{itemize}
        \item \textit{address}: is a unique identifier assigned to a network interface controller for use as a network address in communications within a network segment;
        \item \textit{capabilities}: list of capabilities supported by the network, e.g., WPA and WPS;
        \item \textit{frequency}: the WiFi frequency bands include 2.4 GHz and 5 GHz;
        \item \textit{name}: the name assigned to the WiFi network
        \item \textit{rssi}: Received Signal Strength Indicator is an estimated measure of signal strength that indicates how effectively a device can receive signals from any wireless access point or Wi-Fi router. It provides insight into the quality and reliability of the connection, often measured in decibels (dBm) to represent signal strength.
        The RSSI value range is between 0 and -100, where 0 signifies stronger and more stable connections.
    \end{itemize}\\
    %
    \multicolumn{2}{l}{\textbf{\textsc{Motion}}} \\
    %
    Accelerometer & 
    Measures the acceleration to which the phone is subjected and captures it as a 3D vector. The unit is $m/s^2$. \\
    %
    Gyroscope &
    Measures the rotational forces to which the phone is subjected and it captures it as a 3D vector. The unit is $rad/s$. \\
    %
    Activities &
    It reports the user's activity recognized by the Google Activity Recognition API. The recognized activities are \textit{in vehicle}, \textit{on bicycle}, \textit{on foot}, \textit{running}, \textit{still}, \textit{tilting}, \textit{walking} and \textit{unknown}. The sensor reports a confidence score between 0 and 100, which represents the likelihood that the user is performing the activity. \\
    %
    Step Counter &
    It counts the total number of steps performed by the user (while carrying the phone) since the phone was powered on.\\
    %
    Step detector & 
    An event is triggered each time the user takes a step. \\
    %
    %Orientation & 
    %Returns the position of the device relative to the earth's magnetic north pole.\\
    %
    \multicolumn{2}{l}{\textbf{\textsc{Position}}} \\
    %
    Location & It provides the geographic coordinates of the phone:
    \begin{itemize}
        \item \textit{latitude}: latitude in degrees;
        \item \textit{longitude}: longitude in degrees;
        \item \textit{altitude}: the altitude in meters;
        \item \textit{accuracy}: estimated horizontal accuracy radius in meters;
        \item \textit{speed}: current speed of the phone in $m/s$;
        \item \textit{provider}: the source of the coordinates, i.e., GPS (hardware sensor in the devices), network (based on the WiFi network  the phone is connected to) and passive (retrieve the location from other applications that already requested it);
        \item \textit{bearing}: horizontal direction of travel and it's the angle with respect to the north that is being faced.
    \end{itemize} \\
    %
    Magnetic field &
    Reports the ambient magnetic field along the three sensor axes at the phone location. \\
    %
    Proximity & Measures the distance between the user's head and the phone. Depending on the phone, it may be measured in centimeters (i.e., the absolute distance) or as labels (e.g., 'near', 'far'). \\
    %
    \multicolumn{2}{l}{\textbf{\textsc{App usage}}} \\
    %
    Headset status & Indicates whether the headphones are connected to the phone. \\
    %
    Music Playback & Returns whether music is being played on the phone using the default music player from the operating system. Track information is not collected.\\
    %
    Notifications & It generates an event every time the phone receives a notification and when it is dismissed by the user:
    \begin{itemize}
        \item \textit{identifier}: the unique identifier of notification within the application that generated it
        \item \textit{isclearable}: whether the notification can be canceled when the user clears the notifications;
        \item \textit{isongoing}: the notification refers to an event that is ongoing, e.g., a phone call;
        \item \textit{package}: package name of the application.
        \item \textit{status}: whether the notification is posted or dismissed.
    \end{itemize}\\
    %
    Applications & Reports the name of the application (or application package) currently running in the foreground of the phone. \\
    %
    \multicolumn{2}{l}{\textbf{\textsc{Device usage}}} \\
    %
    Airplane Mode &  Returns whether the phone’s airplane mode is on or off. When off, all the connectivity features of the phone are turned off. Airplane mode also conserves battery life by reducing power-consuming background activities. It’s accessible through the quick settings menu on most devices. \\
    %
    Battery Charge &
    Returns whether the phone is on charge:
    \begin{itemize}
        \item \textit{source}: type of power source connected to the device; possible values are USB, AC charge, wireless power source or unknown.
        \item \textit{status}: whether the device is being charged
    \end{itemize}\\
    %
    Battery Monitoring Log & 
    Returns the phone's battery level:
    \begin{itemize}
        \item \textit{level}: current battery level between 0 and 100;
        \item \textit{scale}: maximum level of the battery represented as a value between 0 and 100.
    \end{itemize}\\
    %
    Doze Mode & Returns whether the phone’s doze mode is on or off.
    Doze mode is a low-power state that a phone enters after a period of inactivity to conserve battery. In this mode, background processes and network access are restricted, allowing only essential tasks, such as high-priority notifications or alarms, to function periodically. This helps significantly reduce battery consumption while the device is idle.\\
    %
    Ring Mode & Reports the current ring status of the phone. When set to \textit{normal}, the smartphone rings on incoming calls, messages and notifications by producing audible alerts. Other statuses are \textit{vibrate} (where the phone vibrates instead of ringing) or \textit{silent} (where all sounds are muted). Ring mode is often managed through the device's sound settings. \\
    %
    Touch event & It generates an event each time the user touches the screen. \\
    %
    Screen status & Returns whether the phone’s screen is on or off.\\
    %
    User Presence & Detects when the user is present near the phone, for example, when the user unlocks the screen. \\
    %
    \multicolumn{2}{l}{\textbf{\textsc{Enviroment}}} \\
    %
    Light &  This component detects ambient light level around the phone, and it is measured in illuminance (lux). 
    This sensor helps the device adjust the screen brightness automatically, optimizing visibility while conserving battery life. For example, in bright conditions, the screen brightness increases for better readability, whereas in darker environments, it dims to reduce eye strain and save power. Light sensors are commonly located near the top of the device, often beside the front-facing camera.
    \\
    %
    Pressure & It measures the ambient air pressure to which the phone is subjected in hPa or mbar. \\
    \bottomrule
\end{longtable}
\end{center}

%\newpage
\subsection{Dataset Files Structure}

The folder structure of the dataset is

\begin{lstlisting}[language=bash]
. dataset root
|- Site_Trento_ITA
   |- Diachronic-Interactions
      - timediaries.parquet
   |- Synchronic-Interactions
      |- matching.csv
      |- survey1.parquet
      |- survey2.parquet
      |- survey3.parquet
   |- Sensors
      |- App-usage
         |- application.parquet
         |- headsetplug.parquet
         |- music.parquet
         |- notification.parquet
      |- Connectivity
         |- bluetooth.parquet
         |- cellularnetwork.parquet
         |- wifi.parquet
         |- wifinetworks.parquet
      |- Device-usage
         |- airplanemode.parquet
         |- doze.parquet
         |- touch.parquet
         |- batterycharge.parquet
         |- ringmode.parquet
         |- userpresence.parquet
         |- batterymonitoringlog.parquet
         |- screen.parquet
      |- Environment
         |- light.parquet
         |- pressure.parquet
      |- Motion
         |- accelerometer.parquet
         |- gyroscope.parquet
         |- stepdetector.parquet
         |- activitiespertime.parquet
         |- stepcounter.parquet
      |- Position
         |- location_poi.parquet
         |- magneticfield.parquet
         |- location_rd.parquet
         |- proximity.parquet
|- Site_Copenhagen_DEN
   |- Diachronic-Interactions
      |-...
   |- Synchronic-Interactions
      |-...
   |- Sensors
      |-...
...
\end{lstlisting}
%
The file \texttt{matching.csv} maps the identifier of the participants in the sensor data with the identifier used in the surveys.

