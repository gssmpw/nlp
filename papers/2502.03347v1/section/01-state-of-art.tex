\subsection{Smartphone Sensing for Behavior Modeling}

Smartphone sensing data is widely used to train machine learning models to infer user behaviors in real-world environments \cite{meegahapola2020smartphone}. Developing these models involves continuously collecting data from sensors like accelerometers, gyroscopes, GPS, Bluetooth, and app usage logs to create labeled datasets. These datasets serve as ground truth for various behaviors, including daily activities, social interactions, mental states, sleep, and dietary habits \cite{meegahapola2020smartphone}. These labeled datasets fuel machine learning models, from classic approaches (e.g., random forest, decision tree, support vector machines) to deep learning techniques, enabling the detection of behavioral patterns and predictions on new, unseen data. Once trained, these models are deployed in-the-wild, running on users' smartphones to provide real-time insights into daily routines, mental well-being, and contextual changes, all without requiring active input from the user. Traditionally, such models have relied heavily on continuous sensing modalities—such as movement patterns and physical proximity—to infer user behavior. More recently, however, interaction sensing modalities (e.g., app usage, typing events, and notification interactions) have gained significance, providing rich insights into users’ engagement patterns and internal states.

\subsection{Continuous and Interaction Sensing Modalities} Smartphone sensors can be broadly classified into two types \cite{meegahapola2020smartphone}: continuous sensing and interaction sensing, both essential for understanding nuances of everyday life.

Continuous sensing involves passive data collection without direct user interaction. For instance, accelerometers and gyroscopes detect movement patterns (e.g., walking, running, or inactivity) that offer insights into physical activity levels and types, relevant for understanding behavioral patterns, energy expenditure, and emotional states, such as depression or anxiety indicated by prolonged inactivity or erratic movements~\cite{elhai2018depression, canzian2015trajectories}. Step count, derived from inertial sensors, further enhances activity monitoring, revealing daily routines and physical well-being. Additionally, proximity sensors and Bluetooth signals can indicate social contexts by identifying close proximity to others, which helps assess social isolation or engagement~\cite{meegahapola2020smartphone, lane2010survey}. GPS and WiFi data are indispensable for determining semantic locations, enabling insights into whether users are at home, at work, or engaged in recreational activities—each of which correlates with mental states or other behavioral patterns~\cite{santani2018drinksense, meegahapola2023generalization, servia2017mobile}.

Interaction sensing captures smartphone user engagement, providing insights into attention, productivity, and emotional states. App usage, for example, can infer attention spans, productivity, and even stress levels, as excessive or reduced engagement with certain apps may be linked to anxiety or depressive symptoms~\cite{guracho2023smartphone}. Studies have shown that interactions with social media, in particular, are tied to mental well-being~\cite{buda2021outliers}. Similarly, frequent typing events, rapid touch interactions, or ignoring notifications can signal heightened stress or distraction \cite{vahedi2018association}. Notifications, app-opening frequencies, and related metrics also provide important context for attention spans and can identify potentially compulsive behaviors, such as frequent app-checking, which may indicate underlying mental health concerns \cite{meegahapola2020smartphone, carlo2019numbers}.

Combining continuous and interaction sensing modalities offers a comprehensive view of behavior, allowing machine learning models to capture nuanced and expressive aspects with a high level of detail. Currently available public smartphone sensing datasets, however, often lack this level of depth and diversity in sensing modalities, limiting the ability to fully understand human behavior at scale. Even if many modalities are present, most publicly available datasets only offer pre-processed versions of data, limiting the use of such datasets for diverse purposes. Therefore, there is an unmet need for raw smartphone sensor datasets that integrate continuous and interaction sensing modalities, offering richer insights into behavioral patterns. 

\subsection{Effect of Country Diversity on Sensing and Self-reported Ground Truth}

Together, continuous and interaction-sensing modalities build a holistic picture of user behavior, capturing both passive and longitudinal patterns and immediate interactions that correlate with mental and emotional states. However, sensor data and self-reported behaviors used as ground truth for training machine learning models can vary significantly across countries, influenced by cultural, social, and environmental differences \cite{phan2022mobile}.

For example, sensor data like accelerometer readings or step counts may reflect distinct physical activity patterns based on country-specific factors, such as urban infrastructure, transportation habits, and climate \cite{ICLEI_UrbanTransport, ZeroHourClimate_UrbanPlanning, EPA_ClimateTransportation}. In countries where walking or cycling is common, step counts may show higher physical activity levels, while regions reliant on driving will have lower activity levels. Similarly, proximity sensor and Bluetooth data capture different social interaction patterns, reflecting cultural norms around personal space, social gatherings, and work environments \cite{ozella2021using, janssen2024tracking, sekara2014strength, hernandez2024proximity}. In collectivist cultures, for instance, users may exhibit more frequent close proximity with others, while individualistic cultures may show more solitary sensor readings \cite{trumbull2001bridging, triandis2001individualism}. Cultural attitudes towards technology could also shape app usage and notification interactions \cite{bombardi2017exploring}. Social media may dominate app engagement in some countries, whereas others might have more work-focused or communication-restricted technology use \cite{poushter2018social, cheng2021prevalence}. Beyond sensing modalities, the way behaviors are reported as ground truth---such as moods, stress, context, or activity types---varies culturally \cite{mesquita1992cultural, meegahapola2023generalization, sebe2005multimodal}. Cultural norms can influence self-reported stress or mood, leading to different labels for similar sensor data patterns across countries \cite{schmidt2019wearable, mesquita1992cultural, meegahapola2024m3bat}. In machine learning terms, such cross-country variations introduce challenges related to data covariate shift and label shift, both of which complicate generalization \cite{bickel2009discriminative, koh2021wilds}.

Despite these evident variations, most prior studies overlook country-level differences, relying on homogeneous datasets from specific regions, often within the Global North \cite{meegahapola2020smartphone, phan2022mobile}. This oversight leads to models that struggle to generalize effectively to diverse populations and cultural contexts. In some instances, even though cross-country data were available, analysis has not focused on country differences \cite{servia2017mobile}. Without datasets that capture country-specific data and behavioral patterns, machine learning models are likely to produce incorrect inferences when applied outside the regions in which they were trained. To improve model robustness, globally diverse datasets are essential for exploring behavioral differences across populations, allowing models to account for the variations in behaviors and sensor data across countries and cultures. Addressing this gap is one of the main objectives of the \dataset dataset.

Furthermore, collecting large-scale, passive smartphone sensing data with accurate labels is challenging due to significant costs, time, and effort, especially when collecting data from diverse international participants \cite{yfantidou2023beyond}. Datasets in this field typically involve fewer than 100 participants (see Table~\ref{tab:relatedwork}) since managing continuous data streams, conducting longitudinal data collection, and providing accurate behavioral labels make scaling these studies complex. Scaling up to engage hundreds of participants requires intensive recruitment, sustained user engagement, and technical infrastructure to handle continuous sensor data collection over extended periods. This is partly why the field of ubiquitous computing (ubicomp) lags behind fields like computer vision or natural language processing, where collecting millions of labeled images, videos, or text is relatively more straightforward and cost-effective, enabling faster progress. However, data collection across countries presents additional challenges, as cultural, behavioral, and infrastructure differences influence how people use smartphones, complicating the labeling process. Unlike images, videos, or text, which can be labeled through standardized methods or crowdsourcing, passive sensing data requires detailed, context-specific labeling of behaviors like mood, activity, and social interaction, often relying on time-consuming self-reports. Cross-country collection further complicates matters with logistical issues such as language differences, varying ethical standards, and compliance with different data privacy laws. These challenges make large-scale, diverse data collection costly and time-intensive. The \dataset dataset’s achievement in collecting data from \nilogusers participants demonstrates the commitment and resources required to overcome these hurdles.

\change{\subsection{Sociological Theory and Approaches to Study Design} 
}\subsubsection{Cultural Diversity as Social Practices} \label{subsubsec:soa-socio}
\change{The concept of culture has been extensively examined in social sciences. Previously, culture was viewed as a cohesive framework that influenced attitudes and practices through socialization \cite{swidler1986culture}. However, recent studies suggest that culture is fragmented and diverse among social groups \cite{lizardo2016dual}. This shift in perspective sees culture as complex structures of quasi-rules that individuals can use strategically \cite{bourdieu1990logic, sewell1992theory} and highlights the need to analyze relationships among various cultural influences, linking them to distinct phenomena and calling for a more nuanced psychological understanding \cite{cerulo2010mining}. }

The study design of the \dataset dataset focuses on a particular definition of culture, deriving from the social practice theory \cite{wittgenstein1953philosophical,goffman1975asylums,giddens1979central,giddens1984society,bourdieu1977outline,bourdieu1990logic,Dreyfus1991world,schatzki2001practice,reckwitz2002toward}. This theory frames human behavior as composed of daily practices that produce societal outcomes and influence individual skills and mindsets, contributing to social structure and culture. According to \citet{shove2012dynamics}, social practices consist of three key components:

\begin{itemize}
\item \textbf{Material}: The physical objects or resources, such as a car or a membership, that enable the execution of a particular practice.
\item \textbf{Competence}: The knowledge, skills, and abilities that make a certain practice possible.
\item \textbf{Meaning}: The cultural and symbolic elements that give significance to social practices, motivating individuals to perform them in alignment with societal norms.
\end{itemize}
\noindent
For instance, environmental consciousness may motivate individuals to choose public transportation, aligning with behaviors like cycling, waste sorting, or adopting a vegetarian lifestyle. These motivations, shaped by material access (such as bike availability or recycling facilities), personal competence, and meaningful commitment to sustainability, collectively define behaviors that, when widely adopted within a community, become recognized social practices \cite{shove2005consumers,ropke2009theories}. 

\change{Thus, there is a distinction between the practitioner (individual) and the social practice (community). While interconnected, social practices exist independently and are established at the social level. Indeed, \citet{ropke2009theories} noted that practices consist of recognizable, interconnected elements that individuals reproduce, with new members being constantly recruited. In this sense, individuals are ``carriers of practices'' recruited based on their backgrounds, not merely choosing practices by utility \cite{reckwitz2002toward}, and the distribution of practices often reflects social inequality, with varying cultural perceptions of what constitutes ethical practice. Participation in a practice leaves lasting effects, such as knowledge and skills, which facilitate future involvement in that practice in a path-dependent process~\cite{ropke2009theories}. Finally, diversity is socially recognized, and practices are inherently social, resembling one another across different contexts \cite{reckwitz2002toward}. In other words, social practices exhibit regularity - models of how certain daily practices are typically and habitually performed in (a considerable part) of a society \cite{holtz2014generating}.}

\change{This perspective enhances our understanding of human behavior in real-world environments, where the activities recorded by sensors may be limited in terms of quantity, quality, and context representation. Identifying whether an individual belongs to a community of practice and their ``career'' within it enables the inference of important characteristics, such as the nature of their activities—distinguishing, for example, between an athlete's training session or recreational activity—the tools utilized, such as appropriate clothing and supportive technologies, and the likelihood of recurring behavioral patterns over time. This understanding is particularly valuable in situations where data may be limited or less accurate.}

\change{Furthermore, gaining insight into the context in which these practices occur allows for culturally informed comparisons. This encompasses both the individual perspective, which considers the factors that may facilitate or impede engagement in a practice or access to a community, and the social perspective, which examines how the reference community typically interacts with that practice.}

\change{\subsubsection{Capturing Social Practices}} \label{subsubsec:soa-meth}
Researchers utilize various methodologies to effectively capture behaviors and lifestyles over time. In European countries, for example, the Harmonized European Time Use Surveys (HETUS) measure time spent on various activities. At the same time, the Experience Sampling Methodology (ESM) focuses on the interoceptive, or internal, aspects of behavior \cite{csikszentmihalyi2014validity, myin2022esm}. Time-use diaries, commonly used in HETUS, are intensive longitudinal surveys where participants self-report activity sequences over a day, detailing the frequency and duration of their actions to reveal intricate social patterns. Participants typically complete these diaries at regular intervals, tracking each activity, which enables a detailed, objective view of daily routines and interactions \cite{sorokin1939time}. In contrast, ESM prompts participants to report their thoughts and behaviors over days or even months, detailing their personal points of view on their daily experiences. A comprehensive review by \citet{van2017experience} highlights the strengths of ESM in capturing these aspects.

\change{Our approach combines these two perspectives, namely time-use diaries and ESM with smartphone sensor data. For every sensor pattern that indicates an action or habit—such as accelerometer data, Movement Activity Labels, or GPS positions—researchers can reconstruct the individual's social and personal context. 
A study review explores this combination of continuous sensing and interaction sensing with self-report data for the well-being of young adults~\cite{meegahapola2020smartphone}.
Here, self-report data can serve as ground truth events. Conversely, in scenarios like interactive classification in the wild \cite{bontempelli2020learning}, algorithms such as skeptical learning can use sensor information to validate user annotation. Furthermore, this approach enables the researcher to observe human behavior from the individual point of view within the community of practice, specifically her understanding and interpretation of the context in which the activity occurs (see, e.g., \cite{zhang2021putting}).} 


\subsection{Currently Available Public Smartphone Sensing Datasets}

\begin{table}[btp]
\caption{\label{tab:relatedwork} Public available datasets for activity and context recognition in the wild. Duration is expressed in days. The number of pilot sites indicates whether the countries are in the Global North (N) and/or Global South (S). (+) Sensors are divided between smartphones and (+) other devices. No '+' means only smartphone sensors. (*) The GLOBEM data collection was done during a semester for 10 weeks each year, for 4 different years.}
\begin{tabularx}{\textwidth}{lrrcXr}
\toprule
\textbf{Datasets}%
& \multicolumn{3}{c}{\textbf{Coverage}}%
& \multicolumn{2}{c}{\textbf{Purpose}}\\  
\cmidrule(rl){2-4}\cmidrule(rl){5-6}
 & \multicolumn{1}{c}{\textbf{Sample}}         & \multicolumn{1}{c}{\textbf{Days}} & \multicolumn{1}{c}{\textbf{Sites}} & \multicolumn{1}{c}{\textbf{Self-reports}}                          & \multicolumn{1}{c}{\textbf{Sensors}}   \\
\midrule
MDC (2013)   \cite{laurila2013}               & 185                      & 365                      & 1 N                 & Location   
                                        & 12+14                     \\
StudentLife  (2014)  \cite{wang2014studentlife}          & 48                      & 70                      & 1 N                 & Sleep-related                                 & 10                    \\
%MobiAct   (2016) \cite{vavoulas2016mobiact}              & 57                      & Trials                  & 1                 & .                                             & 3                     \\
ExtraSensory  (2017)  \cite{vaizman2017recognizing}      & 60                      & 7                       & 1 N                & Activity                                      & 8+2                 \\
Real-life HAR  (2020)  \cite{gonzalez2020}               & 19                      & 28                      & 1 N                &  .                                            & 4                     \\
ContextLabeler  (2021)  \cite{campana2021contextlabeler} & 3                       & 14                      & 1 N                 & Activity                                      & 18                    \\
Qwantify  (2022)   \cite{wilson2022qwantify}          & 242                     & 7   & 1 N                & Desire, emotion,    well-being                & . \\
LifeSnaps  (2022) \cite{yfantidou2022lifesnaps}           & 71                      & 120 & 4 N & Location, Mood, Step Goal & 0+23 \\
ETRI  (2022) \cite{chung2022real}                         & 22                      & 28                      & 1 N               & Activity, Location,   Relation, Mood          & 10+4                \\
SmartUnitn2  (2018)  \cite{li2022representing}           & 158                     & 28                      & 1 N               & Activity, Location,   Relation, Mood          & 28                    \\
LAUREATE   (2023) \cite{laporte2023laureate}             & 42                      & 91                      & 1 N                 & Activities, Health                            & 0+6                   \\
GLOBEM  (2023) \cite{xu2023globem}                        & 534                     & 280*                     & 4 N              & Standard scales (personality, physical, mental and social well-being)                             & $\sim$10              \\
EgoADL  (2024)  \cite{sun2024multimodal}                 & 30                      & 5                       & 1 N                & Automated labels                              & 4+2                   \\


 & & & & & \\
\textbf{\dataset}                                               & \textbf{\nilogusers}            & \textbf{28}             & \textbf{8 (3N,5S)}        & \textbf{Activity, Location,   Relation, Mood} & \textbf{26}   \\        
\bottomrule
\end{tabularx}
\end{table}

Several datasets that leverage smartphone and smartwatch sensors for activity and context recognition have been developed across various research fields. \cref{tab:relatedwork} compares key publicly available datasets, highlighting their sample size, duration, data collection sites, self-reported questionnaires or annotations via intensive longitudinal survey, and number of sensors used, both collected via smartphone and other devices. While these datasets contribute valuable insights, most are constrained by small sample sizes and short data collection durations, which limit their ability to capture daily routines and behavioral patterns. For instance, although datasets like MDC~\cite{laurila2013}, StudentLife~\cite{wang2014studentlife}, and Real-life HAR~\cite{gonzalez2020} extend beyond short-term studies, they still lack the diversity required to observe routine behaviors that often necessitate more extended observation periods and cross-country differences.


Datasets that focus solely on continuous sensor data, such as Real-life HAR \cite{gonzalez2020}, are limited in capturing the complexity of human activities because they do not include annotations reflecting users' experiences or contexts. Moreover, a growing trend involves using experience sampling methodologies (ESM), where users frequently self-report their experiences, to capture behaviors and contextual nuances. Datasets like the Qwantify app~\cite{wilson2022qwantify} provide valuable insights into emotions, desires, and well-being. Still, sensor data in these collections often remain secondary, serving mainly as context rather than central to the analysis. Hence, \cancel{we believe it is fair to say that}the most valuable datasets combine sensor data with self-reports about the collected data. Such datasets can be further divided into those with in-situ (reports provided about the current moment) and retrospective (reports provided about past periods or days) self-reports~\cite{meegahapola2020smartphone}. A notable example of retrospective self-reporting is the work of \citet{krumm2013placer}, which uses time diaries (i.e., similar to the HETUS approach described in \cref{subsubsec:soa-meth}) for labeling. However, the lack of direct user interaction during labeling can lead to inaccuracies and contextual inconsistencies. In contrast, datasets like StudentLife~\cite{wang2014studentlife}, ExtraSensory~\cite{vaizman2017recognizing}, and ContextLabeler~\cite{campana2021contextlabeler} rely on in-situ self-reports from users. Still, they are often limited to specific activities and do not adhere to a standardized reference, which introduces social and cognitive biases.

GLOBEM \cite{xu2023globem} presents a large-scale dataset that enables cross-dataset generalization analysis, which is particularly valuable for examining behavioral patterns across different periods and university environments. Its longitudinal nature, spanning over \change{four years (ten weeks of data collection each year)}, allows researchers to assess how models perform across diverse academic settings over time. However, the dataset predominantly focuses on the USA, limiting its applicability for cross-country studies. Hence, while GLOBEM provides essential insights, its geographic concentration contrasts with the more globally diverse scope of \dataset, which includes participants from eight countries across the global north and south, offering a broader foundation for generalization and cross-country behavior modeling. Moreover, \dataset provides more fine-grained and raw sensor data and frequent in-situ self-reports, giving researchers much flexibility and depth in their analysis. 

Among existing datasets, LifeSnaps \cite{yfantidou2022lifesnaps} is the most comparable to \dataset in terms of countries of data collection. However, despite the richness, the dataset falls short in crucial areas such as sample size, diversity of data collection countries, and breadth of sensor coverage. LifeSnaps focuses primarily on four European countries and only has a sample size of 71 participants, whereas \dataset spans both the global north and south, including a broader cultural and geographic range, with \nilogusers participants. This comprehensive scope makes \dataset a unique resource for understanding everyday life behavior across diverse countries, surpassing other available datasets in terms of depth and scale.


\subsection{The \dataset Dataset}
\change{Our dataset aims at making a valuable contribution by differing significantly from previous works in several key aspects, as outlined below. First, as described in \cref{subsubsec:soa-meth}, our dataset methodology follows best practices from earlier studies by measuring behaviors and psychosocial traits using standardized and validated methods. We propose a combination of self-reported annotations and sensor data from smartphones. This integration enables a nuanced understanding of social practices, considering routine behaviors alongside contextual data about both physical actions and mental states. While many current datasets aim to capture such elements, few achieve the same depth of blending multiple methodologies with sensor data. This multifaceted approach allows for a more accurate, holistic view of human behavior, providing fine-grained labels essential for training robust machine learning models that reflect the complexity of everyday life. Second, to the best of our knowledge, there are currently no publicly available smartphone datasets specifically tailored to examining social practices (see \cref{subsubsec:soa-socio}). This innovative framework allows researchers to enrich their understanding of individuals by situating them within a broader community of profiles that share similar competencies, materials, and meanings. For instance, it distinguishes between those engaged in professional activities and those pursuing recreational interests, highlighting each group's distinct personalities and values. This nuanced approach encourages researchers to explore the diversity of daily activities in a way that emphasizes the cultural context of behaviors, thereby creating a more comprehensive understanding of how social practices shape individual actions and interactions.}

\change{Finally, our sample size and the breadth of data collected internationally, with a particular emphasis on regions in the global south —underrepresented in prior research— is rather unique. In addition to our theoretical and methodological approach, these characteristics enable a more comprehensive exploration of human behavior, shedding light on the diversity among people and their similarities. In summary, by integrating diverse cultural and socio-demographic factors, \dataset facilitates a deeper exploration of behavior modeling, personalization, and cross-cultural adaptation at a more nuanced level. Moreover, incorporating these factors fosters cross-cultural and multidisciplinary analysis of human behavior rather than promoting it solely as a machine learning benchmark dataset~\cite{orr2024ai,raji2021ai}.}
