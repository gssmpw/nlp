Human behavior, routines, habits, and social practices are deeply interwoven with the events, places, interactions, and technologies that compose our everyday lives. Each of these elements contributes to larger behavioral patterns influenced by our continuous engagement with pervasive devices. Our emotional experiences fluctuate daily, shaped by factors such as sleep quality \cite{khalid2024sleepnet}, which, in turn, impact our physical activity, forming both recurring weekly patterns \cite{tseng2016assessing} and longer-term habits \cite{harari202019}. Everyday actions, such as eating and drinking, also have significant effects on our health \cite{biel2018bites, santani2018drinksense}, with additional factors like interactions with technology playing important roles in everyday life  \cite{das2019multisensor}.


Smartphones, as prevalent devices in daily life, influence our behavior, often in nuanced ways that can be both positive \cite{lin2021revisiting} and negative \cite{2017-SOCINFO}. Their use is linked to our internal states \cite{elhai2018depression, buda2021outliers}, and understanding how people engage with their smartphones could provide valuable insights into human behavior. Prior research has shown that smartphone sensors can infer a range of behavioral aspects in young adults, including mood \cite{servia2017mobile, meegahapola2023generalization}, depression \cite{xu2023globem}, activities \cite{haresamudram2021contrastive, assi2023complex}, social context \cite{mader2024learning, hernandez2024proximity}, and eating or drinking episodes \cite{bae2017detecting, santani2018drinksense, thomaz2016automatic, biel2018bites}. This makes smartphone sensor data valuable for improving app design, supporting user well-being, and creating more effective interventions. However, there are profound variations in daily behaviors across different cultural and geographic backgrounds \cite{foner2020introduction}\footnote{\change{We operationalize the concept of culture through the theory of social practices outlined in \cref{subsubsec:soa-socio}. This concept recognizes that cultures are complex and multidimensional, often coexisting or overlapping within the same geographic areas \cite{yuval2004gender}. Additionally, it acknowledges that certain regularities emerge from specific socio-cultural contexts or from the communities of practice that develop locally. The use of terms like ``country,” ``culture,” and ``geographic region” throughout the paper reflects this distinction.}}. Ignoring these variations has often hindered model generalization \cite{meegahapola2023generalization, khwaja2019modeling}, limiting the effectiveness of behavior inference models when applied in diverse real-world settings.


Despite the richness of smartphone sensor data, a critical challenge persists: the lack of datasets that capture behavioral diversity across different cultural and geographic contexts. Existing datasets and research often draw from specific populations typically in the Global North \cite{khwaja2019modeling, meegahapola2023generalization}, and may not fully account for the cultural, environmental, and social norms that shape daily routines and smartphone usage patterns in other regions \cite{phan2022mobile}. This limits the development of machine learning models capable of generalizing across different populations. For instance, eating habits \cite{tobin2018dinner}, sleep routines \cite{stacker2023sleep, cheung2021considering}, and social interactions \cite{gsir2014social, parady2021comparative} vary widely between countries, affecting how individuals interact with their smartphones and how other daily life aspects are inferred from sensor data \cite{lopez2017self, phan2022mobile}. Consequently, models trained on data from one country or region may perform poorly when applied to another. This underscores a fundamental challenge in generalizing machine learning models to real-world applications \cite{meegahapola2023generalization, assi2023complex}. Addressing this challenge requires comprehensive datasets that not only capture a variety of sensor data but also integrate cultural, social, and geographical diversity. Such datasets would enable researchers to study behavioral differences and similarities across populations, to test model robustness in diverse contexts beyond typical experimental settings\change{, and to investigate how to combine model generalization and adaptation to local data to capture local and individual behaviors.} 

The \dataset dataset, developed as part of the large-scale European project \textit{``WeNet - The Internet of US''}\footnote{\url{https://doi.org/10.3030/823783}.} \cite{2025WenetPaper},
%\footnote{Details to be disclosed post-double-blind review},
provides rich, cross-country data for studying everyday life behavior through smartphone sensors and self-reports. \dataset spans data from college students in eight countries: 
{China} (\jlu, \textbf{\JLU}),
{Denmark} (\aau, \textbf{\AAU}), 
{India} (\amrita, \textbf{\AMRITA}), 
{Italy} (\unitn, \textbf{\UNITN}), 
{Mexico} (\ipicyt, \textbf{\IPICYT}), 
{Mongolia} (\num, \textbf{\NUM}), 
{Paraguay} (\uc, \textbf{\UC}), and 
the {United Kingdom} (\lse, \textbf{\LSE}). 
The study involved more than 18,000 students, of which \nilogusers agreed to participate in an intensive longitudinal survey of four weeks. Based on the innovative \texttt{iLog app}~\cite{2014-PERCOM}, adopted for the data collection, \dataset includes raw data from \nsensors smartphone sensors, such as accelerometers, gyroscopes, and GPS, as well as derived information like notification interactions, app usage, activities, and step counts, during these four weeks. These data streams are organized into six category bundles (connectivity, environment, motion, position, app usage, and device usage), allowing modular downloads for ease of use. Additionally, participants provided multiple in-situ self-reports each day, through time diaries, detailing their activities, locations, social contexts, and moods, together with daily reports on sleep quality and daily expectations. This combination of self-reported data and sensor data provides a comprehensive view of behavior, making \dataset a valuable resource for studying the impact of smartphone technology across various contexts in both the Global North and the Global South \cite{biel2018bites, khalid2024sleepnet, qin2019cross, grammenos2018you}. \change{Additionally, DiversityOne enables more computational social science-oriented studies focused on human behavior (see, e.g., \cite{2017-SOCINFO, zhang2021putting, bontempelli2020learning}).}

The \dataset dataset supports the development of machine learning models capable of inferring user behaviors, with potential applications in domain generalization, domain adaptation\change{, transfer learning}, and self-supervised learning. It is a valuable research resource in ubiquitous computing, human-computer interaction, and machine learning, offering a broader scale than previously available datasets. Furthermore, \dataset adheres to the European General Data Protection Regulation (GDPR) \cite{GDPR2016}, with ethical approvals from all participating institutions. The study design and data collection processes were developed by an interdisciplinary team of computer scientists, social scientists, interaction designers, ethicists, and legal experts to meet the highest standards. This extensive dataset has been used in a few initial publications by different authors, showing its various uses \cite{meegahapola2023generalization,  kammoun2023understanding, assi2023complex,meegahapola2024m3bat, mader2024learning,girardini2023adaptation,mercado2023social}, and demonstrating its potential for reuse in studies aiming to understand behaviors across countries through smartphone sensing. The paper describes three contributions:

\begin{itemize}[wide, labelwidth=!, labelindent=0pt]

\item[\textbf{Contribution 1:}] We publicly release \dataset, one of the largest and most geographically diverse datasets combining questionnaires from 18K+ participants and passive smartphone sensor data and self-reports from \nilogusers participants across eight countries, covering both the Global North and the Global South. The dataset includes raw sensor data from \nsensors modalities, providing researchers with a flexible resource for analyzing and developing machine learning models across multiple domains, such as behavior recognition, cross-cultural studies, and multimodal time-series modeling. This dataset facilitates explorations of topics like model generalization and domain adaptation across countries, with detailed self-reports on demographic and psychosocial variables, adding depth for nuanced, multifaceted analyses that can be conducted independently or integrated with sensor data.

\item[\textbf{Contribution 2:}] We provide an in-depth description of the study design and data collection, which was tailored to enable effective data gathering across multiple countries, each with unique languages, privacy regulations, and cultural contexts (Section~\ref{sec:method}). To manage the diversity and sensitivity of the collected data, the dataset was consolidated under rigorous privacy and ethical standards, hosted in a GDPR-compliant secure environment, and respecting the ownership rights of all contributors.


\item[\textbf{Contribution 3:}] We synthesize key insights, lessons learned, and recommendations derived from some initial studies leveraging \dataset, highlighting unique opportunities for further studies in smartphone sensing, behavioral modeling, and machine learning. We provide the insights under two broader themes---Lessons learned from (i) study design and cross-country data collection (\cref{subsec:lessons_design}) and (ii) multi-country sensor data analysis (\cref{subsec:lessons_analysis}). Our 12 recommendations, which we see as actionable, aim to provide ideas for future research, enhancing the adaptability and accuracy of smartphone-based behavioral studies in diverse global contexts.

\end{itemize}


The remainder of the paper is organized as follows: \cref{sec:relatedworks} provides background and context for the methodology used to create \dataset, reviewing existing datasets for modeling everyday behaviors through smartphone sensors alongside various forms of ground truth. \cref{sec:method} details the methodology and protocol adopted to collect diverse, cross-country data, with specific attention to variations in cultural contexts of student's daily routines and data privacy regulations. \cref{sec:validation} presents key results to validate the data collection approach and the dataset's quality. \cref{sec:availability} outlines the resources available for accessing and using the dataset, highlighting compliance with privacy and copyright standards. \cref{sec:discussion} explores potential use cases for the dataset, provides insights and recommendations, emphasizing its value in studying behavioral patterns across different regions and advancing machine learning applications. Lastly, \cref{sec:conclusion} summarizes the paper’s contributions and impacts.