Due to the variety of sources and the sensitive nature of the content, this dataset is made available in a secure environment that complies with GDPR regulations. The main entry point documentation can be accessed in the dataset catalog:

\vspace{1em}
\tikzstyle{background rectangle}=[thin,draw=black]
\begin{center}
\begin{tikzpicture}[show background rectangle]
\node[align=justify, text width=30em, inner sep=1em]{
\url{https://livepeople.disi.unitn.it}
};
\node[xshift=3ex, yshift=-0.7ex, overlay, fill=white, draw=white, above 
right] at (current bounding box.north west) {
\textit{Link to the dataset catalog}
};
\end{tikzpicture}
\end{center}

\noindent
The catalog simplifies the process of finding and selecting datasets of interest by providing metadata and materials to explore their features. \change{The filter options allow to restrict the search on \dataset data.} It also ensures privacy and copyright compliance, fully respecting the ownership of the authors who contributed to its collection. \change{In addition to the catalog, the webpage available at \url{https://datascientia.disi.unitn.it/projects/diversityone/} collects further information.} The descriptions of downloadable datasets, associated documentation, and procedures for dataset requests are detailed below.

\subsection{Downloadable Datasets and Bundles}

\begin{table}[tb]
    \centering
    \small
    \caption{\change{The size of each bundle of data in Gigabytes, with uncompressed sizes in brackets, and bundle names as they appear in the catalog are presented. Synchronic and diachronic interaction refers to questionnaires and time diaries, respectively.}}\label{tab:sizes}
    \iffalse 
\begin{tabular}{lllllllll}
\toprule
Bundle & \AAU & \AMRITA & \IPICYT & \JLU & \LSE & \NUM & \UC & \UNITN \\
\midrule
App-usage & $<$ 0.1 (0.2) & $<$ 0.1 (0.2) & $<$ 0.1 (0.4) & $<$ 0.1 (0.3) & $<$ 0.1 (0.7) & 0.1 (2.3) & $<$ 0.1 (0.3) & 0.3 (2.8) \\
Connectivity & 0.1 (5.7) & $<$ 0.1 (0.4) & $<$ 0.1 (1.2) & $<$ 0.1 (2.5) & 0.4 (16.8) & $<$ 0.1 (3.7) & $<$ 0.1 (2.8) & 0.6 (26.6) \\
Device-usage & $<$ 0.1 (0.3) & $<$ 0.1 (0.2) & $<$ 0.1 (0.2) & $<$ 0.1 (0.4) & $<$ 0.1 (0.9) & 0.2 (2.3) & $<$ 0.1 (0.4) & 0.5 (4.4) \\
Environment & 0.3 (2.6) & 0.1 (1.4) & 0.6 (6.4) & 0.6 (5.6) & 1.5 (12.1) & 3.3 (35.2) & 0.5 (5.4) & 5.2 (44.5) \\
Motion & 1.9 (18.0) & 1.3 (13.2) & 4.0 (36.8) & 1.2 (12.4) & 7.0 (61.3) & 17.1 (161.3) & 1.8 (19.9) & 18.0 (174.8) \\
Position & 0.6 (5.2) & 0.6 (5.5) & 1.3 (10.2) & 0.1 (1.7) & 2.9 (23.0) & 6.8 (54.7) & 0.8 (7.6) & 12.7 (104.8) \\
Questionnaires & $<$ 0.1  & $<$ 0.1  & $<$ 0.1  & $<$ 0.1  & $<$ 0.1  & $<$ 0.1  & $<$ 0.1  & $<$ 0.1 \\
Time diaries & $<$ 0.1  & $<$ 0.1  & $<$ 0.1  & $<$ 0.1  & $<$ 0.1  & $<$ 0.1  & $<$ 0.1  & $<$ 0.1  \\
\midrule
Total & 2.9 & 2.1 & 6.0 & 2.1 & 12.0 & 27.7 & 3.3 & 37.3 \\
\bottomrule
\end{tabular}
\fi 


\begin{tabular}{lllllllll}
\toprule
\textbf{Bundle name} & \textbf{\JLU} & \textbf{\AAU} & \textbf{\AMRITA} & \textbf{\UNITN} & \textbf{\IPICYT} & \textbf{\NUM} & \textbf{\UC} & \textbf{\LSE} \\
\midrule
App-usage & $<$ 0.1 (0.3) & $<$ 0.1 (0.2) & $<$ 0.1 (0.2) & 0.3 (2.8) & $<$ 0.1 (0.4) & 0.1 (2.3) & $<$ 0.1 (0.3) & $<$ 0.1 (0.7) \\
Connectivity & $<$ 0.1 (2.5) & 0.1 (5.7) & $<$ 0.1 (0.4) & 0.6 (26.6) & $<$ 0.1 (1.2) & $<$ 0.1 (3.7) & $<$ 0.1 (2.8) & 0.4 (16.8) \\
Device-usage & $<$ 0.1 (0.4) & $<$ 0.1 (0.3) & $<$ 0.1 (0.2) & 0.5 (4.4) & $<$ 0.1 (0.2) & 0.2 (2.3) & $<$ 0.1 (0.4) & $<$ 0.1 (0.9) \\
Environment & 0.6 (5.6) & 0.3 (2.6) & 0.1 (1.4) & 5.2 (44.5) & 0.6 (6.4) & 3.3 (35.2) & 0.5 (5.4) & 1.5 (12.1) \\
Motion & 1.2 (12.4) & 1.9 (18.0) & 1.3 (13.2) & 18.0 (174.8) & 4.0 (36.8) & 17.1 (161.3) & 1.8 (19.9) & 7.0 (61.3) \\
Position & 0.1 (1.7) & 0.6 (5.2) & 0.6 (5.5) & 12.7 (104.8) & 1.3 (10.2) & 6.8 (54.7) & 0.8 (7.6) & 2.9 (23.0) \\
Synchronic int. & $<$ 0.1 & $<$ 0.1 & $<$ 0.1 & $<$ 0.1 & $<$ 0.1 & $<$ 0.1 & $<$ 0.1 & $<$ 0.1 \\
Diachronic int. & $<$ 0.1 & $<$ 0.1 & $<$ 0.1 & $<$ 0.1 & $<$ 0.1 & $<$ 0.1 & $<$ 0.1 & $<$ 0.1 \\
\midrule
Total & 2.1 & 2.9 & 2.1 & 37.3 & 6.0 & 27.7 & 3.3 & 12.0 \\
\bottomrule
\end{tabular}

\end{table}

The resources are organized and made available separately considering the full dataset size (approximately 94GB in Parquet format) and GDPR’s minimization principle—which mandates that data must be adequate, limited, and relevant for analysis. Researchers can request access to basic datasets from a specific pilot site (e.g., time diaries collected at \UNITN) or a combination of datasets from multiple pilot sites. To streamline dataset selection, we have created thematic bundles that group data commonly used together for main research purposes. For example, the motion bundle includes all motion sensor data relevant to activity recognition studies, while another bundle, combining questionnaires, time diaries, and location data, is tailored for \cancel{social science research}\change{studying social interactions}. \change{The catalog lists both datasets containing one single sensor and bundles.} \cref{tab:sizes} reports the available bundles (\cref{subsec:ils} details the sensors they contain), and Appendix~\ref{app2:sensors} reports the complete list of sensors in each bundle. \change{Any combination of bundles and single sensors can be downloaded.} All datasets are provided in Parquet format\footnote{Apache Parquet \url{https://parquet.apache.org/}}, an efficient storage format with high compression.
%Larger datasets are divided into manageable chunks to facilitate easier downloading and analysis.
Access to the entire \dataset dataset is also available upon request. 
No source code needs to be made available for the release of this dataset. Future benchmarked datasets derived from our raw dataset will be made available, respecting privacy and copyright, along with code for pre-processing and machine learning-based modeling.

\subsection{Metadata and Documentation}
Each single sensor dataset and bundle is accompanied by metadata and comprehensive documentation that outline content, size, format, and other relevant information.
The catalog user can search the datasets through the metadata values such as the acronym, data collection location, and type of bundle or dataset. The metadata for each dataset and bundle includes:
%
%\begin{enumerate}
    \textit{(i)} A technical report detailing the data collection process;
    \textit{(ii)} Dataset metadata and a codebook with summary statistics;
    \textit{(iii)} Data collection information and links to related projects, including articles published using the dataset;
    \textit{(iv)} Documentation and procedures for dataset requests.
%\end{enumerate}

The catalog enhances the findability, accessibility, and reusability of the dataset in several ways. First, it provides an efficient search method for specific data within a secure institutional environment. Second, it includes concise descriptions that guide users through the available resources, along with detailed information on each dataset, including variable values and labels. Lastly, the technical report and dataset descriptions enable users to analyze the dataset and replicate the data collection process for their own research needs.

\subsection{Dataset Request Process}

\change{We outline the procedure to request the data.
%\begin{enumerate}
First, the user retrieves the identifiers of the bundle or single sensor data of interest from the metadata shown on the catalog for each sensor and bundle.
Second, the metadata also includes a link to the request form, designed together with legal and privacy experts to comply with GDPR and privacy regulations. Interested users affiliated with a research institution can request bundles and datasets by listing the identifiers selected in the previous step and presenting a research proposal. Upon completing the form, users submit their request to the catalog manager using the email address specified in the metadata.} Detailed eligibility criteria and required information are available on the catalog website.
Third, once approved, users must sign a \change{Terms} and License Agreement. Key licensing terms include: \textit{(i)} datasets \cancel{may be}\change{are} used exclusively for research purposes; \textit{(ii)} redistribution of the datasets is prohibited; \textit{(iii)} datasets cannot be publicly shared (e.g., on a website); and \change{\textit{(iv)} any attempt to reverse engineer any portion of the data or to re-identify the participants is strictly forbidden and could constitute unlawful processing of personal data. Finally, the catalog manager sends the instructions for downloading the dataset.}
