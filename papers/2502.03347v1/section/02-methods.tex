%% This section details data collection protocols such as apparatus, participants (inclusion/exclusion criteria), duration, and environments. The authors may include technical and scientific aspects related to tools, procedures, and best practices which enabled the dataset to be collected and curated. Following the IMWUT’s policy on Ethics, the authors should explicitly include the ethics statements in the manuscript; i.e., the authors complied with their institutional ethical regulations, and informed consent that the dataset will be publicly released was obtained from all the participants. Any supplementary data that are useful for further research analyses are welcome (e.g., challenges and dropouts in data collection and anonymized video data associated with IMU-based activity datasets).

The design and collection of this dataset posed the challenge of ensuring consistent data comparability between different countries while mapping social practices in a culturally sensitive and diversity-aware manner. Behaviors and practices vary considerably according to time of day and location and are shaped by different cultural and social factors. Local cultural variations also impact ethical and privacy considerations, influencing the adoption of data collection protocols, legal frameworks, and public perceptions. Furthermore, the diversity of participating organizations and the resources available in each context significantly shaped the data collection process.

\begin{figure}[!hbt]
    \includegraphics[width=\textwidth]{figures/2024-IMWUT-Protocol.pdf}
    \Description[Overview of the study set-up and data collection]{The upper part shows the sequence of steps of the study protocol: design, adaptation, approval, test, execution, and consolidation. The lower part shows the data collection timeline: 1 month for the participant invitation and first questionnaire, 1 week for participant selection and second questionnaire, 2 weeks for the first phase of the data collection, 2 weeks for the second data collection, and 1 week for the third questionnaire and closing procedures.}
    \caption{Study set-up and data collection process. Invitation, Selection, and Closing procedures were done using the LimeSurvey platform. The 1st Phase and 2nd Phase were done using the iLog mobile app.}
    \label{fig:ExpPhasesDetailed}
\end{figure}
\noindent
\cref{fig:ExpPhasesDetailed} shows the study set-up and the survey protocol, as applied consistently across all sites, with minor adjustments which still maintained comparability (refer to \cref{subsec:protocol}). The protocol was developed with an adaptive approach which allowed to address local perspectives and requirements. A team of experts, including computer scientists, social scientists, interaction designers, incentive designers, ethicists, and legal representatives, developed the study design, survey protocol, and procedures to ensure compliance with ethics and privacy standards. The experiment partners then adapted these materials in a way to align with local contexts. Each adapted version was validated by the initial team or an authorized body to ensure functionality, adherence to standards, and comparability.

For the study design, \UNITN and \LSE identified the need to measure human behavior over time by adopting both questionnaires and an intensive longitudinal survey approach \cite{iida2023using, mcneish2021measurement} enabled by an adapted version of the iLog app \cite{2014-PERCOM}\footnote{iLog is an app designed to collect intensive longitudinal survey data (in the form of daily self-reports or hourly time diary self-reports) along with passive smartphone sensor data. The initial version of the app, made available by Zeni et al. \cite{2020-zeni1,2014-PERCOM}, was modified with new features to facilitate the collection of the \dataset dataset.}. The content of the questions and the technology were tailored to address local needs (refer to \cref{subsec:questionnaire,subsec:ils}). The questions from the iLog app and the questionnaires were subsequently translated into multiple languages, as detailed in \cref{subsec:questionnaire}. \change{Furthermore, this type of intensive longitudinal survey often faces challenges related to participant dropout, which can impact the quality of the results. According to~\cite{de2014dropouts}, dropout in longitudinal studies can originate from \textit{(i)} failure to locate the research unit, \textit{(ii)} failure to contact the potential respondent, and \textit{(iii)} failure to obtain cooperation.}

\change{Moreover, repeated assessments can become burdensome, potentially leading to decreased compliance over time. However, according to \citet{wrzus2023ecological}, factors such as the total number of assessments or the frequency of assessments do not impact compliance. Indeed, the burden on respondents can be mitigated, as participants who agree to participate are already willing to adhere to demanding schedules. There is a consensus that offering monetary incentives can enhance participation rates. \citet{wrzus2023ecological} also discusses how it is possible to hypothesize that including break options could improve compliance, although their meta-analysis did not find evidence to support this idea. }

\change{A survey protocol was developed in response to these considerations. This protocol involved sending invitations on the same day to all students enrolled at various universities via their institutional email accounts, addressing the issues of locating research units, and contacting potential respondents. We tackled the challenge of obtaining cooperation by ensuring the participants received all the necessary information in advance and incentivizing participation in the one-month survey with monetary compensation. Adjustments were made to the invitation language, sample selection, and incentive amounts to better align with local requirements, as detailed in \cref{subsec:sample}.}

Data collection involved handling personal data across both European and non-European countries. European legislation mandates that any processing of data related to European citizens or conducted using tools based in Europe must comply with the General Data Protection Regulation (GDPR). Consequently, \UNITN prepared procedures and documentation to ensure GDPR compliance. These processes were then adapted for different contexts to ensure effective data management, clear communication with participants, and adherence to privacy requirements where necessary (refer to \cref{subsec:privacy}).

The following sections outline the strategy for adapting measurement tools, privacy protocols, and methods to collect high-quality data for studying diversity in human behavior and social practices over time. Each section addresses both the general approach and specific local adaptations.

\subsection{Questionnaires about Demographics and Social Practices} \label{subsec:questionnaire}

The administered questionnaires measured various aspects of social practices and identified communities of practice relevant to students. Given the wide range of social practices, we focused on a specific set, including social relationships (both online and offline), cultural and sports activities, daily commuting, and shopping and cooking habits. The questionnaire items were designed to capture elements of competence, materiality, and meaning associated with these practices (see \cref{sec:relatedworks}). Each question and scale thus provide foundational information on components of social practices. For example, a profile of an individual's shopping and cooking habits can be built by examining the types of food purchased (materials), cooking skills (competencies, e.g., buying convenience foods \textit{vs.} cooking daily), and motivational values (meaning, e.g., hedonism \textit{vs.} universalism). This combination of components yields a nuanced measure of specific social practices, allowing cross-country comparisons to identify diverse practices. Due to the extensive number of questions, we divided the questionnaire into three parts and administered it throughout the data collection, as mentioned in \cref{fig:ExpPhasesDetailed}.

\begin{enumerate}
    \item The \textbf{First Questionnaire} was sent via an invitation email to all students at each university participating in the study. This questionnaire covered demographic information, cultural interests, and leisure activities, as well as questions about online and offline social interactions. It also included standard scales for measuring personality traits using the \textbf{Big Five Inventory}~\cite{donnellan2006mini} and attitudes toward values measured through the \textbf{Basic Values Survey}~\cite{gouveia2014functional}.
    %
    \item The \textbf{Second Questionnaire} was administered only to participants involved in the sensor and survey data collection. It primarily explored specific social practices such as daily commuting, cooking, grocery shopping, and physical activities. Additionally, a second set of personality questions based on the \textbf{Jungian Scale for Personality Types} \cite{jung1971psychological,briggs1995gifts,mascarenas2016jungian,wilde2011jung} was included, along with a second set of values questions based on the \textbf{Human Values Survey} \cite{schwartz1994there,schwartz2001extending}.
    %
    \item The \textbf{Third Questionnaire} was administered to the same participants as the second questionnaire. It gathered feedback on the user's experience with the data collection app and assessed a \textbf{Multiple Intelligence Scale}~\cite{tirri2008identification}.
\end{enumerate}
\noindent
The structure of the questions reflects the goal of measuring different components of social practices. We adopted standard scales to assess aspects related to meaning—often intangible and challenging to quantify—while additional questions aimed at understanding the other two components of social practices, namely competencies and materials. These components were evaluated through questions about specific practices, such as commuting, sports, cultural activities, and cooking and shopping habits.
To administer the questionnaires, we used \textit{LimeSurvey}~\cite{LS}, a tool widely used in social sciences to protect user privacy. The full questionnaires,  translations and adaptations are available in the data catalog, see \cref{sec:availability}.

\subsubsection{Questionnaire Translation, Testing, and Adaptation}
Each experiment site received the English version of the questionnaire. This version was later translated into the local language and adapted to the local context. We used a validated translated questionnaire whenever available for the standard scales mentioned above. Then, in case of absence, translations were completed by an expert translator and validated through panel and back-translation processes \cite{cha2007translation} by independent translators.
%
Adaptations were made to tailor specific questions and answer options to each site. This was necessary for questions such as field of study, which varies depending on the educational programs offered by each institution, and accommodation, which differs based on personal choices, available facilities, and cultural norms. For example, at \NUM, there is substantial migration to the capital for work and study, intertwining with students' educational journeys and influencing their choices. At \JLU, all students live on the university campus, contrasting with other institutions that offer various accommodation options, including university-provided, public, or private housing. Further adaptations addressed unique local needs. At \JLU, response options related to nationality were modified to exclude countries whose sovereignty is not recognized by the local government. Additionally, \JLU adapted the list of social media and applications list to include those popular in both Western and Asian markets.

Certain items related to sexuality (“To have sexual relationships; to obtain sexual pleasure”) and religiosity (“To believe in God as the savior of humanity; to fulfill the will of God”) from the Basic Values Survey~\cite{gouveia2014functional} were also modified. The content was deemed inappropriate in some locations, such as \AMRITA and \UC, and alternative options like “love” were suggested. For the religiosity item, an option with other religious and spiritual figures, such as Buddha, was explicitly provided for \AMRITA and \JLU. Finally, aware of the cross-cultural variation of personality traits in the case of \JLU, the locally developed personality trait scale ~\cite{zhang2019development} was adopted in addition to the Western version, thus facilitating a comparison.

\noindent
After translation and adaptation, the questionnaires were tested locally. They were distributed to approximately 30 participants, including project partners and university students. This allowed to validate the translations and assess completion times.



\subsection{Intensive Longitudinal Surveys and Passive Sensor Data Collection} \label{subsec:ils}

Temporal aspects are essential for understanding how everyday life behavior evolves over time. Intensive longitudinal surveys are valuable for observing human behavior as it changes, especially when paired with detailed sensor data from everyday devices. Thus, we adopted the \texttt{iLog}~\cite{2014-PERCOM}, an App that effectively collects this information through user interactions and data from smartphone sensors\footnote{The use of iLog and collected data in various experiments has been described in \cite{2020-zeni1,2017-SOCINFO,KD-2017-PERCOM,2017-ICSC,2018-PERCOM2}. At the time of this study, only Android devices were supported. Currently, the \change{new app version} runs on both Android and iOS devices.}. 
\change{\cref{fig:ilogapp} shows the application screenshots.} The app’s ability to gather user self-reports and passive sensor data makes it unique (see \cite{runyan2013smartphone,wang2014studentlife,kreuter2020collecting} for other similar tools). This dual capability significantly enhances traditional time diary methods \cite{2018-PERCOM1,sorokin1939time}, mainly when structured \cite{hellgren2014extracting} to enable real-time user responses. In addition, \change{the application enters power-saving mode when the smartphone reaches around 8\% of the phone battery charge}, thus limiting cases of non-response due to the smartphone being turned off.

\begin{figure}[t]
    \includegraphics[width=\textwidth]{figures/iLog.png}
    \Description[iLog app screenshoots]{Sequence of screenshots of iLog application showing the notification when the application is running in the background, two types of questions and settings.}
    \caption{The iLog app adopted for intensive longitudinal survey and sensor data collection.}
    \label{fig:ilogapp}
\end{figure}


\subsubsection{Diary Methods and Daily Routines}\label{sec:td}
Daily diaries, as a form of self-reports, are foundational for studying daily human behavior. Following standards set by HETUS and ATUS, three types of daily diaries were implemented\footnote{Answer options were supplemented with explanatory notes, not included here for readability. Complete documentation is available in the catalog (see \cref{sec:availability}).}:

\begin{itemize}
%
\item \textbf{Morning and Evening Diaries}: The first diary collects information at the start and end of each day. At 08:00 AM, participants received two qualitative questions about sleep quality and daily expectations, listed in \cref{tab:morning_questions}. At 10:00 PM, participants responded to questions in \cref{tab:evening_questions} regarding \textit{(a)} a rating of their day, \textit{(b)} any challenges encountered, \textit{(c)} how they addressed these challenges, and \textit{(d)} reflections on the COVID-19 pandemic.
%
\item \textbf{Time Diaries}: The second time diary follows the HETUS standard (\cref{tab:td_main}), covering main activities and moods\footnote{The questions design aligns with the definition proposed in \cite{li2022representing, giunchiglia2022context} concerning a model of the situational context of a person which can be applied in organizing and reasoning about massive streams of sensor data and annotations. This alignment expands the dataset’s potential, enabling its adoption in fields such as knowledge graph modeling and engineering.}. Throughout the study, participants were asked to answer four brief questions every thirty minutes during the first two weeks and every hour during the last two weeks:
%
\begin{itemize}
     \item \textit{What are you doing?}: Participants select from 34 activities, such as sleeping, eating, and working.
     \item \textit{Where are you?}: The current location is identified from 26 categories, including home, workplace, university, and restaurant.
     \item \textit{Who is with you?}: Participants indicate their company, with options like ``alone,'' ``with my partner,'' and ``with friends.''
     \item \textit{What is your mood?}: Participants rate their mood on a five-point scale ranging from happy to sad.
\end{itemize}
If a participant reports engaging in ``eating,'' ``traveling,'' or ``doing sport,'' they are prompted to answer additional questions (\cref{tab:td_sub}):

\begin{itemize}
    \item For eating, participants select foods and drinks from 20 categories, such as rice, potatoes, meat, and beer (adapted from~\cite{gatica19}).
    \item For sports, participants specify the type of sport from nine options, such as jogging, running, or water sports.
    \item For traveling, participants state \textit{(a)} the purpose, selecting from seven categories (e.g., study, social life), and \textit{(b)} the mode of transport from 16 options (e.g., car, bus).
\end{itemize}
%
\item \textbf{Snacks Diary}: As listed in \cref{tab:tb_snack}, these questions document food and drink consumption outside of main meal times. These prompts appear every two hours, allowing participants to select multiple items.
\end{itemize}
\noindent
To reduce the participant burden \cite{ROLSTAD20111101, eisele2022effects}, which can lead to non-response missing data or wrong answers \cite{bison2024impacts}, notification breaks were incorporated (see Table \ref{tab:break}). These breaks allow participants to pre-fill answers or pause notifications temporarily. For example, participants can select the “Go to sleep” option in the evening to silence notifications for six hours or pause notifications for two hours during classes or sports.
Moreover, each notification expired only after 12 hours, allowing participants to choose not to respond immediately. They could accumulate a maximum of 24 unanswered notifications. After this limit, the application deleted the oldest notifications and marked them as ``Expired'' in the dataset.

\begin{table}[]
    \footnotesize
    \centering
    \caption{Time diaries collecting contextual information about the participant every half hour.}
    \label{tab:td_main}
    \begin{tabularx}{0.97\textwidth}{p{8cm}X}
    \toprule
    \textbf{A3. What are you doing?}&
    \textbf{A4. Where are you?}\\
    \midrule
    \begin{enumerate}[leftmargin=*]
        \item Sleeping
        \item Personal care
        \item Eating \goto{(go to A3c,  \cref{tab:td_sub})}
        \item Cooking, Food preparation \& management
        \item Study/work group
        \item Lecture, seminar, conference, university meeting
        \item Did not do anything special
        \item Rest/nap
        \item Break
        \item Walking
        \item Travelling \goto{(go to A3a1, A3a2, \cref{tab:td_sub})}
        \item Social life
        \item Happy Hour, Drinking, Party
        \item Phone/Video calling
        \item In chat on Internet or reading, sending e-mail
        \item Surfed or seeking, reading information via Internet
        \item Social media (Facebook, Instagram, etc.)
        \item Watching TV, video, YouTube, etc.
        \item Listening to music
        \item Reading a book, periodicals, news, etc.
        \item Movie Theatre Concert\dots
        \item Entertainment Exhibit, and Culture
        \item Others Entertainment and Culture
        \item Arts
        \item Hobbies
        \item Games
        \item Freetime study
        \item Sport \goto{(go to A3b, \cref{tab:td_sub})}
        \item Voluntary work and participatory activities
        \item Household and family care
        \item Grocery Shopping
        \item Other Shopping
        \item Work
        \item Other
    \end{enumerate}&
    %
    %
    \begin{enumerate}[leftmargin=*]
        \item Home apartment, room
        \item Home garden, patio, courtyard
        \item Relatives Home
        \item House (friends others)
        \item Classroom, Laboratory
        \item Classroom, Study hall
        \item University Library
        \item Other university places
        \item Canteen
        \item Other Library
        \item Gym, swimming pool, Sports centre\ldots
        \item Grocery Shop
        \item Supermarket\ldots
        \item Street markets
        \item Shops, shopping centers, indoor markets, other shops
        \item Café, pub, bar
        \item Restaurant, pizzeria, Street food vendor
        \item Movie Theatre Museum\ldots
        \item In the street
        \item Public Park/Garden
        \item Countryside, mountain, hill, beach
        \item Workplace, office
        \item Weekend home or holiday apartment
        \item Hotel, guesthouse, camping site
        \item Another indoor place
        \item Another outdoor place
    \end{enumerate} \\
    %
    %
    \toprule
    \textbf{A5. With whom are you?}&\textbf{A6a. What is your mood?}\\
    \midrule
    \begin{enumerate}[leftmargin=*]
        \item Alone
        \item Friend(s)
        \item Relative(s)
        \item Classmate(s)
        \item Roommate(s)
        \item Colleague(s)
        \item Partner
        \item Other
    \end{enumerate}&
    \begin{enumerate}[leftmargin=*]
    \item \vcenteredinclude{figures/emoji/emoji_1.png}
    \item \vcenteredinclude{figures/emoji/emoji_2.png}
    \item \vcenteredinclude{figures/emoji/emoji_3.png}
    \item \vcenteredinclude{figures/emoji/emoji_4.png}
    \item \vcenteredinclude{figures/emoji/emoji_5.png}
    \end{enumerate}
    \\
    \bottomrule
\end{tabularx}
\end{table}

\subsubsection{Sensor Data} 
During the study, iLog collected data from all available smartphone sensors, spanning continuous and interaction sensing modalities (see \cref{sec:relatedworks}), at \change{a very high level of granularity} to capture the breadth of daily behaviors, which can be studied through various analytical techniques. Appendix~\ref{app2:sensors} \change{details the sensors and their collection frequency or whether they are event-based}. The collected sensors support research in the following areas:
%
\begin{itemize}
\item \textbf{Connectivity}: includes Bluetooth Low-Energy, Bluetooth Normal, cellular network, connected WiFi network, and discovered WiFi Networks, measuring connections to nearby devices.
\item \textbf{Environment}: includes ambient temperature, light, pressure, and relative humidity sensors, which are hardware-based and available only on compatible devices.
\item \textbf{Motion}: captures participants' movements in different contexts and includes accelerometer, activities, gravity, gyroscope, step counter, and step detector.
\item \textbf{Position}: includes location, magnetic field, proximity sensors, tracking physical positioning and interaction with the device.
\item \textbf{App Usage}: tracks social media and app interactions, including headset plug, music playback, notifications, and running applications.
\item \textbf{Device Usage}: tracks device interaction patterns, i.e., airplane mode, battery charge, battery level, doze, ring mode, screen status, touches, and user presence sensors.
\end{itemize}

\subsubsection{Intensive Longitudinal Survey Translation, Testing, and Adaptation}
Like the previous questionnaires conducted with LimeSurvey, the intensive longitudinal self-report questions and responses were translated into local languages. In addition to the local coordinators, approximately 15 participants downloaded and used the iLog app for two weeks to assess user experience and technical stability in a separate test. We performed adaptations to accommodate specific local interests and needs. Some partners were interested in investigating eating behaviors and eating disorders. Thus, additional questions were added to cover food consumption and dietary routines, balanced with standard time diary items, to minimize invasiveness. Regional variations in response options were also included to reflect local food and beverage consumption. Additionally, \IPICYT opted to collect the notification at a half-hour frequency for the entire month.

Further technical adaptations were made to the iLog app code provided by \cite{2020-zeni1}. For instance, iLog typically relies on Google services to receive notifications from the server. However, these services were unavailable in certain regions, such as at \JLU, or in areas with limited cellular network coverage. An ``offline'' version of iLog was developed to overcome this issue. Here, the notification schedules are downloaded at the beginning using WiFi connections and displayed on participants’ smartphones at planned times, removing the dependency on Google services or data connections. Participants can then synchronize their collected data via WiFi as needed. Initially designed for \JLU, the offline version was also implemented at \AMRITA and \IPICYT for their data collections.

\subsection{Target Population and Sampling Strategy} \label{subsec:sample}

We focused on a specific target population to ensure effective comparisons across study sites: \textit{college students}. Globally, students tend to have similar routines, such as attending classes, studying, working part-time, and participating in extracurricular activities. This shared lifestyle makes it easier to compare social practices, revealing commonalities and differences across diverse contexts—i.e., the ``unity in diversity" of their behaviors.

The sampling strategy was designed with specific criteria to represent the student population. Except for \JLU and \IPICYT, all registered students at each university were invited to participate in the study and complete the first questionnaire. From the participant pool, individuals with incompatible smartphones (e.g., Android OS below 5.0) or low survey attendance were excluded. The sample was designed to be balanced by gender, age, and academic department to minimize routine variation due to scheduling, community characteristics, and demographics. Thus, the dataset is comparable to existing studies, such as \cite{wang2014studentlife}, and aligns with similar studies that follow this methodology and collect comparable variables \cite{li2022representing, 2017-SOCINFO}.

\subsubsection{Recruitment Strategy}
Students at each pilot site received an invitation to their institutional email accounts, followed by four monthly reminders. Recruitment strategies differed at \JLU and \IPICYT, employing snowball sampling via WeChat and WhatsApp. After this initial phase, 350 students were selected according to predefined sampling criteria to ensure a minimum of 250 active participants, accounting for typical longitudinal survey dropout rates of 30\% to 70\% \cite{gustavson2012attrition}. In the pilot sites where achieving this threshold proved challenging, all students who completed the questionnaire and owned compatible smartphones were invited to participate.

The selected students received a second email with attachments, including instructions for downloading the iLog app, survey information, and a privacy statement detailing data processing methods. This email also included the second questionnaire. Participants were given approximately five days to download the app and complete the registration, receiving additional official communications halfway through the survey (along with the third questionnaire) and at the study’s conclusion. The helpdesk provided daily personalized support. \cref{tab:participants} summarizes the sampling and recruitment strategy outcomes.

\subsubsection{Recruitment Adaptations}

Specific adaptations were made to accommodate local needs. During the invitation phase, some modifications were implemented at \JLU and \IPICYT. At \JLU, the survey invitation was posted on social channels and in various WeChat groups, reaching about 5,000 students. Reminders were shared on these channels rather than being sent directly to participants. At \IPICYT, recruitment was conducted through direct participant engagement, with reminders delivered verbally and via messages in a WhatsApp group created specifically for the study. Additionally, at \IPICYT, the selection phase occurred before the first questionnaire was distributed.


\subsection{Incentive Strategy}\label{sec:incentives}

A group of experts designed the incentive strategy. It was informed by sociological literature, particularly the work of \citet{singer2013use}, who extensively reviewed types of incentives (monetary and non-monetary) and their effectiveness in reducing respondent bias and dropout rates. The review suggests that monetary incentives are generally more effective than gifts, and prepaid incentives outperform random prizes. Based on these findings, an incentive strategy was developed, focusing primarily on fixed payments.
%
\cref{tab:incentives} reports the incentive strategy adopted in the different pilots. While no incentives were adopted for the invitation questionnaire, in the first and second phases of iLog data collection, selected students received monetary rewards for completing at least 85\% of notifications. Additionally, daily prizes and three final prizes were awarded in both phases. 

\begin{table}[t]
    \centering
    \caption{\label{tab:incentives} Incentives were paid upon completing at least 85\% of the notifications after the first and second two-week periods. Additionally, daily and final prizes were awarded through random selection. Each pilot site adapted the incentive protocol based on local practices from similar studies they had conducted previously.}
    \begin{tabular}{lccccc}
    \toprule
     & \multicolumn{2}{c}{\textbf{Payments}} & \textbf{Daily} & \multicolumn{2}{c}{\textbf{Final Prizes}}\\
     \cmidrule(rl){2-3}\cmidrule(rl){5-6}
        & \textbf{1st Phase} & \textbf{2nd Phase} & \textbf{Prizes} & \textbf{1st Phase} & \textbf{2nd Phase} \\
    \midrule
    \JLU    & 100 rmb    & 100 rmb   & 1 of 20 rmb   & 3 of 88 rmb   & 3 of 88 rmb \\
    \AAU    & 150 kr     & 150 kr    & 5 of 40 kr    & 3 of 800 kr   & 3 of 1200 kr \\
    \AMRITA & - & - & - & - & - \\
    \UNITN  & 20 €       & 20 €      & 5 of 5 €      & 3 of 100 €    & 3 of 150 € \\
    \IPICYT & - & - & - & - & - \\
    \NUM    & 10k MNT    & 10k MNT   & 5k MNT        & 100k MNT      & 150k MNT \\
    \UC     & 25k GS     & 25k GS    & 10 vouchers   & 1 restaurant voucher & 1 restaurant voucher \\
    \LSE    & -          & -         & -             & £150 (1/50)   & £150 (1/50) \\
    \bottomrule
\end{tabular}
\end{table}

\subsubsection{Incentives Adaptation}
The incentives were adjusted according to the cost of goods in each country. In addition, they were tailored to local cultures and circumstances. Thus, in the case of the \LSE, it was decided to offer a prize of £150 drawn randomly for every 50 participants, aligning with the incentive methods proposed at the same institute. Similarly, at \AMRITA, no monetary incentives were implemented; instead, participation certificates were provided, reflecting the local culture of engagement in science. This strategy was particularly successful for questionnaires but did not yield the same results for the intensive longitudinal survey, as discussed in \cref{sec:limits}. This lack of success may also be attributed to the novelty of the approach. Lastly, at \IPICYT, no monetary incentives were used either. Instead, there was a focus on social engagement by creating a WhatsApp group, which facilitated multiple meetings and exchanges of interests and opinions between participants and researchers. 

\subsection{Ethics and Privacy}\label{subsec:privacy}

Addressing privacy in cross-country research requires adapting to diverse regulations, social norms, and individual attitudes toward privacy \cite{acquisti2015privacy, capurro2005privacy, ess2005lost}. Privacy management requires careful consideration of ethical and legal aspects tailored to each context. Consequently, each university's principal investigator or legal deputy was designated as data controller, thus covering a legal and ethical role of responsibility. Specific data processing responsibilities were also delegated to \UNITN, which managed data collection and management, as outlined in \cref{subsec:management}.

From an ethical perspective, the data controller in each country appointed a local ethics committee to support and approve project activities. If universities or research centers lacked an ethics committee, the local institution designated a suitable one. To facilitate this process, \UNITN, in collaboration with its ethics committee and privacy office, developed templates and documentation that were then adapted to meet local requirements.

Given that the iLog app operates within Europe, the methodology adopted a Eurocentric approach, benefiting from the comprehensive framework provided by the GDPR, which governs personal data processing. \UNITN supplied all necessary documentation—such as privacy statements, information protocols, and additional guidelines—beginning with GDPR standards and extending them with clauses as required by local regulations. In essence, regulations in each country that do not conflict with the GDPR were considered complementary.

This approach ensured that each country's implementation adhered to procedures and documentation based on European legislation while accommodating local laws. For example, in the case of \IPICYT, any participant request for data deletion must be fulfilled within three days—a timeframe stricter than GDPR requirements and incorporated into the local documentation and protocols.

\subsection{Data Collection}\label{subsec:protocol}
The academic calendar and various aspects of university life greatly influence students' routines, often more so than other factors. To ensure the ecological validity of the study, data collection at all sites was carried out during the academic semester. This approach guarantees that participants likely adhered to similar academic routines as their peers in other countries.
%
The questionnaires, longitudinal time diaries, and passive sensor data were collected consistently across the eight pilot sites. The data collection process spanned approximately ten weeks, divided into the following phases (see \cref{fig:ExpPhasesDetailed}):

\begin{itemize}    
    \item \textbf{Invitation}: The first phase involved sending an email with a survey description, an invitation to complete the first questionnaire, and information on the second part of the survey. This invitation was followed by three weekly reminders for students who had not yet completed the survey. 
    \item \textbf{Selection}: At this stage, a subset of eligible participants was selected for the second part of the survey. Selection criteria included consent to personal data processing and possession of a compatible Android smartphone.
    \item \textbf{First Phase}: This phase began with distributing the second questionnaire to the selected participants, followed by a reminder after one week. Along with the second questionnaire, participants received an email with instructions on downloading the iLog app, along with a brief user manual.
    \item \textbf{Second Phase}: During this final data collection phase, the third and final questionnaire was sent, followed by a reminder after one week. In this phase, the frequency of iLog time diary prompts was reduced.
    \item \textbf{Closing Procedures}: At the end of the survey, participants received a final email with instructions on app uninstallation and a last reminder, where necessary, to complete the second and third questionnaires.
\end{itemize}
\noindent
Daily reports were generated throughout the data collection to monitor iLog survey responses and quickly identify potential issues. Based on these reports, local field supervisors contacted inactive participants every three days to provide support as needed. Participants were also contacted with the results of the daily prize draws (see \cref{sec:incentives}). Moreover, due to timing issues, primarily related to technical challenges at \JLU and delays in ethics committee approvals at \AMRITA and \IPICYT, data collection in these countries began several months later than initially planned. 

\subsection{Data Management} \label{subsec:management}

As discussed in \cref{subsec:privacy}, ethical and legal considerations led to establishing protocols to ensure data ownership and privacy compliance, including copyright management, data cleaning, and anonymization, described below.

\subsubsection{Data Ownership and Copyright}

At the end of each data collection, \UNITN, acting as the data processor, securely transferred the entire raw dataset—including personal data—to the designated local data controllers. \UNITN maintained a copy of the non-anonymous data for further preparation, thus sharing a processed copy of the cleaned dataset with all data controllers and researchers involved in refining and analyzing the data. Towards the end of the project, each data controller licensed their cleaned and anonymized dataset to \UNITN, establishing shared management and distribution policies, partly described in \cref{sec:availability}. This procedure ensured local data controllers retained ownership while facilitating legally compliant data sharing for research purposes.

\subsubsection{Data Cleaning and Anonymization}

To ensure GDPR compliance and achieve high data quality, the dataset underwent \textit{Data Cleaning} and \textit{Anonymization}. Data cleaning included standardizing the dataset and addressing non-random missing data due to technical errors. Variable and value labels were adapted for clarity, making the dataset understandable to third-party researchers. On the other hand, Anonymization aimed to prevent individual identification through (i) Personal Data Anonymization, (ii) Network Anonymization, and (iii) GPS Anonymization. 

Regarding personal data anonymization, all personal information, such as email addresses, home addresses, names, and surnames, was removed from the three dataset types (online questionnaire, time diaries, and sensor data). Unique identifiers were assigned to ensure consistency across these datasets. Considering Network Anonymization, potential re-identification sources, such as the WiFi network connected to the smartphone, cellular networks, and available WiFi networks, were anonymized. Relevant sens r file columns were anonymized using irreversible hash functions applied to WiFi network names. Regarding GPS Anonymization, high-precision GPS data were intentionally truncated to maintain anonymity while preserving scientific usability. Anonymization included identifying points of interest (POIs) where participants spent significant time and providing generalized location data. Two versions of this dataset, named \textit{RoundDown} and \textit{POI}, were created. To further protect privacy, a single research institution can access only one of these datasets to minimize re-identification risks associated with combining the two versions.

It is important to note that, despite these measures, the dataset could still allow re-identification if cross-referenced with other datasets. For this reason, the dataset is not publicly available online and can only be accessed under specific conditions (see \cref{sec:availability}).