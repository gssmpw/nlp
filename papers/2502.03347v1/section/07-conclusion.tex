Human behavior varies not only by time of day but also across individuals, countries, and cultures, introducing layers of complexity that most existing smartphone sensing and self-report datasets fail to capture. Current datasets are often limited to specific countries in the Global North, small participant samples, or a narrow set of sensors, which restricts their ability to reflect cross-country behavioral variations and nuanced sensor data patterns. To address this gap, we introduce \dataset, a dataset designed to combine passive smartphone sensor data with self-reported labels at multiple granularities. \dataset is built on a robust interdisciplinary methodology that incorporates standards from computer science, behavioral sciences, social practice theory, and intensive longitudinal surveys via time diaries. This approach supports machine learning models in inferring various everyday aspects, including mood, activities, social context, and eating habits. Data were collected across eight countries from \nilogusers college students over four weeks, encompassing \nsensors smartphone sensors and more than 350,000 self-reports, positioning \dataset as one of the largest and most diverse datasets of its kind, that is publicly available. Additionally, \dataset provides raw sensor data, offering researchers flexibility in processing and usage for diverse analytical purposes. As a significant contribution of this paper, \dataset is publicly released with a comprehensive description of the study design and data collection process, highlighting key insights, lessons learned, and future research directions toward diversity- and privacy-aware studies. We hope this dataset will serve as a valuable resource for the community, enabling studies across multiple disciplines, including ubiquitous computing, human-computer interaction, and machine learning. 



%Extensive demographic and psychosocial survey data add further depth, creating a rich multidimensional perspective on participants. \dataset is uniquely suited for exploring country-level behavioral differences between the Global North and South, as well as cultural and contextual influences on smartphone data. Additionally, it opens up new avenues for machine learning research, particularly in domain adaptation and cross-country generalization, which remain underexplored in ubiquitous computing. Validated by award-winning publications that have already leveraged it, DiversityOne offers substantial opportunities for advancing applications in behavior modeling and beyond.