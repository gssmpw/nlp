\section{Related Work}
To the best of our knowledge, there is no prior work on synergistic traffic assignment, i.e., traffic assignment with decreasing edge cost functions.
    Thus, here we consider research on avoidant traffic assignment (ATA) that may also be relevant for synergistic traffic assignment.
    ATA is a well-studied problem with decades of research in the operations and traffic planning communities.
    Additionally, ATA is often considered in game theory, as a prototypical example of a non-cooperative congestion game in pure strategies~\cite{Owen2013,rosenthal1973class}.

    \parheader{User Equilibrium}
    \citet{wardrop1952some} states that travelers in a road network naturally act selfishly, changing their path if a different path with a better travel time for themselves exists, even to the detriment of others.
    An assignment in which no traveler can change their path to their benefit is called a \emph{user equilibrium (UE)}.
    \citet{beckmann1956studies} find that a UE always exists for ATA, which matches a result from game theory about the existence of pure strategy Nash equilibria in congestion games~\cite{rosenthal1973class}.
    As traffic without central control naturally tends to a UE, it is an important issue of traffic analysis to compute UEs for a given network and traveler demand.

    \parheader{Simple Approaches to Finding a UE}
    A simple approach to finding a UE is a process called iterated \emph{sequential best response}, in which one traveler at a time may change their path while other travelers' paths are kept fixed.
    Edge costs are updated according to the change of one traveler before continuing with the next traveler.
    Sequential best response always converges to a UE~\cite{rosenthal1973class,Monderer1996}.
    However, the number of iterations until convergence can be exponential in the number of travelers~\cite{Durand2016, Fabrikant2004}.

    In iterated \emph{simultaneous best response}, each traveler chooses their next path simultaneously, while edge costs are kept fixed.
    Then, all paths are changed at once, before updating edge costs and repeating the process.
    While the simultaneous method needs fewer edge cost updates than sequential best response, it does not always converge.
    Instead, it may encounter best-response cycles where travelers revert to a previous state in a cyclical manner~\cite{sheffi1985urban}.
    Thus, simultaneous best response is also not suited to find a UE for ATA.

    \parheader{Finding a UE in Practice}
    In practice, more sophisticated methods find a UE by applying convex optimization to \emph{Beckmann's transformation}~\cite{beckmann1956studies} of ATA into a convex program whose minimum is a UE.
    A usual categorization~\cite{florian1995chapter,zhou2010computational,perederieieva2015framework} divides these algorithms into \emph{link-based}~\cite{frank1956algorithm,sheffi1985urban}, \emph{path-based}~\cite{dafermos1968traffic,jayakrishnan1994faster,florian2011new,kumar2011improved}, and \emph{bush-based}~\cite{bargera2002origin,dial2006path,nie2010class} approaches.
    These algorithms perform simultaneous path updates but they utilize the convexity of the problem to ensure progress towards the UE and avoid best-response cycles. 
    Note two limitations common to these approaches:
    First, travelers are treated as non-atomic, i.e., the load of a single O-D pair can be split among multiple paths.
    Second, the algorithms iteratively approach the UE in increasingly smaller steps but do not actually reach it.
    
    Each algorithm mentioned here requires an edge cost update and a re-computation of shortest paths in every iteration.
    The overhead for these updates can be expected to dominate the total running time. 
    \citet{buchhold2019real} show that a recent shortest-path speedup techniques called \emph{customizable contraction hierarchies (CCH)} can adequately reduce running times for cost updates and shortest-path queries to tenths of seconds per iteration for millions of O-D pairs.