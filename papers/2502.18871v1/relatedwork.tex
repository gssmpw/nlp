\section{Related Work}
Research on the automated detection of insulation in architectural and industrial settings remains relatively underexplored. However, related areas such as \textit{blueprint analysis, façade inspection, and thermal defect identification} have seen increasing attention, largely driven by advances in \textit{deep learning} and \textit{computer vision}. Many of these investigations rely on \textit{convolutional neural networks (CNNs)} and \textit{object detection models}, including \textit{YOLO variants}, to identify \textit{structural elements, construction symbols, or drafting errors} within CAD diagrams. While these studies contribute valuable insights into automated detection in construction and industrial applications, they fall short of addressing the specific challenges associated with \textit{auxiliary insulation detection} from \textit{high-resolution blueprint imagery}.  

Several prior works have demonstrated the effectiveness of \textit{object detection models} in analyzing architectural blueprints. For instance, \textit{Kim et al.}~\cite{kim2020automated} explored the use of \textit{deep learning-based object detection} techniques for \textit{automated identification of structural components} such as beams and ducts within construction blueprints. Their study showcased the potential of CNN-driven models in \textit{reducing manual effort} and improving accuracy in \textit{blueprint interpretation}. Similarly, \textit{Ahmed et al.}~\cite{ahmed2020automatic} leveraged CNN-based methods to \textit{detect and localize standard construction symbols} in CAD drawings. Their findings highlighted the \textit{importance of high-resolution input data and rigorous annotation strategies}, reinforcing the need for robust dataset curation when training detection models for architectural applications.  

In a more specialized approach, \textit{Juniat et al.}~\cite{juniat2022automated} proposed an \textit{automated drafting error detection system} based on CNNs. Their work focused on identifying inconsistencies in \textit{architectural CAD symbols}, demonstrating how \textit{symbol recognition models} can be adapted to detect \textit{design inconsistencies and drafting errors}. While their model successfully addressed common blueprint irregularities, it did not extend to identifying \textit{material defects} or evaluating \textit{insulation-related components}, which are the focus of our work.  

In parallel to blueprint-based studies, several researchers have explored \textit{thermal imaging-based insulation defect detection}. \textit{Avdelidis et al.}~\cite{avdelidis2019infrared} and \textit{Pintus et al.}~\cite{pintus2021deep} introduced \textit{deep learning-based two-phase pipelines} that analyze \textit{infrared thermography data} to detect \textit{insulation defects and thermal bridges} in building facades. These methods first \textit{isolate regions exhibiting abnormal heat signatures}, followed by \textit{classification of the defect severity}. Despite their effectiveness in thermal anomaly detection, these approaches are \textit{not applicable to blueprint-based insulation verification}, where defects must be inferred from schematic representations rather than real-world thermal emissions. Nonetheless, these studies underscore the \textit{advantage of combining detection and classification phases}, a strategy we also adopt in our work.  

Unlike prior methods that focus on detecting \textit{generic construction elements}, \textit{drafting errors}, or \textit{thermal anomalies}, our approach introduces a \textit{dedicated solution for auxiliary insulation detection from CAD-derived blueprints}. We propose a \textit{two-phase deep-learning pipeline}, leveraging \textit{YOLOv8x for detection} and \textit{YOLOv8x-CLS for classification}, specifically designed to \textit{localize and evaluate insulation regions} in construction diagrams. Our method surpasses existing approaches in the following key aspects:  

\begin{itemize}
    \item \textbf{Blueprint-Specific Insulation Detection:} Previous studies on blueprint analysis have focused on detecting \textit{structural components or symbols}, whereas our method explicitly \textit{identifies and evaluates insulation placement} within CAD drawings.  
    \item \textbf{Integration of Detection and Classification:} Unlike \textit{symbol detection} or \textit{drafting error analysis}, our pipeline follows a \textit{two-phase workflow} that first \textit{detects insulation regions} and then \textit{classifies them as present or missing}, similar to thermal defect analysis but adapted for blueprint interpretation.  
    \item \textbf{High-Resolution Blueprint Processing:} Our model is designed to process \textit{large-scale construction blueprints} at high resolutions (up to \textit{4800 pixels}), ensuring \textit{detailed pattern recognition} without sacrificing precision.  
    \item \textbf{Dataset Scale and Expert Validation:} A critical limitation in prior research is the \textit{lack of large, verified datasets}. In contrast, our dataset underwent extensive augmentation, expanding from \textit{900 images to over 4,200 detection samples and 23,000 classification patches}. Moreover, \textit{over 10 industry experts} were involved in \textit{revising, reviewing, and double-checking} the labeled dataset, ensuring \textit{annotation accuracy and professional validation}, a step often overlooked in previous studies.  
\end{itemize}  

By combining \textit{state-of-the-art object detection models}, \textit{robust dataset augmentation}, and \textit{expert-driven validation}, our methodology presents a \textit{scalable and industrially viable solution for insulation verification}. In contrast to existing approaches that either \textit{focus on generic structural elements} or \textit{thermal defect detection}, our work directly addresses the challenge of \textit{auxiliary insulation detection in blueprint data}, making it a unique and significant contribution to \textit{computer vision applications in construction and industrial engineering}.