\section{When is ``myopic-greedy'' optimal?}

We have seen that the myopic-greedy policy is not always optimal. Then, the next question will be as follows: \textbf{\textit{When does the myopic-greedy policy succeed?}} This section answers the question with a game-theoretic analysis in the case that $f$ and $\bar{\lambda}$ are linear functions.


Our main finding is that a myopic-greedy policy is nearly optimal when the provider population effects $f$ are homogeneous across provider groups. To formalize this, we define the family of $\epsilon$-greedy policies as follows:
\begin{align*}
    \pi_{k,l}^{(\epsilon)} = (1 - \epsilon) \, \mathbb{I} \{ l = {\arg\max}_{l' \in [L]} b_{k,l'} \} + 
    \epsilon / L,
\end{align*}
where $\mathbb{I}\{\cdot\}$ is the indicator function.
Notably, $\bpi^{(0)}$ corresponds to the myopic-greedy policy, 
while $\bpi^{(1)}$ is the uniform random.
The subsequent results establish that $\bpi^{(0)}$ 
optimal
in the homogeneous-linear setting.

\begin{theorem}[Optimality of the myopic-greedy]\label{thrm:optimal_greedy}
Let $\blambda_{\infty}$ be the population at the NE under policy $\bpi$. For any base utility $B$ and linear increasing and homogeneous functions $\bar{\lambda}$ and $f$, the social welfare $R(\bpi^{(\epsilon)}; \blambda_{\infty}^{(\epsilon)})$ 
under the $\epsilon$-greedy policy 
is decreasing in $\epsilon\in[0, 1]$. In particular, we have
\begin{enumerate}
    \item When $K=1$, $R(\bpi^{(\epsilon)}; \blambda_{\infty}^{(\epsilon)})$ is strictly decreasing in $\epsilon$.
    \item When $K>1$, we can identify functions $g,h$ such that 
    \begin{equation}\label{eq:bound_myopic}
        g(\epsilon) \leq R(\bpi^{(\epsilon)}; \blambda^{(\epsilon)}_{\infty})\leq g(\epsilon)h(\epsilon),
    \end{equation}
    and both $g,h$ are decreasing in $\epsilon$. In addition, when $(\nabla_{\lambda_l} f_{k,l})(\nabla_{e_l} \bar{\lambda}_l) (\nabla_{s_k} \bar{\lambda}_k)$ is sufficiently small, the function $h(\epsilon)\rightarrow 1$ and Eq. \eqref{eq:bound_myopic} is tight.
\end{enumerate}
\end{theorem}
 Theorem~\ref{thrm:optimal_greedy} suggests that when the population effect $f$ is linear and homogeneous across different provider groups, the myopic-greedy policy will be always optimal. This also holds in the case when there is no population effect, i.e., $f_{k,l}(\lambda_l)$ always equals to a constant, $\forall \lambda_l \in \mathbb{R}, \forall (k,l) \in [K] \times [L]$. In such cases, the use of the myopic-greedy policy is recommended.

However, when the population effect becomes heterogeneous across different provider groups, the myopic policy ceases to be optimal, as illustrated by Proposition \ref{prop:heterogeneous_f}.

\begin{proposition}\label{prop:heterogeneous_f}
The myopic-greedy policy can be sub-optimal when $\{f_{k,l}\}$ are heterogeneous across provider groups, even when $\bar{\lambda}$ and $f$ remain linear.
\end{proposition}

We provide a detailed example in Appendix \ref{proof:heterogeneous_f} to support Proposition \ref{prop:heterogeneous_f}. Intuitively, the heterogeneity matters because it results in \textit{cross-over} behaviors (e.g., provider group A starts low utility but becomes high utility, while provider group B has medium utility regardless the population) matter in the policy design. Aside of the linear case, \textit{saturation} behaviors (e.g., having no population effect changes after the population becomes adequately large) also matter. When we encounter such heterogeneous or concave population effects among multiple provider groups, myopic-greedy may not be optimal, as they ignore the impact of policy to future population changes.
These results demonstrate that the myopic-greedy policy is optimal only under highly restrictive conditions, emphasizing the need for practical solutions considering the long-term effect. 

