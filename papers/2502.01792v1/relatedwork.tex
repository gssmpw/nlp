\section{Related work}
\label{app:related_work}

This section summarizes the important related work.

\textbf{Policy optimization under the population departure.} \quad The most relevant existing works to ours are \citet{mladenov2020optimizing} and \citet{huttenlocher2023matching}, which consider the population dynamics by modeling the departure of viewers and providers. Specifically, \citet{mladenov2020optimizing} assume that a provider will leave the platform if the provider cannot receive adequate exposure (i.e., exposure is below some given threshold). Then, \citet{mladenov2020optimizing} solves the constrained optimization problem as linear integer programming and demonstrates that provider fairness is crucial to maintain a high viewer welfare. To extend, \citet{huttenlocher2023matching} additionally consider the departure of viewers who receive less utility than given thresholds. \citet{huttenlocher2023matching} also formulate a matching problem to determine which viewers and providers to keep in the platform to achieve high long-term social welfare. However, both works ignore the possible growth of the platform, and how a policy design affects the ``growing-the-pie'' behavior has remained underexplored. Our work complements these existing work by finding that provider fairness is important to ensure high ``population effects'' in a generalized formulation.

\textbf{Policy optimization under strategic content providers.} \quad
Another related literature is the policy optimization under strategic content providers~\citep{hron2022modeling, jagadeesan2022supply,  yao2023rethinking}. These works often formulate content providers as ``selfish'' agents who maximize only their own utility defined by the amount of exposure minus the cost of content generation. As described in Section~\ref{sec:game_formulation}, our problem setting can also be seen as a variant of policy optimization under strategic viewers and content providers. However, our formulation is distinctive in modeling the increase and decrease of the total population, while existing works assume that the total number of viewers and providers are fixed. This difference results in novel findings: while \citet{hron2022modeling} find that more explorative (i.e., stochastic) policy can be against producing high-quality ``niche'' contents when the population is fixed, provider-fair policy (or stochastic allocation) can be beneficial when taking the population growth of multiple groups into account.

\textbf{Provider fairness, provider diversity, and viewer welfare.}
Fairness and diversity among providers have been considered as necessary metrics or constraints when optimizing policies in two-sided platforms~\citep{singh2018fairness, wang2021user, boutilier2023modeling}. While provider-fairness is initially considered important from provider-side perspectives~\citep{singh2018fairness}, recent works considers the impacts of provider fairness on viewer welfare. Specifically, provider fairness turned out important to maintain provider diversity~\citep{yao2023rethinking, hron2022modeling}, and provider diversity helps maintain viewer welfare in the long-run~\citep{su2023value, mladenov2020optimizing}. Our findings align with these works in pointing out that provider-fairness is important for long-term viewer satisfaction, but from a different viewpoint, suggesting that ``population effect'' should also be taken into account to invest future growth of populations.