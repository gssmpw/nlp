\section{Related Work}
The field of ransomware detection and classification has undergone significant advancements, with various methodologies proposed to address the evolving threat landscape. Given the adaptive capabilities of ransomware, traditional detection methods have struggled to maintain effectiveness, necessitating the development of more sophisticated classification approaches. This section reviews existing techniques, including signature-based detection, behavioral analysis, machine learning methodologies, and heuristic-based anomaly detection, while also highlighting their respective limitations.

\subsection{Signature-Based Detection Methods}

Signature-based detection methods have been widely employed in identifying ransomware through the recognition of unique code patterns. These approaches achieved rapid classification of known ransomware variants through the maintenance of extensive signature databases ____. By comparing executable files against stored patterns, such techniques provided efficient detection mechanisms for previously encountered ransomware families ____. The primary advantage of this approach lay in its ability to generate deterministic and highly precise classifications when identifying known ransomware instances ____. However, the emergence of polymorphic and metamorphic ransomware strains, which modified their code structures dynamically during execution, significantly reduced the effectiveness of signature-based methods ____. The static nature of these techniques rendered them ineffective against ransomware that employed encryption and packing mechanisms to obscure their payloads ____. Obfuscation techniques, including code injection and dynamic recompilation, enabled ransomware to bypass signature-based defenses through subtle modifications in their execution sequences ____. As a consequence, security solutions relying solely on signature-based methodologies required frequent updates, which often resulted in detection lag for newly emerging ransomware strains ____. Moreover, adversaries exploited automation techniques to generate large numbers of unique ransomware variants, effectively overwhelming signature-based classification systems through rapid iteration ____.

\subsection{Behavioral Analysis Techniques}

Behavioral analysis techniques focused on monitoring runtime execution patterns to detect ransomware based on anomalous system activity rather than static characteristics ____. By analyzing operations such as mass file encryption, unauthorized access attempts, and abnormal process executions, behavioral classification methods provided an alternative to signature-based detection ____. Unlike traditional approaches, which relied on predefined indicators of compromise, behavioral analysis aimed to recognize malicious actions based on deviations from standard system behavior ____. The dynamic nature of these techniques allowed for the identification of ransomware variants exhibiting new or previously unseen attack strategies ____. One of the most widely implemented behavioral detection strategies involved monitoring API call sequences, where deviations from expected patterns served as an indicator of ransomware activity ____. Additionally, file system monitoring techniques analyzed irregularities in read-write operations, which enabled the classification of ransomware based on encryption-related behaviors ____. Some approaches also incorporated real-time process memory inspection, detecting in-memory modifications that ransomware employed to evade disk-based detection mechanisms ____. Despite these advancements, behavioral analysis techniques encountered challenges in distinguishing ransomware actions from legitimate applications that engaged in bulk file modifications, leading to false positives ____. Furthermore, sophisticated ransomware families employed evasion techniques, such as delaying encryption routines, executing within virtualized environments to avoid detection, or simulating benign software interactions ____. These adaptive strategies significantly hindered the reliability of behavioral classification methodologies, necessitating complementary approaches to enhance accuracy ____.

\subsection{Machine Learning Approaches}

Machine learning approaches have been widely explored as a means of improving ransomware classification through automated feature extraction and predictive modeling ____. Various supervised learning algorithms, including decision trees, support vector machines, convolutional neural networks, and ensemble models, were trained on datasets containing ransomware and benign software samples to derive classification patterns ____. These models achieved robust detection accuracy through the analysis of complex statistical relationships between program features, including entropy-based indicators, opcode frequency distributions, and system-level behavior traces ____. Deep learning methodologies further expanded the capabilities of ransomware classification through the implementation of recurrent and transformer-based architectures capable of capturing temporal dependencies in ransomware execution patterns ____. The ability of neural networks to generalize across different ransomware families facilitated the classification of novel variants without explicit signature definitions ____. While machine learning-based approaches demonstrated significant improvements in ransomware detection, they were highly dependent on the quality of training data and suffered from potential overfitting to specific ransomware samples ____. Furthermore, adversaries actively devised adversarial machine learning techniques, manipulating feature representations to mislead classifiers and reduce detection effectiveness ____. Additionally, real-time deployment of machine learning-based ransomware classification required substantial computational resources, which introduced scalability constraints in resource-limited environments ____. The reliance on labeled datasets for supervised training posed another limitation, as the continuous evolution of ransomware necessitated frequent retraining to maintain classification accuracy ____.