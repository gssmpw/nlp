\section{Introduction}
\label{sec:intro}
The upcoming 6G networks are widely anticipated to achieve unprecedented levels of autonomous operation and self-X functionalities \cite{chaoub2023hybrid}. These expectations are primarily driven by the recent trend toward seamless integration of \gls{AI} capabilities in the telecommunication landscape. In the meantime, \gls{AI} development has witnessed a remarkable progress, over the last decade, largely due to the cutting-edge transformer schemes \cite{vaswani2017attention} and the explosion of \glspl{LLM}. Both technologies substantially revolutionized natural language understanding- and generation-related tasks, and demonstrated a great potential to drive innovation in various industries.% through injecting vast domain-specific datasets.

The alliance between 6G networks and \glspl{LLM} heavily relies on \gls{LLM} access to rich and specialized knowledge stemming from the accumulated experience of various telecom stakeholders, in particular, \glspl{MNO}. Nowadays, there is a growing interest in adapting various \gls{SOTA} \glspl{LLM} to the telecommunication ecosystem \cite{maatouk2024large}. Examples of \gls{LLM}-empowered \gls{MNO}'s scenarios include network analysis (e.g. using \gls{LLaMA} \cite{Kan2024mobile_llama}), network configuration (e.g. proof-of-concept utilizing \gls{GPT} 4 in \cite{Wang2024NetConfEval}), fault diagnosis (e.g. based on \gls{BERT} \cite{Chen2023Knowledge}), 
standards documentation understanding (e.g. comparative assessment among fine-tuned \gls{LLaMA}3, Solar and Mistral against \gls{GPT}-4 in \cite{Said2024instruct}), and \glspl{SON} \cite{Bariah2024Large}. A comprehensive survey is provided in \cite{Zhou2024survey} for broader details. Nevertheless, it can be clearly seen that the literature on this topic is predominantly focusing on the potential use cases and the resulting gains from integrating the \glspl{LLM} in the realm of 6G networks, leaving the details of how this integration can be implemented in real-world settings largely unexamined. To the best of our knowledge, this paper is the first attempt to provide an in-depth analysis of the architectural design possibilities for an integrated \gls{LLM} and 6G technology. We advocate for incorporating the \gls{LLM} as an intrinsic building block of the \gls{MNO} infrastructure synergistically operating with the remaining components, rather than operating solely as a supplementary add-on or on-demand service. This deep integration will particularly benefit from the \gls{SbD} approach adopted by 6G \cite{Khaloopour2024Resilience}, wherein security is embedded at every stage of the network’s design, deployment, and operation.

Based on the above introduction, the contributions of this article are two-fold:
\begin{itemize}
    \item We build upon the work in \cite{Kan2024mobile_llama}, wherein the focus is on augmenting the \gls{NWDAF} with \gls{LLM} capabilities. \gls{NWDAF} is a \gls{3GPP} network function that collects data from various Core \glspl{NF}, performs network statistical and predictive analytics to be shared with authorized data consumers. We extend the analysis by investigating alternative architectures for incorporating the \gls{LLM} component as a part of 6G \glspl{NF} while describing the merits and the shortcomings of each possible solution. This offers a comparative assessment that assists mobile carriers in understanding the trade-offs and applicability of different architectures. We additionally outline the architectural innovations required to implement these designs.
    \item We provide a concrete example of a complicated anomalous situation to illustrate the inter-exchanges that the \gls{LLM} triggers with the rest of the network to autonomously remedy this situation. 
    \end{itemize}

The remainder of this article is organized as follows. In the next section, we describe how the \gls{MNO} ecosystem can benefit from \glspl{LLM} through features like fine-tuning and prompt engineering. Afterwards, we introduce the possible architectures for building \gls{LLM}-enabled 6G networks, emphasizing the pros and cons of each option. Later, we detail different \gls{LLM} call flows allowing an autonomous resolution of a concrete faulty scenario in 6G networks. The associated challenges and directions of research are identified in the following section. Lastly, we conclude the article.