\section{Conclusions}
\label{sec:conclusion}
This paper investigates the architectural designs for making the \gls{LLM} a key functional element of 6G networks. The deployment form of the \gls{LLM} depends on the \gls{MNO}'s goals. The standalone design enjoys flexibility, sophisticated capabilities, and multi-system interoperability, whereas the fully integrated architecture promotes real-time performance, streamlined architecture, and coherence. The hybrid approach offers the best of both worlds, but remains technically complicated and costly to implement. An external \gls{LLM} poses serious security challenges, as opposed to the aforementioned architectures secured through \gls{SbD} principles. In their path towards integrating \glspl{LLM} in their ecosystem, \glspl{MNO} can begin with the standalone design and smoothly transition to the hybrid model over time through gradually upgrading an existing \gls{NF} such as \gls{NWDAF} to host the \gls{Near-RT} \gls{LLM}. This integration represents a transformative approach toward holistic intelligence and full automation promises of 6G.  A proof-of-concept based on the architectural framework proposed in this study is ongoing and will be a part of a follow-up work to this paper. %A proof of concept based on the general architectural framework proposed in this paper is left for our subsequent work.
%Note that, we leave the study involving URLLC services as part of a follow-up work to this paper.

%we have ongoing R&D work based on the general architectural framework proposed in this paper that we will publish subsequently with more use cases and discourses. %transition from fragmented to holistic intelligence in 6G

