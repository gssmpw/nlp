\section{A Practical Example Showcasing the Autonomous \gls{LLM} Operations in Future 6G Networks}
\label{sec:LLM_applications}
In this section, we delve into a scenario that involves a resource contention between three slices within a 6G network. A first slice, $I$, devoted to \glspl{IC} such as holographic live streaming. A second slice, $H$, assigned to \glspl{HRLLC} including industrial \gls{IoT} applications. A last slice, $M$, dedicated to \glspl{MC} such as \gls{IoT} sensors. The predefined bandwidth allocation among the three slices is 50\%, 30\%, and 20\%, respectively. The latency upper bound for slice $H$ is 1 ms. We consider typical \gls{QoS} policy rules prioritizing \gls{IC} traffics over non-critical \gls{IoT} but ensuring that the stringent reliability requirements of \gls{HRLLC} are never violated. Due to an unforeseen large-scale event (e.g. a concert), slice $I$ experiences a surge in traffic, up to 80\% utilization, consuming excessive compute and bandwidth resources in the \gls{UPF}. This led to an increased latency up to 2 ms in slice $H$, and left insufficient capacity to efficiently meet its bandwidth needs.

In conventional 5G networks, the \gls{NWDAF} continuously monitors the performance of each slice, and spots the slice $I$'s resource overuse as well as the latency spikes for slice $H$. Thereby, the \gls{PCF} is notified to perform appropriate policy-based corrective actions. The \gls{PCF} generates \gls{QoS} updates to deprioritize slice $I$ traffic, e.g. through reducing the allocated bandwidth from 50\% to 30\% and altering the inherent \gls{ARP}, to free up resources for slice $H$. Policy updates are shared with the \gls{SMF} to adjust the corresponding \gls{QoS} parameters. The \gls{SMF} mandates the \gls{AMF} to enforce these parameters via modifying \gls{QoS} rules in the affected \gls{PDU} sessions and adjusting bearer priorities for slice $I$ over the impacted gNBs and UEs. Finally, \gls{QoS} updates are pushed to the \gls{UPF} via \gls{PFCP}, e.g. to reallocate more bandwidth towards slice $H$ and drop non-critical slice $I$ flows. Nevertheless, human decision-making and manual intervention is mandatory to extend the predefined \gls{PCF} policy rules. The \gls{MNO}, in consultation with the external service providers (i.e. the holographic communications platform and the industrial \gls{IoT} service provider) or the corresponding \glspl{SLA}, decided that the least disruptive strategy to restore slice $H$'s \gls{SLA} compliance is to reduce slice $I$'s allocated bandwidth from 50\% to 30\%. This, unfortunately, extends the disruption time.

The \gls{LNF}-based architecture can autonomously and efficiently handle such slicing conflicts. As disparate data get processed by the \gls{LNF}, it tracks the traffic shape of each slice and will capture an unusual pattern indicative of a potential traffic spike in slice $I$. The fine-tuned and \gls{RAG}-powered \gls{LNF} conducts in-depth network-level analysis, identifies the root-cause (i.e. unexpected major event), alerts \gls{NOC} engineers and triggers proactive slicing adjustments to anticipate the anomaly occurrence. When the \gls{LNF} fails to proactively identify the unusual traffic pattern, it can still react to the traffic spike and the latency increase as they occur in real time. Based on its reasoning and optimization capabilities, the \gls{LNF} can find the optimal and precise \gls{QoS} policy that resolves this conflict, e.g. multi-objective optimization among the conflicting slices identifies the Pareto-optimal solutions and selects the most optimal bandwidth allocation for slice $I$, subject to \gls{MNO}'s approval. The \gls{LNF}'s explainable \gls{AI} will reinforce the \gls{MNO} confidence in the \gls{LLM} reasoning, ensuring a fast decision. Then, the required \gls{QoS} policy modifications will be communicated towards the \gls{PCF} for immediate consideration, and will be enforced over the involved \glspl{NF} following the same workflow as the conventional case. Meanwhile, the \gls{LNF} requests the \gls{NWDAF} to monitor throughput- and latency-related metrics in the affected slices. The \gls{NWDAF} reports that both metrics are within their acceptable intervals and thus acknowledging the effectiveness of the enforced actions. The entire resolution workflow is illustrated in Fig.~\ref{fig:LNF_exchanges_with_NFs}. The \gls{LNF} can also generate a report documenting the diagnosis, actions taken, and insights gained for \gls{MNO} records.
\begin{figure*}[t!]
\centering
\includegraphics[width=.99\textwidth]{Fig3.eps}
    \caption{Workflow of \gls{LNF} autonomous operation for anomaly detection, diagnosis and resolution.}
    \label{fig:LNF_exchanges_with_NFs}
\end{figure*}
