The problem of finding the median, or the selection problem in general, is one of the most fundamental computing problems. The median and its variants are widely used in many applications such as sorting, digital signal processing, image processing, and even machine learning \cite{Ferdowsi:2014,Nikolova:2004:Med,Zhang_arxiv.2205.15867}. 
Efficient execution is required for both hardware and software algorithms. To determine the median, two fundamental approaches are the sorting-based method and the histogram-based method~\cite{Adams}.
Hardware implementation offers a variety of energy-saving opportunities. % by using appropriate architectures such as a comparator network consisting of Compare\&Swap (CAS) operations. 
In addition to the architecture optimization, the energy efficiency of HW implementations can be further improved by approximate computing techniques -- the errors are introduced to the computation in order to achieve a reduced energy consumption \cite{Mittal:2016}. Various approaches can be used. In this paper, we focus on functional approximation~\cite{bookaxc} which allows to use the standard synthesis flow.

The median filter is an operation that was approximated even before the advent of approximate computing. A notable example of such an approximation is the so-called Median of Medians (MoM) algorithm, which produces an approximate rather than an exact median~\cite{blum}. %The algorithm assumes a specific distribution of the input data. 
Other approximate approaches, such as those based on separable sorting networks, operate by sharing intermediate results between inputs in stream processing~\cite{Perrot2022, Adams, salvador18}.
The first paper addressing the approximation of 9-input median filters is by Monajati et al.~\cite{Monajati:2015}. In this work, the authors manually approximated 2-bit magnitude comparators, which were then used to construct 8-bit approximate comparators, a crucial component of the CAS operation in HW implementations. The performance median filters built from the 8-bit approximated CAS operations was evaluated using a circuit simulator on a randomly generated set of input data. 
The main limitation of the previous approach is that the quality depends on the data distribution, leading to non-deterministic and hard to predict behavior when used in a real application. In~\cite{vasicek:sekanina:tec}, Vasicek and Sekanina introduced a method for the automatic design of approximate digital circuits using evolutionary algorithms. The objective of the algorithm is to reduce the number of CAS operations of the accurate median network while maintaining the error below a user-defined threshold. Specifically, they reported results for 9-input and 25-input median circuits. Precise CAS are used. The quality of the approximate networks was assessed using a set of training data, and the hardware cost was estimated based on the assumption that the number of CAS operations linearly correlates with hardware cost. This simplification provides a rough estimate for pipelined architectures, accounting for the presence of registers.
Mrazek et al.~\cite{Vasicek2016} focused on the design of approximate median networks suitable for software implementation, enhancing the precision of the quality evaluation process. The quality of candidate approximate circuits was assessed using all permutations generated from a specific monotone sequence, which helped eliminate the bias introduced by randomly generated training vectors. This approach improved the reliability of the evaluation but does not reflect the hardware parameters. %They designed C++ median functions suitable for signal processing in embedded systems and evaluated the candidate solution in $\mathcal{O}(n!)$ complexity. This approach helped to avoid the bias introduced by a randomly generated set of training vectors. However, this work focuses on software median filters and does not reflect the hardware parameters.

In this paper, we propose an improved method for design of approximate medians of high quality suitable for real-time applications.
The primary contribution is \textbf{a formal algorithm} for the data-independent and exact evaluation of the quality of approximate medians, which can be efficiently computed using Binary Decision Diagrams (BDDs)\footnote{BDD-based evaluation algorithm and the proposed approximate circuits are available at \repository}.
We \textbf{reduced the evaluation complexity} from $\mathcal{O}(n!)$ to $\mathcal{O}(2^n)$ by extending the zero-one theorem introduced in~\cite{Knuth:ACP} for exact sorting networks. 
In addition to the error rate, our approach can also determine the average and worst-case errors, as well as the error distribution. %We \textbf{formalized the quality analysis} using \#SAT \footnote{BDD-based evaluation algorithm and the proposed approximate circuits are available at \repository}. 
To estimate the hardware cost, we proposed a new \textbf{HW-oriented estimation method} that reflects parameters of the pipelined circuits.
These approaches were applied in the evolutionary-based \textbf{automated approximation}.