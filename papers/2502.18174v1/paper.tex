\documentclass[conference]{IEEEtran}


\IEEEoverridecommandlockouts
% The preceding line is only needed to identify funding in the first footnote. If that is unneeded, please comment it out.
\usepackage{cite}
\usepackage{amsmath,amssymb,amsfonts}
\usepackage{algorithmic}
\usepackage{graphicx}
\usepackage{url}
\usepackage{textcomp}
\usepackage{xcolor}
\usepackage[ruled,vlined,nofillcomment,linesnumbered,titlenotnumbered]{algorithm2e}

\usepackage{fancyhdr}
% For Arxiv
\fancypagestyle{firstpage}{% Page style for first page
  \fancyhf{}% Clear header/footer
  \renewcommand{\headrulewidth}{0.4pt}% Header rule
  %\renewcommand{\footrulewidth}{0.4pt}% Footer rule
  \fancyhead[C]{ 	To appear at the IEEE International Symposium on Circuits and Systems (ISCAS 2025), May 25-28, 2025, London, UK.}% Header
 
}

\usepackage{subcaption,tabularx,booktabs,arydshln}
\newcolumntype{C}{>{\centering\arraybackslash}X}%
\newcolumntype{P}{>{\centering\arraybackslash}X}%
\newcolumntype{R}{>{\raggedleft\arraybackslash}X}%

\usepackage[switch]{lineno}
\usepackage{askmaps, sparklines, tikz, etex}

%Funkce, ktera umi parsovat argument typu cisla oddelena carkou a automaticky pocitat X souradnici 
% pouziti: 
% \hist{5}{.3,.18,.55,.2,1.0}  prvni parametr je pocet polozek, tj. 5
% \hist[20]{5}{.3,.18,.55,.2,1.0}  nepovinny parametr udava sirku grafu
\makeatletter
\newcommand{\hist}[3][10]{%default width = #1 = 10ex
  \newcount\itmCntr%
  \def\nextitem{\def\nextitem{}}%
  \def\incCntr{\itmCntr=\numexpr\itmCntr+1\relax}%
\begin{sparkline}{#1}
  \@for \el:=#3 \do{\nextitem\pgfmathparse{(\the\itmCntr*1.0+0.5)/#2}\incCntr\let\x=\pgfmathresult\let\y=\el%
      \sparkspike {\x} {\y} 
  }%
\end{sparkline}%
}
\makeatother
\newcommand{\xth}[1]{#1\textsuperscript{th}}
%\newcommand*{\Comb}[2]{{}^{#1}C_{#2}}%
\newcommand*{\Comb}[2]{\binom{#1}{#2}}%
\newcommand{\repository}{\textcolor{blue}{\url{https://github.com/ehw-fit/approximate-medians}}}
\newcommand{\vm}[1]{\textcolor{red}{#1}}
\begin{document}

\title{AxMED: Formal Analysis and Automated Design of Approximate Median Filters using BDDs
\thanks{The work on this paper was supported by the Czech Science Foundation grant 24-10990S.}\vspace{-0.5em}
}

\author{\IEEEauthorblockN{Vojtech Mrazek and Zdenek Vasicek}
\IEEEauthorblockA{\textit{Brno University of Technology}, Brno, Czechia \\
mrazek@fit.vutbr.cz, vasicek@fit.vutbr.cz\vspace{-1.5em}}
}

\maketitle

\thispagestyle{firstpage}

\begin{abstract}
The increasing demand for energy-efficient solutions has led to the emergence of an approximate computing paradigm that enables power-efficient implementations in various application areas such as image and data processing. %The underlying principle is that many of these applications are highly error tolerant, allowing for simplified circuit implementations without significantly compromising accuracy.
The median filter, widely used in image processing and computer vision, is of immense importance in these domains. % Hence, there is a compelling motivation to explore and design power-efficient implementations of median circuits that exploit the advantages of approximate computation and address the increasing demand for energy-efficient solutions without compromising the integrity of image processing and computer vision applications.
We propose a systematic design methodology for the design of power-efficient median networks suitable for on-chip or FPGA-based implementations. 
A search-based design method is used to obtain approximate medians that show the desired trade-offs between accuracy, power consumption and area on chip.
A new metric tailored to this problem is proposed to quantify the accuracy of approximate medians. Instead of the simple error rate, our method analyses the rank error.
A significant improvement in implementation cost is achieved. For example, compared to the well-optimized high-throughput implementation of the exact 9-input median, a 30\% reduction in area and a 36\% reduction in power consumption was achieved by introducing an error by one position (i.e., allowing the 4th or 6th lowest input to be returned instead of the median).\vspace{-0.5em}
\end{abstract}

\begin{IEEEkeywords}
approximate computing, median filter, approximate median filter, formal verification
\end{IEEEkeywords}

\vspace{-0.5em}
\section{Introduction}
\section{Introduction}


\begin{figure}[t]
\centering
\includegraphics[width=0.6\columnwidth]{figures/evaluation_desiderata_V5.pdf}
\vspace{-0.5cm}
\caption{\systemName is a platform for conducting realistic evaluations of code LLMs, collecting human preferences of coding models with real users, real tasks, and in realistic environments, aimed at addressing the limitations of existing evaluations.
}
\label{fig:motivation}
\end{figure}

\begin{figure*}[t]
\centering
\includegraphics[width=\textwidth]{figures/system_design_v2.png}
\caption{We introduce \systemName, a VSCode extension to collect human preferences of code directly in a developer's IDE. \systemName enables developers to use code completions from various models. The system comprises a) the interface in the user's IDE which presents paired completions to users (left), b) a sampling strategy that picks model pairs to reduce latency (right, top), and c) a prompting scheme that allows diverse LLMs to perform code completions with high fidelity.
Users can select between the top completion (green box) using \texttt{tab} or the bottom completion (blue box) using \texttt{shift+tab}.}
\label{fig:overview}
\end{figure*}

As model capabilities improve, large language models (LLMs) are increasingly integrated into user environments and workflows.
For example, software developers code with AI in integrated developer environments (IDEs)~\citep{peng2023impact}, doctors rely on notes generated through ambient listening~\citep{oberst2024science}, and lawyers consider case evidence identified by electronic discovery systems~\citep{yang2024beyond}.
Increasing deployment of models in productivity tools demands evaluation that more closely reflects real-world circumstances~\citep{hutchinson2022evaluation, saxon2024benchmarks, kapoor2024ai}.
While newer benchmarks and live platforms incorporate human feedback to capture real-world usage, they almost exclusively focus on evaluating LLMs in chat conversations~\citep{zheng2023judging,dubois2023alpacafarm,chiang2024chatbot, kirk2024the}.
Model evaluation must move beyond chat-based interactions and into specialized user environments.



 

In this work, we focus on evaluating LLM-based coding assistants. 
Despite the popularity of these tools---millions of developers use Github Copilot~\citep{Copilot}---existing
evaluations of the coding capabilities of new models exhibit multiple limitations (Figure~\ref{fig:motivation}, bottom).
Traditional ML benchmarks evaluate LLM capabilities by measuring how well a model can complete static, interview-style coding tasks~\citep{chen2021evaluating,austin2021program,jain2024livecodebench, white2024livebench} and lack \emph{real users}. 
User studies recruit real users to evaluate the effectiveness of LLMs as coding assistants, but are often limited to simple programming tasks as opposed to \emph{real tasks}~\citep{vaithilingam2022expectation,ross2023programmer, mozannar2024realhumaneval}.
Recent efforts to collect human feedback such as Chatbot Arena~\citep{chiang2024chatbot} are still removed from a \emph{realistic environment}, resulting in users and data that deviate from typical software development processes.
We introduce \systemName to address these limitations (Figure~\ref{fig:motivation}, top), and we describe our three main contributions below.


\textbf{We deploy \systemName in-the-wild to collect human preferences on code.} 
\systemName is a Visual Studio Code extension, collecting preferences directly in a developer's IDE within their actual workflow (Figure~\ref{fig:overview}).
\systemName provides developers with code completions, akin to the type of support provided by Github Copilot~\citep{Copilot}. 
Over the past 3 months, \systemName has served over~\completions suggestions from 10 state-of-the-art LLMs, 
gathering \sampleCount~votes from \userCount~users.
To collect user preferences,
\systemName presents a novel interface that shows users paired code completions from two different LLMs, which are determined based on a sampling strategy that aims to 
mitigate latency while preserving coverage across model comparisons.
Additionally, we devise a prompting scheme that allows a diverse set of models to perform code completions with high fidelity.
See Section~\ref{sec:system} and Section~\ref{sec:deployment} for details about system design and deployment respectively.



\textbf{We construct a leaderboard of user preferences and find notable differences from existing static benchmarks and human preference leaderboards.}
In general, we observe that smaller models seem to overperform in static benchmarks compared to our leaderboard, while performance among larger models is mixed (Section~\ref{sec:leaderboard_calculation}).
We attribute these differences to the fact that \systemName is exposed to users and tasks that differ drastically from code evaluations in the past. 
Our data spans 103 programming languages and 24 natural languages as well as a variety of real-world applications and code structures, while static benchmarks tend to focus on a specific programming and natural language and task (e.g. coding competition problems).
Additionally, while all of \systemName interactions contain code contexts and the majority involve infilling tasks, a much smaller fraction of Chatbot Arena's coding tasks contain code context, with infilling tasks appearing even more rarely. 
We analyze our data in depth in Section~\ref{subsec:comparison}.



\textbf{We derive new insights into user preferences of code by analyzing \systemName's diverse and distinct data distribution.}
We compare user preferences across different stratifications of input data (e.g., common versus rare languages) and observe which affect observed preferences most (Section~\ref{sec:analysis}).
For example, while user preferences stay relatively consistent across various programming languages, they differ drastically between different task categories (e.g. frontend/backend versus algorithm design).
We also observe variations in user preference due to different features related to code structure 
(e.g., context length and completion patterns).
We open-source \systemName and release a curated subset of code contexts.
Altogether, our results highlight the necessity of model evaluation in realistic and domain-specific settings.






\section{Analyzing Approximate Medians}\label{sec:analysis}
%Before describing the algorithm for the analysis of approximate medians based on comparison networks,  let us first discuss the simpler problem -- the formal proof that a comparison network $M$ is the median network.
To prove that a comparison network $M$ is the exact median network, we can use the so-called {\em zero-one theorem}~\cite{Knuth:ACP} introduced to prove the validity of sorting networks. This theorem states that if a comparison network with $n$ inputs sorts all $2^n$ input sequences of 0's and 1's in non-decreasing order, it will sort any arbitrary sequence of $n$ elements in non-decreasing order. As a consequence of that, it is sufficient to restrict the inputs to Boolean values. This allows the verification task to be formulated as a pseudo-Boolean Constraint Satisfaction Problem (CSP).
Let $M(x_1,\cdots,x_{n})$ be a comparison network with $n$-inputs ($n$ is always odd in this paper). Let $x_i \in B$ be a Boolean variable, $B = \{0,1\}$. Then, each median network must satisfy that % $\forall {\bf x} \in \mathrm B^n$.
\begin{equation}
M(x_1,\cdots,x_{n}) = \begin{cases}
         0 & \text{if } x_1 + \cdots + x_n < (n+1)/2 \\
         1 & \text{if } x_1 + \cdots + x_n \ge (n+1)/2 \\
         \end{cases}\label{eq:pbc}
\end{equation}
\noindent holds for all input combinations, i.e. $\forall \{x_1,\cdots,x_n\} \in \mathrm B^n$.

\subsection{Worst-Case Error Analysis}
In the case of an approximate median, an invalid value is returned for some input combination. 
Taking into account the fact that each comparison network performs a selection process, this means that a different value is selected from the sequence of input elements and returned instead of the median.
To model the error introduced by the approximations, we can measure the distance between the rank of the returned element and the rank of the median, i.e., a fixed value equal to $m = (n+1)/2$. For example, the 3\textsuperscript{rd} largest value is returned instead of 5\textsuperscript{th} largest expected for 9-input median, and the rank distance is 2. This can also be applied in the Boolean domain. Let $M({\bf x})=0$ for ${\bf x}=(1,1,0,1,1)$. Let $\widetilde{\bf x}$ denote the sorted sequence $\bf x$, i.e. $\widetilde{\bf x}=(0,1,1,1,1)$. 
In this case, $M$ returns the first lowest element, i.e., $M({\bf x})=\widetilde{x_1}$, instead of the median, i.e., the 3\textsuperscript{rd} lowest element.
Similarly, let $M({\bf x})=1$ for ${\bf x}=(0,1,0,0,0)$. The $\widetilde{\bf x}=(0,0,0,0,1)$ shows that the \xth{5} lowest item was chosen instead of the median. In both cases the distance between the median and the returned value is $d = 2$.

In general, $M$ can have an asymmetric error distribution. Therefore, it may be advantageous to investigate the left ($d^L$) and right ($d^R$) worst-case distance separately.
The worst-case error analysis problem can be formulated using CSP as follows.
Find maximal $d^L \in \{0\ldots m-1\}$ such that $\exists {\bf x} \in B^n : M({\bf x}) = 0 \land x_1 + \cdots + x_n = n - (m - d^L)$. Similarly, find maximal $d^R \in \{0\ldots m-1\}$ such that $\exists {\bf x} \in B^n : M({\bf x}) = 1 \land x_1 + \cdots + x_n = m - d^R$. 
If $d^L=d^R=0$, $M$ is an accurate median network.
Otherwise, $d^L$ ($d^R$) defines the worst case distance between the obtained value and the median that can occur in practice.



\subsection{Error Distribution Analysis}
In most applications, knowing the worst-case error is not enough, as the probability of its occurrence may be negligible.
For such a scenario, we propose a technique to obtain an error distribution that provides information about the probability of occurrence of errors at different distances.

Let $a^L_i=|\{{\bf x} \in B^n: M({\bf x}) = 0 \land x_1 + \cdots + x_n = n - (m - i)\}|$ be the number of input assignments containing exactly $n - (m - i)$ ones for which zero was incorrectly chosen as the output value provided that $0 < i < m$.
Similarly, let $a^R_i=|\{{\bf x} \in B^n: M({\bf x}) = 1 \land x_1 + \cdots + x_n = m - i\}|$ be the number of input assignments containing exactly $n - (m - i)$ zeroes for which 1 was incorrectly chosen as the output value. Note that $a_0 = a^L_0 = a^R_0$ is the number of input assignments for which $M$ provides the expected output value.
The histogram of error distribution $H(M)=(h^L_{m-1},h^L_{m-2},\ldots, h^L_0=h^R_0, \ldots, h^R_{m-2},h^R_{m-1})$ can be determined as 
\begin{equation*}
h^R_i = \begin{cases}

 a^R_i / \Comb{n}{m-i} - a^R_{i+1} / \Comb{n}{m-i-1} & \text{for } 0 \le i < m-1 \\ 
 a^R_i / \Comb{n}{m-i}                               & \text{otherwise} 
 
 \end{cases}
\end{equation*}
\begin{equation*}
h^L_i = \begin{cases}
 a^L_i / \Comb{n}{m-1+i} - a^L_{i+1} / \Comb{n}{m+i} & \text{for } 0 \le i < m-1 \\ 
 a^L_i / \Comb{n}{m-1+i}                             & \text{otherwise} 
 \end{cases}\label{eq:hist}
\end{equation*}
\noindent where $h_i$ is the probability of the error occurence of distance $i$ and binomial coefficient $\Comb{n}{r} = n!/(n-r)!r!$ gives the total number of combinations containing exactly $r$ ones (zeroes).
The calculation was successfully validated according to the exact exhaustive permutation principle \cite{Vasicek2016} for smaller instances.


\begin{figure}[b]\vspace{-15pt}
\centering\includegraphics[width=0.8\columnwidth]{fig/bddcirc3.pdf}%
\caption{Virtual circuit used for an efficient worst-case and error distribution analysis}%
\label{fig:bdd}\vspace{-1em}
\end{figure} 

\subsection{Analysis using BDDs}

To compute the error distribution and efficiently identify the worst case error, we propose to solve the CSP problem using BDDs and the virtual circuit shown in Figure~\ref{fig:bdd}.
This circuit consists of the analysed comparison network, a sorting network and a simple logic. 
The sorting network acts as a counter of the number of input variables that are set to one. 
To determine $a^L_i$ and $a^R_i$, the SatCount (\#SAT) operation available in every BDD package is called for each output, i.e.
$a^L_i = \mathrm{SatCount}(q^L_i)$, $a^R_i = \mathrm{SatCount}(q^R_i)$, $i=0\ldots m-1$.



The construction of BDD is straightforward. 
Due to the zero-one principle, each CAS element corresponds to a pair of logic gates (AND and OR) that must to be included in the BDD. The AND gate determines the minimum value, the OR gate determines the maximum value. 
Finally, the auxiliary logic is included in the BDD.




\section{Proposed Approximation Method}\label{sec:method}
Several approaches have been proposed to approximate digital circuits~\cite{aproxcomp22, Sekanina2021}. Inspired by~\cite{vasicek:sekanina:tec}, we chose Cartesian Genetic Programming (CGP)~\cite{miller:cgp:book}. %because it is easy to implement, it can easily handle constraints on candidate circuits, it is naturally multi-objective, and it has already been used to obtain high-quality approximate circuits.
CGP is an evolutionary algorithm that encodes a circuit to a cartesian grid of nodes. The circuit is represented by an integer netlist specifying the connection of node inputs, functions, and the output node. In our case, a comparison network with $n$ inputs consisting of $k$ CAS elements can be represented by a directed acyclic graph with $k$ two-input two-output nodes, where each node corresponds to a single CAS.
An example is given in Fig.~\ref{fig:cgp}.

\begin{figure}[b]
\centering\vspace{-1.5em}
\includegraphics[width=0.9\columnwidth]{fig/connected.pdf}
\caption{Example of a comparison network consisting of 7 operations encoded using CGP with parameters: $n=5$, $n_c=8$.
This network can be encoded using the sequence $(0,1,0)(2,3,0)(6,7,0)(8,4,0)(5,10,1)$ $(12,4,0)(13,15,0)(15,16,1)(17)$. 
}\vspace{-1em}
\label{fig:cgp}
\end{figure}


The CGP algorithm starts with an exact circuit encoded in the aforementioned encoding. Then, $\lambda$ offsprings are generated using mutation of $h$ randomly chosen integers in the netlist while keeping constraints on generating acyclic representation. A random mutation may disconnect some active nodes that become redundant (see node 7 in the example in Fig.~\ref{fig:cgp}).
For a detailed description, please refer to~\cite {miller:cgp:book}.
We propose to apply a two-stage procedure. At the beginning, the designer specifies the target reduction (area) that should be achieved. The first stage starts with an exact solution. The goal is to gradually modify the initial solution and obtain a reduced network of the target cost $t$, providing that a small difference $\varepsilon$ is tolerated, which prevents the search from getting stuck in a local extreme.
In the second stage, which begins as soon as the target reduction is achieved, the search method reflects the cost and quality. The goal of the second stage is to minimize the fitness determined by the quality of the candidate circuit $M$
\begin{equation}\label{eq:optproblem}
F(M) = \begin{cases}
			\mathcal{Q}(M), & \text{if $t - \varepsilon \le \mathcal{C}(M) \le t + \varepsilon$}\\
                \infty, & \text{otherwise}
		 \end{cases}
\end{equation}

In order to determine the quality of a given comparison network $M$, we use the quality metric $\mathcal Q(M)$ constructed from the histogram of error distribution $H(M)$ defined above as follows
\begin{equation}
\mathcal Q(M) = \sum_{j = -m+1}^{m-1} j^2 H_{m+j}(M) %prostredni se nasobi 0
 \label{eq:qc}
\end{equation}
While this function returns zero value if and only if $M$ is the exact median, a positive non-zero value is returned in the other cases. The $j$ value determining the rank weight is squared because it emphasizes the most important part of the histogram and minimizes the worst-case errors.

In contrast to the previous works we targets fully pipelined architecture which needs to consider the number of stages, registers, CAS operations and multiplexors.
Given a candidate comparison network $M$, the implementation cost $\mathcal C(M)$ is determined as follows.
First, the active nodes (CAS) are identified. 
A node is active if at least one of its outputs is connected to (a) the primary output or (b) the input of an active node. 
The ASAP~\cite{Paulin:1989} scheduling algorithm is then used to determine the position of the pipeline registers.
Outputs connected only to inactive nodes are ignored during this step.
The area on a chip required to implement a candidate comparison network can then be determined as $\mathcal C(M) = A_{mx} (2 n_A + n_P) + A_{cmp}(n_A + n_P) + A_{reg} n_R$,
where $n_A$ is the number of active nodes with both outputs connected to either an active node or primary output, $n_P$ is the number of active nodes with one output connected to an inactive node, $n_R$ is the number of required registers and $A_{mx}$, $A_{cmp}$, $A_{reg}$ are the area on chip required to implement a 2-input $w$-bit multiplexer, a magnitude comparator and a register respectively in a chosen target technology.



\section{Results and Discussion}\label{sec:setup}
For statistical evidence, we carried out 20 repetitions of CGP at each design point. % to eliminate the variation caused by the stochastic nature of CGP. 
In total, 800 experimental runs were performed for each median network, each running 30 minutes.
In the case of 9-input and 25-input median, the known optimal implementations consisting of the minimum number of CAS elements were used as the reference initial design point. \footnote{We report results for 9- and 25-input median networks only, the full results are available in \repository.}
The selected networks were implemented in VHDL as fully pipelined streaming architectures operating at 8-bit precision and synthesized to the 45nm technology with 1~GHz clock using Design Compiler.




\subsection{Analysis of Approximate Medians}
The crucial component of the evolutionary search is the calculation of the histogram of error distribution $H(M)$. Although BDDs are generally known for their poor scalability in certain cases, the proposed approach exhibits excellent scalability. 
On average, 400 milliseconds are required to compute the error distribution for $n=49$. 
If we were to use exhaustive simulation instead of the BDDs, we would have to evaluate the response for $2^{49}$ input combinations, which is intractable. The method presented in~\cite{Vasicek2016} requires evaluating $49!$ input combinations, which is more than $10^{62}$ and also intractable.


Figure~\ref{fig:time} summarizes the run times we measured during our experiments. Each boxplot illustrates the time required to analyze a single candidate solution during the evolutionary search. We compared our proposed exact principle with the permutation testing method presented in \cite{Vasicek2016,vasicek:sekanina:tec}. Even though the permutation testing only examined 1000 random permutations (i.e., 0.2\% and 6$\cdot 10^{-21}$\% of all possible permutations for 9- and 25-input medians, respectively), the BDD-based proposed method is more than two orders of magnitude faster and provides the true $H$ distribution and prove the worst-case errors $d_L, d_H$.

\begin{figure}[t]%\vspace{-1em}
\centering\includegraphics[width=\columnwidth]{fig/evaltime.pdf}\vspace{-1em}
\caption{Computational complexity of the proposed BDD-based analysis framework compared to the subset of 1000 vectors in permutations \cite{Vasicek2016}.}\label{fig:time}%\vspace{-15pt}
\end{figure} 


\subsection{Designed Circuits}
\begin{table}[b!]%\renewcommand{\arraystretch}{1.1}
\setlength{\tabcolsep}{2.5pt}%
\caption{
Parameters of selected Pareto optimal approximations of %(a) 9-input, and (b) 25-input 
median networks.
The $k$ denotes the number of CAS, $l$ is the latency. The area is given in $\mu\mathrm{m}^2$ and power in mW.}\label{tab:results}%\vspace{-0.5em}
\begin{table*}
\centering
\renewcommand{\dashlinedash}{0.5pt} 
\renewcommand{\dashlinegap}{2pt}    
\begin{tabular}{lllll}
\hline\hline
 & \shortstack[l]{\space \\ \space \\ \textbf{Methods} \\ \space } & 
   \shortstack[l]{\space \\ \space \\ \textbf{Venue} \\ \space } & 
   \shortstack[l]{\space \\ \space \\ \textbf{Scheme} \\ \space } & 
   \shortstack[l]{\space \\ \space \\ \textbf{Cross-stage fusion} \\ \space} \\
\bottomrule   \\
Single-stage
& Zhang et al.~\cite{zhang2018single} & CVPR 2018 & $I \rightarrow [T, R]$ & - \\ 
& ERRNet~\cite{wei2019single} & CVPR 2019 & $I \rightarrow T$ & - \\ 
& RobustSIRR~\cite{song2023robust} & CVPR 2023 & $I_{multiscale} \rightarrow T$ & - \\
& YTMT~\cite{hu2021trash} & NeurIPS 2021 & $I \rightarrow [T, R]$ & - \\   \bottomrule   \\
Two-stage
& CoRRN~\cite{wan2019corrn} & TPAMI 2019 & \shortstack[l]{\space \\$I \rightarrow E_{T}$ \\ $[I, E_{T}] \rightarrow T$} & Convolutional Fusion \\ \cline{2-5}
& DMGN~\cite{feng2021deep} & TIP 2021 & \shortstack[l]{\space \\ $I \rightarrow [T_{1}, R]$ \\ $[I, T_{1}, R] \rightarrow T$} & Convolutional Fusion \\ \cline{2-5}
& RAGNet~\cite{li2023two} & Appl. Intell. 2023 & \shortstack[l]{\space \\ $I \rightarrow R$ \\ $ [I, R] \rightarrow T$} & Convolutional Fusion \\ \cline{2-5}
& CEILNet~\cite{fan2017generic} & ICCV 2017 & \shortstack[l]{\space \\ $[I, E_{I}] \rightarrow E_{T}$ \\ $[I, E_{T}] \rightarrow T$ } & Concat \\ \cline{2-5}
& DSRNet~\cite{hu2023single} & ICCV 2023 & \shortstack[l]{\space \\ $I \rightarrow (T_{1}, R_{1})$ \\ $(R_{1}, T_{1}) \rightarrow (R, T, residue)$} & N/A \\ \cline{2-5}
& SP-net BT-net~\cite{kim2020single} & CVPR 2020 & \shortstack[l]{\space \\ $I \rightarrow [T_{1}, R_{1}]$ \\ $R_{1} \rightarrow R$} & N/A \\ \cline{2-5}
& Wan et al.~\cite{wan2020reflection} & CVPR 2020  & \shortstack[l]{\space \\ $[I, E_{I}] \rightarrow R_{1}$ \\ $R_{1} \rightarrow R$} & N/A \\ \cline{2-5}
& Zheng et al.~\cite{zheng2021single} & CVPR 2021 & \shortstack[l]{\space \\ $I \rightarrow e$ \\ $[I, e] \rightarrow T$} & Concat \\ \cline{2-5}
& Zhu et al.~\cite{zhu2024revisiting} & CVPR 2024 & \shortstack[l]{\space \\ $I \rightarrow E_{R}$ \\ $ [I, E_{R}] \rightarrow T$} & Concat \\ \cline{2-5}
& Language-Guided~\cite{zhong2024language} & CVPR 2024 & \shortstack[l]{\space \\ $[I, Texts] \rightarrow R\ or\ T$ \\ $[I, R\ or\ T] \rightarrow T\ or\ R$} & Feature-Level Concat
\\   \bottomrule   \\
Multi-stage
& BDN~\cite{yang2018seeing} & ECCV 2018 & \shortstack[l]{\space \\ $I \rightarrow T_{1}$ \\ $[I, T_{1}] \rightarrow R$ \\ $[I, R] \rightarrow T$ } & Concat \\ \cline{2-5}
& IBCLN~\cite{li2020single} & CVPR 2020 & \shortstack[l]{\space \\ $[I, R_{0}, T_{0}] \rightarrow [R_{1}, T_{1}]$ \\ $[I, R_{1}, T_{1}] \rightarrow [R_{2}, T_{2}]$ \\ \ldots} & \shortstack[l]{Concat \\ Recurrent} \\ \cline{2-5}
& Chang et al.~\cite{chang2021single} & WACV 2021 & \shortstack[l]{\space \\ $I \rightarrow E_{T}$ \\ $[I, E_{T}] \rightarrow T_{1} \rightarrow R_{1} \rightarrow T_{2}$ \\ $[I, E_{T}, T_{2}] \rightarrow R \rightarrow T$} & \shortstack[l]{Concat \\ Recurrent} \\ \cline{2-5}
& LANet~\cite{dong2021location} & ICCV 2021 & \shortstack[l]{\space \\ $[I, T_{0}] \rightarrow R_{1} \rightarrow T_{1}$ \\ $[I, T_{1}] \rightarrow R_{2} \rightarrow T_{2}$ \\ \ldots} & \shortstack[l]{Concat \\ Recurrent} \\ \cline{2-5}
& V-DESIRR~\cite{prasad2021v} & ICCV 2021 & \shortstack[l]{\space \\ $I_{1} \rightarrow T_{1}$ \\ $[I_{1}, T_{1}, I_{2}] \rightarrow T_{2}$ \\ \ldots \\ $[I_{n-1}, T_{n-1}, I_{n}] \rightarrow T$ } & \shortstack[l]{Convolutional Fusion \\ Recurrent} \\
\hline\hline
\end{tabular}
\caption{\textbf{I}, \textbf{R}, \textbf{T}, and \textbf{E} represent the \textbf{I}nput, \textbf{R}eflection, \textbf{T}ransmission, and \textbf{E}dge map, respectively. The subscripts of \textbf{T} and \textbf{R} represent intermediate process outputs. The Absorption Effect $e$ is introduced in~\cite{zheng2021single} to describe light attenuation as it passes through the glass. The output $residue$ term, proposed in~\cite{hu2023single}, is used to correct errors in the additive reconstruction of the reflection and transmission layers. Language descriptions in~\cite{zhong2024language} provide contextual information about the image layers, assisting in addressing the ill-posed nature of the reflection separation problem.}
\label{tab:1}
\end{table*}

\end{table}

Parameters of the selected approximate medians are given in Table~\ref{tab:results}.
The reference implementation of the median filter (i.e. the exact implementation) is denoted as \#1 and \textit{median of medians}~\cite{blum} is denoted as MoM. The parameters of software-defined approximate medians~\cite{Vasicek2016} are not included because they are not suitable for the pipelined architecture.
To compare the estimated value with real area, the value of cost function $\mathcal C(M)$ is also given. 
In addition to that, various quality parameters such as the normalized value of quality function $\mathcal Q(M)$, worst-case error ($d_L$, $d_R$), error probability ($h_0$) and error distribution $H(M)$ are included.
The value of the quality indicator $\mathcal Q(M)$ decreases with decreasing number of CAS elements and it seems that the central tendency of the error distribution is adequately captured by the proposed quality metric.

If we reduce the number of CAS by 26\% for 9-input median, i.e. to $k=14$ operations corresponding to the instance \#6 in Table~\ref{tab:results} (a), the output value is correctly determined in more than 72\% of all possible cases.
In the remaining cases, the output value is incorrectly determined as the \xth{4} or \xth{6} lowest element of the list of sorted input elements. Since the median is equal to the \xth{5} lowest element, the distance between the median and the output value is equal to 1 in both cases. 
If we use this approximate median, e.g. in multimedia, the error will hardly be visible in practice.
By using this architecture instead of the exact one described as \#1, we can reduce power consumption by 35\% without compromising output quality.
In some cases, e.g. in the outlier detection task, we can tolerate even more distant errors and use e.g. implementation \#9, which allows us to achieve more than 69\% power improvement.
The MoM \cite{blum} represent widely used approximate implementation of median. These medians for the same hardware cost exhibits lower worst-case error, however, the overall quality is worser because the probability of the exact solution $h_0$ is significantly lower.

Quality in application was evaluated on ten randomly selected test images from the Berkley image dataset~\cite{berkeley:images}. Each image was corrupted by salt-and-pepper noise and random-valued shot noise of 1\%, 5\%, 10\%, 15\% and 20\% intensity. 
The first six instances of the 9-input median have SSIM above 0.97, which is extremely good. In the case of the 25-input medians, we can observe a very small decrease in SSIM. The SSIM remains above 0.9 even for the most approximated instance (instance \#10). This suggests that the noise filtering task could tolerate an even more aggressive approximation.
Note that the 9-input (25-input) MoM exhibits SSIM=0.971 (0.960). This roughly corresponds to the instance \#6 (\#9) with the same implementation cost.

\section{Conclusions}

In this paper, we propose a systematic design method for search-based synthesis of approximate median architectures. The key idea is to reduce the number of CAS elements while introducing a minimum error.
The obtained implementations can be combined with other approaches, e.g. approximate comparators \cite{Monajati:2015} or composition of medians \cite{blum,Adams,Perrot2022}, to further reduce the power consumption.

In order to measure the quality of the approximation, an efficient method for the analysis of the worst-case error and error distribution has been proposed.
Combined with the search-based design approach, our framework is able to produce high-quality approximations with a guaranteed error distribution that is maintained regardless of the chosen bit width. %


\clearpage
\bibliographystyle{IEEEtran}
\bibliography{IEEEabrv,paper}





\end{document}
