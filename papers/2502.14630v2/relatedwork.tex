\section{Literature Review}
Understanding the electrical load requirements of rural off-grid households is essential to understand individuals needs and ensure appropriate design and maintenance choices are made. Before access to big field datasets was possible, the common method of estimating load demand was consumer use surveys, where households were asked questions on anticipated appliance ownership and times of use \cite{Pandyaswargo2020EstimatingLaos, Sandwell2016AnalysisPradesh, Mandelli2016NovelAreas, Boait2017ESCoBox:World, Boait2015EstimationCountries}. However, surveys have since been proven to be an inaccurate indicator of load demand, both in terms of magnitude and time of use, with errors up to four times greater than actual demand data \cite{Blodgett2017AccuracyUserconsumption, Hartvigsson2018ComparisonData, Allee2021PredictingModels,Gelchu2023ComparisonEthiopia}. 

A data-driven approach is established as more accurate in literature, however, the lack of available field data is a challenge \cite{Bisaga2017ScalableSolar}. In some cases, a limited amount of load demand data is available and this can be used alongside other techniques. Allee et al.\ \cite{Allee2021PredictingModels} combine a data-driven approach with survey results to predict the load demand of 1,378 Tanzanian mini-grid customers, demonstrating the use of surveys can be beneficial when supplemented by measured data. Few et al.\ \cite{Few2022ElectricityDesign} combine mini-grid load demand data with information on climatic conditions and household level characteristics, such as appliance ownership, to simulate household level load profiles. Both cases show that additional information, beyond stand-alone load demand data, further increases load demand estimation capabilities. Yet, reliable load demand data is still the essential starting block.

Greater granularity of the load demand data enables useful insights for system design and maintenance, but sometimes aggregated data is all that is available. Data can be aggregated on a household level (i.e.\ for multiple households) \cite{Bhattacharyya2021AAsia} or a temporal level (i.e.\ multiple households for a single time period) \cite{DenHeeten2017UnderstandingElectrification}, but in both instances, the results are diminished compared to using raw granular data, and this can negatively impact design and maintenance choices \cite{Jurasz2022OnSystems}.

Where ample data is available, clustering can be a useful tool for visualisation and understanding. In the case of load demand clustering, k-centered methods are a common choice due to simple implementation \cite{Yilmaz2019ComparisonManagement}. Some authors have made efforts to compare the effectiveness of different clustering methods \cite{Rajabi2020ASegmentation, Wang2015LoadReview, McLoughlin2015AData}, but different results are found in each case. This highlights the importance of having a clear goal for the clustering process, since this ensures that the algorithms and distance metrics output logical and relevant clusters \cite{VonLuxburg2012Clustering:Art, Hennig2015WhatClusters}. 

Dynamic time warping (DTW) is an elastic distance metric for comparing time-series data which has a one-to-many mapping and is robust to shifts in time \cite{Sakoe1978DynamicRecognition}. This allows a shape-based comparison, as proposed in \cite{Lin2019ClusteringApplications}, without being dependent on dimensionality reduction. The benefits of using DTW over Euclidean distance in load profile clustering are beginning to be recognised in literature \cite{Aghabozorgi2015Time-seriesReview, Ausmus2020ImprovingWarping, Begum2015AcceleratingStrategy, Ratanamahatana2005ThreeMining}. However, due to its quadratic computational complexity, it is not commonly implemented \cite{Rajabi2020ASegmentation}. Teeraratkul et al.\ \cite{Teeraratkul2016CondensedDemand} and Ausmus et al.\ \cite{Ausmus2020ImprovingWarping} clustered household load profiles using DTW with k-medoids and k-means respectively. Both found significant improvement compared to using Euclidean distance, and k-medoids has the advantage of using a time series from the dataset as the representative cluster centre, which can improve interpretability. However, use of k-centered methods in the cluster assignment stage makes the algorithm vulnerable to getting stuck in local optima because the iterative method to solve the k-medoids and k-means problem does not guarantee global optimality. Kumtepeli et al.\ \cite{Kumtepeli2024DTW-C++:Data} instead formulated the k-medoids problem as a mixed integer programming (MIP) problem, allowing global optimality within the cluster assignment while using DTW for the distance metric. MIP is beneficial for alleviating the risk of getting stuck in local optima, however using MIP on very large datasets (with respect to number of time series) is computationally expensive and unfeasible. 

Clustering load profiles from LMIC is much less common than load data for HIC in literature because less data is available. Williams et al.\ \cite{Williams2017LoadMicrogrids} characterised load profiles from eleven mini-grids across Africa finding trends in consumption on daily and monthly time scales. However, no clustering was done, and household level data was not made available, so comprehension of household trends is not possible. Lorenzoni et al.\ \cite{Lorenzoni2020ClassificationApproach} conducted hierarchical clustering on the load demand data of 61 mini-grids in LMICs, finding clusters of aggregated household data. The key differentiation between the clusters was the peaks in load demand. The authors also found that as the systems aged, and the energy connection tier increases, the use profile became flatter.
Lukuyu et al.\ \cite{Lukuyu2023PurchasingAfrica} used k-means to cluster daily load demand profiles of SHS customers in East Africa over a year. However, they used a monthly mean to represent the daily load profile for each household, which loses the granularity of day-to-day variability and gives smoothed profiles for the results.

In addition to understanding day-to-day energy use, it is important to consider long-term behaviour, particularly in the context of energy access for previously unelectrified households. In these situations, behaviour is more unpredictable \cite{Riva2018Long-termPerspective,Muhumuza2018EnergyCountries}. Therefore, long-term load demand analysis is imperative to gain insights into how systems deployed in LMICs are used. %
%
It is commonly thought that energy demands will increase as energy access is obtained \cite{Riva2018Long-termPerspective,Muhumuza2018EnergyCountries,Opiyo2020HowCommunities}, however, this conjecture needs to be supported by real-world data \cite{Riva2019ModellingPlanning}. Bisaga et al.\ \cite{Bisaga2018ToLens} compared energy use of households across Rwanda, splitting them into groups depending on the length of time they had been using systems. They found that households that had systems for over a year generally used less energy each day than those who had obtained systems within the last 6 months, despite the more established households having generally more appliances. Kizilcec et al.\ \cite{Kizilcec2022ForecastingAccess} also conducted work on Rwandan SHS energy use, finding a reduction in electricity demand over the first year of use for the majority of households in their dataset. The decrease in electricity consumption was even more pronounced in households that owned a TV.

Beyond the first year of ownership, Dominguez et al.\ \cite{Dominguez2021UnderstandingHouseholds} found a general increase in monthly electricity expenditure after initial adoption, along with an increase in appliance ownership, but this plateaued at the 2-3 year mark. Conversely, \cite{Masselus202410Rwanda} found that rural households in Rwanda with grid connections decreased their electricity consumption over the first ten years of connection. This analysis includes monthly electrical consumption data from 147,074 rural households, focusing on 174 households with survey data. The surveys found that household appliances largely consisted of lighting and entertainment devices, but economically productive appliances were a rarity. Louie et al.\ \cite{Louie2023DailyNation} analysed the load demand of off-grid households in the Navajo Nation over 2 years and found a large variation both between and within households, the latter being impacted by seasonal changes. Additionally, energy use in the second year was on average 10\% lower than in the first year, highlighting longer-term temporal load shifts. 

The overall picture for long-term temporal changes in energy use remains unclear---increased consumption is anticipated but there are also datasets indicating the opposite. Further information is required to understand underlying causes. The barrier to rural electrification is often economic \cite{Blimpo2020WhyAfrica, Kizilcec2020SolarAfrica}, but there is a research gap in understanding the link between diverse PAYG payment data \cite{Mergulhao2023HowKenya} and energy use. Existing studies establish a link between energy use and payment, but focus mainly on payment behaviour impacts \cite{Guajardo2019HowEconomies} and very coarse-grained energy consumption data \cite{Lukuyu2023PurchasingAfrica}.

This work makes several contributions: firstly, a large-scale analysis of daily SHS load-demand profiles is undertaken, elucidating patterns in real-world data; secondly, progression of energy use over several years is considered, rather than observing a single snapshot in time; finally, payment and energy data are combined, enabling insight into the role of economic factors. This research also provides access to the measured data for others to use, supporting the collective efforts of researchers and industry to achieve SDG7.