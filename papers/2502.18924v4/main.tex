% This must be in the first 5 lines to tell arXiv to use pdfLaTeX, which is strongly recommended.
\pdfoutput=1
% In particular, the hyperref package requires pdfLaTeX in order to break URLs across lines.

\documentclass[11pt]{article}

% Change "review" to "final" to generate the final (sometimes called camera-ready) version.
% Change to "preprint" to generate a non-anonymous version with page numbers.
\usepackage[preprint]{acl}

% Standard package includes
\usepackage{times}
\usepackage{latexsym}

% For proper rendering and hyphenation of words containing Latin characters (including in bib files)
\usepackage[T1]{fontenc}
% For Vietnamese characters
% \usepackage[T5]{fontenc}
% See https://www.latex-project.org/help/documentation/encguide.pdf for other character sets

% This assumes your files are encoded as UTF8
\usepackage[utf8]{inputenc}

% This is not strictly necessary, and may be commented out,
% but it will improve the layout of the manuscript,
% and will typically save some space.
\usepackage{microtype}
\usepackage{booktabs}

% This is also not strictly necessary, and may be commented out.
% However, it will improve the aesthetics of text in
% the typewriter font.
\usepackage{inconsolata}

%Including images in your LaTeX document requires adding
%additional package(s)
\usepackage{graphicx}
\usepackage{float}
\usepackage{overpic}
\usepackage{wrapfig}
\usepackage{caption}
\usepackage{subfig}
\usepackage{multirow}

% math
\usepackage{amsmath}
\usepackage{amsthm}
\usepackage{amssymb}

%symbol
\usepackage{bbding}
\usepackage{enumitem}

% If the title and author information does not fit in the area allocated, uncomment the following
%
%\setlength\titlebox{<dim>}
%
% and set <dim> to something 5cm or larger.

\title{MegaTTS 3: Sparse Alignment Enhanced Latent Diffusion Transformer \\ for Zero-Shot Speech Synthesis}

% Author information can be set in various styles:
% For several authors from the same institution:
% \author{Author 1 \and ... \and Author n \\
%         Address line \\ ... \\ Address line}
% if the names do not fit well on one line use
%         Author 1 \\ {\bf Author 2} \\ ... \\ {\bf Author n} \\
% For authors from different institutions:
% \author{Author 1 \\ Address line \\  ... \\ Address line
%         \And  ... \And
%         Author n \\ Address line \\ ... \\ Address line}
% To start a separate ``row'' of authors use \AND, as in
% \author{Author 1 \\ Address line \\  ... \\ Address line
%         \AND
%         Author 2 \\ Address line \\ ... \\ Address line \And
%         Author 3 \\ Address line \\ ... \\ Address line}

% \author{First Author \\
%   Affiliation / Address line 1 \\
%   Affiliation / Address line 2 \\
%   Affiliation / Address line 3 \\
%   \texttt{email@domain} \\\And
%   Second Author \\
%   Affiliation / Address line 1 \\
%   Affiliation / Address line 2 \\
%   Affiliation / Address line 3 \\
%   \texttt{email@domain} \\}

\author{
 \textbf{Ziyue Jiang\textsuperscript{1,2}\thanks{
Intern at ByteDance.}},
 \textbf{Yi Ren\textsuperscript{2}},
 \textbf{Ruiqi Li\textsuperscript{1,2}},
 \textbf{Shengpeng Ji\textsuperscript{1}},
 \textbf{Boyang Zhang\textsuperscript{1}},
 \textbf{Zhenhui Ye\textsuperscript{1}},
 \textbf{Chen Zhang\textsuperscript{2}},
\\
 \textbf{Bai Jionghao\textsuperscript{1}},
 \textbf{Xiaoda Yang\textsuperscript{1}},
 \textbf{Jialong Zuo\textsuperscript{1}},
 \textbf{Yu Zhang\textsuperscript{1}},
 \textbf{Rui Liu\textsuperscript{3}},
 \textbf{Xiang Yin\textsuperscript{2}},
 \textbf{Zhou Zhao\textsuperscript{1}}
\\
 \textsuperscript{1}Zhejiang University,
 \textsuperscript{2}ByteDance,
 \textsuperscript{3}Inner Mongolia University
\\
 \small{
   \href{ziyuejiang341@gmail.com}{ziyuejiang341@gmail.com},
   \href{zhaozhou@zju.edu.cn}{zhaozhou@zju.edu.cn}
 }
}

\begin{document}
\maketitle
\begin{abstract}
While recent zero-shot text-to-speech (TTS) models have significantly improved speech quality and expressiveness, 
mainstream systems still suffer from issues related to speech-text alignment modeling: 1) models without explicit speech-text alignment modeling exhibit less robustness, especially for hard sentences in practical applications; 2) predefined alignment-based models suffer from naturalness constraints of forced alignments. This paper introduces \textit{MegaTTS 3}, a TTS system featuring an innovative sparse alignment algorithm that guides the latent diffusion transformer (DiT). Specifically, we provide sparse alignment boundaries to MegaTTS 3 to reduce the difficulty of alignment without limiting the search space, thereby achieving high naturalness. Moreover, we employ a multi-condition classifier-free guidance strategy for accent intensity adjustment and adopt the piecewise rectified flow technique to accelerate the generation process. Experiments demonstrate that MegaTTS 3 achieves state-of-the-art zero-shot TTS speech quality and supports highly flexible control over accent intensity. Notably, our system can generate high-quality one-minute speech with only 8 sampling steps. Audio samples are available at~\url{https://sditdemo.github.io/sditdemo/}.
\end{abstract}

% autoregressive large language models are inefficient and not robust in long-sentence inference; 
% The naturalness of predefined alignment-based models is also highly dependent on the performance of the duration model.

\section{Introduction}
Backdoor attacks pose a concealed yet profound security risk to machine learning (ML) models, for which the adversaries can inject a stealth backdoor into the model during training, enabling them to illicitly control the model's output upon encountering predefined inputs. These attacks can even occur without the knowledge of developers or end-users, thereby undermining the trust in ML systems. As ML becomes more deeply embedded in critical sectors like finance, healthcare, and autonomous driving \citep{he2016deep, liu2020computing, tournier2019mrtrix3, adjabi2020past}, the potential damage from backdoor attacks grows, underscoring the emergency for developing robust defense mechanisms against backdoor attacks.

To address the threat of backdoor attacks, researchers have developed a variety of strategies \cite{liu2018fine,wu2021adversarial,wang2019neural,zeng2022adversarial,zhu2023neural,Zhu_2023_ICCV, wei2024shared,wei2024d3}, aimed at purifying backdoors within victim models. These methods are designed to integrate with current deployment workflows seamlessly and have demonstrated significant success in mitigating the effects of backdoor triggers \cite{wubackdoorbench, wu2023defenses, wu2024backdoorbench,dunnett2024countering}.  However, most state-of-the-art (SOTA) backdoor purification methods operate under the assumption that a small clean dataset, often referred to as \textbf{auxiliary dataset}, is available for purification. Such an assumption poses practical challenges, especially in scenarios where data is scarce. To tackle this challenge, efforts have been made to reduce the size of the required auxiliary dataset~\cite{chai2022oneshot,li2023reconstructive, Zhu_2023_ICCV} and even explore dataset-free purification techniques~\cite{zheng2022data,hong2023revisiting,lin2024fusing}. Although these approaches offer some improvements, recent evaluations \cite{dunnett2024countering, wu2024backdoorbench} continue to highlight the importance of sufficient auxiliary data for achieving robust defenses against backdoor attacks.

While significant progress has been made in reducing the size of auxiliary datasets, an equally critical yet underexplored question remains: \emph{how does the nature of the auxiliary dataset affect purification effectiveness?} In  real-world  applications, auxiliary datasets can vary widely, encompassing in-distribution data, synthetic data, or external data from different sources. Understanding how each type of auxiliary dataset influences the purification effectiveness is vital for selecting or constructing the most suitable auxiliary dataset and the corresponding technique. For instance, when multiple datasets are available, understanding how different datasets contribute to purification can guide defenders in selecting or crafting the most appropriate dataset. Conversely, when only limited auxiliary data is accessible, knowing which purification technique works best under those constraints is critical. Therefore, there is an urgent need for a thorough investigation into the impact of auxiliary datasets on purification effectiveness to guide defenders in  enhancing the security of ML systems. 

In this paper, we systematically investigate the critical role of auxiliary datasets in backdoor purification, aiming to bridge the gap between idealized and practical purification scenarios.  Specifically, we first construct a diverse set of auxiliary datasets to emulate real-world conditions, as summarized in Table~\ref{overall}. These datasets include in-distribution data, synthetic data, and external data from other sources. Through an evaluation of SOTA backdoor purification methods across these datasets, we uncover several critical insights: \textbf{1)} In-distribution datasets, particularly those carefully filtered from the original training data of the victim model, effectively preserve the model’s utility for its intended tasks but may fall short in eliminating backdoors. \textbf{2)} Incorporating OOD datasets can help the model forget backdoors but also bring the risk of forgetting critical learned knowledge, significantly degrading its overall performance. Building on these findings, we propose Guided Input Calibration (GIC), a novel technique that enhances backdoor purification by adaptively transforming auxiliary data to better align with the victim model’s learned representations. By leveraging the victim model itself to guide this transformation, GIC optimizes the purification process, striking a balance between preserving model utility and mitigating backdoor threats. Extensive experiments demonstrate that GIC significantly improves the effectiveness of backdoor purification across diverse auxiliary datasets, providing a practical and robust defense solution.

Our main contributions are threefold:
\textbf{1) Impact analysis of auxiliary datasets:} We take the \textbf{first step}  in systematically investigating how different types of auxiliary datasets influence backdoor purification effectiveness. Our findings provide novel insights and serve as a foundation for future research on optimizing dataset selection and construction for enhanced backdoor defense.
%
\textbf{2) Compilation and evaluation of diverse auxiliary datasets:}  We have compiled and rigorously evaluated a diverse set of auxiliary datasets using SOTA purification methods, making our datasets and code publicly available to facilitate and support future research on practical backdoor defense strategies.
%
\textbf{3) Introduction of GIC:} We introduce GIC, the \textbf{first} dedicated solution designed to align auxiliary datasets with the model’s learned representations, significantly enhancing backdoor mitigation across various dataset types. Our approach sets a new benchmark for practical and effective backdoor defense.



\section{Background} \label{section:LLM}

% \subsection{Large Language Model (LLM)}   

Figure~\ref{fig:LLaMA_model}(a) shows that a decoder-only LLM initially processes a user prompt in the “prefill” stage and subsequently generates tokens sequentially during the “decoding” stage.
Both stages contain an input embedding layer, multiple decoder transformer blocks, an output embedding layer, and a sampling layer.
Figure~\ref{fig:LLaMA_model}(b) demonstrates that the decoder transformer blocks consist of a self attention and a feed-forward network (FFN) layer, each paired with residual connection and normalization layers. 

% Differentiate between encoder/decoder, explain why operation intensity is low, explain the different parts of a transformer block. Discuss Table II here. 

% Explain the architecture with Llama2-70B.

% \begin{table}[thb]
% \renewcommand\arraystretch{1.05}
% \centering
% % \vspace{-5mm}
%     \caption{ML Model Parameter Size and Operational Intensity}
%     \vspace{-2mm}
%     \small
%     \label{tab:ML Model Parameter Size and Operational Intensity}    
%     \scalebox{0.95}{
%         \begin{tabular}{|c|c|c|c|c|}
%             \hline
%             & Llama2 & BLOOM & BERT & ResNet \\
%             Model & (70B) & (176B) & & 152 \\
%             \hline
%             Parameter Size (GB) & 140 & 352 & 0.17 & 0.16 \\
%             \hline
%             Op Intensity (Ops/Byte) & 1 & 1 & 282 & 346 \\
%             \hline
%           \end{tabular}
%     }
% \vspace{-3mm}
% \end{table}

% {\fontsize{8pt}{11pt}\selectfont 8pt font size test Memory Requirement}

\begin{figure}[t]
    \centering
    \includegraphics[width=8cm]{Figure/LLaMA_model_new_new.pdf}
    \caption{(a) Prefill stage encodes prompt tokens in parallel. Decoding stage generates output tokens sequentially.
    (b) LLM contains N$\times$ decoder transformer blocks. 
    (c) Llama2 model architecture.}
    \label{fig:LLaMA_model}
\end{figure}

Figure~\ref{fig:LLaMA_model}(c) demonstrates the Llama2~\cite{touvron2023llama} model architecture as a representative LLM.
% The self attention layer requires three GEMVs\footnote{GEMVs in multi-head attention~\cite{attention}, narrow GEMMs in grouped-query attention~\cite{gqa}.} to generate query, key and value vectors.
In the self-attention layer, query, key and value vectors are generated by multiplying input vector to corresponding weight matrices.
These matrices are segmented into multiple heads, representing different semantic dimensions.
The query and key vectors go though Rotary Positional Embedding (RoPE) to encode the relative positional information~\cite{rope-paper}.
Within each head, the generated key and value vectors are appended to their caches.
The query vector is multiplied by the key cache to produce a score vector.
After the Softmax operation, the score vector is multiplied by the value cache to yield the output vector.
The output vectors from all heads are concatenated and multiplied by output weight matrix, resulting in a vector that undergoes residual connection and Root Mean Square layer Normalization (RMSNorm)~\cite{rmsnorm-paper}.
The residual connection adds up the input and output vectors of a layer to avoid vanishing gradient~\cite{he2016deep}.
The FFN layer begins with two parallel fully connections, followed by a Sigmoid Linear Unit (SiLU), and ends with another fully connection.
\section{Study Design}
% robot: aliengo 
% We used the Unitree AlienGo quadruped robot. 
% See Appendix 1 in AlienGo Software Guide PDF
% Weight = 25kg, size (L,W,H) = (0.55, 0.35, 06) m when standing, (0.55, 0.35, 0.31) m when walking
% Handle is 0.4 m or 0.5 m. I'll need to check it to see which type it is.
We gathered input from primary stakeholders of the robot dog guide, divided into three subgroups: BVI individuals who have owned a dog guide, BVI individuals who were not dog guide owners, and sighted individuals with generally low degrees of familiarity with dog guides. While the main focus of this study was on the BVI participants, we elected to include survey responses from sighted participants given the importance of social acceptance of the robot by the general public, which could reflect upon the BVI users themselves and affect their interactions with the general population \cite{kayukawa2022perceive}. 

The need-finding processes consisted of two stages. During Stage 1, we conducted in-depth interviews with BVI participants, querying their experiences in using conventional assistive technologies and dog guides. During Stage 2, a large-scale survey was distributed to both BVI and sighted participants. 

This study was approved by the University’s Institutional Review Board (IRB), and all processes were conducted after obtaining the participants' consent.

\subsection{Stage 1: Interviews}
We recruited nine BVI participants (\textbf{Table}~\ref{tab:bvi-info}) for in-depth interviews, which lasted 45-90 minutes for current or former dog guide owners (DO) and 30-60 minutes for participants without dog guides (NDO). Group DO consisted of five participants, while Group NDO consisted of four participants.
% The interview participants were divided into two groups. Group DO (Dog guide Owner) consisted of five participants who were current or former dog guide owners and Group NDO (Non Dog guide Owner) consisted of three participants who were not dog guide owners. 
All participants were familiar with using white canes as a mobility aid. 

We recruited participants in both groups, DO and NDO, to gather data from those with substantial experience with dog guides, offering potentially more practical insights, and from those without prior experience, providing a perspective that may be less constrained and more open to novel approaches. 

We asked about the participants' overall impressions of a robot dog guide, expectations regarding its potential benefits and challenges compared to a conventional dog guide, their desired methods of giving commands and communicating with the robot dog guide, essential functionalities that the robot dog guide should offer, and their preferences for various aspects of the robot dog guide's form factors. 
For Group DO, we also included questions that asked about the participants' experiences with conventional dog guides. 

% We obtained permission to record the conversations for our records while simultaneously taking notes during the interviews. The interviews lasted 30-60 minutes for NDO participants and 45-90 minutes for DO participants. 

\subsection{Stage 2: Large-Scale Surveys} 
After gathering sufficient initial results from the interviews, we created an online survey for distributing to a larger pool of participants. The survey platform used was Qualtrics. 

\subsubsection{Survey Participants}
The survey had 100 participants divided into two primary groups. Group BVI consisted of 42 blind or visually impaired participants, and Group ST consisted of 58 sighted participants. \textbf{Table}~\ref{tab:survey-demographics} shows the demographic information of the survey participants. 

\subsubsection{Question Differentiation} 
Based on their responses to initial qualifying questions, survey participants were sorted into three subgroups: DO, NDO, and ST. Each participant was assigned one of three different versions of the survey. The surveys for BVI participants mirrored the interview categories (overall impressions, communication methods, functionalities, and form factors), but with a more quantitative approach rather than the open-ended questions used in interviews. The DO version included additional questions pertaining to their prior experience with dog guides. The ST version revolved around the participants' prior interactions with and feelings toward dog guides and dogs in general, their thoughts on a robot dog guide, and broad opinions on the aesthetic component of the robot's design. 

\section{Experiments}
\label{sec:exp}
\subsection{Experimental settings}

\noindent\textbf{Benchmark.}  We conduct experiments on two established 3D occupancy benchmarks: (i) nuScenes~\cite{nuScenes}, which provides instance-level annotations with manually labeled 3D bounding boxes (position/size/orientation) for dynamic objects, and (ii) Occ3D~\cite{Occ3D}, which generates voxel-level occupancy labels (0.4m resolution) through automated LiDAR point cloud aggregation and mesh reconstruction, including occlusion states. Both benchmarks share identical scene configurations of 1,050 driving scenes, each containing up to 40 timestamped frames. Every frame includes six synchronized camera views (front, front-left, front-right, back, back-left, back-right) at 1600$\times$900 resolution. In our experiments, we extend single-frame baselines~\cite{MonoScene,surroundOcc,viewformer} by aggregating features from $N$ historical keyframes. Additionally, we extract unlabeled intermediate frames from the ``sweeps'' folder~\cite{nuScenes} to provide implicit motion cues, enabling self-supervised temporal consistency learning.

\noindent\textbf{Implementation details.} For the nuScenes benchmark~\cite{nuScenes}, we follow the parameter settings of SurroundOcc~\cite{surroundOcc}, using $Cam=6$, $p=6$, $v=32$,$L=116$, and $W=200$. For the Occ3D benchmark~\cite{Occ3D}, we adopt ViewFormer's~\cite{viewformer} standard setup with $Cam=6$, $p=6$, $v=32$, $L=32$, and $W=88$. The output of the occupancy result on both benchmarks is formatted into a vector with dimensions $[200, 200, 16]$. In this vector, the first two dimensions (200 and 200) represent the length and width, while the third (16) indicates the height. The occupancy result covers a range from -50 meters to 50 meters in both width and length, and the vertical height varies from -5 meters to 3 meters. Each voxel corresponds to a cube measuring 0.5 meters on each side. Occupied voxels are categorized into one of 17~\cite{nuScenes,surroundOcc} and 18~\cite{Occ3D} semantic classes.
More details on implementation can be found in the supplementary material.

\subsection{Evaluation Metrics}

To validate the temporal consistency and occupancy accuracy of moving and static objects, objects are divided into two general classes~\cite{Cam4docc}: General Moving Objects (GMO) and General Static Objects (GSO). Detailed classification classes are introduced in the supplementary material.

\noindent\textbf{Occupancy Accuracy Metric.} To ensure rigorous evaluation across different benchmarks, we employ both Intersection over Union (IoU) and Mean Intersection over Union (mIoU) metrics. These metrics are widely adopted in 3D semantic occupancy prediction tasks~\cite{PASCAL, Microsoft_COCO, Cityscapes_dataset, Mask_R_CNN}. The mIoU are calculated separately for three category groups: All classes, GMO classes, and GSO classes.

\noindent\textbf{Temporal Consistency Metric.} \label{para:consistency_metric} To evaluate the effect achieved by integrating \ours\ with baseline models, we propose a temporal consistency metric. We aim to detect and measure changes in a scene from one frame to the next. This metric reflects the stability of prediction results, which directly impacts the user's visual experience. Let $\sigma_{i,n}^{(x,y,z)}$ denote the semantic label of the $n$-th voxel point (with coordinates $(x,y,z)$) in frame $i$, and define the indicator function $\delta(e_1,e_2) = \mathbb{I}(e_1 \neq e_2)$. 

In the occupancy results of frames $i$ and $j$, voxels at corresponding positions may undergo changes, which are categorized into two types: ``Static Object Change"~(SOC) and ``Moving Object Change"~(MOC). The definitions of these changes are as table \ref{tab:moc-soc}.

\begin{wrapfigure}[7]{l}{80mm}
\centering
% \setlength{\tabcolsep}{8pt}
\captionsetup{type=table}
    \begin{tabular}{c|c}
    \toprule
    \textbf{Type} & \textbf{Condition} \\  
    \midrule
    MOC & $\sigma_{i,n}^{(x,y,z)} \in \text{GMO} \lor \sigma_{j,n}^{(x,y,z)} \in \text{GMO}$ \\
    % \midrule
    SOC & $\sigma_{i,n}^{(x,y,z)} \wedge \sigma_{j,n}^{(x,y,z)} \in \text{GSO}$ \\  
    \bottomrule
    \end{tabular}
    \vspace{8pt}
    \caption{Definition of MOC and SOC. $N_{mc}$/$N_{sc}$ denote the number of MOC/SOC voxels, respectively.}
    \label{tab:moc-soc}
    % \vspace{-10pt}
\end{wrapfigure}

Based on these definitions, we can define disparity metrics~($\Delta_{m}$/$\Delta_{s}$) to quantify temporal inconsistencies across frames~($i$ and $j$). The process is defined as:
\begin{equation}
\begin{dcases}
\Delta_{m}(i,j) = \dfrac{1}{N_{mc}} \sum\limits_{n=1}^{N_{mc}} \delta\left(\sigma_{i,n}^{(x,y,z)}, \sigma_{j,n}^{(x,y,z)}\right) \\
\Delta_{s}(i,j) = \dfrac{1}{N_{sc}} \sum\limits_{n=1}^{N_{sc}} \delta\left(\sigma_{i,n}^{(x,y,z)}, \sigma_{j,n}^{(x,y,z)}\right).
\end{dcases}
\label{eq:disparity}
\end{equation}

The temporal consistency metrics -- $S_m$ (moving) and $S_s$ (static) -- are derived through aggregation of $\Delta_{m}$ and $\Delta_{s}$ across sequential frames. Formally, we have:
\begin{equation}
S_{m/s} = 1 - \dfrac{1}{M-1} \sum\limits_{k=1}^{M-1} \Delta_{m/s}(k,k+1),
\label{eq:consistency_scores}
\end{equation}
where $M$ is the scene's total frame count. Final metrics $\overline{S_m}$/$\overline{S_s}$ average across all scenes. A higher temporal consistency score indicates that the predictions within the scene are smoother and more consistent over time.

\begin{table}[t]
    % \centering
    \footnotesize 
    \begin{minipage}[t]{0.48\textwidth}
        \centering
        % \captionsetup{justification=centering, singlelinecheck=false}
        \setlength{\tabcolsep}{2pt}
        \renewcommand{\arraystretch}{1.25}
        \begin{tabular}{r|cccc|cc}
            \toprule
            \multicolumn{1}{r|}{\multirow{2}{*}[-0.4em]{Method}} & \multicolumn{1}{c|}{\multirow{2}{*}[-0.4em]{IoU~$\uparrow$}} & \multicolumn{3}{c|}{mIoU~$\uparrow$} & \multicolumn{1}{c}{\multirow{2}{*}[-0.4em]{$\overline{S_m}\uparrow$}} & \multicolumn{1}{c}{\multirow{2}{*}[-0.4em]{$\overline{S_s}\uparrow$}} \\ \cmidrule(lr){3-5}
            \multicolumn{1}{c|}{} & \multicolumn{1}{c|}{} & All & GMO & GSO & \multicolumn{1}{c}{} & \multicolumn{1}{c}{} \\ 
            \midrule        
            Atlas~\cite{Atlas} & 28.66 & 15.00 & 12.64 & 17.35 & \text{--} & \text{--}  \\
            BEVFormer{~\cite{BEVFormer}} & 30.50 & 16.75 & 14.17 & 19.33 & \text{--} & \text{--}  \\
            TPVFormer~\cite{TPVFormer} & 30.86 & 17.10 & 14.04 & 20.15 & \text{--} & \text{--} \\
            BEVDet4D-Occ~\cite{bevdet4d} & 24.26 & 14.22 & 11.10 & 17.34 & \text{--} & \text{--} \\
            MonoScene~\cite{MonoScene} & 10.04 & 1.15 & 0.24 & 2.07 & 46.53 & 81.77 \\
            % Cam4DOcc~\cite{Cam4docc} & 23.92 & 7.12 & 4.71 & 10.17 & 60.34 & 91.15 \\
            SurroundOcc~\cite{surroundOcc} & 31.49 & 20.30  & \cellcolor{gray!20}18.39 & 22.20 & 58.33 & 91.71 \\
            \midrule
            \makecell[r]{MonoScene \\ \textbf{+\ours}} & \makecell{13.10\\\textbf{+3.06}} & \makecell{1.69\\\textbf{+0.54}}  & \makecell{0.34\\\textbf{+0.10}} & \makecell{3.04\\\textbf{+0.98}} & \makecell{54.21\\\textbf{+7.68}} & \makecell{83.84\\\textbf{+2.07}} \\
            \midrule
            \makecell[r]{SurroundOcc \\ \textbf{+\ours}} & \cellcolor{gray!20}\makecell{33.12 \\ \textbf{+1.63}} & \cellcolor{gray!20}\makecell{20.67\\\textbf{+0.37}}  & \makecell{18.26\\-0.13} & \cellcolor{gray!20}\makecell{23.08\\\textbf{+0.88}} & \cellcolor{gray!20}\makecell{60.64\\\textbf{+2.31}} & \cellcolor{gray!20}\makecell{92.54\\\textbf{+0.83}} \\
            \bottomrule
        \end{tabular}
        \vspace{2mm}
        \caption{Occupancy prediction accuracy on \textbf{nuScenes benchmark~\cite{nuScenes}}. For a fair comparison, we ensure that all models have uniform input data. The best performance is highlighted in gray.}
        \label{tab:main-res-a}
    \end{minipage}\hfill
    \begin{minipage}[t]{0.48\textwidth}
        \centering
        % \captionsetup{justification=centering, singlelinecheck=false}
        \setlength{\tabcolsep}{2pt}
        \begin{tabular}{r|cccc|cc}
            \toprule
            \multicolumn{1}{r|}{\multirow{2}{*}[-0.4em]{Method}} & \multicolumn{1}{c|}{\multirow{2}{*}[-0.4em]{IoU~$\uparrow$}} & \multicolumn{3}{c|}{mIoU~$\uparrow$} & \multicolumn{1}{c}{\multirow{2}{*}[-0.4em]{$\overline{S_m}\uparrow$}} & \multicolumn{1}{c}{\multirow{2}{*}[-0.4em]{$\overline{S_s}\uparrow$}} \\ \cmidrule(lr){3-5}
            \multicolumn{1}{c|}{} & \multicolumn{1}{c|}{} & All & GMO & GSO & \multicolumn{1}{c}{} & \multicolumn{1}{c}{} \\ 
            \midrule    
            MonoScene~\cite{MonoScene} & \text{--} & 6.06 & 5.36 & 6.68 & \text{--} & \text{--} \\
            OccFormer~\cite{OccFormer} & \text{--} & 21.93 & 21.78 & 22.06 & \text{--} & \text{--} \\
            % CTF-Occ~\cite{Occ3D} & 28.53 & 27.42 & 29.52 & \text{--} & \text{--} \\
            FB-OCC~\cite{fb_occ} & \text{--} & 39.11  & 33.74 & 43.88 & \text{--} & \text{--} \\
            SparseOcc~\cite{SparseOcc_Liu} & \text{--} & 30.10  & \text{--} & \text{--} & \text{--} & \text{--} \\
            BEVDet4D-Occ~\cite{bevdet4d} & \text{--} & 39.30  & 29.09 & 42.16 & \text{--} & \text{--} \\ 
            OPUS-L~\cite{opus} & \text{--} & 36.20  & 31.25 & 40.44 & \text{--} & \text{--} \\      
            SurroundOcc~\cite{surroundOcc} & 51.89 & 7.24  & 0.36 & 13.35 & 65.35 & 89.54 \\
            ViewFormer~\cite{viewformer} & 70.39 & 40.46  & 33.73 & 46.45 & 67.26 & 86.06 \\
            \midrule
            \makecell[r]{SurroundOcc \\ \textbf{+\ours}} & \makecell{52.13\\\textbf{+0.24}}& \makecell{10.33\\\textbf{+3.09}}  & \makecell{1.98\\\textbf{+1.62}} & \makecell{17.76\\\textbf{+4.41}} & \makecell{69.60\\\textbf{+4.25}} & \cellcolor{gray!20}\makecell{90.91\\\textbf{+1.37}} \\
            \midrule
            \makecell[r]{ViewFormer \\ \textbf{+\ours}} & \cellcolor{gray!20}\makecell{70.63\\\textbf{+0.24}} & \cellcolor{gray!20}\makecell{41.30\\\textbf{+0.84}}  & \cellcolor{gray!20}\makecell{34.33\\\textbf{+0.60}} & \cellcolor{gray!20}\makecell{47.50\\\textbf{+1.05}} & \cellcolor{gray!20}\makecell{70.13\\\textbf{+2.87}} & \makecell{87.10\\\textbf{+1.04}} \\
            \bottomrule
        \end{tabular}
        \vspace{2mm}
        \caption{Occupancy prediction accuracy on \textbf{Occ3D benchmark~\cite{Occ3D}}. For a fair comparison, we ensure that all models have uniform input data. The best performance is highlighted in gray.}
        \label{tab:main-res-b}
    \end{minipage}
    \vspace{-3mm}
    % \caption{Occupancy prediction accuracy on two benchmarks. The best performance is highlighted in gray.}
    \label{tab:main-res}
    \vspace{-5mm}
\end{table}

\subsection{Comparison Results}

\noindent\textbf{Occupancy accuracy on nuScenes.} We compare our method against several SOTA models, including Atlas~\cite{Atlas}, BEVFormer~\cite{BEVFormer}, TPVFormer~\cite{TPVFormer}, MonoScene~\cite{MonoScene}, and SurroundOcc~\cite{surroundOcc}. For a fair comparison, all methods are trained on the same ground truth and follow the same training procedure. By combining methods such as MonoScene~\cite{MonoScene} and SurroundOcc~\cite{surroundOcc} with \ours, we evaluate the effect of \ours\ in performance enhancement. The results presented in \cref{tab:main-res-a} show that our performance improvement is significant. Notably, the incorporation of \ours\ into SurroundOcc~\cite{surroundOcc} has led to improved metrics that surpass those of all other models listed in this table. The results are improved by 1.63\% and 0.37\% compared with SurroundOcc~\cite{surroundOcc} in IoU and mIoU~(All), respectively.

\noindent\textbf{Occupancy accuracy on Occ3D.} We also conduct experiments on Occ3D~\cite{Occ3D} in \cref{tab:main-res-b}. To validate \ours, we conducted two sets of experiments: First, integrating \ours\ with the 3D VONs~\cite{surroundOcc,viewformer} improved one of the original models'~\cite{viewformer} performance by 0.24\% in IoU and 0.84\% in mIoU. Second, \ours\ consistently outperforms existing history-aware VONs~\cite{opus,bevdet,SparseOcc_Liu,fb_occ} by over 2\% mIoU, demonstrating the efficacy of the \ours.


\noindent\textbf{Temporal Consistency.} The results of $\overline{S_m}$ and $\overline{S_s}$ shown in \cref{tab:main-res} indicate that the integration of \ours\ improved the temporal consistency of occupancy across all frames in all scenes for all models, demonstrating \ours' effectiveness. This enhancement can be attributed to the incorporation of previous keyframes from the dataset~\cite{nuScenes,Occ3D}, along with the addition of intermediate frames from the ``sweeps''~\cite{nuScenes} directory for the SFE and MFE modules. These elements provide critical historical information and motion clues for the model.

\subsection{Ablation study}

Our ablation experiments are all conducted on the nuScenes benchmark~\cite{nuScenes}. The results are presented in~\cref{tab:ablation-studies}.

\begin{wrapfigure}{l}{90mm}
\centering
\captionsetup{type=table}
    \begin{subtable}[t]{0.48\textwidth}
    \centering
    \footnotesize
    \begin{tabular}{c|ccc|cccc}
    \toprule 
    Idx. & Pre & Cur & Mid & IoU$\uparrow$ & mIoU$\uparrow$ & $\overline{S_m}\uparrow$ & $\overline{S_s}\uparrow$ \\
    \midrule
    \textbf{M0} & \ding{55} & \ding{55} & \ding{55} & 31.49 & 20.30 & 58.33 & 91.71 \\
    \textbf{M1} & \ding{55} & \ding{51} & \ding{51} & 33.04 & 20.04 & 60.59 & 92.25 \\
    \textbf{M2} & \ding{51} & \ding{55} & \ding{51} & 33.05 & 19.98 & 60.09 & 92.44 \\
    \textbf{M3} & \ding{51} & \ding{51} & \ding{55} & 32.88 & 20.10 & 60.24 & 92.24 \\
    \textbf{M4} & \ding{51} & \ding{51} & \ding{51} & 31.97 & 20.11 & 60.19 & 92.01 \\
    \textbf{M5} & \ding{51} & \ding{51} & \ding{51} & \textbf{33.12} & \textbf{20.67} & \textbf{60.64} & \textbf{92.54} \\
    \bottomrule
    \end{tabular}
    \vspace{1mm}
    \caption{Ablation study of \ours. \textbf{Cur}, \textbf{Pre} and \textbf{Mid} represent the $f_{cur}$, $f_{pre}$ and $f_{mid}$ input, respectively, in the MSI.}
    \label{tab:ablation-modules}
    \end{subtable}
% \hfill
    \vspace{2mm}
    
    \begin{subtable}[t]{0.48\textwidth}
    \centering
    \footnotesize
    \setlength{\tabcolsep}{7pt}
    \begin{tabular}{c|c|cccc}
    \toprule 
    Idx. & \makecell{Type of\\motion info.} & IoU~$\uparrow$ & mIoU~$\uparrow$ & $\overline{S_m}\uparrow$ & $\overline{S_s}\uparrow$ \\
    \midrule
    \textbf{I0} & -  & 31.49 & 20.30 & 58.33 & 91.71 \\
    \textbf{I1} & Raw Image  & 32.39 & 19.45 & 59.01 & 91.15 \\
    \textbf{I2} & Optical Flow  & 32.80 & 20.27 & 60.53 & 92.13 \\
    \textbf{I3} & Frame Diff.  & \textbf{33.12} & \textbf{20.67}  & \textbf{60.64} & \textbf{92.54} \\
    \bottomrule
    \end{tabular}
    \vspace{1mm}
    \caption{Effect of different types of motion information.}
    \label{tab:ablation-motion}
    \end{subtable}
    % \vspace{-2mm}
\caption{Ablation studies on \ours\ modules and motion information. Best results are \textbf{bolded}.}
\label{tab:ablation-studies}
\end{wrapfigure}

\noindent\textbf{Different combinations of \ours.} \label{para:aba-comb}\cref{tab:ablation-modules} presents the performance results of different combination of \ours's components for $N$=$1$. In \cref{tab:ablation-modules}, there are 6 different combinations: \textbf{M0} shows results from SurroundOcc~\cite{surroundOcc}, which represents the basic model without our method. \textbf{M1} means the model variant in which the part responsible for processing previous keyframes is removed, thereby excluding the input data \(F_{pre}^{1}\). \textbf{M2} refers to the model variant that omits the current feature \(F_{cur}\). \textbf{M3} indicates the model configuration that has \(F_{mid}^{1}\) removed. \textbf{M4} indicates that $I_{cur}$ is used to compute MHAM's query, while $I_{pre}$ and $I_{mid}$ are utilized to compute MHAM's key and value, which differs from the standard design. \textbf{M5} represents the full model with all components included.
% 验证了我们三种输入的必要性
\cref{tab:ablation-modules} clearly demonstrates that the removal of any single input from \ours\ module significantly reduces performance both in prediction accuracy and in temporal consistency. This validates the necessity of the three inputs. Furthermore, the comparison between \textbf{M4} and \textbf{M5} confirms that the cues provided by the previous keyframes and the intermediate frames are crucial for occupancy prediction.

\noindent\textbf{Impact of different types of motion information.} \label{para:motion-extracting} This experiment was conducted on the MFE module to investigate the effects of various types of motion information for $N=1$. The results are presented in \cref{tab:ablation-motion}. Specifically, \textbf{I0} served as the base model~\cite{surroundOcc} without using any motion information. \textbf{I1} employed raw intermediate frames as the input for the MFE. \textbf{I2} used optical flow~\cite{OpticalFlow} as the motion information input. \textbf{I3} used frame difference~\cite{Frame_difference} to capture motion information. It is clear that \textbf{I1} surpasses \textbf{I0} in terms of IoU metrics; however, it exhibits the lowest performance in mIoU, $\overline{S_m}$, and $\overline{S_s}$ metrics compared with \textbf{I1}, \textbf{I2}, and \textbf{I3}. This discrepancy is mainly because of the substantial amount of irrelevant information in the raw, intermediate frames, which complicates the extraction of motion features by the MFE. In addition, the results show that \textbf{I3} significantly outperforms \textbf{I2} in both IoU and mIoU metrics and slightly improves in $\overline{S_m}$ and $\overline{S_s}$ metrics. This indicates that frame difference more effectively captures sudden changes in a scene, such as the abrupt appearance of pedestrians or vehicles exiting intersections, while optical flow may experience delays in processing these sudden events. Furthermore, given the lightweight design of \ours, the frame difference method~\cite{Frame_difference} reduces data processing complexity by only processing simple differential data, thereby contributing to computing speed.

\begin{wrapfigure}[22]{l}{90mm}
    \centering
    \includegraphics[width=90mm]{assets/case_single_light.png}
    \caption{
    Several challenging scenarios are presented: pedestrians are partially occluded by vehicles in the first and second columns, and the road boundary appears visually obscure in the third column. Ours achieves more accurate predictions, while SOTA methods display significant artifacts.
    }
    \label{fig:case-extra}
\end{wrapfigure}

\noindent\textbf{Impact of different numbers of previous keyframes.}\label{para:ntrack} We conduct ablation experiments on $N$ to explore the performance of the model when $N$=$0$, $N$=$1$ and $N$=$2$. $N$=$0$ represents SurroundOcc~\cite{surroundOcc}, which does not use any previous keyframes. Detailed experiment results are documented in the supplementary material.

\begin{figure}[!t]
\centering
% \fbox{\rule{0pt}{2in} \rule{0.9\linewidth}{0pt}}
\includegraphics[width=\linewidth]{assets/case_big_light.png}
% \vspace{-7mm}
\caption{
Comparison under a T-junction scenario, where a pedestrian is partially and dynamically occluded in certain frames. Ours showcases robust predictions, with the pedestrian being consistently tracked, while SOTA methods show a flickering phenomenon.
}
\label{fig:case-study-big}
\vspace{-20pt}
\end{figure}

\subsection{Case analysis}
To visually evaluate the effectiveness of our method~(SurroundOcc+\ours), we compare it with the SOTA 3D VONs~\cite{surroundOcc} and the SOTA history-aware VONs~\cite{bevdet4d}.

\noindent\textbf{Temporal visualization case.} As shown in~\cref{fig:case-study-big} (Scene 277, Frames \#7-\#11), a pedestrian traversing the sidewalk parallel to the ego-motion trajectory is intermittently occluded by roadside vegetation. SurroundOcc~\cite{surroundOcc} exhibits severe instability in predictions (missing in Frames \#7/\#9), revealing fundamental limitations in temporal modeling. BEVDet4D-Occ~\cite{bevdet4d} alleviates this issue through data fusion but still suffers from occasional inconsistencies, such as detection dropout in Frame \#8. In contrast, our method completely eliminates flickering artifacts and maintains consistent detection across all occlusion states.

\noindent\textbf{Extra single frame visualization case.} ~\cref{fig:case-extra} highlights challenging scenarios: 
(i) Vehicle-pedestrian occlusion (Scene-0911 Frame \#15, Scene-0928 Frame \#14): Both SurroundOcc~\cite{surroundOcc} and BEVDet4D-Occ~\cite{bevdet4d} fail to recover the occluded pedestrian’s occupancy, while our method successfully localizes the target with precise geometry.
(ii) Curved road prediction (Scene-0923 Frame \#28): Our approach correctly anticipates the right-turn road geometry where baselines produce fragmented or erroneous occupancy, achieving superior shape consistency with real-world conditions.




\subsection{Overhead analysis}

For a fair comparison, all overhead analysis experiments are performed on a single NVIDIA L20 GPU.

\begin{figure}[htbp]
    \centering
    \begin{minipage}{0.48\textwidth}
        \footnotesize
        \begin{tabular}{r|ccc}
            \toprule
            Model & mIoU~$\uparrow$ & \makecell{Memory (MB)\\ Train~/~Test}~$\downarrow$ & Latency~$\downarrow$ \\
            \midrule
            FB-Occ~\cite{fb_occ} & 39.11 & 32,915~/~5,933 & 0.09s \\
            % SparseOcc~\cite{SparseOcc_Liu} & 30.10 & $>$49,140~/~7,147 & \cellcolor{Gray} 0.05s \\
            OPUS-L~\cite{opus} & 36.20 & OOM~/~10,579 & 0.16s \\
            OPUS-T~\cite{opus} & 33.20 & 48,532~/~6,711 & \textbf{0.03s} \\
            BEVDet4D-Occ~\cite{bevdet4d} & 39.30 & 22,833~/~4,689 & 0.26s \\
            \midrule
            ViewFormer+Ours & \textbf{41.30} & \textbf{16,619~/~4,687} & 0.12s \\
            \bottomrule
        \end{tabular}
        \vspace{2mm}
        \caption{Comparison of computational overhead. All models are benchmarked with ResNet-50 backbones. Our result (ViewFormer+\ours) in this table is measured for $N = 1$. OOM indicates out of CUDA memory. Best results are \textbf{bolded}.}
        \label{tab:efficiency}
    \end{minipage}\hfill
    \begin{minipage}{0.48\textwidth}
        \centering
        \includegraphics[width=\linewidth]{assets/bubble_0304.png}  % 替换成你的图片
        % \vspace{-8mm}
        \caption{Comparison of memory and latency overheads. Lower-left positions indicate superior performance with reduced memory consumption and faster inference. Large circles indicate better mIoU quality.}
        \label{fig:bubble}
    \end{minipage}
\end{figure}



As illustrated in \cref{tab:efficiency} and \cref{fig:bubble}, we conducted a comparative study to evaluate the computational overhead of our model against existing temporal methods~\cite{bevdet4d,opus,fb_occ}. The analysis focuses on GPU memory consumption during the training/testing phases and per-sample inference latency. The result shows that our method establishes an optimal accuracy-memory balance, achieving state-of-the-art mIoU while maintaining minimal GPU memory consumption alongside sustained computational efficiency that avoids runtime bottlenecks. For quantitative benchmarking, we compare two baseline frameworks:
\begin{itemize}
    \item ViewFormer on Occ3D: (i) Training memory: ViewFormer+\ours\ requires 16 GB of GPU memory, with the \ours\ module consuming only 0.22 GB, accounting for \textbf{1.4\%} of total usage; (ii) Inference latency: Full sample processing takes 0.1218s, where \ours\ contributes merely 0.0043s, accounting for \textbf{3.5\%} of total computation.
    \item SurroundOcc on nuScenes: (i) Training memory: SurroundOcc+\ours\ consumes 39 GB of GPU memory, with \ours\ occupying only 0.69 GB, which is \textbf{1.8\%} of total memory; (ii) Inference latency: Complete sample inference requires 0.9200s, while \ours\ takes 0.0065s, contributing to \textbf{0.7\%} of total latency.
\end{itemize}

These measurements confirm that our architecture introduces negligible computational overhead while delivering competitive performance.
\section{Conclusion}
We introduce a novel approach, \algo, to reduce human feedback requirements in preference-based reinforcement learning by leveraging vision-language models. While VLMs encode rich world knowledge, their direct application as reward models is hindered by alignment issues and noisy predictions. To address this, we develop a synergistic framework where limited human feedback is used to adapt VLMs, improving their reliability in preference labeling. Further, we incorporate a selective sampling strategy to mitigate noise and prioritize informative human annotations.

Our experiments demonstrate that this method significantly improves feedback efficiency, achieving comparable or superior task performance with up to 50\% fewer human annotations. Moreover, we show that an adapted VLM can generalize across similar tasks, further reducing the need for new human feedback by 75\%. These results highlight the potential of integrating VLMs into preference-based RL, offering a scalable solution to reducing human supervision while maintaining high task success rates. 

\section*{Impact Statement}
This work advances embodied AI by significantly reducing the human feedback required for training agents. This reduction is particularly valuable in robotic applications where obtaining human demonstrations and feedback is challenging or impractical, such as assistive robotic arms for individuals with mobility impairments. By minimizing the feedback requirements, our approach enables users to more efficiently customize and teach new skills to robotic agents based on their specific needs and preferences. The broader impact of this work extends to healthcare, assistive technology, and human-robot interaction. One possible risk is that the bias from human feedback can propagate to the VLM and subsequently to the policy. This can be mitigated by personalization of agents in case of household application or standardization of feedback for industrial applications. 


\section*{Limitations}
\label{app:limitation_and_future_work}
In this section, we discuss the limitations of the proposed method and outline potential strategies for addressing them in future research. 
\begin{itemize}
\item \textbf{Language Coverage.} Although our model currently supports both English and Chinese, there are far more languages in the world. We plan to incorporate additional training data from a wider range of languages and apply adaptation-based techniques, such as LoRA tuning~\citep{hu2021lora}, to enhance speech quality for low-resource languages.
\item \textbf{Function Coverage.} We can make MegaTTS 3 more user-friendly by enabling it to generate speech in various styles according to text descriptions through instruction-based fine-tuning. We can further fine-tune MegaTTS 3 on the paralinguistic corpus, allowing it to generate speech that is closer to a natural human style.
\end{itemize}


% Bibliography entries for the entire Anthology, followed by custom entries
%\bibliography{anthology,custom}
% Custom bibliography entries only
\bibliography{custom}

\clearpage
\renewcommand{\thefigure}{A\arabic{figure}}
\renewcommand{\thetable}{A\arabic{table}}
\renewcommand{\theequation}{A\arabic{equation}}
\setcounter{figure}{0}
\setcounter{table}{0}
\setcounter{equation}{0}

Our Appendix is organized as follows. First, we present the pseudocode for the key components of iGCT. We also include the proof for unit variance and boundary conditions in preconditioning iGCT's noiser. Next, we detail the training setups for our CIFAR-10 and ImageNet64 experiments. Additionally, we provide ablation studies on using guided synthesized images as data augmentation in image classification. Finally, we present more uncurated results comparing iGCT and CFG-EDM on inversion, editing and guidance, thoroughly of iGCT.

\vspace{-0.2cm}
\label{appendix:iGCT}
\section{Pseudocode for iGCT}
\vspace{-0.2cm}

iGCT is trained under a continuous-time scheduler similar to the one proposed by ECT \cite{ect}. Our noise sampling function follows a lognormal distribution, \(p(t) = \textit{LogNormal}(P_\textit{mean}, P_\textit{std})\), with \(P_\textit{mean}=-1.1\) and \( P_\textit{std}=2.0\). At training, the sampled noise is clamped at \(t_\text{min} = 0.002\) and \(t_\text{max} = 80.0\). Step function \(\Delta t (t)=\frac{t}{2^{\left\lfloor k/d \right\rfloor}}n(t)\), is used to compute the step size from the sampled noise \(t\), with \(k,d\) being the current training iteration and the number of iterations for halfing \(\Delta t\), and \(n(t) = 1 + 8 \sigma(-t)\) is a sigmoid adjusting function. 

In Guided Consistency Training, the guidance mask function determines whether the sampled noise \( t \) should be supervised for guidance training. With probability \( q(t) \in [0,1] \), the update is directed towards the target sample \( \boldsymbol{x}_0^{\text{tar}} \); otherwise, no guidance is applied. In practice, \( q(t) \) is higher in noisier regions and zero in low-noise regions, 
\begin{equation}
    q(t) = 0.9 \cdot \left( \text{clamp} \left( \frac{t - t_{\text{low}}}{t_{\text{high}} - t_{\text{low}}}, 0, 1 \right) \right)^2,
\end{equation}
where \( t_{\text{low}} = 11.0 \) and \( t_{\text{high}} = 14.3 \). For the range of guidance strength, we set \(w_\text{min} = 1\) and \(w_\text{max} = 15\). Guidance strengths are sampled uniformly at training, with \(w_\text{min} = 1\) means no guidance applied. 


\begin{algorithm}
\caption{Guided Consistency Training}
\label{alg:GCT}
\begin{algorithmic}[1]  % Adds line numbers
\setlength{\baselineskip}{0.9\baselineskip} % Adjust line spacing
\INPUT Dataset $\mathcal{D}$, weighting function $\lambda(t)$, noise sampling function $p(t)$, noise range $[t_\text{min}, t_\text{max}]$, step function $\Delta t(t)$, guidance mask function $q(t)$, guidance range $[w_\text{min}, w_\text{max}]$, denoiser $D_\theta$
\STATE \rule{0.96\textwidth}{0.45pt} 
\STATE Sample $(\boldsymbol{x}_0^{\text{src}}, c^{\text{src}}), (\boldsymbol{x}_0^{\text{tar}}, c^{\text{tar}}) \sim \mathcal{D}$ 
\STATE Sample noise $\boldsymbol{z} \sim \mathcal{N}(\boldsymbol{0},\mathbf{I})$, time step $t \sim p(t)$, and guidance weight $w \sim \mathcal{U}(w_\text{min}, w_\text{max})$
\STATE Clamp $t \leftarrow \text{clamp}(t,t_\text{min}, t_\text{max})$
\STATE Compute noisy sample: $\boldsymbol{x}_t = \boldsymbol{x}_0^{\text{src}} + t\boldsymbol{z}$
\STATE Sample $\rho \sim \mathcal{U}(0,1)$  
\vspace{0.3em}
\IF{$\rho > q(t)$}
    \STATE Compute step as normal CT: $\boldsymbol{x}_r = \boldsymbol{x}_t - \Delta t(t) \boldsymbol{z}$
    \STATE Set target class: $c \leftarrow c^{\text{src}}$
\ELSE
    \STATE Compute guided noise: $\boldsymbol{z}^* = (\boldsymbol{x}_t - \boldsymbol{x}_0^{\text{tar}}) / t$
    \STATE Compute guided step: $\boldsymbol{x}_r = \boldsymbol{x}_t - \Delta t(t) [w \boldsymbol{z}^* + (1-w)\boldsymbol{z}]$
    \STATE Set target class: $c \leftarrow c^{\text{tar}}$
\ENDIF
\vspace{0.3em} % Reduces extra vertical space before the loss line
\STATE Compute loss: 
\[
\mathcal{L}_\text{gct} = \lambda(t) \, d(D_{\theta}(\boldsymbol{x}_t, t, c, w), D_{{\theta}^-}(\boldsymbol{x}_r, r, c, w))
\]
\STATE Return $\mathcal{L}_\text{gct}$ 
\end{algorithmic}
\end{algorithm}



A \textit{noiser} trained under \textit{Inverse Consistency Training} maps an image to its latent noise in a single step. In contrast, DDIM Inversion requires multiple steps with a diffusion model to accurately produce an image's latent representation. Since the boundary signal is reversed, spreading from \( t_\text{max} \) down to \( t_\text{min} \), we design the importance weighting function \( \lambda'(t) \) to emphasize higher noise regions, defined as:
\begin{equation}
    \lambda'(t) = \frac{\Delta t (t)}{t_\text{max}},
\end{equation}
where the step size \( \Delta t (t) \) is proportional to the sampled noise level \(t\), and \( t_\text{max} \) is a constant that normalizes the scale of the inversion loss. The noise sampling function \( p(t) \) and the step function \( \Delta t (t) \) used in computing both \(\mathcal{L}_\text{gct}\) and \(\mathcal{L}_\text{ict}\) are the same.



\begin{algorithm}
\caption{Inverse Consistency Training}
\label{alg:iCT}
\begin{algorithmic}[1]  % Adds line numbers
\setlength{\baselineskip}{0.9\baselineskip} % Adjust line spacing
\INPUT Dataset $\mathcal{D}$, weighting function $\lambda'(t)$, noise sampling function $p(t)$, noise range $[t_\text{min}, t_\text{max}]$, step function $\Delta t(t)$, noiser $N_\varphi$
\STATE \rule{0.96\textwidth}{0.45pt} 
\STATE Sample $\boldsymbol{x}_0, c \sim \mathcal{D}$ 
\STATE Sample noise $\boldsymbol{z} \sim \mathcal{N}(\boldsymbol{0},\mathbf{I})$, time step $t \sim p(t)$
\STATE Clamp $t \leftarrow \text{clamp}(t,t_\text{min}, t_\text{max})$
\STATE Compute noisy sample: $\boldsymbol{x}_t = \boldsymbol{x}_0 + t\boldsymbol{z}$
\STATE Compute cleaner sample: $\boldsymbol{x}_r = \boldsymbol{x}_t - \Delta t(t) \boldsymbol{z}$
\vspace{0.3em} 
\STATE Compute loss: 
\[
\mathcal{L}_\text{ict} = \lambda'(t) \, d(N_{\varphi}(\boldsymbol{x}_r, r, c), D_{{\varphi}^-}(\boldsymbol{x}_t, t, c))
\]
\STATE Return $\mathcal{L}_\text{ict}$ 
\end{algorithmic}
\end{algorithm}

Together, iGCT jointly optimizes the two consistency objectives \(\mathcal{L}_\text{gct}, \mathcal{L}_\text{ict}\), and aligns the noiser and denoiser via a reconstruction loss, \(\mathcal{L}_\text{recon}\). To improve training efficiency, \(\mathcal{L}_\text{recon}\) is computed every \(i_\text{skip}\), reducing the computational cost of back-propagation through both the weights of the \textit{denoiser} \(\theta\) and the \textit{noiser} \(\varphi\). Alg. \ref{alg:iGCT} provides an overview of iGCT. 

\begin{algorithm}
\caption{iGCT}
\label{alg:iGCT}
\begin{algorithmic}[1]  % Adds line numbers
\setlength{\baselineskip}{0.9\baselineskip} % Adjust line spacing
\INPUT Dataset $\mathcal{D}$, learning rate $\eta$, weighting functions $\lambda'(t), \lambda(t), \lambda_{\text{recon}}$, noise sampling function $p(t)$, noise range $[t_\text{min}, t_\text{max}]$, step function $\Delta t(t)$, guidance mask function $q(t)$, guidance range $[w_\text{min}, w_\text{max}]$, reconstruction skip iters $i_\text{skip}$, models $N_\varphi, D_\theta$
\STATE \rule{0.9\textwidth}{0.45pt}  % Horizontal line to separate input from main algorithm
\STATE \textbf{Init:} $\theta, \varphi$, $\text{Iters} = 0$
\REPEAT
\STATE Do guided consistency training 
\[
\mathcal{L}_\text{gct}(\theta;\mathcal{D},\lambda(t),p(t),t_\text{min},t_\text{max},\Delta t(t),q(t),w_\text{min},w_\text{max})
\]
\STATE Do inverse consistency training
\[
\mathcal{L}_\text{ict}(\varphi;\mathcal{D},\lambda'(t),p(t),t_\text{min},t_\text{max},\Delta t(t))
\]
\IF{$(\text{Iters} \ \% \ i_\text{skip}) == 0$}
\STATE Compute reconstruction loss
\[
\mathcal{L}_\text{recon} = d(D_{\theta}(N_{\varphi}(\boldsymbol{x}_0,t_\text{min},c),t_\text{max},c,0), \boldsymbol{x}_0)
\]
\ELSE
\STATE \[
\mathcal{L}_\text{recon} = 0
\]
\ENDIF
\STATE Compute total loss: 
\[
\mathcal{L} = \mathcal{L}_\text{gct} + \mathcal{L}_\text{ict} + \lambda_{\text{recon}}\mathcal{L}_\text{recon}
\]
\STATE $\theta \leftarrow \theta - \eta \nabla_{\theta} \mathcal{L}, \ \varphi \leftarrow \varphi - \eta \nabla_{\varphi} \mathcal{L}$
\STATE $\text{Iters} = \text{Iters} + 1$
\UNTIL{$\Delta t \rightarrow dt$}
\end{algorithmic}
\end{algorithm}



\vspace{-0.3cm}
\section{Preconditioning for Noiser}
\label{appendix:unit-variance}
\vspace{-0.1cm}

We define 
\begin{equation}
    N_{\varphi}(\boldsymbol{x}_t, t, c) = c_\text{skip}(t) \, \boldsymbol{x}_t + c_\text{out}(t) \, F_{\varphi}(c_\text{in}(t) \, \boldsymbol{x}_t, t, c),
\end{equation}
where \( c_\text{in}(t) = \frac{1}{\sqrt{t^2 + \sigma_\text{data}^2}} \), \( c_\text{skip}(t) = 1 \), and \( c_\text{out}(t) = t_\text{max} - t \). This setup naturally serves as a boundary condition. Specifically:

\begin{itemize}
    \item When \( t = 0 \),
    \begin{equation}
        c_\text{out}(0) = t_\text{max} \gg c_\text{skip}(0) = 1,
    \end{equation}
    emphasizing that the model's noise prediction dominates the residual information given a relatively clean sample.

    \item When \( t = t_\text{max} \),
    \begin{equation}
        N_{\varphi}(\boldsymbol{x}_{t_\text{max}}, t_\text{max}, c) = \boldsymbol{x}_{t_\text{max}},
    \end{equation}
    satisfying the condition that \( N_{\varphi} \) outputs \( \boldsymbol{x}_{t_\text{max}} \) at the maximum time step.
\end{itemize}



We show that these preconditions ensure unit variance for the model’s input and target. First, \(\text{Var}_{\boldsymbol{x}_0, z}[\boldsymbol{x}_t] = \sigma_\text{data}^2 + t^2\), so setting \( c_\text{in}(t) = \frac{1}{\sqrt{\sigma_\text{data}^2 + t^2}} \) normalizes the input variance to 1. Second, we require the training target to have unit variance. Given the noise target for \( N_{\varphi} \) is \(\boldsymbol{x}_{t_\text{max}} = \boldsymbol{x}_0 + t_\text{max} z\), by moving of terms, the effective target for \( F_{\varphi} \) can be written as,
\begin{equation}
    \frac{\boldsymbol{x}_{t_\text{max}} - c_\text{skip}(t)\boldsymbol{x}_{t}}{c_\text{out}(t)}
\end{equation}
When \(c_\text{skip}(t) = 1\), \(c_\text{out}(t) = t_\text{max} - t \), we verify that target is unit variance,
\begin{align}
    &\text{Var}_{\boldsymbol{x}_0, \boldsymbol{z}} \left[ \frac{\boldsymbol{x}_{t_\text{max}} - c_\text{skip}(t) \, \boldsymbol{x}_{t}}{c_\text{out}(t)} \right] \\ \notag
    = \ &\text{Var}_{\boldsymbol{x}_0, \boldsymbol{z}} \left[ \frac{\boldsymbol{x}_0 + t_\text{max} \, \boldsymbol{z} - (\boldsymbol{x}_0 + t \, \boldsymbol{z})}{t_\text{max} - t} \right] \notag \\
    = \ &\text{Var}_{\boldsymbol{x}_0, \boldsymbol{z}} \left[ \frac{(t_\text{max} - t) \, \boldsymbol{z}}{t_\text{max} - t} \right] \notag \\
    = \ &\text{Var}_{\boldsymbol{x}_0, \boldsymbol{z}}[\boldsymbol{z}] \notag \\
    = \ &1. \notag
\end{align}

\vspace{-0.3cm}
\section{Baselines \& Training Details}
\label{appendix:bs-config}
\vspace{-0.1cm}

\begin{figure}[t!]  
    \centering
    \begin{subfigure}[b]{0.33\textwidth}
    \includegraphics[width=\textwidth]{fig/appendix/guidance_embed.pdf} 
        \caption{Guidance embedding.}
    \end{subfigure}
    \hfill
    \begin{subfigure}[b]{0.33\textwidth}
    \includegraphics[width=\textwidth]{fig/appendix/adm_arch.pdf} 
        \caption{NCSN++ architecture.}
    \end{subfigure}
    \hfill
    \begin{subfigure}[b]{0.33\textwidth}
    \includegraphics[width=\textwidth]{fig/appendix/ncsnpp_arch.pdf} 
        \caption{ADM architecture.}
    \end{subfigure}
    \hfill
    \caption{Design of guidance embedding, and conditioning under different network architectures.}
    \vspace{-1em}
    \label{fig:guidance_conditioning}
\end{figure}

For our diffusion model baseline, we follow \textit{EDM}'s official repository (\href{https://github.com/NVlabs/edm}{https://github.com/NVlabs/edm}) instructions for training and set \textit{label\_dropout} to 0.1 to optimize a CFG (classifier-free guided) DM. We will use this DM as the teacher model for our consistency model baseline via consistency distillation. 

The consistency model baseline \textit{Guided CD} is trained with a discrete-time schedule. We set the discretization steps \( N = 18 \) and use a Heun ODE solver to predict update directions based on the CFG EDM, as in \cite{song2023consistency}. Following \cite{luo2023latent}, we modify the model's architecture and iGCT's denoiser to accept guidance strength \(w\) by adding an extra linear layer. See the detailed architecture design for guidance conditioning of consistency model in Fig. \ref{fig:guidance_conditioning}. A range of guidance scales \(w \in [1,15]\) is uniformly sampled at training. Following \cite{song2023improved}, we replace LPIPS by Pseudo-Huber loss, with \(c=0.03 \) determining the breadth of the smoothing section between L1 and L2. See Table \ref{tab:training_configs} for a summary of the training configurations for our baseline models.


\begin{table}[t!]
\centering
\renewcommand{\arraystretch}{1.3} % Adjust vertical spacing
\small % Reduce text size
\caption{Summary of training configurations for baseline models.}
\begin{tabular}{lccc}
\toprule
\multirow{2}{*}{} & \multicolumn{2}{c}{\textbf{CIFAR-10}} & \textbf{ImageNet64}  \\
                  & EDM & Guided-CD & EDM \\
\midrule
\multicolumn{4}{l}{\textbf{\small Arch. config.}} \\
\hline
model arch.        & NCSN++ & NCSN++ & ADM     \\
channels mult.     & 2,2,2  & 2,2,2  & 1,2,3,4 \\
UNet size          & 56.4M  & 56.4M  & 295.9M  \\
\midrule
\multicolumn{4}{l}{\textbf{\small Training config.}} \\
\hline
lr             & 1e-3  & 4e-4  & 2e-4 \\
batch          & 512   & 512   & 4096 \\
dropout        & 0.13  & 0     & 0.1 \\
label dropout  & 0.1   & (n.a.) & 0.1 \\
loss           & L2    & Huber & L2    \\
training iterations & 390k  & 800k  & 800K \\
\bottomrule
\end{tabular}
\label{tab:training_configs}
\end{table}


\begin{table}[t!]
\centering
\renewcommand{\arraystretch}{1.3} % Adjust vertical spacing
\small % Reduce text size
\caption{Summary of training configurations for iGCT.}
\begin{tabular}{lcc}
\toprule
\multirow{2}{*}{} & \textbf{CIFAR-10} & \textbf{ImageNet64}  \\
                  & iGCT & iGCT \\
\midrule
\multicolumn{3}{l}{\textbf{\small Arch. config.}} \\
\hline
model arch.        & NCSN++ & ADM \\
channels mult.     & 2,2,2  & 1,2,2,3 \\
UNet size          & 56.4M  & 182.4M \\ 
Total size         & 112.9M & 364.8M \\ 
\midrule
\multicolumn{3}{l}{\textbf{\small Training config.}} \\
\hline
lr              & 1e-4 & 1e-4 \\
batch           & 1024 & 1024 \\
dropout            & 0.2 & 0.3 \\
loss               & Huber   & Huber \\
\(c\)                  & 0.03    &  0.06 \\
\(d\)                  & 40k     &  40k \\
\( P_\textit{mean} \) & -1.1 &  -1.1 \\
\( P_\textit{std} \) &  2.0  &  2.0  \\
\( \lambda_{\text{recon}} \) & 2e-5 & \parbox[t]{3.5cm}{\centering 2e-5, (\(\leq\) 180k)\\ 4e-5, (\(\leq\) 200k)\\ 6e-5, (\(\leq\) 260k) } \\  
\( i_{\text{skip}} \)        & 10 &  10 \\  
training iterations & 360k &  260k \\
\bottomrule
\end{tabular}
\label{tab:igct_training_configs}
\end{table}  

\begin{figure*}[t] 
    \centering
    \includegraphics[width=1.0\textwidth]{fig/appendix/inversion_collapse.pdf} 
    \caption{Inversion collapse observed during training on ImageNet64. The left image shows the input data. The middle image depicts the inversion collapse that occurred at iteration 220k, where leakage of signals in the noise latent can be visualized. The right image shows the inversion results at iteration 220k after appropriately increasing $\lambda_{\text{recon}}$ to 6e-5. The inversion images are generated by scaling the model's outputs by $1/80$, i.e., $ 1/t_\text{max}$, then clipping the values to the range [-3, 3] before denormalizing them to the range [0, 255]. }
    \vspace{-1.5em}
    \label{fig:inversion_collpase}
\end{figure*}

iGCT is trained with a continuous-time scheduler inspired by ECT \cite{ect}. To rigorously assess its independence from diffusion-based models, iGCT is trained from scratch rather than fine-tuned from a pre-trained diffusion model. Consequently, the training curriculum begins with an initial diffusion training stage, followed by consistency training with the step size halved every \(d\) iterations. In practice, we adopt the same noise sampling distribution \(p(t)\), same step function \(\Delta t (t) \), and same distance metric \( d(\cdot, \cdot) \) for both guided consistency training and inverse consistency training. 

For CIFAR-10, iGCT adopts the same UNet architecture as the baseline models. However, the overall model size is doubled, as iGCT comprises two UNets: one for the denoiser and one for the noiser. The Pseudo-Huber loss is employed as the distance metric, with a constant parameter \( c = 0.03 \). Consistency training is organized into nine stages, each comprising 400k iterations with the step size halved from the last stage. We found that training remains stable when the reconstruction weight \( \lambda_{\text{recon}} \) is fixed at \( 2 \times 10^{-5} \) throughout the entire training process.
 
For ImageNet64, iGCT employs a reduced ADM architecture \cite{dhariwal2021diffusionmodelsbeatgans} with smaller channel sizes to address computational constraints. A higher dropout rate and Pseudo-Huber loss with \( c = 0.06 \) is used, following prior works \cite{ect,song2023improved}. During our experiments, we observed that training on ImageNet64 is sensitive to the reconstruction weight. Keeping \(\lambda_{\text{recon}}\) fixed throughout training leads to inversion collapse, with significant signal leaked to the latent noise (see Fig. \ref{fig:inversion_collpase}). We found that increasing \(\lambda_{\text{recon}}\) to \( 4 \times 10^{-5} \) at iteration 1800 and to \( 6 \times 10^{-5} \) at iteration 2000 effectively stabilizes training and prevents collapse. This suggests that the reconstruction loss serves as a regularizer for iGCT. Additionally, we observed diminishing returns when training exceeded 240k iterations, leading us to stop at 260k iterations for our experiments. These findings indicate that alternative training strategies, such as framing iGCT as a multi-task learning problem \cite{kendall2018multi,liu2019loss}, and conducting a more sophisticated analysis of loss weighting, may be necessary to enhance stability and improve convergence. See Table \ref{tab:igct_training_configs} for a summary of the training configurations for iGCT.



\begin{table}[t]
\caption{Comparison of GPU hours across the methods used in our experiments on CIFAR-10.}
\centering
\begin{tabular}{|l|c|}
\hline
\textbf{Methods} & \textbf{A100 (40G) GPU hours} \\ \hline
CFG-EDM \cite{karras2022elucidating} & 312 \\ \hline
Guided-CD \cite{song2023consistency} & 3968 \\ \hline
iGCT (ours) & 2032 \\ \hline
\end{tabular}
\label{table:compute_resources}
\end{table}



\begin{figure*}[t!]  
    \centering
    \begin{subfigure}[b]{0.33\textwidth}
    \includegraphics[width=\textwidth]{fig/cls_exp_w1.png} 
        \caption{Accuracy on various ratios of augmented data, guidance scale w=1.}
    \end{subfigure}
    \begin{subfigure}[b]{0.33\textwidth}
    \includegraphics[width=\textwidth]{fig/cls_exp_w3.png} 
        \caption{Accuracy on various ratios of augmented data, guidance scale w=3.}
    \end{subfigure}
    \begin{subfigure}[b]{0.33\textwidth}
    \includegraphics[width=\textwidth]{fig/cls_exp_w5.png} 
        \caption{Accuracy on various ratios of augmented data, guidance scale w=5.}
    \end{subfigure}
    \begin{subfigure}[b]{0.33\textwidth}
    \includegraphics[width=\textwidth]{fig/cls_exp_w7.png} 
        \caption{Accuracy on various ratios of augmented data, guidance scale w=7.}
    \end{subfigure}
    \begin{subfigure}[b]{0.33\textwidth}
    \includegraphics[width=\textwidth]{fig/cls_exp_w9.png} 
        \caption{Accuracy on various ratios of augmented data, guidance scale w=9.}
    \end{subfigure}
    \caption{Comparison of synthesized methods, CFG-EDM vs iGCT, used for data augmentation in image classification. iGCT consistently improves accuracy. Conversely, augmentation data synthesized from CFG-EDM offers only limited gains.}
    \vspace{-1.5em}
    \label{fig:cls_results}
\end{figure*}


\vspace{-0.1cm}
\section{Application: Data Augmentation Under Different Guidance}
\vspace{-0.2cm}

In this section, we show the effectiveness of data augmentation with diffusion-based models, CFG-EDM and iGCT, across varying guidance scales for image classification on CIFAR-10 \cite{article}. High quality data augmentation has been shown to enhance classification performance \cite{yang2023imagedataaugmentationdeep}. Under high guidance, augmentation data generated from iGCT consistently improves accuracy. Conversely, augmentation data synthesized from CFG-EDM offers only limited gains. We describe the ratios of real to synthesized data, the classifier architecture, and the training setup in the following. 

\noindent{\bf Training Details.} We conduct classification experiments trained on six different mixtures of augmented data synthesized by iGCT and CFG-EDM: \(0\%\), \(20\%\), \(40\%\), \(80\%\), and \(100\%\). The ratio represents \(\textit{synthesized data} / \textit{real data}\). For example, \(0\%\) indicates that the training and validation sets contain only 50k of real samples from CIFAR-10, and \(20\%\) includes 50k real \textit{and} 10k synthesized samples. In terms of guidance scales, we choose \(w=1,3,5,7,9\) to synthesize the augmented data using iGCT and CFG-EDM. 
The augmented dataset is split 80/20 for training and validation. For testing, the model is evaluated on the CIFAR-10 test set with 10k samples and ground truth labels. 

The standard ResNet-18 \cite{he2015deepresiduallearningimage} is used to train on all different augmented datasets. All models are trained for 250 epochs, with batch size 64, using an Adam optimizer \cite{kingma2017adammethodstochasticoptimization}. For each augmentation dataset, we train the model six times under different seeds and report the average classification accuracy.

\noindent{\bf Results.} The classifier's accuracy, trained on augmented data synthesized by CFG-EDM and iGCT, is shown in Fig. \ref{fig:cls_results}. With \(w=1\) (no guidance), both iGCT and CFG-EDM provide comparable performance boosts. As guidance scale increases, iGCT shows more significant improvements than CFG-EDM. At high guidance and augmentation ratios, performance drops, but this effect occurs later for iGCT (e.g., at \(100\%\) augmentation and \(w=9\)), while CFG-EDM stops improving accuracy at \(w=7\). This experiment highlights the importance of high-quality data under high guidance, with iGCT outperforming CFG-EDM in data quality.

\section{Uncurated Results}
In this section, we present additional qualitative results to highlight the performance of our proposed iGCT method compared to the multi-step EDM baseline. These visualizations include both inversion and guidance tasks across the CIFAR-10 and ImageNet64 datasets. The results demonstrate iGCT's ability to maintain competitive quality with significantly fewer steps and minimal artifacts, showcasing the effectiveness of our approach.

\subsection{Inversion Results}
We provide additional visualization of the latent noise on both CIFAR-10 and ImageNet64 datasets. Fig. \ref{fig:CIFAR-10_inversion_reconstruction} and Fig. \ref{fig:im64_inversion_reconstruction} compare our 1-step iGCT with the multi-step EDM on inversion and reconstruction.  

\subsection{Editing Results}
In this section, we dump more uncurated editing results on ImageNet64's subgroups mentioned in Sec. \ref{sec:image-editing}. Fig. \ref{fig:im64_edit_1}--\ref{fig:im64_edit_4} illustrate a comparison between our 1-step iGCT and the multi-step EDM approach.

\subsection{Guidance Results}
In Section \ref{sec:guidance}, we demonstrated that iGCT provides a guidance solution without introducing the high-contrast artifacts commonly observed in CFG-based methods. Here, we present additional uncurated results on CIFAR-10 and ImageNet64. For CIFAR-10, iGCT achieves competitive performance compared to the baseline diffusion model, which requires multiple steps for generation. See Figs. \ref{fig:CIFAR-10_guided_1}--\ref{fig:CIFAR-10_guided_10}. For ImageNet64, although the visual quality of iGCT's generated images falls slightly short of expectations, this can be attributed to the smaller UNet architecture used—only 61\% of the baseline model size—and the need for a more robust training curriculum to prevent collapse, as discussed in Section \ref{appendix:bs-config}. Nonetheless, even at higher guidance levels, iGCT maintains style consistency, whereas CFG-based methods continue to suffer from pronounced high-contrast artifacts. See Figs. \ref{fig:im64_guided_1}--\ref{fig:im64_guided_4}.


\begin{figure*}[t]
    \centering
    \begin{subfigure}{0.48\textwidth}
        \centering
        \includegraphics[width=\linewidth]{fig/appendix/recon_c10_data.png}
        \caption{CIFAR-10: Original data}
    \end{subfigure}
    \begin{subfigure}{0.48\textwidth}
        \centering
        \includegraphics[width=\linewidth]{fig/appendix/recon_im64_data.png}
        \caption{ImageNet64: Original data}
    \end{subfigure}

    \begin{subfigure}{0.48\textwidth}
        \centering
        \includegraphics[width=\linewidth]{fig/appendix/inv_c10_edm.png}
    \end{subfigure}
    \begin{subfigure}{0.48\textwidth}
        \centering
        \includegraphics[width=\linewidth]{fig/appendix/inv_im64_edm.png}
    \end{subfigure}

    \begin{subfigure}{0.48\textwidth}
        \centering
        \includegraphics[width=\linewidth]{fig/appendix/recon_c10_edm.png}
        \caption{CIFAR-10: Inversion + reconstruction, EDM (18 NFE)}
    \end{subfigure}
    \begin{subfigure}{0.48\textwidth}
        \centering
        \includegraphics[width=\linewidth]{fig/appendix/recon_im64_edm.png}
        \caption{ImageNet64: Inversion + reconstruction, EDM (18 NFE)}
    \end{subfigure}

    \begin{subfigure}{0.48\textwidth}
        \centering
        \includegraphics[width=\linewidth]{fig/appendix/inv_c10_igct.png}
    \end{subfigure}
    \begin{subfigure}{0.48\textwidth}
        \centering
        \includegraphics[width=\linewidth]{fig/appendix/inv_im64_igct.png}
    \end{subfigure}

    \begin{subfigure}{0.48\textwidth}
        \centering
        \includegraphics[width=\linewidth]{fig/appendix/recon_c10_igct.png}
        \caption{CIFAR-10: Inversion + reconstruction, iGCT (1 NFE)}
    \end{subfigure}
    \begin{subfigure}{0.48\textwidth}
        \centering
        \includegraphics[width=\linewidth]{fig/appendix/recon_im64_igct.png}
        \caption{ImageNet64: Inversion + reconstruction, iGCT (1 NFE)}
    \end{subfigure}

    \caption{Comparison of inversion and reconstruction for CIFAR-10 (left) and ImageNet64 (right).}
    \label{fig:comparison_CIFAR-10_imagenet64}
\end{figure*}




\begin{figure*}[t]
    \centering

    % Left column: corgi -> golden retriever
    \begin{minipage}{0.48\textwidth}
        \centering
        \begin{subfigure}{0.48\textwidth}
            \includegraphics[width=\linewidth]{fig/appendix_edit_igct/src_corgi.png}
            \caption{Original: "corgi"}
        \end{subfigure}

        \begin{subfigure}{0.48\textwidth}
            \includegraphics[width=\linewidth]{fig/appendix_edit_edm/w=0_src_corgi_tar_golden_retriever.png}
            \caption{EDM (18 NFE), w=1}
        \end{subfigure}
        \begin{subfigure}{0.48\textwidth}
            \includegraphics[width=\linewidth]{fig/appendix_edit_edm/w=6_src_corgi_tar_golden_retriever.png}
            \caption{EDM (18 NFE), w=7}
        \end{subfigure}
        \begin{subfigure}{0.48\textwidth}
            \includegraphics[width=\linewidth]{fig/appendix_edit_igct/w=6_src_corgi_tar_golden_retriever.png}
            \caption{iGCT (1 NFE), w=7}
        \end{subfigure}
        \begin{subfigure}{0.48\textwidth}
            \includegraphics[width=\linewidth]{fig/appendix_edit_igct/w=0_src_corgi_tar_golden_retriever.png}
            \caption{iGCT (1 NFE), w=1}
        \end{subfigure}

        \caption{ImageNet64: "corgi" $\rightarrow$ "golden retriever"}
        \label{fig:im64_edit_1}
    \end{minipage}
    \hfill
    % Right column: zebra -> horse
    \begin{minipage}{0.48\textwidth}
        \centering
        \begin{subfigure}{0.48\textwidth}
            \includegraphics[width=\linewidth]{fig/appendix_edit_igct/src_zebra.png}
            \caption{Original: "zebra"}
        \end{subfigure}

        \begin{subfigure}{0.48\textwidth}
            \includegraphics[width=\linewidth]{fig/appendix_edit_edm/w=0_src_zebra_tar_horse.png}
            \caption{EDM (18 NFE), w=1}
        \end{subfigure}
        \begin{subfigure}{0.48\textwidth}
            \includegraphics[width=\linewidth]{fig/appendix_edit_edm/w=6_src_zebra_tar_horse.png}
            \caption{EDM (18 NFE), w=7}
        \end{subfigure}
        \begin{subfigure}{0.48\textwidth}
            \includegraphics[width=\linewidth]{fig/appendix_edit_igct/w=0_src_zebra_tar_horse.png}
            \caption{iGCT (1 NFE), w=1}
        \end{subfigure}
        \begin{subfigure}{0.48\textwidth}
            \includegraphics[width=\linewidth]{fig/appendix_edit_igct/w=6_src_zebra_tar_horse.png}
            \caption{iGCT (1 NFE), w=7}
        \end{subfigure}

        \caption{ImageNet64: "zebra" $\rightarrow$ "horse"}
        \label{fig:im64_edit_2}
    \end{minipage}

\end{figure*}

\begin{figure*}[t]
    \centering

    % Left column: broccoli -> cauliflower
    \begin{minipage}{0.48\textwidth}
        \centering
        \begin{subfigure}{0.48\textwidth}
            \includegraphics[width=\linewidth]{fig/appendix_edit_igct/src_broccoli.png}
            \caption{Original: "broccoli"}
        \end{subfigure}

        \begin{subfigure}{0.48\textwidth}
            \includegraphics[width=\linewidth]{fig/appendix_edit_edm/w=0_src_broccoli_tar_cauliflower.png}
            \caption{EDM (18 NFE), w=1}
        \end{subfigure}
        \begin{subfigure}{0.48\textwidth}
            \includegraphics[width=\linewidth]{fig/appendix_edit_edm/w=6_src_broccoli_tar_cauliflower.png}
            \caption{EDM (18 NFE), w=7}
        \end{subfigure}
        \begin{subfigure}{0.48\textwidth}
            \includegraphics[width=\linewidth]{fig/appendix_edit_igct/w=0_src_broccoli_tar_cauliflower.png}
            \caption{iGCT (1 NFE), w=1}
        \end{subfigure}
        \begin{subfigure}{0.48\textwidth}
            \includegraphics[width=\linewidth]{fig/appendix_edit_igct/w=6_src_broccoli_tar_cauliflower.png}
            \caption{iGCT (1 NFE), w=7}
        \end{subfigure}

        \caption{ImageNet64: "broccoli" $\rightarrow$ "cauliflower"}
        \label{fig:im64_edit_3}
    \end{minipage}
    \hfill
    % Right column: jaguar -> tiger
    \begin{minipage}{0.48\textwidth}
        \centering
        \begin{subfigure}{0.48\textwidth}
            \includegraphics[width=\linewidth]{fig/appendix_edit_igct/src_jaguar.png}
            \caption{Original: "jaguar"}
        \end{subfigure}

        \begin{subfigure}{0.48\textwidth}
            \includegraphics[width=\linewidth]{fig/appendix_edit_edm/w=0_src_jaguar_tar_tiger.png}
            \caption{EDM (18 NFE), w=1}
        \end{subfigure}
        \begin{subfigure}{0.48\textwidth}
            \includegraphics[width=\linewidth]{fig/appendix_edit_edm/w=6_src_jaguar_tar_tiger.png}
            \caption{EDM (18 NFE), w=7}
        \end{subfigure}
        \begin{subfigure}{0.48\textwidth}
            \includegraphics[width=\linewidth]{fig/appendix_edit_igct/w=0_src_jaguar_tar_tiger.png}
            \caption{iGCT (1 NFE), w=1}
        \end{subfigure}
        \begin{subfigure}{0.48\textwidth}
            \includegraphics[width=\linewidth]{fig/appendix_edit_igct/w=6_src_jaguar_tar_tiger.png}
            \caption{iGCT (1 NFE), w=7}
        \end{subfigure}

        \caption{ImageNet64: "jaguar" $\rightarrow$ "tiger"}
        \label{fig:im64_edit_4}
    \end{minipage}

\end{figure*}






\begin{figure*}[b]
    \centering
    % First image
    \begin{subfigure}{0.25\textwidth}
        \includegraphics[width=\linewidth]{fig/appendix_edm/0_0.0_middle_4x4_grid.png}
        \caption{CFG-EDM (18 NFE), w=1.0}
    \end{subfigure}
    \begin{subfigure}{0.25\textwidth}
        \includegraphics[width=\linewidth]{fig/appendix_edm/0_6.0_middle_4x4_grid.png}
        \caption{CFG-EDM (18 NFE), w=7.0}
    \end{subfigure}
    \begin{subfigure}{0.25\textwidth}
        \includegraphics[width=\linewidth]{fig/appendix_edm/0_12.0_middle_4x4_grid.png}
        \caption{CFG-EDM (18 NFE), w=13.0}
    \end{subfigure}
    \begin{subfigure}{0.25\textwidth}
        \includegraphics[width=\linewidth]{fig/appendix_igct/0_0.0_middle_4x4_grid.png}
        \caption{iGCT (1 NFE), w=1.0}
    \end{subfigure}
    \begin{subfigure}{0.25\textwidth}
        \includegraphics[width=\linewidth]{fig/appendix_igct/0_6.0_middle_4x4_grid.png}
        \caption{iGCT (1 NFE), w=7.0}
    \end{subfigure}
    % Third image
    \begin{subfigure}{0.25\textwidth}
        \includegraphics[width=\linewidth]{fig/appendix_igct/0_12.0_middle_4x4_grid.png}
        \caption{iGCT (1 NFE), w=13.0}
    \end{subfigure}
    \caption{CIFAR-10 "airplane"}
    \label{fig:CIFAR-10_guided_1}
\end{figure*}
\begin{figure*}[t]
    \centering
    % First image
    \begin{subfigure}{0.25\textwidth}
        \includegraphics[width=\linewidth]{fig/appendix_edm/1_0.0_middle_4x4_grid.png}
        \caption{CFG-EDM (18 NFE), w=1.0}
    \end{subfigure}
    \begin{subfigure}{0.25\textwidth}
        \includegraphics[width=\linewidth]{fig/appendix_edm/1_6.0_middle_4x4_grid.png}
        \caption{CFG-EDM (18 NFE), w=7.0}
    \end{subfigure}
    \begin{subfigure}{0.25\textwidth}
        \includegraphics[width=\linewidth]{fig/appendix_edm/1_12.0_middle_4x4_grid.png}
        \caption{CFG-EDM (18 NFE), w=13.0}
    \end{subfigure}
    \begin{subfigure}{0.25\textwidth}
        \includegraphics[width=\linewidth]{fig/appendix_igct/1_0.0_middle_4x4_grid.png}
        \caption{iGCT (1 NFE), w=1.0}
    \end{subfigure}
    % Second image
    \begin{subfigure}{0.25\textwidth}
        \includegraphics[width=\linewidth]{fig/appendix_igct/1_6.0_middle_4x4_grid.png}
        \caption{iGCT (1 NFE), w=7.0}
    \end{subfigure}
    % Third image
    \begin{subfigure}{0.25\textwidth}
        \includegraphics[width=\linewidth]{fig/appendix_igct/1_12.0_middle_4x4_grid.png}
        \caption{iGCT (1 NFE), w=13.0}
    \end{subfigure}
    \caption{CIFAR-10 "car"}
    \label{fig:CIFAR-10_guided_2}
\end{figure*}
\begin{figure*}[t]
    \centering
    % First image
    \begin{subfigure}{0.25\textwidth}
        \includegraphics[width=\linewidth]{fig/appendix_edm/2_0.0_middle_4x4_grid.png}
        \caption{CFG-EDM (18 NFE), w=1.0}
    \end{subfigure}
    \begin{subfigure}{0.25\textwidth}
        \includegraphics[width=\linewidth]{fig/appendix_edm/2_6.0_middle_4x4_grid.png}
        \caption{CFG-EDM (18 NFE), w=7.0}
    \end{subfigure}
    \begin{subfigure}{0.25\textwidth}
        \includegraphics[width=\linewidth]{fig/appendix_edm/2_12.0_middle_4x4_grid.png}
        \caption{CFG-EDM (18 NFE), w=13.0}
    \end{subfigure}
    \begin{subfigure}{0.25\textwidth}
        \includegraphics[width=\linewidth]{fig/appendix_igct/2_0.0_middle_4x4_grid.png}
        \caption{iGCT (1 NFE), w=1.0}
    \end{subfigure}
    % Second image
    \begin{subfigure}{0.25\textwidth}
        \includegraphics[width=\linewidth]{fig/appendix_igct/2_6.0_middle_4x4_grid.png}
        \caption{iGCT (1 NFE), w=7.0}
    \end{subfigure}
    % Third image
    \begin{subfigure}{0.25\textwidth}
        \includegraphics[width=\linewidth]{fig/appendix_igct/2_12.0_middle_4x4_grid.png}
        \caption{iGCT (1 NFE), w=13.0}
    \end{subfigure}
    \caption{CIFAR-10 "bird"}
    \label{fig:CIFAR-10_guided_3}
\end{figure*}
\begin{figure*}[t]
    \centering
    % First image
    \begin{subfigure}{0.25\textwidth}
        \includegraphics[width=\linewidth]{fig/appendix_edm/3_0.0_middle_4x4_grid.png}
        \caption{CFG-EDM (18 NFE), w=1.0}
    \end{subfigure}
    \begin{subfigure}{0.25\textwidth}
        \includegraphics[width=\linewidth]{fig/appendix_edm/3_6.0_middle_4x4_grid.png}
        \caption{CFG-EDM (18 NFE), w=7.0}
    \end{subfigure}
    \begin{subfigure}{0.25\textwidth}
        \includegraphics[width=\linewidth]{fig/appendix_edm/3_12.0_middle_4x4_grid.png}
        \caption{CFG-EDM (18 NFE), w=13.0}
    \end{subfigure}
    \begin{subfigure}{0.25\textwidth}
        \includegraphics[width=\linewidth]{fig/appendix_igct/3_0.0_middle_4x4_grid.png}
        \caption{iGCT (1 NFE), w=1.0}
    \end{subfigure}
    % Second image
    \begin{subfigure}{0.25\textwidth}
        \includegraphics[width=\linewidth]{fig/appendix_igct/3_6.0_middle_4x4_grid.png}
        \caption{iGCT (1 NFE), w=7.0}
    \end{subfigure}
    % Third image
    \begin{subfigure}{0.25\textwidth}
        \includegraphics[width=\linewidth]{fig/appendix_igct/3_12.0_middle_4x4_grid.png}
        \caption{iGCT (1 NFE), w=13.0}
    \end{subfigure}
    \caption{CIFAR-10 "cat"}
    \label{fig:CIFAR-10_guided_4}
\end{figure*}
\begin{figure*}[t]
    \centering
    % First image
    \begin{subfigure}{0.25\textwidth}
        \includegraphics[width=\linewidth]{fig/appendix_edm/4_0.0_middle_4x4_grid.png}
        \caption{CFG-EDM (18 NFE), w=1.0}
    \end{subfigure}
    \begin{subfigure}{0.25\textwidth}
        \includegraphics[width=\linewidth]{fig/appendix_edm/4_6.0_middle_4x4_grid.png}
        \caption{CFG-EDM (18 NFE), w=7.0}
    \end{subfigure}
    \begin{subfigure}{0.25\textwidth}
        \includegraphics[width=\linewidth]{fig/appendix_edm/4_12.0_middle_4x4_grid.png}
        \caption{CFG-EDM (18 NFE), w=13.0}
    \end{subfigure}
    \begin{subfigure}{0.25\textwidth}
        \includegraphics[width=\linewidth]{fig/appendix_igct/4_0.0_middle_4x4_grid.png}
        \caption{iGCT (1 NFE), w=1.0}
    \end{subfigure}
    % Second image
    \begin{subfigure}{0.25\textwidth}
        \includegraphics[width=\linewidth]{fig/appendix_igct/4_6.0_middle_4x4_grid.png}
        \caption{iGCT (1 NFE), w=7.0}
    \end{subfigure}
    % Third image
    \begin{subfigure}{0.25\textwidth}
        \includegraphics[width=\linewidth]{fig/appendix_igct/4_12.0_middle_4x4_grid.png}
        \caption{iGCT (1 NFE), w=13.0}
    \end{subfigure}
    \caption{CIFAR-10 "deer"}
    \label{fig:CIFAR-10_guided_5}
\end{figure*}
\begin{figure*}[t]
    \centering
    % First image
    \begin{subfigure}{0.25\textwidth}
        \includegraphics[width=\linewidth]{fig/appendix_edm/5_0.0_middle_4x4_grid.png}
        \caption{CFG-EDM (18 NFE), w=1.0}
    \end{subfigure}
    \begin{subfigure}{0.25\textwidth}
        \includegraphics[width=\linewidth]{fig/appendix_edm/5_6.0_middle_4x4_grid.png}
        \caption{CFG-EDM (18 NFE), w=7.0}
    \end{subfigure}
    \begin{subfigure}{0.25\textwidth}
        \includegraphics[width=\linewidth]{fig/appendix_edm/5_12.0_middle_4x4_grid.png}
        \caption{CFG-EDM (18 NFE), w=13.0}
    \end{subfigure}
    \begin{subfigure}{0.25\textwidth}
        \includegraphics[width=\linewidth]{fig/appendix_igct/5_0.0_middle_4x4_grid.png}
        \caption{iGCT (1 NFE), w=1.0}
    \end{subfigure}
    % Second image
    \begin{subfigure}{0.25\textwidth}
        \includegraphics[width=\linewidth]{fig/appendix_igct/5_6.0_middle_4x4_grid.png}
        \caption{iGCT (1 NFE), w=7.0}
    \end{subfigure}
    % Third image
    \begin{subfigure}{0.25\textwidth}
        \includegraphics[width=\linewidth]{fig/appendix_igct/5_12.0_middle_4x4_grid.png}
        \caption{iGCT (1 NFE), w=13.0}
    \end{subfigure}
    \caption{CIFAR-10 "dog"}
    \label{fig:CIFAR-10_guided_6}
\end{figure*}
\begin{figure*}[t]
    \centering
    % First image
    \begin{subfigure}{0.25\textwidth}
        \includegraphics[width=\linewidth]{fig/appendix_edm/6_0.0_middle_4x4_grid.png}
        \caption{CFG-EDM (18 NFE), w=1.0}
    \end{subfigure}
    \begin{subfigure}{0.25\textwidth}
        \includegraphics[width=\linewidth]{fig/appendix_edm/6_6.0_middle_4x4_grid.png}
        \caption{CFG-EDM (18 NFE), w=7.0}
    \end{subfigure}
    \begin{subfigure}{0.25\textwidth}
        \includegraphics[width=\linewidth]{fig/appendix_edm/6_12.0_middle_4x4_grid.png}
        \caption{CFG-EDM (18 NFE), w=13.0}
    \end{subfigure}
    \begin{subfigure}{0.25\textwidth}
        \includegraphics[width=\linewidth]{fig/appendix_igct/6_0.0_middle_4x4_grid.png}
        \caption{iGCT (1 NFE), w=1.0}
    \end{subfigure}
    % Second image
    \begin{subfigure}{0.25\textwidth}
        \includegraphics[width=\linewidth]{fig/appendix_igct/6_6.0_middle_4x4_grid.png}
        \caption{iGCT (1 NFE), w=7.0}
    \end{subfigure}
    % Third image
    \begin{subfigure}{0.25\textwidth}
        \includegraphics[width=\linewidth]{fig/appendix_igct/6_12.0_middle_4x4_grid.png}
        \caption{iGCT (1 NFE), w=13.0}
    \end{subfigure}
    \caption{CIFAR-10 "frog"}
    \label{fig:CIFAR-10_guided_7}
\end{figure*}
\begin{figure*}[t]
    \centering
    % First image
    \begin{subfigure}{0.25\textwidth}
        \includegraphics[width=\linewidth]{fig/appendix_edm/7_0.0_middle_4x4_grid.png}
        \caption{CFG-EDM (18 NFE), w=1.0}
    \end{subfigure}
    \begin{subfigure}{0.25\textwidth}
        \includegraphics[width=\linewidth]{fig/appendix_edm/7_6.0_middle_4x4_grid.png}
        \caption{CFG-EDM (18 NFE), w=7.0}
    \end{subfigure}
    \begin{subfigure}{0.25\textwidth}
        \includegraphics[width=\linewidth]{fig/appendix_edm/7_12.0_middle_4x4_grid.png}
        \caption{CFG-EDM (18 NFE), w=13.0}
    \end{subfigure}
    \begin{subfigure}{0.25\textwidth}
        \includegraphics[width=\linewidth]{fig/appendix_igct/7_0.0_middle_4x4_grid.png}
        \caption{iGCT (1 NFE), w=1.0}
    \end{subfigure}
    % Second image
    \begin{subfigure}{0.25\textwidth}
        \includegraphics[width=\linewidth]{fig/appendix_igct/7_6.0_middle_4x4_grid.png}
        \caption{iGCT (1 NFE), w=7.0}
    \end{subfigure}
    % Third image
    \begin{subfigure}{0.25\textwidth}
        \includegraphics[width=\linewidth]{fig/appendix_igct/7_12.0_middle_4x4_grid.png}
        \caption{iGCT (1 NFE), w=13.0}
    \end{subfigure}
    \caption{CIFAR-10 "horse"}
    \label{fig:CIFAR-10_guided_8}
\end{figure*}
\begin{figure*}[t]
    \centering
    % First image
    \begin{subfigure}{0.25\textwidth}
        \includegraphics[width=\linewidth]{fig/appendix_edm/8_0.0_middle_4x4_grid.png}
        \caption{CFG-EDM (18 NFE), w=1.0}
    \end{subfigure}
    \begin{subfigure}{0.25\textwidth}
        \includegraphics[width=\linewidth]{fig/appendix_edm/8_6.0_middle_4x4_grid.png}
        \caption{CFG-EDM (18 NFE), w=7.0}
    \end{subfigure}
    \begin{subfigure}{0.25\textwidth}
        \includegraphics[width=\linewidth]{fig/appendix_edm/8_12.0_middle_4x4_grid.png}
        \caption{CFG-EDM (18 NFE), w=13.0}
    \end{subfigure}
    \begin{subfigure}{0.25\textwidth}
        \includegraphics[width=\linewidth]{fig/appendix_igct/8_0.0_middle_4x4_grid.png}
        \caption{iGCT (1 NFE), w=1.0}
    \end{subfigure}
    % Second image
    \begin{subfigure}{0.25\textwidth}
        \includegraphics[width=\linewidth]{fig/appendix_igct/8_6.0_middle_4x4_grid.png}
        \caption{iGCT (1 NFE), w=7.0}
    \end{subfigure}
    % Third image
    \begin{subfigure}{0.25\textwidth}
        \includegraphics[width=\linewidth]{fig/appendix_igct/8_12.0_middle_4x4_grid.png}
        \caption{iGCT (1 NFE), w=13.0}
    \end{subfigure}
    \caption{CIFAR-10 "ship"}
    \label{fig:CIFAR-10_guided_9}
\end{figure*}
\begin{figure*}[t]
    \centering
    % First image
    \begin{subfigure}{0.25\textwidth}
        \includegraphics[width=\linewidth]{fig/appendix_edm/9_0.0_middle_4x4_grid.png}
        \caption{CFG-EDM (18 NFE), w=1.0}
    \end{subfigure}
    \begin{subfigure}{0.25\textwidth}
        \includegraphics[width=\linewidth]{fig/appendix_edm/9_6.0_middle_4x4_grid.png}
        \caption{CFG-EDM (18 NFE), w=7.0}
    \end{subfigure}
    \begin{subfigure}{0.25\textwidth}
        \includegraphics[width=\linewidth]{fig/appendix_edm/9_12.0_middle_4x4_grid.png}
        \caption{CFG-EDM (18 NFE), w=13.0}
    \end{subfigure}
    \begin{subfigure}{0.25\textwidth}
        \includegraphics[width=\linewidth]{fig/appendix_igct/9_0.0_middle_4x4_grid.png}
        \caption{iGCT (1 NFE), w=1.0}
    \end{subfigure}
    % Second image
    \begin{subfigure}{0.25\textwidth}
        \includegraphics[width=\linewidth]{fig/appendix_igct/9_6.0_middle_4x4_grid.png}
        \caption{iGCT (1 NFE), w=7.0}
    \end{subfigure}
    % Third image
    \begin{subfigure}{0.25\textwidth}
        \includegraphics[width=\linewidth]{fig/appendix_igct/9_12.0_middle_4x4_grid.png}
        \caption{iGCT (1 NFE), w=13.0}
    \end{subfigure}
    \caption{CIFAR-10 "truck"}
    \label{fig:CIFAR-10_guided_10}
\end{figure*}


\begin{figure*}[b]
    \centering
    % First image
    \begin{subfigure}{0.25\textwidth}
        \includegraphics[width=\linewidth]{fig/appendix_im64_edm/edm_class_291_w=0.0.png}
        \caption{CFG-EDM (18 NFE), w=1.0}
    \end{subfigure}
    \begin{subfigure}{0.25\textwidth}
        \includegraphics[width=\linewidth]{fig/appendix_im64_edm/edm_class_291_w=6.0.png}
        \caption{CFG-EDM (18 NFE), w=7.0}
    \end{subfigure}
    \begin{subfigure}{0.25\textwidth}
        \includegraphics[width=\linewidth]{fig/appendix_im64_edm/edm_class_291_w=12.0.png}
        \caption{CFG-EDM (18 NFE), w=13.0}
    \end{subfigure}
    \begin{subfigure}{0.25\textwidth}
        \includegraphics[width=\linewidth]{fig/appendix_im64_igct/class_291_w=0.0.png}
        \caption{iGCT (2 NFE), w=1.0}
    \end{subfigure}
    \begin{subfigure}{0.25\textwidth}
        \includegraphics[width=\linewidth]{fig/appendix_im64_igct/class_291_w=6.0.png}
        \caption{iGCT (2 NFE), w=7.0}
    \end{subfigure}
    % Third image
    \begin{subfigure}{0.25\textwidth}
        \includegraphics[width=\linewidth]{fig/appendix_im64_igct/class_291_w=12.0.png}
        \caption{iGCT (2 NFE), w=13.0}
    \end{subfigure}
    \caption{ImageNet64 "lion"}
    \label{fig:im64_guided_1}
\end{figure*}



\begin{figure*}[b]
    \centering
    % First image
    \begin{subfigure}{0.25\textwidth}
        \includegraphics[width=\linewidth]{fig/appendix_im64_edm/edm_class_292_w=0.0.png}
        \caption{CFG-EDM (18 NFE), w=1.0}
    \end{subfigure}
    \begin{subfigure}{0.25\textwidth}
        \includegraphics[width=\linewidth]{fig/appendix_im64_edm/edm_class_292_w=6.0.png}
        \caption{CFG-EDM (18 NFE), w=7.0}
    \end{subfigure}
    \begin{subfigure}{0.25\textwidth}
        \includegraphics[width=\linewidth]{fig/appendix_im64_edm/edm_class_292_w=12.0.png}
        \caption{CFG-EDM (18 NFE), w=13.0}
    \end{subfigure}
    \begin{subfigure}{0.25\textwidth}
        \includegraphics[width=\linewidth]{fig/appendix_im64_igct/class_292_w=0.0.png}
        \caption{iGCT (2 NFE), w=1.0}
    \end{subfigure}
    \begin{subfigure}{0.25\textwidth}
        \includegraphics[width=\linewidth]{fig/appendix_im64_igct/class_292_w=6.0.png}
        \caption{iGCT (2 NFE), w=7.0}
    \end{subfigure}
    % Third image
    \begin{subfigure}{0.25\textwidth}
        \includegraphics[width=\linewidth]{fig/appendix_im64_igct/class_292_w=12.0.png}
        \caption{iGCT (2 NFE), w=13.0}
    \end{subfigure}
    \caption{ImageNet64 "tiger"}
    \label{fig:im64_guided_2}
\end{figure*}


\begin{figure*}[b]
    \centering
    % First image
    \begin{subfigure}{0.25\textwidth}
        \includegraphics[width=\linewidth]{fig/appendix_im64_edm/edm_class_28_w=0.0.png}
        \caption{CFG-EDM (18 NFE), w=1.0}
    \end{subfigure}
    \begin{subfigure}{0.25\textwidth}
        \includegraphics[width=\linewidth]{fig/appendix_im64_edm/edm_class_28_w=6.0.png}
        \caption{CFG-EDM (18 NFE), w=7.0}
    \end{subfigure}
    \begin{subfigure}{0.25\textwidth}
        \includegraphics[width=\linewidth]{fig/appendix_im64_edm/edm_class_28_w=12.0.png}
        \caption{CFG-EDM (18 NFE), w=13.0}
    \end{subfigure}
    \begin{subfigure}{0.25\textwidth}
        \includegraphics[width=\linewidth]{fig/appendix_im64_igct/class_28_w=0.0.png}
        \caption{iGCT (2 NFE), w=1.0}
    \end{subfigure}
    \begin{subfigure}{0.25\textwidth}
        \includegraphics[width=\linewidth]{fig/appendix_im64_igct/class_28_w=6.0.png}
        \caption{iGCT (2 NFE), w=7.0}
    \end{subfigure}
    % Third image
    \begin{subfigure}{0.25\textwidth}
        \includegraphics[width=\linewidth]{fig/appendix_im64_igct/class_28_w=12.0.png}
        \caption{iGCT (2 NFE), w=13.0}
    \end{subfigure}
    \caption{ImageNet64 "salamander"}
    \label{fig:im64_guided_3}
\end{figure*}


\begin{figure*}[b]
    \centering
    % First image
    \begin{subfigure}{0.25\textwidth}
        \includegraphics[width=\linewidth]{fig/appendix_im64_edm/edm_class_407_w=0.0.png}
        \caption{CFG-EDM (18 NFE), w=1.0}
    \end{subfigure}
    \begin{subfigure}{0.25\textwidth}
        \includegraphics[width=\linewidth]{fig/appendix_im64_edm/edm_class_407_w=6.0.png}
        \caption{CFG-EDM (18 NFE), w=7.0}
    \end{subfigure}
    \begin{subfigure}{0.25\textwidth}
        \includegraphics[width=\linewidth]{fig/appendix_im64_edm/edm_class_407_w=12.0.png}
        \caption{CFG-EDM (18 NFE), w=13.0}
    \end{subfigure}
    \begin{subfigure}{0.25\textwidth}
        \includegraphics[width=\linewidth]{fig/appendix_im64_igct/class_407_w=0.0.png}
        \caption{iGCT (2 NFE), w=1.0}
    \end{subfigure}
    \begin{subfigure}{0.25\textwidth}
        \includegraphics[width=\linewidth]{fig/appendix_im64_igct/class_407_w=6.0.png}
        \caption{iGCT (2 NFE), w=7.0}
    \end{subfigure}
    % Third image
    \begin{subfigure}{0.25\textwidth}
        \includegraphics[width=\linewidth]{fig/appendix_im64_igct/class_407_w=12.0.png}
        \caption{iGCT (2 NFE), w=13.0}
    \end{subfigure}
    \caption{ImageNet64 "ambulance"}
    \label{fig:im64_guided_4}
\end{figure*}

\end{document}
