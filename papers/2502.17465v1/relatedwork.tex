\section{Related Work}
Brain-to-speech and brain-to-text decoding related work can be identified by three main entities that they are capturing: namely; motor imagery based, overt speech based, and inner speech based. Several different BCI devices are available and exist respectively, and these can be broadly defined as the usage of Electroencephalography (EEG), Electrocorticography (ECoG), and functional Magnetic Resonance Imaging (fMRI) \cite{Panachakel2021}.

Motor imagery-based systems, such as for instance, point-and-click \cite{Jarosiewicz2015} and imaginary handwriting \cite{Willett2021}, have high accuracy but very slow typing rate. The simultaneous complementary event-related potential (ERP)-based P300 speller, steady state visually evoked potential (SSVEP), as well as code-modulated visual evoked potential (c-VEP) paradigms have been developed that employ neural signals in translating brain signals to text \cite{Lee2018}. The ERP system operates by monitoring neural activity in the brain that occurs in response to sensory events, while SSVEP and c-VEP paradigms exploit correlated neuronal potentials by means of visual evoked potentials, utilizing their different levels of alienation from the user in the course of information circulation, transmission of instant or in-/out-deck via operators or indicators and susceptibility to fatigue.

Decoding or synthesis of speech over speech activity which is real speech activity, this is considered as the overt speech-based method. The method is characterized by faster communication rates \cite{Makin2020} as compared to the existing modes. This technique necessitates the participation of subjects in vocal exercises during the neuro recording process \cite{Anumanchipalli2019} or the subjects have to perform the mental work of saying the sentence aloud \cite{Brigham2010}.

While developing such options, this makes the system itself become language-dependent. There are significant differences in pronunciations in various languages. Inner speech-based methods attempt to resolve the language articulation dependency by decoding language from the speculative speech and reading of text \cite{Defossez2023, Nieto2022}.

A significant limitation of the majority of the approaches under discussion is the restriction of using small closed vocabularies, with a low and limited number of unique words \cite{Brigham2010, Pandarinath2017}. In conclusion, the majority of the current communication through language approaches involve invasive devices such as (ECoG) \cite{Willett2021} or less accessible non-invasive devices like fMRI \cite{Nieto2022}.

This situation makes it more difficult to collect large datasets of the speech and put in place of methods to help paralyzed people who cannot speak anymore. However, the most recent research endeavors are trying to decode inner speech by opening the vocabulary and also using non-invasive technology \cite{Defossez2023, Nieto2022}.

Launching the possibility of similarly decoding studies of brain-to-text conversion of the inner voice. We study EEG signal representation learning, inter-subject variability, human judgment at the sentence level of generated sentences \cite{Wang2022}.