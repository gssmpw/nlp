\documentclass[conference]{IEEEtran}
\IEEEoverridecommandlockouts
% The preceding line is only needed to identify funding in the first footnote. If that is unneeded, please comment it out.
\usepackage{cite}
\usepackage{amsmath,amssymb,amsfonts}
\usepackage{algorithmic}
\usepackage{graphicx}
\usepackage{textcomp}
\usepackage{xcolor}
\usepackage[caption=false,font=footnotesize]{subfig}
\usepackage{float}
\usepackage{makecell}
% \usepackage{subcaption}
\usepackage{multirow}
\usepackage{hyperref}
\usepackage[flushleft]{threeparttable}


\def\BibTeX{{\rm B\kern-.05em{\sc i\kern-.025em b}\kern-.08em
    T\kern-.1667em\lower.7ex\hbox{E}\kern-.125emX}}

\DeclareRobustCommand{\IEEEauthorrefmark}[1]{\smash{\textsuperscript{\footnotesize #1}}}

    
\begin{document}

\title{Design of a Breakaway Utensil Attachment for Enhanced Safety in Robot-Assisted Feeding
\\
\thanks{
The research was conducted at the Future Health Technologies at the Singapore-ETH Centre, which was established collaboratively between ETH Zurich and the National Research Foundation Singapore. This research is supported by the National Research Foundation Singapore (NRF) under its Campus for Research Excellence and Technological Enterprise (CREATE) program.
}


\thanks{
\IEEEauthorrefmark{1}School of Mechanical and Aerospace Engineering, Nanyang Technological University Singapore
}
\thanks{
\IEEEauthorrefmark{2}Singapore-ETH Centre, Future Health Technologies Programme
}
\thanks{
Corresponding author's email: \texttt{janne.yow@ntu.edu.sg}}

% TODO: need to include MAE?
}

\author{
    \IEEEauthorblockN{
        Hau Wen Chang\IEEEauthorrefmark{1}, 
        J-Anne Yow\IEEEauthorrefmark{1,}\IEEEauthorrefmark{2}, 
        Lek Syn Lim\IEEEauthorrefmark{1}
        Wei Tech Ang\IEEEauthorrefmark{1,}\IEEEauthorrefmark{2}
    }
}

\maketitle

\begin{abstract}
Robot-assisted feeding systems enhance the independence of individuals with motor impairments and alleviate caregiver burden. While existing systems predominantly rely on software-based safety features to mitigate risks during unforeseen collisions, this study explores the use of a mechanical fail-safe to improve safety. We designed a breakaway utensil attachment that decouples forces exerted by the robot on the user when excessive forces occur. Finite element analysis (FEA) simulations were performed to predict failure points under various loading conditions, followed by experimental validation using 3D-printed attachments with variations in slot depth and wall loops. To facilitate testing, a drop test rig was developed and validated.
Our results demonstrated a consistent failure point at the slot of the attachment, with a slot depth of 1 mm and three wall loops achieving failure at the target force of 65 N. Additionally, the parameters can be tailored to customize the breakaway force based on user-specific factors, such as comfort and pain tolerance. CAD files and utensil assembly instructions can be found here: \url{https://tinyurl.com/rfa-utensil-attachment}

%This document is a model and instructions for \LaTeX.
%This and the IEEEtran.cls file define the components of your paper [title, text, heads, etc.]. *CRITICAL: Do Not Use Symbols, Special Characters, Footnotes, 
%or Math in Paper Title or Abstract.
%This document is a model and instructions for \LaTeX.
%This and the IEEEtran.cls file define the components of your paper [title, text, heads, etc.]. *CRITICAL: Do Not %Use Symbols, Special Characters, Footnotes, 
%or Math in Paper Title or Abstract.
%This document is a model and instructions for \LaTeX.
%This and the IEEEtran.cls file define the components of your paper [title, text, heads, etc.]. *CRITICAL: Do Not Use Symbols, Special Characters, Footnotes, 
%or Math in Paper Title or Abstract.This document is a model and instructions for \LaTeX.
%This and the IEEEtran.cls file define the components of your paper [title, text, heads, etc.]. *CRITICAL: Do Not Use Symbols, Special Characters, Footnotes, 
%or Math in Paper Title or Abstract.
\end{abstract}

\begin{IEEEkeywords}
Robot-Assisted Feeding, Assistive Robotics, Mechanical Fail-Safe
\end{IEEEkeywords}

%%%%%%%%%%%%%%%%%%%%%%%%%%%%%%%%%%%%%%%%%%%%%%%%%%%%%%%%%
% NOTE: PLEASE GO THROUGH THE HOWTO pdf BEFORE STARTING!
%%%%%%%%%%%%%%%%%%%%%%%%%%%%%%%%%%%%%%%%%%%%%%%%%%%%%%%%%

\section{Introduction}
Robot-assisted feeding systems offer a promising solution for improving the quality of life of those requiring feeding assistance and alleviating the burden on caregivers. Despite significant advancements in developing robust, intelligent, and personalized robot-assisted feeding systems \cite{park2020active, gordon2024adaptable}, a notable gap exists in designing mechanical fail-safes to ensure user safety. One of the critical safety concerns is the excessive force that can be applied during feeding due to the close physical human-robot interaction required for the task, which poses risks such as discomfort, injury, or damage to the user's oral cavity. 

To address this, designing robot systems that limit the force applied during interaction is paramount. Current research systems often focus on sophisticated control algorithms and sensor feedback mechanisms to reduce force \cite{shaikewitz2023mouth, jenamani2024feel}; however, mechanical fail-safes provide an additional layer of protection, particularly in unforeseen scenarios where software-controlled mechanisms may not respond. A mechanical solution offers a passive, reliable response to excessive force without relying on the system's active feedback loops.

This work presents a breakaway utensil attachment designed for robot-assisted feeding. The mechanism ensures that when the robotic arm applies excessive force on the user, the spoon automatically detaches, effectively preventing further force transmission to the user. By incorporating this passive fail-safe, we aim to enhance user safety and provide a safeguard against potential harm in unpredictable scenarios.

\section{Related Work}
\subsection{Utensil Attachments for Robot-Assisted Feeding}
Grasping a utensil directly using a two-finger gripper, commonly found on robot manipulators, is challenging. To address this, utensil attachments are typically employed to enable secure grasping by connecting utensils, such as spoons or forks, to structures that are easier for the robot to handle \cite{park2020active, gordon2024adaptable, belkhale2022balancing, shaikewitz2023mouth, jenamani2024feel}. Additionally, force-torque (FT) sensors are often integrated into these attachments to provide haptic feedback, enabling more precise control during feeding tasks. 

Recently, more innovative utensil designs have been proposed that do not take the form of a traditional spoon or fork -- Kiri-spoon \cite{keely2024kiri} uses a soft kirigami structure, allowing it to encapsulate and release food easily during bite acquisition. Separately, Gordon \textit{et. al.} \cite{gordon2024adaptable} presented their robot-assisted feeding system, which features a utensil attachment with a mechanically engineered weak point. This weak point is designed to break when excessive force is applied, eliminating any physical interaction between the robot and the user. However, the design specifics of this mechanism are not detailed, and there remains limited research exploring the integration of mechanical fail-safes in robot-assisted feeding systems.

% \subsection{Utensil Attachments for Robot-Assisted Feeding}
% Utensil attachments play a critical role in robot-assisted feeding systems by facilitating the acquisition and transfer of food to the user. Existing utensil attachment designs generally consist of a utensil attached directly to the interface of the FT sensor of the end effector \cite{park2020active, gordon2024adaptable, shaikewitz2023mouth, belkhale2022balancing, jenamani2024feel}. % The utensil and its interface can either be manufactured together as one combined attachment \cite{madan2022sparcs} or the utensil can be modified and attached to a 3D printed attachment to the FT sensor through some variation of mechanical attachments such as grub screws \cite{shaikewitz2023mouth} [TODO: Need to rephrase these better]. 


% However, none of these designs have incorporated a mechanical fail-safe in their utensil attachment to account for unforeseen accidental collisions with the users in the event of a technical malfunction. 

% Some existing designs claim to have a mechanical fail-safe engineered to break when subjected to forces above certain thresholds but are still robust enough to withstand forces needed for acquiring and transferring food \cite{gordon2024adaptable}. However, no details are provided on the part's design, and there is still limited research focused on the mechanical design of the utensil attachment and the factors influencing its fail-safe mechanism. 

% Before designing a utensil attachment, the design objectives and constraints must be clearly understood. 
% This utensil attachment should incorporate a mounting interface that must be compatible with the tool arm or be able to be grasped securely by the gripper. We choose to have a design that can be gripped securely and easily by the gripper to allow more flexibility and easier change of attachment for future use. 


\subsection{Safety in Assistive Robotics}
Safety in human-robot interaction is paramount, especially in assistive robotics, where physical human-robot interaction is common, and robots operate in close proximity to vulnerable users. Existing research on safety can be categorized into three main categories -- control methods, motion planning, and behavior prediction. Approaches relying on control methods typically use force-torque sensors, impedance control, or compliance-based control to detect and mitigate excessive forces during interaction \cite{shaikewitz2023mouth, jenamani2024feel}. Motion planning approaches aim to preemptively avoid collisions by generating safe trajectories based on real-time sensing and constraints \cite{lasota2014toward}. However, contact is unavoidable in physically assistive tasks such as feeding. For behavior prediction, the focus is on predicting human intent and motion \cite{lasota2015analyzing, zhao2019walking}, enabling the robot to adapt its actions proactively to prevent potentially unsafe interactions.

While these methods effectively reduce safety risks, they primarily rely on software-based solutions and AI-driven models, which can be limited by sensor inaccuracies, computational latency, and unanticipated failure modes. Furthermore, these approaches often depend on complex system integrations \cite{hamad2023concise} that may not respond adequately in every scenario. In contrast, our work introduces a safety feature based on mechanical principles, providing an additional safety layer in assistive robotic systems.

\begin{figure}[t]
    \centering
    \subfloat[]{%
        \includegraphics[height=5cm]{images/utensil_componenets_breakdown.png}
        \label{fig: utensil-attachment}
    }
    \hfill
    \subfloat[]{%
        \includegraphics[height=5cm]{images/ua_length.png}
        \label{fig: arm-with-utensil-attachment}
    }
    \caption{Utensil attachment with a mechanical failsafe (a) Components of our proposed utensil attachment (b) Utensil attachment grasped by a two-finger gripper}
\end{figure}


\section{Approach}
\subsection{Design Considerations and Constraints}

We designed a utensil attachment for the IKEA F\"{o}rnuft spoon, ensuring compatibility with any robot manipulator with a two-finger gripper. To secure the spoon, its handle was shortened and attached to the utensil holder using two M3 screws and nuts, as shown in Fig. \ref{fig: utensil-attachment}. The utensil and attachment's total length was limited to a maximum of 230mm to maintain the robot's range of motion during feeding and prevent interference with its movements of the surrounding environment (Fig. \ref{fig: arm-with-utensil-attachment}). 


In assistive feeding, excessive force can occur in two scenarios: (1) sudden impact collisions and (2) slow compressive forces. Sudden impact collisions, such as an unintended rapid movement of the arm striking the user, pose a greater hazard as they can deliver high, localized forces in a very short time frame, increasing the risk of injury. Thus, this work focuses on addressing sudden impact collisions. The mechanical fail-safe was optimized to break at an impact force of $F_{\text{impact}} = 65N$, based on the maximum permissible force that collaborative robots can safely exert on a human face, as defined by ISO 15066 \cite{ISO15066}, which details safety requirements for collaborative robots.
% addressing the more hazardous scenario of a sudden collision. This threshold was selected based on the maximum permissible force that collaborative robots can safely exert on a human face, as specified in ISO 15066 \cite{ISO15066}, which details safety requirements for collaborative robots.

Additionally, the utensil attachment must be robust enough to withstand forces encountered during normal feeding operations, such as acquiring food and transferring it to the user's mouth. To determine the minimum force threshold, we recorded force-torque data while scooping uncooked black beans, representing a high-force food scenario. To account for variability in food textures, we applied a safety factor of 2.5, resulting in a minimum force threshold of $F = 25N$. This balance between breakaway safety and functional durability is critical to the attachment's performance in real-world use.


% However, the utensil attachment should withstand forces from acquiring and transferring the food to the user's mouth. From our basic experimental testing, we found that the minimum forces the utensil attachment should withstand are 25N across all axis $F_{\text{x}} = F_{\text{y}} = F_{\text{z}} = 25N$ through scooping and bite transfer. The Cartesian coordinates of the forces are respective to the FT sensor and directions are shown in Fig. \ref{fig: arm-with-utensil-attachment}. 


\subsection{Breakaway Mechanism Design}
Standard breakaway mechanisms like shear pins, slip clutches, and magnetic couplings were initially considered but introduced bulk, complexity, and higher costs. Instead, we integrated the failure point directly into the 3D-printed shaft, eliminating additional components while maintaining a compact design. We achieved controlled failure by leveraging force moments and reducing the cross-sectional diameter, amplifying stress within the part. Localized stress concentration features, such as notches and slots, were introduced to create predictable failure points, ensuring a consistent breakaway force while maintaining a compact and lightweight design. 

% Standard breakaway mechanisms commonly used in robotics, such as shear pins, slip clutches, and magnetic couplings, were initially considered for the utensil attachment. However, each presented limitations in this application. Breakaway clutches, while capable of controlled mechanical failure, add weight, complexity, and cost due to their reliance on springs, ball bearings, or friction plates, making them bulkier and more challenging to manufacture compared to a simple 3D-printed breakaway shaft.

% % To minimize bulk and weight, we opted for a one-time-use design by integrating the failure point directly into the 3D-printed shaft rather than relying on separate components like shear pins. This approach simplifies assembly, reduces part count, and optimizes space utilization—an important consideration given the utensil attachment shaft’s diameter constraints. Furthermore, reusable mechanisms such as magnetic couplings introduce variability in disengagement force over time and require regular maintenance, leading to higher long-term costs.

% For the breakaway mechanism, we opted for a simpler approach to minimize bulk and weight rather than employing standard breakaway mechanisms commonly found in robotics, such as shear pins, slip clutches, or magnetic couplings. While conceptually similar to a shear pin, our design integrates the failure point directly into the 3D-printed shaft, eliminating the need for additional components and housings.
% We achieved controlled failure by leveraging force moments and reducing the cross-sectional diameter, amplifying stress within the part. Localized stress concentration features, such as notches and slots, were introduced to create predictable failure points, ensuring a consistent breakaway force while maintaining a compact and lightweight design.

%For the breakaway mechanism, we opted for a simpler approach to minimize bulk and weight rather than employing standard breakaway mechanisms commonly found in robotics, such as shear pins, slip clutches, or magnetic couplings. Instead, we amplified stress on the part by leveraging force moments and reducing the cross-sectional diameter.Localized stress concentration features, such as notches and slots, were introduced to create predictable failure points by focusing stress in specific regions. 

The reduced diameter \(d\) in these regions increases both torsional stress \(\tau\) and bending stress \(\sigma\), as described by:
\begin{equation}
    \tau = \frac{16T}{\pi d^3},
    \sigma = \frac{32M}{\pi d^3}
\end{equation}
where \(T\) is the applied torque, \(M\) is the bending moment, and 
\(d\) is the diameter of the shaft. These stresses combine to produce higher equivalent (Von-Mises) stresses \(\sigma_{e}\), which predict failure under complex loading conditions. Von-Mises stress is calculated as:
\begin{equation}
    \sigma_{e} = \sqrt{\sigma_{x}^2+\sigma_{y}^2+\sigma_{x}^2\sigma_{y}^2-3\tau_{xy}^2}
\end{equation}
where \(\sigma_{x}\) and \(\sigma_{y}\) are the normal stresses in the x and y directions, and \(\tau_{xy}\) is the shear stress in the x-y plane. Von-Mises stress is used in Section \ref{section FEA} to analyze how the stress is distributed under different loading conditions (Fig. \ref{fig: FEA}).

Guided by these principles, two breakaway mechanism designs were developed: (1) an L-shaped design with notches at the corners and (2) a U-shaped design with a slot in the horizontal section (Fig. \ref{fig: attachment-designs}). These features ensure failure occurs predictably at the intended points under applied forces.

\begin{figure}[t]
    \centering
    \subfloat[]{%
        \includegraphics[height=3.2cm]{images/l_shape_notch_radius.png}
        \label{fig: L-shape design}
    }
    \hfill
    \subfloat[]{%
        \includegraphics[height=3.2cm]{images/u_shape_slot_depth.png}
        \label{fig: U-shape design}
    }
    \caption{Utensil attachment designs -- (a) L-shaped design with notches at both corners (b) U-shaped design with a slot in the horizontal section}
    \label{fig: attachment-designs}
\end{figure}

\subsection{Finite Element Analysis (FEA)}
\label{section FEA}

Finite Element Analysis (FEA) was conducted using SolidWorks to evaluate the stress distribution and predict the failure points for each design under a compressive force of $F_{\text{compressive}}=65N$, corresponding to the ISO 15066 \cite{ISO15066} threshold. 
Regions with high-stress concentrations were identified as likely failure points under accidental impacts.

To simulate real-world conditions, FEA was performed with different ends of the part fixed, accounting for variations in how the utensil attachment might interact with the robot’s end-effector or contact the user. This provided a comprehensive evaluation of stress behavior. We focused on front-facing impacts, as they are the most common during feeding, with the robot and spoon approaching the user from the front. 
% While we initially tested lateral impacts in our simulations, our experimental testing focused solely on the more critical failure scenario—a head-on spoon collision—as it poses a greater risk to the user.

As the parts were designed for 3D printing, anisotropic strength properties resulting from the layer-by-layer construction process introduced challenges in accurately simulating their performance \cite{gibson2021additive, forster2015materials}. Factors such as print orientation, infill pattern, and wall thickness influence strength, making it difficult for FEA to provide exact predictions \cite{abbot2019finite}. Therefore, the FEA results were treated as qualitative guides for identifying potential failure points rather than definitive performance metrics.

\begin{figure}
    \centering
    \subfloat[]{%
        \includegraphics[height=4.5cm]{images/l_shape_fea.jpg}
        \label{fig: L-shape FEA}
    }
    \hfill
    \subfloat[]{%
        \includegraphics[height=4.5cm]{images/u_shape_fea.jpg}
        \label{fig: U-shape FEA}
    }
    \caption{Finite element analysis (FEA) on the L-shaped (left)o and U-shaped (right) design. (Top) FEA where the spoon end is fixed and (bottom) FEA where the opposite end is fixed.}
    \label{fig: FEA}
\end{figure}

FEA results on the L-shaped design revealed that stress concentrations were highest at the corners of the bend (Fig. \ref{fig: L-shape FEA}). However, the specific failure location varied based on which end of the part was fixed. This made predicting the exact failure location challenging, complicating systematic testing and parameter adjustments.

In contrast, the U-shaped design demonstrated consistent stress concentrations at the slot, resulting in a single predictable failure point (Fig. \ref{fig: U-shape FEA}). Given its predictability, the U-shaped design was ultimately selected. The slot depth \(d\) was identified as a key variable for systematically evaluating breakage forces.

% The U-shaped design was ultimately selected due to two primary factors: (1) greater predictability of failure points and (2) ease of parameter modification for systematic testing. 

% This observation is further validated by our experimental results in Section IV, as shown in Fig. \ref{fig: consistent failure}, where the part consistently fails around the slot depth. With only one primary failure point, the U-shaped design significantly simplifies the parameters to consider, enabling a more focused and comprehensive attachment design. Consequently, the design parameter we selected for variation is the slot depth $d$, as discussed in the Experimental Testing IV section. This approach allows us to determine the force required to break the attachment at varying slot depths.


% In our static study, the opposite end of the utensil attachment is assumed to be fixed. However, in real use, it will be connected to the robot arm’s end effector, which can shift in response to high forces, particularly if the software controls fail to detect the increased force.

% utilized similar breakaway mechanisms: (1) the notch radius for the L-shaped design and (2) the slot depth for the U-shaped design. These mechanisms operate because a smaller shaft diameter under identical loading conditions experiences higher normal and shear stresses. This allows for more predictable failure points in areas with reduced shaft diameters. (Consider including bending moment and shear stress formulas for torsion as proof, pending confirmation with J-Anne.) These localized stress concentrations enable us to understand better and predict how the attachment will fail under various loading conditions.

\subsection{Fabrication via 3D printing}
To facilitate rapid prototyping and iterative testing, the utensil attachments were 3D printed. Beyond varying the slot depth \(d\), 3D printing parameters such as wall thickness \(w\) and print orientation (Fig. \ref{fig: Bambu print orientation}) were adjusted to control the strength and performance of each part. A 15\%-density grid pattern was used as the infill for all printed parts. All parts were printed using the Bambu Lab X1C 3D printer (China) and the eSun PLA+ 1.75mm filament (China) to ensure uniformity across iterations and eliminate variability from printer inconsistencies.
% using eSun PLA+ filament, a material chosen for its balance of strength, flexibility, and ease of printing

Print orientation was critical due to the anisotropic strength of 3D-printed parts. Initially, parts were oriented with layer lines perpendicular to the intended breaking line to maximize strength \cite{kelecs2017effect, kovan2017effect}. However, this configuration resulted in inconsistent and unclean breaks (Fig. \ref{fig: Conventional print orientation}), making it more challenging to predict the failure behavior.

To address this, the layer lines were aligned parallel to the breaking line (Fig. \ref{fig: Unconventional print orientation}). This orientation leveraged the inherent layer-to-layer weakness of 3D printing, ensuring predictable failure at the desired location. The mechanical failure point resulting from a collision produces a clean break with smooth edges, minimizing the risk of injury to the patient. Wall thickness \(w\) was also varied to fine-tune the breakaway force, with thicker walls increasing strength and thinner walls reducing the force.

\begin{figure}[t]
    \centering
    \subfloat[]{%
        \includegraphics[height=5.5cm]{images/print_orientation.jpg}
        \label{fig: Bambu print orientation}
    }
    \hfill
    \subfloat[]{%
        \includegraphics[height=4.4cm]{images/Conventional_Print_Orientation.png}
        \label{fig: Conventional print orientation}
    }
    \hfill
    \subfloat[]{%
        \includegraphics[height=4.4cm]{images/Unconventional_Print_Orientation.png}
        \label{fig: Unconventional print orientation}
    }
    \caption{3D printing orientations -- (a) Visualization of two 3D printing orientations in the Bambu slicer: the left corresponds to (b), and the right corresponds to (c). (b) Print orientation with layer lines perpendicular to the breaking line for strength, resulting in an unclean break (c) Print orientation with layer lines parallel to the breaking line, producing a clean break}
    \label{fig: Print orientation}
\end{figure}

\section{Experiments}

\subsection{Experimental Setup and Design}
A drop test rig was developed to evaluate the impact strength of the utensil attachments (Fig. \ref{fig: collision rig updated}). The rig included a 20kg single-point load cell with an intrinsic resolution of 1/3000 of its full-scale range (SIWAREX WL260 SP-S SA, Siemens, Germany) to measure the impact forces. Data acquisition (DAQ) was facilitated using a 24-bit amplifier (DAT 1400, Pavone Systems, Italy), which amplified and conditioned the signals from the load cell. The system was calibrated using the Optimation software (Pavone Systems, Italy) with a scaling factor of 1V = 20N for voltage-to-force conversion. The amplifier's display resolution was set to \(\pm 0.01N\).

A motion capture system (Qualisys AB, Sweden) was employed to record the load cell readings and the kinematics of each drop, including position, velocity, and acceleration. Retro-reflective optical markers were placed on the basket of the test rig for precise motion tracking. The Qualisys Track Manager (QTM) v2019 was used as an integrated software interface for seamless and easy-to-use data recording, which we used to synchronize the cameras with the load cell. The trajectories of the markers and the load cell readings were captured synchronously at 200Hz and 2000Hz, respectively.

% The rig is based on the principles of drop testing, which evaluates the durability of objects subjected to impacts from a given height.

% To measure the impact forces acting on the utensil attachment, we employed a 20 kg off-center load cell capable of accurately capturing forces through its internal strain gauges. The load cell was connected to a DAT 1400 Amplifier and power supply, as depicted in Fig. \ref{fig: Amplifier and Power Supply fig}, to ensure precise data acquisition during testing.


% \begin{figure}
%     \centering
%     \subfloat[Custom drop test rig for breakaway mechanism experiments]{%
%         \includegraphics[height=9.0cm]{images/Collision testing jig.png}
%         \label{fig: collision rig}
%     }
%     \hfill
%     \subfloat[DAT 1400 amplifier and 24V power supply for collision testing rig]{%
%         \includegraphics[height=5.0cm]{images/DAT 1400 Amplifier and Power Supply.png}
%         \label{fig: Amplifier and Power Supply fig}
%     }
%     \caption{Drop test rig components and setup}
% \end{figure}

\begin{figure}
    \centering
        \includegraphics[width=0.6\linewidth]{images/Collision_testing_jig_clear.jpg}
    \caption{Drop test rig}
    \label{fig: collision rig updated}
\end{figure}


To ensure consistent alignment during testing, a 500mm linear guide rail with two linear guide blocks stabilized the weight basket. Grease was applied to the rail interface to reduce friction and ensure smooth motion. The weight basket was designed to hold varying masses, and the test parts were attached to it for drop testing. 

The drop height was controlled using a 3D-printed slot and a brass rod mechanism. The slot, mounted on the central aluminum profile, held the brass rod in place until manually released. 
% (Fig. \ref{fig: brass-rod-mechanism})
This setup ensured consistent and reliable impact force readings at each height.

% \begin{figure}
%     \centering
%     \includegraphics[height=5.0cm]{images/Brass rod mechanism.jpg}
%     \caption{Brass rod mechanism for drop test rig}
%     \label{fig: brass-rod-mechanism}
% \end{figure}


The rig's framework was constructed using 40 x 40mm aluminum profiles and secured with L-brackets to minimize structural damping during collisions. Additionally, two 10kg weights were placed on the rig to enhance stability and further reduce damping effects.

% Force readings were displayed using Qualysis software, while Motion Capture (MOCAP) markers were employed to generate position, velocity, and acceleration versus time graphs during free-fall and collision.

% To minimize the weight of the basket, hexagonal hole patterns were incorporated into its design to reduce filament usage without compromising structural integrity. A small 3D-printed component and a brass rod (material confirmation pending with Alex) attached to the central aluminum profile were used to control the height from which the basket was dropped. Consistently pulling the rod was critical to obtaining reliable impact force readings for each height.

% The rig's framework was constructed using 40 x 40 mm aluminum profiles and L-brackets to securely hold all components and minimize damping during collisions. Additionally, two 10 kg weights were placed on the rig to enhance stability and further reduce damping effects.

\subsection{Validation of Test Rig}
Before conducting our experiments, we conducted some validation checks. Static validation was performed by loading calibrated weights onto the load cell and verifying that the readings matched the known weights. The measured readings were consistent with the known weights, confirming the static accuracy of the load cell.
 % We tested 50 0g, 1kg and 2kg weights individually and in combinations. 

The actual impact force \(F_{actual}\) was determined using the peak voltage \(V_{peak}\) from the voltage-time graph recorded by the load cell. With a scaling factor of 1V/20N, the impact force was calculated as:

\begin{equation}
F_{\text{impact}}=(\frac{20N}{V})V_{\text{peak}}\label{eq}
\end{equation}

The theoretical impact force \(F_{theoretical}\) was calculated based on the principle of energy conservation, assuming no energy losses during impact. The kinetic energy of the weight basket and utensil attachment before impact \(KE\) was equated to the work done \(WD\) on the load cell:

\begin{equation*}
    KE = \frac{1}{2}mv^2, WD=Fd_{\text{stop}}, KE=WD
\end{equation*}

From these equations, the impact force was derived as:
\begin{equation}
    F_{theoretical}=\frac{mv^2}{2d_{\text{stop}}}
\end{equation}
where $m$ is the total mass of the weight basket and attachment, $v$ is max velocity before impact, and $d_{\text{stop}}$ is the stopping distance, approximated as the difference between the resting position \(p_{rest}\) and the lowest position \(p_{lowest}\), i.e. \(d_{\text{stop}} = p_{\text{rest}} - p_{\text{lowest}}\). The lowest position is determined by identifying the lowest point on the displacement vs. time graph, while the resting position is obtained from the initial state when the weight basket rests on the load cell.

\begin{table}[htbp]
\caption{Sample Values for Impact Force Validation}
\begin{center}
\begin{tabular}{|c|c|}
\hline
\textbf{Variable}                     & \textbf{Value} \\ \hline
Mass, $m$                                   & 0.735 kg       \\ \hline
Resting Position, $p_{\text{rest}}$      & 690.489 mm     \\ \hline
Lowest Position, $p_{\text{lowest}}$        & 687.429 mm     \\ \hline
Max Velocity, $v$                           & 860.634 mm/s   \\ \hline
Theoretical Impact Force, $F_{\text{theoretical}}$ & 89.0 N  \\ \hline
\hline
Peak Voltage, $V_{\text{peak}}$             & 3.78 V         \\ \hline
Actual Impact Force, $F_{\text{actual}}$    & 75.6 N         \\ \hline
\hline
Percentage Error, $e$                       & 17.7\%           \\ \hline
\end{tabular}
\label{tab: sample-calc}
\end{center}
\end{table}

% Below is the example calculation and the values used for the second trial of the Utensil Attachment, as shown in Table 1 below, featuring a 1 mm slot depth, printed with 6 wall loops, and dropped from a height of 6.0 cm:
The sample values used in the validation are summarized in Table \ref{tab: sample-calc}.
As expected, the theoretical impact force was greater than the experimental value due to the assumption of no energy losses in the theoretical calculation. In real-life experiments, energy losses occur due to factors such as sound generation and damping in the rig structure. 
The difference between the theoretical and actual impact forces of \(13.4 N\) corresponds to an error of \(17.7\%\), which is within the acceptable range, validating the rig for further testing.
% equivalent to approximately 1.36 kg under standard gravitational acceleration, $g = 9.81m/s^2$. Despite this discrepancy, the collision test rig passes the dynamic validation test, confirming it is reliable and safe to use for further testing.

\subsection{Experiment Procedure}
In our experiments, we evaluated the strength of the utensil attachment by varying two parameters: the slot depth \(d\) and the number of wall loops \(w\) used during 3D printing. To determine the minimum force required to break the attachment, we conducted drop tests by varying the height at which the weight basket and attachment were released onto the load cell. If the attachment did not break, we recorded the peak forces and incrementally increased the drop height by \(1.0cm\) until the attachment broke.

Once the approximate range of heights where the part transitioned between breaking and not breaking, we refined our testing by reducing the height increments to \(0.2cm\). Height increments were limited to \(0.2cm\) during this stage, as smaller increments of \(0.1cm\) were impractical to achieve consistently. For instance, if the part broke at \(h=5.0cm\) but did not break at \(h=4.0cm\), we tested intermediate heights such as \(h=4.8cm\) and \(h=4.6cm\). Starting from the highest height within the identified range, we conducted three trials at each height to ensure consistency and reliability of the results. 

If the part consistently broke at a specific height, we decreased the height incrementally and repeated the trials until the parts stopped breaking. The impact force was the primary peak force in the force-time graph for parts that did not break (Fig. \ref{fig: force-plot-unbroken}). However, for parts that broke, the impact force could not be derived from the graph due to the smaller first peak, as energy is dissipated through the breaking process rather than being fully transferred to the load cell (Fig. \ref{fig: force-plot-broken}). Thus, the breaking height was the highest height where the part did not consistently break across trials. For example, if the part consistently broke at \(h=4.6cm\) but started not breaking at \(h=4.4cm\), we considered \(h=4.4cm\) as the breaking height as shown in Table \ref{tab:impact-force}. 

Once the breaking height was determined, we calculated the average peak force recorded during the trials at that height, where some or all parts did not break, to estimate the breaking force. This approach provided a conservative estimate of the breaking force, reflecting the force just below the threshold of consistent failure. Hence, the actual impact force required to break the attachment is slightly higher than the average impact force recorded.

\begin{table}[htbp]
\caption{Peak Impact Forces for \(d=1.0mm\) and \(w=3\)}
\begin{center}
\begin{threeparttable}
    \begin{tabular}{|c|c|c|c|c|}
    \hline
    Height (cm) & $t_1$ (N) & $t_2$ (N) & $t_3$ (N) & Average (N) \\ \hline
    4.2 & 62.8 & 63.4 & 63.4 & 63.2 \\ \hline
    4.4 & N/A  & 65.0 & N/A  & 65.0 \\ \hline
    4.6 & N/A  & N/A  & N/A  & N/A  \\ \hline
    4.8 & N/A  & N/A  & N/A  & N/A  \\ \hline
    5.0 & N/A  & N/A  & N/A  & N/A  \\ \hline
    \end{tabular}
    \begin{tablenotes}
        \item{N/A indicates that part broke during testing}
    \end{tablenotes}
\end{threeparttable}
\end{center}
\label{tab:impact-force}
\end{table}



% For our systematic testing, we varied two parameters to evaluate their impact on the breakaway mechanism: (1) the slot depth $d$ and (2) the number of wall loops $w$ used in 3D printing the utensil attachment. To determine the minimum force required to break the part, we first varied the height at which the weight basket and attachment were dropped onto the load cell and recorded the peak forces if the attachment did not break.

% We began by identifying a height range where the attachment either breaks or does not break upon impact. For example, at $h=5.0cm$, the attachment does not break, but at $h=6.0cm$, the attachment breaks.

% Once the break/no-break height range was identified, we narrowed the height intervals to 0.2 cm and repeated the procedure. For instance, if at $h=5.8cm$, the attachment does not break, but at $h=6.0cm$, it does, we repeat the testing for each corresponding height for a total of three trials. The average impact force recorded across these trials provides an estimate of the force needed to break the attachment. Therefore, the actual impact force required to break the attachment can be assumed to be slightly higher than the average impact force recorded at the height where the attachment does not break.


% \begin{table}[htbp]
%     \caption{Effect of Slot Depth $d$ and Wall Loop $w$ on Impact Force}
%     \begin{center}
%     \begin{tabular}{|c|c|c|c|c|}
%     \hline
%     \thead{Slot Depth\\$d$ (mm)} & \thead{Wall Loops\\$w$} & \thead{Minimum\\Peak Force\\$F_{\text{min}}$ (N)} & \thead{Maximum\\Peak Force\\ $F_{\text{max}}$ (N)} & \thead{Average\\Peak Force\\ $F_{\text{avg}}$ (N)} \\ \hline
%     1.0 & 6 & 74.80 & 75.60 & 75.20 \\ \hline
%     1.0 & 3 & 63.20 & 65.00 & 64.10 \\ \hline
%     2.0 & 6 & 51.40 & 53.10 & 52.25 \\ \hline
%     2.0 & 3 & 42.07 & 45.00 & 43.53 \\ \hline
%     \end{tabular}
%     \label{tab:slot_depth_forces}
%     \end{center}
% \end{table}

\begin{figure}[htbp]
    \centering
    \subfloat[]{%
        \includegraphics[width=0.45\textwidth]{images/force_plot_combined.png}
        \label{fig: force-plot-unbroken}
    }
    \hfill
    % \subfloat[]{%
    %     \includegraphics[width=0.48\textwidth]{images/force_time_broken.png}
    %     \label{fig: force-plot-broken}
    % }
    \caption{Force-time graph for a part that does not break (blue solid line) and a part that breaks (red dashed line). The impact force is determined from the primary peak in the blue solid line, representing the maximum impact force experienced by the part. In contrast, for the red dashed line, the impact force cannot be derived due to the smaller first peak, which occurs as the part breaks upon impact.}
\end{figure}

\begin{table}[t]
    \caption{Effect of Slot Depth $d$ and Wall Loop $w$ on Impact Force}
    \begin{center}
    \begin{tabular}{|c|c|c|}
    \hline
    \thead{Slot Depth\\$d$ (mm)} & \thead{Wall Loops\\$w$} & \thead{Average\\Peak Force\\ $F_{\text{avg}}$ (N)} \\ \hline
    1.0 & 6 & 75.6 \\ \hline
    1.0 & 3 & 65.0 \\ \hline
    2.0 & 6 & 53.1 \\ \hline
    2.0 & 3 & 45.0 \\ \hline
    \end{tabular}
    \label{tab:slot_depth_forces}
    \end{center}
\end{table}


\begin{figure}[htbp]
    \centerline{\includegraphics[width=0.32\textwidth]{images/consistent_failure.jpg}}
    \caption{Consistent failure point at the slot of the breakaway mechanism}
    \label{fig: consistent failure}
\end{figure}

\section{Results}


The results from the experiments are summarized in Table \ref{tab:slot_depth_forces}. Based on the data, we observe that the impact force required to break the attachment increases with the slot depth \(d\) and decreases with the number of wall loops \(w\).

The configuration that produces an impact force closest to the maximum permissible force \(F_{\text{max}} = 65N\) is achieved when the attachment has a slot depth of \(d = 1mm\) and is printed with \(w = 3\) wall loops, which yielded an average impact force of \(F_{\text{impact}} = 65.0N\). This configuration is selected as the optimal design for the breakaway mechanism.

In scenarios where the maximum permissible force needs to be reduced -- for example, to accommodate lower pain thresholds or to prioritize enhanced safety -- the parameters can be refined to achieve the desired reduction. For instance, if a maximum permissible force of \(F_{\text{max}} = 50N\) is desired, the configuration with \(d=2.0mm\) and \(w=6\) wall loops, which produces an average impact force of \(F_{impact} = 53.1N\), could be adjusted by reducing the number of wall loops to lower the breaking force further.

These results demonstrate the utility of systematically varying slot depth and wall loop parameters to design a breakaway mechanism that can be tailored to meet specific force requirements. Furthermore, our experiments revealed that our design resulted in a consistent failure point, which is at the slot of the utensil attachment (Fig. \ref{fig: consistent failure}).


% From our results, we can deduce that the larger the slot depth $d$ of the attachment, the higher the impact force needed to break the attachment. Conversely, the number of wall loops $w$ and the impact force to break the attachment follow an inverse relationship, where the larger the number of wall loops $w$, the smaller the impact force needed to break the attachment. 

% The impact force that is closest to the maximum permissible force $F_{\text{max}} = 65N$, is when the attachment is printed using 3 wall loops and has a 1 mm slot depth as its breakaway mechanism, $F_{\text{impact}} = 64.1N$. This means we will use the $d = 1 mm$ and $w = 3$ attachment for our design. 

% However, in the scenario, none of the attachments give an average peak force that is close to $65.0 \pm 2.0N$ (Need to check on this with J-Anne, I just guess that this is an acceptable range), we can utilize the information gathered from the table and tweak the variables to achieve the desired impact force to break the attachment. 

% For example, using the $d = 2.0mm$ and $w = 6$ attachment in Table II; where the average impact force to break is $F_{\text{impact}}=52.25N$; we can increase the force required to break the attachment by increasing number of wall loops (to roughly 8 or 9) to achieve the desired force to break the attachment. 

\section{Conclusion and Future Work}

In conclusion, we designed a utensil attachment with a mechanical fail-safe that breaks around the maximum permissible force exerted on a human face by robots $F_{\text{max}} = 65N$. To ensure the part breaks at a consistent breaking point, the parts were printed with the layer lines parallel to the intended breaking line, allowing for controlled and predictable failure behavior. To evaluate the performance of the mechanism, we developed a drop test rig, simulating collisions between the robot arm and the user and conducted systematic experimental testing. Through our experiments, we established how design parameters such as slot depth \(d\) and wall loops \(w\) affect the impact force required to break the attachment.

While the results offer valuable insights into designing effective mechanical fail-safes, the study has some limitations. For instance, the testing procedure did not account for impacts at angles, which could affect the breaking behavior. The experiments were also conducted in a controlled setting using a drop test rig, which does not fully replicate real-world interactions between the utensil and user.

Future work should address these limitations by testing angled impacts and testing with the actual robotic arm and utensil to evaluate performance under real-world conditions. Additionally, other design and 3D printing parameters, such as infill density and infill pattern, could also be investigated to optimize the attachment further. Furthermore, determining the maximum permissible force based on user-specific factors, such as pain tolerance and comfort levels \cite{han2022assessment}, could further enhance the safety and customization of these attachments.

% The insights gained from this study provide a framework for designing utensil attachments with mechanical fail-safes that enhance safety. Furthermore, the maximum permissible force for robots can be tailored to meet the specific needs of patients, such as elderly users who may require lower maximum forces. By combining these findings with future research, researchers can develop more versatile and user-centric solutions that prioritize safety and comfort.

% Future work includes testing additional increments of slot depth and wall loop variations to evaluate their corresponding impact forces. This will help establish a clearer relationship between these variables and identify which factors have the most significant effect on the maximum impact force required to break the part. Additionally, other design and 3D printing parameters such as shaft thickness, infill density, and the infill pattern of the utensil attachment could also be tested to understand their influence on the maximum impact force needed to break the attachment.

% The insights gained from these experiments will assist researchers in designing and implementing utensil attachments with mechanical fail-safes that enhance safety. Moreover, the maximum permissible force for robots can be adjusted based on specific patient needs. For example, an elderly patient would require a lower maximum permissible force, allowing for the design of specialized utensil attachments tailored to their safety and comfort.


% Can include determining the maximum force better, such as through pain tolerance of users\cite{han2022assessment}
% Can also include limitation in scope of testing - eg didn't consider impacts at an angle

%BELOW PARTS ARE PARTS I PLAN TO REMOVE BUT KEEP FOR REFERENCE FOR NOW
%These attachments often incorporate a small Force-Torque (FT) sensor that monitors applied forces. Leveraging this sensor, a force-reactive controller was developed to safely accommodate the user's movements during the bite transfer process \cite{shaikewitz2023mouth}.

% Our current utensil attachment for the xArm-6 robotic arm (UFactory) utilizes a stainless steel spoon. The spoon is slotted into a 3D-printed attachment designed with a slot that closely matches the spoon’s cross-section as shown in Figure 1. The attachment is a cuboid, allowing the gripper to grip onto it before the robot begins feeding the user.

% \begin{figure}[htbp]
% \centerline{\includegraphics{originalSpoon.png}}
% \caption{Left Side: Original Utensil Attachment, Right Side: First Utensil Attachment Design Iteration.}
% \label{fig: sample-fig}
% \end{figure}

% However, this design has several limitations. First, the spoon does not fit snugly within the 3D-printed attachment, leading to potential shifts in the spoon's position during scooping or feeding. Additionally, the gripper struggles to grip the attachment consistently in the same position, complicating the robot’s ability to accurately plan the feeding path.

% To address these issues, our first design iteration involved 3D printing the spoon and attachment as a single piece as shown in Figure 1. This design included a notch radius as a breakaway mechanism, chosen for its simplicity in testing and variation during the experiments. We also varied the shaft thickness, finding that a 15mm diameter provided sufficient strength without breaking easily. However, this design introduced new challenges: (1) 3D printing a food-safe spoon with ergonomic considerations, (2) achieving a secure grip for the attachment, which still allowed movement, and (3) providing adequate strength against compressive forces in the x and z axes while remaining weak in the y-axis (along the shaft center line) — a critical direction to account for, given that collisions are most likely in the y-axis as shown in Figure 2. 

% \begin{figure}[htbp]
% \centerline{\includegraphics{Spoon Axis.png}}
% \caption{x-y-z axis of the utensil attachment}
% \label{fig: sample-fig}
% \end{figure}

%We added a moment arm to the utensil attachment in the second design iteration. This feature is intended to create a bending moment under a compressive force in the y direction. We also returned to using a stainless steel spoon, modifying it by shortening the handle and drilling two M3 holes to secure it to the attachment with M3 screws and nuts, as illustrated in Figure [TODO: Insert figure]. Additionally, we tested the placement and radius of the notch to evaluate its effect on controlled breakage.

%Despite improvements, adjustments were still to be made: the spoon should be left longer to prevent the user from biting the plastic attachment, and systematic testing was needed to measure the force threshold for attachment breakage, covered in Experiment IV.

%In subsequent design iterations, FEA indicated that the utensil attachment had two potential failure points. Basic experimental testing confirmed this, as applying a constant compressive force along the y-axis caused the part to break at two distinct points. Ideally, a single, predictable failure point would be preferable to control breakage during accidental user collisions. We also found that the 3D printing orientation affected breakage consistency due to the material strength variations in FDM-printed parts, which are stronger against layer lines than along them. A failure point along the layer lines is beneficial for predictability and breakage consistency [TODO: Insert diagram explaining concept].

%We adopted a U-shaped utensil attachment for the final design, which FEA suggests is more likely to fail at a single point under compressive y-axis forces. Given the complexity of factors impacting breakage, we identified two critical variables: (1) slot depth at the failure point and (2) the number of wall loops in the 3D print. These parameters will be varied in experimental testing to determine their effects on the attachment's breakage characteristics.

%A custom collision testing rig was designed to evaluate the strength of our breakaway mechanisms during potential accidental collisions with the user. The initial design featured a small weight basket that would slide along parallel aluminum profiles, impacting a securely attached utensil attachment connected to an RS Pro Force Gauge RS232, as shown in Figure [TODO: Insert figure of collision testing jig v2 SolidWorks].

%However, several limitations arose with this design: (1) the slider mechanism required a high level of precision for the basket to slide smoothly, (2) the weight basket itself was too heavy and occasionally broke upon impact, and (3) there was no way to consistently drop the basket from a fixed height.

%To address these issues, the next iteration of the rig incorporated a single linear guide rail mounted to an aluminum profile, with two guide blocks attached to a lighter-weight basket. This setup allowed for smoother, frictionless motion along the rail. Hexagonal pattern holes were also added to the basket to reduce filament use while maintaining structural integrity. A small shaft mechanism was included to lock the basket at a consistent height before testing, allowing the user to release it by pulling the shaft. Additional aluminum bars were added to stabilize the force gauge, minimizing bounce during impact, which had previously caused inaccurate readings.

%Despite these improvements, we discovered that our force gauge, with a 100 Hz sampling rate, could not reliably capture peak impact force. Therefore, we switched to a 20 kg load cell with a 2000 Hz sampling rate for higher accuracy. This load cell connects to a DAT 1400 amplifier and power supply, as shown in Figure [TODO: Insert schematic after confirming with Alex]. Force readings are displayed via the Qualysis software in the Motion Capture (MOCAP) Lab. Additional MOCAP markers are also placed onto the weight basket to allow us to determine the position, velocity and acceleration of the basket during free fall and upon impact with the load cell. The MOCAP system has a sampling rate of 200Hz. 

%We also reoriented the utensil attachment under the weight basket to better simulate the attachment moving toward the user's face, providing a more realistic impact scenario than the previous stationary setup.

% PD ISO/TS 15066:2016 BSI Standards Publication for collaborative robots (TODO: Know how to cite this in with BibTex). 

% Our design also considered the payload limitations of the Lite-6 robotic arm (UFactory), which has a maximum payload of 600g, including the mass of the gripper. In addition to weight constraints, 

% Furthermore, our utensil attachment weight capacity should not exceed the payload of the Lite6 robot arm which is less than 600g - including the mass of the Lite6 gripper [TODO: Cite Lite6 technical manual]. This is because Lite6 has the lower payload compared the xArm6 making it a limiting factor. Other than weight restrictions, the utensil attachment's size should not impede the range of motion of the robot arm during the feeding process. We restricted the maximum length of the utensil attachment from tip of spoon to end of gripper attachment 120 mm (Check actual length with J-Anne as cannot estimate in CAD without spoon). This is because having an overly long spoon can obstruct the camera's field of vision and interfere with the robot's arm trajectory. 

% There are two types of impacts the user can experience with the utensil attachment in the event of a robot technical malfunction (1) sudden impact collision (2) slow compressive force. For our mechanical safe we decided to optimize it to break at an impact force of $F_{\text{impact}}$ = 65N in the event of a sudden impact collision because this is a more dangerous scenario in the event of a robot technical malfunction. $F_{\text{impact}}$ = 65N was chosen as the impact force threshold as it was listed as the maximum permissible force a robot can safely exert on the human face according to the PD ISO/TS 15066:2016 BSI Standards Publication for collaborative robots (TODO: Know how to cite this in with BibTex). 

% To streamline production and support consecutive testing, the utensil attachment prototypes were 3D printed using eSun PLA+ filament. Beyond adjusting design iterations as outlined in the previous section, varying 3D printing parameters such as wall loops, infill, and print orientation allow us to control the strength of each attachment. This approach enables us to establish a clear relationship between these parameters and the force required to break the attachment. For consistency, all parts were printed using the Bambu X1 printer.

% \section*{Acknowledgment}

% The preferred spelling of the word ``acknowledgment'' in America is without 
% an ``e'' after the ``g''. Avoid the stilted expression ``one of us (R. B. 
% G.) thanks $\ldots$''. Instead, try ``R. B. G. thanks$\ldots$''. Put sponsor 
% acknowledgments in the unnumbered footnote on the first page.


\bibliographystyle{IEEEtran}
\bibliography{IEEEabrv, references}

% % TO DELETE AFTER ------------------------------------
% \newpage
% \section{Sample Eqn, Fig, Tables}
% \subsection{Equations}
% Number equations consecutively. To make your 
% equations more compact, you may use the solidus (~/~), the exp function, or 
% appropriate exponents. Italicize Roman symbols for quantities and variables, 
% but not Greek symbols. Use a long dash rather than a hyphen for a minus 
% sign. Punctuate equations with commas or periods when they are part of a 
% sentence, as in:
% \begin{equation}
% a+b=\gamma\label{eq}
% \end{equation}

% Be sure that the 
% symbols in your equation have been defined before or immediately following 
% the equation. Use ``\eqref{eq}'', not ``Eq.~\eqref{eq}'' or ``equation \eqref{eq}'', except at 
% the beginning of a sentence: ``Equation \eqref{eq} is . . .''

% \subsection{Figures and Tables}
% \paragraph{Positioning Figures and Tables} Place figures and tables at the top and 
% bottom of columns. Avoid placing them in the middle of columns. Large 
% figures and tables may span across both columns. Figure captions should be 
% below the figures; table heads should appear above the tables. Insert 
% figures and tables after they are cited in the text. Use the abbreviation 
% ``Fig.~\ref{fig: sample-fig}'', even at the beginning of a sentence.

% \begin{table}[htbp]
% \caption{Table Type Styles}
% \begin{center}
% \begin{tabular}{|c|c|c|c|}
% \hline
% \textbf{Table}&\multicolumn{3}{|c|}{\textbf{Table Column Head}} \\
% \cline{2-4} 
% \textbf{Head} & \textbf{\textit{Table column subhead}}& \textbf{\textit{Subhead}}& \textbf{\textit{Subhead}} \\
% \hline
% copy& More table copy$^{\mathrm{a}}$& &  \\
% \hline
% \multicolumn{4}{l}{$^{\mathrm{a}}$Sample of a Table footnote.}
% \end{tabular}
% \label{tab1}
% \end{center}
% \end{table}

% \begin{figure}[htbp]
% \centerline{\includegraphics{fig1.png}}
% \caption{Example of a figure caption.}
% \label{fig: sample-fig}
% \end{figure}

% Figure Labels: Use 8 point Times New Roman for Figure labels. Use words 
% rather than symbols or abbreviations when writing Figure axis labels to 
% avoid confusing the reader. As an example, write the quantity 
% ``Magnetization'', or ``Magnetization, M'', not just ``M''. If including 
% units in the label, present them within parentheses. Do not label axes only 
% with units. In the example, write ``Magnetization (A/m)'' or ``Magnetization 
% \{A[m(1)]\}'', not just ``A/m''. Do not label axes with a ratio of 
% quantities and units. For example, write ``Temperature (K)'', not 
% ``Temperature/K''.

% % TO DELETE AFTER ------------------------------------

\end{document}
