\section{Related Work}
\subsection{Utensil Attachments for Robot-Assisted Feeding}
Grasping a utensil directly using a two-finger gripper, commonly found on robot manipulators, is challenging. To address this, utensil attachments are typically employed to enable secure grasping by connecting utensils, such as spoons or forks, to structures that are easier for the robot to handle ____. Additionally, force-torque (FT) sensors are often integrated into these attachments to provide haptic feedback, enabling more precise control during feeding tasks. 

Recently, more innovative utensil designs have been proposed that do not take the form of a traditional spoon or fork -- Kiri-spoon ____ uses a soft kirigami structure, allowing it to encapsulate and release food easily during bite acquisition. Separately, Gordon \textit{et. al.} ____ presented their robot-assisted feeding system, which features a utensil attachment with a mechanically engineered weak point. This weak point is designed to break when excessive force is applied, eliminating any physical interaction between the robot and the user. However, the design specifics of this mechanism are not detailed, and there remains limited research exploring the integration of mechanical fail-safes in robot-assisted feeding systems.

% \subsection{Utensil Attachments for Robot-Assisted Feeding}
% Utensil attachments play a critical role in robot-assisted feeding systems by facilitating the acquisition and transfer of food to the user. Existing utensil attachment designs generally consist of a utensil attached directly to the interface of the FT sensor of the end effector ____. % The utensil and its interface can either be manufactured together as one combined attachment ____ or the utensil can be modified and attached to a 3D printed attachment to the FT sensor through some variation of mechanical attachments such as grub screws ____ [TODO: Need to rephrase these better]. 


% However, none of these designs have incorporated a mechanical fail-safe in their utensil attachment to account for unforeseen accidental collisions with the users in the event of a technical malfunction. 

% Some existing designs claim to have a mechanical fail-safe engineered to break when subjected to forces above certain thresholds but are still robust enough to withstand forces needed for acquiring and transferring food ____. However, no details are provided on the part's design, and there is still limited research focused on the mechanical design of the utensil attachment and the factors influencing its fail-safe mechanism. 

% Before designing a utensil attachment, the design objectives and constraints must be clearly understood. 
% This utensil attachment should incorporate a mounting interface that must be compatible with the tool arm or be able to be grasped securely by the gripper. We choose to have a design that can be gripped securely and easily by the gripper to allow more flexibility and easier change of attachment for future use. 


\subsection{Safety in Assistive Robotics}
Safety in human-robot interaction is paramount, especially in assistive robotics, where physical human-robot interaction is common, and robots operate in close proximity to vulnerable users. Existing research on safety can be categorized into three main categories -- control methods, motion planning, and behavior prediction. Approaches relying on control methods typically use force-torque sensors, impedance control, or compliance-based control to detect and mitigate excessive forces during interaction ____. Motion planning approaches aim to preemptively avoid collisions by generating safe trajectories based on real-time sensing and constraints ____. However, contact is unavoidable in physically assistive tasks such as feeding. For behavior prediction, the focus is on predicting human intent and motion ____, enabling the robot to adapt its actions proactively to prevent potentially unsafe interactions.

While these methods effectively reduce safety risks, they primarily rely on software-based solutions and AI-driven models, which can be limited by sensor inaccuracies, computational latency, and unanticipated failure modes. Furthermore, these approaches often depend on complex system integrations ____ that may not respond adequately in every scenario. In contrast, our work introduces a safety feature based on mechanical principles, providing an additional safety layer in assistive robotic systems.

\begin{figure}[t]
    \centering
    \subfloat[]{%
        \includegraphics[height=5cm]{images/utensil_componenets_breakdown.png}
        \label{fig: utensil-attachment}
    }
    \hfill
    \subfloat[]{%
        \includegraphics[height=5cm]{images/ua_length.png}
        \label{fig: arm-with-utensil-attachment}
    }
    \caption{Utensil attachment with a mechanical failsafe (a) Components of our proposed utensil attachment (b) Utensil attachment grasped by a two-finger gripper}
\end{figure}