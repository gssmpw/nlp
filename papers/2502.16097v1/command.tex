% \usepackage{xargs} % Used for new commands with optional arguments
\usepackage{soul}  % Used for custom comments
\usepackage{color} % Used for custom colors in comments
% \usepackage{comment}
\usepackage{xspace}
\usepackage{listings}
\usepackage{enumitem}
% \usepackage{ulem}
%% Note: Some commands for spacing Latin letters/abbreviations
\newcommand{\eg}{{\it e.g.,\ }}
\newcommand{\etal}{{\it et al.\ }}
\newcommand{\etc}{{\it etc.}}
\newcommand{\ie}{{\it i.e.,\ }}
\newcommand{\cf}{{c.f.}\xspace}
\newcommand{\aka}{{a.k.a.}\xspace}

%%%%%%%%%%%%%%%%%%%%%%%%%%%%%%%%%%%%%%%%%%%%%%%%
%% Commands for adding comments to the paper. %%
%%%%%%%%%%%%%%%%%%%%%%%%%%%%%%%%%%%%%%%%%%%%%%%%

\usepackage{booktabs}
\definecolor{oxfordblue}{rgb}{0.0, 0.13, 0.28}
\definecolor{harvardcrimson}{rgb}{0.79, 0.0, 0.09}
\definecolor{dartmouthgreen}{rgb}{0.05, 0.5, 0.06}
\definecolor{princetonorange}{rgb}{1.0, 0.56, 0.0}
\definecolor{yaleblue}{rgb}{0.06, 0.3, 0.57}
\definecolor{usccardinal}{rgb}{0.6, 0.0, 0.0}
\definecolor{uclablue}{rgb}{0.33, 0.41, 0.58}
\definecolor{msugreen}{rgb}{0.09, 0.27, 0.23}
\definecolor{cornellred}{rgb}{0.7, 0.11, 0.11}
\definecolor{pomegranate}{RGB}{192, 57, 43}
\definecolor{anti-pomegranate}{RGB}{43,178,192}
\definecolor{alizarin}{RGB}{231, 76, 60}
\definecolor{anti-belize}{RGB}{185, 41, 56}
\definecolor{belize}{RGB}{41, 128, 185}
\definecolor{peter}{RGB}{52, 152, 219}
\definecolor{green}{RGB}{22, 160, 133}
\definecolor{anti-green}{RGB}{160,22,118}
\definecolor{turquoise}{RGB}{26, 188, 156}
\definecolor{pumpkin}{RGB}{211, 84, 0}
\definecolor{anti-pumpkin}{RGB}{0,22,211}
\definecolor{carrot}{RGB}{230, 126, 34}
\definecolor{wisteria}{RGB}{142, 68, 173}
\definecolor{anti-wisteria}{RGB}{99,173,68}
\definecolor{amethyst}{RGB}{155, 89, 182}
\definecolor{nephritis}{RGB}{39, 174, 96}
\definecolor{anti-nephritis}{RGB}{174,39,117}

% \newcommand{\pzh}[1]{{#1}}
% \newcommand{\peng}[1]{{\color{red} #1}}
% \newcommand{\xingbo}[1]{{\textcolor{black}{#1}}}
% \newcommand{\wxb}[1]{{\textcolor{orange}{#1}}}
% \newcommand{\chen}[1]{{\color{green} #1}}

\newcommand{\penguin}[1]{{#1}}
\newcommand{\pzh}[1]{{#1}}
\newcommand{\peng}[1]{{#1}}
\newcommand{\zhenhui}[1]{{#1}}
\newcommand{\haoxiang}[1]{{#1}}
\newcommand{\yh}[1]{{#1}}
\newcommand{\fhx}[1]{{{#1}}}

% \newcommand{\penguin}[1]{{\color{red} #1}}
% \newcommand{\fhx}[1]{{\textcolor{orange}{#1}}}
% \newcommand{\zhenhui}[1]{{\color{carrot} #1}}

% \newcommand{\penguin}[1]{{\color{blue} #1}}
% \newcommand{\fhx}[1]{{\color{blue}{#1}}}
% \newcommand{\zhenhui}[1]{{\color{blue} #1}}

% \newcommand{\fanhx}[1]{{\color{blue}{#1}}}
\newcommand{\fanhx}[1]{{{#1}}}

%% Note: Comment this in to see all comments and unfinished text.
\newcommand{\todo}[1]{\textcolor{red}{[TODO] \emph{#1}}}
\newcommand{\cut}[1]{\textcolor{red}{\st{#1}}}
\newcommand{\sout}[1]{\cut{#1}}
\newcommand{\gray}[1]{\textcolor{gray}{#1}}


\newcommand{\systemname}{{\textit{SystemName}}}
\newcommand{\name}{{\textit{LitLinker}}}
    
% Capitalizing the first letter for section autorefs
\renewcommand{\sectionautorefname}{Section}
\renewcommand{\subsectionautorefname}{Section}
\renewcommand{\subsubsectionautorefname}{Section}

