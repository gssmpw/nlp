%%
%% This is file `sample-sigconf-authordraft.tex',
%% generated with the docstrip utility.
%%
%% The original source files were:
%%
%% samples.dtx  (with options: `all,proceedings,bibtex,authordraft')
%% 
%% IMPORTANT NOTICE:
%% 
%% For the copyright see the source file.
%% 
%% Any modified versions of this file must be renamed
%% with new filenames distinct from sample-sigconf-authordraft.tex.
%% 
%% For distribution of the original source see the terms
%% for copying and modification in the file samples.dtx.
%% 
%% This generated file may be distributed as long as the
%% original source files, as listed above, are part of the
%% same distribution. (The sources need not necessarily be
%% in the same archive or directory.)
%%
%%
%% Commands for TeXCount
%TC:macro \cite [option:text,text]
%TC:macro \citep [option:text,text]
%TC:macro \citet [option:text,text]
%TC:envir table 0 1
%TC:envir table* 0 1
%TC:envir tabular [ignore] word
%TC:envir displaymath 0 word
%TC:envir math 0 word
%TC:envir comment 0 0
%%
%%
%% The first command in your LaTeX source must be the \documentclass
%% command.
%%
%% For submission and review of your manuscript please change the
%% command to \documentclass[manuscript, screen, review]{acmart}.
%%
%% When submitting camera ready or to TAPS, please change the command
%% to \documentclass[sigconf]{acmart} or whichever template is required
%% for your publication.
%%
%%
% \documentclass[manuscript,review,anonymous]{acmart}
\documentclass[sigconf]{acmart}

%%
%% \BibTeX command to typeset BibTeX logo in the docs
\AtBeginDocument{%
  \providecommand\BibTeX{{%
    Bib\TeX}}}

%% Rights management information.  This information is sent to you
%% when you complete the rights form.  These commands have SAMPLE
%% values in them; it is your responsibility as an author to replace
%% the commands and values with those provided to you when you
%% complete the rights form.
% \setcopyright{acmlicensed}
% \copyrightyear{2018}
% \acmYear{2018}
% \acmDOI{XXXXXXX.XXXXXXX}

%% These commands are for a PROCEEDINGS abstract or paper.
% \acmConference[Conference acronym 'XX]{Make sure to enter the correct
%   conference title from your rights confirmation emai}{June 03--05,
%   2018}{Woodstock, NY}
%%
%%  Uncomment \acmBooktitle if the title of the proceedings is different
%%  from ``Proceedings of ...''!
%%
%%\acmBooktitle{Woodstock '18: ACM Symposium on Neural Gaze Detection,
%%  June 03--05, 2018, Woodstock, NY}
% \acmISBN{978-1-4503-XXXX-X/18/06}


%%
%% Submission ID.
%% Use this when submitting an article to a sponsored event. You'll
%% receive a unique submission ID from the organizers
%% of the event, and this ID should be used as the parameter to this command.
%%\acmSubmissionID{123-A56-BU3}

%%
%% For managing citations, it is recommended to use bibliography
%% files in BibTeX format.
%%
%% You can then either use BibTeX with the ACM-Reference-Format style,
%% or BibLaTeX with the acmnumeric or acmauthoryear sytles, that include
%% support for advanced citation of software artefact from the
%% biblatex-software package, also separately available on CTAN.
%%
%% Look at the sample-*-biblatex.tex files for templates showcasing
%% the biblatex styles.
%%

%%
%% The majority of ACM publications use numbered citations and
%% references.  The command \citestyle{authoryear} switches to the
%% "author year" style.
%%
%% If you are preparing content for an event
%% sponsored by ACM SIGGRAPH, you must use the "author year" style of
%% citations and references.
%% Uncommenting
%% the next command will enable that style.
%%\citestyle{acmauthoryear}

%!TEX root=main.tex
\newif\ifspacehack
%\spacehacktrue
\usepackage{natbib}
\hypersetup{
    colorlinks = blue,
    breaklinks,
    linkcolor = blue,
    citecolor = blue,
    urlcolor  = blue,
}
\usepackage{url} 
\usepackage{graphicx}
\usepackage{mathtools}
\usepackage{footnote}
\usepackage{float}
\usepackage{xspace}
\usepackage{multirow}
\usepackage{xcolor}
\usepackage{wrapfig}
\usepackage{framed}
\usepackage{bbm}
\usepackage[most]{tcolorbox}

\usepackage{footnote}
\usepackage{nicefrac}
\usepackage{makecell}
\usepackage[ruled,vlined]{algorithm2e}
\usepackage{amssymb}
\usepackage{bm}
\makesavenoteenv{tabular}
\makesavenoteenv{table}

\newcommand{\hl}[1]{{\color{red}[HL: #1]}}
\newcommand{\mznote}[1]{{\color{blue}[MZ: #1]}}

% macros@Peng
\newcommand\innerp[2]{\langle #1, #2 \rangle}
\renewcommand{\tilde}{\widetilde}
\renewcommand{\hat}{\widehat}


\newcommand{\TVD}[1]{\norm{#1}_\text{TV}}
\newcommand{\corral}{\textsc{Corral}\xspace}
\newcommand{\expthree}{\textsc{Exp3}\xspace}
\newcommand{\expfour}{\ensuremath{\mathsf{Exp4}}\xspace}
\newcommand{\expthreeP}{\textsc{Exp3.P}\xspace}
\newcommand{\scrible}{\textsc{SCRiBLe}\xspace}

\def \R {\mathbb{R}}
\newcommand{\eps}{\epsilon}
\newcommand{\vecc}{\mathrm{vec}}
\newcommand{\LS}{\mathrm{LS}}
\newcommand{\FG}{\mathrm{FG}}
\newcommand{\DL}{\Delta \ellhat}
\newcommand{\calA}{{\mathcal{A}}}
\newcommand{\smax}{{\mathrm{smax}}}
\newcommand{\calB}{{\mathcal{B}}}
\newcommand{\calX}{{\mathcal{X}}}
\newcommand{\calS}{{\mathcal{S}}}
\newcommand{\calF}{{\mathcal{F}}}
\newcommand{\calI}{{\mathcal{I}}}
\newcommand{\calJ}{{\mathcal{J}}}
\newcommand{\calK}{{\mathcal{K}}}
\newcommand{\calH}{{\mathcal{H}}}
\newcommand{\calD}{{\mathcal{D}}}
\newcommand{\calE}{{\mathcal{E}}}
\newcommand{\calG}{{\mathcal{G}}}
\newcommand{\calU}{{\mathcal{U}}}
\newcommand{\calR}{{\mathcal{R}}}
\newcommand{\calT}{{\mathcal{T}}}
\newcommand{\calP}{{\mathcal{P}}}
\newcommand{\calQ}{{\mathcal{Q}}}
\newcommand{\calZ}{{\mathcal{Z}}}
\newcommand{\calM}{{\mathcal{M}}}
\newcommand{\calN}{{\mathcal{N}}}
\newcommand{\calW}{{\mathcal{W}}}
\newcommand{\calY}{{\mathcal{Y}}}
\newcommand{\cD}{{\mathcal{D}_{\mathcal{X}}}}
\newcommand{\mcD}{{\mathcal{D}}}
\newcommand{\cF}{{\mathcal{F}}}
\newcommand{\cA}{{\mathcal{A}}}
\newcommand{\cX}{{\mathcal{X}}}
\newcommand{\cE}{{\mathcal{E}}}
\newcommand{\cV}{{\mathcal{V}}}
\newcommand{\cR}{{\mathcal{R}}}
\newcommand{\wcR}{\widehat{\mathcal{R}}}
\newcommand{\Reg}{{\mathrm{Reg}}}
\newcommand{\Alg}{{\mathsf{Alg}}}
\newcommand{\wReg}{\widehat{\mathrm{Reg}}}
\newcommand{\cB}{\mathcal{B}}
\newcommand{\cP}{\mathcal{P}}
\newcommand{\nctx}{\text{n-ctx}}
\newcommand{\ctx}{\text{ctx}}
\newcommand{\E}{{\mathbb{E}}}
\newcommand{\V}{\mathbb{V}}
\newcommand{\Prob}{\mathbb{P}}
\newcommand{\1}{\mathbb{I}}
\newcommand{\N}{\mathbb{N}}
\newcommand{\tup}[1]{t^{(#1)}}
\newcommand{\gup}[1]{g^{(#1)}}
\newcommand{\hatfm}{\widehat{f}_m}
\newcommand{\haty}{\widehat{y}}
\newcommand{\hatx}{\widehat{x}}
\newcommand{\yhat}{\widehat{y}}
\newcommand{\xhat}{\widehat{x}}
\newcommand{\fhat}{\widehat{f}}
\newcommand{\ghat}{\widehat{g}}

\newcommand{\inner}[1]{ \left\langle {#1} \right\rangle }
\newcommand{\ind}{\mathbb{I}}
\newcommand{\diag}{\textrm{diag}}
\newcommand{\Nout}{N_{\textrm{out}}}
\newcommand{\nout}{N_{\textrm{out}}}
\newcommand{\Nin}{{\textrm{Nin}}}
\newcommand{\nin}{{\textrm{Nin}}}
\newcommand{\order}{\mathcal{O}}


\newcommand{\Acal}{\mathcal{A}}
\newcommand{\Bcal}{\mathcal{B}}
\newcommand{\Ccal}{\mathcal{C}}
\newcommand{\Dcal}{\mathcal{D}}
\newcommand{\Ecal}{\mathcal{E}}
\newcommand{\Fcal}{\mathcal{F}}
\newcommand{\Gcal}{\mathcal{G}}
\newcommand{\Hcal}{\mathcal{H}}
\newcommand{\Ical}{\mathcal{I}}
\newcommand{\Jcal}{\mathcal{J}}
\newcommand{\Kcal}{\mathcal{K}}
\newcommand{\Lcal}{\mathcal{L}}
\newcommand{\Mcal}{\mathcal{M}}
\newcommand{\Ncal}{\mathcal{N}}
\newcommand{\Ocal}{\mathcal{O}}
\newcommand{\Pcal}{\mathcal{P}}
\newcommand{\Qcal}{\mathcal{Q}}
\newcommand{\Rcal}{\mathcal{R}}
\newcommand{\Scal}{\mathcal{S}}
\newcommand{\Tcal}{\mathcal{T}}
\newcommand{\Ucal}{\mathcal{U}}
\newcommand{\Vcal}{\mathcal{V}}
\newcommand{\Wcal}{\mathcal{W}}
\newcommand{\Xcal}{\mathcal{X}}
\newcommand{\Ycal}{\mathcal{Y}}
\newcommand{\Zcal}{\mathcal{Z}}
\newcommand{\wkdn}{d}


\newcommand{\avgR}{\wh{\cal{R}}}
%\newcommand{\ips}{\wh{r}}
\newcommand{\whpi}{\wh{\pi}}
\newcommand{\whE}{\wh{\E}}
\newcommand{\whV}{\wh{V}}

\newcommand{\whReg}{\wh{\text{\rm Reg}}}
\newcommand{\flg}{\text{\rm flag}}
\newcommand{\one}{\boldsymbol{1}}
\newcommand{\var}{\Delta}
\newcommand{\Var}{\mathrm{Var}}
\newcommand{\bvar}{\bar{\Delta}}
\newcommand{\p}{\prime}
\newcommand{\evt}{\textsc{Event}}
\newcommand{\unif}{\text{\rm Unif}}
\newcommand{\KL}{\text{\rm KL}}
\newcommand{\Lstar}{{L^\star}}
\newcommand{\istar}{{i^\star}}
\newcommand{\dynreg}{\text{Dyn-Reg}}
\newcommand{\tildedynreg}{\widetilde{\text{Dyn-Reg}}}
\newcommand{\Bstar}{{B^\star}}
\newcommand{\Ustar}{\rho}
\newcommand{\Aconst}{a}
\newcommand{\dplus}[1]{\bm{#1}}
\newcommand{\lambdamax}{\lambda_\text{\rm max}}
\newcommand{\biasone}{\textsc{Deviation}\xspace}
\newcommand{\bias}{\textsc{Bias-1}\xspace}
\newcommand{\biastwo}{\textsc{Bias-2}\xspace}
\newcommand{\biasthree}{\textsc{Bias-3}\xspace}
\newcommand{\term}[1]{\texttt{Term} ~(#1)\xspace}
\newcommand{\x}{\mathbf{x}}
\newcommand{\errorterm}{\textsc{Error}\xspace}
\newcommand{\Err}[1]{\textsc{Err-Term}(#1)\xspace}
\newcommand{\regnctx}{\textsc{Reg-NCTX}\xspace}
\newcommand{\regterm}{\textsc{Reg-Term}\xspace}
\newcommand{\LTtilde}{\wt{L}_T}
\newcommand{\Bomega}{B_{\Omega}}
\newcommand{\UOB}{UOB-REPS\xspace}
\newcommand{\Holder}{{H{\"o}lder}\xspace}
\newcommand{\dpg}{\dplus{g}}
\newcommand{\dpM}{\dplus{M}}
\newcommand{\dpf}{\dplus{f}}
\newcommand{\dpX}{\dplus{\calX}}
\newcommand{\dpw}{\dplus{w}}
\newcommand{\dpF}{\dplus{F}}
\newcommand{\dpu}{\dplus{u}}
\newcommand{\dpwtilde}{\dplus{\wtilde}}
\newcommand{\dps}{\dplus{s}}
\newcommand{\dpe}{\dplus{e}}
\newcommand{\dpx}{\dplus{x}}
\newcommand{\dpy}{\dplus{y}}
\newcommand{\dpH}{\dplus{H}}
\newcommand{\dpOmega}{\dplus{\Omega}}
\newcommand{\dpellhat}{\dplus{\ellhat}}
\newcommand{\dpell}{\dplus{\ell}}
\newcommand{\dpr}{\dplus{r}}
\newcommand{\dpxi}{\dplus{\xi}}
\newcommand{\dpv}{\dplus{v}}
\newcommand{\dpI}{\dplus{I}}
\newcommand{\dpA}{\dplus{A}}
\newcommand{\dph}{\dplus{h}}
\newcommand{\cprob}{6}
\newcommand{\sigmoid}{\ensuremath{\mathsf{Sigmoid}}\xspace}
\newcommand{\relu}{\ensuremath{\mathsf{ReLU}}\xspace}

\DeclareMathOperator*{\argmin}{argmin}
\DeclareMathOperator*{\argmax}{argmax}
\DeclareMathOperator*{\argsmax}{argsmax}
%\DeclareMathOperator*{\range}{range}
%\DeclareMathOperator*{\mydet}{det_{+}}
%\DeclarePairedDelimiter\abs{\lvert}{\rvert}
%\DeclarePairedDelimiter\bigabs{\big\lvert}{\big\rvert}
\DeclarePairedDelimiter\ceil{\lceil}{\rceil}
%\DeclarePairedDelimiter\floor{\lfloor}{\rfloor}
%\DeclarePairedDelimiter\bigceil{\big\lceil}{\big\rceil}
%\DeclarePairedDelimiter\bigfloor{\big\lfloor}{\big\rfloor}

\newcommand{\field}[1]{\mathbb{#1}}
\newcommand{\fY}{\field{Y}}
\newcommand{\fX}{\field{X}}
\newcommand{\fH}{\field{H}}
\newcommand{\fR}{\field{R}}
\newcommand{\fN}{\field{N}}
\newcommand{\fS}{\field{S}}
\newcommand{\UCB}{{\operatorname{UCB}}}
\newcommand{\LCB}{{\operatorname{LCB}}}
\newcommand{\testblock}{\textsc{EndofBlockTest}\xspace}
\newcommand{\testreplay}{\textsc{EndofReplayTest}\xspace}

\newcommand{\theset}[2]{ \left\{ {#1} \,:\, {#2} \right\} }
% \newcommand{\inner}[1]{ \langle {#1} \rangle }
\newcommand{\inn}[1]{ \langle {#1} \rangle }
\newcommand{\Ind}[1]{ \field{I}_{\{{#1}\}} }
\newcommand{\eye}[1]{ \boldsymbol{I}_{#1} }
\newcommand{\norm}[1]{\left\|{#1}\right\|}
%\newcommand{\trace}[1]{\text{tr}\left({#1}\right)}
\newcommand{\trace}[1]{\textsc{tr}({#1})}


\newcommand{\defeq}{\stackrel{\rm def}{=}}
\newcommand{\sgn}{\mbox{\sc sgn}}
\newcommand{\scI}{\mathcal{I}}
\newcommand{\scO}{\mathcal{O}}
\newcommand{\scN}{\mathcal{N}}
\newcommand{\msmwc}{\textsc{MsMwC}}

\newcommand{\dt}{\displaystyle}
\renewcommand{\ss}{\subseteq}
\newcommand{\wh}{\widehat}
\newcommand{\wt}{\widetilde}
\newcommand{\wb}{\overline}
\newcommand{\ve}{\varepsilon}
\newcommand{\hlambda}{\wh{\lambda}}

\newcommand{\Jd}{J}
\newcommand{\ellhat}{\wh{\ell}}
\newcommand{\rhat}{\wh{r}}
\newcommand{\elltilde}{\wt{\ell}}
\newcommand{\wtilde}{\wt{w}}
\newcommand{\what}{\wh{w}}

\DeclareMathOperator{\conv}{conv}
\newcommand{\ellprime}{\ellhat^\prime}

\newcommand{\upconf}{\phi}

%\newcommand{\Ltilde}{\wt{L}}

\newcommand{\hDelta}{\wh{\Delta}}
\newcommand{\hdelta}{\wh{\delta}}
\newcommand{\spin}{\{-1,+1\}}

\newcommand{\ep}[1]{\E\!\left[#1\right]}
\newcommand{\LT}{L_T}
\newcommand{\LTbar}{\overline{L}_T}
\newcommand{\LTbarep}{\mathring{L}_T}
\newcommand{\circxhat}{\mathring{\xh}}
\newcommand{\circx}{\mathring{x}}
\newcommand{\circu}{\mathring{u}}
\newcommand{\circcalX}{\mathring{\calX}}
\newcommand{\circg}{\mathring{g}}
\newcommand{\Lubar}{\overline{L}_{u}}
%\newcommand{\Lustarbar}{\overline{L}_{u^\star}}

\newcommand{\Lyr}{J}
\newcommand{\QQ}{{w}}
\newcommand{\Qt}{{\QQ_t}}
\newcommand{\Qstar}{{u}}
\newcommand{\Qpistar}{{\Qstar^{\star}}}
\newcommand{\Qhat}{\wh{\QQ}}
\newcommand{\Ut}{{\upconf_t}}
\newcommand{\intO}{\mathrm{int}(\Omega)}
\newcommand{\intK}{\mathrm{int}(K)}

\newcommand{\squareCB}{\ensuremath{\mathsf{SquareCB}}\xspace}
\newcommand{\feelgood}{\ensuremath{\mathsf{FGTS}}\xspace}
\newcommand{\graphCB}{\ensuremath{\mathsf{SquareCB.G}}\xspace}
\newcommand{\squareCBAuc}{\ensuremath{\mathsf{SquareCB.A}}\xspace}
\newcommand{\AlgSq}{\ensuremath{\mathsf{AlgSq}}\xspace}
\newcommand{\AlgLog}{\ensuremath{\mathsf{AlgLog}}\xspace}
\newcommand{\ips}{\ensuremath{\mathsf{(IPS)}}\xspace}
\newcommand{\optsq}{\ensuremath{\mathsf{(OptSq)}}\xspace}
\newcommand{\sq}{\ensuremath{\mathsf{(Sq)}}\xspace}
\newcommand{\dec}{\ensuremath{\mathsf{dec}_\gamma}\xspace}
\newcommand{\dectwo}{\ensuremath{\mathsf{dec}_{\gamma_1,\gamma_2}}\xspace}
%\newcommand{\theHalgorithm}{\arabic{algorithm}}
\newtheorem{cor}[theorem]{Corollary}
\newcommand{\context}{\text{ctx}}
\newcommand{\noncontext}{\text{n-ctx}}
%\newtheorem{remark}{Remark}
%\newtheorem{prop}{Proposition}
%\newtheorem{definition}{Definition}
%\newtheorem{assumption}{Assumption}
\newtheorem{event}{Event}
%\newtheorem*{main}{Main Result}
%\newtheorem{fact}[theorem]{Fact}

\newcommand{\paren}[1]{\left({#1}\right)}
\newcommand{\brackets}[1]{\left[{#1}\right]}
\newcommand{\braces}[1]{\left\{{#1}\right\}}

\newcommand{\normt}[1]{\norm{#1}_{t}}
\newcommand{\dualnormt}[1]{\norm{#1}_{t,*}}

\newcommand{\otil}{\ensuremath{\tilde{\mathcal{O}}}}

\newcommand{\dist}{\calP}

%%%%  brackets
\newcommand{\rbr}[1]{\left(#1\right)}
\newcommand{\sbr}[1]{\left[#1\right]}
\newcommand{\cbr}[1]{\left\{#1\right\}}
\newcommand{\nbr}[1]{\left\|#1\right\|}
\newcommand{\abr}[1]{\left|#1\right|}

\usepackage{lipsum,booktabs}
\usepackage{amsmath,mathrsfs,amssymb,amsfonts,bm,enumitem}
\usepackage{rotating}
\usepackage{pdflscape}
\usepackage{hyperref,url}
\hypersetup{
    colorlinks,
    breaklinks,
    linkcolor = blue,
    citecolor = blue,
    urlcolor  = blue,
}
\allowdisplaybreaks
\usepackage{appendix}
\usepackage{multirow,makecell}

%\usepackage{algorithmic,algorithm}
%\renewcommand{\algorithmicrequire}{ \textbf{Input:}}
%\renewcommand{\algorithmicensure}{ \textbf{Output:}}

\renewcommand{\tilde}{\widetilde}
\renewcommand{\hat}{\widehat}
\newcommand{\obs}{O}
\newcommand{\unobs}{E}
\newcommand{\unbiasSize}{c}
\newcommand{\unbias}{C}
\newcommand{\cnt}{k}

% define some macros
\def \A {\mathcal{A}}

\def \B {\mathbb{B}}
\def \B {\mathcal{B}}
\def \C {\mathcal{C}}
\def \D {\mathcal{D}}
\def \E {\mathbb{E}}
\def \F {\mathcal{F}}
\def \G {\mathcal{G}}
\def \H {\mathcal{H}}
\def \I {\mathcal{I}}
\def \J {\mathcal{J}}
\def \K {\mathcal{K}}
\def \L {\mathcal{L}}
\def \M {\mathcal{M}}
\def \N {\mathcal{N}}
\def \O {\mathcal{O}}
\def \P {\mathcal{P}}
\def \Q {\mathcal{Q}}
\def \R {\mathbb{R}}
\def \S {\mathcal{S}}
% \def \T {\mathrm{T}}
\def \T {\top}
\def \U {\mathcal{U}}
\def \V {\mathcal{V}}
\def \W {\mathcal{W}}
\def \X {\mathcal{X}}
\def \Y {\mathcal{Y}}
\def \Z {\mathcal{Z}}

\def \a {\mathbf{a}}
\def \b {\mathbf{b}}
\def \c {\mathbf{c}}
\def \d {\mathbf{d}}
\def \e {\mathbf{e}}
\def \f {\mathbf{f}}
\def \g {\mathbf{g}}
\def \h {\mathbf{h}}
\def \m {\mathbf{m}}
\def \p {\mathbf{p}}
\def \q {\mathbf{q}}
\def \u {\mathbf{u}}
\def \w {\mathbf{w}}
\def \s {\mathbf{s}}
\def \t {\mathbf{t}}
\def \v {\mathbf{v}}
\def \x {\mathbf{x}}
\def \y {y}
\def \z {\mathbf{z}}

\def \ph {\hat{p}}

\def \fh {\hat{f}}
\def \fb {\bar{f}}
\def \ft{\tilde{f}}

\def \gh {\hat{\g}}
\def \gb {\bar{\g}}
\def \gt {\tilde{g}}

\def \uh {\hat{\u}}
\def \ub {\bar{\u}}
\def \ut{\tilde{\u}}

\def \vh {\hat{\v}}
\def \vb {\bar{\v}}
\def \vt{\tilde{\v}}

\def \xh {\hat{x}}
\def \xb {\bar{\x}}
\def \xt {\tilde{\x}}

\def \zh {\hat{\z}}
\def \zb {\bar{\z}}
\def \zt {\tilde{\z}}

\def \Ecal {\mathcal{E}}
\def \Rcal {\mathcal{R}}
\def \Ot {\tilde{\O}}
\def \indicator {\mathds{1}}
\def \regret {\mbox{Regret}}
\def \proj {\mbox{Proj}}
\def \Pr {\mathsf{Pr}}
\def \ellb {\boldsymbol{\ell}}
\def \thetah {\hat{\theta}}

\newcommand{\RegSq}{\ensuremath{\mathrm{\mathbf{Reg}}_{\mathsf{Sq}}}\xspace}
\newcommand{\RegCB}{\ensuremath{\mathrm{\mathbf{Reg}}_{\mathsf{CB}}}\xspace}
\newcommand{\RegDyn}{\ensuremath{\mathrm{\mathbf{Reg}}_{\mathsf{Dyn}}}\xspace}
\usepackage{mathtools}
\let\oldnorm\norm   % <-- Store original \norm as \oldnorm
\let\norm\undefined % <-- "Undefine" \norm
\DeclarePairedDelimiter\norm{\lVert}{\rVert}
\DeclarePairedDelimiter\abs{\lvert}{\rvert}
%\newcommand\inner[2]{\langle #1, #2 \rangle}
\newcommand*\diff{\mathop{}\!\mathrm{d}}
\newcommand*\Diff[1]{\mathop{}\!\mathrm{d^#1}}

%\DeclareMathOperator*{\Reg}{Regret}
\DeclareMathOperator*{\AReg}{A-Regret}
\DeclareMathOperator*{\WAReg}{WA-Regret}
\DeclareMathOperator*{\SAReg}{SA-Regret}
\DeclareMathOperator*{\DReg}{\mbox{D-Regret}}
\DeclareMathOperator*{\poly}{poly}
%\DeclareMathOperator*{\argmax}{arg\,max}
%\DeclareMathOperator*{\argmin}{arg\,min}

% define new theorem environments
% \let\proof\relax
% \let\endproof\relax
% \newenvironment{proof}{\par\noindent{\bf Proof\ }}{\hfill\BlackBox\\[2mm]}
% \renewcommand\qedsymbol{$\blacksquare$}
\newtheorem{myThm}{Theorem}
\newtheorem{myFact}{Fact}
\newtheorem{myClaim}{Claim}
\newtheorem{myLemma}[myThm]{Lemma}
\newtheorem{myObservation}{Observation}
\newtheorem{myProp}[myThm]{Proposition}
\newtheorem{myProperty}{Property}

% Define a custom environment for prompts
\newtcolorbox{promptbox}[1][]{
  colback=blue!5!white, colframe=blue!75!black,
  fonttitle=\bfseries, title=Prompt,
  left=2mm, right=2mm, top=2mm, bottom=2mm,
  boxrule=0.5mm,  % Thickness of the frame
  coltitle=black, % Color of the title text
  colbacktitle=blue!15!white, % Background color of the title
  breakable,      % Allows the box to break across pages
  #1
}
\newtheorem{myAssum}{Assumption}
\newtheorem{myConj}{Conjecture}
\newtheorem{myCor}{Corollary}
\newtheorem{myDef}{Definition}
\newtheorem{myExample}{Example}
\newtheorem{myNote}{Note}
\newtheorem{myProblem}{Problem}

\newtheorem{myRemark}{Remark}

% add comments
\usepackage{graphicx,color} % more modern
\newcommand{\red}{\color{red}}
\newcommand{\blue}{\color{blue}}
\definecolor{wine_red}{RGB}{228,48,64}
\definecolor{DSgray}{cmyk}{0,1,0,0}
%\newcommand{\Authornote}[2]{{\small\textcolor{NavyBlue}{\sf$<<<${  #1: #2 }$>>>$}}}
% \newcommand{\Authormarginnote}[2]{\marginpar{\parbox{2cm}{\raggedright\tiny \textcolor{DSgray}{#1: #2}}}}
% \newcommand{\pnote}[1]{{\Authornote{Peng}{#1}}}
% \newcommand{\pmarginnote}[1]{{\Authormarginnote{Peng}{#1}}}

\usepackage{prettyref}
\newcommand{\pref}[1]{\prettyref{#1}}
\newcommand{\pfref}[1]{Proof of \prettyref{#1}}
\newcommand{\savehyperref}[2]{\texorpdfstring{\hyperref[#1]{#2}}{#2}}
\newrefformat{eq}{\savehyperref{#1}{Eq. \textup{(\ref*{#1})}}}
\newrefformat{eqn}{\savehyperref{#1}{Eq.~(\ref*{#1})}}
\newrefformat{lem}{\savehyperref{#1}{Lemma~\ref*{#1}}}
\newrefformat{event}{\savehyperref{#1}{Event~\ref*{#1}}}
\newrefformat{def}{\savehyperref{#1}{Definition~\ref*{#1}}}
\newrefformat{line}{\savehyperref{#1}{Line~\ref*{#1}}}
\newrefformat{thm}{\savehyperref{#1}{Theorem~\ref*{#1}}}
\newrefformat{tab}{\savehyperref{#1}{Table~\ref*{#1}}}
\newrefformat{corr}{\savehyperref{#1}{Corollary~\ref*{#1}}}
\newrefformat{cor}{\savehyperref{#1}{Corollary~\ref*{#1}}}
\newrefformat{sec}{\savehyperref{#1}{Section~\ref*{#1}}}
\newrefformat{app}{\savehyperref{#1}{Appendix~\ref*{#1}}}
\newrefformat{assum}{\savehyperref{#1}{Assumption~\ref*{#1}}}
\newrefformat{asm}{\savehyperref{#1}{Assumption~\ref*{#1}}}
\newrefformat{ex}{\savehyperref{#1}{Example~\ref*{#1}}}
\newrefformat{fig}{\savehyperref{#1}{Figure~\ref*{#1}}}
\newrefformat{alg}{\savehyperref{#1}{Algorithm~\ref*{#1}}}
\newrefformat{rem}{\savehyperref{#1}{Remark~\ref*{#1}}}
\newrefformat{conj}{\savehyperref{#1}{Conjecture~\ref*{#1}}}
\newrefformat{prop}{\savehyperref{#1}{Proposition~\ref*{#1}}}
\newrefformat{proto}{\savehyperref{#1}{Protocol~\ref*{#1}}}
\newrefformat{prob}{\savehyperref{#1}{Problem~\ref*{#1}}}
\newrefformat{claim}{\savehyperref{#1}{Claim~\ref*{#1}}}
\newrefformat{que}{\savehyperref{#1}{Question~\ref*{#1}}}
\newrefformat{op}{\savehyperref{#1}{Open Problem~\ref*{#1}}}
\newrefformat{fn}{\savehyperref{#1}{Footnote~\ref*{#1}}}

\def \p {\boldsymbol{p}}
\def \s {\boldsymbol{s}}
\def \m {\boldsymbol{m}}
\def \epsilon {\varepsilon}

% \def \base {\mathtt{base}\mbox{-}\mathtt{regret}}
% \def \meta {\mathtt{meta}\mbox{-}\mathtt{regret}}
\def \base {\textsc{base-regret}}
\def \meta {\textsc{meta-regret}}
\def \xref {\x_{\text{ref}}}
\def \fb {\bar{f}}
\def \interior {\text{int}}
\def \yh {\hat{\y}}
\def \RegLog {\Reg_{\log}^G}
\newcommand{\bra}[1]{\left[#1\right]}
\newcommand{\pa}[1]{\left(#1\right)}
\newcommand{\hhat}{\wh{h}}
\newcommand{\epsn}{\epsilon_N}
\newcommand{\rad}{\mathsf{rad}}
\newcommand{\hatr}{\wh{r}}
\newcommand{\fl}{\underline{f}^\star}
\usepackage{booktabs}
\usepackage{multirow}
\usepackage{amsmath}
\usepackage{algorithm}
\usepackage{algorithmic}

\copyrightyear{2025}
\acmYear{2025}
\setcopyright{acmlicensed}\acmConference[CHI '25]{CHI Conference on Human Factors in Computing Systems}{April 26-May 1, 2025}{Yokohama, Japan}
\acmBooktitle{CHI Conference on Human Factors in Computing Systems (CHI '25), April 26-May 1, 2025, Yokohama, Japan}
\acmDOI{10.1145/3706598.3714111}
\acmISBN{979-8-4007-1394-1/25/04}


%%
%% end of the preamble, start of the body of the document source.
\begin{document}

%%
%% The "title" command has an optional parameter,
%% allowing the author to define a "short title" to be used in page headers.
% \title{\name: Supporting Ideation of Interdisciplinary Contexts with Large Language Models for Elementary School Literature Instruction}
\title{\name{}: Supporting the Ideation of Interdisciplinary Contexts with Large Language Models for Teaching  Literature in Elementary Schools}


%%
%% The "author" command and its associated commands are used to define
%% the authors and their affiliations.
%% Of note is the shared affiliation of the first two authors, and the
%% "authornote" and "authornotemark" commands
%% used to denote shared contribution to the research.
\author{Haoxiang Fan}
% \authornote{All authors contributed equally to this research.}
\email{fanhx6@mail2.sysu.edu.cn}
\orcid{0009-0000-5729-8491}
\affiliation{%
  \institution{Sun Yat-sen University}
  \city{Zhuhai}
  \country{China}
}

\author{Changshuang Zhou}
% \authornotemark[1]
\authornote{Both authors contributed equally to this research.}
\email{mc34188@um.edu.mo}
\affiliation{%
  \institution{University of Macau}
  \city{Macau}
  \country{Macao}
}

\author{Hao Yu}
\authornotemark[1]
% \authornote{Contributed equally to this research.}
\email{yuhao53@mail2.sysu.edu.cn}
\affiliation{%
  \institution{Sun Yat-sen University}
  \city{Zhuhai}
  \country{China}
}

\author{Xueyang Wu}
% \authornotemark[1]
\email{xwuba@connect.ust.hk}
\affiliation{%
  \institution{NeurlStar}
  \city{Shenzhen}
  \country{China}
}

\author{Jiangyu Gu}
% \authornotemark[1]
% \authornote{Contributed equally to this research.}
\email{3299158551@qq.com}
\affiliation{%
  \institution{Xiangzhou Experimental School of Zhuhai}
  \city{Zhuhai}
  \country{China}
}

\author{Zhenhui Peng}
% \authornotemark[2]
\authornote{Corresponding author.}
\email{pengzhh29@mail.sysu.edu.cn}
\orcid{0000-0002-5700-3136}
\affiliation{%
  \institution{Sun Yat-sen University}
  \city{Zhuhai}
  \country{China}
}

%%
%% By default, the full list of authors will be used in the page
%% headers. Often, this list is too long, and will overlap
%% other information printed in the page headers. This command allows
%% the author to define a more concise list
%% of authors' names for this purpose.
\renewcommand{\shortauthors}{Fan et al.}
\renewcommand{\shorttitle}{\name{}}

%%
%% The abstract is a short summary of the work to be presented in the
%% article.
\begin{abstract}

\pzh{
Teaching literature under interdisciplinary (\eg science, art) contexts that connect reading materials has become popular in elementary schools. However, constructing such contexts is challenging as it requires teachers to explore substantial amounts of interdisciplinary content and link it to the reading materials. In this paper, we develop \name{} via an iterative design process involving 13 teachers to facilitate the ideation of interdisciplinary contexts for teaching literature. Powered by a large language model (LLM), \name{} can recommend interdisciplinary topics and contextualize them with literary elements (\eg paragraphs, viewpoints) in the reading materials. A within-subjects study (N=16) shows that compared to an LLM chatbot, \name{} can improve the integration depth of different subjects and reduce workload in this ideation task. Expert interviews (N=9) also demonstrate \name{}'s usefulness for supporting the ideation of interdisciplinary contexts for teaching literature. We conclude with concerns and design considerations for supporting interdisciplinary teaching with LLMs.  
}

  % Integrating knowledge from various disciplines within literature instruction at the elementary school level is beneficial yet challenging for teachers, especially in identifying suitable interdisciplinary contexts for reading materials. 
  % Large language models (LLMs) can facilitate this process by comprehensively analyzing reading materials, retrieving relevant interdisciplinary topics, and generating contexts, which would otherwise require extensive effort from teachers in terms of text analysis and selection. 
  % Through an iterative design process, which included a foundational study and formative evaluations with seven teachers, we developed \name{}. 
  % This tool supports elementary school teachers in creating interdisciplinary contexts for literature instruction with LLM-generated content.
  % Our within-subjects study ($N=16$) and expert interviews ($N=10$) suggest that \name{} can improve the quality of  interdisciplinary contexts and significantly improve the efficiency of this task.
  % We also address concerns and design considerations for supporting interdisciplinary teaching with LLMs.
\end{abstract}

%%
%% The code below is generated by the tool at http://dl.acm.org/ccs.cfm.
%% Please copy and paste the code instead of the example below.
%%
\begin{CCSXML}
<ccs2012>
   <concept>
       <concept_id>10003120.10003121.10003129</concept_id>
       <concept_desc>Human-centered computing~Interactive systems and tools</concept_desc>
       <concept_significance>500</concept_significance>
       </concept>
   <concept>
       <concept_id>10003120.10003121.10011748</concept_id>
       <concept_desc>Human-centered computing~Empirical studies in HCI</concept_desc>
       <concept_significance>500</concept_significance>
       </concept>
 </ccs2012>
\end{CCSXML}

\ccsdesc[500]{Human-centered computing~Interactive systems and tools}
\ccsdesc[500]{Human-centered computing~Empirical studies in HCI}

%%
%% Keywords. The author(s) should pick words that accurately describe
%% the work being presented. Separate the keywords with commas.
\keywords{Interdisciplinary contexts, ideation, elementary schools, teachers, large language models}
%% A "teaser" image appears between the author and affiliation
%% information and the body of the document, and typically spans the
%% page.
% \begin{teaserfigure}
%   \includegraphics[width=\textwidth]{sampleteaser}
%   \caption{Seattle Mariners at Spring Training, 2010.}
%   \Description{Enjoying the baseball game from the third-base
%   seats. Ichiro Suzuki preparing to bat.}
%   \label{fig:teaser}
% \end{teaserfigure}

% \received{20 February 2007}
% \received[revised]{12 March 2009}
% \received[accepted]{5 June 2009}

%%
%% This command processes the author and affiliation and title
%% information and builds the first part of the formatted document.
\maketitle
\section{Introduction}
\label{sec:introduction}
The business processes of organizations are experiencing ever-increasing complexity due to the large amount of data, high number of users, and high-tech devices involved \cite{martin2021pmopportunitieschallenges, beerepoot2023biggestbpmproblems}. This complexity may cause business processes to deviate from normal control flow due to unforeseen and disruptive anomalies \cite{adams2023proceddsriftdetection}. These control-flow anomalies manifest as unknown, skipped, and wrongly-ordered activities in the traces of event logs monitored from the execution of business processes \cite{ko2023adsystematicreview}. For the sake of clarity, let us consider an illustrative example of such anomalies. Figure \ref{FP_ANOMALIES} shows a so-called event log footprint, which captures the control flow relations of four activities of a hypothetical event log. In particular, this footprint captures the control-flow relations between activities \texttt{a}, \texttt{b}, \texttt{c} and \texttt{d}. These are the causal ($\rightarrow$) relation, concurrent ($\parallel$) relation, and other ($\#$) relations such as exclusivity or non-local dependency \cite{aalst2022pmhandbook}. In addition, on the right are six traces, of which five exhibit skipped, wrongly-ordered and unknown control-flow anomalies. For example, $\langle$\texttt{a b d}$\rangle$ has a skipped activity, which is \texttt{c}. Because of this skipped activity, the control-flow relation \texttt{b}$\,\#\,$\texttt{d} is violated, since \texttt{d} directly follows \texttt{b} in the anomalous trace.
\begin{figure}[!t]
\centering
\includegraphics[width=0.9\columnwidth]{images/FP_ANOMALIES.png}
\caption{An example event log footprint with six traces, of which five exhibit control-flow anomalies.}
\label{FP_ANOMALIES}
\end{figure}

\subsection{Control-flow anomaly detection}
Control-flow anomaly detection techniques aim to characterize the normal control flow from event logs and verify whether these deviations occur in new event logs \cite{ko2023adsystematicreview}. To develop control-flow anomaly detection techniques, \revision{process mining} has seen widespread adoption owing to process discovery and \revision{conformance checking}. On the one hand, process discovery is a set of algorithms that encode control-flow relations as a set of model elements and constraints according to a given modeling formalism \cite{aalst2022pmhandbook}; hereafter, we refer to the Petri net, a widespread modeling formalism. On the other hand, \revision{conformance checking} is an explainable set of algorithms that allows linking any deviations with the reference Petri net and providing the fitness measure, namely a measure of how much the Petri net fits the new event log \cite{aalst2022pmhandbook}. Many control-flow anomaly detection techniques based on \revision{conformance checking} (hereafter, \revision{conformance checking}-based techniques) use the fitness measure to determine whether an event log is anomalous \cite{bezerra2009pmad, bezerra2013adlogspais, myers2018icsadpm, pecchia2020applicationfailuresanalysispm}. 

The scientific literature also includes many \revision{conformance checking}-independent techniques for control-flow anomaly detection that combine specific types of trace encodings with machine/deep learning \cite{ko2023adsystematicreview, tavares2023pmtraceencoding}. Whereas these techniques are very effective, their explainability is challenging due to both the type of trace encoding employed and the machine/deep learning model used \cite{rawal2022trustworthyaiadvances,li2023explainablead}. Hence, in the following, we focus on the shortcomings of \revision{conformance checking}-based techniques to investigate whether it is possible to support the development of competitive control-flow anomaly detection techniques while maintaining the explainable nature of \revision{conformance checking}.
\begin{figure}[!t]
\centering
\includegraphics[width=\columnwidth]{images/HIGH_LEVEL_VIEW.png}
\caption{A high-level view of the proposed framework for combining \revision{process mining}-based feature extraction with dimensionality reduction for control-flow anomaly detection.}
\label{HIGH_LEVEL_VIEW}
\end{figure}

\subsection{Shortcomings of \revision{conformance checking}-based techniques}
Unfortunately, the detection effectiveness of \revision{conformance checking}-based techniques is affected by noisy data and low-quality Petri nets, which may be due to human errors in the modeling process or representational bias of process discovery algorithms \cite{bezerra2013adlogspais, pecchia2020applicationfailuresanalysispm, aalst2016pm}. Specifically, on the one hand, noisy data may introduce infrequent and deceptive control-flow relations that may result in inconsistent fitness measures, whereas, on the other hand, checking event logs against a low-quality Petri net could lead to an unreliable distribution of fitness measures. Nonetheless, such Petri nets can still be used as references to obtain insightful information for \revision{process mining}-based feature extraction, supporting the development of competitive and explainable \revision{conformance checking}-based techniques for control-flow anomaly detection despite the problems above. For example, a few works outline that token-based \revision{conformance checking} can be used for \revision{process mining}-based feature extraction to build tabular data and develop effective \revision{conformance checking}-based techniques for control-flow anomaly detection \cite{singh2022lapmsh, debenedictis2023dtadiiot}. However, to the best of our knowledge, the scientific literature lacks a structured proposal for \revision{process mining}-based feature extraction using the state-of-the-art \revision{conformance checking} variant, namely alignment-based \revision{conformance checking}.

\subsection{Contributions}
We propose a novel \revision{process mining}-based feature extraction approach with alignment-based \revision{conformance checking}. This variant aligns the deviating control flow with a reference Petri net; the resulting alignment can be inspected to extract additional statistics such as the number of times a given activity caused mismatches \cite{aalst2022pmhandbook}. We integrate this approach into a flexible and explainable framework for developing techniques for control-flow anomaly detection. The framework combines \revision{process mining}-based feature extraction and dimensionality reduction to handle high-dimensional feature sets, achieve detection effectiveness, and support explainability. Notably, in addition to our proposed \revision{process mining}-based feature extraction approach, the framework allows employing other approaches, enabling a fair comparison of multiple \revision{conformance checking}-based and \revision{conformance checking}-independent techniques for control-flow anomaly detection. Figure \ref{HIGH_LEVEL_VIEW} shows a high-level view of the framework. Business processes are monitored, and event logs obtained from the database of information systems. Subsequently, \revision{process mining}-based feature extraction is applied to these event logs and tabular data input to dimensionality reduction to identify control-flow anomalies. We apply several \revision{conformance checking}-based and \revision{conformance checking}-independent framework techniques to publicly available datasets, simulated data of a case study from railways, and real-world data of a case study from healthcare. We show that the framework techniques implementing our approach outperform the baseline \revision{conformance checking}-based techniques while maintaining the explainable nature of \revision{conformance checking}.

In summary, the contributions of this paper are as follows.
\begin{itemize}
    \item{
        A novel \revision{process mining}-based feature extraction approach to support the development of competitive and explainable \revision{conformance checking}-based techniques for control-flow anomaly detection.
    }
    \item{
        A flexible and explainable framework for developing techniques for control-flow anomaly detection using \revision{process mining}-based feature extraction and dimensionality reduction.
    }
    \item{
        Application to synthetic and real-world datasets of several \revision{conformance checking}-based and \revision{conformance checking}-independent framework techniques, evaluating their detection effectiveness and explainability.
    }
\end{itemize}

The rest of the paper is organized as follows.
\begin{itemize}
    \item Section \ref{sec:related_work} reviews the existing techniques for control-flow anomaly detection, categorizing them into \revision{conformance checking}-based and \revision{conformance checking}-independent techniques.
    \item Section \ref{sec:abccfe} provides the preliminaries of \revision{process mining} to establish the notation used throughout the paper, and delves into the details of the proposed \revision{process mining}-based feature extraction approach with alignment-based \revision{conformance checking}.
    \item Section \ref{sec:framework} describes the framework for developing \revision{conformance checking}-based and \revision{conformance checking}-independent techniques for control-flow anomaly detection that combine \revision{process mining}-based feature extraction and dimensionality reduction.
    \item Section \ref{sec:evaluation} presents the experiments conducted with multiple framework and baseline techniques using data from publicly available datasets and case studies.
    \item Section \ref{sec:conclusions} draws the conclusions and presents future work.
\end{itemize} 
\section{RELATED WORK}
\label{sec:relatedwork}
In this section, we describe the previous works related to our proposal, which are divided into two parts. In Section~\ref{sec:relatedwork_exoplanet}, we present a review of approaches based on machine learning techniques for the detection of planetary transit signals. Section~\ref{sec:relatedwork_attention} provides an account of the approaches based on attention mechanisms applied in Astronomy.\par

\subsection{Exoplanet detection}
\label{sec:relatedwork_exoplanet}
Machine learning methods have achieved great performance for the automatic selection of exoplanet transit signals. One of the earliest applications of machine learning is a model named Autovetter \citep{MCcauliff}, which is a random forest (RF) model based on characteristics derived from Kepler pipeline statistics to classify exoplanet and false positive signals. Then, other studies emerged that also used supervised learning. \cite{mislis2016sidra} also used a RF, but unlike the work by \citet{MCcauliff}, they used simulated light curves and a box least square \citep[BLS;][]{kovacs2002box}-based periodogram to search for transiting exoplanets. \citet{thompson2015machine} proposed a k-nearest neighbors model for Kepler data to determine if a given signal has similarity to known transits. Unsupervised learning techniques were also applied, such as self-organizing maps (SOM), proposed \citet{armstrong2016transit}; which implements an architecture to segment similar light curves. In the same way, \citet{armstrong2018automatic} developed a combination of supervised and unsupervised learning, including RF and SOM models. In general, these approaches require a previous phase of feature engineering for each light curve. \par

%DL is a modern data-driven technology that automatically extracts characteristics, and that has been successful in classification problems from a variety of application domains. The architecture relies on several layers of NNs of simple interconnected units and uses layers to build increasingly complex and useful features by means of linear and non-linear transformation. This family of models is capable of generating increasingly high-level representations \citep{lecun2015deep}.

The application of DL for exoplanetary signal detection has evolved rapidly in recent years and has become very popular in planetary science.  \citet{pearson2018} and \citet{zucker2018shallow} developed CNN-based algorithms that learn from synthetic data to search for exoplanets. Perhaps one of the most successful applications of the DL models in transit detection was that of \citet{Shallue_2018}; who, in collaboration with Google, proposed a CNN named AstroNet that recognizes exoplanet signals in real data from Kepler. AstroNet uses the training set of labelled TCEs from the Autovetter planet candidate catalog of Q1–Q17 data release 24 (DR24) of the Kepler mission \citep{catanzarite2015autovetter}. AstroNet analyses the data in two views: a ``global view'', and ``local view'' \citep{Shallue_2018}. \par


% The global view shows the characteristics of the light curve over an orbital period, and a local view shows the moment at occurring the transit in detail

%different = space-based

Based on AstroNet, researchers have modified the original AstroNet model to rank candidates from different surveys, specifically for Kepler and TESS missions. \citet{ansdell2018scientific} developed a CNN trained on Kepler data, and included for the first time the information on the centroids, showing that the model improves performance considerably. Then, \citet{osborn2020rapid} and \citet{yu2019identifying} also included the centroids information, but in addition, \citet{osborn2020rapid} included information of the stellar and transit parameters. Finally, \citet{rao2021nigraha} proposed a pipeline that includes a new ``half-phase'' view of the transit signal. This half-phase view represents a transit view with a different time and phase. The purpose of this view is to recover any possible secondary eclipse (the object hiding behind the disk of the primary star).


%last pipeline applies a procedure after the prediction of the model to obtain new candidates, this process is carried out through a series of steps that include the evaluation with Discovery and Validation of Exoplanets (DAVE) \citet{kostov2019discovery} that was adapted for the TESS telescope.\par
%



\subsection{Attention mechanisms in astronomy}
\label{sec:relatedwork_attention}
Despite the remarkable success of attention mechanisms in sequential data, few papers have exploited their advantages in astronomy. In particular, there are no models based on attention mechanisms for detecting planets. Below we present a summary of the main applications of this modeling approach to astronomy, based on two points of view; performance and interpretability of the model.\par
%Attention mechanisms have not yet been explored in all sub-areas of astronomy. However, recent works show a successful application of the mechanism.
%performance

The application of attention mechanisms has shown improvements in the performance of some regression and classification tasks compared to previous approaches. One of the first implementations of the attention mechanism was to find gravitational lenses proposed by \citet{thuruthipilly2021finding}. They designed 21 self-attention-based encoder models, where each model was trained separately with 18,000 simulated images, demonstrating that the model based on the Transformer has a better performance and uses fewer trainable parameters compared to CNN. A novel application was proposed by \citet{lin2021galaxy} for the morphological classification of galaxies, who used an architecture derived from the Transformer, named Vision Transformer (VIT) \citep{dosovitskiy2020image}. \citet{lin2021galaxy} demonstrated competitive results compared to CNNs. Another application with successful results was proposed by \citet{zerveas2021transformer}; which first proposed a transformer-based framework for learning unsupervised representations of multivariate time series. Their methodology takes advantage of unlabeled data to train an encoder and extract dense vector representations of time series. Subsequently, they evaluate the model for regression and classification tasks, demonstrating better performance than other state-of-the-art supervised methods, even with data sets with limited samples.

%interpretation
Regarding the interpretability of the model, a recent contribution that analyses the attention maps was presented by \citet{bowles20212}, which explored the use of group-equivariant self-attention for radio astronomy classification. Compared to other approaches, this model analysed the attention maps of the predictions and showed that the mechanism extracts the brightest spots and jets of the radio source more clearly. This indicates that attention maps for prediction interpretation could help experts see patterns that the human eye often misses. \par

In the field of variable stars, \citet{allam2021paying} employed the mechanism for classifying multivariate time series in variable stars. And additionally, \citet{allam2021paying} showed that the activation weights are accommodated according to the variation in brightness of the star, achieving a more interpretable model. And finally, related to the TESS telescope, \citet{morvan2022don} proposed a model that removes the noise from the light curves through the distribution of attention weights. \citet{morvan2022don} showed that the use of the attention mechanism is excellent for removing noise and outliers in time series datasets compared with other approaches. In addition, the use of attention maps allowed them to show the representations learned from the model. \par

Recent attention mechanism approaches in astronomy demonstrate comparable results with earlier approaches, such as CNNs. At the same time, they offer interpretability of their results, which allows a post-prediction analysis. \par


\pzh{
\section{Design Process and Principles of \name{}}
\label{sec:design_process}
Our work aims to support elementary literature teachers in effectively identifying suitable interdisciplinary contexts for their instructions, which can be used in their later lesson plans and classroom activity designs. 
\penguin{
Our design process and evaluation of \name{} involve in total of 17 Chinese language teachers (\autoref{tab:teachers}) in an elementary school in mainland China. 
% To achieve this, we invite total 17 Chinese language teachers (\autoref{tab:teachers}) in an elementary school in mainland China.
Specifically, in the design process, we involved E1-E7 in the foundational study and I1-I6 in the evaluation of the prototype. 
In the evaluation of \name{} with teachers (\ie Experiment II), we involved E1-E5 again and E8-11. 
In Experiment II, I1-I6 also contributed findings about the unchanged features between \name{} and its prototype. 
% and I1-I6 are involved in the iterative design process, including a foundational study and a evaluation of the prototype.
}

% The initial phase, foundational study, comprised three sessions of semi-structured interviews with an experienced literature teacher (E1).
% Before each session, E1 met with her seven-member interdisciplinary literature course design team (members: E1 - E7), documented the meetings, and subsequently reported these findings during interviews with the authors. 
% The second phase, we developed a workable prototype of \name{} and evaluated it with other six teachers (I1 - I6). 
% }
% We gathered their insights to inform our design goals for \name{} presented in this paper. %conducted a \textbf{formative user evaluation}

% \fhx{E1-E5 and E8-E11 participate in Experiment II (\ie Expert Interviews) following the development of \name{}. We will discuss the results in~\autoref{sec:experiment_2}}


\begin{table*}[htbp]
\caption{17 Chinese language teachers participated in the iterative design process and expert interviews (\ie Experiment II). Among them, there were 6 males and 11 females, with teaching experience ranging from 3 to 29 years. Two participants did not provide information on their teaching experience. This table also includes their experience in reading projects and interdisciplinary projects.}
\Description{17 Chinese language teachers participated in the iterative design process and expert interviews (\ie Experiment II). Among them, there were 6 males and 11 females, with teaching experience ranging from 3 to 29 years. Two participants did not provide information on their teaching experience. This table also includes their experience in reading projects and interdisciplinary projects.}
\label{tab:teachers}
\begin{tabular}{@{}cccccc@{}}
\toprule
\textbf{Involvement}                                                                                                                               & \textbf{ID} & \textbf{Gender} & \begin{tabular}[c]{@{}c@{}}\textbf{Teaching Experience} \cr \textbf{(years)}\end{tabular} & 
\begin{tabular}[c]{@{}c@{}}\textbf{Participation in} \cr \textbf{Reading Projects}\end{tabular} & 
\begin{tabular}[c]{@{}c@{}}\textbf{Participation in} \cr \textbf{Interdisciplinary Projects}\end{tabular} \cr \midrule
\multirow{5}{*}{\begin{tabular}[c]{@{}c@{}}\textit{Foundational}\cr \textit{Study}\cr \textit{\&}\cr \textit{Experiment II}\end{tabular}}                                        & E1          & F               & 5                                                                               & Y                                                                                     & Y                                                                                               \cr
                                                                                                                                                   & E2          & M               & 7                                                                               & Y                                                                                     & Y                                                                                               \cr
                                                                                                                                                   & E3          & F               & 14                                                                              & Y                                                                                     & Y                                                                                               \cr
                                                                                                                                                   & E4          & M               & 6                                                                               & Y                                                                                     & Y                                                                                               \cr
                                                                                                                                                   & E5          & M               & 4                                                                               & Y                                                                                     & Y                                                                                               \cr \hline
\multirow{2}{*}{\begin{tabular}[c]{@{}c@{}}\textit{Foundationall}\cr \textit{Study}\end{tabular}}                                                             & E6          & F               & -                                                                               & Y                                                                                     & Y                                                                                               \cr
                                                                                                                                                   & E7          & F               & -                                                                               & Y                                                                                     & Y                                                                                               \cr \hline
\multirow{6}{*}{\begin{tabular}[c]{@{}c@{}} \textit{\penguin{Evaluation of}}\cr \textit{\penguin{Prototype (re-usable}}\cr \textit{\penguin{findings are presented}}\cr \textit{\penguin{in Experiment II)}}\end{tabular}} & I1          & F               & 27                                                                              & Y                                                                                     & Y                                                                                               \cr
                                                                                                                                                   & I2          & F               & 8                                                                               & Y                                                                                     & Y                                                                                               \cr
                                                                                                                                                   & I3          & F               & 5                                                                               & N                                                                                     & N                                                                                               \cr
                                                                                                                                                   & I4          & M               & 11                                                                              & Y                                                                                     & Y                                                                                               \cr
                                                                                                                                                   & I5          & F               & 5                                                                               & Y                                                                                     & N                                                                                               \cr
                                                                                                                                                   & I6          & M               & 5                                                                               & Y                                                                                     & N                                                                                               \cr \hline
\multirow{4}{*}{\begin{tabular}[c]{@{}c@{}}\textit{\penguin{Expert}}\cr \textit{\penguin{Interviews}}\end{tabular}}                                                              & E8          & M               & 3                                                                               & N                                                                                     & N                                                                                               \cr
                                                                                                                                                   & E9          & F               & 6                                                                               & Y                                                                                     & N                                                                                               \cr
                                                                                                                                                   & E10         & F               & 29                                                                              & N                                                                                     & Y                                                                                               \cr
                                                                                                                                                   & E11         & F               & 17                                                                              & Y                                                                                     & N                                                                                               \cr \bottomrule
\end{tabular}
\end{table*}

\subsection{Design Process}
\penguin{
We generally followed a user-centered approach to plan our design process. 
First, to understand users' practices and involve them in the design of \name{}, we conducted three sessions of semi-structured interviews with an experienced literature teacher E1, who led a seven-member (E1 - E7) interdisciplinary literature course design team that indirectly contributed to the interviews. 
We were not able to have direct discussions with E2 - E7 due to their inconvenience during the semester. 
Then, we developed a workable prototype of \name{} and evaluated it with another six teachers (I1 - I6).  
We gathered their insights to inform our design goals for \name{} presented in this paper. 
}
\subsubsection{Foundational Study}
We closely worked with E1 to identify the practices, challenges and needs for support of ideating interdisciplinary contexts for teaching literature in elementary schools. 
Over the past two years, E1 has spearheaded a team of seven individuals (E1-E7) in the exploration and implementation of interdisciplinary literature instruction within elementary school Chinese courses. 
\fanhx{
To gain a comprehensive understanding of user needs, we progressively conducted three sessions of semi-structured interviews with E1 in April, June, and July 2024, lasting 38 minutes, 53 minutes, and 45 minutes, respectively.
}
Before each session, we communicated the purpose of the interview to E1 and requested that she engage with her team to compile records of their meetings for discussing the topics in the intended interview. 
We documented each interview session with E1 through audio and video recordings. % for each interview.

In Session 1, 
\fanhx{we asked E1 to present their current practices of teaching literature in interdisciplinary contexts, with previously developed lesson plans and assignments in her team.
} 
The discussion also focused on the potential of \fanhx{AI (\eg what do you think AI can support you (in your lesson planning in previous))}, and an interactive system to facilitate the design of interdisciplinary literature contexts, including the integration of art and history into the assignments \peng{of literature reading}. 
\peng{After this section}, two authors brainstormed potential features of a system for supporting the ideation of interdisciplinary contexts and sent E1 a document that explains these features. %(\eg xxx \peng{[any high-level features?]})
% , who subsequently 
We requested E1 to engage in a discussion with her group members to identify any additional or incorrect points about the \peng{potential system}. 
In Session 2, E1 \peng{came back with positive feedback from her team on each potential feature}.
\fanhx{We asked her to further explain their general process for designing interdisciplinary contexts as a team, emphasizing the distinct responsibilities and cognitive processes of each teacher involved.} 
Additionally, she presented the proposed interaction model. 
After this session, E1 had a group meeting with her team and came up with a template that defines the anticipated outcomes of our system. 
% Following a group meeting, a template defining the anticipated outcomes of the system was subsequently provided to the authors.
In Session 3, we introduced how a system works utilizing LLM agents \peng{to simulate roles in a team for ideating the interdisciplinary contexts, as suggested by E1 in Session 2}. 
% , detailing the specific roles of each agent. 
We presented two example outcomes produced using our predefined prompts and intermediate outputs 
\fanhx{to ask for her opinions on these prompts and outputs (\eg whether these intermediate outputs were helpful? If the prompts aligned well with your thoughts?).
}
}


\pzh{
\subsubsection{Development and Evaluation of \name{} Prototype}
\peng{
After Session 3, two of the authors utilized the thematic analysis method to analyze the transcribed recordings and all textual content derived from the foundational study. 
The analysis yielded four summarized Design Principles as described in the following \autoref{sec:principles}. 
We then worked on the implementation of a workable prototype that chains different LLM agents in a structured process to help teachers think of interdisciplinary contexts for teaching the literature materials in the textbooks. 
}
% \subsubsection{Evaluation of \name{} Prototype}
We evaluated our workable prototype with another six elementary school Chinese language teachers (I1 - I6, 3 Male, 3 Female), as shown in \autoref{tab:teachers}. 
Each evaluation lasted approximately 30 to 45 minutes and comprised four parts: (1) an introduction to the background, which included the concepts of interdisciplinary literature instruction, and the theory of contexts of instruction; (2) a brief tutorial on the prototype; (3) a think-aloud study in which participants freely explored the prototype and spoke out their thoughts; and (4) a semi-structured interview \peng{for their comments and suggestions on the prototype}. With the participants' consent, we conducted the evaluation offline and recorded audio and video.

At this stage, we assessed the prototype's usability (\eg whether different functions were well-integrated), user perception (\eg user interaction with the prototype and any additional cognitive load), and the quality of the system's outcomes. Furthermore, we collected suggestions for improving the prototype, particularly regarding user interface design and additional functionalities. 
% \peng{These feedback and suggestions helped us refine the design principles, which are incorporated in \autoref{sec:principles} Design Principles}.
\penguin{These feedback and suggestions are presented in \autoref{sec:formative_findings}, which inform the design principles (\autoref{sec:principles}) of \name{}.}
% \peng{Besides, the prototype in this evaluation study is close to the final version of \name{}.
% The key refinements lie in xxx, xxx, and xxx. 
For the six teachers' (I1-6) feedback on the same features in the prototype and final version of \name{}, we incorporate it in the results of Expert Interviews in \autoref{sec:experiment_2}. 
}
% In this section, we concentrate on gathering feedback and findings to establish design goals and incorporate new features into the prototype. Feedback on other unchanged aspects will be reported in the Section Expert Interviews.

% \fhx{
% \subsection{Findings and Design Principles} \label{sec:principles}
% }

\penguin{
\subsection{Findings}
\label{sec:formative_findings}
}
\fhx{
Two of the authors utilized thematic analysis~\cite{braun2012thematic} to inductively code and summarize the \penguin{practices, challenges, requirements, and concerns} from transcribed recordings \penguin{in the design process. The key themes are shown below.} % and all textual content from the foundational study.
\fanhx{One author first iteratively coded the data, while the other carefully reviewed the codes to ensure accuracy. After discussions, they reached a consensus and identified six primary themes. These findings are shown below.}
}


\penguin{
\textbf{Finding 1: In practice, teachers usually engage in reverse thinking when ideating interdisciplinary context}.}
% \fhx{\textbf{Finding 1: Two opposing intellectual paradigms of interdisciplinary context ideation.}}
% Teachers engage in two distinct intellectual paradigms when ideating contexts. One paradigm, referred to as ``forward thinking'', resembles ``deductive reasoning''. 
\penguin{Our teachers mentioned two intellectual paradigms.}
One paradigm referred to as ``forward thinking'', resembles ``deductive reasoning'', in which
teachers create abstract connections from a limited number of reading materials (typically 3-5 texts) and develop a concrete and reasonable context. 
The other paradigm, which is termed ``reverse thinking'' and analogous to ``inductive reasoning'', is a more habitual cognitive process employed by teachers. 
% \penguin{To conduct ``reverse thinking'', which is analogous to ``inductive reasoning'',}
Teachers would like to first select interdisciplinary contexts that they deem suitable and then identify appropriate texts from a broader text pool, after which they refine the connections between the identified text and the context. 
\textit{``For us, a good context often arises from a sudden inspiration, which we then backtrack to complete the ideation of what texts can connect to this context and how''} (E1).

\penguin{
\textbf{Finding 2: It is challenging to identify the connections between the established context and the reading materials  in the process of ``reverse thinking''.} 
}
% \fhx{\textbf{Finding 2: Extraction of connections from different subjects is challenging.}}
% In the process of ``reverse thinking'', identifying the connections between the established context and the reading materials is challenging. 
Teachers must evaluate the effectiveness of the connections in enhancing students' understanding of both literature and its associated subjects, as well as in stimulating their interest. 
\textit{``It is quite difficult and usually takes a long time for our team to ensure that based on the reading materials, our teaching activities connected by the context can indeed help students gain knowledge''} (E1).

\penguin{
\textbf{Finding 3: Teachers require support at three levels of granularity when analyzing the reading materials and contexts.}
}
% \fhx{\textbf{Finding 3: Typically employing three levels of granularity of analysis are needed.}}
% During the ideation process, to help teachers prepare for the teaching activities, the system should help teachers analyze and understand the reading materials and contexts at three three levels of granularity. 
The first level is in-depth single-text analysis, which \textit{``explains how the elements of a given article relate to the context''} (E1). 
The second level is pairwise comparison, where comparative reading has been demonstrated to be an effective method for understanding texts, \textit{``allowing articles to `disappear in pairs' by analyzing the similarities and differences in relation to the context''} (E1). 
The third level is multi-text-driven exploration, which necessitates that the system should support the comprehensive linking of all texts selected by the teacher. Therefore, this level requires \name{} to conduct a thorough deconstruction of contexts, extract meaningful connections, and convey these connections to the teachers. 
% Besides, the analyses should also be accurate, especially in the filed of education. This not only necessitates high-quality AI-generated content but also emphasizes the assurance of editing all contents by teachers freely within the user interface. \textit{``I want to freely edit rather than just drag or choose fixed options. This allows me to directly correct issues when I discover them''} (E1). 
% Additionally, teachers believe that the system should enable them to pose detailed and flexible questions about the contexts and reading materials.  \textit{``Our team hopes that teachers can input vague, open-ended questions to the LLM''} (E1).

\penguin{
\textbf{Finding 4: 
Teachers require detailed instructional activities based on the selected contexts.}}
% \fhx{\textbf{Finding 4: Ensuring enough details for instructional activities.}}
% \fhx{
% E1 emphasized that finding a precise interdisciplinary context is the first step in preparing a series of instructional activities, and that manually developing comprehensive teaching resources tailored to the teaching practices based on the selected context is necessary but challenging.
% }
\penguin{As E1 summarized after the meeting with her teaching team, the outcome plan of several lessons surrounding a context should include targeted reading materials and analysis in each lesson, an introduction facilitating students' engagement, and related teaching activities.}
% Enough teaching resources should at least include 
% comprehensive lesson planning, relevant explanations for each segment, an introduction to facilitate classroom engagement, and specific analyses of the texts, summarized by E1. 
Also, in the evaluation study of \name{} prototype, three teachers (I1, I2, I5) indicated that the system outputs should be more detailed and reduce human effort in modifying them for the later concrete plans for each lesson. % lesson design and instructional materials.
\textit{``The overall structure of the output is good, but I hope it can be more detailed; for example, providing more in-class and extracurricular activities related to the theme, so we can use them directly''} (I1). Therefore, we incorporated recommendations for literature and interdisciplinary course activities in the refined \name{}.

\penguin{
\textbf{Finding 5: Teachers are concerned the quality and reliability of the content purely generated by LLMs.}
}
% \fhx{\textbf{Finding 6: LLM-generated content is unsatisfying due to high repetitiveness and low reliability.}}
In the evaluation study with six teachers (I1-6), our prototype generated traditional subject-related contexts using the LLM with specific templates and cognitive backgrounds of elementary students, without fine-tuning or retrieval-augmented generation (RAG). 
Three teachers (I4, I5, I6) expressed concerns about the quality of the LLM-generated content. \textit{``The content generated for the art subject is quite repetitive''} (I5). \textit{``We need to establish a dedicated article database for science as well, since many of our articles are highly relevant to science''} (I4). 
% Therefore, in the final version of \name{}, we have collected contexts \peng{and articles} from various subjects and use them as a foundation for generating content, which could reduce repetition and increase reliability of suggested context. 


\penguin{
\textbf{Finding 6: Teachers suggest six metrics for evaluating the outcome of interdisciplinary literature lesson plan.}
% In foundational study, the team of E1-E7 established six metrics for evaluating the quality of contexts and teaching resources.
As established by the team of E1-E7, the metrics are: 
}
% \fhx{\textbf{Finding 5: Metrics for evaluation.}}
% In foundational study, the team of E1-E7 established six metrics for evaluating the quality of contexts and teaching resources.
\begin{itemize}
    \item \textbf{Appropriateness of Context}
    \begin{itemize}
        \item \textit{Content Alignment:} Does the context accurately cover the content of the selected materials?
        \item \textit{Internal Logic:} Is there a logical connection between the context and the selected materials?
    \end{itemize}
    \item \textbf{Alignment with Educational Objectives}
    \begin{itemize}
        \item \textit{Curriculum Standards:} Does the content comply with national curriculum standards and teaching guidelines?
        \item \textit{Subject Goals:} Does it help achieve specific goals of language education, such as reading comprehension and writing skills?
    \end{itemize}
    \item \textbf{Depth of Integration}
    \begin{itemize}
        \item \textit{Subject Integration:} Does it effectively integrate knowledge from different subjects?
        \item \textit{Knowledge Transfer:} Does it promote the application of language arts knowledge in other subject contexts?
    \end{itemize}
\end{itemize}


% We summarize the design principles (\textbf{DPs}) of \name{} derived from the foundation study and evaluation of the prototype as below. 

% We employed the reflexive thematic analysis method to analyze the transcribed recordings and all textual content derived from the foundational study. The analysis yielded four summarized Design Principles. In a similar manner, we analyzed the \textbf{formative user evaluation}, incorporating usability issues identified by the teachers and categorizing them under the corresponding DP1 - DP4.

\subsection{Design Principles} \label{sec:principles}

\penguin{Based on the findings from our design process and related literature, we derive four design principles of \name{}.}
% \fhx{Based on the findings from our iterative design process, we propose four human-centered design principles for the development of \name{}.}

% \textbf{DP1: \name{}'s creative thinking support process should align with teachers' habitual behaviors in interdisciplinary context ideation.} % (Finding 1, Finding 2)
\penguin{
% \textbf{DP1: \name{}'s ideation support process should align with teachers' habitual practices in interdisciplinary context ideation.}
\textbf{DP1: \name{} should provide step-by-step support that aligns with teachers' habitual practices in interdisciplinary context ideation}.
Tailoring the assistance to users' habitual practices (\eg active students' behaviors or teachers' behaviors) is a commonly enacted principle in previous interactive systems in educational scenarios~\cite{fok2024qlarify,liu2024classmeta,fan2024lessonplanner}. 
% For example, LessonPlanner \cite{fan2024lessonplanner} adapts the nine events of Gagne's instructional theory to support the planning of one lesson, 
% and ClassMeta \cite{liu2024classmeta} displays various behaviors commonly observed among active students to promote VR classroom participation. 
In the task of interdisciplinary context ideation, as revealed in Finding 1, \name{} should support step-by-step context ideation through reverse thinking, which is a habitual practice of our teachers. 
\fanhx{In this practice, as E1 shared,} teachers take various roles to analyze the potential contexts 
\fanhx{(we note this role as Context Analyst)}, 
analyze the texts in the reading materials, connect them to the contexts, 
\fanhx{and discuss the approaches (Text Analyst and Text Reviewer)}. 
After that, teachers try to summarize the contexts and associated reading materials into an actionable lesson plan \fanhx{(Context Summarizer)}. 
\name{} can prompt LLMs to play different roles when supporting teachers in each of these steps. 
We do not chase for generating one-step context ideation outcome with one LLM prompt, because teachers desire necessary human input in each step, and enabling multiple LLM agents to simulate human-human collaboration has been proven to improve output quality~\cite{wu2024transagents, du2024multi}. 
}
% \textit{DP1.1: \name{} should support context ideation through reverse thinking (Finding 1).} 
% Teachers engage in two distinct cognitive processes when ideating contexts. The first process, referred to as ``forward thinking'', resembles ``deductive reasoning''. 
% Teachers draw abstract connections from a limited number of reading materials (typically 3-5 texts) and develop a concrete and reasonable context. 
% The second, which is termed ``reverse thinking'' and analogous to ``inductive reasoning'', is a more habitual cognitive process employed by teachers. Teachers would like to first select interdisciplinary contexts that they deem suitable and then identify appropriate texts from a broader text pool, after which they refine the connections between the identified text and the context. \textit{``For us, a good context often arises from a sudden inspiration, which we then backtrack to complete the ideation of what texts can connect to this context and how''} (E1).

% \textit{DP1.2: \name{} should facilitate the extraction of potential connections among the reading materials (Finding 2).}
% \subsubsection{DP1.2: Facilitates the extraction of relevant connections from reading materials.}
% In the process of ``reverse thinking'', identifying the connections between the established context and the reading materials is challenging. 
% Teachers must evaluate the effectiveness of the connections in enhancing students' understanding of both literature and its associated subjects, as well as in stimulating their interest. 
% \textit{``It is quite difficult and usually takes a long time for our team to ensure that based on the reading materials, our teaching activities connected by the context can indeed help students gain knowledge''} (E1).

% \fhx{
% We consider human habitual behaviors to design the system, aligning with the human-centered system for educational scenarios~\cite{fok2024qlarify,liu2024classmeta,fan2024lessonplanner}. 
% Additionally, for LLM-empowered systems, enabling LLM agents to simulate human-human collaboration has been proven to improve output quality~\cite{wu2024transagents, du2024multi}. 
% In summary, DP1 is formulated by integrating the advantages of aligning system's performance with teacher's habitual practices and the insights from Finding 1 and Finding 2.
% }
% \subsubsection{\textbf{DP1: \name{} supports creative thinking modes that align with teachers' habitual practices in context ideation}}

% \penguin{
% \textbf{DP2: \name{} should act as a context analyst that helps  from a comprehensive database of interdisciplinary contexts and articles}
% Especially, \textit{DP1.2) \name{} should facilitate the extraction of potential connections among the reading materials}, as Finding 2 reveals that this is a challenging step in the ideation process. 

% }
\penguin{
\textbf{DP2: \name{} should provide teachers with detailed analyses of the contexts, reading materials, and their relationships.}
% Ideating interdisciplinary context for literature teaching requires integrating knowledge across multiple subjects (\eg science, history), which teachers may not be familiar with. 
Our teachers reported that it was challenging to identify the connections between contexts and reading materials (Finding 2) and desired support during the analyses (Finding 3). 
Previous HCI works have demonstrated the strengths of LLMs in analyzing and connecting complex information~\cite{zheng2024disciplink, chi24_Selenite}. 
% For example, DiscipLink~\cite{zheng2024disciplink} can prompt LLMs to automatically expand queries with disciplinary-specific terminologies and highlight the connections between retrieved papers and questions. 
% Selenite~\cite{chi24_Selenite} employs LLMs to generate comprehensive overviews of options and criteria grounded in search results, guiding users through complex decisions. 
Similarly, to satisfy user requirements (Finding 3) in our task, \name{} can leverage LLMs to recommend contexts, explain them in detail, identify relevant texts in the reading materials, and assess the relationship between the contexts and texts. 
}

% \textbf{DP2: \name{} should provide teachers with detailed analyses of the reading materials and contexts and 
% % precise
% \fhx{support verification of the generated content}. }
% \fhx{
% Interdisciplinary context ideation involves integrating knowledge across multiple subjects, which makes it challenging for elementary literature teachers to assess whether the generated content is comprehensive and accurate in unfamiliar domains (\eg mathematics, science).
% A large amount of complex interdisciplinary information may lead users to feel ``information overload''~\cite{foster2004nonlinear, newby2011entering}.
% Therefore, reducing the cognitive load on educators when understanding generated content is important. 
% On one hand, system-generated content should be detailed to minimize misunderstandings, 
% requiring multiple levels of granularity in analysis and explanation (Finding 3). 
% On the other hand, systems can support users in verifying the generated content to facilitate the integration of information~\cite{foster2004nonlinear} by providing reliable and critical feedback.
% }


% During the ideation process, to help teachers prepare for the later teaching activities, the system should help teachers analyze and understand the reading materials and contexts at three three levels of granularity. 
% The first level is in-depth single-text analysis, which \textit{``explains how the elements of a given article relate to the context''} (E1). 
% The second level is pairwise comparison, where comparative reading has been demonstrated to be an effective method for understanding texts, \textit{``allowing articles to `disappear in pairs' by analyzing the similarities and differences in relation to the context''} (E1). 
% The third level is multi-text driven exploration, which necessitates that the system should support the comprehensive linking of all texts selected by the teacher. Therefore, this level requires \name{} to conduct a thorough deconstruction of contexts, extract meaningful connections, and convey these connections to the teachers. 
% Besides, the analyses should also be accurate, especially in the filed of education. This not only necessitates high-quality AI-generated content but also emphasizes the assurance of editing all contents by teachers freely within the user interface. \textit{``I want to freely edit rather than just drag or choose fixed options. This allows me to directly correct issues when I discover them''} (E1). 
% Additionally, teachers believe that the system should enable them to pose detailed and flexible questions about the contexts and reading materials.  \textit{``Our team hopes that teachers can input vague, open-ended questions to the LLM''} (E1).

\penguin{
\textbf{DP3: \name{} should document the ideation outcomes in a lesson plan that aligns with the established educational practices in interdisciplinary literature teaching.}
A teacher without a lesson plan may struggle to effectively deliver the knowledge and objectives of the lesson~\cite{iqbal2021rethinking}. 
Finding 4 suggests that the lesson plan should contain detailed instructional activities, including the related reading materials and in-class activities, based on the selected contexts. 
To make it further aligned with educational practices, \name{} can adopt the six evaluation metrics of the outcome lesson plan (Finding 6) to guide the generation of ideation outcomes. 
}

% \pzh{
% \textbf{DP3: \name{} should provide structured and high-quality ideation outcomes 
% \fhx{that align with the educational requirements of interdisciplinary literature instruction in elementary school.}
% }}
% \fhx{
% A teacher without a lesson plan may struggle to effectively deliver the knowledge and objectives of the lesson~\cite{iqbal2021rethinking}. 
% Additionally, as Finding 4 revealed, this kind of lesson plan generated from the system should be as detailed as possible, at least including the contents of each lesson, an introduction for engagement, and classroom activities.
% Moreover, to better align the outcomes with the educational requirements, six metrics identified in Finding 6 could be integrated into the generation process as the guidelines.
% }


% \textit{DP3.1: \name{}' outputs should be consistent with educational practices
% \fhx{of literature instruction in elementary school }
% (Finding 4).} 
% % E1 emphasized that system outputs should contain the necessary content tailored to the teaching practices in her team. 
% % E1 summarized that the outputs should at least include comprehensive lesson planning, relevant explanations for each segment, an introduction to facilitate classroom engagement, and specific analyses of the texts. 
% % In the evaluation study of \name{} prototype, three teachers (I1, I2, I5) indicated that the system outputs should be more detailed and reduce human effort in modifying them for the later concrete plans for each lesson. % lesson design and instructional materials.
% % \textit{``The overall structure of the output is good, but I hope it can be more detailed; for example, providing more in-class and extracurricular activities related to the theme, so we can use them directly''} (I1). Therefore, we incorporated recommendations for literature and interdisciplinary course activities in the refined \name{}.

% \textit{DP3.2: \name{} should incorporate metrics that evaluate the quality of 
% outcome 
% \fhx{disciplinary }
% contexts
% \fhx{for literature instruction in elementary school}
% (Finding 5).}
% % The team of E1-E7 established six metrics for evaluating the quality of contexts: 
% }

% \fhx{
% The large amount of complex interdisciplinary information may lead users feeling ``information overload''~\cite{foster2004nonlinear, newby2011entering}. Thus, system outputs should be in a well-designed structure, including useful details while avoiding information unrelated to the instructional activities. 
% This helps prevent teachers from spending more effort on adapting or verifying the generated content.
% In other words, first, outcomes of \name{} require only simple interpretation and minor adjustments to help educators rapidly create lesson plans (Finding 4). 
% Second, systems should support users in verifying output information to facilitate information integration~\cite{foster2004nonlinear}.
% The evaluation metrics from Finding 5 can be integrated into the system to address this issue.
% }

\pzh{
% \begin{itemize}
%     \item \textbf{Appropriateness of Context}
%     \begin{itemize}
%         \item \textit{Content Alignment:} Does the context accurately cover the content of the selected materials?
%         \item \textit{Internal Logic:} Is there a logical connection between the context and the selected materials?
%     \end{itemize}
%     \item \textbf{Alignment with Educational Objectives}
%     \begin{itemize}
%         \item \textit{Curriculum Standards:} Does the content comply with national curriculum standards and teaching guidelines?
%         \item \textit{Subject Goals:} Does it help achieve specific goals of language education, such as reading comprehension and writing skills?
%     \end{itemize}
%     \item \textbf{Depth of Integration}
%     \begin{itemize}
%         \item \textit{Subject Integration:} Does it effectively integrate knowledge from different subjects?
%         \item \textit{Knowledge Transfer:} Does it promote the application of language arts knowledge in other subject contexts?
%     \end{itemize}
% \end{itemize}

\penguin{
\textbf{DP4: \name{} should include database of interdisciplinary contexts and reading materials and provide flexible user control to achieve high-quality ideation outcomes.} 
Prior research on the impact of LLMs in primary education indicates that generating false content is a disadvantage that may lead to ``information pollution'' for children~\cite{adeshola2023opportunities, murgia2023chatgpt}.
Finding 5 also indicates that LLMs sometimes were unable to create content that meets teachers' needs when lacking access to educational resources. 
To generate high-quality outcomes, as inspired by previous works~\cite{yazici2024gelex, khanal2024fathomgpt}, 
\name{} could ground the content generation on diverse real-world contexts and reading materials. 
% This approach not only reduces repetition but also increases the reliability of the suggested material. 
% Apart from high-quality AI generation, 
\name{} should also support teachers to freely edit and question any content (\eg texts in reading materials, outcome lesson plan) to make sure that they understand the content they are going to use in literature teaching. 
}

% \textbf{DP4: \name{} should include a comprehensive database of interdisciplinary contexts and articles.}
% \fhx{
% Prior research on the impact of LLMs in primary education indicates that generating false content is a disadvantage that may lead to ``information pollution'' for children~\cite{adeshola2023opportunities, murgia2023chatgpt}.
% From Finding 5, we also noticed that LLMs are sometimes unable to create content that meets teachers' needs when lacking access to educational resources in evaluating the \name{} prototype. 
% Therefore, inspired by previous works~\cite{yazici2024gelex, khanal2024fathomgpt}, 
% % we recognize that equipping LLMs with a comprehensive database allows them to generate more relevant content by retrieving and clustering educational texts from various subjects. 
% we could try to collect diverse contexts and articles to serve as a solid foundation for content generation in the final version of \name{}. 
% This approach not only reduces repetition but also increases the reliability of the suggested material, helping to prevent potential misinformation that could negatively impact students and teachers.
% }
% In the evaluation study with six teachers, our prototype generated traditional subject-related contexts using the LLM with specific templates and cognitive backgrounds of elementary students, without fine-tuning or retrieval-augmented generation (RAG). 
% Three teachers (I4, I5, I6) expressed concerns about the quality of the generated content. \textit{``The content generated for the art subject is quite repetitive''} (I5). \textit{``We need to establish a dedicated article database for science as well, since many of our articles are highly relevant to science''} (I4). Therefore, in the final version of \name{}, we have collected contexts \peng{and articles} from various subjects and use them as a foundation for generating content, which could reduce repetition and increase reliability of sugggested context. 
}

%PlanGREEN

%GEN-Plan

%G- generate
%R-refine
%E- edit

%% GREEN-plan

%% PURPLE
\begin{figure*}
    \centering
    \Description{PLAID's system architecture diagram. Top part shows the database (a), and bottom part shows the interface (b). The system starts from bottom right as an instructor is interested in a programming domain, then the pipeline described in the text produces reference materials at different levels of granularity, and these are presented in the interface.}
    \includegraphics[width=\textwidth]{img/system-architecture-subgoals.png}
    \caption{PLAID's reference content is generated through an LLM pipeline
    %inspired by the practices of instructors who have successfully identified programming plans. 
    that produces output on three levels.
    First, a wide variety of use cases are generated to create example programs that focus on code's applications. Next, using LLM's explanatory comments that represent subgoals within the code, the examples are segmented into meaningful code snippets. The LLM is queried to generate other plan components for each code snippet. Finally, the code snippets are clustered to identify the most common patterns, representing plan candidates. The full programs are presented in `Programs' views of PLAID interface, whereas snippets are presented in clusters in the `Plan Creation' view.}
    \label{fig:system-pipeline}
\end{figure*}
\section{PLAID: A System for Supporting Plan Identification}
\label{sec:system-design}

Following the design goals devised from the design workshop, we refined our early prototype into PLAID: a
%LLM-powered
tool to assist instructors in their plan identification process.
PLAID synthesizes the capabilities of LLMs in code generation with interactions enabling plan identification practices observed in our studies with instructors.
As we noted in the findings of our design workshop, the LLM-generated candidate plans are not ready to be used as is in instruction, but instructors can readily adapt and correct them (\cref{sec:workshop-findings-condition2}).
PLAID enables collaboration between instructors and LLMs, enhancing the plan identification process by automating its time-intensive information-gathering tasks and facilitating instructors' ability to refine LLM-generated candidate plans based on their knowledge about pedagogy and the programming domain. 



\subsection{Practical Illustration}

To understand how instructors use can PLAID to more easily adopt plan-based pedagogies, we follow Jane, a computer science instructor using PLAID to design programming plans for her course (summarized in \cref{fig:jane-workflow}).

Jane is teaching a programming course for psychology majors and wants to introduce her students to data analysis using Pandas. As her students have limited prior programming experience and use programming for specific goals, she organizes her lecture material around programming plans to emphasize purpose over syntax. 
% that explain practical concepts to students and help them focus on the purpose behind the code they write.
% However, she realizes that all introductory computer science courses offered at her institution only teach basic programming constructs like data structures. After exploring Google Scholar for effective instructional methods to teach application-focused programming to non-computer science majors, she learned about plan-based pedagogies that help them focus on the purpose behind the code they write. In her literature review, she finds out about PLAID, a tool that can help her design domain-specific plans. She reviews the domains supported by the tool (Pandas, Pytorch, Beautifulsoup, and Django) and decides to use Pandas, a popular and powerful data analysis and manipulation library, to create her curriculum. 

She logs in to the PLAID web interface, % and takes time to explore the system's features. 
and asks PLAID to suggest a plan (\cref{fig:jane-workflow}, 1). The first plan recommended to her 
% she sees is a plan to help students learn about
is about reading CSV files. 
She thinks the topic is important and the solution code aligns with her experience; % the solution is promising and represents an important concept that students need to know about.
% She is satisfied with the given solution 
but she finds the generated name and goal to be too generic. She edits (\cref{fig:jane-workflow}, 2) these fields to provide more context that she feels is right for her students.
% She refines those fields and then 
To make this plan more abstract and appropriate for more use cases, %explain how this plan can be used for reading data from different file formats,
she marks the file path as a changeable area (\cref{fig:jane-workflow}, 3), generalizing the plan for reading data from different file formats.

Inspired by the first plan, she decides to create a plan for saving data to disk. She wants to teach the most conventional way of saving data, so she switches to the use case tab (\cref{fig:jane-workflow}, 4) and explores example programs that save data to get a sense of common practices.  %interact with the list of complete programs.
She finds a complete example where a DataFrame is created and and saved to a file. %performs cleaning tasks like deleting NaN values, and exports it.
% She realizes that something she hadn't thought of before: saving new data is almost always necessary after performing data manipulation operations!
She selects the part of the code that exports data to a file and creates a plan from that selection (\cref{fig:jane-workflow}, 5).


For the next plan, she reflects on her own experience with Pandas. She recalls that merging DataFrames was a key concept, but cannot remember the full syntax. 
% Jane reflects on her experience working with Pandas and recalls that merging DataFrames is a key operation when working with big data.
She switches to the full programs tab (\cref{fig:jane-workflow}, 6) that includes complete code examples and searches (\cref{fig:jane-workflow}, 7) for ``\texttt{.merge}'' to find different instances of merging operations. % and tries to use the search bar to find a relevant program that contains ``.merge''. 
After finding a comprehensive example, she selects the relevant section of the code and creates a plan from it.
% She again selects a part of the example, creates plan from the selection, and refines it. She engages with the system iteratively and designs twenty plans for her lecture. 

After designing a set of plans that capture the important topics, she organizes them into groups (\cref{fig:jane-workflow}, 8) 
% also grouped similar plans together
to emphasize sets of plans with similar goals but different implementations. For instance, she takes her plans about \texttt{.merge} and \texttt{.concat} and groups them together to form a category of plans that students can reference when they want to {combine data from different sources}.

% combining data using ``merge'' or ``concat''.

% the the she used plans isn't very good right now
% She exports these plans and starts preparing her lecture slides, using the plans as a way of presenting key concepts to students with minimal programming experience.
She exports these plans to support her students with minimal programming experience by preparing lecture slides that organize the sections around plan goals, using plan solutions as worked examples in class, and providing students with cheat sheets that include relevant plans for their laboratory sessions.
% using the plan goals as titles for different sections of her slides, and using the solutions as references for the examples she creates. Finally, she makes a PDF cheatsheet with all the plans for students to reference during the week's laboratory.
% The next day, she starts preparing her lecture slides and realizes that the names and goals she wrote for her plans represent key concepts in Pandas. She references the plans she created to design annotated examples that she includes on her lecture slides.

%% How does Jane actually use the plans? 
%% > Important to be careful to note that this isn't actually part of the system....
%% > She uses the generated plans to (a) as inspiration for worked examples in teh course, (b) as stems for questions that test how code should be completed
%% > She notices she now has a list of key concepts in the area


\begin{figure*}[h]
        \Description{An annotated screenshot of PLAID's `Programs' view. On the left, a list of use cases such as `Renaming columns in a Frame' and `Plotting a histogram of a column' is shown, with a scrollable list and a search bar. The latter one is selected, and on the right, we see the contents of the program in a monospaced font, with four buttons explained in the caption.}
        \includegraphics[width=\textwidth]{img/system-diagram-1-fixed.png}
        \caption{Plan Identification using PLAID: (a) list of example programs for instructors organized by natural language descriptions, (b) list of full programs of code, (c) search bar enabling easy navigation of given content to find code for specific ideas, (d) button to create a plan using the selected part of the code, (e) button to create a plan using the complete example program, (f) button to view an explanation for a selected code snippet, and (g) button for executing the selected code.}
        \label{fig:system-diagram-1}
\end{figure*}

\subsection{System Design}

At a high level, PLAID\footnote{The code for PLAID can be found at: https://github.com/yosheejain/plaid-interface.} operates on two subsystems: (1) a database of LLM-generated reference materials created through a pipeline that uses \edit{OpenAI's GPT-4o\footnote{https://openai.com/index/hello-gpt-4o/}~\cite{achiam2023gpt}}, inspired by instructors' best practices for identifying programming plans (see ~\cref{fig:system-pipeline})
%LLM for identifying plans in application-focused domains 
and (2) an interface that allows instructors to browse reference materials for relevant code snippets 
% and other plan components to achieve a goal that meets their needs. Then, they refine the candidates to mine plans 
and refine suggested content into programming plans
(see Figures~\ref{fig:system-diagram-1} and~\ref{fig:system-diagram-2}).
% In this section, we describe the implementation of the pipeline generating the reference materials and the key interface features of PLAID.



\subsubsection{Database of Reference Materials for Application-Focused Domains}

PLAID extracts information from reference materials at three levels of granularity to support each instructor's unique workflow: complete programs that address a particular use case, annotated program snippets that include goals and changeable areas, and plan candidates that cluster relevant program snippets together.

\textbf{Generating complete example programs.}
The content at the lowest level of granularity in the PLAID database are the complete programs. 
%These candidate plans were generated using a pipeline to generate \textit{plan-ful examples}, which we define as examples of programming plans in use, with all plan components identified (see Section~\ref{sec:components}). This implementation had three stages: (1) generating in-domain programs, (2) segmenting programs into plan-ful examples, and (3) clustering plan-ful examples into plans. 
\label{sec:llm-pipeline}
% \begin{figure}
% \centering
% % \includegraphics[width=0.5\textwidth]{img/pipeline-new.png}
% \includegraphics[width=\textwidth]{img/new-plan-pipeline.png}
% \caption{The three stage process for generating example programs, segmenting them with plan components, and clustering these plan-ful examples.
% %collecting and processing responses from ChatGPT into plan-ful examples}
% %\caption{The pipeline for LLM plan generation.}
% }
% \label{fig:llm-methods}
% \end{figure}
% \subsubsection{Generating In-Domain Programs}
% Informed by the insights identified in our interview study, we generated programming plans relevant to an application-focused domain: web scraping via BeautifulSoup. We utilized an LLM-based approach to generate these plans with the GPT-4 model from OpenAI using its publicly available API in an iterative workflow. 
% Our participants examined example programs and conducted literature reviews (Section \ref{sec:viewing-programs}) as key parts of their plan identification process. 
As these examples should capture a variance of use cases in the real world, we utilized an LLM trained on a large corpus of computer programs and natural language descriptions~\cite{liu2023isyourcode}.
% Inspired by this, we used Open AI's GPT-4, a state-of-the-art large language model for code generation that is trained on a large corpus of computer programs~\cite{liu2023isyourcode},
% to generate candidate programs along with its respective plan components in the programs.
We prompted\footnote{Full prompts can be found in \cref{sec:appendix-pipeline}.} the model to generate \texttt{specific use cases of <application-focused library>}, defining use case as \texttt{a task you can achieve 
with the given library} (see \cref{sec:use_case_prompt}). Subsequently, we prompted the model to \texttt{write code to do the following: <use case>}, producing a set of 100 example programs with associated tasks (see \cref{sec:code_prompt}). By generating the use cases first and generating the solution later, we avoided the problems with context windows of LLMs where the earlier input might get `forgotten', resulting in the model producing the same output repeatedly. For practical purposes, we generated 100 programs per domain. \edit{To test for potential ``hallucinations'' where the LLM generates plausible yet incorrect code~\cite{Ji_2023_hallucination}, we checked the syntactic validity of the generated programs before developing the rest of our pipeline. No more than one out of 100 generated programs included syntax errors in each of our domains, i.e., Pandas, Django, and PyTorch. Thus, we concluded that hallucinations are not a major threat to the code generation aspect of PLAID.}
%while hallucinations in LLMs are a pressing concern for systems that utilize these models,
% This collection of example programs (which we refer to as 
%dataset 
% $\mathcal{D}$) was used as our primary dataset for further analysis.

\textbf{Generating annotated program snippets.}
% \subsubsection{Segmenting Programs Into Plan-ful Examples}
% We then proceed to compile these examples with each of the plan components generated using ChatGPT. We construct a new dataset with these components, Dataset \((\mathcal{D}^{\textit{Comp}})\).
The second level of granularity in PLAID consists of small program snippets and a goal, with changeable areas annotated. 
We used the generated programs from
% \mathcal{D}$
the prior step as the input to the LLM to add subgoal labels, where we prompted the LLM to annotate subgoals (see \cref{sec:subgoals_prompt}) as comments that describe \texttt{small chunks of code that achieve a task that can be explained in natural language}. These subgoal labels were used to break the full program into shorter snippets. Each snippet was fed back to the model to generate changeable areas (see \cref{sec:ca_prompt}), defined in the prompt as \texttt{parts of the idiom that would change when it is used in different scenarios}. The subgoal label that explained a code snippet corresponded to its goal in the plan view and the list of elements assigned as changeable was used for annotations.
% (see Stage 2 in Figure~\ref{fig:llm-methods}),
%We fragmented these generated programs into smaller code pieces by generating \textit{subgoals} in the program. Then, each goal (Section \ref{sec:goal}) and the accompanying code solution (Section \ref{sec:solution}) were added as a single unit of data in our plan-ful example dataset of components, \(\mathcal{D}^{\textit{Plan-ful}}\). For each of these datapoints, we prompted the model to identify \textit{changeable areas} (Section \ref{sec:changeable}). %The name (Section \ref{sec:name}) was determined later in the pipeline (Stage 2 in Figure \ref{fig:llm-methods}).


% From the results of our qualitative study, we now know about the parts of a programming plan. In order to extract these plans automatically, we used ChatGPT. We accessed it using its publicly available API and we used the GPT-4 model. We selected 3 domains that are interesting for non-majors. This included . 

% For each of these domains, we first asked the LLM to generate 100 use cases. We then re-prompted it with the use cases it generated and asked it to generate code that would be written to accomplish that use case.
% potential for another table?
% add code metrics from stackoverflow github work for chatgpt
% With all these code pieces collected, we then asked ChatGPT to generate each of the plan parts one-by-one.

% \subsubsection*{Extracting Goals and Solutions}Generated programs 
% in \(\mathcal{D}\) 
% typically included a comment before each line, which described that line's functionality. However, these comments did not capture the high-level purpose of the code, as required by a plan goal. To generate more abstract goals for a piece of code, we defined subgoals as \texttt{short descriptions of small pieces of code that do something meaningful} in a prompt and asked the LLM to \texttt{highlight subgoals as comments in the code.} %In our query, we also added the way we define subgoals to provide the relevant context to the model. Specifically, we wrote that 
% The output from this prompt was a modified version of each program
% from \(\mathcal{D}\), 
% where blocks of code are preceded by a comment describing the goal of that block. % of code. % instead of restating functionality. 

% We split each complete program into multiple segments based on these new comments. Thus, the subgoal comments from each complete program I
% n the modified \(\mathcal{D}\) 
% became a plan goal, and the code following that comment became the associated solution. %, collected in \(\mathcal{D}^{\textit{Plan-ful}}\). % After it returned the annotated code piece, we extracted the comment and the following lines of code before the next comment. This pair acted as a subgoal-code piece. We collected all such pairs across all use cases from \(\mathcal{D}\) and added them to \(\mathcal{D}^{\textit{Plan-ful}}\).
% Each goal 
% %(Section \ref{sec:goal}) 
% and solution pair
% %(Section \ref{sec:solution}) 
% was added as a single unit of data in our plan-ful example dataset.
% , \(\mathcal{D}^{\textit{Plan-ful}}\).

% \subsubsection*{Extracting Changeable Areas}To annotate the changeable areas for a plan, we defined changeable areas as \texttt{parts of the plan that would change when it is used in a different context} in our prompt and asked the model to \texttt{return the exact part of the code from the line that would change} for all code pieces from the dataset with plan-ful examples.
% from \(\mathcal{D}^{\textit{Plan-ful}}\). 
% This data was added to \(\mathcal{D}^{\textit{Plan-ful}}\).

% to-do
% \subsubsection{Clustering Plan-ful Examples into Plans}
\textbf{Generating clustered plan candidates.}
\label{sec:clustering}
% We perform k-means clustering on the plans \(\mathcal{D}^{\textit{Plan-ful}}\) to identify clusters of similar code pieces and thus, programming plans.
The highest level of granularity provided in PLAID
%presents users with 
are
plan candidates, in the form of clusters of annotated program snippets. To compare the similarity of program snippets, we used CodeBERT embeddings following prior work~\cite{codebert} and applied Principal Component Analysis (PCA) \cite{PCAanalysis} to reduce the dimensionality of the embedding while preserving 90\% of the variance. The snippets were clustered using the K-means algorithm~\cite{kmeansclustering}, using the mean silhouette coefficient for determining optimal K~\cite{silhouettecoeff}. Each cluster is treated as a plan candidate, with the goal, code, and changeable areas from each program snippet in the cluster presented as a suggested value for the plan attributes.
% We used a clustering algorithm to group similar program snippets 
% plan-ful examples together as a programming plan. For clustering the code pieces, we used the CodeBERT model from Microsoft \cite{codebert} to obtain embeddings for each code piece in our dataset of plan-ful examples
% % in \(\mathcal{D}^{\textit{Plan-ful}}\) 
% and applied Principal Component Analysis (PCA) \cite{PCAanalysis} to reduce the dimensionality of the embedding vectors while preserving 90\% of the variance. These embeddings were clustered using the K-means algorithm~\cite{kmeansclustering}. The optimal number of clusters \(\mathcal{K}\) was determined by assessing all possible \(\mathcal{K}\) values 
% % (where \(\mathcal{K} \in [2, \texttt{length}(\mathcal{D}^{\textit{Plan-ful}})]\))
% using the mean silhouette coefficient \cite{silhouettecoeff}. We assigned each example 
% % in \(\mathcal{D}^{\textit{Plan-ful}}\) 
% to a cluster of similar code pieces. 
% \subsubsection*{Extracting Names}
For each plan candidate, a name (see \cref{sec:name_prompt}) that summarizes all snippets in the cluster was generated by prompting an LLM with the contents of the snippets and stating that it should generate \texttt{a name that reflects the code's purpose} and it should focus on \texttt{what the code is achieving and not the context}. 
% Then, all code snippets from each cluster of examples were provided as input to the LLM along with a prompt asking it to \texttt{devise a name for that cluster of plans}.

% \subsection{Interface for Refining Candidate Plans}

% %nd the back-end server relied on routes written in Flask. The domain-specific candidate plans suggested to the user are queried from the database of candidate plans generated using the LLM. Each participant was required to log in to the web page using their unique credentials, which allowed us to record their activity for analysis. While the complete details of our implementation of the web-based application are out of scope for this paper, we describe its main features in Section~\ref{sec:implementation_of_webinterface}.

% \subsubsection{\edit{Preliminary Technical Evaluation of Generated Content}}

% \edit{syntactic validity and standard code complexity metrics to determine
% their suitability for novices}


\begin{figure*}[h]
    \Description{An annotated screenshot of PLAID's Plan Creation view with three panes, with plans shown as boxes on the left. A plan is highlighted, and we see its components on the middle pane. On the rightmost pane, we see suggested values for the selected component.}
    \includegraphics[width=\textwidth]{img/system-diagram-2-new.png}        
    \caption{Plan Identification using PLAID: (h) button that suggests a domain-specific candidate plan from the system database, (i) pane enabling viewing of similar values for the selected plan component, (j) button to view the solution code as part of a complete program, (k) pane with a structured template for plan design with editable fields to refine plan components, (l) button to copy a selected plan, (m) button to mark snippets of code from the plan solution as changeable areas, and (n) a button to group plans together into a category and add a name.}
    \label{fig:system-diagram-2}
\end{figure*}

% \subsubsection{Key Characteristics}
% PLAID supports the process of plan identification in data processing with Pandas, machine learning with Pytorch, web development using Django, and web scraping using BeautifulSoup. 

\subsubsection{Interface for Designing Programming Plans}
Building on the 
%characteristics addressed in the artifact (Section~\ref{sec:design-artifact}) and 
design goals identified in the design workshop (\cref{sec:design-goals}), PLAID enables a set of key interactions to assist instructors in refining candidates to design plans for their instruction. 



\textbf{Interactions for Initial Plan Identification.}
% Initial Plan Identification with Quick Exploration of Many Authentic Programs
While instructors valued the availability of code examples in the design workshop (Section~\ref{sec:design-workshop-findings}), we observed many opportunities for scaffolding their interaction with the reference material. To this end, PLAID presents example programs in two different views \textbf{(DG1)}. 
% We saw instructors scanning examples, selecting desired code pieces, and copying them over into their plan templates in all conditions in the study. 
The ``Programs (Organized by Use Case)'' (\cref{fig:system-diagram-1}a) tab includes a list of use cases where instructors can click on an item to expand the program for that use case.
The ``Programs (Full Text)''  tab (\cref{fig:system-diagram-1}b) lists all the programs and enables instructors to scroll or search through (\cref{fig:system-diagram-1}c) all the code at once.
% presents the contents of all the programs expanded viewing a list of complete code examples, allowing instructors to look at materials they would typically search for when designing plans.
% equipping instructors with full-code programs organized in a list of short natural language descriptions of common use cases in their domain of expertise. 
Both views support directly creating a plan from the whole example (\cref{fig:system-diagram-1}e), or a selected part of it (\cref{fig:system-diagram-1}d), by copying the solution and the goal of the program into an empty plan template
% < Highlight code in full code and code pane in tab1 and make a plan (D1)
% < Add a button to add full program as a plan too (D1)
further supporting efficient use of the material \textbf{(DG3)}.
% This interaction copies over the selected code and its respective use case into the solution and name fields, respectively. 
% < Code explanation plugin for strange syntax (GPT) (D2)

To facilitate understanding unfamiliar code and syntax, we implemented a ``View Explanation'' button (\textbf{DG2}) that generates a short description of the selected line(s) of code by prompting an LLM (\cref{fig:system-diagram-1}f). 
% In this case, participants hesitated to use the suggested syntax in their plans because its functionality was unclear to them. PLAID supports a button named ``View Explanation'' where the user can select a method, function, or line of code that is unclear and click on it to understand its working \textbf{(D2)}. 
Participants also looked for code execution to validate and understand a program. However, since the code snippets instructors work with are often incomplete in this task, we implemented a ``Run Code'' feature (\textbf{DG2}) that predicts the output of a selected code snippet by prompting an LLM to walk through the code \texttt{step by step}, using Chain-of-Thought prompting~\cite{wei2022chain} (\cref{fig:system-diagram-1}g). Only the predicted output for the code is presented, ignoring other output from the LLM.

% to examine the code behavior and thus mitigate the challenge of being faced with unfamiliar syntax. Thus, using PLAID, instructors are able to run complete programs to view their output \textbf{(D2)}.
% < Search in the use cases (and full progs) (D3)
% Frequently, instructors relied on their expertise and experience to formulate ideas about goals for which they wanted to create plans. While interacting with condition C in the design workshop, interviewees suggested including a mechanism to search for specific keywords within code and  its natural language description. To facilitate the instructor-LLM collaboration, allowing users to find examples implementing their ideas, PLAID includes a search bar that helps users navigate the given use cases, complete programs, and effectively find specific examples they may be looking for \textbf{(D3)}.

\textbf{Interactions for Plan Refinement.}
% Support Plan Refinement with Comparisons of content
% Participants indicated difficulty mining plans from code examples (Section~\ref{sec:challenges_practice}). 
To provide suggestions for code patterns common enough to be potential programming plans,
%To alleviate challenges in identifying content common enough for designing plans, 
we utilize the clustered program snippets from our database. In the ``Plan Creation'' view of PLAID, instructors can ask for suggestions (\cref{fig:system-diagram-2}h) to see a candidate plan to refine (\textbf{DG3}).  \edit{This functionality allows instructors to draw on their experience to recognize common code snippets and decide if they are valuable to teach students.}
% If instructors want to demonstrate their plan as part of a complete code example, they can review these examples reducing the effort that they would need to put in to recall syntax and construct a complete example. 
\edit{This promotes recognition over recall \cite{recognition_over_recall}, thus helping reduce the cognitive effort that instructors may have to put in while designing programming plans traditionally.}
To allow instructors to better understand the context of a plan under refinement, PLAID 
also includes a button for searching for the current solution within the entire set of full programs
%, showing the code snippet in context 
%as part of a complete example
(\textbf{DG3}, \cref{fig:system-diagram-2}j).
% < Keyword search/embedding filter for potential values (D1)

As instructors edit the components of a plan, they are shown similar values from the corresponding component in that cluster (\cref{fig:system-diagram-2}i). By clicking on any suggested value, instructors can replace a plan component with a suggestion that better captures that aspect of the plan \textbf{(DG1)}. \edit{By allowing instructors to view the plan they are working on along with other related code pieces in a split screen view, we promote instructor efficiency by reducing the split-attention effect \cite{tarmizi1988guidance}. In the current plan creation process, even when using LLMs from their chat interface, instructors would have to switch between windows with code examples and their text editor which may increase the load on the instructors' working memory \cite{clark2023learning}. In PLAID, instructors can edit their plans and view similar code pieces at the same time.}

% \edit{By enabling these interactions and thus organizing ``knowledge in the world'' effectively, PLAID reduces the need for instructors to store and retrieve the ``knowledge in their head'' \cite{Norman_DOET}. Thus, PLAID optimizes the plan creation process by allowing efficient search within the ``knowledge in the world'' and reducing the cognitive load while storing and retrieving ``knowledge in the head'', minimizing the total effort required \cite{} by instructors.}
% after searching its code corpus for similar examples using a keyword search \textbf{(DG1)}.
% < Show use case button in solution (add highlighting) (D2)
% To help instructors easily consider the context of a plan as they refine it, PLAID 
% In the design workshop, few instructors emphasized the importance of presenting worked and contextualized examples to students. 

% ‘go to a use case’ button that redirects the user to the tab with full code programs and highlights the plan as part of a complete example \textbf{(D2)}.

\textbf{Interactions for Building Robust and Shareable Plan Descriptions.}
% Support robust/sharable plan descriptions
% From Section~\ref{sec:process_intro_plan_design}, instructors indicated drawing on their experience in the application-specific domain and instructional expertise to think about how to best solve a problem. 
PLAID encourages instructors to design plans in a structured template (\cref{fig:system-diagram-2}k). Moreover, PLAID reinforces the plan template by providing a dedicated method for annotating changeable areas by highlighting any part of the code (\textbf{DG3}, \cref{fig:system-diagram-2}m). Instructors can further explain the changeable areas by adding comments as text.
% \edit{The structured template view of the plan encourages instructors to articulate their mental models of how the plan would generalize to other problems, allowing the transfer of ``knowledge in the head'' to ``knowledge in the world''.}

Our design workshop showed that participants would create a plan and copy it to emphasize alternatives or modifications to the underlying idea. To support this workflow,
% In our design workshop, participants created copies of their plans to display alternative solutions to achieve the same goal, emphasizing that multiple possible solutions in code could accomplish the same goal.
% < Duplicating plans (D3)
% To accelerate this process of teaching a variety of possible solutions, 
PLAID allows users to ``duplicate'' plans on the canvas and further edit them to present alternative solutions for the same plan \textbf{(DG3}, \cref{fig:system-diagram-2}l).
% Highlight text from solution to change it to changeable areas (highlighting code itself) (D4)

% In conditions A and B, instructors highlighted the changeable areas in the code itself.
% To allow participants to emphasize the changeable areas in code in PLAID, we implemented the ``add to changeable areas'' button. After selecting the changeable piece of code, clicking on this button highlights the text in a different color and adds it to the box of changeable areas to complete the templated plan design (\textbf{D4}).
% Grouping plans into categories (D4)
% < Multiple selection of the boxes (D4)
% < Naming groups of boxes (D4)
To encourage instructors to think about organizing plans in ways that they would present them to students, PLAID provides an open canvas view for instructors that allows them to arrange plans as they prefer. In addition, PLAID supports a ``grouping'' feature (\cref{fig:system-diagram-2}n), which allows instructors to combine plans with similar goals together into one category (\textbf{DG4}).

% A handful of users postulated each plan as an example question that can be used on assessments. They intended to create multiple variants of the same question for students. They suggested that being able to visualize the different categories would be helpful. Using PLAID, users can select multiple patterns together, add them to a group, and name the group \textbf{(D4)}.  % :(

\subsubsection{System Architecture}
The pipeline to create reference materials is implemented in Python, using the state-of-the-art large language model GPT-4o (Model Version: 2024-05-13). The interface for PLAID is implemented as a web application in Python as a Flask webserver, with an SQLite database. The user-facing interface is implemented using HTML, CSS, and JavaScript, with the canvas interactions realized with the library `\textit{interact.js}'. 



\begin{table*}[t!]
  \caption{Relative ranking of 12 sequence design methods (descending order) across five random seed replications of the ML-oracle.}
  \vspace{1ex}
  \label{tab:gfp_seeds}
    \begin{minipage}{.48\textwidth}
      \centering
      \begin{adjustbox}{width=\linewidth,center}
      \begin{tabular}{|c|c|c|c|c|}
        \toprule
            \textbf{Seed 1}                                           & \textbf{Seed 2}                                        & \textbf{Seed 3}                                          & \textbf{Seed 4}                    & \textbf{Seed 5}                                        \\ \midrule 
            \cellcolor[HTML]{E6B8AF}BootGen & \cellcolor[HTML]{E6B8AF}BootGen & \cellcolor[HTML]{E6B8AF}BootGen & \cellcolor[HTML]{E6B8AF}BootGen & \cellcolor[HTML]{E6B8AF}BootGen \\ 
\cellcolor[HTML]{EAD1DC}CMA-ES & \cellcolor[HTML]{CFE2F3}GA Min & \cellcolor[HTML]{CFE2F3}GA Min & \cellcolor[HTML]{CFE2F3}GA Min & \cellcolor[HTML]{F4CCCC}GA Mean  \\ 
\cellcolor[HTML]{F4CCCC}GA Mean  & \cellcolor[HTML]{B5DDCA}BO-qEI & \cellcolor[HTML]{F4CCCC}GA Mean  & \cellcolor[HTML]{F4CCCC}GA Mean  & \cellcolor[HTML]{EAD1DC}CMA-ES \\ 
\cellcolor[HTML]{CFE2F3}GA Min & \cellcolor[HTML]{F4CCCC}GA Mean  & \cellcolor[HTML]{D9D2E9}GA & \cellcolor[HTML]{D9D2E9}GA & \cellcolor[HTML]{B5DDCA}BO-qEI \\ 
\cellcolor[HTML]{C9DAF8}COMs  & \cellcolor[HTML]{D9D2E9}GA & \cellcolor[HTML]{D9EAD3}Auto. CbAS & \cellcolor[HTML]{B5DDCA}BO-qEI & \cellcolor[HTML]{CFE2F3}GA Min \\ 
\cellcolor[HTML]{D9EAD3}Auto. CbAS & \cellcolor[HTML]{D9EAD3}Auto. CbAS & \cellcolor[HTML]{EAD1DC}CMA-ES & \cellcolor[HTML]{EAD1DC}CMA-ES & \cellcolor[HTML]{D9EAD3}Auto. CbAS \\ 
\cellcolor[HTML]{D9D2E9}GA & \cellcolor[HTML]{C9DAF8}COMs  & \cellcolor[HTML]{B5DDCA}BO-qEI & \cellcolor[HTML]{D9EAD3}Auto. CbAS & \cellcolor[HTML]{C9DAF8}COMs  \\ 
\cellcolor[HTML]{B5DDCA}BO-qEI & \cellcolor[HTML]{EAD1DC}CMA-ES & \cellcolor[HTML]{C9DAF8}COMs  & \cellcolor[HTML]{C9DAF8}COMs  & \cellcolor[HTML]{D9D2E9}GA \\ 
\cellcolor[HTML]{FCE5CD}CbAS & \cellcolor[HTML]{D0E0E3}MINs & \cellcolor[HTML]{D0E0E3}MINs & \cellcolor[HTML]{D0E0E3}MINs & \cellcolor[HTML]{FCE5CD}CbAS \\ 
\cellcolor[HTML]{D0E0E3}MINs & \cellcolor[HTML]{FCE5CD}CbAS & \cellcolor[HTML]{FCE5CD}CbAS & \cellcolor[HTML]{FCE5CD}CbAS & \cellcolor[HTML]{D0E0E3}MINs \\ 
\cellcolor[HTML]{FFF2CC}REINFORCE & \cellcolor[HTML]{FFF2CC}REINFORCE & \cellcolor[HTML]{FFF2CC}REINFORCE & \cellcolor[HTML]{FFF2CC}REINFORCE & \cellcolor[HTML]{FFF2CC}REINFORCE \\ 
\cellcolor[HTML]{F8E3A6}GFN-AL  & \cellcolor[HTML]{F8E3A6}GFN-AL  & \cellcolor[HTML]{F8E3A6}GFN-AL  & \cellcolor[HTML]{F8E3A6}GFN-AL  & \cellcolor[HTML]{F8E3A6}GFN-AL  \\ 


 
            \bottomrule
            \end{tabular}
            \end{adjustbox}
      \caption*{(a) UTR Design Bench oracle}
      \label{tab:utr_rankings}
    \end{minipage}%
    \hfill
    \begin{minipage}{.48\textwidth}
      \centering
      \begin{adjustbox}{width=\linewidth,center}
      \begin{tabular}{|c|c|c|c|c|}
        \toprule 
            \textbf{Seed 1}                                                 & \textbf{Seed 2}                                          & \textbf{Seed 3}                                          & \textbf{Seed 4}                            & \textbf{Seed 5}                         \\ \midrule                      
            \cellcolor[HTML]{E6B8AF}BootGen & \cellcolor[HTML]{FCE5CD}CbAS & \cellcolor[HTML]{D0E0E3}MINs & \cellcolor[HTML]{D0E0E3}MINs & \cellcolor[HTML]{D0E0E3}MINs \\ 
\cellcolor[HTML]{FFF2CC}REINFORCE & \cellcolor[HTML]{E6B8AF}BootGen & \cellcolor[HTML]{E6B8AF}BootGen & \cellcolor[HTML]{E6B8AF}BootGen & \cellcolor[HTML]{E6B8AF}BootGen \\ 
\cellcolor[HTML]{D0E0E3}MINs & \cellcolor[HTML]{FFF2CC}REINFORCE & \cellcolor[HTML]{FCE5CD}CbAS & \cellcolor[HTML]{D9EAD3}Auto. CbAS & \cellcolor[HTML]{D9EAD3}Auto. CbAS \\ 
\cellcolor[HTML]{D9EAD3}Auto. CbAS & \cellcolor[HTML]{CFE2F3}GA Min & \cellcolor[HTML]{D9EAD3}Auto. CbAS & \cellcolor[HTML]{FFF2CC}REINFORCE & \cellcolor[HTML]{FFF2CC}REINFORCE \\ 
\cellcolor[HTML]{FCE5CD}CbAS & \cellcolor[HTML]{D0E0E3}MINs & \cellcolor[HTML]{FFF2CC}REINFORCE & \cellcolor[HTML]{FCE5CD}CbAS & \cellcolor[HTML]{FCE5CD}CbAS \\ 
\cellcolor[HTML]{C9DAF8}COMs  & \cellcolor[HTML]{F4CCCC}GA Mean  & \cellcolor[HTML]{F4CCCC}GA Mean  & \cellcolor[HTML]{F4CCCC}GA Mean  & \cellcolor[HTML]{F4CCCC}GA Mean  \\ 
\cellcolor[HTML]{F4CCCC}GA Mean  & \cellcolor[HTML]{D9EAD3}Auto. CbAS & \cellcolor[HTML]{CFE2F3}GA Min & \cellcolor[HTML]{CFE2F3}GA Min & \cellcolor[HTML]{CFE2F3}GA Min \\ 
\cellcolor[HTML]{CFE2F3}GA Min & \cellcolor[HTML]{C9DAF8}COMs  & \cellcolor[HTML]{C9DAF8}COMs  & \cellcolor[HTML]{C9DAF8}COMs  & \cellcolor[HTML]{C9DAF8}COMs  \\ 
\cellcolor[HTML]{D9D2E9}GA & \cellcolor[HTML]{D9D2E9}GA & \cellcolor[HTML]{D9D2E9}GA & \cellcolor[HTML]{D9D2E9}GA & \cellcolor[HTML]{D9D2E9}GA \\ 
\cellcolor[HTML]{F8E3A6}GFN-AL  & \cellcolor[HTML]{F8E3A6}GFN-AL  & \cellcolor[HTML]{F8E3A6}GFN-AL  & \cellcolor[HTML]{F8E3A6}GFN-AL  & \cellcolor[HTML]{F8E3A6}GFN-AL  \\ 
\cellcolor[HTML]{EAD1DC}CMA-ES & \cellcolor[HTML]{EAD1DC}CMA-ES & \cellcolor[HTML]{B5DDCA}BO-qEI & \cellcolor[HTML]{EAD1DC}CMA-ES & \cellcolor[HTML]{EAD1DC}CMA-ES \\ 
\cellcolor[HTML]{B5DDCA}BO-qEI & \cellcolor[HTML]{B5DDCA}BO-qEI & \cellcolor[HTML]{EAD1DC}CMA-ES & \cellcolor[HTML]{B5DDCA}BO-qEI & \cellcolor[HTML]{B5DDCA}BO-qEI \\ 
 \bottomrule  
            \end{tabular}
            \end{adjustbox}
      \caption*{(b) GFP Design Bench oracle}
      \label{tab:gfp_rankings}
    \end{minipage}
\end{table*}

\section{Experiment II: Expert Interviews} \label{sec:experiment_2}
In Experiment I, we evaluated novice teachers' perceptions of our system, including the ideation process and their overall experience using \name{}.
Additionally, an experienced teacher, E1, evaluated the outcomes of the novice teachers. In Experiment II, we shifted our focus to the perspectives of experts \pzh{with varying levels of literature teaching experiences} to gain more feedback. %, specifically teachers with varying levels of experience, 
We conducted expert interviews using a think-aloud protocol and a semi-structured interview with nine teachers to collect insights related to our research questions. 
% Furthermore, in the Findings section, we will present the opinions regarding the unchangeable features of the prototype from I1 to I6 during \pzh{the design process}.

\subsection{Participants}
The study involved nine \pzh{Chinese language} teachers (5 female, 4 male, E1-5, E8-11 in \autoref{tab:teachers}) in a local elementary school, 
% including one novice teacher with 1-3 years of experience, four advanced beginner teachers with 4-6 years of experience, three competent teachers with 7-18 years of experience, and one proficient teacher with 29 years of experience. 
including two novice teachers with less than five years of teaching experience and six expert teachers with at least five years of teaching experience~\cite{booth2021mid}.
% \penguin{
% including two novice teachers with less than five years of teaching experience and six expert teachers with at least five years of teaching experience~\cite{booth2021mid}.
% }
Detailed information about the participants is presented in Table 3.


\subsection{Method}
\penguin{
We conducted interviews offline with E8-E11 and I1-I6 in lab sessions and with E1-E5 online after they freely used it for three days. This setup could help us to gain diverse insights, \eg learnability of \name{} in lab and field environments. 
I1-I6 participated in the evaluation of \name{}'s prototype, and their feedback on the unchanged features in \name{} was also presented in this section. 
}

For \textbf{E8 - E11}, who were not involved in the iterative design process, we conducted offline interviews. The process began with a 5-minute introduction to the research background (\ie the background of interdisciplinary literature instruction). 
This was followed by a 10-minute tutorial on how to use \name{}. 
Participants were then allocated 30 minutes to complete exploration tasks while engaging in a think-aloud protocol. Subsequently, a 15-minute semi-structured interview was conducted.

For \textbf{I1 - I6}, who participated in the evaluation of the prototype process, the procedure was consistent with the same methodology above, except that during the semi-structured interview, we asked for more suggestions regarding the prototype.

For \textbf{E1 - E5}, who engaged in the iterative design process, we conducted online interviews. The \name{} was made available for a duration of 3 days, and a tutorial video was provided. Participants were asked to freely use \name{} during this period to complete exploration tasks. Finally, we conducted a 15-minute semi-structured interview with each expert.


% \subsubsection{Exploration Tasks}
\fanhx{
% We developed the following exploration tasks:
Surrounding our RQs, the interview questions (\autoref{sec:appendix}) are about ideation outcomes, ideation process and perception of \name{}, while the participants were assigned the following exploration tasks:
\begin{itemize}
    \item Task 1: Freely choose 4-15 reading materials, explore two contexts of interest, and add them to the collection.
    \item Task 2: Generate an introduction, course plan, and activities.
\end{itemize}
}
% \subsubsection{Interview Questions}


% \penguin{
% \textbf{RQ1: Ideation outcomes}}
% \begin{itemize}
%     \item What do you think about the quality of the outcomes generated by \name{}?
%     \item Do you think the results generated by our system can assist you in preparing lessons or designing a new interdisciplinary context in the classroom setting?
% \end{itemize}

% \penguin{
% \textbf{RQ2: Ideation process}}
% \begin{itemize}
%     \item Did you feel the task load is high while using \name{}? Specifically, did you feel any increased cognitive load or mental demands?
% \end{itemize}

% \penguin{
% \textbf{RQ3: Perception of \name{}}
% }
% \begin{itemize}
%     \item Do you think our system can assist you in ideating interdisciplinary topics more efficiently?
%     \item Do you think our system can help you better explore connections between texts and the contexts?
%     \item If you were to prepare an interdisciplinary reading case, compared to your usual preparation methods (\eg searching, meeting with other subject teachers, or using ChatGPT), do you think \name{} could support you better?
%     \item Do you trust the output of our system compared to your previous experiences with web searching or using ChatGPT?
% \end{itemize}



\subsection{Findings}
\subsubsection{RQ1: Ideation Outcomes}
Overall, seven of nine teachers confirmed that the outcomes generated by \name{} effectively support literature instruction within the classroom environment. E10 emphasized the comprehensiveness of the text analysis,
\penguin{where she could find the content she wanted within a document more strategically, instead of sifting through countless reference books:} 
\penguin{\textit{``I can abandon varied reference books, (because) the system conducts a comprehensive analysis of the specific content of the texts. It includes themes, content, and key points.''}}
Additionally, E11 noted that the activities could be easily implemented in the classroom: \textit{``yes, these activities (such as music appreciation) can be incorporated into upcoming lessons to increase student engagement.''}
I3 mentioned that the structured outcomes could potentially \textit{``make my teaching process more systematic.''}

\penguin{
We observed the different views between novice and expert teachers on how they embrace the integration of generated activities into real classrooms, though almost all teachers consider \name{}'s outcomes are beneficial for instructions. 
Experienced teachers (E9, E10, E11) preferred their self-centered teaching approach and were more conservative in using the generated activities. %, with three of six indicating they critically and carefully select the provided activities. 
\textit{``To be honest, I have never used these (recommended) activities before, and in the future, I might only add a bit of them to my existing lesson plan to make students more interested.'' }(E11)
On the contrary, newer teachers were more open and tended to reconstruct their established curriculum based on the recommended activities.}
E8 stated, 
\textit{``I plan to adjust my instruction method according to them (activities); some recommended activities are excellent in the current context and can be adapted for use in other contexts.''
}

Despite these positive findings, five of nine teachers pointed out that some outcomes did not align well with the objectives of literature instruction: \textit{``the outcomes are still somewhat disconnected from practical application. The content generated based on the provided template does not fully meet the current teaching needs, especially considering the recently revised curriculum standards''} (E2).
Teachers suggested a potential solution: \textit{``Import the curriculum standards and teaching objectives for each text, and to consider these goals when constructing contexts. For instance, what are the learning objectives and abilities required for each grade level?''} (E5)

Despite modifications made to the prototype, 4 out of 9 experts indicated a need for additional outcomes that could directly enhance student learning in literature: \textit{``the documents generated are very useful for lesson preparation; however, they cannot be directly provided to students for learning purposes. I hope it can produce some homework questions''} (E1).


In summary, the comprehensive and structured outcomes facilitate educational activities; however, additional focus is required to ensure the alignment of these outcomes with the literature objectives.

\subsubsection{RQ2: Ideation process}
Experts have reached a consensus that the task load is low when using \name{} to construct contexts for the classroom, attributable to its well-structured layout and functional settings. E11 said, \textit{``the system does not burden my memory due to the collection feature''.} 
\penguin{
However, E5, who used \name{} in the wild for three days, suggested improvements to its UI design and features to further decrease mental demand: 
\textit{``More specific instructions could be incorporated into the page to clearly indicate the available actions. Additionally, I would like to have a feature that synchronizes generated records through user login for long-term usage.''} 
Experts in the lab sessions commented more on \name{}'s error cases which may increase their effort in the ideation process. 
For example, E8 input ``What activities can be designed around `Stepping Stones' [the title of one text]'', and \name{} responded ``Students can observe stone bridges and steps in their daily lives, explore their design principles and practical uses. They can participate in group projects to build stone bridge models using materials like rocks, experiencing the joy of collaboration.''
E8 commented, \textit{``This is indeed related to the article, but only the content, not the core idea or intended message''}.
E8 did not directly incorporate these outputs into the final outcome; instead, he edited manually.
Additionally, teachers may feel frustrated due to LLM's hallucinations.
I3 input 
``Give me more sentences and analysis related to 'Osmanthus Rain' [the title of one text] and the context about poetic life'', and
LLM responded 
```What I like is osmanthus. The osmanthus tree looks clumsy, unlike the plum tree, which has a graceful posture.' This contrast highlights the author's unique affection for osmanthus and reflects the author's ability to discover poetic elements in life through a lens of beauty''.
I3 remarked that \textit{``it did give me one more sentence but not really fit in the context.''}
In summary, \name{} is generally user-friendly but requires more UI and feature refinements to optimize the ideation process.
% It indicate LLMs can generate information that appears believable but is actually incorrect, instead of simply admitting it don't know the answer.
% This issue arises when users repeatedly ask LLMs similar questions.
}
% Additionally, 
% \fhx{
% as a long-term user, E5 suggested that the user interface design could be improved and history tracking should be provided to decrease mental demand for teachers during long-term use: \textit{``more specific instructions could be incorporated into the page to clearly indicate the available actions and the color scheme could be improved. Additionally, I would like to see the capability to synchronize generated records through user login.''}
% }

% \fhx{
% Based on recordings and experts' comments, error responses to input queries may increase user effort. 
% E8 input ``What activities can be designed around 'Stepping Stones' [the title of one text]'', \name{} responded ``Students can observe stone bridges and steps in their daily lives, explore their design principles and practical uses. They can participate in group projects to build stone bridge models using materials like rocks, experiencing the joy of collaboration.''
% E8 commented, ``This is indeed related to the article, but only the content, but not the core idea or intended message.''
% Finally, E8 did not incorporate these outputs into the final outcome; instead, he edited manually.
% Additionally, more frustration may be brought due to hallucination.
% I3  asked \name{} to provide some specific sentences related to the context of "poetic life" and then input 
% ``Give me more sentences and analysis related to 'Osmanthus Rain' [the title of one text] and the context''.
% LLM responded 
% ``'What I like is osmanthus. The osmanthus tree looks clumsy, unlike the plum tree, which has a graceful posture.' This contrast highlights the author's unique affection for osmanthus and reflects the author's ability to discover poetic elements in life through a lens of beauty.''
% I3 remarked ``it did give me one more sentence, but not really fit'' 
% It indicate LLMs can generate information that appears believable but is actually incorrect, instead of simply admitting it don't know the answer.
% This issue arises when users repeatedly ask LLMs similar questions.
% }

% In summary, our system provides support to teachers in constructing contexts, indicating that the system is user-friendly.

\subsubsection{RQ3: Perception of \name{}}
\penguin{
Both users in the lab session, who learned \name{} through verbal instructions from the developer, and users in the wild, who studied it via a video tutorial, agreed that our system is easy to learn.
}
With the exception of E2, all other teachers affirmed that \name{} supported their exploration of contexts and the interrelationships between texts and contexts. \textit{Expanding thinking} and \textit{inspiring creativity} were mentioned as advantages for supporting context exploration.
\textit{``The system provides a broader range of relationships between context and text, gradually expanding the context. I believe AI should work in this manner, incrementally broadening the scope to help me explore more possibilities, rather than rigidly offering a single answer''} (E11).
Furthermore, E5 added: \textit{``It can provide inspiration, demonstrating how to develop the class based on this context.''}
Besides expanding the breadth of thinking, E10 expressed appreciation for \name{}'s summarization capabilities: \textit{``after selecting a substantial number of texts, I was astonished that it could truly organize them and produce a comprehensive design, which is impossible in my typical lesson preparation.''} These comments indicate that \name{} facilitates users in opening and focusing their cognitive processes during interdisciplinary exploration.
% However, 
% E2 found the recommended contexts to be unreasonable:
% \textit{``The content generated is not suitable; some contexts provided are inappropriate... The choice of words in the analysis is also problematic, with certain terms not conforming to the standards for elementary school students''} (E2). 
% \fhx{
% Moreover, some queries
% }
% The generation of unreasonable content has constrained the system's usability, consequently diminishing user support.
\penguin{
However, E1 and E3 raised an issue about the repetition of suggested contexts when they tried different reading materials in their three-day usage of \name{}. 
E1 noted, \textit{``I found that the same contexts reappearing despite my selection of entirely different reading materials''.} 
% two of five long-term users noticed some challenges with \name{}'s reusability due to the lack of original contexts available in the database.
% E1 noted, \textit{``I found a high frequency of context repetition, with the same contexts reappearing despite my selection of entirely different reading materials on multiple times''.} 
}
% the insufficiency of original contexts impacted the exploration experience.
Despite receiving feedback and modifications implemented after the prototype evaluation, this issue still troubled users and will be further discussed in~\ref{sec:discussion}.

In comparison to other generative tools and web search engines, users have reported a high level of trust in the output of our system, despite occasional acceptable errors. I3 stated, \textit{``I feel it is more closely aligned with reading materials compared to previous tools, although a few analyses of the context are not very accurate. Overall, I still trust this system.''} E4, I4 emphasized their reliance on personal experience and subjective judgment during the exploration process: \textit{``I always trust my own design more. When my ideas are limited, I use this system to evaluate its outputs and determine which content is usable''} (E4).
% I4 expressed a similar opinion: \textit{``I trust this system, but I cannot assert that I completely accept all its outputs. I will consider the actual situation before adopting them.''}

In summary, users perceive \name{} as highly usable and effective, and they generally express trust in its outputs.
\section{Discussion of Assumptions}\label{sec:discussion}
In this paper, we have made several assumptions for the sake of clarity and simplicity. In this section, we discuss the rationale behind these assumptions, the extent to which these assumptions hold in practice, and the consequences for our protocol when these assumptions hold.

\subsection{Assumptions on the Demand}

There are two simplifying assumptions we make about the demand. First, we assume the demand at any time is relatively small compared to the channel capacities. Second, we take the demand to be constant over time. We elaborate upon both these points below.

\paragraph{Small demands} The assumption that demands are small relative to channel capacities is made precise in \eqref{eq:large_capacity_assumption}. This assumption simplifies two major aspects of our protocol. First, it largely removes congestion from consideration. In \eqref{eq:primal_problem}, there is no constraint ensuring that total flow in both directions stays below capacity--this is always met. Consequently, there is no Lagrange multiplier for congestion and no congestion pricing; only imbalance penalties apply. In contrast, protocols in \cite{sivaraman2020high, varma2021throughput, wang2024fence} include congestion fees due to explicit congestion constraints. Second, the bound \eqref{eq:large_capacity_assumption} ensures that as long as channels remain balanced, the network can always meet demand, no matter how the demand is routed. Since channels can rebalance when necessary, they never drop transactions. This allows prices and flows to adjust as per the equations in \eqref{eq:algorithm}, which makes it easier to prove the protocol's convergence guarantees. This also preserves the key property that a channel's price remains proportional to net money flow through it.

In practice, payment channel networks are used most often for micro-payments, for which on-chain transactions are prohibitively expensive; large transactions typically take place directly on the blockchain. For example, according to \cite{river2023lightning}, the average channel capacity is roughly $0.1$ BTC ($5,000$ BTC distributed over $50,000$ channels), while the average transaction amount is less than $0.0004$ BTC ($44.7k$ satoshis). Thus, the small demand assumption is not too unrealistic. Additionally, the occasional large transaction can be treated as a sequence of smaller transactions by breaking it into packets and executing each packet serially (as done by \cite{sivaraman2020high}).
Lastly, a good path discovery process that favors large capacity channels over small capacity ones can help ensure that the bound in \eqref{eq:large_capacity_assumption} holds.

\paragraph{Constant demands} 
In this work, we assume that any transacting pair of nodes have a steady transaction demand between them (see Section \ref{sec:transaction_requests}). Making this assumption is necessary to obtain the kind of guarantees that we have presented in this paper. Unless the demand is steady, it is unreasonable to expect that the flows converge to a steady value. Weaker assumptions on the demand lead to weaker guarantees. For example, with the more general setting of stochastic, but i.i.d. demand between any two nodes, \cite{varma2021throughput} shows that the channel queue lengths are bounded in expectation. If the demand can be arbitrary, then it is very hard to get any meaningful performance guarantees; \cite{wang2024fence} shows that even for a single bidirectional channel, the competitive ratio is infinite. Indeed, because a PCN is a decentralized system and decisions must be made based on local information alone, it is difficult for the network to find the optimal detailed balance flow at every time step with a time-varying demand.  With a steady demand, the network can discover the optimal flows in a reasonably short time, as our work shows.

We view the constant demand assumption as an approximation for a more general demand process that could be piece-wise constant, stochastic, or both (see simulations in Figure \ref{fig:five_nodes_variable_demand}).
We believe it should be possible to merge ideas from our work and \cite{varma2021throughput} to provide guarantees in a setting with random demands with arbitrary means. We leave this for future work. In addition, our work suggests that a reasonable method of handling stochastic demands is to queue the transaction requests \textit{at the source node} itself. This queuing action should be viewed in conjunction with flow-control. Indeed, a temporarily high unidirectional demand would raise prices for the sender, incentivizing the sender to stop sending the transactions. If the sender queues the transactions, they can send them later when prices drop. This form of queuing does not require any overhaul of the basic PCN infrastructure and is therefore simpler to implement than per-channel queues as suggested by \cite{sivaraman2020high} and \cite{varma2021throughput}.

\subsection{The Incentive of Channels}
The actions of the channels as prescribed by the DEBT control protocol can be summarized as follows. Channels adjust their prices in proportion to the net flow through them. They rebalance themselves whenever necessary and execute any transaction request that has been made of them. We discuss both these aspects below.

\paragraph{On Prices}
In this work, the exclusive role of channel prices is to ensure that the flows through each channel remains balanced. In practice, it would be important to include other components in a channel's price/fee as well: a congestion price  and an incentive price. The congestion price, as suggested by \cite{varma2021throughput}, would depend on the total flow of transactions through the channel, and would incentivize nodes to balance the load over different paths. The incentive price, which is commonly used in practice \cite{river2023lightning}, is necessary to provide channels with an incentive to serve as an intermediary for different channels. In practice, we expect both these components to be smaller than the imbalance price. Consequently, we expect the behavior of our protocol to be similar to our theoretical results even with these additional prices.

A key aspect of our protocol is that channel fees are allowed to be negative. Although the original Lightning network whitepaper \cite{poon2016bitcoin} suggests that negative channel prices may be a good solution to promote rebalancing, the idea of negative prices in not very popular in the literature. To our knowledge, the only prior work with this feature is \cite{varma2021throughput}. Indeed, in papers such as \cite{van2021merchant} and \cite{wang2024fence}, the price function is explicitly modified such that the channel price is never negative. The results of our paper show the benefits of negative prices. For one, in steady state, equal flows in both directions ensure that a channel doesn't loose any money (the other price components mentioned above ensure that the channel will only gain money). More importantly, negative prices are important to ensure that the protocol selectively stifles acyclic flows while allowing circulations to flow. Indeed, in the example of Section \ref{sec:flow_control_example}, the flows between nodes $A$ and $C$ are left on only because the large positive price over one channel is canceled by the corresponding negative price over the other channel, leading to a net zero price.

Lastly, observe that in the DEBT control protocol, the price charged by a channel does not depend on its capacity. This is a natural consequence of the price being the Lagrange multiplier for the net-zero flow constraint, which also does not depend on the channel capacity. In contrast, in many other works, the imbalance price is normalized by the channel capacity \cite{ren2018optimal, lin2020funds, wang2024fence}; this is shown to work well in practice. The rationale for such a price structure is explained well in \cite{wang2024fence}, where this fee is derived with the aim of always maintaining some balance (liquidity) at each end of every channel. This is a reasonable aim if a channel is to never rebalance itself; the experiments of the aforementioned papers are conducted in such a regime. In this work, however, we allow the channels to rebalance themselves a few times in order to settle on a detailed balance flow. This is because our focus is on the long-term steady state performance of the protocol. This difference in perspective also shows up in how the price depends on the channel imbalance. \cite{lin2020funds} and \cite{wang2024fence} advocate for strictly convex prices whereas this work and \cite{varma2021throughput} propose linear prices.

\paragraph{On Rebalancing} 
Recall that the DEBT control protocol ensures that the flows in the network converge to a detailed balance flow, which can be sustained perpetually without any rebalancing. However, during the transient phase (before convergence), channels may have to perform on-chain rebalancing a few times. Since rebalancing is an expensive operation, it is worthwhile discussing methods by which channels can reduce the extent of rebalancing. One option for the channels to reduce the extent of rebalancing is to increase their capacity; however, this comes at the cost of locking in more capital. Each channel can decide for itself the optimum amount of capital to lock in. Another option, which we discuss in Section \ref{sec:five_node}, is for channels to increase the rate $\gamma$ at which they adjust prices. 

Ultimately, whether or not it is beneficial for a channel to rebalance depends on the time-horizon under consideration. Our protocol is based on the assumption that the demand remains steady for a long period of time. If this is indeed the case, it would be worthwhile for a channel to rebalance itself as it can make up this cost through the incentive fees gained from the flow of transactions through it in steady state. If a channel chooses not to rebalance itself, however, there is a risk of being trapped in a deadlock, which is suboptimal for not only the nodes but also the channel.

\section{Conclusion}
This work presents DEBT control: a protocol for payment channel networks that uses source routing and flow control based on channel prices. The protocol is derived by posing a network utility maximization problem and analyzing its dual minimization. It is shown that under steady demands, the protocol guides the network to an optimal, sustainable point. Simulations show its robustness to demand variations. The work demonstrates that simple protocols with strong theoretical guarantees are possible for PCNs and we hope it inspires further theoretical research in this direction.
\section{Conclusion}
In this work, we propose a simple yet effective approach, called SMILE, for graph few-shot learning with fewer tasks. Specifically, we introduce a novel dual-level mixup strategy, including within-task and across-task mixup, for enriching the diversity of nodes within each task and the diversity of tasks. Also, we incorporate the degree-based prior information to learn expressive node embeddings. Theoretically, we prove that SMILE effectively enhances the model's generalization performance. Empirically, we conduct extensive experiments on multiple benchmarks and the results suggest that SMILE significantly outperforms other baselines, including both in-domain and cross-domain few-shot settings.

%%
%% The acknowledgments section is defined using the "acks" environment
%% (and NOT an unnumbered section). This ensures the proper
%% identification of the section in the article metadata, and the
%% consistent spelling of the heading.
\begin{acks}
This work is supported by the Young Scientists Fund of the National Natural Science Foundation of China with Grant No.: 62202509 and the General Projects Fund of the Natural Science Foundation of Guangdong Province in China with Grant No. 2024A1515012226.
\end{acks}

%%
%% The next two lines define the bibliography style to be used, and
%% the bibliography file.
\bibliographystyle{ACM-Reference-Format}
\bibliography{references}
% \bibliography{sample-base}


%%
%% If your work has an appendix, this is the place to put it.
\appendix
\fanhx{
\section{Interview Questions in Expert Interviews} \label{sec:appendix}
}
The fixed questions for the \pzh{semi-structured} interviews in Experiment II (expert interviews) are shown below.

\penguin{
\textbf{RQ1: Ideation outcomes}}
\begin{itemize}
    \item What do you think about the quality of the outcomes generated by \name{}?
    \item Do you think the results generated by our system can assist you in preparing lessons or designing a new interdisciplinary context in the classroom setting?
\end{itemize}

\penguin{
\textbf{RQ2: Ideation process}}
\begin{itemize}
    \item Did you feel the task load is high while using \name{}? Specifically, did you feel any increased cognitive load or mental demands?
\end{itemize}

\penguin{
\textbf{RQ3: Perception of \name{}}
}
\begin{itemize}
    \item Do you think our system can assist you in ideating interdisciplinary topics more efficiently?
    \item Do you think our system can help you better explore connections between texts and the contexts?
    \item If you were to prepare an interdisciplinary reading case, compared to your usual preparation methods (\eg searching, meeting with other subject teachers, or using ChatGPT), do you think \name{} could support you better?
    \item Do you trust the output of our system compared to your previous experiences with web searching or using ChatGPT?
\end{itemize}

% \section{Research Methods}

% \subsection{Part One}


\end{document}
\endinput
%%
%% End of file `sample-sigconf-authordraft.tex'.
