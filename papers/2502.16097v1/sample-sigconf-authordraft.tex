%%
%% This is file `sample-sigconf-authordraft.tex',
%% generated with the docstrip utility.
%%
%% The original source files were:
%%
%% samples.dtx  (with options: `all,proceedings,bibtex,authordraft')
%% 
%% IMPORTANT NOTICE:
%% 
%% For the copyright see the source file.
%% 
%% Any modified versions of this file must be renamed
%% with new filenames distinct from sample-sigconf-authordraft.tex.
%% 
%% For distribution of the original source see the terms
%% for copying and modification in the file samples.dtx.
%% 
%% This generated file may be distributed as long as the
%% original source files, as listed above, are part of the
%% same distribution. (The sources need not necessarily be
%% in the same archive or directory.)
%%
%%
%% Commands for TeXCount
%TC:macro \cite [option:text,text]
%TC:macro \citep [option:text,text]
%TC:macro \citet [option:text,text]
%TC:envir table 0 1
%TC:envir table* 0 1
%TC:envir tabular [ignore] word
%TC:envir displaymath 0 word
%TC:envir math 0 word
%TC:envir comment 0 0
%%
%%
%% The first command in your LaTeX source must be the \documentclass
%% command.
%%
%% For submission and review of your manuscript please change the
%% command to \documentclass[manuscript, screen, review]{acmart}.
%%
%% When submitting camera ready or to TAPS, please change the command
%% to \documentclass[sigconf]{acmart} or whichever template is required
%% for your publication.
%%
%%
% \documentclass[manuscript,review,anonymous]{acmart}
\documentclass[sigconf]{acmart}

%%
%% \BibTeX command to typeset BibTeX logo in the docs
\AtBeginDocument{%
  \providecommand\BibTeX{{%
    Bib\TeX}}}

%% Rights management information.  This information is sent to you
%% when you complete the rights form.  These commands have SAMPLE
%% values in them; it is your responsibility as an author to replace
%% the commands and values with those provided to you when you
%% complete the rights form.
% \setcopyright{acmlicensed}
% \copyrightyear{2018}
% \acmYear{2018}
% \acmDOI{XXXXXXX.XXXXXXX}

%% These commands are for a PROCEEDINGS abstract or paper.
% \acmConference[Conference acronym 'XX]{Make sure to enter the correct
%   conference title from your rights confirmation emai}{June 03--05,
%   2018}{Woodstock, NY}
%%
%%  Uncomment \acmBooktitle if the title of the proceedings is different
%%  from ``Proceedings of ...''!
%%
%%\acmBooktitle{Woodstock '18: ACM Symposium on Neural Gaze Detection,
%%  June 03--05, 2018, Woodstock, NY}
% \acmISBN{978-1-4503-XXXX-X/18/06}


%%
%% Submission ID.
%% Use this when submitting an article to a sponsored event. You'll
%% receive a unique submission ID from the organizers
%% of the event, and this ID should be used as the parameter to this command.
%%\acmSubmissionID{123-A56-BU3}

%%
%% For managing citations, it is recommended to use bibliography
%% files in BibTeX format.
%%
%% You can then either use BibTeX with the ACM-Reference-Format style,
%% or BibLaTeX with the acmnumeric or acmauthoryear sytles, that include
%% support for advanced citation of software artefact from the
%% biblatex-software package, also separately available on CTAN.
%%
%% Look at the sample-*-biblatex.tex files for templates showcasing
%% the biblatex styles.
%%

%%
%% The majority of ACM publications use numbered citations and
%% references.  The command \citestyle{authoryear} switches to the
%% "author year" style.
%%
%% If you are preparing content for an event
%% sponsored by ACM SIGGRAPH, you must use the "author year" style of
%% citations and references.
%% Uncommenting
%% the next command will enable that style.
%%\citestyle{acmauthoryear}

% \usepackage{xargs} % Used for new commands with optional arguments
\usepackage{soul}  % Used for custom comments
\usepackage{color} % Used for custom colors in comments
% \usepackage{comment}
\usepackage{xspace}
\usepackage{listings}
\usepackage{enumitem}
% \usepackage{ulem}
%% Note: Some commands for spacing Latin letters/abbreviations
\newcommand{\eg}{{\it e.g.,\ }}
\newcommand{\etal}{{\it et al.\ }}
\newcommand{\etc}{{\it etc.}}
\newcommand{\ie}{{\it i.e.,\ }}
\newcommand{\cf}{{c.f.}\xspace}
\newcommand{\aka}{{a.k.a.}\xspace}

%%%%%%%%%%%%%%%%%%%%%%%%%%%%%%%%%%%%%%%%%%%%%%%%
%% Commands for adding comments to the paper. %%
%%%%%%%%%%%%%%%%%%%%%%%%%%%%%%%%%%%%%%%%%%%%%%%%

\usepackage{booktabs}
\definecolor{oxfordblue}{rgb}{0.0, 0.13, 0.28}
\definecolor{harvardcrimson}{rgb}{0.79, 0.0, 0.09}
\definecolor{dartmouthgreen}{rgb}{0.05, 0.5, 0.06}
\definecolor{princetonorange}{rgb}{1.0, 0.56, 0.0}
\definecolor{yaleblue}{rgb}{0.06, 0.3, 0.57}
\definecolor{usccardinal}{rgb}{0.6, 0.0, 0.0}
\definecolor{uclablue}{rgb}{0.33, 0.41, 0.58}
\definecolor{msugreen}{rgb}{0.09, 0.27, 0.23}
\definecolor{cornellred}{rgb}{0.7, 0.11, 0.11}
\definecolor{pomegranate}{RGB}{192, 57, 43}
\definecolor{anti-pomegranate}{RGB}{43,178,192}
\definecolor{alizarin}{RGB}{231, 76, 60}
\definecolor{anti-belize}{RGB}{185, 41, 56}
\definecolor{belize}{RGB}{41, 128, 185}
\definecolor{peter}{RGB}{52, 152, 219}
\definecolor{green}{RGB}{22, 160, 133}
\definecolor{anti-green}{RGB}{160,22,118}
\definecolor{turquoise}{RGB}{26, 188, 156}
\definecolor{pumpkin}{RGB}{211, 84, 0}
\definecolor{anti-pumpkin}{RGB}{0,22,211}
\definecolor{carrot}{RGB}{230, 126, 34}
\definecolor{wisteria}{RGB}{142, 68, 173}
\definecolor{anti-wisteria}{RGB}{99,173,68}
\definecolor{amethyst}{RGB}{155, 89, 182}
\definecolor{nephritis}{RGB}{39, 174, 96}
\definecolor{anti-nephritis}{RGB}{174,39,117}

% \newcommand{\pzh}[1]{{#1}}
% \newcommand{\peng}[1]{{\color{red} #1}}
% \newcommand{\xingbo}[1]{{\textcolor{black}{#1}}}
% \newcommand{\wxb}[1]{{\textcolor{orange}{#1}}}
% \newcommand{\chen}[1]{{\color{green} #1}}

\newcommand{\penguin}[1]{{#1}}
\newcommand{\pzh}[1]{{#1}}
\newcommand{\peng}[1]{{#1}}
\newcommand{\zhenhui}[1]{{#1}}
\newcommand{\haoxiang}[1]{{#1}}
\newcommand{\yh}[1]{{#1}}
\newcommand{\fhx}[1]{{{#1}}}

% \newcommand{\penguin}[1]{{\color{red} #1}}
% \newcommand{\fhx}[1]{{\textcolor{orange}{#1}}}
% \newcommand{\zhenhui}[1]{{\color{carrot} #1}}

% \newcommand{\penguin}[1]{{\color{blue} #1}}
% \newcommand{\fhx}[1]{{\color{blue}{#1}}}
% \newcommand{\zhenhui}[1]{{\color{blue} #1}}

% \newcommand{\fanhx}[1]{{\color{blue}{#1}}}
\newcommand{\fanhx}[1]{{{#1}}}

%% Note: Comment this in to see all comments and unfinished text.
\newcommand{\todo}[1]{\textcolor{red}{[TODO] \emph{#1}}}
\newcommand{\cut}[1]{\textcolor{red}{\st{#1}}}
\newcommand{\sout}[1]{\cut{#1}}
\newcommand{\gray}[1]{\textcolor{gray}{#1}}


\newcommand{\systemname}{{\textit{SystemName}}}
\newcommand{\name}{{\textit{LitLinker}}}
    
% Capitalizing the first letter for section autorefs
\renewcommand{\sectionautorefname}{Section}
\renewcommand{\subsectionautorefname}{Section}
\renewcommand{\subsubsectionautorefname}{Section}


\usepackage{booktabs}
\usepackage{multirow}
\usepackage{amsmath}
\usepackage{algorithm}
\usepackage{algorithmic}

\copyrightyear{2025}
\acmYear{2025}
\setcopyright{acmlicensed}\acmConference[CHI '25]{CHI Conference on Human Factors in Computing Systems}{April 26-May 1, 2025}{Yokohama, Japan}
\acmBooktitle{CHI Conference on Human Factors in Computing Systems (CHI '25), April 26-May 1, 2025, Yokohama, Japan}
\acmDOI{10.1145/3706598.3714111}
\acmISBN{979-8-4007-1394-1/25/04}


%%
%% end of the preamble, start of the body of the document source.
\begin{document}

%%
%% The "title" command has an optional parameter,
%% allowing the author to define a "short title" to be used in page headers.
% \title{\name: Supporting Ideation of Interdisciplinary Contexts with Large Language Models for Elementary School Literature Instruction}
\title{\name{}: Supporting the Ideation of Interdisciplinary Contexts with Large Language Models for Teaching  Literature in Elementary Schools}


%%
%% The "author" command and its associated commands are used to define
%% the authors and their affiliations.
%% Of note is the shared affiliation of the first two authors, and the
%% "authornote" and "authornotemark" commands
%% used to denote shared contribution to the research.
\author{Haoxiang Fan}
% \authornote{All authors contributed equally to this research.}
\email{fanhx6@mail2.sysu.edu.cn}
\orcid{0009-0000-5729-8491}
\affiliation{%
  \institution{Sun Yat-sen University}
  \city{Zhuhai}
  \country{China}
}

\author{Changshuang Zhou}
% \authornotemark[1]
\authornote{Both authors contributed equally to this research.}
\email{mc34188@um.edu.mo}
\affiliation{%
  \institution{University of Macau}
  \city{Macau}
  \country{Macao}
}

\author{Hao Yu}
\authornotemark[1]
% \authornote{Contributed equally to this research.}
\email{yuhao53@mail2.sysu.edu.cn}
\affiliation{%
  \institution{Sun Yat-sen University}
  \city{Zhuhai}
  \country{China}
}

\author{Xueyang Wu}
% \authornotemark[1]
\email{xwuba@connect.ust.hk}
\affiliation{%
  \institution{NeurlStar}
  \city{Shenzhen}
  \country{China}
}

\author{Jiangyu Gu}
% \authornotemark[1]
% \authornote{Contributed equally to this research.}
\email{3299158551@qq.com}
\affiliation{%
  \institution{Xiangzhou Experimental School of Zhuhai}
  \city{Zhuhai}
  \country{China}
}

\author{Zhenhui Peng}
% \authornotemark[2]
\authornote{Corresponding author.}
\email{pengzhh29@mail.sysu.edu.cn}
\orcid{0000-0002-5700-3136}
\affiliation{%
  \institution{Sun Yat-sen University}
  \city{Zhuhai}
  \country{China}
}

%%
%% By default, the full list of authors will be used in the page
%% headers. Often, this list is too long, and will overlap
%% other information printed in the page headers. This command allows
%% the author to define a more concise list
%% of authors' names for this purpose.
\renewcommand{\shortauthors}{Fan et al.}
\renewcommand{\shorttitle}{\name{}}

%%
%% The abstract is a short summary of the work to be presented in the
%% article.
\begin{abstract}

\pzh{
Teaching literature under interdisciplinary (\eg science, art) contexts that connect reading materials has become popular in elementary schools. However, constructing such contexts is challenging as it requires teachers to explore substantial amounts of interdisciplinary content and link it to the reading materials. In this paper, we develop \name{} via an iterative design process involving 13 teachers to facilitate the ideation of interdisciplinary contexts for teaching literature. Powered by a large language model (LLM), \name{} can recommend interdisciplinary topics and contextualize them with literary elements (\eg paragraphs, viewpoints) in the reading materials. A within-subjects study (N=16) shows that compared to an LLM chatbot, \name{} can improve the integration depth of different subjects and reduce workload in this ideation task. Expert interviews (N=9) also demonstrate \name{}'s usefulness for supporting the ideation of interdisciplinary contexts for teaching literature. We conclude with concerns and design considerations for supporting interdisciplinary teaching with LLMs.  
}

  % Integrating knowledge from various disciplines within literature instruction at the elementary school level is beneficial yet challenging for teachers, especially in identifying suitable interdisciplinary contexts for reading materials. 
  % Large language models (LLMs) can facilitate this process by comprehensively analyzing reading materials, retrieving relevant interdisciplinary topics, and generating contexts, which would otherwise require extensive effort from teachers in terms of text analysis and selection. 
  % Through an iterative design process, which included a foundational study and formative evaluations with seven teachers, we developed \name{}. 
  % This tool supports elementary school teachers in creating interdisciplinary contexts for literature instruction with LLM-generated content.
  % Our within-subjects study ($N=16$) and expert interviews ($N=10$) suggest that \name{} can improve the quality of  interdisciplinary contexts and significantly improve the efficiency of this task.
  % We also address concerns and design considerations for supporting interdisciplinary teaching with LLMs.
\end{abstract}

%%
%% The code below is generated by the tool at http://dl.acm.org/ccs.cfm.
%% Please copy and paste the code instead of the example below.
%%
\begin{CCSXML}
<ccs2012>
   <concept>
       <concept_id>10003120.10003121.10003129</concept_id>
       <concept_desc>Human-centered computing~Interactive systems and tools</concept_desc>
       <concept_significance>500</concept_significance>
       </concept>
   <concept>
       <concept_id>10003120.10003121.10011748</concept_id>
       <concept_desc>Human-centered computing~Empirical studies in HCI</concept_desc>
       <concept_significance>500</concept_significance>
       </concept>
 </ccs2012>
\end{CCSXML}

\ccsdesc[500]{Human-centered computing~Interactive systems and tools}
\ccsdesc[500]{Human-centered computing~Empirical studies in HCI}

%%
%% Keywords. The author(s) should pick words that accurately describe
%% the work being presented. Separate the keywords with commas.
\keywords{Interdisciplinary contexts, ideation, elementary schools, teachers, large language models}
%% A "teaser" image appears between the author and affiliation
%% information and the body of the document, and typically spans the
%% page.
% \begin{teaserfigure}
%   \includegraphics[width=\textwidth]{sampleteaser}
%   \caption{Seattle Mariners at Spring Training, 2010.}
%   \Description{Enjoying the baseball game from the third-base
%   seats. Ichiro Suzuki preparing to bat.}
%   \label{fig:teaser}
% \end{teaserfigure}

% \received{20 February 2007}
% \received[revised]{12 March 2009}
% \received[accepted]{5 June 2009}

%%
%% This command processes the author and affiliation and title
%% information and builds the first part of the formatted document.
\maketitle
\documentclass[../main.tex]{subfiles}
\graphicspath{{../images/}}
\makeatletter
\def\input@path{{../images/}}
\makeatother
\begin{document}
\section{Introduction}
\begin{figure}
\centering
\begin{tikzpicture}
\node[inner sep=0pt] (ws) at (0, 0) {
\includegraphics[height=.4\textwidth, trim={10cm 0 10cm 0},clip]{world_space.png}};
\node[inner sep=0pt] (cs) at (6,0) {\includegraphics[height=.4\textwidth, trim={10cm 1cm 10cm 4cm},clip]{conf_space.png}};
\end{tikzpicture}
\vspace{-5pt}
\label{fig:pbrm_intro}
\caption{\textbf{Left}: Shows world space obstacles as grey spheres. Robots start and goal configuration is colored red and green, respectively. Configurations along the computed path are colored transparent blue. \textbf{Right:} Mapped world space scenario to configuration space. Obstacle region is the grey mesh. Red spheres are collision-free regions computed by the neural SCDF. The optimized shortest path in the convex corridor is the blue curve.}
\vspace{-25pt}
\end{figure}
Motion planning is the problem of finding a collision-free trajectory that connects a given start and goal configuration. The planning takes place in the configuration space of the robot. For single body robots, like mobile robots or drones, the configuration space and the world space are usually the same. This simplifies the planning, since explicit obstacle representations are available which enables geometrical tools like separating hyperplanes, smallest distance to obstacles etc., to be used when designing motion planning algorithms. For multi-body robots like manipulators, the situation is completely different. The world space obstacles are usually mapped to non-convex regions, and to make the problem even harder, the mapping is usually not known. Forming explicit representations of the obstacle region in the configuration space is usually too expensive or intractable. Despite all of this, sampling based planners are used with great success, which mainly is due to their use of implicit representations of the obstacle region. The basic idea is to construct a graph in the configuration space that covers and connects the collision-free region. From this graph, a path can be extracted that connects a given start and goal configuration. The approach is computationally expensive, since the graph is constructed with the smallest geometrical building block available, points, which represents a collision-check. Furthermore, the extracted paths from the graph are non-smooth and jagged due to the stochastic nature of the approach. This adds an additional post-processing step to the process, where the paths are shortcutted and smoothened, before the path can be used for tracking. Clearly a lot of time is invested to form this graph and produce smooth paths. Thus, if the obstacles start to move, then all of this work is done in no use, since all points that make up this graph need to be re-verified, which is simply too time consuming to be done in real time.
\\\\
In this work, we want to address the existing drawbacks of the sampling based planners. Our main contribution is an improved motion planner where each vertex in the graph covers a collision-free region in the form of a sphere instead of a point and where the edges are formed with neighboring intersecting spheres. This representation has the advantage of instead of returning piecewise linear paths, returning a sequence of overlapping spheres, i.e. a convex corridor, that connects a given start and goal configuration, illustrated in Figure \ref{fig:pbrm_intro}. This convex corridor allows us to use convex optimization to produce smooth trajectories, instead of computationally expensive post-processing methods. The representation further allows us to estimate the coverage of the collision-free space, which gives us awareness and feedback in the offline roadmap construction phase. Finally, our representation is simple to adapt to moving obstacles, simply requery for the new radii and recheck for intersections. 
\\\\
The spherical collision-free regions are formed using a signed distance function (SDF), which is a function that returns the smallest distance from an arbitrary point to the boundary of an obstacle. As the name implies, the distance is signed, thus if the point is inside the obstacle it is negative otherwise positive. If the distance is positive, a sphere with radius equal to the distance is guaranteed to cover a collision-free region. Using an SDF in motion planning is not new, but what is novel about our approach is that we express the distance in the configuration space instead of the world space and by doing so allows us to form these convex collision-free regions. We refer to the resulting SDF as a signed configuration distance function (SCDF). Computing an SCDF analytically is non-trivial, our approach is therefore to parameterize the SCDF with a deep neural network and learn the mapping by supervised learning. Our resulting neural SCDF can compute distances for different parameter values of obstacle shapes and we also show how multiple distances can be combined, thus making our approach flexible.
\section{Related work}
Motion planning algorithms can roughly be divided into three families, grid-based, sampling based and optimization based methods. Grid-based methods (GBM) discretize the planning space from which a graph is then compiled. A standard search method is A$^\star$ \citep{a_star}, which is classified as an \textit{informed} search method, since it employs a heuristic function to speed up the search. A$^\star$ guarantees to return an optimal path at the level of discretization used. GBMs usually discretize the planning space by a regular lattice and this limits the GBMs to problems with low dimensionality due to the curse of dimensionality. Thus, GBMs are usually limited to single-body robots where the degrees of freedom (DOF) are low. To overcome the inherent scaling problem with the GBMs, stochastic methods are usually used for multi-body robots. These methods are termed as sampling-based methods (SBM) and core members within this family are the rapidly-exploring random trees (RRT) \citep{rrt} and the probabilistic roadmap (PRM) \citep{prm}. RRT grows a tree from the start configuration and explores the collision-free region in a rapid way until it is able to connect to the goal region. RRT is usually improved by bi-directional planning \citep{rrt_connect}, i.e. an additional tree is grown from the goal configuration and the trees are tested for connection after any tree has been expanded. RRT is a single-query method, thus it searches for a path from scratch each time it is queried. Contrary to this, PRM is a multi-query method, which solves for multiple queries without starting from scratch. PRM does this by creating a roadmap (graph) that covers the collision-free space as an offline step. The graph is then used to solve for multiple queries. PRMs are used in cases where the environment does not change since the extra offline step is too computationally costly and needs to be re-done if the environment is changed. In our work, we address this inherent issue by using a different roadmap representation. Our vertices in the graph cover a collision-free region in the form of spheres and we form the edges by checking for intersecting spheres. If something in the environment changes, we recompute the spheres radii and recheck the intersections, without relying on collision detection. We use a trained neural network to compute the sphere radius, therefore querying for the radius can be done fast, hence our representation enables the PRM for dynamic environments.
\\\\
In the recent decades, optimization based methods (OBM) \citep{chomp, schulman, itomp, stomp} have been introduced as an alternative to SBM for multi-body robots. Like the SBM, the OBMs scale well to higher dimensional problems and produce smoother motion. It is common to use a SDF in the optimization since it is a smooth function, thus enabling gradient-based methods. However, the standard way of expressing the SDF is in world space. The distance therefore needs to be mapped to the configuration space by the forward kinematics. This mapping makes the optimization problem a non-linear program (NLP), which is computationally expensive to solve. Recently, a different approach has been proposed. In \cite{mp_gcs} motion planning is formulated as a convex optimization problem by using the graph of convex sets framework \citep{gcs}. The underlying idea is to decompose the collision-free space into intersecting convex sets from which a convex optimization problem is formulated. In cases where an explicit representation of the obstacles in the configuration space exists, like for single-body robots, creating collision-free convex regions can be done fast \citep{iris}. For multi-body robots, this is non-trivial. Existing work does this successfully \citep{iris_nlp, iris_c} by an optimization based approach, but the methods are still too time consuming to be used in the presence of moving obstacles. Our approach is instead to use deep learning to learn an SDF expressed in the configuration space. With this, we can query for shortest distances to the collision boundary, which allows us to expand spherical regions which are collision-free. Our approach is fast and therefore enables our suggested roadmap planner to be used in dynamic environments.
\\\\
Recent research has focused on learning collision detection \citep{fk_kernel_distance, diffco, graphdistnet} by predicting the signed distance between the robot links and the surrounding obstacles in the world space. The learned SDF is used in trajectory optimization but since the distance is expressed in the world space, the problem becomes an NLP and therefore takes a long time to solve. We take a novel approach and suggest to instead express the signed distance in the configuration space. This allows us to improve the PRM at the same time as it enables convex optimization for trajectory optimization, which runs faster and is more reliable than NLP solvers. In \cite{cspf} a learned signed distance function in the configuration space is proposed similar to our approach. However, their approach is restricted to point cloud representations, while we propose to represent the obstacles as parameterized geometric shapes, e.g. spheres. Furthermore, we also show how to use our learned SCDF to improve an existing roadmap planner.
\section{Problem formulation}
A robot is located in the world space, $\W \subset \R^3 $. The unique location of the robot is given by its configuration $\q \in \C$, where $\C$ is the configuration space. The set of points covered by the robots bodies at a certain configuration is expressed as $\B(\q) \subset \W$. The robot is surrounded by $\NrObst$ obstacles $\O = \bigcup_{i=1}^{\NrObst} \O_i$, where  $\O_i \subset \W$. The representation of the obstacle in the configuration space is the set $\C\O_i = \{\q \in \C \: |\: \B(\q) \cap \O_i \neq \emptyset \}$. The obstacle space is formed as $\Co = \bigcup_{i=1}^{\NrObst} \C \O_i$. The complement is referred to as the free space, $\Cf = \C \setminus \Co$. The path planning problem is a tuple, ($\Cf$, $\qStart$, $\qGoal$), where we want to connect a query pair, consisting of a start, $\qStart$, and goal configuration, $\qGoal$, with a geometric path, $\q(s): [0, 1] \mapsto \Cf$, such that $\q(0)=\qStart$ and $\q(1)=\qGoal$, or report correctly when such a path does not exist.
\end{document}
 
\section{Related Work}
% \subsection{Vision Language Model}
% 시각장애인에서 상황을 설명할 DB가 없으니 만들었다. 그리고 이를 VLM에 튜닝했다.
\subsection{Technical approaches for assisting the visually-impaired}


\subsection{Datasets for visual instruction tuning}

\pzh{
\section{Design Process and Principles of \name{}}
\label{sec:design_process}
Our work aims to support elementary literature teachers in effectively identifying suitable interdisciplinary contexts for their instructions, which can be used in their later lesson plans and classroom activity designs. 
\penguin{
Our design process and evaluation of \name{} involve in total of 17 Chinese language teachers (\autoref{tab:teachers}) in an elementary school in mainland China. 
% To achieve this, we invite total 17 Chinese language teachers (\autoref{tab:teachers}) in an elementary school in mainland China.
Specifically, in the design process, we involved E1-E7 in the foundational study and I1-I6 in the evaluation of the prototype. 
In the evaluation of \name{} with teachers (\ie Experiment II), we involved E1-E5 again and E8-11. 
In Experiment II, I1-I6 also contributed findings about the unchanged features between \name{} and its prototype. 
% and I1-I6 are involved in the iterative design process, including a foundational study and a evaluation of the prototype.
}

% The initial phase, foundational study, comprised three sessions of semi-structured interviews with an experienced literature teacher (E1).
% Before each session, E1 met with her seven-member interdisciplinary literature course design team (members: E1 - E7), documented the meetings, and subsequently reported these findings during interviews with the authors. 
% The second phase, we developed a workable prototype of \name{} and evaluated it with other six teachers (I1 - I6). 
% }
% We gathered their insights to inform our design goals for \name{} presented in this paper. %conducted a \textbf{formative user evaluation}

% \fhx{E1-E5 and E8-E11 participate in Experiment II (\ie Expert Interviews) following the development of \name{}. We will discuss the results in~\autoref{sec:experiment_2}}


\begin{table*}[htbp]
\caption{17 Chinese language teachers participated in the iterative design process and expert interviews (\ie Experiment II). Among them, there were 6 males and 11 females, with teaching experience ranging from 3 to 29 years. Two participants did not provide information on their teaching experience. This table also includes their experience in reading projects and interdisciplinary projects.}
\Description{17 Chinese language teachers participated in the iterative design process and expert interviews (\ie Experiment II). Among them, there were 6 males and 11 females, with teaching experience ranging from 3 to 29 years. Two participants did not provide information on their teaching experience. This table also includes their experience in reading projects and interdisciplinary projects.}
\label{tab:teachers}
\begin{tabular}{@{}cccccc@{}}
\toprule
\textbf{Involvement}                                                                                                                               & \textbf{ID} & \textbf{Gender} & \begin{tabular}[c]{@{}c@{}}\textbf{Teaching Experience} \cr \textbf{(years)}\end{tabular} & 
\begin{tabular}[c]{@{}c@{}}\textbf{Participation in} \cr \textbf{Reading Projects}\end{tabular} & 
\begin{tabular}[c]{@{}c@{}}\textbf{Participation in} \cr \textbf{Interdisciplinary Projects}\end{tabular} \cr \midrule
\multirow{5}{*}{\begin{tabular}[c]{@{}c@{}}\textit{Foundational}\cr \textit{Study}\cr \textit{\&}\cr \textit{Experiment II}\end{tabular}}                                        & E1          & F               & 5                                                                               & Y                                                                                     & Y                                                                                               \cr
                                                                                                                                                   & E2          & M               & 7                                                                               & Y                                                                                     & Y                                                                                               \cr
                                                                                                                                                   & E3          & F               & 14                                                                              & Y                                                                                     & Y                                                                                               \cr
                                                                                                                                                   & E4          & M               & 6                                                                               & Y                                                                                     & Y                                                                                               \cr
                                                                                                                                                   & E5          & M               & 4                                                                               & Y                                                                                     & Y                                                                                               \cr \hline
\multirow{2}{*}{\begin{tabular}[c]{@{}c@{}}\textit{Foundationall}\cr \textit{Study}\end{tabular}}                                                             & E6          & F               & -                                                                               & Y                                                                                     & Y                                                                                               \cr
                                                                                                                                                   & E7          & F               & -                                                                               & Y                                                                                     & Y                                                                                               \cr \hline
\multirow{6}{*}{\begin{tabular}[c]{@{}c@{}} \textit{\penguin{Evaluation of}}\cr \textit{\penguin{Prototype (re-usable}}\cr \textit{\penguin{findings are presented}}\cr \textit{\penguin{in Experiment II)}}\end{tabular}} & I1          & F               & 27                                                                              & Y                                                                                     & Y                                                                                               \cr
                                                                                                                                                   & I2          & F               & 8                                                                               & Y                                                                                     & Y                                                                                               \cr
                                                                                                                                                   & I3          & F               & 5                                                                               & N                                                                                     & N                                                                                               \cr
                                                                                                                                                   & I4          & M               & 11                                                                              & Y                                                                                     & Y                                                                                               \cr
                                                                                                                                                   & I5          & F               & 5                                                                               & Y                                                                                     & N                                                                                               \cr
                                                                                                                                                   & I6          & M               & 5                                                                               & Y                                                                                     & N                                                                                               \cr \hline
\multirow{4}{*}{\begin{tabular}[c]{@{}c@{}}\textit{\penguin{Expert}}\cr \textit{\penguin{Interviews}}\end{tabular}}                                                              & E8          & M               & 3                                                                               & N                                                                                     & N                                                                                               \cr
                                                                                                                                                   & E9          & F               & 6                                                                               & Y                                                                                     & N                                                                                               \cr
                                                                                                                                                   & E10         & F               & 29                                                                              & N                                                                                     & Y                                                                                               \cr
                                                                                                                                                   & E11         & F               & 17                                                                              & Y                                                                                     & N                                                                                               \cr \bottomrule
\end{tabular}
\end{table*}

\subsection{Design Process}
\penguin{
We generally followed a user-centered approach to plan our design process. 
First, to understand users' practices and involve them in the design of \name{}, we conducted three sessions of semi-structured interviews with an experienced literature teacher E1, who led a seven-member (E1 - E7) interdisciplinary literature course design team that indirectly contributed to the interviews. 
We were not able to have direct discussions with E2 - E7 due to their inconvenience during the semester. 
Then, we developed a workable prototype of \name{} and evaluated it with another six teachers (I1 - I6).  
We gathered their insights to inform our design goals for \name{} presented in this paper. 
}
\subsubsection{Foundational Study}
We closely worked with E1 to identify the practices, challenges and needs for support of ideating interdisciplinary contexts for teaching literature in elementary schools. 
Over the past two years, E1 has spearheaded a team of seven individuals (E1-E7) in the exploration and implementation of interdisciplinary literature instruction within elementary school Chinese courses. 
\fanhx{
To gain a comprehensive understanding of user needs, we progressively conducted three sessions of semi-structured interviews with E1 in April, June, and July 2024, lasting 38 minutes, 53 minutes, and 45 minutes, respectively.
}
Before each session, we communicated the purpose of the interview to E1 and requested that she engage with her team to compile records of their meetings for discussing the topics in the intended interview. 
We documented each interview session with E1 through audio and video recordings. % for each interview.

In Session 1, 
\fanhx{we asked E1 to present their current practices of teaching literature in interdisciplinary contexts, with previously developed lesson plans and assignments in her team.
} 
The discussion also focused on the potential of \fanhx{AI (\eg what do you think AI can support you (in your lesson planning in previous))}, and an interactive system to facilitate the design of interdisciplinary literature contexts, including the integration of art and history into the assignments \peng{of literature reading}. 
\peng{After this section}, two authors brainstormed potential features of a system for supporting the ideation of interdisciplinary contexts and sent E1 a document that explains these features. %(\eg xxx \peng{[any high-level features?]})
% , who subsequently 
We requested E1 to engage in a discussion with her group members to identify any additional or incorrect points about the \peng{potential system}. 
In Session 2, E1 \peng{came back with positive feedback from her team on each potential feature}.
\fanhx{We asked her to further explain their general process for designing interdisciplinary contexts as a team, emphasizing the distinct responsibilities and cognitive processes of each teacher involved.} 
Additionally, she presented the proposed interaction model. 
After this session, E1 had a group meeting with her team and came up with a template that defines the anticipated outcomes of our system. 
% Following a group meeting, a template defining the anticipated outcomes of the system was subsequently provided to the authors.
In Session 3, we introduced how a system works utilizing LLM agents \peng{to simulate roles in a team for ideating the interdisciplinary contexts, as suggested by E1 in Session 2}. 
% , detailing the specific roles of each agent. 
We presented two example outcomes produced using our predefined prompts and intermediate outputs 
\fanhx{to ask for her opinions on these prompts and outputs (\eg whether these intermediate outputs were helpful? If the prompts aligned well with your thoughts?).
}
}


\pzh{
\subsubsection{Development and Evaluation of \name{} Prototype}
\peng{
After Session 3, two of the authors utilized the thematic analysis method to analyze the transcribed recordings and all textual content derived from the foundational study. 
The analysis yielded four summarized Design Principles as described in the following \autoref{sec:principles}. 
We then worked on the implementation of a workable prototype that chains different LLM agents in a structured process to help teachers think of interdisciplinary contexts for teaching the literature materials in the textbooks. 
}
% \subsubsection{Evaluation of \name{} Prototype}
We evaluated our workable prototype with another six elementary school Chinese language teachers (I1 - I6, 3 Male, 3 Female), as shown in \autoref{tab:teachers}. 
Each evaluation lasted approximately 30 to 45 minutes and comprised four parts: (1) an introduction to the background, which included the concepts of interdisciplinary literature instruction, and the theory of contexts of instruction; (2) a brief tutorial on the prototype; (3) a think-aloud study in which participants freely explored the prototype and spoke out their thoughts; and (4) a semi-structured interview \peng{for their comments and suggestions on the prototype}. With the participants' consent, we conducted the evaluation offline and recorded audio and video.

At this stage, we assessed the prototype's usability (\eg whether different functions were well-integrated), user perception (\eg user interaction with the prototype and any additional cognitive load), and the quality of the system's outcomes. Furthermore, we collected suggestions for improving the prototype, particularly regarding user interface design and additional functionalities. 
% \peng{These feedback and suggestions helped us refine the design principles, which are incorporated in \autoref{sec:principles} Design Principles}.
\penguin{These feedback and suggestions are presented in \autoref{sec:formative_findings}, which inform the design principles (\autoref{sec:principles}) of \name{}.}
% \peng{Besides, the prototype in this evaluation study is close to the final version of \name{}.
% The key refinements lie in xxx, xxx, and xxx. 
For the six teachers' (I1-6) feedback on the same features in the prototype and final version of \name{}, we incorporate it in the results of Expert Interviews in \autoref{sec:experiment_2}. 
}
% In this section, we concentrate on gathering feedback and findings to establish design goals and incorporate new features into the prototype. Feedback on other unchanged aspects will be reported in the Section Expert Interviews.

% \fhx{
% \subsection{Findings and Design Principles} \label{sec:principles}
% }

\penguin{
\subsection{Findings}
\label{sec:formative_findings}
}
\fhx{
Two of the authors utilized thematic analysis~\cite{braun2012thematic} to inductively code and summarize the \penguin{practices, challenges, requirements, and concerns} from transcribed recordings \penguin{in the design process. The key themes are shown below.} % and all textual content from the foundational study.
\fanhx{One author first iteratively coded the data, while the other carefully reviewed the codes to ensure accuracy. After discussions, they reached a consensus and identified six primary themes. These findings are shown below.}
}


\penguin{
\textbf{Finding 1: In practice, teachers usually engage in reverse thinking when ideating interdisciplinary context}.}
% \fhx{\textbf{Finding 1: Two opposing intellectual paradigms of interdisciplinary context ideation.}}
% Teachers engage in two distinct intellectual paradigms when ideating contexts. One paradigm, referred to as ``forward thinking'', resembles ``deductive reasoning''. 
\penguin{Our teachers mentioned two intellectual paradigms.}
One paradigm referred to as ``forward thinking'', resembles ``deductive reasoning'', in which
teachers create abstract connections from a limited number of reading materials (typically 3-5 texts) and develop a concrete and reasonable context. 
The other paradigm, which is termed ``reverse thinking'' and analogous to ``inductive reasoning'', is a more habitual cognitive process employed by teachers. 
% \penguin{To conduct ``reverse thinking'', which is analogous to ``inductive reasoning'',}
Teachers would like to first select interdisciplinary contexts that they deem suitable and then identify appropriate texts from a broader text pool, after which they refine the connections between the identified text and the context. 
\textit{``For us, a good context often arises from a sudden inspiration, which we then backtrack to complete the ideation of what texts can connect to this context and how''} (E1).

\penguin{
\textbf{Finding 2: It is challenging to identify the connections between the established context and the reading materials  in the process of ``reverse thinking''.} 
}
% \fhx{\textbf{Finding 2: Extraction of connections from different subjects is challenging.}}
% In the process of ``reverse thinking'', identifying the connections between the established context and the reading materials is challenging. 
Teachers must evaluate the effectiveness of the connections in enhancing students' understanding of both literature and its associated subjects, as well as in stimulating their interest. 
\textit{``It is quite difficult and usually takes a long time for our team to ensure that based on the reading materials, our teaching activities connected by the context can indeed help students gain knowledge''} (E1).

\penguin{
\textbf{Finding 3: Teachers require support at three levels of granularity when analyzing the reading materials and contexts.}
}
% \fhx{\textbf{Finding 3: Typically employing three levels of granularity of analysis are needed.}}
% During the ideation process, to help teachers prepare for the teaching activities, the system should help teachers analyze and understand the reading materials and contexts at three three levels of granularity. 
The first level is in-depth single-text analysis, which \textit{``explains how the elements of a given article relate to the context''} (E1). 
The second level is pairwise comparison, where comparative reading has been demonstrated to be an effective method for understanding texts, \textit{``allowing articles to `disappear in pairs' by analyzing the similarities and differences in relation to the context''} (E1). 
The third level is multi-text-driven exploration, which necessitates that the system should support the comprehensive linking of all texts selected by the teacher. Therefore, this level requires \name{} to conduct a thorough deconstruction of contexts, extract meaningful connections, and convey these connections to the teachers. 
% Besides, the analyses should also be accurate, especially in the filed of education. This not only necessitates high-quality AI-generated content but also emphasizes the assurance of editing all contents by teachers freely within the user interface. \textit{``I want to freely edit rather than just drag or choose fixed options. This allows me to directly correct issues when I discover them''} (E1). 
% Additionally, teachers believe that the system should enable them to pose detailed and flexible questions about the contexts and reading materials.  \textit{``Our team hopes that teachers can input vague, open-ended questions to the LLM''} (E1).

\penguin{
\textbf{Finding 4: 
Teachers require detailed instructional activities based on the selected contexts.}}
% \fhx{\textbf{Finding 4: Ensuring enough details for instructional activities.}}
% \fhx{
% E1 emphasized that finding a precise interdisciplinary context is the first step in preparing a series of instructional activities, and that manually developing comprehensive teaching resources tailored to the teaching practices based on the selected context is necessary but challenging.
% }
\penguin{As E1 summarized after the meeting with her teaching team, the outcome plan of several lessons surrounding a context should include targeted reading materials and analysis in each lesson, an introduction facilitating students' engagement, and related teaching activities.}
% Enough teaching resources should at least include 
% comprehensive lesson planning, relevant explanations for each segment, an introduction to facilitate classroom engagement, and specific analyses of the texts, summarized by E1. 
Also, in the evaluation study of \name{} prototype, three teachers (I1, I2, I5) indicated that the system outputs should be more detailed and reduce human effort in modifying them for the later concrete plans for each lesson. % lesson design and instructional materials.
\textit{``The overall structure of the output is good, but I hope it can be more detailed; for example, providing more in-class and extracurricular activities related to the theme, so we can use them directly''} (I1). Therefore, we incorporated recommendations for literature and interdisciplinary course activities in the refined \name{}.

\penguin{
\textbf{Finding 5: Teachers are concerned the quality and reliability of the content purely generated by LLMs.}
}
% \fhx{\textbf{Finding 6: LLM-generated content is unsatisfying due to high repetitiveness and low reliability.}}
In the evaluation study with six teachers (I1-6), our prototype generated traditional subject-related contexts using the LLM with specific templates and cognitive backgrounds of elementary students, without fine-tuning or retrieval-augmented generation (RAG). 
Three teachers (I4, I5, I6) expressed concerns about the quality of the LLM-generated content. \textit{``The content generated for the art subject is quite repetitive''} (I5). \textit{``We need to establish a dedicated article database for science as well, since many of our articles are highly relevant to science''} (I4). 
% Therefore, in the final version of \name{}, we have collected contexts \peng{and articles} from various subjects and use them as a foundation for generating content, which could reduce repetition and increase reliability of suggested context. 


\penguin{
\textbf{Finding 6: Teachers suggest six metrics for evaluating the outcome of interdisciplinary literature lesson plan.}
% In foundational study, the team of E1-E7 established six metrics for evaluating the quality of contexts and teaching resources.
As established by the team of E1-E7, the metrics are: 
}
% \fhx{\textbf{Finding 5: Metrics for evaluation.}}
% In foundational study, the team of E1-E7 established six metrics for evaluating the quality of contexts and teaching resources.
\begin{itemize}
    \item \textbf{Appropriateness of Context}
    \begin{itemize}
        \item \textit{Content Alignment:} Does the context accurately cover the content of the selected materials?
        \item \textit{Internal Logic:} Is there a logical connection between the context and the selected materials?
    \end{itemize}
    \item \textbf{Alignment with Educational Objectives}
    \begin{itemize}
        \item \textit{Curriculum Standards:} Does the content comply with national curriculum standards and teaching guidelines?
        \item \textit{Subject Goals:} Does it help achieve specific goals of language education, such as reading comprehension and writing skills?
    \end{itemize}
    \item \textbf{Depth of Integration}
    \begin{itemize}
        \item \textit{Subject Integration:} Does it effectively integrate knowledge from different subjects?
        \item \textit{Knowledge Transfer:} Does it promote the application of language arts knowledge in other subject contexts?
    \end{itemize}
\end{itemize}


% We summarize the design principles (\textbf{DPs}) of \name{} derived from the foundation study and evaluation of the prototype as below. 

% We employed the reflexive thematic analysis method to analyze the transcribed recordings and all textual content derived from the foundational study. The analysis yielded four summarized Design Principles. In a similar manner, we analyzed the \textbf{formative user evaluation}, incorporating usability issues identified by the teachers and categorizing them under the corresponding DP1 - DP4.

\subsection{Design Principles} \label{sec:principles}

\penguin{Based on the findings from our design process and related literature, we derive four design principles of \name{}.}
% \fhx{Based on the findings from our iterative design process, we propose four human-centered design principles for the development of \name{}.}

% \textbf{DP1: \name{}'s creative thinking support process should align with teachers' habitual behaviors in interdisciplinary context ideation.} % (Finding 1, Finding 2)
\penguin{
% \textbf{DP1: \name{}'s ideation support process should align with teachers' habitual practices in interdisciplinary context ideation.}
\textbf{DP1: \name{} should provide step-by-step support that aligns with teachers' habitual practices in interdisciplinary context ideation}.
Tailoring the assistance to users' habitual practices (\eg active students' behaviors or teachers' behaviors) is a commonly enacted principle in previous interactive systems in educational scenarios~\cite{fok2024qlarify,liu2024classmeta,fan2024lessonplanner}. 
% For example, LessonPlanner \cite{fan2024lessonplanner} adapts the nine events of Gagne's instructional theory to support the planning of one lesson, 
% and ClassMeta \cite{liu2024classmeta} displays various behaviors commonly observed among active students to promote VR classroom participation. 
In the task of interdisciplinary context ideation, as revealed in Finding 1, \name{} should support step-by-step context ideation through reverse thinking, which is a habitual practice of our teachers. 
\fanhx{In this practice, as E1 shared,} teachers take various roles to analyze the potential contexts 
\fanhx{(we note this role as Context Analyst)}, 
analyze the texts in the reading materials, connect them to the contexts, 
\fanhx{and discuss the approaches (Text Analyst and Text Reviewer)}. 
After that, teachers try to summarize the contexts and associated reading materials into an actionable lesson plan \fanhx{(Context Summarizer)}. 
\name{} can prompt LLMs to play different roles when supporting teachers in each of these steps. 
We do not chase for generating one-step context ideation outcome with one LLM prompt, because teachers desire necessary human input in each step, and enabling multiple LLM agents to simulate human-human collaboration has been proven to improve output quality~\cite{wu2024transagents, du2024multi}. 
}
% \textit{DP1.1: \name{} should support context ideation through reverse thinking (Finding 1).} 
% Teachers engage in two distinct cognitive processes when ideating contexts. The first process, referred to as ``forward thinking'', resembles ``deductive reasoning''. 
% Teachers draw abstract connections from a limited number of reading materials (typically 3-5 texts) and develop a concrete and reasonable context. 
% The second, which is termed ``reverse thinking'' and analogous to ``inductive reasoning'', is a more habitual cognitive process employed by teachers. Teachers would like to first select interdisciplinary contexts that they deem suitable and then identify appropriate texts from a broader text pool, after which they refine the connections between the identified text and the context. \textit{``For us, a good context often arises from a sudden inspiration, which we then backtrack to complete the ideation of what texts can connect to this context and how''} (E1).

% \textit{DP1.2: \name{} should facilitate the extraction of potential connections among the reading materials (Finding 2).}
% \subsubsection{DP1.2: Facilitates the extraction of relevant connections from reading materials.}
% In the process of ``reverse thinking'', identifying the connections between the established context and the reading materials is challenging. 
% Teachers must evaluate the effectiveness of the connections in enhancing students' understanding of both literature and its associated subjects, as well as in stimulating their interest. 
% \textit{``It is quite difficult and usually takes a long time for our team to ensure that based on the reading materials, our teaching activities connected by the context can indeed help students gain knowledge''} (E1).

% \fhx{
% We consider human habitual behaviors to design the system, aligning with the human-centered system for educational scenarios~\cite{fok2024qlarify,liu2024classmeta,fan2024lessonplanner}. 
% Additionally, for LLM-empowered systems, enabling LLM agents to simulate human-human collaboration has been proven to improve output quality~\cite{wu2024transagents, du2024multi}. 
% In summary, DP1 is formulated by integrating the advantages of aligning system's performance with teacher's habitual practices and the insights from Finding 1 and Finding 2.
% }
% \subsubsection{\textbf{DP1: \name{} supports creative thinking modes that align with teachers' habitual practices in context ideation}}

% \penguin{
% \textbf{DP2: \name{} should act as a context analyst that helps  from a comprehensive database of interdisciplinary contexts and articles}
% Especially, \textit{DP1.2) \name{} should facilitate the extraction of potential connections among the reading materials}, as Finding 2 reveals that this is a challenging step in the ideation process. 

% }
\penguin{
\textbf{DP2: \name{} should provide teachers with detailed analyses of the contexts, reading materials, and their relationships.}
% Ideating interdisciplinary context for literature teaching requires integrating knowledge across multiple subjects (\eg science, history), which teachers may not be familiar with. 
Our teachers reported that it was challenging to identify the connections between contexts and reading materials (Finding 2) and desired support during the analyses (Finding 3). 
Previous HCI works have demonstrated the strengths of LLMs in analyzing and connecting complex information~\cite{zheng2024disciplink, chi24_Selenite}. 
% For example, DiscipLink~\cite{zheng2024disciplink} can prompt LLMs to automatically expand queries with disciplinary-specific terminologies and highlight the connections between retrieved papers and questions. 
% Selenite~\cite{chi24_Selenite} employs LLMs to generate comprehensive overviews of options and criteria grounded in search results, guiding users through complex decisions. 
Similarly, to satisfy user requirements (Finding 3) in our task, \name{} can leverage LLMs to recommend contexts, explain them in detail, identify relevant texts in the reading materials, and assess the relationship between the contexts and texts. 
}

% \textbf{DP2: \name{} should provide teachers with detailed analyses of the reading materials and contexts and 
% % precise
% \fhx{support verification of the generated content}. }
% \fhx{
% Interdisciplinary context ideation involves integrating knowledge across multiple subjects, which makes it challenging for elementary literature teachers to assess whether the generated content is comprehensive and accurate in unfamiliar domains (\eg mathematics, science).
% A large amount of complex interdisciplinary information may lead users to feel ``information overload''~\cite{foster2004nonlinear, newby2011entering}.
% Therefore, reducing the cognitive load on educators when understanding generated content is important. 
% On one hand, system-generated content should be detailed to minimize misunderstandings, 
% requiring multiple levels of granularity in analysis and explanation (Finding 3). 
% On the other hand, systems can support users in verifying the generated content to facilitate the integration of information~\cite{foster2004nonlinear} by providing reliable and critical feedback.
% }


% During the ideation process, to help teachers prepare for the later teaching activities, the system should help teachers analyze and understand the reading materials and contexts at three three levels of granularity. 
% The first level is in-depth single-text analysis, which \textit{``explains how the elements of a given article relate to the context''} (E1). 
% The second level is pairwise comparison, where comparative reading has been demonstrated to be an effective method for understanding texts, \textit{``allowing articles to `disappear in pairs' by analyzing the similarities and differences in relation to the context''} (E1). 
% The third level is multi-text driven exploration, which necessitates that the system should support the comprehensive linking of all texts selected by the teacher. Therefore, this level requires \name{} to conduct a thorough deconstruction of contexts, extract meaningful connections, and convey these connections to the teachers. 
% Besides, the analyses should also be accurate, especially in the filed of education. This not only necessitates high-quality AI-generated content but also emphasizes the assurance of editing all contents by teachers freely within the user interface. \textit{``I want to freely edit rather than just drag or choose fixed options. This allows me to directly correct issues when I discover them''} (E1). 
% Additionally, teachers believe that the system should enable them to pose detailed and flexible questions about the contexts and reading materials.  \textit{``Our team hopes that teachers can input vague, open-ended questions to the LLM''} (E1).

\penguin{
\textbf{DP3: \name{} should document the ideation outcomes in a lesson plan that aligns with the established educational practices in interdisciplinary literature teaching.}
A teacher without a lesson plan may struggle to effectively deliver the knowledge and objectives of the lesson~\cite{iqbal2021rethinking}. 
Finding 4 suggests that the lesson plan should contain detailed instructional activities, including the related reading materials and in-class activities, based on the selected contexts. 
To make it further aligned with educational practices, \name{} can adopt the six evaluation metrics of the outcome lesson plan (Finding 6) to guide the generation of ideation outcomes. 
}

% \pzh{
% \textbf{DP3: \name{} should provide structured and high-quality ideation outcomes 
% \fhx{that align with the educational requirements of interdisciplinary literature instruction in elementary school.}
% }}
% \fhx{
% A teacher without a lesson plan may struggle to effectively deliver the knowledge and objectives of the lesson~\cite{iqbal2021rethinking}. 
% Additionally, as Finding 4 revealed, this kind of lesson plan generated from the system should be as detailed as possible, at least including the contents of each lesson, an introduction for engagement, and classroom activities.
% Moreover, to better align the outcomes with the educational requirements, six metrics identified in Finding 6 could be integrated into the generation process as the guidelines.
% }


% \textit{DP3.1: \name{}' outputs should be consistent with educational practices
% \fhx{of literature instruction in elementary school }
% (Finding 4).} 
% % E1 emphasized that system outputs should contain the necessary content tailored to the teaching practices in her team. 
% % E1 summarized that the outputs should at least include comprehensive lesson planning, relevant explanations for each segment, an introduction to facilitate classroom engagement, and specific analyses of the texts. 
% % In the evaluation study of \name{} prototype, three teachers (I1, I2, I5) indicated that the system outputs should be more detailed and reduce human effort in modifying them for the later concrete plans for each lesson. % lesson design and instructional materials.
% % \textit{``The overall structure of the output is good, but I hope it can be more detailed; for example, providing more in-class and extracurricular activities related to the theme, so we can use them directly''} (I1). Therefore, we incorporated recommendations for literature and interdisciplinary course activities in the refined \name{}.

% \textit{DP3.2: \name{} should incorporate metrics that evaluate the quality of 
% outcome 
% \fhx{disciplinary }
% contexts
% \fhx{for literature instruction in elementary school}
% (Finding 5).}
% % The team of E1-E7 established six metrics for evaluating the quality of contexts: 
% }

% \fhx{
% The large amount of complex interdisciplinary information may lead users feeling ``information overload''~\cite{foster2004nonlinear, newby2011entering}. Thus, system outputs should be in a well-designed structure, including useful details while avoiding information unrelated to the instructional activities. 
% This helps prevent teachers from spending more effort on adapting or verifying the generated content.
% In other words, first, outcomes of \name{} require only simple interpretation and minor adjustments to help educators rapidly create lesson plans (Finding 4). 
% Second, systems should support users in verifying output information to facilitate information integration~\cite{foster2004nonlinear}.
% The evaluation metrics from Finding 5 can be integrated into the system to address this issue.
% }

\pzh{
% \begin{itemize}
%     \item \textbf{Appropriateness of Context}
%     \begin{itemize}
%         \item \textit{Content Alignment:} Does the context accurately cover the content of the selected materials?
%         \item \textit{Internal Logic:} Is there a logical connection between the context and the selected materials?
%     \end{itemize}
%     \item \textbf{Alignment with Educational Objectives}
%     \begin{itemize}
%         \item \textit{Curriculum Standards:} Does the content comply with national curriculum standards and teaching guidelines?
%         \item \textit{Subject Goals:} Does it help achieve specific goals of language education, such as reading comprehension and writing skills?
%     \end{itemize}
%     \item \textbf{Depth of Integration}
%     \begin{itemize}
%         \item \textit{Subject Integration:} Does it effectively integrate knowledge from different subjects?
%         \item \textit{Knowledge Transfer:} Does it promote the application of language arts knowledge in other subject contexts?
%     \end{itemize}
% \end{itemize}

\penguin{
\textbf{DP4: \name{} should include database of interdisciplinary contexts and reading materials and provide flexible user control to achieve high-quality ideation outcomes.} 
Prior research on the impact of LLMs in primary education indicates that generating false content is a disadvantage that may lead to ``information pollution'' for children~\cite{adeshola2023opportunities, murgia2023chatgpt}.
Finding 5 also indicates that LLMs sometimes were unable to create content that meets teachers' needs when lacking access to educational resources. 
To generate high-quality outcomes, as inspired by previous works~\cite{yazici2024gelex, khanal2024fathomgpt}, 
\name{} could ground the content generation on diverse real-world contexts and reading materials. 
% This approach not only reduces repetition but also increases the reliability of the suggested material. 
% Apart from high-quality AI generation, 
\name{} should also support teachers to freely edit and question any content (\eg texts in reading materials, outcome lesson plan) to make sure that they understand the content they are going to use in literature teaching. 
}

% \textbf{DP4: \name{} should include a comprehensive database of interdisciplinary contexts and articles.}
% \fhx{
% Prior research on the impact of LLMs in primary education indicates that generating false content is a disadvantage that may lead to ``information pollution'' for children~\cite{adeshola2023opportunities, murgia2023chatgpt}.
% From Finding 5, we also noticed that LLMs are sometimes unable to create content that meets teachers' needs when lacking access to educational resources in evaluating the \name{} prototype. 
% Therefore, inspired by previous works~\cite{yazici2024gelex, khanal2024fathomgpt}, 
% % we recognize that equipping LLMs with a comprehensive database allows them to generate more relevant content by retrieving and clustering educational texts from various subjects. 
% we could try to collect diverse contexts and articles to serve as a solid foundation for content generation in the final version of \name{}. 
% This approach not only reduces repetition but also increases the reliability of the suggested material, helping to prevent potential misinformation that could negatively impact students and teachers.
% }
% In the evaluation study with six teachers, our prototype generated traditional subject-related contexts using the LLM with specific templates and cognitive backgrounds of elementary students, without fine-tuning or retrieval-augmented generation (RAG). 
% Three teachers (I4, I5, I6) expressed concerns about the quality of the generated content. \textit{``The content generated for the art subject is quite repetitive''} (I5). \textit{``We need to establish a dedicated article database for science as well, since many of our articles are highly relevant to science''} (I4). Therefore, in the final version of \name{}, we have collected contexts \peng{and articles} from various subjects and use them as a foundation for generating content, which could reduce repetition and increase reliability of sugggested context. 
}
\section{System}\label{sec:system}
We consider systems in the form
%
\begin{subequations}\label{eq:system}
	\begin{align}
		\label{eq:system:x0}
		x(t) &= x_0(t), & t &\in (-\infty, t_0], \\
		%
		\label{eq:system:x}
		\dot x(t) &= f(x(t), z(t)), & t &\in [t_0, t_f],
	\end{align}
\end{subequations}
%
where $t \in \R$ is time, $t_0, t_f \in \R$ are the initial and final time, $x: \R \rightarrow \R^{n_x}$ is the state, and $x_0: \R \rightarrow \R^{n_x}$ is the initial state function. Furthermore, $f: \R^{n_x} \times \R^{n_z} \rightarrow \R^{n_x}$ is the right-hand side function, and the memory state, $z: \R \rightarrow \R^{n_z}$, is given by the convolution
%
\begin{subequations}\label{eq:system:delay}
	\begin{align}
		\label{eq:system:z}
		z(t) &= \int\limits_{-\infty}^t \alpha(t - s) \odot r(s) \incr s, \\
		%
		\label{eq:system:r}
		r(t) &= h(x(t)),
	\end{align}
\end{subequations}
%
where $r: \R \rightarrow \R^{n_z}$ is the delayed variable, and each element of $\alpha: \Rnn \rightarrow \Rnn^{n_z}$ is a \emph{regular} kernel (see Definition~\ref{def:regular:kernel}). Furthermore, $h: \R^{n_x} \rightarrow \R^{n_z}$ is the memory function. We assume that $f$ and $h$ are differentiable in their arguments, and we refer to the paper by Ponosov et al.~\cite[Thm.~1]{Ponosov:etal:2004} for more details on the existence and uniqueness of solutions to the initial value problem~\eqref{eq:system}--\eqref{eq:system:delay}. See also the book by Hale and Lunel~\cite{Hale:Lunel:1993}.
%
\begin{definition}\label{def:regular:kernel}
	A scalar-valued kernel, $\alpha: \Rnn \rightarrow \Rnn$, is \emph{regular} if it satisfies the following properties.
	%
	\begin{enumerate}
		\item It is non-negative and bounded, i.e., $0 \leq \alpha(t) \leq K$ for all $t \in \Rnn$ and for some finite $K \in \Rp$.
		%
		\item It is continuous, i.e., for all $\epsilon \in \Rp$ and $t \in \Rnn$, there exists a $\delta \in \Rp$ such that $|\alpha(s) - \alpha(t)| < \epsilon$ for all $s \in \Rnn$ satisfying $|s - t| < \delta$.
		%
		\item It is normalized such that
	\end{enumerate}
	%
	\begin{align}\label{eq:kernel:normalization}
		\int\limits_0^\infty \alpha(t) \incr t &= 1.
	\end{align}
\end{definition}
%
For a given system of DDEs with distributed time delays, each element of $\alpha$ may not satisfy~\eqref{eq:kernel:normalization}. However, as they are assumed to be nonzero and non-negative, it is straightforward to normalize them. Next, we present a few well-known corollaries about the steady states of~\eqref{eq:system:x}--\eqref{eq:system:delay} and their stability.
%
\begin{corollary}\label{thm:steady:state}
	A state $\bar x \in \R^{n_x}$ is a steady state of the system~\eqref{eq:system:x}--\eqref{eq:system:delay} if
	%
	\begin{align}\label{eq:steady:state}
		0 &= f(\bar x, \bar z), &
		\bar z &= \bar r = h(\bar x).
	\end{align}
\end{corollary}

\begin{proof}
	In steady state, $x(t) = \bar x$ for all $t$. Consequently, $r(t) = \bar r = h(\bar x)$ and
	%
	\begin{align}
		z(t)
		&= \int_{-\infty}^t \alpha(t - s) \odot \bar r \incr s
		 = \int_{-\infty}^t \alpha(t - s) \incr s \odot \bar r
		 = \bar r,
	\end{align}
	%
	for all $t$, where we have used the property~\eqref{eq:kernel:normalization} of each element of $\alpha$.
\end{proof}
%
\begin{corollary}\label{thm:stability}
	The system~\eqref{eq:system:x}--\eqref{eq:system:delay} is locally asymptotically stable around a steady state, $\bar x$, satisfying~\eqref{eq:steady:state} if $\real \lambda < 0$ for all $\lambda \in \C$ that satisfy the characteristic equation
	%
	\begin{align}\label{eq:characteristic:equation}
		\det\left(F - \lambda I + G \int_0^\infty e^{-\lambda s} \diag \alpha(s) \incr s H\right) = 0,
	\end{align}
	%
	where $I \in \R^{n_x \times n_x}$ is an identity matrix.
	%
	The matrices $F \in \R^{n_x \times n_x}$, $G \in \R^{n_x \times n_z}$, and $H \in \R^{n_z \times n_x}$ are the Jacobians of the right-hand side function and the delay function evaluated in the steady state:
	%
	\begin{align}\label{eq:jacobians}
		F &= \pdiff{f}{x}(\bar x, \bar z), &
		G &= \pdiff{f}{z}(\bar x, \bar z), &
		H &= \pdiff{h}{x}(\bar x).
	\end{align}
	%
\end{corollary}

\begin{proof}
	The linearized system corresponding to~\eqref{eq:system:x}--\eqref{eq:system:delay} describes the evolution of the deviation variable $X: \R \rightarrow \R^{n_x}$:
	%
	\begin{align}\label{eq:linearized:system}
		\dot X(t) &= F X(t) + G \int_{-\infty}^t \alpha(t - s) \odot H X(s) \incr s, &
		X(t) &= x(t) - \bar x.
	\end{align}
	%
	See, e.g., \cite{Cushing:1975, Cushing:1977, Miller:1972} for proofs of the condition~\eqref{eq:characteristic:equation} for asymptotic stability of the linearized system in~\eqref{eq:linearized:system}.
\end{proof}

\begin{table*}[t!]
  \caption{Relative ranking of 12 sequence design methods (descending order) across five random seed replications of the ML-oracle.}
  \vspace{1ex}
  \label{tab:gfp_seeds}
    \begin{minipage}{.48\textwidth}
      \centering
      \begin{adjustbox}{width=\linewidth,center}
      \begin{tabular}{|c|c|c|c|c|}
        \toprule
            \textbf{Seed 1}                                           & \textbf{Seed 2}                                        & \textbf{Seed 3}                                          & \textbf{Seed 4}                    & \textbf{Seed 5}                                        \\ \midrule 
            \cellcolor[HTML]{E6B8AF}BootGen & \cellcolor[HTML]{E6B8AF}BootGen & \cellcolor[HTML]{E6B8AF}BootGen & \cellcolor[HTML]{E6B8AF}BootGen & \cellcolor[HTML]{E6B8AF}BootGen \\ 
\cellcolor[HTML]{EAD1DC}CMA-ES & \cellcolor[HTML]{CFE2F3}GA Min & \cellcolor[HTML]{CFE2F3}GA Min & \cellcolor[HTML]{CFE2F3}GA Min & \cellcolor[HTML]{F4CCCC}GA Mean  \\ 
\cellcolor[HTML]{F4CCCC}GA Mean  & \cellcolor[HTML]{B5DDCA}BO-qEI & \cellcolor[HTML]{F4CCCC}GA Mean  & \cellcolor[HTML]{F4CCCC}GA Mean  & \cellcolor[HTML]{EAD1DC}CMA-ES \\ 
\cellcolor[HTML]{CFE2F3}GA Min & \cellcolor[HTML]{F4CCCC}GA Mean  & \cellcolor[HTML]{D9D2E9}GA & \cellcolor[HTML]{D9D2E9}GA & \cellcolor[HTML]{B5DDCA}BO-qEI \\ 
\cellcolor[HTML]{C9DAF8}COMs  & \cellcolor[HTML]{D9D2E9}GA & \cellcolor[HTML]{D9EAD3}Auto. CbAS & \cellcolor[HTML]{B5DDCA}BO-qEI & \cellcolor[HTML]{CFE2F3}GA Min \\ 
\cellcolor[HTML]{D9EAD3}Auto. CbAS & \cellcolor[HTML]{D9EAD3}Auto. CbAS & \cellcolor[HTML]{EAD1DC}CMA-ES & \cellcolor[HTML]{EAD1DC}CMA-ES & \cellcolor[HTML]{D9EAD3}Auto. CbAS \\ 
\cellcolor[HTML]{D9D2E9}GA & \cellcolor[HTML]{C9DAF8}COMs  & \cellcolor[HTML]{B5DDCA}BO-qEI & \cellcolor[HTML]{D9EAD3}Auto. CbAS & \cellcolor[HTML]{C9DAF8}COMs  \\ 
\cellcolor[HTML]{B5DDCA}BO-qEI & \cellcolor[HTML]{EAD1DC}CMA-ES & \cellcolor[HTML]{C9DAF8}COMs  & \cellcolor[HTML]{C9DAF8}COMs  & \cellcolor[HTML]{D9D2E9}GA \\ 
\cellcolor[HTML]{FCE5CD}CbAS & \cellcolor[HTML]{D0E0E3}MINs & \cellcolor[HTML]{D0E0E3}MINs & \cellcolor[HTML]{D0E0E3}MINs & \cellcolor[HTML]{FCE5CD}CbAS \\ 
\cellcolor[HTML]{D0E0E3}MINs & \cellcolor[HTML]{FCE5CD}CbAS & \cellcolor[HTML]{FCE5CD}CbAS & \cellcolor[HTML]{FCE5CD}CbAS & \cellcolor[HTML]{D0E0E3}MINs \\ 
\cellcolor[HTML]{FFF2CC}REINFORCE & \cellcolor[HTML]{FFF2CC}REINFORCE & \cellcolor[HTML]{FFF2CC}REINFORCE & \cellcolor[HTML]{FFF2CC}REINFORCE & \cellcolor[HTML]{FFF2CC}REINFORCE \\ 
\cellcolor[HTML]{F8E3A6}GFN-AL  & \cellcolor[HTML]{F8E3A6}GFN-AL  & \cellcolor[HTML]{F8E3A6}GFN-AL  & \cellcolor[HTML]{F8E3A6}GFN-AL  & \cellcolor[HTML]{F8E3A6}GFN-AL  \\ 


 
            \bottomrule
            \end{tabular}
            \end{adjustbox}
      \caption*{(a) UTR Design Bench oracle}
      \label{tab:utr_rankings}
    \end{minipage}%
    \hfill
    \begin{minipage}{.48\textwidth}
      \centering
      \begin{adjustbox}{width=\linewidth,center}
      \begin{tabular}{|c|c|c|c|c|}
        \toprule 
            \textbf{Seed 1}                                                 & \textbf{Seed 2}                                          & \textbf{Seed 3}                                          & \textbf{Seed 4}                            & \textbf{Seed 5}                         \\ \midrule                      
            \cellcolor[HTML]{E6B8AF}BootGen & \cellcolor[HTML]{FCE5CD}CbAS & \cellcolor[HTML]{D0E0E3}MINs & \cellcolor[HTML]{D0E0E3}MINs & \cellcolor[HTML]{D0E0E3}MINs \\ 
\cellcolor[HTML]{FFF2CC}REINFORCE & \cellcolor[HTML]{E6B8AF}BootGen & \cellcolor[HTML]{E6B8AF}BootGen & \cellcolor[HTML]{E6B8AF}BootGen & \cellcolor[HTML]{E6B8AF}BootGen \\ 
\cellcolor[HTML]{D0E0E3}MINs & \cellcolor[HTML]{FFF2CC}REINFORCE & \cellcolor[HTML]{FCE5CD}CbAS & \cellcolor[HTML]{D9EAD3}Auto. CbAS & \cellcolor[HTML]{D9EAD3}Auto. CbAS \\ 
\cellcolor[HTML]{D9EAD3}Auto. CbAS & \cellcolor[HTML]{CFE2F3}GA Min & \cellcolor[HTML]{D9EAD3}Auto. CbAS & \cellcolor[HTML]{FFF2CC}REINFORCE & \cellcolor[HTML]{FFF2CC}REINFORCE \\ 
\cellcolor[HTML]{FCE5CD}CbAS & \cellcolor[HTML]{D0E0E3}MINs & \cellcolor[HTML]{FFF2CC}REINFORCE & \cellcolor[HTML]{FCE5CD}CbAS & \cellcolor[HTML]{FCE5CD}CbAS \\ 
\cellcolor[HTML]{C9DAF8}COMs  & \cellcolor[HTML]{F4CCCC}GA Mean  & \cellcolor[HTML]{F4CCCC}GA Mean  & \cellcolor[HTML]{F4CCCC}GA Mean  & \cellcolor[HTML]{F4CCCC}GA Mean  \\ 
\cellcolor[HTML]{F4CCCC}GA Mean  & \cellcolor[HTML]{D9EAD3}Auto. CbAS & \cellcolor[HTML]{CFE2F3}GA Min & \cellcolor[HTML]{CFE2F3}GA Min & \cellcolor[HTML]{CFE2F3}GA Min \\ 
\cellcolor[HTML]{CFE2F3}GA Min & \cellcolor[HTML]{C9DAF8}COMs  & \cellcolor[HTML]{C9DAF8}COMs  & \cellcolor[HTML]{C9DAF8}COMs  & \cellcolor[HTML]{C9DAF8}COMs  \\ 
\cellcolor[HTML]{D9D2E9}GA & \cellcolor[HTML]{D9D2E9}GA & \cellcolor[HTML]{D9D2E9}GA & \cellcolor[HTML]{D9D2E9}GA & \cellcolor[HTML]{D9D2E9}GA \\ 
\cellcolor[HTML]{F8E3A6}GFN-AL  & \cellcolor[HTML]{F8E3A6}GFN-AL  & \cellcolor[HTML]{F8E3A6}GFN-AL  & \cellcolor[HTML]{F8E3A6}GFN-AL  & \cellcolor[HTML]{F8E3A6}GFN-AL  \\ 
\cellcolor[HTML]{EAD1DC}CMA-ES & \cellcolor[HTML]{EAD1DC}CMA-ES & \cellcolor[HTML]{B5DDCA}BO-qEI & \cellcolor[HTML]{EAD1DC}CMA-ES & \cellcolor[HTML]{EAD1DC}CMA-ES \\ 
\cellcolor[HTML]{B5DDCA}BO-qEI & \cellcolor[HTML]{B5DDCA}BO-qEI & \cellcolor[HTML]{EAD1DC}CMA-ES & \cellcolor[HTML]{B5DDCA}BO-qEI & \cellcolor[HTML]{B5DDCA}BO-qEI \\ 
 \bottomrule  
            \end{tabular}
            \end{adjustbox}
      \caption*{(b) GFP Design Bench oracle}
      \label{tab:gfp_rankings}
    \end{minipage}
\end{table*}

\section{Experiment II: Expert Interviews} \label{sec:experiment_2}
In Experiment I, we evaluated novice teachers' perceptions of our system, including the ideation process and their overall experience using \name{}.
Additionally, an experienced teacher, E1, evaluated the outcomes of the novice teachers. In Experiment II, we shifted our focus to the perspectives of experts \pzh{with varying levels of literature teaching experiences} to gain more feedback. %, specifically teachers with varying levels of experience, 
We conducted expert interviews using a think-aloud protocol and a semi-structured interview with nine teachers to collect insights related to our research questions. 
% Furthermore, in the Findings section, we will present the opinions regarding the unchangeable features of the prototype from I1 to I6 during \pzh{the design process}.

\subsection{Participants}
The study involved nine \pzh{Chinese language} teachers (5 female, 4 male, E1-5, E8-11 in \autoref{tab:teachers}) in a local elementary school, 
% including one novice teacher with 1-3 years of experience, four advanced beginner teachers with 4-6 years of experience, three competent teachers with 7-18 years of experience, and one proficient teacher with 29 years of experience. 
including two novice teachers with less than five years of teaching experience and six expert teachers with at least five years of teaching experience~\cite{booth2021mid}.
% \penguin{
% including two novice teachers with less than five years of teaching experience and six expert teachers with at least five years of teaching experience~\cite{booth2021mid}.
% }
Detailed information about the participants is presented in Table 3.


\subsection{Method}
\penguin{
We conducted interviews offline with E8-E11 and I1-I6 in lab sessions and with E1-E5 online after they freely used it for three days. This setup could help us to gain diverse insights, \eg learnability of \name{} in lab and field environments. 
I1-I6 participated in the evaluation of \name{}'s prototype, and their feedback on the unchanged features in \name{} was also presented in this section. 
}

For \textbf{E8 - E11}, who were not involved in the iterative design process, we conducted offline interviews. The process began with a 5-minute introduction to the research background (\ie the background of interdisciplinary literature instruction). 
This was followed by a 10-minute tutorial on how to use \name{}. 
Participants were then allocated 30 minutes to complete exploration tasks while engaging in a think-aloud protocol. Subsequently, a 15-minute semi-structured interview was conducted.

For \textbf{I1 - I6}, who participated in the evaluation of the prototype process, the procedure was consistent with the same methodology above, except that during the semi-structured interview, we asked for more suggestions regarding the prototype.

For \textbf{E1 - E5}, who engaged in the iterative design process, we conducted online interviews. The \name{} was made available for a duration of 3 days, and a tutorial video was provided. Participants were asked to freely use \name{} during this period to complete exploration tasks. Finally, we conducted a 15-minute semi-structured interview with each expert.


% \subsubsection{Exploration Tasks}
\fanhx{
% We developed the following exploration tasks:
Surrounding our RQs, the interview questions (\autoref{sec:appendix}) are about ideation outcomes, ideation process and perception of \name{}, while the participants were assigned the following exploration tasks:
\begin{itemize}
    \item Task 1: Freely choose 4-15 reading materials, explore two contexts of interest, and add them to the collection.
    \item Task 2: Generate an introduction, course plan, and activities.
\end{itemize}
}
% \subsubsection{Interview Questions}


% \penguin{
% \textbf{RQ1: Ideation outcomes}}
% \begin{itemize}
%     \item What do you think about the quality of the outcomes generated by \name{}?
%     \item Do you think the results generated by our system can assist you in preparing lessons or designing a new interdisciplinary context in the classroom setting?
% \end{itemize}

% \penguin{
% \textbf{RQ2: Ideation process}}
% \begin{itemize}
%     \item Did you feel the task load is high while using \name{}? Specifically, did you feel any increased cognitive load or mental demands?
% \end{itemize}

% \penguin{
% \textbf{RQ3: Perception of \name{}}
% }
% \begin{itemize}
%     \item Do you think our system can assist you in ideating interdisciplinary topics more efficiently?
%     \item Do you think our system can help you better explore connections between texts and the contexts?
%     \item If you were to prepare an interdisciplinary reading case, compared to your usual preparation methods (\eg searching, meeting with other subject teachers, or using ChatGPT), do you think \name{} could support you better?
%     \item Do you trust the output of our system compared to your previous experiences with web searching or using ChatGPT?
% \end{itemize}



\subsection{Findings}
\subsubsection{RQ1: Ideation Outcomes}
Overall, seven of nine teachers confirmed that the outcomes generated by \name{} effectively support literature instruction within the classroom environment. E10 emphasized the comprehensiveness of the text analysis,
\penguin{where she could find the content she wanted within a document more strategically, instead of sifting through countless reference books:} 
\penguin{\textit{``I can abandon varied reference books, (because) the system conducts a comprehensive analysis of the specific content of the texts. It includes themes, content, and key points.''}}
Additionally, E11 noted that the activities could be easily implemented in the classroom: \textit{``yes, these activities (such as music appreciation) can be incorporated into upcoming lessons to increase student engagement.''}
I3 mentioned that the structured outcomes could potentially \textit{``make my teaching process more systematic.''}

\penguin{
We observed the different views between novice and expert teachers on how they embrace the integration of generated activities into real classrooms, though almost all teachers consider \name{}'s outcomes are beneficial for instructions. 
Experienced teachers (E9, E10, E11) preferred their self-centered teaching approach and were more conservative in using the generated activities. %, with three of six indicating they critically and carefully select the provided activities. 
\textit{``To be honest, I have never used these (recommended) activities before, and in the future, I might only add a bit of them to my existing lesson plan to make students more interested.'' }(E11)
On the contrary, newer teachers were more open and tended to reconstruct their established curriculum based on the recommended activities.}
E8 stated, 
\textit{``I plan to adjust my instruction method according to them (activities); some recommended activities are excellent in the current context and can be adapted for use in other contexts.''
}

Despite these positive findings, five of nine teachers pointed out that some outcomes did not align well with the objectives of literature instruction: \textit{``the outcomes are still somewhat disconnected from practical application. The content generated based on the provided template does not fully meet the current teaching needs, especially considering the recently revised curriculum standards''} (E2).
Teachers suggested a potential solution: \textit{``Import the curriculum standards and teaching objectives for each text, and to consider these goals when constructing contexts. For instance, what are the learning objectives and abilities required for each grade level?''} (E5)

Despite modifications made to the prototype, 4 out of 9 experts indicated a need for additional outcomes that could directly enhance student learning in literature: \textit{``the documents generated are very useful for lesson preparation; however, they cannot be directly provided to students for learning purposes. I hope it can produce some homework questions''} (E1).


In summary, the comprehensive and structured outcomes facilitate educational activities; however, additional focus is required to ensure the alignment of these outcomes with the literature objectives.

\subsubsection{RQ2: Ideation process}
Experts have reached a consensus that the task load is low when using \name{} to construct contexts for the classroom, attributable to its well-structured layout and functional settings. E11 said, \textit{``the system does not burden my memory due to the collection feature''.} 
\penguin{
However, E5, who used \name{} in the wild for three days, suggested improvements to its UI design and features to further decrease mental demand: 
\textit{``More specific instructions could be incorporated into the page to clearly indicate the available actions. Additionally, I would like to have a feature that synchronizes generated records through user login for long-term usage.''} 
Experts in the lab sessions commented more on \name{}'s error cases which may increase their effort in the ideation process. 
For example, E8 input ``What activities can be designed around `Stepping Stones' [the title of one text]'', and \name{} responded ``Students can observe stone bridges and steps in their daily lives, explore their design principles and practical uses. They can participate in group projects to build stone bridge models using materials like rocks, experiencing the joy of collaboration.''
E8 commented, \textit{``This is indeed related to the article, but only the content, not the core idea or intended message''}.
E8 did not directly incorporate these outputs into the final outcome; instead, he edited manually.
Additionally, teachers may feel frustrated due to LLM's hallucinations.
I3 input 
``Give me more sentences and analysis related to 'Osmanthus Rain' [the title of one text] and the context about poetic life'', and
LLM responded 
```What I like is osmanthus. The osmanthus tree looks clumsy, unlike the plum tree, which has a graceful posture.' This contrast highlights the author's unique affection for osmanthus and reflects the author's ability to discover poetic elements in life through a lens of beauty''.
I3 remarked that \textit{``it did give me one more sentence but not really fit in the context.''}
In summary, \name{} is generally user-friendly but requires more UI and feature refinements to optimize the ideation process.
% It indicate LLMs can generate information that appears believable but is actually incorrect, instead of simply admitting it don't know the answer.
% This issue arises when users repeatedly ask LLMs similar questions.
}
% Additionally, 
% \fhx{
% as a long-term user, E5 suggested that the user interface design could be improved and history tracking should be provided to decrease mental demand for teachers during long-term use: \textit{``more specific instructions could be incorporated into the page to clearly indicate the available actions and the color scheme could be improved. Additionally, I would like to see the capability to synchronize generated records through user login.''}
% }

% \fhx{
% Based on recordings and experts' comments, error responses to input queries may increase user effort. 
% E8 input ``What activities can be designed around 'Stepping Stones' [the title of one text]'', \name{} responded ``Students can observe stone bridges and steps in their daily lives, explore their design principles and practical uses. They can participate in group projects to build stone bridge models using materials like rocks, experiencing the joy of collaboration.''
% E8 commented, ``This is indeed related to the article, but only the content, but not the core idea or intended message.''
% Finally, E8 did not incorporate these outputs into the final outcome; instead, he edited manually.
% Additionally, more frustration may be brought due to hallucination.
% I3  asked \name{} to provide some specific sentences related to the context of "poetic life" and then input 
% ``Give me more sentences and analysis related to 'Osmanthus Rain' [the title of one text] and the context''.
% LLM responded 
% ``'What I like is osmanthus. The osmanthus tree looks clumsy, unlike the plum tree, which has a graceful posture.' This contrast highlights the author's unique affection for osmanthus and reflects the author's ability to discover poetic elements in life through a lens of beauty.''
% I3 remarked ``it did give me one more sentence, but not really fit'' 
% It indicate LLMs can generate information that appears believable but is actually incorrect, instead of simply admitting it don't know the answer.
% This issue arises when users repeatedly ask LLMs similar questions.
% }

% In summary, our system provides support to teachers in constructing contexts, indicating that the system is user-friendly.

\subsubsection{RQ3: Perception of \name{}}
\penguin{
Both users in the lab session, who learned \name{} through verbal instructions from the developer, and users in the wild, who studied it via a video tutorial, agreed that our system is easy to learn.
}
With the exception of E2, all other teachers affirmed that \name{} supported their exploration of contexts and the interrelationships between texts and contexts. \textit{Expanding thinking} and \textit{inspiring creativity} were mentioned as advantages for supporting context exploration.
\textit{``The system provides a broader range of relationships between context and text, gradually expanding the context. I believe AI should work in this manner, incrementally broadening the scope to help me explore more possibilities, rather than rigidly offering a single answer''} (E11).
Furthermore, E5 added: \textit{``It can provide inspiration, demonstrating how to develop the class based on this context.''}
Besides expanding the breadth of thinking, E10 expressed appreciation for \name{}'s summarization capabilities: \textit{``after selecting a substantial number of texts, I was astonished that it could truly organize them and produce a comprehensive design, which is impossible in my typical lesson preparation.''} These comments indicate that \name{} facilitates users in opening and focusing their cognitive processes during interdisciplinary exploration.
% However, 
% E2 found the recommended contexts to be unreasonable:
% \textit{``The content generated is not suitable; some contexts provided are inappropriate... The choice of words in the analysis is also problematic, with certain terms not conforming to the standards for elementary school students''} (E2). 
% \fhx{
% Moreover, some queries
% }
% The generation of unreasonable content has constrained the system's usability, consequently diminishing user support.
\penguin{
However, E1 and E3 raised an issue about the repetition of suggested contexts when they tried different reading materials in their three-day usage of \name{}. 
E1 noted, \textit{``I found that the same contexts reappearing despite my selection of entirely different reading materials''.} 
% two of five long-term users noticed some challenges with \name{}'s reusability due to the lack of original contexts available in the database.
% E1 noted, \textit{``I found a high frequency of context repetition, with the same contexts reappearing despite my selection of entirely different reading materials on multiple times''.} 
}
% the insufficiency of original contexts impacted the exploration experience.
Despite receiving feedback and modifications implemented after the prototype evaluation, this issue still troubled users and will be further discussed in~\ref{sec:discussion}.

In comparison to other generative tools and web search engines, users have reported a high level of trust in the output of our system, despite occasional acceptable errors. I3 stated, \textit{``I feel it is more closely aligned with reading materials compared to previous tools, although a few analyses of the context are not very accurate. Overall, I still trust this system.''} E4, I4 emphasized their reliance on personal experience and subjective judgment during the exploration process: \textit{``I always trust my own design more. When my ideas are limited, I use this system to evaluate its outputs and determine which content is usable''} (E4).
% I4 expressed a similar opinion: \textit{``I trust this system, but I cannot assert that I completely accept all its outputs. I will consider the actual situation before adopting them.''}

In summary, users perceive \name{} as highly usable and effective, and they generally express trust in its outputs.
This work identifies signal collapse as a critical bottleneck in one-shot neural network pruning. Performance loss in pruned networks is due to \textbf{signal collapse} in addition to the removal of critical parameters. We propose \textbf{REFLOW} (\textbf{Re}storing \textbf{F}low of \textbf{Low}-variance signals), a simple yet effective method that mitigates signal collapse without computationally expensive weight updates. By focusing on signal preservation, REFLOW highlights the importance of mitigating signal collapse in sparse networks and enables magnitude pruning to match or surpass state-of-the-art one-shot pruning methods such as CHITA, CBS, and WF.

REFLOW consistently achieves state-of-the-art accuracy across diverse architectures, restoring ResNeXt-101 from under 4.1\% to 78.9\% top-1 accuracy at 80\% sparsity on ImageNet. Its lightweight design makes it a practical solution for both research and deployment, delivering high-quality sparse models without the overhead of traditional approaches. These findings challenge the traditional emphasis on weight selection strategies and underscore the critical role of signal propagation for achieving high-quality sparse networks in the context of one-shot pruning.



\section*{Conclusion}
This paper aims to enhance our understanding of the computational complexity of computing various Shapley value variants. We found that for various ML models --- including decision trees, regression tree ensembles, weighted automata, and linear regression --- both local and global interventional and baseline SHAP can be computed in polynomial time under HMM modeled distributions. This extends popular algorithms, such as TreeSHAP, beyond their empirical distributional scope. We also establish strict complexity gaps between the various SHAP variants (baseline, interventional, and conditional) and prove the intractability of computing SHAP for tree ensembles and neural networks in simplified scenarios. Overall, we present SHAP as a versatile framework whose complexity depends on four key factors: \begin{inparaenum}[(i)] \item model type, \item SHAP variant, \item distribution modeling approach, \item and local vs. global explanations\end{inparaenum}. We believe this perspective provides deeper insight into the computational complexity of SHAP, paving the way for future work.




%We believe that our framework provides a more intricate understanding of SHAP computation complexity across different models, distributions, and variants, paving the way for further research.

Our work opens promising directions for future research. First, expanding our computational analysis to other SHAP-related metrics, such as asymmetric SHAP~\citep{frye20} and SAGE~\citep{covert2020understanding}, would be valuable. Additionally, we aim to explore more expressive distribution classes and relaxed assumptions beyond those in Section \ref{sec:tractable} while maintaining tractable SHAP computation. Finally, when exact computation is intractable (Section \ref{sec:intractable}), investigating the approximability of SHAP metrics through approximation and parameterized complexity theory~\citep{downey2012parameterized} is an important direction.

%Our work opens several promising avenues for future research on the computational properties of explainable AI methods, with a particular focus on SHAP. First, it would be interesting to broaden the computational analysis conducted in this work to include other popular SHAP-related metrics in the literature, such as asymmetric SHAP \cite{frye20} and SAGE \cite{covert2020understanding}. Also, in the future, we aim to explore more expressive distribution classes and relaxed distributional assumptions—extending beyond those examined in Section \ref{sec:tractable} —that still yield tractable SHAP computation. Finally, when exact computation proves intractable (Section \ref{sec:intractable}), it is worthwhile to theoretically investigate the question of the approximability of computing the SHAP metrics across various configurations, through the lens of approximation and parametrized complexity theory \cite{arora2009computational}.

%This paper aims to deepen our understanding of the computational complexity involved in obtaining different Shapley value variants. We found that for a variety of ML models, including decision trees, tree ensembles for regression, weighted automata, and linear regression models — computing both local and global interventional and baseline SHAP can be done in polynomial time when distributions are modeled by HMMs. This extends the distributional scope of popular algorithms like TreeSHAP, which is limited to empirical distributions. Additionally, we demonstrate a strict complexity gap between SHAP variants, showing that interventional and baseline SHAP can be strictly easier to compute than conditional SHAP. Despite these positive results, we uncovered intractability for various SHAP variants in neural networks and tree ensembles. Finally, we provided generalized complexity relations across SHAP variants. We believe that our framework offers a deeper understanding of the complexity involved in computing SHAP across various variants, models, distributions, as well as in both local and global computations, laying the groundwork for future research.

%%
%% The acknowledgments section is defined using the "acks" environment
%% (and NOT an unnumbered section). This ensures the proper
%% identification of the section in the article metadata, and the
%% consistent spelling of the heading.
\begin{acks}
This work is supported by the Young Scientists Fund of the National Natural Science Foundation of China with Grant No.: 62202509 and the General Projects Fund of the Natural Science Foundation of Guangdong Province in China with Grant No. 2024A1515012226.
\end{acks}

%%
%% The next two lines define the bibliography style to be used, and
%% the bibliography file.
\bibliographystyle{ACM-Reference-Format}
\bibliography{references}
% \bibliography{sample-base}


%%
%% If your work has an appendix, this is the place to put it.
\appendix
\fanhx{
\section{Interview Questions in Expert Interviews} \label{sec:appendix}
}
The fixed questions for the \pzh{semi-structured} interviews in Experiment II (expert interviews) are shown below.

\penguin{
\textbf{RQ1: Ideation outcomes}}
\begin{itemize}
    \item What do you think about the quality of the outcomes generated by \name{}?
    \item Do you think the results generated by our system can assist you in preparing lessons or designing a new interdisciplinary context in the classroom setting?
\end{itemize}

\penguin{
\textbf{RQ2: Ideation process}}
\begin{itemize}
    \item Did you feel the task load is high while using \name{}? Specifically, did you feel any increased cognitive load or mental demands?
\end{itemize}

\penguin{
\textbf{RQ3: Perception of \name{}}
}
\begin{itemize}
    \item Do you think our system can assist you in ideating interdisciplinary topics more efficiently?
    \item Do you think our system can help you better explore connections between texts and the contexts?
    \item If you were to prepare an interdisciplinary reading case, compared to your usual preparation methods (\eg searching, meeting with other subject teachers, or using ChatGPT), do you think \name{} could support you better?
    \item Do you trust the output of our system compared to your previous experiences with web searching or using ChatGPT?
\end{itemize}

% \section{Research Methods}

% \subsection{Part One}


\end{document}
\endinput
%%
%% End of file `sample-sigconf-authordraft.tex'.
