\section{Introduction}

The integration of knowledge from different disciplines within literature instruction at the elementary school level 
% \penguin{\eg in our focused Chinese language courses in China,}
has been proven to enhance student learning outcomes~\cite{navrotskaya2024reading, lindvig2019different}. 
% Language teaching encompasses four skills (listening, speaking, reading, and writing) as well as grammar and vocabulary. 
Contextualization, as a pedagogical approach, makes the curriculum more meaningful and practical for students, which is essential for effective learning~\cite{fernandes2013curricular}. Learning in an interdisciplinary context not only enhances students' comprehensive learning abilities but also increases their motivation to learn. 
However, the ideation of a suitable interdisciplinary context \pzh{is challenging}. %poses considerable challenges.
Firstly, human knowledge is categorized into distinct disciplines, \peng{and people with an occupation usually focus their expertise on a single field~\cite{palmer2010information}.} 
% with numerous researchers focusing their expertise on a single field~\cite{palmer2010information}. 
The development of an interdisciplinary context requires literature teachers to engage in the continuous exploration and evaluation of substantial amounts of information from unfamiliar fields (\eg mathematics, science, and art).
Secondly, elementary school teachers frequently face constraints related to limited time outside of the classroom~\cite{dias2017challenges}, which do not allow them to conduct comprehensive searches for interdisciplinary knowledge. 
% \pzh{Most of these teachers also lack the opportunities of communicating and collaborating with those from different subject areas [ref].}
% Furthermore, the lack of effective communication and collaboration among teachers from different subject areas [ref] presents significant challenges in constructing appropriate interdisciplinary contexts for literature instruction.

Recent advances in large language models (LLMs) show great potential to address the challenges faced by teachers in ideating interdisciplinary contexts.
% The utilization of large language models (LLMs) can address the challenges faced by elementary literature teachers in ideating interdisciplinary contexts.
Such as GPT-4~\footnote{\url{https://openai.com/index/gpt-4}} and GLM-4~\footnote{\url{https://github.com/THUDM/GLM-4}}, possess extensive knowledge across various domains, owing to their pre-training on large text datasets~\cite{NEURIPS2020_1457c0d6}.
\fhx{
Also, LLMs' capabilities in long-context understanding~\cite{naveed2023comprehensive} enable them to synthesize information from extensive textual materials effectively. 
}
% \fhx{
% \sout{
Within the field of human-computer interaction (HCI), the LLM-empowered tools have emerged to facilitate interdisciplinary information exploration~\cite{zheng2024disciplink}, lesson plan preparation based on educational theories~\cite{fan2024lessonplanner}, and support learners in critical and creative thinking~\cite{yuan2023critrainer, shaer2024ai}. %reading
% }
% }
However, neither conversational applications of LLMs (\eg ChatGPT) nor these LLM-powered tools can offer a comprehensive solution for the challenges at hand. 
% \fhx{
% nor proposed LLM-powered tools~\cite{zheng2024disciplink, fan2024lessonplanner, yuan2023critrainer, shaer2024ai} in HCI community can offer a comprehensive solution for the challenges at hand.
% }
Firstly, the process of ideating interdisciplinary contexts from a variety of reading materials necessitates that educators engage in a thorough exploration of both the context and the materials. 
This engagement enables teachers to discern the connections between the context and the reading materials, rather than reading the responses generated by LLMs. Without such an in-depth exploration, fostering a comprehensive understanding of the context and materials becomes challenging, which is essential for effective teaching. 
Secondly, the suggested interdisciplinary contexts need to be effectively integrated into the classroom environment.
% the need for interdisciplinary contexts can be effectively integrated into the classroom environment.
However, due to a limited understanding of 
\haoxiang{teaching practices} and the cognitive backgrounds of elementary students, LLM-generated content often fails to align with the cognitive and pedagogical demands of elementary education~\cite{fan2024lessonplanner}.
Consequently, the generated content may require adjustments by literature teachers or may even be unusable. 
% At present, there is a lack of research investigating the types of contexts that can most effectively enhance teaching activities within an elementary educational setting. % feel that it should be more aligned with the interdispline settings. and this is not HCI contribution.
\haoxiang{
Lastly, educators, particularly those who are not so familiar with LLMs, have to allocate significant time and effort in constructing complex prompts and understanding the intricate outputs generated by these systems.
In summary, there is a need for an interactive tool that assists educators in effectively exploring and generating ideas for interdisciplinary contexts within the classroom.
}

% \fhx{\sout{Literature teaching, as a supporting project for these skill-based teachings, is an essential means of enriching students' language abilities and cultural communication within the language curriculum~\cite{celce1991teaching}.}}
% \fhx{\sout{Similarly, in Chinese language education, literature teaching is also a vital component.}}
\penguin{In this paper, 
\fanhx{we focus on the teaching scenarios of Chinese language courses in China and}
introduce \name{}~\footnote{\fanhx{It is open-sourced and available on \url{https://github.com/fanhaoxiang1/LitLinker}}}, 
an LLM-powered interactive system that supports teachers in ideating diverse interdisciplinary contexts for teaching literature in elementary schools. 
% in China.
}
% In this paper, we focus on the 
% \fhx{teaching scenarios of Chinese literature in elementary schools} and introduce \name{}, an interactive system that supports the teachers in ideating diverse interdisciplinary contexts with LLMs. 
\penguin{To develop \name{}, we follow a user-centered design approach that involves in total 13 Chinese literature teachers in three interview sessions, prototyping, and evaluation.}
% \fhx{We first conduct three interview sessions with an expert who leads a team of seven teachers focused on the development of interdisciplinary literature curricula. }
% \fhx{
% According to the design goals derived from the interviews and a review of relevant literature, we iteratively design a prototype in collaboration with the expert. 
% % Following this, we conduct a formative user evaluation to gather insights from six elementary literature teachers. 
% % After refining our prototype based on this feedback, we develop the final version of \name{} as a web application powered by the LLM.
% A formative user evaluation with six elementary literature teachers provides feedback, which informs the refinement of the prototype.
% }
\penguin{\name{} is} 
a web application powered by the LLM GLM-4. When users select relevant disciplines 
% \penguin{(\eg science)} 
and reading materials for exploration, \name{} initially recommends interdisciplinary topics 
% \penguin{(\eg measuring time)}
and corresponding analyses based on the themes and concepts of the selected materials.
\haoxiang{
Subsequently, \name{} facilitates the exploration of relationships between various literary elements (\eg paragraphs, sentences, and viewpoints within reading materials) and the selected topics.
}
Users can bookmark topics they focused on and ask the LLM any questions related to the topics and reading materials. 
% \fhx{\sout{In \name{}, we simulate the real-world dynamics of multiple teachers engaging in discussions and designing interdisciplinary contexts, carefully crafting the interactions between LLM agents and between agents and humans.}}
Finally, \name{} produces a \penguin{lesson plan that includes a} course outline, an introductory overview, and recommended classroom activities.


% We develop \name{}, a web application powered by the LLM. When users select relevant disciplines and reading materials for exploration, the system initially recommends interdisciplinary topics and corresponding analyses based on the themes and concepts of the selected materials.
% In \name{}, when users select relevant disciplines and reading materials for exploration, the system initially recommends interdisciplinary topics and corresponding analyses based on the themes and concepts of the selected materials.
% Subsequently, the system supports exploration from a literature perspective by providing detailed information (\eg paragraphs, sentences, viewpoints) and relationships between topics. 
% \haoxiang{
% Subsequently, the system facilitates the exploration of relationships between various literary elements (\eg paragraphs, sentences, and viewpoints within reading materials) and the selected topics.
% }
% Users have the capability to bookmark topics they focused on and  ask the LLM any questions related to the topics and reading materials. 
% \fhx{\sout{In \name{}, we simulate the real-world dynamics of multiple teachers engaging in discussions and designing interdisciplinary contexts, carefully crafting the interactions between LLM agents and between agents and humans.}}
% Finally, the system produces a course outline, an introductory overview, and recommended classroom activities.



We evaluate \name{} through two studies
\penguin{that answer three research questions (RQs), \ie how would \name{} affect the interdisciplinary context exploration RQ1) outcomes and RQ2) process, and RQ3) how would users perceive the usability and creative support of \name{}}. 
% \fhx{
% The research questions are the quality of the outcomes, perception of \name{}, and \name{}'s user support.
% We first conduct a within-subjects study to test the usability and the quality of the outcome with 16 student participants by comparing our system with a baseline condition. 
% After that, we invite 10 elementary Chinese literature teachers with varying levels of experience to evaluate \name{}. 
% The quantitative results indicate that~\name{} significantly enhances efficiency, eases users' workload and leads to deeper integration of different subjects. 
% % The qualitative analysis revealed a high level of satisfaction with the generated results and a strong appreciation for \name{} in facilitating the development of interdisciplinary courses.
% }
% In the first study, we tested the usability and the quality of the outcome with 16 participants by comparing our system with a baseline condition in the construction of interdisciplinary context for specific reading materials. The quantitative results indicate that~\name{} significantly enhances efficiency, eases users' workload and leads to deeper integration of different subjects. 
% In the second study, we invited 10 elementary literature teachers with varying levels of experience to use \name{}. 
\penguin{Experiment I is a within-subjects design that quantitatively answers these three RQs with 16 novices of Chinese literature teaching, while Experiment II is a qualitative study that gains answers to the RQs from 10 elementary Chinese literature teachers with varying teaching experience.
The quantitative results indicate that \name{} leads to deeper integration of different subjects in the outcome lesson plan, is deemed significantly more satisfying and efficient, and significantly reduces task workload. 
}
The qualitative analysis reveals high satisfaction with the generated results and a strong appreciation for \name{} in facilitating teachers to develop interdisciplinary courses.

In summary, this paper has three main contributions. 
First, \penguin{we introduce \name{}, 
an interactive system that supports the ideation of interdisciplinary contexts for literature classes.} %via a user-centered design approach with in total 13 elementary teachers, 
% an interactive system designed to support the ideation of interdisciplinary contexts for literature classes, co-designed with \fhx{seven} elementary literature teachers. 
Second, through a within-subjects study and expert interviews, we provide empirical evidence of \name{}'s effectiveness and usefulness. 
Third, based on our design process and findings, we offer design implications for future LLM-based systems that assist teachers. 

\penguin{
In the rest of this paper, we first review literature that inspires our work (\autoref{sec:related_work}), followed by the design process (\autoref{sec:design_process}) and implementation (\autoref{sec:system_design}) of \name{}.
Then, we present two experiments (\autoref{sec:experiment_1}, \autoref{sec:experiment_2}) that evaluates \name{}. 
We finally discuss the implications of our work to HCI and education communities (\autoref{sec:discussion}). 
}