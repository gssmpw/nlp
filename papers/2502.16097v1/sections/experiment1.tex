\section{Experiment I: Within-subject study}\label{sec:experiment_1}
% We conducted two experiments with university students and Chinese language teachers to study the advantages and disadvantages of \name{}. 
\pzh{We evaluate \name{} via two experiments. Experiment I is a within-subjects study that aims at assessing the effectiveness of \name{} in supporting novices of literature teaching to come up with interdisciplinary contexts. 
Experiment II is a qualitative study that examines \name{}'s values in assisting expert Chinese language teachers in ideating interdisciplinary contexts for literature teaching. 
}
In this section, we present the design and results of Experiment I.
% \pzh{that answers the following research questions (RQs) } 
% \penguin{
% \ie how would \name{}
% affect the interdisciplinary context exploration \textbf{RQ1)} outcomes and \textbf{RQ2)} process, and \textbf{RQ3)} how would users perceive
% the usability and creative support of \name{}.
% }
% , a within-subjects study involving 16 participants. 
\fanhx{
The research questions are:
\begin{itemize}
    \item RQ1. How would \name{} affect the interdisciplinary exploration outcomes?
    \item RQ2. How would \name{} affect the process of interdisciplinary exploration?
    \item RQ3. How would users perceive the usability and creative support of \name{}?
\end{itemize}
}

\subsection{Participants}
We recruited 16 students (P1-P16, nine females, seven males; age: \textit{Mean}=23.44, \textit{SD}=2.24) via a post in a group chat and via word-of-mouth from two universities. 
\penguin{This sample size follows the practices of previous work that evaluates systems for supporting ideation and education, such as CreativeConnect (N = 16) \cite{choi2024creativeconnect}, AngleKindling (N = 12) \cite{petridis2023anglekindling}, DiaryHelper (N = 12) \cite{li2024diaryhelper}, and LessonPlanner (N = 12) \cite{fan2024lessonplanner}. 
We also calculated the required sample size with G* Power to conduct Wilcoxon Signed-Rank tests that compare measures in \name{} and baseline conditions. 
With Tails = Two (can not tell in advance which of the two measures is larger), Parent distribution = Normal (default), Effect size = 0.80 (calculated as Cohen's d~\cite{cohen2013statistical}, large effect), $\alpha \text{ err prob} = 0.05$ (default), and power $(1 - \beta \text{ err prob}) = 0.8$ (an acceptable threshold), the output recommended smallest sample size is 15. 
}
\pzh{Five participants, including two third-year undergraduate and three graduate students, major in}
\penguin{the fields related to computer science (CS)}. 
\penguin{They could provide feedback from those who know about the techniques used in \name{}, and they were capable of our ideation tasks as they all learned Chinese literature in elementary schools.} 
% Among the participants, five individuals, comprising two third-year undergraduate students and three graduate students, are majoring in artificial intelligence, while 
The other eleven participants are all graduate students majoring in education-related fields. \penguin{All of them are teachers in training, while P2, P4, and P9 have teaching experience. The others are pre-service teachers.}. 
% The experimental setup is designed to gather insights from interface and technical perspectives, as well as educational viewpoints. 
All participants expressed their interest in utilizing web resources and AI-assisted tools to design interdisciplinary contexts (\textit{Mean} = 4.62, \textit{SD} = 0.48; 1 - no interest at all, 5 - very interest in). All participants reported having experience with large language models (LLMs, \eg ChatGPT) (\textit{Mean} = 4.0, \textit{SD} = 0.94; 1- no experience, 5 - use daily).

\begin{table*}[h]
\caption{Participants involved in the within-subjects study}
\Description{Participants involved in the within-subjects study}
\begin{tabular}{@{}cccccc@{}}
\toprule
\textbf{ID} & \textbf{Gender} & \textbf{Year} & \textbf{Age} & \textbf{Major}                                         & \textbf{Freq. of AI Usage} \\ \midrule
P1  & F & Graduate      & 23 & Curriculum and Instruction – English Language Education & Daily        \\
P2  & M & Graduate      & 23 & Curriculum and Instruction – English Language Education & Weekly       \\
P3  & F & Graduate      & 23 & Curriculum and Instruction – English Language Education & Weekly       \\
P4  & F & Graduate      & 25 & Curriculum and Instruction – English Language Education & Weekly       \\
P5  & F & Graduate      & 24 & Curriculum and Instruction – English Language Education & Daily        \\
P6  & F & Graduate      & 23 & Chinese Literature                                      & Have Tried   \\
P7  & F & Graduate      & 22 & Early Childhood Education and Child Development         & Weekly       \\
P8  & F & Graduate      & 22 & Early Childhood Education and Child Development         & Infrequently \\
P9  & F & Graduate      & 28 & Educational Psychology                                  & Daily        \\
P10         & F               & Graduate      & 23           & TESOL(Teaching English to Speakers of Other Languages) & Have Tried                 \\
P11 & M & Graduate      & 29 & Computer Science and Technology                         & Weekly       \\
P12 & M & Graduate      & 22 & Computer Science and Technology                         & Daily        \\
P13 & M & Graduate      & 23 & Electronics and Information Engineering                 & Weekly       \\
P14 & M & Undergraduate & 20 & Artificial Intelligence                                 & Weekly       \\
P15 & M & Undergraduate & 21 & Artificial Intelligence                                 & Weekly       \\
P16 & M & Graduate      & 24 & Education                                               & Daily        \\ \bottomrule
\end{tabular}
\end{table*}

\subsection{Experimental Design}
Experiment I is a within-subject design. Each participant completed one interdisciplinary exploration task with \name{} and the other with a baseline setup.

\subsubsection{Baseline Condition and \name{} Condition}
The baseline system is the LLM web application, which employs the Zhipu LLM model, the same model implemented in \name{}. 
Participants could upload and read document files, thereby enabling the web app to access selected reading materials. 
Also, the participants had the option to utilize the Google search engine and Microsoft Office Word. 
In the \name{} condition, participants were allowed to use Google, Microsoft Office Word, and \name{}, but were not allowed to use the LLM web app.
In both conditions, participants could freely choose whether, when, and how to utilize the provided tools.

\subsubsection{Tasks-systems Assignment}
\pzh{We invited the experienced Chinese language teacher E1 (\autoref{tab:teachers}), who participated in our design process, to help us prepare the task materials and grade the task outcome}. 
Specifically, E1 selected 16 reading materials from six textbooks, covering 15 different units, and divided them into two groups - Text Set 1 and Text Set 2.
Each participant completed tasks with both groups of texts. The task assignments were as follows:
\begin{itemize}
    \item Baseline (Text Set 1) + \name{} (Text Set 2)
    \item \name{} (Text Set 2) + Baseline (Text Set 1)
    \item \name{} (Text Set 1) + Baseline (Text Set 2)
    \item Baseline (Text Set 2) + \name{} (Text Set 1)
\end{itemize}

\subsubsection{Procedure}
Each participant was assigned two lesson-planning tasks.  
% \pzh{to ideate contexts for teaching the given reading materials}. %\pzh{interdisciplinary exploration}
The task prompt was: 
\begin{quote}
    You are a sixth-grade Chinese language teacher, aiming to implement an interdisciplinary approach over a period of 3-5 days, establishing logical and thematic connections between in-class and extracurricular reading materials. You are facing challenges in organizing these reading materials. Your goal is to analyze the internal relationships among the texts, identify an `interdisciplinary context', integrate the selected texts within this context, develop a course plan in a document explaining the relationship between the reading materials and the selected context, and propose potential literature or interdisciplinary instructional activities. 
    You may refer to the provided documents \pzh{of example lesson plans based on interdisciplinary contexts}. 
    \pzh{You need to choose at least three from eight reading materials and integrate them with the context.}
    % There are 8 reading materials to choose at least three to integrate with the context.
\end{quote}

% As shown in Figure 6, 
Participants received the original reading materials one day before the task and were instructed to spend 10 minutes reading them.
On the day of the task, the four participants assigned to the same task-system group came to our lab. 
We introduced the background knowledge about literature and interdisciplinary instruction, and the components necessary for creating an effective context (\ref{sec:principles} DP2).
To encourage their best performance, we informed the participants that their outputs would be evaluated by an expert based on the appropriateness of Context, alignment with the educational objectives, and depth of integration. The top three participants would be awarded a supplementary reward of 100 RMB.

For each task, we first introduced the task and demonstrated the assigned system (\name{} or baseline).
Subsequently, we assisted participants in setting up their \pzh{task} environment on the provided computers or their own devices, including opening the baseline system or \name{}, Google, and Microsoft Office Word. % learning
Each participant independently completed the lesson planning task. 
We allocated 30 minutes for each task and informed them that they could complete the task ahead of time if they wished. 
Following the completion of each task, participants were required to fill out a questionnaire. A 10-minute break was provided between the two tasks.  %intermission
After completing both tasks, we conducted a final semi-structured interview. Each participant spent approximately 90 minutes in the experiment and received 100 RMB for compensation.

\subsection{Measurements}
% We employed a standard 7-point Likert Scale (1 - strongly disagree, 7 - strongly agree) to measure the quality of interdisciplinary context outcomes for literature instruction, the user experience during the exploration process, and the perceptions of the used system.

\penguin{
\textbf{RQ1. Ideation outcomes.}} 
To assess the quality of the outcomes, we invited E1 to rate all outcomes in a randomized order.
\penguin{E1 did not know which condition each output lesson plan came from and reported that she spent four to six minutes evaluating each outcome, and} 
the assessment was based on three criteria 
\penguin{(derived in Finding 6 in \autoref{sec:principles})}
: Content Alignment, Alignment with Educational Objectives, and Depth of Integration, using a 7-point Likert scale \penguin{(1 - not satisfied at all, 7 - fully satisfied)} to indicate the extent to which each criterion was satisfied. Additionally, \penguin{to make sense of the ratings, we required E1 to} provide comments for each lesson plan, highlighting good and bad aspects.
% \fhx{All the comments and scores are as objective as possible, grounded in the given criteria.}

\penguin{
\textbf{RQ2. Ideation process.}} Based on NASA-TLX, we formulated six questions to measure workload during the lesson planning process, including mental demand, physical demand, temporal demand, performance, effort, and frustration - \penguin{the higher a TLX score suggests the higher the perceived workload.}


\penguin{
\textbf{RQ3. Perception of \name{}.}} We adapted ten questions from the System Usability Scale (SUS)~\cite{brooke2013sus} to study effectiveness, efficiency, and satisfaction. 
To understand how users perceive \name{}'s generated contexts and analysis, we adapted six questions from the Creativity Support Index (CSI)~\cite{cherry2014quantifying} to evaluate Exploration, Expressiveness, and Immersion. \yh{The Creativity Support Index (CSI) includes three key dimensions: Exploratory, which measures how well the system helps users explore diverse ideas; Expressiveness, which reflects the system’s ability to support clear expression of creative ideas; and Immersion, which indicates how well the system allows users to stay focused on creative tasks.}


\pzh{The measured items for task workload, SUS, and CSI are rated using a standard 7-point Likert Scale (1 - strongly disagree, 7 - strongly agree).}

 

\subsection{Results and Analyses}
For quantitative data, we used the Wilcoxon Signed-Rank test.
\penguin{
% We use G*Power software to perform a post hoc power analysis to compute the achieved power. 
% With an effect size dz set at 0.8 (considered a large effect [ref]), an alpha error probability of 0.05, two-tailed tests, and a total sample size of 16, the resulting power (1 - beta error probability) is 0.83, which exceeds the acceptable threshold of 0.8. 
% Also, we calculate the sensitivity with a power of 0.8, and the resulting required effect size is 0.77. 
We used G* Power software to compute sensitivity, given Parent distribution = Normal, $\alpha = 0.05$, Power = 0.8, and Total Sample size = 16, which outputs a required effect size of 0.77. 
% With an effect size dz set at 0.8 (considered a large effect [ref]), an alpha error probability of 0.05, two-tailed tests, and a total sample size of 16, the resulting power (1 - beta error probability) is 0.83, which exceeds the acceptable threshold of 0.8. 
}
For qualitative data,
\fanhx{
we conducted thematic analysis on the semi-structured interview under the same settings in the design process (\autoref{sec:formative_findings}) and presented comments from the outcomes as supporting evidence.
}



\subsubsection{RQ1: Ideation Outcomes}
\autoref{fig:rq1_results} shows the quality of outcome lesson plans from the within-subjects study.
The results indicate high variance in outcome quality for both the baseline and \name{} conditions, suggesting significant differences in user performance. 
\name{} showed improvements in Completion (\name{}: $M = 5.56, Mdn=6, SD = 0.61$; baseline: $M = 4.19, Mdn=4.5, SD = 2.01$; $p = 0.02$\fhx{, Wilcoxon effect size ($r$) = 0.93}), providing more guidance in organizing lesson components. In Subject Integration (\name{}: $M = 4.63, Mdn=6, SD = 2.26$; baseline: $M = 3.56, Mdn=3, SD = 2.00$; $ p = 0.34$\fhx{, $r = 0.50$}), \name{} assisted in combining literature knowledge with disciplinary contexts like science and arts. For Knowledge Transfer (\name{}: $M = 4.44, Mdn=5.5, SD = 2.21$; baseline: $M = 3.56, Mdn=3, SD = 2.00$; $p = 0.38$\fhx{, $r = 0.42$}), outcomes created with \name{} were more likely to promote students' application of disciplinary knowledge in their classroom environment.
\pzh{These results suggest that \name{} performed well in supporting users to connect the interdisciplinary contexts to the reading materials.}

However, regarding Content Alignment (\name{}: $M = 4.25, Mdn=5, SD = 2.80$; baseline: $M = 5.69, Mdn=7, SD = 2.23$; \pzh{$p = 0.22$}\fhx{, $r = 0.57$}), Internal Logic (\name{}: $M = 4.00, Mdn=3.5, SD = 2.65$; baseline: $M = 5.00, Mdn=5.5, SD = 2.06$; \pzh{$p = 0.29$}\fhx{, $r = 0.42$}), Subject Objectives (\name{}: $M = 3.44, Mdn=3, SD = 2.24$; baseline: $M = 4.25, Mdn=5, SD = 1.89$; \pzh{$p = 0.32$}\fhx{, $r = 0.39$}), and Curriculum Standards (\name{}: $M = 3.44, Mdn=3, SD = 2.65$; baseline: $M = 4.56, Mdn=6, SD = 1.93$; \pzh{$p = 0.16$}\fhx{, $r = 0.53$}), \pzh{the outcome lesson plans with the baseline system were rated significantly higher than those with \name{}.}
% the baseline system scored higher than \name{}. 
This indicates that the outputs from \name{} were less appropriate in terms of \pzh{aligning the contexts} with educational objectives compared to the baseline. 
\pzh{We identify two possible reasons for these results based on the experimental setup and E1's comments on the outcome lesson plans.}
% We will attempt to identify the reasons for this from the experimental setup and comments on the outcomes. %context and alignment 
First, few participants had teaching experience, \pzh{and none of them had taught Chinese literature before}.
% , which rendered them 
\pzh{In other words, participants were} unfamiliar with the specific educational objectives associated with literature. 
During the introduction of the system and the task, we overemphasized the concept of ``interdisciplinary'', which led participants to focus more on other subjects rather than the educational objectives of the literature itself. In contrast, the experienced teacher E1 who evaluated the outcomes is highly sensitive to curriculum standards and literature objectives. E1 commented on an outcome that received a Curriculum Standards score of 1 (P08 - \name{}), \textit{``The forced interdisciplinary integration is counterproductive, as it shifts the focus from literature to art or science, losing sight of the primary subject.''}
Second, the system occasionally provided excessive or inaccurate interpretations of the texts, leading to lower appropriateness of context scores. \textit{``I feel that the system's trying to fit the analysis into a specific context, but it does not always feel very relevant.''} (P7). 
%lacked familiarity
Most participants were not familiar with the reading materials, making it difficult for them to identify these issues. In contrast, E1 could easily identify such deviation. E1 commented on an outcome that received an appropriateness of context score of 1 (P11 - \name{}), \textit{``The analysis does not match my understanding of the text.''}

In summary, while \name{} improved Completion, Subject Integration, and Knowledge Transfer \pzh{of the outcome lesson plans}, it resulted in lower ratings concerning the appropriateness of context and alignment with educational objectives. 
This might be due to the experimental setup and the system's misinterpretation of the texts. We will focus on these issues in expert interviews to evaluate the quality of the outcomes.

\begin{figure}[]
  \centering
  \includegraphics[width=1\linewidth]{figures/RQ1.pdf}
  \caption{RQ1 results regarding the outcomes evaluated by E1 in seven different aspects. ***: p<0.001, **: p<0.01, *: p<0.05, +: p<0.1}
  \Description{RQ1 results regarding the outcomes evaluated by E1 in seven different aspects. ***: p<0.001, **: p<0.01, *: p<0.05, +: p<0.1}
  \label{fig:rq1_results}
\end{figure}

\subsubsection{RQ2: Ideation Process}
\autoref{fig:rq3_results} shows the participants' ratings on six NASA-TLX dimensions between the baseline system and \name{} in the user study. 
% \fhx{To better present the result, some scales have been reverse.
% The higher a TLX score is, the higher the perceived workload was.}
Users rated mental demand and physical demand significantly lower with \name{} (\name{}: $M = 3.19, Mdn=3, SD = 1.74$; baseline: $M = 5.38, Mdn=5, SD = 1.17$, $p=0.0009$\fhx{, $r = 1.48$}; \name{}: $M = 2.19, Mdn=2, SD = 1.29$; baseline: $M = 3.38, Mdn=2.5, SD = 1.80$, $p=0.01$\fhx{, $r = 0.76$}). Users felt significantly more satisfied with their performance using \name{} (\name{}: $M = 2.69, Mdn=2, SD = 1.49$; baseline: $M = 4.06, Mdn=4.5, SD = 1.48$; $p = 0.03$\fhx{, $r = 0.93$}) and reported significantly lower Effort (\name{}: $M = 3.31, Mdn=3.5, SD = 1.61$; baseline: $M = 5.38, Mdn=5.5, SD = 1.27$; $p = 0.003$\fhx{, $r = 1.42$}). Frustration Level scores were also significantly lower with \name{} (\name{}: $M = 1.88, Mdn=1.5, SD = 1.27$; baseline: $M = 2.38, Mdn=1, SD = 1.76$; $p = 0.23$\fhx{, $r = 0.33$}), suggesting that users felt more confident during the exploration process. Additionally, users reported lower average Temporal demand with \name{} (\name{}: $M = 2.63, Mdn=2, SD = 1.69$; baseline: $M = 3.44, Mdn=3, SD = 2.00$; $p = 0.27$\fhx{, $r = 0.44$}). 
\pzh{In summary, the participants indicate that \name{} alleviates workload compared to the baseline system across all dimensions.}

\begin{figure}[]
  \centering
  \includegraphics[width=1\linewidth]{figures/RQ3_reverse.pdf}
  \caption{
  \penguin{RQ2 results regarding evaluating how \name{} affects the workload during the task. For each metric, a higher NASA-TLX score suggests a higher perceived workload. ***: p<0.001, **: p<0.01, *: p<0.05, +: p<0.1}}
  \Description{RQ2 results regarding evaluating how \name{} affects the workload during the task. For each metric, a higher NASA-TLX score suggests a higher perceived workload. ***: p<0.001, **: p<0.01, *: p<0.05, +: p<0.1}
  \label{fig:rq3_results}
\end{figure}


\subsubsection{RQ3: Perception of \name{}}
\fhx{\autoref{fig:rq2_results}} presents the statistical results comparing the baseline system and \name{} based on effectiveness, efficiency, and satisfaction scores. 
Participants rated \name{} higher in all three aspects. Satisfaction scores were significantly higher for \name{} (\name{}: $M = 5.13, Mdn=5.5, SD = 1.42$; baseline: $M = 3.88, Mdn=3.83, SD = 1.10$; $ p = 0.03$\fhx{, $r = 0.99$}), indicating greater user satisfaction with \name{}. Efficiency scores also showed a significant increase with \name{} (\name{}: $M = 5.58, Mdn=5.63, SD = 0.93$; baseline: $M = 4.48, Mdn=4.75, SD = 1.36$; $p = 0.02$\fhx{, $r = 0.94$}), suggesting that participants found \name{} more helpful and relevant to their tasks. 
For effectiveness, \name{} received a higher mean score (\name{}: $M = 4.69, Mdn=4.83, SD = 1.48$; baseline: $M = 4.02, Mdn=4, SD = 1.14$; $ p = 0.06$\fhx{, $r = 0.50$}), implying that participants felt more effective using \name{}.

For the three dimensions of the Creativity Support Index (Exploratory, Expressiveness, and Immersion), our system achieved higher average scores in all aspects. Users rated \name{} significantly higher on Exploratory dimension (\name{}: $M = 5.56, Mdn= 5.75, SD = 1.14$; baseline: $M = 4.44, Mdn=4.5, SD = 1.31$; $p = 0.01$\fhx{, $r = 0.91$}), suggesting better support for exploratory creativity. On the Immersion dimension, users also gave \name{} significantly higher ratings (\name{}: $M = 4.91, Mdn=5, SD = 1.28$; baseline: $M = 3.90, Mdn=4.5, SD = 1.30$;$ p = 0.02$\fhx{, $r = 0.75$}), indicating an enhanced ability to focus during creative tasks. For the Expressiveness dimension, users rated \name{} higher (\name{}: $M = 4.19, Mdn=4.5, SD = 1.55$; baseline: $M = 3.88, Mdn=3.75, SD = 1.14$; $p = 0.36$\fhx{, $r = 0.23$}), indicating a moderate improvement in expressiveness.
\peng{Despite the overall strengths of \name{} in creative support, six participants in the semi-structured interviews pointed out ways to further improve its creative support.}
% However, in the semi-structured interviews, 6 of 16 participants pointed out potential reasons why \name{} might reduce their creativity.
First, in terms of exploratory dimensions, the system's interpretation of contexts and the quality of the AI-generated responses did not align with the expectations of some users, thereby diminishing their motivation to engage in further exploration. \textit{``(For a science context,) I expected it to provide more information related to students' everyday lives, but it mostly analyzed from abstract directions like aesthetics, which didn't match my expectations.''} (P6). \textit{``Its answers didn't address the questions I wanted to explore; sometimes it felt like it was using fancy language to evade the core issues.''} (P13). 
Regarding Expressiveness, \pzh{four} participants noted that \name{} tended to ``persuade'' them to accept the contexts established by the system, rather than supporting their own critical thinking and expression. 
\textit{``It felt like scrolling through TikTok, reading the AI-generated decent outcomes without much thought.''} (\pzh{P12}).
\penguin{
We observed that the eleven participants with education-related backgrounds ($M = 3.77, SD = 1.59$) tended to give lower scores on Expressiveness than the five students ($M = 5.10, SD =1.34$) with CS-related backgrounds. 
% An interesting finding is that three of them are from education-related majors, while one is from AI-related majors. 
% Statistical results also reveal that participants from education-related majors demonstrated lower expressiveness compared to those from AI-related majors ($3.77$ vs $5.10$). 
P7, a student from an educational major, stated that 
\textit{
``I find the system too acquiescent; it often responds with `You are right' or `I agree with you'. However, sometimes my thoughts were not necessarily correct, and I would prefer it to provide a more critical perspective.''
}
This suggests the system should further customize the support to users with education majors to construct and refine their initial ideas. 
% \textit{``I felt like I wasn't using my brain, just following its lead.''} (P14). 
}


% which are commonly employed in many AI-powered systems, we focus on the trust in LLM outputs in \name{}.
% Participants generally expressed that the \name{}'s output was more trustworthy \pzh{than the output from the baseline system}. First, the professional and educator-friendly nature of the responses fostered confidence in \name{}. \textit{``I trust this system (\name{}) more because its responses are aligned with many educational theories.''} (P3).
% Second, trust was established based on the accuracy of the system's outcomes for the classroom setting. \textit{``The second system (baseline) struggled to provide practical activity suggestions, whereas the activities recommended by the first system (\name{}) seemed feasible for implementation in my classroom.''} (P10).



% In summary, the statistical results demonstrate that \name{} better supported users' exploration processes in comparison to the baseline system across multiple dimensions. Nevertheless, some users expressed different opinions on Creativity support.


\begin{figure}[]
  \centering
  \includegraphics[width=1\linewidth]{figures/RQ2.pdf}
  \caption{
  \fhx{RQ3 results regarding users' perception of \name{}, based on 3 aspects from SUS, and 3 aspects from CSI. ***: p<0.001, **: p<0.01, *: p<0.05, +: p<0.1}}
  \Description{RQ3 results regarding users' perception of \name{}, based on 3 aspects from SUS, and 3 aspects from CSI. ***: p<0.001, **: p<0.01, *: p<0.05, +: p<0.1}
  \label{fig:rq2_results}
\end{figure}
