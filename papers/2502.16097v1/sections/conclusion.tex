\section{Conclusion}
In this paper, we developed an interactive system, \name{}, through an iterative design process involving 13 Chinese language teachers.
\name{} is to support the ideation of interdisciplinary contexts for teaching literature, with the help of LLM-generated content. \name{} includes three parts \ie Context Exploration View, Text Exploration View, and Collection.
It provides space that users can explore and curate contexts deemed suitable for literature teaching, and then find out the reading materials that can be integrated into the teaching process. 
\name{} provides detailed and precise analyses of the reading materials during exploration, subsequently providing structured outcomes for teachers. The within-subjects experiment involving 16 participants demonstrates that \name{} facilitate the depth of interdisciplinary integration and eases their workload during the exploration process.
Additionally, expert interviews were conducted with 9 teachers, who highlighted they trust in \name{} and believe it expands their thinking. We discuss design considerations and insights derived from the user study for human-AI collaboration in education.