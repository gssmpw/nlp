\pzh{
\section{Design Process and Principles of \name{}}
\label{sec:design_process}
Our work aims to support elementary literature teachers in effectively identifying suitable interdisciplinary contexts for their instructions, which can be used in their later lesson plans and classroom activity designs. 
\penguin{
Our design process and evaluation of \name{} involve in total of 17 Chinese language teachers (\autoref{tab:teachers}) in an elementary school in mainland China. 
% To achieve this, we invite total 17 Chinese language teachers (\autoref{tab:teachers}) in an elementary school in mainland China.
Specifically, in the design process, we involved E1-E7 in the foundational study and I1-I6 in the evaluation of the prototype. 
In the evaluation of \name{} with teachers (\ie Experiment II), we involved E1-E5 again and E8-11. 
In Experiment II, I1-I6 also contributed findings about the unchanged features between \name{} and its prototype. 
% and I1-I6 are involved in the iterative design process, including a foundational study and a evaluation of the prototype.
}

% The initial phase, foundational study, comprised three sessions of semi-structured interviews with an experienced literature teacher (E1).
% Before each session, E1 met with her seven-member interdisciplinary literature course design team (members: E1 - E7), documented the meetings, and subsequently reported these findings during interviews with the authors. 
% The second phase, we developed a workable prototype of \name{} and evaluated it with other six teachers (I1 - I6). 
% }
% We gathered their insights to inform our design goals for \name{} presented in this paper. %conducted a \textbf{formative user evaluation}

% \fhx{E1-E5 and E8-E11 participate in Experiment II (\ie Expert Interviews) following the development of \name{}. We will discuss the results in~\autoref{sec:experiment_2}}


\begin{table*}[htbp]
\caption{17 Chinese language teachers participated in the iterative design process and expert interviews (\ie Experiment II). Among them, there were 6 males and 11 females, with teaching experience ranging from 3 to 29 years. Two participants did not provide information on their teaching experience. This table also includes their experience in reading projects and interdisciplinary projects.}
\Description{17 Chinese language teachers participated in the iterative design process and expert interviews (\ie Experiment II). Among them, there were 6 males and 11 females, with teaching experience ranging from 3 to 29 years. Two participants did not provide information on their teaching experience. This table also includes their experience in reading projects and interdisciplinary projects.}
\label{tab:teachers}
\begin{tabular}{@{}cccccc@{}}
\toprule
\textbf{Involvement}                                                                                                                               & \textbf{ID} & \textbf{Gender} & \begin{tabular}[c]{@{}c@{}}\textbf{Teaching Experience} \cr \textbf{(years)}\end{tabular} & 
\begin{tabular}[c]{@{}c@{}}\textbf{Participation in} \cr \textbf{Reading Projects}\end{tabular} & 
\begin{tabular}[c]{@{}c@{}}\textbf{Participation in} \cr \textbf{Interdisciplinary Projects}\end{tabular} \cr \midrule
\multirow{5}{*}{\begin{tabular}[c]{@{}c@{}}\textit{Foundational}\cr \textit{Study}\cr \textit{\&}\cr \textit{Experiment II}\end{tabular}}                                        & E1          & F               & 5                                                                               & Y                                                                                     & Y                                                                                               \cr
                                                                                                                                                   & E2          & M               & 7                                                                               & Y                                                                                     & Y                                                                                               \cr
                                                                                                                                                   & E3          & F               & 14                                                                              & Y                                                                                     & Y                                                                                               \cr
                                                                                                                                                   & E4          & M               & 6                                                                               & Y                                                                                     & Y                                                                                               \cr
                                                                                                                                                   & E5          & M               & 4                                                                               & Y                                                                                     & Y                                                                                               \cr \hline
\multirow{2}{*}{\begin{tabular}[c]{@{}c@{}}\textit{Foundationall}\cr \textit{Study}\end{tabular}}                                                             & E6          & F               & -                                                                               & Y                                                                                     & Y                                                                                               \cr
                                                                                                                                                   & E7          & F               & -                                                                               & Y                                                                                     & Y                                                                                               \cr \hline
\multirow{6}{*}{\begin{tabular}[c]{@{}c@{}} \textit{\penguin{Evaluation of}}\cr \textit{\penguin{Prototype (re-usable}}\cr \textit{\penguin{findings are presented}}\cr \textit{\penguin{in Experiment II)}}\end{tabular}} & I1          & F               & 27                                                                              & Y                                                                                     & Y                                                                                               \cr
                                                                                                                                                   & I2          & F               & 8                                                                               & Y                                                                                     & Y                                                                                               \cr
                                                                                                                                                   & I3          & F               & 5                                                                               & N                                                                                     & N                                                                                               \cr
                                                                                                                                                   & I4          & M               & 11                                                                              & Y                                                                                     & Y                                                                                               \cr
                                                                                                                                                   & I5          & F               & 5                                                                               & Y                                                                                     & N                                                                                               \cr
                                                                                                                                                   & I6          & M               & 5                                                                               & Y                                                                                     & N                                                                                               \cr \hline
\multirow{4}{*}{\begin{tabular}[c]{@{}c@{}}\textit{\penguin{Expert}}\cr \textit{\penguin{Interviews}}\end{tabular}}                                                              & E8          & M               & 3                                                                               & N                                                                                     & N                                                                                               \cr
                                                                                                                                                   & E9          & F               & 6                                                                               & Y                                                                                     & N                                                                                               \cr
                                                                                                                                                   & E10         & F               & 29                                                                              & N                                                                                     & Y                                                                                               \cr
                                                                                                                                                   & E11         & F               & 17                                                                              & Y                                                                                     & N                                                                                               \cr \bottomrule
\end{tabular}
\end{table*}

\subsection{Design Process}
\penguin{
We generally followed a user-centered approach to plan our design process. 
First, to understand users' practices and involve them in the design of \name{}, we conducted three sessions of semi-structured interviews with an experienced literature teacher E1, who led a seven-member (E1 - E7) interdisciplinary literature course design team that indirectly contributed to the interviews. 
We were not able to have direct discussions with E2 - E7 due to their inconvenience during the semester. 
Then, we developed a workable prototype of \name{} and evaluated it with another six teachers (I1 - I6).  
We gathered their insights to inform our design goals for \name{} presented in this paper. 
}
\subsubsection{Foundational Study}
We closely worked with E1 to identify the practices, challenges and needs for support of ideating interdisciplinary contexts for teaching literature in elementary schools. 
Over the past two years, E1 has spearheaded a team of seven individuals (E1-E7) in the exploration and implementation of interdisciplinary literature instruction within elementary school Chinese courses. 
\fanhx{
To gain a comprehensive understanding of user needs, we progressively conducted three sessions of semi-structured interviews with E1 in April, June, and July 2024, lasting 38 minutes, 53 minutes, and 45 minutes, respectively.
}
Before each session, we communicated the purpose of the interview to E1 and requested that she engage with her team to compile records of their meetings for discussing the topics in the intended interview. 
We documented each interview session with E1 through audio and video recordings. % for each interview.

In Session 1, 
\fanhx{we asked E1 to present their current practices of teaching literature in interdisciplinary contexts, with previously developed lesson plans and assignments in her team.
} 
The discussion also focused on the potential of \fanhx{AI (\eg what do you think AI can support you (in your lesson planning in previous))}, and an interactive system to facilitate the design of interdisciplinary literature contexts, including the integration of art and history into the assignments \peng{of literature reading}. 
\peng{After this section}, two authors brainstormed potential features of a system for supporting the ideation of interdisciplinary contexts and sent E1 a document that explains these features. %(\eg xxx \peng{[any high-level features?]})
% , who subsequently 
We requested E1 to engage in a discussion with her group members to identify any additional or incorrect points about the \peng{potential system}. 
In Session 2, E1 \peng{came back with positive feedback from her team on each potential feature}.
\fanhx{We asked her to further explain their general process for designing interdisciplinary contexts as a team, emphasizing the distinct responsibilities and cognitive processes of each teacher involved.} 
Additionally, she presented the proposed interaction model. 
After this session, E1 had a group meeting with her team and came up with a template that defines the anticipated outcomes of our system. 
% Following a group meeting, a template defining the anticipated outcomes of the system was subsequently provided to the authors.
In Session 3, we introduced how a system works utilizing LLM agents \peng{to simulate roles in a team for ideating the interdisciplinary contexts, as suggested by E1 in Session 2}. 
% , detailing the specific roles of each agent. 
We presented two example outcomes produced using our predefined prompts and intermediate outputs 
\fanhx{to ask for her opinions on these prompts and outputs (\eg whether these intermediate outputs were helpful? If the prompts aligned well with your thoughts?).
}
}


\pzh{
\subsubsection{Development and Evaluation of \name{} Prototype}
\peng{
After Session 3, two of the authors utilized the thematic analysis method to analyze the transcribed recordings and all textual content derived from the foundational study. 
The analysis yielded four summarized Design Principles as described in the following \autoref{sec:principles}. 
We then worked on the implementation of a workable prototype that chains different LLM agents in a structured process to help teachers think of interdisciplinary contexts for teaching the literature materials in the textbooks. 
}
% \subsubsection{Evaluation of \name{} Prototype}
We evaluated our workable prototype with another six elementary school Chinese language teachers (I1 - I6, 3 Male, 3 Female), as shown in \autoref{tab:teachers}. 
Each evaluation lasted approximately 30 to 45 minutes and comprised four parts: (1) an introduction to the background, which included the concepts of interdisciplinary literature instruction, and the theory of contexts of instruction; (2) a brief tutorial on the prototype; (3) a think-aloud study in which participants freely explored the prototype and spoke out their thoughts; and (4) a semi-structured interview \peng{for their comments and suggestions on the prototype}. With the participants' consent, we conducted the evaluation offline and recorded audio and video.

At this stage, we assessed the prototype's usability (\eg whether different functions were well-integrated), user perception (\eg user interaction with the prototype and any additional cognitive load), and the quality of the system's outcomes. Furthermore, we collected suggestions for improving the prototype, particularly regarding user interface design and additional functionalities. 
% \peng{These feedback and suggestions helped us refine the design principles, which are incorporated in \autoref{sec:principles} Design Principles}.
\penguin{These feedback and suggestions are presented in \autoref{sec:formative_findings}, which inform the design principles (\autoref{sec:principles}) of \name{}.}
% \peng{Besides, the prototype in this evaluation study is close to the final version of \name{}.
% The key refinements lie in xxx, xxx, and xxx. 
For the six teachers' (I1-6) feedback on the same features in the prototype and final version of \name{}, we incorporate it in the results of Expert Interviews in \autoref{sec:experiment_2}. 
}
% In this section, we concentrate on gathering feedback and findings to establish design goals and incorporate new features into the prototype. Feedback on other unchanged aspects will be reported in the Section Expert Interviews.

% \fhx{
% \subsection{Findings and Design Principles} \label{sec:principles}
% }

\penguin{
\subsection{Findings}
\label{sec:formative_findings}
}
\fhx{
Two of the authors utilized thematic analysis~\cite{braun2012thematic} to inductively code and summarize the \penguin{practices, challenges, requirements, and concerns} from transcribed recordings \penguin{in the design process. The key themes are shown below.} % and all textual content from the foundational study.
\fanhx{One author first iteratively coded the data, while the other carefully reviewed the codes to ensure accuracy. After discussions, they reached a consensus and identified six primary themes. These findings are shown below.}
}


\penguin{
\textbf{Finding 1: In practice, teachers usually engage in reverse thinking when ideating interdisciplinary context}.}
% \fhx{\textbf{Finding 1: Two opposing intellectual paradigms of interdisciplinary context ideation.}}
% Teachers engage in two distinct intellectual paradigms when ideating contexts. One paradigm, referred to as ``forward thinking'', resembles ``deductive reasoning''. 
\penguin{Our teachers mentioned two intellectual paradigms.}
One paradigm referred to as ``forward thinking'', resembles ``deductive reasoning'', in which
teachers create abstract connections from a limited number of reading materials (typically 3-5 texts) and develop a concrete and reasonable context. 
The other paradigm, which is termed ``reverse thinking'' and analogous to ``inductive reasoning'', is a more habitual cognitive process employed by teachers. 
% \penguin{To conduct ``reverse thinking'', which is analogous to ``inductive reasoning'',}
Teachers would like to first select interdisciplinary contexts that they deem suitable and then identify appropriate texts from a broader text pool, after which they refine the connections between the identified text and the context. 
\textit{``For us, a good context often arises from a sudden inspiration, which we then backtrack to complete the ideation of what texts can connect to this context and how''} (E1).

\penguin{
\textbf{Finding 2: It is challenging to identify the connections between the established context and the reading materials  in the process of ``reverse thinking''.} 
}
% \fhx{\textbf{Finding 2: Extraction of connections from different subjects is challenging.}}
% In the process of ``reverse thinking'', identifying the connections between the established context and the reading materials is challenging. 
Teachers must evaluate the effectiveness of the connections in enhancing students' understanding of both literature and its associated subjects, as well as in stimulating their interest. 
\textit{``It is quite difficult and usually takes a long time for our team to ensure that based on the reading materials, our teaching activities connected by the context can indeed help students gain knowledge''} (E1).

\penguin{
\textbf{Finding 3: Teachers require support at three levels of granularity when analyzing the reading materials and contexts.}
}
% \fhx{\textbf{Finding 3: Typically employing three levels of granularity of analysis are needed.}}
% During the ideation process, to help teachers prepare for the teaching activities, the system should help teachers analyze and understand the reading materials and contexts at three three levels of granularity. 
The first level is in-depth single-text analysis, which \textit{``explains how the elements of a given article relate to the context''} (E1). 
The second level is pairwise comparison, where comparative reading has been demonstrated to be an effective method for understanding texts, \textit{``allowing articles to `disappear in pairs' by analyzing the similarities and differences in relation to the context''} (E1). 
The third level is multi-text-driven exploration, which necessitates that the system should support the comprehensive linking of all texts selected by the teacher. Therefore, this level requires \name{} to conduct a thorough deconstruction of contexts, extract meaningful connections, and convey these connections to the teachers. 
% Besides, the analyses should also be accurate, especially in the filed of education. This not only necessitates high-quality AI-generated content but also emphasizes the assurance of editing all contents by teachers freely within the user interface. \textit{``I want to freely edit rather than just drag or choose fixed options. This allows me to directly correct issues when I discover them''} (E1). 
% Additionally, teachers believe that the system should enable them to pose detailed and flexible questions about the contexts and reading materials.  \textit{``Our team hopes that teachers can input vague, open-ended questions to the LLM''} (E1).

\penguin{
\textbf{Finding 4: 
Teachers require detailed instructional activities based on the selected contexts.}}
% \fhx{\textbf{Finding 4: Ensuring enough details for instructional activities.}}
% \fhx{
% E1 emphasized that finding a precise interdisciplinary context is the first step in preparing a series of instructional activities, and that manually developing comprehensive teaching resources tailored to the teaching practices based on the selected context is necessary but challenging.
% }
\penguin{As E1 summarized after the meeting with her teaching team, the outcome plan of several lessons surrounding a context should include targeted reading materials and analysis in each lesson, an introduction facilitating students' engagement, and related teaching activities.}
% Enough teaching resources should at least include 
% comprehensive lesson planning, relevant explanations for each segment, an introduction to facilitate classroom engagement, and specific analyses of the texts, summarized by E1. 
Also, in the evaluation study of \name{} prototype, three teachers (I1, I2, I5) indicated that the system outputs should be more detailed and reduce human effort in modifying them for the later concrete plans for each lesson. % lesson design and instructional materials.
\textit{``The overall structure of the output is good, but I hope it can be more detailed; for example, providing more in-class and extracurricular activities related to the theme, so we can use them directly''} (I1). Therefore, we incorporated recommendations for literature and interdisciplinary course activities in the refined \name{}.

\penguin{
\textbf{Finding 5: Teachers are concerned the quality and reliability of the content purely generated by LLMs.}
}
% \fhx{\textbf{Finding 6: LLM-generated content is unsatisfying due to high repetitiveness and low reliability.}}
In the evaluation study with six teachers (I1-6), our prototype generated traditional subject-related contexts using the LLM with specific templates and cognitive backgrounds of elementary students, without fine-tuning or retrieval-augmented generation (RAG). 
Three teachers (I4, I5, I6) expressed concerns about the quality of the LLM-generated content. \textit{``The content generated for the art subject is quite repetitive''} (I5). \textit{``We need to establish a dedicated article database for science as well, since many of our articles are highly relevant to science''} (I4). 
% Therefore, in the final version of \name{}, we have collected contexts \peng{and articles} from various subjects and use them as a foundation for generating content, which could reduce repetition and increase reliability of suggested context. 


\penguin{
\textbf{Finding 6: Teachers suggest six metrics for evaluating the outcome of interdisciplinary literature lesson plan.}
% In foundational study, the team of E1-E7 established six metrics for evaluating the quality of contexts and teaching resources.
As established by the team of E1-E7, the metrics are: 
}
% \fhx{\textbf{Finding 5: Metrics for evaluation.}}
% In foundational study, the team of E1-E7 established six metrics for evaluating the quality of contexts and teaching resources.
\begin{itemize}
    \item \textbf{Appropriateness of Context}
    \begin{itemize}
        \item \textit{Content Alignment:} Does the context accurately cover the content of the selected materials?
        \item \textit{Internal Logic:} Is there a logical connection between the context and the selected materials?
    \end{itemize}
    \item \textbf{Alignment with Educational Objectives}
    \begin{itemize}
        \item \textit{Curriculum Standards:} Does the content comply with national curriculum standards and teaching guidelines?
        \item \textit{Subject Goals:} Does it help achieve specific goals of language education, such as reading comprehension and writing skills?
    \end{itemize}
    \item \textbf{Depth of Integration}
    \begin{itemize}
        \item \textit{Subject Integration:} Does it effectively integrate knowledge from different subjects?
        \item \textit{Knowledge Transfer:} Does it promote the application of language arts knowledge in other subject contexts?
    \end{itemize}
\end{itemize}


% We summarize the design principles (\textbf{DPs}) of \name{} derived from the foundation study and evaluation of the prototype as below. 

% We employed the reflexive thematic analysis method to analyze the transcribed recordings and all textual content derived from the foundational study. The analysis yielded four summarized Design Principles. In a similar manner, we analyzed the \textbf{formative user evaluation}, incorporating usability issues identified by the teachers and categorizing them under the corresponding DP1 - DP4.

\subsection{Design Principles} \label{sec:principles}

\penguin{Based on the findings from our design process and related literature, we derive four design principles of \name{}.}
% \fhx{Based on the findings from our iterative design process, we propose four human-centered design principles for the development of \name{}.}

% \textbf{DP1: \name{}'s creative thinking support process should align with teachers' habitual behaviors in interdisciplinary context ideation.} % (Finding 1, Finding 2)
\penguin{
% \textbf{DP1: \name{}'s ideation support process should align with teachers' habitual practices in interdisciplinary context ideation.}
\textbf{DP1: \name{} should provide step-by-step support that aligns with teachers' habitual practices in interdisciplinary context ideation}.
Tailoring the assistance to users' habitual practices (\eg active students' behaviors or teachers' behaviors) is a commonly enacted principle in previous interactive systems in educational scenarios~\cite{fok2024qlarify,liu2024classmeta,fan2024lessonplanner}. 
% For example, LessonPlanner \cite{fan2024lessonplanner} adapts the nine events of Gagne's instructional theory to support the planning of one lesson, 
% and ClassMeta \cite{liu2024classmeta} displays various behaviors commonly observed among active students to promote VR classroom participation. 
In the task of interdisciplinary context ideation, as revealed in Finding 1, \name{} should support step-by-step context ideation through reverse thinking, which is a habitual practice of our teachers. 
\fanhx{In this practice, as E1 shared,} teachers take various roles to analyze the potential contexts 
\fanhx{(we note this role as Context Analyst)}, 
analyze the texts in the reading materials, connect them to the contexts, 
\fanhx{and discuss the approaches (Text Analyst and Text Reviewer)}. 
After that, teachers try to summarize the contexts and associated reading materials into an actionable lesson plan \fanhx{(Context Summarizer)}. 
\name{} can prompt LLMs to play different roles when supporting teachers in each of these steps. 
We do not chase for generating one-step context ideation outcome with one LLM prompt, because teachers desire necessary human input in each step, and enabling multiple LLM agents to simulate human-human collaboration has been proven to improve output quality~\cite{wu2024transagents, du2024multi}. 
}
% \textit{DP1.1: \name{} should support context ideation through reverse thinking (Finding 1).} 
% Teachers engage in two distinct cognitive processes when ideating contexts. The first process, referred to as ``forward thinking'', resembles ``deductive reasoning''. 
% Teachers draw abstract connections from a limited number of reading materials (typically 3-5 texts) and develop a concrete and reasonable context. 
% The second, which is termed ``reverse thinking'' and analogous to ``inductive reasoning'', is a more habitual cognitive process employed by teachers. Teachers would like to first select interdisciplinary contexts that they deem suitable and then identify appropriate texts from a broader text pool, after which they refine the connections between the identified text and the context. \textit{``For us, a good context often arises from a sudden inspiration, which we then backtrack to complete the ideation of what texts can connect to this context and how''} (E1).

% \textit{DP1.2: \name{} should facilitate the extraction of potential connections among the reading materials (Finding 2).}
% \subsubsection{DP1.2: Facilitates the extraction of relevant connections from reading materials.}
% In the process of ``reverse thinking'', identifying the connections between the established context and the reading materials is challenging. 
% Teachers must evaluate the effectiveness of the connections in enhancing students' understanding of both literature and its associated subjects, as well as in stimulating their interest. 
% \textit{``It is quite difficult and usually takes a long time for our team to ensure that based on the reading materials, our teaching activities connected by the context can indeed help students gain knowledge''} (E1).

% \fhx{
% We consider human habitual behaviors to design the system, aligning with the human-centered system for educational scenarios~\cite{fok2024qlarify,liu2024classmeta,fan2024lessonplanner}. 
% Additionally, for LLM-empowered systems, enabling LLM agents to simulate human-human collaboration has been proven to improve output quality~\cite{wu2024transagents, du2024multi}. 
% In summary, DP1 is formulated by integrating the advantages of aligning system's performance with teacher's habitual practices and the insights from Finding 1 and Finding 2.
% }
% \subsubsection{\textbf{DP1: \name{} supports creative thinking modes that align with teachers' habitual practices in context ideation}}

% \penguin{
% \textbf{DP2: \name{} should act as a context analyst that helps  from a comprehensive database of interdisciplinary contexts and articles}
% Especially, \textit{DP1.2) \name{} should facilitate the extraction of potential connections among the reading materials}, as Finding 2 reveals that this is a challenging step in the ideation process. 

% }
\penguin{
\textbf{DP2: \name{} should provide teachers with detailed analyses of the contexts, reading materials, and their relationships.}
% Ideating interdisciplinary context for literature teaching requires integrating knowledge across multiple subjects (\eg science, history), which teachers may not be familiar with. 
Our teachers reported that it was challenging to identify the connections between contexts and reading materials (Finding 2) and desired support during the analyses (Finding 3). 
Previous HCI works have demonstrated the strengths of LLMs in analyzing and connecting complex information~\cite{zheng2024disciplink, chi24_Selenite}. 
% For example, DiscipLink~\cite{zheng2024disciplink} can prompt LLMs to automatically expand queries with disciplinary-specific terminologies and highlight the connections between retrieved papers and questions. 
% Selenite~\cite{chi24_Selenite} employs LLMs to generate comprehensive overviews of options and criteria grounded in search results, guiding users through complex decisions. 
Similarly, to satisfy user requirements (Finding 3) in our task, \name{} can leverage LLMs to recommend contexts, explain them in detail, identify relevant texts in the reading materials, and assess the relationship between the contexts and texts. 
}

% \textbf{DP2: \name{} should provide teachers with detailed analyses of the reading materials and contexts and 
% % precise
% \fhx{support verification of the generated content}. }
% \fhx{
% Interdisciplinary context ideation involves integrating knowledge across multiple subjects, which makes it challenging for elementary literature teachers to assess whether the generated content is comprehensive and accurate in unfamiliar domains (\eg mathematics, science).
% A large amount of complex interdisciplinary information may lead users to feel ``information overload''~\cite{foster2004nonlinear, newby2011entering}.
% Therefore, reducing the cognitive load on educators when understanding generated content is important. 
% On one hand, system-generated content should be detailed to minimize misunderstandings, 
% requiring multiple levels of granularity in analysis and explanation (Finding 3). 
% On the other hand, systems can support users in verifying the generated content to facilitate the integration of information~\cite{foster2004nonlinear} by providing reliable and critical feedback.
% }


% During the ideation process, to help teachers prepare for the later teaching activities, the system should help teachers analyze and understand the reading materials and contexts at three three levels of granularity. 
% The first level is in-depth single-text analysis, which \textit{``explains how the elements of a given article relate to the context''} (E1). 
% The second level is pairwise comparison, where comparative reading has been demonstrated to be an effective method for understanding texts, \textit{``allowing articles to `disappear in pairs' by analyzing the similarities and differences in relation to the context''} (E1). 
% The third level is multi-text driven exploration, which necessitates that the system should support the comprehensive linking of all texts selected by the teacher. Therefore, this level requires \name{} to conduct a thorough deconstruction of contexts, extract meaningful connections, and convey these connections to the teachers. 
% Besides, the analyses should also be accurate, especially in the filed of education. This not only necessitates high-quality AI-generated content but also emphasizes the assurance of editing all contents by teachers freely within the user interface. \textit{``I want to freely edit rather than just drag or choose fixed options. This allows me to directly correct issues when I discover them''} (E1). 
% Additionally, teachers believe that the system should enable them to pose detailed and flexible questions about the contexts and reading materials.  \textit{``Our team hopes that teachers can input vague, open-ended questions to the LLM''} (E1).

\penguin{
\textbf{DP3: \name{} should document the ideation outcomes in a lesson plan that aligns with the established educational practices in interdisciplinary literature teaching.}
A teacher without a lesson plan may struggle to effectively deliver the knowledge and objectives of the lesson~\cite{iqbal2021rethinking}. 
Finding 4 suggests that the lesson plan should contain detailed instructional activities, including the related reading materials and in-class activities, based on the selected contexts. 
To make it further aligned with educational practices, \name{} can adopt the six evaluation metrics of the outcome lesson plan (Finding 6) to guide the generation of ideation outcomes. 
}

% \pzh{
% \textbf{DP3: \name{} should provide structured and high-quality ideation outcomes 
% \fhx{that align with the educational requirements of interdisciplinary literature instruction in elementary school.}
% }}
% \fhx{
% A teacher without a lesson plan may struggle to effectively deliver the knowledge and objectives of the lesson~\cite{iqbal2021rethinking}. 
% Additionally, as Finding 4 revealed, this kind of lesson plan generated from the system should be as detailed as possible, at least including the contents of each lesson, an introduction for engagement, and classroom activities.
% Moreover, to better align the outcomes with the educational requirements, six metrics identified in Finding 6 could be integrated into the generation process as the guidelines.
% }


% \textit{DP3.1: \name{}' outputs should be consistent with educational practices
% \fhx{of literature instruction in elementary school }
% (Finding 4).} 
% % E1 emphasized that system outputs should contain the necessary content tailored to the teaching practices in her team. 
% % E1 summarized that the outputs should at least include comprehensive lesson planning, relevant explanations for each segment, an introduction to facilitate classroom engagement, and specific analyses of the texts. 
% % In the evaluation study of \name{} prototype, three teachers (I1, I2, I5) indicated that the system outputs should be more detailed and reduce human effort in modifying them for the later concrete plans for each lesson. % lesson design and instructional materials.
% % \textit{``The overall structure of the output is good, but I hope it can be more detailed; for example, providing more in-class and extracurricular activities related to the theme, so we can use them directly''} (I1). Therefore, we incorporated recommendations for literature and interdisciplinary course activities in the refined \name{}.

% \textit{DP3.2: \name{} should incorporate metrics that evaluate the quality of 
% outcome 
% \fhx{disciplinary }
% contexts
% \fhx{for literature instruction in elementary school}
% (Finding 5).}
% % The team of E1-E7 established six metrics for evaluating the quality of contexts: 
% }

% \fhx{
% The large amount of complex interdisciplinary information may lead users feeling ``information overload''~\cite{foster2004nonlinear, newby2011entering}. Thus, system outputs should be in a well-designed structure, including useful details while avoiding information unrelated to the instructional activities. 
% This helps prevent teachers from spending more effort on adapting or verifying the generated content.
% In other words, first, outcomes of \name{} require only simple interpretation and minor adjustments to help educators rapidly create lesson plans (Finding 4). 
% Second, systems should support users in verifying output information to facilitate information integration~\cite{foster2004nonlinear}.
% The evaluation metrics from Finding 5 can be integrated into the system to address this issue.
% }

\pzh{
% \begin{itemize}
%     \item \textbf{Appropriateness of Context}
%     \begin{itemize}
%         \item \textit{Content Alignment:} Does the context accurately cover the content of the selected materials?
%         \item \textit{Internal Logic:} Is there a logical connection between the context and the selected materials?
%     \end{itemize}
%     \item \textbf{Alignment with Educational Objectives}
%     \begin{itemize}
%         \item \textit{Curriculum Standards:} Does the content comply with national curriculum standards and teaching guidelines?
%         \item \textit{Subject Goals:} Does it help achieve specific goals of language education, such as reading comprehension and writing skills?
%     \end{itemize}
%     \item \textbf{Depth of Integration}
%     \begin{itemize}
%         \item \textit{Subject Integration:} Does it effectively integrate knowledge from different subjects?
%         \item \textit{Knowledge Transfer:} Does it promote the application of language arts knowledge in other subject contexts?
%     \end{itemize}
% \end{itemize}

\penguin{
\textbf{DP4: \name{} should include database of interdisciplinary contexts and reading materials and provide flexible user control to achieve high-quality ideation outcomes.} 
Prior research on the impact of LLMs in primary education indicates that generating false content is a disadvantage that may lead to ``information pollution'' for children~\cite{adeshola2023opportunities, murgia2023chatgpt}.
Finding 5 also indicates that LLMs sometimes were unable to create content that meets teachers' needs when lacking access to educational resources. 
To generate high-quality outcomes, as inspired by previous works~\cite{yazici2024gelex, khanal2024fathomgpt}, 
\name{} could ground the content generation on diverse real-world contexts and reading materials. 
% This approach not only reduces repetition but also increases the reliability of the suggested material. 
% Apart from high-quality AI generation, 
\name{} should also support teachers to freely edit and question any content (\eg texts in reading materials, outcome lesson plan) to make sure that they understand the content they are going to use in literature teaching. 
}

% \textbf{DP4: \name{} should include a comprehensive database of interdisciplinary contexts and articles.}
% \fhx{
% Prior research on the impact of LLMs in primary education indicates that generating false content is a disadvantage that may lead to ``information pollution'' for children~\cite{adeshola2023opportunities, murgia2023chatgpt}.
% From Finding 5, we also noticed that LLMs are sometimes unable to create content that meets teachers' needs when lacking access to educational resources in evaluating the \name{} prototype. 
% Therefore, inspired by previous works~\cite{yazici2024gelex, khanal2024fathomgpt}, 
% % we recognize that equipping LLMs with a comprehensive database allows them to generate more relevant content by retrieving and clustering educational texts from various subjects. 
% we could try to collect diverse contexts and articles to serve as a solid foundation for content generation in the final version of \name{}. 
% This approach not only reduces repetition but also increases the reliability of the suggested material, helping to prevent potential misinformation that could negatively impact students and teachers.
% }
% In the evaluation study with six teachers, our prototype generated traditional subject-related contexts using the LLM with specific templates and cognitive backgrounds of elementary students, without fine-tuning or retrieval-augmented generation (RAG). 
% Three teachers (I4, I5, I6) expressed concerns about the quality of the generated content. \textit{``The content generated for the art subject is quite repetitive''} (I5). \textit{``We need to establish a dedicated article database for science as well, since many of our articles are highly relevant to science''} (I4). Therefore, in the final version of \name{}, we have collected contexts \peng{and articles} from various subjects and use them as a foundation for generating content, which could reduce repetition and increase reliability of sugggested context. 
}