\section{Discussion}\label{sec:discussion}
% \subsection{Design Considerations}
\subsection{Design Considerations \fanhx{for AI-Teacher Collaboration in Interdisciplinary Contexts}}
Based on the findings from two phases of experiments, we propose three design considerations for interdisciplinary context ideation.

\subsubsection{Encouraging Teacher Reflection Rather Than Full Automation}
One of the important features of \name{} is its high level of automation. During the exploration process, users can generate interdisciplinary contexts with detailed analyses and outcomes without the need for manual input unless asking the LLMs for answers.
This presents a significant advantage for experienced experts who possess a clear understanding of their requirements for classroom practice, as it alleviates their workload (\eg reducing the need for typing and searching), allowing them to concentrate fully on exploration.
However, findings from the within-subjects study indicated that novice teachers' behaviors suggest a potential risk associated with excessive automation. They might rely too heavily on the system, potentially neglecting their own cognitive process and allowing \name{} to dominate the exploration, which could result in content that is disproportionately focused on disciplinary or literature-specific outcomes.
% \name{} is designed to assist teachers in the ideation process rather than be a teacher.
\penguin{
Therefore, \name{} should encourage teachers to make reflections to improve their teaching. 
% This means it should be considered as a reflective tool.
Such a focus is already central to other teaching tools, such as simulations that enhance student engagement~\cite{tang2024vizgroup} and help teachers understand students' learning states in online classrooms~\cite{ma2022glancee}.
}
% Therefore, like many other ideation tools, it should integrate more interactive features that encourage user reflection.

% \subsubsection{Providing a Large Variety of 
% \fhx{High-Availability}
% Contexts in Each Exploration}
\penguin{\subsubsection{Improving the Diversity and Suitability of Informal Educational Context Pool}
Based on DP4, we have collected 113 informal educational contexts from annual popular words in China and 144 subject-related contexts from textbooks to enable \name{} to suggest contexts.
}
% Based on DP4 (include a comprehensive database of interdisciplinary contexts), we have developed a context pool to support \name{}'s RAG tool in offering users more diverse contexts.
In the within-subjects study, participants expressed appreciation for the diversity of contexts available in the laboratory setting. 
\penguin{
However, experts reported the lack of diverse and suitable contexts, especially the informal ones, when they would like to apply them in the literature courses. 
% This issue is due to the repetitive recommendations each time \name{} retrieves similar contexts in its RAG process, 
\fanhx{Besides, in the experiments, we observed 
% in the output of Context Analyst that 
\name{} retrieves similar contexts in its RAG process
to make recommendations.
% , even when users selected different reading materials. 
While implementing advanced RAG strategies
% may help to relieve this issue, 
might alleviate this,
% we suggest future interdisciplinary ideation tools like \name{} should explore approaches to improve the suitability of Informal Educational Contexts pool.
we argue for the need for more diverse datasets.
}
The contexts sourced from Yao Wen Jiao Zi include popular words that Chinese teachers and students are likely to hear about, which could ease teachers' load in understanding the contexts. 
However, these popular words (\eg ``My youth is back!'', ``Prime Spot Debut'') may not always align seamlessly with pedagogical goals~\cite{sidekerskiene2024pedagogical} and may lean toward entertaining purposes. 
Therefore, future work could consider identifying suitable types of informal contexts with teachers and collecting more such contexts from diverse sources.
% like social media posts and blogs. 
}
\fhx{
% However, the long-term users reveal a limitation: the lack of available contexts, especially for the informal educational contexts.
% On one hand, it is due to algorithm-induced repetitive recommendations.
% We found that certain contexts are frequently recommended by the Context Analyst, suggesting that implementing advanced RAG strategies could help reduce context repetition.
% On the other hand, the limitations of context resources decrease the availability of contexts. 
% For the Informal Education Database, the vocabulary in Yao Wen Jiao Zi is updated annually to reflect current events, ensuring it stays relevant. 
% This reduces the cognitive load on teachers when exploring contexts due to ease of understanding.
% However, in the context of elementary literature teaching, popular words may not always align seamlessly with pedagogical goals~\cite{sidekerskiene2024pedagogical}, and some terms might lead to teaching content that leans toward entertainment.
% As a result, teachers may naturally bypass these words during exploration due to their low suitability.
% To support continuous user engagement, it is essential to offer a diverse range of high-availability contexts in each exploration session, drawing from curated databases, textbooks, or websites.
}
% However, in the classroom environment, teachers need to utilize our system each time they conduct interdisciplinary teaching. This necessity leads to a significant limitation: the lack of diverse contexts, as teachers often observe a considerable degree of repetition, even when selecting different reading materials.
% A straightforward solution to this issue is to expand the context database. 
% Additionally, we found that certain contexts are frequently recommended by Context Analyst, suggesting that the implementation of advanced RAG strategies could reduce context repetition. To support continuous user engagement, it is essential to provide a wide variety of unique contexts in each exploration session.

\subsubsection{Meeting the Expectations of Different Users}
\penguin{In accordance with DP1, we carefully designed \name{}'s workflow to align with teachers' practices and developed prompts using templates provided by E1-E7. %, offering structured outputs.
However, our findings reveal that different user groups have varying expectations of the systems. 
For example, users without AI knowledge may need more expressive support that encourages them to criticize their own thoughts as well as the LLM output. 
Pre-service or new teachers could also be more depending on \name{} in the ideation process. 
% Further research should investigate similar tools to \name{} that 
\fanhx{To}
support and motivate specific users, 
% \eg students that become teachers soon, to conduct critical thinking in the context ideation process. 
\fanhx{collecting various templates from them and replace the original ones in \name{} can be beneficial (\eg prompting Text Reviewer to offer more incisive feedback can motivate teachers to think deeply).
Also, providing visualization of agent interactions and outputs can help users better understand the meaning of each step, to promote their critical thinking.}
% The fixed workflow and outcomes of \name{} cannot meet the expectations of specific users, so supporting user personalization, \eg encouraging critical thinking should be considered.
% For example, \name{} should provide more persuasive descriptions for contexts to encourage expert and conservative teachers, while offering a variety of inspirations for novice teachers to choose.
}
% However, teachers have different expectations of the system.
% Some teachers anticipate that the system will only provide inspiration, while others expect it to generate comprehensive resources, including lesson plans and student assignments. 
% The fixed workflow and outcomes of \name{} cannot meet the expectations of specific users, so supporting user personalization should be considered.
% \fhx{
% % \subsection{Context Pool Resources from Yao Wen Jiao Zi}
% \subsection{Benefits and Concerns of the Context Resources}
% We extract ``Annual Popular Words'' from the Chinese journal Yao Wen Jiao Zi for Informal Education Database, and collect topics from widely used textbooks for context resources of traditional subjects.
% Our evaluation revealed that although this method is usable, it's not ideal. 
% We further analyzed its strengths, weaknesses, and potential biases as follows.}

% \fhx{
% For the Informal Education Database, the vocabulary in Yao Wen Jiao Zi is updated annually according to current events, making it highly relevant and ensuring that curriculum design is no longer static but keeps pace with the times.
% This enhances the system’s potential for practical application in teaching and reduces the cognitive load on teachers when exploring various contexts because of ease of understanding.
% However, in the context of elementary literature teaching, popular words may not always align seamlessly with pedagogical goals~\cite{sidekerskiene2024pedagogical}. 
% Certain terms might lead to teaching content that leans toward entertainment or overly socialized topics, neglecting themes of greater academic value or cultural significance.
% % Content generated using annual popular words may diverge from the educational objectives of elementary literature courses. 
% Moreover, since these words are manually selected, they inherently reflect the evaluators’ subjective standards and values—for instance, prioritizing widely disseminated or trending terms while overlooking less popular but education-relevant vocabulary.
% In the context of an AI-human collaboration, this bias may amplify the use of certain inappropriate terms, undermining diversified teaching and eroding teachers’ trust in the generated content, ultimately impacting student learning outcomes.
% }
\pzh{
\subsection{Reflections on AI-Teacher Collaboration}
% \subsection{AI-teacher collaboration}
}

\penguin{
% \textbf{Point 0: emphasizing the importance of AI-teacher collaboration -- ``Many would argue the idea of AI replacing teachers misunderstands the complexity of teaching. What this work does is provide a concrete example of how AI can assist teachers, both in the design process and the final product.''}

Previous researchers on human-computer interaction and education have proposed various intelligent tutoring agents, \eg QuizBot \cite{ruan2019quizbot}, Sara \cite{winkler2020sara}, and DesignQuizzer \cite{peng2024designquizzer}. 
These agents act like teachers to prompt questions and give adaptive feedback to learners, which do supplement the lack of accessible teachers in online learning scenarios. 
Nevertheless, we argue that human educators are still irreplaceable, especially in offline classrooms. 
Teaching is not just about transferring knowledge but also about shaping students' cognitive, social, and emotional development~\cite{adeshola2023opportunities, chan2024will, guilherme2019ai}, which normally requires human teachers' interaction and empathy with students. 
% Besides, human teachers can tailor their teaching to fit the cultural, social, and environmental contexts that influence how students learn,
Yet, these are qualities that AI currently lacks due to its generalization and lack of contextual awareness. 
% Yet, teachers traditionally need to spend a large amount of effort exploring substantial materials to ideate and prepare the interdisciplinary teaching activities, while AI is good at processing complex and large-scale data.

Our design, development, and evaluation of \name{} provide a concrete example and implications of AI-teacher collaboration in such interdisciplinary teaching scenarios. 
% \textbf{Point 1: LLMs for easing teachers' load in some tasks (and not in some tasks) -- what \name{} does and can be improved}
In the design process, we identified with teachers the places where LLMs can help in their context ideation process, such as detailed analyses of the contexts, reading materials, and their relationships (DP2 \autoref{sec:principles}) as well as documented outcome (DP3). 
This human support helps AI improve its outcome lesson plans and promotion of knowledge transfer to that subject (\autoref{fig:rq1_results}). 
In addition, \name{} reduced users' workload in ideating contexts for literature teaching (\autoref{fig:rq2_results}), supporting that AI can serve as an auxiliary tool or a collaborator to improve the overall efficiency of teachers' tasks~\cite{holter2024deconstructing}. 
% With teachers' suggestions and feedback, w
We implemented \name{} that aligns its LLM output as closely as possible with teachers' practices (DP1)
% and provides flexible user control (DP4). 
% Nevertheless, 
, but the novices' outcome lesson plans with \name{} did not align effectively with the teaching goals (\autoref{fig:rq1_results}), and experts also pointed out that some recommended contexts were unreasonable. 
% They speculated that this misalignment might stem from a lack of detailed descriptions of curriculum standards or relevant aspects of students' backgrounds and experiences. 
% This feedback from teachers is instrumental in the development of AI systems like \name{}, particularly in the creation of customized teaching templates and strategies~\cite{nguyen2023exploring}. 
It indicates that the iterative design of AI-teacher collaborative systems can be longitudinal, wherein the diverse teaching experiences can be utilized as data to refine AI-generated content through techniques such as fine-tuning or multi-modal fusion. 

We look forward to building an extensible version of \name{} in which teachers without programming skills can easily customize the system based on their needs to achieve more effective collaboration with AI. 
% For example, the utilization of LLMs by educators shared in social media \cite{mogavi2024chatgpt} may provide template functions (\eg generate a definition) that users could add in the \name{}. 
% \textbf{Point 2: teachers' input for enhancing LLMs' performance in generating teaching materials -- what \name{} does and can be improved}
However, the ultimate responsibility for decisions of enacted teaching activities must remain with human educators. 
% Nevertheless, 
}

\begin{comment}
In our study, teachers not only employed AI-based systems to improve educational outcomes with the help of LLMs, but they also played a significant role in supporting the LLM in generating higher-quality content in the iterative design process, by incorporating templates into prompts, and providing the mode of human-simulated LLM workflows. 

Teachers, as primary stakeholders and direct implementers of educational practices, possess a unique ability to identify the actual needs present in classroom instruction. 
Research indicates that feedback from teachers is instrumental in the development of AI systems, particularly in the creation of customized teaching templates and strategies~\cite{nguyen2023exploring}. 
In contrast, LLMs that have lack specific knowledge in these fields, so teachers' practical experience helps to mitigate this shortcoming. This process can be longitudinal, wherein the diverse teaching experiences can be utilized as data to refine LLM outputs through techniques such as prompt engineering or fine-tuning. For example, insights derived from User Study may contribute to the template of the prompts to enhance the performance of the LLM in \name{}. This iterative process promotes a more integrated collaboration between AI and humans. 

Conversely, AI can serve as an auxiliary tool or collaborator to improve the overall efficiency of teachers' tasks~\cite{holter2024deconstructing}. As AI takes on more tasks, it is crucial to maintain human dominance in workflows and ensure transparency and controllability~\cite{abedin2022designing}. For instance,in the domains of design and creative work, AI can assist humans by proposing and generating a variety of options; however, the ultimate responsibility for creative decisions must remain with humans~\cite{figoli2022ai}.

\fhx{
In the field of education, it is believed that AI cannot substitute for human educators as it fails to account for the multifaceted nature of education, where teachers not only deliver information but also play an irreplaceable role in shaping students' cognitive, social, and emotional development~\cite{adeshola2023opportunities, chan2024will, guilherme2019ai}.
Despite our concerted effort to align the LLM output as closely as possible with teachers' practices during the design of \name{}, some participants believed that the system's outcomes did not align effectively with the teaching goals. 
They speculated that this misalignment might stem from a lack of detailed descriptions of curriculum standards or relevant aspects of students' backgrounds and experiences.
This suggests a potentially discouraging conclusion: the additional information provided to the AI might never be sufficient, as educational practice is intertwined with complex factors such as social context, students’ knowledge stage, and teachers’ perceptions, which are challenging to conceptualize.
Nevertheless, our work demonstrates that AI has the potential to assist teachers in ideating how to instruct courses. 
This implies a diminishing in responsibility, reducing AI's role in directly managing the classroom and instead focusing on generating precise descriptions of key concepts and activities.
}
% In the domain of education, the role of AI should not be conceptualized as "replacing teachers"~\cite{guilherme2019ai}, but rather as a decision-support tool that augments teachers' professional judgment.
% This perspective is consistent with users' concerns about over-automation in Experiment I. Furthermore, the lower alignment scores in educational outcomes suggest that, in the absence of teachers' comprehensive understanding of curriculum standards, AI alone is insufficient to create a well-rounded classroom environment nowadays.

% In summary, the collaboration between AI and teachers mutually reinforces one another, and this ongoing process has the potential to continuously improve the quality of education through the integration of AI interventions.

\end{comment}


\penguin{
\subsection{Generalizability}
\fanhx{\name{} is open-source, allowing elementary school Chinese literature teachers to easily deploy and use it.} \name{} also has great potential to be generalized to support interdisciplinary teaching in elementary subjects apart from literature, in higher education, in other cultural contexts, and in self-learning tasks. 
For example, many elementary schools have explored STEM (science, technology, engineering, and mathematics) as an interdisciplinary approach to help students develop critical thinking, creativity, and problem-solving skills~\cite{english2015stem, sinatra2017speedometry}, and some universities have opened interdisciplinary programs in which students can take courses in various domains. 
% These teachers should also engage in the continuous exploration and evaluation of information from unfamiliar fields to prepare for an effective course. 
\name{} can facilitate these educators with detailed analyses of the contexts and teaching materials, but to ensure it aligns with teachers' habitual practices, teachers' involvement in the design process is essential. 
To generalize \name{} and our findings to other cultural or language contexts, we should further be aware of the biases embedded in the context pool. 
Many of \name{}'s suggested contexts are sourced from annual popular words in China, which may potentially lead to misunderstanding or miscommunication in a multicultural environment. 
Lastly, while we position \name{} as a support tool for teachers, its analyses of reading materials and contexts could be useful for self-learners. 
For instance, learners can select the literature they are interested in, get recommended contexts from \name{}, and read them within a context in their after-school time. 
}

\begin{comment}
The design of \name{} is intended for interdisciplinary literature instruction in elementary schools. 
Its high usability, combined with its ability to empower teachers to delve into interdisciplinary opportunities, indicates that this tool could be applied across a wide spectrum of educational contexts.
\name{} offers a unified context ideation system that includes a modular back-end and a general interface, requiring only minor modifications to be applicable to other language environments.
For this purpose, the Informal Context Pool should be replaced with contexts that has specific cultural backgrounds. 
Since its original content from Yao Wen Jiao Zi lacks a global perspective, it is regarded as biased.
\name{} is also capable to extend to more subjects, such as STEM (science, technology, engineering, and mathematics). 
% To achieve this, the text pool should be customized by replacing the content with relevant chapters. 
A co-design session with expert teachers, as we did in the iterative design process, is necessary to assist the developer in crafting the prompts to define the relevant subject-specific information, utilizing the provided templates and guidelines. 


\fhx{
\name{} also facilitates personalized learning for learners.
The recommended generated-contexts enable learners to engage with specific content in scenarios that are more familiar or of interest to them. 
These recommendations can serve as the integral component of human-centered learning applications, thereby promoting educational equity.
}
\end{comment}

\subsection{Limitations and Future Work}
This study has several limitations. 
\penguin{
First, we only met with the head teacher E1 of the interdisciplinary literature teaching team (E1-E7) in the foundational study.
\fanhx{
It introduced limitations such as a one-sided perspective or inaccurate representation of information, which might challenge the robustness of the derived knowledge.
}
% Though the team members contributed to the design of \name{} via our contact with E1, it
It would be more effective and insightful if we could directly reach them, \eg via a brainstorming session or design workshop, 
\fanhx{and involve more teaching teams}. 
}
Second, the participants in the within-subjects were students assigned to work as teachers, which means they do not possess adequate knowledge of teaching standards and the objectives of literature lessons.
Therefore, the data obtained may not accurately reflect the perspectives of actual teachers.
\fanhx{
Also, the within-subjects study design with a small sample size ($N=16$) significantly limits the robustness of our quantitative findings. 
Participants in our user study directly compare the systems rather than evaluate each system independently, which may lead to biased judgment.
Future studies should consider a between-subjects study design with a larger sample size.
}
\penguin{
Third, though we had an experienced teacher (E1) to score and comment on the outcome of ideation in Experiment I following the given criteria, it would further reduce the level of subjectivity involved in this rating task by inviting multiple experienced raters. 
% we invited one teacher with expertise in interdisciplinary literature instruction to evaluate the outcomes in the within-subjects study.
% However, a more ideal way to assess the outcomes would be to invite multiple experts to evaluate them, thereby reducing the level of subjectivity involved.
}
% Second, \name{} currently focuses on elementary school literature teachers. 
% In the future, we plan to expand the targeted user group by co-designing with a broader range of educators. 
% For example, collaborating with science teachers to create templates could enhance the applicability of \name{} for interdisciplinary instruction in science subjects.
\penguin{
Fourth, we evaluated \name{} in a short period and did not systematically study participants' usage patterns. 
A future field study is needed to observe teachers' usage of \name{} in their classrooms over a long period, \eg the adopted contexts and associated instructional methods they actually use.  
% Fourth, \name{} focuses on the process of contexts ideation, but we have not yet studied how users utilize our tool over a long period (\eg over six months) and apply these contexts in real classroom settings.
% In future research, a field study could be conducted to observe the ways in which teachers use \name{} to ideate contexts, the instructional methods they employ based on these contexts, and the homework assignments they provide. Additionally, collecting feedback from the students in the process to further improve \name{}.
}
Fifth, our system is limited to textual content generation. Research indicates that the integration of multi-modal information (\eg images, and videos) can enhance educational activities. Future work could explore the incorporation of these elements into the ideation process within interdisciplinary contexts.
Finally, the algorithm implemented in \name{} is not yet the most advanced LLM technology. 
Incorporating more advanced RAG algorithms and multi-agent LLM architectures has the potential to enhance the quality of the system’s output.