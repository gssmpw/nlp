\section{Related Work}
\label{sec:related_work}
\subsection{Interdisciplinary and Literature Instruction}
Interdisciplinary education equips students to think across subject boundaries, addressing complex global challenges and enhancing critical thinking, problem-solving, and collaboration skills~\cite{rhoten2006interdisciplinary}.
Its growing importance is evident in fostering adaptable, innovative learners for a rapidly changing world~\cite{you2017teach}. 
By applying language in real-world contexts, disciplinary approaches promote engagement~\cite{towhidnejad2011introducing}. Interdisciplinary education offers key advantages in K-12 language learning, promoting critical thinking, creativity, and motivation by connecting learning to real-world issues~\cite{bestelmeyer2015collaboration, hardre2013teachers}. 
Empirical studies often focus on STEM fields, such as~\citet{english2016stem}’s integration approach and~\citet{sarama2017interdisciplinary}'s C4L curriculum, which combines language learning with early childhood development. Also, \citet{czerniak2014interdisciplinary} demonstrated how interdisciplinary science teaching enhances language development.

With the introduction of the new curriculum standards by China's Ministry of Education in 2022, interdisciplinary curriculum design was formally included in the ``expansive learning task group'' of Chinese language education, explicitly outlining requirements for interdisciplinary teaching~\footnote{\url{http://www.moe.gov.cn/srcsite/A26/s8001/202204/W020220420582344386456.pdf}}. 
Guided by these new standards, interdisciplinary teaching in primary school Chinese language classes can bring a fresh perspective to Chinese language education. By the design of diverse interdisciplinary activities, course content is integrated to cultivate students' active learning and inquiry skills. 
Additionally, reading instruction fosters students' interdisciplinary thinking abilities, promotes cross-subject integration, and stimulates students' imagination. 
\pzh{However,}
few works focus on integrating literature with other subjects and customizing technologies for diverse teaching environments. 
\peng{Teachers could face challenges in preparing for interdisciplinary literature teaching due to inadequate professional development}~\cite{boix2010interdisciplinarity, yarker2012analysis}. 
% Long-term impacts on language proficiency and cognitive development are underexplored, and teachers face challenges due to inadequate professional development~\cite{boix2010interdisciplinarity, yarker2012analysis}. 
% Standardized assessment methods are also needed~\cite{moss2008interdisciplinary}.
\peng{Our work is motivated by the trend and benefits of interdisciplinary literature teaching and aims to help teachers ideate interdisciplinary contexts for teaching literature in elementary schools.} 

\subsection{Interactive Systems that Support Brainstorming and Ideation}

Interactive systems have been widely used to facilitate the creative thinking processes, including art design~\cite{heyrani2021creativegan, frich2018hci}, writing~\cite{gero2022sparks}, and learning~\cite{jin2024codetree}. 
Creativity Support Tools (CSTs) provide numerous advantages due to their interactive nature. 
For instance, C2Ideas~\cite{hou2024c2ideas}, an innovative system for designers to ideate color schemes, provides users with an interactive workflow that aligns with traditional interior design methods to facilitate the design process. 
Users are required to input their initial design intentions and customize the intermediate results, after which the system generates good outcomes. Furthermore, in the field of supporting creative activities within interdisciplinary environments, DiscipLink~\cite{zheng2024disciplink} assists users in making sense of information by generating exploratory questions, expanding queries, and extracting themes and connections among academic papers. 
Users can freely explore the carefully designed Orientation View, Exploration View, and Collection View of the user interface, a layout that we also use.

More and more creativity support interactive systems are being designed for specific tasks in particular professions~\cite{louie2020novice, petridis2023anglekindling, choi2024creativeconnect, li2024diaryhelper}. 
% \fhx{
% Inspired by these works, we maintain a similar sample size and experimental setup.
% }
For example, AngleKindling~\cite{petridis2023anglekindling} is designed to assist journalists in exploring diverse angles for reporting on press releases. 
This work is similar to ours, as the creativity support also comes from analyzing large amounts of textual material, and our work also leverages the capabilities of LLMs. 
There also has been a growing trend toward the integration of human-computer interaction (HCI) and education. 
This integration has led to the development of interactive systems aimed at enhancing educational outcomes by promoting users' creative thinking~\cite{zhang2022storydrawer}. NaCava~\cite{yan2023nacanva} is a mobile interactive system designed to facilitate nature-inspired creativity for children. 
It enhances children's multidimensional observation and engagement with nature by encouraging the collection of multi-modal materials and using them in a creation process.

However, few studies focus on developing creativity support tools for teachers to help them design classroom activities to promote student learning, especially in interdisciplinary setups. In our work, we create \name{} aimed at teachers to help them ideate interdisciplinary contexts in education, offering creativity support to facilitate the ideation process.

\subsection{Large Language Models in Education}
Large Language Models (LLMs) are increasingly utilized across various educational domains due to their capacity to facilitate personalized learning experiences and enhance accessibility for all learners~\cite{mogavi2024chatgpt}. 
They support self-learning to enhance personal skills in diverse areas, including programming~\cite{yilmaz2023augmented}, problem-solving ability~\cite{kasneci2023chatgpt}, and writing skills~\cite{shidiq2023use}. Additionally, they also support various activities within classroom environments, such as helping with homework, reviewing~\cite {mogavi2024chatgpt}, and assisting students in project-based learning~\cite{zheng2024charting}. 
Specifically, within the K-12 educational domain, LLMs have demonstrated their effectiveness in providing assistance across multiple subjects, including foreign languages, programming, and mathematics~\cite{mogavi2023exploring}. This capability is attributed to their pre-training on large datasets across various fields~\cite{NEURIPS2020_1457c0d6}.
\penguin{
Recent work also suggested that designing multiple LLM agents to simulate human collaboration can enhance the quality of the outputs on domain-specific tasks, such as translation~\cite{wu2024transagents} and software development~\cite{du2024multi}.
Inspired by these works, we leverage the potential of LLMs and deploy them as agents to generate precise and comprehensive interdisciplinary information in our study.
}

LLMs also promote instructional activities for teachers. 
For instance, RetLLM-E~\cite{mitra2024retllm} provides instructional support by delivering context-aware, high-quality answers to student questions using LLMs. This research illustrates that retrieved context can significantly increase the quality of LLM-generated responses, thereby informing the advantages of 
utilizing retrieval-augmented generation~\cite{lewis2020retrieval} for precise instruction. 
Furthermore, LessonPlanner~\cite{fan2024lessonplanner}, an LLM-driven tool that formulates structured lesson plans based on educational theories, has been empirically validated to assist teachers in increasing the efficiency of lesson preparation and the quality of lesson plans. 
The objectives of our work are similar to theirs, as both emphasize the quality of outcomes and the user experience.

Previous studies have demonstrated the potential of LLMs in education, but few works focus on teachers' preparation for interdisciplinary instruction scenarios.
In our work, we conduct an iterative design with 13 teachers to understand their views on how LLMs can facilitate interdisciplinary literature instruction and support them.