\section{Experiment II: Expert Interviews} \label{sec:experiment_2}
In Experiment I, we evaluated novice teachers' perceptions of our system, including the ideation process and their overall experience using \name{}.
Additionally, an experienced teacher, E1, evaluated the outcomes of the novice teachers. In Experiment II, we shifted our focus to the perspectives of experts \pzh{with varying levels of literature teaching experiences} to gain more feedback. %, specifically teachers with varying levels of experience, 
We conducted expert interviews using a think-aloud protocol and a semi-structured interview with nine teachers to collect insights related to our research questions. 
% Furthermore, in the Findings section, we will present the opinions regarding the unchangeable features of the prototype from I1 to I6 during \pzh{the design process}.

\subsection{Participants}
The study involved nine \pzh{Chinese language} teachers (5 female, 4 male, E1-5, E8-11 in \autoref{tab:teachers}) in a local elementary school, 
% including one novice teacher with 1-3 years of experience, four advanced beginner teachers with 4-6 years of experience, three competent teachers with 7-18 years of experience, and one proficient teacher with 29 years of experience. 
including two novice teachers with less than five years of teaching experience and six expert teachers with at least five years of teaching experience~\cite{booth2021mid}.
% \penguin{
% including two novice teachers with less than five years of teaching experience and six expert teachers with at least five years of teaching experience~\cite{booth2021mid}.
% }
Detailed information about the participants is presented in Table 3.


\subsection{Method}
\penguin{
We conducted interviews offline with E8-E11 and I1-I6 in lab sessions and with E1-E5 online after they freely used it for three days. This setup could help us to gain diverse insights, \eg learnability of \name{} in lab and field environments. 
I1-I6 participated in the evaluation of \name{}'s prototype, and their feedback on the unchanged features in \name{} was also presented in this section. 
}

For \textbf{E8 - E11}, who were not involved in the iterative design process, we conducted offline interviews. The process began with a 5-minute introduction to the research background (\ie the background of interdisciplinary literature instruction). 
This was followed by a 10-minute tutorial on how to use \name{}. 
Participants were then allocated 30 minutes to complete exploration tasks while engaging in a think-aloud protocol. Subsequently, a 15-minute semi-structured interview was conducted.

For \textbf{I1 - I6}, who participated in the evaluation of the prototype process, the procedure was consistent with the same methodology above, except that during the semi-structured interview, we asked for more suggestions regarding the prototype.

For \textbf{E1 - E5}, who engaged in the iterative design process, we conducted online interviews. The \name{} was made available for a duration of 3 days, and a tutorial video was provided. Participants were asked to freely use \name{} during this period to complete exploration tasks. Finally, we conducted a 15-minute semi-structured interview with each expert.


% \subsubsection{Exploration Tasks}
\fanhx{
% We developed the following exploration tasks:
Surrounding our RQs, the interview questions (\autoref{sec:appendix}) are about ideation outcomes, ideation process and perception of \name{}, while the participants were assigned the following exploration tasks:
\begin{itemize}
    \item Task 1: Freely choose 4-15 reading materials, explore two contexts of interest, and add them to the collection.
    \item Task 2: Generate an introduction, course plan, and activities.
\end{itemize}
}
% \subsubsection{Interview Questions}


% \penguin{
% \textbf{RQ1: Ideation outcomes}}
% \begin{itemize}
%     \item What do you think about the quality of the outcomes generated by \name{}?
%     \item Do you think the results generated by our system can assist you in preparing lessons or designing a new interdisciplinary context in the classroom setting?
% \end{itemize}

% \penguin{
% \textbf{RQ2: Ideation process}}
% \begin{itemize}
%     \item Did you feel the task load is high while using \name{}? Specifically, did you feel any increased cognitive load or mental demands?
% \end{itemize}

% \penguin{
% \textbf{RQ3: Perception of \name{}}
% }
% \begin{itemize}
%     \item Do you think our system can assist you in ideating interdisciplinary topics more efficiently?
%     \item Do you think our system can help you better explore connections between texts and the contexts?
%     \item If you were to prepare an interdisciplinary reading case, compared to your usual preparation methods (\eg searching, meeting with other subject teachers, or using ChatGPT), do you think \name{} could support you better?
%     \item Do you trust the output of our system compared to your previous experiences with web searching or using ChatGPT?
% \end{itemize}



\subsection{Findings}
\subsubsection{RQ1: Ideation Outcomes}
Overall, seven of nine teachers confirmed that the outcomes generated by \name{} effectively support literature instruction within the classroom environment. E10 emphasized the comprehensiveness of the text analysis,
\penguin{where she could find the content she wanted within a document more strategically, instead of sifting through countless reference books:} 
\penguin{\textit{``I can abandon varied reference books, (because) the system conducts a comprehensive analysis of the specific content of the texts. It includes themes, content, and key points.''}}
Additionally, E11 noted that the activities could be easily implemented in the classroom: \textit{``yes, these activities (such as music appreciation) can be incorporated into upcoming lessons to increase student engagement.''}
I3 mentioned that the structured outcomes could potentially \textit{``make my teaching process more systematic.''}

\penguin{
We observed the different views between novice and expert teachers on how they embrace the integration of generated activities into real classrooms, though almost all teachers consider \name{}'s outcomes are beneficial for instructions. 
Experienced teachers (E9, E10, E11) preferred their self-centered teaching approach and were more conservative in using the generated activities. %, with three of six indicating they critically and carefully select the provided activities. 
\textit{``To be honest, I have never used these (recommended) activities before, and in the future, I might only add a bit of them to my existing lesson plan to make students more interested.'' }(E11)
On the contrary, newer teachers were more open and tended to reconstruct their established curriculum based on the recommended activities.}
E8 stated, 
\textit{``I plan to adjust my instruction method according to them (activities); some recommended activities are excellent in the current context and can be adapted for use in other contexts.''
}

Despite these positive findings, five of nine teachers pointed out that some outcomes did not align well with the objectives of literature instruction: \textit{``the outcomes are still somewhat disconnected from practical application. The content generated based on the provided template does not fully meet the current teaching needs, especially considering the recently revised curriculum standards''} (E2).
Teachers suggested a potential solution: \textit{``Import the curriculum standards and teaching objectives for each text, and to consider these goals when constructing contexts. For instance, what are the learning objectives and abilities required for each grade level?''} (E5)

Despite modifications made to the prototype, 4 out of 9 experts indicated a need for additional outcomes that could directly enhance student learning in literature: \textit{``the documents generated are very useful for lesson preparation; however, they cannot be directly provided to students for learning purposes. I hope it can produce some homework questions''} (E1).


In summary, the comprehensive and structured outcomes facilitate educational activities; however, additional focus is required to ensure the alignment of these outcomes with the literature objectives.

\subsubsection{RQ2: Ideation process}
Experts have reached a consensus that the task load is low when using \name{} to construct contexts for the classroom, attributable to its well-structured layout and functional settings. E11 said, \textit{``the system does not burden my memory due to the collection feature''.} 
\penguin{
However, E5, who used \name{} in the wild for three days, suggested improvements to its UI design and features to further decrease mental demand: 
\textit{``More specific instructions could be incorporated into the page to clearly indicate the available actions. Additionally, I would like to have a feature that synchronizes generated records through user login for long-term usage.''} 
Experts in the lab sessions commented more on \name{}'s error cases which may increase their effort in the ideation process. 
For example, E8 input ``What activities can be designed around `Stepping Stones' [the title of one text]'', and \name{} responded ``Students can observe stone bridges and steps in their daily lives, explore their design principles and practical uses. They can participate in group projects to build stone bridge models using materials like rocks, experiencing the joy of collaboration.''
E8 commented, \textit{``This is indeed related to the article, but only the content, not the core idea or intended message''}.
E8 did not directly incorporate these outputs into the final outcome; instead, he edited manually.
Additionally, teachers may feel frustrated due to LLM's hallucinations.
I3 input 
``Give me more sentences and analysis related to 'Osmanthus Rain' [the title of one text] and the context about poetic life'', and
LLM responded 
```What I like is osmanthus. The osmanthus tree looks clumsy, unlike the plum tree, which has a graceful posture.' This contrast highlights the author's unique affection for osmanthus and reflects the author's ability to discover poetic elements in life through a lens of beauty''.
I3 remarked that \textit{``it did give me one more sentence but not really fit in the context.''}
In summary, \name{} is generally user-friendly but requires more UI and feature refinements to optimize the ideation process.
% It indicate LLMs can generate information that appears believable but is actually incorrect, instead of simply admitting it don't know the answer.
% This issue arises when users repeatedly ask LLMs similar questions.
}
% Additionally, 
% \fhx{
% as a long-term user, E5 suggested that the user interface design could be improved and history tracking should be provided to decrease mental demand for teachers during long-term use: \textit{``more specific instructions could be incorporated into the page to clearly indicate the available actions and the color scheme could be improved. Additionally, I would like to see the capability to synchronize generated records through user login.''}
% }

% \fhx{
% Based on recordings and experts' comments, error responses to input queries may increase user effort. 
% E8 input ``What activities can be designed around 'Stepping Stones' [the title of one text]'', \name{} responded ``Students can observe stone bridges and steps in their daily lives, explore their design principles and practical uses. They can participate in group projects to build stone bridge models using materials like rocks, experiencing the joy of collaboration.''
% E8 commented, ``This is indeed related to the article, but only the content, but not the core idea or intended message.''
% Finally, E8 did not incorporate these outputs into the final outcome; instead, he edited manually.
% Additionally, more frustration may be brought due to hallucination.
% I3  asked \name{} to provide some specific sentences related to the context of "poetic life" and then input 
% ``Give me more sentences and analysis related to 'Osmanthus Rain' [the title of one text] and the context''.
% LLM responded 
% ``'What I like is osmanthus. The osmanthus tree looks clumsy, unlike the plum tree, which has a graceful posture.' This contrast highlights the author's unique affection for osmanthus and reflects the author's ability to discover poetic elements in life through a lens of beauty.''
% I3 remarked ``it did give me one more sentence, but not really fit'' 
% It indicate LLMs can generate information that appears believable but is actually incorrect, instead of simply admitting it don't know the answer.
% This issue arises when users repeatedly ask LLMs similar questions.
% }

% In summary, our system provides support to teachers in constructing contexts, indicating that the system is user-friendly.

\subsubsection{RQ3: Perception of \name{}}
\penguin{
Both users in the lab session, who learned \name{} through verbal instructions from the developer, and users in the wild, who studied it via a video tutorial, agreed that our system is easy to learn.
}
With the exception of E2, all other teachers affirmed that \name{} supported their exploration of contexts and the interrelationships between texts and contexts. \textit{Expanding thinking} and \textit{inspiring creativity} were mentioned as advantages for supporting context exploration.
\textit{``The system provides a broader range of relationships between context and text, gradually expanding the context. I believe AI should work in this manner, incrementally broadening the scope to help me explore more possibilities, rather than rigidly offering a single answer''} (E11).
Furthermore, E5 added: \textit{``It can provide inspiration, demonstrating how to develop the class based on this context.''}
Besides expanding the breadth of thinking, E10 expressed appreciation for \name{}'s summarization capabilities: \textit{``after selecting a substantial number of texts, I was astonished that it could truly organize them and produce a comprehensive design, which is impossible in my typical lesson preparation.''} These comments indicate that \name{} facilitates users in opening and focusing their cognitive processes during interdisciplinary exploration.
% However, 
% E2 found the recommended contexts to be unreasonable:
% \textit{``The content generated is not suitable; some contexts provided are inappropriate... The choice of words in the analysis is also problematic, with certain terms not conforming to the standards for elementary school students''} (E2). 
% \fhx{
% Moreover, some queries
% }
% The generation of unreasonable content has constrained the system's usability, consequently diminishing user support.
\penguin{
However, E1 and E3 raised an issue about the repetition of suggested contexts when they tried different reading materials in their three-day usage of \name{}. 
E1 noted, \textit{``I found that the same contexts reappearing despite my selection of entirely different reading materials''.} 
% two of five long-term users noticed some challenges with \name{}'s reusability due to the lack of original contexts available in the database.
% E1 noted, \textit{``I found a high frequency of context repetition, with the same contexts reappearing despite my selection of entirely different reading materials on multiple times''.} 
}
% the insufficiency of original contexts impacted the exploration experience.
Despite receiving feedback and modifications implemented after the prototype evaluation, this issue still troubled users and will be further discussed in~\ref{sec:discussion}.

In comparison to other generative tools and web search engines, users have reported a high level of trust in the output of our system, despite occasional acceptable errors. I3 stated, \textit{``I feel it is more closely aligned with reading materials compared to previous tools, although a few analyses of the context are not very accurate. Overall, I still trust this system.''} E4, I4 emphasized their reliance on personal experience and subjective judgment during the exploration process: \textit{``I always trust my own design more. When my ideas are limited, I use this system to evaluate its outputs and determine which content is usable''} (E4).
% I4 expressed a similar opinion: \textit{``I trust this system, but I cannot assert that I completely accept all its outputs. I will consider the actual situation before adopting them.''}

In summary, users perceive \name{} as highly usable and effective, and they generally express trust in its outputs.