\pzh{
\section{Design of \name{}}
\label{sec:system_design}
Based on the identified design principles, we introduce \name{}, a human-AI collaborative system to support elementary school literature teachers in ideating interdisciplinary contexts.
\name{} not only facilitates teachers' ability to explore various contexts and reading materials but also delivers structured outputs derived from this exploration process and the teachers' thought processes. 
Inspired by prior research~\cite{zheng2024disciplink} and tailored to our interdisciplinary context exploration scenario, the primary interface of \name{} consists of three parts: Contexts Exploration View, Texts Exploration View, and Collection View. %, all integrated on a single page to better support interdisciplinary exploration.
% The interactions within the system have been carefully crafted based on design principles established through an iterative design process. 
After confirming the subjects and importing the reading materials, the system aligns with the user's habitual thinking process by selecting or specifying potentially related interdisciplinary contexts within Context Exploration View \penguin{(DP1)}. % (DP1.1). 
Based on the selected context, the system enables multi-text-driven exploration (DP2). 
\penguin{Specifically,} \name{} extracts connections between the context and literacy elements, displaying specific details in either Context Exploration View or Texts Exploration View, decided upon whether they are context-oriented or text-oriented. % (DP1.2).  %the nature of these connections (
In Texts Exploration View, users can conduct in-depth analyses of individual texts and perform pairwise comparisons of any selected texts. % (DP2). % easily
% Throughout both the Context Exploration View and Texts Exploration View, 
\peng{In these two exploration processes}, users have the opportunity to chat with the LLM at any time and modify the analyses provided by the system \penguin{(DP4)}. % (DP2). 
Users can add analyzed themes and corresponding texts into the collection, allowing the system to generate structured ideation outcomes based on their selections (DP3). 
In \name{}, all generated content relies on a context database, \peng{prompt} templates, and generation guidelines (DP3, DP4). 
In the following subsections, we will describe how users can interact with \name{} and the detailed design and implementation of \name{}.
}

\begin{figure*}[ht]
  \centering
  \includegraphics[width=\linewidth]{figures/system.pdf}
  \caption{The user interface of \name{}, translated from Chinese either by Google Translator or manually.}
  \Description{The user interface of \name{}, translated from Chinese either by Google Translator or manually.}
\label{fig:system_interface}
\end{figure*}

\subsection{\name{} Interface and Interaction Design}
% Before initiating 
\pzh{To initiate}
the exploration process, users are required to select the interdisciplinary subjects and reading materials they wish to explore \pzh{(\autoref{fig:system_interface} A)}.
By clicking the ``Collapse Config Panel'' button (~\autoref{fig:system_interface} A turned into ``Open config panel'' at the top of the page after clicking on), users confirm their settings, which subsequently expands Contexts Exploration View. Upon clicking the ``Generate Recommended Contexts'' \pzh{button, \name{}} displays a series of context cards (B), each containing the context title, the titles of the most relevant reading materials, the original context from Context Pool (see~\autoref{sbsbsc:contextpool}), and a detailed description of the context.
Users may click the ``Find'' button on any context card of interest to further explore and edit the description of the context, allowing them to ask the LLM any questions  \peng{(D)} related to the theme and its connections (\eg relate this context to additional musical works?) to aid in understanding and exploration. The description becomes editable, enabling users to reconstruct it based on the LLM's responses. Users can also click ``Star'' to favorite a context and click the ``Right Arrow'' to specify the theme associated with texts in Texts Exploration View  \pzh{(B)}. 
If users are dissatisfied with a context, they have the option to click the ``Delete'' button to remove it. Users can click ``Manually Add Context'' to provide the title and background of a context in a pop-up window \pzh{(C)}, allowing \name{} to tailor a suitable context based on the information provided.

The middle part of the page is designated as Texts Exploration View (\autoref{fig:system_interface}), which facilitates a comprehensive examination of the interrelationships among various reading materials and specified contexts. 
Texts that are most relevant to the selected context are displayed at the top in Texts Exploration View, with each reading material accompanied by the analysis.
This analysis encompasses the relationships among sentences, paragraphs, and the context (\ie in-depth single-text analysis). 
To further support user exploration, inspired by~\cite{wu2024transagents} and the workflow of the teachers, additional ratings and critical recommendations are provided for LLM-generated analysis.
Similar to Contexts View, users have the option to click the ``Find'' button on any card (E) to ask the LLM any questions about the contexts related to the reading material (\eg Can you elaborate on the relationship between the descriptions of scenery in the text and the context?), and users can also modify the analysis. 
Users can click ``Star'' to favorite a reading material under the relevant context or click ``Delete'' if they no longer consider the text. If a user believes a reading material fits the context but was not recommended, they have the option to manually click ``Add Text'' to request the system to analyze it based on the context.
To optimize generation speed, the system generates a limited number of contexts and analyzes a limited number of reading materials at a time (set as 8 during experiments). Users can click ``More Contexts'' and ``More Contexts'' to generate additional contexts and analyze more texts for further exploration.

When clicking a context within Collection View  \pzh{(\autoref{fig:system_interface})}, users are directed to the Outcome Generation Panel \pzh{(\autoref{fig:system_interface2})}.
In this panel, users can examine the details of their chosen context as well as the analyses of the associated reading materials. 
Subsequently, users are prompted to enter the ``Expected number of lessons'' (\autoref{fig:system_interface} G) and select the ``Generate Introduction and Course Plan'' (\autoref{fig:system_interface2}) to create a comprehensive course plan that includes explanations for each segment and an introduction aligned with the course plan. 
In this interface, all content can be edited by double-clicking to adjust the final outcomes. Clicking ``Generate the Activities for Classroom'' recommends teaching activities related to the literature and the interdisciplinary subjects. Users can click on the titles to delete unnecessary activities.
At this point, a complete outcome based on a specific context has been generated.
Clicking the download button at the bottom right of the panel allows users to download the content in txt and HTML formats for sharing or further editing.

\begin{figure*}[]
  \centering
  \includegraphics[width=\linewidth]{figures/system2.pdf}
  \caption{The user interface of \name{}'s Outcome Generation Panel, translated from Chinese either by Google Translator or manually. The content in this figure serves as an example of outcomes.}
  \Description{The user interface of \name{}'s Outcome Generation Panel, translated from Chinese either by Google Translator or manually. The content in this figure serves as an example of outcomes.}
\label{fig:system_interface2}
\end{figure*}

\subsection{Implementation of the System}
\fanhx{
Inspired by~\cite{hou2024c2ideas, wu2024transagents}, \name{} aligns its behavior with teachers' habitual practices in interdisciplinary ideation (DP1).
% we utilize LLMs to construct a workflow that simulates team roles for ideating interdisciplinary contexts.
We designed an LLM-based multi-agent system to simulate human collaboration.
Following the multi-agent collaborative framework by~\citet{hong2024metagpt}, we decompose the task into multiple steps, allowing agents to use tools when necessary and interact with the user for feedback.
Also, the prompt design follows the Co-Star framework~\cite{napoli2024leveraging} to optimize outputs.
}
The architecture of our workflow is illustrated in~\autoref{fig:implement}, which will be discussed in~\autoref{sbsbsc:architecture}.
The effectiveness of the content generated through this architecture relies on a comprehensive Context Pool (DP4) and carefully designed prompts (DP2, DP3), which will be introduced in~\autoref{sbsbsc:contextpool} and~\autoref{sbsbsc:Iterativeprompt}, respectively.

\begin{figure*}[]
  \centering
  \includegraphics[width=0.90\linewidth]{figures/impl.png}
  \caption{\fhx{The architecture of \name{}. The illustration of the main workflow of \name{}. The responses to user's queries with gray background are not shown in this figure.}}
  \Description{The architecture of \name{}. The illustration of the main workflow of \name{}. The responses to user's queries with gray background are not shown in this figure.}
\label{fig:implement}
\end{figure*}

\subsubsection{Architecture of \name{}}
\label{sbsbsc:architecture}
\penguin{In line with the teachers' habitual practices and DPs,} 
we simulate four roles \fanhx{(DP1)} that participate in the ideation of interdisciplinary contexts: Context Analyst \penguin{(DP2)}, Text Analyst \penguin{(DP2)}, Text Reviewer \penguin{(DP2, DP4)}, and Context Summarizer \penguin{(DP3)}.
To facilitate their collaboration, similar to MetaGPT~\cite{hong2024metagpt}, we enable these agents to remember, think (\ie determine subsequent actions based on observations), act (\ie utilize tools and communicate with one another), and, crucially, observe user behavior through the interface to respond appropriately.
Each role is equipped with the GLM4 model as its ``brain.''
Before the application is implemented, all original contexts and reading materials are embedded using the Zhipu embedding-3 model\footnote{\url{https://bigmodel.cn/dev/api/vector/embedding-3}}. The texts and embedding arrays are stored in the database.
For contexts imported by users, \name{} will complete this embedding process in real-time.

\textbf{Context Analyst} acts as a ``teacher with expertise across various disciplines.'' When a user clicks ``Generate Recommended Contexts,'' it retrieves background information of the original contexts that are most relevant to the reading materials from the database
% , utilizing a "vote-based RAG"
. For each context, Context Analyst conducts an analysis based on all associated reading materials and generates a detailed description of the context. The Context Analyst's memory continuously retains the texts and analyzes contexts, similar to the role of a teacher. When a user initiates a request in ``Find'' mode, Context Analyst is tasked with providing detailed information related to the context.

\textbf{Text Analyst} acts as an ``active-minded literature teacher.'' When a user requests an analysis of texts based on the selected context, Context Analyst provides Text Analyst with the details of the selected context. Initially, Text Analyst
% employs a "vote-based RAG" algorithm to 
retrieves the most relevant texts. Subsequently, Text Analyst conducts a thorough analysis of each text individually, producing an in-depth single-text analysis. Furthermore, Text Analyst responds to users' inquiries related to the text.

\textbf{Text Reviewer} acts as a ``conservative and experienced literature teacher.'' When Text Analyst generates an analysis, it seeks Text Reviewer's opinion, which entails Text Reviewer assessing the relevance and accuracy of the analysis based on the reading material and the selected context, providing evaluative information to the user. This additional review process provides users with critical information and corrects potential inaccuracies in the RAG.

\textbf{Context Summarizer} acts as a ``literature teacher skilled in summarization,'' responsible for integrating the context with synthesizing contextual information with selected reading materials from the collection. It delivers structured outputs step-by-step in the Outcome Generation Panel (\ie a detailed lesson plan with explanations for each segment, the introduction based on the lesson plan, and the associated activities).

In summary, various user actions within the interface trigger the actions of different roles. Upon reflection, these roles employ a range of tools and capabilities, which depend on carefully designed prompts, to accomplish their respective tasks.

\subsubsection{Context Pool}
\label{sbsbsc:contextpool}
To enhance the generation of reasonable and evidence-based contexts, we compiled a database of original contexts accompanied by background information across various subjects (including general contexts). 

The Informal Education Database contains 113 original contexts derived from a famous Chinese journal Yao Wen Jiao Zi's ``Annual Popular Words (2008-2023) \footnote{\url{http://yaowenjiaozi.cn/}}.''
These materials claim to be based on data from the National Language Resources Corpus, with additional content selected manually. 
One author read the 160 original contexts and included those comprehensible to elementary school students in the database. 
The context database for the subjects of art, science, mathematics, and music is extracted from the unit titles and descriptions of widely used textbooks for grades 3-6, including 144 original texts.
These contexts were selected to fulfill teachers' requirements for accuracy and diversity, providing a positive experience during the user study. Notably, teachers can easily add contexts in a batch; they only need to provide the subject, title, and background information, similar to the operations performed on the interface.

\subsubsection{Iterative Prompt Engineering}
\label{sbsbsc:Iterativeprompt}
\name{} aims to provide users with high-quality, structured outcomes based on specific evaluation criteria. We have incorporated valuable insights from experts gathered during the foundational study into the prompt design. The prompts are structured in accordance with the co-star~\cite{napoli2024leveraging} framework, which encompasses context, objective, style, tone, audience, and response in a single prompt. The style, tone, and audience specifications for each role are shown in the supplementary materials.
The design details of the context, objective, and response are shown below.

The \textbf{context part} of the prompt not only explains the task (\eg find multiple points related to the context across various texts and generate a highly specific analysis based on the context) but also provides all the requisite knowledge for the task. Although the GLM-4 supports a 128k context length, overly long prompts may lead the LLM to neglect certain information. Therefore, we precisely designed the minimum information necessary for each role's actions. For instance, Text Analyst is provided solely with the content of the text to be analyzed and the relevant context for each request, and this procedure is repeated for multiple texts to ensure the most precise analysis for each individual text. Meanwhile, Context Summarizer concentrates on generating relevant activities in accordance with the course plan, incorporating only the course plan and introduction into the prompt, without the content of the reading materials.

The \textbf{objectives part} in all prompts are consistently aligned with the six metrics established by the team (DP3) for evaluating the quality of contexts and outcomes. These clearly defined objective statements ensure that the outputs are in close accordance with the expectations of teachers. The response, which specifies the required output format and content, is embedded with templates developed by experts during the foundational study. Following the evaluation of the prototype, we provided feedback to experts E1-E6 and revised the templates to enhance their content richness. The template for Context Summarizer for generating a course plan is as follows:


\begin{quote}
Segment 1: Initial Encounter with the Scene · Uncontrollable Emotions
\begin{itemize}
    \item ``Prairie'' [the title of the text] (The natural scenery and ethnic friendship in the prairie)
    \begin{itemize}
        \item Lesson 1: Appreciate the beautiful sentences in the text and feel the beauty of the prairie.
        \item Lesson 2: Feel the enthusiasm of the prairie people and the deep friendship between Mongolian and Han people.
    \end{itemize}
\end{itemize}

Segment 2: Encounter with the Scene Again · Touching the Heartstrings
\begin{itemize}
    \item ``Lilac Knot'' [the title of the text] (The feelings in the lilac knot)
    \begin{itemize}
        \item Lesson 3: Understand the author's way of expressing associations triggered by objects and explain your understanding.
        \item Lesson 4: Understand the symbolic meaning of the lilac knot and appreciate the emotions embedded by the author.
    \end{itemize}
    \item ``Lodging by the River'' [the title of the text] + ``Visiting an Old Friend'' [the title of the text] (The feelings in the mountains and rivers under the moonlight)
    \begin{itemize}
        \item Lesson 5: Use ``Lodging by the River'' as an example to teach the method of learning ancient poetry and explore the imagery of the ``moon.''
        \item Lesson 6: Imagine the scenes in ``Lodging by the River'' and ``Visiting an Old Friend'' and understand the unique poetic feelings evoked by the ``mountains and rivers.''
    \end{itemize}
\end{itemize}

Segment 3: Another Encounter with the Scene · Endless Thoughts
\begin{itemize}
    \item ``Song of Flowers'' [the title of the text] (Unity of man and nature, understanding philosophy)
    \begin{itemize}
        \item Lesson 7: Feel the attitude towards life and inner ideals after transforming into a ``flower,'' and appreciate the author's positive attitude towards life.
    \end{itemize}
\end{itemize}
\end{quote}

The template includes all the elements expected by teachers during the foundational study. The LLM can utilize the template to verify that the response includes all necessary content.
