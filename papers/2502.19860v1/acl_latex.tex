% This must be in the first 5 lines to tell arXiv to use pdfLaTeX, which is strongly recommended.
% \pdfoutput=1
% In particular, the hyperref package requires pdfLaTeX in order to break URLs across lines.

\documentclass[11pt]{article}
% \usepackage{import}
% Change "review" to "final" to generate the final (sometimes called camera-ready) version.
% Change to "preprint" to generate a non-anonymous version with page numbers.

\usepackage[preprint]{acl}

% Standard package includes
\usepackage{times}
\usepackage{latexsym}
% \usepackage[UTF-8]{ctex}
% For proper rendering and hyphenation of words containing Latin characters (including in bib files)
\usepackage[T1]{fontenc}
% For Vietnamese characters
% \usepackage[T5]{fontenc}
% See https://www.latex-project.org/help/documentation/encguide.pdf for other character sets

% This assumes your files are encoded as UTF8
\usepackage[utf8]{inputenc}

% This is not strictly necessary, and may be commented out,
% but it will improve the layout of the manuscript,
% and will typically save some space.
\usepackage{microtype}

% This is also not strictly necessary, and may be commented out.
% However, it will improve the aesthetics of text in
% the typewriter font.
\usepackage{inconsolata}

%Including images in your LaTeX document requires adding
%additional package(s)
\usepackage{graphicx}
\usepackage{enumitem}
\usepackage{booktabs}
\usepackage{multirow}
\usepackage{amsmath}

\usepackage{algorithm}
\usepackage{algpseudocode}
% If the title and author information does not fit in the area allocated, uncomment the following
%
%\setlength\titlebox{<dim>}
%
% and set <dim> to something 5cm or larger.

\newcommand{\MIND}{\textsc{Mind}}

\title{\MIND: Towards Immersive Psychological Healing with \\Multi-agent Inner Dialogue}

% Author information can be set in various styles:
% For several authors from the same institution:
% \author{Author 1 \and ... \and Author n \\
%         Address line \\ ... \\ Address line}
% if the names do not fit well on one line use
%         Author 1 \\ {\bf Author 2} \\ ... \\ {\bf Author n} \\
% For authors from different institutions:
% \author{Author 1 \\ Address line \\  ... \\ Address line
%         \And  ... \And
%         Author n \\ Address line \\ ... \\ Address line}
% To start a separate ``row'' of authors use \AND, as in
% \author{Author 1 \\ Address line \\  ... \\ Address line
%         \AND
%         Author 2 \\ Address line \\ ... \\ Address line \And
%         Author 3 \\ Address line \\ ... \\ Address line}

\author{
 \textbf{Yujia Chen\textsuperscript{1}},
 \textbf{Changsong Li\textsuperscript{1}$^\dagger$},
 \textbf{Yiming Wang\textsuperscript{2}},
 \textbf{Qingqing Xiao\textsuperscript{3,4}},
\\
 \textbf{Nan Zhang\textsuperscript{1}},
 \textbf{Zifan Kong\textsuperscript{1}},
 \textbf{Peng Wang\textsuperscript{1}},
 \textbf{Binyu Yan\textsuperscript{1}\thanks{Corresponding author.}}
\\
\\
 \textsuperscript{1}Sichuan University
 ~~\textsuperscript{2}Shanghai Jiao Tong University
\\ 
 \textsuperscript{3}Mental Health Center, West China Hospital, Sichuan University\\
 \textsuperscript{4}WestChina School of Nursing, Sichuan University
\\
   \texttt{\{yujiachen0058,changsongli0018,alsaceym\}@gmail.com} \quad \texttt{\{yanby\}@scu.edu.cn
 }
}
\begin{document}
\maketitle
\begin{abstract}
Mental health issues are worsening in today's competitive society, such as depression and anxiety. 
Traditional healings like counseling and chatbots fail to engage effectively, they often provide generic responses lacking emotional depth. Although large language models (LLMs) have the potential to create more human-like interactions, they still struggle to capture subtle emotions. This requires LLMs to be equipped with human-like adaptability and warmth.

To fill this gap, we propose the {\bf \MIND}~ (\textbf{M}ulti-agent \textbf{IN}ner \textbf{D}ialogue), a novel paradigm that provides more immersive psychological healing environments.
Considering the strong generative and role-playing ability of LLM agents, we predefine an interactive healing framework and assign LLM agents different roles within the framework to engage in interactive inner dialogues with users, thereby providing an immersive healing experience.
% Our key insight is that advanced LLMs can act both as role-playing agents and dynamic scenario generators. By orchestrating multi-agent interactions within carefully designed virtual scenarios, \MIND~ transforms conventional empathy training into immersive, interactive dialogue games tailored to individual needs.
We conduct extensive human experiments in various real-world healing dimensions, and find that \MIND ~provides a more user-friendly experience than traditional paradigms. This demonstrates that \MIND ~effectively leverages the significant potential of LLMs in psychological healing.
% We conducted two experiments to validate our approach: (1) A comparative evaluation where human evaluators assessed \MIND~ against psychologist counseling, chatbots, and traditional empathy training across various themes, showing significant improvements in immersion, satisfaction, emotional relief, engagement, and coherence. (2) An agent performance benchmark where seven LLMs were evaluated in simulating patient roles based on metrics like dialogue stability, language fluency, emotional expression, personalization, and accuracy, with the Gemini-2.0-Flash model outperforming others. Our results demonstrate \MIND~'s potential to advance psychological interventions by combining LLM scalability with human-centered interaction design, offering accessible and effective mental health support.

\end{abstract}

\section{Introduction}

\begin{figure}[t]
  \includegraphics[width=\columnwidth]{image/overview.pdf}
  \caption{Examples of our \MIND ~paradigm with multi-agent inner dialogue compared to the traditional empathy training healing method.}
  \vspace{-0.1in}
  \label{fig:overview}
\end{figure}


Mental health issues are worsening in today's competitive society, with rising cases of disorders like depression \citep{moitra2023global}. This lead to a growing market for psychological healing. Traditional healing paradigms like Cognitive Behavioral Therapy \citep{beck1979cognitive} and Dialectical Behavior Therapy \citep{lynch2007dialectical} are widely used but rely on face-to-face interactions, making them time-consuming and costly \citep{duruz2003recherche} that limits large-scale accessibility.
Another healing paradigm is VR-based Empathy Training \citep{halim2023individualized,hidding2024single,dollinger2024virtual}, which is rooted in the mechanisms of virtual body ownership illusion. This method involves the creation of a virtual self-image within VR, where the individual provides verbal encouragement to this virtual self. Subsequently, the individual swaps perspectives to adopt the role of the one being comforted, receiving the previously offered words of comfort. This ``self-dialogue'' process enables patients to experience and understand empathy from a new vantage point, thereby promoting self-compassion and reducing self-criticism. However, current systems are limited by static scenarios and scripted interactions. The lack of counselor-guided mechanisms and dynamic responsiveness in these predefined frameworks hampers the effectiveness of emotional regulation and reduces the adaptability of the therapeutic process, potentially leading to emotional deterioration in self-administered interventions.


\begin{figure*}[t]
  \centering
  \includegraphics[width=\textwidth]{image/framework.pdf}
  \caption{Overview of our \MIND ~paradigm: Trigger, Devil, Guide and Strategist interact with Player.}
    \label{fig:framework}
\end{figure*}

Recently, large language models (LLMs) have quickly advanced \citep{minaee2024largelanguagemodelssurvey,zhao2024surveylargelanguagemodels}, gaining strong abilities in generation \citep{X2025Learning}, reasoning \citep{huang-chang-2023-towards}, and role-playing \citep{wang-etal-2024-rolellm}. They also show great promise in mental health support \citep{hu2024psycollm,obradovich2024opportunities,bhatia2022cognitive}, offering new opportunities for psychological healing.
% With the advent of Large Language Models (LLMs), the field of psychological healing has undergone a new transformation. LLMs, with their vast data reserves and advanced natural language processing capabilities, show great potential for providing scalable and easily accessible mental health support \citep{hu2024psycollm,obradovich2024opportunities,bhatia2022cognitive}. 
% For example, the "AI Mood Town" platform \citep{park2023generative} utilizes LLMs to provide users with virtual listening partners capable of offering empathetic responses and emotional guidance. These systems can operate 24/7, making mental health support more accessible, especially for individuals who may not have the opportunity or means to engage in traditional therapy.
%For example, the ``AI Mood Town'' platform \citep{park2023generative} uses LLMs to provide users with round-the-clock virtual listeners that offer empathetic responses and emotional guidance. This makes mental health support more accessible for those who cannot access traditional therapy.
Despite these advancements, LLMs still face numerous challenges in the field of psychological healing. One major issue is {\it the lack of human empathy and the inability to form genuine therapeutic alliances}, which are crucial for effective treatment \citep{iftikhar2024therapy,guo2024large,obradovich2024opportunities,volkmer2024large}. LLMs often generate overly generic responses, failing to capture the subtle emotional nuances of patients \citep{sanu2024limitations}. These limitations highlight the need for a more sophisticated approach that blends LLMs' strengths with the warmth and flexibility of human interaction.

The emergence of multi-agent technology \citep{guo2024largelanguagemodelbased} offers potential solutions to these challenges. Multi-agent systems comprise specialized agents that collaborate and adapt to individual needs, ensuring a more immersive, interactive, dynamic healing experience \citep{guo2024multi,rocha2023applying}.
Each agent can focus on different aspects of psychological support, including emotional regulation, cognitive restructuring, and social interaction.
% For example, the AutoCBT framework \citep{xu2025autocbt} employs multiple agents to dynamically allocate tasks during cognitive behavioral therapy, offering personalized support. This approach not only enhances the system's adaptability but also makes interactions with users more detailed and empathetic.
By utilizing the collective intelligence of multiple agents, they can provide a more comprehensive and effective experience.



Based on the above motivations, we propose {\bf M}ulti-agent {\bf IN}ner {\bf D}ialogue ({\bf \MIND}), a novel immersive and interactive psychological healing paradigm.
As illustrated in Figure \ref{fig:overview}, our approach is the first to introduce a multi-agent system into an empathy training paradigm, significantly enhancing the interaction between the patient and their inner self through dynamic narrative scenarios.
We first decompose the patient's conflicting self into interactive conversational entities, creating an embodied space for inner dialogue, and introducing a dynamic scene generation mechanism that adapts the narrative path based on the user's emotional state.
Next, we allocate four core roles for LLM agents, including trigger, devil, guide, and strategist. Through multi-perspective dialogue history review, it provides the user with a cognitive scaffold for metacognitive reflection. This design allows the patient to not only experience empathy transformation from a dual perspective but also, through the agent-mediated dialogue process, systematically deconstruct the cognitive roots of emotional generation, leading to deeper self-reconciliation.
% Our \MIND ~paradigm consists of four core LLM agents: (1) {\it Trigger Agent}, creating emotionally inducing virtual scenarios; (2) {\it Devil Agent}, the reveal of the user's internal conflicts; (3) {\it Guide Agent}, providing cognitive restructuring and emotional regulation strategies, guiding the user to comfort the Devil; (4) {\it Strategist Agent}, managing the narrative progression and the devil's cognitive changes.

% Compared to the traditional empathy training paradigm, our paradigm achieves three key innovations: 
% \begin{itemize}[leftmargin=0.3cm]
%     \item We decompose the patient's conflicting self into interactive conversational entities, creating an embodied space for inner dialogue.
%     \item We introduce a dynamic scene generation mechanism that adapts the narrative path based on the user's emotional state.
%     \item Through multi-perspective dialogue history review, it provides the user with a cognitive scaffold for metacognitive reflection. This design allows the patient to not only experience empathy transformation from a dual perspective but also, through the agent-mediated dialogue process, systematically deconstruct the cognitive roots of emotional generation, leading to deeper self-reconciliation.
% \end{itemize}

We conduct extensive experiments to validate our approach. Initially, we benchmark the performance of seven large language models (LLMs) in simulating patient roles. The main experiment involved a comparative evaluation where human evaluators assessed \MIND~ against traditional psychologist counseling, chatbots, and conventional empathy training methods. The results demonstrate that \MIND~ has the potential to advance psychological interventions by combining the scalability of LLMs with human-centered interaction design, thereby offering accessible and effective mental health support. 
Additional experiments based on different themes and ablation experiments based on agents further underscored the stability in generated scenarios and content and the rationality of the overall framework.

% Specifically, our contributions are as follows:

% \begin{itemize}
% \item[$\bullet$] We propose a novel psychological game therapy paradigm that optimizes traditional therapeutic methods with the help of LLM agents. This paradigm enhances participant engagement and offers an innovative perspective on psychological healing methods in the AI era. 

% \item[$\bullet$] We evaluate the authenticity of LLMs acting as SPs (simulated patients) and their accuracy in identifying cognitive distortions, with the Gemini-2.0-Flash model outperforming others.

% \item[$\bullet$] We have developed a framework based on LLM agents for psychological healing in various scenarios, offering personalization and immersion. In comparison to traditional therapeutic methods available in the market, experiments show that our framework demonstrates effectiveness and superiority in user experience.
% \end{itemize}







\section{\MIND: Multi-agent Inner Dialogue}

\subsection{Overall Framework}

The overall framework of our \MIND ~paradigm is shown in Figure \ref{fig:framework}, composed of four agents responsible for inner dialogue generation, in addition to an agent simulating patients with cognitive distortions. The subsequent section will commence with an overview of the workflow: the trigger, the devil, the guide, the strategist and the human simulated patient. Detailed prompt templates used by each agent are presented in Appendix \ref{sec:MIND_prompt}.


In the initial phase of the psychological interaction framework, the player articulates their recent concerns and selects a theme to guide the narrative direction. These inputs serve as the foundational trigger for dynamically generating virtual scenarios ($S_i$), which are designed to mirror the player's cognitive patterns. Subsequently, the devil processes the scenario ($S_i$) and concerns to distorted thoughts ($D_i$), emulating maladaptive cognitive biases aligned with the player's mental state.

The guide then integrates $S_i$ and $D_i$ to generate professional psychological guidance ($G_i$), aimed at facilitating empathetic responses from the player. Upon receiving $G_i$, the player engages in a reflective dialogue to provide comforting words ($C_i$) and counter the distorted thoughts ($D_i$), thereby advancing the therapeutic narrative.



To ensure narrative coherence and progression, the strategist analyzes the cumulative memory ($M_{i-1}$) --- a structured summary of prior scenarios, cognitive distortions, and guidance --- alongside the player's comforting words ($C_i$). This analysis produces strategic directives ($P_i$) that govern the generation of subsequent triggers ($S_{i+1}$) and the devil's adaptive cognitive evolution ($D_{i+1}$).

Through iterative cycles of scenario generation, cognitive reflection, and guided intervention, the framework progressively refines its alignment with the player's psychological profile. Each iteration enhances the system's capacity to model nuanced mental states while maintaining narrative continuity, ultimately achieving a balance between therapeutic efficacy and immersive storytelling.



\subsection{Trigger: Scenario Generation}

The trigger generates artificial scenes within the interactive fiction game, drawing from the chosen theme and the player's concerns. It begins by creating an initial scene that reflects the player's psychological state and evolves the narrative based on previous interactions. The agent adapts the storyline according to the player's emotional context and worries, ensuring a coherent progression in the scene's development. Through this process, the trigger sets the stage for therapeutic reflection by crafting a dynamic and consistent narrative that mirrors the player's thoughts and psychological growth.

Let the first-round trigger agent be $\pi_{t_{0}}$ and non-first rounds trigger agent be $\pi_{t_{i}}$, the process can be formulated as:
\begin{equation}
    \begin{aligned}
        &S_{0}=\pi_{t_{0}}(W,T), \\
        &S_{i}=\pi_{t_{i}}(C_{i-1},P_{i-1};W,T) ~(i>0),
        \label{eq:trigger}
    \end{aligned}
\end{equation}
where $W$ is the player's concerns and $T$ is the theme, $C_{i-1}$ is the player's last-round comforting words and $P_{i-1}$ is the strategist's last-round storyline progression.

We adopt the chain-of-thought prompting technique \citep{wei2022chain} to enhance the quality of the trigger in scenario generation. Specifically, the trigger is instructed to generate a simulation scene based on the theme and the patient's concerns, while also explaining how to incorporate the scene history and the patient's thought processes to create a logical extension. %Additionally, the trigger is required to provide detailed justifications for why the generated simulation scene effectively captures and reproduces the patient's troubles.

\subsection{Devil: Cognitive Distortion Simulation}

The devil simulates the cognitive distortions that a patient might experience within the context of the scenario. It functions as the player's ``virtual embodiment'' representing an ``alternate self'' within the simulated environment.

Based on the simulated scenario provided by the trigger, the devil produces thoughts that align with common cognitive distortions, such as catastrophizing or emotional reasoning. These distortions are personalized to the player's specific context, offering an authentic simulation of how negative thinking can influence behavior and perceptions.

Let the first-round devil agent be $\pi_{d_{0}}$ and non-first rounds devil agent be $\pi_{d_{i}}$, the process can be formulated as:

\begin{equation}
    \begin{aligned}
        &D_{0}=\pi_{d_{0}}(W,S_{0}),\\
        &D_{i}=\pi_{d_{i}}(C_{i-1},P_{i-1},S_{i}) ~(i>0),
        \label{eq:devil}
    \end{aligned}
\end{equation}

To refine the simulation of the player's psychological state, we incorporate descriptions and definitions of five personality traits into the prompt design, aiming to create a more precise and personalized cognitive model. In the initial iteration, the devil agent generates responses solely based on the player's initial input and the scenario created by the trigger. However, in each subsequent iteration, the devil reacts to the player's comforting words, gradually weakening its cognitive distortions over time. This dynamic adjustment optimizes the player's interactive experience by allowing the devil's responses to evolve in alignment with the player's engagement and cognitive restructuring efforts.

\subsection{Guide: Cognitive Restructuring Guidance}

The guide aims to assist the player in recognizing, challenging, and reframing negative thought patterns through cognitive restructuring. The process begins with the guide identifying cognitive distortions in the player's thinking, which may have been amplified by the devil. The guide then offers alternative perspectives to counter these irrational beliefs and provides practical suggestions, such as taking a deep breath or writing down worries to evaluate their validity. The guide's goal is not to enforce immediate change, but to support gradual shifts in thinking, ensuring that each new perspective is integrated at the player's own pace.

Denote the guide agent as $\pi_{g}$. The process can be formulated as:

\begin{equation}
  \label{eq:guide}
  (G_{i},M_{i})=\pi_{g}(S_{i},D_{i})
\end{equation}

As the game progresses, the growing history becomes burdensome for the LLM to process efficiently. To mitigate this issue, a summarization mechanism is employed to maintain coherent narrative memory \citep{zhou2023recurrentgpt}. By operating in this way, the guide ensures that the player is not only challenged but also supported in a structured, manageable way, encouraging long-term emotional resilience and rational thinking. Ultimately, the guide helps transform the player from a passive recipient of distorted thoughts, as influenced by the devil, into an active participant in their own cognitive change, laying the foundation for healthier thought patterns and emotional well-being.

\begin{figure*}[ht]
  \centering
  \includegraphics[width=\textwidth]{image/paradigm_comparison.pdf}
  \caption{Comparison between three healing paradigms: Traditional counseling, traditional empathy training and our paradigm. \MIND~ transfers a traditional healing environment into an artificial interactive scenario where players show empathy to their ``internal-self''.}
  \label{fig:paradigmcomparison}
\end{figure*}

\subsection{Strategist: Storyline Progression}

The strategist is responsible for planning the next stage of the narrative and determining the mental shifts of the antagonist based on previous events and the comfort provided by the player. The primary goal of the strategist is to ensure that the protagonist's cognitive distortions are gradually restructured through the unfolding of the story.

Denote the strategist agent as $\pi_{s}$. The process can be formulated as:

\begin{equation}
  \label{eq:strategist}
  P_{i}=\pi_{s}(M_{i},C_{i})
\end{equation}
where $C_{i}$ is the player's this round comforting words.

In each iteration, the strategist carefully evaluates whether the devil's mindset has evolved. If the comforting words successfully address the devil's cognitive distortions, a shift in their thought process occurs, leading to a more balanced and realistic perspective on their circumstances. This change catalyzes the natural progression of the story, with the devil's actions and decisions reflecting a healthier mindset. Conversely, if no change takes place, the narrative remains consistent with the devil's previous emotional state, allowing the player's guidance to continue influencing their emotional transformation. The objective is to ensure that every story development is not only logically coherent but also aligns with the devil's cognitive journey toward self-awareness and emotional resilience.

\begin{algorithm}[h]
\caption{\MIND~ Paradigm}
\begin{algorithmic}[1]
\State \textbf{Input:} Player's concerns, theme
\State \textbf{Output:} The player reaches a reconciliation with their own concerns.

\State \textbf{Initialize:} 
\State \quad Memory $M_0 \gets \emptyset$, iteration counter $i \gets 0$
\State \quad Generate initial scenario $S_0$ and initial distortion thoughts $D_0$ based on Player's concerns and theme

\While{Player Engaged $\land$ $\neg$Therapeutic Goal Reached} \label{line:loop}
    \State \textbf{Step 1: Scenario Generation}
    \State \quad $S_i \gets \textsc{Scenario}(C_{i-1}, P_{i-1})$

    \State \textbf{Step 2: Distorted Thought Processing}
    \State \quad $D_i \gets \textsc{Distortions}(S_i, C_{i-1}, P_{i-1})$
    
    \State \textbf{Step 3: Psychological Guidance}
    \State \quad $G_i \gets \textsc{Guidance}(S_i, D_i)$
    
    \State \textbf{Step 4: Comforting Dialogue}
    \State \quad Present $S_i$, $D_i$, and $G_i$ to player
    \State \quad $C_i \gets \textsc{GetComfortingWords}()$
    
    \State \textbf{Step 5: Storyline Progression}
    \State \quad $P_i \gets \textsc{AnalyzeMemory}(M_{i-1}, C_i)$
    
    \State $i \gets i + 1$ \label{line:increment}
\EndWhile

\State \textbf{Output: }Enhanced therapeutic engagement and narrative continuity
\end{algorithmic}
\end{algorithm}



\subsection{Human Simulated Patient: Empathy and Interaction}

To facilitate the automated operation and evaluation of our framework, and drawing upon the validated psychological characteristics and annotation capabilities of LLM, we employ LLMs to simulate human interactions by providing comforting words to the devil. Based on the guidance from the guide, the virtual scenario generated by the trigger, and the cognitive distortions produced by the devil, human simulated patient assumes the role of the Player, engaging in empathetic reassurance toward the devil . This process also incorporates the chain-of-thought (CoT) technique\citep{wei2022chain}, allowing for a structured and coherent response generation that aligns with the psychological progression of the player-agent interaction.

\section{Experiments}

% In this section, the experiment is divided into three main parts. In Section \ref{sec:content evaluation}, we primarily discuss the comparison between this paradigm and other baseline approaches, propose evaluation metrics, and provide evidence. Next, in Section \ref{sec:sp role-playing assessment}, we will introduce the SP role-playing assessment to evaluate the ability of different baseline models in SP role-playing. In the following Section \ref{sec:ablations}, we compare different themes in \MIND~ and get the conclusion that this framework is more suitable for certain themes. Besides, we demonstrate the contribution of each agent to the overall framework and user experience through ablation experiments.

\begin{table*}[t]
\centering
\begin{tabular}{l|l}
\toprule
\textbf{Metric} & \textbf{Description} \\
\midrule
       Immersion & Measures whether the user feels fully engaged and captivated by the interaction. \\
       Coherence & Assesses if the generated content is logical and transitions smoothly. \\
      Engagement & Evaluates if the system encourages sustained and meaningful interaction. \\
Emotional Relief & Measures if the interaction reduces user stress or anxiety. \\
    Satisfaction & Reflects the user's overall contentment with the system. \\
        Interest & Assesses whether the content grabs attention and sparks curiosity. \\
\bottomrule
\end{tabular}
\caption{Our six evaluation dimensions and corresponding descriptions.}
\label{tab:content evaluation}
\end{table*}

\subsection{Setup}
\label{sec:content evaluation}

\paragraph{Scenario Setting.}
The real-life scenarios, thinking patterns, and cognitive distortion types of the Human Simulated Patient simulated by the LLM are derived from the C2D2 dataset \citep{wang2023c2d2}. This dataset is the first publicly available resource focused on cognitive distortion analysis, solving the problem of data scarcity in this field. The dataset covers eight major topics, including work issues, interpersonal issues, economic issues, random negative events, family issues, physical stress, and discrepancy between ideal and reality.


\paragraph{Baseline Paradigms.}
To evaluate the effectiveness of our \MIND ~paradigm, we compare it with traditional counseling methods(face-to-face dialogue and Q\&A) and the traditional empathy training paradigm \citep{halim2023individualized,hidding2024single,dollinger2024virtual}. Figure \ref{fig:paradigmcomparison} presents a comparison between these three paradigms, with the detailed implementation of baseline methods provided in Appendix \ref{sec:Baseline_methods}.

\paragraph{LLM Agents.}
We used several LLM agents including both open-source and closed-sourced models with varying parameter scales.
For closed-source models, we chose Gemini-2.0-flash \citep{gemini_url}, GPT-4o \citep{openai2024gpt4ocard}, GPT-3.5-Turbo \citep{ye2023comprehensivecapabilityanalysisgpt3}.
For open-source models, we chose Llama-3.1-8B-Instruct \citep{grattafiori2024llama3herdmodels}, Qwen2.5-72B-Instruct \citep{qwen2025qwen25technicalreport}, Qwen2.5-7B-Instruct \citep{qwen2025qwen25technicalreport} and Deepseek-R1 \citep{deepseekai2025deepseekr1incentivizingreasoningcapability}. To control variables, we set the temperature of each model to 0.7. 

In Section \ref{sec:sp role-playing assessment}, a preliminary role-playing experiment was conducted to evaluate the performance of various models in the SP role-playing task. Based on the results of this initial assessment, we selected the Gemini-2.0-flash model, which exhibited the best performance, for our main experiments. This model was chosen due to its superior ability to handle the complexities of the role-playing scenario, making it the most appropriate candidate for further investigation in the primary phase of our study.


\paragraph{Evaluation Metrics.}
The generation quality of the devil agent is crucial for the implementation of this framework, as it mirrors the player's internal ``cognitive distortions'' and can better serve its purpose only if it closely aligns with the player's ``inner voice.'' Therefore, before conducting our main experiments, we have specifically designed an Simulated Patient(SP) role-playing evaluation system to verify that the model can correctly identify the player's type of cognitive distortion and accurately reflect the patient's thoughts, achieving a realistic effect. We recruited 5 mental health professionals. Each evaluator had 10 rounds of conversation with each model. The evaluators rated the content based on the five evaluation metrics \citep{johri2025evaluation}, with a scoring range of 1 to 5. Detailed illustration is shown in Appendix \ref{sec:SP metrics}. 

The paradigm comparison
evaluation metrics we propose cover three main aspects: user experience, interaction quality, and emotional comfort, with six different metrics \citep{hua2024large,kumaran2023scenecraft,jennett2008measuring,ryan2015narrative,nacke2011towards}. The evaluation metrics is shown in Table \ref{tab:content evaluation}.
We recruited 7 mental health practitioners with professional expertise in psychological therapy. For the different paradigms, the evaluators rated the content based on the six evaluation metrics, with a scoring range of 1 to 5. 

\begin{table}
\begin{tabular}{p{3.5cm}p{0.3cm}p{0.3cm}p{0.3cm}p{0.3cm}p{0.3cm}}
    \toprule
    \textbf{Model Name} & \textbf{DS} & \textbf{LF} & \textbf{EE} & \textbf{PD} & \textbf{Acc} \\
    
    \midrule
    \multicolumn{6}{c}{\it Closed-Source Model} \\
    \midrule
    Gemini-2.0-flash & \textbf{\underline{4.8}} & 4.2 & \textbf{\underline{4.4}} & \textbf{\underline{4.6}} & 4.2 \\
    GPT-4o & \textbf{\underline{4.8}} & \textbf{\underline{4.4}} & 4.0 & 3.6 & \textbf{\underline{4.4}} \\
    GPT-3.5-Turbo & 4.2 & 4.2 & 3.6 & 3.4 & 3.4 \\
    
    \midrule
    \multicolumn{6}{c}{\it Open-Source Model} \\
    \midrule
    Qwen2.5-72B-Instruct & 3.2 & 2.8 & 3.0 & 2.6 & 3.0 \\
    Llama-3.1-8B-Instruct & 3.8 & 3.2 & 3.4 & 3.4 & 3.2 \\
    Qwen2.5-7B-Instruct & 3.2 & 2.8 & 3.0 & 2.8 & 3.0 \\
    Deepseek-R1 & 3.0 & 3.6 & 3.4 & 3.2 & 3.4 \\
    \bottomrule

\end{tabular}
\caption{SP role-playing results between different models. DS=Dialogue Stability, LF=Language Fluency, EE=Emotional Expression, PD=Personalization \& Diversity, Acc=Accuracy.}
\label{tab:spmodel}
\end{table}

\subsection{Prelinimary: SP Role-playing Evaluation}
\label{sec:sp role-playing assessment}

The results are shown in Table \ref{tab:spmodel} and the evaluation metrics is shown in Appendix \ref{sec:SP metrics}. The Gemini-2.0-flash model emerged as the best performer among the tested LLMs.  In comparison, GPT-4o performed well in some areas but lagged in Emotional Expression and Personalization. Models like GPT-3.5-Turbo, Llama-3.1-8B-Instruct, and Deepseek-R1 showed weaker results, particularly in emotional and personalized responses. Qwen2.5 models performed the poorest, especially in emotional expression and accuracy, with scores not surpassing 3.2 in any dimension. 


\begin{figure}
  \centering
  \includegraphics[width=\columnwidth]{image/Paradigm_result.png}
  \caption{Comparisons among various healing methods through human evaluations. It is evident that our paradigm surpasses other paradigms in all aspects.}
  \label{fig:paradigmresult}
  \vspace{-0.1in}
\end{figure}

\begin{table*}
\centering
\begin{tabular}{cccccccc}
\hline
\textbf{Theme} & \textbf{IM}& \textbf{CO}& \textbf{EN}& \textbf{ER}& \textbf{SA} &\textbf{IN}\\
\hline
Work issues& 4.14& \textbf{\underline{4.71}}& \textbf{\underline{4.14}}& 4.14& 4.14& \textbf{\underline{4.49}}\\
Random negative events& 3.57& 3.86& 4.00& \textbf{\underline{4.49}}& 3.71& 3.86\\
Interpersonal issues& 3.57& 3.86& 3.86& \textbf{\underline{4.49}}& 4.00& 4.14\\
Economic issues& 4.00& 4.57& 4.00& 4.14& 3.71& 4.14\\
Family issues& 4.14& 4.29& 4.00& 3.71& 4.29& 3.86\\
Physical stress& 3.71& 4.57& 4.00& \textbf{\underline{4.49}}& 4.00& 4.14\\
Discrepancy between ideal and reality& \textbf{\underline{4.29}}& 4.14& 4.00& 4.00& \textbf{\underline{4.49}}& 3.86\\
\hline
\end{tabular}
\caption{Content evaluation results between different themes. IM=Immersion, CO=Coherence, EN=Engagement, ER=Emotional Relief, SA=Satisfaction, IN=Interest. }
\label{tab:contenttheme}
\end{table*}

\subsection{Main Results}
The mean scores of each paradigm are shown in Figure \ref{fig:paradigmresult}. \MIND ~demonstrated significant strengths in all six core assessment dimensions. Quantitative analysis showed that our paradigm performed particularly well on the dimensions of interest and satisfaction, reaching a perfect score of 5, compared to all the baseline methods of traditional counseling, traditional empathy training, and chatbot. Notably, in terms of the engagement index, \MIND ~achieved an absolute improvement of 17.1\% over the suboptimal method of traditional counseling, which reflects the increased motivation of the caller users that \MIND ~can improve, so that they cooperate and participate in psychotherapy. On the dimensions of immersion, coherence and emotional relief, MIND also outperforms/equals the remaining three paradigms, which fully demonstrates that \textbf{\MIND ~has the potential to advance psychological interventions by combining the scalability of LLMs with human-centered interaction design.}

\section{Analysis}
\label{sec:ablations}

\subsection{Thematic Scenarios Ablation}

This framework is applicable to a variety of thematic scenarios, including but not limited to work, family, and interpersonal issues. To analyze the differences in effectiveness across different themes within this framework, we independently generated five examples for each of the seven themes in the C2D2 dataset. Similarly, we invited evaluators with psychological therapy expertise to score these examples. As shown in Table \ref{tab:contenttheme}, the performance of different themes varies under our framework. Most themes perform well in ``Immersion'' and ``Coherence,'' indicating that the system effectively engages users and maintains logical consistency.
Emotional relief and satisfaction are high, especially in themes like ``Work issues'' and ``Discrepancy between ideal and reality.''
However, ``Random negative events'' and ``Family issues'' score lower in certain dimensions, such as engagement and interest, and may require further optimization.

\subsection{Agent Involvement Ablation}
Our framework consists of four agents: trigger, devil, guide, and strategist. To evaluate the effectiveness of \MIND's two core agents (i.e., the guide and strategist) as well as the memorization mechanism, we conducted several ablation experiments to assess their impact on user experience and demonstrate the importance of each component. Specifically, we randomly generated three examples for each ablation experiment. We recruited 7 clinical psychology researchers with professional expertise to evaluate six content evaluation metrics, as outlined in Table \ref{tab:content evaluation}. The experimental results, presented in Figure \ref{fig:agentablation}, show that {\bf each agent significantly contributes to the overall framework}. The removal of any agent or the memorization mechanism notably diminishes the quality of the generated content, underscoring the collective importance of all agents in the framework.

\begin{figure}[ht]
  \centering
  \includegraphics[width=0.77\columnwidth]{image/Agent_ablation.png}
  \caption{Ablations to assess the effectiveness
of \MIND~'s two agents (i.e., the guide and strategist) and the memorization mechanism}
  \label{fig:agentablation}
\end{figure}

\subsection{Case Study}

To demonstrate \MIND's effectiveness in real-world applications, we present a case study in Appendix \ref{sec:case study}, featuring a four-round dialogue on the theme of ``work issues,'' with the concern: ``Despite studying hard, my grades remain poor, and effort seems useless in a talent-driven society.'' The case study shows how the devil agent gains confidence through the player's comforting words, while the player also develops greater self-compassion and reconciles with their own concerns.

\section{Related Work}

\subsection{LLM Agent}
An agent refers to an entity capable of perceiving its environment and taking action to achieve its goals. AI agents are increasingly seen as a promising direction toward achieving Artificial General Intelligence (AGI) \citep{durante2024agent}. Agents leverage the capabilities of Large Language Models (LLMs) to perform various tasks. In the construction of LLM agents, two of the most crucial aspects are (1) the architecture and (2) the method of acquiring capabilities. The architecture of LLM agents consists of four parts: Profile (primarily involving character background, written as prompts), Memory (including environmental and contextual information), Planning (allowing the agent to rationally execute according to a plan), and Action (transforming the agent's decisions into reasonable outputs)\citep{wang2024survey}. The method of acquiring capabilities is mainly divided into whether fine-tuning is performed. ReAct \citep{yao2022react} proposed a framework that combines reasoning and action, utilizing prompt engineering for task decomposition. Later, AutoGPT \citep{yang2023auto} introduced memory mechanisms and tool invocation capabilities, supporting multi-step task execution. HuggingGPT \citep{shen2024hugginggpt} coordinated multimodal models through LLMs, validating the potential of LLMs as the control hub. In multi-agent systems, early research borrowed from traditional multi-agent system architecture designs, proposing two mainstream frameworks: hierarchical (e.g., MetaGPT \citep{hong2023metagpt}) and decentralized (e.g., AutoGen \citep{wu2023autogen}). To enhance collaboration efficiency, researchers have explored various interaction paradigms, such as role-playing (CAMEL \citep{li2023camel} promotes task decomposition through predefined role divisions), debate negotiation (e.g., the debate decision-making framework MAD \citep{liang2024encouragingdivergentthinkinglarge}), and knowledge sharing (AgentVerse \citep{chen2023agentverse} uses dynamic memory banks to achieve experience transfer).

\subsection{LLM-assisted Psychology}
The powerful capabilities of LLMs in natural language processing and simulating interpersonal interactions have provided opportunities to assist in mental health. LLMs can play a role in various areas such as medical diagnosis, expansion of mental health resources, and therapy \citep{hua2024large}. In diagnosis, LLMs are widely used for screening and diagnosing mental health issues, including depression, anxiety, and post-traumatic stress disorder (PTSD). In mental health resource development, LLMs address the scarcity of mental health data by generating synthetic data (e.g., simulated counseling dialogues) or expanding existing clinical questionnaires. In psychological therapy, the application of LLMs offers new possibilities for improving mental health services. By increasing accessibility, providing personalized treatment plans, and reducing treatment costs, LLMs have the potential to enhance mental health care. SMILE utilizes ChatGPT to convert single-turn long conversations into multi-turn dialogues for the development of specialized dialogue systems for mental health support \citep{qiu2023smile}. SoulChat constructs the SoulChatCorpus dataset based on psychological consultation questions and answers, fine-tuning it to significantly enhance LLMs' abilities to provide empathy, listening, and comfort when offering emotional support \citep{chen2023soulchat}. MindChat is trained on one million high-quality multi-turn mental health conversation data to communicate in a more empathetic and guiding manner with users \citep{MindChat}. 

\section{Conclusion}

In this study, we propose \MIND ~paradigm, a novel paradigm for psychological healing. Our framework consists of four LLM agents: trigger, devil, guide, and strategist. Through iterative interactions between these agents and the player, the system comforts the player's ``inner self'' within a virtual scenario, thereby enhancing empathy and emotional resonance, reducing self-criticism, and fostering a stronger sense of self-identity. Experimental results validate the significant potential of this paradigm, demonstrating an improved user experience compared to both traditional psychological counseling models and the prototype of our framework. Our work provides a new perspective on gamified psychological healing and opens an innovative path for utilizing LLM agents in therapeutic applications. We hope this research offers a fresh outlook on the intersection of LLMs and psychological healing, encouraging the public to pay greater attention to and improve their mental health.

\section*{Ethics Statement}
The system used in this study is not intended to replace professional psychological treatment but rather to provide an effective option for clinical therapy. Before deployment, it is essential to ensure the presence of licensed professionals for supervision. Our evaluation method ensures the participation of mental health professionals aged 18 and above. The human evaluators' ages range from 25 to 45 years, and their professions include one psychiatrist, two rehabilitation therapists, two psychotherapists, and two nurses. Prior to the experiment, we provided the human evaluators with detailed experimental guidelines.

We have taken rigorous precautions to exclude individuals currently experiencing mental illness or those at risk of self-harm or suicidal tendencies. Our experiments are designed to avoid exposing participants to potentially harmful or misleading content. Participation in our evaluation experiment is entirely voluntary, and participants may withdraw at any time. We also ensured that a member of the research team was present throughout the process to guarantee its safety and effectiveness.

In our human study, we refrained from collecting any personally identifiable information, ensuring the anonymization of data before analysis. All research data were securely stored in a dedicated computing environment, accessible exclusively to trained research personnel.

\section*{Limitations}

This framework has been evaluated exclusively in a Chinese-language context, which poses a limitation in terms of localizing psychological healing applications for different linguistic and cultural settings. While this study represents a significant step forward in shifting the paradigm of psychological healing, moving beyond the focus on training LLMs specifically for the psychological domain., it remains an initial attempt. To effectively implement this research into everyday psychological therapy, further extensive studies and clinical trials involving real mental health patients are necessary. Additionally, the framework's guide agent could benefit from being replaced with a more specialized therapeutic model, which could enhance the system's performance. Moreover, the framework used in this study is a simplified prototype. In the original theory , characters interact within a VR setting. There is significant potential for expanding this framework into more sophisticated formats, such as VR-based applications, to provide users with a more immersive and enriching therapeutic experience. Further exploration is required to address challenges related to the scalability of the system across various therapeutic scenarios and languages. Additionally, it remains unclear how the integration of this framework will scale in real-world settings with diverse patient populations, which presents another area for future research.

% Bibliography entries for the entire Anthology, followed by custom entries
%\bibliography{anthology,custom}
% Custom bibliography entries only
\bibliography{custom}
\clearpage
\appendix

\section{Baseline Methods}
\label{sec:Baseline_methods}

This section provides a comprehensive overview of the baseline methods that we have employed. These methods serve as the foundational approaches in our study, and we introduce two distinct LLM-based baselines: (1) \textbf{Chat-Bot}; (2) \textbf{Traditional Empathy Training}.

\textbf{Chat-Bot} employs a simulated psychologist agent to engage in communication with patients suffering from cognitive distortions. During the conversation, it identifies the types of cognitive distortions and provides comfort and cognitive restructuring to the patients. 

\textbf{Traditional Empathy Training} employs role reversal in four phases to address cognitive distortions. In Phase 1, self-critical participants interact with a crying child avatar as an adult, demonstrating empathy. In Phase 2, some participants switch to the child avatar to receive comfort from their past selves, while others observe from a third-person perspective as a control. Phase 3 involves adapting to new perspectives: first-person participants embody the child avatar, while third-person participants observe without a virtual body. In Phase 4, participants re-experience empathy from the child's perspective, with real-time replays of the adult's gestures and voice.To better align with our current work, we simulated this process using LLMs. An agent, describing actions, demeanor, and emotions, played the role of the crying child. Participants provided verbal comfort and interacted with the agent, observing changes in the crying child. Once the interaction concluded (i.e., when the crying child stopped crying), the comforter assumed the child's perspective to review their comforting words and the child's responses, describing their psychological state. This approach, using agents, replicated the role reversal process typically conducted in Virtual Reality (VR), with specific prompts detailed in Appendix \ref{sec:MIND_prompt}.

\section{SP Role-playing Assessment}
\label{sec:SP metrics}
We provide mental health professionals with the following statement to help them better comprehend tasks and assess models' all-round abilities.

\noindent
(1) \textbf{Dialogue Stability} 

Does the model consistently exhibit characteristics of cognitive distortion across all rounds of dialogue, rather than intermittently deviating from these traits? The simulated patient should maintain a stable mental state throughout the conversation, with consistency in the display of cognitive distortions. Furthermore, the content generated should reflect varying degrees of the same cognitive distortion type.

\noindent
(2) \textbf{Language Fluency}

Is the language coherent and fluent? Cognitive distortion patients may demonstrate features such as slowed speech, increased pauses, and disrupted speech patterns. The SP should replicate these linguistic tendencies, ensuring the language style aligns with the patient's condition and avoids inconsistencies.

\noindent
(3) \textbf{Emotional Expression}

Does the emotional content generated align with the emotional traits typical of cognitive distortion patients? The simulation should accurately reflect common emotional responses observed in these patients, such as persistent low mood, anhedonia, feelings of helplessness, and hopelessness.

\noindent
(4) \textbf{Personalization \& Diversity} 

In addition to core characteristics, does the model incorporate a wide range of individualized traits, such as how different personality traits, life experiences, and educational backgrounds influence the patient's expression and behavior? For example, introverted patients may exhibit more passive and reticent communication styles, while extroverted patients may display more outward and active engagement. The model should construct diverse cognitive profiles to ensure the simulated patient is both authentic and personalized by considering various influencing factors.

\noindent
(5) \textbf{Accuracy}

Is the identification of cognitive distortion types precise? This should be particularly evident in distinguishing the predominant distortion types when multiple cognitive distortions are present in the same interaction.

\onecolumn 
\section{Prompt Templates}
\label{sec:MIND_prompt}

In this section, we present some prompt templates used in this work,and its ablated versions.

\begin{figure*}[h]
    \centering
    \includegraphics[width=1\linewidth]{image/vr_patient.pdf}
\end{figure*}

\begin{figure*}[h]
    \centering
    \includegraphics[width=1\linewidth]{image/vr_change_role.pdf}
\end{figure*}

\begin{figure*}[h]
    \centering
    \includegraphics[width=1\linewidth]{image/vr_user.pdf}
\end{figure*}

\begin{figure*}[h]
  \centering
  \includegraphics[width=\textwidth]{image/trigger_0.pdf}
\end{figure*}

\begin{figure*}[h]
    \centering
    \includegraphics[width=1\linewidth]{image/trigger_i.pdf}
\end{figure*}

\begin{figure*}[h]
    \centering
    \includegraphics[width=1\linewidth]{image/guide.pdf}
\end{figure*}

\begin{figure*}[h]
    \centering
    \includegraphics[width=1\linewidth]{image/devil_0.pdf}
\end{figure*}

\begin{figure*}[h]
    \centering
    \includegraphics[width=1\linewidth]{image/devil_i.pdf}
\end{figure*}

\begin{figure*}[h]
    \centering
    \includegraphics[width=1\linewidth]{image/strategist.pdf}
\end{figure*}

\begin{figure*}[h]
    \centering
    \includegraphics[width=1\linewidth]{image/user.pdf}
\end{figure*}

\begin{figure*}[h]
    \centering
    \includegraphics[width=1\linewidth]{image/trigger_without_memory.pdf}
\end{figure*}

\begin{figure*}[h]
    \centering
    \includegraphics[width=1\linewidth]{image/strategist_without_memory.pdf}
\end{figure*}

\begin{figure*}[h]
    \centering
    \includegraphics[width=1\linewidth]{image/trigger_without_strategist.pdf}
\end{figure*}

\begin{figure*}[h]
    \centering
    \includegraphics[width=1\linewidth]{image/user_without_guide.pdf}
\end{figure*}

\clearpage
\section{Case Study}
\label{sec:case study}
\begin{figure*}[h]
    \centering
    \includegraphics[width=0.98\linewidth]{image/case1.pdf}
\end{figure*}

\begin{figure*}[h]
    \centering
    \includegraphics[width=1\linewidth]{image/case2.pdf}
\end{figure*}

\begin{figure*}[h]
    \centering
    \includegraphics[width=1\linewidth]{image/case3.pdf}
\end{figure*}

\begin{figure*}[h]
    \centering
    \includegraphics[width=1\linewidth]{image/case4.pdf}
\end{figure*}

\end{document}
% 
\definecolor{titlecolor}{rgb}{0.9, 0.5, 0.1}
\definecolor{anscolor}{rgb}{0.2, 0.5, 0.8}
\definecolor{labelcolor}{HTML}{48a07e}
\begin{table*}[h]
	\centering
	
 % \vspace{-0.2cm}
	
	\begin{center}
		\begin{tikzpicture}[
				chatbox_inner/.style={rectangle, rounded corners, opacity=0, text opacity=1, font=\sffamily\scriptsize, text width=5in, text height=9pt, inner xsep=6pt, inner ysep=6pt},
				chatbox_prompt_inner/.style={chatbox_inner, align=flush left, xshift=0pt, text height=11pt},
				chatbox_user_inner/.style={chatbox_inner, align=flush left, xshift=0pt},
				chatbox_gpt_inner/.style={chatbox_inner, align=flush left, xshift=0pt},
				chatbox/.style={chatbox_inner, draw=black!25, fill=gray!7, opacity=1, text opacity=0},
				chatbox_prompt/.style={chatbox, align=flush left, fill=gray!1.5, draw=black!30, text height=10pt},
				chatbox_user/.style={chatbox, align=flush left},
				chatbox_gpt/.style={chatbox, align=flush left},
				chatbox2/.style={chatbox_gpt, fill=green!25},
				chatbox3/.style={chatbox_gpt, fill=red!20, draw=black!20},
				chatbox4/.style={chatbox_gpt, fill=yellow!30},
				labelbox/.style={rectangle, rounded corners, draw=black!50, font=\sffamily\scriptsize\bfseries, fill=gray!5, inner sep=3pt},
			]
											
			\node[chatbox_user] (q1) {
				\textbf{System prompt}
				\newline
				\newline
				You are a helpful and precise assistant for segmenting and labeling sentences. We would like to request your help on curating a dataset for entity-level hallucination detection.
				\newline \newline
                We will give you a machine generated biography and a list of checked facts about the biography. Each fact consists of a sentence and a label (True/False). Please do the following process. First, breaking down the biography into words. Second, by referring to the provided list of facts, merging some broken down words in the previous step to form meaningful entities. For example, ``strategic thinking'' should be one entity instead of two. Third, according to the labels in the list of facts, labeling each entity as True or False. Specifically, for facts that share a similar sentence structure (\eg, \textit{``He was born on Mach 9, 1941.''} (\texttt{True}) and \textit{``He was born in Ramos Mejia.''} (\texttt{False})), please first assign labels to entities that differ across atomic facts. For example, first labeling ``Mach 9, 1941'' (\texttt{True}) and ``Ramos Mejia'' (\texttt{False}) in the above case. For those entities that are the same across atomic facts (\eg, ``was born'') or are neutral (\eg, ``he,'' ``in,'' and ``on''), please label them as \texttt{True}. For the cases that there is no atomic fact that shares the same sentence structure, please identify the most informative entities in the sentence and label them with the same label as the atomic fact while treating the rest of the entities as \texttt{True}. In the end, output the entities and labels in the following format:
                \begin{itemize}[nosep]
                    \item Entity 1 (Label 1)
                    \item Entity 2 (Label 2)
                    \item ...
                    \item Entity N (Label N)
                \end{itemize}
                % \newline \newline
                Here are two examples:
                \newline\newline
                \textbf{[Example 1]}
                \newline
                [The start of the biography]
                \newline
                \textcolor{titlecolor}{Marianne McAndrew is an American actress and singer, born on November 21, 1942, in Cleveland, Ohio. She began her acting career in the late 1960s, appearing in various television shows and films.}
                \newline
                [The end of the biography]
                \newline \newline
                [The start of the list of checked facts]
                \newline
                \textcolor{anscolor}{[Marianne McAndrew is an American. (False); Marianne McAndrew is an actress. (True); Marianne McAndrew is a singer. (False); Marianne McAndrew was born on November 21, 1942. (False); Marianne McAndrew was born in Cleveland, Ohio. (False); She began her acting career in the late 1960s. (True); She has appeared in various television shows. (True); She has appeared in various films. (True)]}
                \newline
                [The end of the list of checked facts]
                \newline \newline
                [The start of the ideal output]
                \newline
                \textcolor{labelcolor}{[Marianne McAndrew (True); is (True); an (True); American (False); actress (True); and (True); singer (False); , (True); born (True); on (True); November 21, 1942 (False); , (True); in (True); Cleveland, Ohio (False); . (True); She (True); began (True); her (True); acting career (True); in (True); the late 1960s (True); , (True); appearing (True); in (True); various (True); television shows (True); and (True); films (True); . (True)]}
                \newline
                [The end of the ideal output]
				\newline \newline
                \textbf{[Example 2]}
                \newline
                [The start of the biography]
                \newline
                \textcolor{titlecolor}{Doug Sheehan is an American actor who was born on April 27, 1949, in Santa Monica, California. He is best known for his roles in soap operas, including his portrayal of Joe Kelly on ``General Hospital'' and Ben Gibson on ``Knots Landing.''}
                \newline
                [The end of the biography]
                \newline \newline
                [The start of the list of checked facts]
                \newline
                \textcolor{anscolor}{[Doug Sheehan is an American. (True); Doug Sheehan is an actor. (True); Doug Sheehan was born on April 27, 1949. (True); Doug Sheehan was born in Santa Monica, California. (False); He is best known for his roles in soap operas. (True); He portrayed Joe Kelly. (True); Joe Kelly was in General Hospital. (True); General Hospital is a soap opera. (True); He portrayed Ben Gibson. (True); Ben Gibson was in Knots Landing. (True); Knots Landing is a soap opera. (True)]}
                \newline
                [The end of the list of checked facts]
                \newline \newline
                [The start of the ideal output]
                \newline
                \textcolor{labelcolor}{[Doug Sheehan (True); is (True); an (True); American (True); actor (True); who (True); was born (True); on (True); April 27, 1949 (True); in (True); Santa Monica, California (False); . (True); He (True); is (True); best known (True); for (True); his roles in soap operas (True); , (True); including (True); in (True); his portrayal (True); of (True); Joe Kelly (True); on (True); ``General Hospital'' (True); and (True); Ben Gibson (True); on (True); ``Knots Landing.'' (True)]}
                \newline
                [The end of the ideal output]
				\newline \newline
				\textbf{User prompt}
				\newline
				\newline
				[The start of the biography]
				\newline
				\textcolor{magenta}{\texttt{\{BIOGRAPHY\}}}
				\newline
				[The ebd of the biography]
				\newline \newline
				[The start of the list of checked facts]
				\newline
				\textcolor{magenta}{\texttt{\{LIST OF CHECKED FACTS\}}}
				\newline
				[The end of the list of checked facts]
			};
			\node[chatbox_user_inner] (q1_text) at (q1) {
				\textbf{System prompt}
				\newline
				\newline
				You are a helpful and precise assistant for segmenting and labeling sentences. We would like to request your help on curating a dataset for entity-level hallucination detection.
				\newline \newline
                We will give you a machine generated biography and a list of checked facts about the biography. Each fact consists of a sentence and a label (True/False). Please do the following process. First, breaking down the biography into words. Second, by referring to the provided list of facts, merging some broken down words in the previous step to form meaningful entities. For example, ``strategic thinking'' should be one entity instead of two. Third, according to the labels in the list of facts, labeling each entity as True or False. Specifically, for facts that share a similar sentence structure (\eg, \textit{``He was born on Mach 9, 1941.''} (\texttt{True}) and \textit{``He was born in Ramos Mejia.''} (\texttt{False})), please first assign labels to entities that differ across atomic facts. For example, first labeling ``Mach 9, 1941'' (\texttt{True}) and ``Ramos Mejia'' (\texttt{False}) in the above case. For those entities that are the same across atomic facts (\eg, ``was born'') or are neutral (\eg, ``he,'' ``in,'' and ``on''), please label them as \texttt{True}. For the cases that there is no atomic fact that shares the same sentence structure, please identify the most informative entities in the sentence and label them with the same label as the atomic fact while treating the rest of the entities as \texttt{True}. In the end, output the entities and labels in the following format:
                \begin{itemize}[nosep]
                    \item Entity 1 (Label 1)
                    \item Entity 2 (Label 2)
                    \item ...
                    \item Entity N (Label N)
                \end{itemize}
                % \newline \newline
                Here are two examples:
                \newline\newline
                \textbf{[Example 1]}
                \newline
                [The start of the biography]
                \newline
                \textcolor{titlecolor}{Marianne McAndrew is an American actress and singer, born on November 21, 1942, in Cleveland, Ohio. She began her acting career in the late 1960s, appearing in various television shows and films.}
                \newline
                [The end of the biography]
                \newline \newline
                [The start of the list of checked facts]
                \newline
                \textcolor{anscolor}{[Marianne McAndrew is an American. (False); Marianne McAndrew is an actress. (True); Marianne McAndrew is a singer. (False); Marianne McAndrew was born on November 21, 1942. (False); Marianne McAndrew was born in Cleveland, Ohio. (False); She began her acting career in the late 1960s. (True); She has appeared in various television shows. (True); She has appeared in various films. (True)]}
                \newline
                [The end of the list of checked facts]
                \newline \newline
                [The start of the ideal output]
                \newline
                \textcolor{labelcolor}{[Marianne McAndrew (True); is (True); an (True); American (False); actress (True); and (True); singer (False); , (True); born (True); on (True); November 21, 1942 (False); , (True); in (True); Cleveland, Ohio (False); . (True); She (True); began (True); her (True); acting career (True); in (True); the late 1960s (True); , (True); appearing (True); in (True); various (True); television shows (True); and (True); films (True); . (True)]}
                \newline
                [The end of the ideal output]
				\newline \newline
                \textbf{[Example 2]}
                \newline
                [The start of the biography]
                \newline
                \textcolor{titlecolor}{Doug Sheehan is an American actor who was born on April 27, 1949, in Santa Monica, California. He is best known for his roles in soap operas, including his portrayal of Joe Kelly on ``General Hospital'' and Ben Gibson on ``Knots Landing.''}
                \newline
                [The end of the biography]
                \newline \newline
                [The start of the list of checked facts]
                \newline
                \textcolor{anscolor}{[Doug Sheehan is an American. (True); Doug Sheehan is an actor. (True); Doug Sheehan was born on April 27, 1949. (True); Doug Sheehan was born in Santa Monica, California. (False); He is best known for his roles in soap operas. (True); He portrayed Joe Kelly. (True); Joe Kelly was in General Hospital. (True); General Hospital is a soap opera. (True); He portrayed Ben Gibson. (True); Ben Gibson was in Knots Landing. (True); Knots Landing is a soap opera. (True)]}
                \newline
                [The end of the list of checked facts]
                \newline \newline
                [The start of the ideal output]
                \newline
                \textcolor{labelcolor}{[Doug Sheehan (True); is (True); an (True); American (True); actor (True); who (True); was born (True); on (True); April 27, 1949 (True); in (True); Santa Monica, California (False); . (True); He (True); is (True); best known (True); for (True); his roles in soap operas (True); , (True); including (True); in (True); his portrayal (True); of (True); Joe Kelly (True); on (True); ``General Hospital'' (True); and (True); Ben Gibson (True); on (True); ``Knots Landing.'' (True)]}
                \newline
                [The end of the ideal output]
				\newline \newline
				\textbf{User prompt}
				\newline
				\newline
				[The start of the biography]
				\newline
				\textcolor{magenta}{\texttt{\{BIOGRAPHY\}}}
				\newline
				[The ebd of the biography]
				\newline \newline
				[The start of the list of checked facts]
				\newline
				\textcolor{magenta}{\texttt{\{LIST OF CHECKED FACTS\}}}
				\newline
				[The end of the list of checked facts]
			};
		\end{tikzpicture}
        \caption{GPT-4o prompt for labeling hallucinated entities.}\label{tb:gpt-4-prompt}
	\end{center}
\vspace{-0cm}
\end{table*}