\begin{figure}[t!]
	\centering
\begin{tikzpicture}
	\begin{axis}[
    xlabel={$ p $},
    ylabel={$ \clr $},
    axis lines=left,
    ymax=0.09,
    scaled ticks=false,
    yticklabel style={
	/pgf/number format/precision=3,
	/pgf/number format/fixed,
    }
]

\addplot [
    domain=1/2:1, 
    samples=\figSamples,
    color=blue,
    smooth
]
{-(1/2)*x*(1+x-12*x^2+10*x^3+10*x^4-14*x^5+4*x^6)};
\addlegendentry{\(\pone-\pzero\)}

\addplot [
    domain=1/2:1, 
    samples=\figSamples,
    color=red,
    smooth
]
{-(1/4)*(-1+x)^2*x^2*(4+5*x-28*x^2-14*x^3+35*x^4+14*x^5-28*x^6+8*x^7)};
\addlegendentry{\(\pthree-\pone\)}

\end{axis}
\end{tikzpicture}
\caption{The plot of the numerator and the denominator of the condition on $ M-1 $ from \Cref{lemma:bayesian_clause}.
The denominator approaches zero noticably faster, suggesting the fraction approaches infinity as $ p $ goes to 1.}
\label{fig:pdff}
\Description{The difference $\pone-\pzero$ is always strictly larger than $\pthree-\pone$. As $p$ goes to 1, their ratio grows.}
\end{figure}
