\begin{figure}[t!]
\centering
\begin{tikzpicture}
	\begin{axis}[
    xlabel={$ p $},
    ylabel={$ \varepsilon(p) $},
    legend pos=south east,
    ymax=0.001,
    ymin=-0.003,
    axis lines=center,
    scaled ticks=false,
    yticklabel style={
	/pgf/number format/precision=3,
	/pgf/number format/fixed,
    }
]

\addplot [
    domain=1/2:1, 
    samples=\figSamples,
    color=blue,
    smooth
]
 {(7*x*(-2+26*x+3*x^2-258*x^3+309*x^4+273*x^5-498*x^6-147*x^7+546*x^8-308*x^9+56*x^10))/1536};

\end{axis}
\end{tikzpicture}
\caption{The value of $ \varepsilon $ w.r.t. choice of $p$ below which (additive) approximation is hard (that is, $ \delta = \frac 78 $ \cite{Hastad2001-fg}). The requirement of $\varepsilon > 0$ gives us that $p \geq \sqrt{7/8}$.}
\Description{A plot of $\varepsilon$, initially zero, then negative, and in the interval $(\sqrt{7/8}, 1)$ slightly positive.}
\label{fig:jie_epsilon}
\end{figure}
