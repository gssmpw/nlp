\begin{figure}[t!]
	\centering
\begin{tikzpicture}
	\begin{axis}[
    xlabel={$ p $},
    ylabel={$ \clr(\gadget C) $},
    legend pos=south east,
    axis lines=left,
]

\addplot [
    domain=1/2:1, 
    samples=\figSamples,
    color=red,
    smooth
]
{x*(2+14*x+27*x^2-6*x^3-59*x^4-7*x^5+62*x^6+21*x^7-78*x^8+44*x^9-8*x^10)/4};
\addlegendentry{\(\pthree\)}

\addplot [
    domain=1/2:1, 
    samples=\figSamples,
    color=green,
    smooth
]
{x*(2+15*x+22*x^2+7*x^3-84*x^4+14*x^5+92*x^6-72*x^7+16*x^8)/4};
\addlegendentry{\(\ptwo\)}

\addplot [
    domain=1/2:1, 
    samples=\figSamples,
    color=blue,
    smooth
]
{(x+9*x^2+12*x^3-20*x^4-6*x^5+14*x^6-4*x^7)/2};
\addlegendentry{\(\pone\)}

\addplot [
    domain=1/2:1, 
    samples=\figSamples,
    color=black,
    smooth
]
{x*(1+5*x-5*x^3+2*x^4)};
\addlegendentry{\(\pzero\)}

\end{axis}
\end{tikzpicture}
\caption{The relationship between learning rates of $ \gadget C $, depending on the number of ``on'' literals, as a function of $ p $.}
\Description{All are equal to $\frac 12$ for $p=\frac 12$, and $1$ for $p=1$. In the interval between those two values, $\pzero \leq \pone \leq \ptwo \leq \pthree$.}
\label{fig:bayes_ps}
\end{figure}
