\begin{figure}[t!]
	\centering
	\begin{tikzpicture}[
		xsh/.style = { yshift=-10mm },
		node distance = 5mm and 5mm,
		dcs/.style = { yshift=-3mm,xshift=-6mm },
	stff/.style={circle, draw=black, \figThickness, minimum size=\figCircSize, inner sep=0pt},
	]
		%Nodes
		\node[stff]        (x_i)                  {$ x $};
		\node[stff]        (nx_i)   [left=of x_i]   {$ \lnot x $};
		\node[stff]        (d_i)   [above=of x_i,dcs]   {$ d_x $};

		\node[stff]        (x_j)    [left=of nx_i]  {$ y $};
		\node[stff]        (nx_j)   [left=of x_j]   {$ \lnot y $};
		\node[stff]        (d_j)   [above=of x_j,dcs]   {$ d_y $};

		\node[stff]        (x_k)    [left=of nx_j]  {$ z $};
		\node[stff]        (nx_k)   [left=of x_k]   {$ \lnot z $};
		\node[stff]        (d_k)   [above=of x_k,dcs]   {$ d_z $};

		\node[stff]        (l_m^1)    [below=of x_i,xsh]  {$ x_C $};
		\node[stff]        (l_m^2)    [below=of x_j]  {$ y_C $};
		\node[stff]        (l_m^3)    [below=of nx_k,xsh]  {$ \lnot z_C $};


		%Lines
		\draw[->, \figThickness] (d_i)  to  (x_i);
		\draw[->, \figThickness] (d_i)  to (nx_i);
		\draw[<->, \figThickness] (x_i)  to (nx_i);

		\draw[->, \figThickness] (d_j)  to  (x_j);
		\draw[->, \figThickness] (d_j)  to (nx_j);
		\draw[<->, \figThickness] (x_j)  to (nx_j);

		\draw[->, \figThickness] (d_k)  to  (x_k);
		\draw[->, \figThickness] (d_k)  to (nx_k);
		\draw[<->, \figThickness] (x_k)  to (nx_k);

		\draw[->, \figThickness] (x_i)  to (l_m^1);
		\draw[->, \figThickness] (x_j)  to (l_m^2);
		\draw[->, \figThickness] (nx_k)  to (l_m^3);

		\draw[<->, \figThickness] (l_m^1)  to (l_m^2);
		\draw[<->, \figThickness] (l_m^1)  to (l_m^3);
		\draw[<->, \figThickness] (l_m^2)  to (l_m^3);
	\end{tikzpicture}
	\caption{The graph $ G_\varphi $ for $ \varphi = C = \left( x \lor y \lor \lnot z \right)$.}
 \Description{A construction for the Majority vote reduction, as described in \Cref{def:bayesian_graph} for a simple formula.}
	\label{fig:gphi_bayesian}
\end{figure}
