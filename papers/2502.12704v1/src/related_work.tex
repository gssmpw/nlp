\subsection*{Related Work}

Our proof is perhaps similar to that of \cite{hazla2019reasoning}, where they make a reduction of a similar problem to PSPACE.
We also make a reduction to the \sat{} problem, but while they were working under the model where decisions are made repeatedly, we work under the sequential decision-making model.
Furthermore, our decision problem focuses on the overall learning rate of the entire network, as opposed to the complexity of computing the single probability of learning the ground truth by one agent of \cite{hazla2019reasoning}.

More broadly, much work has been done in determining how well sequential social networks are able to learn on particular families of graphs. In general, most existing literature in this field focuses on determining variants of network learning rates for these families, in both synchronous and asynchronous (sequential) cases. It is understood that determining this for arbitrary, unstructured graphs is hard, but prior to this work this was unproven. Our work justifies the study and construction of particular families of graphs for sequential networks, since our \np-hardness result implies that no efficient algorithm can be used to determine the best-case learning rate for arbitrary sequential social networks on directed graphs. For instance, \cite{multiissue} give a construction of a ``celebrity graph'' for which the network is able to effectively aggregate information under random network orderings. Similarly, \cite{Mossel_2013} show that for sufficiently biased private signals, the network on the expander graph achieves network learning approaching $1$. 

