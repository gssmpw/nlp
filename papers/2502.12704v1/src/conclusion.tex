\section{Conclusion}

In this paper, we tackle the \emph{complexity} of judging how well-equipped a given network is for social truth learning.
We consider this question in the setting of sequential decision-making by agents with bounded belief.
We then show that it is \np-hard to decide whether a large proportion of the network successfully learns a binary ground truth, both when agents are fully rational and when agents have bounded rationality. Finally, we show that it is actually hard to even approximate the learning rate of a fully rational network.

\subsection*{Future Work}

There are many remaining open directions for future work.
A natural question to ask is whether this problem belongs to the class \np, or not. Namely, it remains open whether there exists some efficiently verifiable characterization of networks which achieve high learning rates.
We conjecture that the answer is no, but we have so far been unable to prove it.

Yet another interesting direction is that of finding connections between \netlearn{} and other combinatorial problems.
\netlearn{} seems to be somewhat connected to finding big independent sets, which can allow for successful aggregation of information, along with big enough cliques for efficient spread of information. Clarifying the balance between these two contributing factors could improve our understanding of both network learning and its relation to the rest of combinatorics.

Finally, it would be fascinating to look for classes of networks for which \netlearn{} is easy.
For example, the problem of graph coloring is \np-hard in general, but easy for certain classes of graphs, such as perfect graphs. Similarly, it would be useful to develop characterizations of graphs which achieve high network learning rates and graphs whose learning rates remain bounded away from $1$. An alternative quantity of interest is the network learning rate under random, rather than optimal, decision orderings. Further, one can also consider robustness of different networks under networks with adversarial agents. 
We suspect that finding such characterizations and families of networks would give us a better understanding of the previously mentioned question of connecting this problem to the rest of the field of combinatorics.
