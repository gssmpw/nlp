%%%%%%%%%%%%%%%%%%%%%%%%%%%%%%%%%%%%%%%%%%%%%%%%%%%%%%%%%%%%%%%%%%%%%%%%
%%%%%%%%%%%%%%%%%%%%%%%%%%%%%%%%%%%%%%%%%%%%%%%%%%%%%%%%%%%%%%%%%%%%%%%%

%%% LaTeX Template for AAMAS-2025 (based on sample-sigconf.tex)
%%% Prepared by the AAMAS-2025 Program Chairs based on the version from AAMAS-2025. 

%%%%%%%%%%%%%%%%%%%%%%%%%%%%%%%%%%%%%%%%%%%%%%%%%%%%%%%%%%%%%%%%%%%%%%%%

%%% Start your document with the \documentclass command.


%%% == IMPORTANT ==
%%% Use the first variant below for the final paper (including auithor information).
%%% Use the second variant below to anonymize your submission (no authoir information shown).
%%% For further information on anonymity and double-blind reviewing, 
%%% please consult the call for paper information
%%% https://aamas2025.org/index.php/conference/calls/submission-instructions-main-technical-track/

%%%% For anonymized submission, use this
\documentclass[sigconf]{aamas} 

%%%% For camera-ready, use this
% \documentclass[sigconf]{aamas} 


%%% Load required packages here (note that many are included already).

\usepackage{balance} % for balancing columns on the final page

%%%%%%%%%%%%%%%%%%%%%%%%%%%%%%%%%%%%%%%%%%%%%%%%%%%%%%%%%%%%%%%%%%%%%%%%

%%% AAMAS-2025 copyright block (do not change!)

\makeatletter
\gdef\@copyrightpermission{
  \begin{minipage}{0.2\columnwidth}
   \href{https://creativecommons.org/licenses/by/4.0/}{\includegraphics[width=0.90\textwidth]{by}}
  \end{minipage}\hfill
  \begin{minipage}{0.8\columnwidth}
   \href{https://creativecommons.org/licenses/by/4.0/}{This work is licensed under a Creative Commons Attribution International 4.0 License.}
  \end{minipage}
  \vspace{5pt}
}
\makeatother

\setcopyright{ifaamas}
\acmConference[AAMAS '25]{Proc.\@ of the 24th International Conference
on Autonomous Agents and Multiagent Systems (AAMAS 2025)}{May 19 -- 23, 2025}
{Detroit, Michigan, USA}{Y.~Vorobeychik, S.~Das, A.~Nowé  (eds.)}
\copyrightyear{2025}
\acmYear{2025}
\acmDOI{}
\acmPrice{}
\acmISBN{}

\newcommand{\thought}[1]{{\color[rgb]{0.2,0.39,0.66}(#1)}}
\newcommand{\todo}[1]{{\color[rgb]{1.0,0.0,0.0}(#1)}}
\newcommand{\hsh}[1]{{\color{green!50!black} Henrik: #1}}
\newcommand{\st}[1]{{\color{red!50!black} Sebastian: #1}}

\newcommand{\ulm}[1]{_{\scaleto{\mathrm{#1}}{3pt}}}
\newcommand\at[2]{\left.#1\right|_{#2}}











\newtheorem{assumption}{Assumption}

\DeclareMathOperator*{\argmax}{arg\,max}
\DeclareMathOperator*{\argmin}{arg\,min}

\newcommand{\swname}[1]{\texttt{#1}}
\newcommand{\ie}{i\/.\/e\/.,\/~}
\newcommand{\eg}{e\/.\/g\/.,\/~}
\newcommand{\cf}{cf\/.\/~}

\newcommand{\fig}{Fig\/.\/~}
\newcommand{\defn}{Def\/.\/~}
\newcommand{\sect}{Sec\/.\/~}
\newcommand{\tabl}{Tab\/.\/~}
\newcommand{\algo}{Algorithm~}
\newcommand{\theo}{Theorem~}

\newcommand{\bnnl}{3 hidden layers}
\newcommand{\bnnn}{50 neurons}
\newcommand{\bnna}{tanh activations}

\newcommand{\capt}[1]{\mdseries{\emph{#1}}}

\newcommand{\videolink}{at \url{https://youtu.be/_d7AqTRjz6g}}
\newcommand{\codelink}{\url{https://github.com/wheelbot/mini-wheelbot}}

\newcommand{\fakepar}[1]{\vspace{0mm}\noindent\textbf{#1.}}

\newcommand{\needref}{\textcolor{red}{[REF]}}

\newcommand{\plotfontsize}{9pt}

%\renewcommand{\appmaj}{Appendix~B}
%\renewcommand{\appbayes}{Appendix~A}

%%%%%%%%%%%%%%%%%%%%%%%%%%%%%%%%%%%%%%%%%%%%%%%%%%%%%%%%%%%%%%%%%%%%%%%%

%%% == IMPORTANT ==
%%% Use this command to specify your submission number.
%%% In anonymous mode, it will be printed on the first page.

\acmSubmissionID{564}

%%% Use this command to specify the title of your paper.

\title{Maximizing Truth Learning in a Social Network is NP-hard}

%%% Provide names, affiliations, and email addresses for all authors.

\author{Filip \'Uradn\'ik}
\authornote{These authors contributed equally.}
\affiliation{
  \institution{Charles University}
  \city{Prague}
  \country{Czech Republic}}
\email{uradnik@kam.mff.cuni.cz}

\author{Amanda Wang}
\authornotemark[1]
\affiliation{
  \institution{Princeton University}
  \city{Princeton, NJ}
  \country{United States}}
\email{aw4309@princeton.edu}

\author{Jie Gao}
\affiliation{
  \institution{Rutgers University}
  \city{Piscataway, NJ}
  \country{United States}}
\email{jg1555@rutgers.edu}


%%% Use this environment to specify a short abstract for your paper.

\begin{abstract}
  \Abstract{}
\end{abstract}

%%% The code below was generated by the tool at http://dl.acm.org/ccs.cfm.
%%% Please replace this example with code appropriate for your own paper.

%%% Use this command to specify a few keywords describing your work.
%%% Keywords should be separated by commas.

\keywords{\Keywords{}}

%%%%%%%%%%%%%%%%%%%%%%%%%%%%%%%%%%%%%%%%%%%%%%%%%%%%%%%%%%%%%%%%%%%%%%%%

%%% Include any author-defined commands here.
         
\newcommand{\BibTeX}{\rm B\kern-.05em{\sc i\kern-.025em b}\kern-.08em\TeX}

%%%%%%%%%%%%%%%%%%%%%%%%%%%%%%%%%%%%%%%%%%%%%%%%%%%%%%%%%%%%%%%%%%%%%%%%

\begin{document}

%%% The following commands remove the headers in your paper. For final 
%%% papers, these will be inserted during the pagination process.

\pagestyle{fancy}
\fancyhead{}

%%% The next command prints the information defined in the preamble.

%BEGIN
\maketitle 

%%%%%%%%%%%%%%%%%%%%%%%%%%%%%%%%%%%%%%%%%%%%%%%%%%%%%%%%%%%%%%%%%%%%%%%%

\section{Introduction}
\label{sec:intro}

\begin{figure*}[tb]
    \centering
    \includegraphics[width=0.848\linewidth]{figs/circuitnn.pdf} 
    \caption{Illustration of differentiable CircuitNN. CircuitNN is designed based on differentiable NAND gates. After DAS is guided by PI and PO pairs of the truth table, CircuitNN can get the precise circuit architecture logic equivalent to the truth table.}
    \label{fig:circuitnn}
\end{figure*}

% 1. Describe the importance of logic synthesis
% 2. Existing Problems
% (a) Neural Architecture Search: Unstable, Predefined Setting, etc.
% (b) Circuit Generation: Probabilistic Model, Logic Equivalence

With the rapid advancement of technology, the scale of integrated circuits (ICs) has expanded exponentially. 
This expansion has introduced significant challenges in chip manufacturing, particularly concerning power and area metrics.
A primary objective in IC design is achieving the same circuit function with fewer transistors, thereby reducing power usage and area occupancy.

Logic synthesis~\cite{hachtel2005logicsynth}, a critical step in electronic design automation (EDA), transforms behavioral-level circuit designs into optimized gate-level circuits, ultimately yielding the final IC layout. 
The primary goal of logic synthesis is to identify the physical implementation with the fewest gates for a given circuit function. 
This task constitutes a challenging NP-hard combinatorial optimization problem. 
Current logic synthesis tools~\cite{brayton2010abc, wolf2013yosys} rely on human-designed heuristics, often leading to sub-optimal outcomes.

Differentiable architecture search (DAS) techniques~\cite{liu2018darts, chu2020darts} offer novel perspectives on addressing challenges in this problem.
Circuit functions can be represented through truth tables, which map binary inputs to their corresponding outputs. 
Truth tables provide a precise representation of input-output relationships, ensuring the design of functionally equivalent circuits.
Inspired by this, researchers~\cite{deepmind2024ai4sys, wang2024tnet} have begun exploring the application of DAS to synthesize circuits directly from truth tables.
Specifically, \citet{deepmind2024ai4sys} proposed CircuitNN, a framework that learns differentiable connection structures with logic gates, enabling the automatic generation of logic circuits from truth tables.
This approach significantly reduces the complexity of traditional circuit generation. 
Building on this, \citet{wang2024tnet} introduced T-Net, a triangle-shaped variant of CircuitNN, incorporating regularization techniques to enhance the efficiency of DAS.

Despite these advancements, several challenges remain. 
The computational complexity of DAS grows quadratically with the number of gates, posing scalability issues.
Although triangle-shaped architecture~\cite{wang2024tnet} partially mitigates this problem, redundancy persists. 
%Additionally, DAS is susceptible to converging to local optima, limiting the ability to search architectures that satisfy the given truth tables~\cite{liu2018darts}. 
%Furthermore, hyperparameters (network depth and layer width) require extensive searches, introducing complexity and prolonging the synthesis process. 
Additionally, DAS is susceptible to converging to local optima~\cite{liu2018darts} and hyperparameters (network depth and layer width) require extensive searches. 
The challenges arise from the vast search space in DAS. 
% Even with predefined settings for CircuitNN, finding a configuration that meets the truth table requires extensive trial and error during the DAS process. 
Intuitively, limiting the search space through predefined parameters (network depth, gates per layer, and connection probabilities) can significantly reduce the complexity.

Recent advances~\cite{openai2023gpt4, abramson2024alphafold3, esser2024sd3, li2024mar} in conditional generative models have demonstrated remarkable performance across language, vision, and graph generation tasks. 
Motivated by these developments, we propose a novel approach to circuit generation that generates preliminary circuit structures to guide DAS in generating refined circuits matching specified truth tables. 
Firstly, we introduce CircuitVQ, a tokenizer with a discrete codebook for circuit tokenization. 
Built upon our Circuit AutoEncoder framework~\cite{hou2022graphmae,li2023maskgae,wu2025mgvga}, CircuitVQ is trained through a circuit reconstruction task. 
Specifically, the CircuitVQ encoder encodes input circuits into discrete tokens using a learnable codebook, while the decoder reconstructs the circuit adjacency matrix based on these tokens.
Subsequently, the CircuitVQ encoder serves as a circuit tokenizer for CircuitAR pretraining, which employs a masked autoregressive modeling paradigm~\cite{chang2022maskgit, li2023mage}. 
In this process, the discrete codes function as supervision signals. 
After training, CircuitAR can generate discrete tokens progressively, which can be decoded into initial circuit structures by the decoder of the CircuitVQ. 
These prior insights can guide DAS in producing refined circuits that match the target truth tables precisely.

Our key contributions can be summarized as follows:
\begin{itemize}
\item We introduce CircuitVQ, a circuit tokenizer that facilitates graph autoregressive modeling for circuit generation, based on our Circuit AutoEncoder framework;
\item Develop CircuitAR, a model trained using masked autoregressive modeling, which generates initial circuit structures conditioned on given truth tables;
\item Propose a refinement framework that integrates differentiable architecture search to produce functionally equivalent circuits guided by target truth tables;
\item Comprehensive experiments demonstrating the scalability and capability emergence of our CircuitAR and the superior performance of the proposed circuit generation approach.
\end{itemize}

% Motivation
% (a) Diffusion (Vision, Graph), Autoregressive (Language, Vision)
% (b) Circuit Generation for Predefined Setting
% (c) Neural Architecture Search for Strict Logic Equivalence

% Contribution
% (a) Circuit Tokenizer (new transformer arch, training strategy)
% (b) CircuitAR (train and gen strategies, post-ar strategy)
% (c) Extensive Evaluation including BitD (Bit Distance) for Scalability

\section{Problem Statement}
\label{sec:problem_statement}


\begin{figure*}[tb!]
    %\begin{figure}[t]
 \centering
  \includegraphics[width=1\linewidth]{image/motivation_example_svg2pdf.pdf}
 \caption{Overview of human preference judgments for a pair of paraphrase ad texts.}
 \label{fig:motivation_example}
\end{figure}
    \begin{subfigure}[b]{0.35\textwidth}
        \centering
        \includegraphics[]{motivation_legend.pdf}
        \includegraphics[]{b8.pdf}
        \includegraphics[]{b32.pdf}
        
    \end{subfigure}
    \hspace{6mm}
    \begin{subfigure}[b]{0.46\textwidth}
        \includegraphics[]{TT_acc_fig.pdf}
    \end{subfigure}

    \caption{The training time on fixed number of samples~$T_s$ and the total training time~$TT_\text{acc}$ to reach an accuracy threshold of~$78\%$ using two batch sizes 8 and 32, while considering the maximum feasible GPU frequencies under three power constraints; $P_1=\SI{4.5}{\watt}$, $P_2=\SI{5}{\watt}$, and~$P_3=\SI{7}{\watt}$. We observe that for~$P_1$ and~$P_2$, selecting $b=8$ will lead to lower~$TT_\text{acc}$, while for~$P_3$ selecting $b=32$ is better. This is in contrast with our observation for~$T_s$, where selecting $b=32$ is the best option in all cases.}
    \label{fig:b8_b32_power_constraint}
\end{figure*}

We consider the following scenario: for a specific training task, an edge device requests a pre-trained \ac{NN} model~$\nn$ with its weights~$\theta$ from a server in order to fine-tune it on local data~$D$ till reaching a given accuracy threshold. 
Importantly, the edge device has a power constraint~$P_{\text{max}}$, which should not be exceeded during the training process. 
Our goal in this paper is to \textit{minimize the training (fine-tuning) time at edge devices under their given power constraints.} 



 We introduce~$T_s$ as the training time required to apply training using a fixed number of samples~$s$. As shown in \cref{fig:3d_profiling}, the joint selection of $b$ and $f$ will help reduce $T_s$  under a power constraint. However, the ultimate optimization goal is to minimize the total training time required to reach a target accuracy, which we label~$TT_{\text{acc}}$.
A set of parameters, i.e., frequency and batch size~$(f,b)$, that are optimal for training a fixed number of samples ($T_s$) might not be necessarily optimal for the training to accuracy ($TT_{\text{acc}})$.

We display in~\cref{fig:b8_b32_power_constraint} the training time to reach an accuracy  of~$78\%$ (~$TT_{\text{acc}}$) for ResNet18 using two batch sizes of~$b_1=8$ and~$b_2=32$ under three different power constraints (i.e., $P_1=\SI{4.5}{\watt}$, $P_2=\SI{5}{\watt}$, and $P_3=\SI{7}{\watt}$). 
We notice that for the three power constraints, selecting~$b_1$ allows to utilize a higher frequency than~$b_2$.
%For the feasible operating points (frequency, batch size) that satisfy the power constraints, we compare $T_S$ and $TT_{\text{acc}}$.
For each batch size, we select the highest frequency that satisfies the power constraint, and measure~$T_s$ and~$TT_{\text{acc}}$.
We observe that using~$b_2$ (the higher batch size) always leads to a lower~$T_s$. %\textcolor{red}{Furthermore, the ratio of ~$T_s$ for ~$b_1$ to ~$T_s$ for ~$b_2$ increases as the power limit value increases, indicating that the larger batch size ~$b_2$ becomes more efficient relative to ~$b_1$ under higher power limits. }
However, selecting the same batch size over~$b_1$ leads to a longer~$TT_{\text{acc}}$ for~$P_1$ and~$P_2$, and shorter~$TT_{\text{acc}}$ for~$P_3$. 
%This shows the complexity of the targeted problem, as the effect of the power constraint on the feasible frequency and the different number of training iterations to reach accuracy for batch sizes highly influences the optimal batch size to minimize~$TT_{\text{acc}}$.

This shows the complexity of the targeted problem. In particular, $TT_{\text{acc}}$ does not only depend on $T_s$, but it also depends on the number of times of processing~$s$ to reach target accuracy ($N_{{s}_\text{acc}}$). In this example, ~$N_{{s}_\text{acc}}$ for $b_2$ is equal to $15$ while ~$N_{{s}_\text{acc}}$ for $b_1$ is equal to $10$. These values and the effect of power constraint on the feasible frequency highly influence the optimal batch size to minimize~$TT_{\text{acc}}$. 
In summary, there is no clear indication on how to select the optimal operating points $(f,b)$ to achieve the target goal. 
%These values have led to higher $TT_{\text{acc}}$ at $b_2$ for $P_1$ and $P_2$, but still lower value for $P_3$, due to the impact of the selected frequencies on $T_s$. 

We formulate our optimization problem as follows:
\begin{equation}
    \begin{aligned}
        & \underset{b\in\mathcal{B}, f\in\mathcal{F}}{\text{min}}
        & & TT_{\text{acc}}(b, f, \nn, D) \\
        & \text{subject to}
        & & P(b, f, \nn) \leq P_{\text{max}}
    \end{aligned}
\label{eq:optimization}
\end{equation}
where $\mathcal{B}$ is the set of feasible batch sizes, $\mathcal{F}$ is the set of available GPU's frequencies, and $P(b, f, \nn)$ is the required power to training~$M$ using~$b$ and~$f$. We rewrite ~$TT_{\text{acc}}$ as the multiplication of~$T_{s}$ and ~$N_{{s}_{\text{acc}}}$, we thus have:
\begin{equation}
    TT_{\text{acc}}(b, f, \nn, D) = T_{s}(b, f, \nn) \times  N_{{s}_{\text{acc}}}(b, \nn, D)
    \label{eq:time_to_acc2}
\end{equation}
$s$ is selected, s.t.~$b_{\text{max}} \leq s \leq |D|$, where~$b_{\text{max}}$ is the largest batch size that can fit into the memory of the devices. 
This detached formulation enables our proposed optimization method, presented in \cref{sec:methodology}.
In particular, the first factor~$T_{s}$ does not depend on the training data~$D$, nor on the accuracy threshold. The second factor~$N_{{s}_\text{acc}}$ is independent of the GPU frequency of the device.

%the number of times of processing~$s$ to reach target accuracy ($N_{{s}_{\text{acc}}}$)
%, and therefore, we can perform profiling on the device prior to the training to obtain feasible operating points. The second factor~$N_{{s}_\text{acc}}$ is independent of the GPU frequency of the device.% and hence it can be estimated on the server. %by performing predictions on the server on a proxy data set. 
%The details of our method are discussed in the following section. 


\section{Proof of \np-hardness for the Bayesian Model}
\label{sec:bayes}

We now state one of the main results of this paper---the hardness of \netlearn{}.

\begin{theorem}[ ]\label{thm:nphardness_bayes}
	\netlearn{} with the Bayesian learning rule $\mu = \mu^B$ is \np-hard.
\end{theorem}

\subsection{Proof Idea}

We perform a reduction from \sat{} to \netlearn{}.
Specifically, we assume in our reduction that all formulas have exactly 3 distinct literals in each clause, and never a literal along with its negation.
For a given formula $ \varphi $, we construct a network $ \network $ and an $ \varepsilon > 0 $, such that
\[
    \olr(\network, \mu^B) \geq 1-\varepsilon \quad \iff \quad \text{$ \varphi $ is satisfiable.}
\]

The network $ \network $ consists of a directed graph $ G $, the ground truth prior $ q $, and the prior of the agents' private signals $ p $.
Our reduction requires setting $ p $ and $ G $ based on the formula, but it allows us to set $ q=\frac 12 $, regardless of $ \varphi $.

\subsection{Graph Construction \& Notation}
\label{ssec:graph}

First, we construct the directed graph $G$.
Our construction consists of $ N $ \emph{variable cells}, and $ M $ \emph{clause gadgets}, where $ N $ and $ M $ are the number of variables and clauses in $ \varphi $, respectively.
We define the cell and gadget first, and then define the full graph in \Cref{def:bayesian_graph}.

\begin{restatable}[Variable cell]{definition}{variableCell}
\label{def:cell}
    Let $ x $ be a variable of a formula $ \varphi $.
    A \emph{variable cell} of $ x $ is a directed graph $ \cell x = \left( V_x, E_x \right) $, where \begin{enumerate}[ ]
    	\item $ V_x = \left\{ x, \lnot x, d_x \right\} $, and
	\item $ E_x = \left\{ \left( d_x, x \right), \left( d_x, \lnot x \right), \left( x, \lnot x \right), \left( \lnot x, x \right) \right\} $.
    \end{enumerate}
\end{restatable}

\Cref{fig:cell_bayesian} depicts a cell for some variable $x_i$.

\begin{definition}[Clause gadget]
    Let $ C = j \lor k \lor \ell $ be a clause of a 3-CNF formula $ \varphi $, where $ j \neq k \neq \ell $ are some literals.
    Then the \emph{clause gadget} is $ \gadget C = \left( V_C, E_C \right) $, where \begin{enumerate}[ ]
    	\item $ V_C = \left\{ j,k,\ell \right\} $, and
    	\item $ E_C = \left\{ \left( x,y \right) \suchthat x,y \in \left\{ j,k,\ell \right\} \land x \neq y \right\} $.
    \end{enumerate}
\end{definition}

\begin{restatable}[Formula graph]{definition}{bayesianGraph}\label{def:bayesian_graph}
	Let $ \varphi = C_1 \land C_2 \land \dots \land C_M $ be a CNF formula of variables $ \vars = \left\{ x_1, \ldots, x_N \right\} $, where each clause $ C_i $ is the disjunction of exactly three literals.
	Then $ G_\varphi $ is a disjoint union of the graphs $ \cell {x_i} $ for all $ {x_i} \in \vars $, and $ \gadget {C_i} $ for all clauses $ C_i \in \varphi $.
        Additionally, there is an edge to each of the vertices in every $ \gadget {C_i} $ from the corresponding literal nodes in the respective variable cell.
\end{restatable}

Note that that there are no incoming edges to any cell, except from within the same cell, so the learning rate of each cell is determined only by the ordering of its own vertices. Also, no two clause gadgets share any vertices, so the ordering of vertices within a gadget only affects that gadget.
For an illustration of the formula graph construction, refer to \Cref{fig:gphi_bayesian}, where we give a sample formula graph for the formula $ \varphi = x \lor y \lor \lnot z $.

\begin{figure}[t!]
	\centering
	\begin{tikzpicture}[
		xsh/.style = { xshift=10mm },
		node distance = 5mm and 5mm,
	stff/.style={circle, draw=black, \figThickness, minimum size=\figCircSize, inner sep=0pt},
	]
		%Nodes
		\node[stff]        (x)                  {$ x_i $};
		\node[stff]        (nx)   [below=of x]   {$ \lnot x_i $};
		\node[stff]        (d)   [left=of x,yshift=-6mm]   {$ d_i $};

		%Lines
		\draw[->, \figThickness] (d)  to  (x);
		\draw[->, \figThickness] (d)  to (nx);
		\draw[<->, \figThickness] (x)  to (nx);
	\end{tikzpicture}
 \Description[A graph of three vertices, connected together]{A graph containing vertices x and non-x, connected both-ways, along with a degree d, from which two edges go to x and non-x, but not the other way around.}
	\caption{The cell for variable $ x_i $.}
 \label{fig:cell_bayesian}
\end{figure}

\input{src/figures/bayes_gadget}

\ourparagraph{Ordering-assignment relation}\label{par:ordering_literals}
To determine satisfiability of $\varphi$ from the learning rate of $ G_\varphi $, we map vertex orderings to variable assignments.
We then show that orderings achieving higher learning rates correspond to assignments with more satisfied clauses.
For a more detailed description of both the mapping and its properties, see \Cref{ssec:assignment_bayes}.

For clarity, we introduce some more notation. Let $ \ell $ be a literal of some variable $ x \in \vars $, meaning $ \ell = x $ or $ \ell = \lnot x $.
We say that cell $\cell x$ is in one of two states: it is ``on'' under a decision ordering $\sigma$ if $\sigma(\lnot x) < \sigma(x)$; otherwise, cell $\cell x$ is ``off''.
We say that the literal $ \ell $ is ``on'' if $ \ell = x $ and $ \cell x $ is on, or $ \ell = \lnot x $ and $ \cell x $ is off; otherwise, literal $ \ell $ is ``off''.
For brevity, we further denote $ \cell \ell \deq \cell x $, regardless of whether $ \ell = x $ or $ \ell = \lnot x $.

\subsection{Gadget Learning Rates}
This section lists the learning rates of the cells and clause gadgets under Bayesian aggregation. We assume WLOG for this section that $\theta = 1$, and compute all probabilities in this section conditioned on $\theta = 1$. Since we always take $\theta$ to be uniform on $\{0,1\}$, this yields the same values as taking the probability over $\theta$ as well. We note that our computations were verified using Wolfram Mathematica.

We begin by examining the learning rate of an arbitrary cell under a pair of orderings in which the cell is either ``on'' or ``off''. The following lemma shows that cells achieve the same learning rate under either of these orderings, and that this learning rate is the best possible over all orderings. 

\begin{lemma}[Bayesian Cell LR] \label{lemma:bayesian_cellLearningRate}
    Let $x \in \vars$.
    Let $ q = \frac 12 $, and $ p > \frac 12 $ be given.
    Then \[
	\oclr (\cell x) = \tfrac 52 p + \tfrac 32 p^2 - p^3.
    \]
    In particular, if under an optimal ordering $ \sigma^* $ a literal $ \ell $ is on, then its corresponding literal node has learning rate $ \clr (\ell, \sigma^*, \mu^B) = \frac p2 + \frac 32 p^2 - p^3 $; otherwise $ \clr (\ell, \sigma^*, \mu^B) = p $.
\end{lemma}

\begin{proof}
    First, observe that for the optimal ordering $ \sigma^* $, it is always beneficial to put the dummy node $ d_x $ \emph{before} the nodes $ x $ and $ \lnot x $.
    This is because $ d_x $ has no incoming edges, so it cannot acquire more information by going later, and it has edges going to $ x $ and $ \lnot x $, which can only increase their chances of getting the correct answer.
    Hence, we can see that $ \oclr(d_x) = p $.

    The case of the remaining two nodes is symmetric, so WLOG, let us assume that $ \sigma^*(x) < \sigma^*(\lnot x) $ (so the cell $ \cell x $ is ``off'').
    The node $ x $ then receives the action of $ d_x $, which is i.i.d. from its private information.
    So node $ x $ chooses its action correctly either if both $ s_x $ and $ a_{d_x} $ are correct, where $a_{d_x}$ is the action chosen by node $d_x$, or if exactly one of the two are correct and $ x $ tiebreaks correctly. The first outcome occurs with probability $ p^2 $, and the second with probability $ 2p (1-p) \frac 12 $.
    Thus, the learning rate of node $ x $ is \[
        \oclr(x) = p^2 + p - p^2 = p.
    \]

    Finally, the node $ \lnot x $ receives its private signal, $ s_{\lnot x} $ and the actions $ a_{d_x} $ and $ a_x $.
    Notice that these three pieces of information are \emph{not} independent, since $ x $ was influenced by $ d_x $.
    We perform case analysis on the relative likelihood \[
        \Lambda \deq \frac{\pr{\theta = 1 \mid X_{\lnot x}}}{\pr{\theta = 0 \mid X_{\lnot x}}} = \frac{\pr{X_{\lnot x} \mid \theta = 1}}{\pr{X_{\lnot x} \mid \theta = 0}},
    \]
    where the equality follows from Bayes' theorem and from the fact that the prior $ q = \frac 12 $.
    Recall that $ X_{\lnot x} = \{s_{\lnot x}, a_x, a_{d_x}\} $, and that $ s_{\lnot x} $ is independent of the other two.
    Thus, $ \Lambda $ can be expressed as
    \begin{align*}
        \Lambda = \frac{\pr{s_{\lnot x} \mid \theta = 1}}{\pr{s_{\lnot x} \mid \theta = 0}} \cdot
        \frac{\pr{a_{d_x} \mid \theta = 1}}{\pr{a_{d_x} \mid \theta = 0}} \cdot
        \frac{\pr{a_x \mid \theta = 1, a_{d_x}}}{\pr{a_x \mid \theta = 0, a_{d_x}}}.
    \end{align*}
    We compute $ \Lambda $ for each case of $(s_{\lnot x}, a_x, a_{d_x})$:
    \begin{enumerate}
        \item $ \left( 1,1,1 \right) $: \[
                \Lambda = \tfrac p{1-p} \cdot \tfrac p{1-p} \cdot \tfrac{p + (1-p)/2}{(1-p) + p/2} = \tfrac {p^2}{(1-p)^2} \cdot \tfrac {1/2+ p/2}{1- p/2} > 1.
        \]
        \item $ \left( 1,1,0 \right) $: \[
                \Lambda = \tfrac p{1-p} \cdot \tfrac {1-p}p \cdot \tfrac{p/2}{(1-p)/2} = \tfrac {p}{(1-p)} > 1.
        \]

        \item $ \left( 1,0,1 \right) $: \[
                \Lambda = \tfrac p{1-p} \cdot \tfrac p{1-p} \cdot \tfrac{(1-p) /2}{p/2} = \tfrac {p}{(1-p)} > 1.
        \]

        \item $ \left( 1,0,0 \right) $: \[
                \Lambda = \tfrac p{1-p} \cdot \tfrac {1-p}p \cdot \tfrac{(1-p) + p/2}{p + (1-p) /2} =  \tfrac {1-p/2}{1/2+ p/2} < 1.
        \]
    \end{enumerate}
    The inequalities hold for $ \tfrac 12 < p < 1 $.
    The remaining cases (that is $(0,1,1)$, $(0,1,0)$, $(0,0,1)$, $(0,0,0)$) follow from symmetry.
    It is now clear that if $ \theta = 1 $, $ a_{\lnot x} $ is correct in cases 1, 2, 3, 5.
    We now compute $ \oclr(\lnot x) $, which is the probability of cases 1, 2, 3, or 5 occurring. \begin{align*}
        \oclr(\lnot x) &= \pr{X_{\lnot x} \in \left\{ \left( 1,1,1 \right) ,  \left( 1,0,1 \right) ,  \left( 1,1,0 \right) ,  \left( 0,1,1 \right) \right\}} \\
                       &= p ( p + (1-p)\tfrac p2)) + (1-p)p(p + (1-p)\tfrac 12) \\
                       &= \tfrac p2 + \tfrac 32 p^2 - p^3.
    \end{align*}

    The cumulative learning rate of the clause gadget is then the sum of the above, \[
        \oclr(\cell x) = \tfrac 52 p + \tfrac 32 p^2 - p^3.
    \]
\end{proof}

To summarize, there exist two orderings for each cell which yield the same cell learning rate.
The two orderings place $d_x$ first, and correspond to either the ``on'' state (when $ \sigma(x) < \sigma(\lnot x) $) or the ``off'' state ($ \sigma(x) > \sigma(\lnot x) $).
For now, we will defer the question of which of the two is optimal in the overall network ordering.
Addressing this first requires examining the learning rates of the clause gadgets.
Since the graph contains edges from cells to clause gadgets, there is an optimal ordering which orders all cell nodes before all clauses gadgets.
We begin by examining the learning rate for arbitrary clause gadgets. 

\begin{lemma}\label{lemma:bayesian_clause}
    Let $ C = \alpha \lor \beta \lor \gamma $ be a clause of $ \varphi $.
    Let $ \sigma^* $ be an optimal ordering on $ \network $.
    Let $p'\deq\frac p2 + \frac 32 p^2 - p^3 $.
    Then if under $ \sigma^* $, exactly $ i $ cells of $ \alpha, \beta, \gamma $ are ``on'', then $ \clr(\gadget C, \sigma^*) = \clr_i $, where \begin{align*}
        \pzero &\deq p (2 p^4-5 p^3+5 p+1). \\
        \pone &\deq  p^4 (2 p'-1)+p^3 (2-4 p')+p^2 (1-2 p')+4 p p'+p', \\
        \ptwo &\deq 4 p^3 (p'-1) p'+p^2 (-6 (p')^2+4 p'+1)+2 p p'+p' (p'+1), \\
        \pthree &\deq p' (p^2 (2 (p')^2-3 p'+1)-p (2 (p')^2+p'-3)+2 p'+1).
    \end{align*}
    Furthermore, if $ D \in \varphi $ is a satisfied clause under the ordering $ \sigma^* $, meaning at least one of the literals of $ D $ is ``on'' under $ \sigma^* $, then 
    \[
        \pthree \geq \clr(\gadget D, \sigma^*) \geq \pone \geq \pzero.
    \]
\end{lemma}

\begin{figure}[t!]
	\centering
\begin{tikzpicture}
	\begin{axis}[
    xlabel={$ p $},
    ylabel={$ \clr(\gadget C) $},
    legend pos=south east,
    axis lines=left,
]

\addplot [
    domain=1/2:1, 
    samples=\figSamples,
    color=red,
    smooth
]
{x*(2+14*x+27*x^2-6*x^3-59*x^4-7*x^5+62*x^6+21*x^7-78*x^8+44*x^9-8*x^10)/4};
\addlegendentry{\(\pthree\)}

\addplot [
    domain=1/2:1, 
    samples=\figSamples,
    color=green,
    smooth
]
{x*(2+15*x+22*x^2+7*x^3-84*x^4+14*x^5+92*x^6-72*x^7+16*x^8)/4};
\addlegendentry{\(\ptwo\)}

\addplot [
    domain=1/2:1, 
    samples=\figSamples,
    color=blue,
    smooth
]
{(x+9*x^2+12*x^3-20*x^4-6*x^5+14*x^6-4*x^7)/2};
\addlegendentry{\(\pone\)}

\addplot [
    domain=1/2:1, 
    samples=\figSamples,
    color=black,
    smooth
]
{x*(1+5*x-5*x^3+2*x^4)};
\addlegendentry{\(\pzero\)}

\end{axis}
\end{tikzpicture}
\caption{The relationship between learning rates of $ \gadget C $, depending on the number of ``on'' literals, as a function of $ p $.}
\Description{All are equal to $\frac 12$ for $p=\frac 12$, and $1$ for $p=1$. In the interval between those two values, $\pzero \leq \pone \leq \ptwo \leq \pthree$.}
\label{fig:bayes_ps}
\end{figure}


\begin{proof}[Proof Idea.]
    The full proof is long and technical.
    Here we offer the main idea, which is similar to that of \Cref{lemma:bayesian_cellLearningRate}.

    The actions of the nodes in $ \gadget C $ are exactly determined by \begin{enumerate}[ ]
        \item the ordering of the nodes in $ \gadget C $,
        \item the private signals of the nodes in $ \gadget C $,
        \item the actions taken by $ \alpha, \beta, \gamma $, as well as $ \alpha, \beta, \gamma $ being on or off.
    \end{enumerate}
    We compute the learning rate in each case.
    We then compute the expected value over the private signals and the actions of $ \alpha,\beta,\gamma $, the probabilities of which are given by \Cref{lemma:bayesian_cellLearningRate}.

    The resulting cumulative learning rate of $ \gadget C $ depends only on the states of $ \alpha, \beta, \gamma $.
    Among these are the learning rates $ \pone $ and $ \pthree $, which are the learning rates of the clause gadget when one or all of the literals are on, respectively.
     
    The case of $\pone$ actually corresponds to three sub-cases, depending on whether the node corresponding to the ``on'' literal is first, second, or third in the ordering. Notice that swapping the ordering of the three vertices inside the clause gadget does not affect the learning rate of other vertices in the network. Thus, since $ \sigma^* $ is an optimal ordering, it maximizes the learning rate for this clause, and as such, $\pone$ is the maximum over the three subcases.

    The full proof can be found in \appbayes.
\end{proof}

See \Cref{fig:bayes_ps} for the relationship between the different values as a function of $ p $.
Notice that $ \pthree,\ptwo,\pone$ and $\pzero $ converge as $p$ approaches $1$ and as it approaches $\frac 12$, regardless of the graph topology and ordering.
This is exactly what we expect: if $ p=1 $, the agents have perfect information from their private signals alone, while if $ p=\frac 12 $, then the private signals give the agents no extra information, and thus their the LR approaches $ \frac 12 $.


\subsection{Optimal Ordering \& Restrictions on p}
\label{ssec:assignment_bayes}
We can now determine the optimal ordering by examining the learning rates derived in the previous section. First, we define an \emph{induced ordering} below:

\begin{definition}
    Let $\mathcal{A}: \chi \to \{0,1\}$ be any assignment of values to variables. Define the (partial) \emph{ordering $\sigma(\mathcal{A})$ induced by $\mathcal{A}$} as follows: if $\mathcal{A}(x_i)=1$, then cell $\cell {x_i}$ is on; otherwise, $\cell {x_i}$ is off.
\end{definition}

In particular, the above definition gives a bijection between assignments and partial ordering over variables. We further say that a total ordering $\sigma$ \emph{respects} an assignment $\mathcal{A}$ (denoted by $\sigma \sim \mathcal{A}$) if it contains $\sigma(\mathcal{A})$ as a partial ordering over variable nodes. We also write $\sigma^*(\mathcal{A}) \deq \arg\max_{\sigma \sim \mathcal{A}} \mathcal{L}(\mathcal{N}, \sigma, \mu^B)$.

\begin{definition} \label{def:max_assignment}
    Let $\mathcal{A}: \chi \to \{0,1\}$ be an assignment of values to variables maximizing the number of satisfied clauses. Then $\mathcal{A}$ is a \emph{maximal assignment}.
\end{definition}

\begin{lemma}[Optimal Ordering] \label{lemma:bayes_bestOrder}
    Let $\mathcal{A}^*$ be a maximal assignment.  Let $p(M) \deq (3M-4)/(3M-3)$ be a threshold probability determined by $M$, the number of clauses. Then for all $p \geq p(M) $, $\sigma^*(\mathcal{A}^*)$ is an optimal ordering.
\end{lemma}
\begin{proof}
    Note that an assignment-induced ordering only specifies whether each cell is on or off. From \Cref{lemma:bayesian_cellLearningRate},  an optimal cell ordering exists both if the cell is on or off, and yields the same learning rate in either case. Therefore, the optimal ordering is determined by comparing clause learning rates. 
    
    Also note that any total ordering must respect some assignment. So we argue the optimality of $\sigma^*(\mathcal{A}^*)$ by comparing it to $\sigma^*(\mathcal{A}')$ for any non-maximal $\mathcal{A'}$. Let $S^*$ be the number of satisfied clauses under $\mathcal{A}^*$ and $S' < S^*$ that under $\mathcal{A'}$. By \Cref{lemma:bayesian_clause}, for any clause satisfied under an arbitrary assignment $\mathcal{A}$, the corresponding clause gadget under an ordering $\sigma \sim \mathcal{A}$ achieves learning rate lower bounded by $\pone$. Further, by \Cref{lemma:bayesian_clause}, $\pone >\pzero$,  so having more satisfied clauses in an assignment can never decrease the CLR under the induced ordering. However, note also that $\pthree >\pone$  (by \Cref{lemma:bayesian_clause}). Hence, in the most extreme case, all $S'$ satisfied clauses under $\mathcal{A}'$ are satisfied with three true literals, while all $S^*$ satisfied clauses under $\mathcal{A}^*$ are satisfied with one true literal. 

    Consider that extreme case, and further impose the worst-case choices of $S'$ and $S^*$ by setting $S^* = M$ and $S' = M-1$. Then over all clause gadgets, $\sigma^*(\mathcal{A}^*)$ gives a CLR of $M \pone$, $\sigma^*(\mathcal{A}')$ gives a CLR of $(M-1) \pthree + \pzero$. We can now solve for conditions on $p$ such that $\sigma^*(\mathcal{A}^*)$ achieves a higher network learning rate:
    \begin{align*}
        M \pone &> (M-1) \pthree + \pzero\\
        \tfrac{\pone - \pzero}{\pthree - \pone} &> M-1.
    \end{align*}
    \begin{figure}[t!]
	\centering
\begin{tikzpicture}
	\begin{axis}[
    xlabel={$ p $},
    ylabel={$ \clr $},
    axis lines=left,
    ymax=0.09,
    scaled ticks=false,
    yticklabel style={
	/pgf/number format/precision=3,
	/pgf/number format/fixed,
    }
]

\addplot [
    domain=1/2:1, 
    samples=\figSamples,
    color=blue,
    smooth
]
{-(1/2)*x*(1+x-12*x^2+10*x^3+10*x^4-14*x^5+4*x^6)};
\addlegendentry{\(\pone-\pzero\)}

\addplot [
    domain=1/2:1, 
    samples=\figSamples,
    color=red,
    smooth
]
{-(1/4)*(-1+x)^2*x^2*(4+5*x-28*x^2-14*x^3+35*x^4+14*x^5-28*x^6+8*x^7)};
\addlegendentry{\(\pthree-\pone\)}

\end{axis}
\end{tikzpicture}
\caption{The plot of the numerator and the denominator of the condition on $ M-1 $ from \Cref{lemma:bayesian_clause}.
The denominator approaches zero noticably faster, suggesting the fraction approaches infinity as $ p $ goes to 1.}
\label{fig:pdff}
\Description{The difference $\pone-\pzero$ is always strictly larger than $\pthree-\pone$. As $p$ goes to 1, their ratio grows.}
\end{figure}

    The numerator and denominator is plotted in \Cref{fig:pdff} as a function of $ p $.
    Since the left-hand side can be lower bounded by $\frac{1}{3-3p}$ for all $p \in (0.5,1)$, it is sufficient to have $ \frac 1{3-3p} \geq M-1 $.
    We can thus define $ p(M) \deq \frac{3M-4}{3M-3} $, which respects the condition. Note that $p(M)$ is well-defined for any $ M \geq 2 $, and lies in the interval $ (\frac 12, 1) $.
\end{proof}

To recap, given any instance $\varphi$ of \sat{} with $N$ variables and $M$ clauses, we can construct a network $G_\varphi$ with ground truth prior $q = 1/2$ and signal accuracy $p = p(M)$. In particular, whenever $\varphi$ is satisfiable under some maximal assignment $\mathcal{A}^*$, there is an optimal decision ordering $\sigma^*$ which respects the ordering induced by $\mathcal{A}^*$, and which achieves a network CLR of at least $N(2p+3p^2-2p^3) + M\pone$. Otherwise, if $\varphi$ is non-satisfiable under any maximal assignment $\mathcal{A}^*$, then the optimal decision ordering $\sigma^*$ respecting the ordering induced by $\mathcal{A}^*$ achieves a network CLR of no more than $N(2p+3p^2-2p^3) + (M-1)\pthree + \pzero$. Choosing $p = p(M)$ allowed us to show the optimality of $\sigma^*$, as well as to separate the learning rates in networks corresponding to satisfiable and non-satisfiable formulas. All that remains to complete the reduction is to pick an appropriate choice of $\varepsilon$, such that a formula graph $G_\varphi$ achieves expected network learning rate greater than $\varepsilon$ under an optimal ordering iff $\varphi$ is satisfiable.

\subsection{Picking the Epsilon}\label{sec:bayes_epsilon}
We will simply pick $\varepsilon$ to lie exactly halfway between the satisfiable and non-satisfiable network learning rates. Recalling that each variable and clause gadget contains $3$ vertices, we can compute the learning rates below. For networks corresponding to satisfiable formulas, we have $\frac{N\pcell + M\pone}{3(N+M)},$ and for networks corresponding to non-satisfiable formulas, we have $\frac{N\pcell + (M-1)\pthree + \pzero}{3(N+M)}.$ This gives 
\begin{align*}
    \varepsilon = \frac{1}{2}\left(\frac{2N\pcell + M\pone + (M-1)\pthree + \pzero}{3(N+M)}\right),
\end{align*}
thus completing the reduction. 

\subsection{Generalization for a constant range of p}

Finally, we remark that, while this construction proves the Theorem, it uses $ p $ approaching 1 as the size of $ \varphi $ increases.
A natural question, then, is whether it is hard to compute the maximum learning rate for other values of $p$, especially when $ p $ is a fixed constant.
In fact, \Cref{sec:approx} shows a stronger version of \Cref{thm:nphardness_bayes}, stated as follows. 

\begin{theorem}[ ]
	\netlearn{} with Bayesian learning rule $\mu = \mu^B$ and a fixed $ p \in ( \sqrt{7/8},1 ) $ is \np-hard.
\end{theorem}

This follows as a direct corollary of \Cref{thm:approx}.


\section{Proof of \np-hardness for Majority Dynamics}
\label{sec:maj}


In this section, we adapt the statement of \Cref{thm:nphardness_bayes} to the Majority dynamics setting.

\begin{theorem}[ ]\label{thm:nphardness_maj}
	\netlearn{} with the majority vote rule $\mu = \mu^M$ is \np-hard.
\end{theorem}

The proof is again a reduction from \sat.
The main idea is identical to that of \Cref{thm:nphardness_bayes}, so we offer here only the main points and construction, with emphasis on differences from the proof of \Cref{thm:nphardness_bayes}.
We offer the full proof in \appmaj.

\subsection{Adapted Graph Construction}
\label{ssec:maj_graph}

Unfortunately, if we wanted to directly apply the previous construction, we would find that the worst-case satisfied ordering reaches a \emph{lower} learning rate than the best-case unsatisfied ordering (as discussed in \Cref{ssec:assignment_bayes}).
Then there is no $ \varepsilon $ which separates the learning rates corresponding to satisfied and unsatisfied assignments.
We thus adapt the construction from \Cref{ssec:graph}, modifying the clause gadget.
We keep the variable cell unchanged.

\variableCell*

Intuitively, we need to give the nodes in the clause gadgets more input, so they are better equipped to use the input from the ``on'' cells.
To achieve this, we add two dummy nodes to the clause gadget.

\begin{definition}[Clause gadget]
    Let $ C = j \lor k \lor \ell $ be a clause of a 3-CNF formula $ \varphi $, where $ j \neq k \neq \ell $ are some literals.
    Then the \emph{clause gadget} is $ \gadget C = \left( V_C, E_C \right) $, where \begin{enumerate}[ ]
    	\item $ V_C = \left\{ j,k,\ell, d^1, d^2 \right\} $,
    	\item $ E_C = \{ ( x,y ) \mid x,y \in \{ j,k,\ell \}, x \neq y \} \cup \{ ( d^i, x ) \mid i \in \{ 1,2 \} , x \in \{ j,k,\ell \} \} $.
    \end{enumerate}
\end{definition}

We now define the formula graph using the same definition, only with the new clause gadget.

\bayesianGraph*

See \Cref{fig:gphi} for an illustration of this construction for the simple formula $ \varphi = x \lor y \lor z $.
Note that we also use the ``on''/``off'' states of a variable/literal, as defined for the Bayesian proof (see \Cref{par:ordering_literals}).

\begin{figure}[t!]
	\centering
	\begin{tikzpicture}[
		xsh/.style = { yshift=-9mm },
		xxsh/.style = { yshift=-8mm },
		dcs/.style = { yshift=-3mm,xshift=-6mm },
		node distance = 5mm and 5mm,
	stff/.style={circle, draw=black, \figThickness, minimum size=\figCircSize, inner sep=0pt},
	]
		%Nodes
		\node[stff]        (x_i)                  {$ x $};
		\node[stff]        (nx_i)   [left=of x_i]   {$ \lnot x $};
		\node[stff]        (d_i)   [above=of x_i,dcs]   {$ d_x $};

		\node[stff]        (x_j)    [left=of nx_i]  {$ y $};
		\node[stff]        (nx_j)   [left=of x_j]   {$ \lnot y $};
		\node[stff]        (d_j)   [above=of x_j,dcs]   {$ d_y $};

		\node[stff]        (x_k)    [left=of nx_j]  {$ z $};
		\node[stff]        (nx_k)   [left=of x_k]   {$ \lnot z $};
		\node[stff]        (d_k)   [above=of x_k,dcs]   {$ d_z $};

		\node[stff]        (l_m^1)    [below=of x_i,xsh]  {$ x_C $};
		\node[stff]        (l_m^2)    [below=of x_j]  {$ y_C $};
		\node[stff]        (l_m^3)    [below=of nx_k,xsh]  {$ \lnot z_C $};
		\node[stff]        (d1)    [below=of l_m^2, xxsh,xshift=20]  {$ d_C^1 $};
		\node[stff]        (d2)    [below=of l_m^2, xxsh,xshift=-40]  {$ d_C^2 $};

		%Lines
		\draw[->, \figThickness] (d_i)  to  (x_i);
		\draw[->, \figThickness] (d_i)  to (nx_i);
		\draw[<->, \figThickness] (x_i)  to (nx_i);

		\draw[->, \figThickness] (d_j)  to  (x_j);
		\draw[->, \figThickness] (d_j)  to (nx_j);
		\draw[<->, \figThickness] (x_j)  to (nx_j);

		\draw[->, \figThickness] (d_k)  to  (x_k);
		\draw[->, \figThickness] (d_k)  to (nx_k);
		\draw[<->, \figThickness] (x_k)  to (nx_k);

		\draw[->, \figThickness] (x_i)  to (l_m^1);
		\draw[->, \figThickness] (x_j)  to (l_m^2);
		\draw[->, \figThickness] (nx_k)  to (l_m^3);

		\draw[<-, \figThickness] (l_m^1)  to  (d1);
		\draw[<-, \figThickness] (l_m^2)  to  (d1);
		\draw[<-, \figThickness] (l_m^3)  to  (d1);
		\draw[<-, \figThickness] (l_m^1)  to  (d2);
		\draw[<-, \figThickness] (l_m^2)  to  (d2);
		\draw[<-, \figThickness] (l_m^3)  to  (d2);
		\draw[<->, \figThickness] (l_m^1)  to (l_m^2);
		\draw[<->, \figThickness] (l_m^1)  to (l_m^3);
		\draw[<->, \figThickness] (l_m^2)  to (l_m^3);
	\end{tikzpicture}
	\caption{The graph $ G_\varphi $ for $ \varphi = C = \left( x \lor y \lor \lnot z \right)$.}
 \Description{A construction for the Majority vote reduction, as described in \Cref{def:bayesian_graph} for a simple formula.}
	\label{fig:gphi}
\end{figure}


Next, we compute the learning rates of the variable cell, and of the clause gadgets, depending on whether its literals are on or off.

\begin{lemma}[Majority Dynamics Cell LR] \label{lemma:maj_MD_cellLearningRate}
    Let $x \in \vars$.
    Let $ q = \frac 12 $, and $ p $ be given.
    Then \[
	\oclr (\cell x) = 2p + 3p^2 -2p^3.
    \]
\end{lemma}

\begin{lemma}[Majority Dynamics Gadget Learning Rate]\label{lemma:maj_100CLR} 
    Let $C$ be a clause.
    Suppose that $\sigma^*$ is an optimal learning rate.
    Then, in the gadget for $C$, $\sigma^*$ places the cells first, then the dummy nodes, and finally the three literal nodes.
    Further, \begin{enumerate}
        \item if one literal is on, then $\clr(\gadget C, \sigma^*) $ is
        \begin{align*}
                \pone \deq p &\left(2 + 2 p + 6 p^2 + 11 p^3 + 4 p^4 - 51 p^5 - 6 p^6  + 21 p^7 \right. \\
                               & \left. + 115 p^8 - 136 p^9 + 13 p^{10} + 36 p^{11} - 12 p^{12} \right).
            \end{align*}
        \item if one literal is on, then $\clr(\gadget C, \sigma^*)$ is \begin{align*}
                \ptwo \deq p & \left(12 p^8-54 p^7+76 p^6-14 p^5 \right. \\
                               & \left. -40 p^4+9 p^3+12 p^2+2 p+2\right).
            \end{align*}
        \item if one literal is on, then $\clr(\gadget C, \sigma^*) $ is \begin{align*}
                \pthree \deq p & \left(2 + 2 p + 3 p^2 + 14 p^3 + 22 p^4 - 66 p^5 - 69 p^6 + 310 p^7 \right.\\
                            & \left.- 688 p^8 + 710 p^9 + 756 p^{10} - 2581 p^{11} + 2304 p^{12} - 558 p^{13}\right.\\ 
                            & \left.- 372 p^{14} + 264 p^{15} - 48 p^{16} \right).
            \end{align*}
    \end{enumerate}
    Furthermore, for a clause $ D $ satisfied under $ \sigma^* $, it holds \[
	    \pthree \geq \clr(\gadget D, \sigma^*) \geq \pone \geq \pzero.
	\]
\end{lemma}

Finally, we can now determine the optimal ordering, compute its learning rate, and show that there is an $ \varepsilon $ such that the learning rate is above $ \varepsilon $ if and only if the induced ordering is satisfied.

\begin{lemma}[Optimal Ordering] \label{lemma:maj_bestOrder}
    Let $\mathcal{A}^*$ be a maximal assignment.  Let $p(M) < 1$ be a threshold probability determined by $M$, the number of clauses. Then for all $p \geq p(M)$, the decision ordering $\sigma^*$ which places all dummy nodes first, then all variable nodes respecting the partial ordering induced by $\mathcal{A}^*$, and finally all literal nodes in the clause gadgets, maximizes the network learning rate.
\end{lemma}

The proof of this Lemma is very similar that of \Cref{lemma:bayes_bestOrder}, and is included in \appmaj.

We now apply the same reasoning as in \Cref{sec:bayes_epsilon}.
If $ p = p(M) $, then the worst-case learning rate of a satisfying assignment is $\frac{N\pcell + M\pone}{3N + 5M},$ strictly higher than the best-case non-satisfying LR, $\frac{N\pcell + (M-1)\pthree + \pzero}{3N + 5M}$. We can thus define the threshold (mind the new number of vertices---$ G_\varphi $ now has 5 for each clause):
\begin{align*}
    \varepsilon = \frac{1}{2}\left(\frac{2N\pcell + M\pone + (M-1)\pthree + \pzero}{3N + 5M}\right).
\end{align*}
This concludes the proof.

\section{Approximating the Optimal LR}
\label{sec:approx}

In this section, we extend the ideas from the 3-SAT reduction 
to show that even approximating a solution to \netlearn{} is hard.
More precisely, we show hardness for the optimization version of the problem, defined in \Cref{sec:problem} as \netlearnopt{}.
We give a proof for the Bayesian learning rule; we believe for the proof for majority vote to be similar.

\begin{theorem}[ ]\label{thm:approx}
    \netlearnopt{} with the Bayesian inference $\mu = \mu^B$ is \apx-hard. 
\end{theorem}

\begin{proof}
    We perform a PTAS reduction to \maxsat, where again each clause has exactly three literals (also referred to as \problem{Max E3-SAT}), which is known to be \apx-hard.
    In particular, we show that an approximation scheme of $ \olr(\network) $ implies an approximation scheme of \maxsat{}.

    Given an instance of \maxsat, we construct the graph $ G_\varphi $ from \Cref{def:bayesian_graph}.
    Let $ \asg^* $ be a maximal assignment, defined in \Cref{def:max_assignment}, satisfying $ M^* $ clauses.
    To prove the theorem, it suffices to show that for every $ \delta \in \left( 0,1 \right) $, there is an $ \alpha \in \left( 0,1 \right) $ such that if an ordering $ \sigma $ achieves a LR which is $ \alpha  $-close to the optimum, then the induced assignment $ \asg(\sigma) $ satisfies $ \delta M^*$ clauses.
    Abusing notation, for an assignment $ \asg $ we denote $ \clr(\asg) \deq \clr(\network,\sigma^*(\asg)) $.
    
    Note that for any 3-CNF formula, assigning truth values independently and uniformly at random satisfies $\frac 7 8$ of the clauses in expectation, implying that $ M^* \geq \frac 78 M $~\cite{Hastad2001-fg}. 
    From \Cref{lemma:bayesian_clause}, it follows that unsatisfied clauses receive optimal CLR $ \pzero $, and satisfied clauses receive at least $ \pone $ and at most $ \pthree $.
    It then follows that \begin{align*}
         \clr(\asg^*)&\geq M^*\pone+(M-M^*)\pzero+N \pcell.
    \end{align*}
    Let $ \asg' $ be an assignment which achieves $ \lr(\asg') \geq \olr(\network) - \varepsilon \geq \lr(\asg^*)  - \varepsilon $ for some $ \varepsilon > 0 $ specified later.
    Multiplying by the number of vertices in $ G_\varphi $, we get \[
        \clr(\asg') \geq \clr(\asg^*) - \varepsilon (3M+3N) \geq \clr(\asg^*) - 6\varepsilon M,
    \]
    where the second inequality holds since WLOG $ N \leq M $, by duplication of clauses.
    Re-arranging, we have that the number of satisfied clauses under $\asg'$ is
    $$M'\geq \tfrac{\pone - \pzero}{\pthree - \pzero}M^*-\tfrac{6\eps M}{\pthree - \pzero} \geq M^*\left( \tfrac{\pone - \pzero}{\pthree - \pzero}-\tfrac{48\eps}{7(\pthree - \pzero)}\right), $$
    where the second inequality holds since $M^*\geq \frac 78 M$.
    If we now set the final expression equal to $ \delta M^* $, we get an equality in $ \varepsilon $ and $ p $.
    Solving for $ \varepsilon $, we get
    \begin{equation}\label{eq:eps}
        \eps=\tfrac{7}{48}( (\pone - \pzero)-\delta(\pthree - \pzero)).
    \end{equation}
    This is a function of $ p $ and $ \delta $. 
    We only require that this $\varepsilon > 0$.
    Using that $p \in (\frac 12,1)$, this gives us \[
    \delta < \tfrac{4 p^4-8 p^3-4 p^2+8 p+2}{4 p^8-16 p^7+13 p^6+17 p^5-12 p^4-23 p^3+5 p^2+12 p+2}.
    \]
    On $p \in (\frac 12, 1)$, the RHS can be strictly lower-bounded by $p^2$. It thus suffices to set $p \geq \sqrt \delta$.
    
    This proves that an approximation to \netlearnopt{} within an additive bound $ \varepsilon $ for any $ p \geq \sqrt \delta$ implies an approximation to \maxsat{} within a factor of $\delta$. 
    Converting $\varepsilon$ to a multiplicative bound, since $ \olr \in (0,1) $, \[
        \olr - \varepsilon = \left( 1- \tfrac{\varepsilon}{\olr} \right)\olr \leq \left( 1-\varepsilon \right) \olr.
    \]
    Thus we also have a multiplicative bound $ \alpha = (1-\varepsilon) $ for each $ \delta $, which is what we wanted to prove. 
\end{proof}


\begin{corollary}[ ]
    For any fixed constant $ p \in (\sqrt{ {7}/{8}}, 1) $, an $ \alpha $-approximation of \netlearnopt{} with Bayesian inference rule is \np-hard for every $ \alpha > \alpha(p) $, defined as \[
        \alpha(p) \deq 1-\tfrac{7}{48}( (\pone - \pzero)-\tfrac 78(\pthree - \pzero)).
    \]
\end{corollary}

\begin{proof}
    This follows from \Cref{thm:approx}, since approximating \maxsat{} is \np-hard for $ \frac 78 + \xi $, where $ \xi > 0 $ \cite{Hastad2001-fg}.
    The condition of $ \alpha(p) $ is achieved by substituting $ \delta = \frac 78 $ to \Cref{eq:eps}.
\end{proof}


\begin{figure}[t!]
\centering
\begin{tikzpicture}
	\begin{axis}[
    xlabel={$ p $},
    ylabel={$ \varepsilon(p) $},
    legend pos=south east,
    ymax=0.001,
    ymin=-0.003,
    axis lines=center,
    scaled ticks=false,
    yticklabel style={
	/pgf/number format/precision=3,
	/pgf/number format/fixed,
    }
]

\addplot [
    domain=1/2:1, 
    samples=\figSamples,
    color=blue,
    smooth
]
 {(7*x*(-2+26*x+3*x^2-258*x^3+309*x^4+273*x^5-498*x^6-147*x^7+546*x^8-308*x^9+56*x^10))/1536};

\end{axis}
\end{tikzpicture}
\caption{The value of $ \varepsilon $ w.r.t. choice of $p$ below which (additive) approximation is hard (that is, $ \delta = \frac 78 $ \cite{Hastad2001-fg}). The requirement of $\varepsilon > 0$ gives us that $p \geq \sqrt{7/8}$.}
\Description{A plot of $\varepsilon$, initially zero, then negative, and in the interval $(\sqrt{7/8}, 1)$ slightly positive.}
\label{fig:jie_epsilon}
\end{figure}


\section*{Conclusion}
This paper aims to enhance our understanding of the computational complexity of computing various Shapley value variants. We found that for various ML models --- including decision trees, regression tree ensembles, weighted automata, and linear regression --- both local and global interventional and baseline SHAP can be computed in polynomial time under HMM modeled distributions. This extends popular algorithms, such as TreeSHAP, beyond their empirical distributional scope. We also establish strict complexity gaps between the various SHAP variants (baseline, interventional, and conditional) and prove the intractability of computing SHAP for tree ensembles and neural networks in simplified scenarios. Overall, we present SHAP as a versatile framework whose complexity depends on four key factors: \begin{inparaenum}[(i)] \item model type, \item SHAP variant, \item distribution modeling approach, \item and local vs. global explanations\end{inparaenum}. We believe this perspective provides deeper insight into the computational complexity of SHAP, paving the way for future work.




%We believe that our framework provides a more intricate understanding of SHAP computation complexity across different models, distributions, and variants, paving the way for further research.

Our work opens promising directions for future research. First, expanding our computational analysis to other SHAP-related metrics, such as asymmetric SHAP~\citep{frye20} and SAGE~\citep{covert2020understanding}, would be valuable. Additionally, we aim to explore more expressive distribution classes and relaxed assumptions beyond those in Section \ref{sec:tractable} while maintaining tractable SHAP computation. Finally, when exact computation is intractable (Section \ref{sec:intractable}), investigating the approximability of SHAP metrics through approximation and parameterized complexity theory~\citep{downey2012parameterized} is an important direction.

%Our work opens several promising avenues for future research on the computational properties of explainable AI methods, with a particular focus on SHAP. First, it would be interesting to broaden the computational analysis conducted in this work to include other popular SHAP-related metrics in the literature, such as asymmetric SHAP \cite{frye20} and SAGE \cite{covert2020understanding}. Also, in the future, we aim to explore more expressive distribution classes and relaxed distributional assumptions—extending beyond those examined in Section \ref{sec:tractable} —that still yield tractable SHAP computation. Finally, when exact computation proves intractable (Section \ref{sec:intractable}), it is worthwhile to theoretically investigate the question of the approximability of computing the SHAP metrics across various configurations, through the lens of approximation and parametrized complexity theory \cite{arora2009computational}.

%This paper aims to deepen our understanding of the computational complexity involved in obtaining different Shapley value variants. We found that for a variety of ML models, including decision trees, tree ensembles for regression, weighted automata, and linear regression models — computing both local and global interventional and baseline SHAP can be done in polynomial time when distributions are modeled by HMMs. This extends the distributional scope of popular algorithms like TreeSHAP, which is limited to empirical distributions. Additionally, we demonstrate a strict complexity gap between SHAP variants, showing that interventional and baseline SHAP can be strictly easier to compute than conditional SHAP. Despite these positive results, we uncovered intractability for various SHAP variants in neural networks and tree ensembles. Finally, we provided generalized complexity relations across SHAP variants. We believe that our framework offers a deeper understanding of the complexity involved in computing SHAP across various variants, models, distributions, as well as in both local and global computations, laying the groundwork for future research.

%%%%%%%%%%%%%%%%%%%%%%%%%%%%%%%%%%%%%%%%%%%%%%%%%%%%%%%%%%%%%%%%%%%%%%%%

%%% The acknowledgments section is defined using the "acks" environment
%%% (rather than an unnumbered section). The use of this environment 
%%% ensures the proper identification of the section in the article 
%%% metadata as well as the consistent spelling of the heading.

\begin{acks}
  \Acknowledgments{}
\end{acks}

%%%%%%%%%%%%%%%%%%%%%%%%%%%%%%%%%%%%%%%%%%%%%%%%%%%%%%%%%%%%%%%%%%%%%%%%

%%% The next two lines define, first, the bibliography style to be 
%%% applied, and, second, the bibliography file to be used.

\bibliographystyle{ACM-Reference-Format} 
\bibliographystyle{siamplain}
\bibliography{./ref/ref}


\clearpage
%APPENDIX
\counterwithin{figure}{section}
\onecolumn
\appendix

\section{Bayesian General Learning Rates}
\label{app:bayesian_gadgets}

\begin{lemma}[ ]\label{lemma:twoInformedOneUninformed}
    Let $ \frac 12 < p < 1 $.
    Let $ p' = \frac p2 + \frac 32 p^2 - p^3 $.
    Then \[
        \left( \frac p{1-p} \right)^2 \frac {1-p'}{p'} > 1.
    \]
\end{lemma}

\begin{proof}
    This Lemma is proven by solving the stated inequality.
    We check all our calculations using Wolfram Mathematica.
\end{proof}

\ourparagraph{General Clause Setup}
Consider a clause gadget with literal nodes $a$, $b$, and $c$.
WLOG, suppose the optimal ordering $\sigma^*$ orders $a$ before $b$ before $c$.
Let $\alpha$, $\beta$, and $\gamma$ be the literals corresponding to $ a $, $ b $, and $ c $, respectively.
We also use $\alpha$, $\beta$, and $\gamma$, to denote the corresponding nodes from their respective cells. (see \Cref{fig:bayesian_alphabetagamma} for details of the construction).
Suppose that $\alpha$, $\beta$, and $\gamma$ correctly predict the ground truth with probabilities $p_\alpha$, $p_\beta$, and $p_\gamma$, respectively.
Note that since the literal nodes have no outgoing edges except to each other, ordering all literal nodes after variable nodes gives the literal nodes more information at no cost to any other nodes.
So we assume $ \sigma^* $ orders all literal nodes after variable nodes, and also $a$ before $b$ before $c$. 

\begin{figure}[t!]
	\centering
	\begin{tikzpicture}[
		xsh/.style = { xshift=10mm },
		node distance = 5mm and 5mm,
	stff/.style={circle, draw=black, \figThickness, minimum size=\figCircSize, inner sep=0pt},
	]
		%Nodes
		\node[stff]        (alpha)                  {$ \alpha $};
		\node[stff]        (beta)    [below=of alpha]  {$ \beta $};
		\node[stff]        (gamma)    [below=of beta]  {$ \gamma $};

		\node[stff]        (a)    [right=of alpha,xsh]  {$ a $};
		\node[stff]        (b)    [right=of beta]  {$ b $};
		\node[stff]        (c)    [right=of gamma,xsh]  {$ c $};

		%Lines
		\draw[->, \figThickness] (alpha)  to (a);
		\draw[->, \figThickness] (beta)  to (b);
		\draw[->, \figThickness] (gamma)  to (c);

		\draw[<->, \figThickness] (a)  to (c);
		\draw[<->, \figThickness] (b)  to (c);
		\draw[<->, \figThickness] (b)  to (a);
	\end{tikzpicture}
	\caption{Construction used in General Ordering Learning Rate proof. $\alpha, \beta$ and $\gamma$ are the vertices from the cell gadgets corresponding to the literals from $C$.}
 \Description{A graph of the Bayes gadget  with literals $a,b,c$, along with $alpha,beta,gamma$, from which edges go only to $a,b,c,$ respectively.}
     \label{fig:bayesian_alphabetagamma}
\end{figure}


\begin{observation}
    The predictions of $ \alpha $, $ \beta $, $ \gamma $ are independent.
\end{observation}

\begin{proof}
    The nodes $ \alpha $, $ \beta $, $ \gamma $ are from their variable cells.
    From each cell, there are only outward edges.
    This means that nothing outside of a cell can affect any of its nodes.
    This includes the nodes of other cells.
    Hence, the predictions are independent.
\end{proof}

\begin{lemma}\label{lemma:bayesian_l1_LR}
    For $a$ the earliest literal node in its clause gadget, $\oclr(a) = p$ when its corresponding literal $\alpha$ is off, and $\oclr(a) = p' \deq \frac p2 + \frac 32 p^2 - p^3$ when $\alpha$ is on. 
\end{lemma}
\begin{proof}
    We compute the learning rate of $ a $ under Bayesian aggregation.
    Recall that Bayesian agents base their predictions on the posterior probability for $\theta$ given their inputs, tie-breaking by selecting an action randomly.
    So, the prediction of node $a$ is determined by the posterior of $\theta$ given their private signal $s_a$ and their in-neighbor $\alpha$.
    Therefore, $a$ predicts correctly whenever its posterior for $\theta = 1$ exceeds $1/2$ or tiebreaks correctly.
    Equivalently, $a$ predicts correctly whenever its corresponding likelihood ratio exceeds $1$ or tiebreaks correctly.
    We denote the ratio as $ \Lambda $, defined as
    \[
        \Lambda(\theta = 1 : \theta = 0 \mid s_a, \alpha) \deq \frac {\Pr[s_a, \alpha \mid \theta = 1]} {\Pr[s_a = 1, \alpha = 0 \mid \theta = 0]}                                                            = \frac {\Pr[s_a \mid \theta = 1]} {\Pr[s_a \mid \theta = 0]} \cdot \frac {\Pr[ \alpha \mid \theta = 1]} {\Pr[\alpha \mid \theta = 0]},
    \]
    where the equality holds due to $ s_a $ and $ \alpha $'s prediction being independent.

    In general, there are four cases which may arise.
    \[
                \Lambda(1:0 \mid s_a, \alpha) = \begin{cases}
                     \frac p{1-p} \cdot \frac {p_\alpha}{1-p_\alpha} > 1 & \text{if $s_a = 1, \alpha = 1$,} \\
                     \frac {1-p}p \cdot \frac {1-{p_\alpha}}{p_\alpha} < 1 & \text{if $s_a = 0, \alpha = 0$,} \\
                     \frac {1-p}p\cdot \frac{p_\alpha}{1-p_\alpha} & \text{if $s_a = 0, \alpha = 1$,} \\
                     \frac p{1-p} \cdot\frac{1-p_\alpha}{p_\alpha} & \text{if $s_a = 1, \alpha = 0$.} \\
                \end{cases}
    \]
    The inequality in the first two cases holds because $ p_\alpha \geq p $.
    Cases 3 and 4 need to be analyzed separately for $ \alpha $ on and off.
    \begin{itemize}
        \item If $\alpha$ is off, then $ p_\alpha = \pr{\alpha = 1 \mid  \theta = 1} = \pr{\alpha = 0 \mid \theta = 0} = p $, as discussed in \Cref{lemma:bayesian_cellLearningRate}.
            Thus \begin{align*}
                \Lambda(1:0 \mid s_a = 0, \alpha=1) = 
                \Lambda(1:0 \mid s_a = 1, \alpha=0)                 = \frac p{1-p} \cdot\frac{1-p}{p} = 1.
            \end{align*}
        \item If $\alpha$ is on, then $ p_\alpha = \pr{\alpha = 1 \mid  \theta = 1} = \pr{\alpha = 0 \mid \theta = 0} = \frac p2 + \frac 32 p^2 - p^3 > p $, again, as discussed in \Cref{lemma:bayesian_cellLearningRate}.
            Thus $\Lambda(1:0 \mid s_a = 0, \alpha=1) > 1$ and $\Lambda(1:0 \mid s_a = 1, \alpha=0) < 1.$ \end{itemize}
    
            Now, we compute the final learning rate of $ a $, which is the probability $ \pr{a=\theta} $. Since we assume $ q = \frac 12 $, this is the same as $ \pr{a=1 \suchthat \theta=1} $.
    
	Therefore, the probability of $a$ being correct is the probability of any input configuration yielding either $\Lambda > 1$, plus the probability of a configuration yielding $\Lambda = 1$ and tie breaking successfully with probability $ \frac 12 $.
    \begin{align*}
        \Pr[a = 1 \mid \alpha \text{ off}\,] &= \Pr[s_a = \alpha = 1 \mid \theta = 1] + \frac 12 \pr{s_a \neq \alpha \mid \theta = 1} = p^2 + \frac 12 \left( p(1-p) + (1-p)p \right) = p, \\
            \Pr[a = 1 \mid \alpha \text{ on}\,] &= \Pr[\alpha = 1 \mid \theta = 1] = p_\alpha = \frac p2 + \frac 32 p^2 - p^3.
	\end{align*}
\end{proof}

\begin{lemma}\label{lemma:bayesian_l2_LR}
    For $b$ the second earliest literal node in its clause gadget, $\oclr(b)$ takes the following values:
    \begin{align*}
        \oclr(b) = \begin{cases}
        3p^2-2p^3 & \text{if $ \alpha $, $ \beta $ both off,} \\
        p'(p'+2p-2pp') & \text{if $ \alpha $, $ \beta $ both on,} \\
        p(p+2p'-2pp') & \text{otherwise,} \\
    \end{cases}
    \end{align*}
    where $p' = \frac p2 + \frac 32 p^2 - p^3 $.
\end{lemma}
\begin{proof}
    As in the previous Lemmas, we first compute the ratio of $ \frac{\pr{\theta=1 \mid X_b}}{\pr{\theta=0 \mid X_b}} $, denoted as $ \Lambda $, to get $ b $'s predictions for each of its input configurations. We then compute the probability that $b$ sees inputs that yield a correct prediction.

    The information available to $ b $ is the predictions of $ a $ and $ \beta $, along with $ s_b $, $ b $'s private signal.
    Observe that these are all independent.
    We again define the ratio $ \Lambda $ as \begin{align*}
        \Lambda &\deq \frac {\Pr[s_b, a, \beta \mid \theta = 1]} {\Pr[s_b, a, \beta \mid \theta = 0]} = \frac {\Pr[s_b \mid \theta = 1]} {\Pr[s_b \mid \theta = 0]} \cdot \frac {\Pr[ a \mid \theta = 1]} {\Pr[a \mid \theta = 0]}\cdot\frac {\Pr[ \beta \mid \theta = 1]} {\Pr[\beta \mid \theta = 0]}.
    \end{align*}
    The values of $ \Lambda $ are once again determined by $ p \deq \pr{s_b = \theta} $ and $ p_a \deq \pr{a = \theta} $, $ p_\beta = \pr{\beta = \theta} $.
    Notice also that $ p_a, p_\beta \geq p $. We now list $\Lambda$ for all possible inputs.

    \begin{enumerate}[ref=(\theenumi)]
        \item If $ s_b = a = \beta = 1 $, \[
            \Lambda = \frac p{1-p} \cdot \frac {p_a}{1-p_a} \cdot \frac {p_\beta}{1-p_\beta} > 1
        \]
    \item\label{badcase1} If $ s_b = a = 1, \beta = 0 $, \[
                \Lambda = \frac p{1-p} \cdot \frac {p_a}{1-p_a} \cdot \frac {1-p_\beta}{p_\beta}.
            \]
        \item\label{badcase2} If $ s_b = 1,  a = 0, \beta = 1 $, \[
                \Lambda = \frac p{1-p} \cdot \frac {1-p_a}{p_a} \cdot \frac {p_\beta}{1-p_\beta}.
            \]
	\item If $ s_b = 1,  a = \beta = 0 $, \[
                \Lambda = \frac p{1-p} \cdot \frac {1-p_a}{p_a} \cdot \frac {1-p_\beta}{p_\beta} < 1.
            \]
    \end{enumerate}
    The rest of the cases follow trivially from symmetry, since
    \[
        \Lambda(s_b, a, \beta) = 1/\Lambda(\bar s_b, \bar a, \bar \beta).
    \]

    For the cases \labelcref{badcase1,badcase2}, we perform case analysis on whether $ \alpha $ and $ \beta $ are on or off.
    By \Cref{lemma:bayesian_l1_LR}, $p_a = p$ when $\alpha$ is off and $p_a = p'$ when $\alpha$ is on.
    From \Cref{lemma:bayesian_cellLearningRate}, $\beta$ is correct with probability $p$ when $\beta$ is off and $p'$ when $\beta$ is on.
    We consider two cases:
    \begin{itemize}[ ]
        \item If $ \alpha, \beta $ are both on, or both off, then $ p_a = p_\beta $, leaving $\Lambda = \frac p{1-p} > 1 $ in both of the middle cases.
        \item Otherwise, WLOG $ \alpha $ is on, $ \beta $ off.
            Then $ p_a = p', p_\beta = p $, resulting in $\Lambda( s_b= a=1, \beta=0) = \frac {p'}{1-{p'}} > 1 $ and $\Lambda( s_b= 
            \beta=1, a=0) = \left( \frac{p}{1-p} \right)^2 \cdot \frac {1-p'}{p'} > 1 $ by \Cref{lemma:twoInformedOneUninformed}.
    \end{itemize}

    We can see that $ b $ simply follows the majority of signals.
    \begin{align*}
        \pr{b = \theta} &= \pr{b = 1 \mid \theta = 1} = \pr{s_b + a + \beta \geq 2} = 1-\pr{s_b + a + \beta \leq 1} \\
                        &= 1 - (1-p)(1-p_\alpha)(1-p_\beta) - p(1-p_\alpha)(1-p_\beta) - (1-p)p_\alpha(1-p_\beta) - (1-p)(1-p_\alpha)p_\beta.
    \end{align*}
    Substituting $ p_a = p $ (similarly, $p_b = p$) if $ \alpha $ ($\beta$) is off and $ p' $ otherwise yields the desired probabilities.
\end{proof}

\begin{lemma}\label{lemma:bayesian_l3_LR}
    For $c$ the last literal node in its clause gadget, $\oclr(c)$ takes the following values, depending on the state of $ \left( \alpha, \beta, \gamma \right) $:
    \begin{align*}
\text{(off},\text{off},\text{off)}&\to            p^2 \left(2 p^3-5 p^2+2 p+2\right), \\
\text{(off},\text{off},\text{on)}&\to            p \left(-p^2-p (p'-2)+p'\right), \\
\text{(off},\text{on},\text{off)}&\to            p \left(p^3 (2 p'-1)+p^2 (1-4 p')+p (p'+1)+p'\right), \\
\text{(off},\text{on},\text{on)}&\to            p \left(2 p^2 (p'-1) p'+p \left(-3 (p')^2+p'+1\right)+p' (p'+1)\right), \\
\text{(on},\text{off},\text{off)}&\to            p^4 (2 p'-1)+p^3 (2-4 p')+2 p p', \\
\text{(on},\text{off},\text{on)}&\to            p \left(p^2 \left(2 (p')^2-2 p'+1\right)-3 p (p')^2+p' (p'+2)\right), \\
\text{(on},\text{on},\text{off)}&\to            p \left(4 p^2 (p'-1) p'+p \left(-6 (p')^2+4 p'+1\right)+2 (p')^2\right), \\
\text{(on},\text{on},\text{on)}&\to            p' \left(p^2 \left(2 (p')^2-3 p'+1\right)+p \left(-2 (p')^2+p'+1\right)+p'\right).
    \end{align*}
\end{lemma}

\begin{proof}
    We again compute the value of $\Lambda = \frac{\pr{\theta = 1 \mid X_c}}{\pr{\theta = 0 \mid X_c}}$ for all possible configurations of $ X_c $.
    From that, we determine the action $ c $ chooses and compute the learning rate.

    First, let us rewrite \[
        \Lambda = \frac{\pr{s_c \mid \theta = 1}}{\pr{s_c \mid \theta = 0}} \cdot \frac{\pr{\gamma \mid \theta = 1}}{\pr{\gamma \mid \theta = 0}} \cdot \frac{\pr{a \mid \theta = 1}}{\pr{a \mid \theta = 0}} \cdot \frac{\pr{b \mid a,\theta = 1}}{\pr{b \mid a,\theta = 0}}.
    \]

    From the proofs of the previous two lemmas, we have \begin{align*}
        \pr{b = 1 \mid a = 1, \theta = 1} &= \pr{b = 0 \mid a = 0, \theta = 0} = p + p_\beta - pp_\beta, \\
        \pr{b = 1 \mid a = 0, \theta = 1} &= \pr{b = 0 \mid a = 1, \theta = 0} = pp_\beta, \\
        \pr{a = 1 \mid \theta = 1} &= \pr{a = 0 \mid \theta = 0} = p_\alpha, \\
    \end{align*}
    where $ p_\alpha = p' $ if $ \alpha  $ is on, $ p $ otherwise.
    We again use $ p_\beta, p_\gamma  $ similarly.

    We now perform case analysis on $ \left( s_c, \gamma, a, b \right) $:
    \begin{enumerate}[ ]
        \item $ \left( 1,1,1,1 \right) $: \[
            \Lambda = \frac{p}{1-p} \cdot \frac{p_\gamma}{1-p_\gamma} \cdot \frac{p_\alpha}{1-p_\alpha} \cdot \frac{p+p_\beta - pp_\beta}{1-pp_\beta} > 1,
        \]
        since it holds that $ 1>p,p_\alpha, p_\beta, p_\gamma> \frac 12  $.
        \item $ \left( 1,1,1,0 \right) $: \begin{align*}
                \Lambda &= \frac{p}{1-p} \cdot \frac{p_\gamma}{1-p_\gamma} \cdot \frac{p_\alpha}{1-p_\alpha} \cdot \frac{\left( 1-p \right)\left( 1-p_\beta \right)}{pp_\beta} = \frac{p_\gamma}{1-p_\gamma} \cdot \frac{p_\alpha}{1-p_\alpha} \cdot \frac{1-p_\beta}{p_\beta} \geq \frac{p}{1-p} \cdot \frac{p}{1-p}  \cdot \frac{1-p'}{p'} > 1.
        \end{align*}
        \item $ \left( 1,1,0,1 \right) $: \begin{align*}
                \Lambda &= \frac{p}{1-p} \cdot \frac{p_\gamma}{1-p_\gamma} \cdot \frac{1-p_\alpha}{p_\alpha} \cdot \frac{pp_\beta}{\left( 1-p \right)\left( 1-p_\beta \right)} \geq \left( \frac{p}{1-p} \right)^4 \cdot \frac{1-p'}{p'} > 1.
        \end{align*}
        \item $ \left( 1,1,0,0 \right) $: \begin{align*}
                \Lambda &= \frac{p}{1-p} \cdot \frac{p_\gamma}{1-p_\gamma} \cdot \frac{1-p_\alpha}{p_\alpha} \cdot \frac{1-pp_\beta}{p+p_\beta - pp_\beta}.
        \end{align*}
        This is $ <1 $ if $ \alpha, \beta  $ are on and $ \gamma $ is off, and $ >1 $ otherwise. (By Wolfram, need to write this out)
        \item $ \left( 1,0,1,1 \right) $: \begin{align*}
                \Lambda &= \frac{p}{1-p} \cdot \frac{1-p_\gamma}{p_\gamma} \cdot \frac{p_\alpha}{1-p_\alpha} \cdot \frac{p+p_\beta - pp_\beta}{1-pp_\beta} \geq \left( \frac{p}{1-p} \right)^2 \cdot \frac{1-p'}{p'} \cdot \frac{p+p_\beta - pp_\beta}{1-pp_\beta} > \frac{p+p_\beta - pp_\beta}{1-pp_\beta} > 1.
        \end{align*}
        \item $ \left( 1,0,1,0 \right) $: \begin{align*}
                \Lambda &= \frac{p}{1-p} \cdot \frac{1-p_\gamma}{p_\gamma} \cdot \frac{p_\alpha}{1-p_\alpha} \cdot \frac{\left( 1-p \right)\left( 1-p_\beta \right)}{pp_\beta} \leq \frac{p'}{1-p'} \cdot \left( \frac{1-p}{p} \right)^2 <1.
        \end{align*}
        \item $ \left( 1,0,0,1 \right) $: \begin{align*}
                \Lambda &= \frac{p}{1-p} \cdot \frac{1-p_\gamma}{p_\gamma} \cdot \frac{1-p_\alpha}{p_\alpha} \cdot \frac{pp_\beta}{\left( 1-p \right)\left( 1-p_\beta \right)} \geq \left( \frac{1-p'}{p'} \right)^2 \cdot \left( \frac{p}{1-p} \right)^3.
        \end{align*}
        The lower bound on $ \Lambda $ is reached for $ \gamma $ on and $ \alpha, \beta $ off, and in such a case it is $ <1 $.
        However, in any other state of $ \alpha, \beta, \gamma $, the value of $ \Lambda $ is lower-bounded by \[
             \frac{1-p'}{p'} \cdot \left( \frac{p}{1-p} \right)^2 > 1,
        \]
        by \Cref{lemma:twoInformedOneUninformed}.
        \item $ \left( 1,0,0,0 \right) $: \begin{align*}
                \Lambda &= \frac{p}{1-p} \cdot \frac{1-p_\gamma}{p_\gamma} \cdot \frac{1-p_\alpha}{p_\alpha} \cdot \frac{1-pp_\beta}{1-\left( 1-p \right)\left( 1-p_\beta \right)} \leq \frac{1-p}{p} \cdot \frac{1-p^2}{2p-p^2}<1.
        \end{align*}
    \end{enumerate}
    The remaining cases follow from symmetry.

    We now compute the learning rates in each case, and multiply them by the probability of that case taking place.
    Due to symmetry of $ \theta $ and $ 1-\theta $, this is equivalent to $\pr{\Lambda > 1 \mid \theta = 1}$.
    This can be written as the sum of the probabilities of all cases yielding $ \Lambda > 1 $ assuming $ \theta = 1 $.

    The configurations which always yield $ \Lambda > 1 $ are $ \left( 1,1,1,1 \right)$, $(1,1,1,0)$, $(1,1,0,1)$, $(1,0,1,1)$, $(0,1,0,1)$, $(0,1,1,1) $.
    Then, if $ \alpha,\beta $ are off and $ \gamma $ is on, then $ (0,1,1,0) $ gives $ \Lambda > 1 $, otherwise $ (1,0,0,1) $ gives $ \Lambda > 1 $.
    Finally, if $ \alpha,\beta $ are on and $ \gamma $ is off, then $ (1,1,0,0) $ gives $ \Lambda > 1 $, otherwise $ (0,0,1,1) $ gives $ \Lambda > 1 $.
    The learning rate of $ c $ is then the sum of the probabilities of these cases, given that $ \theta = 1 $.
\end{proof}


We now compute the CLR in the following three cases: when all literals are off, when all literals are on, and when only one literal is on.

\begin{lemma}\label{lemma:bayesian_000CLR} 
    Suppose that an optimal ordering $\sigma^*$ sets all literals in a clause $ C $ to be off. Then the resulting CLR of $ \gadget C $ is \[
        \oclr (\gadget C) = \pzero \deq p \left(2 p^4-5 p^3+5 p+1\right).
     \]
\end{lemma}

\begin{proof}
    Follows directly from \Cref{lemma:bayesian_l1_LR,lemma:bayesian_l2_LR,lemma:bayesian_l3_LR}.
\end{proof}

\begin{lemma}
    \label{lemma:bayesian_111CLR}
    Suppose that an optimal ordering $\sigma^*$ sets all literals in a clause $ C $ to be on. Then the resulting CLR of $ \gadget C $ is \begin{align*}
        \oclr(\gadget C) = \pthree \deq p' \left(p^2 \left(2 (p')^2-3 p'+1\right)-p \left(2 (p')^2+p'-3\right)+2 p'+1\right).
     \end{align*}
\end{lemma}

\begin{proof}
    Follows again directly from \Cref{lemma:bayesian_l1_LR,lemma:bayesian_l2_LR,lemma:bayesian_l3_LR}.
    The expression in the lemma statement is just the sum of $ \oclr(a),\oclr(b),\oclr(c) $.
\end{proof}

\begin{lemma}
    \label{lemma:bayesian_100CLR}
    Suppose that an optimal ordering $\sigma^*$ sets exactly one literal to be on. Then the resulting CLR is \[
        \oclr(\gadget C) =\pone \deq p^4 (2 p'-1)+p^3 (2-4 p')+p^2 (1-2 p')+4 p p'+p'.
     \]
\end{lemma}

\begin{proof}
    This lemma is a bit more involved than the two ealrier ones, since there are now three cases which are not symmetrical---the vertex corresponding to the ``on'' literal is either first, second, or third (in other words, it corresponds to either $ a $, $ b $, or $ c $ from the general lemmas).

    It turns out that the highest learning rate is achieved when $ \alpha $ is on.
    Intuitively, the fact that $ \alpha $ is on means that $ a $ receives a stronger signal, and it is able to be spread all over the gadget.
    Formally, we show that the learning rate corresponding to $ \left( \alpha, \beta,\gamma \right) = \left( \text{on},\text{off},\text{off} \right) $ is higher than $ \left( \text{off},\text{off},\text{on} \right) $ and $ \left( \text{off},\text{on},\text{off} \right) $.

    By \Cref{lemma:bayesian_l1_LR,lemma:bayesian_l2_LR,lemma:bayesian_l3_LR}, and assuming that all the variable cells are ordered before the clause gadget, the learning rate of the clause $ C $ is as follows.
    \begin{enumerate}[ ]
        \item If $ \alpha $ is on, then \[
             \clr(\gadget C) = p^4 (2 p'-1)+p^3 (2-4 p')+p^2 (1-2 p')+4 p p'+p',
        \]
        \item if $ \beta $ is on, then \[
            \clr(\gadget C) = p \left(p^3 (2 p'-1)+p^2 (1-4 p')-p (p'-2)+3 p'+1\right),
        \]
        \item if $ \gamma $ is on, then \[
            \clr(\gadget C) = p \left(-3 p^2-p (p'-5)+p'+1\right).
        \]
    \end{enumerate}
    It is not hard to see that indeed the first quantity is, assuming $ \frac 12 > p > 1 $, the largest of the three, and it is thus the optimal value for the cumulative learning rate of a cell with only one variable turned on.
\end{proof}

\begin{lemma}
    \label{lemma:bayesian_110CLR}
    Suppose that an optimal ordering $\sigma^*$ sets exactly two literals to be on. Then the resulting CLR is \[
        \oclr(\gadget C) =\ptwo \deq 4 p^3 (p'-1) p'+p^2 \left(-6 (p')^2+4 p'+1\right)+2 p p'+p' (p'+1).
     \]
\end{lemma}

\begin{proof}
    As in the previous lemma, we need to decide which of the literals should be the ``off'' one---in this case it is $ \gamma $.
    Intuitively, again, the fact that $ \alpha $ and $ \beta $ are on means that their stronger signals can be better spread over the gadget.
    Formally, we show that the learning rate corresponding to $ \left( \alpha, \beta,\gamma \right) = \left( \text{on},\text{on},\text{off} \right) $ is the highest.

    By \Cref{lemma:bayesian_l1_LR,lemma:bayesian_l2_LR,lemma:bayesian_l3_LR}, and assuming that all the variable cells are ordered before the clause gadget, the learning rate of the clause $ C $ is as follows.
    \begin{enumerate}[ ]
        \item If $ \alpha $ is off, then \[
             \clr(\gadget C) = p \left(2 p^2 (p'-1) p'-p \left(3 (p')^2+p'-2\right)+(p')^2+3 p'+1\right),
        \]
        \item if $ \beta $ is off, then \[
            \clr(\gadget C) = p'+p p' (4+p')+p^2 (1-2 p'-3 (p')^2)+p^3 (1-2 p'+2 (p')^2),
        \]
        \item if $ \gamma $ is off, then \[
            \clr(\gadget C) = 4 p^3 (p'-1) p'+p^2 \left(-6 (p')^2+4 p'+1\right)+2 p p'+p' (p'+1).
        \]
    \end{enumerate}
    It is not hard to see that indeed the last quantity is, assuming $ \frac 12 > p > 1 $, the largest of the three, and it is thus the optimal value for the cumulative learning rate of a cell with exactly two variables turned on.
\end{proof}

Lastly, observe that as the number of literals which are set to on increases, the optimal gadget learning rate increases because the literal nodes get stronger signals for $\theta$. We can verify this for the cases computed above:
\begin{corollary}\label{cor:bayesian_LR_comparison}
    For $p \in (\frac 1 2, 1)$, \[
    \pthree \geq \ptwo \geq \pone \geq \pzero.
    \]
\end{corollary}

\begin{proof}
    Follows by substituting the expressions from \Cref{lemma:bayesian_000CLR,lemma:bayesian_111CLR,lemma:bayesian_100CLR,lemma:bayesian_110CLR}.
    The calculations were verified by Wolfram Mathematica.
    For a visual representation of these learning rates, see \Cref{fig:bayes_ps}.
\end{proof}


\begin{corollary} \label{obs:satisfied_bounds}
    For clause gadgets with at least one on literal, the optimal gadget learning rate is lower bounded by $\pone$ and upper bounded by $\pthree$.
\end{corollary} 

\section{Full Proof of Theorem~\ref{thm:nphardness_maj}}
\label{app:majority_proof}

As stated in \Cref{sec:maj}, we perform a reduction from \sat{} to \netlearn{}, adapting the proof of \Cref{thm:nphardness_bayes}.
The graph construction we use is described in \Cref{ssec:maj_graph}.

\ourparagraph{Literal notation}
We now introduce some notation for use in subsequent sections.
Let $ \ell $ be a literal of some variable $ x \in \vars $, meaning either $ \ell = x $ or $ \ell = \lnot x $.
We extend the notation of the cell to be $ \cell \ell \deq \cell x $.
We say that cell $\cell x$ is ``on'' under a decision ordering $\sigma$ if $\sigma(\lnot x_i) < \sigma(x_i)$; otherwise, cell $\cell x$ is ``off''.
We say that a \emph{literal} $ \ell $ is \emph{on} if $ \ell = x $ and $ \cell x $ is on, or $ \ell = \lnot x $ and $ \cell x $ is off; otherwise, literal $ \ell $ is off.


\subsection{Gadget Learning Rates}
We begin by examining the learning rate of an arbitrary cell under a pair of orderings in which the cell is either ``on'' or ``off''. The following lemma shows that cells achieve the same learning rate under either of these orderings, and that this learning rate is the best possible over all orderings. 

\begin{lemma}[Majority Dynamics Cell LR] \label{lemma:MD_cellLearningRate}
    Let $x \in \vars$.
    Let $ q = \frac 12 $, and $ p $ be given.
    Then \[
	\oclr (\cell x) = 2p + 3p^2 -2p^3.
    \]
\end{lemma}

\begin{proof}
    First, observe that for the optimal ordering $ \sigma^* $, it is always beneficial to put the dummy node $ d_x $ \emph{before} the nodes $ x $ and $ \lnot x $.
    This is because $ d_x $ itself has no incoming edges, so it cannot benefit from any further information, and it has edges going to $ x $ and $ \lnot x $, which can only increase their chances of getting the correct answer.
    Hence, we can see that $ \oclr(d_x) = p $.

    The case of the remaining two nodes is symmetric, WLOG let us assume that $ \sigma^*(x) < \sigma^*(\lnot x) $ (so the cell $ \cell x $ is ``off'').
    The node $ x $ then receives, apart from its private information, the action of $ d_x $.
    Due to tie-breaking, it still always chooses to believe its private signal, so $ \oclr (x) = p $.

    Finally, the node $ \lnot x $ receives the private signal, $ s_{\lnot x} $, the action of $ d_x $, $ a_{d_x} $ and the action of $ x $, $ a_x $.
    Notice that all these three pieces of information are independent (since $ x $ was not influenced by $ d_x $ due to tie-breaking).
    This makes it easy to compute the learning rate of node $ \lnot x $ as the probability that at least two of the three pieces of information are correct \begin{align*}
        \oclr(\lnot x) &= \prt{at least two of $ s_{\lnot x}, a_x, a_{d_x} $ are correct} \\
        &= 3p^2 - 2p^3.
    \end{align*}

    By linearity of expectation, it then holds that \[
	\oclr(\cell x) = 2p + 3p^2 -2p^3.
    \]
\end{proof}

We defer the question of which of the two orderings is optimal to the next section. Addressing this first requires examining the learning rates of the clause gadgets, which are also affected by cell node orderings. We begin with a general expression for the CLR of any clause gadget under a particular partial ordering. We then use this expression to compute the CLR for specific clauses, observing that the optimal clause node ordering must respect the partial ordering given earlier.

\begin{lemma}[Majority Clause LR] \label{lemma:MD_genOrderingCLR}
    For any clause $ C = \left( \ell_1 \lor \ell_2 \lor \ell_3 \right) $, let $ a,b,c $ be the literal nodes in the clause gadget corresponding to $\ell_1$, $\ell_2$, and $\ell_3$. Also, let $ \alpha, \beta, \gamma $ be the cell variable nodes corresponding to $\ell_1$, $\ell_2$, and $\ell_3$ (see \Cref{fig:alphabetagamma} for details of the construction). Suppose that $\alpha$, $\beta$, and $\gamma$ correctly predict the ground truth with probabilities $p_\alpha$, $p_\beta$, and $p_\gamma$, respectively.
    Consider a partial ordering $ \sigma $ which orders all variable and dummy nodes before all literal nodes. WLOG, suppose $ \sigma $ also orders $a$ before $b$ before $c$.
    Then the cumulative learning rate of the clause gadget under $\sigma$ is \begin{align*}
	\clr(\gadget C, \sigma) = 2p + \Pr[a] + \Pr[b] + \Pr[c], 
    \end{align*}
    where 
    \begin{align*}
        \Pr[a] &= p((1-p)p_\alpha + p(p+2)),\\
        \Pr[b] &= p^2 \left((1-p)^2 (p_\alpha + 5{p_\beta})+p (4-3 p)\right),\\
        \Pr[c] &= p^2 \left((1-p)^2 (p_\alpha(1-{p_\gamma} ({p_\beta}+p-1)+{p_\beta} p)+ 2p_\beta p -{p_\gamma}({p_\beta} (5 p-3)-3 p-2)) + p(4-3p)\right).
    \end{align*}
\end{lemma}


\begin{proof}
     Note that the dummy nodes have no incoming edges, and therefore each contributes a learning rate of $p$. We now compute the learning rates of $ a $, $ b $, and $ c $.
    
     \begin{figure}[t!]
	\centering
	\begin{tikzpicture}[
		node distance = 5mm and 5mm,
		xsh/.style = { xshift=10mm },
	stff/.style={circle, draw=black, \figThickness, minimum size=\figCircSize, inner sep=0pt},
	]
		%Nodes
		\node[stff]        (alpha)                  {$ \alpha $};
		\node[stff]        (beta)    [below=of alpha]  {$ \beta $};
		\node[stff]        (gamma)    [below=of beta]  {$ \gamma $};

		\node[stff]        (a)    [right=of alpha,xsh]  {$ a $};
		\node[stff]        (b)    [right=of beta]  {$ b $};
		\node[stff]        (c)    [right=of gamma,xsh]  {$ c $};
		\node[stff]        (d1)    [right=of b, xsh,yshift=20]  {$ d_1 $};
		\node[stff]        (d2)    [right=of b, xsh,yshift=-6mm]  {$ d_2 $};

		%Lines
		\draw[->, \figThickness] (alpha)  to (a);
		\draw[->, \figThickness] (beta)  to (b);
		\draw[->, \figThickness] (gamma)  to (c);

		\draw[<-, \figThickness] (a)  to  (d1);
		\draw[<-, \figThickness] (b)  to  (d1);
		\draw[<-, \figThickness] (c)  to  (d1);
		\draw[<-, \figThickness] (a)  to  (d2);
		\draw[<-, \figThickness] (b)  to  (d2);
		\draw[<-, \figThickness] (c)  to  (d2);
		\draw[<->, \figThickness] (a)  to (c);
		\draw[<->, \figThickness] (b)  to (c);
		\draw[<->, \figThickness] (b)  to (a);
	\end{tikzpicture}
	\caption{Construction used in General Ordering Learning Rate proof. $\alpha, \beta$ and $\gamma$ are the vertices from the cell gadgets corresponding to the literals from $C$.}
 \Description{A graph of the Majority gadget  with literals $a,b,c$, along with $alpha,beta,gamma$, from which edges go only to $a,b,c,$ respectively.}
     \label{fig:alphabetagamma}
\end{figure}


    The first node, $a$, is correct iff \begin{enumerate}[ ]
        \item both $d_1$ and $d_2$ are correct, and at least one of their private signal and $\alpha$ is correct: \[
        \Pr[a \mid d_1, d_2] = 1 - (1-p)(1-p_\alpha),
        \]
        \item or one of $d_1$ and $d_2$ are correct, and their private signal is correct: \[
        \Pr[a \mid d_1, \bar d_2] = \Pr[a \mid \bar d_1, d_2] = p,
        \]
        \item or neither $d_1$ nor $d_2$ are correct, but both $\alpha$ and their private signal are correct: \[
        \Pr[a \mid \bar d_1, \bar d_2] = pp_\alpha.
        \]
	\end{enumerate}

	Therefore, the total probability of $a$ being correct is \begin{align*}
            \Pr[a] &= p^2 \left( 1-(1-p)(1-p_\alpha) \right) + 2p(1-p)p + (1-p)^2pp_\alpha = p (p_\alpha- p(p_\alpha -p-2))= p((1-p)p_\alpha + p(p+2)).
	\end{align*}
    
    Node $b$ is correct with conditional probabilities:
    \begin{align*}
        \Pr[b \mid d_1, d_2] &= 1 - (1-p_\beta)(1-\Pr[a \mid d_1, d_2])(1-p), \\
        \Pr[b \mid d_1, \bar d_2] &= p_\beta\Pr[a \mid d_1, \bar d_2]p + (1-p_\beta)\Pr[a \mid d_1, \bar d_2]p + p_\beta(1-\Pr[a \mid d_1, \bar d_2])p + p_\beta\Pr[a \mid d_1, \bar d_2](1-p) \\
                                  & = \Pr[b \mid \bar d_1, d_2], \\
        \Pr[b \mid \bar d_1, \bar d_2] &= p_\beta\Pr[a \mid \bar d_1, \bar d_2]p.
    \end{align*}

    The total probability of $b$ being correct is then \[
	\pr{b} = p^2 \left((1-p)^2 (p_\alpha + 5{p_\beta})+p (4-3 p)\right).
    \]

    We can also compute the following joint probabilities between $a$ and $b$, conditioning again on the dummy nodes: 
    \begin{align*}
        \Pr[a,b \mid d_1, d_2] &= \Pr[a \mid d_1, d_2],\\
        \Pr[\bar a, \bar b \mid d_1, d_2] &= (1-p_\beta)(1-p)(1-\Pr[a \mid d_1, d_2]),\\
        \Pr[a,b \mid d_1, \bar d_2] &= (1 - (1-p_\beta)(1-p)) \Pr[a \mid d_1, \bar d_2],\\
        \Pr[\bar a, \bar b \mid d_1, \bar d_2] &= (1-p_\beta p)(1-\Pr[a \mid d_1, \bar d_2]),\\
        \Pr[a,b \mid \bar d_1, \bar d_2] &= p_\beta p \Pr[a \mid \bar d_1, \bar d_2],\\
        \Pr[\bar a, \bar b \mid \bar d_1, \bar d_2] &= 1-\Pr[a \mid \bar d_1, \bar d_2].
    \end{align*}
    
    Similarly, $c$ is correct with conditional probabilities 
    \begin{align*}
        \Pr[c \mid d_1, d_2] &= p + (1-p)(p_\gamma(1-\Pr[\bar a, \bar b \mid d_1, d_2]) + (1-p_\gamma)\Pr[a, b \mid d_1, d_2]), \\
        \Pr[c \mid d_1, \bar d_2] &= p(1-(1-p_\gamma)\Pr[\bar a, \bar b \mid d_1, \bar d_2]) + (1-p) p_\gamma \Pr[a, b \mid d_1, \bar d_2] \\
        & = \Pr[c \mid \bar d_1, d_2], \\
        \Pr[c \mid \bar d_1, \bar d_2] &= p(p_\gamma(1-\Pr[\bar a, \bar b \mid \bar d_1, \bar d_2]) + (1-p_\gamma)\Pr[a, b \mid \bar d_1, \bar d_2]).
    \end{align*}
    
    Therefore, the total probability of $c$ being correct is: \[
	\pr{c} = p^2 \left((1-p)^2 (p_\alpha(1-{p_\gamma} ({p_\beta}+p-1)+{p_\beta} p)+ 2p_\beta p -{p_\gamma}({p_\beta} (5 p-3)-3 p-2)) + p(4-3p)\right).
    \]
    
    Finally, nodes $ d_1 $ and $ d_2 $ are both correct with probability $ p $.
    So the total learning rate of the clause is \begin{align*}
	2p + \pr{a} + \pr{b} + \pr{c}.
    \end{align*}
\end{proof}

\begin{lemma}[$1 \lor 0 \lor 0$ Learning Rate]\label{lemma:100CLR} 
    Let $C=\ell_1 \lor \ell_2 \lor \ell_3$ be a clause.
    Suppose that an optimal ordering $\sigma^*$ sets the literal $\ell_1$ to be ``on'', and the rest of the literals to be ``off''.
    Then, in the gadget for $C$, $\sigma^*$ places $\ell_1$ first, then the other two, and finally the node $C$.
    Further, the resulting CLR is \[
        \pone = p \left(2 + 2 p + 6 p^2 + 11 p^3 + 4 p^4 - 51 p^5 - 6 p^6 + 21 p^7 + 
   115 p^8 - 136 p^9 + 13 p^{10} + 36 p^{11} - 12 p^{12} \right).
    \]
\end{lemma}

\begin{proof}
    Since $\sigma^*$ is an optimal ordering, it must place the dummy nodes $d_1$ and $d_2$, as well as all variable nodes, before all literal nodes. Otherwise, the literal nodes receive strictly less information.
    To determine the order in which the literal nodes $\ell_1, \ell_2, \ell_3$ are placed, we can simply compute the expected learning rate for every permutation of them, and then pick the maximizing order.
    Note that $\ell_2$ and $\ell_3$ are identical, so we can treat permutations which just exchange these two as identical as well.

    We use the learning rate for the general clause gadget, as computed in \Cref{lemma:MD_genOrderingCLR}.
    We simply analyze the three non-isomorphic cases: $\ell_1=a, \ell_1=b,$ and $\ell_1=c$. Note that the first case corresponds to setting $p_\alpha = 3p^2 - 2p^3$ and $p_\beta = p_\gamma = p$, and similarly for the latter cases. We get the following cumulative learning rates: \begin{align*}
        (\ell_1=a) &\to p \left(2 + 2 p + 6 p^2 + 11 p^3 + 4 p^4 - 51 p^5 - 6 p^6 + 21 p^7 + 115 p^8 - 136 p^9 + 13 p^{10} + 36 p^{11} - 12 p^{12} \right),\\
        (\ell_1=b) &\to -p \left(-2 - 3 p + p^2 - 28 p^3 + 6 p^4 + 85 p^5 - 95 p^6 + 132 p^7 - 248 p^8 + 165 p^9 + 42 p^{10} - 84 p^{11} + 24 p^{12}\right),\\
        (\ell_1=c) &\to p \left(2 + 3 p + 2 p^2 + 21 p^3 - 25 p^4 - 29 p^5 + 143 p^6 - 440 p^7 + 684 p^8 - 458 p^9 + 54 p^{10} + 72 p^{11} - 24 p^{12}\right).
    \end{align*}
    
    The first polynomial is always strictly larger than the other two for $\frac 12 < p < 1$, so placing $l_1$ first in the ordering gives the highest learning rate in the clause. So we conclude that 
    \begin{align*}
        \pone = p \left(2 + 2 p + 6 p^2 + 11 p^3 + 4 p^4 - 51 p^5 - 6 p^6 + 21 p^7 + 
   115 p^8 - 136 p^9 + 13 p^{10} + 36 p^{11} - 12 p^{12} \right).
    \end{align*}
    Moving forward, we refer to this probability simply by $\pone$, regardless of the order that the literals appear in the clause itself.
\end{proof}

Before we continue with other clauses, note that the CLR of any clause increases with the number of true literals in the clause. One can verify this by checking that for any literal node or the clause node $C$ in the gadget, its probability of correctly predicting the ground truth increases monotonically with each of the variable probabilities. So we will only compute the CLR for clauses which involve the smallest and largest possible variable probabilities.

\begin{lemma}[{$0 \lor 0 \lor 0$, $1 \lor 1 \lor 1$ Learning Rates}]
    \label{lemma:000/111CLR}
    Let $C=\ell_1 \lor \ell_2 \lor \ell_3$ be a clause.
    Suppose that an optimal ordering $\sigma^*$ sets all three literals to false.
    Then the resulting CLR is \[
        \pzero = p \left(2 + 3 p + 2 p^2 + 25 p^3 - 42 p^4 + 23 p^5 - 67 p^6 + 102 p^7 - 
   31 p^8 - 24 p^9 + 12 p^{10}\right).
    \]
    If instead $\sigma^*$ sets all three literals to true, then the resulting CLR is  
    \begin{align*}
     \pthree = p \left(2 + 2 p + 3 p^2 + 14 p^3 + 22 p^4 - 66 p^5 - 69 p^6 + 310 p^7 - 688 p^8 + 710 p^9 + \right.\\
     \left.756 p^{10} - 2581 p^{11} + 2304 p^{12} - 558 p^{13} - 
   372 p^{14} + 264 p^{15} - 48 p^{16} \right).
    \end{align*}
\end{lemma}
\begin{proof}
    As before, the optimal ordering $\sigma^*$ must place node $C$ after all literal nodes. Observe that by symmetry, the ordering of $l_1$, $l_2$, and $l_3$ doesn't matter, since all three variable probabilities are identical. Using the \Cref{lemma:MD_genOrderingCLR} and supplying variable probabilities $p_\alpha = p_\beta = p_\gamma = p$, we get the desired CLR $p^000$.
    Supplying variable probabilities $p_\alpha = p_\beta = p_\gamma = 3p^2 - 2p^3$ gives the desired value of $\pthree$.
\end{proof}

\begin{lemma}[$ 0 \lor 1 \lor 1 $ Learning Rate]
    \label{lemma:110CLR}
    Let $C=\ell_1 \lor \ell_2 \lor \ell_3$ be a clause.
    Suppose that an optimal ordering $\sigma^*$ sets one literal to false, the other literals to true.
    Then the resulting CLR is \[
	\ptwo = p \left(12 p^8-54 p^7+76 p^6-14 p^5-40 p^4+9 p^3+12 p^2+2 p+2\right).
    \]
\end{lemma}

\begin{proof}
    WLOG assume that $ \ell_2 = \ell_3 = 1 $ and $ \ell_1 = 0 $
    We analyze the probabilities of all three (non-isomorphic) possible orderings by setting proper $ p_\alpha, p_\beta, p_\gamma $ from \Cref{lemma:MD_genOrderingCLR}. 
    This results in the following CLRs: \begin{align*}
        (\ell_1=\alpha) &\to p \left(-24 p^{10}+132 p^9-282 p^8+279 p^7-116 p^6+28 p^5-36 p^4+11 p^3+8 p^2+3 p+2\right),\\
        (\ell_1=\beta) &\to -8 p^{10}+44 p^9-86 p^8+57 p^7+18 p^6-21 p^5-11 p^4-7 p^3+15 p^2+2 p+2,\\
        (\ell_1=\gamma) &\to p \left(12 p^8-54 p^7+76 p^6-14 p^5-40 p^4+9 p^3+12 p^2+2 p+2\right).
    \end{align*}
    It is not hard to prove, that for $ p \in \left( \frac 12, 1 \right) $, the biggest value is received when $ \ell_1=\gamma $, and thus its CLR is the optimal CLR.
    This finishes the proof.
\end{proof}

\begin{lemma}[Clause Learning Rate Comparison]
\label{lemma:CLRComp}
	For $ p \in \left( \frac 12, 1 \right) $, it holds \[
	    \pthree \geq \ptwo \geq \pone \geq \pzero.
	\]
\end{lemma}

\begin{proof}
    We know the learning rates of $ \pthree ,\ptwo , \pone , \pzero $ by \Cref{lemma:100CLR,lemma:110CLR,lemma:000/111CLR}.
    The statement then follows by simple algebra.
\end{proof}


\subsection{Optimal Ordering \& Restrictions on p} 
Finally, we can now determine the optimal ordering by examining the learning rates derived in the previous section. First, we define an \emph{assignment-induced ordering} below:

\begin{definition}
    Let $\mathcal{A}: \chi \to \{0,1\}$ be any assignment of values to variables. Define the (partial) \emph{ordering induced by $\mathcal{A}$} as follows: if $\mathcal{A}(x_i)=1$, node $x_i$ is ordered after node $\lnot x_i$ (cell $\cell {x_i}$ is ``on''); otherwise, $x_i$ comes before $\lnot x_i$ (cell $\cell {x_i}$ is ``off'').
\end{definition}

\begin{definition}
    Let $\mathcal{A}: \chi \to \{0,1\}$ be an assignment of values to variables maximizing the number of satisfied clauses. Then $\mathcal{A}$ is a \emph{maximal assignment}.
\end{definition}

\begin{lemma}[Optimal Ordering] \label{lemma:bestOrder}
    Let $\mathcal{A}^*$ be a maximal assignment.  Let $p(M) < 1$ be a threshold probability determined by $M$, the number of clauses. Then for all $p \geq p(M)$, the decision ordering $\sigma^*$ which places all dummy nodes first, then all variable nodes respecting the partial ordering induced by $\mathcal{A}^*$, then all literal nodes, and finally all clause nodes, maximizes the network learning rate.
\end{lemma}
\begin{proof}
    First note that by the same reasoning as in Lemma \ref{lemma:MD_cellLearningRate}, there is an optimal ordering $\sigma^*$ which places all dummy nodes before all variable nodes. Similarly, the same reasoning as in Lemma \ref{lemma:100CLR} gives that there is an optimal $\sigma^*$ in which all literal nodes are ordered after all variable nodes, and all clause nodes are ordered after all literal nodes. So all that remains is to show that there is an optimal $\sigma^*$ which respects the ordering induced by $\mathcal{A}^*$. 

    First recall that by Lemma \ref{lemma:MD_cellLearningRate}, an optimal ordering always gives the same cumulative learning rate $p_i^0 = p_i^1$ per cell regardless of whether the cell is ``on'' or ``off''. So the ordering of variable nodes only affects the network learning rate through the clauses. Consider any other assignment $\mathcal{A}$ which is not maximal. Let $S^*$ be the number of satisfied clauses under $\mathcal{A}^*$ and $S < S^*$ that under $\mathcal{A}$. By \Cref{lemma:100CLR,lemma:000/111CLR}, the CLR for any satisfied clause is lower bounded by $\pone$, which is strictly greater than that for an unsatisfied clause, $\pzero$. So having more satisfied clauses should improve the CLR. However, note also that $\pthree \geq \ptwo \geq \pone \geq \pzero$, by \Cref{lemma:110CLR}. So in the most extreme case, all $S$ satisfied clauses under $\mathcal{A}$ are satisfied by having three true literals, while all $S^*$ satisfied clauses under $\mathcal{A}^*$ are satisfied by having one true literal. 

    Consider that extreme case, and further impose the worst-case choices of $S$ and $S^*$ by setting $S^* = M$ and $S = M-1$. Then over all clause gadgets, the ordering induced by $\mathcal{A}^*$ gives a CLR of $M \pone$, while the ordering induced by $\mathcal{A}$ gives a CLR of $(M-1) \pthree + \pzero$. In order for the ordering induced by $\mathcal{A}^*$ to maximize the network learning rate, we must have
    \begin{align*}
        M \pone &\geq (M-1) \pthree + \pzero\\
        \frac{\pone - \pzero}{\pthree - \pone} &\geq M-1
    \end{align*}
    One can verify that the left-hand side can be lower bounded by $\frac{1}{6-6p}$ for all $p \in (0.5,1)$, so the following is sufficient:
    \begin{align*}
        \frac{1}{6-6p} \geq M-1 \Longrightarrow p \geq p(M) \deq \frac{6M-7}{6M-6},
    \end{align*}
    which is well-defined for any $M \geq 2$. Thus, the $\sigma^*$ respecting the ordering induced by $\mathcal{A}^*$ is optimal for all $p \geq p(M)$.
\end{proof}

To recap, given any instance $\varphi$ of \sat{} with $N$ variables and $M$ clauses, we can construct a network $G_\varphi$ with ground truth prior $q = 1/2$ and signal accuracy $p \geq p(M)$. In particular, whenever $\varphi$ is satisfiable under some maximal assignment $\mathcal{A}^*$, there is an optimal decision ordering $\sigma^*$ which respects the ordering induced by $\mathcal{A}^*$, and which achieves a network CLR of at least $N(2p+3p^2-2p^3) + M\pone$. Otherwise, if $\varphi$ is non-satisfiable under any maximal assignment $\mathcal{A}^*$, then the optimal decision ordering $\sigma^*$ respecting the ordering induced by $\mathcal{A}^*$ achieves a network CLR of no more than $N(2p+3p^2-2p^3) + (M-1)\pthree + \pzero$. Choosing $p \geq p(M)$ allowed us to show the optimality of $\sigma^*$, as well as to separate the learning rates in networks corresponding to satisfiable and non-satisfiable formulas. All that remains to complete the reduction is to pick an appropriate choice of $\varepsilon$, such that a formula graph $G_\varphi$ achieves expected network learning rate greater than $\varepsilon$ under an optimal ordering iff $\varphi$ is satisfiable.

\subsection{Picking the Epsilon}
We will simply pick $\varepsilon$ to lie exactly halfway between the satisfiable and non-satisfiable network learning rates. Recalling that each variable gadget contains $3$ vertices and each clause gadget contains $4$ vertices, we can compute the learning rates below. For networks corresponding to satisfiable formulas, we have at least
\begin{align*}
    \frac{N\pcell + M\pone}{3N + 5M},
\end{align*}
and for networks corresponding to non-satisfiable formulas, we have at most
\begin{align*}
    \frac{N\pcell + (M-1)\pthree + \pzero}{3N + 5M}.
\end{align*}
This gives 
\begin{align*}
    \varepsilon = \frac{1}{2}\left(\frac{2N\pcell + M\pone + (M-1)\pthree + \pzero}{3N + 5M}\right),
\end{align*}
thus completing the reduction. 


%%%%%%%%%%%%%%%%%%%%%%%%%%%%%%%%%%%%%%%%%%%%%%%%%%%%%%%%%%%%%%%%%%%%%%%%
%END

\end{document}

%%%%%%%%%%%%%%%%%%%%%%%%%%%%%%%%%%%%%%%%%%%%%%%%%%%%%%%%%%%%%%%%%%%%%%%%

