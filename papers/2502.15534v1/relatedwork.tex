\section{Related Work}
\label{sec:related_work}



Scheduling has been an active area of research. For our work, we focus on related works in serverless computing and queuing theory as the most relevant ones. In the following, we position pull-based scheduling within these two areas.

\altparagraph{FaaS scheduling:}
Almost all prior work on FaaS scheduling follows a \textit{push-based} model, where the scheduler tries to select suitable workers for incoming function requests. Examples include the following:
Fuerst and Sharma~\cite{fuerst2022locality} achieve function locality in serverless clusters with a variation of consistent hashing, which randomly updates loads to prevent server overload during bursty workloads.
Abdi et al.~\cite{abdi2023palette} increase data locality in serverless computing by caching results in idle function instances, and introducing a color-based system for pre-defined locality hints that uses consistent hashing for color mappings, which requires additional configuration from developers.
Aumala et al.~\cite{aumala2019beyond} propose a different method to increase data locality using package-aware scheduling, which caches packages at workers and uses consistent hashing to assign requests to workers with preloaded packages.

Our approach fundamentally shifts from this push-based scheduling to \textit{pull-based} scheduling, where idle workers proactively request tasks. Furthermore, unlike most previous work, we do not rely on consistent hashing to achieve locality, thus avoiding its fundamental limitations (see Section~\ref{sec:pull_based_scheduling}).
The closest work to ours is by Kim and Roh~\cite{kim2021scheduling}, who propose predictive pre-warming of containers based on the length of request queues while containers continuously pull tasks for execution, which has similarities to the pull mechanism we propose in this paper. Several other pre-warming techniques have been proposed for serverless computing~\cite{roy2022icebreaker, agarwal2021reinforcement, silva2020prebaking}, but the timing of instance pre-warming and the number of pre-warmed instances required can be inaccurate and incur significant costs, not present in pull-based scheduling.


\altparagraph{Queuing theory:}
In queuing theory, traditional algorithms such as Join-Shortest Queue~\cite{eschenfeldt2018join} or Power-of-d-choices~\cite{hellemans2018power} are push-based.
Lu et. al~\cite{lu2011join} are the first to propose the class of Join-Idle-Queue (JIQ) algorithms, where idle servers inform the scheduler of their readiness for new tasks. JIQ has been successfully applied in large distributed systems to reduce system load and response times compared to push-based scheduling algorithms~\cite{wang2018distributed}. Although pull-based scheduling has received attention in the queuing theory literature, its application to serverless computing has remained largely unexplored.