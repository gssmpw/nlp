\section{Related work\label{sec:Related-work}
}

The literature on optimisation aspects at different layers of the
network architecture for the coexistence of D2D communications within
a cellular network, and on D2D offloading in particular, is vast.
One of the parameters that are typically subject to optimisation is
the transmit power of the devices. In layer 2-oriented studies, the
D2D transmit power control is used to enable an optimal radio resource
sharing of D2D communications with I2D (and viceversa) ones by mitigating
the interference among transmissions that share the same resources,
assuming the two types of links are eligible for using the same resources
____. In most network-level studies, the transmission
range settings are taken as an input to the problem. To the best of
our knowledge, linking the selection of the maximum D2D transmission
range to specific classes of traffic demand is an approach that has
not yet been investigated in the literature.

In the networking community, D2D offloading problems have been framed
under different assumptions on the system model. The interested reader
can check ____ for large spectrum classification.
Scenarios assuming synchronous request make it possible to organize
the content delivery by partially leveraging multicast, see e.g.,
____. In this work, we consider requests arising
from users independently of each other (except for the fact that they
are all subsequent to a content generation). Pioneering studies date
back to the works ____. Our interest
is in the energy savings that can be achieved without requiring the
network to explicitly organize the diffusion of each content among
the interested users. Therefore, we do not consider selective content
injection ____, and multicast planning
operation either. While these strategies are without question worth
for reducing the network resources usage, they require a specific
planning for each content. The solution presented in this paper requires
the network controller to execute minimal operation, as the content
diffusion process evolves accordingly to the spontaneous requests
from the users. We do require, however, that the cellular network
has a certain degree of at least \emph{monitoring} how popular contents
spread within the population of users. An assumption we make, which
is also popular in works dealing with the 5G and beyond network architecture
(see e.g. ____) is that the network has knowledge of the
nodes positions and of which contents the users are caching at any
time. This is used to identify neighboring nodes and instruct them
to hand content of mutual interest to each other.

Regarding the characterisation of the content classes, an important
feature in our work is the request intensity profile, i.e., a function
of time which tells how the number of requests (per unit time) evolves
after the generation of a content. For the scope of this work, we
target contents that become of interest for many users in a matter
of tens of minutes, according to the already mentioned application
domains. While there are many studies covering the topic of how the
popularity of a given content evolves in time, they are mostly targeted
at videos shared on popular platforms, for which popularity evolves
on timescales in the order of days, or even months ____.
This type of works, however, are not focused on capturing the initial,
raising phase of the popularity dynamics, as they are more interested
in the long-term behavior. In ____, the authors leverage
the rank-and shift model to establish raises and decays in relative
popularity of the contents in a content library. While their setup
is completely different from our one, this model confirms a typical
behavior of the popularity dynamics, in which there is an initially
slowly raising phases, followed by an acceleration and deceleration
in the growth of the popularity, until a peak of request is reached.
In this work, we adopt a parametrical approach, similar to that in
____ which, however, is still focused on the tail
of the distribution of the requests.