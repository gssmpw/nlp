Quality-Diversity (QD) has demonstrated potential in discovering collections of diverse solutions to optimisation problems. Originally designed for deterministic environments, QD has been extended to noisy, stochastic, or uncertain domains through various Uncertain-QD (UQD) methods.
However, the large number of UQD methods, each with unique constraints, makes selecting the most suitable one challenging.
To remedy this situation, we present two contributions: first, the \Longframework{} (\framework{}), and second, \Longname{} (\name{}), a new method derived from it.
The \framework{} unifies existing approaches within a modular view, and facilitates developing novel methods by interchanging modules. 
We use it to derive \name{}, a novel method that consistently outperforms or matches the best existing methods on standard benchmarks, while previous methods show varying performance.
In a second experiment, we show how our \framework{} can be used to augment existing QD algorithms and in particular the well-established Policy-Gradient-Assisted-MAP-Elites method, and demonstrate improved performance in uncertain domains at no additional evaluation cost. 
For any new uncertain task, our contributions now provide \name{} as a reliable "first guess" method, and the \framework{} as a tool for developing task-specific approaches.
Together, these contributions aim to lower the cost of adopting UQD insights in QD applications.
