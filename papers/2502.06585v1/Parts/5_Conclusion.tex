\section{Discussion and Conclusion}


This work focuses on QD algorithms applied to uncertain domains and introduces two main contributions: the \framework{}, a generic framework that encompasses existing UQD approaches and facilitates the design of new ones, and \name{}, a new UQD algorithm that directly instantiates this framework.
These contributions are intended as tools for addressing new UQD tasks: \name{} provides a reliable "first guess" method, likely to perform well, while \framework{} serves as a toolbox for developing tailored UQD approaches for specific tasks or accounting for uncertainty in existing QD methods.
We present an experimental study demonstrating the strong performance of \name{} on widely used UQD tasks. Additionally, we demonstrate the capability of our \framework{} to easily create new and high-performing UQD algorithms, showing that accounting for uncertainty and integrating UQD insights can significantly enhance the performance of standard algorithms.

A limitation of this work is that it focuses on the Performance Estimation problem, without addressing the two other known UQD problems, namely Reproducibility Maximisation and Fitness-Reproducibility Trade-off. While this choice was made for clarity, we demonstrate in Section~\ref{sec:framework} and Table~\ref{tab:approaches} that our \framework{} encompasses existing approaches that account for these two problems. This also highlights the potential of the framework for deriving new approaches that address these two problems, although a detailed study of this is left for future work.

This work consolidates existing knowledge from previous UQD research while advancing our understanding of how existing methods can be adapted to novel UQD tasks. Our contributions aim to lower the entry barrier to UQD methods and insights, making an important step to extend the applicability of QD algorithms.

