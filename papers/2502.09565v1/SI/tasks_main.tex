\begin{longtable}[]{|p{0.1\textwidth}|p{0.6\textwidth}|p{0.05\textwidth}|p{0.25\textwidth}|}
\caption{Details of 25 task prompts used in experiments} \\

\hline
\textbf{Prompt ID} & \textbf{Prompt} & \textbf{\# subtasks} & \textbf{List of required subtasks} \\
\hline
\endfirsthead

\hline
\textbf{Prompt ID} & \textbf{Prompt} & \textbf{\# subtasks} & \textbf{List of required subtasks} \\
\hline
\endhead

\hline
\endfoot
\hline
\endlastfoot

1 & Simulate PDB ID 1MBN at two different temperatures: 300 K and 400 K for 1 ns each. Plot the RMSD of both over time and compare the final secondary structures at the end of the simulations. Get information about this protein, such as the number of residues and chains, etc. & 8 & Download PDB, simulate (x2), RMSD (x2), DSSP (x2), summarize\_protein \\ \hline
2 & Download the PDB file for protein 1LYZ. & 1 & Download PDB \\ \hline
3 & Download the PDB file for protein 1GZX. Then, analyze the secondary structure of the protein and provide information on how many helices, sheets, and other components are present. Get the gene names for this protein. & 3 & Download PDB, DSSP, GetProteinFunction (or literature) \\ \hline
4 & What are the common parameters used to simulate fibronectin? & 1 & literature search \\ \hline
5 & Simulate 1VII for 1 ns at a temperature of 300 K. Then, tell me if the secondary structure changed from the beginning of the simulation to the end of the simulation. & 5 & Download PDB, simulate, DSSP before, DSSP after, comparison \\ \hline
6 & Simulate 1A3N and 7VDE (two PDB IDs matching hemoglobin) with identical parameters. Find the appropriate parameters for simulating hemoglobin from the literature. Then, plot the radius of gyration throughout both simulations. & 7 & Download PDB (x2), literature, simulate (x2), RGy (x2) \\ \hline
7 & Simulate 1ZNI for 1 ns at a temperature of 300 K in water. Then, simulate it again in acetonitrile. Compute the RMSD, final secondary structure, and PCA for each simulation. & 10 & Download PDB, simulate (x2), packmol, RMSD (x2), SASA (x2), PCA (x2) \\ \hline
8 & Simulate 4RMB at 100K, 200K, and 300K. Then, for each simulation, plot the radius of gyration over time and compare the secondary structure before and after the simulation. & 10 & Download PDB, simulate (x3), RGy (x3), DSSP (x3) \\ \hline
9 & Download the PDB file for 1AEE. Then tell me how many chains and atoms are present in the protein. & 2 & download PDB, count atoms/chains \\ \hline
10 & Simulate protein 1ZNI at 300 K for 1 ns and calculate the RMSD. & 3 & Download PDB, simulate, RMSD \\ \hline
11 & Download the PDB files for 8PFK and 8PFQ. Then, compare the secondary structures of the two proteins, including the number of atoms, secondary structures, number of chains, etc. & 4 & Download PDB (x2), DSSP (x2) \\ \hline
12 & Simulate fibronectin (PDB ID 1FNF) for 1 ns, using an appropriate temperature found in the literature. Compute the RMSD and the final secondary structure. By using the PDB ID to get the Uniprot ID, obtain the subunit structure and the number of beta sheets, helices, etc. Compare this information to the structure we computed. & 8 & Download PDB, literature, simulate, RMSD, DSSP, get uniprot, subunit structure, get beta sheets/helices \\ \hline
13 & Compare the RMSF of 1UBQ under high pressure and low pressure. Perform the simulation for 1 ns, varying only the pressure. Plot the moments of inertia over time for both simulations. & 7 & Download PDB, simulate (x2), RMSF (x2), MOI (x2) \\ \hline
14 & Simulate deoxygenated hemoglobin (1A3N) and oxygenated hemoglobin (6BB5). Plot the PCA of both trajectories. & 6 & Download PDB (x2), simulate (x2), PCA (x2) \\ \hline
15 & Simulate trypsin (1TRN) for 1 ns at 300 K and plot eneRGy over time. Compute SASA, RMSF, and radius of gyration. Get the subunit structure, sequence, active and binding sites. & 9 & Download PDB, simulate, output figures, SASA, RMSF, RGy, subunit structure, sequence info, all known sites \\ \hline
16 & Download the PDB file for 1C3W and describe the secondary structure. Then, simulate the protein at 300 K for 1 ns. Plot the RMSD over time and the radius of gyration over time. & 5 & Download PDB, DSSP, simulate, RMSD, RGy \\ \hline
17 & Download the PDB file for 1XQ8, and then save the visualization for it. & 2 & Download PDB, visualize \\ \hline
18 & Download the PDB for 2YXF. Tell me about its stability as found in the literature. Then, simulate it for 1 ns and plot its RMSD over time. & 4 & Download PDB, literature search, simulate, RMSD \\ \hline
19 & Simulate 1MBN in water and methanol solutions. & 4 & Download PDB, packmol to get appropriate non-water solvent, simulate (x2) \\ \hline
20 & Download protein 1ATN. & 1 & Download PDB \\ \hline
21 & Download and clean protein 1A3N. & 2 & Download PDB, clean \\ \hline
22 & Perform a brief simulation of protein 1PQ2. & 2 & Download PDB, simulate \\ \hline
23 & Analyze the RDF of the simulation of 1A3N solvated in water. & 3 & Download PDB, simulate, RDF \\ \hline
24 & Simulate oxygenated hemoglobin (1A3N) and deoxygenated hemoglobin (6BB5). Then analyze the RDF of both. & 6 & Download PDB (x2), simulate (x2), RDF (x2) \\ \hline
25 & Simulate 1L6X at pH 5.0 and 8.8, then analyze the SASA and RMSF under both pH conditions. & 9 & Download PDB, clean at pH 5.5 and 8.0, simulate(x2), SASA(x2), RMSF(x2) \\ 
\bottomrule
\end{longtable}