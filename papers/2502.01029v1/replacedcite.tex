\section{Related Work}
Research in cryptocurrency fee prediction has evolved significantly, spanning multiple methodological approaches. We organize the relevant literature into three main categories: economic analysis, time series approaches, and advanced statistical methods.

\subsection{Economic Analysis of Transaction Fees}
Early work in cryptocurrency fee analysis focused on economic fundamentals. Easley et al.____ provided foundational insights into Bitcoin fee markets, establishing the relationship between transaction fees and mining economics. Their research demonstrated how fee dynamics emerge from the interaction between users and miners, though their analysis focused on equilibrium patterns rather than prediction. Houy____ developed one of the first analytical frameworks for Bitcoin fee economics, proposing models based on game theory and market equilibrium.

Building on these economic foundations, Moser and Böhme____ conducted longitudinal analysis of Bitcoin transaction fees, identifying key patterns in fee rate evolution. Kasahara and Kawahara____ extended this work by examining the effect of fees on transaction confirmation times, introducing temporal considerations that inform current predictive approaches.

\subsection{Time Series Modeling}
The application of time series analysis to cryptocurrency data marked a significant advancement in fee prediction methodology. Box et al.____ established the statistical foundations for analyzing temporal patterns in financial markets, providing techniques that would later prove crucial for cryptocurrency analysis. Building on these methods, Möser and Böhme____ expanded the understanding of Bitcoin transaction dynamics through empirical analysis of fee structures and network characteristics, demonstrating key relationships between network metrics and transaction costs.

Recent studies have extended time series approaches to cryptocurrency forecasting. For example, 
Estimating and forecasting bitcoin daily prices using ARIMA-GARCH models.
Phung Duy et al.____ explored hybrid ARIMA-GARCH models for Bitcoin price prediction, which offer insights for fee forecasting due to their shared temporal characteristics. Similarly, Fischer and Krauss____ utilized LSTM networks for financial forecasting, demonstrating their ability to capture long-term dependencies and volatility in sequential data.

\subsection{Advanced Statistical Methods}
Recent developments in machine learning and explainable AI have enabled more sophisticated approaches to cryptocurrency analysis. Vaswani et al.____ introduced transformer architectures that revolutionized temporal data processing, while Kazemi et al.____ developed temporal embeddings specifically designed for time series analysis. These advances in temporal modeling were complemented by Friedman's____ gradient boosting framework, which has proven particularly effective for cryptocurrency applications.

Transformer-based models have gained significant attention for financial forecasting. Lim et al.____ introduced the Temporal Fusion Transformer (TFT), which remains a cornerstone for interpretable multi-horizon time series forecasting. More recently, Muhammad et al.____ demonstrated how attention-based transformers can be applied to stock market prediction, emphasizing feature importance and interpretability. These approaches align with advancements in explainable AI, as discussed by Arrieta et al.____, ensuring transparency in model predictions.

Hybrid approaches have also gained traction. Vargas et al.____ proposed a hybrid approach combining convolutional neural networks (CNNs) and LSTMs for cryptocurrency price prediction, showing how feature extraction and sequential modeling can be integrated effectively. Similarly, Chong et al.____ utilized hybrid deep learning frameworks for financial time series forecasting, capturing both short-term trends and long-term dependencies.

Our research builds upon these foundations while addressing several critical gaps in the literature. Most existing studies focus on equilibrium analysis or price prediction rather than dynamic fee forecasting. Additionally, the interaction between network metrics, market conditions, and fee rates remains inadequately modeled for longer time horizons. Our work extends both the theoretical understanding and practical applications of fee prediction while incorporating comprehensive feature engineering, hybrid modeling, and transformer-based approaches.