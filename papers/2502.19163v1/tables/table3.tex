\begin{table*}[!th]
\centering
\resizebox{\textwidth}{!}{%
\begin{tabular}{@{}lcccccccccc@{}}
\toprule
\textbf{Aggregation Strategy} & \textbf{BANKING} & \textbf{CLINC} & \textbf{Reddit} & \textbf{StackEx} & \textbf{MTOP} & \textbf{CLINC(D)} & \textbf{FewEvent} & \textbf{GoEmotion} & \textbf{AVG} \\ \midrule \midrule
Standard Prompting & 0.652 & 0.792 & 0.534 & 0.482 & 0.896 & 0.536 & 0.630 & 0.378 & 0.613 \\ \midrule \midrule
Naive Majority Voting & 0.756 & 0.862 & 0.644 & 0.538 & 0.948 & 0.550 & 0.668 & 0.392 & 0.670 \\
Weighted Majority Voting (Distance) & 0.764 & 0.864 & 0.646 & 0.540 & 0.948 & 0.554 & 0.680 & 0.414 & 0.676 \\
Weighted Majority Voting (Distance \& Confidence) & \textbf{0.768} & 0.870 & 0.656 & 0.542 & 0.948 & 0.552 & 0.684 & \textbf{0.416} & 0.680 \\
Filtered Weighted Majority Voting & 0.762 & \textbf{0.876} & \textbf{0.688} & \textbf{0.542} & \textbf{0.954} & \textbf{0.572} & \textbf{0.688} & 0.410 & \textbf{0.687} \\
\bottomrule
\end{tabular}%
}
\caption{Comparison of aggregation strategies across diverse datasets. Naive majority voting already significantly improves accuracy over standard prompting. Weighted Majority Voting with distance and confidence further enhances performance, and filtering low-confidence predictions achieves the highest average result.}
\label{tab:voting_comparison}
\end{table*}