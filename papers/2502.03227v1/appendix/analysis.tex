\section{Ablation Studies}

\subsection{Reduncancy and Downstream Performance} \label{subapp:redundancy_analysis}

This section assesses the impact of the two problem formulations on downstream performance and correlation. 
We train the classification method from Section~\ref{sec:results_infomax} in three different settings: with standardization of the representations, without standardization but with a margin, and with neither of these. 
From the results reported in Table~\ref{tab:ablation_loss_clevr4_cls}, it can be observed that the standardized version (first row) achieves by far the lowest correlation level, but its accuracy on the \textit{texture}, \textit{color} and \textit{count} taxonomies is not better than the baseline from Table \ref{tab:results_clevr4_infomax}. 
In contrast, the margin loss (second row) performs well on all taxonomies. 
Finally, the method with neither a margin nor standardization (third row) performs poorly on the unknown taxonomies and has a much higher correlation level. This can be explained by analyzing that this method indefinitely increased the representation norm instead of decorrelating the variables, reaching an average representation norm of 584.4, while the version with a margin stabilized to around 56.1. We finally report the average accuracy and correlation for different margin values in Figure \ref{fig:clevr4_cls_margin_influence}. The figure demonstrates that accuracy increases with the margin up to a margin of 0.4, but that too large margins lead to poor performance on the unknown taxonomies. 

\begin{table}
    \small
    \centering
    \caption{Impact of different adversarial loss formulations on a classification problem.  The classifiers are trained on the \textit{shape} taxonomy from the Clevr-4 dataset.}
    \label{tab:ablation_loss_clevr4_cls}
    \vskip 0.15in
    \begin{tabular}{cc rrrr r} %
    \toprule
        \multicolumn{2}{c}{formulation} & \multicolumn{4}{c}{kNN top-1 accuracy} & \multirow{2}{*}{mean $\mathcal{R}^2$} \\
        std. & margin & shape & texture & color & count &  \\  %
        \midrule
        \cmark & \xmark & 100.0 & 21.2 & 16.9 & 22.1 & 0.009 \\ %
        \xmark & 0.4 & 100.0 & 83.7 & 100.0 & 39.6 & 0.068 \\ %
        \xmark & \xmark & 100.0 & 18.4 & 40.4 & 32.2 & 0.287 \\ %
    \bottomrule
    \end{tabular}
\end{table} %
\begin{figure} %
    \vspace{-10px}
    \centering
    \def\svgwidth{0.42\textwidth}
    {\footnotesize \input{CLEVR4_CLS_margin_and_acc_vs_corr.pdf_tex}}
    % \includesvg[pretex=\footnotesize,width=0.42\textwidth]{CLEVR4_CLS_margin_and_acc_vs_corr.svg}
    \vspace{-10px}
    \caption{Influence of the loss margin on the correlation and accuracy for classification models trained on the Clevr-4 dataset. The models are trained on the \textit{shape} taxonomy and the kNN accuracy is averaged over the three remaining taxonomies.}
    \label{fig:clevr4_cls_margin_influence}
\end{figure}

