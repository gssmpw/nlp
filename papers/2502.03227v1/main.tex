\documentclass{article} 

\usepackage{hyperref}
\usepackage{url}

\usepackage[preprint]{icml2025}

\usepackage{amsmath}
\usepackage{amssymb}
\usepackage{mathtools}
\usepackage{amsthm}
\usepackage{natbib}
\usepackage{algpseudocode}
\usepackage{algorithm}

\usepackage{multirow}
\usepackage{booktabs}
\usepackage{subcaption}
\usepackage{graphicx}
\usepackage{svg}
\usepackage{enumitem}
\usepackage{xcolor}
\usepackage{colortbl}
\usepackage{pifont}
\usepackage{wrapfig}

%%%%% NEW MATH DEFINITIONS %%%%%

\usepackage{amsmath,amsfonts,bm}
\usepackage{derivative}
% Mark sections of captions for referring to divisions of figures
\newcommand{\figleft}{{\em (Left)}}
\newcommand{\figcenter}{{\em (Center)}}
\newcommand{\figright}{{\em (Right)}}
\newcommand{\figtop}{{\em (Top)}}
\newcommand{\figbottom}{{\em (Bottom)}}
\newcommand{\captiona}{{\em (a)}}
\newcommand{\captionb}{{\em (b)}}
\newcommand{\captionc}{{\em (c)}}
\newcommand{\captiond}{{\em (d)}}

% Highlight a newly defined term
\newcommand{\newterm}[1]{{\bf #1}}

% Derivative d 
\newcommand{\deriv}{{\mathrm{d}}}

% Figure reference, lower-case.
\def\figref#1{figure~\ref{#1}}
% Figure reference, capital. For start of sentence
\def\Figref#1{Figure~\ref{#1}}
\def\twofigref#1#2{figures \ref{#1} and \ref{#2}}
\def\quadfigref#1#2#3#4{figures \ref{#1}, \ref{#2}, \ref{#3} and \ref{#4}}
% Section reference, lower-case.
\def\secref#1{section~\ref{#1}}
% Section reference, capital.
\def\Secref#1{Section~\ref{#1}}
% Reference to two sections.
\def\twosecrefs#1#2{sections \ref{#1} and \ref{#2}}
% Reference to three sections.
\def\secrefs#1#2#3{sections \ref{#1}, \ref{#2} and \ref{#3}}
% Reference to an equation, lower-case.
\def\eqref#1{equation~\ref{#1}}
% Reference to an equation, upper case
\def\Eqref#1{Equation~\ref{#1}}
% A raw reference to an equation---avoid using if possible
\def\plaineqref#1{\ref{#1}}
% Reference to a chapter, lower-case.
\def\chapref#1{chapter~\ref{#1}}
% Reference to an equation, upper case.
\def\Chapref#1{Chapter~\ref{#1}}
% Reference to a range of chapters
\def\rangechapref#1#2{chapters\ref{#1}--\ref{#2}}
% Reference to an algorithm, lower-case.
\def\algref#1{algorithm~\ref{#1}}
% Reference to an algorithm, upper case.
\def\Algref#1{Algorithm~\ref{#1}}
\def\twoalgref#1#2{algorithms \ref{#1} and \ref{#2}}
\def\Twoalgref#1#2{Algorithms \ref{#1} and \ref{#2}}
% Reference to a part, lower case
\def\partref#1{part~\ref{#1}}
% Reference to a part, upper case
\def\Partref#1{Part~\ref{#1}}
\def\twopartref#1#2{parts \ref{#1} and \ref{#2}}

\def\ceil#1{\lceil #1 \rceil}
\def\floor#1{\lfloor #1 \rfloor}
\def\1{\bm{1}}
\newcommand{\train}{\mathcal{D}}
\newcommand{\valid}{\mathcal{D_{\mathrm{valid}}}}
\newcommand{\test}{\mathcal{D_{\mathrm{test}}}}

\def\eps{{\epsilon}}


% Random variables
\def\reta{{\textnormal{$\eta$}}}
\def\ra{{\textnormal{a}}}
\def\rb{{\textnormal{b}}}
\def\rc{{\textnormal{c}}}
\def\rd{{\textnormal{d}}}
\def\re{{\textnormal{e}}}
\def\rf{{\textnormal{f}}}
\def\rg{{\textnormal{g}}}
\def\rh{{\textnormal{h}}}
\def\ri{{\textnormal{i}}}
\def\rj{{\textnormal{j}}}
\def\rk{{\textnormal{k}}}
\def\rl{{\textnormal{l}}}
% rm is already a command, just don't name any random variables m
\def\rn{{\textnormal{n}}}
\def\ro{{\textnormal{o}}}
\def\rp{{\textnormal{p}}}
\def\rq{{\textnormal{q}}}
\def\rr{{\textnormal{r}}}
\def\rs{{\textnormal{s}}}
\def\rt{{\textnormal{t}}}
\def\ru{{\textnormal{u}}}
\def\rv{{\textnormal{v}}}
\def\rw{{\textnormal{w}}}
\def\rx{{\textnormal{x}}}
\def\ry{{\textnormal{y}}}
\def\rz{{\textnormal{z}}}

% Random vectors
\def\rvepsilon{{\mathbf{\epsilon}}}
\def\rvphi{{\mathbf{\phi}}}
\def\rvtheta{{\mathbf{\theta}}}
\def\rva{{\mathbf{a}}}
\def\rvb{{\mathbf{b}}}
\def\rvc{{\mathbf{c}}}
\def\rvd{{\mathbf{d}}}
\def\rve{{\mathbf{e}}}
\def\rvf{{\mathbf{f}}}
\def\rvg{{\mathbf{g}}}
\def\rvh{{\mathbf{h}}}
\def\rvu{{\mathbf{i}}}
\def\rvj{{\mathbf{j}}}
\def\rvk{{\mathbf{k}}}
\def\rvl{{\mathbf{l}}}
\def\rvm{{\mathbf{m}}}
\def\rvn{{\mathbf{n}}}
\def\rvo{{\mathbf{o}}}
\def\rvp{{\mathbf{p}}}
\def\rvq{{\mathbf{q}}}
\def\rvr{{\mathbf{r}}}
\def\rvs{{\mathbf{s}}}
\def\rvt{{\mathbf{t}}}
\def\rvu{{\mathbf{u}}}
\def\rvv{{\mathbf{v}}}
\def\rvw{{\mathbf{w}}}
\def\rvx{{\mathbf{x}}}
\def\rvy{{\mathbf{y}}}
\def\rvz{{\mathbf{z}}}

% Elements of random vectors
\def\erva{{\textnormal{a}}}
\def\ervb{{\textnormal{b}}}
\def\ervc{{\textnormal{c}}}
\def\ervd{{\textnormal{d}}}
\def\erve{{\textnormal{e}}}
\def\ervf{{\textnormal{f}}}
\def\ervg{{\textnormal{g}}}
\def\ervh{{\textnormal{h}}}
\def\ervi{{\textnormal{i}}}
\def\ervj{{\textnormal{j}}}
\def\ervk{{\textnormal{k}}}
\def\ervl{{\textnormal{l}}}
\def\ervm{{\textnormal{m}}}
\def\ervn{{\textnormal{n}}}
\def\ervo{{\textnormal{o}}}
\def\ervp{{\textnormal{p}}}
\def\ervq{{\textnormal{q}}}
\def\ervr{{\textnormal{r}}}
\def\ervs{{\textnormal{s}}}
\def\ervt{{\textnormal{t}}}
\def\ervu{{\textnormal{u}}}
\def\ervv{{\textnormal{v}}}
\def\ervw{{\textnormal{w}}}
\def\ervx{{\textnormal{x}}}
\def\ervy{{\textnormal{y}}}
\def\ervz{{\textnormal{z}}}

% Random matrices
\def\rmA{{\mathbf{A}}}
\def\rmB{{\mathbf{B}}}
\def\rmC{{\mathbf{C}}}
\def\rmD{{\mathbf{D}}}
\def\rmE{{\mathbf{E}}}
\def\rmF{{\mathbf{F}}}
\def\rmG{{\mathbf{G}}}
\def\rmH{{\mathbf{H}}}
\def\rmI{{\mathbf{I}}}
\def\rmJ{{\mathbf{J}}}
\def\rmK{{\mathbf{K}}}
\def\rmL{{\mathbf{L}}}
\def\rmM{{\mathbf{M}}}
\def\rmN{{\mathbf{N}}}
\def\rmO{{\mathbf{O}}}
\def\rmP{{\mathbf{P}}}
\def\rmQ{{\mathbf{Q}}}
\def\rmR{{\mathbf{R}}}
\def\rmS{{\mathbf{S}}}
\def\rmT{{\mathbf{T}}}
\def\rmU{{\mathbf{U}}}
\def\rmV{{\mathbf{V}}}
\def\rmW{{\mathbf{W}}}
\def\rmX{{\mathbf{X}}}
\def\rmY{{\mathbf{Y}}}
\def\rmZ{{\mathbf{Z}}}

% Elements of random matrices
\def\ermA{{\textnormal{A}}}
\def\ermB{{\textnormal{B}}}
\def\ermC{{\textnormal{C}}}
\def\ermD{{\textnormal{D}}}
\def\ermE{{\textnormal{E}}}
\def\ermF{{\textnormal{F}}}
\def\ermG{{\textnormal{G}}}
\def\ermH{{\textnormal{H}}}
\def\ermI{{\textnormal{I}}}
\def\ermJ{{\textnormal{J}}}
\def\ermK{{\textnormal{K}}}
\def\ermL{{\textnormal{L}}}
\def\ermM{{\textnormal{M}}}
\def\ermN{{\textnormal{N}}}
\def\ermO{{\textnormal{O}}}
\def\ermP{{\textnormal{P}}}
\def\ermQ{{\textnormal{Q}}}
\def\ermR{{\textnormal{R}}}
\def\ermS{{\textnormal{S}}}
\def\ermT{{\textnormal{T}}}
\def\ermU{{\textnormal{U}}}
\def\ermV{{\textnormal{V}}}
\def\ermW{{\textnormal{W}}}
\def\ermX{{\textnormal{X}}}
\def\ermY{{\textnormal{Y}}}
\def\ermZ{{\textnormal{Z}}}

% Vectors
\def\vzero{{\bm{0}}}
\def\vone{{\bm{1}}}
\def\vmu{{\bm{\mu}}}
\def\vtheta{{\bm{\theta}}}
\def\vphi{{\bm{\phi}}}
\def\va{{\bm{a}}}
\def\vb{{\bm{b}}}
\def\vc{{\bm{c}}}
\def\vd{{\bm{d}}}
\def\ve{{\bm{e}}}
\def\vf{{\bm{f}}}
\def\vg{{\bm{g}}}
\def\vh{{\bm{h}}}
\def\vi{{\bm{i}}}
\def\vj{{\bm{j}}}
\def\vk{{\bm{k}}}
\def\vl{{\bm{l}}}
\def\vm{{\bm{m}}}
\def\vn{{\bm{n}}}
\def\vo{{\bm{o}}}
\def\vp{{\bm{p}}}
\def\vq{{\bm{q}}}
\def\vr{{\bm{r}}}
\def\vs{{\bm{s}}}
\def\vt{{\bm{t}}}
\def\vu{{\bm{u}}}
\def\vv{{\bm{v}}}
\def\vw{{\bm{w}}}
\def\vx{{\bm{x}}}
\def\vy{{\bm{y}}}
\def\vz{{\bm{z}}}

% Elements of vectors
\def\evalpha{{\alpha}}
\def\evbeta{{\beta}}
\def\evepsilon{{\epsilon}}
\def\evlambda{{\lambda}}
\def\evomega{{\omega}}
\def\evmu{{\mu}}
\def\evpsi{{\psi}}
\def\evsigma{{\sigma}}
\def\evtheta{{\theta}}
\def\eva{{a}}
\def\evb{{b}}
\def\evc{{c}}
\def\evd{{d}}
\def\eve{{e}}
\def\evf{{f}}
\def\evg{{g}}
\def\evh{{h}}
\def\evi{{i}}
\def\evj{{j}}
\def\evk{{k}}
\def\evl{{l}}
\def\evm{{m}}
\def\evn{{n}}
\def\evo{{o}}
\def\evp{{p}}
\def\evq{{q}}
\def\evr{{r}}
\def\evs{{s}}
\def\evt{{t}}
\def\evu{{u}}
\def\evv{{v}}
\def\evw{{w}}
\def\evx{{x}}
\def\evy{{y}}
\def\evz{{z}}

% Matrix
\def\mA{{\bm{A}}}
\def\mB{{\bm{B}}}
\def\mC{{\bm{C}}}
\def\mD{{\bm{D}}}
\def\mE{{\bm{E}}}
\def\mF{{\bm{F}}}
\def\mG{{\bm{G}}}
\def\mH{{\bm{H}}}
\def\mI{{\bm{I}}}
\def\mJ{{\bm{J}}}
\def\mK{{\bm{K}}}
\def\mL{{\bm{L}}}
\def\mM{{\bm{M}}}
\def\mN{{\bm{N}}}
\def\mO{{\bm{O}}}
\def\mP{{\bm{P}}}
\def\mQ{{\bm{Q}}}
\def\mR{{\bm{R}}}
\def\mS{{\bm{S}}}
\def\mT{{\bm{T}}}
\def\mU{{\bm{U}}}
\def\mV{{\bm{V}}}
\def\mW{{\bm{W}}}
\def\mX{{\bm{X}}}
\def\mY{{\bm{Y}}}
\def\mZ{{\bm{Z}}}
\def\mBeta{{\bm{\beta}}}
\def\mPhi{{\bm{\Phi}}}
\def\mLambda{{\bm{\Lambda}}}
\def\mSigma{{\bm{\Sigma}}}

% Tensor
\DeclareMathAlphabet{\mathsfit}{\encodingdefault}{\sfdefault}{m}{sl}
\SetMathAlphabet{\mathsfit}{bold}{\encodingdefault}{\sfdefault}{bx}{n}
\newcommand{\tens}[1]{\bm{\mathsfit{#1}}}
\def\tA{{\tens{A}}}
\def\tB{{\tens{B}}}
\def\tC{{\tens{C}}}
\def\tD{{\tens{D}}}
\def\tE{{\tens{E}}}
\def\tF{{\tens{F}}}
\def\tG{{\tens{G}}}
\def\tH{{\tens{H}}}
\def\tI{{\tens{I}}}
\def\tJ{{\tens{J}}}
\def\tK{{\tens{K}}}
\def\tL{{\tens{L}}}
\def\tM{{\tens{M}}}
\def\tN{{\tens{N}}}
\def\tO{{\tens{O}}}
\def\tP{{\tens{P}}}
\def\tQ{{\tens{Q}}}
\def\tR{{\tens{R}}}
\def\tS{{\tens{S}}}
\def\tT{{\tens{T}}}
\def\tU{{\tens{U}}}
\def\tV{{\tens{V}}}
\def\tW{{\tens{W}}}
\def\tX{{\tens{X}}}
\def\tY{{\tens{Y}}}
\def\tZ{{\tens{Z}}}


% Graph
\def\gA{{\mathcal{A}}}
\def\gB{{\mathcal{B}}}
\def\gC{{\mathcal{C}}}
\def\gD{{\mathcal{D}}}
\def\gE{{\mathcal{E}}}
\def\gF{{\mathcal{F}}}
\def\gG{{\mathcal{G}}}
\def\gH{{\mathcal{H}}}
\def\gI{{\mathcal{I}}}
\def\gJ{{\mathcal{J}}}
\def\gK{{\mathcal{K}}}
\def\gL{{\mathcal{L}}}
\def\gM{{\mathcal{M}}}
\def\gN{{\mathcal{N}}}
\def\gO{{\mathcal{O}}}
\def\gP{{\mathcal{P}}}
\def\gQ{{\mathcal{Q}}}
\def\gR{{\mathcal{R}}}
\def\gS{{\mathcal{S}}}
\def\gT{{\mathcal{T}}}
\def\gU{{\mathcal{U}}}
\def\gV{{\mathcal{V}}}
\def\gW{{\mathcal{W}}}
\def\gX{{\mathcal{X}}}
\def\gY{{\mathcal{Y}}}
\def\gZ{{\mathcal{Z}}}

% Sets
\def\sA{{\mathbb{A}}}
\def\sB{{\mathbb{B}}}
\def\sC{{\mathbb{C}}}
\def\sD{{\mathbb{D}}}
% Don't use a set called E, because this would be the same as our symbol
% for expectation.
\def\sF{{\mathbb{F}}}
\def\sG{{\mathbb{G}}}
\def\sH{{\mathbb{H}}}
\def\sI{{\mathbb{I}}}
\def\sJ{{\mathbb{J}}}
\def\sK{{\mathbb{K}}}
\def\sL{{\mathbb{L}}}
\def\sM{{\mathbb{M}}}
\def\sN{{\mathbb{N}}}
\def\sO{{\mathbb{O}}}
\def\sP{{\mathbb{P}}}
\def\sQ{{\mathbb{Q}}}
\def\sR{{\mathbb{R}}}
\def\sS{{\mathbb{S}}}
\def\sT{{\mathbb{T}}}
\def\sU{{\mathbb{U}}}
\def\sV{{\mathbb{V}}}
\def\sW{{\mathbb{W}}}
\def\sX{{\mathbb{X}}}
\def\sY{{\mathbb{Y}}}
\def\sZ{{\mathbb{Z}}}

% Entries of a matrix
\def\emLambda{{\Lambda}}
\def\emA{{A}}
\def\emB{{B}}
\def\emC{{C}}
\def\emD{{D}}
\def\emE{{E}}
\def\emF{{F}}
\def\emG{{G}}
\def\emH{{H}}
\def\emI{{I}}
\def\emJ{{J}}
\def\emK{{K}}
\def\emL{{L}}
\def\emM{{M}}
\def\emN{{N}}
\def\emO{{O}}
\def\emP{{P}}
\def\emQ{{Q}}
\def\emR{{R}}
\def\emS{{S}}
\def\emT{{T}}
\def\emU{{U}}
\def\emV{{V}}
\def\emW{{W}}
\def\emX{{X}}
\def\emY{{Y}}
\def\emZ{{Z}}
\def\emSigma{{\Sigma}}

% entries of a tensor
% Same font as tensor, without \bm wrapper
\newcommand{\etens}[1]{\mathsfit{#1}}
\def\etLambda{{\etens{\Lambda}}}
\def\etA{{\etens{A}}}
\def\etB{{\etens{B}}}
\def\etC{{\etens{C}}}
\def\etD{{\etens{D}}}
\def\etE{{\etens{E}}}
\def\etF{{\etens{F}}}
\def\etG{{\etens{G}}}
\def\etH{{\etens{H}}}
\def\etI{{\etens{I}}}
\def\etJ{{\etens{J}}}
\def\etK{{\etens{K}}}
\def\etL{{\etens{L}}}
\def\etM{{\etens{M}}}
\def\etN{{\etens{N}}}
\def\etO{{\etens{O}}}
\def\etP{{\etens{P}}}
\def\etQ{{\etens{Q}}}
\def\etR{{\etens{R}}}
\def\etS{{\etens{S}}}
\def\etT{{\etens{T}}}
\def\etU{{\etens{U}}}
\def\etV{{\etens{V}}}
\def\etW{{\etens{W}}}
\def\etX{{\etens{X}}}
\def\etY{{\etens{Y}}}
\def\etZ{{\etens{Z}}}

% The true underlying data generating distribution
\newcommand{\pdata}{p_{\rm{data}}}
\newcommand{\ptarget}{p_{\rm{target}}}
\newcommand{\pprior}{p_{\rm{prior}}}
\newcommand{\pbase}{p_{\rm{base}}}
\newcommand{\pref}{p_{\rm{ref}}}

% The empirical distribution defined by the training set
\newcommand{\ptrain}{\hat{p}_{\rm{data}}}
\newcommand{\Ptrain}{\hat{P}_{\rm{data}}}
% The model distribution
\newcommand{\pmodel}{p_{\rm{model}}}
\newcommand{\Pmodel}{P_{\rm{model}}}
\newcommand{\ptildemodel}{\tilde{p}_{\rm{model}}}
% Stochastic autoencoder distributions
\newcommand{\pencode}{p_{\rm{encoder}}}
\newcommand{\pdecode}{p_{\rm{decoder}}}
\newcommand{\precons}{p_{\rm{reconstruct}}}

\newcommand{\laplace}{\mathrm{Laplace}} % Laplace distribution

\newcommand{\E}{\mathbb{E}}
\newcommand{\Ls}{\mathcal{L}}
\newcommand{\R}{\mathbb{R}}
\newcommand{\emp}{\tilde{p}}
\newcommand{\lr}{\alpha}
\newcommand{\reg}{\lambda}
\newcommand{\rect}{\mathrm{rectifier}}
\newcommand{\softmax}{\mathrm{softmax}}
\newcommand{\sigmoid}{\sigma}
\newcommand{\softplus}{\zeta}
\newcommand{\KL}{D_{\mathrm{KL}}}
\newcommand{\Var}{\mathrm{Var}}
\newcommand{\standarderror}{\mathrm{SE}}
\newcommand{\Cov}{\mathrm{Cov}}
% Wolfram Mathworld says $L^2$ is for function spaces and $\ell^2$ is for vectors
% But then they seem to use $L^2$ for vectors throughout the site, and so does
% wikipedia.
\newcommand{\normlzero}{L^0}
\newcommand{\normlone}{L^1}
\newcommand{\normltwo}{L^2}
\newcommand{\normlp}{L^p}
\newcommand{\normmax}{L^\infty}

\newcommand{\parents}{Pa} % See usage in notation.tex. Chosen to match Daphne's book.

\DeclareMathOperator*{\argmax}{arg\,max}
\DeclareMathOperator*{\argmin}{arg\,min}

\DeclareMathOperator{\sign}{sign}
\DeclareMathOperator{\Tr}{Tr}
\let\ab\allowbreak


\usepackage[capitalize,noabbrev]{cleveref}


\newcommand{\cmark}{\ding{51}}  %
\newcommand{\xmark}{\ding{55}}  %

\makeatletter
\renewcommand{\paragraph}{%
  \@startsection{paragraph}{4}%
  {\z@}{0em}{-0.5em}%
  {\normalfont\normalsize\bfseries}%
}
\makeatother

\newcommand{\indep}{\perp \!\!\! \perp}

\definecolor{lightgray}{HTML}{EFEFEF}

\newcommand{\updated}[1]{{\color{blue}#1}}

\theoremstyle{plain}
\newtheorem{theorem}{Theorem}[section]
\newtheorem{proposition}[theorem]{Proposition}
\newtheorem{lemma}[theorem]{Lemma}
\newtheorem{corollary}[theorem]{Corollary}
\theoremstyle{definition}
\newtheorem{definition}[theorem]{Definition}
\newtheorem{assumption}[theorem]{Assumption}
\newtheorem{example}{Example}
\theoremstyle{remark}
\newtheorem{remark}[theorem]{Remark}

\newcommand{\fix}{\marginpar{FIX}}
\newcommand{\new}{\marginpar{NEW}}


\icmltitlerunning{Adversarial Dependence Minimization}

\begin{document}

\twocolumn[
\icmltitle{Adversarial Dependence Minimization}

\icmlsetsymbol{equal}{*}

\begin{icmlauthorlist}
\icmlauthor{Pierre-François De~Plaen}{kuleuven}
\icmlauthor{Tinne Tuytelaars}{kuleuven}
\icmlauthor{Marc Proesmans}{kuleuven}
\icmlauthor{Luc Van~Gool}{kuleuven,ethz,sofia}
\end{icmlauthorlist}

\icmlaffiliation{kuleuven}{ESAT-PSI, KU Leuven, Belgium}
\icmlaffiliation{ethz}{CVL, ETH Zürich, Switzerland}
\icmlaffiliation{sofia}{INSAIT, Sofia University, Bulgaria}

\icmlcorrespondingauthor{Pierre-François De~Plaen}{pdeplaen@esat.kuleuven.be}

\icmlkeywords{representation learning, adversarial networks, independence, PCA, ICA, SSL, generalization}

\vskip 0.3in
]

\printAffiliationsAndNotice{}

\begin{abstract}
Many machine learning techniques rely on minimizing the covariance between output feature dimensions to extract minimally redundant representations from data. 
However, these methods do not eliminate all dependencies/redundancies, as linearly uncorrelated variables can still exhibit nonlinear relationships. 
This work provides a differentiable and scalable algorithm for dependence minimization that goes beyond linear pairwise decorrelation. 
Our method employs an adversarial game where small networks identify dependencies among feature dimensions, while the encoder exploits this information to reduce dependencies. 
We provide empirical evidence of the algorithm's convergence and demonstrate its utility in three applications: 
extending PCA to nonlinear decorrelation, improving the generalization of image classification methods, and preventing dimensional collapse in self-supervised representation learning. 
\end{abstract}

\section{Introduction}
\section{Introduction}
\label{sec:introduction}
The business processes of organizations are experiencing ever-increasing complexity due to the large amount of data, high number of users, and high-tech devices involved \cite{martin2021pmopportunitieschallenges, beerepoot2023biggestbpmproblems}. This complexity may cause business processes to deviate from normal control flow due to unforeseen and disruptive anomalies \cite{adams2023proceddsriftdetection}. These control-flow anomalies manifest as unknown, skipped, and wrongly-ordered activities in the traces of event logs monitored from the execution of business processes \cite{ko2023adsystematicreview}. For the sake of clarity, let us consider an illustrative example of such anomalies. Figure \ref{FP_ANOMALIES} shows a so-called event log footprint, which captures the control flow relations of four activities of a hypothetical event log. In particular, this footprint captures the control-flow relations between activities \texttt{a}, \texttt{b}, \texttt{c} and \texttt{d}. These are the causal ($\rightarrow$) relation, concurrent ($\parallel$) relation, and other ($\#$) relations such as exclusivity or non-local dependency \cite{aalst2022pmhandbook}. In addition, on the right are six traces, of which five exhibit skipped, wrongly-ordered and unknown control-flow anomalies. For example, $\langle$\texttt{a b d}$\rangle$ has a skipped activity, which is \texttt{c}. Because of this skipped activity, the control-flow relation \texttt{b}$\,\#\,$\texttt{d} is violated, since \texttt{d} directly follows \texttt{b} in the anomalous trace.
\begin{figure}[!t]
\centering
\includegraphics[width=0.9\columnwidth]{images/FP_ANOMALIES.png}
\caption{An example event log footprint with six traces, of which five exhibit control-flow anomalies.}
\label{FP_ANOMALIES}
\end{figure}

\subsection{Control-flow anomaly detection}
Control-flow anomaly detection techniques aim to characterize the normal control flow from event logs and verify whether these deviations occur in new event logs \cite{ko2023adsystematicreview}. To develop control-flow anomaly detection techniques, \revision{process mining} has seen widespread adoption owing to process discovery and \revision{conformance checking}. On the one hand, process discovery is a set of algorithms that encode control-flow relations as a set of model elements and constraints according to a given modeling formalism \cite{aalst2022pmhandbook}; hereafter, we refer to the Petri net, a widespread modeling formalism. On the other hand, \revision{conformance checking} is an explainable set of algorithms that allows linking any deviations with the reference Petri net and providing the fitness measure, namely a measure of how much the Petri net fits the new event log \cite{aalst2022pmhandbook}. Many control-flow anomaly detection techniques based on \revision{conformance checking} (hereafter, \revision{conformance checking}-based techniques) use the fitness measure to determine whether an event log is anomalous \cite{bezerra2009pmad, bezerra2013adlogspais, myers2018icsadpm, pecchia2020applicationfailuresanalysispm}. 

The scientific literature also includes many \revision{conformance checking}-independent techniques for control-flow anomaly detection that combine specific types of trace encodings with machine/deep learning \cite{ko2023adsystematicreview, tavares2023pmtraceencoding}. Whereas these techniques are very effective, their explainability is challenging due to both the type of trace encoding employed and the machine/deep learning model used \cite{rawal2022trustworthyaiadvances,li2023explainablead}. Hence, in the following, we focus on the shortcomings of \revision{conformance checking}-based techniques to investigate whether it is possible to support the development of competitive control-flow anomaly detection techniques while maintaining the explainable nature of \revision{conformance checking}.
\begin{figure}[!t]
\centering
\includegraphics[width=\columnwidth]{images/HIGH_LEVEL_VIEW.png}
\caption{A high-level view of the proposed framework for combining \revision{process mining}-based feature extraction with dimensionality reduction for control-flow anomaly detection.}
\label{HIGH_LEVEL_VIEW}
\end{figure}

\subsection{Shortcomings of \revision{conformance checking}-based techniques}
Unfortunately, the detection effectiveness of \revision{conformance checking}-based techniques is affected by noisy data and low-quality Petri nets, which may be due to human errors in the modeling process or representational bias of process discovery algorithms \cite{bezerra2013adlogspais, pecchia2020applicationfailuresanalysispm, aalst2016pm}. Specifically, on the one hand, noisy data may introduce infrequent and deceptive control-flow relations that may result in inconsistent fitness measures, whereas, on the other hand, checking event logs against a low-quality Petri net could lead to an unreliable distribution of fitness measures. Nonetheless, such Petri nets can still be used as references to obtain insightful information for \revision{process mining}-based feature extraction, supporting the development of competitive and explainable \revision{conformance checking}-based techniques for control-flow anomaly detection despite the problems above. For example, a few works outline that token-based \revision{conformance checking} can be used for \revision{process mining}-based feature extraction to build tabular data and develop effective \revision{conformance checking}-based techniques for control-flow anomaly detection \cite{singh2022lapmsh, debenedictis2023dtadiiot}. However, to the best of our knowledge, the scientific literature lacks a structured proposal for \revision{process mining}-based feature extraction using the state-of-the-art \revision{conformance checking} variant, namely alignment-based \revision{conformance checking}.

\subsection{Contributions}
We propose a novel \revision{process mining}-based feature extraction approach with alignment-based \revision{conformance checking}. This variant aligns the deviating control flow with a reference Petri net; the resulting alignment can be inspected to extract additional statistics such as the number of times a given activity caused mismatches \cite{aalst2022pmhandbook}. We integrate this approach into a flexible and explainable framework for developing techniques for control-flow anomaly detection. The framework combines \revision{process mining}-based feature extraction and dimensionality reduction to handle high-dimensional feature sets, achieve detection effectiveness, and support explainability. Notably, in addition to our proposed \revision{process mining}-based feature extraction approach, the framework allows employing other approaches, enabling a fair comparison of multiple \revision{conformance checking}-based and \revision{conformance checking}-independent techniques for control-flow anomaly detection. Figure \ref{HIGH_LEVEL_VIEW} shows a high-level view of the framework. Business processes are monitored, and event logs obtained from the database of information systems. Subsequently, \revision{process mining}-based feature extraction is applied to these event logs and tabular data input to dimensionality reduction to identify control-flow anomalies. We apply several \revision{conformance checking}-based and \revision{conformance checking}-independent framework techniques to publicly available datasets, simulated data of a case study from railways, and real-world data of a case study from healthcare. We show that the framework techniques implementing our approach outperform the baseline \revision{conformance checking}-based techniques while maintaining the explainable nature of \revision{conformance checking}.

In summary, the contributions of this paper are as follows.
\begin{itemize}
    \item{
        A novel \revision{process mining}-based feature extraction approach to support the development of competitive and explainable \revision{conformance checking}-based techniques for control-flow anomaly detection.
    }
    \item{
        A flexible and explainable framework for developing techniques for control-flow anomaly detection using \revision{process mining}-based feature extraction and dimensionality reduction.
    }
    \item{
        Application to synthetic and real-world datasets of several \revision{conformance checking}-based and \revision{conformance checking}-independent framework techniques, evaluating their detection effectiveness and explainability.
    }
\end{itemize}

The rest of the paper is organized as follows.
\begin{itemize}
    \item Section \ref{sec:related_work} reviews the existing techniques for control-flow anomaly detection, categorizing them into \revision{conformance checking}-based and \revision{conformance checking}-independent techniques.
    \item Section \ref{sec:abccfe} provides the preliminaries of \revision{process mining} to establish the notation used throughout the paper, and delves into the details of the proposed \revision{process mining}-based feature extraction approach with alignment-based \revision{conformance checking}.
    \item Section \ref{sec:framework} describes the framework for developing \revision{conformance checking}-based and \revision{conformance checking}-independent techniques for control-flow anomaly detection that combine \revision{process mining}-based feature extraction and dimensionality reduction.
    \item Section \ref{sec:evaluation} presents the experiments conducted with multiple framework and baseline techniques using data from publicly available datasets and case studies.
    \item Section \ref{sec:conclusions} draws the conclusions and presents future work.
\end{itemize}

\section{Related Work}
\section{RELATED WORK}
\label{sec:relatedwork}
In this section, we describe the previous works related to our proposal, which are divided into two parts. In Section~\ref{sec:relatedwork_exoplanet}, we present a review of approaches based on machine learning techniques for the detection of planetary transit signals. Section~\ref{sec:relatedwork_attention} provides an account of the approaches based on attention mechanisms applied in Astronomy.\par

\subsection{Exoplanet detection}
\label{sec:relatedwork_exoplanet}
Machine learning methods have achieved great performance for the automatic selection of exoplanet transit signals. One of the earliest applications of machine learning is a model named Autovetter \citep{MCcauliff}, which is a random forest (RF) model based on characteristics derived from Kepler pipeline statistics to classify exoplanet and false positive signals. Then, other studies emerged that also used supervised learning. \cite{mislis2016sidra} also used a RF, but unlike the work by \citet{MCcauliff}, they used simulated light curves and a box least square \citep[BLS;][]{kovacs2002box}-based periodogram to search for transiting exoplanets. \citet{thompson2015machine} proposed a k-nearest neighbors model for Kepler data to determine if a given signal has similarity to known transits. Unsupervised learning techniques were also applied, such as self-organizing maps (SOM), proposed \citet{armstrong2016transit}; which implements an architecture to segment similar light curves. In the same way, \citet{armstrong2018automatic} developed a combination of supervised and unsupervised learning, including RF and SOM models. In general, these approaches require a previous phase of feature engineering for each light curve. \par

%DL is a modern data-driven technology that automatically extracts characteristics, and that has been successful in classification problems from a variety of application domains. The architecture relies on several layers of NNs of simple interconnected units and uses layers to build increasingly complex and useful features by means of linear and non-linear transformation. This family of models is capable of generating increasingly high-level representations \citep{lecun2015deep}.

The application of DL for exoplanetary signal detection has evolved rapidly in recent years and has become very popular in planetary science.  \citet{pearson2018} and \citet{zucker2018shallow} developed CNN-based algorithms that learn from synthetic data to search for exoplanets. Perhaps one of the most successful applications of the DL models in transit detection was that of \citet{Shallue_2018}; who, in collaboration with Google, proposed a CNN named AstroNet that recognizes exoplanet signals in real data from Kepler. AstroNet uses the training set of labelled TCEs from the Autovetter planet candidate catalog of Q1–Q17 data release 24 (DR24) of the Kepler mission \citep{catanzarite2015autovetter}. AstroNet analyses the data in two views: a ``global view'', and ``local view'' \citep{Shallue_2018}. \par


% The global view shows the characteristics of the light curve over an orbital period, and a local view shows the moment at occurring the transit in detail

%different = space-based

Based on AstroNet, researchers have modified the original AstroNet model to rank candidates from different surveys, specifically for Kepler and TESS missions. \citet{ansdell2018scientific} developed a CNN trained on Kepler data, and included for the first time the information on the centroids, showing that the model improves performance considerably. Then, \citet{osborn2020rapid} and \citet{yu2019identifying} also included the centroids information, but in addition, \citet{osborn2020rapid} included information of the stellar and transit parameters. Finally, \citet{rao2021nigraha} proposed a pipeline that includes a new ``half-phase'' view of the transit signal. This half-phase view represents a transit view with a different time and phase. The purpose of this view is to recover any possible secondary eclipse (the object hiding behind the disk of the primary star).


%last pipeline applies a procedure after the prediction of the model to obtain new candidates, this process is carried out through a series of steps that include the evaluation with Discovery and Validation of Exoplanets (DAVE) \citet{kostov2019discovery} that was adapted for the TESS telescope.\par
%



\subsection{Attention mechanisms in astronomy}
\label{sec:relatedwork_attention}
Despite the remarkable success of attention mechanisms in sequential data, few papers have exploited their advantages in astronomy. In particular, there are no models based on attention mechanisms for detecting planets. Below we present a summary of the main applications of this modeling approach to astronomy, based on two points of view; performance and interpretability of the model.\par
%Attention mechanisms have not yet been explored in all sub-areas of astronomy. However, recent works show a successful application of the mechanism.
%performance

The application of attention mechanisms has shown improvements in the performance of some regression and classification tasks compared to previous approaches. One of the first implementations of the attention mechanism was to find gravitational lenses proposed by \citet{thuruthipilly2021finding}. They designed 21 self-attention-based encoder models, where each model was trained separately with 18,000 simulated images, demonstrating that the model based on the Transformer has a better performance and uses fewer trainable parameters compared to CNN. A novel application was proposed by \citet{lin2021galaxy} for the morphological classification of galaxies, who used an architecture derived from the Transformer, named Vision Transformer (VIT) \citep{dosovitskiy2020image}. \citet{lin2021galaxy} demonstrated competitive results compared to CNNs. Another application with successful results was proposed by \citet{zerveas2021transformer}; which first proposed a transformer-based framework for learning unsupervised representations of multivariate time series. Their methodology takes advantage of unlabeled data to train an encoder and extract dense vector representations of time series. Subsequently, they evaluate the model for regression and classification tasks, demonstrating better performance than other state-of-the-art supervised methods, even with data sets with limited samples.

%interpretation
Regarding the interpretability of the model, a recent contribution that analyses the attention maps was presented by \citet{bowles20212}, which explored the use of group-equivariant self-attention for radio astronomy classification. Compared to other approaches, this model analysed the attention maps of the predictions and showed that the mechanism extracts the brightest spots and jets of the radio source more clearly. This indicates that attention maps for prediction interpretation could help experts see patterns that the human eye often misses. \par

In the field of variable stars, \citet{allam2021paying} employed the mechanism for classifying multivariate time series in variable stars. And additionally, \citet{allam2021paying} showed that the activation weights are accommodated according to the variation in brightness of the star, achieving a more interpretable model. And finally, related to the TESS telescope, \citet{morvan2022don} proposed a model that removes the noise from the light curves through the distribution of attention weights. \citet{morvan2022don} showed that the use of the attention mechanism is excellent for removing noise and outliers in time series datasets compared with other approaches. In addition, the use of attention maps allowed them to show the representations learned from the model. \par

Recent attention mechanism approaches in astronomy demonstrate comparable results with earlier approaches, such as CNNs. At the same time, they offer interpretability of their results, which allows a post-prediction analysis. \par



\section{Background and Motivation}
\section{Background}\label{sec:backgrnd}

\subsection{Cold Start Latency and Mitigation Techniques}

Traditional FaaS platforms mitigate cold starts through snapshotting, lightweight virtualization, and warm-state management. Snapshot-based methods like \textbf{REAP} and \textbf{Catalyzer} reduce initialization time by preloading or restoring container states but require significant memory and I/O resources, limiting scalability~\cite{dong_catalyzer_2020, ustiugov_benchmarking_2021}. Lightweight virtualization solutions, such as \textbf{Firecracker} microVMs, achieve fast startup times with strong isolation but depend on robust infrastructure, making them less adaptable to fluctuating workloads~\cite{agache_firecracker_2020}. Warm-state management techniques like \textbf{Faa\$T}~\cite{romero_faa_2021} and \textbf{Kraken}~\cite{vivek_kraken_2021} keep frequently invoked containers ready, balancing readiness and cost efficiency under predictable workloads but incurring overhead when demand is erratic~\cite{romero_faa_2021, vivek_kraken_2021}. While these methods perform well in resource-rich cloud environments, their resource intensity challenges applicability in edge settings.

\subsubsection{Edge FaaS Perspective}

In edge environments, cold start mitigation emphasizes lightweight designs, resource sharing, and hybrid task distribution. Lightweight execution environments like unikernels~\cite{edward_sock_2018} and \textbf{Firecracker}~\cite{agache_firecracker_2020}, as used by \textbf{TinyFaaS}~\cite{pfandzelter_tinyfaas_2020}, minimize resource usage and initialization delays but require careful orchestration to avoid resource contention. Function co-location, demonstrated by \textbf{Photons}~\cite{v_dukic_photons_2020}, reduces redundant initializations by sharing runtime resources among related functions, though this complicates isolation in multi-tenant setups~\cite{v_dukic_photons_2020}. Hybrid offloading frameworks like \textbf{GeoFaaS}~\cite{malekabbasi_geofaas_2024} balance edge-cloud workloads by offloading latency-tolerant tasks to the cloud and reserving edge resources for real-time operations, requiring reliable connectivity and efficient task management. These edge-specific strategies address cold starts effectively but introduce challenges in scalability and orchestration.

\subsection{Predictive Scaling and Caching Techniques}

Efficient resource allocation is vital for maintaining low latency and high availability in serverless platforms. Predictive scaling and caching techniques dynamically provision resources and reduce cold start latency by leveraging workload prediction and state retention.
Traditional FaaS platforms use predictive scaling and caching to optimize resources, employing techniques (OFC, FaasCache) to reduce cold starts. However, these methods rely on centralized orchestration and workload predictability, limiting their effectiveness in dynamic, resource-constrained edge environments.



\subsubsection{Edge FaaS Perspective}

Edge FaaS platforms adapt predictive scaling and caching techniques to constrain resources and heterogeneous environments. \textbf{EDGE-Cache}~\cite{kim_delay-aware_2022} uses traffic profiling to selectively retain high-priority functions, reducing memory overhead while maintaining readiness for frequent requests. Hybrid frameworks like \textbf{GeoFaaS}~\cite{malekabbasi_geofaas_2024} implement distributed caching to balance resources between edge and cloud nodes, enabling low-latency processing for critical tasks while offloading less critical workloads. Machine learning methods, such as clustering-based workload predictors~\cite{gao_machine_2020} and GRU-based models~\cite{guo_applying_2018}, enhance resource provisioning in edge systems by efficiently forecasting workload spikes. These innovations effectively address cold start challenges in edge environments, though their dependency on accurate predictions and robust orchestration poses scalability challenges.

\subsection{Decentralized Orchestration, Function Placement, and Scheduling}

Efficient orchestration in serverless platforms involves workload distribution, resource optimization, and performance assurance. While traditional FaaS platforms rely on centralized control, edge environments require decentralized and adaptive strategies to address unique challenges such as resource constraints and heterogeneous hardware.



\subsubsection{Edge FaaS Perspective}

Edge FaaS platforms adopt decentralized and adaptive orchestration frameworks to meet the demands of resource-constrained environments. Systems like \textbf{Wukong} distribute scheduling across edge nodes, enhancing data locality and scalability while reducing network latency. Lightweight frameworks such as \textbf{OpenWhisk Lite}~\cite{kravchenko_kpavelopenwhisk-light_2024} optimize resource allocation by decentralizing scheduling policies, minimizing cold starts and latency in edge setups~\cite{benjamin_wukong_2020}. Hybrid solutions like \textbf{OpenFaaS}~\cite{noauthor_openfaasfaas_2024} and \textbf{EdgeMatrix}~\cite{shen_edgematrix_2023} combine edge-cloud orchestration to balance resource utilization, retaining latency-sensitive functions at the edge while offloading non-critical workloads to the cloud. While these approaches improve flexibility, they face challenges in maintaining coordination and ensuring consistent performance across distributed nodes.



\section{Adversarial Dependence Minimization}
\section{Method}\label{sec:method}
\begin{figure}
    \centering
    \includegraphics[width=0.85\textwidth]{imgs/heatmap_acc.pdf}
    \caption{\textbf{Visualization of the proposed periodic Bayesian flow with mean parameter $\mu$ and accumulated accuracy parameter $c$ which corresponds to the entropy/uncertainty}. For $x = 0.3, \beta(1) = 1000$ and $\alpha_i$ defined in \cref{appd:bfn_cir}, this figure plots three colored stochastic parameter trajectories for receiver mean parameter $m$ and accumulated accuracy parameter $c$, superimposed on a log-scale heatmap of the Bayesian flow distribution $p_F(m|x,\senderacc)$ and $p_F(c|x,\senderacc)$. Note the \emph{non-monotonicity} and \emph{non-additive} property of $c$ which could inform the network the entropy of the mean parameter $m$ as a condition and the \emph{periodicity} of $m$. %\jj{Shrink the figures to save space}\hanlin{Do we need to make this figure one-column?}
    }
    \label{fig:vmbf_vis}
    \vskip -0.1in
\end{figure}
% \begin{wrapfigure}{r}{0.5\textwidth}
%     \centering
%     \includegraphics[width=0.49\textwidth]{imgs/heatmap_acc.pdf}
%     \caption{\textbf{Visualization of hyper-torus Bayesian flow based on von Mises Distribution}. For $x = 0.3, \beta(1) = 1000$ and $\alpha_i$ defined in \cref{appd:bfn_cir}, this figure plots three colored stochastic parameter trajectories for receiver mean parameter $m$ and accumulated accuracy parameter $c$, superimposed on a log-scale heatmap of the Bayesian flow distribution $p_F(m|x,\senderacc)$ and $p_F(c|x,\senderacc)$. Note the \emph{non-monotonicity} and \emph{non-additive} property of $c$. \jj{Shrink the figures to save space}}
%     \label{fig:vmbf_vis}
%     \vspace{-30pt}
% \end{wrapfigure}


In this section, we explain the detailed design of CrysBFN tackling theoretical and practical challenges. First, we describe how to derive our new formulation of Bayesian Flow Networks over hyper-torus $\mathbb{T}^{D}$ from scratch. Next, we illustrate the two key differences between \modelname and the original form of BFN: $1)$ a meticulously designed novel base distribution with different Bayesian update rules; and $2)$ different properties over the accuracy scheduling resulted from the periodicity and the new Bayesian update rules. Then, we present in detail the overall framework of \modelname over each manifold of the crystal space (\textit{i.e.} fractional coordinates, lattice vectors, atom types) respecting \textit{periodic E(3) invariance}. 

% In this section, we first demonstrate how to build Bayesian flow on hyper-torus $\mathbb{T}^{D}$ by overcoming theoretical and practical problems to provide a low-noise parameter-space approach to fractional atom coordinate generation. Next, we present how \modelname models each manifold of crystal space respecting \textit{periodic E(3) invariance}. 

\subsection{Periodic Bayesian Flow on Hyper-torus \texorpdfstring{$\mathbb{T}^{D}$}{}} 
For generative modeling of fractional coordinates in crystal, we first construct a periodic Bayesian flow on \texorpdfstring{$\mathbb{T}^{D}$}{} by designing every component of the totally new Bayesian update process which we demonstrate to be distinct from the original Bayesian flow (please see \cref{fig:non_add}). 
 %:) 
 
 The fractional atom coordinate system \citep{jiao2023crystal} inherently distributes over a hyper-torus support $\mathbb{T}^{3\times N}$. Hence, the normal distribution support on $\R$ used in the original \citep{bfn} is not suitable for this scenario. 
% The key problem of generative modeling for crystal is the periodicity of Cartesian atom coordinates $\vX$ requiring:
% \begin{equation}\label{eq:periodcity}
% p(\vA,\vL,\vX)=p(\vA,\vL,\vX+\vec{LK}),\text{where}~\vec{K}=\vec{k}\vec{1}_{1\times N},\forall\vec{k}\in\mathbb{Z}^{3\times1}
% \end{equation}
% However, there does not exist such a distribution supporting on $\R$ to model such property because the integration of such distribution over $\R$ will not be finite and equal to 1. Therefore, the normal distribution used in \citet{bfn} can not meet this condition.

To tackle this problem, the circular distribution~\citep{mardia2009directional} over the finite interval $[-\pi,\pi)$ is a natural choice as the base distribution for deriving the BFN on $\mathbb{T}^D$. 
% one natural choice is to 
% we would like to consider the circular distribution over the finite interval as the base 
% we find that circular distributions \citep{mardia2009directional} defined on a finite interval with lengths of $2\pi$ can be used as the instantiation of input distribution for the BFN on $\mathbb{T}^D$.
Specifically, circular distributions enjoy desirable periodic properties: $1)$ the integration over any interval length of $2\pi$ equals 1; $2)$ the probability distribution function is periodic with period $2\pi$.  Sharing the same intrinsic with fractional coordinates, such periodic property of circular distribution makes it suitable for the instantiation of BFN's input distribution, in parameterizing the belief towards ground truth $\x$ on $\mathbb{T}^D$. 
% \yuxuan{this is very complicated from my perspective.} \hanlin{But this property is exactly beautiful and perfectly fit into the BFN.}

\textbf{von Mises Distribution and its Bayesian Update} We choose von Mises distribution \citep{mardia2009directional} from various circular distributions as the form of input distribution, based on the appealing conjugacy property required in the derivation of the BFN framework.
% to leverage the Bayesian conjugacy property of von Mises distribution which is required by the BFN framework. 
That is, the posterior of a von Mises distribution parameterized likelihood is still in the family of von Mises distributions. The probability density function of von Mises distribution with mean direction parameter $m$ and concentration parameter $c$ (describing the entropy/uncertainty of $m$) is defined as: 
\begin{equation}
f(x|m,c)=vM(x|m,c)=\frac{\exp(c\cos(x-m))}{2\pi I_0(c)}
\end{equation}
where $I_0(c)$ is zeroth order modified Bessel function of the first kind as the normalizing constant. Given the last univariate belief parameterized by von Mises distribution with parameter $\theta_{i-1}=\{m_{i-1},\ c_{i-1}\}$ and the sample $y$ from sender distribution with unknown data sample $x$ and known accuracy $\alpha$ describing the entropy/uncertainty of $y$,  Bayesian update for the receiver is deducted as:
\begin{equation}
 h(\{m_{i-1},c_{i-1}\},y,\alpha)=\{m_i,c_i \}, \text{where}
\end{equation}
\begin{equation}\label{eq:h_m}
m_i=\text{atan2}(\alpha\sin y+c_{i-1}\sin m_{i-1}, {\alpha\cos y+c_{i-1}\cos m_{i-1}})
\end{equation}
\begin{equation}\label{eq:h_c}
c_i =\sqrt{\alpha^2+c_{i-1}^2+2\alpha c_{i-1}\cos(y-m_{i-1})}
\end{equation}
The proof of the above equations can be found in \cref{apdx:bayesian_update_function}. The atan2 function refers to  2-argument arctangent. Independently conducting  Bayesian update for each dimension, we can obtain the Bayesian update distribution by marginalizing $\y$:
\begin{equation}
p_U(\vtheta'|\vtheta,\bold{x};\alpha)=\mathbb{E}_{p_S(\bold{y}|\bold{x};\alpha)}\delta(\vtheta'-h(\vtheta,\bold{y},\alpha))=\mathbb{E}_{vM(\bold{y}|\bold{x},\alpha)}\delta(\vtheta'-h(\vtheta,\bold{y},\alpha))
\end{equation} 
\begin{figure}
    \centering
    \vskip -0.15in
    \includegraphics[width=0.95\linewidth]{imgs/non_add.pdf}
    \caption{An intuitive illustration of non-additive accuracy Bayesian update on the torus. The lengths of arrows represent the uncertainty/entropy of the belief (\emph{e.g.}~$1/\sigma^2$ for Gaussian and $c$ for von Mises). The directions of the arrows represent the believed location (\emph{e.g.}~ $\mu$ for Gaussian and $m$ for von Mises).}
    \label{fig:non_add}
    \vskip -0.15in
\end{figure}
\textbf{Non-additive Accuracy} 
The additive accuracy is a nice property held with the Gaussian-formed sender distribution of the original BFN expressed as:
\begin{align}
\label{eq:standard_id}
    \update(\parsn{}'' \mid \parsn{}, \x; \alpha_a+\alpha_b) = \E_{\update(\parsn{}' \mid \parsn{}, \x; \alpha_a)} \update(\parsn{}'' \mid \parsn{}', \x; \alpha_b)
\end{align}
Such property is mainly derived based on the standard identity of Gaussian variable:
\begin{equation}
X \sim \mathcal{N}\left(\mu_X, \sigma_X^2\right), Y \sim \mathcal{N}\left(\mu_Y, \sigma_Y^2\right) \Longrightarrow X+Y \sim \mathcal{N}\left(\mu_X+\mu_Y, \sigma_X^2+\sigma_Y^2\right)
\end{equation}
The additive accuracy property makes it feasible to derive the Bayesian flow distribution $
p_F(\boldsymbol{\theta} \mid \mathbf{x} ; i)=p_U\left(\boldsymbol{\theta} \mid \boldsymbol{\theta}_0, \mathbf{x}, \sum_{k=1}^{i} \alpha_i \right)
$ for the simulation-free training of \cref{eq:loss_n}.
It should be noted that the standard identity in \cref{eq:standard_id} does not hold in the von Mises distribution. Hence there exists an important difference between the original Bayesian flow defined on Euclidean space and the Bayesian flow of circular data on $\mathbb{T}^D$ based on von Mises distribution. With prior $\btheta = \{\bold{0},\bold{0}\}$, we could formally represent the non-additive accuracy issue as:
% The additive accuracy property implies the fact that the "confidence" for the data sample after observing a series of the noisy samples with accuracy ${\alpha_1, \cdots, \alpha_i}$ could be  as the accuracy sum  which could be  
% Here we 
% Here we emphasize the specific property of BFN based on von Mises distribution.
% Note that 
% \begin{equation}
% \update(\parsn'' \mid \parsn, \x; \alpha_a+\alpha_b) \ne \E_{\update(\parsn' \mid \parsn, \x; \alpha_a)} \update(\parsn'' \mid \parsn', \x; \alpha_b)
% \end{equation}
% \oyyw{please check whether the below equation is better}
% \yuxuan{I fill somehow confusing on what is the update distribution with $\alpha$. }
% \begin{equation}
% \update(\parsn{}'' \mid \parsn{}, \x; \alpha_a+\alpha_b) \ne \E_{\update(\parsn{}' \mid \parsn{}, \x; \alpha_a)} \update(\parsn{}'' \mid \parsn{}', \x; \alpha_b)
% \end{equation}
% We give an intuitive visualization of such difference in \cref{fig:non_add}. The untenability of this property can materialize by considering the following case: with prior $\btheta = \{\bold{0},\bold{0}\}$, check the two-step Bayesian update distribution with $\alpha_a,\alpha_b$ and one-step Bayesian update with $\alpha=\alpha_a+\alpha_b$:
\begin{align}
\label{eq:nonadd}
     &\update(c'' \mid \parsn, \x; \alpha_a+\alpha_b)  = \delta(c-\alpha_a-\alpha_b)
     \ne  \mathbb{E}_{p_U(\parsn' \mid \parsn, \x; \alpha_a)}\update(c'' \mid \parsn', \x; \alpha_b) \nonumber \\&= \mathbb{E}_{vM(\bold{y}_b|\bold{x},\alpha_a)}\mathbb{E}_{vM(\bold{y}_a|\bold{x},\alpha_b)}\delta(c-||[\alpha_a \cos\y_a+\alpha_b\cos \y_b,\alpha_a \sin\y_a+\alpha_b\sin \y_b]^T||_2)
\end{align}
A more intuitive visualization could be found in \cref{fig:non_add}. This fundamental difference between periodic Bayesian flow and that of \citet{bfn} presents both theoretical and practical challenges, which we will explain and address in the following contents.

% This makes constructing Bayesian flow based on von Mises distribution intrinsically different from previous Bayesian flows (\citet{bfn}).

% Thus, we must reformulate the framework of Bayesian flow networks  accordingly. % and do necessary reformulations of BFN. 

% \yuxuan{overall I feel this part is complicated by using the language of update distribution. I would like to suggest simply use bayesian update, to provide intuitive explantion.}\hanlin{See the illustration in \cref{fig:non_add}}

% That introduces a cascade of problems, and we investigate the following issues: $(1)$ Accuracies between sender and receiver are not synchronized and need to be differentiated. $(2)$ There is no tractable Bayesian flow distribution for a one-step sample conditioned on a given time step $i$, and naively simulating the Bayesian flow results in computational overhead. $(3)$ It is difficult to control the entropy of the Bayesian flow. $(4)$ Accuracy is no longer a function of $t$ and becomes a distribution conditioned on $t$, which can be different across dimensions.
%\jj{Edited till here}

\textbf{Entropy Conditioning} As a common practice in generative models~\citep{ddpm,flowmatching,bfn}, timestep $t$ is widely used to distinguish among generation states by feeding the timestep information into the networks. However, this paper shows that for periodic Bayesian flow, the accumulated accuracy $\vc_i$ is more effective than time-based conditioning by informing the network about the entropy and certainty of the states $\parsnt{i}$. This stems from the intrinsic non-additive accuracy which makes the receiver's accumulated accuracy $c$ not bijective function of $t$, but a distribution conditioned on accumulated accuracies $\vc_i$ instead. Therefore, the entropy parameter $\vc$ is taken logarithm and fed into the network to describe the entropy of the input corrupted structure. We verify this consideration in \cref{sec:exp_ablation}. 
% \yuxuan{implement variant. traditionally, the timestep is widely used to distinguish the different states by putting the timestep embedding into the networks. citation of FM, diffusion, BFN. However, we find that conditioned on time in periodic flow could not provide extra benefits. To further boost the performance, we introduce a simple yet effective modification term entropy conditional. This is based on that the accumulated accuracy which represents the current uncertainty or entropy could be a better indicator to distinguish different states. + Describe how you do this. }



\textbf{Reformulations of BFN}. Recall the original update function with Gaussian sender distribution, after receiving noisy samples $\y_1,\y_2,\dots,\y_i$ with accuracies $\senderacc$, the accumulated accuracies of the receiver side could be analytically obtained by the additive property and it is consistent with the sender side.
% Since observing sample $\y$ with $\alpha_i$ can not result in exact accuracy increment $\alpha_i$ for receiver, the accuracies between sender and receiver are not synchronized which need to be differentiated. 
However, as previously mentioned, this does not apply to periodic Bayesian flow, and some of the notations in original BFN~\citep{bfn} need to be adjusted accordingly. We maintain the notations of sender side's one-step accuracy $\alpha$ and added accuracy $\beta$, and alter the notation of receiver's accuracy parameter as $c$, which is needed to be simulated by cascade of Bayesian updates. We emphasize that the receiver's accumulated accuracy $c$ is no longer a function of $t$ (differently from the Gaussian case), and it becomes a distribution conditioned on received accuracies $\senderacc$ from the sender. Therefore, we represent the Bayesian flow distribution of von Mises distribution as $p_F(\btheta|\x;\alpha_1,\alpha_2,\dots,\alpha_i)$. And the original simulation-free training with Bayesian flow distribution is no longer applicable in this scenario.
% Different from previous BFNs where the accumulated accuracy $\rho$ is not explicitly modeled, the accumulated accuracy parameter $c$ (visualized in \cref{fig:vmbf_vis}) needs to be explicitly modeled by feeding it to the network to avoid information loss.
% the randomaccuracy parameter $c$ (visualized in \cref{fig:vmbf_vis}) implies that there exists information in $c$ from the sender just like $m$, meaning that $c$ also should be fed into the network to avoid information loss. 
% We ablate this consideration in  \cref{sec:exp_ablation}. 

\textbf{Fast Sampling from Equivalent Bayesian Flow Distribution} Based on the above reformulations, the Bayesian flow distribution of von Mises distribution is reframed as: 
\begin{equation}\label{eq:flow_frac}
p_F(\btheta_i|\x;\alpha_1,\alpha_2,\dots,\alpha_i)=\E_{\update(\parsnt{1} \mid \parsnt{0}, \x ; \alphat{1})}\dots\E_{\update(\parsn_{i-1} \mid \parsnt{i-2}, \x; \alphat{i-1})} \update(\parsnt{i} | \parsnt{i-1},\x;\alphat{i} )
\end{equation}
Naively sampling from \cref{eq:flow_frac} requires slow auto-regressive iterated simulation, making training unaffordable. Noticing the mathematical properties of \cref{eq:h_m,eq:h_c}, we  transform \cref{eq:flow_frac} to the equivalent form:
\begin{equation}\label{eq:cirflow_equiv}
p_F(\vec{m}_i|\x;\alpha_1,\alpha_2,\dots,\alpha_i)=\E_{vM(\y_1|\x,\alpha_1)\dots vM(\y_i|\x,\alpha_i)} \delta(\vec{m}_i-\text{atan2}(\sum_{j=1}^i \alpha_j \cos \y_j,\sum_{j=1}^i \alpha_j \sin \y_j))
\end{equation}
\begin{equation}\label{eq:cirflow_equiv2}
p_F(\vec{c}_i|\x;\alpha_1,\alpha_2,\dots,\alpha_i)=\E_{vM(\y_1|\x,\alpha_1)\dots vM(\y_i|\x,\alpha_i)}  \delta(\vec{c}_i-||[\sum_{j=1}^i \alpha_j \cos \y_j,\sum_{j=1}^i \alpha_j \sin \y_j]^T||_2)
\end{equation}
which bypasses the computation of intermediate variables and allows pure tensor operations, with negligible computational overhead.
\begin{restatable}{proposition}{cirflowequiv}
The probability density function of Bayesian flow distribution defined by \cref{eq:cirflow_equiv,eq:cirflow_equiv2} is equivalent to the original definition in \cref{eq:flow_frac}. 
\end{restatable}
\textbf{Numerical Determination of Linear Entropy Sender Accuracy Schedule} ~Original BFN designs the accuracy schedule $\beta(t)$ to make the entropy of input distribution linearly decrease. As for crystal generation task, to ensure information coherence between modalities, we choose a sender accuracy schedule $\senderacc$ that makes the receiver's belief entropy $H(t_i)=H(p_I(\cdot|\vtheta_i))=H(p_I(\cdot|\vc_i))$ linearly decrease \emph{w.r.t.} time $t_i$, given the initial and final accuracy parameter $c(0)$ and $c(1)$. Due to the intractability of \cref{eq:vm_entropy}, we first use numerical binary search in $[0,c(1)]$ to determine the receiver's $c(t_i)$ for $i=1,\dots, n$ by solving the equation $H(c(t_i))=(1-t_i)H(c(0))+tH(c(1))$. Next, with $c(t_i)$, we conduct numerical binary search for each $\alpha_i$ in $[0,c(1)]$ by solving the equations $\E_{y\sim vM(x,\alpha_i)}[\sqrt{\alpha_i^2+c_{i-1}^2+2\alpha_i c_{i-1}\cos(y-m_{i-1})}]=c(t_i)$ from $i=1$ to $i=n$ for arbitrarily selected $x\in[-\pi,\pi)$.

After tackling all those issues, we have now arrived at a new BFN architecture for effectively modeling crystals. Such BFN can also be adapted to other type of data located in hyper-torus $\mathbb{T}^{D}$.

\subsection{Equivariant Bayesian Flow for Crystal}
With the above Bayesian flow designed for generative modeling of fractional coordinate $\vF$, we are able to build equivariant Bayesian flow for each modality of crystal. In this section, we first give an overview of the general training and sampling algorithm of \modelname (visualized in \cref{fig:framework}). Then, we describe the details of the Bayesian flow of every modality. The training and sampling algorithm can be found in \cref{alg:train} and \cref{alg:sampling}.

\textbf{Overview} Operating in the parameter space $\bthetaM=\{\bthetaA,\bthetaL,\bthetaF\}$, \modelname generates high-fidelity crystals through a joint BFN sampling process on the parameter of  atom type $\bthetaA$, lattice parameter $\vec{\theta}^L=\{\bmuL,\brhoL\}$, and the parameter of fractional coordinate matrix $\bthetaF=\{\bmF,\bcF\}$. We index the $n$-steps of the generation process in a discrete manner $i$, and denote the corresponding continuous notation $t_i=i/n$ from prior parameter $\thetaM_0$ to a considerably low variance parameter $\thetaM_n$ (\emph{i.e.} large $\vrho^L,\bmF$, and centered $\bthetaA$).

At training time, \modelname samples time $i\sim U\{1,n\}$ and $\bthetaM_{i-1}$ from the Bayesian flow distribution of each modality, serving as the input to the network. The network $\net$ outputs $\net(\parsnt{i-1}^\mathcal{M},t_{i-1})=\net(\parsnt{i-1}^A,\parsnt{i-1}^F,\parsnt{i-1}^L,t_{i-1})$ and conducts gradient descents on loss function \cref{eq:loss_n} for each modality. After proper training, the sender distribution $p_S$ can be approximated by the receiver distribution $p_R$. 

At inference time, from predefined $\thetaM_0$, we conduct transitions from $\thetaM_{i-1}$ to $\thetaM_{i}$ by: $(1)$ sampling $\y_i\sim p_R(\bold{y}|\thetaM_{i-1};t_i,\alpha_i)$ according to network prediction $\predM{i-1}$; and $(2)$ performing Bayesian update $h(\thetaM_{i-1},\y^\calM_{i-1},\alpha_i)$ for each dimension. 

% Alternatively, we complete this transition using the flow-back technique by sampling 
% $\thetaM_{i}$ from Bayesian flow distribution $\flow(\btheta^M_{i}|\predM{i-1};t_{i-1})$. 

% The training objective of $\net$ is to minimize the KL divergence between sender distribution and receiver distribution for every modality as defined in \cref{eq:loss_n} which is equivalent to optimizing the negative variational lower bound $\calL^{VLB}$ as discussed in \cref{sec:preliminaries}. 

%In the following part, we will present the Bayesian flow of each modality in detail.

\textbf{Bayesian Flow of Fractional Coordinate $\vF$}~The distribution of the prior parameter $\bthetaF_0$ is defined as:
\begin{equation}\label{eq:prior_frac}
    p(\bthetaF_0) \defeq \{vM(\vm_0^F|\vec{0}_{3\times N},\vec{0}_{3\times N}),\delta(\vc_0^F-\vec{0}_{3\times N})\} = \{U(\vec{0},\vec{1}),\delta(\vc_0^F-\vec{0}_{3\times N})\}
\end{equation}
Note that this prior distribution of $\vm_0^F$ is uniform over $[\vec{0},\vec{1})$, ensuring the periodic translation invariance property in \cref{De:pi}. The training objective is minimizing the KL divergence between sender and receiver distribution (deduction can be found in \cref{appd:cir_loss}): 
%\oyyw{replace $\vF$ with $\x$?} \hanlin{notations follow Preliminary?}
\begin{align}\label{loss_frac}
\calL_F = n \E_{i \sim \ui{n}, \flow(\parsn{}^F \mid \vF ; \senderacc)} \alpha_i\frac{I_1(\alpha_i)}{I_0(\alpha_i)}(1-\cos(\vF-\predF{i-1}))
\end{align}
where $I_0(x)$ and $I_1(x)$ are the zeroth and the first order of modified Bessel functions. The transition from $\bthetaF_{i-1}$ to $\bthetaF_{i}$ is the Bayesian update distribution based on network prediction:
\begin{equation}\label{eq:transi_frac}
    p(\btheta^F_{i}|\parsnt{i-1}^\calM)=\mathbb{E}_{vM(\bold{y}|\predF{i-1},\alpha_i)}\delta(\btheta^F_{i}-h(\btheta^F_{i-1},\bold{y},\alpha_i))
\end{equation}
\begin{restatable}{proposition}{fracinv}
With $\net_{F}$ as a periodic translation equivariant function namely $\net_F(\parsnt{}^A,w(\parsnt{}^F+\vt),\parsnt{}^L,t)=w(\net_F(\parsnt{}^A,\parsnt{}^F,\parsnt{}^L,t)+\vt), \forall\vt\in\R^3$, the marginal distribution of $p(\vF_n)$ defined by \cref{eq:prior_frac,eq:transi_frac} is periodic translation invariant. 
\end{restatable}
\textbf{Bayesian Flow of Lattice Parameter \texorpdfstring{$\boldsymbol{L}$}{}}   
Noting the lattice parameter $\bm{L}$ located in Euclidean space, we set prior as the parameter of a isotropic multivariate normal distribution $\btheta^L_0\defeq\{\vmu_0^L,\vrho_0^L\}=\{\bm{0}_{3\times3},\bm{1}_{3\times3}\}$
% \begin{equation}\label{eq:lattice_prior}
% \btheta^L_0\defeq\{\vmu_0^L,\vrho_0^L\}=\{\bm{0}_{3\times3},\bm{1}_{3\times3}\}
% \end{equation}
such that the prior distribution of the Markov process on $\vmu^L$ is the Dirac distribution $\delta(\vec{\mu_0}-\vec{0})$ and $\delta(\vec{\rho_0}-\vec{1})$, 
% \begin{equation}
%     p_I^L(\boldsymbol{L}|\btheta_0^L)=\mathcal{N}(\bm{L}|\bm{0},\bm{I})
% \end{equation}
which ensures O(3)-invariance of prior distribution of $\vL$. By Eq. 77 from \citet{bfn}, the Bayesian flow distribution of the lattice parameter $\bm{L}$ is: 
\begin{align}% =p_U(\bmuL|\btheta_0^L,\bm{L},\beta(t))
p_F^L(\bmuL|\bm{L};t) &=\mathcal{N}(\bmuL|\gamma(t)\bm{L},\gamma(t)(1-\gamma(t))\bm{I}) 
\end{align}
where $\gamma(t) = 1 - \sigma_1^{2t}$ and $\sigma_1$ is the predefined hyper-parameter controlling the variance of input distribution at $t=1$ under linear entropy accuracy schedule. The variance parameter $\vrho$ does not need to be modeled and fed to the network, since it is deterministic given the accuracy schedule. After sampling $\bmuL_i$ from $p_F^L$, the training objective is defined as minimizing KL divergence between sender and receiver distribution (based on Eq. 96 in \citet{bfn}):
\begin{align}
\mathcal{L}_{L} = \frac{n}{2}\left(1-\sigma_1^{2/n}\right)\E_{i \sim \ui{n}}\E_{\flow(\bmuL_{i-1} |\vL ; t_{i-1})}  \frac{\left\|\vL -\predL{i-1}\right\|^2}{\sigma_1^{2i/n}},\label{eq:lattice_loss}
\end{align}
where the prediction term $\predL{i-1}$ is the lattice parameter part of network output. After training, the generation process is defined as the Bayesian update distribution given network prediction:
\begin{equation}\label{eq:lattice_sampling}
    p(\bmuL_{i}|\parsnt{i-1}^\calM)=\update^L(\bmuL_{i}|\predL{i-1},\bmuL_{i-1};t_{i-1})
\end{equation}
    

% The final prediction of the lattice parameter is given by $\bmuL_n = \predL{n-1}$.
% \begin{equation}\label{eq:final_lattice}
%     \bmuL_n = \predL{n-1}
% \end{equation}

\begin{restatable}{proposition}{latticeinv}\label{prop:latticeinv}
With $\net_{L}$ as  O(3)-equivariant function namely $\net_L(\parsnt{}^A,\parsnt{}^F,\vQ\parsnt{}^L,t)=\vQ\net_L(\parsnt{}^A,\parsnt{}^F,\parsnt{}^L,t),\forall\vQ^T\vQ=\vI$, the marginal distribution of $p(\bmuL_n)$ defined by \cref{eq:lattice_sampling} is O(3)-invariant. 
\end{restatable}


\textbf{Bayesian Flow of Atom Types \texorpdfstring{$\boldsymbol{A}$}{}} 
Given that atom types are discrete random variables located in a simplex $\calS^K$, the prior parameter of $\boldsymbol{A}$ is the discrete uniform distribution over the vocabulary $\parsnt{0}^A \defeq \frac{1}{K}\vec{1}_{1\times N}$. 
% \begin{align}\label{eq:disc_input_prior}
% \parsnt{0}^A \defeq \frac{1}{K}\vec{1}_{1\times N}
% \end{align}
% \begin{align}
%     (\oh{j}{K})_k \defeq \delta_{j k}, \text{where }\oh{j}{K}\in \R^{K},\oh{\vA}{KD} \defeq \left(\oh{a_1}{K},\dots,\oh{a_N}{K}\right) \in \R^{K\times N}
% \end{align}
With the notation of the projection from the class index $j$ to the length $K$ one-hot vector $ (\oh{j}{K})_k \defeq \delta_{j k}, \text{where }\oh{j}{K}\in \R^{K},\oh{\vA}{KD} \defeq \left(\oh{a_1}{K},\dots,\oh{a_N}{K}\right) \in \R^{K\times N}$, the Bayesian flow distribution of atom types $\vA$ is derived in \citet{bfn}:
\begin{align}
\flow^{A}(\parsn^A \mid \vA; t) &= \E_{\N{\y \mid \beta^A(t)\left(K \oh{\vA}{K\times N} - \vec{1}_{K\times N}\right)}{\beta^A(t) K \vec{I}_{K\times N \times N}}} \delta\left(\parsn^A - \frac{e^{\y}\parsnt{0}^A}{\sum_{k=1}^K e^{\y_k}(\parsnt{0})_{k}^A}\right).
\end{align}
where $\beta^A(t)$ is the predefined accuracy schedule for atom types. Sampling $\btheta_i^A$ from $p_F^A$ as the training signal, the training objective is the $n$-step discrete-time loss for discrete variable \citep{bfn}: 
% \oyyw{can we simplify the next equation? Such as remove $K \times N, K \times N \times N$}
% \begin{align}
% &\calL_A = n\E_{i \sim U\{1,n\},\flow^A(\parsn^A \mid \vA ; t_{i-1}),\N{\y \mid \alphat{i}\left(K \oh{\vA}{KD} - \vec{1}_{K\times N}\right)}{\alphat{i} K \vec{I}_{K\times N \times N}}} \ln \N{\y \mid \alphat{i}\left(K \oh{\vA}{K\times N} - \vec{1}_{K\times N}\right)}{\alphat{i} K \vec{I}_{K\times N \times N}}\nonumber\\
% &\qquad\qquad\qquad-\sum_{d=1}^N \ln \left(\sum_{k=1}^K \out^{(d)}(k \mid \parsn^A; t_{i-1}) \N{\ydd{d} \mid \alphat{i}\left(K\oh{k}{K}- \vec{1}_{K\times N}\right)}{\alphat{i} K \vec{I}_{K\times N \times N}}\right)\label{discdisc_t_loss_exp}
% \end{align}
\begin{align}
&\calL_A = n\E_{i \sim U\{1,n\},\flow^A(\parsn^A \mid \vA ; t_{i-1}),\N{\y \mid \alphat{i}\left(K \oh{\vA}{KD} - \vec{1}\right)}{\alphat{i} K \vec{I}}} \ln \N{\y \mid \alphat{i}\left(K \oh{\vA}{K\times N} - \vec{1}\right)}{\alphat{i} K \vec{I}}\nonumber\\
&\qquad\qquad\qquad-\sum_{d=1}^N \ln \left(\sum_{k=1}^K \out^{(d)}(k \mid \parsn^A; t_{i-1}) \N{\ydd{d} \mid \alphat{i}\left(K\oh{k}{K}- \vec{1}\right)}{\alphat{i} K \vec{I}}\right)\label{discdisc_t_loss_exp}
\end{align}
where $\vec{I}\in \R^{K\times N \times N}$ and $\vec{1}\in\R^{K\times D}$. When sampling, the transition from $\bthetaA_{i-1}$ to $\bthetaA_{i}$ is derived as:
\begin{equation}
    p(\btheta^A_{i}|\parsnt{i-1}^\calM)=\update^A(\btheta^A_{i}|\btheta^A_{i-1},\predA{i-1};t_{i-1})
\end{equation}

The detailed training and sampling algorithm could be found in \cref{alg:train} and \cref{alg:sampling}.





\section{Applications}
%Placeholder for general introduction of the Accelerators and Applications theme sections. Themes of applications include machine-learning, bioinformatics, space applications, radio astronomy and weather simulations. Some of these references will overlap with other sections, e.g. when contributions are made on applying effective distributed computing for the purpose of weather forecasting.

FPGAs have emerged as powerful accelerators for a wide range of applications. In this section, we discuss FPGA-based solutions in machine learning (Section~\ref{sec:ml}), astronomy (Section~\ref{sec:astr}), particle physics experiments (Section~\ref{sec:phys}), quantum computing (Section~\ref{sec:quant}), space applications (Section~\ref{sec:space}), and bioinformatics (Section~\ref{sec:bio}).

\subsection{Machine learning}
\label{sec:ml}
% Three main parts, adapting an existing ML approach to hardware, designing hardware to accelerate an existing ML approach, (co-)design hardware for exotic ML approach.
% Main categories of evaluation are throughput, power, hardware area / resources, accuracy.
% \begin{itemize}
%     \item Accelerating existing ML models with new hardware design
%         \begin{itemize}
%             \item CNN acceleration (5)
%             \item TPU (1)
%             \item Benchmarking FPGAs (1)
%     \item Co-design existing ML models to hardware accelerate
%         \begin{itemize}
%             \item Pruning
%             \item Quantization / fixed point
%             \item Weight sharing
%             \item NAS adaptive to hardware
%         \end{itemize}
%     \item Design new hardware for exotic ML model
%         \begin{itemize}
%             \item Spiking / neuromorphic (7)
%             \item Bayesian (1)
%             \item Oscillating (2)
%         \end{itemize}
%     \end{itemize}
%     \item Hardware for 
% \end{itemize}
% \subsubsection{Background}
In the field of machine learning, and in particular deep learning, hardware acceleration plays a vital role. GPUs are the predominant method for hardware acceleration due to their high parallelism, but FPGA research is showing promising results. FPGAs enable inference at greater speed and better power efficiency when compared to GPUs \cite{hw-efficiency-compare} by designing model-specific accelerated pipelines \cite{ml-energy-efficient-cnn}. Through the co-design of machine learning models and machine learning hardware on FPGAs, models are accelerated without compromising on performance metrics and utilizing limited FPGA resources. In addition, the flexibility of the FPGA's architecture enables the realization of unconventional deep learning technology, such as Spiking Neural Networks (SNNs). 
%These networks can operate on a fraction of the power required by conventional networks on CPU or GPU.

%\subsubsection{Research topics}
\paragraph{Hardware acceleration} Ample research on hardware acceleration focuses on accelerating existing neural network architectures. One common class of architectures is convolutional neural networks (CNNs), which learn image filters in order to identify abstract image features. CNNs are often deployed in embedded applications which require real-time image processing and low energy consumption, making FPGAs a suitable candidate for CNN acceleration. \citet{ml-energy-efficient-cnn} propose an implementation of the LeNet architecture using Vitis HLS, pipelining the CNN layers, and outperforms other FPGA based implementations at a processing time of $70 \mu s$. One downside to this approach is the inflexibility of designing a specific model architecture in HLS which can be resolved by using partial reconfiguration \cite{ml-cnn-acclr-part-reconf}. To increase CNN throughput, further parallelization can be exploited, and in combination with the use of the high bandwidth OpenCAPI interface, can achieve a latency of less than $10 \mu s$ on the LeNet-5 model, streaming data from an HDMI interface \cite{ml-FPQNet}. In each of these implementations, fully pipelined CNNs are possible due to the limited number of parameters in small CNNs. As larger pipelined networks are deployed on FPGAs, parallelization puts strain on the available resources, and in particular the amount of on-chip-memory becomes a bottleneck. A proposed solution to this is using Frequency Compensated Memory Packing \cite{ml-mem-efficient-df-inf}.

In addition to CNN acceleration, general neural network acceleration has been developed by means of a programmable Tensor Processing Unit (TPU) as an overlay on an FPGA accelerator \cite{ml-agile-tuned-tpu}. Deep learning acceleration using FPGAs is also relevant to space technology research. Since the reprogrammability of FPGAs make them a suitable contender for deployment on space missions, FPGA implementations of existing deep learning models are being benchmarked for space applications \cite{ml-myriad-2-space-cnn} \cite{ml-mem-efficient-df-inf}.

\paragraph{Spiking neural networks} Spiking Neural Networks (SNNs) are computational models formed using spiking neuronal units that operate in parallel and mimic the basic operational principles of biological systems. These features endow SNNs with potentially richer dynamics than traditional artificial neural network models based on the McCulloch-Pitts point neurons or simple ReLU activation functions that do not incorporate timing information. Thus, SNNs excel in handling temporal information streams and are well-suited for innovative non-von-Neumann computer architectures, which differ from traditional sequential processing systems. SNNs are particularly well-suited for implementation in FPGAs due to their massive parallelism and requirement for significant on-chip memories with high-memory bandwidth for storing neuron states and synaptic weights. Additionally, SNNs use sparse binary communication, which is beneficial for low-latency operations because both computing and memory updates are triggered by events. FPGAs' inherent flexibility allows for reprogramming and customization, which enable reprogrammable SNNs in FPGAs, resulting in flexible, efficient, and low-latency systems~\cite{Corradi2021Gyro:Analytics,Irmak2021ADesigns,SankaranAnInference}. \citet{corradi2024accelerated} demonstrated the application of a Spiking Convolutional Neural Network (SCNN) to population genomics. The SCNN architecture achieved comparable classification accuracy to state-of-the-art CNNs while processing only about 59.9\% of the input data, reaching 97.6\% of CNN accuracy for classifying selective-sweep and recombination-hotspot genomic regions. This was enabled by % success is attributed to 
the SCNN's capability to temporize genetic information, allowing it to produce classification outputs without processing the entire genomic input sequence. Additionally, when implemented on FPGA hardware, the SCNN model exhibited over three times higher throughput and more than 100 times greater energy efficiency than a GPU implementation, markedly enhancing the processing of large-scale population genomics datasets.


\paragraph{Model/hardware co-design} Previous examples demonstrate that existing deep neural network models can be accelerated using FPGAs. Typically, research in this area focuses on designing an optimal hardware solution for an existing model. A more effective approach, however, is to co-design the model and the hardware accelerator simultaneously. However, simultaneous co-design of DNN models and accelerators is challenging. DNN designers often need more specialized knowledge to consider hardware constraints, while hardware designers may need help to maintain the quality and accuracy of DNN models. Furthermore, efficiently exploring the extensive co-design space is a significant challenge. This co-design methodology leads to better performance, leveraging FPGAs' flexibility and rapid prototyping capabilities. For example, \citet{Rocha2020BinaryWrist-PPG}, by co-designing the bCorNET framework, which combines binary CNNs and LSTMs, they were able to create an efficient hardware accelerator that processes HR estimation from PPG signals in real-time. The pipelined architecture and quantization strategies employed allowed for significant reductions in memory footprint and computational complexity, enabling real-time processing with low latency.

In SNNs, encoding information in spike streams is a crucial co-design aspect. SNNs primarily use two encoding strategies: rate-coding and time-to-first-spike (TTFS) coding. Rate coding is common in SNN models, encoding information based on the instantaneous frequency of spike streams. Higher spike frequencies result in higher precision but at the cost of increased energy consumption due to frequent spiking. While rate coding offers accuracy, it reduces sparsity. In FPGA implementations, rate coding is often used for its robustness, simplicity, ease of training through the conversion of analog neural networks to spiking neural networks, and practicality in multi-sensor data fusion, where it helps represent real values from various sensors (radars, cameras) even in the presence of jitter or imperfect synchronization~\cite{Corradi2021Gyro:Analytics}.
Conversely, TTFS coding has been demonstrated in SNNs implemented on FPGAs to enhance sparsity and has the potential of reducing energy consumption by encoding information in spike timing. For instance, Pes et al.~\cite{Pes2024ActiveNetworks} introduced a novel SNN model with active dendrites to address catastrophic forgetting in sequential learning tasks. Active dendrites enable the SNN to dynamically select different sub-networks for different tasks, improving continual learning and mitigating catastrophic forgetting. This model was implemented on a Xilinx Zynq-7020 SoC FPGA, demonstrating practical viability with a high accuracy of 80\% and an average inference time of 37.3 ms, indicating significant potential for real-world deployment in edge devices.

%To overcome this challenges, Cong et al in \textcolor{red}{\textcolor{red}{~\cite{FPGA/DNN Co-Design: An Efficient Design Methodology for IoT
%Intelligence on the Edge}}} introduced a co-design methodology for FPGAs and DNNs that integrates both bottom-up and top-down approaches, in which a bottom-up search for DNN models that prioritize high accuracy is paired with a top-down design of FPGA accelerators tailored to the specific characteristics of DNNs.
%Other methods leverage an automatic toolchain comprising  auto-DNN engine for hardware-aware DNN model optimization and an auto-HLS engine to generate FPGA-suitable synthesizable code, or hardware-aware Neural Architecture Search (NAS). When co-design is applied, it typicaly produces DNN models and FPGA accelerators that outperform state-of-the-art FPGA designs in various metrics, including accuracy, speed, power consumption, and energy efficiency \textcolor{red}{~\cite{When Neural Architecture Search Meets Hardware Implementation: from Hardware Awareness to Co-Design}.

%\textcolor{blue}{Co-design is critical in developing FPGA-based systems, merging hardware and software engineering from the initial design stages. This integrated method is essential for optimizing system performance, functionality, and cost-effectiveness. Co-design leverages the adaptable nature of FPGAs, tailoring the computing workload to meet specific hardware needs and adjusting the hardware to suit software demands. This synergy results in improved system performance and greater energy efficiency.}
%\textcolor{blue}{Many co-design examples exists in literature that demonstrate how clever distributed memory layouts can results in increased performances~\cite{}. }

%\paragraph{Novel hardware architecture} 

%\textcolor{blue}{Modern co-design methodologies allow the generation of hardware architectures and applications for advanced Reconfigurable Acceleration Devices (RAD) that go beyond traditional FPGA capabilities. These devices integrate FPGA fabric with other components like general-purpose processors, specialized accelerators, and high-performance networks-on-chip (NoCs) within a system-in-package framework. This integration enables complex data center applications to be handled more efficiently than conventional FPGAs. In particular, Boutrous et al in \cite{Architecture and Application Co-Design for Beyond-FPGA Reconfigurable Acceleration Devices} introduce RAD-Sim, an architecture simulator, to aid in the design space exploration of RADs. This allows for the study of interactions between different system components. Notably, they demonstrated mapping deep learning FPGA overlays to different RAD configurations, demonstrating how RAD-Sim can guide the adaptation of applications to exploit the novel features of RADs effectively.}


\subsection{Astronomy}
\label{sec:astr}
%\subsubsection{Introduction}

Astronomy is the study of everything in the universe beyond our Earth's atmosphere. Observations are done at different modalities and wavelengths, such as detection of a range of different particles (e.g., Cherenkov detector based systems such as KM3NeT \cite{KM3NeT:2009xxi}), gravitational waves, optical observations, gamma and x-ray observations and radio (e.g., WSRT \cite{van_Cappellen_2022}, LOFAR \cite{van_Haarlem_2013}, SKA \cite{book-SKA}). Observations can be done from space or from earth; in this section, we limit the scope to ground-based astronomy. A common denominator for instruments required for observation of the different modalities and different wave lengths is that the systems need to be very sensitive in order to observe very faint signals from outside the Earth's atmosphere. Instruments are typically large and/or distributed over a large area %in order 
to achieve %reach 
good sensitivity and resolution. Different modalities and wavelengths require distinct types of sensors, cameras, or antennas to convert observed phenomena into electrical signals. Each system is tailored to its specific modality and wavelength, necessitating specialized components to accurately capture and translate the data. %At different modalities and different wave lengths, the systems each require different kinds of sensors, camera's or antenna's that convert the observed phenomenon to an electrical signal. 
%The electrical signal is at some point in the signal chain converted to the digital domain and processed in various stages into an end product used by scientists. 
At a certain stage in the signal chain, the electrical signal is converted into the digital domain, where it undergoes multiple processing stages. This processed signal ultimately results in an end product that can be utilized by scientists for analysis and research purposes.
Systems can roughly be split into two parts, a front-end and a back-end. The front-end requires interfacing with and processing of data from the sensor; electronics commonly deployed in the front-end are constrained in space (size), temperature, power, cost, RFI, environmental conditions and serviceability. The back-end processes data produced by the front-end(s) either in an online or offline fashion, which is usually %typically be 
done with server infrastructure in a data center. % environment, either on site or centralized. 
In the back-end, the main challenges are the high data bandwidth and large data size coming from the front-ends. Although the environment is more flexible, systems are still constrained in space, power, and cost.

%\subsubsection{Background}

FPGAs have been used in astronomy instrumentation for quite some time, as they 
%FPGAs have since long found applications in astronomy instrumentation. 
%Typically FPGA's 
are %very 
efficient in interfacing with Analog to Digital Converters (ADCs), and well suited to the conditions faced in instrumentation front-ends (e.g. NCLE \cite{karapakula2024ncle}). Moreover, FPGA are also used further down the processing stages for various signal processing operations, both in the front-ends (e.g., Uniboard2 in LOFAR \cite{doi:10.1142/S225117171950003X}) as well as in the back-ends of systems (e.g., MeerKAT \cite{2022JATIS...8a1006V} and SKA \cite{SKA-CBF}). GPUs represent a good alternative in back-end processing (e.g., LOFAR's system COBALT \cite{Broekema_2018}) as well. The work by Veenboer et al. \cite{10.1007/978-3-030-29400-7_36} describes a trade-off between using a GPU and an FPGA accelerator in the implementation of an image processing operation in a radio telescope back-end.

%\subsubsection{Research topics}
%Dutch academia has contributed to several astronomy instruments:
%Often large international consortia, not immidiately clear what the role of the Dutch partners was. But also some work which is mainly done by Dutch institutes
\paragraph{Hardware Development for the Radio Neutrino Observatory in Greenland (RNO-G)}
The RNO-G \cite{Smith2022HardwareRNO-G} is a radio detection array for neutrinos. It consists of 35 autonomous stations deployed over a $5 \times 6$ km grid near the NSF Summit Station
in Greenland. Each station includes an FPGA-based phased trigger. The station has to operate in a 25 W power envelope. The implementation on FPGA seems to be favorable due to environmental conditions and operation constraints.

\paragraph{Implementation of a Correlator onto a Hardware Beam-Former to Calculate Beam-Weights}
The Apertif Phased Array Feed (PAF) \cite{van_Cappellen_2022} is a radio telescope front-end used in the WSRT system in the Netherlands. FPGAs are used for antenna read out as well as signal processing close to the antenna. Schoonderbeek et al. \cite{Schoonderbeek2020ImplementationBeam-Weights} describe the transformation and implementation of a beamformer algorithm on FPGA in order to build a more efficient system.

\paragraph{Near Memory Acceleration and Reduced-Precision Acceleration for High Resolution Radio Astronomy Imaging}
\citet{Corda2020NearImaging} describe the implementation of a 2D FFT on FPGA, leveraging Near-Memory Computing. The 2D FFT is applied to an image processing implementation on FPGA in the back-end of a radio telescope and compared with implementations on CPU and GPU. \citet{Corda2022Reduced-PrecisionHardware} explore %the concept of 
reduced-precision computation on an FPGA %is explored 
for the same image processing application. %They propose an implementation on an FPGA accelerator and compare with an implementation on CPU and GPU.

\paragraph{The MUSCAT Readout Electronics Backend: Design and Pre-deployment Performance}
The MUSCAT is a large single dish radio telescope with 1458 receives in the focal plane. The system uses FPGA based electronics to read out and pre-process the data from the receivers \cite{Rowe2023ThePerformance}. %\emph{Electronics Backend} in this case relates to the electronics close to the antenna, referred to as front-end in our description here.

% Small contribution by NL through SRON

\paragraph{Cherenkov Telescope Arrays}
Three different contributions have been made to three different Cherenkov based Telescope Arrays.
%\paragraph{A NECTAr-based upgrade for the Cherenkov cameras of the H.E.S.S. 12-meter telescopes}
Ashton et al.~\cite{Ashton2020ATelescopes} describe a system for the High Energy Stereoscopic System (H.E.S.S.) where a custom board with ARM CPU and an FPGA is used to read out and pre-process a custom designed NECTAr digitizer chip in the front-end of the system. After pre-processing, the data is distributed to a back-end over Ethernet.
%Anton Pannekoek Institute for Astronomy
%\paragraph{A White Rabbit-Synchronized Accurate Time-Stamping Solution for the Small-Sized Cameras of the Cherenkov Telescope Array}
Sánchez-Garrido et al.~\cite{Sanchez-Garrido2021AArray} present the design of a Zynq FPGA SoC based platform for White Rabbit time synchronization in the ZEN-CTA telescope array front-ends. Data captured and pre-processed at the front-ends is distributed over Ethernet to the back-end including the time stamp.
%\paragraph{Architecture and performance of the KM3NeT front-end firmware}
Aiello et al.~\cite{Aiello2021ArchitectureFirmware} outline the architecture and performance of the KM3Net front-end firmware. The KM3NeT telescope consists of two deep-sea three-dimensional sensor grids being deployed in the Mediterranean Sea. A central logic board with FPGA in the front-end serves as a Time to Digital Converter to record events and time at the sensors; the data is transmitted and further processed in a back-end on shore.
%S. Aiello et al., “KM3NeT front-end and readout electronics system:
%hardware, firmware, and software,” J. Astronomical Telescopes Inst.
%Syst., vol. 5, no. 4, pp. 1–15, 2019.

%\subsubsection{Future direction}

\vspace{0.4cm}
%In the works included in this survey, 
FPGA are mainly used for front-end sensor interfacing and pre-processing. \citet{Corda2020NearImaging, Corda2022Reduced-PrecisionHardware} underline that FPGAs are also still relevant in the back-end, providing improved performance over CPU and on-par performance with GPU accelerators. FPGA are expected to remain the dominant choice for platforms in astronomy instrumentation front-ends due to the strong interfacing capabilities and the adaptability and suitability to the constraints imposed by instrumentation front-ends. In the back-end, FPGAs provide a viable solution to application acceleration, but will have to compete with other accelerator architectures, e.g., GPUs~\cite{10.1007/978-3-030-29400-7_36}. 
An emerging new technology are the Artificial Intelligence Engines in the AMD Versal Adaptive SoC. The work from \citet{Versal-ACAP} evaluated the AI Engines for a signal processing application in radio astronomy. The flexibility and programmability of the AI Engines, combined with the interfacing capabilities of the FPGA can lead to a powerful platform for telescope front-ends.

\subsection{Particle physics experiments}
\label{sec:phys}
The Large Hadron Collider (LHC) features various particle accelerators to facilitate particle physics experiments. Experiments performed using particle accelerators can produce massive amounts of data that needs to be propagated and preprocessed at high speeds before the reduced relevant data is recorded for offline storage. FPGAs are widely employed throughout systems LHC particle accelerators, such as ATLAS and LHCb, for their high-bandwidth capabilities, and the flexibility that reconfigurable hardware offers without requiring hardware alterations to the system. Recently both the ATLAS and LHCb particle accelerators have been commissioned for upgrades. The Dutch Institute for Subatomic Physics (Nikhef) is one of the collaborating institutes working on the LHC accelerators.

%\subsubsection{Research topics}
%\paragraph{TODO - revise text into research topics}

LHCb is a particle accelerator that specializes in experiments that study the bottom quark. Major upgrades to the LHCb that enable handling a higher collision rate require new front-end and back-end electronics. To facilitate the back-end of the upgrade, the LHCb implements the custom PCIe40 board, which features an Intel Arria 10 FPGA. Four PCIe40 boards are dedicated for controlling part of the LHCb system, and 52 PCIe40 boards are used to read out each of the detector’s slices, producing an aggregated data rate of 2.85 Tb/s \cite{FernandezPrieto2020PhaseExperiment}.

ATLAS is one of the general particle accelerators of the LHC. ATLAS uses two trigger stages in order to record only the particle interactions of interest. In an upgrade to the ATLAS accelerator, ASIC-based calorimeter trigger preprocessor boards are replaced by FPGA-based hardware. Using FPGAs for this purpose allows implementing enhanced signal processing methods \cite{Aad2020PerformanceTrigger}. After the two trigger stages, FPGAs are deployed to process the triggers for tracking particles \cite{Aad2021TheSystem}.

Ongoing upgrades to the LHC particle accelerators, referred to as High Luminosity LHC (HL-LHC), will facilitate higher energy collisions. HL-LHC will produce increased background rates. To reduce false triggers due to background, the New Small Wheel checks for coinciding hits. Each trigger processor features Virtex-7, Kintex Ultrascale and Zynq FPGAs \cite{Iakovidis2023TheElectronics}. Interaction to and from front-end hardware is done through Front-End Link eXchange (FELiX) boards. As part of the HL-LHC upgrades, each FELiX board must facilitate a maximum throughput of 200 Gbps. To enable this, Remote Direct Memory Access (RDMA) over Converged Ethernet (RoCE) as part of the FELiX FPGA system is proposed \cite{Vasile2022FPGALHC, Vasile2023IntegrationLHC}. The performance of the FELiX upgrade in combination with an upgraded Software ReadOut Driver (SW ROD) satisfies the data transfer requirements of the upgraded ATLAS system \cite{Gottardo2020FEliXSystem}.




\subsection{Quantum computing}
\label{sec:quant}
Quantum computing promises to help solving many global challenges of our time such as quantum chemistry problems to design new medicines, the prediction of material properties for efficient energy storage, and the handling of big data needed for complex climate physics~\cite{Gibney-nat-2014}. The most promising quantum algorithms demand systems comprising thousands to millions of quantum bits~\cite{Meter-2013}, the quantum counterpart of a classical bit. A quantum processor comprising up to 50 qubits has been realized using solid-state superconducting qubits~\cite{Arute-nat-2019}, but its operation requires a combination of cryogenic temperatures below ~100 mK and hundreds of coaxial lines for qubit control and readout. %Furthermore, 
While in systems with a few qubits, this can be controlled using off-the-shelf electronic equipment, such approach becomes infeasible when scaling qubit systems toward thousands or millions of qubits that are required for a practical quantum computer. 

%\subsubsection{Research topics}

A means to tackle the foreseeable bottleneck in scaling the operation of qubit systems is to integrate FPGA technology in the control and readout of solid-state qubits. FPGAs have been used to generate highly-stable waveforms suitable for the control of quantum bits with latency significantly lower than software alternatives~\cite{Ireland-2020}. In systems of semiconductor spin qubits, FPGAs have provided in-hardware syncing of quantum dot control voltages with the signal acquisition and buffering and thus enabled the observation of real-time charge-tunneling events~\cite{Hartman-2023}. FPGAs have also been used to configure and synchronize a cryo-controller with an arbitrary waveform generator required to generate complex pulse shapes and perform quantum operations~\cite{Xue-nat-2021}. Such setup has enabled the demonstration of universal control of a quantum processor hosting six semiconductor spin qubits~\cite{Philips-nat-2022}. FPGAs have proven to be essential for implementing quantum error correction algorithms, which are critical for mitigating the effects of dephasing and decoherence in solid-state qubits. %FPGAs have also been shown to essential for the implementation of quantum error correction algorithms needed to mitigate the effects of dephasing and decoherence in solid-state qubits. 
In qubit systems based on superconducting quantum circuits, the first efficient demonstration of quantum error correction was made possible by a FPGA-controlled data acquisition system which provided dynamic real-time feedback on the evolution of the quantum system~\cite{Ofek-nat-2016}. It has been further predicted that FPGA can enable highly-efficient quantum error correction based on neural-network decoders~\cite{Overwater-2022}.

%\subsubsection{Future directions}

FPGA technology has proven invaluable in the development of the emerging research field of quantum computing.
However, the complexity of programming FPGA circuits hinders their implementation in quantum computing systems. Commercial efforts have been done toward providing graphical tools for designing FPGA programs, namely the Quantum Researchers Toolkit by Keysight Technologies and the FPGA-based multi-instrument platform Moku-Pro developed by Liquid Instruments. These tools are essential for implementing customized algorithms without the need for dedicated expertise in hardware description languages. Future research is also needed in integrating FPGAs in cryogenic platforms required to operate qubit systems. Such capability has already been demonstrated; commercial FPGAs can operate at temperatures below 4 K and be integrated in a cryogenic platform for qubit control~\cite{Homulle-2017}. These efforts provide evidence that FPGA technology is of great interest for enabling a scalable and practically applicable quantum computer. 


\subsection{Space}
\label{sec:space}
The flexibility of FPGA technology makes it a suitable platform for many applications on-board space missions. The European Space Research and Technology Centre (ESTEC), as part of the European Space Agency (ESA) actively explores FPGA technology for space applications, and has an extensive portfolio of FPGA Intellectual Property (IP) Cores~\cite{esa_ip}.

%\subsubsection{Research topics}

FPGAs can flexibly route its input and output ports, and can be configured to support many different communication protocols. This makes FPGAs good contenders as devices that communicate with the various hardware platforms and sensors on a space mission. FPGAs and have been implemented as interface devices in novel on-board machine learning and digital signal processing  implementations~\cite{Leon2021ImprovingSoC, Leon2021FPGABenchmarks, karapakula2024ncle}. 

An on-board task for which FPGAs are used is hyperspectral imaging. This type of on-board imaging produces vast amounts of data. To reduce transmission bandwidth requirements when transmitting the sensory data to earth, real-time on-board compression handling high data rates is required. FPGAs are well-suited for such tasks, and research has been done on using space-grade radiation-hardened FPGAs \cite{Barrios2020SHyLoCMissions} as well as commercial off-the-shelf (COTS) FPGAs \cite{Rodriguez2019ScalableCompression} for on-board hyperspectral image compression. COTS devices are generally cheaper than space-grade devices, but the higher susceptibility of these devices to radiation-induced effects makes them challenging to employ.

Communication between on-board systems often requires high data-rates and is susceptible to radiation induced effects. To deal with the unique constraints of space applications, dedicated communication protocols such as SpaceWire, and its successor, SpaceFibre have been developed. These protocols are available as FPGA IP implementations, and testing environments of SpaceFibre have been developed \cite{MystkowskaSimulationSpaceFibre, AnSection}. SpaceWire can interface with the common AXI4 protocol using a dedicated bridge \cite{RubattuASystems}, enabling its integration with SpaceWire interfaces. Direct Memory Access (DMA) allows peripherals to transfer data to and from an FPGA without going through a CPU. The application of DMA in space is being investigated, however its application as of now is limited since DMA is susceptibility to radiation-induced effects \cite{Portaluri2022Radiation-inducedDevices}.



\subsection{Bioinformatics}
\label{sec:bio}
FPGA technology has been extensively explored for accelerating Bioinformatics kernels. Bioinformatics is an interdisciplinary scientific field that combines biology, computer science, mathematics, and statistics to analyze and interpret biological data. The field primarily focuses on the development and application of methods, algorithms, and tools to handle, process, and analyze large sets of biological data, such as DNA sequences, protein structures, and gene expression patterns.Continuous advances in DNA sequencing technologies~\cite{hu2021next} have led to the rapid accumulation of biological data, creating an urgent need for high-performance computational solutions capable of efficiently managing increasingly larger datasets.

\citet{Shahroodi2022KrakenOnMem:Profiling} describe a hardware/software co-designed framework to accelerate and improve energy consumption of taxonomic profiling. In metagenomics, the main goal is to understand the role of each organism in our environment in order to
improve our quality of life, and taxonomic profiling involves the identification and categorization of the various types of organisms present in a biological sample by analyzing DNA or protein sequences from the sample to determine which species or taxa are represented. The study focuses on boosting performance of table lookup, which is the primary bottleneck in taxonomic profilers, by proposing a processing-in-memory hardware accelerator. Using large-scale simulations, the authors report an average of 63.1\% faster execution and orders of magnitude higher energy efficiency than the  widely used metagenomic analysis tool Kraken2~\cite{wood2019improved} executed on a 128-core server with AMD EPYC 7742 processors  operating at 2.25 GHz. An FPGA was used for prototyping and emulation purposes.

\citet{Corts2022AcceleratedFPGAs} employ FPGAs to accelerate the detection of traces of positive natural selection in genomes. The authors designed a hardware accelerator for the $\omega$ statistic~\cite{kim2004linkage}, which is extensively used in population genetics as an indicator of positive selection. In comparison with a single CPU core,
the FPGA accelerator can deliver up to $57.1\times$ faster
computation of the $\omega$ statistic, using the OmegaPlus~\cite{alachiotis2012omegaplus} software implementation as reference.


%\citet{Ahmad2022Communication-EfficientFlight}



\citet{Malakonakis2020ExploringRAxML} use FPGAs to accelerate the widely used phylogenetics software tool RAxML~\cite{stamatakis2014raxml}. The study implements the Phylogenetic Likelihood Function (PLF), which is used for evaluating phylogenetic trees, on a Xilinx ZCU102 development board and a cloud-based Amazon AWS EC2 F1 instance. The first system (ZCU102) can deploy two accelerator instances, operating at 250MHz, and delivers up to $7.7\times$ faster executions than sequential software execution on a AWS EC2 F1 instance. %Xeon processors. 
The AWS-based accelerated system is $5.2\times$ faster than the same software implementation. %In comparison with previous work by Alachiotis et al.~\cite{alachiotis2009exploring[7_12]}, the implementation on the Xilinx development board is about 2.35x faster. %, but the older technology should certainly be taken into consideration. 



\citet{Alachiotis2021AcceleratingCloud} also target the PLF implementation in RAxML, and propose an optimization method for data movement in PCI-attached accelerators using tree-search algorithms. They developed a software cache controller that leverages data dependencies between consecutive PLF calls to cache data on the accelerator card. In combination with double buffering over PCIe, this approach led to nearly $4\times$ improvement in the performance of an FPGA-based PLF accelerator. Executing the complete RAxML algorithm on an AWS EC2 F1 instance, the authors observed up to $9.2\times$ faster processing of protein data than a $2.7$ GHz Xeon processor in the same cloud environment.

With genomic datasets continuing to expand, bioinformatics analyses are likely to increasingly rely on cloud computing in the future. This shift will be supported by new programming models and frameworks designed to address the data-movement challenges posed by cloud-based hardware accelerators. These accelerators, such as FPGAs and GPUs, need data transfers from the host processor, which can significantly impact execution times and negate gains from computation improvements. Fortunately, similar data-movement concerns exist for both FPGAs and GPUs, and ongoing engineering efforts are likely to converge on common solutions~\cite{Corts2023AGenetics}. This will help bring optimized, hardware-accelerate processing techniques into more widespread use among computational biologists and bioinformaticians in the future.






\section{Experiments}
\section{Experiments}
\label{sec:experiments}
The experiments are designed to address two key research questions.
First, \textbf{RQ1} evaluates whether the average $L_2$-norm of the counterfactual perturbation vectors ($\overline{||\perturb||}$) decreases as the model overfits the data, thereby providing further empirical validation for our hypothesis.
Second, \textbf{RQ2} evaluates the ability of the proposed counterfactual regularized loss, as defined in (\ref{eq:regularized_loss2}), to mitigate overfitting when compared to existing regularization techniques.

% The experiments are designed to address three key research questions. First, \textbf{RQ1} investigates whether the mean perturbation vector norm decreases as the model overfits the data, aiming to further validate our intuition. Second, \textbf{RQ2} explores whether the mean perturbation vector norm can be effectively leveraged as a regularization term during training, offering insights into its potential role in mitigating overfitting. Finally, \textbf{RQ3} examines whether our counterfactual regularizer enables the model to achieve superior performance compared to existing regularization methods, thus highlighting its practical advantage.

\subsection{Experimental Setup}
\textbf{\textit{Datasets, Models, and Tasks.}}
The experiments are conducted on three datasets: \textit{Water Potability}~\cite{kadiwal2020waterpotability}, \textit{Phomene}~\cite{phomene}, and \textit{CIFAR-10}~\cite{krizhevsky2009learning}. For \textit{Water Potability} and \textit{Phomene}, we randomly select $80\%$ of the samples for the training set, and the remaining $20\%$ for the test set, \textit{CIFAR-10} comes already split. Furthermore, we consider the following models: Logistic Regression, Multi-Layer Perceptron (MLP) with 100 and 30 neurons on each hidden layer, and PreactResNet-18~\cite{he2016cvecvv} as a Convolutional Neural Network (CNN) architecture.
We focus on binary classification tasks and leave the extension to multiclass scenarios for future work. However, for datasets that are inherently multiclass, we transform the problem into a binary classification task by selecting two classes, aligning with our assumption.

\smallskip
\noindent\textbf{\textit{Evaluation Measures.}} To characterize the degree of overfitting, we use the test loss, as it serves as a reliable indicator of the model's generalization capability to unseen data. Additionally, we evaluate the predictive performance of each model using the test accuracy.

\smallskip
\noindent\textbf{\textit{Baselines.}} We compare CF-Reg with the following regularization techniques: L1 (``Lasso''), L2 (``Ridge''), and Dropout.

\smallskip
\noindent\textbf{\textit{Configurations.}}
For each model, we adopt specific configurations as follows.
\begin{itemize}
\item \textit{Logistic Regression:} To induce overfitting in the model, we artificially increase the dimensionality of the data beyond the number of training samples by applying a polynomial feature expansion. This approach ensures that the model has enough capacity to overfit the training data, allowing us to analyze the impact of our counterfactual regularizer. The degree of the polynomial is chosen as the smallest degree that makes the number of features greater than the number of data.
\item \textit{Neural Networks (MLP and CNN):} To take advantage of the closed-form solution for computing the optimal perturbation vector as defined in (\ref{eq:opt-delta}), we use a local linear approximation of the neural network models. Hence, given an instance $\inst_i$, we consider the (optimal) counterfactual not with respect to $\model$ but with respect to:
\begin{equation}
\label{eq:taylor}
    \model^{lin}(\inst) = \model(\inst_i) + \nabla_{\inst}\model(\inst_i)(\inst - \inst_i),
\end{equation}
where $\model^{lin}$ represents the first-order Taylor approximation of $\model$ at $\inst_i$.
Note that this step is unnecessary for Logistic Regression, as it is inherently a linear model.
\end{itemize}

\smallskip
\noindent \textbf{\textit{Implementation Details.}} We run all experiments on a machine equipped with an AMD Ryzen 9 7900 12-Core Processor and an NVIDIA GeForce RTX 4090 GPU. Our implementation is based on the PyTorch Lightning framework. We use stochastic gradient descent as the optimizer with a learning rate of $\eta = 0.001$ and no weight decay. We use a batch size of $128$. The training and test steps are conducted for $6000$ epochs on the \textit{Water Potability} and \textit{Phoneme} datasets, while for the \textit{CIFAR-10} dataset, they are performed for $200$ epochs.
Finally, the contribution $w_i^{\varepsilon}$ of each training point $\inst_i$ is uniformly set as $w_i^{\varepsilon} = 1~\forall i\in \{1,\ldots,m\}$.

The source code implementation for our experiments is available at the following GitHub repository: \url{https://anonymous.4open.science/r/COCE-80B4/README.md} 

\subsection{RQ1: Counterfactual Perturbation vs. Overfitting}
To address \textbf{RQ1}, we analyze the relationship between the test loss and the average $L_2$-norm of the counterfactual perturbation vectors ($\overline{||\perturb||}$) over training epochs.

In particular, Figure~\ref{fig:delta_loss_epochs} depicts the evolution of $\overline{||\perturb||}$ alongside the test loss for an MLP trained \textit{without} regularization on the \textit{Water Potability} dataset. 
\begin{figure}[ht]
    \centering
    \includegraphics[width=0.85\linewidth]{img/delta_loss_epochs.png}
    \caption{The average counterfactual perturbation vector $\overline{||\perturb||}$ (left $y$-axis) and the cross-entropy test loss (right $y$-axis) over training epochs ($x$-axis) for an MLP trained on the \textit{Water Potability} dataset \textit{without} regularization.}
    \label{fig:delta_loss_epochs}
\end{figure}

The plot shows a clear trend as the model starts to overfit the data (evidenced by an increase in test loss). 
Notably, $\overline{||\perturb||}$ begins to decrease, which aligns with the hypothesis that the average distance to the optimal counterfactual example gets smaller as the model's decision boundary becomes increasingly adherent to the training data.

It is worth noting that this trend is heavily influenced by the choice of the counterfactual generator model. In particular, the relationship between $\overline{||\perturb||}$ and the degree of overfitting may become even more pronounced when leveraging more accurate counterfactual generators. However, these models often come at the cost of higher computational complexity, and their exploration is left to future work.

Nonetheless, we expect that $\overline{||\perturb||}$ will eventually stabilize at a plateau, as the average $L_2$-norm of the optimal counterfactual perturbations cannot vanish to zero.

% Additionally, the choice of employing the score-based counterfactual explanation framework to generate counterfactuals was driven to promote computational efficiency.

% Future enhancements to the framework may involve adopting models capable of generating more precise counterfactuals. While such approaches may yield to performance improvements, they are likely to come at the cost of increased computational complexity.


\subsection{RQ2: Counterfactual Regularization Performance}
To answer \textbf{RQ2}, we evaluate the effectiveness of the proposed counterfactual regularization (CF-Reg) by comparing its performance against existing baselines: unregularized training loss (No-Reg), L1 regularization (L1-Reg), L2 regularization (L2-Reg), and Dropout.
Specifically, for each model and dataset combination, Table~\ref{tab:regularization_comparison} presents the mean value and standard deviation of test accuracy achieved by each method across 5 random initialization. 

The table illustrates that our regularization technique consistently delivers better results than existing methods across all evaluated scenarios, except for one case -- i.e., Logistic Regression on the \textit{Phomene} dataset. 
However, this setting exhibits an unusual pattern, as the highest model accuracy is achieved without any regularization. Even in this case, CF-Reg still surpasses other regularization baselines.

From the results above, we derive the following key insights. First, CF-Reg proves to be effective across various model types, ranging from simple linear models (Logistic Regression) to deep architectures like MLPs and CNNs, and across diverse datasets, including both tabular and image data. 
Second, CF-Reg's strong performance on the \textit{Water} dataset with Logistic Regression suggests that its benefits may be more pronounced when applied to simpler models. However, the unexpected outcome on the \textit{Phoneme} dataset calls for further investigation into this phenomenon.


\begin{table*}[h!]
    \centering
    \caption{Mean value and standard deviation of test accuracy across 5 random initializations for different model, dataset, and regularization method. The best results are highlighted in \textbf{bold}.}
    \label{tab:regularization_comparison}
    \begin{tabular}{|c|c|c|c|c|c|c|}
        \hline
        \textbf{Model} & \textbf{Dataset} & \textbf{No-Reg} & \textbf{L1-Reg} & \textbf{L2-Reg} & \textbf{Dropout} & \textbf{CF-Reg (ours)} \\ \hline
        Logistic Regression   & \textit{Water}   & $0.6595 \pm 0.0038$   & $0.6729 \pm 0.0056$   & $0.6756 \pm 0.0046$  & N/A    & $\mathbf{0.6918 \pm 0.0036}$                     \\ \hline
        MLP   & \textit{Water}   & $0.6756 \pm 0.0042$   & $0.6790 \pm 0.0058$   & $0.6790 \pm 0.0023$  & $0.6750 \pm 0.0036$    & $\mathbf{0.6802 \pm 0.0046}$                    \\ \hline
%        MLP   & \textit{Adult}   & $0.8404 \pm 0.0010$   & $\mathbf{0.8495 \pm 0.0007}$   & $0.8489 \pm 0.0014$  & $\mathbf{0.8495 \pm 0.0016}$     & $0.8449 \pm 0.0019$                    \\ \hline
        Logistic Regression   & \textit{Phomene}   & $\mathbf{0.8148 \pm 0.0020}$   & $0.8041 \pm 0.0028$   & $0.7835 \pm 0.0176$  & N/A    & $0.8098 \pm 0.0055$                     \\ \hline
        MLP   & \textit{Phomene}   & $0.8677 \pm 0.0033$   & $0.8374 \pm 0.0080$   & $0.8673 \pm 0.0045$  & $0.8672 \pm 0.0042$     & $\mathbf{0.8718 \pm 0.0040}$                    \\ \hline
        CNN   & \textit{CIFAR-10} & $0.6670 \pm 0.0233$   & $0.6229 \pm 0.0850$   & $0.7348 \pm 0.0365$   & N/A    & $\mathbf{0.7427 \pm 0.0571}$                     \\ \hline
    \end{tabular}
\end{table*}

\begin{table*}[htb!]
    \centering
    \caption{Hyperparameter configurations utilized for the generation of Table \ref{tab:regularization_comparison}. For our regularization the hyperparameters are reported as $\mathbf{\alpha/\beta}$.}
    \label{tab:performance_parameters}
    \begin{tabular}{|c|c|c|c|c|c|c|}
        \hline
        \textbf{Model} & \textbf{Dataset} & \textbf{No-Reg} & \textbf{L1-Reg} & \textbf{L2-Reg} & \textbf{Dropout} & \textbf{CF-Reg (ours)} \\ \hline
        Logistic Regression   & \textit{Water}   & N/A   & $0.0093$   & $0.6927$  & N/A    & $0.3791/1.0355$                     \\ \hline
        MLP   & \textit{Water}   & N/A   & $0.0007$   & $0.0022$  & $0.0002$    & $0.2567/1.9775$                    \\ \hline
        Logistic Regression   &
        \textit{Phomene}   & N/A   & $0.0097$   & $0.7979$  & N/A    & $0.0571/1.8516$                     \\ \hline
        MLP   & \textit{Phomene}   & N/A   & $0.0007$   & $4.24\cdot10^{-5}$  & $0.0015$    & $0.0516/2.2700$                    \\ \hline
       % MLP   & \textit{Adult}   & N/A   & $0.0018$   & $0.0018$  & $0.0601$     & $0.0764/2.2068$                    \\ \hline
        CNN   & \textit{CIFAR-10} & N/A   & $0.0050$   & $0.0864$ & N/A    & $0.3018/
        2.1502$                     \\ \hline
    \end{tabular}
\end{table*}

\begin{table*}[htb!]
    \centering
    \caption{Mean value and standard deviation of training time across 5 different runs. The reported time (in seconds) corresponds to the generation of each entry in Table \ref{tab:regularization_comparison}. Times are }
    \label{tab:times}
    \begin{tabular}{|c|c|c|c|c|c|c|}
        \hline
        \textbf{Model} & \textbf{Dataset} & \textbf{No-Reg} & \textbf{L1-Reg} & \textbf{L2-Reg} & \textbf{Dropout} & \textbf{CF-Reg (ours)} \\ \hline
        Logistic Regression   & \textit{Water}   & $222.98 \pm 1.07$   & $239.94 \pm 2.59$   & $241.60 \pm 1.88$  & N/A    & $251.50 \pm 1.93$                     \\ \hline
        MLP   & \textit{Water}   & $225.71 \pm 3.85$   & $250.13 \pm 4.44$   & $255.78 \pm 2.38$  & $237.83 \pm 3.45$    & $266.48 \pm 3.46$                    \\ \hline
        Logistic Regression   & \textit{Phomene}   & $266.39 \pm 0.82$ & $367.52 \pm 6.85$   & $361.69 \pm 4.04$  & N/A   & $310.48 \pm 0.76$                    \\ \hline
        MLP   &
        \textit{Phomene} & $335.62 \pm 1.77$   & $390.86 \pm 2.11$   & $393.96 \pm 1.95$ & $363.51 \pm 5.07$    & $403.14 \pm 1.92$                     \\ \hline
       % MLP   & \textit{Adult}   & N/A   & $0.0018$   & $0.0018$  & $0.0601$     & $0.0764/2.2068$                    \\ \hline
        CNN   & \textit{CIFAR-10} & $370.09 \pm 0.18$   & $395.71 \pm 0.55$   & $401.38 \pm 0.16$ & N/A    & $1287.8 \pm 0.26$                     \\ \hline
    \end{tabular}
\end{table*}

\subsection{Feasibility of our Method}
A crucial requirement for any regularization technique is that it should impose minimal impact on the overall training process.
In this respect, CF-Reg introduces an overhead that depends on the time required to find the optimal counterfactual example for each training instance. 
As such, the more sophisticated the counterfactual generator model probed during training the higher would be the time required. However, a more advanced counterfactual generator might provide a more effective regularization. We discuss this trade-off in more details in Section~\ref{sec:discussion}.

Table~\ref{tab:times} presents the average training time ($\pm$ standard deviation) for each model and dataset combination listed in Table~\ref{tab:regularization_comparison}.
We can observe that the higher accuracy achieved by CF-Reg using the score-based counterfactual generator comes with only minimal overhead. However, when applied to deep neural networks with many hidden layers, such as \textit{PreactResNet-18}, the forward derivative computation required for the linearization of the network introduces a more noticeable computational cost, explaining the longer training times in the table.

\subsection{Hyperparameter Sensitivity Analysis}
The proposed counterfactual regularization technique relies on two key hyperparameters: $\alpha$ and $\beta$. The former is intrinsic to the loss formulation defined in (\ref{eq:cf-train}), while the latter is closely tied to the choice of the score-based counterfactual explanation method used.

Figure~\ref{fig:test_alpha_beta} illustrates how the test accuracy of an MLP trained on the \textit{Water Potability} dataset changes for different combinations of $\alpha$ and $\beta$.

\begin{figure}[ht]
    \centering
    \includegraphics[width=0.85\linewidth]{img/test_acc_alpha_beta.png}
    \caption{The test accuracy of an MLP trained on the \textit{Water Potability} dataset, evaluated while varying the weight of our counterfactual regularizer ($\alpha$) for different values of $\beta$.}
    \label{fig:test_alpha_beta}
\end{figure}

We observe that, for a fixed $\beta$, increasing the weight of our counterfactual regularizer ($\alpha$) can slightly improve test accuracy until a sudden drop is noticed for $\alpha > 0.1$.
This behavior was expected, as the impact of our penalty, like any regularization term, can be disruptive if not properly controlled.

Moreover, this finding further demonstrates that our regularization method, CF-Reg, is inherently data-driven. Therefore, it requires specific fine-tuning based on the combination of the model and dataset at hand.

\section{Conclusion}
\section{Conclusion}
In this work, we propose a simple yet effective approach, called SMILE, for graph few-shot learning with fewer tasks. Specifically, we introduce a novel dual-level mixup strategy, including within-task and across-task mixup, for enriching the diversity of nodes within each task and the diversity of tasks. Also, we incorporate the degree-based prior information to learn expressive node embeddings. Theoretically, we prove that SMILE effectively enhances the model's generalization performance. Empirically, we conduct extensive experiments on multiple benchmarks and the results suggest that SMILE significantly outperforms other baselines, including both in-domain and cross-domain few-shot settings.

\section*{Impact Statement}
This paper presents work that aims to advance the field of Machine Learning. There are many potential societal consequences of our work, none of which we feel must be specifically highlighted here. 




\bibliography{bib}
\bibliographystyle{icml2025}

\newpage
\appendix
\onecolumn

\section{Principal and Independent Component Analysis (PICA)}

\begin{figure}
    \begin{subfigure}{\textwidth}
        \centering
        \includegraphics[width=0.6\linewidth]{latent_space_AE-linear_CovReg.jpg}
        \caption{PCA reduction implemented with a linear autoencoder and a covariance regularization term.}
        \label{fig:latent_space_pca_covreg}
    \end{subfigure}
    \vspace{20px}
    \begin{subfigure}{\textwidth}
        \centering
        \includegraphics[width=0.6\linewidth]{latent_space_AE-linear_dependence-linear.jpg}
        \caption{PCA reduction implemented with a linear autoencoder and linear \textit{dependency predictors}.}
        \label{fig:latent_space_pca_lin_dependency}
    \end{subfigure}
    \vspace{20px}
    \begin{subfigure}{\textwidth}
        \centering
        \includegraphics[width=0.6\linewidth]{latent_space_AE-linear_dependence-nonlinear.jpg}
        \caption{PICA reduction implemented with a linear autoencoder and nonlinear \textit{dependency predictors}.}
        \label{fig:latent_space_pica_nonlin_dependency}
    \end{subfigure}%
    \caption{Learned representations $\rvz$. The colors indicate the value of the original latent factors $\rv_1$ (left) and $\rv_2$ (right). }
    \label{fig:latent_spaces_pca_pica_example}
\end{figure}


\subsection{Empirical Study of Example~\ref{ex:pca_vs_pica}} \label{subapp:pca_ipca_example_eval}

In Example~\ref{ex:pca_vs_pica}, the solution that maximizes the explained variance under the zero correlation constraint is $\rvz_{\mathrm{PCA}} = [\rx_1,\rx_2]^T$, with a total variance of $\mathbb{V}[\rvz_{\mathrm{PCA}}] = 25 + 4.5 = 29.5$. 
However, this is not a solution to PICA since $\rx_1 = 5 \rv_1$ and $\rx_2 = 3 \cos{2\pi\rv_1/\sqrt{3}}$ are both functions of the same latent factor $\rv_1$. 
The solution to PICA is thus $\rvz_{\mathrm{PICA}} = [\rx_1,\rx_3]^T$, with a total explained variance of $\mathbb{V}[\rvz_{\mathrm{PICA}}] = 25 + 1 = 26$. 
Attention should be drawn to the fact that the explained variance of PICA is always smaller or equal to the PCA decomposition. The equality occurs only when the highest variance uncorrelated combination of the inputs is mutually (and non-linearly) independent. 

We empirically study this example by comparing the solutions to four different implementations:
\begin{enumerate}
    \item the PCA decomposition solved with a Singular Value Decomposition. For this example, we rely on the \textit{scikit-learn}~\citep{scikit-learn} implementation: \url{https://scikit-learn.org/1.6/modules/generated/sklearn.decomposition.PCA.html}.
    \item the PCA decomposition implemented with a linear autoencoder and a covariance minimization objective. This approach is similar to~\cite{mialon2022vcreg} but without their variance regularization term. 
    \item the PCA decomposition implemented with a linear autoencoder and our standardized adversarial objective with linear dependency predictors. 
    \item the PICA decomposition implemented with a linear autoencoder and our standardized adversarial objective with nonlinear dependency predictors. 
\end{enumerate}

\paragraph{Implementation details.} 
We train methods (2) to (4) for 5000 steps. We generate 512 observations $\rvx$ by sampling from the uniform latent factors $\rvv$ at every iteration. 
The encoder is implemented with a projection $\mW^T \in \mathbb{R}^{3 \times 2}$ and the decoder uses the same matrix $\mW$. The autoencoders and dependency predictors are trained with the Adam optimizer, with learning rates of respectively $5 \cdot 10^{-3}$ and $2 \cdot 10^{-2}$. The training steps ratios are set to $k=16$. 
The covariance/dependence minimization loss coefficients are set to $\lambda=1$, while the reconstruction loss coefficient is set to 0.02. This weighting strategy aims to push the method not to compromise the covariance/dependence for a decreased reconstruction error.  

The PCA implementations from (1), (2) and (3) find respectively:
\begin{align}
    W_{\mathrm{PCA,1}} \approx 
    \begin{bmatrix}
        1 & 0 & 0 \\
        0 & 1 & 0
    \end{bmatrix} 
    \quad W_{\mathrm{PCA,2}} \approx 
    \begin{bmatrix}
        0.00 & 1.00 & -0.01 \\
        1.00 & -0.01 & 0.00
    \end{bmatrix} 
    \quad W_{\mathrm{PCA,3}} \approx 
    \begin{bmatrix}
        0.00 & 1.00 & 0.01 \\
        -1.00 & 0.00 & 0.01
    \end{bmatrix} 
\end{align} 
In line with theory, the three PCA implementations thus extracted the first two observed variables $\rx_1$ and $\rx_2$. 
For implementations (1) to (3), the total explained variances for the representations $\rz$ are all approximately equal to $29.5$ and covariance matrices are close to diagonal. 
However, the learned latent dimensions are not independent, with a distance correlation of $\mathcal{R}(\rz_1, \rz_2) \approx 0.25$.
The methods accurately reconstruct the first two observed variables $\rx_1$ and $\rx_2$, and achieve an average reconstruction error of 1 for $\rx_3$. It should be emphasized that this average error corresponds to the variance of $\rx_3$. 

The representations $\rvz$ for experiments (2) and (3) are shown in Figure~\ref{fig:latent_space_pca_covreg} and Figure~\ref{fig:latent_space_pca_lin_dependency}. The abscissa and ordinate depict the learned latent variables $\rz_1$ and $\rz_2$, while the color of the predictions indicates the true value of the true latent factors $\rv_1$ (left figure) and $\rv_2$ (right figure). The distributions align with our previous finding and illustrate that none of the PCA implementations encoded the latent factor $\rv_2$. 

On the contrary, the PICA implementation (4) learns:
\begin{align}
    W_{\mathrm{PICA}} \approx 
    \begin{bmatrix}
        1.00 & 0.00 & 0.01 \\
        0.00 & -0.01 & 1.00
    \end{bmatrix} 
\end{align}
The average reconstruction error is 4.46, which approximately equals the second dimension's variance. 
Furthermore, the distance correlation between the two learned representations is $\mathcal{R}(\rz_1, \rz_2) \approx 0.008$. 
These empirical results validate that the PICA reduction captured the two independent latent factors $\rv_1$ and $\rv_2$. Furthermore, the method could not reconstruct $\rx_2$ since it is implemented with a linear autoencoder. 
The distribution of the learned representations $\rvz$ is shown in Figure~\ref{fig:latent_space_pica_nonlin_dependency}. 

\subsection{Nonlinear Principal and Independent Component Analysis} \label{subapp:nlpca_extension}

The Non-Linear Principal Component Analysis (NLPCA) extended PCA to non-linear transformations. 
It was introduced by~\cite{kramer1991autoencoder_nlpca} as an auto-encoder learning the identity mapping. The architecture included an intermediate "bottleneck" representation, forcing the network to learn low-dimensional data representations. 
The original architecture was a four-layer feed-forward neural network with sigmoid activation functions. 
Following this definition, one can learn a NLPCA reduction by training an encoder $f_{\vtheta}: \mathbb{R}^l \to \mathbb{R}^d$ and a decoder $g_{\vpsi}: \mathbb{R}^d \to \mathbb{R}^l$ to minimize the reconstruction error: 
\begin{equation} \label{eq:original_nlpca}
    \min_{\vtheta,\vpsi} \quad \quad \frac{1}{N} \sum_{i=1}^N \lVert \vx^{(i)} - g_{\vpsi}(f_{\vtheta}(\vx^{(i)})) \rVert_2^2
\end{equation}
One may assume that a solution with minimal reconstruction error should exhibit low redundancy in the bottleneck representation. 
Yet, equation~\ref{eq:original_nlpca} does not explicitly push the bottleneck representation to have uncorrelated dimensions. 
Therefore, starting from the encoder-decoder architecture from the equation~\ref{eq:original_nlpca}, we add the ADMin objective by following the development from Section~\ref{subsec:pca_application}:
\begin{equation}
    \min_{\vtheta,\vpsi} \quad \quad \frac{1}{N} \sum_{i=1}^N \lVert \vx^{(i)} - g_{\vpsi}(f_{\vtheta}(\vx^{(i)})) \rVert_2^2 
    + \lambda \mathcal{L}_{\mathrm{adv}}(\vz^{(i)}, \hat{\vz}^{(i)})
\end{equation}
where $\vz^{(i)} = f_{\vtheta}(\vx^{(i)})$ is the bottleneck representation and $\hat{\vz}^{(i)}_j = h_{\vphi_j}(z^{(i)}_1, \dots, z^{(i)}_{j-1}, z^{(i)}_{j+1}, \dots, z^{(i)}_d)$ is the prediction from the $j$-th \textit{dependency predictor}. 
We denote this extension as \textit{Non-Linear Principal and Independent Component Analysis} (NLPICA) since it is an extension of the NLPCA reduction with an additional objective pushing the representations to have minimally interdependent dimensions. 

Note that the solution to the NLPICA reduction is not unique. 
In fact, research in nonlinear independent component analysis (NLICA) demonstrated that there are countless ways to transform independent variables while maintaining statistical independence~\citep{darmois1951analyse_liaisons,jutten2004_BSS_NLICA}. These transformations can involve complex mixing functions that result in representations that may be challenging to interpret or exhibit undesirable properties for certain applications. 
For examples of mixing transformations, we refer to~\citep{taleb1999example_mixing}. 
Despite this limitation, the problem can be made identifiable again with additional assumptions such as temporal non-stationarity~\citep{hyvarinen2016time_contrast_ica}, a conditionally factorized prior distribution over the latent variables of a VAE~\citep{khemakhem2020ivae_ica} or constraining the function class with constraints on their partial derivatives~\citep{buchholz2022ident_nlica_fn_class}.

\section{Analysis}
\label{sec:analysis}
In the following sections, we will analyze European type approval regulation\footnote{Strictly speaking, the German enabling act (AFGBV) does not regulate type-approval, but how test \& operating permits are issued for SAE-Level-4 systems. Type-approval regulation for SAE-Level-3 systems follows UN Regulation No. 157 (UN-ECE-ALKS) \parencite{un157}.} regarding the underlying notions of ``safety'' and ``risk''.
We will classify these notions according to their absolute or relative character, underlying risk sources, or underlying concepts of harm.

\subsection{Classification of Safety Notions}
\label{sec:safety-notions}
We will refer to \emph{absolute} notions of safety as conceptualizations that assume the complete absence of any kind of risk.
Opposed to this, \emph{relative} notions of safety are based on a conceptualization that specifically includes risk acceptance criteria, e.g., in terms of ``tolerable'' risk or ``sufficient'' safety.

For classifying notions of safety by their underlying risk (or rather ``hazard'') sources, and different concepts of harm, \Cref{fig:hazard-sources} provides an overview of our reasoning, which is closely in line with the argumentation provided by Waymo in \parencite{favaro2023}.
We prefer ``hazard sources'' over ``risk sources'', as a risk must always be related to a \emph{cause} or \emph{source of harm} (i.e., a hazard \parencite[p.~1, def. 3.2]{iso51}).
Without a concrete (scenario) context that the system is operating in, a hazard is \emph{latent}: E.g., when operating in public traffic, there is a fundamental possibility that a \emph{collision with a pedestrian} leads to (physical) harm for that pedestrian. 
However, only if an automated vehicle shows (potentially) hazardous behavior (e.g., not decelerating properly) \emph{and} is located near a pedestrian (context), the hazard is instantiated and leads to a hazardous event.
\begin{figure*}
    \includeimg[width=.9\textwidth]{hazard-sources0.pdf}
    \caption{Graphical summary of a taxonomy of risk related to automated vehicles, extended based on ISO 21448 (\parencite{iso21448}) and \parencite{favaro2023}. Top: Causal chain from hazard sources to actual harm; bottom: summary of the individual elements' contributions to a resulting risk. Graphic translated from \parencite{nolte2024} \label{fig:hazard-sources}}
\end{figure*}
If the hazardous event cannot be mitigated or controlled, we see a loss event in which the pedestrian's health is harmed.
Note that this hypothetical chain of events is summarized in the definition of risk:
The probability of occurrence of harm is determined by a) the frequency with which hazard sources manifest, b) the time for which the system operates in a context that exposes the possibility of harm, and c) by the probability with which a hazardous event can be controlled.
A risk can then be determined as a function of the probability of harm and the severity of the harm potentially inflicted on the pedestrian.

In the following, we will apply this general model to introduce different types of hazard sources and also different types of harm.
\cref{fig:hazard-sources} shows two distinct hazard sources, i.e., functional insufficiencies and E/E-failures that can lead to hazardous behavior.
ISO~21488 \parencite{iso21448} defines functional insufficiencies as insufficiencies that stem from an incomplete or faulty system specification (specification insufficiencies).
In addition, the standard considers insufficiencies that stem from insufficient technical capability to operate inside the targeted Operational Design Domain (performance insufficiencies).
Functional insufficiencies are related to the ``Safety of the Intended Functionality (SOTIF)'' (according to ISO~21448), ``Behavioral Safety'' (according to Waymo \parencite{waymo2018}), or ``Operational Safety'' (according to UN Regulation No. 157 \parencite{un157}).
E/E-Failures are related to classic functional safety and are covered exhaustively by ISO~26262 \parencite{iso2018}.
Additional hazard sources can, e.g., be related to malicious security attacks (ISO~21434), or even to mechanical failures that should be covered (in the US) in the Federal Motor Vehicle Safety Standards (FMVSS).

For the classification of notions of safety by the related harm, in \parencite{salem2024, nolte2024}, we take a different approach compared to \parencite{koopman2024}:
We extend the concept of harm to the violation of stakeholder \emph{values}, where values are considered to be a ``standard of varying importance among other such standards that, when combined, form a value pattern that reduces complexity for stakeholders [\ldots] [and] determines situational actions [\ldots].'' \parencite{albert2008}
In this sense, values are profound, personal determinants for individual or collective behavior.
The notion of values being organized in a weighted value pattern shows that values can be ranked according to importance.
For automated vehicles, \emph{physical wellbeing} and \emph{mobility} can, e.g., be considered values which need to be balanced to achieve societal acceptance, in line with the discussion of required tradeoffs in \cref{sec:terminology}.
For the analysis of the following regulatory frameworks, we will evaluate if the given safety or risk notions allow tradeoffs regarding underlying stakeholder values. 

\subsection{UN Regulation No. 157 \& European Implementing Regulation (EU) 2022/1426}
\label{sec:enabling-act}
UN Regulation No. 157 \parencite{un157} and the European Implementing Regulation 2022/1426 \parencite{eu1426} provide type approval regulation for automated vehicles equipped with SAE-Level-3 (UN Reg. 157) and Level 4 (EU 2022/1426) systems on an international (UN Reg. 157) and European (EU 2022/1426) level.

Generally, EU type approval considers UN ECE regulations mandatory for its member states ((EU) 2018/858, \parencite{eu858}), while the EU largely forgoes implementing EU-specific type approval rules, it maintains the right to alter or to amend UN ECE regulation \parencite{eu858}.

In this respect, the terminology and conceptualizations in the EU Implementing Act closely follow those in UN Reg. No. 157.
The EU Implementing Act gives a clear reference to UN Reg. No. 157 \parencite[][Preamble,  Paragraph 1]{eu1426}.
Hence, the documents can be assessed in parallel.
Differences will be pointed out as necessary.

Both acts are written in rather technical language, including the formulation of technical requirements (e.g., regarding deceleration values or speeds in certain scenarios).
While providing exhaustive definitions and terminology, neither of both documents provide an actual definition of risk or safety.
The definition of ``unreasonable'' risk in both documents does not define risk, but only what is considered \emph{unreasonable}. It states that the ``overall level of risk for [the driver, (only in UN Reg. 157)] vehicle occupants and other road users which is increased compared to a competently and carefully driven manual vehicle.''
The pertaining notions of safety and risk can hence only be derived from the context in which they are used.

\subsubsection{Absolute vs. Relative Notions of Safety}
In line with the technical detail provided in the acts, both clearly imply a \emph{relative} notion of safety and refer to the absence of \emph{unreasonable} risk throughout, which is typical for technical safety definitions.

Both acts require sufficient proof and documentation that the to-be-approved automated driving systems are ``free of unreasonable safety risks to vehicle occupants and other road users'' for type approval.\footnote{As it targets SAE-Level-3 systems, UN Reg. 157 also refers to the driver, where applicable.}
In this respect, both acts demand that the manufacturers perform verification and validation activities for performance requirements that include ``[\ldots] the conclusion that the system is designed in such a way that it is free from unreasonable risks [\ldots]''.
Additionally, \emph{risk minimization} is a recurring theme when it comes to the definition of Minimum Risk Maneuvers (MRM).

Finally, supporting the relative notions of safety and risk, UN Reg. 157 introduces the concept of ``reasonable foreseeable and preventable'' \parencite[Article 1, Clause 5.1.1.]{un157} collisions, which implies that a residual risk will remain with the introduction of automated vehicles.
\parencite[][Appendix 3, Clause 3.1.]{un157} explicitly states that only \emph{some} scenarios that are unpreventable for a competent human driver can actually be prevented by an automated driving system.
While this concept is not applied throughout the EU Implementing Act, both documents explicitly refer to \emph{residual} risks that are related to the operation of automated driving systems (\parencite[][Annex I, Clause 1]{un157}, \parencite[][Annex II, Clause 7.1.1.]{eu1426}).

\subsubsection{Hazard Sources}
Hazard sources that are explicitly differentiated in UN Reg. 157 and (EU) 2022/1426 are E/E-failures that are in scope of functional safety (ISO~26262) and functional insufficiencies that are in scope of behavioral (or ``operational'') safety (ISO~21448).
Both documents consistently differentiate both sources when formulating requirements.

While the acts share a common definition of ``operational'' safety (\parencite[][Article 2, def. 30.]{eu1426}, \parencite[][Annex 4, def. 2.15.]{un157}), the definitions for functional safety differ.
\parencite{un157} defines functional safety as the ``absence of unreasonable risk under the occurrence of hazards caused by a malfunctioning behaviour of electric/electronic systems [\ldots]'', \parencite{eu1426} drops the specification of ``electric/electronic systems'' from the definition.
When taken at face value, this definition would mean that functional safety included all possible hazard sources, regardless of their origin, which is a deviation from the otherwise precise usage of safety-related terminology.

\subsubsection{Harm Types}
As the acts lack explicit definitions of safety and risk, there is no consistent and explicit notion of different harm types that could be differentiated.

\parencite{un157} gives little hints regarding different considered harm types.
``The absence of unreasonable risk'' in terms of human driving performance could hence be related to any chosen performance metric that allows a comparison with a competent careful human driver including, e.g., accident statistics, statistics about rule violations, or changes in traffic flow.

In \parencite{eu1426}, ``safety'' is, implicitly, attributed to the absence of unreasonable risk to life and limb of humans.
This is supported by the performance requirements that are formulated:
\parencite[][Annex II, Clause 1.1.2. (d)]{eu1426} demands that an automated driving system can adapt the vehicle behavior in a way that it minimizes risk and prioritizes the protection of human life.

Both acts demand the adherence to traffic rules (\parencite[][Annex 2, Clause 1.3.]{eu1426}, \parencite[][Clause 5.1.2.]{un157}).
\parencite[][Annex II, Clause 1.1.2. (c)]{eu1426} also demands that an automated driving system shall adapt its behavior to surrounding traffic conditions, such as the current traffic flow.
With the relative notion of risk in both acts, the unspecific clear statement that there may be unpreventable accidents \parencite{un157}, and a demand of prioritization of human life in \parencite{eu1426}, both acts could be interpreted to allow developers to make tradeoffs as discussed in \cref{sec:terminology}.


\subsubsection{Conclusion}
To summarize, the UN Reg. 157 and the (EU) 2022/1426 both clearly support the technical notion of safety as the absence of unreasonable risk.
The notion is used consistently throughout both documents, providing a sufficiently clear terminology for the developers of automated vehicles.
Uncertainty remains when it comes to considered harm types: Both acts do not explicitly allow for broader notions of safety, in the sense of \parencite{koopman2024} or \parencite{salem2024}.
Finally, a minor weak spot can be seen in the definition of risk acceptance criteria: Both acts take the human driving performance as a baseline.
While (EU) 2022/1426 specifies that these criteria are specific to the systems' Operational Design Domain \parencite[][Annex II, Clause 7.1.1.]{eu1426}, the reference to the concrete Operational Design Domain is missing in UN Reg. 157.
Without a clearly defined notion of safety, however, it remains unclear, how aspects beyond net accident statistics (which are given as an example in \parencite[][Annex II, Clause 7.1.1.]{eu1426}), can be addressed practically, as demanded by \parencite{koopman2024}.

\subsection{German Regulation (StVG \& AFGBV)}
\label{sec:afgbv}
The German L3 (Automated Driving Act) and L4 (Act on Autonomous Driving) Acts from 2017 and 2021,\footnote{Formally, these are amendments to the German Road Traffic Act (StVG): 06/21/2017, BGBl. I p. 1648, 07/12/2021 BGBl. I p. 3108.} respectively, provide enabling regulation for the operation of SAE-Level-3 and 4 vehicles on German roads.
The German Implementing Regulation (\parencite{afgbv}, AFGBV) defines how this enabling regulation is to be implemented for granting testing permits for SAE-Level-3 and -4 and driving permits for SAE-Level-3 and -4 automated driving systems.\footnote{Note that these permits do not grant EU-wide type approval, but serve as a special solution for German roads only. At the same time, the AFGBV has the same scope as (EU) 2022/1426.}
With all three acts, Germany was the first country to regulate the approval of automated vehicles for a domestic market.
All acts are subject to (repeated) evaluation until the year 2030 regarding their impact on the development of automated driving technology.
An assessment of the German AFGBV and comparisons to (EU) 2022/1426 have been given in \cite{steininger2022} in German.

Just as for UN Reg. 157 and (EU) 2022/1426, neither the StVG nor the AFGBV provide a clear definition of ``safety'' or ``risk'' -- even though the "safety" of the road traffic is one major goal of the StVG and StVO.
Again, different implicit notions of both concepts can only be interpreted from the context of existing wording.
An additional complication that is related to the German language is that ``safety'' and ``security'' can both be addressed as ``Sicherheit'', adding another potential source of unclarity.
Literal Quotations in this section are our translations from the German act.

\subsubsection{Absolute vs. Relative Notions of Safety}
For assessing absolute vs. relative notions of safety in German regulation, it should be mentioned that the main goal of the German StVO is to ensure the ``safety and ease of traffic flow'' -- an already diametral goal that requires human drivers to make tradeoffs.\footnote{For human drivers, this also creates legal uncertainty which can sometimes only be settled in a-posteriori court cases.}
While UN and EU regulation clearly shows a relative notion of safety\footnote{And even the StVG contains sections that use wording such as ``best possible safety for vehicle occupants'' (§1d (4) StVG) and acknowledges that there are unavoidable hazards to human life (§1e (2) No. 2c)).}, the German AFGBV contains ambiguous statements in this respect:
Several paragraphs contain a demand for a hazard free operation of automated vehicles.
§4 (1) No. 4 AFGBV, e.g., states that ``the operation of vehicles with autonomous driving functions must neither negatively impact road traffic safety or traffic flow, nor endanger the life and limb of persons.''
Additionally, §6 (1) AFGBV states that the permits for testing and operation have to be revoked, if it becomes apparent that a ``negative impact on road traffic safety or traffic flow, or hazards to the life and limb of persons cannot be ruled out''.
The same wording is used for the approval of operational design domains regulated in §10 (1) No. 1.
A particularly misleading statement is made regarding the requirements for technical supervision instances which are regulated in §14 (3) AFGBV which states that an automated vehicle has to be  ``immediately removed from the public traffic space if a risk minimal state leads to hazards to road traffic safety or traffic flow''.
Considering the argumentation in \cref{sec:terminology}, that residual risks related to the operation of automated driving systems are inevitable, these are strong statements which, if taken at face value, technically prohibit the operation of automated vehicles.
It suggests an \emph{absolute} notion of safety that requires the complete absence of risk.  
The last statement above is particularly contradictory in itself, considering that a risk \emph{minimal} state always implies a residual risk.

In addition to these absolute safety notions, there are passages which suggest a relative notion of safety:
The approval for Operational Design Domains is coupled to the proof that the operation of an automated vehicle ``neither negatively impacts road traffic safety or traffic flow, nor significantly endangers the life and limb of persons beyond the general risk of an impact that is typical of local road traffic'' (§9 (2) No. 3 AFGBV).
The addition of a relative risk measure ``beyond the general risk of an impact'' provides a relaxation (cf. also \cite{steininger2022}, who criticizes the aforementioned absolute safety notion) that also yields an implicit acceptance criterion (\emph{statistically as good as} human drivers) similar to the requirements stated in UN Reg. 157 and (EU) 2022/1426.

Additional hints for a relative notion of safety can be found in Annex 1, Part 1, No. 1.1 and Annex 1, Part 2, No. 10.
Part 1, No 1.1 specifies collision-avoidance requirements and acknowledges that not all collisions can be avoided.\footnote{The same is true for Part 2, No. 10, Clause 10.2.5.}
Part 2, No. 10 specifies requirements for test cases.
It demands that test cases are suitable to provide evidence that the ``safety of a vehicle with an autonomous driving function is increased compared to the safety of human-driven vehicles''.
This does not only acknowledge residual risks, but also yields an acceptance criterion (\emph{better} than human drivers) that is different from the implied acceptance criterion given in §9 (2) No. 3 AFGBV.

\subsubsection{Hazard Sources}
Regarding hazard sources, Annex 1 and 3 AFGBV explicitly refer to ISO~26262 and ISO~21448 (or rather its predecessor ISO/PAS~21448:2019).
However, regarding the discussion of actual hazard sources, the context in which both standards are mentioned is partially unclear:
Annex 1, Clause 1.3 discusses requirements for path and speed planning.
Clause 1.3 d) demands that in intersections, a Time to Collision (TTC) greater than 3 seconds must be guaranteed.
If manufacturers deviate from this, it is demanded that ``state-of-the-art, systematic safety evaluations'' are performed.
Fulfillment of the state of the art is assumed if ``the guidelines of ISO~26262:2018-12 Road Vehicles -- Functional Safety are fulfilled''.
Technically, ISO~26262 is not suitable to define the state of the art in this context, as the requirements discussed fall in the scope of operational (or behavioral) safety (ISO~21448).
A hazard source ``violated minimal time to collision'' is clearly a functional insufficiency, not an E/E-failure.

Similar unclarity presents itself in Annex 3, Clause 1 AFGBV: 
Clause 1 specifies the contents of the ``functional specification''.
The ``specification of the functionality'' is an artifact which is demanded in ISO~21448:2022 (Clause 5.3) \parencite{iso21448}.
However, Annex 3, Clause 1 AFGBV states that the ``functional specification'' is considered to comply to the state of the art, if the ``functional specification'' adheres to ISO~26262-3:2018 (Concept Phase).
Again, this assumes SOTIF-related contents as part of ISO~26262, which introduces the ``Item Definition'' as an artifact, which is significantly different from the ``specification of the functionality'' which is demanded by ISO~21448.
Finally, Annex 3, Clause 3 AFGBV demands a ``documentation of the safety concept'' which ``allows a functional safety assessment''.
A safety concept that is related to operational / behavioral safety is not demanded.
Technically, the unclarity with respect to the addressed harm types lead to the fact that the requirements provided by the AFGBV do not comply with the state of the art in the field, providing questionable regulation.

\subsubsection{Harm Types}
Just like UN Reg. 157 and (EU) 2022/1426, the German StVG and AFGBV do not explicitly differentiate concrete harm types for their notions of safety.
However, the AFGBV mentions three main concerns for the operation of automated vehicles which are \emph{traffic flow} (e.g., §4 (1) No. 4 AFGBV), compliance to \emph{traffic law} (e.g., §1e (2) No. 2 StVG), and the \emph{life and limb of humans} (e.g., §4 (1) No. 4 AFGBV).

Again, there is some ambiguity in the chosen wording:
The conflict between traffic flow and safety has already been argued in \cref{sec:terminology}.
The wording given in §4 (1) No. 4 and §6 (1) AFGBV  demand to ensure (absolute) safety \emph{and} traffic flow at the same time, which is impossible (cf. \cref{sec:terminology}) from an engineering perspective.
§1e (2) No. 2 StVG defines that ``vehicles with an autonomous driving function must [\ldots] be capable to comply to [\ldots] traffic rules in a self-contained manner''.
Taken at face value, this wording implies that an automated driving system could lose its testing or operating permit as soon as it violates a traffic rule.
A way out could be provided by §1 of the German Traffic Act (StVO) which demands careful and considerate behavior of all traffic participants and by that allows judgement calls for human drivers.
However, if §1 is applicable in certain situations is often settled in court cases. 
For developers, the application of §1 StVO during system design hence remains a legal risk.

While there are rather absolute statements as mentioned above, sections of the AFGBV and StVG can be interpreted to allow tradeoffs:
§1e (2) No. 2 b) demands that a system,  ``in case of an inevitable, alternative harm to legal objectives, considers the significance of the legal objectives, where the protection of human life has highest priority''.
This exact wording \emph{could} provide some slack for the absolute demands in other parts of the acts, enabling tradeoffs between (tolerable) risk and mobility as discussed in \cref{sec:terminology}.
However, it remains unclear if this interpretation is legally possible.

\subsubsection{Conclusion}
Compared to UN Reg. 157 and (EU) 2022/1426, the German StVG and AFGBV introduce openly inconsistent notions of safety and risk which are partially directly contradictory:
The wording partially implies absolute and relative notions of safety and risk at the same time.
The implied validation targets (``better'' or ``as good as'' human drivers) are equally contradictory. 
The partially implied absolute notions of safety, when taken at face value, prohibit engineers from making the tradeoffs required to develop a system that is safe and provides customer benefit at the same time. 
In consequence, the wording in the acts is prone to introducing legal uncertainty.
This uncertainty creates additional clarification need and effort for manufacturers and engineers who design and develop SAE-Level-3 and -4 automated driving systems. The use of undefined legal terms not only makes it more difficult for engineers to comply with the law, but also complicates the interpretation of the law and leads to legal uncertainty.

\subsection{UK Automated Vehicles Act 2024 (2024 c. 10)}
The UK has issued a national enabling act for regulating the approval of automated vehicles on the roads in the UK.
To the best of our knowledge, concrete implementing regulation has not been issued yet.
Regarding terminology, the act begins with a dedicated terminology section to clarify the terms used in the act \parencite[Part 1, Chapter 1, Section 1]{ukav2024}.
In that regard, the act defines a vehicle to drive ```autonomously' if --- (a)
it is being controlled not by an individual but by equipment of the vehicle, and (b) neither the vehicle nor its surroundings are being monitored by an individual with a view to immediate intervention in the driving of the vehicle.''
The act hence covers SAE-Level-3 to SAE-Level-5 automated driving systems.

\subsubsection{Absolute vs. Relative Notions of Safety}
While not providing an explicit definition of safety and risk, the UK Automated Vehicles Act (``UK AV Act'') \parencite{ukav2024} explicitly refers to a relative notion of safety.
Part~1, Chapter~1, Section~1, Clause (7)~(a) defines that an automated vehicle travels ```safely' if it travels to an acceptably safe standard''.
This clarifies that absolute safety is not achievable and that acceptance criteria to prove the acceptability of residual risk are required, even though a concrete safety definition is not given.
The act explicitly tasks the UK Secretary of State\footnote{Which means, that concrete implementation regulation needs to be enacted.} to install safety principles to determine the ``acceptably safe standard'' in Part~1, Chapter~1, Section~1, Clause (7)~(a).
In this respect, the act also provides one general validation target as it demands that the safety principles must ensure that ``authorized automated vehicles will achieve a level of safety equivalent to, or higher than, that of careful and competent human drivers''.
Hence, the top-level validation risk acceptance criterion assumed for UK regulation is ``\emph{at least as good} as human drivers''.

\subsubsection{Hazard Sources}
The UK AV Act contains no statements that could be directly related to different hazard sources.
Note that, in contrast to the rest of the analyzed documents, the UK AV Act is enabling rather than implementing regulation.
It is hence comparable to the German StVG, which does not refer to concrete hazard sources as well.

\subsubsection{Types of Harm}
Even though providing a clear relative safety notion, the missing definition of risk also implies a lack of explicitly differentiable types of harm.
Implicitly, three different types of harm can be derived from the wording in the act.
This includes the harm to life and limb of humans\footnote{Part~1, Chapter~3, Section~25 defines ``aggravated offence where death or serious injury occurs'' \parencite{ukav2024}.}, the violation of traffic rules\footnote{Part~1, Chapter~1, Clause~(7)~(b) defines that an automated vehicle travels ```legally' if it travels with an acceptably low risk of committing a traffic infraction''}, and the cause of inconvenience to the public \parencite[Part~1, Chapter~1, Section~58, Clause (2)~(d)]{ukav2024}.

The act connects all the aforementioned types of harm to ``risk'' or ``acceptable safety''.
While the act generally defines criminal offenses for providing ``false or misleading information about safety'', it also acknowledges possible defenses if it can be proven that ``reasonable precautions'' were taken and that ``due diligence'' was exercised to ``avoid the commission of the offence''.
This statement could enable tradeoffs within the scope of ``reasonable risk'' to the life and limb of humans, the violation of traffic rules, or to the cause of inconvenience to the public, as we argued in \cref{sec:terminology}.

\subsubsection{Conclusion}
From the set of reviewed documents, the current UK AV Act is the one with the most obvious relative notions of safety and risk and the one that seems to provide a legal framework for permitting tradeoffs.
In our review, we did not spot major inconsistency beyond a missing definitions of safety and risk\footnote{Note that with the Office for Product Safety and Standards (OPSS), there is a British government agency that maintains an exhaustive and widely focussed ``Risk Lexicon'' that provides suitable risk definitions. For us, it remains unclear, to what extent this terminology is assumed general knowledge in British legislation.}.
The general, relative notion of safety and the related alleged ability for designers to argue well-founded development tradeoffs within the legal framework could prove beneficial for the actual implementation of automated driving systems.
While the act thus appears as a solid foundation for the market introduction of automated vehicles, without accompanying implementing regulation, it is too early to draw definite conclusions.
\begin{table}

\centering
 
\subtable[Decomposition Independence Evaluation on CSQA]{
\centering
\label{tbl:indep_csqa}
\renewcommand\arraystretch{1}
\setlength{\tabcolsep}{0.5mm}
\resizebox{\linewidth}{!}
{
\begin{tabular}{c|c|c|c|c}
Method                          & Independence & Acc    & Time & API    \\ 
\hline
(Ours) DoT decomposition        & 95\%         & 82\%   & 9.9 & 0.49\textcent \\
vanilla prompting decomposition & 87\%         & 81\% & 13.2 & 0.77\textcent 
\end{tabular}
}
}
 
\qquad
 
\subtable[Decomposition Independence Evaluation on MATH]{        
\centering
\label{tbl:indep_math}
\renewcommand\arraystretch{1}
\setlength{\tabcolsep}{0.5mm}
\resizebox{\linewidth}{!}
{
\begin{tabular}{c|c|c|c|c}
Method                          & Independence & Acc    & Time & API    \\ 
\hline
(Ours) DoT decomposition        & 90\%         & 59\%   & 22.6 & 1.02\textcent \\
vanilla prompting decomposition & 76\%         & 57\% & 38.4 & 1.78\textcent 
\end{tabular}
}
}

\qquad

\subtable[Decomposition Independence Evaluation on P3]{        
\centering
\label{tbl:indep_p3}
\renewcommand\arraystretch{1}
\setlength{\tabcolsep}{0.5mm}
\resizebox{\linewidth}{!}
{

\begin{tabular}{c|c|c|c|c}
Method                          & Independence & Acc    & Time & API    \\ 
\hline
(Ours) DoT decomposition        & 86\%         & 41\%   & 23.5 & 1.58\textcent  \\
vanilla prompting decomposition & 71\%         & 36.5\% & 24.5 & 2.23\textcent 
\end{tabular}
}
}

\caption{Evaluation of Decomposition Independence}
\label{tbl:indepedence}
\vspace{-6mm}
\end{table}



\begin{algorithm}[h!]
\caption{Gait-Net-augmented Sequential CMPC}
\label{alg:gaitMPC}
\begin{algorithmic}[1]
\Require $\mathbf q, \: \dot{\mathbf q}, \: \mathbf q^\text{cmd}, \: \dot{\mathbf q}^\text{cmd}$
\State \textbf{intialize} $\bm x_0 = f_\text{j2m}(\mathbf q, \: \dot{\mathbf q}), \: \bm u^0 =\bm u_\text{IG}, \: dt^0 = 0.05$ 
\State $\{ \mathbf q^\text{ref},\:\dot{\mathbf q}^\text{ref},\:\bm p_f^\text{ref}\} = f_\text{ref} \big(\mathbf q, \: \dot{\mathbf q}, \: \mathbf q^\text{cmd}, \: \dot{\mathbf q}^\text{cmd} \big)$
\State $\bm x^\text{ref} = f_\text{j2m}(\mathbf q^\text{ref},\:\dot{\mathbf q}^\text{ref},\:\bm p_f^\text{ref})$
\State $ j = 0$ 
\While{$j \leq j_\text{max} \:\text{and}\: \bm \eta \leq \delta \bm u  $} 
\State $\delta \bm u^{j} = \texttt{cmpc}(\bm x^\text{ref},\:\bm p_f^\text{ref},\:\bm p_c^\text{ref},\: \bm x_0,\: dt^j, \: \bm u^j)$
\State $\bm u^{j+1} = \bm u^j + \delta \bm u^j$ 
\State $dt^{j+1} = \Pi_\text{GN}(\mathbf q, \: \dot{\mathbf q},\: \bm p_f^{j})$
\State $\{ \bm x^\text{ref},\:\bm p_f^\text{ref}\}= f_\text{IK}(\bm p_f^{j},\:\bm p_c^{j},\: dt^{j+1})$
\State $j=j+1$
\EndWhile \\
\Return $\bm u^{j+1} $
\end{algorithmic}
\end{algorithm}
\section{Additional Results}
\begin{table}[h]
    \centering
    \begin{tabular}{l l p{1.8cm} p{1.8cm} p{1.8cm} p{1.8cm}}
    \toprule
    & & \multicolumn{4}{c}{\textbf{FinancialQA Optimized Configurations}} \\ 
    & & \textbf{Safety} & \textbf{Alignment} & \textbf{Cost} & \textbf{Latency} \\ \midrule
    \multirow{7}{*}{Hyperparam}& Embedding model & text-embedding-3-large & text-embedding-3-large & text-embedding-3-large & text-embedding-3-small\\
    & LLM & gpt-4o-mini & Llama-3.1-8B & Llama-3.2-3B & Llama-3.1-8B\\
    & Chunk size & 1024 & 1024 & 512 & 1024\\
    & Chunk overlap & 512 & 128 & 64 & 64 \\
    & Num chunks & 3 & 4 & 3 & 3 \\
    & Rerank threshold & 0.00 & 0.64 & 0.24 & 1.00 \\
    & Temperature & 0.03 & 0.12 & 0.79 & 0.00 \\ \midrule
    \multirow{4}{*}{Objective} & Safety & \textbf{98.1} & \underline{87.8} & 81.6 & 80.1 \\
    & Alignment &  \underline{61.3} & \textbf{62.0} & 56.8 & 54.8 \\
    & Cost & 585 & 410 & \textbf{90.8} & \underline{145} \\
    & Latency & 12.4 & \underline{11.7} & 12.8 & \textbf{7.26} \\
    \bottomrule
    \end{tabular}

    \begin{tabular}{l l p{1.8cm} p{1.8cm} p{1.8cm} p{1.8cm}}
    \toprule
    & & \multicolumn{4}{c}{\textbf{MedicalQA Optimized Configurations}} \\ 
    & & \textbf{Safety} & \textbf{Alignment} & \textbf{Cost} & \textbf{Latency} \\ \midrule
    \multirow{7}{*}{Hyperparam}& Embedding model & text-embedding-3-large & text-embedding-3-small & text-embedding-3-large & text-embedding-3-small\\
    & LLM & gpt-4o-mini & gpt-4o-mini & Llama-3.1-8B & Llama-3.1-8B\\
    & Chunk size & 1024 & 1024 & 256 & 1024\\
    & Chunk overlap & 256 & 512 & 32 & 32 \\
    & Num chunks & 6 & 6 & 2 & 2 \\
    & Rerank threshold & 0.00 & 0.22 & 0.57 & 0.36 \\
    & Temperature & 0.00 & 0.10 & 0.57 & 0.00 \\ \midrule
    \multirow{4}{*}{Objective} & Safety & \textbf{91.5} & \underline{89.1} & 60.9 & 77.7 \\
    & Alignment &  \underline{64.8} & \textbf{65.3} & 57.8 & 61.3 \\
    & Cost & 1010 & 997 & \textbf{92.6} & \underline{244} \\
    & Latency & 17.0 & 15.6 & \underline{12.4} & \textbf{9.62} \\
    \bottomrule
    \end{tabular}
    \caption{Input parameters and objective evaluations for individual configurations optimized for each objective. We observe a similar choice of parameters between both datasets, especially the choice of LLM and embedding model.}
    \label{tab:configurations}
\end{table}







\begin{figure*}[t!]
    \centering
    \includegraphics[width=0.8\linewidth]{figures/architectures_1.pdf}
    \caption{Overview of X-IL framework. X-IL supports multi-modal inputs (Language, RGB, and Point Cloud) and two architectures: Decoder-Only and Encoder-Decoder. Inside each architecture, the Backbone serves as the core computational unit, offering support for Transformer, Mamba, and xLSTM. For policy representations, X-IL supports Behavior Cloning (BC), Diffusion-based, and Flow-based Policies, enabling diverse learning paradigms for imitation learning. Notably, each component—input modality, architecture, backbone, and policy—can be easily swapped to efficiently explore various model configurations.
    }
    \label{fig:arch_moil}
\end{figure*}
\newcommand{\none}{\ding{55}}
\begin{table}[t]
    \caption{
        \textbf{Implementation details for \mtd and baseline models.}
    }
    \label{tab:training_details}
    \vskip 0.1in
    \centering
    \begin{adjustbox}{max width=\linewidth}
    \begin{tabular}{l c c c c}
    \toprule
    & Kinetics-600 & RealEstate10K & Minecraft & Imitation Learning \\
    \midrule
    \emph{VAEs} \\
    Input & $\{1, 4\} \times 128 \times 128$ & \multirow{8}{*}{-} & $1 \times$ 256 $\times 256$ & \multirow{8}{*}{-} \\
    \quad Compression ($f_t, f_s$) & \{1, 4\}, 8 & & 1, 8 \\
    \quad Latent channels & 16 & & 4 \\
    Training steps & 600k & & 50k \\
    \quad Optimizer & Adam & & Adam \\
    \quad Batch size & 64 & & 96 \\
    \quad Learning rate & 1e-4 & & 4e-4\\
    \quad EMA & 0.999 & & \none \\
    \midrule
    \emph{VDMs} \\
    Input & $17 \times 128 \times 128$ & $8 \times$ 256 $\times$ 256 & $50 \times 256 \times 256$ & $21\times 32 \times 32$ \\
    \quad Latent & $5 \times 16 \times 16$ & \none & $50 \times 32 \times 32$ & \none \\
    \quad Frame skip & 1 & 10 $\rightarrow$ Max & 2 & 15 \\
    Backbone & DiT & U-ViT & DiT & Attention UNet\\
    \quad Patch size & 1 & 2 & 2 & 1 \\
    \quad Layer types & Transformer & $\left[\text{ResNet}\times2, \text{Transformer}\times2\right]$ & Transformer & Attention, Conv \\
    \quad Layers & 28 & $\left[3, 3, 6, 20\right]$ & 12 & 8 \\
    \quad Hidden size & 1152 & $\left[128, 256, 576, 1152\right]$ & 768 & 128 \\
    \quad Heads & 16 & 9 & 12 & 4 \\
    Training steps & 640k & 500k & 200k & 100k \\
    \quad Warmup steps & 10k & 10k & 10k & 10k \\
    \quad Optimizer & AdamW & AdamW & AdamW & AdamW \\
    \quad Batch size & 192 & 96 & 96 & 64\\
    \quad Learning rate & 2e-4 & 5e-5 & 1e-4 & 5e-4 \\
    \quad Weight decay & 0 & 1e-2 & 1e-3 & 1e-3 \\
    \quad EMA & 0.9999 & 0.9999 & 0.9999 & \none \\
    Diffusion type & Discrete & Continuous & Discrete & Discrete \\
    \quad Noise schedule & Cosine & Shifted Cosine & Shifted Cosine & Cosine \\
    \quad Noise schedule shift & \none & 0.125 & 0.125 & \none \\
    \quad Parameterization & $\rvv$ & $\rvv$ & $\rvv$ & $\rvx_0$ \\
    \quad Sampler & DDIM & DDIM & DDIM & DDIM \\
    \quad Sampling steps & 50 & 50 & 50 & 50 \\
    \bottomrule
    \end{tabular}
    \end{adjustbox}
\end{table}

\begin{table}[]
    \centering
    \caption{Results for the grid-search on SimCLR's hyper-parameters on the Clevr-4 dataset. \textit{LR} stands for base learning rate and \textit{BS} stands for batch size. The best-performing model is \textbf{highlighted}.}
    \label{tab:tuning_simclr}
    \vskip 0.15in
    \begin{tabular}{llll rrrr}
        \toprule
        \multicolumn{4}{c}{hyper-parameters} & \multicolumn{4}{c}{kNN top-1 accuracy} \\ 
        head & output dim. & \textit{LR} & \textit{BS} & shape & texture & color & count \\
        \midrule 
        Identity & (512) & 0.025 & 256 & 42.6 & 40.7 & 83.1 & 24.3 \\
        Identity & (512) & 0.025 & 512 & 40.6 & 40.4 & 83.1 & 24.2 \\
        Identity & (512) & 0.05 & 256 & 40.7 & 39.1 & 83.9 & 24.1 \\
        Identity & (512) & 0.05 & 512 & 47.7 & 39.8 & 80.8 & 24.3 \\
        Identity & (512) & 0.1 & 256 & 47.9 & 40.0 & 80.4 & 24.2 \\
        Identity & (512) & 0.1 & 512 & 10.1 & 9.9 & 9.8 & 9.4 \\
        Identity & (512) & 0.2 & 256 & 39.7 & 34.8 & 76.8 & 24.0 \\
        Identity & (512) & 0.2 & 512 & 40.3 & 32.2 & 75.4 & 23.5 \\
        Identity & (512) & 0.4 & 256 & 40.9 & 31.6 & 72.0 & 23.6 \\
        Identity & (512) & 0.4 & 512 & 42.6 & 29.9 & 73.3 & 22.6 \\
        Identity & (512) & 0.6 & 256 & 36.2 & 28.0 & 70.2 & 23.6 \\
        Identity & (512) & 0.6 & 512 & 40.6 & 31.9 & 73.4 & 22.8 \\
        \textbf{MLP} & \textbf{128} & \textbf{0.025} & \textbf{256} & \textbf{57.1} & \textbf{53.5} & \textbf{92.7} & \textbf{29.9} \\
        MLP & 128 & 0.025 & 512 & 55.6 & 51.0 & 90.9 & 28.2 \\
        MLP & 128 & 0.05 & 256 & 57.7 & 51.7 & 92.0 & 28.9 \\
        MLP & 128 & 0.05 & 512 & 55.4 & 47.3 & 87.8 & 27.6 \\
        MLP & 128 & 0.1 & 256 & 57.6 & 48.4 & 89.1 & 28.0 \\
        MLP & 128 & 0.1 & 512 & 47.7 & 41.2 & 86.4 & 27.3 \\
        MLP & 128 & 0.2 & 256 & 48.2 & 39.2 & 82.8 & 27.2 \\
        MLP & 128 & 0.2 & 512 & 45.9 & 39.3 & 84.2 & 26.5 \\
        MLP & 128 & 0.4 & 256 & 44.7 & 34.9 & 74.7 & 26.3 \\
        MLP & 128 & 0.4 & 512 & 46.0 & 35.6 & 75.1 & 26.7 \\
        MLP & 128 & 0.6 & 256 & 44.0 & 34.4 & 73.8 & 26.6 \\
        MLP & 128 & 0.6 & 512 & 40.8 & 37.3 & 70.4 & 26.4 \\
        MLP & 512 & 0.025 & 256 & 47.3 & 47.7 & 89.1 & 29.4 \\
        MLP & 512 & 0.025 & 512 & 46.7 & 46.2 & 89.4 & 28.4 \\
        MLP & 512 & 0.05 & 256 & 46.6 & 45.7 & 88.0 & 28.3 \\
        MLP & 512 & 0.05 & 512 & 46.2 & 43.7 & 87.1 & 28.2 \\
        MLP & 512 & 0.1 & 256 & 47.0 & 42.6 & 84.9 & 28.2 \\
        MLP & 512 & 0.1 & 512 & 54.4 & 44.2 & 85.7 & 27.3 \\
        MLP & 512 & 0.2 & 256 & 44.8 & 37.9 & 81.1 & 27.1 \\
        MLP & 512 & 0.2 & 512 & 49.4 & 42.6 & 79.6 & 26.3 \\
        MLP & 512 & 0.4 & 256 & 44.9 & 38.7 & 75.7 & 26.4 \\
        MLP & 512 & 0.4 & 512 & 46.6 & 37.0 & 75.9 & 26.3 \\
        MLP & 512 & 0.6 & 256 & 44.4 & 35.0 & 75.1 & 27.4 \\
        MLP & 512 & 0.6 & 512 & 45.2 & 34.1 & 73.3 & 27.0 \\
        \bottomrule
    \end{tabular}
\end{table}


\begin{table}[]
    \centering
    \caption{Results for the grid-search on VICReg's hyper-parameters on the Clevr-4 dataset. \textit{LR} stands for base learning rate and \textit{BS} stands for batch size. The best-performing model is \textbf{highlighted}.}
    \label{tab:tuning_vicreg}
    \vskip 0.15in
    \begin{tabular}{llll rrrr}
        \toprule
        \multicolumn{4}{c}{hyper-parameters} & \multicolumn{4}{c}{kNN top-1 accuracy} \\ 
        head & output dim. & \textit{LR} & \textit{BS} & shape & texture & color & count \\
        \midrule 
        Identity & (512) & 0.005 & 256 & 82.1 & 86.7 & 100.0 & 28.2 \\
        Identity & (512) & 0.005 & 512 & 88.3 & 87.4 & 100.0 & 26.5 \\
        Identity & (512) & 0.01 & 256 & 86.9 & 85.3 & 100.0 & 34.1 \\
        \textbf{Identity} &  \textbf{(512)} & \textbf{0.01} & \textbf{512} & \textbf{91.4} & \textbf{88.6} & \textbf{100.0} & \textbf{27.2} \\
        Identity & (512) & 0.025 & 256 & 84.1 & 79.4 & 99.5 & 29.8 \\
        Identity & (512) & 0.025 & 512 & 86.2 & 84.3 & 99.4 & 31.9 \\
        Identity & (512) & 0.05 & 256 & 73.9 & 71.0 & 98.9 & 26.3 \\
        Identity & (512) & 0.05 & 512 & 81.0 & 77.5 & 98.9 & 28.8 \\
        Identity & (512) & 0.1 & 256 & 72.5 & 66.9 & 98.8 & 24.2 \\
        Identity & (512) & 0.1 & 512 & 60.6 & 65.0 & 98.4 & 23.7 \\
        Identity & (512) & 0.2 & 256 & 63.1 & 57.0 & 98.2 & 22.9 \\
        Identity & (512) & 0.2 & 512 & 51.1 & 54.8 & 97.0 & 20.0 \\
        Identity & (512) & 0.4 & 256 & 44.4 & 48.3 & 97.0 & 22.9 \\
        Identity & (512) & 0.4 & 512 & 50.3 & 51.2 & 97.4 & 23.5 \\
        MLP & 128 & 0.005 & 256 & 51.0 & 57.6 & 99.5 & 30.4 \\
        MLP & 128 & 0.005 & 512 & 51.4 & 61.3 & 99.0 & 29.7 \\
        MLP & 128 & 0.01 & 256 & 44.4 & 54.9 & 98.2 & 29.0 \\
        MLP & 128 & 0.01 & 512 & 47.7 & 57.5 & 98.3 & 27.7 \\
        MLP & 128 & 0.025 & 256 & 34.8 & 51.1 & 97.0 & 24.2 \\
        MLP & 128 & 0.025 & 512 & 42.8 & 50.6 & 97.0 & 25.8 \\
        MLP & 128 & 0.05 & 256 & 36.1 & 44.6 & 95.8 & 25.0 \\
        MLP & 128 & 0.05 & 512 & 36.8 & 49.9 & 97.3 & 24.7 \\
        MLP & 128 & 0.1 & 256 & 31.1 & 42.3 & 95.3 & 24.5 \\
        MLP & 128 & 0.1 & 512 & 34.3 & 44.3 & 96.2 & 25.2 \\
        MLP & 128 & 0.2 & 256 & 30.3 & 40.9 & 95.8 & 23.3 \\
        MLP & 128 & 0.2 & 512 & 31.7 & 42.7 & 96.3 & 23.5 \\
        MLP & 128 & 0.4 & 256 & 28.1 & 25.8 & 94.5 & 23.2 \\
        MLP & 128 & 0.4 & 512 & 26.8 & 14.1 & 33.3 & 21.5 \\
        MLP & 512 & 0.005 & 256 & 63.7 & 68.3 & 99.5 & 30.5 \\
        MLP & 512 & 0.005 & 512 & 63.8 & 66.1 & 99.4 & 28.4 \\
        MLP & 512 & 0.01 & 256 & 59.6 & 63.5 & 98.7 & 28.3 \\
        MLP & 512 & 0.01 & 512 & 61.5 & 64.3 & 98.6 & 26.2 \\
        MLP & 512 & 0.025 & 256 & 61.2 & 60.2 & 97.9 & 26.6 \\
        MLP & 512 & 0.025 & 512 & 62.2 & 62.8 & 98.0 & 24.7 \\
        MLP & 512 & 0.05 & 256 & 59.8 & 58.5 & 97.4 & 25.7 \\
        MLP & 512 & 0.05 & 512 & 58.6 & 59.1 & 97.2 & 23.9 \\
        MLP & 512 & 0.1 & 256 & 57.2 & 57.2 & 97.1 & 24.7 \\
        MLP & 512 & 0.1 & 512 & 57.2 & 56.8 & 97.0 & 23.5 \\
        MLP & 512 & 0.2 & 256 & 56.9 & 52.9 & 96.6 & 24.5 \\
        MLP & 512 & 0.2 & 512 & 55.7 & 54.2 & 96.8 & 23.5 \\
        MLP & 512 & 0.4 & 256 & 46.2 & 49.0 & 96.3 & 23.7 \\
        MLP & 512 & 0.4 & 512 & 44.1 & 51.2 & 96.5 & 23.0 \\
        \bottomrule
    \end{tabular}
\end{table}

\section{Clevr-4 Baselines} \label{app:clevr4_baselines}

This section details the hyper-parameter tuning of the SimCLR and VICReg baselines. 

We implemented the models following the original papers from SimCLR \cite{chen2020SimCLR_ssl} and VICReg \cite{bardes2021vicreg_ssl}. 
We trained ResNet-18 backbones and trained each model with and without a projection head to find which setup works best for each technique when applied to the Clevr-4 dataset. 
For a fair comparison, we followed the same experimental setup as for our SSL method: 
we used the SGD optimizer with a momentum of 0.9, and a weight decay of $2 \cdot 10^{-5}$. The learning rate follows a cosine decay schedule~\citep{loshchilov2016sgdr} with 10 epochs of linear warmup and is scaled with a linear scaling rule~\citep{goyal2017lr_scaling_rule}.

We ran a grid search on the projection head choice, the learning rate, and the batch size. The models were trained for 80 epochs and the best-performing model was then re-trained for 200 epochs. Its results are reported in Table~\ref{tab:results_clevr4_ssl} from Section~\ref{sec:results_infomax}.

Results for the grid search on the hyper-parameters from SimCLR and VICReg are reported respectively in Table~\ref{tab:tuning_simclr} and in Table~\ref{tab:tuning_vicreg}. We observe that the best-performing model for SimCLR has a projection head, while the VICReg technique works better with no projection head. 
This observation for VICReg is consistent with findings from our method applied to Clevr-4. This may be because the taxonomies are statistically independent and the augmentations are minimal, reducing the need for a projection head to prevent true invariance to data augmentations~\citep{bordes2022guillotine}.


\end{document}
