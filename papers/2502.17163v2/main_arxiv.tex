% This must be in the first 5 lines to tell arXiv to use pdfLaTeX, which is strongly recommended.
\pdfoutput=1
% In particular, the hyperref package requires pdfLaTeX in order to break URLs across lines.

\documentclass[11pt]{article}

% Change "review" to "final" to generate the final (sometimes called camera-ready) version.
% Change to "preprint" to generate a non-anonymous version with page numbers.
% Change the following package for Review or Preprint Purpose
% \usepackage[review]{acl}

\usepackage[preprint]{acl}


% Standard package includes
\usepackage{times}
\usepackage{latexsym}

% For proper rendering and hyphenation of words containing Latin characters (including in bib files)
\usepackage[T1]{fontenc}
% For Vietnamese characters
% \usepackage[T5]{fontenc}
% See https://www.latex-project.org/help/documentation/encguide.pdf for other character sets

% This assumes your files are encoded as UTF8
\usepackage[utf8]{inputenc}

% This is not strictly necessary, and may be commented out,
% but it will improve the layout of the manuscript,
% and will typically save some space.
\usepackage{microtype}

% This is also not strictly necessary, and may be commented out.
% However, it will improve the aesthetics of text in
% the typewriter font.
\usepackage{inconsolata}

%Including images in your LaTeX document requires adding
%additional package(s)
\usepackage{graphicx}
\usepackage{tikz}
\usetikzlibrary{shapes.geometric, positioning, fit}

% for checkmark, crossmark
\usepackage{pifont}
\usepackage{stackengine}
% If the title and author information does not fit in the area allocated, uncomment the following
%
%\setlength\titlebox{<dim>}
%
% and set <dim> to something 5cm or larger.


% Paper packages
\usepackage{multirow}
\usepackage{comment}
\usepackage{cuted,tcolorbox,lipsum} % for prompt rendering
\usepackage{xspace}
\usepackage{booktabs}
\newcommand{\dataname}{MEMERAG\xspace}
\newcommand{\gptfomini}{GPT-4o mini\xspace}
\newcommand{\qwenthrirtytwob}{Qwen 2.5 32B\xspace}
\usepackage{float}
\usepackage{placeins}
\usepackage{afterpage}
\usepackage{enumitem}
\usepackage{booktabs,siunitx}
\usepackage{colortbl}  % Add this to your preamble
\usepackage{xcolor}    % For custom colors
\usepackage{tabularx} % using tabluar x

\usepackage{hyperref}
\usepackage{soul}
\sisetup{
  table-auto-round = true, % Round numbers in S-columns
  % detect-weight=true,
  detect-all = true,
  % detect-inline-weight=math
}

\newcommand{\roundo}[1]{\num[round-mode=places,round-precision=1]{#1}}

\title{\dataname: A Multilingual End-to-End Meta-Evaluation Benchmark for Retrieval Augmented Generation}
% \title{Benchmarking Factuality of LLMs on Multilingual Long-Form Question Answering}

% Author information can be set in various styles:
% For several authors from the same institution:
% \author{Author 1 \and ... \and Author n \\
%         Address line \\ ... \\ Address line}
% if the names do not fit well on one line use
%         Author 1 \\ {\bf Author 2} \\ ... \\ {\bf Author n} \\
% For authors from different institutions:
% \author{Author 1 \\ Address line \\  ... \\ Address line
%         \And  ... \And
%         Author n \\ Address line \\ ... \\ Address line}
% To start a separate ``row'' of authors use \AND, as in
% \author{Author 1 \\ Address line \\  ... \\ Address line
%         \AND
%         Author 2 \\ Address line \\ ... \\ Address line \And
%         Author 3 \\ Address line \\ ... \\ Address line}

\author{
    \textbf{Mar\'ia Andrea Cruz Bland\'on\textsuperscript{1}}\thanks{Work done as an intern at Amazon},
    \textbf{Jayasimha Talur\textsuperscript{2}}\thanks{\textbf{Correspondence:} \href{mailto:talurj@amazon.com}{talurj@amazon.com}},
    \textbf{Bruno Charron\textsuperscript{2}}, 
\\
    \textbf{Dong Liu\textsuperscript{2}},
    \textbf{Saab Mansour\textsuperscript{2}},
    \textbf{Marcello Federico\textsuperscript{2}}
\\
    \textsuperscript{1}Tampere University,
    \textsuperscript{2}Amazon
}


%\author{
%  \textbf{First Author\textsuperscript{1}},
%  \textbf{Second Author\textsuperscript{1,2}},
%  \textbf{Third T. Author\textsuperscript{1}},
%  \textbf{Fourth Author\textsuperscript{1}},
%\\
%  \textbf{Fifth Author\textsuperscript{1,2}},
%  \textbf{Sixth Author\textsuperscript{1}},
%  \textbf{Seventh Author\textsuperscript{1}},
%  \textbf{Eighth Author \textsuperscript{1,2,3,4}},
%\\
%  \textbf{Ninth Author\textsuperscript{1}},
%  \textbf{Tenth Author\textsuperscript{1}},
%  \textbf{Eleventh E. Author\textsuperscript{1,2,3,4,5}},
%  \textbf{Twelfth Author\textsuperscript{1}},
%\\
%  \textbf{Thirteenth Author\textsuperscript{3}},
%  \textbf{Fourteenth F. Author\textsuperscript{2,4}},
%  \textbf{Fifteenth Author\textsuperscript{1}},
%  \textbf{Sixteenth Author\textsuperscript{1}},
%\\
%  \textbf{Seventeenth S. Author\textsuperscript{4,5}},
%  \textbf{Eighteenth Author\textsuperscript{3,4}},
%  \textbf{Nineteenth N. Author\textsuperscript{2,5}},
%  \textbf{Twentieth Author\textsuperscript{1}}
%\\
%\\
%  \textsuperscript{1}Affiliation 1,
%  \textsuperscript{2}Affiliation 2,
%  \textsuperscript{3}Affiliation 3,
%  \textsuperscript{4}Affiliation 4,
%  \textsuperscript{5}Affiliation 5
%\\
%  \small{
%    \textbf{Correspondence:} \href{mailto:email@domain}{email@domain}
%  }
%}


\graphicspath{ {./plots/} }
% changes
\usepackage{xcolor}
\newcommand{\change}
{\textcolor{red}}


\def\showcomments{}  % Comment this line out to hide comments
\ifdefined\showcomments
    \newcommand{\sm}[1]{\textcolor{cyan}{$_{Saab}${[#1]}}}
    \newcommand{\bc}[1]{\textcolor{magenta}{$_{Bruno}${[#1]}}}
    \newcommand{\dl}[1]{\textcolor{orange}{$_{Dong}${[#1]}}}
    \newcommand{\jt}[1]{\textcolor{red}{$_{Jayasimha}${[#1]}}}
    \newcommand{\mf}[1]{\textcolor{red}{$_{MF}${[#1]}}}
    
\else
    \newcommand{\sm}[1]{}
    \newcommand{\bc}[1]{}
    \newcommand{\dl}[1]{}
    \newcommand{\jt}[1]{}
    \newcommand{\mf}[1]{}
\fi

% Command to disable sections 
\newif\ifsectionenabled
\sectionenabledfalse

% Commands to highlight best and second best results
% Option 1
\newcommand{\best}[1]{{\textbf{#1}}}
\newcommand{\comparable}[1]{#1{$^{\dagger}$}}
% Option 2
% \usepackage{colortbl}
% \newcommand{\best}[1]{\textbf{#1}}
% \newcommand{\comparable}[1]{\cellcolor{gray!20}#1}
% Option 3
% \usepackage{ulem}
% \newcommand{\best}[1]{\textbf{#1}}
% \newcommand{\comparable}[1]{\uline{#1}}

\usepackage{setspace}
%\doublespacing

\begin{document}
\maketitle
\begin{abstract}
Automatic evaluation of retrieval augmented generation (RAG) systems relies on fine-grained dimensions like faithfulness and relevance, as judged by expert human annotators. Meta-evaluation benchmarks support the development of automatic evaluators that correlate well with human judgement. However, existing benchmarks predominantly focus on English or use translated data, which fails to capture cultural nuances. A native approach provides a better representation of the end user experience.

In this work, we develop a Multilingual End-to-end Meta-Evaluation RAG benchmark (\dataname).
%\dl{say "Meta Evaluation Multilingual End-to-End RAG" by capitalizing the letter?} 
Our benchmark builds on the popular MIRACL dataset, using native-language questions and generating responses with diverse large language models (LLMs), which are then assessed by expert annotators for faithfulness and relevance. We describe our annotation process and show that it achieves high inter-annotator agreement. We then analyse the performance of the answer-generating LLMs across languages as per the human evaluators. Finally we apply the dataset to our main use-case which is to benchmark multilingual automatic evaluators (LLM-as-a-judge). We show that our benchmark can reliably identify improvements offered by advanced prompting techniques and LLMs~\footnote{We release~\url{https://anonymous} our benchmark to support the community developing accurate evaluation methods for multilingual RAG systems.}. 
%We report generation performance of diverse LLMs across the languages, but, most importantly, we use \dataname to develop automatic evaluators using the LLM-as-a-judge framework and compare different prompting techniques. 
%Experimental results using \dataname expose performance gaps across languages, for both using LLMs-as-a-judge for automatic evaluation as well as LLMs for response generation in a RAG setup. Finally, we compare our native language benchmark with a translated benchmark, highlighting key limitations in translation based evaluation. 
%We release\footnote{\url{https://anonymous}} our benchmark to support the community developing accurate evaluation methods for multilingual RAG systems.
\end{abstract}

\section{Introduction}
%\section{Introduction}
\label{sec:introduction}
The business processes of organizations are experiencing ever-increasing complexity due to the large amount of data, high number of users, and high-tech devices involved \cite{martin2021pmopportunitieschallenges, beerepoot2023biggestbpmproblems}. This complexity may cause business processes to deviate from normal control flow due to unforeseen and disruptive anomalies \cite{adams2023proceddsriftdetection}. These control-flow anomalies manifest as unknown, skipped, and wrongly-ordered activities in the traces of event logs monitored from the execution of business processes \cite{ko2023adsystematicreview}. For the sake of clarity, let us consider an illustrative example of such anomalies. Figure \ref{FP_ANOMALIES} shows a so-called event log footprint, which captures the control flow relations of four activities of a hypothetical event log. In particular, this footprint captures the control-flow relations between activities \texttt{a}, \texttt{b}, \texttt{c} and \texttt{d}. These are the causal ($\rightarrow$) relation, concurrent ($\parallel$) relation, and other ($\#$) relations such as exclusivity or non-local dependency \cite{aalst2022pmhandbook}. In addition, on the right are six traces, of which five exhibit skipped, wrongly-ordered and unknown control-flow anomalies. For example, $\langle$\texttt{a b d}$\rangle$ has a skipped activity, which is \texttt{c}. Because of this skipped activity, the control-flow relation \texttt{b}$\,\#\,$\texttt{d} is violated, since \texttt{d} directly follows \texttt{b} in the anomalous trace.
\begin{figure}[!t]
\centering
\includegraphics[width=0.9\columnwidth]{images/FP_ANOMALIES.png}
\caption{An example event log footprint with six traces, of which five exhibit control-flow anomalies.}
\label{FP_ANOMALIES}
\end{figure}

\subsection{Control-flow anomaly detection}
Control-flow anomaly detection techniques aim to characterize the normal control flow from event logs and verify whether these deviations occur in new event logs \cite{ko2023adsystematicreview}. To develop control-flow anomaly detection techniques, \revision{process mining} has seen widespread adoption owing to process discovery and \revision{conformance checking}. On the one hand, process discovery is a set of algorithms that encode control-flow relations as a set of model elements and constraints according to a given modeling formalism \cite{aalst2022pmhandbook}; hereafter, we refer to the Petri net, a widespread modeling formalism. On the other hand, \revision{conformance checking} is an explainable set of algorithms that allows linking any deviations with the reference Petri net and providing the fitness measure, namely a measure of how much the Petri net fits the new event log \cite{aalst2022pmhandbook}. Many control-flow anomaly detection techniques based on \revision{conformance checking} (hereafter, \revision{conformance checking}-based techniques) use the fitness measure to determine whether an event log is anomalous \cite{bezerra2009pmad, bezerra2013adlogspais, myers2018icsadpm, pecchia2020applicationfailuresanalysispm}. 

The scientific literature also includes many \revision{conformance checking}-independent techniques for control-flow anomaly detection that combine specific types of trace encodings with machine/deep learning \cite{ko2023adsystematicreview, tavares2023pmtraceencoding}. Whereas these techniques are very effective, their explainability is challenging due to both the type of trace encoding employed and the machine/deep learning model used \cite{rawal2022trustworthyaiadvances,li2023explainablead}. Hence, in the following, we focus on the shortcomings of \revision{conformance checking}-based techniques to investigate whether it is possible to support the development of competitive control-flow anomaly detection techniques while maintaining the explainable nature of \revision{conformance checking}.
\begin{figure}[!t]
\centering
\includegraphics[width=\columnwidth]{images/HIGH_LEVEL_VIEW.png}
\caption{A high-level view of the proposed framework for combining \revision{process mining}-based feature extraction with dimensionality reduction for control-flow anomaly detection.}
\label{HIGH_LEVEL_VIEW}
\end{figure}

\subsection{Shortcomings of \revision{conformance checking}-based techniques}
Unfortunately, the detection effectiveness of \revision{conformance checking}-based techniques is affected by noisy data and low-quality Petri nets, which may be due to human errors in the modeling process or representational bias of process discovery algorithms \cite{bezerra2013adlogspais, pecchia2020applicationfailuresanalysispm, aalst2016pm}. Specifically, on the one hand, noisy data may introduce infrequent and deceptive control-flow relations that may result in inconsistent fitness measures, whereas, on the other hand, checking event logs against a low-quality Petri net could lead to an unreliable distribution of fitness measures. Nonetheless, such Petri nets can still be used as references to obtain insightful information for \revision{process mining}-based feature extraction, supporting the development of competitive and explainable \revision{conformance checking}-based techniques for control-flow anomaly detection despite the problems above. For example, a few works outline that token-based \revision{conformance checking} can be used for \revision{process mining}-based feature extraction to build tabular data and develop effective \revision{conformance checking}-based techniques for control-flow anomaly detection \cite{singh2022lapmsh, debenedictis2023dtadiiot}. However, to the best of our knowledge, the scientific literature lacks a structured proposal for \revision{process mining}-based feature extraction using the state-of-the-art \revision{conformance checking} variant, namely alignment-based \revision{conformance checking}.

\subsection{Contributions}
We propose a novel \revision{process mining}-based feature extraction approach with alignment-based \revision{conformance checking}. This variant aligns the deviating control flow with a reference Petri net; the resulting alignment can be inspected to extract additional statistics such as the number of times a given activity caused mismatches \cite{aalst2022pmhandbook}. We integrate this approach into a flexible and explainable framework for developing techniques for control-flow anomaly detection. The framework combines \revision{process mining}-based feature extraction and dimensionality reduction to handle high-dimensional feature sets, achieve detection effectiveness, and support explainability. Notably, in addition to our proposed \revision{process mining}-based feature extraction approach, the framework allows employing other approaches, enabling a fair comparison of multiple \revision{conformance checking}-based and \revision{conformance checking}-independent techniques for control-flow anomaly detection. Figure \ref{HIGH_LEVEL_VIEW} shows a high-level view of the framework. Business processes are monitored, and event logs obtained from the database of information systems. Subsequently, \revision{process mining}-based feature extraction is applied to these event logs and tabular data input to dimensionality reduction to identify control-flow anomalies. We apply several \revision{conformance checking}-based and \revision{conformance checking}-independent framework techniques to publicly available datasets, simulated data of a case study from railways, and real-world data of a case study from healthcare. We show that the framework techniques implementing our approach outperform the baseline \revision{conformance checking}-based techniques while maintaining the explainable nature of \revision{conformance checking}.

In summary, the contributions of this paper are as follows.
\begin{itemize}
    \item{
        A novel \revision{process mining}-based feature extraction approach to support the development of competitive and explainable \revision{conformance checking}-based techniques for control-flow anomaly detection.
    }
    \item{
        A flexible and explainable framework for developing techniques for control-flow anomaly detection using \revision{process mining}-based feature extraction and dimensionality reduction.
    }
    \item{
        Application to synthetic and real-world datasets of several \revision{conformance checking}-based and \revision{conformance checking}-independent framework techniques, evaluating their detection effectiveness and explainability.
    }
\end{itemize}

The rest of the paper is organized as follows.
\begin{itemize}
    \item Section \ref{sec:related_work} reviews the existing techniques for control-flow anomaly detection, categorizing them into \revision{conformance checking}-based and \revision{conformance checking}-independent techniques.
    \item Section \ref{sec:abccfe} provides the preliminaries of \revision{process mining} to establish the notation used throughout the paper, and delves into the details of the proposed \revision{process mining}-based feature extraction approach with alignment-based \revision{conformance checking}.
    \item Section \ref{sec:framework} describes the framework for developing \revision{conformance checking}-based and \revision{conformance checking}-independent techniques for control-flow anomaly detection that combine \revision{process mining}-based feature extraction and dimensionality reduction.
    \item Section \ref{sec:evaluation} presents the experiments conducted with multiple framework and baseline techniques using data from publicly available datasets and case studies.
    \item Section \ref{sec:conclusions} draws the conclusions and presents future work.
\end{itemize}
Retrieval augmented generation (RAG) is emerging as a popular application of large language models (LLMs) and a powerful paradigm to improve LLMs factuality \cite{gao_retrieval-augmented_2024}. A RAG pipeline first retrieves relevant documents from an index based on a query and then composes a response using an LLM. Grounding the LLM response on retrieved knowledge helps mitigate outdated knowledge, lack of domain expertise and reduce hallucinations \cite{lewis2020retrieval, gao_retrieval-augmented_2024}.
Collecting benchmarking data for RAG is challenging due to the complexity of the pipeline that includes {\em information retrieval} and {\em text generation}. Text generation, our focus in this paper, has, in general,  two modes of {\em automatic evaluation}: reference-based and reference-free, which differ in the availability of human-generated gold references for each model input. Both modes can either leverage single (e.g. BERTScore \cite{Zhang2020BERTScore:}) or multidimensional (e.g. autoMQM  \cite{fernandes_devil_2023}) scores. Multidimensional evaluation 
\cite{burchardt-2013-mqm} provides more comprehensive understanding of text generation systems and is the de facto % \dl{typo here?}
standard in the machine translation (MT) community\footnote{Since 2021 in the WMT metrics shared task \url{https://www2.statmt.org/wmt24/metrics-task.html}}. %On the other hand, latest insights from the MT community show that reference-free methods are competitive with reference-based ones bringing into question the usefulness of composing human references \cite{rei-etal-2021-references, freitag-etal-2024-wmt24refnoref}. % Here, we focus on reference-free evaluation for RAG, following recent trends in the summarization and RAG communities\sm{cite, rephrase or remove}. 

In a reference-free evaluation setup,  gold multidimensional judgements (factuality, relevance, etc) of model generations can be leveraged in a {\em meta-evaluation} framework. In this framework, automated evaluators are evaluated against the human judgement to measure correlation. The automated evaluators can then be applied to measure the performance of new models' outputs. 

Previous work for RAG meta-evaluation mainly focused on English \cite{fan-2024-rag-survey} or leveraged human or machine translation of English datasets \cite{sharma2024fauxpolyglotstudyinformation}. Multilingual meta-evaluation is important to reliably measure performance across languages which can vary depending on language characteristics (low vs. high resource, complex morphology, etc) and scripts (Latin vs. non-Latin). Translation-based benchmarks, while permitting cross-language comparisons,  suffer from translationese phenomena such as introducing simpler syntax and  lexical choices \cite{Baker1993CorpusLA, graham-etal-2020-statistical}, thus leading to data distributionally different from native data and not necessarily reflecting native users preferences \cite{chen-etal-2024-good-data}. Our position is that translation-based (parallel) benchmarks should be complemented by native multilingual benchmarks. 

To bridge those gaps, we propose a native meta-evaluation multilingual benchmark for RAG systems. Our benchmark is built on top of the popular MIRACL \cite{zhang_miracl_2023} dataset\footnote{\url{https://huggingface.co/datasets/miracl/miracl}} that includes native questions across 18 languages and relevance judgements of retrieved passages for multilingual retrieval evaluation. We extend MIRACL by generating answers in five languages with a diverse set of LLMs, and collecting judgements on the faithfulness and relevance of the answers using native expert human annotators. For the latter, we devised a structured annotation process that achieved a high rate of inter-annotator agreement. To evaluate the benchmark and set reference baseline results for others to compare against, we run LLM-as-a-judge experiments with  various prompting techniques and state-of-the-art LLMs. 

% recent trend in multilingual evaluation is using translation of English benchmarks, whether MT or human. Frontier models such as GPT-4~\cite{openai2024gpt4technicalreport}, Claude 3~\cite{claude32024} and LLama 3~\cite{dubey2024llama3herdmodels} evaluate multilingual performance on machine translated versions of the MMLU dataset~\cite{hendrycks2021mmlu}. On the other hand, \cite{chen-etal-2024-good-data} reveal a notable difference between native and translated data for benchmarking LLM capabilities in terms of model ranking. Following \cite{chen-etal-2024-good-data}, we evaluate the usefulness of machine translated data in the meta-evaluation setup for RAG applications.

To summarize, our main contributions are:
\begin{itemize}[noitemsep,topsep=0pt]
%\setlength\itemsep{0.05em}
    \item We built and publicly release the first (to the best of our knowledge) native multilingual  meta-evaluation RAG benchmark.
    \item We developed a rigorous flow chart-based annotation process to achieve high inter-annotator agreement rate for both faithfulness and relevance judgements. 
    \item We evaluated the quality of the benchmark on three multi-lingual meta-evaluation aspects: prompt selection, model selection, and fine-grained analysis.
    \item We establish reference baselines of multilingual automatic evaluators on our benchmark, showcasing performance improvements when using advanced prompting and LLMs.
\end{itemize}

\begin{comment}
\begin{itemize}
    \item Multilinguality a requirement for RAG systems
    \item RAG investigation new in literature, only EN 
    \item lack of multilingual dataset in RAG format 
    \item New dataset introduced in this paper
\end{itemize}
\end{comment}

% TODO (talurj) Create a flow diagram to represent dataset pre-processing steps
\begin{figure*}[!t]
  \centering
  \includegraphics[width=1\textwidth]{plots/figures/sample_annotation_v9.pdf} \caption{Examples from our Multilingual End-to-end Meta-Evaluation for RAG (MEMERAG) dataset. We select MIRACL native multilingual questions (for a subset of 5 languages), generate responses using diverse LLMs, and annotate each generated sentence with human experts for faithfulness and relevance. The annotation includes both coarse-grained (\ding{51}, \ding{53}) and fine-grained labels. Our dataset forms a meta-evaluation benchmark where automated evaluators can be developed and assessed on their correlation to human judgement. Due to space constraints we omit the retrieved documents (context) for the question and show only one sentence per LLM response. %The check/cross marks represent coarse-grained labels and the associated fine-grained labels are shown to their right. %\sm{Relevance is a bit complex and I would expect yes/no, to simplify I would suggest faithful v/x symbols, fine-grained, relevant v/x, fine-grained, so 4 columns for human judgement} %\sm{change to pdf}\dl{in pdf format now}
  }
  \label{fig:derive_dataset}
\end{figure*}
\section{Related Work}
Due to the pipeline approach of RAG systems, the evaluation can be split into 3 main components: 1) retriever metrics to identify relevant chunks of information to the input typically measured with recall/precision@K \cite{manning2008ir}; 2) generator metrics to identify the “usefulness” of the generated answers in relation to the input, this is done across fine grained dimensions such as faithfulness and relevance, either leveraging references answers \cite{es-etal-2024-ragas}; 3) end-to-end (overall) metrics that take into account additional components such as preprocessing, chunking, query reformulation and cascading errors. Retrieval metrics have been extensively studied in the information retrieval community, hence recent work focused on the text generation performance of RAG systems. This is also the focus of our work. One important difference between generation with and without retrieved documents is the conflict between the parametric “world knowledge” and the non-parametric retrieved documents knowledge, hence the distinction between faithfulness against the retrieved documents (RAG specific) and factuality according to general knowledge~\cite{maynez-etal-2020-faithfulness, wu2024clasheval}.

%\jt{I have added additional related work as suggested by MF}

Recent studies have investigated the importance of various components within multilingual RAG systems. \cite{chirkova2024s} utilized existing multilingual QA datasets to evaluate different combinations of retrievers and generator models, finding that task-specific prompt engineering is crucial for high-quality multilingual generation. In another study, \cite{mirage_bench} extend MIRACL dataset to develop ``MIRAGE-BENCH'' a synthetic arena-based benchmark for ranking multilingual LLM generators. They employed preference judgments from a GPT-4o ``judge'' to train a ranking model. However, an important limitation of such synthetic benchmarks is the potential for self-preference bias \cite{self_preference_bias}, where the LLM judge may favor its own generations.

In this work, we focus on the faithfulness and relevance aspects to ensure meaningful results. These dimensions are typically assessed by human evaluators based on model-generated outputs, highlighting the need for developing automatic evaluation metrics that correlate well with human judgment—a process known as meta-evaluation. This need has led to a recent trend in the English-language research community of publishing meta-evaluation datasets and developing automated evaluators \cite{es-etal-2024-ragas, saad-falcon_ares_2024}.

%For multilingual RAG evaluation, the focus has been on evaluating the retrieval %performance. E.g. MIRACL \cite{zhang_miracl_2023} is a retrieval focused %multilingual dataset covering 18 languages and based on Wikipedia articles. %Questions about the articles are composed by native speakers and relevance of the %etrieved documents are judged. NoMIRACL~\cite{thakur-etal-2024-nomiracl} extends the MIRACL dataset by incorporating a subset of questions for which none of the retrieved passages are relevant, specifically to benchmark LLMs' ability to recognize when they lack sufficient information to provide an answer—a capability referred to as "knowing when you don't know".

%There are works evaluating multilingual generation capabilities but are either span-selection focused or lack a retrieval component \cite{clark-etal-2020-tydi, lewis_mlqa_2020}. 

%\mf{SOMETHING ABOUT ENGLISH AND TRANSLATION-BASED META-EVAL OF RAG TEXT GENERATION}

To the best of our knowledge, no multilingual meta-evaluation benchmark for RAG systems currently exists. In this work, we address this gap by developing such a benchmark. Our dataset facilitates the creation of multilingual automatic evaluators that correlate well with human judgments. This, in turn, enables comprehensive end-to-end benchmarking of RAG systems. We believe our dataset is the first to offer this capability in a multilingual context.


\section{Dataset Construction}
\label{sec:dataset_construction}
Meta-evaluation datasets enable the development of reliable automatic evaluators. In a RAG setup, the input to the evaluator is composed of a question $q$ (user input), a context $c$ (set of passages automatically retrieved to the question) and an answer $a$ generated by a language model to answer the question based on the context. The end-to-end evaluator then needs to judge the quality of the answer $a$ given the context and the question $(c,q)$. 
Following previous work \cite{saad-falcon_ares_2024, es-etal-2024-ragas} we focus on two quality dimensions: %\dl{What do we mean here the "fine-grained dimensions"?}
\begin{center}
\noindent\fbox{\parbox[t]{0.96\linewidth}{%
\setlength{\parskip}{0pt}%
\noindent\textbf{Faithfulness } Is the answer grounded on the context, regardless of your world knowledge?\\[0.5em]
\noindent\textbf{Relevance } Is the answer relevant to the question, regardless of the context?
}}
\end{center}

We build a multilingual end-to-end meta-evaluation RAG (\dataname) dataset by extending the MIRACL dataset~\citep{zhang_miracl_2023} to include \textit{model-generated answers and human-based quality judgements}. %which enables the former in a natively multilingual way across 18 languages.
More precisely, we select relevant question-context pairs, generate answers using various language models and gather expert human annotations on the quality of those answers.
Our dataset encompasses 5 languages: English (EN), German (DE), Spanish (ES), French (FR), and Hindi (HI), which represent multiple language families and both high- and low-resource languages.
Figure~\ref{fig:derive_dataset} shows examples from the dataset, with LLM-generated answers and coarse- to fine-grained human-assigned labels for the faithfulness and relevance dimensions. %while Table~\ref{tab:dataset_stats} includes data statistics.
% \setlength{\parskip}{12pt}

The MIRACL dataset is composed of questions written by humans in their \textit{native} languages, one or more passages automatically retrieved from the Wikipedia, and human annotations about the relevance of each passage.
Building a dataset starting from native questions in each language allows to evaluate RAG pipelines without resorting to (machine) translations, thus avoiding limitations and biases associated with translation. % and to avoid its caveats.
Note, however, that as questions were elicited from native speakers independently across different languages, the resulting data set is not parallel. 


\subsection{Question Selection}
The questions in the MIRACL dataset were generated by humans based on prompts. 
This leads to questions that may be answerable by the prompt but may have ambiguity outside of that context.
In particular, we identified as problematic the questions for which the right answer can change over time.
For example ``\textit{Who is the president of Spain?}''
%\sm{why is this problematic as we care about faithfulness and not factuality? This could be confusing so a better explanation is needed}\bc{Updated the explanation}
or ``\textit{How old is Drake Hogestyn?}''.
%The former question requires up-to-date knowledge while the latter question may not require up-to-date knowledge (the birth date is fixed) but its answer depends on the current date.
In a RAG setting, different passages may have been written at different times and provide conflicting context.
% The generated answers may also be affected by the cutoff date of the generating model.
Additionally, the time of reference is usually not explicit in the question.
To remove those complications we automatically filtered out time-dependent questions across all languages\footnote{
See Appendix~\ref{sec:prompts_dataset_construction} for all prompts used during dataset construction. 
}.
We combined the train and dev splits of the MIRACL dataset, which corresponds to a total of 3,662, 305, 2,810, 1,486 and 1,519 questions respectively for EN, DE, ES, FR and HI.
Of those, 244, 13, 120, 64 and 86 were identified as time-dependent and filtered out.

\subsection{Context Selection}

The MIRACL dataset has an average of 10.3 passages per question, which corresponds to an average of 1,218 words of context in the English train and dev splits, with similar numbers in other languages.
To reduce the cognitive load on human annotators, we limit the context per query to 5 passages.


The source dataset provides human-annotated binary relevance labels for passages. To more accurately simulate an automated retrieval process, we rank the passages for each question using BM25~\citep{schutze2008introduction}, as implemented in \cite{bm25s}. We then select the top-5 ranked passages for each question.
If these top-5 passages do not contain any human-annotated relevant passages, we replace the lowest-ranked passage with the highest-ranked relevant passage from the full set. This approach ensures that each question has at least one relevant passage in its context, avoiding scenarios where annotators would evaluate responses without any relevant information.

It is worth noting that simulating scenarios where no relevant passages exist is straightforward (e.g., by including only irrelevant passages). In such cases, for faithfulness evaluation, we would expect responses like "The provided documents do not contain a relevant answer." Our method focuses on faithfulness while efficiently utilizing human annotation efforts by ensuring that each evaluated case has at least some relevant context.

\subsection{Answer Generation}
\label{sec:answer_gen}

After question and passage selection, we generate an answer for each question-context pair and each of five state-of-the-art LLMs.\footnote{We generated answers with Claude 3 Sonnet, Llama3 70B, Llama3 8B, Mistral 7B, and \gptfomini.}
Those LLMs were selected to cover a range of model sizes, open weight and proprietary models.
We prompted all the models in English\footnote{There is evidence in the literature for better model accuracy when the models are prompted to "think" in English~\cite{lai-etal-2023-chatgpt, liu2024translationneedstudysolving}, though under particular scenarios other languages could perform better~\cite{behzad-etal-2024-ask}. We leave additional prompting experiments for future work.}, asking to answer the question based only on the given context, and requesting the answer to be provided in the same language as the context and question.
For all models, we set the temperature to $0.1$, and maximum number of output tokens to 1000.

We thus produced answers for more than 1000 questions per language, except for German for which MIRACL only contains 305 questions. As our focus is on long-form answers, we further filtered out questions for which any of the 5 models generated an answer shorter than 10 words.
% This removes trivial questions, without bias from any particular model

\subsection{Annotation Guidelines}

% \sm{This section can be significantly reduced to along the lines: "We improved our annotation by running piltos, enforcing meaningful better labels and a flowchart (details, this one is interesting and innovative), ending up the following set of labels ... and improving AC1 from-to for faithfulness and from-to for relevance. Further details in appendix X." This is because some of the details are not very interesting, and also if we need more space}
The task of annotating answers with faithfulness is challenging  due to several factors.
First, it involves some subjectivity which might impact Inter-Annotator Agreement (IAA) \citep{kryscinski_evaluating_2020, tang_tofueval_2024}.
Then,  its label space is not precisely defined in the literature \citep{tang_tofueval_2024,laban_summedits_2023, malaviya_expertqa_2024}. 
Finally, although faithfulness should be ideally evaluated for atomic facts, it is generally evaluated at the sentence or even document level, due to annotation costs.

Starting with the factuality error taxonomy introduced in~\citet{tang_tofueval_2024}, we ran a number of annotation pilots to refine the label space and guidelines.

Finally, we converged to three coarse-grained labels (\textit{Supported}, \textit{Not supported}, \textit{Challenging to determine}), explained through 10 fine-grained labels.
To increase the consistency of the annotation (IAA), we guide the annotation process through a flow chart (documented in Figure~\ref{fig:annotation_guide} in Appendix~\ref{appendix:annotation_guideline}).
For relevance, which is significantly less ambiguous to evaluate, we device a simple annotation process with three labels: \textit{Directly answers the question}, \textit{Adds context to the answer}, and \textit{Unrelated to the question}. Note that the first two labels can be used to describe "relevant" sentences, while the last label identifies "irrelevant" sentences. 
(See Appendix~\ref{appendix:annotation_guideline} for more details.)


\begin{table}%[h]
    \centering
    \normalsize{
    \begin{tabular}{lcccc}
    \hline
    \multirow{2}{*}{\textbf{Lang}} & \multirow{2}{*}{\textbf{\#Q}} & \multicolumn{2}{c}{\textbf{Answer}} & \textbf{Context} \\
    &  & \textbf{\#S} & \textbf{Avg. \#W}  & \textbf{Avg. \#W} \\ 
    \hline
    EN & 250 & 400 & 30.3 & 613.5 \\
    DE & 250 & 468 & 27.3 & 455.0 \\
    ES & 250 & 563 & 52.1 & 522.3 \\
    FR & 250 & 540 & 48.7 & 478.3 \\
    HI & 250 & 351 & 23.8 & 571.5 \\\hline
    Total & 1,250 & 2,322 & & \\\hline
  \end{tabular}
  }
    \caption{General statistics of the \dataname dataset. \#Q: number of questions, \#S: number of sentences, \#W: number of words.
    Each answer is annotated at the sentence level, leading to 2,322 total sentences annotated by experts for faithfulness and relevance.
    }
    \label{tab:dataset_stats}
\end{table}

\subsection{Annotation Process}
\label{sec:annotations_process}

From the question-context-answer triplets obtained in Section~\ref{sec:answer_gen}, we randomly sampled 250 questions per language (50 per answer-generating model, without overlapping questions for diversity).
We employed a professional vendor with native annotators\footnote{Annotators were  compensated with a competitive hourly rate that is benchmarked against similar
roles in their country of residence.} to gather annotations for each sentence \footnote{Sentences were segmented using the pySBD\cite{sadvilkar-neumann-2020-pysbd} package.} of the 250 answers per language.
Among the 250 answers per language, a random subset of 10 were assigned to 3 annotators for computing the IAA and the rest to a single annotator. The statistics of the annotated dataset are presented in Table~\ref{tab:dataset_stats}.
% \sm{did Andrea have some analysis that those numbers are sufficient? we can include here or have some argumentation why this is sufficient?}


% \bc{Do we move this paragraph to appendix A?}
The annotations were gathered via a web-based tool that implemented the flow chart of the annotation guidelines (see Appendix~\ref{appendix:annotation_guideline} for details).
% For relevance, the annotators directly selected among the 3 labels.
%To further improve IAA, we relied on the findings of \citet{krishna-etal-2023-longeval} %in which highlighting relevant information was found to help the annotators to perform %the task and to agree. For this purpose, we used Llama 3 70B to identify the sentences %in the retrieved passages that could be supportive information for the answer sentences.
%See Appendix~\ref{sec:passage_highlight} for the prompts used.
To further enhance IAA, we drew upon the findings of \citet{krishna-etal-2023-longeval}, which demonstrated that highlighting relevant information aids annotators in performing tasks and reaching consensus. Thus, we utilized the Llama 3 70B LLM to identify sentences within the retrieved passages that could potentially serve as supporting information to the answer sentences.
%(see prompt in Appendix~\ref{sec:passage_highlight})%\sm{We have too many see appendix for details, can we bring some of this info back into main?}

The English annotations required approximately 25 hours of total annotation time, averaging 5.5 minutes per question. This covered 250 questions, including 10 that were annotated by three different annotators for quality control. Similar time investments were observed for the other four languages.

Table~\ref{tab:iaa} summarizes the IAA per language for faithfulness and relevance labels assigned by 3 annotators. 
We report IAA using Gwet's AC1~\cite{gwet2008computing} %which is robust to label imbalance
and Fleiss Kappa \cite{fleiss_kappa}.
We observe high agreement for faithfulness (0.84-0.93 Gwet's AC1 and 0.70-0.88 Fleiss Kappa) and even higher agreement for relevance (0.95-1.0 Gwet's AC1 and 0.63-1.0 Fleiss Kappa).
%\sm{relevance is a bit confusing as we don't actually show the results, rephrase for clarity}.
This shows that the annotators are aligned and indicates a high quality of the annotations. Comparing to previous work, \cite{tang_tofueval_2024} report a Fleiss Kappa of 0.34-0.42 on  faithfulness labels which they deem fair to moderate agreement. Note that a direct comparison with this work is not possible as they deal with different tasks, nevertheless the high IAA we are reporting is a testament of the effectiveness of our flow chart-based annotation design.
%The IAA for fine-grained faithfulness ranges from moderate to low (Gwet's AC1 0.63 for EN to 0.18 for HI).
%Those annotations are shared for reference and improving their IAA could be the focus of future work.
%The annotations allow us to gauge LLMs RAG performance across languages, which we %present in Section~\ref{sec:details-on-dataset}.
Further details on IAA including fine-grained explanatory labels can be found in Appendix A, Table~\ref{tab:iaa_appendix}.
%\jt{Should we report Fleiss Kappa as well}\sm{yes, I added parts of the rebuttal}
%\dl{We have moved and rewrite the dataset statics to result section.}


% data -> https://quip-amazon.com/AsfjAJjWZ2Xy/Final-report-internship-Benchmarking-factuality-of-LLMs-on-multilingual-long-form-QA#temp:C:QGAe2957bea2d2349798a955cb1b




\begin{table}[h]
\centering
% \small
\begin{tabular}{@{}ccccl@{}}
\toprule
\textbf{Lang} & \multicolumn{2}{c}{\textbf{Faithfulness}}              & \multicolumn{2}{c}{\textbf{Relevance}} \\ \midrule
              & \textbf{Gwet's} & \multicolumn{1}{l|}{\textbf{Fleiss}} & \textbf{Gwet's}    & \textbf{Fleiss}   \\
\multicolumn{1}{l}{} & \multicolumn{1}{l}{\textbf{AC1}} & \multicolumn{1}{l|}{\textbf{Kappa}} & \multicolumn{1}{l}{\textbf{AC1}} & \textbf{Kappa} \\ \midrule
EN            & 0.93            & \multicolumn{1}{c|}{0.77}            & 1.00               & 1.00               \\
DE            & 0.84            & \multicolumn{1}{c|}{0.81}            & 0.95               & 0.73              \\
ES            & 0.91            & \multicolumn{1}{c|}{0.76}            & 1.00               & 1.00               \\
FR            & 0.89            & \multicolumn{1}{c|}{0.88}            & 0.93               & 0.63              \\
HI            & 0.89            & \multicolumn{1}{c|}{0.70}             & 1.00               & 1.00               \\ \bottomrule
\end{tabular}
\caption{Inter-annotator agreement (IAA) on the  faithfulness and  relevance dimensions with 3 annotators. Annotations are at the answer sentence level.}
\label{tab:iaa}
\end{table}



% Perhaps we need to add the basic statistics of the dataset. 

% \subsection{Off-the-shelf multilingual factuality and relevance}




% \subsection{Dataset statistics}
% \begin{itemize}
%     \item Distribution of labels
%     \item Distribution of mistakes
%     \item Scenario analyses (LM results: numbers)
%     \item Perhaps small analysis about challenging to determine sentences (if space; otherwise one summary sentence)
% \end{itemize}




% \bc{Just give high-level statistics, we don't focus on analyzing faithfulness here}
% \dl{it seems that the human results are presented in this section. Do you sugest that we move it anwhere else?}




% Once the annotation quality has been assessed, we present the performance of models according to human judgement in the Table~\ref{tab:factuality}.
% The results show that all together, models are producing more than 67\% supported sentences, thus showing the complexity of generating factual long answers. Moreover, combining the labels "Directly answers the question" and "Adds context to the answer" as relevant information, all models are producing more than 92\% relevant sentences. These results shows that there is still room for improvement when using off-the-shelf LLMs for multilingual long QA in terms of their factuality. 

% Interestingly, when analysing the results per model and language (see E. Appendix: Results models' performance), we did no find general pattern, with different languages obtaining the largest amount of supported sentences per model. This is counter intuitive to our assumption of having more supported sentences for English language. Surprisingly, even though Hindi is a low-resource language the models are producing 70\% supported sentences, except for Mistral that produced only around 62\% supported sentences. 

% Overall, when looking at the results per model and language as shown in Table~\ref{tab:factuality}, we found that the percentage of supported generated sentences varies greatly within model across languages (a range of 10-20 percentage points between the language with lowest percentage and the language with highest percentage of supported sentences). This supports our assumption of variable behaviour/performance across different languages. 


% \begin{table}
%     \centering
%     \begin{tabular}{lrr}
%     \hline
%         Mistake & Overall & GPT4o-mini\\
%         \hline
%         Contradiction &  175& 23\\
%         Hallucination &  296& 40\\
%         Mis-referencing &  54& 9\\
%         Nuance shift &  90& 18\\
%         Opinion as fact &  19& 1\\
%         Other mistake &  34& 1\\
%         Wrong reasoning & 62& 5\\
%         \hline
%     \end{tabular}
%     \caption{Distribution of mistakes. Overall column indicates distribution of mistakes across all models and languages. }
%     \label{tab:mistake_analysis}
% \end{table}

% \begin{figure*}[!htbp]
%   \centering
%   \includegraphics[width=\textwidth]{plots/figures/mistake_distribution_by_model_lang.png}
%   \caption{Mistake distribution by model and language\sm{i assume this needs to come after table 2 as a deep dive? obviously right now it's too small, suggestion to keep eg gpt4 across langauges one column figure, and move all models to appendix}\dl{I think the mistake analysis is the diving deep of Table 1. I added the overall and GPT4 error analysis in Table~\ref{tab:mistake_analysis}. Please let us know which you prefer to keep.}}
%   \label{fig:mistakes_analysis}
% \end{figure*}


% \section{Meta-evaluator Benchmark}

\section{Annotation Results}
\label{sec:details-on-dataset}
We present in this section the results of the human annotations for the 2,322 sentences of the \dataname dataset.

% An ideal multi-lingual benchmark for meta-evaluation would provide comparable distributions of phenomena across all languages. As we did not design our benchmark directly with this intent, but rather sampled independently questions from native data and answers from real LLMs, our distribution of labels will more likely reflect the average performance achieved by SOTA models. 


\begin{table}[htbp]
    \centering
  %   \begin{tabular}{lll}
  %   \hline
  %   \textbf{Language} & \textbf{Faithfulness} & \textbf{Relevance} \\
  %   \hline
  %   English  & 65.2 / 31.5 / 3.2 & 65.2 / 32.5 / 2.2 \\
  %   German   & 71.2 / 26.7 / 2.1 & 61.3 / 26.5 / 12.2 \\
  %   Spanish  & 65.7 / 32.9 / 1.4 & 48.8 / 43.9 / 7.3 \\
  %   French   & 62.0 / 37.8 / 0.2 & 63.3 / 29.3 / 7.4 \\
  %   Hindi    & 73.8 / 25.6 / 0.6 & 68.9 / 21.4 / 9.7 \\
  %   \hline
  % \end{tabular}
\begin{tabular}{lccc|ccc}
% \multirow{3}{*}{\textbf{Lang}} & \multirow{3}{*}{\textbf{\#S}} & \multicolumn{4}{c}{\textbf{IAA (Gwet's AC1)}} \\
%  &   & \multicolumn{2}{c}{\textbf{Faithfulness}} & \multicolumn{2}{c}{\textbf{Relevance}} \\
% &   & \multicolumn{1}{c}{\textbf{Coarse}} & \multicolumn{1}{l}{\textbf{Fine}} & \textbf{Coarse} & \textbf{Fine} \\ 
\toprule
\textbf{Lang} & \multicolumn{3}{c|}{\textbf{Faithfulness}} & \multicolumn{3}{c}{\textbf{Relevance}} \\
    & \ding{52} & \ding{53} & \multicolumn{1}{c|}{$\text{\sffamily ?}$} & \ding{52} & \ding{51} & \ding{53} \\
\midrule
EN & 65.2 & 31.5 & 3.2 & 65.2 & 32.5 & 2.2 \\
DE & 71.2 & 26.7 & 2.1 & 61.3 & 26.5 & 12.2 \\
ES & 65.7 & 32.9 & 1.4 & 48.8 & 43.9 & 7.3 \\
FR & 62.0 & 37.8 & 0.2 & 63.3 & 29.3 & 7.4 \\
HI & 73.8 & 25.6 & 0.6 & 68.9 & 21.4 & 9.7 \\
\bottomrule
\end{tabular}
    \caption{Label distribution in the benchmark. Percentage of sentences labelled as supported (\ding{52}), not supported (\ding{53}), challenging to determine ($\text{\sffamily ?}$) for  faithfulness, and as directly answers the question (\ding{52}), adds context to the answer (\ding{51}), unrelated to the question (\ding{53}) for relevance, per language.
    \label{tab:factuality}
    }
\end{table}
\noindent
Table~\ref{tab:factuality} shows the distribution of faithfulness and relevance labels across the five languages.
The distribution of labels is consistent across all languages, with a few exceptions.
On faithfulness, German and Hindi show higher percentages of \textit{Supported} answers.
On relevance, Spanish presents a significantly higher percentage of labels \textit{Adds context to the answer} compared to other languages while English had a very low share of labels \textit{Unrelated to the question}.
As a partial explanation for this, we note that Spanish questions generated the largest numbers of output sentences, 563 vs. 400 (see Table~\ref{tab:dataset_stats}) for English.
Hence, we expect that the relevance statistics reflect the tendency of Spanish answers to be more verbose. 




% \section{Data Error Analysis}
% \label{app-sec:error_analysis}

% \begin{table}[!htbp]
%     \centering
%     \begin{tabular}{lrr}
%     \hline
%         Mistake & Overall & \gptfomini\\
%         \hline
%         Contradiction &  175& 23\\
%         Hallucination &  296& 40\\
%         Mis-referencing &  54& 9\\
%         Nuance shift &  90& 18\\
%         Opinion as fact &  19& 1\\
%         Other mistake &  34& 1\\
%         Wrong reasoning & 62& 5\\
%         \hline
%     \end{tabular}
%     \caption{Distribution of mistakes. Overall column indicates distribution of mistakes across all models and languages. }
%     \label{tab:mistake_analysis}
% \end{table}
% \begin{table}[!htbp]
%     \centering
%     \begin{tabular}{lr}
%     \hline
%         Mistake & Overall \\
%         \hline
%         Adds new information &  296\\
%         Contradiction &  175\\
%         Nuance shift &  90\\
%         Wrong reasoning & 62\\
%         Mis-referencing &  54\\
%         Other mistake &  34\\
%         Opinion as fact &  19\\
%         \hline
%     \end{tabular}
%     \caption{Distribution of mistakes. Overall column indicates distribution of mistakes across all models and languages. %\jt{1. Sorting the labels by count would make the reading easier}
%     }
%     \label{tab:mistake_analysis}
% \end{table}



% \begin{table}[!htbp]
%     \centering
%     \resizebox{\columnwidth}{!}{
%     \begin{tabular}{lrrrrr}
%     \hline
%     %  Mistake & {Claude 3} & {Llama3} & Llama3 & Mistral & GPT4o- \\
%     % & Sonnet & 70B & 8B & & mini \\
%     Mistake & EN & DE & ES & FR & HI \\
%     \hline
%     Contradiction & 56 & 15 & 17 & 109 & 23 \\
%     Hallucination & 47 & 19 & 31 & 83 & 40 \\
%     Nuance shift & 5 & 8 & 15 & 26 & 18 \\
%     Opinion as fact & 3 & 3 & 4 & 8 & 1 \\
%     Mis-referencing & 14 & 6 & 11 & 10 & 9 \\
%     Wrong reasoning & 136 & 8 & 6 & 13 & 5 \\
%     Other mistake & 10 & 1 & 2 & 22 & 1 \\
%     \hline
%     \end{tabular} 
% }
% \caption{Distribution of mistakes, breakdown on languages.}
% \label{tab:mistake_analysis3}
% \end{table}

%\begin{table}[thbp]
\begin{table}[t]
\centering
    \resizebox{\columnwidth}{!}{
    % {
% \begin{tabular}{lrrrrr}
% \hline
% \textbf{Mistake} & \textbf{EN} & \textbf{DE} & \textbf{ES} & \textbf{FR} & \textbf{HI} \\
% \hline
% Logical conclusion \\
% Direct paraphrase \\
% \hline
% Contradiction & 18 & \textbf{53} & 47 & 32 & 25 \\
% Adds new info. & 28 & 45 & \textbf{90} & \textbf{81} & \textbf{52} \\
% Mis-referencing & 6 & 14 & 13 & 20 & 1 \\
% Nuance shift & 27 & 3 & 24 & 30 & 6 \\
% Opinion as fact & 2 & 3 & 1 & 12 & 1 \\
% Other mistake & 5 & 4 & 2 & 19 & 4 \\
% Wrong reasoning & \textbf{40} & 3 & 8 & 10 & 1 \\
% \hline
% \end{tabular}
% \begin{tabular}{lrrrrr}
\begin{tabular}{l*{5}{S[table-format=2.1]}}
\toprule
\textbf{Label} & {\textbf{EN}} & {\textbf{DE}} & {\textbf{ES}} & {\textbf{FR}} & {\textbf{HI}} \\
\midrule
Direct paraphrase & 8.5 & 41.7 & 25.0 & 33.7 & 40.7 \\
Logical conclusion & 41.2 & 28.2 & 30.4 & 28.0 & 5.7 \\
Other & 15.5 & 1.3 & 10.3 & 0.4 & 27.4 \\
\midrule
Adds new info & 7.0 & 9.6 & 16.0 & 15.0 & 14.8 \\
Contradiction & 4.5 & 11.3 & 8.3 & 5.9 & 7.1 \\
Mis-referencing & 1.5 & 3.0 & 2.3 & 3.7 & 0.3 \\
Nuance shift & 6.8 & 0.6 & 4.3 & 5.6 & 1.7 \\
Opinion as fact & 0.5 & 0.6 & 0.2 & 2.2 & 0.3 \\
Wrong reasoning & 10.0 & 0.6 & 1.4 & 1.9 & 0.3 \\
Other  & 1.2 & 0.9 & 0.4 & 3.5 & 1.1 \\
\midrule
Challeng. to determ. & 3.2 & 2.1 & 1.4 & 0.2 & 0.6 \\
\bottomrule
\end{tabular}
}
    \caption{Fine-grained faithfulness label distribution in the benchmark. Percentage of sentences with each label per language. The three sections correspond to the coarse-grained labels \textit{Supported/Not supported/Challenging to determine}.
    }
\label{tab:mistake_analysis}
\end{table}


\noindent
For a more granular insights into the annotations, we show in  Table~\ref{tab:mistake_analysis} the distribution of the explanatory fine-grained faithfulness labels for each language. (As label names are quite self-explanatory, we refer for their precise meaning to Figure 3 in Appendix A.)   Table~\ref{tab:mistake_analysis} is split into three blocks, respectively, addressing fine-grained labels for \textit{Supported} (top), \textit{Not supported} (middle), and \textit{Challenging to determine} (bottom) answers.  We observe significant differences across languages. 
For example, supported answers for English are prominently under form of a \textit{logical conclusion} from the context (41.2\%),  for German and Hindi from \textit{direct paraphrasing} of information in the context (41.7\% and 40.7\%), 
while for French and Spanish from a more balanced combination of the two reasons. 
On the side of unsupported answers,  the main mistake type in English is \textit{Wrong reasoning} (10\%), i.e. answers are non logical conclusions from the context, while this type or error is significantly rarer (under 2\%) for all other languages. The rate of  \textit{Adds new information} errors, a.k.a. hallucinations, ranges from 7\% for English to 16\% for Spanish. 
The observed cross-linguistic variations in the distribution of labels can be attributed to  the different nature of the questions and accuracy of the models across the 5 languages. 

%\mf{Overall, our dataset presents a quite balanced distribution of faithfulness and relevance labels,  which makes for a challenging benchmark for the meta-evaluation of multilingual automated evaluators.}
% In summary, our benchmark presents a range of challenges for multilingual meta-evaluation that require innovative solutions. In the following section, we discuss these challenges in detail and provide baseline results to address them.

%One plausible explanation is that the annotators in each language had different interpretations of the fine-grained labels despite the detailed guidelines.
%To investigate this, we show in Table~\ref{tab:mistake_analysis_llm} the distribution of faithfulness mistakes by generating LLM.
%As our purpose is not to compare the LLMs but to provide a diverse dataset, we anonymize the LLM names.
%We see significantly different profiles across LLMs.
%For example, LLM-A has a significantly larger mistake rate than LLM-B and LLM-D has a different main mistake type than LLM-A.
%These being aggregated across languages lends credence to the fact that the fine-grained faithfulness labels capture important difference in the ways that LLMs generate answers.
%Additional label distributions at the LLM-level can be found in Appendix~\ref{appendix:memereg-detail}. 

% Furthermore, we look into the factuality mistakes in the responses generated in our dataset.
% To study the error patterns, we categorize all not-supported responses into nine classes and report the statics in mistake types of models in Table~\ref{tab:mistake_analysis}.
% When inspecting the type of mistakes that these models are making, the most common mistake is hallucination and the second most common mistake type is contradiction. We also report the number of mistakes in mis-referencing, nuance shift, opinion as fact and wrong reasoning. For the mistake types that do not fall into these listed ones, we label them as other mistake.\sm{from table 4 I mostly understand there are 7 types of mistakes, when aggregating across languages and LLMs it could mask an issue with a specific LLM, eg if one is generating many nuance shifts errors. If we focus on a specific language, is it useful to compare two LLMs and any insights?}%\dl{Added Table mistake_analysis2 in appendix, is that more helpful?}
%showing the tendency of this models (regardless of their size) to introduce extrinsic information (As shown in Table~\ref{tab:mistake_analysis}). 
%However, certain models displayed more hallucination mistakes than others (e.g., Sonnet in ES, FR, and HI or  \gptfomini in all languages except EN), or more contradiction than others (e.g., Mistral). Interestingly, for DE and HI, there is less representation of several mistakes per model, for instance there is only one instance of mis-referencing mistake with Llama3 8B.


\section{Dataset Applications}
\label{sec:experimental}

The \dataname dataset is designed to support the development of reliable automatic evaluation methods.
For that purpose, we describe in this section how our dataset can be used as a benchmark to enable various meta-evaluation use cases.
% In this section, we focus on the use cases it enables rather than the underlying capabilities of the LLMs.
% The focus of this section is deriving an effective benchmark from our dataset and showcasing
We focus on two applications: 1) \textbf{Prompt selection}: The ability of our benchmark to effectively select prompts for automatic evaluation, 2) \textbf{Model selection}: The effectiveness of our benchmark to distinguish and select models. %, 3) \textbf{Fine-grained analysis}: The insights provided by our benchmark on the specific strengths and weaknesses of automatic evaluators.
By concentrating on these aspects, we can evaluate the benchmark's utility as a comprehensive tool for multilingual model assessment.
While we provide reference baselines for each application, our focus is on showcasing the effectiveness of the benchmark rather than the underlying capabilities of the LLMs.

\subsection{Experimental Setup}
% Our meta-evaluation benchmark enables the development of reliable automatic evaluation methods. Hence our focus is on the effectiveness of our benchmark and the usecases it enables rather than the underlying capabilities of the LLMs. We focus on two critical aspects of our benchmark: 1) Performance differentiation across languages - We aim to determine how effectively the benchmark distinguishes models' performance in faithfulness evaluation across various languages, and 2) Language-specific behavioral insights - We seek to determine the benchmark's capacity to expose distinct, language-specific behaviors exhibited by different models. By concentrating on these aspects, we can evaluate the benchmark's utility as a comprehensive tool for multilingual model assessment. This approach allows us to gauge its effectiveness in providing valuable insights into language-specific performance variations, cross-linguistic generalization abilities, and potential biases or limitations in multilingual language model capabilities.
% \looseness=-1


% \dataname enables three main usecases: 1) multilingual RAG - generation performance across the languages, 2) multilingual meta-evaluation - developing and benchmarking multilingual automatic evaluators, and 3) native vs. translated data - how effective is translated data for model ranking vs. native data. 
%We demonstrate two applications of our meta-evaluation benchmark in an LLM-as-a-Judge setting, namely 1) Evaluating various prompting strategies and LLMs 2) Ranking evaluators on translated datasets. We focus on faithfulness and setup meta evaluation as a binary classification task, where automatic annotator is expected to output \textit{Supported} or \textit{Not Supported} labels. 
% To develop automatic evaluation methods, we implemented an LLM-as-a-judge framework. We experiment with various LLMs and prompting techniques.

\paragraph{Benchmark Tasks}
Our benchmark is composed of multiple tasks defined by the annotation dimension and subset considered.
On the annotation dimension, we focus our experiments on the coarse-grained faithfulness dimension.
This dimension is more challenging than relevance as highlighted by the lower IAA, while retaining a high-enough IAA to make for a trustworthy benchmark.
We invite benchmark users to also experiment on the other dimensions provided by the dataset depending on their use case.
Note that we remove the sentences labelled as \textit{Challenging to determine} by human annotators.
On the annotation subset, we first distinguish the \textit{multilingual} task which uses the full dataset and the \textit{monolingual} task, which only considers a single language.
Those task can then be further broken down at the \textit{fine-grained} level by considering the subset of sentences with a certain fine-grained label.
Performance is evaluated with Balanced Accuracy (BAcc) (see Appendix ~\ref{sec:bacc_def} for definition), with equal weights on each coarse-grained label and language.
We conduct significance testing using permutation tests~\cite{good2013permutation}, with further details in Appendix~\ref{sec:stat_test}.


\begin{comment}
\mf{Our primary objective is to assess the effectiveness of a benchmark designed for multilingual factuality evaluation. We focus on two critical aspects:\\
\begin{itemize}
\item Performance Differentiation Across Languages: We aim to determine how effectively the benchmark distinguishes models' performance in factuality evaluation across various languages. 
\item Language-Specific Behavioral Insights: We seek to determine the benchmark's capacity to expose distinct, language-specific behaviors exhibited by different models.
\end{itemize}
By concentrating on these aspects, we can evaluate the benchmark's utility as a comprehensive tool for multilingual model assessment. This approach allows us to gauge its effectiveness in providing valuable insights into language-specific performance variations, cross-linguistic generalization abilities, and potential biases or limitations in multilingual language model capabilities.}
\end{comment}

\ifsectionenabled
\subsection{Multilingual RAG Performance}
As we annotated the models' generated answers with faithfulness and relevance labels, we can measure the performance of those generators across the languages. The answer generation during the data labeling phase was conducted using five LLMs as mentioned in Section~\ref{sec:answer_gen}. %, though we mostly focus on the performance differences across the languages rather than assessing the models themselves.
\fi

\label{sec:prompting_strategies}

% \paragraph{Prompt Selection}
\paragraph{Reference Prompts}
To demonstrate how the benchmark can be used to select the appropriate prompt, we experiment with multiple prompting strategies from simple to advanced techniques, starting with zero-shot prompting, where LLMs directly classify statements as \textit{Supported} or \textit{Not supported}. We then implement chain-of-thought (COT) prompting \cite{chain_of_thought_jason_wei}, which incorporates an intermediate reasoning step. While these basic prompting strategies provide a good starting point for evaluation, our initial experiments revealed limitations in their ability to capture the nuanced requirements of our specific task. Without explicit guidelines in the prompt, automatic evaluators rely on their ``world knowledge'', which may not align with the specific requirements of the evaluation task. To overcome this, we add  instructions from the annotation guidelines (AG) in the prompt. % as illustrated by Figure~\ref{fig:annotation_guide} (Appendix) in the prompt. 
Adding annotation guidelines provides clear criteria for what constitutes \textit{Supported} versus \textit{Not Supported} sentences, reducing ambiguity in the evaluation process. The various prompts are presented in Appendix~\ref{sec:eval_prompts}.

% \paragraph{Model Selection}
\paragraph{Reference Models}
We experiment with four LLMs with varying model sizes and capabilities: \gptfomini, \qwenthrirtytwob, and two versions of Llama 3.2 (11B and 90B)\footnote{In comparison to the LLMs selected for answer generation, see Section~\ref{sec:answer_gen}, we upgraded Llama from 3 to 3.2 as the context length of 8K tokens was not sufficient for all prompts. We picked GPT-4o mini as representative of proprietary models. Additionally, Mistral was excluded as it performed poorly in initial experiments.}. In case an LLM does not produce one of the required labels, we repeatedly prompt the LLM up to five times with temperature and \verb|top_p| equal to 0.1 to get a valid label. If the LLM fails to generate a label after five retries, we treat the datapoint as an error (wrong label).



\begin{table}[]% [!t]
% \small
\centering
%\begin{tabular}{@{}cccccc@{}}
\begin{tabular}{
    l
    S[table-format = 2.1]
    S[table-format = 2.1]
    S[table-format = 2.1]
    S[table-format = 2.1]
}
\toprule
\textbf{Prompt} & \textbf{\small \shortstack{GPT-4o\\mini}}    & \textbf{\small \shortstack{Qwen\\2.5 32B}}    & \textbf{\small \shortstack{Llama\\3.2 90B}}    & \textbf{\small \shortstack{Llama\\ 3.2 11B}}    \\ \midrule
ZS & 59.7 & 66.7 & 58.0 & 55.4 \\
COT & 61.4 & 68.8 & 59.9 & \best{62.5} \\
AG & \comparable{71.6} & \best{72.6} & \comparable{62.8} & 57.9 \\
AG+COT & \best{71.7} & \comparable{71.8} & \best{64.4} & \comparable{61.6} \\
\bottomrule
\end{tabular}
\caption{Reference baselines for the multilingual task on coarse-grained faithfulness. Balanced accuracy (BAcc) of the automatic evaluators averaged across the 5 languages (EN, DE, ES, FR, HI) using zero-shot (ZS), chain-of-thought (COT), annotation guidelines (AG) and AG+COT prompting strategies. Bold indicates best performance for the column, $\dagger$ indicates results not statistically different from the best (p > 0.05). Additional results on monolingual tasks and standard errors can be found in Appendix~\ref{sec:stat_test}, Tables \ref{tab:avg_llm_bacc_with_se}-\ref{tab:metaeval_all}.}
\label{tab:avg_prompt_bacc}
\end{table}



\ifsectionenabled
\begin{table}%[!t]
\centering
\begin{tabular}{
    m{1.5cm}
    S[table-format = 2.1]
    S[table-format = 2.1]
    S[table-format = 2.1]
    S[table-format = 2.1]
    S[table-format = 2.1]
}
\toprule
\textbf{Model}                  & \textbf{EN} & \textbf{DE} & \textbf{ES} & \textbf{FR} & \textbf{HI} \\ \midrule
\gptfomini & \best{68.4} & \comparable{73.7} & \comparable{69.9} & \comparable{73.7} & \comparable{74.2} \\
\qwenthrirtytwob & 62.5 & \best{76.8} & \best{71.1} & \best{74.4} & \best{75.5} \\
Llama 3.2 90B & 62.6 & 63.2 & 63.4 & 63.2 & \comparable{75.1} \\
Llama 3.2 11B & 60.2 & 60.0 & 59.1 & 65.0 & 65.2 \\\bottomrule
\end{tabular}
\caption{Automatic evaluator results. Balanced Accuracy (BAcc) of automatic evaluators using various LLMs-as-a-judge, across five languages using the best prompt. Bold indicates best performance, $\dagger$ indicates results not statistically different from the best (p > 0.05).}
\label{tab:avg_llm_bacc}
\end{table}
\fi


\begin{figure*}[t]
  \centering\small
  \begin{tabular}{cc}
    \includegraphics[width=0.48\linewidth]{plots/bacc_per_language/bigger_font/bacc_plot_DE.pdf} &
    \includegraphics[width=0.48\linewidth]{plots/bacc_per_language/bigger_font/bacc_plot_EN.pdf} \\
    \includegraphics[width=0.48\linewidth]{plots/bacc_per_language/bigger_font/bacc_plot_ES.pdf} &
    \includegraphics[width=0.48\linewidth]{plots/bacc_per_language/bigger_font/bacc_plot_FR.pdf} \\
  \end{tabular}
  \includegraphics[width=0.48\linewidth]{plots/bacc_per_language/bigger_font/bacc_plot_HI.pdf}

  \caption{Reference baselines for the monolingual task  on coarse-grained faithfulness. Balanced Accuracy (BAcc) of the automatic evaluators using various LLMs across five languages: EN, DE, ES, FR and HI. Each plot compares the performance of four models: Llama 3.2 11B, Qwen 2.5 32B, Llama 3.2 90B, and \gptfomini using AG + COT prompt. The best-performing model for each language is highlighted with a darker blue bar. Bars with diagonal hatching indicate results not statistically different from the best (p > 0.05). }%\jt{Shrink the figures}}
  \label{fig:bacc_per_lang}
\end{figure*}


% \section{Automatic Evaluators Results}
\subsection{Experimental Results}
Our benchmark enables systematic evaluation of different approaches to automated faithfulness evaluation. To illustrate this, we examine how the benchmark can surface the effectiveness of various automatic evaluation models and prompting strategies.

Table \ref{tab:avg_prompt_bacc} demonstrates the benchmark's ability to compare different prompting approaches across languages. The benchmark reveals consistent patterns, showing how different prompt designs impact evaluation quality. As expected, adding a reasoning step (COT) improves over zero-shot prompting. In addition, adding annotation guidelines (AG) helps align automated evaluators with human judgments across all languages. %, demonstrating the benchmark's utility in identifying robust cross-lingual patterns.
Comparing the two best models \gptfomini and \qwenthrirtytwob, \qwenthrirtytwob excels in the zero-shot and COT setups, showcasing higher ``out-of-the-box'' alignment with human judgements. \gptfomini achieves similar performance once the annotation guidelines are added to the prompt. 

% \sm{rewrite, as we talked one option is bars: next, for a specific language, we compare different LLMs, fixing the best prompt as discussed before, gpt best for english, quen for rest...} 

Figure ~\ref{fig:bacc_per_lang}, shows the performance per language of various automatic evaluators with a fixed prompt (AG + COT), which allows us to select the best model for each language. We observe that \gptfomini performs best in English. For the rest of the languages \qwenthrirtytwob performs the best however, the results are not statistically different from \gptfomini. Our benchmark also provides users with the capability to conduct detailed, fine-grained analyses of model performance across various dimensions of faithfulness. The breakdown of automatic evaluation performance by error type is shown in Appendix ~\ref{appendix:fine_grained_analysis}.


\label{sec:prompt_perf}


\looseness=-1


\section{Conclusions}
We introduced a high-quality and challenging multilingual end-to-end meta-evaluation benchmark for RAG (MEMERAG). Our carefully designed flow-chart-based annotation achieved a high inter-annotator agreement rate supporting the reliability of the benchmark. The introduced MEMERAG dataset opens the door for multiple application scenarios, including but not limited to the demonstrated cases, i.e. prompt selection and model selection.

%\dl{We further show insights on the fine-grained faithfulness distribution of MEMERAG acorss five languages. The introduced MEMERAG dataset opens the door for multiple application scenarios, including but not limited to the demonstrated cases, i.e. prompt selection, model selection and fine-grained analysis. }
% For answer generation, we show that LLMs performance varies across the languages where \sm{TODO eg bigger LLMs are better, llm-c better on french why?}%English performs best in terms of faithfulness. 
% \jt{something on fine-grained labels}
For the meta-evaluation setup, we develop and compare various LLMs-as-a-judge and observe that automatic evaluators performance varies across the languages, influenced by language characteristics, native-question complexity and LLM generation nuances. These variations underscore the importance of our testbed, which demonstrated consistent results when comparing the prompting techniques (COT+guidelines > COT > zero-shot) and provides a foundation for developing better multilingual evaluators. 


\newpage
\clearpage

\section{Limitations}
Due to time and cost constraints, our annotations and experiments are limited in terms of prompting techniques, LLMs we experimented with and languages we annotated. Nevertheless, we diversified our LLMs across size and ``openness'' while the languages represent two families and low and high resource ones. In addition, there exists in the literature fine-tuned factuality evaluators for English, though we expect those to not work as well on non-English languages. Another method is to approximate factuality through entailment tasks (i.e. XNLI dataset) though such methods were shown (for English) to be inferior to multi-task training and distillation and data augmentation from LLMs~\cite{tang_minicheck_2024}. Fine-tuning multilingual evaluators and examining transfer learning across languages is interesting but is left for future work that can leverage our dataset for this purpose.

As we advocate for a native testing approach, the questions across the languages are not parallel, which could introduce a dimension of different questions and LLM generations complexities across the different language test data. The data we collected presents different challenges which are captured according to our fine grained error labels (Table~\ref{tab:mistake_analysis}). Future work could balance the challenges and complexities by collecting data for specific challenging phenomena. Note that this balancing is not straightforward, it can be done on the question side though this is insufficient as it does not control for the answer complexity. Controlling for the answer complexity is a challenging problem as the answer side is model generated (one method is to generate many answers and select for certain phenomena with human in the loop which is costly).
% We conducted the following mitigations to remedy the issue \sm{TODO}. 
% In addition, one could have a parallel culture independent questions and a non-parallel culture specific questions. We leave such extensions to future work. %\sm{TODO if we add ranking results, need to add that if models are closer together results might not hold and related work}




\begin{comment}
\begin{table}[]
    \centering
    \small{
    \begin{tabular}{lccccc}
         \hline
         \textbf{Evaluator} & \textbf{EN} & \textbf{DE} & \textbf{ES} & \textbf{FR} & \textbf{HI} \\\hline
         Claude 3 Sonnet & & & & & \\
         \textit{Zero-Shot} & 0.55 & 0.66 & 0.58 & 0.59 & 0.68 \\
         \textit{CoT} & 0.54 & 0.65 & 0.57 & 0.60 & 0.71\\
         Llama3 70B & & & & & \\
         \textit{Zero-Shot} & 0.58 & 0.59 & 0.55 & 0.55 & 0.68 \\
         \textit{CoT} & 0.61 & 0.62 & 0.56 & 0.57 & 0.69 \\
         Llama3 8B & & & & & \\
         \textit{Zero-Shot} & 0.51 & 0.55 & 0.50 & 0.54 & 0.61 \\
         \textit{CoT} & 0.57 & 0.61 & 0.59 & 0.64 & 0.66\\
         Mistral & & & & & \\
         \textit{Zero-Shot} & 0.48 & 0.51 & 0.48 & 0.51 & 0.55 \\
         \textit{CoT} & 0.48 & 0.57 & 0.49 & 0.52 & 0.56 \\
         \hline 
    \end{tabular}
    \caption{Balanced accuracy of evaluator models per prompt strategy and language.}
    \label{tab:automatic_evaluators}}
\end{table}
\end{comment}



% Bibliography entries for the entire Anthology, followed by custom entries
%\bibliography{anthology,custom}
% Custom bibliography entries only
\bibliography{custom}


\newpage
\appendix

\begin{figure*}[!h]
  \centering
  \includegraphics[width=\textwidth]{plots/figures/annotation_guide.png}
  \caption{Annotation guideline for faithfulness labelling by human}
  \label{fig:annotation_guide}
\end{figure*}

\begin{figure*}[!tbp]
  \centering
  \includegraphics[width=\textwidth]{plots/figures/labelling_ui.png}
  \caption{User interface used by human annotators for labelling faithfulness and relevance.}
  \label{fig:labelling_ui}
\end{figure*}



\section{Human Annotation Guidance}
\label{appendix:annotation_guideline}

Before conducting large-scale annotations, we conducted a pilot with 10 English RAG outputs and 3 annotators.
We asked annotators to evaluate faithfulness at the sentence level either as \textit{Supported} or with a subset of the factuality mistakes typology in~\citet{tang_tofueval_2024} (developed for summarization): \textit{Contradiction}, \textit{Hallucination}, \textit{Mis-referencing}, \textit{Nuance meaning shift}, \textit{Opinion stated as fact}, \textit{Wrong reasoning}, to which was an \textit{Other mistake} label was added to account for unforeseen mistakes in the RAG setting.
As this led to very low IAA, we conducted another round reducing the non-supported labels to \textit{Stating opinion as fact}, \textit{Drawing wrong conclusions}, \textit{Other mistake} but this still resulted in low IAA (Gwet's AC1 0.45).

Upon careful analysis of the annotator disagreements in the pilot and further rounds of calibration, we developed the annotation workflow shown in Figure~\ref{fig:annotation_guide}.
The key improvements were:
(i) add a \textit{Challenging to determine} label,
(ii) have two levels of labels, a coarse-grained level (\textit{Supported}, \textit{Not supported}, \textit{Challenging to determine}) and a fine-grained level for precision on the mistakes, 
(iii) enforce annotators to follow a specific reasoning with a flow chart,
(iv) use numbers for the fine-grained level rather than labels which could be misinterpreted.
Those guidelines allowed to reach significantly higher IAA on faithfulness (Gwet's AC1 0.81 coarse-grained).
Likewise we iterated on the relevance labels, starting from \textit{Must have}, \textit{Nice to have}, or \textit{Irrelevant} and converging to the more explicit \textit{Directly answers the question}, \textit{Adds context to the answer}, or \textit{Unrelated to the question}.
% 0.9 on EN from commit 5aa342f
This also increased the IAA significantly. The screenshot of the user interface used by human annotators is shown in Figure ~\ref{fig:labelling_ui}
%(Cohen's kappa 0.15 to 0.90). \bc{Find AC1 numbers or remove}





\begin{table}[h]
    \centering
    \normalsize{
    \begin{tabular}{ccclll}
    \hline
    \multirow{3}{*}{\textbf{Lang}} & \multirow{3}{*}{\textbf{\#S}} & \multicolumn{4}{c}{\textbf{IAA (Gwet's AC1)}} \\
     &   & \multicolumn{2}{c}{\textbf{Faithfulness}} & \multicolumn{2}{c}{\textbf{Relevance}} \\
    &   & \multicolumn{1}{c}{\textbf{Coarse}} & \multicolumn{1}{l}{\textbf{Fine}} & \textbf{Coarse} & \textbf{Fine} \\ 
    \hline
    EN  & 13 & 0.93 & 0.63 &    1 & 0.92 \\
    DE  & 31 & 0.84 & 0.47 & 0.95 & 0.91 \\
    ES  & 20 & 0.91 & 0.3  &    1 & 0.93 \\
    FR  & 23 & 0.89 & 0.58 & 0.93 & 0.92 \\
    HI  & 17 & 0.89 & 0.18 &    1 & 1    \\\hline
  \end{tabular}
  }
    \caption{Inter-annotator agreement (IAA) on the faithfulness and relevance dimensions with 3 annotators, for coarse-grained and fine-grained levels. Annotations are at the answer sentence level (\#S number of answer sentences is provided).
    }
    \label{tab:iaa_appendix}
\end{table}



\newpage

\newpage



\section{Detailed numbers on MEMEREG dataset}
\label{appendix:memereg-detail}
\begin{table*}[htbp]
    \centering
    \small{
    \begin{tabular}{lccccc}
    \hline
    \textbf{Model} & \textbf{EN} & \textbf{DE} & \textbf{ES} & \textbf{FR} & \textbf{HI}\\
    \hline
    % Claude 3 Sonnet (615) & & & & & \\
    LLM-A (615) & & & & & \\
    \textit{Faithfulness} & 52.7 / 43.8 / 3.5 & 74.3 / 17.1 / 8.6 & 58.6 / 41.4 / 0 & 59.6 / 40.4 / 0 & 73.2 / 26.8 / 0\\ 
    \textit{Relevance} & 42.5 / 51.7 / 5.8 & 60.9 / 37.4 / 1.7 & 42.3 / 54.5 / 3.2 & 48 / 50.3 / 1.7 & 46.8 / 41.3 / 11.9 \\
    % Llama3 70B (315) & & & & & \\
    LLM-B (315) & & & & & \\
    \textit{Faithfulness} & {76.1} / 17.9 / 6 & {88.4} / 11.6 / 0 & {88.2} / 11.8 / 0 & 58.6 / 41.4 / 0 & {84.9} / 15.1 / 0 \\
     \textit{Relevance} & 67.1 / 31.5 / 1.4 & 72.7 / 22.1 / 5.2 & {75} / 22.2 / 2.8 & {87.1} / 12.9 / 0 & 80.7 / 10.5 / 8.8 \\
    %Llama3 8B (315) & & & & & \\
    LLM-C (315) & & & & & \\
    \textit{Faithfulness} & 61.4 / 35.1 / 3.5 & 69.8 / 30.2 / 0 & 68.1 / 31.9 / 0 & {80.3} / 19.7 / 0 & 80 / 20 / 0 \\
     \textit{Relevance} & {86.9} / 11.5 / 1.6 & {78.3} / 20.3 / 1.4 & 69.3 / 28 / 2.7 & {87.1} / 10 / 2.9 & 85.9 / 9.4 / 4.7\\
    % Mistral (743) & & & & & \\
    LLM-D (743) & & & & & \\
    \textit{Faithfulness} & 67.3 / 30.8 / 1.9 & 61.4 / 38 / 0.6 & 61 / 35.7 / 3.3  & 59.8 / 40.2 / 0 & 63 / 34.6 / 2.5\\
     \textit{Relevance} & 70.3 / 29.7 / 0 & 41.5 / 33 / 25.5 & 33 / 52.3 / 14.7 & 54.8 / 26.1 / 19.1 & 57.1 / 24.2 / 18.7\\
    % \gptfomini (334) & & & & & \\
    LLM-E (334) & & & & & \\
    \textit{Faithfulness} & 77.2 / 21.1 / 1.7 & 73.9 / 26.1 / 0 & 69.7 / 29 / 1.3 & 59.2 / 39.5 / 1.3 & 73.3 / 26.7 / 0\\
     \textit{Relevance} & 80.3 / 19.7 / 0 & 75.4 / 17.4 / 7.3 & 63.4 / 36.6 / 0 & 73.8 / 25 / 1.2 & {90.6} / 9.4 / 0 \\\hline
  \end{tabular}
    \caption{Percentage of sentences labelled as \textit{Supported/Not Supported/Challenging to determine} for the faithfulness, and as \textit{Directly answers the question/Adds context to the answer/Unrelated to the question} for the relevance, per language and model. The total number of sentences generated by the model across all languages is given in parenthesis besides the model name.
    }
    }
\end{table*}

In this section, we look at the detailed statics of faithfulness and relevance in the MEMERAG dataset itself. Overall, the faithfulness and relevance in the dataset variate a lot from one language to another language and from one generator model to another.
There is usually a range of 10-20 percentage points between the language with lowest percentage and the language with highest percentage of supported sentences. This supports our assumption of variable behaviour/performance across different languages. 
 
Given the large variation, the percentage of supported answers is larger than that of unsuported answers. Similarly, the ratio of relevant answers is in general larger than that of other relevance buckets (except for ES with LLM-D; EN, ES and FR with LLM-A).
Interestingly, when analysing the vairous results per model and language, we did no find general pattern. This is counter intuitive to our assumption of having more supported sentences for English language. Surprisingly, even though Hindi is a low-resource language the models are producing 70\% supported sentences, except for LLM-D that produced only around 62\% supported sentences. 



\section{Fine grained automatic annotation analysis}
\begin{figure*}[!htbp]
  \centering
  \includegraphics[width=\textwidth]{plots/automatic_annotator/finegrained_labels/en/heatmap_cot_with_annotation_guidelines_English_new.pdf}
  \caption{Fine-grained faithfulness errors when automatic evaluator disagrees with ground truth.}
  \label{fig:zero_shot_en_heat_map}
\end{figure*}

Figure \ref{fig:zero_shot_en_heat_map} shows a heatmap representing the failure modes of automatic evaluators. We observe that improving the prompting strategy from ZS to AG + COT, reduces automatic evaluation errors across all error categories, except for Llama 3.2 11B, where the model makes more errors in the ``Logical conclusion category''. We also observe that ``Wrong reasoning'', ``Nuance shift'' and ``Logical conclusion'' are the top error categories for all the models tested. Future work could explore prompting or fine-tuning techniques designed to handle specific error types.

\section{Statistical Significance Test Details}
\label{sec:stat_test}
In Table~\ref{tab:avg_prompt_bacc} and Figure~\ref{fig:bacc_per_lang}, we aim to determine whether the scores (in terms of BAcc) for the best prompt and best model was significantly different from the other scores in the table. To test statistical significance, we use permutation test \cite{good2013permutation}, a non-parametric method for comparing two related samples. The null hypothesis for this test suggests that there is no significant difference between the performance of the best-performing prompt/models and the other prompts/models, while the alternative hypothesis suggests a significant difference exists. We consider $\alpha = 0.05$ as significance level. Consequently, when p > 0.05, we are not able to reject the null hypothesis, indicating that prompts and models have similar performance.
\section{Balanced Accuracy}
\label{sec:bacc_def}
Due to the presence of class imbalance in our dataset, we employ balanced accuracy as the primary metric to assess the performance of our automatic evaluators. Balanced accuracy provides a more robust measure of performance when dealing with imbalanced classes, as it equally weights the recall of each class.

For our binary classification task, where the automatic evaluators predict either "Supported" or "Not Supported", the balanced accuracy (BAcc) is calculated as follows:

$$BAcc = \frac{1}{2} \left( \frac{TP}{TP + FN} + \frac{TN}{TN + FP} \right)$$

\noindent Where TP, TN, FN, and FP denote True Positives (correctly identified ``Supported'' instances), True Negatives (correctly identified ``Not Supported'' instances), False Negatives (``Supported'' instances misclassified as ``Not Supported''), and False Positives (``Not Supported'' instances misclassified as ``Supported''), respectively.


% \newpage


% \newpage
% \section{Computing Infrastructure}
% Inference for Qwen 2.5 model is done via VLLM server running on a host with 8 V100 GPUs.\cite{kwon2023efficient}. We access proprietary models via cloud provider APIs.

\section{Prompts used for dataset construction}
\label{sec:prompts_dataset_construction}
In section we describe the various prompts used in our pipeline. Out prompts are written as  Jinja2\footnote{\url{https://jinja.palletsprojects.com/en/stable/}} templates.
\subsection{Time-dependent answer filtering}
\label{prompt:time_dependent}
During internal pilots, we identified answers that relied on current date / current affairs that could be challenging to determine their faithfulness. For example, to the question "How old is Yann LeCun?" is the answer correct if the LLM uses the time when it was trained? To avoid such cases we used Llama3 70B to filter out those cases using the prompt shown below:

\begin{tcolorbox}[colback=gray!10, colframe=gray!50, title= Time-dependent Answer filtering]
\small{
\label{prompt:time_dependent}
{\ttfamily You are an NLP assistant that helps to identify if a question requires to know  when is today (day, or month, or year), current affairs, or up-to-date  information. Give the step by step of how to answer the question in between the labels <rationale></rationale>. Then verify if the steps included to know 
any information about the current time and give your answer in between the 
tags: <label></label>. Please only use 'yes' or 'no' in your final answer.  You will be given the question in \{\{language\}\}. 

Provide your rationale and label in English. 
}
}
\end{tcolorbox}

\subsection{Highlighting relevant segments}
We highlight relevant statements in the passage to help annotators focus on important parts of the context. We prompt Llama 3 70B, to predict all the statements that support the passage. We use temperature equal to 0.1 in this step.

\subsection{Answer generation prompt}
\label{sec:prompts_answers}
We generate the answers using the same prompt across all models and languages. All the instructions were given in English, while the context and question were given in the testing language. For all models we set the temperature to 0.1 and the maximum number of token to 1000.\\

\begin{tcolorbox}[colback=gray!10, colframe=gray!50, title= Highlighting Relevant Segments]
\small
\label{sec:passage_highlight}
{\ttfamily You are an agent that verifies if sentences are supported by a given context. \\
You will be given a context made of several passages referred as "Passage" split 
into sentences, each with a numeral identifier, and each referred as "Sentence". 
Your task is to determine if the sentence is supported by some of the passages  
(using label 1)  or is not supported (using label 0). Please follow this process: \\ \\
(1) Explain why the sentence is supported or not supported, please write your reasoning in between the tags <rationale></rationale>. If a sentence is supported, 
list the number of the sentences in the passage or several passages that supported 
it. If a sentence is not supported, explain why is not supported.\\  \\
(2) Write the  final label for the sentence in between the tags <label></label>. \\ \\
(3) Write the id  of the supporting sentences from passages in between the tags  <references></references> separated by comma. Remember to use only 0 or 1 for label. \\

\begin{verbatim}
### Context

Passage {{loop.index}}: 


    {{outer_loop.index}}.{{loop.index}}: 
    {{sent|safe}}




### Sentence: 
{{sentence|safe}}
\end{verbatim}

}

\end{tcolorbox}



\begin{tcolorbox}[colback=gray!10, colframe=gray!50, title= Answer Generation Prompt]
\small
\label{prompt:answer_generation}
{\ttfamily You are an NLP assistant whose purpose is to answer a question based on given passages. The passages may or may not help answer the question. You will need to 
provide the answer based only on the passages. The answer must be in {{language}} 
without fail. Be concise and direct without referring to passages in the answer. 
Avoid expressions such as "According to the passages" or "Based on the passages". \\

\begin{verbatim}

    - {{passage.text|safe}}


Question: {{question}} 

Please answer directly the question 
above in {{language}}.
\end{verbatim}
}

\end{tcolorbox}



\section{Prompts for automated evaluation}
\label{sec:eval_prompts}
The four prompts evaluate answer faithfulness to source passages with increasing complexity: Zero-shot (ZS) provides basic supported/not-supported classification, Chain of Thought (COT) adds explicit reasoning steps, Annotation Guidelines (AG) includes detailed evaluation criteria, and AG+COT combines detailed guidelines with reasoning steps. All prompts output their final classification in <answer> tags.

\begin{tcolorbox}[colback=gray!10, colframe=gray!50, title= Zero-shot Prompt (ZS)]
\label{sec:prompt_zs}
\small
{\ttfamily You are an automatic annotator tasked with determining whether a given answer is grounded in the list of provided evidence passages. Your role is to carefully analyze the relationship between the answer and the evidence, and then classify the answer as either "Supported" or "Not 
Supported". Provide your answer directly in <answer> </answer> tag. \\

Evidence Passages:
\begin{verbatim}


  .  {{loop.index}}: {{passage.text}}



Answer: {{answer segment}}

\end{verbatim}

Now provided your label directly as ''Supported'' or ''Not Supported''.
}

\end{tcolorbox}

% COT prompt

\subsubsection*{}
\begin{tcolorbox}[colback=gray!10, colframe=gray!50, title= Chain of thought (COT)]
\label{sec:prompt_cot}
\small
{\ttfamily You are an automatic annotator tasked with determining whether a given answer is grounded in the list of provided evidence passages. Your role is to carefully analyze the relationship between the answer and the evidence, write your reasoning in between the tags <rationale></rationale> then classify the answer as either ``Supported'' or ``Not Supported'' and write answer in <answer> </answer> tag \\

Evidence Passages:
\begin{verbatim}



    {{loop.index}}: {{passage.text}}



Answer: {{answer segment}}

\end{verbatim}
Now provided your label directly as "Supported" or "Not Supported".

}



\end{tcolorbox}



\subsubsection*{}

\begin{tcolorbox}[colback=gray!10, colframe=gray!50, title= Annotation Guidelines (AG)]
\small
\label{sec:prompt_ag}
{\ttfamily Given a set of evidence passages, and an answer, determine if the answer is fully supported by the evidence passages or not. Analyze each sentence of the answer carefully and verify that all information it contains is explicitly stated in or can be directly inferred from the evidence passages. \\\\

Output "Not Supported" if ANY of the following are true: \\

The answer contains any information not explicitly stated in or directly inferable from the passages. \\\\
The answer contradicts any information in the passages. \\\\
The answer introduces any new information not found in the passages.\\\\
The answer misrepresents or inaccurately paraphrases information from the passages. \\\\
The answer draws conclusions not logically supported by the given information.\\\\
The answer changes the level of certainty, specificity, or nuance from what is expressed in the passages.\\\\
The answer does not directly address the specific aspect asked about in the question.\\\\
The answer conflates or misrepresents separate pieces of information when summarizing multiple passages. \\\\

Output Supported otherwise.

Provide this determination without any additional explanation in <answer></answer> tags. Analyze thoroughly but output only the single-word label. \\

Evidence Passages:
\begin{verbatim}


    {{loop.index}}: {{passage.text}}



Answer: {{answer segment}}
\end{verbatim}\\


Now provided your label directly as "Supported" or "Not Supported".

}

\end{tcolorbox}
\pagebreak
\subsubsection*{}
\begin{strip}
\begin{tcolorbox}[colback=gray!10, colframe=gray!50, title= Annotation guidelines with Chain of thought (AG + COT)]
\label{sec:prompt_ag_cot}
{\ttfamily Given a set of evidence passages, and an answer, determine if the answer is fully supported by the evidence passages or not. Analyze each sentence of the answer carefully and verify that all information it contains is explicitly stated in or can be directly inferred from the evidence passages. \\

Output \textbf{"Not Supported"} if ANY of the following are true:\\

The answer contains any information not explicitly stated in or directly inferable from the passages.\\

The answer contradicts any information in the passages. \\

The answer introduces any new information not found in the passages. \\

The answer misrepresents or inaccurately paraphrases information from the passages. \\

The answer draws conclusions not logically supported by the given information. \\

The answer changes the level of certainty, specificity, or nuance from what is expressed in the passages. \\

The answer does not directly address the specific aspect asked about in the question. \\

The answer conflates or misrepresents separate pieces of information when summarizing multiple passages. \\

Output \textbf{Supported} otherwise.

write your reasoning in between the tags \textbf{<rationale></rationale>} and Provide your final answer in \textbf{<answer></answer>} tags. \\

Evidence Passages:
\begin{verbatim}

    {{loop.index}}: {{passage.text}}

Answer: {{answer segment}}
\end{verbatim}
Now provided your label directly as "Supported" or "Not Supported".
}
\end{tcolorbox}
\end{strip}


\section{More Results on Meta-Evaluators}
\begin{table*}[!t]
\begin{tabular}{@{}cccccc@{}}
\toprule
\multicolumn{1}{l}{} & EN                        & DE               & ES               & FR               & HI               \\ \midrule
GPT4o mini           & \textbf{68.37 $\pm$ 2.84} & 73.58 $\pm$ 2.86 & 69.84 $\pm$ 2.26 & 73.74 $\pm$ 2.29 & 74.10 $\pm$ 2.61 \\
Qwen 2.5 32B & 62.45 $\pm$ 3.00 & \textbf{76.79 $\pm$ 2.76} & \textbf{71.09 $\pm$ 1.98} & \textbf{74.38 $\pm$ 2.28} & \textbf{75.41 $\pm$ 2.72} \\
Llama 3.2 90B        & 62.60 $\pm$ 2.73          & 63.11 $\pm$ 2.70 & 63.36 $\pm$ 2.09 & 63.17 $\pm$ 1.97 & 75.05 $\pm$ 2.87 \\
Llama 3.2 11B        & 60.29 $\pm$ 2.84          & 59.91 $\pm$ 3.13 & 59.02 $\pm$ 2.51 & 65.02 $\pm$ 2.46 & 65.22 $\pm$ 2.75 \\ \bottomrule
\end{tabular}
\caption{Balanced Accuracy for five languages  and four LLMs using AG+COT prompting strategy. Standard errors are calculated using 1000 Bootstrap iterations.}
\label{tab:avg_llm_bacc_with_se}
\end{table*}

\begin{table*}[!htbp]
\begin{tabular}{@{}l
c
cccc@{}}
\toprule
\textbf{Prompt} & \textbf{EN}      & \textbf{DE}               & \textbf{ES}      & \textbf{FR}      & \textbf{HI}      \\ \midrule
ZS              & 57.22 {$\pm$} 2.14 & 61.54 $\pm$ 1.98          & 56.94 $\pm$ 1.26 & 60.00 $\pm$ 1.39 & 68.96 $\pm$ 2.22 \\
COT             & 60.71 $\pm$ 2.34 & 63.83 $\pm$ 2.15          & 60.61 $\pm$ 1.44 & 63.85 $\pm$ 1.53 & 69.68 $\pm$ 2.42 \\
AG              & 62.72 $\pm$ 2.24 & \textbf{68.87 $\pm$ 2.27} & 63.61 $\pm$ 1.36 & 68.32 $\pm$ 1.53 & 70.20 $\pm$ 2.02 \\
AG + COT & \textbf{63.42 $\pm$ 2.44} & 68.35 $\pm$ 2.37 & \textbf{65.83 $\pm$ 1.64} & \textbf{69.08 $\pm$ 1.71} & \textbf{72.44 $\pm$ 2.07} \\ \bottomrule
\end{tabular}
\caption{Balanced Accuracy for five languages using four different prompting strategies, averaged across four LLMs (\gptfomini, Llama 3.2 90B, Llama 3.2 11B, and \qwenthrirtytwob).  Standard errors are calculated using 1000 Bootstrap iterations.}
\label{tab:avg_prompt_bacc_with_se}
\end{table*}


\begin{table*}[!htbp]
\centering
\begin{tabular}{@{}ccS[table-format=2.1]S[table-format=2.1]S[table-format=2.1]S[table-format=2.1]@{}}
\toprule
\textbf{Language} & \textbf{Prompt} & \textbf{GPT4o mini} & \textbf{Llama 3.2 90B} & \textbf{Llama 3.2 11B} & \textbf{Qwen 2.5 32B} \\ \midrule
\multirow{4}{*}{EN} & ZS       & 59.84 & 58.00 & 51.00 & 60.07 \\
                    & COT      & 61.78 & 59.06 & 59.99 & 61.97 \\
                    & AG       & 69.95 & 59.40 & 52.97 & 68.54 \\
                    & AG + COT & 68.43 & 62.62 & 60.22 & 62.47 \\ \midrule
\multirow{4}{*}{DE} & ZS       & 61.10 & 58.60 & 54.84 & 71.89 \\
                    & COT      & 63.30 & 60.00 & 59.59 & 72.70 \\
                    & AG       & 71.99 & 68.60 & 60.64 & 74.64 \\
                    & AG + COT & 73.69 & 63.25 & 59.98 & 76.84 \\ \midrule
\multirow{4}{*}{ES} & ZS       & 53.78 & 56.76 & 53.65 & 63.65 \\
                    & COT      & 56.22 & 58.24 & 61.89 & 66.22 \\
                    & AG       & 69.73 & 59.59 & 54.32 & 70.81 \\
                    & AG + COT & 69.86 & 63.38 & 59.05 & 71.08 \\ \midrule
\multirow{4}{*}{FR} & ZS       & 59.54 & 56.31 & 55.99 & 68.45 \\
                    & COT      & 60.36 & 58.52 & 65.13 & 71.62 \\
                    & AG       & 74.47 & 62.41 & 60.28 & 76.24 \\
                    & AG + COT & 73.73 & 63.18 & 65.00 & 74.44 \\ \midrule
\multirow{4}{*}{HI} & ZS       & 72.23 & 64.33 & 66.12 & 73.27 \\
                    & COT      & 72.06 & 67.61 & 65.35 & 73.78 \\
                    & AG       & 73.78 & 67.10 & 65.52 & 74.53 \\
                    & AG + COT & 74.19 & 75.06 & 65.23 & 75.52 \\ \bottomrule
\end{tabular}
\caption{Meta evaluation results for all combination of LLMs and prompts we have tested.}
\label{tab:metaeval_all}
\end{table*}

\clearpage
\mbox{~}

\section{Fine grained Analysis}
\label{appendix:fine_grained_analysis}
\begin{table}[!t]
\centering
\small
% \begin{tabular}{@{}llll@{}}
\begin{tabular}{lr*{2}{S[table-format=2.1]}}
\toprule
\textbf{Fine-Grained Label} & \textbf{\#S} & \textbf{\gptfomini} & \textbf{\begin{tabular}[c]{@{}l@{}}Llama 3.2\\  90B\end{tabular}} \\ \midrule
Logical conclusion   & 151 & 80.13   & 87.42 \\
Direct paraphrase    & 182 & 93.41   & 97.25 \\
Other                & 2   & 50.00      & 50.00    \\ \midrule
We. Acc Sup.          &     & 87.16   & 92.54 \\ \midrule
Adds new information & 81  & 67.90    & 33.33 \\
Nuance shift         & 30  & 26.67   & 16.67 \\
Mis-referencing      & 20  & 35.00   & 20.00 \\
Contradiction        & 32  & 65.63   & 50.00    \\
Opinion as fact      & 12  & 66.67   & 25.00    \\
Other mistake        & 19  & 84.21   & 52.63 \\
Wrong reasoning      & 10  & 80.00      & 40.00    \\ \midrule
We. Acc NS.           &     & 60.29   & 33.82 \\ \midrule
BAcc                 &     & 73.73 & 63.18
\end{tabular}
\caption{Accuracy of \gptfomini and Llama 3.2 90B on various fine-grained labels, when evaluating French language using AG + COT prompt}
\label{tab:gpt_lama_fr_fine_grained}
\end{table}

To demonstrate the benchmark's capability for fine-grained analysis, Table \ref{tab:gpt_lama_fr_fine_grained} breaks down automatic evaluation performance by error type for two models: \gptfomini and Llama 3.2 90B, both evaluating French answers using AG + COT prompt. The upper section of the table shows the distribution of errors when the ground truth class is \textit{Supported}, categorized into logical conclusion, direct paraphrase, or other correct categories. The lower section details the types of mistakes made by each model for the \textit{Not supported} category. We observe that both models show a similar pattern for the \textit{Supported} category, with logical conclusions being the most common (69.8\% for \gptfomini and 76.0\% for Llama 3.2 90B), followed by direct paraphrases. For the \textit{Not supported} category, both models struggle most with detecting ``adding new information'' (32.1\% and 40.0\% of mistakes, respectively) and ``nuance shifts'' (27.2\% and 18.5\%). The ranking of error types is consistent across both models, despite their different overall BAcc (73.7\% for \gptfomini vs 63.2\% for Llama 3.2 90B see Table ~\ref{tab:metaeval_all} in Appendix), suggesting similar challenges in evaluation. The benchmark's fine-grained labeling system provides a detailed view of evaluation challenges across languages. By categorizing different types of faithfulness violations, it reveals which specific errors are harder to detect for each model, offering insights into both the strengths and limitations of automated evaluation methods.

\end{document}
