\section{Experiments}
In this section, we conduct extensive experiments to verify the effectiveness and efficiency of CSGM. 
% We aim to answer the following questions: \textbf{RQ1:} Can our methods discover accounts that are involved in money laundering activities? \textbf{RQ2:} How do the hyperparameters affect the performance of our methods? and \textbf{RQ3:} How efficient are our methods?


\begin{table*}[!ht]
    \centering
    \caption{Experiments on AMLSim and AMLWorld datasets. "-" represents that the metric is unsuitable for the method. We bold the best experimental results and underline the second-best results.}
    \label{tab:exp_effectiveness}
    \resizebox{\textwidth}{!}{
    \begin{tabular}{ccccccccccc}
        \toprule  % 顶部线
         & \multicolumn{5}{c}{\bal} & \multicolumn{5}{c}{\unb} \\
        \cmidrule(r){2-6} \cmidrule(r){7-11} 
        \textbf{Methods} & ACC & Precision & Recall & F1-score & AUC & ACC & Precision & Recall & F1-score & AUC \\
        SGM & 0.9761 & 0.8627  & 0.9047 & 0.8832 & \textbf{0.9743} & 0.9805 & 0.8665 & 0.9678 & 0.9144 & 0.9876\\
        GIN~\cite{xu2018powerful, hu2019strategies} & 0.9497$\pm$0.0021 & 0.8397$\pm$0.0096 & 0.8992$\pm$0.0033 & 0.8684$\pm$0.0047 &0.9301$\pm$0.0016 & 0.8978$\pm$0.0064 & 0.6926$\pm$0.0159 & 0.9045$\pm$0.0034 & 0.7844$\pm$0.0109 & 0.9003$\pm$0.0049 \\ 
        GAT~\cite{velivckovic2017graph} & 0.8332$\pm$0.0123 & 0.5285$\pm$0.0201 & 0.9173$\pm$0.0021 & 0.6704$\pm$0.0159 & 0.8657$\pm$0.0071 & 0.8235$\pm$0.0181 & 0.5424$\pm$0.0288 & 0.9251$\pm$0.0035 & 0.6835$\pm$0.0225 & 0.8611$\pm$0.0113 \\
        PNA~\cite{velickovic2019deep} & 0.9533$\pm$0.0017 & 0.8508$\pm$0.0078 & 0.9061$\pm$0.0018 & 0.8776$\pm$0.0039 & 0.9351$\pm$0.0011 & 0.9177$\pm$0.0043 & 0.7470$\pm$0.0125 & 0.9066$\pm$0.0022 & 0.8190$\pm$0.0079 & 0.9136$\pm$0.0032\\
        LaundroGraph~\cite{cardoso2022laundrograph} & 0.936$\pm$0.0014 & 0.7870$\pm$0.0048 & 0.8961$\pm$0.0034 & 0.8380$\pm$0.0031 & 0.9206$\pm$0.0018 & 0.9136$\pm$0.0043 & 0.7350$\pm$0.0126 & 0.9070$\pm$0.0077 & 0.812$\pm$0.0077 & 0.9112$\pm$0.0029 \\
        MultiGIN~\cite{egressy2024provably} & 0.9827$\pm$0.0003 & \textbf{0.9949$\pm$0.001} & \uline{0.9108$\pm$0.0016} & \uline{0.9510$\pm$0.0008} & 0.9549$\pm$0.0008 & 0.9809$\pm$0.0005 & \textbf{0.9955$\pm$0.0012} & \uline{0.9110$\pm$0.0018} & \uline{0.9514$\pm$0.0012} & \uline{0.9550$\pm$0.0009}\\
        \midrule
        Prob-CSGM & \uline{0.9858$\pm$0.0007} & \uline{0.9926 $\pm$0.0003} & 0.8638$\pm$0.0072 & 0.9237$\pm$0.0041 & - &\uline{0.9880$\pm$0.0019} & 0.9928$\pm$0.0010 & 0.8943$\pm$0.0176 & 0.9409 $\pm$0.0096 & - \\
        Sim-CSGM & 
        \textbf{0.9908$\pm$0.0006} & 0.9833$\pm$0.0012 & \textbf{0.9231$\pm$0.0068} & \textbf{0.9522$\pm$0.0033} & \uline{0.9607$\pm$0.0034} & \textbf{0.9964$\pm$0.0012} & \uline{0.9930$\pm$0.0002} & \textbf{0.9737$\pm$0.0114} & \textbf{0.9833$\pm$0.0058} & \textbf{0.9865$\pm$0.0057} \\
        \bottomrule 
        \toprule
        
         & \multicolumn{5}{c}{\hi} & \multicolumn{5}{c}{\li}\\
         \cmidrule(r){2-6} \cmidrule(r){7-11} 
          \textbf{Methods} & ACC & Precision & Recall & F1-score & AUC & ACC & Precision & Recall & F1-score & AUC\\
         \midrule
         SGM & 0.9992 & 0.5187  & 0.6501 & 0.5770 & 0.8250 & 0.9989 & 0.0314 & 0.1765 & 0.0533 & 0.5878 \\         
         GIN~\cite{xu2018powerful, hu2019strategies} & 0.9984$\pm$0.0004 & 0.2938$\pm$0.0781 & 0.5526$\pm$0.0892 & 0.3811$\pm$0.0828 & 0.7757$\pm$0.0447 & 0.9997$\pm$0.0001 & 0.1791$\pm$0.0350 & 0.1647$\pm$0.0738 & 0.1598$\pm$0.0474 & 0.5823$\pm$0.0369\\ 
         GAT~\cite{velivckovic2017graph} & 0.9992$\pm$0.0001 & 0.5572$\pm$0.1188 & 0.2143$\pm$0.0136 & 0.3081$\pm$0.0326 & 0.6071$\pm$0.0068 & 0.998 & 0.0 & 0.0 & 0.0 & 0.5 \\
         PNA~\cite{velickovic2019deep} & 0.9985$\pm$0.0001 & 0.3565$\pm$0.0208 & \textbf{0.9380$\pm$0.0097} & 0.5165$\pm$0.0234 & \textbf{0.9683$\pm$0.0049} & 0.9997$\pm$0.0001 & 0.3321$\pm$0.0052 & \uline{0.8873$\pm$0.0069} & \uline{0.4833$\pm$0.0065} & \uline{0.9435$\pm$0.0035}\\
         LaundroGraph~\cite{cardoso2022laundrograph} & 0.9992$\pm$0.0001 & 0.5412$\pm$0.0840 & 0.6193$\pm$0.0613 & 0.5710$\pm$0.0299 & 0.8094$\pm$0.0306 & 0.9998$\pm$0.0043 & 0.3846$\pm$0.0126 & 0.0490$\pm$0.0077 & 0.0870$\pm$0.0077 & 0.5245$\pm$0.0029 \\
         MultiGIN~\cite{egressy2024provably} & 0.9996$\pm$0.0002 & 0.6945$\pm$0.0959 & \uline{0.9366$\pm$0.0173} & \uline{0.7943$\pm$0.0658} & \uline{0.9681$\pm$0.0086} & 0.9996$\pm$0.0001 & 0.1746$\pm$0.0365 & 0.3353$\pm$0.2252 & 0.2104$\pm$0.1101 & 0.6675$\pm$0.1125\\
         \midrule
         Prob-CSGM & \uline{0.9996$\pm$0.0001} & \textbf{0.8747$\pm$0.0242} & 0.6413$\pm$0.0643 & 0.7392 $\pm$0.0499 & - &
        \uline{0.9998$\pm$0.0001} & \uline{0.4370$\pm$0.0684} & 0.3529$\pm$0.1038 & 0.3878 $\pm$0.086 & - \\
        Sim-CSGM & \textbf{0.9997$\pm$0.0001} & \uline{0.7718$\pm$0.0191} & 0.8292$\pm$0.0136 & \textbf{0.7995$\pm$0.0128} & 0.9145$\pm$0.0068& \textbf{0.9999 $\pm$0.0009} & \textbf{0.6458$\pm$0.0008} & \textbf{0.9118$\pm$0.0001} & \textbf{0.7561$\pm$0.0041} & \textbf{0.9558$\pm$0.0002} \\
        \bottomrule 
    \end{tabular}}
\end{table*}

\begin{figure*}
\centering
\begin{tabular}{cccc}
     % Accuracy & Precision & Recall & F1-score\\[-0.5ex]
    \subfloat{
        \includegraphics[width=0.23\textwidth,valign=c]{./figures/sim-amlsim-bal.png}} &
    \subfloat{
        \includegraphics[width=0.23\textwidth,valign=c]{./figures/sim-amlsim-unb.png}} &
    \subfloat{
        \includegraphics[width=0.23\textwidth,valign=c]{./figures/sim-amlworld-hi.png}} &
    \subfloat{
        \includegraphics[width=0.23\textwidth,valign=c]{./figures/sim-amlworld-li.png}}\\
        % \addlinespace
\end{tabular}
\caption{Experiments for the similarity-based method with the four synthetic datasets.}
\label{fig:sim_thd}
\end{figure*}

\subsection{Experimental setting}

\Paragraph{Dataset.}
To evaluate whether CSGM can detect money laundering activities in practice, we construct a real-world dataset called \textit{Alipay-ECB}, based on daily transactions recorded on Alipay~\cite{alipay} and E-Commerce Bank~\cite{ECB}. It comprises over 200 million accounts and 300 million transactions.
To the best of our knowledge, it is the largest transaction dataset that tracks currency flow in the real world, providing a comprehensive reflection of money laundering activities. Detailed information on the dataset is provided in Appendix~\ref{ssec:dataset}.
We also conduct experiments on synthetic datasets that simulate transactions for money laundering activities.
We utilize AMLSim~\cite{AMLSim} and AMLWorld~\cite{altman2024realistic}, which supports building a multi-agent simulator of anti-money laundering and has been widely used in previous works~\cite{karim2023catch,weber2018scalable,usman2023intelligent,egressy2024provably}.
For AMLSim~\cite{AMLSim}, we synthetic two datasets with \numprint{100000} nodes, and simulate two scenarios where two institutions have balanced transaction subgraphs or unbalanced subgraphs. We refer to $\textit{AMLSim\_bal}$ and $\textit{AMLSim\_unb}$, respectively. For AMLWorld~\cite{altman2024realistic}, we choose two datasets of size 5 million and 7 million for experiments. We refer to $\textit{AMLWorld\_HI}$ and $\textit{AMLWorld\_LI}$, where HI stands for relatively higher illicit rate and LI stands for lower illicit rate.
The statistics of the datasets are presented in Table~\ref{tab: datasets}.


\Paragraph{Baselines.}
\textbf{SGM} refers to the centralized scatter-gather mining method described in Section~\ref{ssec: MLSD}.
A line of research leverages graph neural networks (GNNs) for identifying money laundering transactions. \textbf{GIN}\cite{xu2018powerful, hu2019strategies}, \textbf{GAT}\cite{velivckovic2017graph}, and \textbf{PNA}\cite{velickovic2019deep} are commonly used GNN models for general graph classification tasks. Two additional studies propose GNNs specifically tailored for money laundering detection. \textbf{LaundroGraph}\cite{cardoso2022laundrograph} introduces a self-supervised graph representation learning method to detect money laundering. \textbf{MultiGIN}\cite{egressy2024provably}, incorporates a range of adaptations, including multigraph port numbering, ego IDs, and reverse message passing, to enhance GNNs' ability to detect various patterns of illicit activities. 

% \Paragraph{Metrics.}
% We use the following metrics to evaluate the performance of the algorithm.
\newenvironment{packeditemize}{
\begin{list}{$\bullet$}{
\setlength{\labelwidth}{8pt}
\setlength{\itemsep}{0pt}
\setlength{\leftmargin}{\labelwidth}
\addtolength{\leftmargin}{\labelsep}
\setlength{\parindent}{0pt}
\setlength{\listparindent}{\parindent}
\setlength{\parsep}{0pt}
\setlength{\topsep}{3pt}}}{\end{list}}

% \begin{packeditemize}
% \item \textbf{Accuracy (ACC)}, which is the fraction of correctly classified accounts involving money laundering accounts and normal accounts.
% \item \textbf{Precision}, which is the fraction of money laundering accounts among the detected accounts.
% \item \textbf{Recall}, which is the probability of the detected money laundering accounts, conditioned on the account truly being positive.
% \item \textbf{F1-score}, which is another measure of accuracy defined as $2\times \frac{\textit{precision}\times \textit{recall}}{(\textit{precision} + \textit{recall})}$.
% % \item \textbf{Set matching precision (SMP)}, which is the fraction of money laundering groups among the matched sets.
% \item \textbf{AUC}, which stands for "Area under the ROC Curve". This metric is not suitable for the probability-based method. The ROC curve is plotted by choosing all possible thresholds.
% \end{packeditemize}

\subsection{Experiments on Alipay-ECB.}
\Paragraph{Data process.}
We first preprocess the dataset, including account segregation, transaction aggregation, and transaction filtering to remove noise transactions from the dataset. As a result, we obtain a transaction graph with 48.95 million accounts and 34.45 million transactions. The detailed process is presented in Appendix~\ref{ssec:dataset}.


\Paragraph{Results.}
We experiment with similarity-based methods and set the threshold to $0.1$.
% We detect more than 1000 groups with the scatter-gather pattern. 
Examples of detected groups are presented in Figure~\ref{fig:detected_groups}.
We randomly selected 100 detected groups and evaluated them based on whether the accounts in each group had been reported as illegal~\footnote{For reasons related to corporate confidentiality and data safety, we regret that we cannot disclose the exact figures of detected groups.}.
% In addition, we evaluate each group based on transaction patterns, geographical information, and transaction time to determine whether our method can detect previously undiscovered nodes.

% Among the 100 detected groups, 59 were identified as money laundering groups, with more than half of its accounts having been identified as illicit. 2 were identified as non-money laundering groups, and the remaining 39 groups contained at least one illicit account. 
Among the 100 detected groups, 59 were identified as money laundering groups, with more than half of their accounts recognized as illicit in actual business operations. Two groups were identified as non-money laundering, while the remaining 39 groups contained at least one illicit account.
The results demonstrate that our method can effectively detect money laundering groups. With the group-level information, our methods would assist in uncovering previously undiscovered illicit accounts.
% Of the 42 undetermined groups, 21 include illicit accounts, 7 include abnormal transaction patterns, and 13 include abnormal geographical and transaction time clustering. 

\subsection{Experiments on synthetic datasets.}
We experiment with Prob-CSGM and Sim-CSGM on four synthetic datasets and compare them with all five baselines. 
We set the number of hash functions $m$ used in MinHash to $100$, and $r$ to $5$ for Prob-CSGM. 
For Sim-CSGM, $m$ is set to $100$, $r=2$, and the threshold is $0.2$ for $AMLSim$ and $0.3$ for $AMLWorld$.
The size of the Bloom filter is 500,000 bits for AMLSim and 3,000,000 bits for AMLWorld, resulting in a false positive probability of approximately $0.01$.

The results are shown in Table~\ref{tab:exp_effectiveness}. On the two AMLSim datasets, our methods have a performance comparable to SGM. Sim-CSGM, in particular, outperforms the centralized method in terms of recall, indicating that it is more effective at identifying abnormal nodes comprehensively. On the AMLWorld datasets, SGM exhibits low precision, with even poorer performance on AMLWorld-LI. This is because SGM tends to identify small transaction groups as money laundering groups, which is normal in AMLWorld-HI. However, Sim-CSGM can still identify money laundering groups, demonstrating its robustness.

Compared to GNN-based methods, our approach has a comparable performance on the AMLSim dataset, which achieves the best results with both precision and recall rates exceeding 90\%. However, when the proportion of illicit transactions is exceptionally low, such as in the AMLWorld-HI dataset, MultiGIN suffers from low precision, leading to a high false positive rate. In contrast, our methods maintain strong performance on the AMLWorld datasets, highlighting the generalizability of our approach.


% Focusing on precision, we observe that our methods, both Prob and Sim, surpass the centralized in precision, indicating a higher level of correctness in the money laundering accounts identified by our methods. 
% Moreover, the precision of Sim is higher than Prob, which is because those similar sets are only identified with a certain probability in Prob, whereas Sim can identify similar sets as long as their similarity is higher than the predefined threshold.
% In contrast, Sim exhibits higher recall than MLSD, indicating its proficiency in discovering similar sets.






% \subsection{Effectiveness of the banding technique.}
% To further study the effectiveness of reducing the number of repeated elements by concatenating a group of hash values into a new one when inserting them into Bloom filters, we first conduct experiments to see how the Bloom filter affects the performance when integrated into MinHash. In more detail, we compare the performance of our methods with the one that does not use the Bloom filter. 
% We vary the size of groups from 2 to 10 and report the F1-score for the probability-based methods, and the AUC for the similarity-based methods. The results are shown in Figure~\ref{}.



% To further study the effectiveness of reducing the number of repeated elements, we count the number of repeated elements before and after concatenating and present the results in Figure~\ref{}. The x-axis represents the number of repeated elements, and the y-axis represents the frequency. It shows that by concatenating the hash values, we can greatly reduce the number of repeated elements, which aligns with our theoretical analysis. Meanwhile, with the group size getting bigger, the number of repeated elements is getting less.
% Moreover, combined with the experimental results presented in Table~\ref{tab: exp_effectiveness}, it shows that it is not required to make all elements different to achieve a good performance, demonstrating the robustness of our algorithm.

% \begin{figure}
% \begin{center}
% \centerline{
%     \includegraphics[width=0.46\textwidth]{figures/example4sgd.png}}
% \caption{Example of discovering money laundering group by treating $A$ as the sender. The red number represents the marked money, and the numbers in the box represent the money the nodes possess, and the numbers with no box represent the money in a transaction.}
% \label{fig: example4sgd}
% \end{center}
% \end{figure}


% \begin{figure*}[!htbp]
% \centering
% % \setlength\tabcolsep{1.0pt}
% % \renewcommand{\arraystretch}{0}
% % \begin{tabular}{cccc}
%      % Accuracy & Precision & Recall & F1-score\\[-0.5ex]
%     \subfloat{
%         \includegraphics[width=0.24\linewidth,valign=c]{./figures/prob_fix_hash_accuracy.png}} 
%     \subfloat{
%         \includegraphics[width=0.24\linewidth,valign=c]{./figures/prob_fix_hash_precision.png}}
%     \subfloat{
%         \includegraphics[width=0.24\linewidth,valign=c]{./figures/prob_fix_hash_recall.png}} 
%     \subfloat{
%         \includegraphics[width=0.24\linewidth,valign=c]{./figures/prob_fix_hash_f1.png}}
% % \end{tabular}
% \caption{Experiments for the probability-based method. Fix $M=120$ and range $r$ from 2 to 6.}
% \label{fig:prob_fix_hash}
% \end{figure*}




\subsection{Ablation study.}
% We then vary the choice of the number of hash functions used in MinHash $M$, and the number of rows in each band $r$ to study how they impact the performance of our algorithm.
To explore the impact of different parameters on the performance of our methods, we conducted experiments by varying the number of hash functions used in MinHash $m$, the number of rows $r$ in each band as well as the threshold. 



% We first focus on the probability-based method and experiment on the two datasets. We fix $M$ to 120 and vary $r$ from 2 to 6. Such a combination of parameters allows two sets of the similarity $0.5$ to have a probability of 99.99\%, 99.52\%, 85.57\%, 53.33\%, and 27.01\% to be discovered, respectively. 
% We report all available metrics, including accuracy, precision, recall, and f1-score. The results are presented in Figure~\ref{fig:prob_fix_hash}. It shows the precision increases as $r$ becomes larger, while recall decreases. This is because the probability of sets with a specific similarity to be discovered diminishes as $r$ increases. Consequently, only sets with higher similarity, which is more likely to be involved in money laundering activities, are discovered. This leads to an increase in precision. However, sets with lower similarity, which could still be associated with money laundering, may remain undetected, resulting in a decreased recall. 
% The results also show that 

Here, we mainly focus on Sim-CSGM. 
% The experimental results of Prob-CSGM are presented in Appendix.
. We vary the threshold from 0.2 to 0.6 and observe the change of F1-score with different $r$.
% We further conduct experiments to study how the threshold affects the performance of our algorithm in similarity-based methods. We range the threshold from 0.2 to 0.6 and fix the number of bands to 160. We also experiment with choosing different $r$. 
The results are depicted in Figure~\ref{fig:sim_thd}.
It shows that the F1-score when $r>1$ performs better than when $r=1$, showing the effectiveness of the banding technique in the similarity-based method. Furthermore, when $r=1$, the method prefers a higher threshold, illustrating that repeated elements in a band lead to an overestimation of similarity when using the bloom filter. 

Additionally, experiments in Appendix~\ref{ssec:rows_r} show that the banding technique could significantly reduce the number of repeated elements. We also evaluate the efficiency of our methods in terms of the communication costs as well as the running time in Appendix~\ref{ssec:efficiency}. The results show that our methods take only a few minutes.

% \begin{figure*}[!htbp]
% \centering
% % \setlength\tabcolsep{1.0pt}
% % \renewcommand{\arraystretch}{0}
% \begin{tabular}{cccc}
%      % Accuracy & Precision & Recall & F1-score\\[-0.5ex]
%     \subfloat{
%         \includegraphics[width=0.4\textwidth,valign=c]{./figures/sim_thd_accuracy_bal.png}} 
%     \subfloat{
%         \includegraphics[width=0.4\textwidth,valign=c]{./figures/sim_thd_accuracy_unb.png}} \\
%     \subfloat{
%         \includegraphics[width=0.4\textwidth,valign=c]{./figures/sim_thd_precision_bal.png}}
%     \subfloat{
%         \includegraphics[width=0.4\textwidth,valign=c]{./figures/sim_thd_precision_unb.png}} \\
%     \subfloat{
%         \includegraphics[width=0.4\textwidth,valign=c]{./figures/sim_thd_recall_bal.png}}
%     \subfloat{
%         \includegraphics[width=0.4\textwidth,valign=c]{./figures/sim_thd_recall_unb.png}} \\
%     \subfloat{
%         \includegraphics[width=0.4\textwidth,valign=c]{./figures/sim_thd_f1_bal.png}}
%         % \addlinespace
%     \subfloat{
%         \includegraphics[width=0.4\textwidth,valign=c]{./figures/sim_thd_f1_unb.png}}
% \end{tabular}
% \caption{Experiments for the similarity-based method with the $\textit{Syn\_bal}$ (left) and $\textit{Syn\_unb}$ (right). The number of bands is fixed to 160.}
% \label{fig:sim_thd}
% \end{figure*}


