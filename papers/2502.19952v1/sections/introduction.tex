\section{Introduction}
% Money laundering (money laundering) is a pervasive issue that involves the process of disguising, concealing, and converting funds obtained from illegal activities into legitimate assets. Criminals engaged in money laundering employ various techniques, including tax evasion, illicit commodity trading, and fraudulent loan acquisition, among others, to obfuscate the origins of illicit funds. 
% % The scale of this problem is staggering, with approximately 1.6 trillion USD, equivalent to 2.7\% of the global GDP, being laundered annually~\cite{frumerie2021money, unitednations}.
% Moreover, as electronic commerce and online transactions continue to expand, the prevalence of money laundering crimes is projected to increase~\cite{zhang2003applying}.
% Through the execution of multiple money transfers within or between financial institutions, a multitude of transactions is employed as a means to conceal illicit assets, thereby significantly impeding detection efforts.
Money laundering is a process that attempts to conceal or disguise the origins of dirty money derived from illicit activities, making it appear as if the funds have been obtained through legitimate means~\cite{aml}.
%Although the complicity of money laundering, yet generally, 
It typically consists of three primary steps: a \textit{placement} step first introduces the dirty money into existing financial systems; 
a \textit{layering} step then carries out complex transactions to hide the source of the funds; 
and a \textit{integration} step withdraws the fund from a destination bank account before using it for legitimate activities~\cite{unitednations}.
The transaction relationship of accounts can be represented as a graph, where an individual account is denoted as a node, and transactions between two accounts are denoted as edges. Due to the distinctive nature of money laundering activities, the transaction graph associated with money launderers exhibits a unique pattern known as \textbf{scatter-gather}~\cite{michalak2011graph, altman2024realistic, egressy2024provably}, as illustrated in Fig.~\ref{fig: topology1}.
% shows a  money laundering transaction graph in a common {\em scatter-gather} pattern~\cite{michalak2011graph, karim2023catch, altman2024realistic, egressy2024provably}. %In these typologies, the sender and receiver represent the placement and integration stages, respectively. The intermediaries signify the layering phase, where the fund is systematically transferred among intermediate nodes, either by being divided into multiple transactions.


\begin{figure}[t]
\centering
\subfloat[]{
    \label{fig: topology1}
    \includegraphics[width=0.48\linewidth]{figures/topology1.pdf}}~~~~~~~~~~~
\subfloat[]{
    \label{fig: topology2}
    \includegraphics[width=0.48\linewidth]{figures/topology_inst.pdf}}
\caption{(a) Scatter-gather pattern  money laundering; (b) Scatter-gather distributed across two institutions.}
\end{figure}

%Anti-money laundering (AML) refers to the task of preventing criminals from moving illicit funds through the financial system. 
It is the responsibility of financial institutions to conduct {\em anti-money laundering} (AML): 
diligently monitor transactions, take necessary actions like shutting down or imposing restrictions on suspicious accounts, and promptly report any suspicious activities through %Suspicious Activities Reports (SARs) 
to law enforcement agencies. 
% Numerous efforts have been dedicated to detecting money laundering activities within transaction systems. Among them, rule-based methods have gained significant popularity~\cite{rajput2014ontology}. The monitoring system detects suspicious transactions based on pre-defined rules, such as instances where a substantial amount of money is debited immediately after being credited to an account. However, rule-based algorithms are easy to be evaded by fraudsters.
% % Machine learning algorithms are also applied for detecting money laundering activities. 
% With the development of machine learning, a number of other approaches also have been proposed, such as radial basis function (RBF) neural networks~\cite{lv2008rbf}, support vector machines (SVM)~\cite{tang2005developing}, and multi-layer perception (MLP)~\cite{wu2020comprehensive}.  Some earlier approaches~\cite{awasthi2012clustering,le2010application} also employed clustering-based methods to detect money laundering activities by grouping transactions into clusters. 
% While these methods have demonstrated effectiveness, they pose significant challenges when dealing with vast transaction graphs containing hundreds of millions of vertices and edges. The scalability and performance of these methods in such scenarios remain unclear, especially considering the substantial computational costs involved.
To detect money laundering activities, a common idea is to identify the ultimate beneficiary, which refers to the individual or entity that ultimately receives the funds, even if those funds have been obscured through multiple layers of transactions~\cite{aml}. 
% is a crucial strategy for detecting money laundering activities. This approach aims to uncover the individual or entity that ultimately receives the funds, even when the funds have been obscured through multiple layers of transactions~\cite{aml}. 
To achieve that, a simple approach is to calculate the ratio to which funds in one account originate from another account~\cite{michalak2011graph}. 
If the ratio exceeds a predefined threshold, it indicates a potential association between the two accounts, raising suspicions of money laundering activities with one account being the source and the other the destination.
%This is achieved by propagating funds along the transaction graph, following the direction of fund flow. 
%We refer to this approach as \textit{money laundering group discovery} (MLGD). 
% The rationale behind the method can be observed in Figure~\ref{fig: topology}, where it is evident that funds in intermediaries and receiver nodes can be traced back to the sender node.


However, money laundering has evolved into a highly sophisticated process, spanning across multiple financial institutions s.t. the subgraph within one institution appears to be normal (Fig.~\ref{fig: topology2}). 
As a result, relying solely on the transaction graph within a single institution for AML is no longer sufficient. 
A straightforward solution is to combine the transaction graphs from multiple institutions. However, due to regulatory, legal, commercial, and customer privacy concerns, institutions tend not to share data.

% However, money laundering activities often encompass multiple financial institutions, such as banks and mobile payment platforms, resulting in each institution possessing only a portion of the overall money laundering transaction graph. The sharing of transaction graphs among institutions is hindered due to privacy and policy concerns, further complicating the task of money laundering group detection for an institution that owns a fragmented graph. Additionally, it is common that the sender and the receiver within one group are distributed across different institutions, resulting in a situation where the subgraph within one institution appears to be normal.

\Paragraph{Our contribution.}
In this paper, we make the \textit{first} step towards collaborative AML,
which allows multiple institutions to jointly conduct AML without exposing their individual transaction graphs.

Our primary contribution lies in the introduction of a novel algorithm for scatter-gather subgraph mining, specifically tailored to suit the collaborative setting. 
% The key idea is that the set of cross-institution transactions scattered from the source are identical to transactions gathered at the destination, as long as the source and destination are involved in the same money laundering subgraph.
In more detail, this algorithm first employs a breadth-first search (BFS) approach for each node to identify a set of cross-institution transactions associated with that node, which can be either scattered from or gathered towards the node.
If two nodes, belonging to different institutions, share the same set of cross-institution transactions, it indicates a potential scatter-gather relationship within a money laundering subgraph, with one node being the source and the other being the destination.
Building upon this observation, the algorithm considers two institutions, denoted by $\Cli_A$ and $\Cli_B$, and iterates through their respective nodes ($\{N_1^A, N_2^A, \ldots, N_n^A\}$ and $\{N_1^B, N_2^B, \ldots, N_n^B\}$) to identify the sets of cross-institution transactions: $\SSS^A = \{S_1^A, S_2^A, \ldots, S_n^A\}$ and $\SSS^B = \{S_1^B, S_2^B, \ldots, S_n^B\}$, where e.g., $S_i^A$ is the set of cross-institution transactions associated with node $N_i^A$. 
If two sets $S_i^A$ and $S_j^B$ exhibit a high degree of similarity, it suggests that $N_i^A$ and $N_j^B$ are potentially involved in scatter-gather activities within a money laundering subgraph.

This approach requires $\Cli_A$ and $\Cli_B$ to exchange $\SSS^A$ and $\SSS^B$, and measure the similarity between each pair (e.g., $S_i^A$ and $S_j^B$). This is costly in terms of both communication and computation.
To solve the problem, we use locality-sensitive hashing (LSH)~\cite{lsh} and Bloom filter~\cite{bloomfilter} to minimize the amount of information to be exchanged between $\Cli_A$ and $\Cli_B$. 
LSH enables the estimation of similarity between two sets by comparing the minimum hash values of their elements. Combined with Bloom filters, the approach transforms pairwise comparisons into a process of testing the presence of an element within a Bloom filter.  The Bloom filter is memory-efficient, and this testing process is computationally efficient.


Specifically, an LSH is computed for each set, resulting in $\{\lsh_1^A, \lsh_2^A, \allowbreak\ldots, \lsh_n^A\}$ and $\{\lsh_1^B, \lsh_2^B, \ldots, \lsh_n^B\}$.
Notice that $\lsh_i^A=\lsh_j^B$ if $S_i^A$ and $S_j^B$ exhibit a high degree of similarity.
Next, one institution, say $\Cli_A$, inserts $\{\lsh_1^A, \lsh_2^A, \ldots, \lsh_n^A\}$ into a bloom filter $BF_A$, and transfers $BF_A$ to $\Cli_B$;
$\Cli_B$ iterates through $\{\lsh_1^B, \lsh_2^B, \ldots, \lsh_n^B\}$ to check if each $\lsh^B$ is present in $BF_A$.
If $\lsh_j^B$ is found in $BF_A$, $\Cli_B$ learns that $N_j$ is one end node in the scatter-gather activity. 
At this stage, $\Cli_B$ reveals the corresponding $\lsh_j^B$ to $\Cli_A$, enabling $\Cli_A$ to identify the other end node in the scatter-gather activity.
By leveraging this optimization,  the communication overhead is significantly reduced as it only requires the transfer of a bloom filter. 
Moreover, by comparing against a bloom filter, the computational complexity is reduced to $O(n)$, rather than $O(n^2)$ when comparing each pair individually. 


To evaluate whether our methods can detect money laundering activities across multiple institutions in a real-world setting, we construct Alipay-ECB, a multi-institution transaction dataset that includes digital currency transactions from Alipay and E-Commerce Bank (ECB) users. The dataset contains over 200 million accounts and 300 million transactions. To the best of our knowledge, it is the largest real-world transaction dataset available. 

By analyzing the dataset, we find that money laundering groups possess a much more intricate structure in real-world settings, encompassing multiple simple patterns such as fan-in, fan-out, cycles, random, and bipartite, etc. However, our method can effectively identify money laundering subgroups. Experiments on synthetic datasets also demonstrate our methods can effectively and efficiently identify money laundering subgroups.




% Experiments on Alipay-ECB show that money laundering groups possess a much more intricate structure in real-world settings, encompassing multiple simple patterns such as fan-in, fan-out, cycles, random, and bipartite, etc. Despite the complexity, money laundering groups basically  

% This demonstrates the effectiveness of our method in studying the scatter-gather pattern, which aligns well with the three stages of money laundering activities.
% Our experimental results show that our methods can effectively and efficiently identify money laundering subgroups. 
% Experiments on synthetic datasets also demonstrate our methods can effectively and efficiently identify money laundering subgroups.






% Specifically, we propose a collaborative AML protocol that enables the discovery of money laundering groups within graphs owned by multiple institutions while ensuring privacy and efficiency. The heart of the framework is a novel distributed MLGD algorithm, which not only propagates funds with the direction of fund flow but also propagates in the inverse direction. We name it \textbf{distribute-MLGD}.
% In more detail, our approach involves the recursive propagation of funds from a given node in a specified direction. When a transaction occurs between two nodes that are distributed in different institutions, we incorporate these nodes into a set until no new nodes are included or propagated to the $K$-hop neighbor nodes. Finally, we establish sets for each node and the designated direction. Moreover, we construct two sets for all nodes, representing forward and backward propagation, respectively. The key observation is that the sender's forward set overlaps with the receiver's backward set of when both nodes belong to the same money laundering group. Based on this premise, we propose that institutions can share their sets with one another, allowing for the discovery of AML groups by comparing the sets constructed on local graphs with the received sets. If a pair of sets exhibit a high degree of similarity (overlap mostly), it indicates the involvement of the source node of the set in money laundering activities. 
%This process is carried out collaboratively between institutions, enabling each institution to identify money laundering accounts within its own domain. 


% To reduce communication costs induced by the transformed algorithm, our algorithm makes use of MinHash~\cite{minhash} and Bloom filter~\cite{bloomfilter} techniques. Specifically, we first hash each set into a vector of fixed length and insert them into Bloom filters such that the communication costs are significantly reduced to only $\mathcal{O}(M)$, where $K_m$ is the number of hash functions used in MinHash, and $M$ is the bits of a bloom filter. By applying MinHash and the Bloom filter, the algorithm also ensures that the information of accounts within the institution, as well as the transactions between them, would not leak to other institutions.
% % However, directly integrating the Bloom filter into MinHash may degrade the performance of our algorithm, as the Bloom filter loses the positional information when inserting a list of data into it. 
% % We provide a comprehensive analysis of the problem and propose two methods.
% Experimental results on synthetic datasets show the effectiveness of our algorithm in detecting
% money-laundering accounts.



% \textbf{Notations.} The frequently used notations are summarized in Table~\ref{tab:notations} in appendix.
