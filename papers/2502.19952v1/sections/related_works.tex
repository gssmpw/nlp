\section{Related works}
The term money laundering was first used at the beginning of the 20th Century to label the operations that in some way intended to legalize the income derived from illicit activity,  thus facilitating their entry into the monetary flow of the economy~\cite{AMLDef}. Since then, numerous methods have been proposed to identify money laundering activities~\cite{hooi2016fraudar,le2010application,michalak2011graph,rajput2014ontology,soltani2016new,starnini2021smurf,zhang2003applying}.  
Rule-based approaches were first widely used in the early days~\cite{michalak2011graph,rajput2014ontology}. Rajput et al.~\cite{rajput2014ontology} propose an ontology-based expert system to detect suspicious transactions, and Michalak et al.~\cite{michalak2011graph} propose a method that integrates the fuzzing method and decision rules to detect suspicious transactions. 
Although easy to deploy, rule-based methods can easily be evaded by fraudsters.

With the popularity of machine learning, learning-based methods have become an emergency. Tang et al.~\cite{tang2005developing} propose to use the support vector machine method (SVM) to detect unusual behaviors in transactions. 
% \cite{awasthi2012clustering} and \cite{le2010application} using the clustering method for grouping transactions with bank accounts in different clusters that have the most similarities with each other. Decision tree-based methods are also combined with clustering to detect money laundering~\cite{liu2011research}.
% Michalak et al. and Chen et al.~\cite{chen2011fuzzy,michalak2011graph} use fuzzy matching to catch subgraphs that may contain suspicious accounts.
Lv et al.~\cite{lv2008rbf} judge whether the capital flow is involved in money laundering activities using RBF neural networks calculated from time to time. Paula et al.~\cite{paula2016deep} also show some success for AML by using deep neural networks. However, these methods detect money laundering activities in a supervised manner, suffering from highly skewed labels and limited adaptability.
% Recently, Li et al.~\cite{li2020flowscope} propose a metric to evaluate the anomalousness of a subgraph induced by a subset of nodes and propose an algorithm to find subsets that maximize the metric. The subsets are treated as suspicious money laundering groups.

Graphs have the advantage of better characterizing the association between objects. Many graph-based anomaly detection techniques have been developed for discovering structural anomalies. 
Zhang et al.~\cite{zhang2003applying} use financial transaction networks and community detection algorithms to find money laundering groups.
% Michalak et al.~\cite{michalak2011graph} propose a graph-mining algorithm to detect money laundering activities.
% Soltani et al.~\cite{hooi2016fraudar} proposed a suspiciousness metric that considers the density of the subgraph. \cite{soltani2016new} find transactions involved in money laundering activities based on the structural similarity of subgraphs that the transactions are composed of. 
Cardoso et al.~\cite{cardoso2022laundrograph} introduces a self-supervised graph representation learning method aimed at detecting money laundering. Recently, Béni et al.~\cite{egressy2024provably}, incorporates a range of adaptations, including multigraph port numbering, ego IDs, and reverse message passing, to enhance GNNs' ability to detect various patterns of illicit activities. 

Despite the advance of all those methods, they work based on the prerequisite that the transaction graph is centralized, while in practice, money laundering activities span across multiple institutions s.t. the transaction subgraph within one institution appears to be normal.
% However, the transaction graph of money laundering activities may not be complete in one institution, which degrades the performance of those methods. 
Our methods make the \textit{first} steps towards collaborative anti-money laundering among institutions without exposing the transaction graphs.