\section{Problem Statement}
\label{sec: problem_statement}

\begin{figure*}[t]
\begin{center}
\centerline{
    \includegraphics[width=0.85\linewidth]{figures/workflow.pdf}}
\caption{Workflow of CSGM. The dotted lines on the graphs indicate cross-institution transactions.}
\label{fig:workflow}
\end{center}
\end{figure*}

Let $\mathcal{G=(V,E,X)}$ be a money transaction graph, where $\mathcal{V}$ is the vertex set represents accounts, $\mathcal{E}$ is the edge set represents transactions, and $\mathcal{X} \in \mathbb{R}^d$ is the feature matrix of all edges. 
An edge $(i, j) \in \mathcal{E}$ indicates that the account $i$ transfers money to $j$ and the corresponding $\mathbf{x}\in\mathcal{X}$ indicates the attributes of the transaction, such as the amount of money, the time, to name a few. In this paper, we mainly focus on two attributes: the amount of money and whether the transaction is an external transaction, denoted as $a$ and $c$ separately. Specifically, $\mathbf{x} = [a, c]^\top$. For ease of presentation, we denote $\mathbf{x}_{i\rightarrow j}$ the attributes for the transaction from $i$ to $j$. 


In our setting of collaborative learning, we consider two institutions $\Cli_A$ and $\Cli_B$; each holds a subgraph $\mathcal{G}_A = (\mathcal{V}_A,\mathcal{E}_A,\mathcal{X}_A)$ and $\mathcal{G}_B = (\mathcal{V}_B,\mathcal{E}_B,\mathcal{X}_B)$, where $\mathcal{V}_i,\mathcal{E}_i,\mathcal{X}_i$ are subsets of $\mathcal{V},\mathcal{E},\mathcal{X}$, separately.
In the rest of the paper, we use the notations $p$ and $q$ to denote the indices of the two institutions. Specifically, \(\Cli_p\) refers to one institution and \(\Cli_q\) to the other.

To comply with Know Your Customer (KYC) standards~\cite{kyc}, financial institutions are required to gather basic information about both the initiator and recipient of each transaction. This rule remains applicable even when accounts are held across different institutions. Based on this requirement, we assume an overlap between $\mathcal{V}_A$ and $\mathcal{V}_B$. The overlapping nodes represent accounts involved in cross-institution transactions between $\Cli_A$ and $\Cli_B$.

% Specifically, let \(\mathcal{V}_p\) denote the set of accounts associated with institution \(p\), where \(\mathcal{V}_p = \mathcal{I}_p \cup \mathcal{O}_p\), with \(\mathcal{I}_p\) representing the internal accounts of the institution and \(\mathcal{O}_p\) representing the external accounts involved in transactions with the institution. Our assumption implies that \(\mathcal{O}_B \subseteq \mathcal{I}_A\) and simultaneously, \(\mathcal{O}_A \subseteq \mathcal{I}_B\). 
We further assume that the overlapping accounts are recorded with identical identifiers by both institutions. This identification can be performed privately through multi-party private set intersection methods~\cite{kolesnikov2017practical}, which is orthogonal to our paper.

Given the above setting, we aim to discover money laundering groups of typologies presented in figure~\ref{fig: topology1} based on two subgraphs
$\mathcal{V}_A$ and $\mathcal{V}_B$.

% In particular, we mainly target the first typology as it is more challenging to detect in practice. The second typology is more easily detected as large-sum transactions are suffering more restricted supervision~\cite{paymentsystemsUS,taxguidelinesUS}.




