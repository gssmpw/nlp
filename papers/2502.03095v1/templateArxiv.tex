\documentclass{article}

\usepackage{microtype}
\usepackage{graphicx}
\usepackage{subfigure}
\usepackage{booktabs} % for professional tables

\usepackage{threeparttable}
% hyperref makes hyperlinks in the resulting PDF.
% If your build breaks (sometimes temporarily if a hyperlink spans a page)
% please comment out the following usepackage line and replace
% \usepackage{icml2024} with \usepackage[nohyperref]{icml2024} above.
\usepackage{amsmath}
\usepackage{xcolor}
\usepackage{amssymb}
\usepackage{graphicx}
\usepackage{booktabs}
\usepackage[colorlinks=true, allcolors=blue]{hyperref}
\definecolor{red}{HTML}{D9423C}
\definecolor{purple}{HTML}{854C98}

% hyperref makes hyperlinks in the resulting PDF.
% If your build breaks (sometimes temporarily if a hyperlink spans a page)
% please comment out the following usepackage line and replace
% \usepackage{icml2025} with \usepackage[nohyperref]{icml2025} above.
\usepackage{hyperref}

\newcommand{\yi}[1]{\textcolor{red}{#1}}
\newcommand{\yue}[1]{\textcolor{blue}{#1}}
\newcommand{\xuerui}[1]{\textcolor{green}{#1}}

% Attempt to make hyperref and algorithmic work together better:
\newcommand{\theHalgorithm}{\arabic{algorithm}}

% For theorems and such
\usepackage{amsmath}
\usepackage{amssymb}
\usepackage{mathtools}
\usepackage{amsthm}

% if you use cleveref..
\usepackage[capitalize,noabbrev]{cleveref}

%%%%%%%%%%%%%%%%%%%%%%%%%%%%%%%%
% THEOREMS
%%%%%%%%%%%%%%%%%%%%%%%%%%%%%%%%
\theoremstyle{plain}
\newtheorem{theorem}{Theorem}[section]
\newtheorem{proposition}[theorem]{Proposition}
\newtheorem{lemma}[theorem]{Lemma}
\newtheorem{corollary}[theorem]{Corollary}
\theoremstyle{definition}
\newtheorem{definition}[theorem]{Definition}
\newtheorem{assumption}[theorem]{Assumption}
\theoremstyle{remark}
\newtheorem{remark}[theorem]{Remark}
\newcommand\numberthis{\addtocounter{equation}{1}\tag{\theequation}}




\usepackage{PRIMEarxiv}

\usepackage[utf8]{inputenc} % allow utf-8 input
\usepackage[T1]{fontenc}    % use 8-bit T1 fonts
\usepackage{hyperref}       % hyperlinks
\usepackage{url}            % simple URL typesetting
\usepackage{booktabs}       % professional-quality tables
\usepackage{amsfonts}       % blackboard math symbols
\usepackage{nicefrac}       % compact symbols for 1/2, etc.
\usepackage{microtype}      % microtypography
\usepackage{lipsum}
\usepackage{fancyhdr}       % header
\usepackage{graphicx}       % graphics
\graphicspath{{media/}}     % organize your images and other figures under media/ folder

%Header
\pagestyle{fancy}
\thispagestyle{empty}
\rhead{ \textit{ }} 

% Update your Headers here
\fancyhead[LO]{Reveal the Mystery of DPO: The Connection between DPO and RL Algorithms}
% \fancyhead[RE]{Firstauthor and Secondauthor} % Firstauthor et al. if more than 2 - must use \documentclass[twoside]{article}


%% Title
\title{Reveal the Mystery of DPO: The Connection between DPO and RL Algorithms 
}


\author{
  Xuerui Su\thanks{These authors contributed equally to this work.} \\
School of Mathematics and Statistics \\
Beijing Jiaotong University \\
\texttt{24110486@bjtu.edu.cn} \\
\And
Yue Wang\footnotemark[1] \\
Independent Researcher \\
\texttt{yuewang\_yw@foxmail.com} \\
   \And
   Jinhua Zhu \\
  University of Science and Technology of China \\
\texttt{teslazhu@mail.ustc.edu.cn} \\
   \And
   Mingyang Yi \\
  School of information \\
  Renmin University of China \\
  \texttt{yimingyang@ruc.edu.cn} \\
   \And
   Feng Xu \\
  School of Management \\
  Fudan University \\
  \texttt{fxu23@m.fudan.edu.cn} \\
   \And
  Zhiming Ma \\
  Academy of Mathematics and Systems Science \\
  \texttt{mazm@amt.ac.cn} \\
   \And
  Yuting Liu\thanks{Corresponding Author}  \\
School of Mathematics and Statistics \\
  Beijing Jiaotong University \\
  \texttt{ytliu@bjtu.edu.cn} \\
  %% \AND
  %% Coauthor \\
  %% Affiliation \\
  %% Address \\
  %% \texttt{email} \\
  %% \And
  %% Coauthor \\
  %% Affiliation \\
  %% Address \\
  %% \texttt{email} \\
  %% \And
  %% Coauthor \\
  %% Affiliation \\
  %% Address \\
  %% \texttt{email} \\
}

\renewcommand{\thefootnote}{\fnsymbol{footnote}}
\begin{document}
\footnotetext[1]{These authors contributed equally to this work.}
\maketitle

\begin{abstract}
With the rapid development of Large Language Models (LLMs), numerous Reinforcement Learning from Human Feedback (RLHF) algorithms have been introduced to improve model safety, and alignment with human preferences. These algorithms can be divided into two main frameworks based on whether they require an explicit reward (or value) function for training: actor-critic-based Proximal Policy Optimization (PPO) and alignment-based Direct Preference Optimization (DPO). The mismatch between DPO and PPO, such as DPO’s use of a classification loss driven by human-preferred data, has raised confusion about whether DPO should be classified as a Reinforcement Learning (RL) algorithm. To address these ambiguities, we focus on three key aspects related to DPO, RL, and other RLHF algorithms: (1) the construction of the loss function; (2) the target distribution at which the algorithm converges; (3) the impact of key components within the loss function. Specifically, we first establish a unified framework named UDRRA connecting these algorithms based on the construction of their loss functions. Next, we uncover their target policy distributions within this framework. Finally, we investigate the critical components of DPO to understand their impact on the convergence rate. Our work provides a deeper understanding of the relationship between DPO, RL, and other RLHF algorithms, offering new insights for improving existing algorithms.
\end{abstract}


% keywords can be removed
\keywords{LLMs, RLHF, DPO, Soft Policy Iteration, Boltzmann Distribution, Offline Dataset Design}



\section{Introduction}
\label{Intro}


Large Language Models (LLMs) are considered one of the most promising advancements toward Artificial General Intelligence (AGI) \cite{zhao2023survey, ouyang2022training, minaee2024large, achiam2023gpt}. In practice, post-training the LLM with Reinforcement Learning from Human Feedback (RLHF) has significantly improved its safety, compliance, and the alignment with human preferences \cite{arumugam2019deep, singh2022flava, bai2022training, dai2023safe}.

{While well-established RLHF approaches, such as the Proximal Policy Optimization (PPO-RLHF) \cite{PPO-basedRLHF}, have demonstrated significant success, Direct Preference Optimization (DPO) \cite{DPO} has also emerged and garnered widespread attention. DPO differs from PPO in that it does not require explicit modeling of the reward function, making it simpler and more efficient in practice. Instead, DPO directly optimizes the language model to align with human preferences by minimizing a simple classification loss function, which is based solely on the log probability of outputs generated by the LLMs. Since its inception, DPO has gained considerable traction, leading to the development of several derivatives, such as IPO \cite{IPO} and DRO \cite{DRO_Deepmind}, among others.}

Despite its practical success, DPO differs significantly from typical Reinforcement Learning (RL) algorithms, particularly in its underlying algorithmic structure. While most RL algorithms, such as PPO-RL \cite{PPO}, rely on an actor-critic framework and explicitly model value functions, DPO takes a distinct approach. Under a Bradly-Terry \cite{bradley1952rank} framework to modeling human preference, DPO transferred training policy model into minimizing a human-preference labeled loss function, without requiring an explicit reward model or training value function with policy gradient. This fundamental difference leads to limited theoretical understanding of how DPO connects with standard RL algorithms, particularly in terms of its theoretical advantages. As a result, more research is needed to clarify the specific scenarios where DPO might offer distinct benefits over typical RL algorithms.

% Due to the difference, theoretical understanding of DPO's relationship with typical RL algorithms is relative limited, especially for its connection with standard RL algorithms \cite{} and theoretical advantages.

% Despite its practical success, there exists a significant difference between DPO and the standard actor-critic framework based standard RL methods\citep{}. Concretely, under a Bradly-Terry \citep{bradley1952rank} framework to modeling human preference, training policy model with DPO is transferred as minimizing a human-preference labeled loss function, without requiring an explicit reward model or training value function with policy gradient\cite{}. 

 
However, addressing the relationship between DPO and RL algorithms is a non-trivial task. On one hand, DPO is rooted in the optimal solution analysis of the KL-constrained reward maximization problem, based on PPO-RLHF. On the other hand, the absence of a reward function in DPO complicates its classification within the typical RL framework. As a result, recent research has even suggested that DPO may not be a true RL algorithm \cite{rlhfwithoutrl}, further highlighting the need for a deeper theoretical investigation. In this paper, we aim to bridge this gap by investigating the connections between DPO, RL, and other RLHF algorithms. Specifically, we focus on three key aspects:
% To address these ambiguities, we focus on three key aspects related to DPO, RL, and other RLHF algorithms: (1) the construction of the loss function; (2) the target distribution at which the algorithm converges; (3) the impact of key components within the loss function. 
\begin{itemize}
\item 1. What are the distinctions and connections of the construction of the loss function between DPO, RL and other representative RLHF algorithms? 
\item 2. What is the target distribution of these loss function?  
\item 3. How do the key components within these algorithms affect the algorithm performance?  
\end{itemize}
% To do so, we first propose a unified framework (Figure \ref{Framework_figure}) that connects DPO and other RLHF algorithms with typical RL methods by analyzing their target distributions and application scenarios. This framework not only clarifies the theoretical relationships between DPO and RL algorithms but also serves as a foundation for understanding how RL principles can be leveraged to enhance DPO. 
% To explore the relationship between them, we first establish a unified framework named UDRRA connecting these algorithms based on the construction of their loss functions. Next, we uncover their target policy distributions within this framework. Finally, we investigate the critical components of DPO to understand their impact on the convergence rate. 


To do so, we fist construct a unified framework (Figure \ref{Framework_figure}) that uniformly covers the standard RL algorithms (i.e., PPO and Soft Actor Critic (SAC) \cite{SAC}) and the DPO-based algorithms i.e., (IPO, DRO, DPO). Based on the construction of loss functions of these algorithms, we clearly reveal the connections between them in our framework. Besides, the target distribution under this framework is also clearly revealed. Notably, though some existing literature \cite{xu2024dpo,ivison2024unpacking,yan20243d} have also explored the connection between DPO and PPO. However, they mainly focus on the technical details of training LLM and the comparison of experimental results between DPO and PPO.
%However, they mainly focus on training LLM. \xuerui{Concretely, formalizing sequence-level DPO into a standard token-level RL framework, by properly constructing token-level reward model as log-likelihood loss}. 
In contrast, our framework is more general, and is not restricted to any specific model scenario.

%Furthermore, based on this framework, we constructed an algorithm tree (See Figure \ref{Tree}). In this tree, some leaf nodes represent well-established algorithms, such as DPO and IPO, while others correspond to promising yet unexplored algorithms. This provides a unified and comprehensive perspective for understanding the relationships among these methods and offers guidance for future algorithmic improvements.

% To answer these questions, we first propose a unified framework that connects DPO and other RLHF algorithms with typical RL methods through their target distributions and application scenarios. This framework not only helps clarify the theoretical relationships between DPO and RL algorithms but also serves as a tool for understanding how RL principles can be applied to improve DPO.   Furthermore, based on this framework, we constructed an algorithm tree (See Figure ). In this tree, some leaf nodes correspond to established powerful algorithms, such as DPO and IPO, while others represent promising algorithms that remain unexplored which leads to 一个更统一而全面的视角来理解所有算法,并且指导了future work about 算法改进.
%下面这里要说明是怎么分析的。

% We then analyze the target distribution of the loss function for algorithm in our Framework containing DPO. Our analysis reveals that the target distribution of DPO aligns more closely with that of the Soft Actor-Critic (SAC) algorithm, rather than PPO-RL as proposed in the RL literature \cite{PPO}. This finding suggests that DPO's behavior fundamentally differs from typical RL methods such as PPO-RL. Additionally, we highlight how the use of an offline dataset in DPO introduces distribution shift issues, distinguishing it from the online learning settings of methods like SAC and PPO.

Then we analyze the target distributions of the series of methods mentioned in our framework. Although these methods have different requirements for the reward function, the same versions (e.g. the posterior version) all share the same target distribution. Furthermore, by analyzing the relationship between DPO and the PRA-P method in our framework, we proved that the target distribution of DPO is not $\bar{\pi}^\tau$ introduced in the original DPO's paper. Finally, considering the importance of DPO in training RLHF models, we further investigate the convergence rate of the DPO algorithm. Building on our theoretical results, we analyze how the hyper-parameter $\tau$ and the offline dataset of the algorithm influence its convergence performance.
% Finally, considering the importance of DPO in training RLHF model, we further investigate the convergence rate of DPO under non-convex optimization settings \yi{as in \cite{} studied PPO}. Building upon on the theoretical results, we further explore the influence of the hyperparameter in DPO's loss function, %$\tau$ (express $\beta$ in the paper of DPO \cite{DPO} as $\frac{1}{\tau}$ in this paper for convenience) 
% and the pairwise human preference feedback dataset on DPO's performance.

 
% Finally, we investigate the convergence properties of DPO under non-convex settings. Building on our theoretical results, we further explore the impact of the hyperparameter $\tau$ (express $\beta$ in the paper of DPO \cite{DPO} as $\frac{1}{\tau}$ in this paper for convenience) and the pairwise preference feedback dataset on DPO's performance.

% These questions are essential for achieving a deeper understanding of DPO and enhancing its practical applications in training LLMs. 
% To the best of our knowledge, we are the first to establish such a unified framework that provides a cohesive perspective on typical RL algorithms and a variety of RLHF methods. Using this framework, we theoretically address the aforementioned questions, offering novel insights into DPO and related algorithms. %
%这里有framework和问题23之间的逻辑问题,因此在后文应该说这些点:问题2中distribution shift的问题是基于这个framework发现的。问题3的中我们分析了这些framework中的不同算法的收敛性质,并且分析了DPO这一offline算法的数据选择策略对算法效果的影响。 

To the best of our knowledge, we are the first to establish such a unified framework that provides a cohesive perspective on DPO Algorithm and a variety of RLHF algorithms. Our explorations provide us deeper understanding and new insights in improving the existing algorithms for RLHF. The contributions in this paper can be summarized as follows:
 
\begin{itemize}
\item  We established a unified framework to bridge the theoretical gap between DPO and RL algorithms.
\item We analyzed the target distribution of the loss function for algorithm in our Framework.
\item We explored the impact of the hyper parameter $\tau$ and the preference dataset on algorithm performance.
\end{itemize}





\section{Preliminary}\label{Preliminary}

\subsection{Problem Setup}

Consider a set of state or prompt  $\mathbb{X}$. Denote the response space as $\mathbb{Y}$, and the reward function as $r: \mathbb{X} \times \mathbb{Y} \rightarrow \mathbb{R}$. The optimization problem that most RLHF algorithm aim to solve is defined as where $\mathcal{D}$ is arbitrary distribution:
\begin{equation}\label{RLOpt}
\small{\min _{\pi_\theta} J \triangleq \min _{\pi_\theta} \mathbb{E}_{x \sim \mathcal{D}, y \sim \pi_\theta(y \mid x)}\left[-r(x, y)\right],}
\end{equation}
and define the target distribution $\pi^\delta=\arg\min _{\pi_\theta} J$. 

\begin{definition}\label{BDA}
Boltzmann Distribution Approximation Problem (BDAP). Motivated by the idea of Soft Actor Critic algorithm \cite{SAC} that they update the policy towards the exponential of the new Q-function, another important problem we consider is the Boltzmann Distribution Approximation Problem:
\begin{equation}\label{BDA_loss_eq}\small
\begin{gathered}
\min _{\pi^{\prime} \in \Pi} \mathbb{E}_{x \sim \mathcal{D}}\left[\mathrm{D}_{\mathrm{KL}}\left(\pi^{\prime}\left(\cdot \mid x\right) \| \pi^\tau\left(\cdot \mid x\right)\right)\right],\quad \pi^\tau\left(\cdot \mid x\right)\triangleq \frac{\exp \left(\tau r\left(x, \cdot\right)\right)}{Z\left(x\right)}.
\end{gathered}\normalsize
\end{equation}
\end{definition}
Problem (\ref{BDA_loss_eq}) describes the objective of optimizing $\pi^{\prime}$ towards the Boltzmann distribution of reward function by the KL divergence, though in principle any distribution distance is suitable. The partition function $Z\left(x\right)=\sum_{y\in\mathbb{Y}}\exp \left(\tau r\left(x, y\right)\right)$ normalizes the distribution.  $\tau$ is the temperature parameter. $\Pi$ is the set of policies. $\pi^\tau$ is essentially a soft approximation of the optimal solution $\pi^\delta$ of Problem (\ref{RLOpt}), while $\pi^\delta$ is sharp. This approximation is referred to as the Boltzmann approximation in physics. Proposition \ref{tau_tend_to_delta} shows that $\pi^\tau$ converges to $\pi^\delta$.

 


% Indeed, $\pi^\tau\left(\cdot \mid x\right)$ is essentially a soft approximation of the optimal solution $\pi^\delta(y|x)$ of Problem ((\ref{RLOpt})), while $\pi^\delta(y|x)$ is sharp. This approximation is referred to as the Boltzmann approximation in Physics. Proposition \ref{tau_tend_to_delta} shows that $\pi^\tau\left(\cdot \mid x\right)$ will converge to $\pi^\delta(\cdot|x)$. We defer the proof to Appendix \ref{tau_tend_to_delta_proof}.
\begin{proposition}\label{tau_tend_to_delta}
Without loss of generality, assume that the reward function $r(x, y)$ has a unique maximum for any given $x$. Recall
$$\small \pi^\delta(y|x) = 
\begin{cases} 
1, & \text{if } y = \arg\max_{y^{\prime}} r(x, y^{\prime}), \\
0, & \text{otherwise}.
\end{cases} \normalsize$$
Then we have:
$  \lim_{\tau \to \infty} \pi^\tau(y|x) = \pi^\delta(y|x).$  
\end{proposition}






\subsection{Related Works}

% where $\mathcal{D}$ is a given distribution. 
% Without loss of generality, we assume that the reward lies in the [0, 1] interval:
% \assumption{(\small{Bounded Reward}.) $\forall(x, y), r(x, y) \in[0,1]$.}

\textbf{PPO-RL:} The Proximal Policy Optimization (PPO) algorithm \cite{PPO} in RL is based on the Trust Region Policy Optimization (TRPO) algorithm \cite{TRPO}. The KL penalty version of PPO in RL is below: 
\begin{equation}
\begin{aligned}
\mathcal{L}_{\text{PPO}}(\pi_\theta)=\hat{\mathbb{E}}_t[&-\frac{\pi_\theta\left(a_t \mid s_t\right)}{\pi_{\theta_{\text {old }}}\left(a_t \mid s_t\right)}\hat{A}_t +\frac{1}{\tau}\mathrm{D}_{\mathrm{KL}}\left(\pi_{\theta_{\text {old }}}(\cdot|s_t)\|\pi_\theta(\cdot|s_t)\right)].
\end{aligned}
\end{equation}
where $\theta_{\text{old}}$ is the vector of policy parameters before the update. $\hat{A}_t$ is an estimator of the advantage function at timestep $t$. For one step policy optimization problem, $\hat{A}_t=r(s_0,a_0)$ where $s_0, a_0$ are taken as the prompt and response separately. Based on the theory of TRPO, PPO-RL will converge to the optimal solution $\pi^\delta$ of Problem (\ref{RLOpt}).

\textbf{PPO-RLHF:} The PPO based RLHF \cite{PPO-basedRLHF} typically consists of two major stages: reward modeling and RL optimization. In the first stage, human annotators select the preferred answer $y_w$ over the less preferred one $y_l$, forming the preference pair $(y_w, y_l, x)$ based on a given input $x$. A reward model $r_\phi(x, y)$ is optimized by: 
\begin{equation}
\small{\mathcal{L}_R\left(r_\phi\right)=-\mathbb{E}_{\left(x, y_w, y_l\right) \sim \mathcal{D}_R}\left[\log \sigma\left(r_\phi\left(x, y_w\right)-r_\phi\left(x, y_l\right)\right)\right].}
\end{equation}
where $\sigma(\cdot)$ is the sigmoid function, $\mathcal{D}_{R}\triangleq \{(x,y_w,y_l)|x\sim\mathcal{D},y_w,y_l\sim\pi_0(\cdot|x),(y_w\succ y_l)\sim p^*(1|y_w,y_l,x)\}$, $\pi_0$ is the offline data sampling distribution. $\{z=1|y_1,y_2,x\}\triangleq \{r(x,y_1)\geq r(x,y_2)\}$ and $p^*$ is modeled by the BT model. In the RL optimization stage, the trained reward model evaluates the outputs of the language model, guiding policy optimization with loss function of the classic PPO algorithm:
\begin{equation}\label{PPO_loss}
\small{\mathbb{E}_{x \sim \mathcal{D}, y \sim \pi_\theta(\cdot\mid x)}\left[-r_\phi(x, y)+\frac{1}{\tau} \mathrm{D}_{\mathrm{KL}}\left(\pi_\theta(\cdot|x) \| \pi_{ref}(\cdot|x)\right)\right].}
\end{equation}
Typically the reference policy $\pi_{ref}(\cdot|x)$ is a pretrained or Supervised Fine-Tuned (SFT) model.

\textbf{DPO:} The Direct Preference Optimization (DPO) \cite{DPO} leverages the optimal policy form in Eq.\ref{PPO_loss} as theoretical support by representing the comparison probability (calculated under the assumption of BT model) of human preferences through the ratio between the policy $\pi_\theta$ and the reference policy $\pi_{ref}$. This approach eliminates the need for explicitly modeling the reward function. Thus DPO directly optimizes the policy by maximizing the log-likelihood function based on human preference feedback. 
\begin{equation}\label{DPO_eq}
\small{\begin{gathered}
\bar{h}_\theta\left(x, y_w, y_l\right)=\frac{1}{\tau} \log \frac{\pi_\theta\left(y_w \mid x\right)}{\pi_{\text {ref }}\left(y_w \mid x\right)}-\frac{1}{\tau} \log \frac{\pi_\theta\left(y_l \mid x\right)}{\pi_{\text {ref }}\left(y_l \mid x\right)},\ 
\mathcal{L}_{\mathrm{DPO}}\left(\pi_\theta ; \pi_{\text {ref }}\right)=-\mathbb{E}_{\left(x, y_w, y_l\right) \sim \mathcal{D}_R}\left[\log \sigma\left(\bar{h}_\theta\left(x, y_w, y_l\right)\right)\right].
\end{gathered}}
\end{equation}


% 改:determine的箭头再长一点,reward function那里改成填充图,跟别的图统一。
\begin{figure*}[!ht]
% \vskip -0.1in
\begin{center}
\centerline{\includegraphics[width=0.9\columnwidth,]{figure/framework.pdf}}
\vskip -0.1in
% 改delta r,在底部给不同的scenario加上不同颜色的区间做区分。
\caption{The unified UDRRA framework to connect DPO and other RLHF algorithms with typical RL methods. The algorithms in our framework can construct loss functions for four different scenarios (from right to left): (1) between \textcolor{purple}{$\pi_\theta(y|x)$} and \textcolor{red}{$\pi^\delta(y|x)$} (or \textcolor{red}{$\pi^\tau(y|x)$}); (2) between \textcolor{purple}{$r_\theta(x,y)$} and \textcolor{red}{$r(x,y)$}; (3) between \textcolor{purple}{$\Delta r_\theta(x,y_1,y_2)$} and \textcolor{red}{$\Delta r(x,y_1,y_2)$}; and (4) between \textcolor{purple}{$p_\theta(z|y_1,y_2,x)$} and \textcolor{red}{$p^*(z|y_1,y_2,x)$}. From scenario (1) to scenario (4), the requirements of the reward function $r(x,y)$ are progressively relaxed. The notations in the figure will be elaborated in Section \ref{Preliminary} and Section \ref{TRPRA_sec}. Notably, we utilize Proposition \ref{tau_tend_to_delta} to illustrate the relationship between $\pi^\delta$ and $\pi^\tau$. }
\label{Framework_figure}
\end{center}
\vskip -0.3in
\end{figure*}



\section{The Framework to \underline{U}nify \underline{D}PO, \underline{R}L and Other Representative \underline{R}LHF \underline{A}lgorithms}\label{TRPRA_sec}
In this section, we unify DPO, RL, and other representative RLHF algorithms within a unified framework (see Figure \ref{Framework_figure}) based on
the construction of loss functions. The distinctions of loss functions are mainly because how loss functions are defined across various reward function scenarios. Our framework clarifies these relationships, highlighting each method's strengths and use cases while bridging RL principles with RLHF techniques to improve methods like DPO.
% In this section, we present the relationship between DPO, RL and other representative RLHF algorithms, such as PPO-RLHF, through a unified framework. The core idea of this framework is that while many different algorithms may appear to have entirely distinct objective functions, they are, in fact, approximations of the same target distribution. The differences lie in how they define loss functions across various spaces, leveraging equivalence relationships between these spaces. Each approach has its own applicable scenarios, and our framework positions these algorithms appropriately, providing a clearer understanding of their respective advantages and characteristics.

% We introduce the framework (see Figure \ref{Framework_figure}), which naturally leads to a variety of algorithms, including PPO, SAC, IPO, DRO, and DPO. This framework not only introduces the theoretical relationships between DPO, RLHF, and RL algorithms but also establishes a foundation for understanding how RL principles can be harnessed to enhance RLHF methods like DPO.

%Furthermore, in Section \ref{Tree_sec}, we extend this framework to construct an algorithm tree (see Figure \ref{Tree}). In this tree, some leaf nodes represent well-established algorithms, such as DPO and IPO, while others correspond to promising but unexplored approaches. This comprehensive perspective not only deepens our understanding of the relationships among these methods but also offers valuable guidance for future algorithmic advancements.



% In this section, we will show the relationship between DPO and other representative RLHF algorithms such as PPO-RLHF through a unified framework. The key idea behind the framework is that 许多不同的算法看起来拥有完全不同的目标函数,但是实际上却只是对相同target distribution的逼近,区别在于他们通过考虑不同的空间之间的等价关系,在不同的空间上构造损失函数。不同的情形有各自的适用场景,我们的framework将不同的算法放在了合适的位置上,从而有助于更加清楚的理解不同的算法的优势和特点。
% We will first show the framework, see Figure \ref{Framework_figure}, which naturally derives a variety of algorithms, including PPO, SAC, IPO, DRO, and DPO, etc. This framework not only clarifies the theoretical relationships between DPO, RLHF and RL algorithms but also serves as a foundation for understanding how RL principles can be leveraged to enhance the RLHF methods such as DPO. 
% Furthermore, in Section \ref{Tree_sec}, based on this framework, we constructed an algorithm tree, see Figure \ref{Tree}. In this tree, some leaf nodes represent well-established algorithms, such as DPO and IPO, while others correspond to promising yet unexplored algorithms. This provides a comprehensive perspective for understanding the relationships among these methods and offers guidance for future algorithmic improvements.


% \section{Our Unified Framework}\label{TRPRA_sec}
% In this section, we will establish a unified framework, see Figure \ref{Framework_figure}, which naturally derives a variety of algorithms, including PPO, SAC, IPO, DRO, and DPO, etc. This framework not only clarifies the theoretical relationships between DPO, RLHF and RL algorithms but also serves as a foundation for understanding how RL principles can be leveraged to enhance the RLHF methods such as DPO. Furthermore, in Section \ref{Tree_sec}, based on this framework, we constructed an algorithm tree, see Figure \ref{Tree}. In this tree, some leaf nodes represent well-established algorithms, such as DPO and IPO, while others correspond to promising yet unexplored algorithms. This provides a comprehensive perspective for understanding the relationships among these methods and offers guidance for future algorithmic improvements.
\subsection{The UDRRA Framework}\label{Framework_sec}
Most RL algorithms aim to solve Problem (\ref{RLOpt}): given a reward function, find the optimal policy $\pi^\delta$. To balance exploration and exploitation, the target distribution $\pi^\delta$ is often approximated by a Boltzmann distribution $\pi^\tau$, turning the optimization into a distribution approximation problem. 
% The key challenge is designing a loss function for $\pi_\theta$ to approximate $\pi^\tau$ using gradient-based methods. Our framework highlights differences in loss function design, focusing on four scenarios: policy distributions, reward functions, reward differences, and preferences, as shown in Figure \ref{Framework_figure}.
The critical challenge lies in designing a loss function that enables the parameterized policy $\pi_\theta$ to effectively approximate the target distribution $\pi^\tau$ through gradient-based methods. Our framework emphasizes the differences in loss function construction across various algorithms. Specifically, the construction of loss functions in current algorithms primarily addresses four types of scenarios involving policy distributions, reward functions, reward differences, and preferences, corresponding to scenarios (1)-(4) in Figure \ref{Framework_figure}. 

%Most RL algorithms aim to solve the Problem (\ref{RLOpt}): given a reward function (not necessarily an analytic function, but rather the values of the reward), find the optimal policy $\pi^\delta$ uniquely determined by this reward function. To address the Exploration-Exploitation trade-off, a common approach is to approximate the target distribution $\pi^\delta$ using a Boltzmann approximation, resulting in $\pi^\tau$. This transforms the optimization problem in the probability space into a distribution approximation problem. 

In scenario (1), the loss function aims to directly approximate $\pi^\delta$ or $\pi^\tau$ using $\pi_\theta$, as seen in algorithms like PPO-RL, SAC, and PPO-RLHF. In scenario (2), an implicit reward function $r_\theta(x, y)$ is constructed from $\pi_\theta$ (Eq.\ref{r_theta_def}) to minimize its difference from the true reward function $r(x, y)$, guiding $\pi_\theta$ towards $\pi^\tau$. However, this requires the partition function $Z(x)$, which is discussed later. Scenario (3) avoids calculating $Z(x)$ by aligning reward differences $\Delta r_\theta(x, y_1, y_2)$ and $\Delta r(x, y_1, y_2)$, focusing on differences rather than exact values.

%In scenario (1), the loss function focuses on using $\pi_\theta$ to directly approximate $\pi^\delta$ or $\pi^\tau$, as seen in algorithms like PPO-RL, SAC, and PPO-RLHF. In scenario (2), the implicit reward function $r_\theta(x, y)$ is constructed using $\pi_\theta$ according to Eq.\ref{r_theta_def}. The difference between this implicit reward function and the ground truth reward function $r(x, y)$ is leveraged to build the loss function and guide the parameterized policy $\pi_\theta$ to converge to $\pi^\tau$. One constraint above is that it requires the partition function $Z(x)$, which will be discussed in detail later. Consequently, in scenario (3), the loss function is designed to align the differences in reward values, $\Delta r_\theta(x, y_1, y_2)$ and $\Delta r(x, y_1, y_2)$, shifting the focus from the exact reward values to their differences and avoiding the calculation of $Z(x)$. 



% The critical challenge lies in designing a loss function that enables the parameterized policy $\pi_\theta$ to approximate the target distribution $\pi^\tau$ effectively through gradient-based optimization methods. 
% Our framework focuses on the differences in loss function construction among various algorithms. Specifically, the construction of loss functions in current algorithms primarily focuses on four types of scenarios: those involving policy distributions, reward functions, reward differences, and preferences, corresponding to the scenarios (1)-(4) in Figure \ref{Framework_figure}. In scenarios (1), the loss function focus on using $\pi_\theta$ to approximate $\pi^\delta$ or $\pi^\tau$ directly, such as the PPO-RL \cite{PPO}, SAC \cite{SAC} and PPO-RLHF \cite{PPO-basedRLHF}, etc. In scenario (2), the implicit reward function $r_\theta(x, y)$ is constructed by using current parameterized policy  $\pi_\theta$ according to Eq.\ref{r_theta_def} and its difference between   the ground truth reward function $r(x, y)$ can be leveraged to build the loss function and guide the parameterized policy  $\pi_\theta$  to converge to the target policy distribution. 

% In this scenario, we not only relax the requirement for exact reward values to the differences between two rewards but also eliminate the dependency on the partition function $Z(x)$. 


% One constrained in scenario (2) is that it requires the partition  function $Z(x)$ which we will show detailed later. Thus in scenario (3), the loss function is constructed to align the differences in reward values, $\Delta r_\theta(x, y_1, y_2)$ and $\Delta r(x, y_1, y_2)$, where the focus shifts from exact reward values to their differences. In this scenario (3) we both relax the required information from the rewrd value to the difference between two rewards and the requirements of partitionfunction $Z(x)$.

% In many real world application, the exact difference between two reward is have to get due to noise such as commonly situation considered by RLHF algorithm, 容易获取的信息是与此相关的一个随机变量, 例如BT 模型。因此 In scenario (4),  we consider the distribution of $z$ condition on  the reward difference. Specifically, we denote $p_\theta(z|y_1, y_2, x)$ as the  approximation of the ground truth conditional preference distribution $p^*(z|y_1, y_2, x)$. We can construct the loss function through minimize the distance between these two distribution. This scenario is relying primarily on preference data without assuming access to the true reward function $r(x, y)$. In summary, from scenario (1) to scenario (4), the requirements of the reward function $r(x, y)$ are progressively relaxed.  In the following content, we will illustrate the characteristics of loss function construction in different scenarios by using specific algorithms.
In many real-world applications, noise often makes it difficult to determine the exact reward difference, as is common in RLHF scenarios. Instead, accessible information is typically a related random variable $ z $, reflecting the comparisons between reward values. In scenario (4), we model the conditional distribution of $ z $ based on the reward difference. Let $ p_\theta(z|y_1, y_2, x) $ approximate the true conditional preference distribution $ p^*(z|y_1, y_2, x) $. The loss function minimizes the distance between these distributions, relying on preference data without requiring the true reward function $ r(x, y) $.


%In many real-world applications, obtaining the exact difference between two rewards can be challenging due to noise, as often seen in RLHF scenarios. Instead, easily accessible information is often a related random variable (r.v.), showing comparison information between reward values. Therefore, in scenario (4), we consider the distribution of r.v. $z$ conditioned on the reward difference. Denote $p_\theta(z|y_1, y_2, x)$ as the approximation of the ground truth conditional preference distribution $p^*(z|y_1, y_2, x)$. The loss function can then be constructed by minimizing the distance between these two distributions. This scenario relies primarily on preference data without assuming access to the true reward function $r(x, y)$. 

In summary, from scenario (1) to scenario (4), the requirements on the reward function $r(x, y)$ are progressively relaxed. Our paper will illustrate the characteristics of loss function construction in different scenarios and prove the equivalence of different loss designs in the sense of target distribution. Furthermore, we will analyze the position of DPO within these scenarios and its relationship with our proposed method PRA-P for offering a new perspective for the theoretical investigation of DPO.

\subsubsection{Scenario (1): Boltzmann Distribution Approximation}\label{BDA_sec}
%In scenario (1), optimizing $\pi_\theta$ towards the optimal policy $\pi^\delta$ is typically the paradigm for Problem (\ref{RLOpt}). %Many typical RL algorithms such as PPO-RL \cite{PPO} and TRPO \cite{TRPO} are all following this paradigm. 
Proposition \ref{tau_tend_to_delta} supports the Boltzmann approximation, allowing $\pi^\tau$ to relax $\pi^\delta$. This relaxation is validated by RL algorithms like SAC. Therefore, in scenario (1) there is a natural relaxation method to approximate the solution to Problem (\ref{RLOpt}) by solving $\pi^\tau$. Here, we summarize the method based on the Boltzmann Distribution Approximation Problem in Definition \ref{BDA} using the KL divergence as the loss function. For the Forward-KL and Reverse-KL, we have Eq.\ref{forward-KL} and Eq.\ref{reverse-KL} (See derivation in Appendix \ref{forward-KL_proof}, \ref{reverse-KL_proof}):
{\begin{align*}
&\mathcal{L}_{\mathrm{Forward-BDA}}(\pi_\theta)=\mathbb{E}_{x\sim \mathcal{D}}\left[\mathrm{D}_{\mathrm{KL}}\left(\pi_\theta\left(\cdot \mid x\right) \| \pi^\tau\left(\cdot \mid x\right)\right)\right]=-\mathbb{E}_{x\sim \mathcal{D},y\sim\pi_\theta(\cdot|x)}\left[(\tau r(x,y) - \log(\pi_\theta(y|x)))\right], \numberthis\label{forward-KL} \\
&\mathcal{L}_{\mathrm{Reverse-BDA}}(\pi_\theta)=\mathbb{E}_{x\sim \mathcal{D}}\left[\mathrm{D}_{\mathrm{KL}}\left( \pi^\tau\left(\cdot \mid x\right) \| \pi_\theta\left(\cdot \mid x\right) \right)\right]=-\mathbb{E}_{x\sim \mathcal{D},y\sim\pi^\tau\left(\cdot \mid x\right)}[\log\pi_\theta(y|x)].\numberthis\label{reverse-KL}
\end{align*}}
We name the method using Eq.\ref{forward-KL} or Eq.\ref{reverse-KL} as the loss function as the \textbf{Boltzmann Distribution Approximation (BDA)} method. Eq.\ref{forward-KL} is exactly the loss function of SAC. 
\subsubsection{Scenario (2): Reward Approximation}
Denote the implicit reward function $r_\theta(x, y)$:
\begin{equation}\label{r_theta_def}
\small{r_\theta(x,y)\triangleq \frac{1}{\tau}\log(Z(x)\pi_\theta(y|x)).}
\end{equation}
Scenario (2) focuses on the intuition of using $r_\theta(x,y)$ to approximate the ground truth $r(x,y)$. Considering using $r_\theta(x,y)$ to approximate $r(x,y)$ is a typical regression problem, the Mean Square Error (MSE) is natural to be the loss function (See derivation in Appendix \ref{Derivation_RA_eq}):
\begin{equation}\label{RA_eq}
\small{\begin{aligned}
&\mathcal{L}_{\mathrm{RA}}(\pi_\theta)=\mathbb{E}_{x\sim \mathcal{D},y\sim\pi_\theta(\cdot|x)}\left[\left(r_\theta(x,y)- r(x,y)\right)^2\right] =\mathbb{E}_{x\sim \mathcal{D},y\sim\pi_\theta(\cdot|x)}\left[\frac{1}{\tau^2}\left(\log(\frac{\pi_\theta(y|x)}{\pi^\tau(y|x)})\right)^2\right].
\end{aligned}}
\end{equation}
We name the method using Eq.\ref{RA_eq} as the loss function as the \textbf{Reward Approximation (RA)} method. Due to the special design of $r_\theta(x,y)$, although the forms are different, Eq.\ref{forward-KL}, Eq.\ref{reverse-KL} and Eq.\ref{RA_eq} all effectively share the same target distribution. We will formally prove it later in Theorem \ref{policy_equivalence_1}.

In RLHF, the alignment task builds upon a reference model $\pi_{ref}(\cdot|x)$ rather than an initialized language model policy. Thus we incorporate $\pi_{ref}(y|x)$ via the posterior implicit reward function $\bar{r}_\theta(x,y)$. Treating $\pi_{ref}$ as the prior, the posterior distribution becomes $\bar{\pi}^\tau$:
\begin{equation}\label{pi_bar}
\small{\bar{\pi}^\tau(y|x)\triangleq \frac{\pi_{ref}(y|x)\exp(\tau r(x,y))}{\sum_{y'\in Y}\pi_{ref}(y'|x)\exp({\tau r(x,y')})}.}
\end{equation}
The corresponding posterior implicit reward function is 
\begin{equation}\label{r_theta_def_posterior}
\small{\bar{r}_\theta(x,y)=\frac{1}{\tau}\log(Z'(x)\frac{\pi_\theta(y|x)}{\pi_{ref}(y|x)}),}
\end{equation}
where $Z'(x)=\sum_{y'\in Y}\pi_{ref}(y'|x)\exp({\tau r(x,y')})$. Then Eq.\ref{RA_eq} will be changed into:
\begin{equation}\label{RA_eq_posterior}
\small{\begin{aligned}
&\mathcal{L}_{\mathrm{RA-P}}(\pi_\theta)=\mathbb{E}_{x\sim \mathcal{D},y\sim\pi_\theta(\cdot|x)}\left[\left(\bar{r}_\theta(x,y)- r(x,y)\right)^2\right] =\mathbb{E}_{x\sim \mathcal{D},y\sim\pi_\theta(\cdot|x)}\left[\left(\frac{1}{\tau}\log(Z'(x)\frac{\pi_\theta(y|x)}{\pi_{ref}(y|x)})- r(x,y)\right)^2\right].
\end{aligned}}
\end{equation}
We call the method using Eq.\ref{RA_eq_posterior} as loss function as the \textbf{{Reward Approximation-Posterior} (RA-P)} method. 

\subsubsection{Scenario (3): Reward Difference Approximation}
Define the general difference function $\Delta f(x, y_1, y_2)\triangleq f(x, y_1)-f(x, y_2)$. In scenario (3), the loss function is constructed to align the reward difference function, $\Delta r_\theta(x, y_1, y_2)=r_\theta(x,y_1)-r_\theta(x,y_2)$ and $\Delta r(x, y_1, y_2)=r(x,y_1)-r(x,y_2)$. Denote $\mathcal{D}_{\text{pw}}\triangleq\{(x, y_1, y_2) \mid x \in \mathcal{D}, y_1, y_2 \sim \pi_\theta(y|x)\}$. Similar to the RA method, using MSE, we have the loss function:
\begin{equation}\label{pair-wised_RA_eq}
\small{\begin{aligned}
&\mathcal{L}_{\mathrm{RDA}}(\pi_\theta)=\mathbb{E}_{\mathcal{D}_{\text{pw}}}\left[\left( \frac{1}{\tau}\log\frac{\pi_\theta(y_1|x)}{\pi_\theta(y_2|x)}-\Delta r(x, y_1, y_2)\right)^2\right].
\end{aligned}}
\end{equation}
The method using Eq.\ref{pair-wised_RA_eq} as the loss function is named \textbf{Reward Difference Approximation (RDA)}. Its posterior version, RDA-P, replaces $r_\theta$ with $\bar{r}_\theta$. All our methods have posterior versions, this paper only formulate RA-P and PRA-P as example. The RDA method simplifies computation by canceling the partition function $Z(x)$, which requires multiple reward function queries. Additionally, RDA only needs reward differences, not absolute values, while maintaining the same target distribution as BDA and RA. This is formally proven in Theorem \ref{policy_equivalence_2}.

%We name the method using Eq.\ref{pair-wised_RA_eq} as the loss function as the \textbf{Reward Difference Approximation (RDA)} method. Similar to the RA-P method, the RDA-P (posterior version of RDA) is simply changing $r_\theta$ into $\bar{r}_\theta$. Actually all our methods have the posterior version, but our paper mainly analyzes the RA-P and PRA-P methods. On the one hand, since the subtraction of $r_\theta(x,y_1)$ and $r_\theta(x,y_2)$ cancels out the partition function $Z(x)$ term, comparing the RA method, the RDA method avoids the calculation of $Z(x)$ which requires multiple queries to the reward function and then complicates the computation and optimization. On the other hand, the RDA method also relaxes the requirement on reward function to only know the difference between two reward values. Moreover, the RDA method still keeps the same target distribution as BDA and RA methods. We will formally claim this conclusion in Theorem \ref{policy_equivalence_2}.


\subsubsection{Scenario (4): Preference Reward Approximation}\label{Scenario4}
In scenario (4), we consider using $p^*(z \mid y_1, y_2, x) = \omega(r(x, y_{2-z}), r(x, y_{z+1}))$, where $z = 1$ if $r(x, y_1) > r(x, y_2)$, and $z = 0$ otherwise, to model the comparison probability between $r(x, y_1)$ and $r(x, y_2)$. Only $(x, y_1, y_2, z)$ sampled with $p^*(z \mid y_1, y_2, x)$ is available, while $r(x, y)$ is unobservable.  
%In many real-world applications, easily accessible information is often a related r.v., such as those used in the BT model. Therefore, in scenario (4), we consider the distribution of r.v. $z$ conditioned on the reward difference. Specifically, $p^*(z \mid y_1, y_2, x)=\omega(r(x,y_{2-z}),r(x,y_{z+1}))$ where $z = 1 \text{ if } r(x, y_1) > r(x, y_2); 0 \text{ otherwise}$, is modeling the comparison probability between $r(x,y_1)$ and $r(x,y_2)$ such as the BT model. In this scenario, the reward function is invisible but only the pair-wised data $(x, y_1, y_2, z)$ is sampled with probability $p^*(z \mid y_1, y_2, x)$.
The BDA, RA, and RDA methods are less effective in scenario (4). These methods rely on inverting $\omega$ to estimate $r(x, y)$. Table \ref{omega_table} and Appendix \ref{omega} discuss various $\omega$ formulations. In general, most $\omega^{-1}$ are difficult to compute because $p^*$ is not analytic. Thus developing methods that rely only on the forward evaluation of $\omega$ instead of $\omega^{-1}$ would greatly improve practicality in this scenario.

%The BDA, RA and RDA methods cannot be well applied to scenario (4) because of the complexity of the inverse function $\omega^{-1}$ determines. When only preference data $(x, y_1, y_2, z)$ is available, the above methods rely on estimating the reward $r(x, y)$ using the inverse of $\omega$, making their applicability highly dependent on the properties of $\omega^{-1}$. We analyze different formulations of $\omega$ in Table \ref{omega_table} and provide a discussion in Appendix \ref{omega}. In general, most $\omega^{-1}$ are difficult to compute because $p^*$ is not analytic. Therefore, developing new methods that bypass the need for $\omega^{-1}$ and instead operate solely with the forward evaluation of $\omega$ could significantly enhance practicality in this scenario.
Therefore we propose the \textbf{Preference Reward Approximation (PRA)} method, assuming $\omega$ is known and data is sampled as $(x, y_w, y_l)$ where $\{r(x, y_w) > r(x, y_l)\}$ is true with probability $p^*(1\mid y_w, y_l, x)$. Define the parameterized comparison distribution as $p_\theta(z|y_1, y_2, x) = \omega(r_\theta(x, y_{2-z}), r_\theta(x, y_{z+1}))$. Assuming $\omega(x, y)$ is injective and satisfies the symmetric complementarity property $\omega(x, y) = 1 - \omega(y, x)$, the winning probability of $x$ against $y$ equals $y$'s losing probability against $x$, then we have:
% Therefore, we propose the \textbf{Preference Reward Approximation (PRA)} method. Assume that the function $\omega$ is known and the data is sampled in the form of $(x, y_w, y_l)$. Given $(x, y_1)$ and $(x, y_2)$, the pair $(x, y_1, y_2)$ is identified as $r(x, y_1) > r(x, y_2)$ and is sampled with a comparison probability $p^*(1 \mid y_1, y_2, x)$, denoted as $y_1 \succ y_2$; similarly, $(x, y_2, y_1)$ is identified as $r(x, y_1) < r(x, y_2)$ and is sampled with probability $p^*(0 \mid y_1, y_2, x)$, denoted as $y_2 \succ y_1$. Define a parameterized conditional comparison distribution $p_\theta(z|y_1,y_2,x)=\omega(r_\theta(x,y_{2-z}),r_\theta(x,y_{z+1}))$. Assume $\omega(x,y)$ is injective with respect to $(x,y)$ and satisfies the symmetric complementarity property of $\omega(x,y)=1-\omega(y,x)$, which means that the probability of $x$ winning against $y$ is equal to the probability of $y$ losing against $x$. Then we have:
\begin{equation}\label{PRA_eq}
\small{    \begin{aligned}
\mathcal{L}_{\mathrm{PRA}}(\pi_\theta)=&\mathbb{E}_{\mathcal{D}_{\theta}}\left[\mathrm{D}_{\mathrm{KL}}(p^*(z|y_1,y_2,x)||p_\theta(z|y_1,y_2,x))\right]\\
=&-\mathbb{E}_{(x,y_w,y_l)\sim \mathcal{D}_{\theta}}\left[\log \left( p_\theta(1|y_w,y_l,x)\right)\right]+\mathbb{E}_{x\sim \mathcal{D},y_1,y_2\sim \pi_\theta(y|x)}\left[M(x,y_1,y_2)\right].
\end{aligned}}
\end{equation}
Defer the proof in Appendix \ref{Derivation_PRA}. $\mathcal{D}_{\theta}\triangleq \{(x,y_w,y_l)|x\sim\mathcal{D},y_w,y_l\sim\pi_\theta(\cdot|x),(y_w\succ y_l)\sim p^*(1|y_w,y_l,x)\}$ and $M(x,y_1,y_2)=\sum_{z=0,1}p^*(z|y_1,y_2,x)\log p^*(z|y_1,y_2,x)$. 
PRA uses the KL divergence between $p^*$ and $p_\theta$ as the loss function, combining a cross-entropy term that avoids computing gradients of the reward function or $p^*$, and a regularization term involving integration over $p^*$ and $\pi_\theta$.
% \begin{figure*}[!ht]
% \begin{center}
% \centerline{\includegraphics[width=1.8\columnwidth]{figure/Tree.pdf}}
% % \vskip -0.2in
% \caption{The algorithm Tree containing different algorithms like DPO, DRO, IPO, SAC, etc.}
% \label{Tree}
% \end{center}
% % \vskip -0.3in
% \end{figure*}

Similar to the RA-P method, we also propose a posterior version of PRA. Define $\bar{p}_\theta(z|y_1,y_2,x)=\omega(\bar{r}_\theta(x,y_{2-z}),\bar{r}_\theta(x,y_{z+1}))$. Based on Eq.\ref{PRA_eq}, we have:
\begin{equation}\label{PRA_posterior}
\small{    \begin{aligned}
\mathcal{L}_{\mathrm{PRA-P}}(\pi_\theta)=&\mathbb{E}_{\mathcal{D}_{\theta}}\left[\mathrm{D}_{\mathrm{KL}}(p^*(z|y_1,y_2,x)||\bar{p}_\theta(z|y_1,y_2,x))\right]\\
=&-\mathbb{E}_{(x,y_w,y_l)\sim \mathcal{D}_{\theta}}\left[\log \bar{p}_{\theta}\left(1|y_w, y_l,x\right)\right]+\mathbb{E}_{x\sim \mathcal{D},y_1,y_2\sim \pi_\theta(y|x)}\left[M(x,y_1,y_2)\right].
\end{aligned}}
\end{equation}
The method using Eq.\ref{PRA_posterior} as the loss function is name \textbf{{Preference Reward Approximation-Posterior} (PRA-P)} method. 
%Firstly, the PRA and PRA-P methods avoid the calculation of $\omega^{-1}$ while maintaining the consistency of the target distribution with the previous methods. In Theorem \ref{policy_equivalence_3}, we demonstrate that the PRA method and methods such as BDA share the same target distribution, while in Theorem \ref{policy_equivalence_4}, the PRA-P and RA-P methods also share an identical target distribution. Secondly, we can view that the DPO algorithm is a solution for scenario (4). Furthermore, the PRA-P method and the DPO algorithm are closely related. Actually, the first term of Eq.\ref{PRA_posterior} is exactly the loss function of DPO \cite{DPO} only if changing $\mathcal{D}_{\theta}$ into an offline dataset, although doing so will bring about the problem of distribution shift. We will discuss this in detail in Theorem \ref{DPO_PRA}.
Firstly, PRA and PRA-P avoid calculating $\omega^{-1}$ while maintaining the same target distribution as previous methods. Theorem \ref{policy_equivalence_3} shows that PRA shares its target distribution with methods like BDA, while Theorem \ref{policy_equivalence_4} demonstrates that PRA-P and RA-P share the same target distribution. Secondly, the DPO algorithm aligns with scenario (4) and is closely related to PRA-P. In fact, the first term of Eq.\ref{PRA_posterior} matches the DPO loss \cite{DPO} if $\mathcal{D}_{\theta}$ is replaced with an offline dataset, though this introduces a distribution shift issue, which is discussed in Theorem \ref{DPO_PRA}.

\subsection{Summary}\label{Tree_sec}
%这一subsection是summary 四个方法和当前的RL与RLHF算法的对应,作为3.2 overview的内容,3.2一边总结,一边说他们的关系

% \subsection{Summary}\label{Tree_sec}
% %这一subsection是summary  因为rl通常要解决的是0阶优化的问题,beyond前面的target distribution和se。还应该考虑算法利用了reward的1阶还是0阶信息,
%Our framework demonstrates not only logical coherence but also empirical validity. After derived the framework for designing the solution method for Problem \ref{BDA} under different reward function requirements from a logical structural perspective, we observe that prior studies have explored specific methodologies corresponding to one of the four scenarios mentioned earlier and proposed efficient algorithms accordingly. For example, as discussed in Section \ref{BDA_sec}, the loss function of the SAC \cite{SAC} algorithm aligns with the loss function of the Forward-BDA method. 
Our framework is both logically reasonable and empirically valid. After developing the solution method for Problem \ref{BDA} based on different reward function scenarios, we found that previous studies have explored specific methodologies for the some scenarios and proposed efficient algorithms. For example, as shown in Section \ref{BDA_sec}, the loss function of SAC (Eq.10 in \cite{SAC}) aligns with the Forward-BDA method. To highlight these relationships, we summarize the correspondence between these algorithms and our methods in Table \ref{Corresponding_Algorithms}. Due to space constraints we only include a few algorithms to demonstrate the practical significance of our framework.
% To clarify these relationships, we have organized the correspondence between these algorithms and our proposed methods in Table \ref{Corresponding_Algorithms}. Please note that we only list a small number of algorithms to prove that our framework has practical significance rather than a simple theoretical deduction, considering the length of the article.

\begin{table}[ht] %[width=1\textwidth,cols=4,pos=h]
\centering
   \resizebox{0.4\hsize}{!}{
   \begin{tabular}{cc}%{\tblwidth}{@{}CCCCCC@{}}
   \toprule
    Method & \#Corresponding Algorithms  \\
   \midrule
    Forward-BDA & SAC \cite{SAC}, DPG \cite{DPG} \\
    Reverse-BDA & RERPI \cite{RERPI} \\
    RA (RA-P)  & DRO \cite{DRO_Deepmind} \\
    RDA (RDA-P) & IPO \cite{IPO}, SVPO \cite{SVPO} \\
    PRA (PRA-P) & DPO \cite{DPO} \\
   \bottomrule
  \end{tabular}
  }
  % \vspace{-0.05in}
  \caption{Correspondence between the series of methods in our Framework and existing algorithms.}
  \label{Corresponding_Algorithms}
\end{table}
For scenario (1), the Reverse-BDA method, using reverse-KL (Eq.\ref{reverse-KL}) as a loss function for policy improvement, has been applied in finite state-action space control problems (see Eq.2 in RERPI\cite{RERPI}). However, in continuous state-action spaces, the term $ Z(x) $ is hard to compute, limiting the method's applicability. For scenario (2), the loss function of the Direct Reward Optimization (DRO) method matches the RA-P method’s loss function (Eq.\ref{RA_eq_posterior}), as shown the Eq.4 in \cite{DRO_Deepmind}. % For scenario (1), the Reverse-BDA method, utilizing the reverse-KL Eq.\ref{reverse-KL} as a loss function for policy improvement has been explored within the control problems of finite state-action spaces (See Eq.2 in RERPI\cite{RERPI}). In continuous state-action spaces, however, the term $ Z(x) $ cannot be accurately computed, limiting the development of this method. For scenario (2), the loss function of the Direct Reward Optimization (DRO) method proposed by Deepmind, corresponds to the loss function (Eq.\ref{RA_eq_posterior}) of RA-P method (see Eq.4 in \cite{DRO_Deepmind} where $V(x)=\frac{1}{\tau}\log Z'(x)$). 
For scenario (3), the Step-level Value Preference Optimization (SVPO) algorithm \cite{SVPO} uses $\Delta r_\pi\left(\mathbf{s}_{t+1}^w, \mathbf{s}_{t+1}^l\right)$ to learn towards $\operatorname{sg}\left[\Delta r_\phi\left(\mathbf{s}_{t+1}^w, \mathbf{s}_{t+1}^l\right)\right]$ (Eq.10 in \cite{SVPO}), which corresponds to our RDA-P method. The RDA-P method is also a generalized version of Identity-PO (IPO) \cite{IPO}, relaxing the data requirement from querying reward differences to only ranking queries (i.e., $\mathbb{I}(r(x, y_1) - r(x, y_2))$, where $\mathbb{I}$ is the indicator function). Though $\mathbb{I}(r(x,y_1)-r(x,y_2))$ is more like a preference, we classify IPO as a relaxed version of RDA due to its similar form as RDA-P. Finally, for scenario (4), we will show in Theorem \ref{DPO_PRA} that DPO is simply an offline version of PRA-P.

% For scenario (3), the Step-level Value Preference Optimization (SVPO) algorithm \cite{SVPO} uses $\Delta r_\pi\left(\mathbf{s}_{t+1}^w, \mathbf{s}_{t+1}^l\right)$ to learning towards $\operatorname{sg}\left[\Delta r_\phi\left(\mathbf{s}_{t+1}^w, \mathbf{s}_{t+1}^l\right)\right]$ (See Eq.10 in \cite{SVPO}), corresponding to our RDA-P method. And, the RDA-P method is a also generalized version of Identity-PO (IPO) \cite{IPO} by relaxing the data requirement from querying the reward difference into only querying the ranking (i.e. $\mathbb{I}(r(x,y_1)-r(x,y_2))$, where $\mathbb{I}$ is the indicative function) between $r(x,y_1)$ and $r(x,y_2)$, see Eq.16 in paper \cite{IPO}. Although $\mathbb{I}(r(x,y_1)-r(x,y_2))$ is more like a preference, considering that the highlight of IPO lies in its form and that the form is more similar to the RDA method, we classify IPO as a relaxed version of the RDA method here. And lastly for scenario (4), we will strictly introduce that DPO is just an offline version of PRA-P in Theorem \ref{DPO_PRA}.


%In the previous subsection, we proposed distinct methods for four different scenarios, each corresponding to loss functions of varying form. In this subsection, we summarize the relationships between these algorithms, as well as their connections to existing algorithms, offering a classification and correspondence. Please note that the RL problems are typically a zero-order optimization problems. To address this, algorithms have evolved to incorporate both zero-order and first-order methods. Therefore, when classifying these algorithms, the order must be taken into account. Based on the target distribution, different scenarios, and the algorithmic order, we present an overview of an algorithmic tree to illustrate these relationships as illustrated in Figure \ref{Tree}. The leaf nodes of this tree represent well-established methods, such as DPO and IPO, while others correspond to promising, yet unexplored approaches. This classification aims to offer a comprehensive perspective for understanding these methods and to guide future developments in algorithmic research.

% At the first level of the tree, algorithms are categorized based on whether they utilize first-order derivative information from the reward function. The second level of the classification focuses on the target distribution ($\pi^\delta$ and $\pi^\tau$) used by the algorithm. Finally, at the third level, we further distinguish the algorithms according to the four scenarios outlined in Section \ref{Framework_sec} of our framework.

% For the nodes associated with zero-order methods and those that use $\pi^\tau$ as the target distribution, all scenarios have already been explored in either the RLHF domain or the typical RL domain. In contrast, algorithms like SAC and DPG directly calculate the gradient of the policy $\pi_\theta$ by differentiating the Q-network, which models the sum of the reward function. Therefore, in our classification, the SAC series of algorithms are categorized as first-order methods.

% For the nodes under other categories, there is still insufficient research to fully explore these algorithms. From the perspective of optimizing Problem (\ref{RLOpt}), methods that use $\pi^\delta$ as the target distribution are more effective in solving this problem. However, current algorithms primarily focus on using $\pi^\tau$ as the target distribution. Our framework can serve as a principled approach to inspire both the RL and RLHF communities to develop more algorithms based on $\pi^\delta$ as the target distribution. Such methods would theoretically ensure that large language models (LLMs) can more accurately achieve the optimal solution (i.e., $\pi^\delta$) for Problem (\ref{RLOpt}).



\section{Target Distribution Analysis}\label{RL_Classification_sec}
In this section, we analyze the target distribution of the loss function mentioned in our framework. Although these methods have different requirements for the reward function, the same versions (e.g. the posterior version) all share the same target distribution. Additionally, as noted in Section \ref{Framework_sec}, we highlight how using an offline dataset in DPO introduces distribution shift issues, distinguishing it from online methods like SAC and PPO-RLHF.

% In this section, we will analyze the target distribution of the loss function for algorithm in our Framework (see Figure \ref{Framework_figure}) containing DPO. Our analysis reveals that the target distribution of DPO aligns more closely with that of the SAC algorithm, rather than PPO-RL. This finding suggests that DPO's behavior fundamentally differs from typical RL methods such as PPO-RL. Additionally, as mentioned at the end of Section \ref{Framework_sec}, we highlight how the use of an offline dataset in DPO introduces distribution shift issues, distinguishing it from the online learning settings of methods like SAC and PPO.

\subsection{The Target Distribution Equivalence of Our Framework}
Here we state the approximation equivalence among the BDA, RA, RDA and PRA methods. % Recall Eq.\ref{forward-KL} as $\mathcal{L}_{\mathrm{Forward-BDA}}(\pi_\theta)$, Eq.\ref{reverse-KL} as $\mathcal{L}_{\mathrm{Reverse-BDA}}(\pi_\theta)$ and Eq.\ref{RA_eq} as $\mathcal{L}_{\mathrm{RA}}(\pi_\theta)$.

\begin{theorem}\label{policy_equivalence_1}
 Define $\pi^*_{\mathrm{RA}}=\arg\min_{\pi_\theta}\mathcal{L}_{\mathrm{RA}}(\pi_\theta),\ \pi^*_{\mathrm{Forward-BDA}}=\arg\min_{\pi_\theta}\mathcal{L}_{\mathrm{Forward-BDA}}(\pi_\theta),\ \pi^*_{\mathrm{Reverse-BDA}}=\arg\min_{\pi_\theta}\mathcal{L}_{\mathrm{Reverse-BDA}}(\pi_\theta)$.
The following property holds:
\begin{equation}
    \pi^*_{\mathrm{Forward-BDA}} = \pi^*_{\mathrm{Reverse-BDA}} = \pi^*_{\mathrm{RA}} = \pi^\tau.
\end{equation}
\end{theorem}

See proof on Appendix \ref{Proof_Reward_policy}. Theorem \ref{policy_equivalence_1} demonstrate that the ``$\arg\min$'' of Eq.\ref{forward-KL}, \ref{reverse-KL}, \ref{RA_eq} are equivalent and equal to $\pi^\tau$. 

\begin{theorem}\label{policy_equivalence_2}
Recall $\mathcal{L}_{\mathrm{RDA}}(\pi_\theta)$ in Eq.\ref{pair-wised_RA_eq}. Define $\pi^*_{\mathrm{RDA}}=\arg\min_{\pi_\theta}\mathcal{L}_{\mathrm{RDA}}(\pi_\theta).$
Then $\pi^*_{\mathrm{RDA}} = \pi^\tau$.
\end{theorem}
Theorem \ref{policy_equivalence_2} proves that the target distribution corresponding to the loss function of the RDA method is $\pi^\tau$, which is consistent with the target distribution of the BDA and RA methods. See proof in Appendix \ref{Proof_RDA_lemma}.

\begin{theorem}\label{policy_equivalence_3}
Recall $\mathcal{L}_{\mathrm{PRA}}(\pi_\theta)$ in Eq.\ref{PRA_eq}. Define $\pi^*_{\mathrm{PRA}}=\arg\min_{\pi_\theta}\mathcal{L}_{\mathrm{PRA}}(\pi_\theta)$.
Then $\pi^*_{\mathrm{PRA}} = \pi^\tau$.
\end{theorem}
Theorem \ref{policy_equivalence_3} shows that although the loss function of the PRA method is designed by minimizing the distance between the two distributions $p^*, p_\theta$, its target distribution remains $\pi^\tau$. Defer proof in Appendix \ref{PRA_optimal_proof}. Similarly, we use Theorem \ref{policy_equivalence_4} to demonstrate the relationship between the target distributions of the posterior versions of the RA and PRA methods. See the proof in Appendix \ref{Reward_policy_posterior_proof}.

\begin{theorem}\label{policy_equivalence_4}
Recall $\mathcal{L}_{\mathrm{RA-P}}(\pi_\theta)$ in Eq.\ref{RA_eq_posterior} and $\mathcal{L}_{\mathrm{PRA-P}}(\pi_\theta)$ in Eq.\ref{PRA_posterior}. Define 
\begin{equation}
    \begin{aligned}
&\pi^*_{\mathrm{RA-P}}=\arg\min_{\pi_\theta}\mathcal{L}_{\mathrm{RA-P}}(\pi_\theta),\ \pi^*_{\mathrm{PRA-P}}=\arg\min_{\pi_\theta}\mathcal{L}_{\mathrm{PRA-P}}(\pi_\theta).
    \end{aligned}
\end{equation}
The following property holds:
\begin{equation}
    \pi^*_{\mathrm{RA-P}} = \pi^*_{\mathrm{PRA-P}} = \bar{\pi}^\tau.
\end{equation}
\end{theorem}
In summary, while the BDA, RA, RDA, and PRA methods are designed for different scenarios, they all share the same target distribution and aim to solve the same problem (BDAP) under varying conditions of reward function access. Specifically, BDA and RA require exact reward values, RDA only needs reward differences across variables $y$, and PRA relies on ordinal relationships of the reward function. Based on our analyzing about the methods in our UDRRA framework, we can more accurately select the appropriate method for different scenarios. %Further distinctions among these methods will be discussed in Section \ref{Convergence_sec}.

% In summary, while the BDA, RA, RDA, and PRA methods are tailored to address four distinct scenarios, they share the same target distribution. In other words, these methods aim to solve the same problem (BDAP) but under different conditions regarding how the reward function is accessed. Specifically, the BDA and RA methods necessitate the exact values of the reward function, the RDA method only requires the differences in the reward function across different variables $y$, while the PRA method relies solely on the ordinal relationships of the reward function across different variables $y$. By consolidating these methods within our framework, we enable more precise selection of the appropriate method for various practical scenarios. In Section \ref{Convergence_sec}, we will further elaborate on the distinctions among these methods.

\subsection{The Relationship between DPO and PRA-P (The Target Distribution of DPO)}\label{DPO_PRA_P_sec}
% 分析distribution shift的问题。其他的内容往后面放。
This section analyzes the target distribution of DPO. DPO mentions that the DPO target distribution is $\bar{\pi}^\tau$ which is also the the target distribution of the PRA-P method (Eq.4 in \cite{DPO}). However, we will show that there is a distribution shift between the the target distribution $\bar{\pi}^\tau$ of the PRA-P method and the the target distribution of DPO. We use Theorem \ref{DPO_PRA} to strictly analyze the differences and connections between DPO and PRA-P. 
\begin{theorem}
    \label{DPO_PRA}
% Recall $$
% \small{\begin{aligned}
% &\mathcal{L}_{\mathrm{DPO}}\left(\pi_\theta\right) =\mathbb{E}_{\mathcal{D}_R}\left[-\log \sigma\left(\bar{h}_\theta\left(x, y_w, y_l\right)\right)\right], \\
% &\mathcal{L}_{\mathrm{PRA-P}}\left(\pi_\theta\right) =\mathbb{E}_{\mathcal{D}_{\theta}}\left[\mathrm{D}_{\mathrm{KL}}(p^*(z|y_1,y_2,x)||\bar{p}_\theta(z|y_1,y_2,x))\right].
%     \end{aligned}}$$
%where $\mathcal{D}_{R}\triangleq \{(x,y_w,y_l)|x\sim\mathcal{D},y_w,y_l\sim\pi_0(\cdot|x),(y_w\succ y_l)\sim p^*(1|y_w,y_l,x)\}$ and $\mathcal{D}_{\theta}\triangleq \{(x,y_1,y_2)|x\sim\mathcal{D},y_1,y_2\sim\pi_\theta(\cdot|x),(y_1\succ y_2)\sim p^*(1|y_1,y_2,x)\}$. 
When $p^*$ is modeled by the BT model:
\begin{equation}\label{BTmodel}
\begin{aligned}
p^*(1|y_1,y_2,x) &= \omega(r(x,y_{2-z}),r(x,y_{z+1}))=\sigma(r(x,y_1)- r(x,y_2) ),
\end{aligned}
\end{equation}
then we have the following equality:
\begin{equation}
\small{\mathcal{L}_{\mathrm{PRA-P}}\left(\pi_\theta\right) = \mathcal{L}_{\mathrm{DPO}}\left(\pi_\theta\right) + \eta_1(\pi_\theta, \pi_0) + \eta_2(\pi_\theta),}
\end{equation}
where $\eta_1(\pi_\theta, \pi_0)$ equals to 0 if and only if $\pi_\theta=\pi_0$ and $\eta_2(\pi_\theta)\triangleq\mathbb{E}_{x\sim \mathcal{D},y_1,y_2\sim \pi_\theta(y|x)}\left[M(x,y_1,y_2)\right]$. 
\end{theorem}
Theorem \ref{DPO_PRA} tells us that since $\mathcal{D}_{R}$ in DPO is an offline dataset, a shift term $\eta_1(\pi_\theta, \pi_0)$ is introduced between DPO and PRA-P. Because of the difference between $\pi_\theta$ and $\pi_0$, the existence of the non-zero term $\eta_1(\pi_\theta, \pi_0)$ makes the target distribution of DPO deviate from the target distribution $\bar{\pi}^\tau$ of the PRA-P method. We call this phenomenon as distribution shift. Defer the proofs to Appendix \ref{DPO_PRA_proof}. 
Moreover, $\eta_2(\pi_\theta)$ is a regularization term which serves to increase the information entropy ($-M(x,y_1,y_2)$) of the comparison probability $p^*$ under the current policy distribution $\pi_\theta$. The ignorance of this regularization term will make the DPO deviate further from the target distribution $\bar{\pi}^\tau$. Lastly, we emphasize that the $\omega$ function in the PRA method only needs to satisfy the symmetric complementarity property and is not limited to the BT model, which is a special case. We list more $\omega$ functions meeting this requirement in Table \ref{omega_table}.


% Although DPO has achieved higher efficiency on some open source datasets comparing PPO based RLHF, DPO is still challenging for being used in the state-of-the-art production-level LLMs, implying its potential pathologies \cite{yan20243d}. In this part, we analyze the potential pathologies of DPO from the relationship between DPO and RL.

% Firstly, from the perspective of the target distribution, in the original PPO-RL algorithm, the two policies in the KL divergence constraints are continuously updated during the training process, so the final policy will converge to $\pi^\delta$ instead of $\pi^\tau$ or $\bar{\pi}^\tau$ which is supported by TRPO. However, in DPO and PPO-RLHF, the reference policy $\pi_{ref}$ is fixed, causing the target distribution of DPO without distribution shift to become $\bar{\pi}^\tau$, which will be proved in the second property in Appendix \ref{DPO_theorem_proof}.


% Secondly, from the perspective of the PRA-P method in our framework, because DPO radicalizes the distribution $\mathcal{D}_{\theta}$ to another distribution $\mathcal{D}_R$ which is not related to policy $\pi_\theta$, thus DPO retains the Eq.\ref{PRA_eq}'s main term and removes the regularization term, see details in Appendix \ref{DPO_eq_from_PRA_proof}. The second term of Eq.\ref{PRA_posterior} is a regularization term which serves to increase the information entropy of the comparison probability $p^*$ under the current policy distribution $\pi_\theta$. The ignorance of this regularization term will make the target distribution of the DPO algorithm align with that of the PRA-P algorithm with a bias of distribution shift. This observation explains why the DPO algorithm, which omits this regularization term, tends to overfit \cite{xu2024dpo,fisch2024robust}.


\section{Components Influence in Algorithms within UDRRA}\label{Convergence_sec}% In the previous sections, we primarily focused on our framework and explored the relationship between the DPO algorithm and the PRA-P method in Scenario (4) of our framework. Our framework addresses the requirements of different reward functions across four distinct scenarios, which are represented by the BDA, RA, RDA, and PRA methods. Some of these methods correspond to classic RL or RLHF algorithms, while others remain under-explored, despite the fact that, from a logical perspective, these algorithms should be equally deserving of further investigation.

%The DPO algorithm, as a representative method in RLHF, has received considerable attention in recent research. In Section \ref{DPO_PRA_P_sec}, we analyzed the relationship between the DPO algorithm and the PRA-P algorithm. 
%In this section, we further investigate the theoretical properties of the DPO algorithm and discuss how different components of the DPO loss function influence the algorithm's performance. Our analysis focuses on two issues: 
After analyzing the connections and distinctions between algorithms like DPO, the remaining question is how the common components in algorithms within UDRRA affect their performance. %Are they similar or different? 
Here we focus on the impact of hyperparameters $\tau$ and preference datasets on algorithm performance.

% In this section, we further discuss Q1: How $\tau$ influence the convergence rate of DPO? Q2: how to choose the winner response and loser response, or what kind of offline dataset should we choose to speed up the training of DPO?
% \begin{itemize}
% \item Q1: How $\tau$ influence the convergence rate of DPO?
% \item Q2: As for DPO's offline dataset, how to choose the winner response and loser response, or what kind of data should we choose to speed up the training of DPO?
% \end{itemize}

\subsection{Q1: The Influence of $\tau$ on Algorithm Performance}\label{DPO_rate}
For Q1, Increasing hyper-parameter $\tau$ can speed up the convergence of DPO, which will formulate in Theorem \ref{DPO_theorem}. As $ \tau $ increases, the optimal solution $\bar{\pi}^\tau$ of DPO tends to $ \pi^\delta $ (See Proposition \ref{tau_tend_to_delta}), while deviating from $ \pi_{\text{ref}} $. Thus there exists a trade-off between the convergence rate and maintaining proximity to $ \pi_{\text{ref}}$. 
\begin{theorem}\label{DPO_theorem}
Assume $\pi_\theta$ is constructed by Definition \ref{Softmax}. Consider the DPO loss function  $\mathcal{L}_{\mathrm{DPO}}(\pi_{\theta} ; \pi_{\text {ref }})$. Given the learning rate $\alpha_t$ satisfying finite squared summability, suppose the parameters $\theta$ are updated by:
\begin{equation}\label{SGD_update}
    \theta_{t+1}=\theta_{t} - \alpha_t g(x,y_w,y_l,\theta_t),
\end{equation}
where $g(x,y_w,y_l,\theta_t)$ is a stochastic gradient of $\mathcal{L}_{\mathrm{DPO}}(\pi_{\theta_t} ; \pi_{\text {ref }})$. Assume $\|g(\cdot,\cdot,\cdot,\cdot)\|^2\leq G^2$. Denote $\mathcal{L}_{\mathrm{DPO}}^*=\min_{\pi_\theta} \mathcal{L}_{\mathrm{DPO}}(\pi_{\theta} ; \pi_{\text {ref }})$, then:
\begin{equation}
{    \begin{aligned}
\min_{1\leq i\leq T}&||\nabla_\theta \mathcal{L}_{\mathrm{DPO}}(\pi_{\theta_i} ; \pi_{\text {ref }})||^2_2\leq \frac{2G^2\sum_{t=1}^{T-1}\alpha_t^2}{\tau^2\sum_{t=1}^{T-1}\alpha_t} + \frac{\mathcal{L}_{\mathrm{DPO}}(\pi_{\theta_1})-\mathcal{L}_{\mathrm{DPO}}^*}{\sum_{t=1}^{T-1}\alpha_t}.
    \end{aligned}}
\end{equation}
\end{theorem}
See the proof in Appendix \ref{DPO_theorem_proof}. We assume the policy is a softmax policy (Definition \ref{Softmax}) and show that the gradient norm of the DPO loss function has an upper bound under non-convexity, which is negatively correlated with the hyper-parameter $\tau$. Thus we have that $\tau$ is inversely related to the convergence rate of DPO.

% See proof in Appendix \ref{DPO_theorem_proof}. Our proof idea is to assume that the strategy is a softmax policy (Definition \ref{Softmax}) and prove the upper bound of the gradient norm of the loss function of the DPO algorithm under the assumption of non-convexity. The upper bound is negatively correlated with the hyper-parameter $\tau$. For Q1, Theorem \ref{DPO_theorem}.(2) demonstrate that $\tau$ has an inverse relationship with the convergence rate of DPO.
This paper focuses on the impact of $\tau$, rather than the design of the LLMs. Thus, we simplify Theorem \ref{DPO_theorem} to apply the convergence theorem for SGD in non-convex settings. Our argument is that, as long as the policy distribution is a softmax policy (Definition \ref{Softmax}), we show that the loss functions for all four methods in our framework, including DPO, are $L$-smooth with respect to the policy parameters $\theta$ leading to a sub-linear convergence rate of $O(\frac{1}{T})$.

% We hope to focus on the impact of different components in the DPO algorithm rather than focusing on the impact on the complex model of LLM itself. Thus we essentially simplify Theorem \ref{DPO_theorem} into an application of the convergence theorem for SGD in non-convex settings. Our argument is that, as long as the policy distribution is assumed to be a softmax policy (Definition \ref{Softmax}), the loss functions for all four methods in our framework, as well as for DPO, can be shown to be $L$-smooth with respect to the policy parameters $\theta$. Consequently, it can be proven that these methods exhibit a sub-linear convergence rate, i.e. $O(\frac{1}{T})$.
\begin{table*}[ht] %[width=1\textwidth,cols=4,pos=h]
\centering
   \resizebox{0.85\hsize}{!}{
   \begin{tabular}{cccc}%{\tblwidth}{@{}CCCCCC@{}}
   \toprule
    Method & \#Loss Functions & \#Smooth Coefficient ($L$) & \#$p(x,y;\pi_\theta)$ \\
   \midrule
    Forward-BDA & Eq.\ref{forward-KL}   & $6\epsilon_1+10$ & $\pi_\theta(y|x), x\sim\mathcal{D}$ \\
    Reverse-BDA  & Eq.\ref{reverse-KL}   & $2$ & $\pi^\tau(y|x), x\sim\mathcal{D}$\\
    RA & Eq.\ref{RA_eq} and Eq.\ref{RA_eq_posterior}   & $3\epsilon_1^2 + \frac{18\epsilon_1}{\tau} + \frac{8}{\tau^2} + \max\left\{ \epsilon_1^2 + \frac{2}{\tau} \epsilon_1, \frac{1}{\tau} \right\}$  & $\pi_\theta(y|x), x\sim\mathcal{D}$ \\
    RDA  & Eq.\ref{pair-wised_RA_eq}   & $20\epsilon_2^2+\frac{32\epsilon_2}{\tau}+\frac{8}{\tau^2}$  & $\pi_\theta(y|x), x\sim\mathcal{D}$ \\
    PRA  & Eq.\ref{PRA_eq} and Eq.\ref{PRA_posterior}   & $20\log(1+ e^{\frac{d}{\tau}})+\frac{16\epsilon_3}{\tau}+\frac{4}{\tau^2}+16\log2$  & $\pi_\theta(y|x), x\sim\mathcal{D}$ \\
    DPO & Eq.\ref{DPO_eq}  & $\frac{4}{\tau^2}$  & $\pi_0(y|x), x\sim\mathcal{D}$ \\
   \bottomrule
  \end{tabular}
  }
  % \vspace{-0.1in}
  \caption{Smooth Coefficients for different methods. $\pi_\theta$ is the current policy to be optimized, and $\pi_0$ is the sampling distribution for DPO's offline dataset. Let $|\log(\pi_\theta(y|x)) - \log(\pi^\tau(y|x))| \leq \epsilon_1$, where $\epsilon_1$ represents the maximum likelihood difference between $\pi_\theta$ and $\pi^\tau$. Let $|(r_\theta(x,y_1) - r_\theta(x,y_2)) - (r(x,y_1) - r(x,y_2))| \leq \epsilon_2$, where $\epsilon_2$ is the maximum deviation between $r_\theta$ and $r$. Define $d$ as the diameter of the smallest manifold sphere containing the domain of $\theta$ (as per Definition \ref{Softmax}), i.e. $|\theta(x_1,y_1) - \theta(x_2,y_2)| \leq d$. Similarly, let $|p^*(z|y_1,y_2,x) - p_\theta(z|y_1,y_2,x)| \leq \epsilon_3$, where $\epsilon_3$ bounds the difference between $\bar{p}_\theta(z|y_1,y_2,x)$ and $p^*(z|x,y_1,y_2)$. %$\pi_\theta$ is the current parameterized policy to be optimized. $\pi_0$ the sampling distribution for DPO's offline dataset. Let $|\log(\pi_\theta(y|x)) - \log(\pi^\tau(y|x))|\leq\epsilon_1$ for all $x\in\mathbb{X},y\in\mathbb{Y}$ where $\epsilon_1$ represents the maximum likelihood probability values between the initial distribution $\pi_\theta$ and the target distribution $\pi^\tau$. Let $|(r_\theta(x,y_1)-r_\theta(x,y_2))-(r(x,y_1)-r(x,y_2))|\leq\epsilon_2$ for all $x\in\mathbb{X},y_1,y_2\in\mathbb{Y}$ where $\epsilon_2$ represents the maximum distance on pair-wised data in reward function values between the parameterized reward function $r_\theta(x,y)$ and the target reward function $r(x,y)$. Let $|\theta(x_1,y_1)-\theta(x_2,y_2)|\leq d$ for all $x_1,x_2\in\mathbb{X},y_1,y_2\in\mathbb{Y}$ where $d$ represents the diameter of the smallest manifold sphere that contains the domain of the function $\theta$ (See the definition of the function $\theta$ in Definition \ref{Softmax}). Similar to $\epsilon_1,\epsilon_2$, define $|p^* - \sigma(\frac{1}{\tau}\log\frac{\pi_\theta(y_1|x)}{\pi_\theta(y_2|x)}) |\leq \epsilon_3$ representing the upper bound of the differences between $\bar{p}_\theta\left(z|y_1,y_2,x\right)$ and $p^*(z|x,y_1,y_2)$ where $p^*=p^*(1|x,y_1,y_2)$.
  }
  \label{smoothCoefficients}
  \vspace{-0.1in}
\end{table*}

Thus we summarize the convergence properties of DPO and related methods (BDA, RA, RDA, PRA) in Table \ref{smoothCoefficients}, corresponding to Lemma \ref{BDA_L_RKL_coef}-\ref{PRA_L_coef}. Our goal is to compare these methods from a unified perspective, focusing on the smoothness coefficient $L$, which influences convergence. The variations in $L$ arise from differences in how their loss functions are constructed. For Table \ref{smoothCoefficients}, we summarized several interesting conclusions below. 

% Furthermore, we summarize the convergence properties of DPO and other related methods, including BDA, RA, RDA, and PRA, as mentioned before in Table \ref{smoothCoefficients}, corresponding to Lemma \ref{BDA_L_RKL_coef}-\ref{PRA_L_coef}. We aim to study the differences among the methods from a unified perspective. We analyzed the core factor influencing the convergence properties: the smoothness coefficient ``$L$'', and explore how it varies across different algorithms. Essentially the variations of ``$L$'' arise from the differences in the construction patterns of their respective loss functions. For Table \ref{smoothCoefficients}, we summarized several interesting conclusions. 
First, the Forward-BDA method has a larger smooth coefficient ($6\epsilon_1 + 10$) than the Reverse-BDA method, indicating slower convergence but a more sampling-friendly policy $\pi_\theta$. Second, the RA method outperforms both BDA methods by setting $\epsilon_1$ and $\tau$ to reduce the smooth coefficient. For instance, when $\epsilon_1$ and $\tau$ satisfy $0 \leq \epsilon_1\leq\frac{\sqrt{1+\tau}-1}{\tau} \leq \frac{\sqrt{6}-1}{5}$, the smooth coefficient of the RA method is bounded by 2, ensuring faster convergence. Third, the smooth coefficients of the RDA and PRA methods can be reduced by adjusting $\epsilon_2, \epsilon_3, \tau$, and the parameter domain diameter $d$, to speed up their convergence. Lastly, the DPO method has a higher smooth coefficient than PRA, with additional terms like $20\log(1 + e^{\frac{d}{\tau}}) + \frac{16\epsilon_3}{\tau} + 16\log 2$, indicating that DPO converges faster than PRA due to the distribution shift bias.

% Firstly, comparing the Reverse-BDA method, the Forward-BDA method has bigger smooth coefficient ($6\epsilon_1+10$), which means slower convergence rate but with a policy $\pi_\theta$ that can facilitate sampling. Secondly, comparing the two BDA methods, the RA method shows better property that we can control $\epsilon_1$ and $\tau$ to construct a smaller smooth coefficient. For example, when $\epsilon_1$ and $\tau$ satisfy $0\leq \epsilon_1\leq\frac{\sqrt{1+\tau}-1}{\tau}\leq\frac{\sqrt{6}-1}{5}$, then the smooth coefficient of RA method $3\epsilon_1^2 + \frac{18\epsilon_1}{\tau} + \frac{8}{\tau^2} + \max\left\{ \epsilon_1^2 + \frac{2}{\tau} \epsilon_1, \frac{1}{\tau} \right\}\leq 2$, which means the RA method will converge faster than the two BDA methods. Thirdly, we can control $\epsilon_2, \epsilon_3, \tau$ and the diameter $d$ of the domain of the parameter $\theta$ to decrease the smooth coefficient of the RDA and PRA methods and then speed up their convergence rate. Lastly, comparing the smooth coefficient of the DPO algorithm, the one of PRA method has additional value $20\log(1+ e^{\frac{d}{\tau}})+\frac{16\epsilon_3}{\tau}+16\log2$. From this perspective, DPO speed up its convergence rate comparing the PRA method with the bias of distribution shift.

% How Dataset influennce algorithm's convergence rate
\subsection{Q2: The Influence of Preference Dataset on Algorithm Performance}
% For Q2, we use Theorem \ref{data_select_coro} to demonstrate that we should select the $(x, y_1, y_2)$ data pairs where both the reward function values and the current policy probability values consistently exhibit a large margin (i.e. large difference, or distance) between $y_1$ and $y_2$. One the one hand, the larger margin, the faster the convergence speed. On the other hand, the margin consistence between the reward $r(x,y)$ and policy $\pi_\theta$ ensures a smaller distribution shift.

Since DPO is a representative algorithm in RLHF, this paper uses it to answer Q2. Essentially, the design of preference dataset $\mathcal{D}_R$ depends on the design of the sampling strategy $\pi_0$, in other words, the relationship between the sampling strategy $\pi_0$ and the current strategy $\pi_\theta$. 
Thus for Q2, we use Theorem \ref{data_select_coro} to show that data $(x, y_1, y_2)$ should be selected where both the reward values and policy probabilities have a large and consistent margin between $y_1$ and $y_2$. Larger margins accelerate convergence, while consistent margins reduce distribution shift.

To prove Theorem \ref{data_select_coro}, we first consider a simplified question: should the sampling strategy $\pi_0$ prioritize samples with large or small reward differences or margins when identifying winners and losers? The challenge is that $\pi_0$ and $\pi_\theta$ are interdependent, as we optimize $\pi_\theta$. To address this, Lemma \ref{data_select} quantifies their relationship when $\pi_0$ is a uniform distribution. Define event sets $\Omega_1 = \{(y_1, y_2, x) \mid |\log\frac{p^*(1 \mid y_1, y_2, x)}{p^*(0 \mid y_1, y_2, x)}| \geq \epsilon_0\}$ and $\Omega_2 = \{(y_1, y_2, x) \mid |\log\frac{\pi_\theta(y_1 \mid x)\pi_{\text{ref}}(y_2 \mid x)}{\pi_\theta(y_2 \mid x)\pi_{\text{ref}}(y_1 \mid x)}| \geq \epsilon_0\}$. Let $\gamma(x) = \frac{|\Omega_1 \cap \Omega_2|}{K^2}\leq\gamma$, where $K=|\mathbb{Y}|$. $\epsilon_0$ represents the log-probability significance threshold of $p^*$. Intuitively, the parameter $\gamma$ reflects the alignment between $\pi_0$ and $\pi_\theta$ for large-margin data pairs $(x, y_1, y_2)$. 

% To prove Theorem \ref{data_select_coro}, we start with a simplified question: when determining the responses of winners and losers, should the sample strategy $\pi_0$ prioritizes samples with large or small reward differences, in other word, large or small ``margin''? The difficulty of analyzing $\pi_0$ is that since we are optimizing $\pi_\theta$, the relationship between $\pi_0$ and $\pi_\theta$ must be considered. Thus We first quantified this relationship in Lemma \ref{data_select} when $\pi_0$ is a uniform distribution. Define event set $\Omega_1=\{(y_1,y_2,x)|\ |\log\frac{p^*(1|y_1,y_2,x)}{p^*(0|y_1,y_2,x)}|\geq\epsilon_0 \}$ and $\Omega_2=\{(y_1,y_2,x)|\ |\log\frac{\pi_\theta(y_1|x)\pi_{\text{ref}}(y_2|x)}{\pi_\theta(y_2|x)\pi_{\text{ref}}(y_1|x)}|\geq\epsilon_0 \}$. Let $\gamma(x)=\frac{ |\Omega_1\cap\Omega_2| }{K^2}$ where $K=|\mathbb{Y}|$ representing the number of elements in the discrete space $\mathbb{Y}$.
\begin{lemma}\label{data_select}
Assume $\|g_t\|^2\leq G^2$. See $\mathcal{L}_{\mathrm{DPO}}(\pi_{\theta} ; \pi_{\text {ref }})$ on Eq.\ref{DPO_eq}. Given the learning rate $\alpha_t$ satisfying finite squared summability, for DPO, let $\gamma(x)\leq\gamma$, when $\pi_0$ is a uniform distribution, then:
\begin{equation}\label{data_select_eq}
\small{    \begin{aligned}
\min_{1\leq i\leq T}&||\nabla_\theta \mathcal{L}_{\mathrm{DPO}}(\pi_{\theta_i} ; \pi_{\text {ref }})||^2_2\leq \frac{\mathcal{L}_{\mathrm{DPO}}(\pi_{\theta_1})-\mathcal{L}_{\mathrm{DPO}}^*}{\sum_{t=1}^{T-1}\alpha_t} + \frac{2(\gamma c_0+1)G^2\sum_{t=1}^{T-1}\alpha_t^2}{\tau^2\sum_{t=1}^{T-1}\alpha_t}.
    \end{aligned}}
\end{equation} 
where $\mathcal{L}_{\mathrm{DPO}}^*=\min_{\pi_\theta} \mathcal{L}_{\mathrm{DPO}}(\pi_{\theta} ; \pi_{\text {ref }})$ and $c_0=\sigma(\frac{\epsilon_0}{\tau})\sigma(-\frac{\epsilon_0}{\tau})-1\in (-1,0)$. 
\end{lemma}

%\textbf{Remark of Lemma \ref{data_select}:} 
Assuming $\pi_0$ is uniform, Lemma \ref{data_select} prepares for analyzing how sampling certain types of data affects convergence. Since $c_0$ is negative, a higher $\gamma$ value indicates better alignment and faster convergence, as shown in the second term of Eq.\ref{data_select_eq}.

% See proof in Appendix \ref{data_select_proof}. The purpose of assuming that $\pi_0$ is uniformly distributed in Lemma \ref{data_select} is to prepare for the subsequent analysis, where we explore how sampling probabilities for which types of data can accelerate the convergence rate. Intuitively, the parameter $\gamma$ reflects the degree of alignment between $\pi_0$ and $\pi_\theta$ for large-margin data pairs $(x,y_1, y_2)$. Please note because $c_0$ is negative, according to the second term on the right-hand side of Eq.\ref{data_select_eq}, a higher value of $\gamma$ indicates a greater degree of alignment and a faster convergence rate. 

Based on Lemma \ref{data_select}, we further demonstrate with Theorem \ref{data_select_coro} that when the data pairs $(x, y_1, y_2)$ satisfy that $(x, y_1, y_2)\in\Omega_1 \cap \Omega_2$ and are sampled more, the convergence rate can be accelerated. In Theorem \ref{data_select_coro}, $\mu$ represents a preference for selecting data pairs with large margins that are also consistent with $\pi_\theta$. Compared to the upper bound of inequality \ref{data_select_eq}, the upper bound of inequality \ref{data_select_coro_eq} is smaller, indicating that the data selection distribution $\pi_1$ plays a beneficial role in accelerating the DPO optimization process. Thus in practice, we can sample data according to $\pi_1$ to achieve faster convergence for DPO. See proof for Theorem \ref{data_select_coro} in Appendix \ref{data_select_coro_proof}.
\begin{theorem}\label{data_select_coro}
Define joint conditional probability distribution $\pi_1(y_1,y_2|x)= \frac{\mu}{K^2}\ \text{if}\ (x,y_1,y_2)\in(\Omega_1 \cap \Omega_2); \frac{1-\mu\gamma}{(1-\gamma)K^2} \text{else}$ where $\mu\in(0,1)$. Assume $\|g_t\|^2\leq G^2$. Given the learning rate $\alpha_t$  satisfying finite squared summability, for DPO, we have:
\begin{equation}\label{data_select_coro_eq}
\small{    \begin{aligned}
\min_{1\leq i\leq T}&||\nabla_\theta \mathcal{L}_{\mathrm{DPO}}(\pi_{\theta_i} ; \pi_{\text {ref }})||^2_2\leq \frac{\mathcal{L}_{\mathrm{DPO}}(\pi_{\theta_1})-\mathcal{L}_{\mathrm{DPO}}^*}{\sum_{t=1}^{T-1}\alpha_t} + \frac{2(\mu\gamma c_0+1)G^2\sum_{t=1}^{T-1}\alpha_t^2}{\tau^2\sum_{t=1}^{T-1}\alpha_t}.
    \end{aligned}}
\end{equation}
\end{theorem}

% \textbf{Remark of Theorem \ref{data_select_coro}:} %


In summary, we conduct the answers for Q1 and Q2:
\begin{itemize}
\item About Q1: %The parameter $ \tau $ essentially corresponds to the temperature parameter in the Boltzmann distribution. 
A larger value of $ \tau $ leads to faster convergence of DPO. As $ \tau $ increases, the optimal solution $ \bar{\pi}^\tau $ of DPO converges to $ \pi^\delta$, while deviating from $ \pi_{\text{ref}} $. Hence, there is a trade-off between the convergence speed and the preservation of proximity to $\pi_{\text{ref}}$.
\item About Q2: Selecting data where both the current policy probability values ($\pi(y_1|x)$ and $\pi(y_2|x)$) and the reward values ($r(x,y_1)$ and $r(x,y_1)$) consistently exhibit a large distance (i.e. large margin) can accelerate the convergence of the DPO algorithm. And with the consistence, the distribution shift problem will reduce.
\end{itemize}

\section{Conclusion}\label{conclusion_sec}
In this paper, we investigated the connections of DPO, RL and other RLHF algorithms. We proposed UDRRA to link these algorithms through their different reward function scenarios. Then we analyzed the target distributions corresponding to the methods in UDRRA and pointed out that the distribution shift problem of DPO comparing PRA-P. Furthermore, we examined the impact of key components of the DPO loss on algorithm performance. Our findings offer a deeper understanding of DPO’s theoretical positioning and practical implications, providing a foundation for future developments in RLHF algorithms.






% \section*{Acknowledgments}
% This was was supported in part by......

%Bibliography
\bibliographystyle{unsrt}  
\bibliography{references}  





\newpage
\appendix
\section*{Appendix Contents}
\begin{itemize}
\item Appendix \ref{appendix_related_work}: Related Work.
\item Appendix \ref{omega}: Discussion about $p^*(z \mid y_1, y_2, x)=\omega(r(x,y_{2-z}),r(x,y_{z+1}))$.
\item Appendix \ref{Proof_Preliminary}: Proof in Section \ref{Preliminary}.
\item \quad Appendix \ref{tau_tend_to_delta_proof}: Proof of Proposition \ref{tau_tend_to_delta}.
\item \quad Appendix \ref{Softmax_Transform_sec}: {The Softmax Transform of Policy $\pi_\theta$}.
\item Appendix \ref{TRPRA_sec_proof}: Proof in Section \ref{TRPRA_sec}.
\item \quad Appendix \ref{forward-KL_proof}: Derivation of Equation \ref{forward-KL}.
\item \quad Appendix \ref{reverse-KL_proof}: Derivation of Equation \ref{reverse-KL}.
\item \quad Appendix \ref{Derivation_PRA}: Derivation of Equation \ref{PRA_eq}.
\item \quad Appendix \ref{Derivation_RA_eq}: Derivation of Equation \ref{RA_eq}.
\item \quad Appendix \ref{DPO_eq_from_PRA_proof}: Derivation of Equation \ref{DPO_eq}.
\item Appendix \ref{RL_Classification_sec_proof}: Proof in Section \ref{RL_Classification_sec}.
\item \quad Appendix \ref{Proof_Reward_policy}: The Target Distribution Equivalence of Eq.\ref{forward-KL}, Eq.\ref{reverse-KL} and Eq.\ref{RA_eq}.
\item \quad Appendix \ref{Proof_RDA_lemma}: The Target Distribution Equivalence of Eq.\ref{RA_eq} and Eq.\ref{pair-wised_RA_eq}.
\item \quad Appendix \ref{PRA_optimal_proof}: The Target Distribution Equivalence of Eq.\ref{pair-wised_RA_eq} and Eq.\ref{PRA_eq}.
\item \quad Appendix \ref{Reward_policy_posterior_proof}: Proof of The Target Distribution Equivalence of Eq.\ref{RA_eq_posterior} and Eq.\ref{PRA_posterior}.
\item \quad Appendix \ref{DPO_PRA_proof}: Proof of Theorem \ref{DPO_PRA}.
\item Appendix \ref{Convergence_sec_proof}: Proof in Section \ref{Convergence_sec}.
\item \quad Appendix \ref{SGD_theorem_proof}: Proof of Proposition \ref{SGD_theorem}.
\item \quad Appendix \ref{DPO_theorem_proof}: Proof of Theorem \ref{DPO_theorem}.
\item \quad Appendix \ref{data_select_proof}: Proof of Lemma \ref{data_select}.
\item \quad Appendix \ref{data_select_coro_proof}: {Proof of Theorem \ref{data_select_coro}}.
\item \quad Appendix \ref{pq_equal_sec}: {Proposition \ref{pq_equal}}.
\item \quad Appendix \ref{SubLemma_sec}: {Sub Lemma for Lemma \ref{BDA_L_RKL_coef}-\ref{PRA_L_coef}}.
\item \quad Appendix \ref{BDA_L_RKL_coef_proof}: Proof of Lemma \ref{BDA_L_RKL_coef}.
\item \quad Appendix \ref{BDA_L_FKL_coef_proof}: {Proof of Lemma \ref{BDA_L_FKL_coef}}.
\item \quad Appendix \ref{RA_L_coef_proof}: {Proof of Lemma \ref{RA_L_coef}}.
\item \quad Appendix \ref{RDA_L_coef_proof}: {Proof of Lemma \ref{RDA_L_coef}}.
\item \quad Appendix \ref{PRA_L_coef_proof}: {Proof of Lemma \ref{PRA_L_coef}}.
\end{itemize}
\newpage

\section{Related Work}\label{appendix_related_work}
\textbf{Reinforcement Learning from Human Feedback (RLHF).} The Large Language Models (LLMs) \cite{zhao2023survey,chang2024survey,hadi2024large,minaee2024large,hadi2023survey,achiam2023gpt,bubeck2023sparks} is one of the most promising evolutions towards Artificial General Intelligence (AGI). The success of this transformation lies a critical component: Reinforcement Learning from Human Feedback (RLHF) or Human Alignment (HA), which is the final and crucial step in LLMs' training \cite{arumugam2019deep,ouyang2022training,singh2022flava,bai2022training,dai2023safe}. Motivated by the instability, complexity, and incurring significant computational costs of the RLHF process, \cite{DPO} proposed an algorithmic framework for directly optimizing the language model to follow human preferences, namely the DPO algorithm, based on the analysis of the optimal solutions of classic RL algorithm PPO \cite{schulman2017proximal,engstrom2019implementation}. DPO has received widespread attention since its inception, and hundreds of algorithms for improving DPO have been derived within a year, e.g. IPO\cite{IPO}, KTO\cite{KTO}, SimPO\cite{SimPO}, ORPO\cite{ORPO}, etc. 

\textbf{DPO $\&$ PPO discussing.}
The success of DPO \cite{DPO} benefits from the optimal solution analysis of the KL-constrained reward maximization objective (Eq.3 in \cite{DPO}) come from the classical Reinforcement Learning (RL) algorithm, PPO \cite{PPO}. Therefore, the performance difference between the DPO algorithm and the PPO based RLHF method \cite{PPO-basedRLHF} has gradually attracted the attention of researchers in the RL community \cite{ivison2024unpacking}. In general, while the DPO algorithm can fit the static training dataset comparably, it generalizes less effectively than PPO based RLHF \cite{lin2024limited}. While PPO based RLHF usually performs better in the state-of-the-art production-level LLMs \cite{yan20243d}, its correct fine-tuning usually requires more sophisticated techniques \cite{xu2024dpo}.

There are many works that empirically discuss the relationship between the DPO algorithm and the PPO algorithm in RLHF. From the perspective of algorithm design, \cite{xu2024dpo,ivison2024unpacking,yan20243d} pointed out that the PPO algorithm is generally better than the DPO algorithm in human preference alignment tasks, but the PPO algorithm requires various additional tricks, such as advantage normalization, large batch size, and exponential moving average update for the reference model. \cite{zhong2024dpo} combined the advantages of DPO and PPO to create a more effective algorithm named RTO at the token-level. From the perspective of data source, \cite{tang2024understanding} classified algorithms such as DPO as offline algorithms, while the PPO based RLHF is an online algorithm. This demonstration is consistent with our paper. \cite{li2023policy,lin2024limited} showed that compared with the DPO algorithm that only relies on static data sets, the PPO algorithm can use sufficient non-preferred data for policy optimization to significantly improve performance by relying on the generalization ability of its Reward model. 


\section{Discussion about $p^*(z \mid y_1, y_2, x)=\omega(r(x,y_{2-z}),r(x,y_{z+1}))$}\label{omega}
As for modeling the comparison probability $p^*(z \mid y_1, y_2, x)$ between $r(x,y_1)$ and $r(x,y_2)$, i.e. the pair $(x, y_1, y_2, z)$ is sampled with probability $p^*(z \mid y_1, y_2, x)=\omega(r(x,y_{2-z}),r(x,y_{z+1}))$ where $z\in\{0,1\}$, in general, the difficulty of estimating the reward depends on the complexity of $\omega^{-1}$. Table \ref{omega_table} presents several special forms of the $\omega$ function, including cases based on the BT model assumptions \cite{bradley1952rank}, along with an analysis of their relevant properties.
\begin{table}[ht] %[width=1\textwidth,cols=4,pos=h]
\centering
\begin{threeparttable} 
   \resizebox{1\hsize}{!}{
   \begin{tabular}{cccc}%{\tblwidth}{@{}CCCCCC@{}}
   \toprule
    $\omega(r(x,y_{2-z}),r(x,y_{z+1}))$ & \#Requirement about $r(x,y)$ & \#Inverse Formula about $r(x,y_{1})-r(x,y_{2})$ & \#Corresponding Algorithm \\
   \midrule
    \tnote{1}$\quad \frac{\exp(\eta r(x,y_{2-z}))}{\exp(\eta r(x,y_{2-z}))+\exp(\eta r(x,y_{z+1}))}$  & $[-\infty,\infty]$ & $\frac{1}{\eta}\log(\frac{p^*(1 \mid y_1, y_2, x)}{1-p^*(1 \mid y_1, y_2, x)})$ & DPO\cite{DPO} \\
    $\frac{(\eta r(x,y_{2-z}))}{(\eta r(x,y_{2-z}))+(\eta r(x,y_{z+1}))}$  & $[0,\infty]$ & N/A  & TYPO\cite{TYPO}  \\
    \tnote{2}$\quad \frac{1}{2}+\frac{1}{2}\tanh(r(x,y_{2-z})-r(x,y_{z+1}))$  & $[-\infty,\infty]$ & $\frac{1}{2}\log(\frac{p^*(1 \mid y_1, y_2, x)}{1-p^*(1 \mid y_1, y_2, x)})$ & N/A \\
    $\frac{1}{2}+\frac{1}{2}\sin(r(x,y_{2-z})-r(x,y_{z+1}))$  & $[-\frac{\pi}{2},\frac{\pi}{2}]$ & $\arcsin(2p^*(1 \mid y_1, y_2, x)-1)$ & N/A \\
    $\mathbb{I}(r(x,y_{2-z})-r(x,y_{z+1}))$ & $[-\infty,\infty]$ & $ 1 $ & IPO\cite{IPO} \\
    \tnote{4}$\quad \max\{0,1-\eta(r(x,y_{2-z})-r(x,y_{z+1}))\}$ & $[-\infty,\infty]$ & N/A & SLiC\cite{SLiC} \\
    \tnote{3}$\ $\tnote{4}$\quad \frac{\exp(\eta r(x,y_{2-z}))}{\exp(\eta r(x,y_{2-z}))+\exp(\eta r(x,y_{\text{average}}))}$ & $[-\infty,\infty]$ & $\frac{1}{\eta}\log(\frac{p^*(1 \mid y_1, y_2, x)(1-p^*(0 \mid y_1, y_2, x))}{p^*(0 \mid y_1, y_2, x)(1-p^*(1 \mid y_1, y_2, x))})$ & KTO\cite{KTO} \\
    \tnote{4}$\quad \frac{1}{(1+\exp(\eta(r(x,y_{2-z})-r(x,y_{z+1}))))^2}$  & $[-\infty,\infty]$ & $\frac{1}{\eta}\log(\frac{\sqrt{p^*(1 \mid y_1, y_2, x)}}{1-\sqrt{p^*(1 \mid y_1, y_2, x)}})$  & N/A  \\
    \tnote{4}$\quad \exp(\eta(r(x,y_{2-z})-r(x,y_{z+1})))$  & $[-\infty,\infty]$ & $\frac{1}{\eta}\log(p^*(1 \mid y_1, y_2, x))$  & N/A  \\
   \bottomrule
  \end{tabular}
  }
  \begin{tablenotes}
  \footnotesize  
     \item[1] Here is the generalized BT model expression, like other work did, we simplified $\eta=1$ in Eq.\ref{BTmodel}.
     \item[2] $\frac{1}{2}+\frac{1}{2}\tanh(r(x,y_{1})-r(x,y_{2}))$ actually is a BT model expression when $\eta=2$.
     \item[3] $r(x,y_{\text{average}})$ is the reference point $z_{\text{ref}}$ in KTO\cite{KTO}. 
     \item[4] These forms of $\omega$ may not strictly satisfy $p^*(1 \mid y_1, y_2, x)=(1-p^*(0 \mid y_1, y_2, x))$.
   \end{tablenotes}
   \end{threeparttable}       % 添加命令
   % \vspace{-0.1in}
   \caption{Various expressions of function $\omega(r(x,y_{2-z}),r(x,y_{z+1}))$.}
  \label{omega_table}
\end{table}


\section{Proof in Section \ref{Preliminary}}\label{Proof_Preliminary}

\subsection{Proof of Proposition \ref{tau_tend_to_delta}}\label{tau_tend_to_delta_proof}
\textbf{Proposition \ref{tau_tend_to_delta}:} Without loss of generality, assume that the reward function $r(x, y)$ has a unique maximum for any given $x$. Define
$$\small \pi^\delta(y|x) = 
\begin{cases} 
1, & \text{if } y = \arg\max_{y^{\prime}} r(x, y^{\prime}), \\
0, & \text{otherwise}.
\end{cases} \normalsize$$
Then the following limit holds:
$  \lim_{\tau \to \infty} \pi^\tau(y|x) = \pi^\delta(y|x).$  

\textbf{Proof:} Given $x$ and defining $y^* = \arg\max\ r(x,y)$, we consider the limit of $\pi^\tau(y|x)$ at the point $(x, y^*)$ as $\tau$ approaches infinity:
\begin{equation}
    \small{\begin{aligned}
&\lim_{\tau \to \infty} \pi^\tau(y^*|x) = \lim_{\tau \to \infty} \frac{\exp(\tau r(x, y^*))}{\sum_{y \in Y} \exp(\tau r(x, y))} = \frac{1}{\lim_{\tau \to \infty} \sum_{y \in Y} \exp(\tau (r(x, y) - r(x, y^*)))} \\
&= \frac{1}{\sum_{y \in Y} \lim_{\tau \to \infty} \exp(\tau (r(x, y) - r(x, y^*)))} = \frac{1}{1 + 0 + \cdots + 0} = 1.
\end{aligned}}
\end{equation}

Next, for $y \neq y^*$, we consider the limit of $\pi^\tau(y|x)$ at the point $(x, y)$:
\begin{equation}
    \small{\begin{aligned}
&\lim_{\tau \to \infty} \pi^\tau(y|x) = \lim_{\tau \to \infty} \frac{\exp(\tau r(x, y))}{\sum_{y' \in Y} \exp(\tau r(x, y'))} = \frac{1}{\lim_{\tau \to \infty} \sum_{y' \in Y} \exp(\tau (r(x, y') - r(x, y)))} \\
&= \frac{1}{\sum_{y' \in Y} \lim_{\tau \to \infty} \exp(\tau (r(x, y') - r(x, y)))} \\
&= \frac{1}{\sum_{y' \in Y, \ r(x, y') > r(x, y)} \lim_{\tau \to \infty} \exp(\tau (r(x, y') - r(x, y))) + \sum_{y' \in Y, \ r(x, y') \leq r(x, y)} \lim_{\tau \to \infty} \exp(\tau (r(x, y') - r(x, y)))} \\
&= \frac{1}{\sum_{y' \in Y, \ r(x, y') > r(x, y)} \lim_{\tau \to \infty} \exp(\tau (r(x, y') - r(x, y)))} = 0.
\end{aligned}}
\end{equation}

Thus, we conclude that $\lim_{\tau \to \infty} \pi^\tau(y|x) = \pi^\delta(y|x)$. Proof finished.

\subsection{The Softmax Transform of Policy $\pi_\theta$}\label{Softmax_Transform_sec}
Given a policy $\pi: \mathbb{X} \rightarrow \Delta(\mathbb{Y})$, the softmax transform of a vector exponentiates the components of the vector and normalizes it so that the result lies in the simplex. This can be used to transform vectors assigned to state-action pairs into policies:
\begin{definition}
\label{Softmax}
    (Softmax transform.) Given the function $\theta: \mathbb{X} \times \mathbb{Y} \rightarrow$ $\mathbb{R}$, the softmax transform of $\theta$ is defined as $\pi_\theta(\cdot \mid x)\triangleq \operatorname{softmax}(\theta(x, \cdot))$, where for all $y \in \mathbb{Y}$,
\begin{equation}
    \pi_\theta(y \mid x)\triangleq \frac{\exp \{\theta(x, y)\}}{\sum_{y^{\prime}} \exp \left\{\theta\left(x, y^{\prime}\right)\right\}}.
\end{equation}
\end{definition}

We call the values $\theta(x, y)$ the logit values and the function $\theta$ itself a logit function due to its origin in logistic regression. This paper assume the set is parameterized by a softmax function, i.e., $\Pi = \{\pi_\theta \mid \pi_\theta(\cdot \mid x) \triangleq  \operatorname{softmax}(\theta(x, \cdot)), \, \theta : \mathbb{X} \times \mathbb{Y} \to \mathbb{R}\}$, which is commonly used in the proof of the policy gradient theorem \cite{mei2020global}.

\section{Proof in Section \ref{TRPRA_sec}}\label{TRPRA_sec_proof}
\subsection{Derivation of Equation \ref{forward-KL}}\label{forward-KL_proof}
\begin{equation}
    \begin{aligned}  
&\mathbb{E}_{x\sim \mathcal{D}}\left[\mathrm{D}_{\mathrm{KL}}\left(\pi_\theta\left(\cdot \mid x\right) \| \pi^\tau\left(\cdot \mid x\right)\right)\right]=\mathbb{E}_{x\sim \mathcal{D}}[\mathrm{D}_{\mathrm{KL}}(\pi_\theta(y|x)||\frac{\exp(\tau r(x,y))}{Z(x)})] \\ 
=& \mathbb{E}_{x\sim \mathcal{D}}\left[\sum_{y\in Y} \pi_\theta(y|x) \log\left(\frac{\pi_\theta(y|x)}{\frac{\exp(\tau r(x,y))}{Z(x)}}\right)\right] = \mathbb{E}_{x\sim \mathcal{D}}\left[\sum_{y\in Y} \pi_\theta(y|x) \log\left(\frac{Z(x)\pi_\theta(y|x)}{\exp(\tau r(x,y))}\right)\right] \\   
=& \mathbb{E}_{x\sim \mathcal{D}}\left[\sum_{y\in Y} -\pi_\theta(y|x)(\tau r(x,y) - \log(\pi_\theta(y|x))) + \log(Z(x))\right] \\  
=& \mathbb{E}_{x\sim \mathcal{D}}\left[\sum_{y\in Y} -\pi_\theta(y|x)(\tau r(x,y) - \log(\pi_\theta(y|x)))\right] + \mathbb{E}_{x\sim \mathcal{D}}[\log(Z(x))] \\
\equiv&\mathbb{E}_{x\sim \mathcal{D}}\left[\sum_{y\in Y} -\pi_\theta(y|x)(\tau r(x,y) - \log(\pi_\theta(y|x)))\right]\\
=&\mathbb{E}_{x\sim \mathcal{D},y\sim\pi_\theta(y|x)}\left[-(\tau r(x,y) - \log(\pi_\theta(y|x)))\right].
\end{aligned}
\end{equation}
The ``$\equiv$" means that the left and right ends differ by a term that is unrelated to $\theta$.


\subsection{Derivation of Equation \ref{reverse-KL}}\label{reverse-KL_proof}
\begin{equation}
    \begin{aligned}
&\mathbb{E}_{x\sim \mathcal{D}}\left[\mathrm{D}_{\mathrm{KL}}\left( \pi^\tau\left(\cdot \mid x\right) \| \pi_\theta\left(\cdot \mid x\right) \right)\right] \equiv\mathbb{E}_{x\sim \mathcal{D}}[\sum_{y\in Y}\frac{\exp(\tau r(x,y))}{Z(x)}(-\log\pi_\theta(y|x))]\\
=&-\mathbb{E}_{x\sim \mathcal{D},y\sim\pi^\tau\left(\cdot \mid x\right)}[\log\pi_\theta(y|x)]=\mathbb{E}_{x\sim \mathcal{D}}[\sum_{y\in Y}\pi_\theta(y|x)\frac{1}{\pi_\theta(y|x)}\frac{\exp(\tau r(x,y))}{Z(x)}(-\log\pi_\theta(y|x))]\\
=&-\mathbb{E}_{x\sim \mathcal{D},y\sim\pi_\theta(\cdot|x)}[\frac{\exp(\tau r(x,y))}{Z(x)\pi_\theta(y|x)}\log\pi_\theta(y|x)].
\end{aligned}
\end{equation}


\subsection{Derivation of Equation \ref{PRA_eq}}\label{Derivation_PRA}
\begin{equation}
    \begin{aligned}
&\mathcal{L}_{\mathrm{PRA}}(\pi_\theta)=\mathbb{E}_{\mathcal{D}_{\theta}}\left[\mathrm{D}_{\mathrm{KL}}(p^*(z|y_1,y_2,x)||p_\theta(z|y_1,y_2,x))\right]\\
=&-\mathbb{E}_{x\sim \mathcal{D},y_1,y_2\sim \pi_\theta(y|x)}\left[p^*(1|y_1,y_2,x)\log p_\theta\left( 1|y_1,y_2,x\right)+p^*(0|y_1,y_2,x)\log p_\theta\left( 0|y_1,y_2,x\right)\right]\\
&+\mathbb{E}_{x\sim \mathcal{D},y_1,y_2\sim \pi_\theta(y|x)}\left[p^*(1|y_1,y_2,x)\log p^*(1|y_1,y_2,x)+p^*(0|y_1,y_2,x)\log p^*(0|y_1,y_2,x)\right]\\
=&-\mathbb{E}_{x\sim \mathcal{D},y_1,y_2\sim \pi_\theta(y|x)}\left[p^*(1|y_1,y_2,x)\log p_\theta\left( 1|y_1,y_2,x\right)+p^*(0|y_1,y_2,x)\log p_\theta\left( 0|y_1,y_2,x\right)\right]\\
&+\mathbb{E}_{x\sim \mathcal{D},y_1,y_2\sim \pi_\theta(y|x)}\left[M(x,y_1,y_2)\right]\\
=&-\mathbb{E}_{x\sim \mathcal{D},y_1,y_2\sim \pi_\theta(y|x)}\left[p^*(1|y_1,y_2,x)\log p_\theta\left( 1|y_1,y_2,x\right)+p^*(1|y_2,y_1,x)\log p_\theta\left( 1|y_2,y_1,x\right)\right]\\
&+\mathbb{E}_{x\sim \mathcal{D},y_1,y_2\sim \pi_\theta(y|x)}\left[M(x,y_1,y_2)\right]\\
=&-\mathbb{E}_{x\sim \mathcal{D},y_1,y_2\sim \pi_\theta(y|x)}\left[p^*(1|y_w,y_l,x)\log p_\theta\left( 1|y_w,y_l,x\right)+p^*(1|y_l,y_w,x)\log p_\theta\left( 1|y_l,y_w,x\right)\right]\\
&+\mathbb{E}_{x\sim \mathcal{D},y_1,y_2\sim \pi_\theta(y|x)}\left[M(x,y_1,y_2)\right]\\
=&-\mathbb{E}_{(x,y_w,y_l)\sim \mathcal{D}_{\theta}}\left[\log p_\theta\left( 1|y_w,y_l,x\right)\right]+\mathbb{E}_{x\sim \mathcal{D},y_1,y_2\sim \pi_\theta(y|x)}\left[M(x,y_1,y_2)\right].
\end{aligned}
\end{equation}
where $\sigma(x)=\frac{1}{1+e^{-x}}$ is the sigmoid function, $M(x,y_1,y_2)=\sum_{z=0,1}p^*(z|y_1,y_2,x)\log p^*(z|y_1,y_2,x)$ and $(x,y_1,y_2)\sim \mathcal{D}_{\theta}\triangleq y_1\succ y_2\sim p^*(z=1|y_1,y_2,x),y_1,y_2\sim\pi_\theta(y|x),x\sim\mathcal{D}$. The third equality is because the symmetric complementarity property of $\omega(\cdot,\cdot)$:
\begin{equation}
p^*(1|y_2,y_1,x)=\omega(r(x,y_2),r(x,y_1))=1-\omega(r(x,y_1),r(x,y_2))=1-p^*(1|y_1,y_2,x)=p^*(0|y_1,y_2,x).
\end{equation}




\subsection{Derivation of Equation \ref{RA_eq}}\label{Derivation_RA_eq}
\begin{equation}
\begin{aligned}
&\mathbb{E}_{x\sim \mathcal{D},y\sim\pi_\theta(\cdot|x)}\left[\left(r_\theta(x,y)- r(x,y)\right)^2\right]=\mathbb{E}_{x\sim \mathcal{D},y\sim\pi_\theta(y|x)}\left(\frac{1}{\tau}\log(\pi_\theta(y|x)) + \frac{1}{\tau}\log(Z(x))-r(x,y) \right)^2\\
=&\frac{1}{\tau^2}\mathbb{E}_{x\sim \mathcal{D},y\sim\pi_\theta(\cdot|x)}\left[\left(\log(\pi_\theta(y|x)) - \log(\frac{\exp(\tau r(x,y))}{Z(x)})\right)^2\right]=\frac{1}{\tau^2}\mathbb{E}_{x\sim \mathcal{D},y\sim\pi_\theta(\cdot|x)}\left[\left(\log(\frac{\pi_\theta(y|x)}{\pi^\tau(y|x)})\right)^2\right].
\end{aligned}
\end{equation}


\subsection{Derivation of Equation \ref{DPO_eq}}\label{DPO_eq_from_PRA_proof}
\begin{equation}
    \begin{aligned}
&\mathbb{E}_{x\sim \mathcal{D},y_1,y_2\sim \pi_0(y|x)}\left[\mathbb{D}_{\mathrm{KL}}(p^*(z|y_1,y_2,x)||\bar{p}_\theta(z|y_1,y_2,x))\right]\\
\equiv&-\mathbb{E}_{x\sim \mathcal{D},y_1,y_2\sim \pi_0(y|x)}\left[p^*(1|y_1,y_2,x)\log\sigma\left( \bar{h}_\theta(x,y_1,y_2)\right)+p^*(0|y_1,y_2,x)\log\sigma\left( \bar{h}_\theta(x,y_2,y_1)\right)\right]\\
=&-\mathbb{E}_{x\sim \mathcal{D},y_1,y_2\sim \pi_0(y|x)}\left[p^*(1|y_1,y_2,x)\log\sigma\left( \bar{h}_\theta(x,y_1,y_2)\right)+p^*(1|y_2,y_1,x)\log\sigma\left( \bar{h}_\theta(x,y_2,y_1)\right)\right]\\
=&-\mathbb{E}_{x\sim \mathcal{D},y_1,y_2\sim \pi_0(y|x)}\left[p^*(1|y_w,y_l,x)\log\sigma\left( \bar{h}_\theta(x,y_w,y_l)\right)+p^*(1|y_l,y_w,x)\log\sigma\left( \bar{h}_\theta(x,y_l,y_w)\right)\right]\\
=&-\mathbb{E}_{(x,y_w,y_l)\sim \mathcal{D}_R}\left[\log\sigma\left( \bar{h}_\theta(x,y_w,y_l)\right)\right]\\
=&-\mathbb{E}_{(x,y_w,y_l)\sim \mathcal{D}_R}\left[\log\sigma\left( \frac{1}{\tau}\log\frac{\pi_\theta(y_w|x)}{\pi_{ref}(y_w|x)}-\frac{1}{\tau}\log\frac{\pi_\theta(y_l|x)}{\pi_{ref}(y_l|x)}\right)\right].
\end{aligned}
\end{equation}
where $z\in\{0,1\}$, $\bar{p}_\theta(0|y_1,y_2,x)=\sigma\left( \bar{h}_\theta(x,y_2,y_1)\right)$ and $\bar{p}_\theta(1|y_1,y_2,x)=\sigma\left( \bar{h}_\theta(x,y_1,y_2)\right)$.



\section{Proof in Section \ref{RL_Classification_sec}}\label{RL_Classification_sec_proof}
% \subsection{Proof of the Target Distribution Equivalence of Our Framework 1}
% Proposition \ref{Reward_policy}: The $\arg\min$ of Eq.\ref{forward-KL}, \ref{reverse-KL}, \ref{RA_eq}, \ref{pair-wised_RA_eq}, \ref{PRA_eq} are equivalent and equal to $\pi^\tau$. 

% \textbf{Proof of Sketch:} For Proposition \ref{Reward_policy}, we first proved the equivalent target distribution of Eq.\ref{forward-KL}, Eq.\ref{reverse-KL} and Eq.\ref{RA_eq} is $\pi^\tau$ (Appendix \ref{Proof_Reward_policy}). Then we proved the equivalence of Eq.\ref{RA_eq} and Eq.\ref{pair-wised_RA_eq} (Appendix \ref{Proof_RDA_lemma}). As for the equivalence of Eq.\ref{pair-wised_RA_eq} and Eq.\ref{PRA_eq}, we introduce the BT model \cite{bradley1952rank} as an assumption for $p^*$ and $p_\theta$ to simplify the proof (Appendix \ref{PRA_optimal_proof}).

\subsection{The Target Distribution Equivalence of Eq.\ref{forward-KL}, Eq.\ref{reverse-KL} and Eq.\ref{RA_eq}}\label{Proof_Reward_policy}
Denote:
\begin{equation}
\begin{aligned}
\pi^*_{\mathrm{Forward-KL}}&=\arg\min_{\pi_\theta} -\mathbb{E}_{x\sim \mathcal{D},y\sim\pi_\theta(\cdot|x)}\left[(\tau r(x,y) - \log(\pi_\theta(y|x)))\right], \\
\pi^*_{\mathrm{Reverse-KL}}&=\arg\min_{\pi_\theta} -\mathbb{E}_{x\sim \mathcal{D},y\sim\pi^\tau\left(\cdot \mid x\right)}[\log\pi_\theta(y|x)], \\
\pi^*_{\mathrm{RA}}&=\arg\min_{\pi_\theta}\mathbb{E}_{x\sim \mathcal{D},y\sim\pi_\theta(\cdot|x)}\left[\frac{1}{\tau^2}\left(\log(\frac{\pi_\theta(y|x)}{\pi^\tau(y|x)})\right)^2\right],
\end{aligned}
\end{equation}
then $\pi^*_{\mathrm{Forward-KL}}=\pi^*_{\mathrm{Reverse-KL}}=\pi^*_{\mathrm{RA}}=\pi^\tau$.

\textbf{Proof:} First, we prove $\pi^*_{\mathrm{Forward-KL}}=\pi^\tau$:
\begin{equation}
    \begin{aligned}
&-\mathbb{E}_{x\sim \mathcal{D},y\sim\pi_\theta(\cdot|x)}\left[\tau r(x,y) - \log(\pi_\theta(y|x))\right]-(-\mathbb{E}_{x\sim \mathcal{D},y\sim\pi^\tau(\cdot|x)}\left[\tau r(x,y) - \log(\pi^\tau(y|x))\right] )\\
=& -\mathbb{E}_{x\sim \mathcal{D},y\sim\pi_\theta(\cdot|x)}\left[\tau r(x,y) - \log(\pi_\theta(y|x))\right]+\mathbb{E}_{x\sim \mathcal{D},y\sim\pi^\tau(\cdot|x)}\left[\tau r(x,y) - \log(\frac{\exp \left(\tau r\left(x, y\right)\right)}{Z(x)} )\right] \\
=& -\mathbb{E}_{x\sim \mathcal{D},y\sim\pi_\theta(\cdot|x)}\left[\tau r(x,y) - \log(\pi_\theta(y|x))\right]+\mathbb{E}_{x\sim \mathcal{D},y\sim\pi^\tau(\cdot|x)}\left[\log({Z(x)} )\right] \\
=& \mathbb{E}_{x\sim \mathcal{D},y\sim\pi_\theta(\cdot|x)}\left[-\tau r(x,y) + \log(\pi_\theta(y|x)) + \frac{\pi^\tau(y|x)}{\pi_\theta(y|x)}\log({Z(x)} ) \right] \\
=& \mathbb{E}_{x\sim \mathcal{D},y\sim\pi_\theta(\cdot|x)}\left[-(\log\exp(\tau r(x,y)) -\log({Z(x)} )+\log({Z(x)})) + \log(\pi_\theta(y|x)) + \frac{\pi^\tau(y|x)}{\pi_\theta(y|x)}\log({Z(x)} ) \right] \\
=& \mathbb{E}_{x\sim \mathcal{D},y\sim\pi_\theta(\cdot|x)}\left[-\log(\pi^\tau(y|x))-\log({Z(x)}) + \log(\pi_\theta(y|x)) + \frac{\pi^\tau(y|x)}{\pi_\theta(y|x)}\log({Z(x)} ) \right] \\
=& \mathbb{E}_{x\sim \mathcal{D},y\sim\pi_\theta(\cdot|x)}\left[\log(\frac{\pi_\theta(y|x)}{\pi^\tau(y|x)}) + \frac{\pi^\tau(y|x)-\pi_\theta(y|x)}{\pi_\theta(y|x)}\log({Z(x)} ) \right] \\
=& \mathbb{E}_{x\sim \mathcal{D}}\left[\mathrm{D}_{\mathrm{KL}}\left(\pi_\theta\left(\cdot \mid x\right) \| \pi^\tau\left(\cdot \mid x\right)\right)\right]\geq0.\\
    \end{aligned}
\end{equation}
From the above equation, we can see that when $\pi_\theta$ is equal to $\pi^\tau$, $-\mathbb{E}_{x\sim \mathcal{D},y\sim\pi_\theta(\cdot|x)}\left[\tau r(x,y) - \log(\pi_\theta(y|x))\right]$ reaches the minimum value $-\mathbb{E}_{x\sim \mathcal{D},y\sim\pi^\tau(\cdot|x)}\left[\tau r(x,y) - \log(\pi^\tau(y|x))\right]$. Therefore $\pi^*_{\mathrm{Forward-KL}}=\pi^\tau$.

Second, similarly we can prove $\pi^*_{\mathrm{Reverse-KL}}=\pi^\tau$:
\begin{equation}
    \begin{aligned}
&-\mathbb{E}_{x\sim \mathcal{D},y\sim\pi^\tau\left(\cdot \mid x\right)}[\log\pi_\theta(y|x)]-(-\mathbb{E}_{x\sim \mathcal{D},y\sim\pi^\tau\left(\cdot \mid x\right)}[\log\pi^\tau(y|x)] )=\mathbb{E}_{x\sim \mathcal{D}}\left[\mathrm{D}_{\mathrm{KL}}\left(\pi^\tau\left(\cdot \mid x\right) \| \pi_\theta\left(\cdot \mid x\right)\right)\right] \geq0.\\
    \end{aligned}
\end{equation}

At last, we prove $\pi^*_{\mathrm{RA}}=\pi^\tau$ like above:
\begin{equation}
\small{    \begin{aligned}
\mathbb{E}_{x\sim \mathcal{D},y\sim\pi_\theta(\cdot|x)}\left[\frac{1}{\tau^2}\left(\log(\frac{\pi_\theta(y|x)}{\pi^\tau(y|x)})\right)^2\right]-\mathbb{E}_{x\sim \mathcal{D},y\sim\pi_\theta(\cdot|x)}\left[\frac{1}{\tau^2}\left(\log(\frac{\pi^\tau(y|x)}{\pi^\tau(y|x)})\right)^2\right]=\mathbb{E}_{x\sim \mathcal{D},y\sim\pi_\theta(\cdot|x)}\left[\frac{1}{\tau^2}\left(\log(\frac{\pi_\theta(y|x)}{\pi^\tau(y|x)})\right)^2\right] \geq0.
    \end{aligned}}
\end{equation}
Proof finished.


\subsection{The Target Distribution Equivalence of Eq.\ref{RA_eq} and Eq.\ref{pair-wised_RA_eq} }\label{Proof_RDA_lemma}
Here we prove the target distribution equivalence of Eq.\ref{RA_eq} and Eq.\ref{pair-wised_RA_eq}. Denote
\begin{equation}
\begin{aligned}
\pi^*_{\mathrm{RDA}}&=\arg\min_{\pi_\theta}\mathbb{E}_{\mathcal{D}_{\text{pw}}}\left[\left( \frac{1}{\tau}\log\frac{\pi_\theta(y_1|x)}{\pi_\theta(y_2|x)}-(r(x,y_1)-r(x,y_2))\right)^2\right],
\end{aligned}
\end{equation}
then $\pi^*_{\mathrm{RDA}}=\pi^*_{\mathrm{RA}}=\pi^\tau$.

\textbf{Proof:} 
\begin{equation}
\begin{aligned}
&\mathbb{E}_{\mathcal{D}_{\text{pw}}}\left[\left( \frac{1}{\tau}\log\frac{\pi_\theta(y_1|x)}{\pi_\theta(y_2|x)}-(r(x,y_1)-r(x,y_2))\right)^2\right]-\mathbb{E}_{\mathcal{D}_{\text{pw}}}\left[\left( \frac{1}{\tau}\log\frac{\pi^\tau(y_1|x)}{\pi^\tau(y_2|x)}-(r(x,y_1)-r(x,y_2))\right)^2\right]\\
=&\mathbb{E}_{\mathcal{D}_{\text{pw}}}\left[\left( \frac{1}{\tau}\log\frac{\pi_\theta(y_1|x)}{\pi_\theta(y_2|x)}-(r(x,y_1)-r(x,y_2))\right)^2\right] \geq0.
    \end{aligned}
\end{equation} 
When $\pi_\theta$ is equal to $\pi^\tau$, $\mathbb{E}_{\mathcal{D}_{\text{pw}}}\left[\left( \frac{1}{\tau}\log\frac{\pi_\theta(y_1|x)}{\pi_\theta(y_2|x)}-(r(x,y_1)-r(x,y_2))\right)^2\right]$ reaches the minimum value 0. Therefore $\pi^*_{\mathrm{RDA}}=\pi^*_{\mathrm{RA}}=\pi^\tau$. Proof finished.
\subsection{The Target Distribution Equivalence of Eq.\ref{pair-wised_RA_eq} and Eq.\ref{PRA_eq}}\label{PRA_optimal_proof}
Denote
\begin{equation} 
    \begin{aligned}
&\pi^*_{\mathrm{PRA}}&=\arg\min_{\pi_\theta}\mathbb{E}_{\mathcal{D}_{\theta}}\left[\mathrm{D}_{\mathrm{KL}}(p^*(z|y_1,y_2,x)||p_\theta(z|y_1,y_2,x))\right],
\end{aligned}
\end{equation}
then $\pi^*_{\mathrm{PRA}}=\pi^*_{\mathrm{RDA}}=\pi^\tau$.

\textbf{Proof:} Because $r_\theta(x, y) = \frac{1}{\tau} \log(Z(x)\pi_\theta(y|x))$, $p^*(z \mid y_1, y_2, x)=\omega(r(x,y_{2-z}),r(x,y_{z+1}))$ and $p_\theta(z|y_1,y_2,x)=\omega(r_\theta(x,y_{2-z}),r_\theta(x,y_{z+1}))$, when $\pi_\theta$ is equal to $\pi^\tau$, $r_\theta(x, y) = \frac{1}{\tau} \log(Z(x)\pi_\theta(y|x))=r(x, y)$. Then $p_\theta(z|y_1,y_2,x)=p^*(z|y_1,y_2,x)$ and $\mathbb{E}_{\mathcal{D}_{\theta}}\left[\mathrm{D}_{\mathrm{KL}}(p^*(z|y_1,y_2,x)||p_\theta(z|y_1,y_2,x))\right]=0$, then $\pi^*_{\mathrm{PRA}}=\pi^\tau=\pi^*_{\mathrm{RDA}}$. Proof finished.


\subsection{Proof of The Target Distribution Equivalence of Eq.\ref{RA_eq_posterior} and Eq.\ref{PRA_posterior}}\label{Reward_policy_posterior_proof}
\textbf{Proof:}
See the definition of $\bar{\pi}^\tau$ in Eq.\ref{pi_bar}. Denote
\begin{equation}
{\begin{aligned}
\pi^*_{\mathrm{RA-P}}&=\arg\min_{\pi_\theta}\mathbb{E}_{x\sim \mathcal{D},y\sim\pi_\theta(\cdot|x)}\left[\left(\frac{1}{\tau}\log(Z'(x)\frac{\pi_\theta(y|x)}{\pi_{ref}(y|x)})- r(x,y)\right)^2\right] \\
\pi^*_{\mathrm{PRA-P}}&=\arg\min_{\pi_\theta}\mathbb{E}_{\mathcal{D}_{\theta}}\left[\mathrm{D}_{\mathrm{KL}}(p^*(z|y_1,y_2,x)||\bar{p}_\theta(z|y_1,y_2,x))\right].
\end{aligned}}
\end{equation}
First, when $\pi_\theta$ is equal to $\bar{\pi}^\tau$, then
\begin{equation}
{\begin{aligned}
&\mathbb{E}_{x\sim \mathcal{D},y\sim\pi_\theta(\cdot|x)}\left[\left(\frac{1}{\tau}\log(Z'(x)\frac{\pi_\theta(y|x)}{\pi_{ref}(y|x)})- r(x,y)\right)^2\right] \\
=&\mathbb{E}_{x\sim \mathcal{D},y\sim\bar{\pi}^\tau(\cdot|x)}\left[\left(\frac{1}{\tau}\log(Z'(x)\frac{\bar{\pi}^\tau(y|x)}{\pi_{ref}(y|x)})- r(x,y)\right)^2\right] \\
=& \mathbb{E}_{x\sim \mathcal{D},y\sim\bar{\pi}^\tau(\cdot|x)}\left[\left(\frac{1}{\tau}\log(\exp(\tau r(x,y))- r(x,y)\right)^2\right]=0.
\end{aligned}}
\end{equation}
Because $\mathbb{E}_{x\sim \mathcal{D},y\sim\pi_\theta(\cdot|x)}\left[\left(\frac{1}{\tau}\log(Z'(x)\frac{\pi_\theta(y|x)}{\pi_{ref}(y|x)})- r(x,y)\right)^2\right]\geq0$, then $\pi^*_{\mathrm{RA-P}}=\bar{\pi}^\tau$. 

Second, when $\pi_\theta$ is equal to $\bar{\pi}^\tau$, $r_\theta(x, y) = \frac{1}{\tau} \log(Z'(x)\frac{\pi_\theta(y|x)}{\pi_{ref}(y|x)}))=\frac{1}{\tau} \log(Z'(x)\frac{\bar{\pi}^\tau(y|x)}{\pi_{ref}(y|x)}))=r(x, y)$. Then $\bar{p}_\theta(z|y_1,y_2,x)=p^*(z|y_1,y_2,x)$ and $\mathbb{E}_{\mathcal{D}_{\theta}}\left[\mathrm{D}_{\mathrm{KL}}(p^*(z|y_1,y_2,x)||\bar{p}_\theta(z|y_1,y_2,x))\right]=0$, then $\pi^*_{\mathrm{PRA-P}}=\bar{\pi}^\tau$. Then $\pi^*_{\mathrm{RA-P}}=\pi^*_{\mathrm{PRA-P}}=\bar{\pi}^\tau$. Proof finished.

\subsection{Proof of Theorem \ref{DPO_PRA}}\label{DPO_PRA_proof}
Theorem \ref{DPO_PRA}: Denote $$
\small{\begin{aligned}
&\mathcal{L}_{\mathrm{DPO}}\left(\pi_\theta\right) =\mathbb{E}_{\mathcal{D}_R}\left[-\log \sigma\left(\bar{h}_\theta\left(x, y_w, y_l\right)\right)\right], \\
&\mathcal{L}_{\mathrm{PRA-P}}\left(\pi_\theta\right) =\mathbb{E}_{\mathcal{D}_{\theta}}\left[\mathrm{D}_{\mathrm{KL}}(p^*(z|y_1,y_2,x)||\bar{p}_\theta(z|y_1,y_2,x))\right].
    \end{aligned}}$$
where $\mathcal{D}_{R}\triangleq \{(x,y_w,y_l)|x\sim\mathcal{D},y_w,y_l\sim\pi_0(\cdot|x),(y_w\succ y_l)\sim p^*(1|y_w,y_l,x)\}$ and $\mathcal{D}_{\theta}\triangleq \{(x,y_1,y_2)|x\sim\mathcal{D},y_1,y_2\sim\pi_\theta(\cdot|x),(y_1\succ y_2)\sim p^*(1|y_1,y_2,x)\}$. When $p^*$ is modeled by the BT model with Eq.\ref{BTmodel}:
\begin{equation}
\begin{aligned}
    p^*(1|y_1,y_2,x) &= \omega(r(x,y_{2-z}),r(x,y_{z+1}))=\sigma(r(x,y_1)- r(x,y_2) ),
\end{aligned}
\end{equation}
then we have the following equality:
\begin{equation}
\small{\mathcal{L}_{\mathrm{PRA-P}}\left(\pi_\theta\right) = \mathcal{L}_{\mathrm{DPO}}\left(\pi_\theta\right) + \eta_1(\pi_\theta, \pi_0) + \eta_2(\pi_\theta).}
\end{equation}
where $\eta_1(\pi_\theta, \pi_0)$ equals to 0 if and only if $\pi_\theta=\pi_0$ and $\eta_2(\pi_\theta)\triangleq\mathbb{E}_{x\sim \mathcal{D},y_1,y_2\sim \pi_\theta(y|x)}\left[M(x,y_1,y_2)\right]$. 

\textbf{Proof:} 

\begin{equation}
    \begin{aligned}
&\mathcal{L}_{\mathrm{PRA-P}}\left(\pi_\theta\right) =\mathbb{E}_{\mathcal{D}_{\theta}}\left[\mathrm{D}_{\mathrm{KL}}(p^*(z|y_1,y_2,x)||\bar{p}_\theta\left( x,y_1,y_2\right))\right]\\
=&-\mathbb{E}_{x\sim \mathcal{D},y_1,y_2\sim \pi_\theta(y|x)}\left[p^*(z=1|y_1,y_2,x)\log \bar{p}_\theta\left( x,y_1,y_2\right)+p^*(z=0|y_1,y_2,x)\log \bar{p}_\theta\left( x,y_2,y_1\right)\right]\\
&+\mathbb{E}_{x\sim \mathcal{D},y_1,y_2\sim \pi_\theta(y|x)}\left[p^*(z=1|y_1,y_2,x)\log p^*(z=1|y_1,y_2,x)+p^*(z=0|y_1,y_2,x)\log p^*(z=0|y_1,y_2,x)\right]\\
=&-\mathbb{E}_{x\sim \mathcal{D},y_1,y_2\sim \pi_\theta(y|x)}\left[p^*(z=1|y_1,y_2,x)\log \bar{p}_\theta\left( x,y_1,y_2\right)+p^*(z=0|y_1,y_2,x)\log \bar{p}_\theta\left( x,y_2,y_1\right)\right]\\
&+\mathbb{E}_{x\sim \mathcal{D},y_1,y_2\sim \pi_\theta(y|x)}\left[M(x,y_1,y_2)\right].
\end{aligned}
\end{equation}
where $\sigma(x)=\frac{1}{1+e^{-x}}$ is the sigmoid function, $M(x,y_1,y_2)=\sum_{z=0,1}p^*(z|y_1,y_2,x)\log p^*(z|y_1,y_2,x)$ and $(x,y_1,y_2)\sim \mathcal{D}_{\theta}\triangleq y_1\succ y_2\sim p^*(z=1|y_1,y_2,x),y_1,y_2\sim\pi_\theta(y|x),x\sim\mathcal{D}$.

Denote $\eta_2(\pi_\theta)=\mathbb{E}_{x\sim \mathcal{D},y_1,y_2\sim \pi_\theta(y|x)}\left[M(x,y_1,y_2)\right]$, $\zeta(\bar{p}_\theta;x,y_1,y_2)=p^*(z=1|y_1,y_2,x)\log \bar{p}_\theta\left( x,y_1,y_2\right)+p^*(z=0|y_1,y_2,x)\log \bar{p}_\theta\left( x,y_2,y_1\right)$. Then we have:
\begin{equation}
    \begin{aligned}
&\mathcal{L}_{\mathrm{PRA-P}}\left(\pi_\theta\right) = -\mathbb{E}_{x\sim \mathcal{D},y_1,y_2\sim \pi_\theta(y|x)}\left[ \zeta(\bar{p}_\theta;x,y_1,y_2) \right] + \eta_2(\pi_\theta)\\
=&-\mathbb{E}_{x\sim \mathcal{D},y_1,y_2\sim \pi_0(y|x)}\left[ \zeta(\bar{p}_\theta;x,y_1,y_2) \right] + \left(\sum_{x\in\mathbb{X},y_1,y_2\in\mathbb{Y}}(\pi_0(y|x)-\pi_\theta(y|x))\cdot\zeta(\bar{p}_\theta;x,y_1,y_2)\right) + \eta_2(\pi_\theta).
    \end{aligned}
\end{equation}
Denote $\sum_{x\in\mathbb{X},y_1,y_2\in\mathbb{Y}}(\pi_0(y|x)-\pi_\theta(y|x))\cdot\zeta(\bar{p}_\theta;x,y_1,y_2)$ as $\eta_1(\pi_\theta, \pi_0)$. Based on Eq.\ref{DPO_eq_from_PRA_proof} and $p^*(1|y_1,y_2,x) = \omega(r(x,y_{2-z}),r(x,y_{z+1}))=\sigma(r(x,y_1)- r(x,y_2) )$ because of the BT model, then we have the final conclusion:
\begin{equation}
    \mathcal{L}_{\mathrm{PRA-P}}\left(\pi_\theta\right) = \mathcal{L}_{\mathrm{DPO}}\left(\pi_\theta\right) + \eta_1(\pi_\theta, \pi_0) + \eta_2(\pi_\theta).
\end{equation}
Proof finished.






\section{Proof in Section \ref{Convergence_sec}}\label{Convergence_sec_proof}

\begin{proposition}\label{SGD_theorem}
If $F(\theta)$ is $L$-smooth for $\theta$ and $\|g_t\|^2\leq G^2$. Suppose the parameters $\theta$ are updated by:
\begin{equation}
    \theta_{t+1}=\theta_{t} - \alpha_tg(x,y,\theta_t),
\end{equation}
Then the following inequality holds:
\begin{equation}\label{SGD_eq}
    % \centering
    \small{\begin{aligned}
        \min_{1\leq t\leq T-1}||\nabla F(\theta_t)||^2_2\leq \frac{LG^2\sum_{t=1}^{T-1}\alpha_t^2+2(F(\theta_1)-F^*)}{2\sum_{t=1}^{T-1}\alpha_t}.
    \end{aligned}}
\end{equation}
where $F^*=\arg\min_\theta F(\theta)$. 
\end{proposition}
\textbf{Proof of Sketch:} Before proof, firstly we need to introduce a formal setting. Assume that the parameterized policy $\pi_\theta$ is given by Definition \ref{Softmax} and the parameters $\theta$ is optimized by the Stochastic Gradient Descent (SGD) method. Denote $F(\theta)=\mathbb{E}_{x,y\sim p(x,y;\pi_\theta)}[f(x,y, \theta)]$ where $p(x,y;\pi_\theta)$ is a distribution and $f(x,y, \theta)$ is the loss function on data $(x,y)$ for policy parameters $\theta$. Let $g(x,y,\theta_k)$ be the stochastic gradient of $\nabla F(\theta_k)$, i.e. $E_{x,y\sim p(x,y;\pi_\theta)}[g(x,y,\theta)]=\nabla_\theta F(\theta)$. Usually $g(x,y,\theta_t)$ is simplified to $g_t$ when it does not cause ambiguity. With learning rate $\alpha_t$, the SGD update rule is Eq.\ref{SGD_eq}. For Theorem \ref{SGD_theorem}, since the $L$-smooth property  cannot guarantee convex, thus for a non-convex optimization problem, the commonly used measure is the gradient norm. Therefore, with the properties of $L$-smooth and SGD, Proposition \ref{SGD_theorem} holds. See details in Appendix \ref{SGD_theorem_proof}.

Observing Eq.\ref{SGD_eq}, $||\nabla F(\theta_t)||^2_2$ will converge to 0 when $T\rightarrow\infty$ if $\sum_{t=1}^\infty\alpha_t^2\leq\infty$. The smaller the variance $G$ of stochastic gradient $g_t$ is, the faster the convergence rate of $||\nabla F(\theta_t)||^2_2$ is. For the methods in Figure \ref{Framework_figure}, we prove that the loss functions of these methods all satisfy the $L$-smooth assumption with different coefficients under the softmax parametrization of policy $\pi_\theta$ (See Table \ref{smoothCoefficients}, corresponding to Lemma \ref{BDA_L_RKL_coef}-\ref{PRA_L_coef}). 
\subsection{Proof of Proposition \ref{SGD_theorem}}\label{SGD_theorem_proof}
\textbf{Proof:} Because $F(\theta)$ is $L$-smooth for $\theta$, we have:
\begin{equation}
    \left|F(\theta_{t+1})-F(\theta_{t})-\nabla_\theta F(\theta_{t})(\theta_{t+1}-\theta_{t})\right|\leq \frac{L}{2}\|\theta_{t+1}-\theta_{t}\|_2^2.
\end{equation}
where $\theta_{t+1}=\theta_{t}-\alpha_tg_t$.

Substituting into $\theta_{t+1}=\theta_{t}-\alpha_tg_t$, we get:
\begin{equation}\label{L_smooth_proof2}
    \begin{aligned}
& (F(\theta_{t+1})-F(\theta_{t})+\alpha_t\nabla_\theta F(\theta_{t})g_t) \\
\leq& \left|F(\theta_{t+1})-F(\theta_{t})+\alpha_t\nabla_\theta F(\theta_{t})g_t\right|\\
=&\left|F(\theta_{t+1})-F(\theta_{t})-\nabla_\theta F(\theta_{t})(\theta_{t+1}-\theta_{t})\right| \\
\leq& \frac{L}{2}\|\theta_{t+1}-\theta_{t}\|_2^2 = \frac{L}{2}\|\alpha_t g_t\|_2^2 \leq \frac{L}{2}\alpha_t^2 G^2.
    \end{aligned}
\end{equation}
Base on Eq.\ref{L_smooth_proof2}, we get:
\begin{equation}
    \alpha_t\nabla_\theta F(\theta_{t})g_t \leq \frac{L}{2}\alpha_t^2 G^2 + F(\theta_{t}) -F(\theta_{t+1}).
\end{equation}
Summing both sides with respect to $t$ yields:
\begin{equation}
    \begin{aligned}
        \sum_{t=1}^{T-1}\alpha_t\nabla_\theta F(\theta_{t})g_t \leq \frac{L}{2} G^2\sum_{t=1}^{T-1}\alpha_t^2 + F(\theta_{1}) -F(\theta_{T})\leq \frac{L}{2} G^2\sum_{t=1}^{T-1}\alpha_t^2 + F(\theta_{1}) -F^*.
    \end{aligned}
\end{equation}
where $F^*=\arg\min_\theta F(\theta)$.

Expect the stochastic gradient $g_t$ to get:
\begin{equation}
    \begin{aligned}
        \min_{1\leq t\leq T-1}||\nabla_\theta F(\theta_{t})||_2^2\sum_{t=1}^{T-1}\alpha_t\leq \sum_{t=1}^{T-1}\alpha_t||\nabla_\theta F(\theta_{t})||_2^2 \leq \frac{L}{2} G^2\sum_{t=1}^{T-1}\alpha_t^2 + F(\theta_{1}) -F^*.
    \end{aligned}
\end{equation}
Finally we have:
\begin{equation}
    % \centering
    \begin{aligned}
        \min_{1\leq t\leq T-1}||\nabla F(\theta_t)||^2_2\leq \frac{LG^2\sum_{t=1}^{T-1}\alpha_t^2+2(F(\theta_1)-F^*)}{2\sum_{t=1}^{T-1}\alpha_t}.
    \end{aligned}
\end{equation}
Proof finished.






\subsection{Proof of Theorem \ref{DPO_theorem}}\label{DPO_theorem_proof}
\textbf{Theorem \ref{DPO_theorem}:} Assume $\|g_t\|^2\leq G^2$. Given the Definition \ref{Softmax} for policy $\pi_\theta$ and the learning rate $\alpha_t$, for DPO algorithm, the following properties hold:
    
    (0). $\forall r, \tau$, then $\theta\rightarrow \mathcal{L}_{\mathrm{DPO}}\left(\pi_\theta ; \pi_{\text {ref }}\right)$ is $\frac{4}{\tau^2}$-smooth.

    (1). If there is  $\pi_\theta^*$ such that $\mathbb{E}_{x\sim \mathcal{D},y_1,y_2\sim \pi_0(y|x)}[(p^*(1|y_1,y_2,x) - \sigma(\bar{h}_\theta(x,y_1,y_2)))]=0$, then $\pi_\theta^* = \arg\min_{\pi_\theta} \mathcal{L}_{\mathrm{DPO}}(\pi_{\theta} ; \pi_{\text {ref }}, \pi_0)$. And $\bar{\pi}^\tau(y|x)=\frac{\pi_{ref}(y|x)\exp(\tau r(x,y))}{Z'(x)}$ is an example $\pi_\theta^*$ with the Bradley-Terry (BT) model.

    (2). Denote $\mathcal{L}_{\mathrm{DPO}}^*=\min_{\pi_\theta} \mathcal{L}_{\mathrm{DPO}}(\pi_{\theta} ; \pi_{\text {ref }}, \pi_0)$. $\min_{1\leq i\leq T}||\nabla_\theta \mathcal{L}_{\mathrm{DPO}}(\pi_{\theta_i} ; \pi_{\text {ref }})||^2_2\leq \frac{2G^2\sum_{t=1}^{T-1}\alpha_t^2}{\tau^2\sum_{t=1}^{T-1}\alpha_t} + \frac{\mathcal{L}_{\mathrm{DPO}}(\pi_{\theta_T})-\mathcal{L}_{\mathrm{DPO}}^*}{\sum_{t=1}^{T-1}\alpha_t}$. 

See $\mathcal{L}_{\mathrm{DPO}}(\pi_{\theta} ; \pi_{\text {ref }})$ on Eq.\ref{DPO_eq}.  $\mathcal{D}_{R}\triangleq \{(x,y_w,y_l)|x\sim\mathcal{D},y_w,y_l\sim\pi_0(\cdot|x),(y_w\succ y_l)\sim p^*(1|y_w,y_l,x)\}$ which $\pi_0$ is an unanalytical distribution, $\mathcal{D}$ is an arbitrary distribution and $\{z=1|y_1,y_2,x\}\triangleq \{r(x,y_1)\geq r(x,y_2)\}$.
\begin{equation}
\begin{gathered}
\bar{h}_\theta\left(x, y_w, y_l\right)=\frac{1}{\tau} \log \frac{\pi_\theta\left(y_w \mid x\right)}{\pi_{\text {ref }}\left(y_w \mid x\right)}-\frac{1}{\tau} \log \frac{\pi_\theta\left(y_l \mid x\right)}{\pi_{\text {ref }}\left(y_l \mid x\right)}, \\
\mathcal{L}_{\mathrm{DPO}}\left(\pi_\theta ; \pi_{\text {ref }}\right)=-\mathbb{E}_{\left(x, y_w, y_l\right) \sim \mathcal{D}_R}\left[\log \sigma\left(\bar{h}_\theta\left(x, y_w, y_l\right)\right)\right].
\end{gathered}
\end{equation}

\textbf{Proof:}

\textbf{(0).} $\forall r, \tau$, then $\theta\rightarrow \mathcal{L}_{\mathrm{DPO}}\left(\pi_\theta ; \pi_{\text {ref }}\right)$ is $\frac{8}{\tau^2}$-smooth.

Proof: By Lemma \ref{spectral_radius}, it suffices to show that the spectral radius of the hessian matrix of the second derivative of $\mathcal{L}_{\mathrm{DPO}}(\pi_{\theta} ; \pi_{\text {ref }})$, i.e.
\begin{equation}\label{spectral_radius_DPO}
\begin{aligned}
&\left|\sum_{x,x'\in\mathbb{X}}\sum_{y_i,y_j\in\mathbb{Y}}z(x,y_i) \frac{\partial^2 \mathcal{L}_{\mathrm{DPO}}(\pi_{\theta} ; \pi_{\text {ref }})}{\partial \theta(x,y_i)\partial\theta(x',y_j)} z(x',y_j)\right|    \leq\left(\frac{4}{\tau^2}\right)||z(\cdot,\cdot)||_2^2.
\end{aligned}
\end{equation}

Denote $h(\pi_\theta,x,y_1,y_2)=-p^*(1|y_1,y_2,x)\log\sigma\left( \bar{h}_\theta(x,y_1,y_2)\right)-p^*(0|y_1,y_2,x)\log\sigma\left( \bar{h}_\theta(x,y_2,y_1)\right)$, we have:
\begin{equation}\label{dpo_decompose_psi_with_M_eq}
\begin{aligned}
&\left|\sum_{x,x'\in\mathbb{X}}\sum_{y_i,y_j\in\mathbb{Y}}z(x,y_i) \frac{\partial^2 \mathcal{L}_{\mathrm{DPO}}\left(\pi_\theta ; \pi_{\text {ref }}\right)}{\partial \theta(x,y_i)\partial\theta(x',y_j)} z(x',y_j)\right|\\
=&\left|\sum_{x\in\mathbb{X}}\sum_{y_i,y_j\in\mathbb{Y}}z(x,y_i) \frac{\partial^2 \sum_{x\in\mathbb{X}}\mathcal{D}(x)\sum_{y_1,y_2\in\mathbb{Y}}\pi_0(y_1|x)\pi_0(y_2|x)h(\pi_\theta,x,y_1,y_2)}{\partial \theta(x,y_i)\partial\theta(x,y_j)}  z(x,y_j)\right|\\ =&\left|\sum_{x\in\mathbb{X}}\mathcal{D}(x)\sum_{y_i,y_j\in\mathbb{Y}}z(x,y_i)  \frac{\partial^2 f_{\text{DPO}}(x,\theta)}{\partial \theta(x, y_i) \partial \theta(x, y_j)}  z(x,y_j)\right| \\
\triangleq& |\sum_{x\in\mathbb{X}}\mathcal{D}(x) \psi(x)|  \leq \|\mathcal{D}(\cdot)\|_1\|\psi(\cdot)\|_\infty  = 1\cdot\|\psi(\cdot)\|_\infty  .
\end{aligned}
\end{equation}
where $f_{\text{DPO}}(x,\theta)=\sum_{y_1,y_2\in\mathbb{Y}}\pi_0(y_1|x)\pi_0(y_2|x)h(\pi_\theta,x,y_1,y_2)$ and $\psi(x)=\sum_{y_i,y_j\in\mathbb{Y}}z(x,y_i)  \frac{\partial^2 f_{\text{DPO}}(x,\theta)}{\partial \theta(x, y_i) \partial \theta(x, y_j)}  z(x,y_j)$.

The second derivative of $f_{\text{DPO}}(x,\theta)$ is:
\begin{equation}\label{DPO_decompose}
    \begin{aligned}
        \frac{\partial^2 f_{\text{DPO}}(x, \theta)}{\partial \theta(x, y_i) \partial \theta(x, y_j)} = \sum_{y_1,y_2\in\mathbb{Y}}& \pi_0(y_1|x)\pi_0(y_2|x)\frac{\partial^2 h(\pi_\theta, x, y_1,y_2)}{\partial \theta(x, y_i) \partial \theta(x, y_j)}.
    \end{aligned}
\end{equation}

Consider the first derivative of  $h(\pi_\theta,x,y_1,y_2)$, denote $p^*=p^*(1|y_1,y_2,x)$:
\begin{equation}
    \begin{aligned}
&\frac{\partial h(\pi_\theta,x,y_1,y_2)}{\partial \theta(x, y_i)} \\
=& -p^*\frac{\partial }{\partial \theta(x, y_i) }(\log\sigma(\frac{1}{\tau}\log\frac{\pi_\theta(y_1|x)\pi_{\text{ref}}(y_2|x)}{\pi_\theta(y_2|x)\pi_{\text{ref}}(y_1|x)})) - (1-p^*)\frac{\partial }{\partial \theta(x, y_i) }(\log\sigma(\frac{1}{\tau}\log\frac{\pi_\theta(y_1|x)\pi_{\text{ref}}(y_2|x)}{\pi_\theta(y_2|x)\pi_{\text{ref}}(y_1|x)})) \\
=& -\left(p^*(1-\sigma(\frac{1}{\tau}\log\frac{\pi_\theta(y_1|x)\pi_{\text{ref}}(y_2|x)}{\pi_\theta(y_2|x)\pi_{\text{ref}}(y_1|x)})) - (1-p^*)\sigma(\frac{1}{\tau}\log\frac{\pi_\theta(y_1|x)\pi_{\text{ref}}(y_2|x)}{\pi_\theta(y_2|x)\pi_{\text{ref}}(y_1|x)}) \right)\frac{1}{\tau}\frac{\partial }{\partial \theta(x, y_i) }(\log\frac{\pi_\theta(y_1|x)}{\pi_\theta(y_2|x)}) \\
=& -\left(p^* - \sigma(\frac{1}{\tau}\log\frac{\pi_\theta(y_1|x)\pi_{\text{ref}}(y_2|x)}{\pi_\theta(y_2|x)\pi_{\text{ref}}(y_1|x)}) \right) \frac{1}{\tau}(\delta_{y_1y_i}-\delta_{y_2y_i}).
    \end{aligned}
\end{equation}

Consider the second derivative of  $h(\pi_\theta,x,y_1,y_2)$:
\begin{equation}
    \begin{aligned}
&\frac{\partial^2 h(\pi_\theta,x,y_1,y_2)}{\partial \theta(x, y_i)\partial \theta(x, y_j)}\\ =& -\frac{\partial }{\partial \theta(x, y_j) }( \sigma(\frac{1}{\tau}\log\frac{\pi_\theta(y_2|x)\pi_{\text{ref}}(y_1|x)}{\pi_\theta(y_1|x)\pi_{\text{ref}}(y_2|x)}) \frac{1}{\tau}(\delta_{y_1y_i}-\delta_{y_2y_i}) ) \\
=& \frac{1}{\tau^2}(\delta_{y_1y_i}-\delta_{y_2y_i})(\delta_{y_1y_j}-\delta_{y_2y_j})\sigma(\frac{1}{\tau}\log\frac{\pi_\theta(y_1|x)\pi_{\text{ref}}(y_2|x)}{\pi_\theta(y_2|x)\pi_{\text{ref}}(y_1|x)})\sigma(\frac{1}{\tau}\log\frac{\pi_\theta(y_2|x)\pi_{\text{ref}}(y_1|x)}{\pi_\theta(y_1|x)\pi_{\text{ref}}(y_2|x)}).
    \end{aligned}
\end{equation}

Then:
\begin{equation}\label{DPO_4term_decompose}
\begin{aligned}
&|\psi(x)|=|\sum_{y_i,y_j\in\mathbb{Y}}z(x,y_i)  \frac{\partial^2 f_{\text{DPO}}(x,\theta)}{\partial \theta(x, y_i) \partial \theta(x, y_j)}  z(x,y_j)| \\
=&|\sum_{y_i,y_j\in\mathbb{Y}}z(x,y_i)  \sum_{y_1,y_2\in\mathbb{Y}} \pi_0(y_1|x)\pi_0(y_2|x)\frac{\partial^2 h(\pi_\theta, x, y_1,y_2)}{\partial \theta(x, y_i) \partial \theta(x, y_j)}  z(x,y_j)| \\
\leq &|\sum_{y_i,y_j\in\mathbb{Y}}z(x,y_i)  \sum_{y_1,y_2\in\mathbb{Y}} \pi_0(y_1|x)\pi_0(y_2|x)\frac{1}{\tau^2}(\delta_{y_1y_i}-\delta_{y_2y_i})(\delta_{y_1y_j}-\delta_{y_2y_j})  z(x,y_j)| \\
=& |\sum_{y_i\in\mathbb{Y}}z(x,y_i)   \pi_0(y_i|x)\frac{1}{\tau^2}z(x,y_i) - \sum_{y_i\in\mathbb{Y}}z(x,y_i)   \pi_0(y_i|x)\sum_{y_j\in\mathbb{Y}}\pi_0(y_j|x)\frac{1}{\tau^2}z(x,y_j) \\
&- \sum_{y_i\in\mathbb{Y}}z(x,y_i)   \pi_0(y_i|x)\sum_{y_j\in\mathbb{Y}}\pi_0(y_j|x)\frac{1}{\tau^2}z(x,y_j) + \sum_{y_i\in\mathbb{Y}}z(x,y_i)   \pi_0(y_i|x)\frac{1}{\tau^2}z(x,y_i)| \\
\leq& \frac{2}{\tau^2}|\sum_{y_i\in\mathbb{Y}}z(x,y_i)   \pi_0(y_i|x)z(x,y_i)| + |\sum_{y_i\in\mathbb{Y}}z(x,y_i)   \pi_0(y_i|x)\sum_{y_j\in\mathbb{Y}}\pi_0(y_j|x)z(x,y_j)|\\
=& \frac{2}{\tau^2} (\pi^T_0(\cdot|x)z^2(x,\cdot) + (\pi^T_0(\cdot|x)z(x,\cdot))^2) \leq \frac{2}{\tau^2} (\|\pi_0(\cdot|x)\|_1\|z^2(x,\cdot)\|_\infty + (\|\pi_0(\cdot|x)\|_1\|z(x,\cdot)\|_\infty)^2) \\
\leq& \frac{4}{\tau^2}\|z(x,\cdot)\|_2^2.
\end{aligned}
\end{equation}
The first inequality is because $\sigma(\cdot)\leq 1$. Therefore,
\begin{equation}
    \begin{aligned}
\|\psi(\cdot)\|_\infty &= \max_{x\in\mathbb{X}}|\sum_{y_i,y_j\in\mathbb{Y}}z(x,y_i)  \frac{\partial^2 f_{\text{DPO}}(x,\theta)}{\partial \theta(x, y_i) \partial \theta(x, y_j)}  z(x,y_j)| \leq \max_{x\in\mathbb{X}}\frac{4}{\tau^2}\|z(x,\cdot)\|_2^2 \leq \frac{4}{\tau^2}\|z(\cdot,\cdot)\|_2^2.
    \end{aligned}
\end{equation}
Then Eq.\ref{spectral_radius_DPO} is proved. Proof finished. 

\textbf{(1).} If $\pi_\theta$ such that $\mathbb{E}_{x\sim \mathcal{D},y_1,y_2\sim \pi_0(y|x)}[(p^*(1|y_1,y_2,x) - \sigma(\bar{h}_\theta(x,y_1,y_2)))]=0$, then the DPO's loss function get a local optimal solution (zero gradient point) where $\pi_0$ is another unknown distribution. And $\bar{\pi}^\tau(y|x)=\frac{\pi_{ref}(y|x)\exp(\tau r(x,y))}{Z'(x)}$ is an example with the Bradley-Terry (BT) model.

Proof: From the above theorem, it is evident that the optimal solution for $\pi_\theta$ is influenced by the comparison distribution $p^*(z|y_1,y_2,x)$. Here, we provide an existence proof. We first propose a candidate solution $\bar{\pi}^\tau(y|x)=\frac{\pi_{ref}(y|x)\exp(\tau r(x,y))}{Z'(x)}$ where $Z'(x)=\sum_{y'\in Y}\pi_{ref}(y'|x)\exp({\tau r(x,y')})$, and subsequently demonstrate that if $p^*(z|y_1,y_2,x)$ is modeled by the Bradley-Terry (BT) model \cite{bradley1952rank}, solution $\bar{\pi}^\tau(y|x)=\frac{\pi_{ref}(y|x)\exp(\tau r(x,y))}{Z'(x)}$ satisfies condition $\mathbb{E}_{x\sim \mathcal{D},y_1,y_2\sim \pi_0(y|x)}[(p^*(1|y_1,y_2,x) - \sigma(\bar{h}_\theta(x,y_1,y_2)))]=0$ as a locally optimal solution. But it is not necessarily a global optimum, as we can only verify that $\bar{\pi}^\tau(y|x)=\frac{\pi_{ref}(y|x)\exp(\tau r(x,y))}{Z'(x)}$ is a point where the gradient of the loss function $\mathcal{L}_{\mathrm{DPO}}(\pi_{\theta} ; \pi_{\text {ref }})$ equals zero.

To compute the gradient of the loss function $\mathbb{E}_{x \sim \mathcal{D}, y_1, y_2 \sim \pi_0(y|x)}\left[\mathrm{D}_{\mathrm{KL}}(p^*(z|y_1, y_2, x) || \bar{p}_\theta(z|y_1, y_2,x))\right]$ with respect to $\theta$, we proceed as follows:
\begin{equation}
    \begin{aligned}
&\nabla_\theta \mathbb{E}_{x \sim \mathcal{D}, y_1, y_2 \sim \pi_0(y|x)}\left[\mathrm{D}_{\mathrm{KL}}(p^*(z|y_1, y_2, x) || \bar{p}_\theta(z|y_1, y_2,x))\right] \\
=& -\nabla_\theta \mathbb{E}_{x \sim \mathcal{D}, y_1, y_2 \sim \pi_0(y|x)} \left[p^*(1|y_1, y_2, x) \log \sigma(\bar{h}_\theta(x, y_1, y_2)) + p^*(0|y_1, y_2, x) \log \sigma(\bar{h}_\theta(x, y_2, y_1)) \right] \\
=& -\mathbb{E}_{x \sim \mathcal{D}, y_1, y_2 \sim \pi_0(y|x)} \left[p^*(1|y_1, y_2, x) \nabla_\theta \log \sigma(\bar{h}_\theta(x, y_1, y_2)) + p^*(0|y_1, y_2, x) \nabla_\theta \log \sigma(\bar{h}_\theta(x, y_2, y_1)) \right] \\
=& -\mathbb{E}_{x \sim \mathcal{D}, y_1, y_2 \sim \pi_0(y|x)} \left[ \frac{p^*(1|y_1, y_2, x)}{\sigma(\bar{h}_\theta(x, y_1, y_2))} \nabla_\theta \sigma(\bar{h}_\theta(x, y_1, y_2)) + \frac{p^*(0|y_1, y_2, x)}{\sigma(\bar{h}_\theta(x, y_2, y_1))} \nabla_\theta \sigma(\bar{h}_\theta(x, y_2, y_1)) \right] \\
=& -\mathbb{E}_{x \sim \mathcal{D}, y_1, y_2 \sim \pi_0(y|x)} \left[\frac{p^*(1|y_1, y_2, x)}{\sigma(\bar{h}_\theta(x, y_1, y_2))} \nabla_\theta \sigma(\bar{h}_\theta(x, y_1, y_2)) - \frac{1 - p^*(1|y_1, y_2, x)}{1 - \sigma(\bar{h}_\theta(x, y_1, y_2))} \nabla_\theta \sigma(\bar{h}_\theta(x, y_1, y_2)) \right] \\
=& -\mathbb{E}_{x \sim \mathcal{D}, y_1, y_2 \sim \pi_0(y|x)} \left[\left(\frac{p^*(1|y_1, y_2, x)(1 - \sigma(\bar{h}_\theta)) - (1 - p^*(1|y_1, y_2, x)) \sigma(\bar{h}_\theta)}{\sigma(\bar{h}_\theta(x, y_1, y_2))(1 - \sigma(\bar{h}_\theta)(x, y_1, y_2))}\right) \nabla_\theta \sigma(\bar{h}_\theta)\right] \\
=& -\mathbb{E}_{x \sim \mathcal{D}, y_1, y_2 \sim \pi_0(y|x)} \left[\left(\frac{p^*(1|y_1, y_2, x) - \sigma(\bar{h}_\theta(x, y_1, y_2))}{\sigma(\bar{h}_\theta(x, y_1, y_2))(1 - \sigma(\bar{h}_\theta(x, y_1, y_2)))}\right) \nabla_\theta \sigma(\bar{h}_\theta(x, y_1, y_2))\right] \\
=& -\mathbb{E}_{x \sim \mathcal{D}, y_1, y_2 \sim \pi_0(y|x)} \left[\left(p^*(1|y_1, y_2, x) - \sigma(\bar{h}_\theta(x, y_1, y_2))\right) \nabla_\theta \bar{h}_\theta(x, y_1, y_2)\right].
\end{aligned}
\end{equation}
where $z\in\{0,1\}$, $\bar{p}_\theta(0|y_1,y_2,x)=\sigma\left( \bar{h}_\theta(x,y_2,y_1)\right)$ and $\bar{p}_\theta(1|y_1,y_2,x)=\sigma\left( \bar{h}_\theta(x,y_1,y_2)\right)$.

Under the assumption of the Bradley-Terry (BT) model, which posits:
\begin{equation}
    p^*(1|y_1, y_2, x) = \frac{1}{1 + e^{-(r(x, y_1) - r(x, y_2))}} = \sigma(h(x, y_1, y_2)),
\end{equation}
substituting this into the gradient of the target function, we obtain:
\begin{equation}
    \begin{aligned}
&\nabla_\theta \mathbb{E}_{x \sim \mathcal{D}, y_1, y_2 \sim \pi_0(y|x)}\left[\mathrm{D}_{\mathrm{KL}}(p^*(z|y_1, y_2, x) || \bar{p}_\theta(z|y_1, y_2,x))\right] \\
=& -\mathbb{E}_{x \sim \mathcal{D}, y_1, y_2 \sim \pi_0(y|x)} \left[\left(p^*(1|y_1, y_2, x) - \sigma(\bar{h}_\theta(x, y_1, y_2))\right) \nabla_\theta \bar{h}_\theta(x, y_1, y_2)\right] \\
=& -\mathbb{E}_{x \sim \mathcal{D}, y_1, y_2 \sim \pi_0(y|x)} \left[\left(\sigma(h(x, y_1, y_2)) - \sigma(\bar{h}_\theta(x, y_1, y_2))\right) \nabla_\theta \bar{h}_\theta(x, y_1, y_2)\right].
\end{aligned}
\end{equation}
where $\bar{h}_\theta\left(x, y_1, y_2\right)=\frac{1}{\tau} \log \frac{\pi_\theta\left(y_1 \mid x\right)}{\pi_{\text {ref }}\left(y_1 \mid x\right)}-\frac{1}{\tau} \log \frac{\pi_\theta\left(y_2 \mid x\right)}{\pi_{\text {ref }}\left(y_2 \mid x\right)}$ and ${h}\left(x, y_1, y_2\right)=r(x,y_1)-r(x,y_2)$.

Given that $\sigma(\cdot)$ is a strictly increasing function, it follows that when $\pi_\theta(y|x)=\bar{\pi}^\tau(y|x)=\frac{\pi_{ref}(y|x)\exp(\tau r(x,y))}{Z'(x)}$, we have $\bar{h}_\theta(x, y_1, y_2) = h(x, y_1, y_2)$ and $\nabla_\theta \mathbb{E}_{x \sim \mathcal{D}, y_1, y_2 \sim \pi_0(y|x)} \left[\mathrm{D}_{\mathrm{KL}}(p^*(z|y_1, y_2, x) || \bar{p}_\theta(z|y_1, y_2,x))\right] = 0$. Proof finished.



\textbf{(2).} Denote $\mathcal{L}_{\mathrm{DPO}}^*=\min_{\pi_\theta} \mathcal{L}_{\mathrm{DPO}}(\pi_{\theta} ; \pi_{\text {ref }}, \pi_0)$. $\min_{1\leq i\leq T}||\nabla_\theta \mathcal{L}_{\mathrm{DPO}}(\pi_{\theta_i} ; \pi_{\text {ref }})||^2_2\leq \frac{2G^2\sum_{t=1}^{T-1}\alpha_t^2}{\tau^2T} + \frac{\mathcal{L}_{\mathrm{DPO}}(\pi_{\theta_T})-\mathcal{L}_{\mathrm{DPO}}^*}{T}$. 

This theorem can be obtained from Theorem \ref{DPO_theorem}.(1) and Theorem \ref{SGD_theorem}.

\subsection{Proof of Lemma \ref{data_select}}\label{data_select_proof}
\textbf{Lemma \ref{data_select}:} Assume $\|g_t\|^2\leq G^2$. Given the Definition \ref{Softmax} for policy $\pi_\theta$ and the learning rate $\alpha_t$, for DPO algorithm, let $\gamma(x)\leq\gamma$, when $\pi_0$ is uniform distribution, then:
\begin{equation}  
\small{    \begin{aligned}
\min_{1\leq i\leq T}&||\nabla_\theta \mathcal{L}_{\mathrm{DPO}}(\pi_{\theta_i} ; \pi_{\text {ref }})||^2_2\leq \frac{\mathcal{L}_{\mathrm{DPO}}(\pi_{\theta_T})-\mathcal{L}_{\mathrm{DPO}}^*}{\sum_{t=1}^{T-1}\alpha_t} + \frac{2(\gamma c_0+1)G^2\sum_{t=1}^{T-1}\alpha_t^2}{\tau^2\sum_{t=1}^{T-1}\alpha_t}.
    \end{aligned}}
\end{equation} 
where $\mathcal{L}_{\mathrm{DPO}}^*=\min_{\pi_\theta} \mathcal{L}_{\mathrm{DPO}}(\pi_{\theta} ; \pi_{\text {ref }}, \pi_0)$ and $c_0=\sigma(\frac{\epsilon_0}{\tau})\sigma(-\frac{\epsilon_0}{\tau})-1\in (-1,0)$. See $\mathcal{L}_{\mathrm{DPO}}(\pi_{\theta} ; \pi_{\text {ref }})$ on Eq.\ref{DPO_eq}.

\textbf{Proof:} Based on Theorem \ref{SGD_theorem}, we only need to know $\mathcal{L}_{\mathrm{DPO}}(\pi_{\theta}; \pi_{\text {ref }})$ is $\frac{4(\gamma c_0+1)}{\tau^2}$-smooth.

Denote $h(\pi_\theta,x,y_1,y_2)=-p^*(1|y_1,y_2,x)\log\sigma\left( \bar{h}_\theta(x,y_1,y_2)\right)-p^*(0|y_1,y_2,x)\log\sigma\left( \bar{h}_\theta(x,y_2,y_1)\right)$, we have:
\begin{equation}
\begin{aligned}
&\left|\sum_{x,x'\in\mathbb{X}}\sum_{y_i,y_j\in\mathbb{Y}}z(x,y_i) \frac{\partial^2 \mathcal{L}_{\mathrm{DPO}}\left(\pi_\theta ; \pi_{\text {ref }}\right)}{\partial \theta(x,y_i)\partial\theta(x',y_j)} z(x',y_j)\right|\\
=&\left|\sum_{x\in\mathbb{X}}\sum_{y_i,y_j\in\mathbb{Y}}z(x,y_i) \frac{\partial^2 \sum_{x\in\mathbb{X}}\mathcal{D}(x)\sum_{y_1,y_2\in\mathbb{Y}}\pi_0(y_1|x)\pi_0(y_2|x)h(\pi_\theta,x,y_1,y_2)}{\partial \theta(x,y_i)\partial\theta(x,y_j)}  z(x,y_j)\right|\\ =&\left|\sum_{x\in\mathbb{X}}\mathcal{D}(x)\sum_{y_i,y_j\in\mathbb{Y}}z(x,y_i)  \frac{\partial^2 f_{\text{DPO}}(x,\theta)}{\partial \theta(x, y_i) \partial \theta(x, y_j)}  z(x,y_j)\right| 
\triangleq |\sum_{x\in\mathbb{X}}\mathcal{D}(x) \psi(x)|  \leq \|\mathcal{D}(\cdot)\|_1\|\psi(\cdot)\|_\infty  = 1\cdot\|\psi(\cdot)\|_\infty  .
\end{aligned}
\end{equation}
where $f_{\text{DPO}}(x,\theta)=\sum_{y_1,y_2\in\mathbb{Y}}\pi_0(y_1|x)\pi_0(y_2|x)h(\pi_\theta,x,y_1,y_2)$ and $\psi(x)=\sum_{y_i,y_j\in\mathbb{Y}}z(x,y_i)  \frac{\partial^2 f_{\text{DPO}}(x,\theta)}{\partial \theta(x, y_i) \partial \theta(x, y_j)}  z(x,y_j)$.

Because $\pi_0$ is uniform distribution, the second derivative of $f_{\text{DPO}}(x,\theta)$ is:
\begin{equation}
    \begin{aligned}
\frac{\partial^2 f_{\text{DPO}}(x, \theta)}{\partial \theta(x, y_i) \partial \theta(x, y_j)} = \sum_{y_1,y_2\in\mathbb{Y}}& \frac{1}{K^2} \frac{\partial^2 h(\pi_\theta, x, y_1,y_2)}{\partial \theta(x, y_i) \partial \theta(x, y_j)}.
    \end{aligned}
\end{equation}

Consider the second derivative of  $h(\pi_\theta,x,y_1,y_2)$:
\begin{equation}
    \begin{aligned}
&\frac{\partial^2 h(\pi_\theta,x,y_1,y_2)}{\partial \theta(x, y_i)\partial \theta(x, y_j)}= -\frac{\partial }{\partial \theta(x, y_j) }( \sigma(\frac{1}{\tau}\log\frac{\pi_\theta(y_2|x)\pi_{\text{ref}}(y_1|x)}{\pi_\theta(y_1|x)\pi_{\text{ref}}(y_2|x)}) \frac{1}{\tau}(\delta_{y_1y_i}-\delta_{y_2y_i}) ) \\
=& \frac{1}{\tau^2}(\delta_{y_1y_i}-\delta_{y_2y_i})(\delta_{y_1y_j}-\delta_{y_2y_j})\sigma(\frac{1}{\tau}\log\frac{\pi_\theta(y_1|x)\pi_{\text{ref}}(y_2|x)}{\pi_\theta(y_2|x)\pi_{\text{ref}}(y_1|x)})\sigma(\frac{1}{\tau}\log\frac{\pi_\theta(y_2|x)\pi_{\text{ref}}(y_1|x)}{\pi_\theta(y_1|x)\pi_{\text{ref}}(y_2|x)})\\
=& \frac{1}{\tau^2}(\delta_{y_1y_i}-\delta_{y_2y_i})(\delta_{y_1y_j}-\delta_{y_2y_j})\zeta(x,y_1,y_2).
    \end{aligned}
\end{equation}
where $\zeta(x,y_1,y_2)=\sigma(\frac{1}{\tau}\log\frac{\pi_\theta(y_1|x)\pi_{\text{ref}}(y_2|x)}{\pi_\theta(y_2|x)\pi_{\text{ref}}(y_1|x)})\sigma(\frac{1}{\tau}\log\frac{\pi_\theta(y_2|x)\pi_{\text{ref}}(y_1|x)}{\pi_\theta(y_1|x)\pi_{\text{ref}}(y_2|x)})$.

Let $m_\theta(x,y_1,y_2)=\frac{\pi_\theta(y_1|x)\pi_{\text{ref}}(y_2|x)}{\pi_\theta(y_2|x)\pi_{\text{ref}}(y_1|x)}$ and . Because $\Omega_1(y_1,y_2,x)=\{|\log\frac{p^*(1|y_1,y_2,x)}{p^*(0|y_1,y_2,x)}|\geq\epsilon_0 \}$, $\Omega_2(y_1,y_2,x)=\{|\log m_\theta(x,y_1,y_2)|\geq\epsilon_0 \}$ and $\gamma(x)=\frac{\sum_{y_1,y_2\in\mathbb{Y}} \mathbb{I}(\Omega_1\cap\Omega_2)}{K^2}$.
\begin{equation}
\begin{aligned}
&|\psi(x)|=|\sum_{y_i,y_j\in\mathbb{Y}}z(x,y_i)  \frac{\partial^2 f_{\text{DPO}}(x,\theta)}{\partial \theta(x, y_i) \partial \theta(x, y_j)}  z(x,y_j)| \\
=&|\sum_{y_i,y_j\in\mathbb{Y}}z(x,y_i)  \sum_{y_1,y_2\in\mathbb{Y}} \frac{1}{K^2}\frac{\partial^2 h(\pi_\theta, x, y_1,y_2)}{\partial \theta(x, y_i) \partial \theta(x, y_j)}  z(x,y_j)| \\
= &|\sum_{y_i,y_j\in\mathbb{Y}}z(x,y_i)  \sum_{y_1,y_2\in\mathbb{Y}} \frac{1}{K^2}\frac{1}{\tau^2}(\delta_{y_1y_i}-\delta_{y_2y_i})(\delta_{y_1y_j}-\delta_{y_2y_j})\zeta(x,y_1,y_2)  z(x,y_j)| \\
=& |\sum_{y_i,y_2\in\mathbb{Y}}z(x,y_i)   \frac{1}{K^2}\zeta(x,y_i,y_2)\frac{1}{\tau^2}z(x,y_i) - \sum_{y_j,y_i\in\mathbb{Y}}z(x,y_i)\frac{1}{K^2}\zeta(x,y_i,y_j)\frac{1}{\tau^2}z(x,y_j) \\
&- \sum_{y_i,y_j\in\mathbb{Y}}z(x,y_i)   \frac{1}{K^2}\zeta(x,y_i,y_j)\frac{1}{\tau^2}z(x,y_j) + \sum_{y_1,y_j\in\mathbb{Y}}z(x,y_i)   \frac{1}{K^2}\zeta(x,y_1,y_j)\frac{1}{\tau^2}z(x,y_i)| \\
\leq& \frac{2}{\tau^2}|\sum_{y_i,y_2\in\mathbb{Y}}z(x,y_i)   \frac{1}{K^2}\zeta(x,y_i,y_2)z(x,y_i)|+\frac{2}{\tau^2}|\sum_{y_i,y_j\in\mathbb{Y}}z(x,y_i)   \frac{1}{K^2}\zeta(x,y_i,y_j)z(x,y_j)|\\
=& \frac{2}{\tau^2}|\sum_{y_i\in\mathbb{Y}}z^2(x,y_i)   \sum_{y_2\in\mathbb{Y}}\frac{1}{K^2}\zeta(x,y_i,y_2) |+\frac{2}{\tau^2}|\sum_{y_i,y_j\in\mathbb{Y}}z(x,y_i)   \frac{1}{K^2}\zeta(x,y_i,y_j)z(x,y_j)|.
\end{aligned}
\end{equation}

For the first term $\frac{2}{\tau^2}|\sum_{y_i\in\mathbb{Y}}z^2(x,y_i)   \sum_{y_2\in\mathbb{Y}}\frac{1}{K^2}\zeta(x,y_i,y_2) |$, based on Hölder's inequality we have:
\begin{equation}\label{data_proof_term1_eq}
    \begin{aligned}
&\frac{2}{\tau^2}|\sum_{y_i\in\mathbb{Y}}z^2(x,y_i)   \sum_{y_2\in\mathbb{Y}}\frac{1}{K^2}\zeta(x,y_i,y_2) | \leq \frac{2}{\tau^2}\sum_{y_i,y_2\in\mathbb{Y}}|\frac{1}{K^2}\zeta(x,y_i,y_2)|\cdot \|z^2(x,\cdot)\|_\infty \\
\leq& \frac{2}{\tau^2}(\gamma(x)\sigma(\frac{\epsilon_0}{\tau})\sigma(-\frac{\epsilon_0}{\tau}) + 1-\gamma(x) )\|z^2(x,\cdot)\|_\infty =  \frac{2}{\tau^2}(\gamma(x) c_0+1)\|z(x,\cdot)\|_\infty^2 \leq \frac{2}{\tau^2}(\gamma(x) c_0+1)\|z(x,\cdot)\|_2^2.
    \end{aligned}
\end{equation}

For the second term $\frac{2}{\tau^2}|\sum_{y_i,y_j\in\mathbb{Y}}z(x,y_i)   \frac{1}{K^2}\zeta(x,y_i,y_j)z(x,y_j)|$, let $\mathbb{Y}_1(x)=\{(y_1,y_2)| \mathbb{I}(\Omega_1(x,y_1,y_2)\cap\Omega_2(x,y_1,y_2) )=1 \}$ and $\mathbb{Y}_2(x)=\{(y_1,y_2)| \mathbb{I}(\Omega_1(x,y_1,y_2)\cap\Omega_2(x,y_1,y_2) )=0 \}$, we have:
\begin{equation}\label{data_proof_term2_eq}
    \begin{aligned}
&\frac{2}{\tau^2}|\sum_{y_i,y_j\in\mathbb{Y}}z(x,y_i)   \frac{1}{K^2}\zeta(x,y_i,y_j)z(x,y_j)| \\
\leq&  \frac{2}{\tau^2}|\sum_{(y_i,y_j)\in\mathbb{Y}_1}z(x,y_i)   \frac{1}{K^2}\zeta(x,y_i,y_j)z(x,y_j)| + \frac{2}{\tau^2}|\sum_{(y_i,y_j)\in\mathbb{Y}_2}z(x,y_i)   \frac{1}{K^2}\zeta(x,y_i,y_j)z(x,y_j)| \\
\leq& \frac{2}{\tau^2}|\sum_{(y_i,y_j)\in\mathbb{Y}_1}z(x,y_i)   \frac{1}{K^2}\sigma(\frac{\epsilon_0}{\tau})\sigma(-\frac{\epsilon_0}{\tau})z(x,y_j)| + \frac{2}{\tau^2}|\sum_{(y_i,y_j)\in\mathbb{Y}_2}z(x,y_i)   \frac{1}{K^2}z(x,y_j)| \\
\leq& \frac{2}{\tau^2}\sigma(\frac{\epsilon_0}{\tau})\sigma(-\frac{\epsilon_0}{\tau})\sqrt{\sum_{(y_i,y_j)\in\mathbb{Y}_1}z^2(x,y_i)\frac{1}{K^2}} \sqrt{\sum_{(y_i,y_j)\in\mathbb{Y}_1}\frac{1}{K^2}z^2(x,y_j)} + \frac{2}{\tau^2}|\sum_{(y_i,y_j)\in\mathbb{Y}_2}z(x,y_i)   \frac{1}{K^2}z(x,y_j)|  \\
\leq& \frac{2}{\tau^2}\sigma(\frac{\epsilon_0}{\tau})\sigma(-\frac{\epsilon_0}{\tau})\gamma(x)\|z^2(x,\cdot)\|_\infty + \frac{2}{\tau^2}|\sum_{(y_i,y_j)\in\mathbb{Y}_2}z(x,y_i)   \frac{1}{K^2}z(x,y_j)| \\
\leq& \frac{2}{\tau^2}\sigma(\frac{\epsilon_0}{\tau})\sigma(-\frac{\epsilon_0}{\tau})\gamma(x)\|z^2(x,\cdot)\|_\infty +  \frac{2}{\tau^2}\sqrt{\sum_{(y_i,y_j)\in\mathbb{Y}_2}z^2(x,y_i)\frac{1}{K^2}} \sqrt{\sum_{(y_i,y_j)\in\mathbb{Y}_2}\frac{1}{K^2}z^2(x,y_j)} \\
\leq& \frac{2}{\tau^2}\sigma(\frac{\epsilon_0}{\tau})\sigma(-\frac{\epsilon_0}{\tau})\gamma(x)\|z^2(x,\cdot)\|_\infty + \frac{2}{\tau^2}(1-\gamma(x))\|z^2(x,\cdot)\|_\infty \leq \frac{2}{\tau^2}(\gamma(x) c_0+1)\|z(x,\cdot)\|_2^2.
    \end{aligned}
\end{equation}
Because $c_0=\sigma(\frac{\epsilon_0}{\tau})\sigma(-\frac{\epsilon_0}{\tau})-1\in (-1,0)$.

Therefore,
\begin{equation}
    \begin{aligned}
\|\psi(\cdot)\|_\infty \leq \max_{x\in\mathbb{X}}\frac{4(\gamma(x) c_0+1)}{\tau^2}\|z(x,\cdot)\|_2^2 \leq \frac{4(\gamma c_0+1)}{\tau^2}\|z(\cdot,\cdot)\|_2^2.
    \end{aligned}
\end{equation}
Then $\mathcal{L}_{\mathrm{DPO}}(\pi_{\theta}; \pi_{\text {ref }})$ is $\frac{4(\gamma c_0+1)}{\tau^2}$-smooth is proved. Proof finished. 



\subsection{Proof of Theorem \ref{data_select_coro}}\label{data_select_coro_proof}
\textbf{Theorem \ref{data_select_coro}:} Define joint conditional probability distribution $\pi_1(y_1,y_2|x)= \frac{\mu}{K^2}\ \text{if}\ \mathbb{I}(\Omega_1 \cap \Omega_2) = 1; \frac{1-\mu\gamma}{(1-\gamma)K^2} \text{else}$ where $\mu\in(0,1)$. Assume $\|g_t\|^2\leq G^2$. Given the Definition \ref{Softmax} for policy $\pi_\theta$ and the learning rate $\alpha_t$, for DPO algorithm, let $\gamma(x)\leq\gamma$, then:
\begin{equation}
\small{    \begin{aligned}
\min_{1\leq i\leq T}&||\nabla_\theta \mathcal{L}_{\mathrm{DPO}}(\pi_{\theta_i} ; \pi_{\text {ref }})||^2_2\leq \frac{\mathcal{L}_{\mathrm{DPO}}(\pi_{\theta_T})-\mathcal{L}_{\mathrm{DPO}}^*}{\sum_{t=1}^{T-1}\alpha_t}  + \frac{2(\mu\gamma c_0+1)G^2\sum_{t=1}^{T-1}\alpha_t^2}{\tau^2\sum_{t=1}^{T-1}\alpha_t}.
    \end{aligned}}
\end{equation}
where $\mathcal{L}_{\mathrm{DPO}}^*=\min_{\pi_\theta} \mathcal{L}_{\mathrm{DPO}}(\pi_{\theta} ; \pi_{\text {ref }}, \pi_0)$ and $c_0=\sigma(\frac{\epsilon_0}{\tau})\sigma(-\frac{\epsilon_0}{\tau})-1\in (-1,0)$. See $\mathcal{L}_{\mathrm{DPO}}(\pi_{\theta} ; \pi_{\text {ref }})$ on Eq.\ref{DPO_eq}. 


\textbf{About $\Omega_1(y_1,y_2,x)$:} As the optimization progresses, $\pi_\theta$ gradually approaches $\bar{\pi}^\tau$, causing the proportion of the event $\Omega_1 \cap \Omega_2$ (denoted by $\gamma$) to increase (See Proposition \ref{pq_equal}), thereby reducing the upper bound in inequality \ref{data_select_coro_eq}. This leads to an increasingly faster convergence rate for DPO. Conversely, if small-margin pairs are selected as event $\Omega_1$ (i.e. $\Omega_1(y_1,y_2,x)=\{|\log\frac{p^*(1|y_1,y_2,x)}{p^*(0|y_1,y_2,x)}|\leq\epsilon_0 \}$), the convergence rate of DPO will slow down as the optimization advances and intensify the distribution shift problem cause $\Omega_1$ and $\Omega_2$ will more and more insistent as the optimization advances.

\textbf{Proof:} Based on Theorem \ref{SGD_theorem}, we only need to know $\mathcal{L}_{\mathrm{DPO}}(\pi_{\theta}; \pi_{\text {ref }})$ is $\frac{4(\mu\gamma c_0+1)}{\tau^2}$-smooth. This corollary shares the same proof process as Lemma \ref{data_select}. The only two differences are Eq.\ref{data_proof_term1_eq} and Eq.\ref{data_proof_term2_eq}. Their $\gamma(x)$ will be changed into $\mu\gamma(x)$. Then we will get the smooth coefficient of $\mathcal{L}_{\mathrm{DPO}}(\pi_{\theta}; \pi_{\text {ref }})$ is $\frac{4(\mu\gamma c_0+1)}{\tau^2}$. Proof finished.


\subsection{Proposition \ref{pq_equal}}\label{pq_equal_sec}
\begin{proposition}\label{pq_equal}
   \begin{equation}
       \lim_{\epsilon_0\rightarrow 0} P(|\log\frac{p^*(z=1)}{p^*(z=0)}|\geq \epsilon_0) - P(|\log\frac{\bar{\pi}^\tau(y_1|x)\pi_{\text{ref}}(y_2|x)}{\bar{\pi}^\tau(y_2|x)\pi_{\text{ref}}(y_1|x)}|\geq \epsilon_0) =0.
   \end{equation} 
\end{proposition}
\textbf{Proof:} comparison probability $p^*(1 \mid y_1, y_2, x)\propto r(x,y_1)-r(x,y_2)$, where it is identified that $r(x, y_1) > r(x, y_2)$, denoted as $y_1 \succ y_2$; similarly, $(x, y_2, y_1)$ is sampled with probability $p^*(0 \mid y_1, y_2, x)\propto r(x,y_2)-r(x,y_1)$, and is identified as $r(x, y_1) < r(x, y_2)$, denoted as $y_2 \succ y_1$. $\bar{\pi}^\tau(y|x)=\frac{\pi_{ref}(y|x)\exp(\tau r(x,y))}{Z'(x)}$. Then
\begin{equation}
\begin{aligned}
    \{|\log\frac{p^*(z=1)}{p^*(z=0)}|\geq \epsilon_0\}\ &\underrightarrow{\epsilon_0\rightarrow0}\ \{r(x,y_1)\geq r(x,y_2)\}, \\
    \{|\log\frac{\bar{\pi}^\tau(y_1|x)\pi_{\text{ref}}(y_2|x)}{\bar{\pi}^\tau(y_2|x)\pi_{\text{ref}}(y_1|x)}|\geq \epsilon_0\}\ &\underrightarrow{\epsilon_0\rightarrow0}\ \{r(x,y_1)\geq r(x,y_2)\}.
\end{aligned}
\end{equation}
Proof finished.

As the optimization progresses, $\pi_\theta$ gradually approaches $\bar{\pi}^\tau$, causing $\Omega_1(y_1,y_2,x)$ close to $\Omega_2(y_1,y_2,x)$, then the proportion of the event $\Omega_1 \cap \Omega_2$ (denoted by $\gamma$) increases. Conversely, if small-margin pairs are selected as event $\Omega_1$ (i.e. $\Omega_1(y_1,y_2,x)=\{|\log\frac{p^*(1|y_1,y_2,x)}{p^*(0|y_1,y_2,x)}|\leq\epsilon_0 \}$), the convergence rate of DPO will slow down as the optimization advances and intensify the distribution shift problem cause $\Omega_1$ and $\Omega_2$ will more and more insistent as the optimization advances.


\subsection{Sub Lemma for Lemma \ref{BDA_L_RKL_coef}-\ref{PRA_L_coef}}\label{SubLemma_sec}
\begin{lemma}\label{softmax_1derivative}
    By the properties of the softmax derivative, we have: 
\begin{equation}
    \frac{\partial \pi_\theta(y|x)}{\partial \theta(x,y')} = \pi_\theta(y|x) (\delta_{yy'} - \pi_\theta(y'|x)),\ \text{where}\ \delta_{yy'}= \begin{cases}1, & \text { if } y=y' \\ 0, & \text { otherwise }\end{cases}.
\end{equation}
\end{lemma}
\begin{lemma}\label{spectral_radius}
    $\forall x$, $x\rightarrow f(x)$ is $L$-smooth is equivalent to the following property:
    \begin{equation}
\left|\sum_{i,j=1}^{|\mathbb{X}|}x_i \frac{\partial^2 f(x)}{\partial x_i\partial x_j} x_j\right|\leq L||\vec{x}||_2^2.
\end{equation}
By Taylor's theorem, it suffices to show that the spectral radius of the hessian matrix of the second derivative of $f(x)$ is bounded by $L$.
\end{lemma}
\begin{lemma}\label{Pair_second_derivative_lemma}
$f(x,\theta)=\sum_{y_1,y_2\in\mathbb{Y}}\pi_\theta(y_1|x)\pi_\theta(y_2|x)h(\pi_\theta,x,y_1,y_2)$. Then the second derivative of $f(x,\theta)$ is:
\begin{equation}
\small{    \begin{aligned}
        \frac{\partial^2 f(x, \theta)}{\partial \theta(x, y_i) \partial \theta(x, y_j)} = \sum_{y_1,y_2\in\mathbb{Y}}& 2\frac{\partial^2 \pi_\theta(y_1|x)}{\partial \theta(x, y_i) \partial \theta(x, y_j)} \pi_\theta(y_2|x)h(\pi_\theta,x,y_1,y_2) + 2\frac{\partial \pi_\theta(y_1|x)}{\partial \theta(x, y_i)}\frac{\partial \pi_\theta(y_2|x)}{\partial \theta(x, y_j)}h(\pi_\theta,x,y_1,y_2) \\
        &+ 2\frac{\partial \pi_\theta(y_1|x)}{\partial \theta(x, y_i)}\pi_\theta(y_2|x) \frac{\partial h(\pi_\theta, x, y_1,y_2)}{\partial \theta(x, y_j)}+ 2\frac{\partial \pi_\theta(y_1|x)}{\partial \theta(x, y_j)}\pi_\theta(y_2|x) \frac{\partial h(\pi_\theta, x, y_1,y_2)}{\partial \theta(x, y_i)}\\
        & + \pi_\theta(y_1|x)\pi_\theta(y_2|x)\frac{\partial^2 h(\pi_\theta, x, y_1,y_2)}{\partial \theta(x, y_i) \partial \theta(x, y_j)}.
    \end{aligned} }
\end{equation}
\end{lemma}
\begin{lemma}\label{SelfEntropy}
    For $f(p)=-p\log p-(1-p)\log(1-p)$, $f(p)\leq\log2$.

    Proof: For $f(p)=-p\log p-(1-p)\log(1-p)\geq 0, p\in[0,1]$, $f'(p)=\log(\frac{1}{p}-1)$ is a monotonically decreasing function that reaches 0 when $p=0.5$. Thus $f(p)$ increases first and then decreases, and reaches its maximum value at $p=0.5$. So $f(p)\leq\log2$.
\end{lemma}

\begin{lemma}\label{TV_KL_inequal}
$D_{\mathrm{TV}}(p \| q)^2 \leq D_{\mathrm{KL}}(p \| q)$.
\end{lemma}
\subsection{Proof of Lemma \ref{BDA_L_RKL_coef}}\label{BDA_L_RKL_coef_proof}
\begin{lemma}\label{BDA_L_RKL_coef}
    (BDA(Reverse-KL) Smoothness ) Given softmax parametrization of Definition \ref{Softmax} for policy $\pi_\theta$,  $\forall r, \tau$, $\theta\rightarrow \mathbb{E}_{x\sim \mathcal{D}}\left[\mathrm{D}_{\mathrm{KL}}\left( \pi^\tau\left(\cdot \mid x\right) \| \pi_\theta\left(\cdot \mid x\right) \right)\right]$ is 2-smooth (see Eq.\ref{reverse-KL}).

    See proof in Appendix \ref{BDA_L_RKL_coef_proof}.
\end{lemma}

\textbf{Proof:} By Lemma \ref{spectral_radius}, it suffices to show that the spectral radius of the hessian matrix of the second derivative of $\mathbb{E}_{x\sim \mathcal{D}}\left[\mathrm{D}_{\mathrm{KL}}\left( \pi^\tau\left(\cdot \mid x\right) \| \pi_\theta\left(\cdot \mid x\right) \right)\right]$ is bounded by $2$, i.e.
\begin{equation}
\left|\sum_{x,x'\in\mathbb{X}}\sum_{y,y'\in\mathbb{Y}}z(x,y) \frac{\partial^2 \mathbb{E}_{x\sim \mathcal{D}}\left[\mathrm{D}_{\mathrm{KL}}\left( \pi^\tau\left(\cdot \mid x\right) \| \pi_\theta\left(\cdot \mid x\right) \right)\right]}{\partial \theta(x,y)\partial\theta(x',y')} z(x',y')\right|\leq2||z(\cdot,\cdot)||_2^2.
\end{equation}
By Lemma \ref{softmax_1derivative}, the first derivative of $\mathbb{E}_{x\sim \mathcal{D}}\left[\mathrm{D}_{\mathrm{KL}}\left( \pi^\tau\left(\cdot \mid x\right) \| \pi_\theta\left(\cdot \mid x\right) \right)\right]$ is:
\begin{equation}
    \begin{aligned}
& \frac{\partial \mathbb{E}_{x\sim \mathcal{D}}\left[\mathrm{D}_{\mathrm{KL}}\left( \pi^\tau\left(\cdot \mid x\right) \| \pi_\theta\left(\cdot \mid x\right) \right)\right]}{\partial \theta(x,y)}  \\
=& -\frac{\partial \mathbb{E}_{x\sim \mathcal{D}}[\sum_{y'\in Y}\frac{\exp(\tau r(x,y'))}{Z(x)}(\log\pi_\theta(y'|x))]}{\partial \theta(x,y)} \\
=&-\mathcal{D}(x)\sum_{y'\in Y}\frac{\exp(\tau r(x,y'))}{Z(x)}\frac{\partial \log\pi_\theta(y'|x))}{\partial \theta(x,y)} \\
=&-\mathcal{D}(x)\sum_{y'\in Y}\frac{\exp(\tau r(x,y'))}{Z(x)} (\delta_{y'y} - \pi_\theta(y|x))\\
=& \mathcal{D}(x)\pi_\theta(y|x) - \mathcal{D}(x)\frac{\exp(\tau r(x,y))}{Z(x)}.
\end{aligned}
\end{equation}
The second derivative of $\mathbb{E}_{x\sim \mathcal{D}}\left[\mathrm{D}_{\mathrm{KL}}\left( \pi^\tau\left(\cdot \mid x\right) \| \pi_\theta\left(\cdot \mid x\right) \right)\right]$ is:
\begin{equation}
    \begin{aligned}
        & \frac{\partial^2 \mathbb{E}_{x\sim \mathcal{D}}\left[\mathrm{D}_{\mathrm{KL}}\left( \pi^\tau\left(\cdot \mid x\right) \| \pi_\theta\left(\cdot \mid x\right) \right)\right]}{\partial \theta(x,y)\partial \theta(x',y')}  \\
        =& \frac{\partial}{\partial \theta(x',y')}(\frac{\partial \mathbb{E}_{x\sim \mathcal{D}}\left[\mathrm{D}_{\mathrm{KL}}\left( \pi^\tau\left(\cdot \mid x\right) \| \pi_\theta\left(\cdot \mid x\right) \right)\right]}{\partial \theta(x,y)})\\
        =& \frac{\partial \mathcal{D}(x)\pi_\theta(y|x)}{\partial \theta(x',y')} = \delta_{xx'}\mathcal{D}(x)\pi_\theta(y|x)(\delta_{yy'} - \pi_\theta(y'|x)).
    \end{aligned}
\end{equation}
Then the spectral radius of $\mathbb{E}_{x\sim \mathcal{D}}\left[\mathrm{D}_{\mathrm{KL}}\left( \pi^\tau\left(\cdot \mid x\right) \| \pi_\theta\left(\cdot \mid x\right) \right)\right]$ is:
\begin{equation}
\begin{aligned}
&\left|\sum_{x,x'\in\mathbb{X}}\sum_{y,y'\in\mathbb{Y}}z(x,y) \frac{\partial^2 \mathbb{E}_{x\sim \mathcal{D}}\left[\mathrm{D}_{\mathrm{KL}}\left( \pi^\tau\left(\cdot \mid x\right) \| \pi_\theta\left(\cdot \mid x\right) \right)\right]}{\partial \theta(x,y)\partial\theta(x',y')} z(x',y')\right|\\
=&\left|\sum_{x\in\mathbb{X}}\sum_{y,y'\in\mathbb{Y}}z(x,y) \delta_{xx'}\mathcal{D}(x)\pi_\theta(y|x)(\delta_{yy'} - \pi_\theta(y'|x)) z(x,y')\right| \\
=&\left|\sum_{x\in\mathbb{X}}\mathcal{D}(x)\sum_{y,y'\in\mathbb{Y}}z(x,y)  \pi_\theta(y|x)(\delta_{yy'} - \pi_\theta(y'|x))  z(x,y')\right|\\
\triangleq& |\sum_{x\in\mathbb{X}}\mathcal{D}(x) \psi(x) |\leq \|\mathcal{D}(\cdot)\|_1 \|\psi(\cdot)\|_\infty=1\cdot\|\psi(\cdot)\|_\infty.
\end{aligned}
\end{equation}
Then we have:
\begin{equation}
\begin{aligned}
    &\|\psi(\cdot)\|_\infty = \max_{x\in\mathbb{X}}\left| \sum_{y,y'\in\mathbb{Y}}z(x,y)  \pi_\theta(y|x)(\delta_{yy'} - \pi_\theta(y'|x))  z(x,y')\right| \\
    =& \max_{x\in\mathbb{X}}\left| \sum_{y,y'\in\mathbb{Y}}z(x,y)  \pi_\theta(y|x)\delta_{yy'}z(x,y') - z(x,y)\pi_\theta(y|x)\pi_\theta(y'|x)  z(x,y')\right| \\
    \leq& \max_{x\in\mathbb{X}}\left| \sum_{y,y'\in\mathbb{Y}}z(x,y)  \pi_\theta(y|x)\delta_{yy'}z(x,y')\right|+ \max_{x\in\mathbb{X}}\left| \sum_{y,y'\in\mathbb{Y}}z(x,y)\pi_\theta(y|x)\pi_\theta(y'|x)  z(x,y')\right| \\
    =& \max_{x\in\mathbb{X}}\left| \sum_{y,y'\in\mathbb{Y}}z(x,y)  \pi_\theta(y|x)z(x,y)\right|+ \max_{x\in\mathbb{X}} (\sum_{y\in\mathbb{Y}}z(x,y)\pi_\theta(y|x))^2\\
    \leq& \max_{x\in\mathbb{X}} \sum_{y\in\mathbb{Y}}z(x,y)z(x,y) + \max_{x\in\mathbb{X}}(z(x,\cdot)^T\pi_\theta(\cdot|x))^2 \\
    \leq& \max_{x\in\mathbb{X}}  ||z(x,\cdot)||_2^2+ \max_{x\in\mathbb{X}}(||\pi_\theta(\cdot|x)||_1 ||z(x,\cdot)||_\infty )^2 \\
    \leq& 2\max_{x\in\mathbb{X}}  ||z(x,\cdot)||_2^2 = 2 \|||z(x,\cdot)||_2^2\|_\infty \leq 2||z(\cdot,\cdot)||_2^2.
\end{aligned}
\end{equation}
where $\|||z(x,\cdot)||_2^2\|_\infty$ is taking the infinite norm of $||z(x,\cdot)||_2^2$ with $x$ as the coordinate axis. 

Therefore, we have the $2$-smooth conclusion:
\begin{equation}
\left|\sum_{x,x'\in\mathbb{X}}\sum_{y,y'\in\mathbb{Y}}z(x,y) \frac{\partial^2 \mathbb{E}_{x\sim \mathcal{D}}\left[\mathrm{D}_{\mathrm{KL}}\left( \pi^\tau\left(\cdot \mid x\right) \| \pi_\theta\left(\cdot \mid x\right) \right)\right]}{\partial \theta(x,y)\partial\theta(x',y')} z(x',y')\right|\leq2||z(\cdot,\cdot)||_2^2.
\end{equation}
Proof finished.





\subsection{Proof of Lemma \ref{BDA_L_FKL_coef}}\label{BDA_L_FKL_coef_proof}
\begin{lemma}\label{BDA_L_FKL_coef}
    (BDA(Forward-KL) Smoothness ) Given softmax parametrization of Definition \ref{Softmax} for policy $\pi_\theta$, assume $|\log(\pi_\theta(y|x)) - \log(\pi^\tau(y|x))|\leq\epsilon_1$,  $\forall r, \tau$, $\theta\rightarrow \mathbb{E}_{x\sim \mathcal{D}}\left[\mathrm{D}_{\mathrm{KL}}\left(\pi_\theta\left(\cdot \mid x\right) \| \pi^\tau\left(\cdot \mid x\right)\right)\right]$ is $\left((4+K)\epsilon_1+6+2K\right)$-smooth (see Eq.\ref{forward-KL}).

    See proof in Appendix \ref{BDA_L_FKL_coef_proof}.
\end{lemma}

\textbf{Proof:} By Lemma \ref{spectral_radius}, it suffices to show that the spectral radius of the hessian matrix of the second derivative of $\mathbb{E}_{x\sim \mathcal{D}}\left[\mathrm{D}_{\mathrm{KL}}\left(\pi_\theta\left(\cdot \mid x\right) \| \pi^\tau\left(\cdot \mid x\right)\right)\right]$ is bounded by $\left(6\epsilon_1+10\right)$, i.e.
\begin{equation}
\left|\sum_{x,x'\in\mathbb{X}}\sum_{y,y'\in\mathbb{Y}}z(x,y) \frac{\partial^2 \mathbb{E}_{x\sim \mathcal{D}}\left[\mathrm{D}_{\mathrm{KL}}\left(\pi_\theta\left(\cdot \mid x\right) \| \pi^\tau\left(\cdot \mid x\right)\right)\right]}{\partial \theta(x,y)\partial\theta(x',y')} z(x',y')\right|\leq\left(6\epsilon_1+10\right)||z(\cdot,\cdot)||_2^2.
\end{equation}
For Equation \ref{forward-KL}, we have:
\begin{equation}
\begin{aligned}
&\mathbb{E}_{x\sim \mathcal{D}}\left[\mathrm{D}_{\mathrm{KL}}\left(\pi_\theta\left(\cdot \mid x\right) \| \pi^\tau\left(\cdot \mid x\right)\right)\right]\\ 
=&-\mathbb{E}_{x\sim \mathcal{D},y\sim\pi_\theta(\cdot|x)}\left[(\log(\pi^\tau(y|x) - \log(\pi_\theta(y|x)))\right]\\
=&\sum_{x\in\mathbb{X}}\mathcal{D}(x)\pi_\theta(\cdot|x)^Th(\pi_\theta,x)\quad \text{where}\ h(\pi_\theta,x,y)=(-\log(\pi^\tau(y|x) + \log(\pi_\theta(y|x))),\ h(\pi_\theta,x)=h(\pi_\theta,x,\cdot).
\end{aligned}
\end{equation}
Denote $f_{\text{fKL}}(x,\theta) =\pi_\theta(\cdot|x)^Th(\pi_\theta,x)$. We can calculate the spectral radius of $\frac{\partial^2 \mathbb{E}_{x\sim \mathcal{D}}\left[\mathrm{D}_{\mathrm{KL}}\left(\pi_\theta\left(\cdot \mid x\right) \| \pi^\tau\left(\cdot \mid x\right)\right)\right]}{\partial \theta(x,y)\partial\theta(x',y')}$:
\begin{equation}
\begin{aligned}
&\left|\sum_{x,x'\in\mathbb{X}}\sum_{y,y'\in\mathbb{Y}}z(x,y) \frac{\partial^2 \mathbb{E}_{x\sim \mathcal{D}}\left[\mathrm{D}_{\mathrm{KL}}\left(\pi_\theta\left(\cdot \mid x\right) \| \pi^\tau\left(\cdot \mid x\right)\right)\right]}{\partial \theta(x,y)\partial\theta(x',y')} z(x',y')\right|\\
=&\left|\sum_{x\in\mathbb{X}}\sum_{y,y'\in\mathbb{Y}}z(x,y) \frac{\partial^2 \sum_{x\in\mathbb{X}}\mathcal{D}(x)\pi_\theta(\cdot|x)^Th(\pi_\theta,x)}{\partial \theta(x,y)\partial\theta(x,y')}  z(x,y')\right|\\ =&\left|\sum_{x\in\mathbb{X}}\sum_{y,y'\in\mathbb{Y}}z(x,y) \mathcal{D}(x) \frac{\partial^2 f_{\text{fKL}}(x,\theta)}{\partial \theta(x, y) \partial \theta(x, y')}  z(x,y')\right|.
\end{aligned}
\end{equation}
The first derivative of $f_{\text{fKL}}(x,\theta)$ is:
\begin{equation}
    \frac{\partial f_{\text{fKL}}(x, \theta)}{\partial \theta(x, y_i)} = \sum_{y} \frac{\partial \pi_\theta(y|x)}{\partial \theta(x, y_i)} h(\pi_\theta, x, y) + \pi_\theta(y|x) \frac{\partial h(\pi_\theta, x, y)}{\partial \theta(x, y_i)}.
\end{equation}

The second derivative of $f_{\text{fKL}}(x,\theta)$ is:
\begin{equation}
    \begin{aligned}
        \frac{\partial^2 f_{\text{fKL}}(x, \theta)}{\partial \theta(x, y_i) \partial \theta(x, y_j)} = \sum_{y}& \frac{\partial^2 \pi_\theta(y|x)}{\partial \theta(x, y_i) \partial \theta(x, y_j)} h(\pi_\theta, x, y) + \frac{\partial \pi_\theta(y|x)}{\partial \theta(x, y_i)} \frac{\partial h(\pi_\theta, x, y)}{\partial \theta(x, y_j)}\\
        &+ \frac{\partial \pi_\theta(y|x)}{\partial \theta(x, y_j)} \frac{\partial h(\pi_\theta, x, y)}{\partial \theta(x, y_i)} + \pi_\theta(y|x) \frac{\partial^2 h(\pi_\theta, x, y)}{\partial \theta(x, y_i) \partial \theta(x, y_j)}.
    \end{aligned}
\end{equation}
By Lemma \ref{softmax_1derivative}, we have:
\begin{equation}
    \frac{\partial \pi_\theta(y|x)}{\partial \theta(x, y_i)} = \pi_\theta(y|x)(\delta_{yy_i}-\pi_\theta(y_i|x)).
\end{equation}
\begin{equation}
    \frac{\partial\left(\pi_\theta(y | x)(\delta_{yy_i} - \pi_\theta(y_i | x))\right)}{\partial \theta(x, y_j)} = \pi_\theta(y | x)(\delta_{yy_j} - \pi_\theta(y_j | x))(\delta_{yy_i} - \pi_\theta(y_i | x)) - \pi_\theta(y | x) \pi_\theta(y_i | x)(\delta_{y_iy_j} - \pi_\theta(y_j | x)).
\end{equation}
And:
\begin{equation}
    \frac{\partial h(\pi_\theta, x, y)}{\partial \theta(x, y_j)} = \frac{\partial (-\log(\pi^\tau(y|x) + \log(\pi_\theta(y|x)))}{\partial \theta(x, y_j)} = (\delta_{yy_j}-\pi_\theta(y_j|x)).
\end{equation}
\begin{equation}
    \frac{\partial^2 h(\pi_\theta, x, y)}{\partial \theta(x, y_i) \partial \theta(x, y_j)} = \frac{\partial}{\partial \theta(x, y_j)}\left( \frac{\partial h(\pi_\theta, x, y)}{\partial \theta(x, y_i)} \right) = \frac{\partial (\delta_{yy_i}-\pi_\theta(y_i|x))}{\partial \theta(x, y_j)}=-\pi_\theta(y_i|x)(\delta_{y_iy_j}-\pi_\theta(y_j|x)).
\end{equation}
Thus the second derivative of $f_{\text{fKL}}(x,\theta)$ is changed into:
\begin{equation}
    \begin{aligned}
        &\frac{\partial^2 f_{\text{fKL}}(x, \theta)}{\partial \theta(x, y_i) \partial \theta(x, y_j)} \\
        =&\sum_{y} \pi_\theta(y | x)(\delta_{yy_j} - \pi_\theta(y_j | x))(\delta_{yy_i} - \pi_\theta(y_i | x))(h(\pi_\theta, x, y) +2) \\
        &\quad\quad - \pi_\theta(y | x) \pi_\theta(y_i | x)(\delta_{y_iy_j} - \pi_\theta(y_j | x))(h(\pi_\theta, x, y) + 1). \\
    \end{aligned}
\end{equation}
Let $\vec{h}(\theta,x) = (h(\pi_\theta, x, y_1), h(\pi_\theta, x, y_2), \ldots, h(\pi_\theta, x, y_{|\mathbb{Y}|}))$, $\vec{H}_1(\theta,x) =-\vec{h}(\theta,x) - 1$, and $\vec{H}_2(\theta,x) = \vec{h}(\theta,x) + 2 $. So ${H}_1(\pi_\theta, x, y)=-h(\pi_\theta, x, y)-1$ and $H_2(\pi_\theta, x, y)$ is similar. Then, the Hessian matrix of $f_{\text{fKL}}(x,\theta) = \pi_\theta(\cdot|x)^Th(\pi_\theta,x)$ simplifies as follows:
\begin{equation}
    \begin{aligned}
\nabla^2_\theta f_{\text{fKL}}(x,\theta)_{i,j} = \sum_{y \in \mathbb{Y}} [&\pi_\theta(y | x) \pi_\theta(y_i | x)(\delta_{y_iy_j} - \pi_\theta(y_j | x)) {H}_1(\pi_\theta, x, y) \\
&+ \pi_\theta(y | x)(\delta_{yy_j} - \pi_\theta(y_j | x))(\delta_{yy_i} - \pi_\theta(y_i | x)) {H}_2(\pi_\theta, x, y)].
\end{aligned}
\end{equation}

Then we compute the quadratic form of the Hessian matrix:
\begin{equation}
    \small{\begin{aligned}
&|\sum_{x\in\mathbb{X}}\sum_{y,y'\in\mathbb{Y}}z(x,y) \mathcal{D}(x) \frac{\partial^2 f_{\text{fKL}}(x,\theta)}{\partial \theta(x, y) \partial \theta(x, y')}  z(x,y')| \\
=& |\sum_{x\in\mathbb{X}}\mathcal{D}(x) \sum_{i=1}^{|\mathbb{Y}|} \sum_{j=1}^{|\mathbb{Y}|} z(x,y_i) \left(\nabla^2_\theta f_{\text{fKL}}(x,\theta)_{i,j} \right) z(x,y_j)| \\
=& |\sum_{x\in\mathbb{X}}\mathcal{D}(x) ( \pi_\theta(\cdot|x)^\top \vec{H}_1(\theta,x) (\pi_\theta(\cdot|x)^\top z(x,\cdot))^2 + \pi_\theta(\cdot|x)^\top (\vec{H}_2(\theta,x) - \vec{H}_1(\theta,x)) (\pi_\theta(\cdot|x)^\top z(x,\cdot))^2 \\
&\quad + \sum_{i=1}^K z(x,y_i)^2 \pi_\theta(y_i | x) H_2(\pi_\theta, x, y_i) - 2 \pi_\theta(\cdot|x)^\top z(x,\cdot) \pi_\theta(y_i | x) H_2(\pi_\theta, x, y_i) z(x,y_i) ) |\\
=& | \sum_{x\in\mathbb{X}}\mathcal{D}(x) ( \pi_\theta(\cdot|x)^\top \vec{H}_1(\theta,x) (\pi_\theta(\cdot|x)^\top z(x,\cdot))^2 + \pi_\theta(\cdot|x)^\top (\vec{H}_2(\theta,x) - \vec{H}_1(\theta,x)) (\pi_\theta(\cdot|x)^\top z(x,\cdot))^2 \\
&\quad+ \sum_{i=1}^K (z(x,y_i) - 2 \pi_\theta(\cdot|x)^\top z(x,\cdot)) \pi_\theta(y_i | x) H_2(\pi_\theta, x, y_i) z(x,y_i) ) | \\
\triangleq& |\sum_{x\in\mathbb{X}}\mathcal{D}(x) \psi(x) |\leq \|\mathcal{D}(\cdot)\|_1 \|\psi(\cdot)\|_\infty=1\cdot\|\psi(\cdot)\|_\infty.
\end{aligned}}
\end{equation}

To establish an upper bound for the spectral radius of the Hessian matrix, let $\vec{H}_3(\theta,x) = \vec{H}_2(\theta,x) - \vec{H}_1(\theta,x)$. We can then write:
\begin{equation}
    \begin{aligned}
& \|\psi(\cdot)\|_\infty \\ 
=& \max_{x\in\mathbb{X}}\left|\pi_\theta(\cdot|x)^\top \vec{H}_1(\theta,x) \cdot (\pi_\theta(\cdot|x)^\top z(x,\cdot))^2 + \pi_\theta(\cdot|x)^\top \vec{H}_3(\theta,x) \cdot (\pi_\theta(\cdot|x)^\top z(x,\cdot) )^2 \right.\\
&\quad+ \left. \sum_{i=1}^K z(x,y_i)^2 \pi_\theta(y_i | x) H_2(\pi_\theta, x, y_i) + \sum_{i=1}^K 2 \pi_\theta(\cdot|x)^\top z(x,\cdot) \pi_\theta(y_i | x) H_2(\pi_\theta, x, y_i) z(x,y_i) \right| \\
\leq& \max_{x\in\mathbb{X}}\left|\pi_\theta(\cdot|x)^\top \vec{H}_1(\theta,x) \cdot (\pi_\theta(\cdot|x)^\top z(x,\cdot))^2\right| + \left|\pi_\theta(\cdot|x)^\top \vec{H}_3(\theta,x) \cdot (\pi_\theta(\cdot|x)^\top z(x,\cdot))^2\right| \\
&\quad+ \left|\sum_{i=1}^K z(x,y_i)^2 \pi_\theta(y_i | x) H_2(\pi_\theta, x, y_i)\right| + \left|\sum_{i=1}^K 2 \pi_\theta(\cdot|x)^\top z(x,\cdot) \pi_\theta(y_i | x) H_2(\pi_\theta, x, y_i) z(x,y_i)\right| ,
\end{aligned}
\end{equation}
Continuing from the previous equation:
\begin{equation}
    \begin{aligned}
& \|\psi(\cdot)\|_\infty \\ 
\leq& \max_{x\in\mathbb{X}}\|\pi_\theta(\cdot|x)\|_1 \|\vec{H}_1(\theta,x)\|_\infty \|z(x,\cdot)\|_2^2 + \|\pi_\theta(\cdot|x)\|_1 \|\vec{H}_3(\theta,x)\|_\infty \|z(x,\cdot)\|_2^2 \\
&\quad+ \|\vec{H}_2(\theta,x)\|_\infty \|z(x,\cdot)\|_2^2 + 2\|\pi_\theta(\cdot|x)\odot\vec{H}_2(\theta,x)\|_1 \|z(x,\cdot)\|_2^2 \\
=& \max_{x\in\mathbb{X}}(\|\vec{H}_1(\theta,x)\|_\infty + \|\vec{H}_2(\theta,x)\|_\infty + \|\vec{H}_3(\theta,x)\|_\infty) \|z(x,\cdot)\|_2^2 + 2\|\pi_\theta(\cdot|x)\odot\vec{H}_2(\theta,x)\|_1 \|z(x,\cdot)\|_2^2 \\
\leq& \max_{x\in\mathbb{X}}(\|\vec{H}_1(\theta,x)\|_\infty + \|\vec{H}_2(\theta,x)\|_\infty + \|\vec{H}_3(\theta,x)\|_\infty) \|z(x,\cdot)\|_2^2 + 2\|\vec{H}_2(\theta,x)\|_\infty \|z(x,\cdot)\|_2^2 \\
=& \max_{x\in\mathbb{X}}(\|\vec{H}_1(\theta,x)\|_\infty + 3\|\vec{H}_2(\theta,x)\|_\infty + \|\vec{H}_3(\theta,x)\|_\infty) \|z(x,\cdot)\|_2^2\\
\leq& \max_{x\in\mathbb{X}}(\|\vec{H}_1(\theta,x)\|_\infty + 3\|\vec{H}_2(\theta,x)\|_\infty + \|\vec{H}_3(\theta,x)\|_\infty) \max_{x\in\mathbb{X}}\|z(x,\cdot)\|_2^2 \\
\leq& \max_{x\in\mathbb{X}}(\|\vec{H}_1(\theta,x)\|_\infty + 3\|\vec{H}_2(\theta,x)\|_\infty + \|\vec{H}_3(\theta,x)\|_\infty) \|z(\cdot,\cdot)\|_2^2 .
\end{aligned}
\end{equation}

In the second inequality, the first term arises from Hölder's inequality, which states that $ \pi_\theta^\top z(x,\cdot) \leq \|\pi_\theta\|_1 \|z(x,\cdot)\|_\infty = \|z(x,\cdot)\|_\infty $, where $ \|\pi_\theta\|_1 = 1 $ and $ \|z^2(x,\cdot)\|_1 = \|z(x,\cdot)\|_2^2 $. For the second term, we have the bound on $ |(\pi_\theta^\top z(x,\cdot))^2| $:
\begin{equation}
    |(\pi_\theta^\top z(x,\cdot))^2| = \pi_\theta^\top z(x,\cdot) \cdot \pi_\theta^\top z(x,\cdot) \leq \|\pi_\theta\|_1 \|z(x,\cdot)\|_\infty \cdot \|\pi_\theta\|_1 \|z(x,\cdot)\|_\infty \leq \|z(x,\cdot)\|_2^2.
\end{equation}

For the third term:
\begin{equation}
    \left| \sum_{i=1}^K z(x,y_i)^2 \pi_\theta(y_i | x) H_2(\pi_\theta, x, y_i) \right| \leq \left| \sum_{i=1}^K z(x,y_i)^2 |H_2(\pi_\theta, x, y_i)| \right| \leq \|\vec{H}_2(\theta,x)\|_\infty \|z(x,\cdot)\|_2^2.
\end{equation}

For the fourth term:
\begin{equation}
    \begin{aligned}
|\sum_{i=1}^K 2\pi_\theta^Tz(x,\cdot)\pi_\theta(y_i|x)H_2(\pi_\theta, x, y_i)z(x,y_i)|&\leq2||\pi_\theta||_1||z(x,\cdot)||_\infty|\sum_{i=1}^K \pi_\theta(y_i|x)H_2(\pi_\theta, x, y_i)z(x,y_i)||\\
&\leq2||z(x,\cdot)||_2||\pi_\theta(\cdot|x)\odot\vec{H}_2(\theta,x)||_1||z(x,\cdot)||_\infty\\
&\leq2||\vec{H}_2(\theta,x)||_\infty||z(x,\cdot)||_2^2.
\end{aligned}
\end{equation}
In this Lemma, we assume $|h(\pi_\theta,x,y)|=|\log(\pi_\theta(y|x)) - \log(\pi^\tau(y|x))|\leq\epsilon_1$ where $ \epsilon_1 $ denotes the upper bound of the error of the current parameterized distribution $ \pi_\theta $.

For $ ||\vec{H}_1(\theta,x)||_\infty $, we have:
\begin{equation}
    ||\vec{H}_1(\theta,x)||_\infty = || -\vec{h}(\theta,x) - 1 ||_\infty \leq \epsilon_1 + 1.
\end{equation}

For $ ||\vec{H}_2(\theta,x)||_\infty $, it follows that:
\begin{equation}
    ||\vec{H}_2(\theta,x)||_\infty = ||\vec{h}(\theta,x) + 2||_\infty \leq \epsilon_1 + 2.
\end{equation}

For $ ||\vec{H}_3(\theta,x)||_\infty $, we find:
\begin{equation}
    ||\vec{H}_3(\theta,x)||_\infty = ||\vec{H}_2(\theta,x) - \vec{H}_1(\theta)||_\infty = ||\vec{h}(\theta,x) + 2 + \vec{h}(\theta,x) + 1||_\infty \leq 2\epsilon_1 + 3.
\end{equation}

Combining these results yields:
\begin{equation}
    \begin{aligned}
&||\vec{H}_1(\theta)||_\infty + ||\vec{H}_2(\theta)||_\infty + ||\vec{H}_3(\theta)||_\infty \leq 6\epsilon_1+10.
\end{aligned}
\end{equation}
Thus,
\begin{equation}
\left|\sum_{x,x'\in\mathbb{X}}\sum_{y,y'\in\mathbb{Y}}z(x,y) \frac{\partial^2 \mathbb{E}_{x\sim \mathcal{D}}\left[\mathrm{D}_{\mathrm{KL}}\left(\pi_\theta\left(\cdot \mid x\right) \| \pi^\tau\left(\cdot \mid x\right)\right)\right]}{\partial \theta(x,y)\partial\theta(x',y')} z(x',y')\right|\leq\left(6\epsilon_1+10\right)||z(\cdot,\cdot)||_2^2.
\end{equation}
Proof finished.

\subsection{Proof of Lemma \ref{RA_L_coef}}\label{RA_L_coef_proof}
\begin{lemma}\label{RA_L_coef}
    % RA L smooth coefficient%算法的梯度力普希次系数
(RA Smoothness) Given softmax parametrization of Definition \ref{Softmax} for policy $\pi_\theta$, assume $|r_\theta(x,y)- r(x,y)|\leq\epsilon_2$,  $\forall r, \tau$, $\theta\rightarrow \mathbb{E}_{x\sim \mathcal{D},y\sim\pi_\theta(\cdot|x)}\left[\left(r_\theta(x,y)- r(x,y)\right)^2\right]$ is $( 3\epsilon_1^2 + \frac{18\epsilon_1}{\tau} + \frac{8}{\tau^2} + \max\left\{ \epsilon_1^2 + \frac{2}{\tau} \epsilon_1, \frac{1}{\tau} \right\} )$-smooth (see Eq.\ref{RA_eq}). See proof in Appendix \ref{RA_L_coef_proof}.

\end{lemma}

\textbf{Proof:} Denote $\mathcal{L}_{RA}(\pi_\theta)=\mathbb{E}_{x\sim \mathcal{D},y\sim\pi_\theta(\cdot|x)}\left[\left(r_\theta(x,y)- r(x,y)\right)^2\right]$ Let $S\triangleq S(\theta) \in \mathbb{R}^{K \times K}$ be the second derivative of the value map $\theta \rightarrow \mathcal{L}_{RA}(\pi_\theta)$, i.e. $S(\theta,x_1,y_1,x_2,y_2)=\frac{\partial^2 \mathcal{L}_{RA}(\pi_\theta)}{\partial \theta(x_1,y_1)\partial\theta(x_2,y_2)}$. Denote $L_{\text{RA}}=( 3\epsilon_1^2 + \frac{18\epsilon_1}{\tau} + \frac{8}{\tau^2} + \max\left\{ \epsilon_1^2 + \frac{2}{\tau} \epsilon_1, \frac{1}{\tau} \right\} )$. By Lemma \ref{spectral_radius}, it suffices to show that the spectral radius of $S$ is bounded by $L_{\text{RA}}$. Because $r_\theta(x, y) = \frac{1}{\tau} \log(Z(x)\pi_\theta(y|x))$, denote $z(x,y)=\frac{1}{\tau}\log\sum_{y\in Y}e^{\tau r(x,y)}- r(x,y)$, then we have:
\begin{equation}
\begin{aligned}
&\mathbb{E}_{x\sim \mathcal{D},y\sim\pi_\theta(\cdot|x)}\left[\left(r_\theta(x,y)- r(x,y)\right)^2\right]\\
=&\mathbb{E}_{x\sim \mathcal{D},y\sim\pi_\theta(y|x)}\left[\left(\frac{1}{\tau}\log\pi_\theta(y)+z(x,y)\right)^2\right] \\
=&\sum_{x\in\mathbb{X},y\in\mathbb{Y}}\mathcal{D}(x)\pi_\theta(y|x)\left(\frac{1}{\tau}\log\pi_\theta(y|x)+z(x,y)\right)^2\\
=&\sum_{x\in\mathbb{X}}\mathcal{D}(x)\pi_\theta(\cdot|x)^Th(\pi_\theta,x)\quad \text{where}\ h(\pi_\theta,x,y)=\left(\frac{1}{\tau}\log\pi_\theta(y|x)+z(x,y)\right)^2,\ h(\pi_\theta,x)=h(\pi_\theta,x,\cdot).
\end{aligned}
\end{equation}

Now, by Definition \ref{Softmax} we have:
\begin{equation}
    \pi_\theta(y \mid x)=\frac{\exp \{\theta(x, y)\}}{\sum_{y^{\prime}} \exp \left\{\theta\left(x, y^{\prime}\right)\right\}}.
\end{equation}
By Lemma \ref{softmax_1derivative}, we have: 
\begin{equation}
    \frac{\partial \pi_\theta(y|x)}{\partial \theta(x,y')} = \pi_\theta(y|x) (\delta_{yy'} - \pi_\theta(y'|x)),\ \text{where}\ \delta_{yy'}= \begin{cases}1, & \text { if } y=y' \\ 0, & \text { otherwise }\end{cases}.
\end{equation}

If $x\neq x'$, then the second derivative of $\pi_\theta(y|x)$ is 0:
\begin{equation}
    \frac{\partial^2 \pi_\theta(y|x)}{\partial \theta(x,y_1)\partial\theta(x',y_2)} = \frac{\partial \pi_\theta(y|x) (\delta_{yy_1} - \pi_\theta(y_1|x))}{\partial \theta(x',y_2)}=0.
\end{equation}
Similarly, because $\pi_\theta(y|x)$ and $h(\pi_\theta,x)$ don't have relevant parameters to $\theta(x',\cdot)$, for $x\neq x'$, we have:
\begin{equation}
    \frac{\partial^2 \pi_\theta(\cdot|x)^Th(\pi_\theta,x)}{\partial \theta(x,y_1)\partial\theta(x',y_2)} =0.
\end{equation}
Therefore if $x_1\neq x_2$,
\begin{equation}
    \frac{\partial^2 \mathbb{E}_{x\sim \mathcal{D},y\sim\pi_\theta(\cdot|x)}\left[\left(r_\theta(x,y)- r(x,y)\right)^2\right]}{\partial \theta(x_1,y_1)\partial\theta(x_2,y_2)} = \mathcal{D}(x_1) \frac{\partial}{\partial\theta(x_2,y_2)} \left(\frac{\partial \pi_\theta(\cdot|x_1)^Th(\pi_\theta,x_1)}{\partial \theta(x_1,y_1)}\right)=0.
\end{equation}

So the only care about the term when $x_1=x_2$, i.e.
\begin{equation}
    \frac{\partial^2 \mathbb{E}_{x\sim \mathcal{D},y\sim\pi_\theta(\cdot|x)}\left[\left(r_\theta(x,y)- r(x,y)\right)^2\right]}{\partial \theta(x,y_1)\partial\theta(x,y_2)} = \mathcal{D}(x) \frac{\partial}{\partial\theta(x,y_2)} \left(\frac{\partial \pi_\theta(\cdot|x)^Th(\pi_\theta,x)}{\partial \theta(x,y_1)}\right).
\end{equation}
Denote $f_{RA}(x,\theta) = \pi_\theta(\cdot|x)^Th(\pi_\theta,x) = \sum_{y \in \mathbb{Y}} \pi_\theta(y|x)h(\pi_\theta,x,y)$ and $h(\pi_\theta,x,y)={g}^2(\pi_\theta,x,y),\ g(\pi_\theta,x,y) = \frac{1}{\tau} \log \pi_\theta(y|x) + z(x,y)=r_\theta(x,y)- r(x,y)$, the first derivative of $ f_{RA}(x,\theta) $ is given by:
\begin{equation}
    \frac{\partial f_{RA}(x,\theta)}{\partial \theta(x,y')} = \sum_{y \in \mathbb{Y}} \left[\frac{\partial \pi_\theta(y|x)}{\partial \theta(x,y')} g^2(\pi_\theta,x,y) + 2\pi_\theta(y|x) g(\pi_\theta,x,y) \frac{\partial g(\pi_\theta,x,y)}{\partial \theta(x,y')}\right].
\end{equation}
where, $\frac{\partial g(\pi_\theta,x,y)}{\partial \theta(x,y')} = \frac{1}{\tau} \frac{\partial\log \pi_\theta(y|x)}{\partial \theta(x,y')}  = \frac{1}{\tau} (\delta_{yy'} - \pi_\theta(y'|x))$.

To compute the Hessian matrix, we take the derivative of this expression with respect to $\theta(y_2)$, yielding the $(y_1, y_2)$-th element as
\begin{equation}
    \frac{\partial^2 f_{RA}(x,\theta)}{\partial \theta(x,y_1) \partial \theta(x,y_2)} = \frac{\partial}{\partial \theta(x,y_2)} \left[\sum_{y \in \mathbb{Y}} \left(\frac{\partial \pi_\theta(y|x)}{\partial \theta(x,y_1)} g^2(\pi_\theta,x,y) + 2\pi_\theta(y|x) g(\pi_\theta,x,y) \frac{\partial g(\pi_\theta,x,y)}{\partial \theta(x,y_1)}\right)\right].
\end{equation}
We proceed by decomposing this expression into two parts for further derivation. The first part of our derivation is given by:
\begin{equation}
    \sum_{y \in \mathbb{Y}} \frac{\partial \pi_\theta(y | x)}{\partial \theta(x, y_1)} g^2(\pi_\theta, x, y) = \sum_{y \in \mathbb{Y}} \pi_\theta(y | x) (\delta_{yy_1} - \pi_\theta(y_1 | x)) g^2(\pi_\theta, x, y).
\end{equation}
Applying the chain rule, we first differentiate $\pi_\theta(y | x)(\delta_{yy_1} - \pi_\theta(y_1 | x))$ with respect to $\theta(x, y_2)$:
\begin{equation}
    \frac{\partial}{\partial \theta(x, y_2)} \left(\pi_\theta(y | x)(\delta_{yy_1} - \pi_\theta(y_1 | x))\right) = \frac{\partial \pi_\theta(y | x)}{\partial \theta(x, y_2)} (\delta_{yy_1} - \pi_\theta(y_1 | x)) - \pi_\theta(y | x) \frac{\partial \pi_\theta(y_1 | x)}{\partial \theta(x, y_2)}.
\end{equation}
Since $\frac{\partial \pi_\theta(y | x)}{\partial \theta(x, y_2)} = \pi_\theta(y | x)(\delta_{yy_2} - \pi_\theta(y_2 | x))$ and $\frac{\partial \pi_\theta(y_1 | x)}{\partial \theta(x, y_2)} = \pi_\theta(y_1 | x)(\delta_{y_1y_2} - \pi_\theta(y_2 | x))$, the derivative of the first part is
\begin{equation}
    \frac{\partial\left(\pi_\theta(y | x)(\delta_{yy_1} - \pi_\theta(y_1 | x))\right)}{\partial \theta(x, y_2)} = \pi_\theta(y | x)(\delta_{yy_2} - \pi_\theta(y_2 | x))(\delta_{yy_1} - \pi_\theta(y_1 | x)) - \pi_\theta(y | x) \pi_\theta(y_1 | x)(\delta_{y_1y_2} - \pi_\theta(y_2 | x)).
\end{equation}

Next, we differentiate $g^2(\pi_\theta, x, y)$ with respect to $\theta(x, y_2)$. Noting that $\frac{\partial g(\pi_\theta, x, y)}{\partial \theta(x, y_2)} = \frac{1}{\tau} (\delta_{yy_2} - \pi_\theta(y_2 | x))$, we obtain
\begin{equation}
    \frac{\partial g^2(\pi_\theta, x, y)}{\partial \theta(x, y_2)} = 2 g(\pi_\theta, x, y) \frac{\partial g(\pi_\theta, x, y)}{\partial \theta(x, y_2)} = \frac{2}{\tau} g(\pi_\theta, x, y)(\delta_{yy_2} - \pi_\theta(y_2 | x)).
\end{equation}
Combining all results, we obtain the derivative of the first part as follows:
\begin{equation}
    \begin{aligned}
\sum_{y \in \mathbb{Y}} &\left[\left(\pi_\theta(y | x)(\delta_{yy_2} - \pi_\theta(y_2 | x))(\delta_{yy_1} - \pi_\theta(y_1 | x)) - \pi_\theta(y | x) \pi_\theta(y_1 | x)(\delta_{y_1y_2} - \pi_\theta(y_2 | x))\right) g^2(\pi_\theta, x, y)\right. \\
&+ \left.\pi_\theta(y | x)(\delta_{yy_1} - \pi_\theta(y_1 | x)) \frac{2}{\tau} g(\pi_\theta, x, y)(\delta_{yy_2} - \pi_\theta(y_2 | x)) \right].
\end{aligned}
\end{equation}

For the second part, we begin with the expression:
\begin{equation}
    2\sum_{y \in \mathbb{Y}} \pi_\theta(y | x) g(\pi_\theta, x, y) \frac{\partial g(\pi_\theta, x, y)}{\partial \theta(x, y_1)} = \frac{2}{\tau} \sum_{y \in \mathbb{Y}} \pi_\theta(y | x) g(\pi_\theta, x, y)(\delta_{yy_1} - \pi_\theta(y_1 | x)).
\end{equation}

Using the product rule, this derivative with respect to $\theta(x, y_2)$ consists of two components:

1. The derivative of $\pi_\theta(y | x)$: 
   $\frac{2}{\tau} \sum_{y \in \mathbb{Y}} \frac{\partial \pi_\theta(y | x)}{\partial \theta(x, y_2)} g(\pi_\theta, x, y)(\delta_{yy_1} - \pi_\theta(y_1 | x))$,
   
2. The derivative of $g(\pi_\theta, x, y)(\delta_{yy_1} - \pi_\theta(y_1 | x))$: $\frac{2}{\tau} \sum_{y \in \mathbb{Y}} \pi_\theta(y | x) \left[\frac{\partial g(\pi_\theta, x, y)}{\partial \theta(x, y_2)}(\delta_{yy_1} - \pi_\theta(y_1 | x)) - g(\pi_\theta, x, y)\frac{\partial \pi_\theta(y_1 | x)}{\partial \theta(x, y_2)}\right]$.

For the derivative of $\pi_\theta(y | x)$, noting that $\frac{\partial \pi_\theta(y | x)}{\partial \theta(x, y_2)} = \pi_\theta(y | x)(\delta_{yy_2} - \pi_\theta(y_2 | x))$, we obtain
\begin{equation}
    \frac{2}{\tau} \sum_{y \in \mathbb{Y}} \pi_\theta(y | x)(\delta_{yy_2} - \pi_\theta(y_2 | x)) g(\pi_\theta, x, y)(\delta_{yy_1} - \pi_\theta(y_1 | x)).
\end{equation}

For the derivative of $g(\pi_\theta, x, y)(\delta_{yy_1} - \pi_\theta(y_1 | x))$, using $g(\pi_\theta, x, y) = \frac{1}{\tau} \log \pi_\theta(y | x) + z(y)$, we find
\begin{equation}
    \frac{\partial g(\pi_\theta, x, y)}{\partial \theta(x, y_2)} = \frac{1}{\tau} (\delta_{yy_2} - \pi_\theta(y_2 | x)),
\end{equation}
and since $\frac{\partial \pi_\theta(y_1 | x)}{\partial \theta(x, y_2)} = \pi_\theta(y_1 | x)(\delta_{y_1y_2} - \pi_\theta(y_2 | x))$, the derivative of the second part simplifies to:
\begin{equation}
    \small{\begin{aligned}
&2\sum_{y \in \mathbb{Y}} \pi_\theta(y | x) g(\pi_\theta, x, y) \frac{\partial g(\pi_\theta, x, y)}{\partial \theta(x, y_1)} = \frac{2}{\tau} \sum_{y \in \mathbb{Y}} \pi_\theta(y | x)(\delta_{yy_2} - \pi_\theta(y_2 | x)) g(\pi_\theta, x, y)(\delta_{yy_1} - \pi_\theta(y_1 | x))\\
&\quad+ \frac{2}{\tau^2} \sum_{y \in \mathbb{Y}} \pi_\theta(y | x)(\delta_{yy_2} - \pi_\theta(y_2 | x))(\delta_{yy_1} - \pi_\theta(y_1 | x)) - \frac{2}{\tau} \sum_{y \in \mathbb{Y}} \pi_\theta(y | x) g(\pi_\theta, x, y) \pi_\theta(y_1 | x)(\delta_{y_1y_2} - \pi_\theta(y_2 | x)).
\end{aligned}}
\end{equation}

Finally, combining terms, we derive the compact expression for the derivative of the second part with respect to $\theta(x, y_2)$:
\begin{equation}
    \small{\begin{aligned}
&2\sum_{y \in \mathbb{Y}} \pi_\theta(y | x) g(\pi_\theta, x, y) \frac{\partial g(\pi_\theta, x, y)}{\partial \theta(x, y_1)}\\
=& \sum_{y \in \mathbb{Y}} \left(\frac{2}{\tau} g(\pi_\theta, x, y) + \frac{2}{\tau^2}\right)\pi_\theta(y | x)(\delta_{yy_2} - \pi_\theta(y_2 | x))(\delta_{yy_1} - \pi_\theta(y_1 | x)) - \frac{2}{\tau} \pi_\theta(y | x) g(\pi_\theta, x, y) \pi_\theta(y_1 | x)(\delta_{y_1y_2} - \pi_\theta(y_2 | x)).
\end{aligned}}
\end{equation}

By gathering all terms, we derive the full expression for the Hessian matrix:
\begin{equation}
\small{    \begin{aligned}
&\frac{\partial}{\partial\theta(x,y_2)} \left(\frac{\partial \pi_\theta(\cdot|x)^Th(\pi_\theta,x)}{\partial \theta(x,y_1)}\right) = \frac{\partial^2 f_{RA}(x,\theta)}{\partial \theta(x, y_1) \partial \theta(x, y_2)} \\
=& \frac{\partial}{\partial \theta(x, y_2)} \left[\sum_{y \in \mathbb{Y}} \left(\frac{\partial \pi_\theta(y | x)}{\partial \theta(x, y_1)} g(\pi_\theta, x, y)^2 + 2 \pi_\theta(y | x) g(\pi_\theta, x, y) \frac{\partial g(\pi_\theta, x, y)}{\partial \theta(x, y_1)}\right)\right] \\
=& \sum_{y \in \mathbb{Y}} \left[\left(\pi_\theta(y | x)(\delta_{yy_2} - \pi_\theta(y_2 | x))(\delta_{yy_1} - \pi_\theta(y_1 | x)) - \pi_\theta(y | x) \pi_\theta(y_1 | x)(\delta_{y_1y_2} - \pi_\theta(y_2 | x))\right) g(\pi_\theta, x, y)^2\right. \\
&\quad+ \left.\pi_\theta(y | x)(\delta_{yy_1} - \pi_\theta(y_1 | x)) \frac{2}{\tau} g(\pi_\theta, x, y)(\delta_{yy_2} - \pi_\theta(y_2 | x))\right] \\
&+ \sum_{y \in \mathbb{Y}} \left(\frac{2}{\tau} g(\pi_\theta, x, y) + \frac{2}{\tau^2}\right) \pi_\theta(y | x)(\delta_{yy_2} - \pi_\theta(y_2 | x))(\delta_{yy_1} - \pi_\theta(y_1 | x)) \\
&\quad- \frac{2}{\tau} \pi_\theta(y | x) g(\pi_\theta, x, y) \pi_\theta(y_1 | x)(\delta_{y_1y_2} - \pi_\theta(y_2 | x)) \\
=& \sum_{y \in \mathbb{Y}} \left(g(\pi_\theta, x, y)^2 + \frac{4}{\tau} g(\pi_\theta, x, y) + \frac{2}{\tau^2}\right) \pi_\theta(y | x)(\delta_{yy_2} - \pi_\theta(y_2 | x))(\delta_{yy_1} - \pi_\theta(y_1 | x)) \\
&\quad - \sum_{y \in \mathbb{Y}} \left(g(\pi_\theta, x, y)^2 + \frac{2}{\tau} g(\pi_\theta, x, y)\right) \pi_\theta(y | x) \pi_\theta(y_1 | x)(\delta_{y_1y_2} - \pi_\theta(y_2 | x)).
\end{aligned}}
\end{equation}

Let $\vec{g}(\theta,x) = (g(\pi_\theta, x, y_1), g(\pi_\theta, x, y_2), \ldots, g(\pi_\theta, x, y_{|\mathbb{Y}|}))$, $\vec{G}_1(\theta,x) = -\vec{g}^2(\theta,x) - \frac{2}{\tau} \vec{g}(\theta,x)$, and $\vec{G}_2(\theta,x) = (\vec{g}(\theta,x) + \frac{2}{\tau})^2 - \frac{2}{\tau^2}$. So ${G}_1(\pi_\theta, x, y)=-g^2(\pi_\theta, x, y)-\frac{2}{\tau}g(\pi_\theta, x, y)$ and $G_2(\pi_\theta, x, y)$ is similar. Then, the Hessian matrix of $f_{RA}(x,\theta) = \pi_\theta(\cdot|x)^Th(\pi_\theta,x)$ simplifies as follows:
\begin{equation}
    \begin{aligned}
\nabla^2_\theta f_{RA}(x,\theta)_{i,j} = \sum_{y \in \mathbb{Y}} [&\pi_\theta(y | x) \pi_\theta(y_i | x)(\delta_{y_iy_j} - \pi_\theta(y_j | x)) {G}_1(\pi_\theta, x, y) \\
&+ \pi_\theta(y | x)(\delta_{yy_j} - \pi_\theta(y_j | x))(\delta_{yy_i} - \pi_\theta(y_i | x)) {G}_2(\pi_\theta, x, y)].
\end{aligned}
\end{equation}

Then we can calculate the spectral radius of $S(\theta)$, because if $x_1\neq x_2$, $\frac{\partial^2 \mathbb{E}_{x\sim \mathcal{D},y\sim\pi_\theta(\cdot|x)}\left[\left(r_\theta(x,y)- r(x,y)\right)^2\right]}{\partial \theta(x_1,y_1)\partial\theta(x_2,y_2)}=0$. So
\begin{equation}
\begin{aligned}
&\left|\sum_{x,x'\in\mathbb{X}}\sum_{y,y'\in\mathbb{Y}}z(x,y)S(\theta,x,y,x',y')z(x',y')\right| = \left|\sum_{x,x'\in\mathbb{X}}\sum_{y,y'\in\mathbb{Y}}z(x,y) \frac{\partial^2 \mathcal{L}_{RA}(\pi_\theta)}{\partial \theta(x,y)\partial\theta(x',y')} z(x',y')\right|\\
=& \left|\sum_{x\in\mathbb{X}}\sum_{y,y'\in\mathbb{Y}}z(x,y) \frac{\partial^2 \mathcal{L}_{RA}(\pi_\theta)}{\partial \theta(x,y)\partial\theta(x,y')} z(x,y')\right| =\left|\sum_{x\in\mathbb{X}}\sum_{y,y'\in\mathbb{Y}}z(x,y) \mathcal{D}(x) \frac{\partial^2 f_{RA}(x,\theta)}{\partial \theta(x, y) \partial \theta(x, y')}  z(x,y')\right|.
\end{aligned}
\end{equation}

To compute the quadratic form of the Hessian matrix, we proceed as follows:
\begin{equation}
    \small{\begin{aligned}
&|\sum_{x\in\mathbb{X}}\sum_{y,y'\in\mathbb{Y}}z(x,y) \mathcal{D}(x) \frac{\partial^2 f_{RA}(x,\theta)}{\partial \theta(x, y) \partial \theta(x, y')}  z(x,y')| \\
=& |\sum_{x\in\mathbb{X}}\mathcal{D}(x) \sum_{i=1}^{|\mathbb{Y}|} \sum_{j=1}^{|\mathbb{Y}|} z(x,y_i) \left(\nabla^2_\theta f_{RA}(x,\theta)_{i,j} \right) z(x,y_j)| \\
=& |\sum_{x\in\mathbb{X}}\mathcal{D}(x) ( \pi_\theta(\cdot|x)^\top \vec{G}_1(\theta,x) (\pi_\theta(\cdot|x)^\top z(x,\cdot))^2 + \pi_\theta(\cdot|x)^\top (\vec{G}_2(\theta,x) - \vec{G}_1(\theta,x)) (\pi_\theta(\cdot|x)^\top z(x,\cdot))^2 \\
&\quad + \sum_{i=1}^K z(x,y_i)^2 \pi_\theta(y_i | x) G_2(\pi_\theta, x, y_i) - 2 \pi_\theta(\cdot|x)^\top z(x,\cdot) \pi_\theta(y_i | x) G_2(\pi_\theta, x, y_i) z(x,y_i) ) |\\
=& | \sum_{x\in\mathbb{X}}\mathcal{D}(x) ( \pi_\theta(\cdot|x)^\top \vec{G}_1(\theta,x) (\pi_\theta(\cdot|x)^\top z(x,\cdot))^2 + \pi_\theta(\cdot|x)^\top (\vec{G}_2(\theta,x) - \vec{G}_1(\theta,x)) (\pi_\theta(\cdot|x)^\top z(x,\cdot))^2 \\
&\quad+ \sum_{i=1}^K (z(x,y_i) - 2 \pi_\theta(\cdot|x)^\top z(x,\cdot)) \pi_\theta(y_i | x) G_2(\pi_\theta, x, y_i) z(x,y_i) ) | \\
\triangleq& |\sum_{x\in\mathbb{X}}\mathcal{D}(x) \psi(x) |\leq \|\mathcal{D}(\cdot)\|_1 \|\psi(\cdot)\|_\infty=1\cdot\|\psi(\cdot)\|_\infty.
\end{aligned}}
\end{equation}

To establish an upper bound for the spectral radius of the Hessian matrix, let $\vec{G}_3(\theta,x) = \vec{G}_2(\theta,x) - \vec{G}_1(\theta,x)$. We can then write:
\begin{equation}
    \begin{aligned}
& \|\psi(\cdot)\|_\infty \\ 
=& \max_{x\in\mathbb{X}}\left|\pi_\theta(\cdot|x)^\top \vec{G}_1(\theta,x) \cdot (\pi_\theta(\cdot|x)^\top z(x,\cdot))^2 + \pi_\theta(\cdot|x)^\top \vec{G}_3(\theta,x) \cdot (\pi_\theta(\cdot|x)^\top z(x,\cdot) )^2 \right.\\
&\quad+ \left. \sum_{i=1}^K z(x,y_i)^2 \pi_\theta(y_i | x) G_2(\pi_\theta, x, y_i) + \sum_{i=1}^K 2 \pi_\theta(\cdot|x)^\top z(x,\cdot) \pi_\theta(y_i | x) G_2(\pi_\theta, x, y_i) z(x,y_i) \right| \\
\leq& \max_{x\in\mathbb{X}}\left|\pi_\theta(\cdot|x)^\top \vec{G}_1(\theta,x) \cdot (\pi_\theta(\cdot|x)^\top z(x,\cdot))^2\right| + \left|\pi_\theta(\cdot|x)^\top \vec{G}_3(\theta,x) \cdot (\pi_\theta(\cdot|x)^\top z(x,\cdot))^2\right| \\
&\quad+ \left|\sum_{i=1}^K z(x,y_i)^2 \pi_\theta(y_i | x) G_2(\pi_\theta, x, y_i)\right| + \left|\sum_{i=1}^K 2 \pi_\theta(\cdot|x)^\top z(x,\cdot) \pi_\theta(y_i | x) G_2(\pi_\theta, x, y_i) z(x,y_i)\right| \\
\leq& \max_{x\in\mathbb{X}}\|\pi_\theta(\cdot|x)\|_1 \|\vec{G}_1(\theta,x)\|_\infty \|z(x,\cdot)\|_2^2 + \|\pi_\theta(\cdot|x)\|_1 \|\vec{G}_3(\theta,x)\|_\infty \|z(x,\cdot)\|_2^2 \\
&\quad+ \|\vec{G}_2(\theta,x)\|_\infty \|z(x,\cdot)\|_2^2 + 2\|\pi_\theta(\cdot|x)\odot\vec{G}_2(\theta,x)\|_1 \|z(x,\cdot)\|_2^2 \\
=& \max_{x\in\mathbb{X}}\|\vec{G}_1(\theta,x)\|_\infty \|z(x,\cdot)\|_2^2 + \|\vec{G}_3(\theta,x)\|_\infty \|z(x,\cdot)\|_2^2 + \|\vec{G}_2(\theta,x)\|_\infty \|z(x,\cdot)\|_2^2 + 2\|\vec{G}_2(\theta,x)\|_\infty \|z(x,\cdot)\|_2^2 \\
=& \max_{x\in\mathbb{X}}(\|\vec{G}_1(\theta,x)\|_\infty + 3\|\vec{G}_2(\theta,x)\|_\infty + \|\vec{G}_3(\theta,x)\|_\infty) \|z(x,\cdot)\|_2^2\\
\leq& \max_{x\in\mathbb{X}}(\|\vec{G}_1(\theta,x)\|_\infty + 3\|\vec{G}_2(\theta,x)\|_\infty + \|\vec{G}_3(\theta,x)\|_\infty) \max_{x\in\mathbb{X}}\|z(x,\cdot)\|_2^2 \\
\leq& \max_{x\in\mathbb{X}}(\|\vec{G}_1(\theta,x)\|_\infty + 3\|\vec{G}_2(\theta,x)\|_\infty + \|\vec{G}_3(\theta,x)\|_\infty) \|z(\cdot,\cdot)\|_2^2 .
\end{aligned}
\end{equation}
In the second inequality, the first term arises from Hölder's inequality, which states that $ \pi_\theta^\top z(x,\cdot) \leq \|\pi_\theta\|_1 \|z(x,\cdot)\|_\infty = \|z(x,\cdot)\|_\infty $, where $ \|\pi_\theta\|_\infty \leq 1 $ and $ \|z^2(x,\cdot)\|_1 = \|z(x,\cdot)\|_2^2 $. For the second term, we have the bound on $ |(\pi_\theta^\top z(x,\cdot))^2| $:
\begin{equation}
    |(\pi_\theta^\top z(x,\cdot))^2| = \pi_\theta^\top z(x,\cdot) \cdot \pi_\theta^\top z(x,\cdot) \leq \|\pi_\theta\|_1 \|z(x,\cdot)\|_\infty \cdot \|\pi_\theta\|_1 \|z(x,\cdot)\|_\infty \leq \|z(x,\cdot)\|_2^2.
\end{equation}

For the third term:
\begin{equation}
    \left| \sum_{i=1}^K z(x,y_i)^2 \pi_\theta(y_i | x) G_2(y_i) \right| \leq \left| \sum_{i=1}^K z(x,y_i)^2 |G_2(y_i)| \right| \leq \|\vec{G}_2(\theta)\|_\infty \|z^2(x,\cdot)\|_1 = \|\vec{G}_2(\theta)\|_\infty \|z(x,\cdot)\|_2^2.
\end{equation}

For the fourth term:
\begin{equation}
    \begin{aligned}
|\sum_{i=1}^K 2\pi_\theta^Tz(x,\cdot)\pi_\theta(y_i)G_2(y_i)z(x,y_i)|&\leq2||\pi_\theta||_1||z(x,\cdot)||_\infty|\sum_{i=1}^K \pi_\theta(y_i)G_2(y_i)z(x,y_i)||\\
&\leq2||z(x,\cdot)||_2||\pi_\theta(\cdot|x)\odot\vec{G}_2(\theta)||_1||z(x,\cdot)||_\infty\\
&\leq2||\vec{G}_2(\theta)||_\infty||z(x,\cdot)||_2^2.
\end{aligned}
\end{equation}
Next, we analyze the coefficient $ (\|\vec{G}_1(\theta,x)\|_\infty + \|\vec{G}_2(\theta,x)\|_\infty + \|\vec{G}_3(\theta,x)\|_\infty + \|\vec{G}_2(\theta,x)\|_1) $. The term $ |g(\pi_\theta, x, y)| $ represents the error between the parameterized distribution $ \pi_\theta $ and the true distribution for component $ (\pi_\theta, x, y) $. In this Lemma, we assume $|r_\theta(x,y)- r(x,y)|\leq\epsilon_1$. Therefore $ |g(\pi_\theta, x, y)| \leq \epsilon_1 $ for all components $ (\pi_\theta, x, y) $, where $ \epsilon_1 $ denotes the upper bound of the error of the current parameterized distribution $ \pi_\theta $.

For $ ||\vec{G}_1(\theta,x)||_\infty $, we have:
\begin{equation}
    ||\vec{G}_1(\theta,x)||_\infty = ||\vec{g}^2(\theta,x) + \frac{2}{\tau} \vec{g}(\theta,x)||_\infty \leq \max\{\epsilon_1^2 + \frac{2}{\tau} \epsilon_1, \frac{1}{\tau}\}.
\end{equation}

For $ ||\vec{G}_2(\theta,x)||_\infty $, it follows that:
\begin{equation}
    ||\vec{G}_2(\theta,x)||_\infty = ||\vec{g}^2(\theta,x) + \frac{4}{\tau} \vec{g}(\theta,x) + \frac{2}{\tau^2}||_\infty \leq \epsilon_1^2 + \frac{4\epsilon_1}{\tau} + \frac{2}{\tau^2}.
\end{equation}

For $ ||\vec{G}_3(\theta,x)||_\infty $, we find:
\begin{equation}
    ||\vec{G}_3(\theta,x)||_\infty = ||\vec{G}_2(\theta,x) - \vec{G}_1(\theta)||_\infty = ||\frac{6}{\tau} \vec{g}(\theta,x) + \frac{2}{\tau^2}||_\infty \leq \frac{6\epsilon_1}{\tau} + \frac{2}{\tau^2}.
\end{equation}

Combining these results yields:
\begin{equation}
    \begin{aligned}
&||\vec{G}_1(\theta)||_\infty + 3||\vec{G}_2(\theta)||_\infty + ||\vec{G}_3(\theta)||_\infty \\
\leq & 3\epsilon_1^2 + \frac{18\epsilon_1}{\tau} + \frac{8}{\tau^2} + \max\left\{ \epsilon_1^2 + \frac{2}{\tau} \epsilon_1, \frac{1}{\tau} \right\} .
\end{aligned}
\end{equation}

Thus, the Lipschitz coefficient of the gradient of the target function with respect to the parameter $ \theta $ is given by:
\begin{equation}
    \begin{aligned}
&\left|\sum_{x,x'\in\mathbb{X}}\sum_{y,y'\in\mathbb{Y}}z(x,y)S(\theta,x,y,x',y')z(x',y')\right| \\
\leq& \max_{x\in\mathbb{X}}(\|\vec{G}_1(\theta,x)\|_\infty + 3\|\vec{G}_2(\theta,x)\|_\infty + \|\vec{G}_3(\theta,x)\|_\infty) \|z(\cdot,\cdot)\|_2^2\\
\leq & ( 3\epsilon_1^2 + \frac{18\epsilon_1}{\tau} + \frac{8}{\tau^2} + \max\left\{ \epsilon_1^2 + \frac{2}{\tau} \epsilon_1, \frac{1}{\tau} \right\} )\|z(\cdot,\cdot)\|_2^2.
    \end{aligned}
\end{equation}
Proof finished.


\subsection{Proof of Lemma \ref{RDA_L_coef}}\label{RDA_L_coef_proof}
\begin{lemma}\label{RDA_L_coef}
    % RDA L smooth coefficient%算法的梯度力普希次系数 
    (RDA Smoothness) Given softmax parametrization of Definition \ref{Softmax} for policy $\pi_\theta$, assume $|(r_\theta(x,y_1)-r_\theta(x,y_2))-(r(x,y_1)-r(x,y_2))|\leq\epsilon_3$,  $\forall r, \tau$, $\theta\rightarrow \mathbb{E}_{\mathcal{D}_{\text{pw}}}\left[\left( \frac{1}{\tau}\log\frac{\pi_\theta(y_1|x)}{\pi_\theta(y_2|x)}-(r(x,y_1)-r(x,y_2))\right)^2\right]$ is $\left(20\epsilon_2^2+\frac{32\epsilon_2}{\tau}+\frac{8}{\tau^2}\right)$-smooth where $\mathcal{D}_{\text{pw}}\triangleq\{(x, y_1, y_2) \mid x \in \mathcal{D}, y_1, y_2 \sim \pi_\theta(y|x)\}$ (see Eq.\ref{pair-wised_RA_eq}). See proof in Appendix \ref{RDA_L_coef_proof}.

\end{lemma}

\textbf{Proof:} By Lemma \ref{spectral_radius}, it suffices to show that the spectral radius of the hessian matrix of the second derivative of $\mathcal{L}_{\text{RDA}}(\pi_\theta) = \mathbb{E}_{\mathcal{D}_{\text{pw}}}\left[\left( \frac{1}{\tau}\log\frac{\pi_\theta(y_1|x)}{\pi_\theta(y_2|x)}-(r(x,y_1)-r(x,y_2))\right)^2\right]$ is bounded by $\left(20\epsilon_2^2+\frac{32\epsilon_2}{\tau}+\frac{8}{\tau^2}\right)$, i.e.
\begin{equation}\label{spectral_radius_RDA}
\left|\sum_{x,x'\in\mathbb{X}}\sum_{y,y'\in\mathbb{Y}}z(x,y) \frac{\partial^2 \mathcal{L}_{\text{RDA}}(\pi_\theta)}{\partial \theta(x,y)\partial\theta(x',y')} z(x',y')\right|\leq\left(20\epsilon_2^2+\frac{32\epsilon_2}{\tau}+\frac{8}{\tau^2}\right)||z(\cdot,\cdot)||_2^2.
\end{equation}
To show the bound, denote $h(\pi_\theta,x,y_1,y_2)=g^2\left( \pi_\theta,x,y_1,y_2\right)$ where $g(\pi_\theta,x,y_1,y_2)=\frac{1}{\tau}\log\frac{\pi_\theta(y_1|x)}{\pi_\theta(y_2|x)}-(r(x,y_1)-r(x,y_2))$, we have:
\begin{equation}
\begin{aligned}
&\left|\sum_{x,x'\in\mathbb{X}}\sum_{y,y'\in\mathbb{Y}}z(x,y) \frac{\partial^2 \mathbb{E}_{\mathcal{D}_{\text{pw}}}\left[\left( \frac{1}{\tau}\log\frac{\pi_\theta(y_1|x)}{\pi_\theta(y_2|x)}-(r(x,y_1)-r(x,y_2))\right)^2\right]}{\partial \theta(x,y)\partial\theta(x',y')} z(x',y')\right|\\
=&\left|\sum_{x\in\mathbb{X}}\sum_{y,y'\in\mathbb{Y}}z(x,y) \frac{\partial^2 \sum_{x\in\mathbb{X}}\mathcal{D}(x)\sum_{y_1,y_2\in\mathbb{Y}}\pi_\theta(y_1|x)\pi_\theta(y_2|x)h(\pi_\theta,x,y_1,y_2)}{\partial \theta(x,y)\partial\theta(x,y')}  z(x,y')\right|\\ =&\left|\sum_{x\in\mathbb{X}}\mathcal{D}(x)\sum_{y,y'\in\mathbb{Y}}z(x,y)  \frac{\partial^2 f_{\text{RDA}}(x,\theta)}{\partial \theta(x, y) \partial \theta(x, y')}  z(x,y')\right| \\
\triangleq& |\sum_{x\in\mathbb{X}}\mathcal{D}(x) \psi(x) |\leq \|\mathcal{D}(\cdot)\|_1 \|\psi(\cdot)\|_\infty=1\cdot\|\psi(\cdot)\|_\infty.
\end{aligned}
\end{equation}
where $f_{\text{RDA}}(x,\theta)=\sum_{y_1,y_2\in\mathbb{Y}}\pi_\theta(y_1|x)\pi_\theta(y_2|x)h(\pi_\theta,x,y_1,y_2)$.

Based on Lemma \ref{Pair_second_derivative_lemma}, the second derivative of $f_{\text{RDA}}(x,\theta)$ is:
\begin{equation}\label{RDA_decompose}
    \begin{aligned}
        \frac{\partial^2 f_{\text{RDA}}(x, \theta)}{\partial \theta(x, y_i) \partial \theta(x, y_j)} = \sum_{y_1,y_2\in\mathbb{Y}}& 2\frac{\partial^2 \pi_\theta(y_1|x)}{\partial \theta(x, y_i) \partial \theta(x, y_j)} \pi_\theta(y_2|x)h(\pi_\theta,x,y_1,y_2) + 2\frac{\partial \pi_\theta(y_1|x)}{\partial \theta(x, y_i)}\frac{\partial \pi_\theta(y_2|x)}{\partial \theta(x, y_j)}h(\pi_\theta,x,y_1,y_2) \\
        &+ 2\frac{\partial \pi_\theta(y_1|x)}{\partial \theta(x, y_i)}\pi_\theta(y_2|x) \frac{\partial h(\pi_\theta, x, y_1,y_2)}{\partial \theta(x, y_j)}+ 2\frac{\partial \pi_\theta(y_1|x)}{\partial \theta(x, y_j)}\pi_\theta(y_2|x) \frac{\partial h(\pi_\theta, x, y_1,y_2)}{\partial \theta(x, y_i)}\\
        & + \pi_\theta(y_1|x)\pi_\theta(y_2|x)\frac{\partial^2 h(\pi_\theta, x, y_1,y_2)}{\partial \theta(x, y_i) \partial \theta(x, y_j)}.
    \end{aligned}
\end{equation}

Because $|(r_\theta(x,y_1)-r_\theta(x,y_2))-(r(x,y_1)-r(x,y_2))|\leq\epsilon_2$, $|g(\pi_\theta,x,y_1,y_2)|\leq\epsilon_2$.

According to the absolute value inequality, $|\psi(x)|=\left|\sum_{y_i,y_j\in\mathbb{Y}}z(x,y_i) \frac{\partial^2 f_{\text{RDA}}(x,\theta)}{\partial \theta(x, y_i) \partial \theta(x, y_j)}  z(x,y_j)\right|$ can be decomposed into five parts based on Eq.\ref{RDA_decompose} for further analysis.
\begin{equation}\label{RDA_decompose_abs}
    \begin{aligned}
&\left|\sum_{y_i,y_j\in\mathbb{Y}}z(x,y_i) \frac{\partial^2 f_{\text{RDA}}(x,\theta)}{\partial \theta(x, y_i) \partial \theta(x, y_j)}  z(x,y_j)\right| \\
\leq& \left|\sum_{y_i,y_j\in\mathbb{Y}}z(x,y_i)  \sum_{y_1,y_2\in\mathbb{Y}} 2\frac{\partial^2 \pi_\theta(y_1|x)}{\partial \theta(x, y_i) \partial \theta(x, y_j)} \pi_\theta(y_2|x)h(\pi_\theta,x,y_1,y_2)  z(x,y_j)\right| \\
&+\left|\sum_{y_i,y_j\in\mathbb{Y}}z(x,y_i) \sum_{y_1,y_2\in\mathbb{Y}}  2\frac{\partial \pi_\theta(y_1|x)}{\partial \theta(x, y_i)}\frac{\partial \pi_\theta(y_2|x)}{\partial \theta(x, y_j)}h(\pi_\theta,x,y_1,y_2) z(x,y_j)\right| \\
&+\left|\sum_{y_i,y_j\in\mathbb{Y}}z(x,y_i) \sum_{y_1,y_2\in\mathbb{Y}}  2\frac{\partial \pi_\theta(y_1|x)}{\partial \theta(x, y_i)}\pi_\theta(y_2|x) \frac{\partial h(\pi_\theta, x, y_1,y_2)}{\partial \theta(x, y_j)}  z(x,y_j)\right|  \\
&+ \left|\sum_{y_i,y_j\in\mathbb{Y}}z(x,y_i) \sum_{y_1,y_2\in\mathbb{Y}}   2\frac{\partial \pi_\theta(y_1|x)}{\partial \theta(x, y_j)}\pi_\theta(y_2|x) \frac{\partial h(\pi_\theta, x, y_1,y_2)}{\partial \theta(x, y_i)}  z(x,y_j)\right| \\
&+\left|\sum_{y_i,y_j\in\mathbb{Y}}z(x,y_i) \sum_{y_1,y_2\in\mathbb{Y}} \pi_\theta(y_1|x)\pi_\theta(y_2|x)\frac{\partial^2 h(\pi_\theta, x, y_1,y_2)}{\partial \theta(x, y_i) \partial \theta(x, y_j)}  z(x,y_j)\right|.
    \end{aligned}
\end{equation}

For the first and second terms of Eq.\ref{RDA_decompose_abs}, we have:
\begin{equation}\label{RDA_1_2_terms}
\begin{aligned}
    & \left|\sum_{y_i,y_j\in\mathbb{Y}}z(x,y_i)  \sum_{y_1,y_2\in\mathbb{Y}} 2\frac{\partial^2 \pi_\theta(y_1|x)}{\partial \theta(x, y_i) \partial \theta(x, y_j)} \pi_\theta(y_2|x)h(\pi_\theta,x,y_1,y_2)  z(x,y_j)\right| \\
&+ \left|\sum_{y_i,y_j\in\mathbb{Y}}z(x,y_i) \sum_{y_1,y_2\in\mathbb{Y}}  2\frac{\partial \pi_\theta(y_1|x)}{\partial \theta(x, y_i)}\frac{\partial \pi_\theta(y_2|x)}{\partial \theta(x, y_j)}h(\pi_\theta,x,y_1,y_2) z(x,y_j)\right| \\
=&\left|\sum_{y_i,y_j\in\mathbb{Y}}z(x,y_i)  \sum_{y_1,y_2\in\mathbb{Y}}4  \pi_\theta(y_1 | x)\pi_\theta(y_2 | x)(\delta_{y_1y_i} - \pi_\theta(y_i | x))(\delta_{y_2y_j} - \pi_\theta(y_j | x))h(\pi_\theta,x,y_1,y_2)  z(x,y_j)\right| \\
&+\left|\sum_{y_i,y_j\in\mathbb{Y}}z(x,y_i) \sum_{y_1,y_2\in\mathbb{Y}}2\pi_\theta(y_1 | x)\pi_\theta(y_2 | x) \pi_\theta(y_i | x)(\delta_{y_iy_j} - \pi_\theta(y_j | x))h(\pi_\theta,x,y_1,y_2)  z(x,y_j)\right| \\
\leq&\left|\sum_{y_i,y_j\in\mathbb{Y}}z(x,y_i)  \sum_{y_1,y_2\in\mathbb{Y}}4  \pi_\theta(y_1 | x)\pi_\theta(y_2 | x)(\delta_{y_1y_i} - \pi_\theta(y_i | x))(\delta_{y_2y_j} - \pi_\theta(y_j | x))\epsilon_2^2  z(x,y_j)\right| \\
&+\left|\sum_{y_i,y_j\in\mathbb{Y}}z(x,y_i) \sum_{y_1,y_2\in\mathbb{Y}}2\pi_\theta(y_1 | x)\pi_\theta(y_2 | x) \pi_\theta(y_i | x)(\delta_{y_iy_j} - \pi_\theta(y_j | x))\epsilon_2^2  z(x,y_j)\right|.
\end{aligned}
\end{equation}
The first equality is because:
\begin{equation}
    \frac{\partial^2 \pi_\theta(y|x)}{\partial \theta(x, y_i) \partial \theta(x, y_j)} = \pi_\theta(y | x)(\delta_{yy_j} - \pi_\theta(y_j | x))(\delta_{yy_i} - \pi_\theta(y_i | x)) - \pi_\theta(y | x) \pi_\theta(y_i | x)(\delta_{y_iy_j} - \pi_\theta(y_j | x)).
\end{equation}
The second inequality is because:
\begin{equation}
|h(\pi_\theta,x,y_1,y_2)|=|g^2(\pi_\theta,x,y_1,y_2)|\leq\epsilon_2^2.
\end{equation}
Then:
\begin{equation}\label{RDA_1_2_terms_1part}
\begin{aligned}
&\left|\sum_{y_i,y_j\in\mathbb{Y}}z(x,y_i)   \sum_{y_1,y_2\in\mathbb{Y}}4  \pi_\theta(y_1 | x)\pi_\theta(y_2 | x)(\delta_{y_1y_i} - \pi_\theta(y_i | x))(\delta_{y_2y_j} - \pi_\theta(y_j | x))\epsilon_2^2  z(x,y_j)\right| \leq 16\epsilon_2^2 \|z(x,\cdot)\|_2^2.
\end{aligned}
\end{equation}
The inequality is because we can expand the absolute value operation in the second-to-last line into four terms, and these four terms can be deduced to be less than $\|z(x,\cdot)\|_2^2$ using some existing conclusions. The relevant conclusions are:
$ \pi_\theta^\top z(x,\cdot) \leq \|\pi_\theta\|_1 \|z(x,\cdot)\|_\infty = \|z(x,\cdot)\|_\infty\leq\|z(x,\cdot)\|_2 $, $ \pi_\theta^\top z^2(x,\cdot) \leq \|z(x,\cdot)\|_2^2$, $ \|\pi_\theta\|_\infty \leq 1 $, $ \|z^2(x,\cdot)\|_1 = \|z(x,\cdot)\|_2^2 $ and $|(\pi_\theta^\top z(x,\cdot))^2|  \leq \|z(x,\cdot)\|_2^2$.

And similarly, we have:
\begin{equation}\label{RDA_1_2_terms_2part}
    \begin{aligned}
&\left|\sum_{y_i,y_j\in\mathbb{Y}}z(x,y_i) \sum_{y_1,y_2\in\mathbb{Y}}2\pi_\theta(y_1 | x)\pi_\theta(y_2 | x) \pi_\theta(y_i | x)(\delta_{y_iy_j} - \pi_\theta(y_j | x))\epsilon_2^2  z(x,y_j)\right| \\
\leq& 2\left|\sum_{y_i,y_j\in\mathbb{Y}}z(x,y_i) \delta_{y_iy_j}\epsilon_2^2  z(x,y_j)\right| +2\left|\sum_{y_i,y_j\in\mathbb{Y}}z(x,y_i)\pi_\theta(y_i | x)\pi_\theta(y_j | x)\epsilon_2^2  z(x,y_j)\right| \\
\leq& 2\epsilon_2^2+2\left|\sum_{y_i,y_j\in\mathbb{Y}}z(x,y_i)\pi_\theta(y_i | x)\pi_\theta(y_j | x)\epsilon_2^2  z(x,y_j)\right| \\
\leq& 2\epsilon_2^2+2\epsilon_2^2\left|\sum_{y_i\in\mathbb{Y}}z(x,y_i)\pi_\theta(y_i | x)\sum_{y_j\in\mathbb{Y}}\pi_\theta(y_j | x)  z(x,y_j)\right| \\
\leq& 2\epsilon_2^2+2\epsilon_2^2\left|\|\pi(\cdot|x)\|_1\|z(x,\cdot)\|_\infty\|\pi(\cdot|x)\|_1\|z(x,\cdot)\|_\infty\right|
\leq 4\epsilon_2^2\|z(x,\cdot)\|_2^2.
    \end{aligned}
\end{equation}

For the third and forth terms of Eq.\ref{RDA_decompose_abs}, we have:
\begin{equation}\label{RDA_3_4_terms}
\begin{aligned}
    &\left|\sum_{y_i,y_j\in\mathbb{Y}}z(x,y_i) \sum_{y_1,y_2\in\mathbb{Y}}  2\frac{\partial \pi_\theta(y_1|x)}{\partial \theta(x, y_i)}\pi_\theta(y_2|x) \frac{\partial h(\pi_\theta, x, y_1,y_2)}{\partial \theta(x, y_j)}  z(x,y_j)\right|  \\
&+ \left|\sum_{y_i,y_j\in\mathbb{Y}}z(x,y_i) \sum_{y_1,y_2\in\mathbb{Y}}   2\frac{\partial \pi_\theta(y_1|x)}{\partial \theta(x, y_j)}\pi_\theta(y_2|x) \frac{\partial h(\pi_\theta, x, y_1,y_2)}{\partial \theta(x, y_i)}  z(x,y_j)\right| \\
=& 2\left|\sum_{y_i,y_j\in\mathbb{Y}}z(x,y_i) \sum_{y_1,y_2\in\mathbb{Y}}  2\frac{\partial \pi_\theta(y_1|x)}{\partial \theta(x, y_i)}\pi_\theta(y_2|x) \frac{\partial h(\pi_\theta, x, y_1,y_2)}{\partial \theta(x, y_j)}  z(x,y_j)\right|  \\
=& 8\left|\sum_{y_i,y_j\in\mathbb{Y}}z(x,y_i) \sum_{y_1,y_2\in\mathbb{Y}}  \frac{\partial \pi_\theta(y_1|x)}{\partial \theta(x, y_i)}\pi_\theta(y_2|x) g(\pi_\theta, x, y_1,y_2)\frac{\partial g(\pi_\theta,x,y_1,y_2)}{\partial \theta(x, y_j)}  z(x,y_j)\right| \\
=& 8\left|\sum_{y_i,y_j\in\mathbb{Y}}z(x,y_i) \sum_{y_1,y_2\in\mathbb{Y}}  \pi_\theta(y_1|x)(\delta_{y_1y_i}-\pi_\theta(y_i|x))\pi_\theta(y_2|x) g(\pi_\theta, x, y_1,y_2)\frac{1}{\tau}(\delta_{y_1y_j} - \delta_{y_2y_j})  z(x,y_j)\right| \\
\leq&\frac{8\epsilon_2}{\tau}\left|\sum_{y_i,y_j\in\mathbb{Y}}z(x,y_i) \sum_{y_1,y_2\in\mathbb{Y}}  \pi_\theta(y_1|x)(\delta_{y_1y_i}-\pi_\theta(y_i|x))\pi_\theta(y_2|x) \frac{1}{\tau}(\delta_{y_1y_j} - \delta_{y_2y_j})  z(x,y_j)\right| \\
\leq& \frac{32\epsilon_2}{\tau}\|z(x,\cdot)\|_2^2.
\end{aligned}
\end{equation}
where the first derivative of $g(\pi_\theta,x,y_1,y_2)$ is:
\begin{equation}
    \begin{aligned}
\frac{\partial g(\pi_\theta,x,y_1,y_2)}{\partial \theta(x, y_i)} = \frac{\partial \frac{1}{\tau}(\log{\pi_\theta(y_1|x)} -\log{\pi_\theta(y_2|x)})}{\partial \theta(x, y_i)} = \frac{1}{\tau}(\delta_{y_1y_i} - \pi_\theta(y_i|x) - \delta_{y_2y_i} + \pi_\theta(y_i|x)) = \frac{1}{\tau}(\delta_{y_1y_i} - \delta_{y_2y_i}).
    \end{aligned}
\end{equation}

And the last inequality is because we can expand the absolute value operation in the second-to-last line into four terms, and these four terms can be deduced to be less than $\|z(x,\cdot)\|_2^2$ using some existing conclusions. The relevant conclusions are:
$ \pi_\theta^\top z(x,\cdot) \leq \|\pi_\theta\|_1 \|z(x,\cdot)\|_\infty = \|z(x,\cdot)\|_\infty\leq\|z(x,\cdot)\|_2 $, $ \pi_\theta^\top z^2(x,\cdot) \leq \|z(x,\cdot)\|_2^2$, $ \|\pi_\theta\|_\infty \leq 1 $, $ \|z^2(x,\cdot)\|_1 = \|z(x,\cdot)\|_2^2 $ and $|(\pi_\theta^\top z(x,\cdot))^2|  \leq \|z(x,\cdot)\|_2^2$.


For the fifth term of Eq.\ref{RDA_decompose_abs}, we have:
\begin{equation}\label{RDA_5_terms}
    \begin{aligned}
&\left|\sum_{y_i,y_j\in\mathbb{Y}}z(x,y_i) \sum_{y_1,y_2\in\mathbb{Y}} \pi_\theta(y_1|x)\pi_\theta(y_2|x)\frac{\partial^2 h(\pi_\theta, x, y_1,y_2)}{\partial \theta(x, y_i) \partial \theta(x, y_j)}  z(x,y_j)\right| \\
=& \left|\sum_{y_i,y_j\in\mathbb{Y}}z(x,y_i) \sum_{y_1,y_2\in\mathbb{Y}} \pi_\theta(y_1|x)\pi_\theta(y_2|x)2\frac{\partial g(\pi_\theta, x, y_1,y_2)}{\partial \theta(x, y_i)}\frac{\partial g(\pi_\theta, x, y_1,y_2)}{\partial \theta(x, y_j)}  z(x,y_j)\right| \\
=& \left|\sum_{y_i,y_j\in\mathbb{Y}}z(x,y_i) \sum_{y_1,y_2\in\mathbb{Y}} \pi_\theta(y_1|x)\pi_\theta(y_2|x)\frac{2}{\tau^2}(\delta_{y_1y_i} - \delta_{y_2y_i})(\delta_{y_1y_j} - \delta_{y_2y_j})  z(x,y_j)\right|\\
=& |\sum_{y_i\in\mathbb{Y}}z(x,y_i)   \pi_\theta(y_i|x)\frac{2}{\tau^2}z(x,y_i) - \sum_{y_i\in\mathbb{Y}}z(x,y_i)   \pi_\theta(y_i|x)\sum_{y_j\in\mathbb{Y}}\pi_\theta(y_j|x)\frac{2}{\tau^2}z(x,y_j) \\
&- \sum_{y_i\in\mathbb{Y}}z(x,y_i)   \pi_\theta(y_i|x)\sum_{y_j\in\mathbb{Y}}\pi_\theta(y_j|x)\frac{2}{\tau^2}z(x,y_j) + \sum_{y_i\in\mathbb{Y}}z(x,y_i)   \pi_\theta(y_i|x)\frac{2}{\tau^2}z(x,y_i)|,
    \end{aligned}
\end{equation}
$$    \begin{aligned}
\leq& \frac{4}{\tau^2}|\sum_{y_i\in\mathbb{Y}}z(x,y_i)   \pi_\theta(y_i|x)z(x,y_i)| + |\sum_{y_i\in\mathbb{Y}}z(x,y_i)   \pi_\theta(y_i|x)\sum_{y_j\in\mathbb{Y}}\pi_\theta(y_j|x)z(x,y_j)|\\
=& \frac{4}{\tau^2} (\pi^T_\theta(\cdot|x)z^2(x,\cdot) + (\pi^T_\theta(\cdot|x)z(x,\cdot))^2) \leq \frac{4}{\tau^2} (\|\pi_\theta(\cdot|x)\|_1\|z^2(x,\cdot)\|_\infty + (\|\pi_\theta(\cdot|x)\|_1\|z(x,\cdot)\|_\infty)^2) \\
\leq& \frac{8}{\tau^2}\|z(x,\cdot)\|_2^2.
    \end{aligned}$$
where the second derivative of $g(\pi_\theta,x,y_1,y_2)$ is:
\begin{equation}
    \begin{aligned}
\frac{\partial^2 g(\pi_\theta,x,y_1,y_2)}{\partial \theta(x, y_i)\partial \theta(x, y_j)} = \frac{\partial}{\partial \theta(x, y_j)}(\frac{1}{\tau}(\delta_{y_1y_i} - \delta_{y_2y_i}))=0.
    \end{aligned}
\end{equation}
In summary, we have the upper bound of Eq.\ref{RDA_decompose_abs}:
\begin{equation}
    \begin{aligned}
\|\psi(x)\|_\infty=&\max_{x\in\mathbb{X}}|\psi(x)|=\max_{x\in\mathbb{X}}\left|\sum_{y_i,y_j\in\mathbb{Y}}z(x,y_i) \frac{\partial^2 f_{\text{RDA}}(x,\theta)}{\partial \theta(x, y_i) \partial \theta(x, y_j)}  z(x,y_j)\right| \\
\leq& \left(20\epsilon_2^2+\frac{32\epsilon_2}{\tau}+\frac{8}{\tau^2}\right)\max_{x\in\mathbb{X}}\|z(x,\cdot)\|_2^2 \\
\leq& \left(20\epsilon_2^2+\frac{32\epsilon_2}{\tau}+\frac{8}{\tau^2}\right)\|z(\cdot,\cdot)\|_2^2
    \end{aligned}
\end{equation}
Then Eq.\ref{spectral_radius_RDA} is proved. Proof finished.



\subsection{Proof of Lemma \ref{PRA_L_coef}}\label{PRA_L_coef_proof}
\begin{lemma}\label{PRA_L_coef}
    % PRA L smooth coefficient%算法的梯度力普希次系数
    (PRA Smoothness) Given softmax parametrization of Definition \ref{Softmax} for policy $\pi_\theta$, assume $|\theta(x_1,y_1)-\theta(x_2,y_2)|\leq d$ and $|p^* - \sigma(\frac{1}{\tau}\log\frac{\pi_\theta(y_1|x)}{\pi_\theta(y_2|x)}) |\leq \epsilon_4$,  $\forall r, \tau$, $\theta\rightarrow \mathbb{E}_{\mathcal{D}_{\theta}}\left[\mathrm{D}_{\mathrm{KL}}(p^*(z|y_1,y_2,x)||p_\theta(z|y_1,y_2,x))\right]$ is $(20\log(1+ e^{\frac{d}{\tau}})+\frac{16\epsilon_3}{\tau}+\frac{4}{\tau^2}+16\log2)$-smooth where $\mathcal{D}_{\theta} \triangleq \{(x, y_1, y_2, z) \mid x \in \mathcal{D}, y_1, y_2\sim\pi_\theta(\cdot|x), z = 1 \text{ if } r(x, y_1) > r(x, y_2); 0 \text{ otherwise}\}$ (see Eq.\ref{PRA_eq}). See proof in Appendix \ref{PRA_L_coef_proof}.
\end{lemma}

\textbf{Proof:} Based on Appendix \ref{Derivation_PRA}, we have:
\begin{equation}
\begin{aligned}
&\mathbb{E}_{\mathcal{D}_{\theta}}\left[\mathrm{D}_{\mathrm{KL}}(p^*(z|y_1,y_2,x)||\sigma\left( h_\theta(x,y_1,y_2)\right))\right]\\
=&-\mathbb{E}_{x\sim \mathcal{D},y_1,y_2\sim \pi_\theta(y|x)}\left[p^*(z=1|y_1,y_2,x)\log\sigma\left( h_\theta(x,y_1,y_2)\right)+p^*(z=0|y_1,y_2,x)\log\sigma\left( h_\theta(x,y_2,y_1)\right)\right]\\
&+\mathbb{E}_{x\sim \mathcal{D},y_1,y_2\sim \pi_\theta(y|x)}\left[p^*(z=1|y_1,y_2,x)\log p^*(z=1|y_1,y_2,x)+p^*(z=0|y_1,y_2,x)\log p^*(z=0|y_1,y_2,x)\right]\\
=&-\mathbb{E}_{x\sim \mathcal{D},y_1,y_2\sim \pi_\theta(y|x)}\left[p^*(z=1|y_1,y_2,x)\log\sigma\left( h_\theta(x,y_1,y_2)\right)+p^*(z=0|y_1,y_2,x)\log\sigma\left( h_\theta(x,y_2,y_1)\right)\right]\\
&+\mathbb{E}_{x\sim \mathcal{D},y_1,y_2\sim \pi_\theta(y|x)}\left[M(x,y_1,y_2)\right]\\
\triangleq& \mathcal{L}_{\text{PRA}}(\pi_\theta).
\end{aligned}
\end{equation}
where $\sigma(x)=\frac{1}{1+e^{-x}}$ is the sigmoid function, $M(x,y_1,y_2)=\sum_{z=0,1}p^*(z|y_1,y_2,x)\log p^*(z|y_1,y_2,x)$ and $(x,y_1,y_2)\sim \mathcal{D}_{\theta}\triangleq y_1\succ y_2\sim p^*(z=1|y_1,y_2,x),y_1,y_2\sim\pi_\theta(y|x),x\sim\mathcal{D}$. 

By Lemma \ref{spectral_radius}, it suffices to show that the spectral radius of the hessian matrix of the second derivative of $\mathcal{L}_{\text{PRA}}(\pi_\theta)$, i.e.
\begin{equation}\label{spectral_radius_pra}
\begin{aligned}
    &\left|\sum_{x,x'\in\mathbb{X}}\sum_{y_i,y_j\in\mathbb{Y}}z(x,y_i) \frac{\partial^2 \mathcal{L}_{\text{PRA}}(\pi_\theta)}{\partial \theta(x,y_i)\partial\theta(x',y_j)} z(x',y_j)\right|\\
    \leq&\left(20\log(1+ e^{\frac{d}{\tau}})+\frac{16\epsilon_3}{\tau}+\frac{4}{\tau^2} + 16\log2\right)||z(\cdot,\cdot)||_2^2.
\end{aligned}
\end{equation}
Denote $h(\pi_\theta,x,y_1,y_2)=-p^*(z=1|y_1,y_2,x)\log\sigma\left( h_\theta(x,y_1,y_2)\right)-p^*(z=0|y_1,y_2,x)\log\sigma\left( h_\theta(x,y_2,y_1)\right)$, we have:
\begin{equation}\label{pra_decompose_psi_with_M_eq}
\begin{aligned}
&\left|\sum_{x,x'\in\mathbb{X}}\sum_{y_i,y_j\in\mathbb{Y}}z(x,y_i) \frac{\partial^2 \mathcal{L}_{\text{PRA}}(\pi_\theta)}{\partial \theta(x,y_i)\partial\theta(x',y_j)} z(x',y_j)\right|\\
=&\left|\sum_{x\in\mathbb{X}}\sum_{y_i,y_j\in\mathbb{Y}}z(x,y_i) \frac{\partial^2 \sum_{x\in\mathbb{X}}\mathcal{D}(x)\sum_{y_1,y_2\in\mathbb{Y}}\pi_\theta(y_1|x)\pi_\theta(y_2|x)(h(\pi_\theta,x,y_1,y_2)+M(x,y_1,y_2))}{\partial \theta(x,y_i)\partial\theta(x,y_j)}  z(x,y_j)\right|\\ =&\left|\sum_{x\in\mathbb{X}}\mathcal{D}(x)\sum_{y_i,y_j\in\mathbb{Y}}z(x,y_i)  \frac{\partial^2 f_{\text{PRA}}(x,\theta)}{\partial \theta(x, y_i) \partial \theta(x, y_j)}  z(x,y_j)\right| \\
&+ \left|\sum_{x\in\mathbb{X}}\mathcal{D}(x)\sum_{y_i,y_j\in\mathbb{Y}}z(x,y_i)  \frac{\partial^2 \sum_{y_1,y_2\in\mathbb{Y}}\pi_\theta(y_1|x)\pi_\theta(y_2|x)M(x,y_1,y_2)}{\partial \theta(x, y_i) \partial \theta(x, y_j)}  z(x,y_j)\right| \\
\triangleq& |\sum_{x\in\mathbb{X}}\mathcal{D}(x) \psi(x) | + \left|\sum_{x\in\mathbb{X}}\mathcal{D}(x)\sum_{y_i,y_j\in\mathbb{Y}}z(x,y_i)  \frac{\partial^2 \sum_{y_1,y_2\in\mathbb{Y}}\pi_\theta(y_1|x)\pi_\theta(y_2|x)M(x,y_1,y_2)}{\partial \theta(x, y_i) \partial \theta(x, y_j)}  z(x,y_j)\right|  \\
\leq& \|\mathcal{D}(\cdot)\|_1 \|\psi(\cdot)\|_\infty  + \left|\sum_{x\in\mathbb{X}}\mathcal{D}(x)\sum_{y_i,y_j\in\mathbb{Y}}z(x,y_i)  \frac{\partial^2 \sum_{y_1,y_2\in\mathbb{Y}}\pi_\theta(y_1|x)\pi_\theta(y_2|x)M(x,y_1,y_2)}{\partial \theta(x, y_i) \partial \theta(x, y_j)}  z(x,y_j)\right|\\
=& 1\cdot\|\psi(\cdot)\|_\infty  + \left|\sum_{x\in\mathbb{X}}\mathcal{D}(x)\sum_{y_i,y_j\in\mathbb{Y}}z(x,y_i)  \frac{\partial^2 \sum_{y_1,y_2\in\mathbb{Y}}\pi_\theta(y_1|x)\pi_\theta(y_2|x)M(x,y_1,y_2)}{\partial \theta(x, y_i) \partial \theta(x, y_j)}  z(x,y_j)\right|.
\end{aligned}
\end{equation}
where $f_{\text{PRA}}(x,\theta)=\sum_{y_1,y_2\in\mathbb{Y}}\pi_\theta(y_1|x)\pi_\theta(y_2|x)h(\pi_\theta,x,y_1,y_2)$ and $\psi(x)=\sum_{y_i,y_j\in\mathbb{Y}}z(x,y_i)  \frac{\partial^2 f_{\text{PRA}}(x,\theta)}{\partial \theta(x, y_i) \partial \theta(x, y_j)}  z(x,y_j)$.

Because $|\theta(x_1,y_1)-\theta(x_2,y_2)|\leq d$ and the Definition \ref{Softmax} for policy $\pi_\theta$, consider the upper bound of $h(\pi_\theta,x,y_1,y_2)$, we have:
\begin{equation}
\begin{aligned}
    |\log\sigma\left( h_\theta(x,y_1,y_2)\right)| =& |\log\sigma\left( \frac{1}{\tau}\log\frac{\pi_\theta(y_1|x)}{\pi_\theta(y_2|x)}\right)| = |\log\sigma\left( \frac{1}{\tau}\log\frac{\exp(\theta(y_1,x))}{\exp(\theta(y_2,x))}\right)| \\
    =& |\log\sigma\left( \frac{1}{\tau}(\theta(y_1,x)-\theta(y_2,x))\right)|
    \leq -\log\sigma\left( \frac{-d}{\tau}\right)\\
    =& \log \sigma^{-1}\left( \frac{-d}{\tau}\right) = \log(1+ e^{\frac{d}{\tau}}).
\end{aligned}
\end{equation}
Then 
\begin{equation}
\begin{aligned}
    &|h(\pi_\theta,x,y_1,y_2)|\\
    =&|p^*(z=1|y_1,y_2,x)\log\sigma\left( h_\theta(x,y_1,y_2)\right)+p^*(z=0|y_1,y_2,x)\log\sigma\left( h_\theta(x,y_2,y_1)\right)| \\
    \leq& p^*(z=1|y_1,y_2,x)\log(1+ e^{\frac{d}{\tau}}) + p^*(z=0|y_1,y_2,x)\log(1+ e^{\frac{d}{\tau}})=\log(1+ e^{\frac{d}{\tau}}).
\end{aligned}
\end{equation}

Consider the first derivative of  $h(\pi_\theta,x,y_1,y_2)$, denote $p^*=p^*(1|y_1,y_2,x)$:
\begin{equation}
    \begin{aligned}
\frac{\partial h(\pi_\theta,x,y_1,y_2)}{\partial \theta(x, y_i)} =& -p^*\frac{\partial }{\partial \theta(x, y_i) }(\log\sigma(\frac{1}{\tau}\log\frac{\pi_\theta(y_1|x)}{\pi_\theta(y_2|x)})) - (1-p^*)\frac{\partial }{\partial \theta(x, y_i) }(\log\sigma(\frac{1}{\tau}\log\frac{\pi_\theta(y_2|x)}{\pi_\theta(y_1|x)})) \\
=& -\left(p^*(1-\sigma(\frac{1}{\tau}\log\frac{\pi_\theta(y_1|x)}{\pi_\theta(y_2|x)})) - (1-p^*)\sigma(\frac{1}{\tau}\log\frac{\pi_\theta(y_1|x)}{\pi_\theta(y_2|x)}) \right)\frac{1}{\tau}\frac{\partial }{\partial \theta(x, y_i) }(\log\frac{\pi_\theta(y_1|x)}{\pi_\theta(y_2|x)}) \\
=& -\left(p^* - \sigma(\frac{1}{\tau}\log\frac{\pi_\theta(y_1|x)}{\pi_\theta(y_2|x)}) \right) \frac{1}{\tau}(\delta_{y_1y_i}-\delta_{y_2y_i}).
    \end{aligned}
\end{equation}
Consider the second derivative of  $h(\pi_\theta,x,y_1,y_2)$:
\begin{equation}\label{pra_h_pi_theta_2rd_eq}
    \begin{aligned}
\frac{\partial^2 h(\pi_\theta,x,y_1,y_2)}{\partial \theta(x, y_i)\partial \theta(x, y_j)} =& -\frac{\partial }{\partial \theta(x, y_j) }( \sigma(\frac{1}{\tau}\log\frac{\pi_\theta(y_2|x)}{\pi_\theta(y_1|x)}) \frac{1}{\tau}(\delta_{y_1y_i}-\delta_{y_2y_i}) ) \\
=& \frac{1}{\tau^2}(\delta_{y_1y_i}-\delta_{y_2y_i})(\delta_{y_1y_j}-\delta_{y_2y_j})\sigma(\frac{1}{\tau}\log\frac{\pi_\theta(y_2|x)}{\pi_\theta(y_1|x)})\sigma(\frac{1}{\tau}\log\frac{\pi_\theta(y_1|x)}{\pi_\theta(y_2|x)}).
    \end{aligned}
\end{equation}
Based on Lemma \ref{Pair_second_derivative_lemma}, the second derivative of $f_{\text{PRA}}(x,\theta)$ is:
\begin{equation}\label{PRA_decompose}
    \begin{aligned}
        \frac{\partial^2 f_{\text{PRA}}(x, \theta)}{\partial \theta(x, y_i) \partial \theta(x, y_j)} = \sum_{y_1,y_2\in\mathbb{Y}}& 2\frac{\partial^2 \pi_\theta(y_1|x)}{\partial \theta(x, y_i) \partial \theta(x, y_j)} \pi_\theta(y_2|x)h(\pi_\theta,x,y_1,y_2) + 2\frac{\partial \pi_\theta(y_1|x)}{\partial \theta(x, y_i)}\frac{\partial \pi_\theta(y_2|x)}{\partial \theta(x, y_j)}h(\pi_\theta,x,y_1,y_2) \\
        &+ 2\frac{\partial \pi_\theta(y_1|x)}{\partial \theta(x, y_i)}\pi_\theta(y_2|x) \frac{\partial h(\pi_\theta, x, y_1,y_2)}{\partial \theta(x, y_j)}+ 2\frac{\partial \pi_\theta(y_1|x)}{\partial \theta(x, y_j)}\pi_\theta(y_2|x) \frac{\partial h(\pi_\theta, x, y_1,y_2)}{\partial \theta(x, y_i)}\\
        & + \pi_\theta(y_1|x)\pi_\theta(y_2|x)\frac{\partial^2 h(\pi_\theta, x, y_1,y_2)}{\partial \theta(x, y_i) \partial \theta(x, y_j)}.
    \end{aligned}
\end{equation}

According to the absolute value inequality, $|\psi(x)|=\left|\sum_{y_i,y_j\in\mathbb{Y}}z(x,y_i) \frac{\partial^2 f_{\text{PRA}}(x,\theta)}{\partial \theta(x, y_i) \partial \theta(x, y_j)}  z(x,y_j)\right|$ can be decomposed into five parts based on Eq.\ref{PRA_decompose} for further analysis.
\begin{equation}\label{PRA_decompose_abs}
    \begin{aligned}
&\left|\sum_{y_i,y_j\in\mathbb{Y}}z(x,y_i) \frac{\partial^2 f_{\text{PRA}}(x,\theta)}{\partial \theta(x, y_i) \partial \theta(x, y_j)}  z(x,y_j)\right| \\
\leq& \left|\sum_{y_i,y_j\in\mathbb{Y}}z(x,y_i)  \sum_{y_1,y_2\in\mathbb{Y}} 2\frac{\partial^2 \pi_\theta(y_1|x)}{\partial \theta(x, y_i) \partial \theta(x, y_j)} \pi_\theta(y_2|x)h(\pi_\theta,x,y_1,y_2)  z(x,y_j)\right| \\
&+\left|\sum_{y_i,y_j\in\mathbb{Y}}z(x,y_i) \sum_{y_1,y_2\in\mathbb{Y}}  2\frac{\partial \pi_\theta(y_1|x)}{\partial \theta(x, y_i)}\frac{\partial \pi_\theta(y_2|x)}{\partial \theta(x, y_j)}h(\pi_\theta,x,y_1,y_2) z(x,y_j)\right| \\
&+\left|\sum_{y_i,y_j\in\mathbb{Y}}z(x,y_i) \sum_{y_1,y_2\in\mathbb{Y}}  4\frac{\partial \pi_\theta(y_1|x)}{\partial \theta(x, y_i)}\pi_\theta(y_2|x) \frac{\partial h(\pi_\theta, x, y_1,y_2)}{\partial \theta(x, y_j)}  z(x,y_j)\right|  \\
&+\left|\sum_{y_i,y_j\in\mathbb{Y}}z(x,y_i) \sum_{y_1,y_2\in\mathbb{Y}} \pi_\theta(y_1|x)\pi_\theta(y_2|x)\frac{\partial^2 h(\pi_\theta, x, y_1,y_2)}{\partial \theta(x, y_i) \partial \theta(x, y_j)}  z(x,y_j)\right|.
    \end{aligned}
\end{equation}

Similar to Eq.\ref{RDA_1_2_terms} and its following derivation, we have:
\begin{equation} 
\begin{aligned}
    & \left|\sum_{y_i,y_j\in\mathbb{Y}}z(x,y_i)  \sum_{y_1,y_2\in\mathbb{Y}} 2\frac{\partial^2 \pi_\theta(y_1|x)}{\partial \theta(x, y_i) \partial \theta(x, y_j)} \pi_\theta(y_2|x)h(\pi_\theta,x,y_1,y_2)  z(x,y_j)\right| \\
&+ \left|\sum_{y_i,y_j\in\mathbb{Y}}z(x,y_i) \sum_{y_1,y_2\in\mathbb{Y}}  2\frac{\partial \pi_\theta(y_1|x)}{\partial \theta(x, y_i)}\frac{\partial \pi_\theta(y_2|x)}{\partial \theta(x, y_j)}h(\pi_\theta,x,y_1,y_2) z(x,y_j)\right| \\
\leq&\left|\sum_{y_i,y_j\in\mathbb{Y}}z(x,y_i)  \sum_{y_1,y_2\in\mathbb{Y}}4  \pi_\theta(y_1 | x)\pi_\theta(y_2 | x)(\delta_{y_1y_i} - \pi_\theta(y_i | x))(\delta_{y_2y_j} - \pi_\theta(y_j | x))\log(1+ e^{\frac{d}{\tau}}) z(x,y_j)\right| \\
&+\left|\sum_{y_i,y_j\in\mathbb{Y}}z(x,y_i) \sum_{y_1,y_2\in\mathbb{Y}}2\pi_\theta(y_1 | x)\pi_\theta(y_2 | x) \pi_\theta(y_i | x)(\delta_{y_iy_j} - \pi_\theta(y_j | x))\log(1+ e^{\frac{d}{\tau}})  z(x,y_j)\right| \\
\leq& 20\log(1+ e^{\frac{d}{\tau}}).
\end{aligned}
\end{equation}



Similar to Eq.\ref{RDA_3_4_terms}, for the third term of Eq.\ref{PRA_decompose_abs}, we have:
\begin{equation}
\begin{aligned}
& \left|\sum_{y_i,y_j\in\mathbb{Y}}z(x,y_i) \sum_{y_1,y_2\in\mathbb{Y}}  4\frac{\partial \pi_\theta(y_1|x)}{\partial \theta(x, y_i)}\pi_\theta(y_2|x) \frac{\partial h(\pi_\theta, x, y_1,y_2)}{\partial \theta(x, y_j)}  z(x,y_j)\right|  \\
=& 4\left|\sum_{y_i,y_j\in\mathbb{Y}}z(x,y_i) \sum_{y_1,y_2\in\mathbb{Y}}  \frac{\partial \pi_\theta(y_1|x)}{\partial \theta(x, y_i)}\pi_\theta(y_2|x) \left(p^* - \sigma(\frac{1}{\tau}\log\frac{\pi_\theta(y_1|x)}{\pi_\theta(y_2|x)}) \right) \frac{1}{\tau}(\delta_{y_1y_j}-\delta_{y_2y_j})  z(x,y_j)\right| \\
\leq& 4\left|\sum_{y_i,y_j\in\mathbb{Y}}z(x,y_i) \sum_{y_1,y_2\in\mathbb{Y}}  \frac{\partial \pi_\theta(y_1|x)}{\partial \theta(x, y_i)}\pi_\theta(y_2|x) \epsilon_3    \frac{1}{\tau}(\delta_{y_1y_j}-\delta_{y_2y_j})  z(x,y_j)\right| \\
=& \frac{4\epsilon_3}{\tau} \left|\sum_{y_i,y_j\in\mathbb{Y}}z(x,y_i) \sum_{y_1,y_2\in\mathbb{Y}}  \pi_\theta(y_1|x)(\delta_{y_1y_i}-\pi_\theta(y_i|x)) \pi_\theta(y_2|x) \epsilon_3    \frac{1}{\tau}(\delta_{y_1y_j}-\delta_{y_2y_j})  z(x,y_j)\right| \\
\leq&  \frac{16\epsilon_3}{\tau}\|z(x,\cdot)\|_2^2.
\end{aligned}
\end{equation}
The first inequality is because $|p^* - \sigma(\frac{1}{\tau}\log\frac{\pi_\theta(y_1|x)}{\pi_\theta(y_2|x)}) |\leq \epsilon_3$ where $p^*=p^*(1|x,y_1,y_2)$. The last inequality is because we can expand the absolute value operation in the second-to-last line into four terms, and these four terms can be deduced to be less than $\|z(x,\cdot)\|_2^2$ using some existing conclusions. The relevant conclusions are:
$ \pi_\theta^\top z(x,\cdot) \leq \|\pi_\theta\|_1 \|z(x,\cdot)\|_\infty = \|z(x,\cdot)\|_\infty\leq\|z(x,\cdot)\|_2 $, $ \pi_\theta^\top z^2(x,\cdot) \leq \|z(x,\cdot)\|_2^2$, $ \|\pi_\theta\|_\infty \leq 1 $, $ \|z^2(x,\cdot)\|_1 = \|z(x,\cdot)\|_2^2 $ and $|(\pi_\theta^\top z(x,\cdot))^2|  \leq \|z(x,\cdot)\|_2^2$.

Similar to Eq.\ref{RDA_5_terms}, for the fifth term of Eq.\ref{PRA_decompose_abs}, based on Eq.\ref{pra_h_pi_theta_2rd_eq} and Eq.\ref{DPO_4term_decompose}, we have:
\begin{equation}
        \begin{aligned}
&\left|\sum_{y_i,y_j\in\mathbb{Y}}z(x,y_i) \sum_{y_1,y_2\in\mathbb{Y}} \pi_\theta(y_1|x)\pi_\theta(y_2|x)\frac{\partial^2 h(\pi_\theta, x, y_1,y_2)}{\partial \theta(x, y_i) \partial \theta(x, y_j)}  z(x,y_j)\right| \\
=& \left|\sum_{y_i,y_j\in\mathbb{Y}}z(x,y_i) \sum_{y_1,y_2\in\mathbb{Y}} \pi_\theta(y_1|x)\pi_\theta(y_2|x)  (\delta_{y_1y_i}-\delta_{y_2y_i})(\delta_{y_1y_j}-\delta_{y_2y_j})(\frac{1}{\tau^2}\sigma(\frac{1}{\tau}\log\frac{\pi_\theta(y_2|x)}{\pi_\theta(y_1|x)})\sigma(\frac{1}{\tau}\log\frac{\pi_\theta(y_1|x)}{\pi_\theta(y_2|x)}))
    z(x,y_j)\right| \\
\leq& \left|\sum_{y_i,y_j\in\mathbb{Y}}z(x,y_i) \sum_{y_1,y_2\in\mathbb{Y}} \pi_\theta(y_1|x)\pi_\theta(y_2|x)(\delta_{y_1y_i}-\delta_{y_2y_i})(\delta_{y_1y_j}-\delta_{y_2y_j})\frac{1}{\tau^2}  z(x,y_j)\right| \\
\leq& \frac{4}{\tau^2}\|z(x,\cdot)\|_2^2.
    \end{aligned}
\end{equation}
In summary, we have the upper bound of Eq.\ref{PRA_decompose_abs}:
\begin{equation}
    \begin{aligned}
\|\psi(x)\|_\infty=&\max_{x\in\mathbb{X}}|\psi(x)|=\max_{x\in\mathbb{X}}\left|\sum_{y_i,y_j\in\mathbb{Y}}z(x,y_i) \frac{\partial^2 f_{\text{PRA}}(x,\theta)}{\partial \theta(x, y_i) \partial \theta(x, y_j)}  z(x,y_j)\right| \\
\leq& \left(20\log(1+ e^{\frac{d}{\tau}})+\frac{16\epsilon_3}{\tau}+\frac{4}{\tau^2}\right)\|z(\cdot,\cdot)\|_2^2
    \end{aligned}
\end{equation}
As for the second term $\left|\sum_{x\in\mathbb{X}}\mathcal{D}(x)\sum_{y_i,y_j\in\mathbb{Y}}z(x,y_i)  \frac{\partial^2 \sum_{y_1,y_2\in\mathbb{Y}}\pi_\theta(y_1|x)\pi_\theta(y_2|x)M(x,y_1,y_2)}{\partial \theta(x, y_i) \partial \theta(x, y_j)}  z(x,y_j)\right|$ of Eq.\ref{pra_decompose_psi_with_M_eq}, we have:
\begin{equation}
    \begin{aligned}
&|\frac{\partial^2 \sum_{y_1,y_2\in\mathbb{Y}}\pi_\theta(y_1|x)\pi_\theta(y_2|x)M(x,y_1,y_2)}{\partial \theta(x, y_i) \partial \theta(x, y_j)}| \\
=&| \frac{\partial}{\partial \theta(x, y_j) }(\frac{\partial \sum_{y_1,y_2\in\mathbb{Y}}\pi_\theta(y_1|x)\pi_\theta(y_2|x)M(x,y_1,y_2)}{\partial \theta(x, y_i) })|\\
=& |\frac{\partial}{\partial \theta(x, y_j) }\left(\sum_{y_1,y_2\in\mathbb{Y}}\pi_\theta(y_1|x)\pi_\theta(y_2|x)(\delta_{y_1y_i}+\delta_{y_2y_i} -2\pi_\theta(y_i|x))M(x,y_1,y_2)\right) |\\
=& |\sum_{y_1,y_2\in\mathbb{Y}}\pi_\theta(y_1|x)\pi_\theta(y_2|x)(\delta_{y_1y_i}+\delta_{y_2y_i} -2\pi_\theta(y_i|x))(\delta_{y_1y_j}+\delta_{y_2y_j} -2\pi_\theta(y_j|x))M(x,y_1,y_2).
    \end{aligned}
\end{equation}
Based on Lemma \ref{SelfEntropy}. We have:
\begin{equation}
    \begin{aligned}
&\left|\sum_{x\in\mathbb{X}}\mathcal{D}(x)\sum_{y_i,y_j\in\mathbb{Y}}z(x,y_i)  \frac{\partial^2 \sum_{y_1,y_2\in\mathbb{Y}}\pi_\theta(y_1|x)\pi_\theta(y_2|x)M(x,y_1,y_2)}{\partial \theta(x, y_i) \partial \theta(x, y_j)}  z(x,y_j)\right| 
\leq 16\log2 \cdot\|z(\cdot,\cdot)\|_2^2
    \end{aligned}
\end{equation}
This inequality is because we can expand the absolute value operation in the second-to-last line into nine terms, and these nine terms can be deduced to be less than $\|z(x,\cdot)\|_2^2$ or $2\|z(x,\cdot)\|_2^2$ using some existing conclusions. The relevant conclusions are:
$ \pi_\theta^\top z(x,\cdot) \leq \|\pi_\theta\|_1 \|z(x,\cdot)\|_\infty = \|z(x,\cdot)\|_\infty\leq\|z(x,\cdot)\|_2 $, $ \pi_\theta^\top z^2(x,\cdot) \leq \|z(x,\cdot)\|_2^2$, $ \|\pi_\theta\|_\infty \leq 1 $, $ \|z^2(x,\cdot)\|_1 = \|z(x,\cdot)\|_2^2 $ and $|(\pi_\theta^\top z(x,\cdot))^2|  \leq \|z(x,\cdot)\|_2^2$.


Then Eq.\ref{spectral_radius_pra} is proved. Proof finished.












\end{document}
