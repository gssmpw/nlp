We now provide a formal definition of obviously dominant strategies. 
and note that our definitions and setup closely follows \cite{Ron24}.  Fixing a protocol and behavior $B_i$ of bidder $i$ we say that vertex $u$ of the protocol is \emph{attainable} if there exists some $B_{-i}\in \mathcal{B}_{-i}$ (i.e., some profile of behaviors for the other players) such that $u\in Path(B_i,B_{-i})$. 
% For example, in the mechanism depicted in Figure \ref{fig-prems}, vertex $N_2$ is attainable given the behavior $(N_1:"1")$ of the sunglasses-wearing duck, whereas vertex $N_3$ is not attainable given this behavior. 
% For example,  given the mechanism described in Figure \ref{fig-prems}, the behavior profiles $B=\{B_s=(N_1:"2"),B_j=(N_2:"2",N_3:"2")\}$ and $B'=\{B_s'=(N_1:"2"),B_j'=(N_2:"1",N_3:"1")\}$ satisfy that  $Path(B)\cap Path(B')=\{N_1,N_3\}$. 
We now may formally define an \emph{obviously dominant behavior}:
\begin{definition}[Definition 2.1 in \cite{Ron24}]\label{def-obvs-behavior}
Consider a deterministic mechanism $A$, together with a behavior $B_i$ and a valuation $v_i$ of some player $i$.
Fix a vertex $u\in \mathcal N_i$ that is attainable given the behavior $B_i$. 
Behavior $B_i$ is an \emph{obviously dominant behavior for player $i$ at vertex $u$ given the valuation $v_i$} if for every behavior profiles $B_{-i}\in \mathcal B_{-i}$ and  $(B_1',\ldots,B_n')\in \mathcal B_1 \times \cdots \times \mathcal{B}_n$ such that:
\begin{enumerate}
    \item $u\in Path(B_1,\ldots,B_n)\cap Path(B_1',\ldots,B_n')$ \emph{and} 
    \item $B_i$ and $B_i'$ dictate sending different messages at vertex $u$.
\end{enumerate}
it holds that:
$$
v_i(f_i(B_i,B_{-i})) - p_i(B_i,B_{-i}) \geq v_i(f_i(B_i',B_{-i}')) - p_i(B_i',B_{-i}')
$$
\end{definition}

Note that \cref{def-obvs-behavior} only deals with behaviors at individual nodes in the protocol.  We then say that a behavior $B_i$ is an obviously dominant behavior given valuation $v_i$ if it is an obviously dominant behavior for player $i$ at
all  attainable vertices.  Formally:
\begin{definition}[Definition 2.2 in \cite{Ron24}]
Fix a behavior $B_i$ together with the subset of vertices in $\mathcal N_i$ that are attainable for it,  which we denote with $U_{B_i}$. Fix a valuation $v_i$ of player $i$.
The behavior $B_i$ is an \emph{obviously dominant behavior for player $i$ given the valuation $v_i$} if
it is an obviously dominant behavior for player $i$ given the valuation $v_i$ for every vertex $u\in U_{B_i}$. 
\end{definition} 

Finally, the definition of \emph{obviously dominant strategies} naturally follows.  Particularly, an obviously dominant strategy is one that defines an obviously dominant behavior for each possible valuation.  Formally:
\begin{definition}[Definition 2.3 in \cite{Ron24}]
A strategy $\mathcal{S}_i$ of player $i$ is an \emph{obviously dominant strategy} if for every $v_i$, the behavior $\mathcal S_i(v_i)$ is an obviously dominant behavior.
% for player $i$ given the valuation $v_i$. 
\end{definition}
 

\paragraph{Weak Monotonicity}
Weak monotonicity is a property of social choice functions.  To define it, we denote with $f_i$ the function that outputs for every player $i$ the bundle that she wins given $f$.

    An allocation rule $f:V\to\allocs$ is \emph{weakly monotone} if for every player $i$, for every valuation profile $v_{-i}$ of the bidders $N\setminus \{i\}$, and every two valuations $v_i,v_i'\in V_i$, it holds that if $f_i(v_i,v_{-i})=S$ and $f_i(v_i',v_{-i})=S'$, then $v_i(S)-v_i(S')\ge v_i'(S)-v_i'(S')$. 
 % The following well-known property of dominant-strategy mechanisms will be useful to us: 
    It is well known that:
\begin{lemma}\cite{BCLMNS06,LMN03}\label{wmon-lemma}
    Every allocation rule that is implemented by a dominant-strategy mechanism is weakly monotone. 
\end{lemma}

% \subsection{Classes of Valuations}\label{classes-of-valuations}
% \src{TODO}

