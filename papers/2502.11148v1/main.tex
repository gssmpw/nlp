\documentclass[11pt]{article}
\usepackage[T1]{fontenc}
\usepackage{csquotes}
\usepackage{tgheros}

\usepackage{comment}

\newcommand{\ESymbol}{\mathds{E}}



\usepackage{amsmath,amsfonts,amsthm,mathtools,color}
\usepackage{mathtools,color}
\usepackage{fullpage}
\usepackage{thmtools,thm-restate}
\usepackage[algo2e,linesnumbered,noend,ruled,noline]{algorithm2e} 
\usepackage{setspace}
\usepackage{tikz}
\renewcommand{\algorithmcfname}{MECHANISM}
\SetAlFnt{\small}
\SetAlCapFnt{\small}
\SetAlCapNameFnt{\small}
\SetAlCapHSkip{0pt}
\IncMargin{-\parindent}
\DeclarePairedDelimiter{\floor}{\lfloor}{\rfloor}
\usepackage{graphicx}
\usepackage{amsthm}
\usepackage{nicefrac}
\usepackage{hyperref}
\usepackage{cleveref}
\crefname{theorem}{Theorem}{Theorems}
\crefname{lemma}{Lemma}{Lemmata}
\crefname{claim}{Claim}{Claims}
\crefname{definition}{Definition}{Definitions}
\crefname{appendix}{Appendix}{Appendices}
\crefname{example}{Example}{Examples}
\crefname{equation}{Equation}{Equations}
\crefname{section}{Section}{Sections}
\crefname{subsection}{Subsection}{Subsections}

\usepackage[numbers]{natbib}
\setcitestyle{acmnumeric}
\DontPrintSemicolon
\newcommand\mycommfont{\footnotesize}
\SetCommentSty{mycommfont}
\SetArgSty{textnormal}
\usepackage{soul}
\usepackage[title]{appendix}
\usepackage{apxproof}
\usepackage{cancel}
\usepackage{authblk}
\usepackage[normalem]{ulem}
\usepackage{xcolor}



\newtheorem{theorem}{Theorem}
\newtheorem{example}{Example}
\newtheorem{lemma}[theorem]{Lemma}
\newtheorem*{lemma*}{Lemma}
\newtheorem*{remark}{Remark}
\newtheorem*{theorem*}{Theorem}
\newtheorem{corollary}{Corollary}
\newtheorem{claim}[theorem]{Claim}
\newtheorem{definition}[theorem]{Definition}
%%%%% A PACKAGE MODIFIWED BY SHIRI -- CAN BE REMOVED IF CAUSES TROUBLE:
\usepackage[inline]{enumitem}
% \usepackage{enumitem}

\usepackage{authblk}
\newcommand{\E}{\mbox{\bf E}}

\DeclareFontShape{T1}{cmr}{m}{scit}{<->ssub*cmr/m/sc}{}

%%%%%TODO - remove before submission %%%%%%%%
% \definecolor{turquoise}{rgb}{0.19, 0.84, 0.78}
% \usepackage[colorinlistoftodos,textsize=tiny,textwidth=2cm,color=red!25!white,obeyFinal]{todonotes}
% \newcommand{\note}[1]{\todo[color=green!25!white]{ {#1} }}
% \newcommand{\dnote}[1]{{\color{blue} (Dan: #1)}}
% \newcommand{\src}[1]{{\color{magenta} (Shiri: #1)}}
% \newcommand{\shirinote}[1]{\todo[color=pink!100!white]{ {#1} }}
% \newcommand{\sre}[1]{{\color{magenta} #1}} % edit
% \newcommand{\toedit}[1]{{\color{violet} #1}}
% \newcommand{\missing}[1]{{\color{turquoise!90} \textbf{#1}}} %parts to fill in 
% \newcommand{\cout}[1]{}
% \usepackage[normalem]{ulem}
% \newcommand\pinksout{\bgroup\markoverwith{\textcolor{magenta}{\rule[0.5ex]{2pt}{0.4pt}}}\ULon}





\title{On the Power of Randomization for Obviously Strategy-Proof Mechanisms}

% \author{
%   Shiri Ron\thanks{Weizmann Institute of Science. Email: \email{shiriron@weizmann.ac.il}.}
%   \and Clayton Thomas\thanks{Microsoft Research. E-mail: \email{clathomas@microsoft.com}.}}

\author[1]{Shiri Ron}
\author[2]{Daniel Schoepflin}

\affil[1]{Weizmann Institute of Science}
\affil[2]{Rutgers University}

% \author{
%   Shiri Ron\thanks{Weizmann Institute of Science} \\
%   \and
%   Daniel Schoepflin\thanks{Rutgers University}
% }

\begin{document}
\titlepage
\maketitle
\begin{abstract}
\begin{abstract}
Retrieval-Augmented Generation (RAG) is often used with Large Language Models (LLMs) to infuse domain knowledge or user-specific information. In RAG, given a user query, a retriever extracts chunks of relevant text from a knowledge base. These chunks are sent to an LLM as part of the input prompt. Typically, any given chunk is repeatedly retrieved across user questions. However, currently, for every question, attention-layers in LLMs fully compute the key values (KVs) repeatedly for the input chunks, as state-of-the-art methods cannot reuse KV-caches when chunks appear at arbitrary locations with arbitrary contexts. Naive reuse leads to output quality degradation.  This leads to potentially redundant computations on expensive GPUs and increases latency. In this work, we propose \sys, a system for managing and reusing precomputed KVs corresponding to the text chunks (we call \textit{chunk-caches}) in RAG-based systems. We present how to identify \hl{\textit{chunk-caches} that are reusable}, how to efficiently perform a small fraction of recomputation to \textit{fix} the cache to maintain output quality, and how to efficiently store and evict \textit{chunk-caches} in the hardware for maximizing reuse while masking any overheads. With real production workloads as well as synthetic datasets, we show that \sys reduces redundant computation by \textbf{51\%} over SOTA prefix-caching and \textbf{75\%} over full recomputation.
\hl{Additionally, with continuous batching on a real production workload, we get a \textbf{1.6$\times$} speedup in throughput and a \textbf{2$\times$} reduction in end-to-end response latency over prefix-caching while maintaining quality, for both the \llama-3-8B and \llama-3-70B models. 
}
\end{abstract}





\end{abstract}
\clearpage
\newcommand{\p}{\textsc{p}}

\newcommand{\data}{\mathcal{D}}
\newcommand{\extdata}{\mathcal{D}_e}
\newcommand{\shots}{S}
\newcommand{\classes}{C}
\newcommand{\numbershadowmodels}{M}

\newcommand{\hpofunction}{$\texttt{HPO}$}
\newcommand{\trainfunction}{$\textsc{train}$}

\newcommand{\hyperparamsdata}{\eta_{{}_{\data}}}
\newcommand{\hyperparamsextdata}{\eta_{{}_{\extdata}}}

\newcommand{\model}{\mathcal{M}} 
\newcommand{\modeltar}{\mathcal{M}_{\mathcal{T}}} 
\newcommand{\modelshadow}{\mathcal{M}_{\mathcal{S}}} 
\newcommand{\return}{\textbf{return }}
\newcommand{\grid}{\mathbb{M}}

\newcommand{\tpr}{\textsc{tpr}}
\newcommand{\fpr}{\textsc{fpr}}


\newcommand{\lira}{\mathrm{LiRA}} 

\newcommand{\datashadow}{\mathcal{D}_{\mathrm{shadow}}}
\newcommand{\architecture}{\mathcal{A}}
\newcommand{\prob}{\mathbb{P}}
\newcommand{\normal}{\mathcal{N}}
\newcommand{\attack}{\texttt{KNOWN}}
\newcommand{\bb}{\texttt{BLACK-BOX}}
\newcommand{\logits}{\textsc{logits}}

% \section*{Possible Titles}
% \src{On a second thought, I think that this title is a bit too similar to the title "On the power of randomization in algorithmic mechanism design". Another suggestion is "On the Power of Sampling for Simplicity in Mechanism Design". Also -- maybe using "Harnessing Randomization/Sampling" to make our title slightly more original?
% }
% \dnote{I like the sound of ``sample'' and ``simple'' together, but I wonder if using that in the title might suggest that we are doing a sample complexity/prophet inequality with samples paper... maybe we use ``samples'' as a section/paragraph title and use randomization for the paper title?}
% \shirinote{I wrote some title for now, please let me know what you think of it :)}
% \begin{itemize}  
% \item On The Power of Randomized Obviously Strategy-Proof Mechanisms
%     \item On The Power and Limitations of the Sampling in Algorithmic Mechanism Design
%     \item (Im)possibilities for randomized obviously strategyproof mechanisms 
%     \item The Unreasonable Effectiveness of Randomization for Simplicity in Mechanism Design
    
% \end{itemize}
\section{Introduction}
\section{Introduction}
\label{sec:intro}

\begin{figure*}[tb]
    \centering
    \includegraphics[width=0.848\linewidth]{figs/circuitnn.pdf} 
    \caption{Illustration of differentiable CircuitNN. CircuitNN is designed based on differentiable NAND gates. After DAS is guided by PI and PO pairs of the truth table, CircuitNN can get the precise circuit architecture logic equivalent to the truth table.}
    \label{fig:circuitnn}
\end{figure*}

% 1. Describe the importance of logic synthesis
% 2. Existing Problems
% (a) Neural Architecture Search: Unstable, Predefined Setting, etc.
% (b) Circuit Generation: Probabilistic Model, Logic Equivalence

With the rapid advancement of technology, the scale of integrated circuits (ICs) has expanded exponentially. 
This expansion has introduced significant challenges in chip manufacturing, particularly concerning power and area metrics.
A primary objective in IC design is achieving the same circuit function with fewer transistors, thereby reducing power usage and area occupancy.

Logic synthesis~\cite{hachtel2005logicsynth}, a critical step in electronic design automation (EDA), transforms behavioral-level circuit designs into optimized gate-level circuits, ultimately yielding the final IC layout. 
The primary goal of logic synthesis is to identify the physical implementation with the fewest gates for a given circuit function. 
This task constitutes a challenging NP-hard combinatorial optimization problem. 
Current logic synthesis tools~\cite{brayton2010abc, wolf2013yosys} rely on human-designed heuristics, often leading to sub-optimal outcomes.

Differentiable architecture search (DAS) techniques~\cite{liu2018darts, chu2020darts} offer novel perspectives on addressing challenges in this problem.
Circuit functions can be represented through truth tables, which map binary inputs to their corresponding outputs. 
Truth tables provide a precise representation of input-output relationships, ensuring the design of functionally equivalent circuits.
Inspired by this, researchers~\cite{deepmind2024ai4sys, wang2024tnet} have begun exploring the application of DAS to synthesize circuits directly from truth tables.
Specifically, \citet{deepmind2024ai4sys} proposed CircuitNN, a framework that learns differentiable connection structures with logic gates, enabling the automatic generation of logic circuits from truth tables.
This approach significantly reduces the complexity of traditional circuit generation. 
Building on this, \citet{wang2024tnet} introduced T-Net, a triangle-shaped variant of CircuitNN, incorporating regularization techniques to enhance the efficiency of DAS.

Despite these advancements, several challenges remain. 
The computational complexity of DAS grows quadratically with the number of gates, posing scalability issues.
Although triangle-shaped architecture~\cite{wang2024tnet} partially mitigates this problem, redundancy persists. 
%Additionally, DAS is susceptible to converging to local optima, limiting the ability to search architectures that satisfy the given truth tables~\cite{liu2018darts}. 
%Furthermore, hyperparameters (network depth and layer width) require extensive searches, introducing complexity and prolonging the synthesis process. 
Additionally, DAS is susceptible to converging to local optima~\cite{liu2018darts} and hyperparameters (network depth and layer width) require extensive searches. 
The challenges arise from the vast search space in DAS. 
% Even with predefined settings for CircuitNN, finding a configuration that meets the truth table requires extensive trial and error during the DAS process. 
Intuitively, limiting the search space through predefined parameters (network depth, gates per layer, and connection probabilities) can significantly reduce the complexity.

Recent advances~\cite{openai2023gpt4, abramson2024alphafold3, esser2024sd3, li2024mar} in conditional generative models have demonstrated remarkable performance across language, vision, and graph generation tasks. 
Motivated by these developments, we propose a novel approach to circuit generation that generates preliminary circuit structures to guide DAS in generating refined circuits matching specified truth tables. 
Firstly, we introduce CircuitVQ, a tokenizer with a discrete codebook for circuit tokenization. 
Built upon our Circuit AutoEncoder framework~\cite{hou2022graphmae,li2023maskgae,wu2025mgvga}, CircuitVQ is trained through a circuit reconstruction task. 
Specifically, the CircuitVQ encoder encodes input circuits into discrete tokens using a learnable codebook, while the decoder reconstructs the circuit adjacency matrix based on these tokens.
Subsequently, the CircuitVQ encoder serves as a circuit tokenizer for CircuitAR pretraining, which employs a masked autoregressive modeling paradigm~\cite{chang2022maskgit, li2023mage}. 
In this process, the discrete codes function as supervision signals. 
After training, CircuitAR can generate discrete tokens progressively, which can be decoded into initial circuit structures by the decoder of the CircuitVQ. 
These prior insights can guide DAS in producing refined circuits that match the target truth tables precisely.

Our key contributions can be summarized as follows:
\begin{itemize}
\item We introduce CircuitVQ, a circuit tokenizer that facilitates graph autoregressive modeling for circuit generation, based on our Circuit AutoEncoder framework;
\item Develop CircuitAR, a model trained using masked autoregressive modeling, which generates initial circuit structures conditioned on given truth tables;
\item Propose a refinement framework that integrates differentiable architecture search to produce functionally equivalent circuits guided by target truth tables;
\item Comprehensive experiments demonstrating the scalability and capability emergence of our CircuitAR and the superior performance of the proposed circuit generation approach.
\end{itemize}

% Motivation
% (a) Diffusion (Vision, Graph), Autoregressive (Language, Vision)
% (b) Circuit Generation for Predefined Setting
% (c) Neural Architecture Search for Strict Logic Equivalence

% Contribution
% (a) Circuit Tokenizer (new transformer arch, training strategy)
% (b) CircuitAR (train and gen strategies, post-ar strategy)
% (c) Extensive Evaluation including BitD (Bit Distance) for Scalability


% \subsection{Technical Overview}
% \src{I currently write here information about the technical sections that is relevant to all lower bounds. Not sure that it should be in the introduction, or that it should be a subsection, I just throw it here for now :)}

% \toedit{Note that when proving lower bounds for deterministic mechanisms, the set of valuations being analyzed can be arbitrarily large. To overrule the existence of a deterministic mechanism with approximation $\alpha$,
% we can come up with a set of instances which is arbitrarily large, 
% as long as  we show that no deterministic obviously strategy-proof mechanism manages to give $\alpha$ approximation on all the instances.

% However, when proving lower bounds for randomized mechanisms, we face a new challenge. Now, as our proofs show, the larger is the number of instances that we analyze, the weaker are our lower bounds. We do not know whether this obstacle is a limitation of our approach, or it is unavoidable.}  

\subsection{Open Questions}
This work identifies signal collapse as a critical bottleneck in one-shot neural network pruning. Performance loss in pruned networks is due to \textbf{signal collapse} in addition to the removal of critical parameters. We propose \textbf{REFLOW} (\textbf{Re}storing \textbf{F}low of \textbf{Low}-variance signals), a simple yet effective method that mitigates signal collapse without computationally expensive weight updates. By focusing on signal preservation, REFLOW highlights the importance of mitigating signal collapse in sparse networks and enables magnitude pruning to match or surpass state-of-the-art one-shot pruning methods such as CHITA, CBS, and WF.

REFLOW consistently achieves state-of-the-art accuracy across diverse architectures, restoring ResNeXt-101 from under 4.1\% to 78.9\% top-1 accuracy at 80\% sparsity on ImageNet. Its lightweight design makes it a practical solution for both research and deployment, delivering high-quality sparse models without the overhead of traditional approaches. These findings challenge the traditional emphasis on weight selection strategies and underscore the critical role of signal propagation for achieving high-quality sparse networks in the context of one-shot pruning.




\section{Preliminaries and Useful Observations}\label{sec-prelims}

\section{Preliminaries}\label{sec:preliminaries}



%We denote by $(\Ac(x_\Ac),\Bc(x_\Bc))(z)$ a random execution of $\pi$ with private inputs $(x_\Ac,y_\Ac)$, and common input $z$.

%\Jnote{Move to DP}
% At the end of such an execution, the protocol outputs a public transcript denoted by the random variable $\trans_\pi(x_\Ac,x_\Ac,z)$ we denotes the common as $\out(\trans_\pi(x_\Ac,x_\Ac,z)$, and each party $\Pc \in \set{\Ac,\Bc}$ obtains his view denoted $\view^\Pc_\pi(x_\Ac,x_\Bc,z)$, which may also contain a ``local output'' \Jnote{Local} $\out^\Pc(x_\Ac,x_\Bc,z)$ (if the protocol specifies such an output). \Jnote{Common output, and parties output}


\subsection{Distributions and Random Variables}\label{sec:prelim:dist}
The support of a distribution $P$ over a finite set $\cS$ is defined by $\Supp(P) \eqdef \set{x\in \cS: P(x)>0}$. For a distribution or a random variable $D$, let $d\from D$ denote that $d$ was sampled according to $D$. Similarly,  for a set $\cS$, let $x \from \cS$ denote that $x$ is drawn uniformly from $\cS$, and denote by $\cU_{\cS}$ the uniform distribution over $\cS$. For a finite set $\cX$ and a distribution $C_X$ over $\cX$, we use the capital letter $X$ to denote the random variable that takes values in $\cX$ and is sampled according to $C_X$. The {\sf statistical distance} (\aka {\sf~variation distance}) of two distributions $P$ and $Q$ over a discrete domain $\cX$ is defined by $\sdist{P}{Q} \eqdef \max_{\cS\subseteq \cX} \size{P(\cS)-Q(\cS)} = \frac{1}{2} \sum_{x \in \cS}\size{P(x)-Q(x)}$. 
For a vector $x = (x_1,\ldots,x_n)$ and index $i\in [n]$, we let $x_{-i} = (x_1,\ldots,x_{i-1},x_{i+1},\ldots,x_n)$ and $x^{(i)} = (x_1,\ldots,x_{i-1}, -x_i, x_{i+1},\ldots,x_n)$, for a set $\cS \subseteq [n]$ we let $x_{\cS} = (x_i)_{i \in \cS}$ and $x_{-\cS} = (x_i)_{i \in [n]\setminus \cS}$, and for a vector $r \in \zo^n$ we let $x_r = (x_i)_{\set{i \colon r_i = 1}}$ and $x_{-r} = (x_i)_{\set{i \colon r_i = 0}}$.

%For $n \in \N$ we let $U_n$ be the uniform distribution over $\oo^n$, and let $S_n$ be the distribution induces by the sum of $n$ i.i.d.\ random variables, each is distributed according to $U_1$. Let $\cN(0,1)$ be the standard normal distribution.
%For a distribution $\cD$ and a function $f$, we define by $f(\cD)$ the distribution that is induced by the output of $f(x)$ for $x \from \cD$. 





% \begin{theorem}[\cite{McGregorMPRTV10}]\label{thm:sv-extracotr}
% 	\Enote{Remove if not needed}
% 	There is a constant $c$ to make the following holds. Let $X$ be an $\alpha$-SV source on $\{0,1\}^n$, let $Y$ be a source on $\{0,1\}^n$ with min-entropy at least $\beta n$ (independent from $X$), and let $Z=\ip{X,Y}\mbox{mod m}$ for some $m\in\mathbb{N}$. Then for every $\delta\in[0,1]$, the random variable $(Y,Z)$ is $\delta$-close to $(Y,U)$ where $U$ is uniform on $\mathbb{Z}_m$ and independent of $Y$, provided that
% 	$$
% 	n\geq c\cdot\frac{m^2}{\alpha\beta}\cdot\log(\frac{m}{\beta})\cdot\log(\frac{m}{\delta}).
% 	$$
% \end{theorem}



\Enote{I removed the definition of DP since it already appears in the intro}
\remove{
\subsection{Differential Privacy}\label{sec:prelim:DP}
We use the following standard definition of (information theoretic) differential privacy, due to \citet{DMNS06}. For notational convenience, we focus on databases over $\oo$.
\begin{definition}[Differentially private mechanisms]\label{def:mech}
	A randomized function $f\colon\oo^n\mapsto \zs$ is an {\sf $n$-size, $(\eps,\delta)$-differentially private mechanism} (denoted $(\eps,\delta)$-\DP) if for every neighboring $w,w'\in \oo^n$ and every function $g\colon \zs\mapsto \zo$, it holds that 
	$$
	\pr{g(f(w))=1}\leq e^{\eps}\cdot \pr{g(f(w'))=1} +\delta.
	$$ 	
	If $\delta=0$, we omit it from the notation.
\end{definition}
}


\subsubsection{Computational Differential Privacy}
There are several ways for defining computational differential privacy (see \cref{sec:related-works}). We use the most relaxed version due to \cite{BNO08}. For notational convenience, we focus on databases over $\oo$.
\begin{definition}[Computational differentially private mechanisms]\label{def:ComMech}
	A randomized function ensemble $f=\set{f_\pk\colon\oo^{n(\pk)}\mapsto \zs}$ is an {\sf $n$-size, $(\eps,\delta)$-computationally differentially private} (denoted $(\eps,\delta)$-$\CDP$) if for every poly-size circuit family $\set{\Ac_\pk}_{\pk\in \N}$, the following holds for every large enough $\pk$ and every neighboring $w,w'\in\oo^{n(\pk)}$:
	$$
	\pr{\Ac_\pk(f_\pk(w))=1}\leq e^{\eps(\pk)}\cdot \pr{\Ac_\pk(f_\pk(w'))=1} +\delta(\pk).
	$$ 
	If $\delta(\pk) = \negl(\pk)$, we omit it from the notation. 
\end{definition}



\subsubsection{Two-Party Differential Privacy}\label{sec:DP}
In this section we formally define distributed differential privacy mechanism (\ie protocols). %For the ease of notation, we consider protocol with no common input.

\begin{definition}\label{def:DP}%\Nnote{fix security parameter}
	A two-party protocol $\Pi=(\Ac,\Bc)$ is {\sf $(\eps,\delta)$-differentially private}, denoted $(\eps,\delta)$-$\DP$, if the following holds for every algorithm $\Dc$: let $\V^\Pc(x,y)(\pk)$ be the view of party $\Pc$ in a random execution of $\Pi(x,y)(1^\pk)$. Then for every $\pk,n \in \N$, $x\in \oo^n$ and neighboring $y,y'\in\oo^n$:
	\begin{align*}
	\pr{\Dc(V^\Ac(x,y)(\pk))=1}\le e^{\eps(\pk)}\cdot \pr{\Dc(V^\Ac (x,y')(\pk))=1}+\delta(\pk),
	\end{align*} 
	and for every $y\in \oo^n$ and neighboring $x,x'\in\oo^{n}$:
	\begin{align*}
	\pr{\Dc(V^\Bc(x,y)(\pk))=1}\le e^{\eps(\pk)}\cdot \pr{\Dc(V^\Bc (x',y)(\pk))=1}+\delta(\pk).
	\end{align*} 	
	Protocol $\Pi$ is {\sf $(\eps,\delta)$-computational differentially private}, denoted $(\eps,\delta)$-$\CDP$, if the above inequalities only hold for a non-uniform \ppt $\Dc$ and large enough $\pk$. We omit $\delta = \negl(\pk)$ from the notation. \footnote{Note that define we give for two-party differentially private protocols is a semi-honest definition, in which we ask for the security to hold when the parties interact in an honest execution of the protocol. Since we are proving a lower bound, starting from this weaker guarantee (as opposed to security against malicious players), yields a stronger result.}
\end{definition}
%We omit $\delta$ from the notation if $\delta$ is a negligible function of $n$.

%\Enote{simulation-based}
\begin{remark}[The definition for computational differential privacy we use]\label{rem:comDPChannel} 
	An alternative, stronger definition of computational differential privacy, known as simulation-based computational differential privacy, requires that the distribution of each party’s view be computationally indistinguishable from a distribution that ensures privacy in an information-theoretic sense. \cref{def:DP} is a weaker notion in comparison. Consequently, establishing a lower bound for a protocol that satisfies this weaker guarantee (as we do in this work) yields a stronger result.%Actually, our lower bound only requires the privacy to hold against \emph{uniform} external observer.
	%\Nnote{Maybe add: When only interesting in \Dp against external observer, the two definitions can be achieve using key-agreement and (single-party) \Dp mechanism. }
\end{remark}




\subsection{Useful Claims}
\remove{
In this section, we state generic lemmas and propositions that we will use later in our proofs.

The following lemma which we prove in \cref{sec:missing-proofs:distance-I}, measures the distance between two uniform stings conditioned one a random index $i$ either being fixed to $0$ or to $1$.

\def\distanceILemma{
    Let $R \la \zo^n$. For any (randomized) function $f:\{0,1\}^n\rightarrow \{0,1\}$ and $\alpha > 0$, it holds that
    \begin{align}\label{eq:f-alpha}
        \ppr{i \la [n]}{\size{\:\ex{f(R) \mid R_i = 0}-\ex{f(R) \mid R_i = 1}\:}\geq \alpha} \leq \frac{2}{n \alpha^2},
    \end{align}
    where the expectations are taken over $R$ and the randomness of $f$.
}

\begin{lemma}\label{lem:distance-I}
    \distanceILemma
\end{lemma}
}

The following two propositions state that given the output of a differentially private function, it is not possible to predict well even a random index (even if all other indexes are leaked). The first proposition handles the information-theoretic case and the second handles the computation case. Both propositions are proven in \cref{sec:missing-proofs:hard-to-guess}. 

\def\propHardToGuessInf{
    Let $f\colon \oo^n \rightarrow \cY$ be an $(\eps,\delta)$-\DP function, let $g \colon [n] \times \oo^{n-1} \times \cY \rightarrow \set{-1,1,\bot}$ be a (randomized) function, and let $X = (X_1,\ldots,X_n) \la \oo^n$. Then the following holds for every $i \in [n]$ where $X_i^* = g(i,X_{-i},f(X_1,\ldots,X_n))$:
    \begin{align*}
        \pr{X_i^* = X_i} \leq e^{\eps}\cdot \pr{X_i^* = -X_i} + \delta.
    \end{align*}
}

\begin{proposition}\label{prop:hard-to-guess-inf}
    \propHardToGuessInf
\end{proposition}


\def\propHardToGuessComp{
    Let $f = \set{f_{\pk} \colon \oo^{n(\pk)} \rightarrow \zo^{m(\pk)}}_{\pk \in \bbN}$ be an $(\eps,\delta)$-\CDP function ensemble, and let $\set{g_{\pk}}_{\pk \in \bbN}$ be a poly-size circuit family. Then, for large enough $\pk$ and $X = (X_1,\ldots,X_{n(\pk)}) \la \oo^{n(\pk)}$, the following holds for every $i \in [n(\pk)]$ where $X_i^* = g_{\pk}(i,X_{-i},f_{\pk}(X_1,\ldots,X_n))$:
    \begin{align*}
        \pr{X_i^* = X_i} \leq e^{\eps(\pk)}\cdot \pr{X_i^* = -X_i} + \delta(\pk).
    \end{align*}
}

\begin{proposition}\label{prop:hard-to-guess-comp}
    \propHardToGuessComp
\end{proposition}





\remove{
\Enote{Chao's old statement:}
\begin{lemma}\label{lem:distance-I-old}
        Let $R \la \zo^n$. 
	For any function $f:\{0,1\}^n\rightarrow \{0,1\}$ and $\alpha<0.01$, it holds that
	$$
	\Pr_{i\la[n]}\left[\: \size{\:\mathbb{E}[f(R) \mid R_i = 0]-\mathbb{E}[f(R) \mid R_i = 1]\:}\geq \alpha\right]\leq \frac{2+2\log(\frac{1}{\alpha})}{n\alpha^2}.
	$$
\end{lemma}
\begin{proof}
	Define $S_1=\{r \in \zo^n \colon f(r)=1\}$. Then for any $i\in[n]$, we have
	$$
	\begin{array}{rl}
		\size{\mathbb{E}[f(R) \mid R_i = 0]-\mathbb{E}[f(R) \mid R_i = 1]}
		&=\size{\Pr[R\in S_1|R_i=0]-\Pr[R\in S_1|R_i=1]}\\
		&=\size{\frac{\Pr[R_i=0|R\in S_1]\cdot\Pr[R\in S_1]}{\Pr[R_i=0]}-\frac{\Pr[R_i=1|R\in S_1]\cdot\Pr[R\in S_1]}{\Pr[R_i=1]}}\\
		&=\frac{2\size{S_1}}{2^n}\size{\Pr[R_i=0|R\in S_1]-\Pr[R_i=1|R\in S_1]}
	\end{array}
	$$
	When $|S_1|\leq \alpha\cdot 2^{n-1}$, we have $\size{\mathbb{E}[f(R) \mid R_i = 0]-\mathbb{E}[f(R) \mid R_i = 1]}\leq\frac{2\size{S_1}}{2^n}\leq \alpha$ for any $i\in[n]$. Hence, in the following, we assume $|S_1|> \alpha\cdot 2^{n-1}$.

	%Define $I_{bad}=\{i|\size{\Pr[R_i=0|R\in S_1]-\Pr[R_i=1|R\in S_1]}>2\alpha\}$ and $k=\size{I_{bad}}$, then for any $i\notin I_{bad}$, we have 
    %$$
    %\begin{array}{rl}
    %    2\alpha&\geq \size{\Pr[R_i=0|R\in S_1]-\Pr[R_i=1|R\in S_1]}\\
    %    &=\size{\frac{\Pr[R\in S_1|R_i=0]\cdot\Pr[R_i=0]}{\Pr[R\in S_1]}-\frac{\Pr[R\in S_1|R_i=1]\cdot\Pr[R_i=1]}{\Pr[R\in S_1]}}\\
    %    &=\size{\Pr[R\in S_1|R_i=0]-\Pr[R\in S_1|R_i=1]}\cdot\frac{1}{2\Pr[R\in S_1]}\\
    %    &\geq \size{\mathbb{E}[f(R) \mid R_i = 0]-\mathbb{E}[f(R) \mid R_i = 1]}\cdot \frac{1}{2},
    %\end{array}
    %$$ 
    %where the last inequality is because $\Pr[R\in S_1]\leq 1$. So that $\size{\mathbb{E}}[f(R) \mid R_i = 0]-\mathbb{E}[f(R) \mid R_i = 1]\leq %4\alpha$.
    Define $I_{bad}=\{i \colon \size{\Pr[R_i=0|R\in S_1]-\Pr[R_i=1|R\in S_1]} \geq 2\alpha\}$ and $k=\size{I_{bad}}$, and denote $I_{bad}=\{i_1,\dots,i_k\}$. Define $(X_{i_1}, \ldots X_{i_k}) = (R_{i_1},\dots,R_{i_k})\mid_{R \in S_1}$. 
    Consider the min-entropy
	$$
	\begin{array}{rl}
		H_{min}(X_{i_1},\dots,X_{i_k})&\leq H(X_{i_1},\dots,X_{i_k})\\
		&\leq \sum_{j=1}^k H(X_{i_j})\\
		&\leq k\cdot \left(-(\frac{1}{2}+2\alpha)\cdot\log(\frac{1}{2}+2\alpha)-(\frac{1}{2}-2\alpha)\cdot\log(\frac{1}{2}-2\alpha)\right)\\
            &=k\cdot \left(-(\frac{1}{2}+2\alpha)\cdot(\log(1+4\alpha)-1)-(\frac{1}{2}-2\alpha)\cdot(\log(1-4\alpha)-1)\right)\\
            &=k\cdot \left(1-(\frac{1}{2}+2\alpha)\cdot\log(1+4\alpha)-(\frac{1}{2}-2\alpha)\cdot\log(1-4\alpha)\right),
		
	\end{array}
	$$
	where $H_{min}(Y)$ is the minimum entropy of $Y$ and $H(Y)$ is the Shannon entropy of $Y$.\Enote{add to preliminaries.}
        The third inequality holds since by the definition of $I_{bad}$, for every $j \in [k]$ it holds that $\size{\pr{X_{i_j} = 1}-\pr{X_{i_j} = 0}} > 2\alpha$, and therefore $H(X_{i_j}) \leq H(1/2 + 2\alpha)$\Enote{define}.
	
	Therefore, there exists $b_1,\dots,b_k\in\{0,1\}$, such that 
	
	\begin{align}\label{eq:min-entropy-result}
		\Pr\left[(R_{i_1},\ldots,R_{i_k}) = (b_1,\ldots,b_k) \mid R\in S_1\right]
		&= \pr{(X_{i_1},\ldots,X_{i_k}) = (b_1,\ldots,b_k)}\\
		&= 2^{-H_{min}(X_{i_1},\dots,X_{i_k})}\nonumber\\
		&\geq 2^{k\cdot \left(-1+(\frac{1}{2}+2\alpha)\cdot\log(1+4\alpha)+(\frac{1}{2}-2\alpha)\cdot\log(1-4\alpha)\right)}.\nonumber
	\end{align}
	
	Let $S_{bad}=\{r \in \zo^n  \colon \set{(r_{i_1},\ldots,r_{i_k}) = (b_1,\ldots,b_k)} \land \set{r\in S_1}\}$.
	It holds that
	\begin{align*}
		|S_{bad}|
		&= \size{S_1} \cdot \Pr\left[(R_{i_1},\ldots,R_{i_k}) = (b_1,\ldots,b_k) \mid R\in S_1\right]\\
		&\geq \alpha\cdot 2^{n-1}\cdot2^{k\cdot \left(-1+(\frac{1}{2}+2\alpha)\cdot\log(1+4\alpha)+(\frac{1}{2}-2\alpha)\cdot\log(1-4\alpha)\right)},
	\end{align*} 
	where the inequality holds by \cref{eq:min-entropy-result} and since $\size{S_1} \geq \alpha\cdot 2^{n-1}$.
	Notice that any string in $S_{bad}$ depends on at most $n-k$ bits. It implies that $|S_{bad}|\leq 2^{n-k}$. Therefore, we have
	$$
	\begin{array}{rl}
		&2^{n-k}\geq \alpha\cdot 2^{n-1}\cdot2^{k\cdot \left(-1+(\frac{1}{2}+2\alpha)\cdot\log(1+4\alpha)+(\frac{1}{2}-2\alpha)\cdot\log(1-4\alpha)\right)} \\
		\Rightarrow& n-k \geq \log \alpha+n-1+k\cdot \left(-1+(\frac{1}{2}+2\alpha)\cdot\log(1+4\alpha)+(\frac{1}{2}-2\alpha)\cdot\log(1-4\alpha)\right)\\
		\Rightarrow& 1-\log \alpha \geq k\cdot((\frac{1}{2}+2\alpha)\cdot\log(1+4\alpha)+(\frac{1}{2}-2\alpha)\cdot\log(1-4\alpha))\\
		\Rightarrow& 1-\log \alpha \geq k\cdot(4\alpha\cdot\log(1+4\alpha)+(\frac{1}{2}-2\alpha)\cdot\log(1-16\alpha^2))\\
        \Rightarrow& 1-\log\alpha \geq k\cdot(15.9\alpha^2-8\alpha^2+32\alpha^3)=k\cdot(7.9\alpha^2+32\alpha^3)>0.5k\alpha^2\\
		\Rightarrow& k\leq \frac{2-2\log \alpha}{\alpha^2} = \frac{2+2\log (1/\alpha)}{\alpha^2},
	\end{array}
	$$
	Where the third transition holds since 
	\begin{align*}
		\lefteqn{(\frac{1}{2}+2\alpha)\cdot\log(1+4\alpha)+(\frac{1}{2}-2\alpha)\cdot\log(1-4\alpha)}\\
		&= 4\alpha\cdot\log(1+4\alpha) + (\frac{1}{2}-2\alpha)\paren{\log(1+4\alpha)+\log(1-4\alpha)}\\
		&= 4\alpha\cdot\log(1+4\alpha)+(\frac{1}{2}-2\alpha)\cdot\log(1-16\alpha^2),
	\end{align*}
	and the forth transition holds since $4\alpha\cdot\log(1+4\alpha)+(\frac{1}{2}-2\alpha)\cdot\log(1-16\alpha^2) > 15.9\alpha^2-8\alpha^2+32\alpha^3$ for $\alpha < 0.01$.
	Thus, we conclude that 
	$$
	\Pr_{i\la[n]}\left[\size{\mathbb{E}[f(R) \mid R_i=0]-\mathbb{E}[f(R) \mid R_i = 1]}\geq \alpha\right]\leq \frac{k}{n}\leq \frac{2+2\log (1/\alpha)}{n\alpha^2}.
	$$
\end{proof}
}


\subsection{Channels and Two-Party Protocols}\label{sec:protocol}

\paragraph{Channels.}A channel is simply a distribution of a pair of tuples defined as follows. 
\begin{definition}[Channels]\label{def:channel} A {\sf channel} $C_{(X,U)(Y,V)}$ of size $\isize$ over alphabet $\Sigma$ is a probability distribution over $(\Sigma^\isize \times\zo^\ast) \times(\Sigma^\isize \times\zo^\ast)$. The ensemble $C_{(X,U)(Y,V)}= \set{C_{(X_\pk,U_\pk)(Y_\pk,V_\pk)}}_{\pk\in \N}$ is an $\isize$-size channel ensemble, if for every $\pk\in \N$, $C_{(X_\pk,U_\pk)(Y_\pk,V_\pk)}$ is an $\isize(\pk)$-size channel. %We denote a channel of size one by a \emph{single-bit} channel. 
We refer to $X$ and $Y$ as the {\sf local outputs}, and to $U$ and $V$ as the {\sf views}.	
\end{definition}

We view a  channel as the experiment in which there are two parties $\Ac$ and $\Bc$.  Party $\Ac$ receives ``output'' $X$ and ``view'' $U$, and party $\Bc$ receives ``output'' $Y$ and ``view'' $V$. Unless stated otherwise, the channels we consider are over the alphabet $\Sigma = \oo$. We naturally identify channels with the distribution that characterizes their output.








\subsubsection{Two-Party Protocols}

A two-party protocol $\Pi=(\Ac,\Bc)$ is \ppt if the running time of both parties is polynomial in their input length. We let $\Pi(x,y)(z)$ or $(\Ac(x),\Bc(y))(z)$ denote a random execution of $\Pi$ on a common input $z$, and private inputs $x,y$.%We assume \wlg that a protocol has a common output (part of its transcript).\Jnote{This is not really the case we consider in this paper..}

\begin{definition}[Oracle-aided protocols]\label{def:ChannelAidedProtocol}
	In a two-party protocol $\Pi$ with oracle access to a {\sf protocol} $\Psi$, denoted $\Pi^\Psi$, the parties make use of the \textit{next-message function} of $\Psi$.\footnote{The function that on a partial view of one of the parties, returns its next message.} In a two-party protocol $\Pi$ with oracle access to a {\sf channel} $C_{Z W}$, denoted $\Pi^C$, the parties can jointly invoke $C$ for several times. In each call, an independent pair $(z,w)$ is sampled according to $C_{Z W}$, one party gets $z$, the other gets $w$.
\end{definition}


\begin{definition}[The channel of a protocol]\label{def:ChannlOfProtocol}
	For a no-input two-party protocol $\Pi= (\Ac,\Bc)$, we associate the channel $C_\Pi$, defined by $\C_\Pi= C_{(X, U),(Y, V)}$, where $X$ and $Y$ are the local outputs of $\Ac$ and $\Bc$ (respectively) and
	$U$ and $V$ are the local views of $\Ac$ and $\Bc$ (respectively).
    
	For a two-party protocol $\Pi$ that gets a security parameter $1^\pk$ as its (only, common) input, we associate the channel ensemble $ \set{C_{\Pi(1^\pk)}}_{\pk\in \N}$. 
\end{definition}

\begin{definition}[$(\alpha,\gamma)$-Accurate channel]\label{def:accurate-func}
	A channel $C = C_{(X, U),(Y, V)}$ is {\sf $(\alpha,\gamma)$-accurate for the function $f$}, if $\ppr{C}{\size{\out(V)-f(X,Y)}\leq \alpha}\ge \gamma$, where $\out(V)$ is the designated output.
    A channel ensemble $C_{(X, U),(Y, V)}= \set{C_{(X_\pk, U_\pk),(Y_\pk, V_\pk)}}_{\pk\in \N}$ is  $(\alpha,\gamma)$-accurate for  $f$ if $C_{(X_\pk, U_\pk),(Y_\pk, V_\pk)}$ is $(\alpha(\pk),\gamma(\pk))$-accurate for $f$, for every $\pk \in \N$.
\end{definition}

\subsubsection{Differentially Private Channels}\label{sec:DPChannel}
Differentially private channels are naturally defined as follows:
\begin{definition}[Differentially private channels]\label{def:DPChannel}
	An $n$-size channel $C = C_{(X, U),(Y, V)}$ with $X, Y$ over $\oo^n$ 
	is {\sf$(\eps,\delta)$-differentially private} (denoted $(\eps,\delta)$-$\DP$) if for every $x \in \Supp(X)$ there exists an $n$-size $(\eps,\delta)$-$\DP$ mechanisms $\Mc_x$ such that $(X,Y,U) \equiv (X,Y,\Mc_X(Y))$, and for every $y \in \Supp(Y)$ there exists an $n$-size $(\eps,\delta)$-$\DP$ mechanisms $\Mc_y'$ such that $(X,Y,V) \equiv (X,Y,\Mc_Y'(X))$. In addition, we say that the channel is \emph{uniform} if $X$ and $Y$ are independent random variables uniformly distributed in $\oo^n$. 
\end{definition}

\begin{definition}[Computational differentially private channels]\label{def:CDPChannel}
	An $n$-size channel ensemble $C = \set{C_{(X_\pk, U_\pk),(Y_\pk, V_\pk)}}_{\pk\in\N}$ with $X_\pk, Y_\pk$ over $\oo^n$ 
	is {\sf$(\eps,\delta)$-computationally differentially private} (denoted $(\eps,\delta)$-$\CDP$) if for every ensemble $\set{x_\pk \in \Supp(X_\pk)}_{\pk\in\N}$ there exists an $n$-size $(\eps,\delta)$-\CDP mechanisms ensemble $\set{\Mc_{x_\pk}}_{\pk\in\N}$ such that $(X_\pk,Y_\pk,U_\pk) \equiv (X_\pk,Y_\pk,\Mc_{X_\pk}(Y_\pk))$, for every $\pk\in\N$, and for every ensemble $\set{y_\pk \in \Supp(Y_\pk)}_{\pk\in\N}$ there exists an $n$-size $(\eps,\delta)$-$\CDP$ mechanisms ensemble $\set{\Mc'_{y_\pk}}_{\pk\in\N}$ such that $(X_\pk,Y_\pk,V_\pk) \equiv (X_\pk,Y_\pk,\Mc_{Y_\pk}'(X_\pk))$ for every $\pk\in \N$. In addition, we say that the channel is \emph{uniform} if $X_\pk$ and $Y_\pk$ are independent random variables uniformly distributed in $\{\pm 1\}^n$ for all $\pk\in\N$.
\end{definition}




% \begin{lemma}~\label{lem:dp-sv-source}
% 	Let $P$ be an $\varepsilon$-DP randomized protocol. Let $X$ and $Y$ be independent random variables uniformly distributed in $\{\pm 1\}^n$ and let random variable $\Pi(X,Y)$ denote the transcript of running $P(X,y)$. Then for every $\pi\in Supp(\Pi)$, the random variables corresponding to the inputs conditioned on transcript $\pi$, $X_\pi$ and $Y_\pi$, are independent $e^{-\varepsilon}$-strong SV source.
% \end{lemma}





\subsubsection{Weak Erasure Channel (\WEC)}

\begin{definition}[\WEC]\label{def:WEC}
	A channel $((O_A,V_A), (O_B,V_B))$ with $O_A \in \set{0,1}$ and $O_B \in \set{0,1,\bot}$ is a {\sf weak erasure channel}, denoted $(\alpha,p,q)$-$\WEC$, if:
	\begin{itemize}
		%\item $O_A\in \set{-1,1}$ and $O_B\in \set{-1,1,\bot}$.
		\item Random erasure: $\pr{O_B = \perp} = 1/2$.
		
		\item Agreement: $\pr{O_A\ne O_B\mid O_B\ne \bot}\le \alpha$.
		
		\item Secrecy:
		
		\begin{enumerate}
			\item For every algorithm $\Dc$ it holds that\label{WEC:item:A}
			\begin{align*}
				%\size{\pr{\Ac(O_A,V_A) = 1 \mid O_B \neq \perp} - \pr{\Ac(O_A,V_A) = 1 \mid O_B = \perp}} \le p
				\size{\pr{\Dc(V_A) = 1 \mid O_B \neq \perp} - \pr{\Dc(V_A) = 1 \mid O_B = \perp}} \le p
			\end{align*}
			(Alice doesn't know if $O_B = \perp$.)
			
			\item For every algorithm $\Dc$ it holds that\label{WEC:item:B}
			\begin{align*}
				\pr{\Dc(V_B) = O_A \mid O_B=\bot} \leq \frac{1+q}{2}.
			\end{align*}
			(i.e., if $O_B=\bot$, Bob don't know what is the value of $O_A$).
			
			%\item $SD((O_A U|O_B=\bot),(O_A U|O_B\ne \bot))\le p$ (The sender don't know if $O_B=\bot$).
			
			%\item $SD(V O_A|O_B=\bot,V(-O_A)|O_B=\bot)\le q$ (If $O_B=\bot$, Bob don't know what the value of $O_A$).
		\end{enumerate}
	\end{itemize}
   We say that a channel ensemble $C=\set{C_\pk}_{\pk\in N}$ is a {\sf computational weak erasure channel}, denoted $(\alpha,p,q)$-\CompWEC, if for every \ppt algorithm $\Dc$ and every sufficiently large $\pk\in\N$, $C_\pk$ satisfies the properties stated in the items above, where the secrecy property holds with respect to a \ppt algorithm $\Dc$. A protocol $\Lambda$ is said to be $(\alpha,p,q)$-$\CompWEC$, if the ensemble induces by the protocol (that is, $C=\set{C_{\Lambda(\pk)}}_{\pk\in\N}$) is $(\alpha,p,q)$-$\CompWEC$.  
\end{definition}



\subsubsection{Approximate Weak Erasure Channel (\AWEC)}\label{sec:AWEC}

\begin{definition}[\AWEC]\label{def:AWEC}
	A channel $C = ((O_A,V_A), (O_B,V_B))$ over $([-n,n] \times \zo^*) \times (([-n,n] \cup \bot)  \times \zo^*)$ is an {\sf approximate weak erasure channel}, denoted $(\ell,\alpha,p,q)$-\AWEC if:
	\begin{itemize}
		
		\item Random erasure: $\pr{O_B = \perp} = 1/2$.
		
		\item Accuracy: $\pr{\size{O_A - O_B} > \ell \mid O_B \ne \bot}\le \alpha$.
		
		\item Secrecy:
		
		\begin{enumerate}
			\item For every algorithm $\Dc$ it holds that\label{AWEC:item:A}
			\begin{align*}
				%\size{\pr{\Ac(O_A,V_A) = 1 \mid O_B \neq \perp} - \pr{\Ac(O_A,V_A) = 1 \mid O_B = \perp}} \le p
				\size{\pr{\Dc(V_A) = 1 \mid O_B \neq \perp} - \pr{\Dc(V_A) = 1 \mid O_B = \perp}} \le p
			\end{align*}
			(Alice doesn't know if $O_B=\bot$).
			
			\item For every algorithm $\Dc$ it holds that\label{AWEC:item:B}
			\begin{align*}
				\pr{\size{\Dc(V_B) - O_A} \leq 1000 \ell \mid O_B=\bot} \leq q.
			\end{align*}
			(i.e., if $O_B=\bot$, Bob can't estimate the value of $O_A$ with error $\leq 1000 \ell$).
		\end{enumerate}
	\end{itemize}
     We say that a channel ensemble $C=\set{C_\pk}_{\pk\in N}$ is a {\sf computational approximate weak erasure channel}, denoted $(\ell,\alpha,p,q)$-\CompAWEC, if for every \ppt algorithm $\Dc$ and every sufficiently large $\pk\in\N$, $C_\pk$ satisfies the properties stated in the items above. A protocol $\Gamma$ is said to be $(\ell,\alpha,p,q)$-$\CompAWEC$, if the ensemble induced by the protocol (that is, $C=\set{C_{\Gamma(\pk)}}_{\pk\in\N}$) is $(\ell,\alpha,p,q)$-$\CompAWEC$.  
\end{definition}

We will make use of the following lemma, which shows that for some choices of the parameters, \AWEC implies \WEC. The lemma is proven in \cref{sec:AWEC-to-WEC}.

\begin{lemma}\label{lemma:AWEC-to-WEC}
	For every $\ell> 0$, there exists a \ppt protocol $\Lambda = (\Pc_1,\Pc_2)$ such that given an oracle access to an $(\ell,\alpha,p,q)$-\AWEC $C$, the channel $\tilde{C}$ induced by $\Lambda^C$ is $(\alpha'=\alpha+0.001,\: p' = p ,\:  q' = 1/2 + 2(q+0.01))$-\WEC.
	Furthermore, the proof is constructive in a black-box manner:
	\begin{enumerate}
		\item There exists an oracle-aided \ppt algorithm $\Ec_1$ such that for every channel $C = ((\OA,\VA), (\OB,\VB))$ and algorithm $\Dc$ violating the \WEC secrecy property~\ref{WEC:item:A} of $\tilde{C}$, algorithm $\Ec_1^{\Dc}$ violates the \AWEC secrecy property~\ref{AWEC:item:A} of $C$.
		
		\item There exists an oracle-aided \ppt algorithm $\Ec_2$ such that for every channel $C = ((\OA,\VA), (\OB,\VB))$ and algorithm $\Dc$ violating the \WEC secrecy property~\ref{WEC:item:B} of $\tilde{C}$, algorithm $\Ec_2^{\Dc}$ violates the \AWEC secrecy property~\ref{AWEC:item:B} of $C$.
	\end{enumerate}
\end{lemma}

Since \cref{lemma:AWEC-to-WEC} is constructive, the following is an immediate corollary.
\begin{corollary}\label{cor:CompAWEC to CompWEC}
There exists an oracle aided \ppt protocol $\Lambda$, such that given a protocol $\Gamma$ that induces $(\ell,\alpha,p,q)$-\CompAWEC, it holds that $\Lambda^\Gamma$ is $(\alpha'=\alpha+0.001,\: p' = p ,\:  q' = 1/2 + 2(q+0.01))$-\CompWEC.  
\end{corollary}
\begin{proof}[Proof of \ref{cor:CompAWEC to CompWEC}]
Let $\Lambda$ be the \ppt algorithm guaranteed  by Lemma \ref{lemma:AWEC-to-WEC}. Given an $(\ell,\alpha,p,q)$-\CompAWEC protocol $\Gamma$, we define $\Lambda(\pk)={\Lambda^{\Gamma(\pk)}(\pk)}$. Assume towards a contradiction that $\Lambda$ is not a $(\alpha',p',q')$-\CompWEC. It follows that there exists a \ppt $\Dc$ that for infinity many $\pk\in\N$ contradicts one of the \WEC secrecy properties of channel ensemble $\set{C_{\Lambda(\pk)}}_{\pk\in\N}$. Fix $\pk\in\N$ for which this holds. By Lemma \ref{lemma:AWEC-to-WEC}, there exists a \ppt $\Ec^\Dc$ that for every such $\pk$  contradicts one of the secrecy properties of the channel $C_{\Gamma(\pk)}$. This implies that for infinity many $\pk\in\N$, $\Ec^\Dc$  contradict the secrecy of the channel ensemble $\set{C_{\Gamma(\pk)}}_{\pk\in\N}$, which is a contradiction since this would means that $\Gamma$ is not a $(\ell,\alpha,p,q)$-\CompAWEC.       
\end{proof}



\subsection{Oblivious Transfer (\OT)}

\paragraph{Secure Computation.}
We use the standard notion of securely computing a functionality, \cf  \cite{Goldreich04}.
\begin{definition}[Secure computation]\label{def:SFE}
	A two-party protocol {\sf securely computes a functionality $f$}, if it does so according to the real/ideal paradigm.   We add the term perfectly/statistically/computationally/non-uniform computationally, if the simulator's output is  perfect/statistical/computationally indistinguishable/  non-uniformly indistinguishable from  the real distribution.  The protocol have the above notions of security {\sf against semi-honest  adversaries}, if its security only  guaranteed to holds against an adversary that follows the prescribed protocol.   Finally, for the case of perfectly secure computation, we naturally apply the above notion also to the non-asymptotic case: the protocol with no security parameter perfectly  compute a functionality $f$.
	
	A two-party protocol {\sf securely computes a functionality ensemble $f$ with oracle to a channel $C$}, if it does so according to the above definition when the parties have access to a trusted party computing $C$. All the above adjectives naturally extend to this setting.
\end{definition}

\paragraph{Oblivious Transfer.}
The (one-out-of-two) oblivious transfer functionality is defined as follows.
\begin{definition}[oblivious transfer functionality $f_{\OT}$]\label{def:OTfunc}
	The oblivious transfer functionality over $\zo \times (\zs)^2$ is defined by  $f_{\OT} (i,(\sigma_0,\sigma_1)) = (\perp,\sigma_i)$.
\end{definition}
A protocol is $\ast$ secure OT,   for \\$\ast\in \set{\text{semi-honest statistically/computationally/computationally non-uniform}}$, if it  compute the $f_{\OT}$  functionality with $\ast$ security.





% \begin{definition}[Computational oblivious transfer, semi-honest model]
% A protocol $\Pi=(\Ac,\Bc)$ is a semi-honest 1-out-of-2 computational oblivious transfer (comp-OT) protocol if the following holds. Given a common input $1^{\pk}$, the parties $\Ac$ and $\Bc$ run the protocol $\Pi(1^\pk)$ (in an honest manner) and    
% $\Ac$ outputs $X=(m_1,m_2)\in \zo\times\zo$ and has a view $U$ and $\Bc$ outputs $Y=(i,\hat{m})\in\zo\times\zo$ and has a view $V$, and the following properties are satisfied:
% \begin{enumerate}
%     \item \textbf{Correctness:} 
%     $\pr{\hat{m}\neq m_i}<\negl(\pk).$ 
    
%     \item \textbf{A's Privacy:} For every \ppt $\Dc$ and every sufficiently large $\pk$:
%     $\pr{\Dc(V)=m_{i-1}}<(1+\negl(\pk))/2$
    
%     \item \textbf{B's Privacy:} For every \ppt $\Dc$ and every sufficiently large $\pk$:
%     $\pr{\Dc(U)=i}<(1+\negl(\pk))/2$  
% \end{enumerate}
% \end{definition}

We make use of the following useful results by Wullschleger on oblivious transfer amplification from weak channels.
\begin{theorem}[\cite{Wullschleger09}, from \WEC to statistically secure \OT]\label{thm:WEC TO OT IT}
    There exists an oracle aided protocol $\Pi$ such that the following holds: Given a $(\alpha,p,q)$-\WEC $C$, if $44(\alpha+p)\le 1-q$ then $\Pi^{C}(1^\pk)$ is a semi-honest statistically secure \OT.
\end{theorem}

The following computational version of \cref{thm:WEC TO OT IT} is implicit in \cite{Wullschleger09} and is based on the computational proof explicitly stated in \cite{Wul07} (see Section 6 in \cite{Wullschleger09} for discussion).   

\begin{theorem}[\cite{Wullschleger09,   Wul07}, from \CompWEC to computinally secure \OT]\label{thm:WEC TO OT Comp}
    There exists an oracle aided protocol $\Pi$ such that the following holds: Given a $(\alpha,p,q)$-\CompWEC protocol $\Lambda$, if $44(\alpha+p)\le 1-q$ then $\Pi^{\Lambda}$ is a semi-honest computational secure \OT.
\end{theorem}



% \begin{definition}[Computational 1-out-of-2 Oblivious Transfer, semi-honest model]
% A protocol $\Pi=(\Ac,\Bc)$ is a semi-honest 1-out-of-2 $(\eps,\alpha,\beta)$-oblivious transfer (OT) protocol if the following holds. 

% The parties $\Ac$ and $\Bc$ run the protocol (in an honest manner) and    
% $\Ac$ outputs $X=(m_1,m_2)\in \zo\times\zo$ and has a view $U$ and $\Bc$ outputs $Y=(i,\hat{m})\in\zo\times\zo$ and has a view $V$, and following properties are satisfied:
% \begin{enumerate}
%     \item \textbf{Correctness:} 
%     $\pr{\hat{m}\neq m_i}<\eps.$ 
    
%     \item \textbf{A's Privacy:} For every adversary $\Dc$:
%     $\pr{\Dc(V)=m_{i-1}}<(1+\alpha)/2$
    
%     \item \textbf{B's Privacy:} For every adversary $\Dc$: $\pr{\Dc(U)=i}<(1+\beta)/2$  
% \end{enumerate}
% \end{definition}



\section{Multi-Unit Auctions}\label{sec-mua}
%\src{To write that everything here is for Single-minded, identical items, unknown bundle sizes.}
We begin with the setting of multi-unit auctions. A multi-unit auction comprises of $m$ \emph{identical} items, where the valuation of every player $i$ is given by $v_i:[m]\to \mathbb R^{+}$. 
We consider two classes of valuations: unknown single-minded bidders (\cref{subsec::mua-sm}) and bidders whose valuations exhibit decreasing marginal values (\cref{subsec::mua-decreasing}). 




\subsection{Single-Minded Valuations} \label{subsec::mua-sm}
In this section, we address two types of single-minded bidders: known and unknown. We begin with some definitions and then provide background on each.  

A valuation $v_i$ is \emph{single-minded} if there is a scalar $x_i$ and a quantity $d_i$ such that $v_i(q)=x_i$
if $q\ge d_i$, and otherwise $v_i(q)=0$.
In particular, we investigate the setting of \emph{unknown single-minded bidders}, where both the demand $d_i$ and the value $x_i$ of every player $i$ are private information (if only $x_i$ is private, then it is a setting with \emph{known single-minded bidders}).


For deterministic obviously strategy-proof mechanisms, the class of  known single-minded bidders admit strategy-proof mechanisms that give ${\mathcal O}(\min\{\log m,\log n\})$ approximation to the welfare \cite{DGR14,CGS22}, and no mechanism  gives an approximation better than $\Omega(\sqrt{\log n})$ of the welfare \cite{FPV21}.  For unknown single-minded bidders, \cite{Ron24} has shown a tight lower bound of $\min\{m,n\}$. 

We note that our proposed mechanisms (both the first attempt in \cref{subsub::first-attempt} and the actual one in \cref{subsub::actual-ub-mua-sm}) are for both unknown and known obviously strategy-proof mechanisms, whilst the impossibility that we provide in \cref{subsec-lbs-22-sm-mua} holds solely for unknown single-minded bidders. We leave open the question of understanding the approximation power of randomized obviously strategy-proof mechanisms for known single-minded bidders.  



% We first consider the case of \emph{unknown single-minded bidders}, where each bidder $i$ receives value $v_i$ for receiving at least $d_i$ items and $0$ otherwise.\footnote{This setting is commonly known as unknown single-minded bidders, since both the demand $d_i$ and the value $v_i$ of every player $i$ are private.} 

\subsubsection{Upper Bound: a First Attempt}\label{subsub::first-attempt}
Since an ascending-price auction for the grand bundle of goods is the optimal deterministic OSP mechanism in this setting \cite{Ron24}, we begin with a natural randomized analogue of this approach.  Namely, we ``guess'' a bundle size and run an ascending price auction for bundles of this size.  

Formally,
%\src{Not sure whether it should be that formal, or that a non-formal intuition in the introduction suffices}
%Let $m$ denote the number of items and $n$ denote the number of agents.  
let $k = \lceil \log{m} \rceil$.  Consider the mechanism \textsc{Random-Bundles} in which 
an integer bundle size $\ell =2^{j}$ is sampled uniformly at random from the set of powers of two $\ell \in \{1,2,4,8,\dots,\frac{m}{2}, m\}$. Given this fixed bundle size, every bidder either wins exactly $\ell$ items or wins no items at all.  Observe, then, that at most $\nicefrac{m}{\ell}$ bidders can win a bundle. We now increase the price of being served $\ell$ items, until at most $\nicefrac{m}{\ell}$ bidders remain. All these bidders win $\ell$ items and pay the price at which we stop, while the remainder get nothing and pay nothing.
Note that \textsc{Random-Bundles} is a generalized ascending auction, so by \cref{lemma-partial} it is OSP.

% \begin{theorem}
%     \dnote{\textsc{Random-Bundles}} is universally OSP.
% \end{theorem}
% \begin{proof}
%     Fix an arbitrary bundle size $\ell$ selected by the mechanism.  We will argue that the mechanism is OSP for this fixed realization and, hence, is universally OSP.  Observe that once the bundle sizes are fixed, bidder $i$ can evaluate her value for receiving a bundle by comparing its size to her demand.  If any fixed bundle has size exceeding her demand, she will be satisfied by any bundle and the inverse is also true.  Ergo, remaining in the auction while the uniform asking price is less than or equal to her private value for receiving an arbitrary bundle and exiting as soon as the price exceeds this value is an obviously dominant strategy.
% \end{proof}

We now argue that this mechanism obtains a logarithmic approximation to the optimal social welfare.  Due to space constraints, we provide a proof sketch of the approximation guarantee and defer the complete proof 
% to the full version.
%full-version-change-tag%
\cref{app-missing-mua}.

\begin{theorem}\label{claim:ascending-bundles-approx}
{\textsc{Random-Bundles}} obtains an $\mathcal O(\log{m})$ approximation to the optimal social welfare.
\end{theorem}
\begin{proof}[Proof Sketch]
First, observe that we can partition bidders into groups depending on their demand as follows:  for each bidder $i$, we place bidder $i$ in group $p$ if her demand  $d_i$ is between $2^{p}$ and $2^{p+1}-1$.
%(inclusive).  
Since each bidder has demand at least $1$ and at most $m$, there are at most $\log{m}$ groups in total.  

Now, we  compare the portion of the optimal social welfare coming from bidders in group $p$ against the welfare \textsc{Random-Bundles} obtains when
%\pinksout{conditioned on it}
 selecting bundles of size $2^{p+1}$.  On one hand, the optimal solution selects at most twice as many bidders appearing in group $p$ as the total number of bidders \textsc{Random-Bundles} serves conditioned on it selecting bundles of size $2^{p+1}$.  On the other hand, the bidders served in \textsc{Random-Bundles} conditioned on selecting bundles of size $2^{p+1}$ have the \emph{highest} value among bidders satisfied by receiving $2^{p+1}$ goods.  In total, we obtain an $\mathcal O(\log{m})$-approximation.
\end{proof}

\subsubsection{A Constant Upper Bound }\label{subsub::actual-ub-mua-sm}
Unfortunately, an approximation of $\mathcal O(\log m)$ appears to be the best achievable using the approach of randomly choosing fixed bundle sizes.  As such, we need to turn to a different approach.  We, thus, adopt the ``balanced sampling'' approach utilized extensively in other areas of mechanism design (see, e.g., \cite{feige2005competitive,goldberg2006competitive,dobzinski2012truthful,dobzinski2007two,badanidiyuru2012learning,bei2017worst})  in the form of Mechanism \ref{alg:single-minded},  below:

\begin{theorem}\label{thm-ub-mua-sm}
        There is a universally OSP mechanism for unknown single-minded bidders in a multi-unit auction  that gives a $400$-approximation to the optimal welfare.
\end{theorem}
We prove \cref{thm-ub-mua-sm} by describing a randomized mechanism, i.e., Mechanism \ref{alg:single-minded}, and showing it is universally OSP (\cref{claim-mua-sm-mechanism-osp}) and indeed gives a $400$-approximation to the optimal social welfare (\cref{lem:single-minded-approx}). 


\begin{algorithm2e}
%\setstretch{1.1}
\SetKwInOut{Input}{Input}
\Input{A set of bidders $\bidders$ and $m$ identical items}
 With probability $\nicefrac{1}{2}$: 

 \quad Bundle all $m$ items together and run an ascending price auction on the grand bundle

 With remaining probability $\nicefrac{1}{2}$:

 \quad Let $S \leftarrow \emptyset$, $U \leftarrow \emptyset$
 
 \quad Place each bidder independently in $S$ w.p. $\nicefrac{1}{2}$ and each bidder in $U$ with the remaining probability

 \quad ``Discard'' each bidder and $S$ and learn their valuation function

 \quad Compute the optimal solution among bidders only in $S$ and let $O$ denote the value of this solution

 % \quad Set a price of $\nicefrac{O}{10m}$ for each of the $m$ goods

\quad  Iterate over the bidders (in an arbitrary) order, and for each bidder $i \in \bidders$, let them purchase their preferred bundle from the remaining items at a price of $\nicefrac{O}{10m}$ per item
 
 % \quad Let each bidder in $U$ purchase their preferred (possibly empty) bundle at current prices
 
 \caption{``\textsc{Single-Minded}''}
 \label{alg:single-minded}
\end{algorithm2e}


\begin{lemma}\label{claim-mua-sm-mechanism-osp}
    Mechanism \ref{alg:single-minded} is universally OSP.
\end{lemma}
\begin{proof}
Under the realization of randomness where we auction the grand bundle, we utilize a generalized ascending auction for the grand bundle which is OSP by \cref{lemma-partial}.  
If we run the uniform pricing auction, then no bidder in $S$ can win any items and, thus, reporting their valuations truthfully is an obviously dominant strategy. Bidders in $U$ select their preferred bundle of goods, so reporting their preferences truthfully is also an obviously dominant strategy for them.
% the auction is OSP for them as well.
\end{proof}

% \begin{theorem}
%     $\mathcal{M}_5$ obtains a $\Omega(1)$-approximation (specifically a $1/200$-approximation) to the optimal social welfare.
% \end{theorem}

To prove the approximation factor of Mechanism \ref{alg:single-minded}, 
we define a bidder as \emph{critical} if her value for the grand bundle of goods is at least $\nicefrac{1}{100}$ of the total optimal social welfare. We also use the notation $OPT$ to denote the optimal social welfare for a given valuation profile $(v_1, \ldots, v_n)$ and define $OPT(S)$ as the optimal social welfare attainable by allocating all items exclusively among the bidders in a specified subset $S \subseteq N$.
% where $S$ is a subset of bidders $S\subseteq \bidders$,
% to denote the optimal welfare achieved by allocating all items exclusively among the bidders in $S$.

\cref{lem:auction-sampling} establishes that the sampling phase yields 
\textquote{representative}
%``accurate'' 
sampled and unsampled sets in the case that there are no critical bidders with \textquote{high enough} probability. Note that the proof utilizes a lemma of \cite{bei2017worst}. %regarding partitions of sets and subadditive functions.  
We defer the proof of \cref{lem:auction-sampling}  to 
% the full version.
% full-version-change-tag%
\cref{subsec::proof-auction-sampling}. 

\begin{lemma}\label{lem:auction-sampling}
Consider an instance $(v_1,\ldots,v_n)$ of bidders
in a multi-unit auction\footnote{
Our lemma actually holds also for bidders in a combinatorial auctions,  
%more general valuations
%Our lemma actually holds also for more general valuations, 
but for simplicity we state it solely for multi-unit auctions.} where no bidder is critical.  Suppose each bidder is placed in a ``sampled'' set $S$ with probability $\nicefrac 1 2$ and placed in an ``unsampled'' set $U$ with the remaining probability independently. Then the optimal welfare obtained from bidders in the sampled set $\text{OPT}(S)$ and the optimal welfare obtained from bidders in the unsampled set $\text{OPT}(U)$ are such that $\text{OPT}(S) \geq \nicefrac{\text{OPT}}{5}$ and $\text{OPT}(U) \geq \nicefrac{\text{OPT}}{5}$ with probability at least $\nicefrac{1}{2}$.
\end{lemma}

With Lemma \ref{lem:auction-sampling} in hand, we are ready to prove that: 
% our main theorem for this class of valuation functions.  

\begin{lemma}\label{lem:single-minded-approx}
Mechanism \ref{alg:single-minded} obtains 
a $400$-approximation
to the optimal social welfare.
\end{lemma}
\begin{proof}
First we handle the case that there is a critical bidder, i.e., 
a bidder whose value for the grand bundle is at least $\nicefrac{OPT}{100}$. 
The existence of a critical bidder $i$ implies that allocating $i$ the grand bundle gives a $\nicefrac{1}{100}$-approximation to the optimal welfare.  Since we run an ascending auction on the grand bundle with probability $\nicefrac{1}{2}$ we obtain a $\nicefrac{1}{200}$-approximation  
in this case.


We now turn to the case that there does not exist a critical bidder.
%, i.e., all bidders contribute relatively little to the optimal solution.  
In this case, Lemma \ref{lem:auction-sampling} implies that with probability $\nicefrac{1}{2}$ over the random sampling of bidders,
the optimal welfare achievable by the sampled set is within a factor $5$ of the optimal welfare.
%and the optimal welfare achievable by the unsampled set is within a factor $5$ of the optimal welfare.
As such when we proceed to the pricing phase, we set a price per item $p \in [\nicefrac{\text{OPT}}{50m}, \nicefrac{\text{OPT}}{10m}]$. Since we run a uniform price auction with probability $\nicefrac{1}{2}$, these conditions hold simultaneously with probability at least $\nicefrac{1}{4}$.
We perform case analysis on the number of items sold during this phase. 
%\shirinote{If I understand you correctly, only the sampled set affects $p$. So I don't understand why the following phrase is relevant at this point -  "and the optimal welfare achievable by the unsampled set is within a factor $5$ of the optimal welfare."} \note{We'll need both to be accurate later to ensure that we don't have ``nothing left'', i.e., there is a lot of welfare for us to collect in the demand query phase.  But, yes, at this point the only relevant piece is that the sampled set is reasonably close to OPT.} \shirinote{Yep, my point was that it should be said later :)}

Suppose the uniform pricing phase sells at least $\nicefrac{m}{2}$ goods.  In this case, since an unsampled bidder buying $t$ goods spends at least $\frac{t\text{OPT}}{50m}$, their value for the purchased bundle is at least $\frac{t\text{OPT}}{50m}$.  
Then, the total value of all bidders who purchase goods is at least $\frac{m}{2}\cdot\frac{\text{OPT}}{50m} = \frac{\text{OPT}}{100}$. Altogether, since we run uniform sampling with probability $\nicefrac{1}{2}$ and the estimation is
\textquote{good} with probability $\nicefrac{1}{2}$, we obtain  a $400$-approximation to the welfare.

% Now suppose the uniform pricing phase sells fewer than $m/2$ goods.  In this case, observe that any bidder who wanted strictly more goods than were available when they were approached to make a demand query wanted to purchase strictly more than $m/2$ goods at a price of at least $\frac{\text{OPT}}{50m}$ and thus had a value of at least $\frac{\text{OPT}}{100}$.  Since we assume, however, that there are no critical bidders, it must be that all bidders had enough items available to satiate them when they were asked to purchase their demand set.  
% Note, however, that even if we sold the entire grand bundle of items to bidders who purchased no items, they would, in total, produce welfare at most $\frac{\text{OPT}}{50}$ (since we would allocate at most $m$ goods at average values less than $\frac{\text{OPT}}{50m}$ each).  As such, the portion of the optimal value we lost from setting prices too high was at most $\frac{\text{OPT}}{50}$.  However, we assumed that the total optimal welfare within the ``unsampled'' set was at least $\frac{\text{OPT}}{5}$ and as such, the bidders who did purchase goods must have total value at least $\frac{\text{OPT}}{5} - \frac{\text{OPT}}{50} \geq \frac{\text{OPT}}{10}$.  In either case, we obtain welfare at least $\frac{\text{OPT}}{100}$ if we run the uniform pricing auction when there does not exist a critical bidder.  Since we run this auction with probability $1/2$, we thus obtain a $1/200$-approximation to the optimal welfare in the case that there does not exist a critical bidder.  As such, we may conclude by observing that our auction obtains a $1/200$-approximation to the optimal welfare whether a critical bidder exists or not, completing the proof.

% \src{I'm rewriting the end of the proof starting from "Now suppose" to better understand it.}

Now, we analyze the complementary case 
%the more challenging case that 
where the uniform pricing phase sells fewer than $\nicefrac{m}{2}$ goods. For that, let $\vec{q}=(q_1,\dots,q_n)$ be the optimal allocation if the items are divided only among the bidders in $U$ (clearly, every bidder not in $U$ is allocated zero items).  
%\sre{TODO - specify at the beginning of the phrase that a bidder might not be allocated for two reasons: being blocked or having too small value compared to the price. "We define two kinds of bidders" does not capture that well enough, I think, and I think that it will be clearer to say that in the beginning and not in the end.}

For that, observe that if  a bidder is allocated in $\vec q$ but not allocated in the allocation of the algorithm, it necessarily happens because of one of the following reasons. The first possibility is that the bidder is \emph{blocked}, meaning that   the number of items that she wants $d_i$ is not available when it is her turn. The other reason is that the bidder is \emph{small}, meaning that  $v_i(d_i)\le p\cdot d_i$, i.e., 
$d_i$ items are available  but
the 
%\pinksout{uniform} 
price set is too high for her. 
% Now, we define two kinds of bidders that are allocated in $\vec{q}$ but not allocated in the allocation of the algorithm. Bidder $j\in U$ is \emph{blocked} if the number of items that she wants is not available when it is her turn. Bidder $j\in U$ is \emph{small} if $v_i(q_i)\le p\cdot q_i$, i.e. the 
%\pinksout{uniform} 
% price is too high for her. 
Note that every bidder $i$ that 
is neither blocked nor small, is also satisfied in the algorithm.  

We will bound the loss of welfare from both kinds of bidders conditioned on our assumptions
of running a uniform price auction and having a ``balanced''  sampling  (i.e., both $OPT(S) \geq OPT/5$ and $OPT(U) \geq OPT/5$).
% which, by \cref{lem:auction-sampling} occurs with probability at least $\nicefrac{1}{4}$.
First, we show that blocked bidders do not exist, so they do not cause any loss of welfare. 
%we do not lose any welfare from not allocating to them:
For that, we remind that by assumption the uniform phase sells less than $\frac
m 2$ items. Thus, a  blocked bidder wants to 
%Any bidder who wanted strictly more goods than were available when they were approached to make a demand query wanted to 
purchase strictly more than $\frac m 2
$ goods at a price of at least $\frac{\text{OPT}}{50m}$ and thus has a value of at least $\frac{\text{OPT}}{100}$, meaning that she is critical.  By assumption, there are no critical bidders, which implies that there are no blocked bidders, so they incur no loss of welfare.

We will now bound the welfare that comes from small bidders in $\vec q$. Note that the price $p$ that we set per item is at most $\nicefrac{OPT}{10m}$ and that the number of items allocated to small bidders in $\vec q$ is at most $m$. Since by definition $v_i(d_i)\le p\cdot d_i$, those bidders contribute to the welfare of $\vec q$ at most
% the welfare of all those bidders together is at most 
$\frac{OPT}{10}$. 
% observe the definition of small bidders combined with the fact that  the sum of items allocated to small bidders in $\vec q$ is at most $m$ and that $p$
% since , we have that their total welfare in the  allocation $\vec q$ is at most $\frac{OPT}{10}$. 
Since the welfare of $\vec{q}$ is at least $\frac{OPT}{5}$,  bidders who are neither blocked or small contribute at least $\frac{OPT}{10}$ to the welfare of $\vec q$. Since 
Mechanism \ref{alg:single-minded} allocates to these bidders their desired  number of items, it achieves welfare of at least $\frac{OPT}{10}$. 
As we said before, this 
% happens with probability $\frac{1}{4}$,
depends on finding a \textquote{good} partition of $U,S$ and running a uniform price auction which occurs in probability $\nicefrac{1}{4}$, 
so overall the expected welfare of the mechanism in this case is at least $\frac{OPT}{40}$. 

Combining all cases, we conclude that the expected welfare of Mechanism \ref{alg:single-minded} is at least $\frac{OPT}{400}$, thereby completing the proof.
\end{proof}
% {Before describing the lower bound for bidders with single-minded valuations,} 
We note that
Mechanism \ref{alg:single-minded} also achieves a constant-approximation for bidders with 
% the case that bidders have  
{decreasing marginal valuations}. We discuss this in greater detail in \cref{subsec::mua-decreasing}.

% I.e., in the case that each bidder $i$ has a concave, monotone valuation function $v_i : \mathbb{N} \rightarrow \mathbb{R}^{\geq 0}$ over all possible \emph{quantities} of items she could receive, Mechanism \ref{alg:single-minded} gives a constant approximation.  


\subsubsection{Lower Bound}\label{subsec-lbs-22-sm-mua}


\begin{theorem}\label{thm-mua-sm-lb}
For a multi-unit auction with $m\ge 2$ items and $n \ge 2$ unknown single-minded bidders,
no randomized mechanism that satisfies OSP, individual rationality and no negative transfers has approximation better than $\nicefrac{6}{5}$.
\end{theorem}
We note that that the proof uses a distribution of valuations that is based on the 
construction of \cite{Ron24}.  
\begin{proof}
We assume the domain $V_i$ of each bidder consists of single-minded valuations with values in  $\{0,1,\ldots,k^4\}$, where $k$ is an arbitrarily large number. 
Our example has only two bidders,  but it can be  extended to any number of bidders by adding bidders with the all-zero valuation. 

    % For the proof, 
To  use 
    our variant of Yao's principle, 
    % (which is stated in the full version),
    %full-version-change-tag%
we define a distribution $\mathcal D$ of valuation profiles
    and show that no deterministic mechanism that satisfies
    obvious strategy-proofness, individual rationality and no negative transfers with respect to $V=V_1\times V_2$ has approximation better than $\frac{6}{5}$ in expectation over $\mathcal D$. 
    To define it, 
consider the following valuations:  
\[
\renewcommand{\arraystretch}{1}
\begin{aligned}
v_i^{\text{one}}(x) &= \begin{cases}
1 & x \geq 1,\\
0 & \text{else.}
\end{cases}
\quad
v_i^{\text{ONE}}(x) = \begin{cases}
k^2 + 1 & x \geq 1,\\
0 & \text{else.}
\end{cases}\\[4pt]
v_i^{\text{all}}(x) &= \begin{cases}
k^2 & x=m,\\
0 & \text{else.}
\end{cases}
\quad
v_i^{\text{ALL}}(x) = \begin{cases}
k^4 & x=m,\\
0 & \text{else.}
\end{cases}
\end{aligned}
\]
% \[
% v_i^{\text{one}}(x) = 
% \begin{cases}
% 1 & x \geq 1,\\
% 0 & \text{else.}
% \end{cases}
%  \hspace{0.25em}
% v_i^{\text{ONE}}(x) = 
% \begin{cases}
% k^2 + 1 & x \geq 1,\\
% 0 & \text{else.}
% \end{cases}
% \]
% \[
% v_i^{\text{all}}(x) = 
% \begin{cases}
% k^2 & x=m,\\
% 0 & \text{else.}
% \end{cases}
%  \hspace{0.25em}
% v_i^{\text{ALL}}(x) = 
% \begin{cases}
% k^4 & x=m,\\
% 0 & \text{else.}
% \end{cases}
% \]
Consider the following valuation profiles:
\[
\begin{aligned}
& I_1 = (v_1^{\text{one}}, v_2^{\text{one}}) \quad 
I_2 = (v_1^{\text{all}}, v_2^{\text{one}}) \quad 
I_3 = (v_1^{\text{ONE}}, v_2^{\text{ALL}}) \quad 
I_4 = (v_1^{\text{one}}, v_2^{\text{all}}) \quad
I_5 = (v_1^{\text{ALL}}, v_2^{\text{ONE}})
\end{aligned}
\]
% $I_1 = (v_1^{\text{one}}, v_2^{\text{one}})$, 
% $I_2 = (v_1^{\text{all}}, v_2^{\text{one}})$,
% $I_3 = (v_1^{\text{ONE}}, v_2^{\text{ALL}})$,
% $I_4 = (v_1^{\text{one}}, v_2^{\text{all}})$,
% and $I_5 =(v_1^{\text{ALL}}, v_2^{\text{ONE}})$.
Denote with $\mathcal D$ the distribution over 
%valuation 
profiles where the probability of  $I_1$ is $\frac{1}{3}$, and the probability of $I_2,I_3,I_4$ and $I_5$ is $\frac{1}{6}$ each. 
Observe that:
%Our goal is to show that no deterministic mechanism that satisfies all of the desired properties extracts more than $\frac{5}{6}$ of the optimal welfare. For that, we begin by making the following observation:
\begin{claim}\label{claim-mua-sm-instances}
Every deterministic mechanism that has approximation strictly better than $\nicefrac{6}{5}$ necessarily satisfies all of the following conditions:
\begin{enumerate}
    \item Given the valuation profile $I_1=(v_1^{one},v_2^{one})$, the mechanism
    allocates at least one item to every bidder. \label{condi-1}
    % $A$ outputs an allocation with welfare at most $1$. 
    \item Given the valuation profile $I_2 = (v_1^{\text{all}}, v_2^{\text{one}})$, the mechanism allocates all items to bidder $1$.
    \label{condi-3}
\item Given the valuation profile $I_3 = (v_1^{\text{ONE}}, v_2^{\text{ALL}})$, the mechanism allocates all items to bidder $2$. \label{condi-2}
\item Given the valuation profile $I_4 = (v_1^{\text{one}}, v_2^{\text{all}})$, the mechanism allocates all items to bidder $2$. 
\item Given the valuation profile $I_5 =(v_1^{\text{ALL}}, v_2^{\text{ONE}})$, the mechanism allocates all items to bidder $1$. \label{condi-5}
\end{enumerate}
\end{claim}
The proof of \cref{claim-mua-sm-instances} is straightforward: if a deterministic mechanism violates one of the conditions, then since $k$ is arbitrarily large, then it  extracts at most $\frac{5}{6}$ of the optimal welfare in expectation over the distribution $\mathcal D$. 


Fix a deterministic mechanism $A$ and strategies
$(\mathcal S_1,\mathcal S_2)$ that are individually rational and satisfy no negative transfers with respect to the valuations $V_1\times V_2$ 
and give approximation better than $\frac{6}{5}$
in expectation over the valuation profiles in the distribution $\mathcal D$. 
Let  $(f,P_1,P_2)$ be the allocation and payment rules that the mechanism $A$ and the strategies $(\mathcal S_1,\mathcal S_2)$ jointly realize. Assume towards a contradiction that $A$ and $(\mathcal S_1,\mathcal S_2)$ are OSP. 



To analyze the mechanism, we focus on the following subsets of the domains of the valuations:
$
\mathcal{V}_1=\{v_1^{one},v_1^{ONE},v_1^{ALL}\}$ and  $\mathcal{V}_2=\{v_2^{one},v_2^{ONE},v_2^{ALL}\}$.\footnote{The cautious reader may have noticed that $\mathcal V_i$ does not contain  $v_i^{all}$. This is intentional, and it will be clear from the remainder of the proof why including this valuation is not necessary.} 
We begin by observing that there necessarily exists a vertex $u$, and valuations $v_1,v_1' \in \mathcal{V}_1$, and  $v_2,v_2' \in \mathcal{V}_2$ such that $(\mathcal{S}_1(v_1), \mathcal{S}_2(v_2))$ diverge at vertex $u$. This follows from \cref{claim-mua-sm-instances}, which implies that the mechanism $A$ must output different allocations for different valuation profiles in $\mathcal{V}_1 \times \mathcal{V}_2$. Consequently, not all valuation profiles end up in the same leaf, meaning that divergence must occur at some point. 
% given the valuation profiles $I_1=(v_1^{one},v_2^{one})$ and
% either bidder $1$ or $2$ has to send different messages for different valuations in $\mathcal V_i$ at some vertex. This is an immediate implication of \cref{claim-mua-sm-instances}, as 
% the mechanism $A$ necessarily outputs different allocations given the valuation profiles $I_1=(v_1^{one},v_2^{one})$ and
%  $I_2=(v_1^{all},v_2^{one})$, meaning that the behaviors 
%  % \dnote{Missing parentheses here?}
%  $(\mathcal S_1(v_1^{one}),\mathcal S_2(v_2^{one}))$ and $(\mathcal S_1(v_1^{all}),\mathcal S_2(v_2^{one}))$ reach different leaves and thus have to diverge at some point.  
 
 Let $u$ be the first vertex in the protocol such that 
 %the behavior profiles 
 $(\mathcal{S}_1(v_1),\mathcal{S}_2(v_2))$ and $(\mathcal{S}_1(v_1'),\mathcal{S}_2(v_2'))$ diverge, i.e., dictate different messages. 
Note that by definition this implies that $u\in Path(\mathcal{S}_1(v_1),\mathcal{S}_2(v_2))\cap Path(\mathcal{S}_1(v_1'),\mathcal{S}_2(v_2'))$ and that either bidder $1$ or bidder $2$ sends different messages for the valuations in $\mathcal{V}_1$ or $\mathcal V_2$, respectively. 
Without loss of generality, we assume that bidder $1$ sends different messages, meaning that there exist $v_1,v_1'\in \mathcal{V}_1$ such that $\mathcal S_1(v_1)$ and $\mathcal S_1(v_1')$ dictate different messages at vertex $u$.  
We remind that $\mathcal{V}_1=\{v_1^{one},v_1^{ONE},v_1^{all}\}$, so the  following claims jointly imply a contradiction, completing the proof: 
\begin{claim}\label{claim-oneone-same}
    The strategy $\mathcal S_1$ dictates the same message at vertex $u$ for the valuations $v_1^{one}$ and $v_1^{ONE}$. 
\end{claim}
\begin{claim}\label{claim-ONE-ALL-same}
        The strategy $\mathcal S_1$ dictates the same message at vertex $u$ for the valuations $v_1^{ONE}$ and $v_1^{ALL}$.
\end{claim}
% We defer the proofs of \cref{claim-oneone-same,claim-ONE-ALL-same} to the full version.
% \cref{subsec:mua-sm-claims-proofs}. 
We include the proofs for the sake of completeness, {but note that} they are identical to the proofs provided in \cite{Ron24}.
% The proofs are are based on the properties of the mechanism: its approximation guarantee, obvious strategy-proofness, individual rationality and no-negative-transfers.  
The proofs make use of the following lemma, which is a collection of observations about the allocation and the payment scheme of player $1$:    
\begin{lemma}\label{lemma-small-pay}
    The allocation rule $f$ and the payment scheme $P_1$ of bidder $1$ satisfy that:
    % Let $f$ be an allocation rule and let $P_1$ be the payment scheme of bidder $1$ that 
    % The allocation rule $f$ and the  payment scheme $P_1$ $(f,P_1,\ldots,P_n):V_1\times \cdots \times V_n\to \mathbb{R}^{n}$
    % are realized by a dominant-strategy, individually rational and no-negative-transfers mechanism. Then: 
    \begin{enumerate}
        \item Given $(v_1^{one},v_{2}^{one})$, bidder $1$ wins at least one item and pays at most $1$.  \label{item-1}
        \item  Given $(v_1^{ONE},v_2^{ALL})$, bidder $1$ gets the empty bundle and pays zero.   \label{item-2}
        \item Given $(v_1^{ALL},v_2^{one})$, bidder $1$ wins all the items and pays at most $k^2$. \label{item-3}  
    \end{enumerate}
\end{lemma}
The lemma is a direct consequence of the approximation guarantees of the mechanism, together with the fact that it is obviously strategy-proof and satisfies individual rationality and no negative transfers.  We use \cref{lemma-small-pay} 
 now and defer the proof to \cref{subsec::proof-lemma-small-pay}.
\begin{proof}[Proof of \cref{claim-oneone-same}]
     Note that by Lemma \ref{lemma-small-pay} item \ref{item-1}, $f(v_1^{one},v_2^{one})$ allocates at least one item to player $1$ and $P_1(v_1^{one},v_2^{one})\le 1$. Therefore:
\begin{equation}\label{eq-good-leaf1}
 v_1^{ONE}(f(v_1^{one},v_2^{one}))-P_1(v_1^{one},v_2^{one})\ge k^2   
\end{equation}
 In contrast, by part \ref{item-2} of Lemma \ref{lemma-small-pay},   $f(v_1^{ONE},v_2^{ALL})$ allocates no items to player $1$ and $P_1(v_1^{ONE},v_2^{ALL})=0$, so:
 \begin{equation}\label{eq-bad-leaf1}
 v_1^{ONE}(f(v_1^{ONE},v_2^{ALL}))-P_1(v_1^{ONE},v_2^{ALL})= 0   
\end{equation}
Combining inequalities (\ref{eq-good-leaf1}) and (\ref{eq-bad-leaf1}) gives:
\begin{align*}
  v_1^{ONE}(f(v_1^{ONE},v_2^{ALL}))-P_1(v_1^{ONE},v_2^{ALL})< 
  v_1^{ONE}(f(v_1^{one},v_2^{one}))-P_1(v_1^{one},v_2^{one})  
\end{align*}
We remind that vertex $u$ belongs in $Path(\mathcal S_1(v_1^{one}),\mathcal S_2(v_2^{one}))$ and also in
$Path(\mathcal{S}_1(v_1^{ONE}),\allowbreak\mathcal{S}_2(v_2^{ALL}))$. Therefore, Lemma \ref{lemma-bad-leaf-good-leaf} gives that the strategy $\mathcal S_1$ dictates the same message for  $v_1^{one}$ and $v_1^{ONE}$ at vertex $u$. 
% See Figure \ref{subfig-1} for an illustration.
\end{proof}
\begin{proof}[Proof of \cref{claim-ONE-ALL-same}]
    Following the same approach as in the proof of Claim \ref{claim-oneone-same}, note that by \cref{lemma-small-pay} \cref{item-3}: 
\begin{equation}\label{break-align}
v_1^{ONE}(f(v_1^{ALL},v_2^{one}))-P_1(v_1^{ALL},v_2^{one}) \ge k^2+1-k^2     
\end{equation}
Also, by \cref{lemma-small-pay}  \cref{item-2}:
\begin{equation}\label{break-align2}
  v_1^{ONE}(f(v_1^{ONE},v_2^{ALL}))  
-P_1(v_1^{ONE},v_2^{ALL})=0   
\end{equation}
Combining \cref{break-align} and \cref{break-align2}:
\begin{equation*}
    v_1^{ONE}(f(v_1^{ONE},v_2^{ALL}))  
-P_1(v_1^{ONE},v_2^{ALL})<  v_1^{ONE}(f(v_1^{ALL},v_2^{one}))-P_1(v_1^{ALL},v_2^{one})
\end{equation*}
% where the first inequality is by Lemma \ref{lemma-small-pay} part \ref{item-3} and the equality is by Lemma \ref{lemma-small-pay} part \ref{item-2}.
Given the above inequality with the fact that 
vertex $u$ belongs in $Path(\mathcal S_1(v_1^{ALL}),\mathcal S_2(v_2^{one}))$ and in
$Path(\mathcal{S}_1(v_1^{ONE}),\mathcal{S}_2(v_2^{ALL}))$,
Lemma \ref{lemma-bad-leaf-good-leaf}
implies  
% applying Lemma \ref{lemma-bad-leaf-good-leaf} gives
that the strategy $\mathcal S_1$ dictates the same message for the valuations $v_1^{ONE}$ and $v_1^{ALL}$ at vertex $u$. 
\end{proof}


\end{proof}


\subsection{Decreasing Marginal Valuations}\label{subsec::mua-decreasing}
In this section, we consider valuations that exhibit decreasing marginals. 
A valuation $v:[m]\to \mathbb R$  has decreasing marginals if  for every quantity $j\in [m]$, $v(j)-v(j-1)\ge v(j+1)-v(j)$. 
This is the only class of multi-parameter valuations for which the power of deterministic obviously strategy-proof mechanisms is not yet understood: 
% We note that for deterministic obviously strategy-proof mechanisms, 
the best-to-date mechanism achieves an $\mathcal O(\log n)$ approximation, and no mechanism for two bidders and two items obtains approximation better than $\sqrt 2$ \cite{GMR17}. We begin by showing that if we allow randomization:
\begin{theorem}\label{thm:decreasing-marginals}
There exists a randomized obviously strategy-proof mechanism that achieves a $400$\allowbreak-approximation to the optimal social welfare for bidders with decreasing marginal valuations.
\end{theorem}
We note that the mechanism described in the proof of \cref{thm:decreasing-marginals} corresponds to Mechanism \ref{alg:single-minded}, which was previously introduced for the class of single-minded bidders. The proof of \cref{thm:decreasing-marginals} is deferred to \cref{subsec::proof--dec-mua}.

Having established a constant-factor randomized obviously strategy-proof mechanism for decreasing marginal valuations, a natural question arises: is this result tight? Specifically, can we establish impossibility results for this class? To further deepen our understanding of this class of valuations, we now describe a phenomenon that highlights the challenges in proving such impossibilities.

\subsubsection{A Non-Monotonicity Effect for Bidders with Decreasing Marginal Values}\label{subsub::non-mono}
In this section, we describe a non-monotonicity phenomenon that occurs for obviously strategy-proof mechanisms in multi-unit auction with decreasing marginal valuations.  In contrast to the rest of the paper, we focus on deterministic mechanisms rather than randomized ones. 
% Currently, this is the only multi-parameter setting for which we do not know the exact approximation ratio of deterministic obviously strategy-proof mechanisms \cite{Ron24,GMR17}.\footnote{There is an obviously strategy-proof mechanism that gives $O(\log n)$ approximation, and no deterministic mechanism gives approximation better than $\sqrt 2$ \cite{GMR17}.} 
The phenomenon is that    \emph{deterministic} obviously strategy-proof mechanisms for bidders with decreasing marginal valuations, 
 adding an item improves the approximation power:
\begin{theorem}\label{thm-lb-mua-dec}
    For $2$ bidders and $2$ items, no obviously strategy-proof \emph{deterministic} mechanism that satisfies individual rationality and no negative transfers gives approximation better than $2$.  
\end{theorem}

\begin{lemma}\label{lemma:mono-mua-dec}
        For $2$ bidders and $3$ items, there is an obviously strategy-proof 
\emph{deterministic} mechanism that gives an approximation of  $1.5$.
\end{lemma}


Observe that this is not typical, as the approximation guarantee of mechanisms typically deteriorates as the number of items increases: intuitively, the more items there are, the \textquote{harder} it becomes to allocate them optimally. From a more formal perspective, impossibility results for auctions with $m$ items can be extended to those with $m+1$ items by introducing a \textquote{dummy} item that no bidder values. However, for the class of valuations with decreasing marginals, the additional item enables a new mechanism: we can now 
 allocate one item to each bidder and then run an ascending auction on the remaining item. 

% However, when considering \emph{deterministic} obviously strategy-proof mechanisms for the bidders that have valuations with decreasing marginals, adding an item improves the approximation power:

Note that the previously known lower bound on deterministic obviously strategy-proof mechanisms is $\sqrt 2$ \cite{GMR17}.\footnote{Note that \cite{GMR17} actually prove an impossibility for all mechanisms that are weakly group strategy-proof. This implies an impossibility for obviously strategy-proof mechanisms because as \cite{li} shows, every obviously strategy-proof mechanism is weakly group strategy-proof.} However, in contrast to \cite{GMR17}, \cref{thm-lb-mua-dec} applies solely to mechanisms that satisfy individual rationality and no negative transfers.   
The proofs  of \cref{thm-lb-mua-dec} and \cref{lemma:mono-mua-dec} can be found in \cref{subsec-lb-proof-mua-dec,sec-impos-mua-dec}.  

 % The proof can be found in  




\section{Combinatorial Auctions}\label{sec-combi}
We now turn to settings with heterogeneous items.
We explore settings involving additive and unit-demand bidders and conclude by considering mechanisms for subadditive and general valuations.
The proofs of all theorems can be found in 
%the full version.
\cref{app-missing-combi}. 
%full-version-change-tag%
\subsection{Additive Valuations}
A valuation $v_i$ is \emph{additive} if bidder $i$ has a value $v_{ij} \geq 0$ for item $j$ and the value bidder $i$ has for receiving a set of items $A_i$ is equal to $\sum_{j \in A_i}{v_{ij}}$. 
\subsubsection{Upper Bound}
% \sre{Notes to ourselves:
% \begin{enumerate}
%     \item Shahar mentioned in a meeting that we can probably improve the constant by sampling $\frac{1}{e}$ of the bidders. This is a reminder for us to check this out at some point.
%     \item A way to solve the tie breaking without loss in the approximation factor: Do the \textquote{"Rebecca Way"}: set an arbitrary ranking on the bidders, and allocate an item to a bidder only if either her value is strictly higher than the price or her value is equal to the price and this bidder is ranked higher than the bidder who set the price. 
%     \end{enumerate}
%     }
We show that the sampling approach yields a $4$-approximation for this setting:
\begin{theorem}\label{thm-ub-add}
    There is a universally OSP mechanism for bidders with additive valuations that gives a $4$-approximation to the optimal welfare.
\end{theorem}
We now describe Mechanism \ref{alg:additive}. Simply put, Mechanism \ref{alg:additive} samples a threshold price for each item and then uses these threshold prices as a posted-price mechanism for the unsampled bidders. To handle tie-breaking, priority is given to bidders with higher indices. 
% \cref{app-missing-combi}.
% that it is universally OSP and gives a $4$-approximation to the optimal welfare. 
\begin{algorithm2e}
%\setstretch{1.1}
\SetKwInOut{Input}{Input}
\Input{A set of bidders $\bidders$ and a set of $M$ items}

 Index the bidders in some arbitrary fixed order


 $S \leftarrow \emptyset$, $U \leftarrow \emptyset$
 
Independently assign each bidder to set $S$ with probability $\nicefrac{1}{2}$ and to set $U$ with probability $\nicefrac{1}{2}$.

 ``Discard'' each bidder in $S$ and learn their value for each item

 For each $j \in M$: set a price $p_j$ on item $j$ equal to $\max_{i \in S}{v_{ij}}$ and let $n(j)$ denote the smallest index among bidders in $\arg\max_{i \in S}{v_{ij}}$
 
 For each $i \in U$ in an arbitrary order:  Let $i$ purchase all previously unsold items $j \in M$ for which either: (i) $v_{ij} > p_j$; or (ii) $v_{ij} = p_j$ and $i$ has a lower index than $n(j)$
 
 \caption{``\textsc{Additive}''}
 \label{alg:additive}
\end{algorithm2e}
The following two lemmata jointly provide the proof of \cref{thm-ub-add}.
% We show the following two lemmata below, which jointly provide the proof of \cref{thm-ub-add}. 
\begin{lemma}\label{lemma-add-osp}
Mechanism \ref{alg:additive} is universally OSP.
\end{lemma}
The proof of \cref{lemma-add-osp} is straightforward, and we write it for completeness in \cref{subsec::proof-add-osp}. 


\begin{lemma}\label{lem-add-approx}
Mechanism \ref{alg:additive} obtains a $4$-approximation to the optimal social welfare in the presence of additive bidders.
\end{lemma}
\begin{proof}
Observe that since the valuation functions are additive,  optimal solutions must allocate each item $j$ to some bidder $i \in \text{argmax}_{i \in \bidders}{v_{ij}}$.  In particular, one optimal solution allocates each item $j$ to the bidder $i^*_j \in \text{argmax}_{i \in \bidders}{v_{ij}}$ with the smallest index according to the order over the bidders that the mechanism specifies.  We argue that each item $j$ is allocated to its corresponding bidder $i^*_j$ by Mechanism \ref{alg:additive} with probability at least $\frac 1 4$. Linearity of expectation directly implies a $4$-approximation to the optimal welfare.

To see that we allocate each item $j$ to
$i^*_j$
% an optimal bidder 
with probability $\frac 1 4$, first observe that $i^*_j$ is placed in $U$ with probability $\frac 1 2$.  Moreover, let $\tilde{i}_j$ denote the \textquote{second-highest bidder}, which we define as
the bidder in $\text{argmax}_{i \in \bidders \setminus \{i^*_j\}}{v_{ij}}$
that has the smallest index.
 This bidder is placed in $S$ (independently of the placement of $i^*_j$) with probability $1/2$.  
Observe that when bidder $\tilde{i}_j$ is in $S$ and $i^*_j$ is in $U$,
then item $j$ is necessarily available for bidder $i^*_j$ who gets it indeed. 
This occurs with probability at least $\frac{1}{4}$, as desired.\qedhere
\end{proof}

% \sre{Simply put, Mechanism \ref{alg:additive} generates a threshold price for each item and then applies these thresholds as posted prices for the bidders who were not sampled. We prioritize higher-indexed bidders to tie breaking issues.}

%The correctness is based on the fact that for additive valuations, allocating each item to the highest bidder is optimal. 
%The added layer of allowing bidders to take items only if their threshold exceeds the threshold of the bidders 


% Let $m$ denote the number of items and $n$ denote the number of agents. 
%Consider the mechanism $\mathcal{M}_4$ which independently for every bidder $i$ with probability $1/2$ marks $i$ as a ``sampled'' bidder (she is marked ``unsampled'' otherwise).  The sampled bidders are then eliminated from winning the items and the mechanism learns their valuation functions (using polynomially many value queries).  $\mathcal{M}_4$ then assigns a price to each item equal to the highest value for that item among the sampled bidders.  Finally, with probability $1/3$ we allow each unsampled bidder to claim her maximum sized demand set at the prices set by the sampling phase, and with the remaining $2/3$ probability we allow each bidder, in an arbitrary order to claim her minimum sized demand set.


\subsubsection{Lower Bound}
\begin{theorem}\label{thm-lb-add}
For a combinatorial auction with $m\ge 2$ items and $n\ge 2$ additive bidders, there 
is no randomized obviously strategy-proof mechanism that satisfies individual rationality and no negative transfers and gives approximation better than $\frac{8}{7}$ to the optimal social welfare.
\end{theorem}
The proof follows the same structure as the proof of \cref{thm-mua-sm-lb} of describing 
a distribution $\mathcal D$ and showing that it is hard for every deterministic mechanism. By applying Yao's Lemma (\cref{lem:yaos}), we get hardness for randomized obviously strategy-proof mechanisms. 
The proof can be found in \cref{lb-add-proof-place}. 
We note that the construction of \cref{thm-lb-add} is similar to a construction in \cite{Ron24}.  However, the case analysis we employ is more involved as it includes additional valuation profiles. 


\subsection{Unit-Demand Valuations}
We now address bidders with \emph{unit-demand} valuations, where there is a value $v_{ij} \geq 0$ for each $i \in \bidders$ and  $j \in \items$ and the value bidder $i$ has for a set $A_i$ is equal to $\max_{j \in A_i}{v_{ij}}$. 



% Consider the following example that demonstrates it. If the number of items $m$ is equal to the number of bidders $n$, and there 
% Assume that there are two bidders that have value $1$ for all the items, and the rest of the bidders have values $1-\epsilon$ for all the items. Then, the price 
% \begin{example}
%     The following example demonstrates why  unit demand bidders are harder than additive bidders. For additive bidders, if we would have had an oracle that says to us for every item $j$, $\max_i{v_i(j)}$, i.e. the highest value for this item, then by setting the price to be $p_j=\max_i{v_i(j)}$ for every item $j$ would have give us the optimal allocation (up to tie breaking). 

%     However, this approach gives a bad approximation for unit-demand bidders. Assume that bidder $1$ has value $1$ for all the items, and the rest of the bidders have values $1-\epsilon$ for all the items. Then, setting those prices gives only an $n$-approximation.   
% \end{example}
% \shirinote{For some reason after writing it down this example felt less compelling to me, maybe I missed some details? Anyhow, hopefully it will be handy in the writeup!}

% Without loss of generality, we may assume that the optimal solution is a matching of items to bidders.  For convenience we denote the optimal matching $\mu^*$ and the item matched to bidder $i$ in the optimal matching as $\mu^*(i)$ whereas we denote the bidder matched to item $j$ in the optimal matching as $\mu^{-*}(j)$ (e.g., $\mu^*$ is a function from bidders to items and $\mu^{-*}$ is its inverse).

% \shirinote{do we want to keep \cref{claim-ud-1} and \cref{claim-ud-2}?}
% \begin{claim}\label{claim-ud-1}
% Consider an instance with unit demand bidders and its optimal solution $\mu^*$.  Consider setting a price of $\frac{v_{\mu^{-*}(j)}(j)}{10}$ for each item $j$.  If bidders are queried in an arbitrary order and purchase arbitrary items in their demand sets at these prices the resulting matching yields welfare within a constant factor of the optimum.
% \end{claim} 

% \begin{proof}
%     Consider the allocation produced by approaching the bidders in an arbitrary order and allowing them to purchase arbitrary items in their demand sets.  We consider separately the items allocated in the optimal allocation which are sold by the demand query process and those which are left unsold.  Consider an arbitrary item $j$ allocated in the optimal solution which is sold in the demand query process.  Since its price is $\frac{v_{\mu^{-*}(j)}(j)}{10}$ we have that the value of the bidder who purchases it is at least $1/10$ of the value of the bidder who receives it in the optimal solution.  Now consider an item $j$ which is allocated in the optimal solution which is left unsold in the demand query process.  Since it is unsold it was necessarily available for purchase by $\mu^{-*}(j)$ but this bidder selected a different good.  The utility that $\mu^{-*}(j)$ obtains in the demand query process is then at least $\frac{9v_{\mu^{-*}(j)}(j)}{10}$.  Finally, observe that each bidder buys at most one item in the demand query process and is allocated at most one item in the optimal solution.  As such, we may assign at most one sold item and one unsold item to each bidder and we obtain 
% \end{proof}

% \begin{claim}\label{claim-ud-2}
% Lower bound on ``accurately'' priced goods.
% \end{claim}
\subsubsection{Upper Bound}
This setting appears more complicated than the setting of additive valuations, as the approach of setting the price of each item to be the price of the second highest bid fails miserably:
\begin{example}\label{ex:ud-failure}
    Consider the instance where there are $\sqrt{n}$ \textquote{high} bidders with value $2$ for all items, and the rest of the bidders are \textquote{low}, in the sense that they value all items at $1$. Assume that the number of items, $m$, is equal to the number of bidders, $n$.
    
    Note that if we sample roughly half of the bidders and use their highest values to determine the prices for the unsampled bidders, as we
    do in Mechanism \ref{alg:additive},
    %did in the mechanism $\mathcal M_3$ 
    % used for additive valuations,
    we get $\approx\frac{1}{\sqrt n}$ of the optimal welfare. This is because at least one of the \textquote{high} bidders is in the sample with probability $1-\frac{1}{2^{\sqrt n}}$. Thus in this very likely case, the price of all items is set to be $2$ and none of the \textquote{low} bidders take any item, so the welfare obtained in expectation is at most $\sqrt{n}$. However, the  optimal welfare is $n+\sqrt{n}$. 
\end{example}
% (we provide an example that demonstrates it in the full version.)
%full-version-change-tag%
% (see \cref{ex:ud-failure} in Appendix \ref{app-missing-combi}).
Thus, to obtain a constant factor approximation for this setting,  we use the beautiful algorithm of \cite{reiffenhauser2019optimal}, originally formulated for the problem of strategy-proof online matching: 
\begin{algorithm2e}
%\setstretch{1.1}
\SetKwInOut{Input}{Input}
\Input{A set of bidders $\bidders$ and a set of  items $M$}

Ensure uniqueness of all optimal solutions for any fixed subset of bidders and items by fixing a consistent tie-breaking rule between optimal allocations

 Choose uniformly at random permutation $\sigma$ over the bidders and index the bidders in this order

 $S \leftarrow \emptyset$, $M_A \leftarrow M$
  
 ``Discard'' the first $\lfloor n/e \rfloor$ bidders and learn their value for each item and add these bidders to $S$

 For each consecutive bidder $i \in \{\lfloor n/e \rfloor + 1, \dots, n\}$:

    \quad Compute a price $p_j$ for each item $j \in M_A$ equal to $OPT(S, M_A) - OPT(S, M_A \setminus \{j\})$ (i.e., the decrease in welfare if $j$ were taken away from bidders in $S$)

    \quad Let $i$ purchase her favorite item (i.e., the item for which $v_{ij} - p_j$ is maximized and greater than $0$) at the current prices and let $j^i$ denote this item (if any). Use the tie breaking rule to determine whether player $i$ can take items for which $v_{ij}-p_j=0$

    \quad $M_A \leftarrow M_A \setminus \{j^i\}$, $S \leftarrow S \cup \{i\}$

    \quad Ask $i$ for her value of $i$ for each item
 
 \caption{``\textsc{Unit-Demand}'' (adapted from Algorithm 2 of \cite{reiffenhauser2019optimal})}
 \label{alg:unit-demand}
\end{algorithm2e}

We note that Mechanism \ref{alg:unit-demand} is Algorithm 2 of \cite{reiffenhauser2019optimal}, which we slightly adapt to our offline setting and rephrase to make the fact that the mechanism is universally obviously strategy-proof more clear. 


% By adapting the mechanism of \cite{reiffenhauser2019optimal}, we get: 
\begin{theorem}\label{thm:ud-upper}
Mechanism \ref{alg:unit-demand} is universally OSP and achieves an $e$-approximation
to the optimal social welfare in the presence of unit-demand bidders.
\end{theorem}
\begin{proof}
The approximation ratio of the mechanism directly follows from Theorem 1 of \cite{reiffenhauser2019optimal}.  To see that the mechanism is obviously strategy-proof, observe that the discarded bidders obtain no utility regardless of their report (and, thus, true value reporting is weakly obviously dominant). As for the remaining bidders, they get to select their most preferred remaining item. After that, their reported information does not affect their utility. Therefore, picking their favorite remaining item and then answer the queries afterwards truthfully is an obviously dominant strategy.   
\end{proof}

% The mechanism referred to in \cref{thm:ud-upper} is Mechanism \ref{alg:unit-demand} below.   






%We defer the proof to \cref{app-claims-proofs-ud}. 


% \begin{proof}
% Follows closely from mechanism in \citet{reiffenhauser2019optimal}. \sre{TODO - further elaborate.}
% \end{proof}
\subsubsection{Lower Bound}
\begin{theorem}\label{thm-lb-ud}
For a combinatorial auction with $m\ge 2$ items and $n\ge 2$ unit-demand bidders, there 
is no randomized obviously strategy-proof mechanism that satisfies individual rationality and no negative transfers and gives approximation better than $\frac{8}{7}$ to the optimal social welfare.
\end{theorem}
Similarly to \cref{thm-mua-sm-lb} and \cref{thm-lb-add}, the proof follows the structure of describing a distribution $\mathcal{D}$, proving that it is hard for every deterministic mechanism, and applying Yao's lemma (\cref{lem:yaos}). In fact, the proof closely resembles that of \cref{thm-lb-add} due to the fact that most valuations used in both constructions are simultaneously additive and unit-demand. The full proof is provided in \cref{lb-ud-proof-place}.



%The proof of Theorem \ref{thm-lb-ud} can be found in \cref{lb-ud-proof-place}.
% is extremely similar to the proof of \cref{thm-lb-add} and can be found in  
%We write them both for completeness. Note that the similarity holds because  most of the valuations used in both proofs are both unit-demand and additive. 
% can be found in \cref{lb-ud-proof-place}. It follows the same structure as the proof of \cref{thm-mua-sm-lb} of describing 
% a distribution $\mathcal D$ and showing that it is hard for every deterministic mechanism. \toedit{Also similarly, 
% we prove for the case of two bidders and two items, but the proof extends to any number of bidders and any number of items by adding bidders with the all-zero valuation and assume that the bidders in our construction have zero values for the additional items.} However, the case analysis we employ in this proof is significantly more involved.   

\subsection{More General Valuations}
In light of our previous results, one may wonder whether there exists a \textquote{rich enough} class of valuations for which randomized OSP mechanisms are provably unable to extract more than a constant fraction of the welfare. Perhaps surprisingly, the answer is no. However, the state-of-the-art \emph{computationally efficient} randomized mechanisms for subadditive and general valuations\footnote{A function $v$ is subadditive if for every two bundles of items $A, B \subseteq M$, it holds that $v(A \cup B) \leq v(A) + v(B)$. A valuation is general monotone if for every two bundles $A\subseteq B$, it holds that $v(B)\ge v(A)$.} are, in fact, universally OSP:
%(although neither mechanism is presented as such).  
%\subsection{Subadditive to Submodular valuations}
\begin{claim}\label{cl:subadditive}
    The $O((\log\log(m))^3)$-approximate randomized mechanism of \cite{assadi2021improved} for subadditive valuations is universally OSP.
\end{claim}
\begin{claim}\label{cl:general}
The $O(\sqrt{m})$-approximate randomized mechanism of \cite{dobzinski2012truthful} for general valuations is universally OSP.
\end{claim}
The proofs are straightforward and can be found in \cref{subsec::proofs-subadd-general}. 
% \subsection{General Valuations}
% \cite{DNS12} is obviously universally OSP, so if we don't improve it by the submission, we \sre{should} state for the class of general valuations and in particular for the class of unknown-single minded bidders we can get $\mathcal O(\sqrt{m})$ approximation.   




\bibliographystyle{alpha}
\bibliography{bibliography}

\clearpage

% \begin{center}
%     \LARGE\textbf{Supplemental Material: On The Power of Randomization for Obviously Strategy-Proof Mechanisms
%     }
% \end{center}
% \setcounter{page}{1}
\appendix

\section{Additional Formalities}\label{app-formalities}
We now provide a formal definition of obviously dominant strategies. 
and note that our definitions and setup closely follows \cite{Ron24}.  Fixing a protocol and behavior $B_i$ of bidder $i$ we say that vertex $u$ of the protocol is \emph{attainable} if there exists some $B_{-i}\in \mathcal{B}_{-i}$ (i.e., some profile of behaviors for the other players) such that $u\in Path(B_i,B_{-i})$. 
% For example, in the mechanism depicted in Figure \ref{fig-prems}, vertex $N_2$ is attainable given the behavior $(N_1:"1")$ of the sunglasses-wearing duck, whereas vertex $N_3$ is not attainable given this behavior. 
% For example,  given the mechanism described in Figure \ref{fig-prems}, the behavior profiles $B=\{B_s=(N_1:"2"),B_j=(N_2:"2",N_3:"2")\}$ and $B'=\{B_s'=(N_1:"2"),B_j'=(N_2:"1",N_3:"1")\}$ satisfy that  $Path(B)\cap Path(B')=\{N_1,N_3\}$. 
We now may formally define an \emph{obviously dominant behavior}:
\begin{definition}[Definition 2.1 in \cite{Ron24}]\label{def-obvs-behavior}
Consider a deterministic mechanism $A$, together with a behavior $B_i$ and a valuation $v_i$ of some player $i$.
Fix a vertex $u\in \mathcal N_i$ that is attainable given the behavior $B_i$. 
Behavior $B_i$ is an \emph{obviously dominant behavior for player $i$ at vertex $u$ given the valuation $v_i$} if for every behavior profiles $B_{-i}\in \mathcal B_{-i}$ and  $(B_1',\ldots,B_n')\in \mathcal B_1 \times \cdots \times \mathcal{B}_n$ such that:
\begin{enumerate}
    \item $u\in Path(B_1,\ldots,B_n)\cap Path(B_1',\ldots,B_n')$ \emph{and} 
    \item $B_i$ and $B_i'$ dictate sending different messages at vertex $u$.
\end{enumerate}
it holds that:
$$
v_i(f_i(B_i,B_{-i})) - p_i(B_i,B_{-i}) \geq v_i(f_i(B_i',B_{-i}')) - p_i(B_i',B_{-i}')
$$
\end{definition}

Note that \cref{def-obvs-behavior} only deals with behaviors at individual nodes in the protocol.  We then say that a behavior $B_i$ is an obviously dominant behavior given valuation $v_i$ if it is an obviously dominant behavior for player $i$ at
all  attainable vertices.  Formally:
\begin{definition}[Definition 2.2 in \cite{Ron24}]
Fix a behavior $B_i$ together with the subset of vertices in $\mathcal N_i$ that are attainable for it,  which we denote with $U_{B_i}$. Fix a valuation $v_i$ of player $i$.
The behavior $B_i$ is an \emph{obviously dominant behavior for player $i$ given the valuation $v_i$} if
it is an obviously dominant behavior for player $i$ given the valuation $v_i$ for every vertex $u\in U_{B_i}$. 
\end{definition} 

Finally, the definition of \emph{obviously dominant strategies} naturally follows.  Particularly, an obviously dominant strategy is one that defines an obviously dominant behavior for each possible valuation.  Formally:
\begin{definition}[Definition 2.3 in \cite{Ron24}]
A strategy $\mathcal{S}_i$ of player $i$ is an \emph{obviously dominant strategy} if for every $v_i$, the behavior $\mathcal S_i(v_i)$ is an obviously dominant behavior.
% for player $i$ given the valuation $v_i$. 
\end{definition}
 

\paragraph{Weak Monotonicity}
Weak monotonicity is a property of social choice functions.  To define it, we denote with $f_i$ the function that outputs for every player $i$ the bundle that she wins given $f$.

    An allocation rule $f:V\to\allocs$ is \emph{weakly monotone} if for every player $i$, for every valuation profile $v_{-i}$ of the bidders $N\setminus \{i\}$, and every two valuations $v_i,v_i'\in V_i$, it holds that if $f_i(v_i,v_{-i})=S$ and $f_i(v_i',v_{-i})=S'$, then $v_i(S)-v_i(S')\ge v_i'(S)-v_i'(S')$. 
 % The following well-known property of dominant-strategy mechanisms will be useful to us: 
    It is well known that:
\begin{lemma}\cite{BCLMNS06,LMN03}\label{wmon-lemma}
    Every allocation rule that is implemented by a dominant-strategy mechanism is weakly monotone. 
\end{lemma}

% \subsection{Classes of Valuations}\label{classes-of-valuations}
% \src{TODO}



\section{Improved Approximation Ratios For Two Agents and Two Items}
\label{sec-22}
In this section, we describe improved approximation guarantees for the special case of two bidders and two items. Observe that this is the simplest case for which the power of randomized obviously strategy-proof mechanisms is not resolved.\footnote{If we consider only one bidder, then a mechanism that always allocates to her all the items is optimal and obviously strategy-proof. If there is only one item, then an ascending auction on this item among the bidders is optimal and obviously strategy-proof.} We believe that understanding the power of randomized obviously strategy-proof mechanisms in this simple case is a first step towards understanding their power for the more general settings of arbitrary number of bidders and items. 

We consider three classes of valuations: multi-unit auctions with unknown single-minded bidders (\cref{subsec:22-mua-sm}), which admit a $\frac{4}{3}$-approximation, combinatorial auctions with subadditive bidders (\cref{subsec:22-ca-subadd}), which also admit a $\frac{4}{3}$-approximation, and combinatorial auctions with general bidders (\cref{subsec:22-ca-general}), which admit a $\frac{3}{2}$-approximation.

We remind the reader that for unknown single-minded bidders in a multi-unit auction, even with just two bidders and two items, no mechanism achieves an approximation better than $\frac{5}{6}$ (\cref{thm-mua-sm-lb}). Thus, there remains a gap of $\frac{1}{12}$ in our understanding of this setting.
Since unit-demand and additive valuations are subadditive, these classes also admit a $\frac{4}{3}$-approximation. As we have shown that even for two bidders and two items (\cref{thm-lb-ud,thm-lb-add}) no mechanism achieves an approximation better than $\frac{7}{8}$, there remains a gap of $\frac{1}{8}$.

In the proofs, we denote $v_i(j~|~j') = v_i(\{j,j'\}) - v(\{j'\})$ as the marginal gain in value of bidder $i$ when adding item $j$ to her bundle already containing item $j'$. We denote the optimal welfare with $\opt$. 

\subsection{Multi Unit Auction for Single-Minded Bidders with Unknown Demands} \label{subsec:22-mua-sm}
% \src{TO DO - mention that it is for unknown bundle sizes}
 % \paragraph{Single-minded, identical items, unknown bundle sizes}
Let $p = 1/2$ and consider the mechanism $\mathcal{M}_1$ which with probability $p$ runs a second-price auction for the grand bundle and with probability $(1-p)$ allocates a uniformly random agent a single item and runs a second-price auction for the second item.  
\begin{claim}\label{claim-mec-sm-mua-osp}
   The mechanism $\mathcal{M}_1$ is universally OSP.  
\end{claim}
\begin{proof}
Observe that $\mathcal{M}_1$ is a probability distribution over two deterministic mechanisms, both of which are generalized ascending auctions. By \cref{lemma-partial}, they are therefore obviously strategy-proof. Thus, $\mathcal M_1$ is obviously strategy-proof in the universal sense.
\end{proof}

\begin{claim}
The mechanism $\mathcal{M}_1$ achieves a $\nicefrac 4 3$-approximation to the optimal social welfare.
\end{claim}
\begin{proof}
    Let $v_i$ denote the private value of agent $i$ if she receives a bundle which satiates her demand. 
    
    First suppose that the optimal allocation obtains positive value from exactly one agent.  Without loss of generality, say that this is agent $1$.  We may immediately conclude that $v_1 \geq v_2$.  Consider that, by definition, agent $1$ is satiated by the grand bundle.  Thus, with probability $p$, the mechanism $\mathcal{M}_1$ satiates agent $1$ and receives a value $v_1$.  On the other hand, with probability $(1-p)/2$, the mechanism $\mathcal{M}_1$ allocates agent $1$ the first item and runs a second-price auction for the second item.  This necessarily yields total welfare at least $v_1$ since $v_1 \geq v_2$ and since agent $1$ will either have marginal value for the second item equal to $v_1$ if she is satiated only by both items or marginal value $0$ if she is satiated by a single item.  Thus, in the case where the optimal allocation serves exactly one agent, $\mathcal{M}_1$ achieves an approximation ratio of at least $(p + (1-p)/2)$.

    Now suppose that the optimal allocation obtains positive value from both agents.  Note that we may immediately conclude that each agent is satiated when she receives at least one item.  Without loss of generality, suppose that $v_1 \geq v_2$.  But then, with probability $p$ $\mathcal{M}_1$ allocates both items to bidder $1$ and obtains welfare $p\cdot v_1$.  On the other hand, in the case that $\mathcal{M}_1$ allocates a single item to a uniformly random agent and runs a second-price auction for the second item, we have that the agent who is randomly given the first item has $0$ marginal value for the second item whereas the other agent has positive marginal value for the item.  But then, in this case, which occurs with probability $(1-p)$, $\mathcal{M}_1$ obtains the optimal welfare.  As such, the expected welfare of $\mathcal{M}_1$ is $p \cdot v_1 + (1-p)\cdot (v_1 + v_2)$.  Since $v_1 \geq v_2$ we then obtain an approximation of $p/2 + (1-p)$.

    The approximation in the first case is monotonically increasing in $p$ whereas the approximation in the second case is monotonically decreasing.  Taking $1/2 + p/2 = p/2 + (1-p)$ gives $p = 1/2$ and thus setting $p=\nicefrac{1}{2}$ implies that $\mathcal M_1$ an approximation ratio of $\nicefrac 4 3$ as desired.
\end{proof}



\subsection{Combinatorial Auction with Subadditive Bidders}\label{subsec:22-ca-subadd}
Consider the mechanism $\mathcal{M}_2$ which uniformly at random selects a bidder $i$ and uniformly at random selects an item $j$ and allocates item $j$ to bidder $i$ and after this allocation occurs runs an ascending second-price auction for the remaining item.  Note that $\mathcal M_2$ is in fact a probability distribution over generalized ascending auctions, so by \cref{lem:auction-sampling} it is OSP in the universal sense.
We argue that it achieves a $\frac 4 3$-approximation for the broad class of \emph{monotone subadditive} valuations.  A function
$v:2^M\to \mathbb{R}^{\geq 0}$ is \emph{monotone} if for all bundles $T \subseteq T' \subseteq M$ we have that $v(T) \geq v(T')$.\footnote{Monotonicity is sometimes called ``free-disposal'' in the literature since a bidder is always weakly happier to receive more goods (because she can always ``dispose'' of goods she is unhappy to receive).}  A function $v:2^M\to \mathbb{R}^{\geq 0}$
is \emph{subadditive}  if for all bundles $T, T' \subseteq M$ we have that $v(T) + v(T') \geq v(T \cup T')$. 

Subadditive valuations are more general than submodular valuations and monotone subadditive valuations capture, as special cases, both additive and unit-demand valuations.  As such, $\mathcal{M}_2$ provides improved approximations for two bidders and two items in combinatorial auctions with both unit-demand and additive valuations.
\begin{claim}
The mechanism $\mathcal{M}_2$ achieves a $\nicefrac 4 3$-approximation to the optimal social welfare for monotone subadditive valuations.
\end{claim}
\begin{proof}
    Let $\{1,2\}$ denote the set of two bidders and $\{a,b\}$ denote the set of two items.  
    We consider two cases depending on whether the optimal allocation is such that one bidder receives both items or  each bidder receives an item.

    We begin with the former case and without loss of generality assume that the optimal allocation awards both items to bidder $1$ (the case where bidder $2$ obtains both items is symmetric), i.e., $\opt = v_1(\{a,b\})$.   Since $\opt = v_1(\{a,b\})$, it must be that $v_1(a~|~b) \geq v_2(\{a\})$ and $v_1(b~|~a) \geq v_2(\{b\})$. 
    
    We now consider all possible outcomes of the randomized mechanism $\mathcal M_2$. If the mechanism $\mathcal M_2$ randomly allocates either item $a$ or $b$ to bidder $1$ our auction obtains welfare equal to $v_1(\{a,b\})$.  On the other hand, if we randomly allocate $a$ to bidder $2$ then our auction obtains welfare of $v_2(\{a\}) + \max\{v_2(b~|~a),v_1(\{b\})\} \geq v_1(\{b\})$ and, similarly, if we allocate $b$ to bidder $2$ then our auction obtains $v_2(\{b\}) + \max\{v_2(a~|~b),v_1(\{a\})\} \geq v_1(\{a\})$.  Combining these cases with the probabilities they occur gives that $\mathcal M_2$ achieves expected welfare at least $\frac{1}{2}\cdot v_1(\{a,b\}) + \frac{1}{4}\cdot v_1(\{a\}) + \frac{1}{4}\cdot v_1(\{b\})$.  But then, by subadditivity, we have that the auction obtains welfare at least $\frac{3}{4}\cdot v_1(\{a,b\})$.

    Now consider the latter case where the optimal allocation awards each bidder a single item and suppose that bidder $1$ receives $a$ and bidder $2$ receives $b$ (the opposite case is symmetric), i.e., $\opt = v_1(\{a\}) + v_2(\{b\})$.  We then have that $v_1(\{a\}) \geq v_2(\{a~|~b\})$ and $v_2(\{b\}) \geq v_1(\{b ~|~ a\})$.  This means that if $\mathcal M_2$  randomly allocates item $a$ to bidder $1$ or item $b$ to bidder $2$ our auction then obtains the optimal social welfare.  On the other hand, if we allocate $b$ to $1$ our auction obtains welfare $v_1(\{b\}) + \max\{v_1(a~|~b) ,v_2(\{a\})\} \geq v_1(\{a,b\}) \geq v_1(\{a\})$ by monotonicity and similarly if we allocate $a$ to $2$ our auction obtains welfare $v_2(\{a\} + \max\{v_2(b~|~a) ,v_1(\{b\})\} \geq v_2(\{a,b\}) \geq v_2(\{b\})$.  
    
    Combining these cases with the probabilities they occur implies that the expected welfare of $\mathcal M_2$ is at least $\frac{1}{2}\left(v_1(\{a\}) + v_2(\{b\})\right) + \frac{1}{4}v_1(\{a\}) + \frac{1}{4}v_2(\{b\}) = \frac{3}{4}\left(v_1(\{a\}) + v_2(\{b\})\right)$, as desired.
\end{proof}

\subsection{Combinatorial Auction with General Monotone Bidders}\label{subsec:22-ca-general}
Let $p$ be equal to $\frac{1}{3}$. 
Consider the mechanism $\mathcal M_3$ that with probability $p$ runs an ascending second-price auction on the grand bundle and with probability $(1-p)$ allocates a uniformly random agent a uniformly random item and runs an ascending second-price auction for the remaining item. 
\begin{claim}
    The mechanism $\mathcal{M}_3$ is universally OSP.  
\end{claim}
The proof is identical to the proof of Claim \ref{claim-mec-sm-mua-osp}. 
\begin{claim}
The mechanism $\mathcal{M}_3$ achieves a $\nicefrac 3 2$-approximation to the optimal social welfare.
\end{claim}
\begin{proof}
    Let $\{1,2\}$ denote the set of two bidders and $\{a,b\}$ denote the set of the two items. 
Throughout the proof, we slightly abuse notation by writing the value of a valuation $v$ for item $a$ as  $v(a)$ instead of $v_1(\{a\})$. 

The proof goes by a case  analysis.  First, we analyze the approximation guarantee of the mechanism if the optimal allocation allocates both items to the same agent, and then we analyze the case where the optimal allocation assigns a different item to each bidder. 

We begin with the first case:  assume that the optimal allocation assigns both $a$ and $b$ to the same bidder, say bidder $1$ without loss of generality. 
% The mechanism is anonymous so we can assume without loss of generality that bidder $1$ wins both items. 
Note that if we run ascending auction on the grand bundle, then by assumption it outputs the optimal allocation.  Now, consider the case where $\mathcal M_3$ randomly allocates bidder $1$ with item $a$, and then we runs an ascending auction on item $b$.  Note that:
$v_1(a,b) \ge v_1(a)+v_2(b)$, so $v_1(a ~|~ b)\ge v_2(b)$. Therefore, the mechanism obtains welfare of $v_1(a)+\max\{v_1(b~|~a)+v_2(b)\}\ge v_1(a,b)=\opt$, i.e., $\mathcal M_3$ obtains the optimal welfare. Due to the same reasons, $\mathcal M_3$ obtains the optimal welfare also for the case where $b$ is randomly allocated to bidder $1$. 
% Therefore, in this case bidder $1$ wins item $b$ and item $a$ (or, alternatively, bidder $1$ wins item $a$ and bidder $2$ wins item $b$ but $v_1(a) + v_2(b) = v_1(a,b)$) so we optimize the welfare.
%\shirinote{I feel like this is true up to tie breaking, although in case of a tie we are also good because then if bidder $2$ wins $b$ it just means that there are two optimal allocations and we necessarily output one of them. However, I'm not sure right now how to write it both formally and elegantly.} 
% For the same reasons exactly, if we randomly allocate item $b$ to bidder $1$, then she ends up winning item $a$ as well. 
Therefore, the expected welfare is  at least $\big(p+\frac{1-p}{2}\big)\cdot \opt$. 

Consider the complementary case where the optimal allocation assigns one item to each bidder, without loss of generality 
item $a$ to  $1$ and  $b$ to $2$. 
The expected welfare of $\mathcal M_6$ is:
\begin{align*}
    &p\cdot \underbrace{\max\{v_1(a,b),v_2(a,b)\}}_{\ge OPT/2} \quad&\text{(ascending auction on $\{a,b\}$)} &\\
    +&\frac{1-p}{4}\cdot\Big(v_1(a)+\max\big\{v_2(b),v_1(a,b)-v_1(a)\big\}\Big) &(1\gets a) \\ 
    +&\frac{1-p}{4}\cdot\Big(v_1(b)+
    \max\big\{v_2(a),v_1(a,b)-v_1(b)\big\} \Big) &(1\gets b) \\
    +&\frac{1-p}{4}\cdot\Big(v_2(a)+
    \max\big\{v_1(b),v_2(a,b)-v_2(a)\big\} \Big) &(2\gets a) \\
    +&\frac{1-p}{4}\cdot\Big(v_2(b)+
    \max\big\{v_1(a),v_2(a,b)-v_2(b)\big\} \Big) &(2\gets b)
\end{align*}
where $(1\gets a)$ signifies the case where bidder $1$ is randomly allocated with $a$ and an ascending auction is run on $b$, and $(1\gets b)$, $(2\gets a)$ and $(2\gets b)$ are defined analogously. 
Now, observe that since $\opt=v_1(a)+v_2(b)$, we have that: 
 $v_2(b)\ge v_1(b~|~a))$ and also $v_1(a)\ge v_2(a~|~b))$. Therefore, we obtain the optimal welfare in cases $(1\gets a)$ and $(2\gets b)$.  
 % if either bidder $1$ is randomly allocated with item $a$ or bidder $2$ is randomly allocated with item $b$. 

 The contribution of the remaining cases,
where the bidders are allocated the \textquote{wrong} items, i.e., $(1\gets b)$ and $(2\gets a)$, to the expected welfare is: 
 \begin{multline*}
     \frac{1-p}{4}\cdot\Big(v_1(b)+
    \max\big\{v_2(a),v_1(a~|~b)\big\}+ v_2(a)+
    \max\big\{v_1(b),v_2(b~|~a)\big\}\Big)   \\  
    \ge \frac{1-p}{4}\cdot\Big( v_1(a,b)+v_2(a,b)  \Big) \ge \frac{1-p}{4}\cdot\big(v_1(a)+v_2(b)\big)=\frac{1-p}{4}\cdot \opt
 \end{multline*}
Therefore, the expected welfare is $\frac{p\cdot \opt}{2}$ + $\frac{3(1-p)\cdot \opt}{4}$.  
Since $p=\frac{1}{3}$, in both cases the expected welfare is at least $\frac{2}{3}$ of the optimum, which completes the proof. 
\end{proof}




% \section{Non-Monotonicity for Bidders with Decreasing Marginal Values}\label{app:mua-dec}
% \input{app-non-monotonicity}


\section{The Connection of Obviously Strategy-Proof Auctions and Weakly Group Strategy-Proof Mechanisms}\label{app:wgsp}
In his seminal work, \cite{li} has observed that  obviously strategy-proof mechanisms satisfy the quality of being weakly group strategy-proof\footnote{Loosely speaking, a mechanism $A$ is weakly group strategy-proof if no coalition of bidders can all simultaneously increase their utility by deviating from the dominant strategy.}, but the converse is not true: the top trading cycles mechanism is weakly group strategy-proof, but is not obviously strategy-proof (see also \cite{troyan2019obviously}). This gives rise to the question of understanding the differences between the two notions.

On one hand, weak group strategy-proofness is already
quite restrictive. For illustration, in certain settings (such as multi-unit auction with decreasing marginal valuations)  enforcing weak group strategy-proofness precludes exact welfare approximation \cite{GMR17}. 
This raises the question of how much more restrictive a notion can become beyond weak group strategy-proofness, and in particular whether obvious strategy-proofness is more restrictive than weak group strategy-proofness.  We ask:
%As such, a crisp separation between OSP and group strategy-proofness in auction settings is unclear -- 
are there settings in which OSP mechanisms must necessarily obtain worse approximation guarantees than mechanisms which are ``only'' weakly group strategy-proof? 

We answer this question affirmatively by examining the setting with heterogeneous items and unit-demand bidders.  First, as \cite{Ron24} has shown, deterministic obviously strategy-proof mechanisms that satisfy individual rationality and no negative transfers cannot obtain approximation better than $\min\{m,n\}$. Furthermore, as we show in \cref{thm-lb-ud},  one cannot achieve a $(\nicefrac 8 7-\varepsilon)$-approximation to the social welfare with any randomized OSP mechanism in this setting (similarly, assuming individual rationality and no negative transfers).  On the other hand, we will now explain why  for this setting the VCG mechanism (which is clearly optimal and deterministic) \emph{is} weakly group strategy-proof.  

This holds due to the combination of the following two well known facts.
First, the outcome corresponding to the minimum-price Walrasian equilibrium is weakly group strategy-proof in unit-demand settings even beyond quasi-linear utilities (see, e.g., \cite{morimoto2015strategy}). The second known fact is that for unit-demand settings the minimum-price Walrasian equilibrium corresponds to the VCG outcome and prices (see, e.g., Theorem 5 of \cite{gul1999walrasian}).
Therefore, we obtain a separation between the approximation ratio achievable by OSP mechanisms (which is $\min\{m,n\}$ for deterministic mechanisms \cite{Ron24}, and bounded away from $1$ for randomized ones) and the ratio of weakly group strategy-proof mechanisms (which is $1$ exactly).  


We leave open the question of understanding the power of weakly group strategy-proof mechanisms for additional auction settings. In particular, we have yet to understand the power of weakly group strategy-proof mechanisms for additive bidders.  
For this setting, we cannot apply the arguments previously made for unit-demand bidders, as the the Vickrey-Clarke-Groves (VCG) mechanism 
%\cite{Vic61,C1971,G1973}
is not weakly group strategy-proof\footnote{Take an example with two bidders and two items where the first bidder has value $10$ for both items and the second has value $9$ for both items.  By colluding and reporting that they only value distinct items, both bidders receive an item at a price of $0$ (which is preferable to the VCG outcome for both bidders).}. 

  

%To our knowledge, this is the first known separation for auction settings between the approximation guarantees achievable between OSP mechanisms and group strategy-proof mechanisms and 
% An intriguing line of future work would be to see how large this separation is for various auction settings.

\section{Missing Proofs from Section \ref{sec-prelims}}\label{app:missing-proofs-prems}
% \subsection{Proof of \cref{lem:yaos}} \label{subsec-yaos-lem}

\begin{proof}[Proof of Lemma \ref{lemma-partial}]
    We establish the proof by describing an obviously dominant strategy for every bidder.
We argue that for each bidder $i$ the strategy which has the bidder remain in the auction if and only if her current clock price $p_i$ is less than or equal to $v_i(X_i^{P}) - v_i(X_i^{B})$, i.e., the increase in value she has for receiving $X_i^{P}$ compared to $X_i^{B}$, is obviously dominant. 
We prove it by a straightforward case analysis. In particular, we demonstrate that the worst case given the truthful strategy is always more profitable than the best case given any other strategy. 
% The proof of this fact is rather straightforward case analysis. 

Consider first the case where $p_i \leq v_i(X_i^{P}) - v_i(X_i^{B})$.  The best-case utility $i$ can receive by deviating from the strategy and exiting the auction early is $v_i(X_i^{B})$ (by receiving $X_i^{B}$ for a price of $0$), but the worst-case utility she receives by following the strategy and staying in the auction until the next price increment is also $v_i(X_i^{B})$ (since she can always exit the auction if the price then becomes too high).  

Now consider the case where $p_i > v_i(X_i^{P}) - v_i(X_i^{B})$.  The worst-case utility bidder $i$ can receive by following the strategy and exiting the auction is $v_i(X_i^{B})$. 
On the other hand, if she deviates from it by remaining an active player and the auction were to terminate she would obtain utility 
less than  $v_i(X_i^{P}) - (v_i(X_i^{P}) - v_i(X_i^B)) \leq v_i(X_i^B)$.  Thus, in this case as well, her worst case utility from following the strategy is better than the best case when deviating, which completes the proof.
\end{proof}


\begin{proof}[Proof of \cref{lem:yaos}] 
Fix a randomized mechanism $\mathcal M$ and 
denote with $\mathcal{D}_{\mathcal{M}}$ its distribution over the deterministic mechanisms in its support.
Note that by assumption every deterministic mechanism $A$ in the support of $\mathcal M$ satisfies that:
$$
    \E_{(v_1,\ldots,v_n)\sim {\mathcal{D}}}\Big[\dfrac{A(v_1,\ldots,v_n)}{OPT(v_1,\ldots,v_n)}\Big]\le 
    \frac{1}{\alpha}
$$
Since this is true for every deterministic mechanism in the support, averaging over the randomness of the mechanism $\mathcal M$:
$$
   \E_{(v_1,\ldots,v_n)\sim \mathcal{D},A\sim \mathcal {D}_M}\Big[\dfrac{A(v_1,\ldots,v_n)}{OPT(v_1,\ldots,v_n)}\Big]\le \frac{1}{\alpha} 
$$
Denote with $(v_1^\ast,\ldots,v_n^\ast)$ the valuation profile in the support of $\mathcal D$ that minimizes $\E_{A\sim \mathcal \mathcal D_M}\Big[\frac{A(v_1^\ast,\ldots,v_n^\ast)}{OPT(v_1^\ast,\ldots,v_n^\ast)}\Big]$. Since the minimum is  smaller or equal than the average:
$$
   \E_{A\sim \mathcal \mathcal D_M}\Big[\dfrac{A(v_1^\ast,\ldots,v_n^\ast)}{OPT(v_1^\ast,\ldots,v_n^\ast)}\Big]\le  \frac{1}{\alpha}
$$
Thus, the randomized mechanism $\mathcal M$ does not give an  approximation better than $\alpha$ to the optimal welfare given the valuation profile $(v_1^\ast,\ldots,v_n^\ast)$, which completes the proof.
\end{proof}






\section{Missing Proofs from Section \ref{sec-mua}: Multi-Unit Auctions}\label{app-missing-mua}
\subsection{Proof of Theorem \ref{claim:ascending-bundles-approx}}
% \begin{proof}
Fix a valuation profile $(v_1,d_1),\ldots,(v_n,d_n)$ where $d_i$ is the desired set and $v_i$ is the value for them and
consider a social welfare optimizing allocation $\vec O=(O_1,\ldots,O_n)$, whose  welfare we denote with $\opt$. In particular, Let $\vec O$ be the allocation that allocates the minimum number of items among the optimal ones.\footnote{Note that we slightly abuse notation here, since $v_i$ also stands for a valuation function.}   

% Without loss of generality, we may assume  that bidders who receive at least one item in this allocation receive exactly the minimum number of items which yields them positive value.  
Consider partitioning the bidders who receive at least one item in this allocation into $k+1$ disjoint subsets based on the number of items allocated to them 
% such that bidders in part $i \in \{0,1, 2, \dots, k=\lceil \log{m} \rceil\}$ receive between $2^{i}$ and $2^{i+1}-1$ items. 
such that bidders in group $i\in \{0,1, 2, \dots, k=\lceil \log{m} \rceil-1\}$ receive between $2^{i}+1$ and $2^{i+1}$ items. We denote the subset of bidders in group $i$ with $\mathcal B_i$. 

Denote with $\mathcal U[0,k]$ the uniform distribution over the integers $\{0,\ldots,k\}$. Observe that the fact there exist at most $\log m$ groups implies that: 
% Observe that the expected welfare of bidders belonging to a group $I$ sampled uniformly at random  is equal to:
\begin{equation}\label{eq-opt-lb}
\mathbb{E}_{i \sim \mathcal{U}[0, k]} \big[\sum_{i \in \mathcal B_i} v_i\big] = 
\frac{1}{\log m} \cdot \sum_i v_i(O_i)= \frac{OPT}{\log m}
\end{equation}
% Observe that there necessarily exists a group $I^\ast$ of bidders such that: 
% \begin{equation}\label{eq-opt-lb}
%  \sum_{i\in I^\ast} v_i \ge \sum_i \frac{v_i(O_i)}{\log m}= \frac{OPT}{ \log m}   
% \end{equation}

For every $i\in \{0,\ldots,k\}$, denote with $\delta_i$ the total number of bidders whose demand $d_i$ is smaller than or equal to $2^{i+1}$ (including bidders who are not allocated in the optimal allocation).  Observe that since
each bidder in the group $\mathcal B_i$ receives between $2^{i}+1$ and $2^{i+1}$ items in the optimal allocation, the number of the bidders in each subset  $\mathcal B_i$ is at most 
$\min\{\Big\lfloor \frac{m}{2^{i}+1}\Big\rfloor ,\delta_i\}$.  

Now, fix an integer $i\in \{0,1,\ldots,k\}$ and
consider the allocation of \textsc{Random-Bundles} whenever it allocates bundles of size $2^{i+1}$. Denote the bidders who are satiated in this allocation with $\mathcal A_i$. We make the following two observations regarding this set of bidders.

First, note that by construction the bidders in $\mathcal A_i$ are the bidders with the highest values who are  satiated by bundles of size  $2^{i+1}$. Since the bidders in $\mathcal B_i$ are also satiated by bundles of size $2^{i+1}$, we get that every bidder who is in $\mathcal A_i$ has a higher value than any bidder who is in $\mathcal B_i\setminus \mathcal A_i$. Formally: 
% since bidders in $I^\ast \setminus A$ are satiated by bundles of size $2^{i^\ast+1}$ and the bidders in $A$ are the bidders with highest values that are satiated by this size of bundles, we get that: 
\begin{equation}\label{eq-diff}
  \max_{i\in \mathcal B_i \setminus \mathcal A_i} v_i \le \min_{i\in \mathcal A_i} v_i  
\end{equation}
Second, since \textsc{Random-Bundles} allocates bundles of size $2^{i+1}$, it implies that $|\mathcal A_i|\ge \min\{\lfloor \frac{m}{2^{i+1}}\rfloor,\delta_i\}$ and therefore: 
\begin{equation}\label{eq-diff-count}
2|\mathcal A_i|\ge \min \{\Big\lfloor \frac{m}{2^{i}}\Big\rfloor,2\cdot \delta_i\} \ge |\mathcal B_i|
\end{equation}
where we remind that the rightmost inequality holds because $|\mathcal B_i|\le \min\{\Big\lfloor \frac{m}{2^{i}+1}\Big\rfloor ,\delta_i\}$.
Combining (\ref{eq-diff}) and (\ref{eq-diff-count}) gives that for every subset of bidders $\mathcal B_i$:
\begin{equation}\label{eq-comp}
    \sum_{i\in \mathcal B_i} v_i \le    \sum_{i\in \mathcal B_i\cap \mathcal A_i} v_i  + 
        \sum_{i\in \mathcal B_i\setminus \mathcal A_i} v_i \le \sum_{i\in \mathcal A_i} v_i + |\mathcal B_i \setminus \mathcal A_i|\cdot \min_{i\in \mathcal A_i} v_i \le 3\cdot \sum_{i\in \mathcal A_i} v_i
\end{equation}
where the rightmost inequality holds since clearly $|\mathcal B_i\setminus \mathcal A_i| \le |\mathcal B_i|$. 


Now, denote with $ALG$ the expected welfare of 
\textsc{Random-Bundles}. We remind that \textsc{Random-Bundles} samples $i\in \{0,\ldots,k\}$ uniformly at random, and therefore:
% chooses a group of bidders $I$ from $\mathcal{U}[u,k]$ uniformly at random, and therefore: 
% allocates to the bidders in $A^\ast$ with probability at least $\frac{1}{\log m}$, and therefore:
% \begin{equation}\label{eq-alg-lb}
%     ALG \ge \frac{\sum_{i\in A}}{\log m}
% \end{equation}
% Combining all of the above gives that:
\begin{align*}
     ALG &=  \mathbb{E}_{i \sim \mathcal{U}[0, k]} \Big[\sum_{i \in \mathcal A_i} v_i\Big] \\
     % \frac{\sum_{i\in A}v_i}{\log m} \\ 
%&\text{\small{(\textsc{Random-Bundles} allocates items to $A$ with probability at least $\nicefrac{1}{\log m}$})} \\
& \ge \mathbb{E}_{i \sim \mathcal{U}[0, k]} \Big[\frac{\sum_{i \in \mathcal B_i} v_i}{3}\Big] &\text{(by (\ref{eq-comp}))} \\
% \frac{\sum_{i \in \mathcal B_i}  v_i\Big]}{3}  &\text{(by (\ref{eq-opt-lb}))} \\
 &= \frac{OPT}{3\log m} &\text{(by (\ref{eq-opt-lb}))} 
\end{align*}
which completes the proof. 









\begin{comment}
Consider a social welfare optimizing allocation.  Without loss of generality, we may assume that each bidder who receives at least one item in this allocation receives exactly the minimum number of items which yields them positive value.  
Consider partitioning the bidders who receive at least one item in this allocation into $k+1$ disjoint subsets based on the number of items allocated to them 
% such that bidders in part $i \in \{0,1, 2, \dots, k=\lceil \log{m} \rceil\}$ receive between $2^{i}$ and $2^{i+1}-1$ items. 
such that bidders in part $i \in \{0,1, 2, \dots, k=\lceil \log{m} \rceil-1\}$ receive between $2^{i}+1$ and $2^{i+1}$ items.
\src{did some numbering changes here}


  
We first consider the contribution of bidders to the optimal social welfare who are in the ``large'' bundle parts of our bidder partition.  Observe that there are at most $8$ bidders in total served in any optimal allocation who demand at least $\nicefrac{m}{8}$ items.  As such there are at most $8$ bidders across parts $k-3,k-2,k-1$ and $k$.  Thus, we know that welfare contributed by bidders in these parts is no more than $8$ times the welfare \textsc{Random-Bundles} obtains if it bundles all items together and auctions this grand bundle.  


Now consider an arbitrary part $i$ in our bidder partition with $i \leq k-3$.  Note that by definition $\left \lfloor \frac{m}{2^{i+1}}\right\rfloor \geq 2$.
Observe, however, that since the bidders in part $i$ have demand between $2^{i}$ and $2^{i+1} - 1$ we know that there are at most $\left \lfloor \frac{m}{2^{i}}\right\rfloor$ bidders in part $i$ and they are all satiated by receiving $2^{i+1}$ items.  Since $\left\lfloor \frac{m}{2^{i}}\right\rfloor \leq 2\cdot \left(\left\lfloor\frac{m}{2^{i+1}}\right\rfloor + 1\right)$ for all positive integers $m$ and $i$ and since $2\cdot \left(\left\lfloor\frac{m}{2^{i+1}}\right\rfloor + 1\right) \leq 3\cdot \left\lfloor\frac{m}{2^{i+1}}\right\rfloor$ whenever $\left\lfloor\frac{m}{2^{i+1}}\right\rfloor \geq 2$, we have that the optimal welfare contributed by bidders in any part $i \leq k - 3$ is no more than $3$ times the welfare \textsc{Random-Bundles} obtains when the bundle size is $\ell = 2^{i+1}$.

As such the contribution to the social welfare from bidders in each part $i \leq k-3$ is covered within a factor $3$ by a unique auction in the support of \textsc{Random-Bundles} with bundle size $\ell = 2^{i+1}$ and the contribution to the to the social welfare from bidders in parts $i > k - 3$ is covered within a factor $8$ by the auction for the grand bundle.  Since there are $\lceil \log{m} \rceil + 1$ auctions in the support of \textsc{Random-Bundles} and each are run uniformly at random, we obtain the $O(\log{m})$-approximation as desired.
\end{comment}
% \end{proof}

\subsection{Proof of Lemma \ref{lem:auction-sampling}}\label{subsec::proof-auction-sampling}

To prove \cref{lem:auction-sampling}, we rely on the following useful and general lemma from \cite{bei2017worst}.  
\begin{lemma}[Lemma 2.1 from \cite{bei2017worst}]
Consider any subadditive function $f : A \rightarrow \mathbb{R}$.  For a given subset $S \subseteq A$ and a positive integer $k$ we assume that $f(S) \geq k \cdot f(\{i\})$ for any $i \in S$.  Further, suppose that $S$ is divided uniformly at random into two groups $T_1$ and $T_2$.  Then, with probability of at least $1/2$, we have $f(T_1) \geq \frac{k-1}{4k} \cdot  f(S)$ and $f(T_2) \geq \frac{k-1}{4k} \cdot f(S)$.
\end{lemma}

Applying this lemma to our auction setting, we can prove our desired statement.

\begin{proof}[Proof of Lemma \ref{lem:auction-sampling}]
To show the desired statement we first argue that the optimal welfare $f$ achievable in any  auction setting (either with homogeneous or heterogeneous items) is a \emph{subadditive} function \emph{over the bidders}. To this end,  consider a set of bidders $S$ and the optimal solution over these bidders.  Observe that each item is allocated to any bidder at most once.  Thus, for any two sets $T_1$ and $T_2$ such that $T_1 \cup T_2 = S$ we have that any feasible solution $x$ in $S$ comprises bidders which appear either in $T_1$ or $T_2$ (or possibly both).  Since we can then construct feasible solutions in $T_1$ and $T_2$ which capture the allocation $x$ it must be that the value of the optimal solution in $T_1$ plus the value of the optimal solution in $T_2$ is greater than or equal to the value of the optimal solution in $S$.

But then, since we have that the optimal welfare on an instance of combinatorial auctions is subadditive over the ground set of bidders and we assume that no bidder is critical, we have that the $k$ in the statement of Lemma 2.1 of \cite{bei2017worst} is $100$.  As such, we have that when we randomly partition the bidders into unsampled and sampled sets, both sets have an optimal welfare that is sets are within a factor $99/400 > 1/5$ of the optimal welfare with probability at least $1/2$.
\end{proof}

\subsection{Proof of Lemma \ref{lemma-small-pay}}\label{subsec::proof-lemma-small-pay}
% We now prove Lemma \ref{lemma-small-pay}. 
% Lemma \ref{lemma-small-pay-add} and Lemma \ref{lemma-small-pay-unit}. 
The proof is a direct consequence of the approximation guarantees of the mechanisms considered, as well 
as  the fact that they are
% the mechanisms in all of them are 
obviously strategy-proof (and thus also dominant-strategy incentive compatible) and satisfy individual rationality and no negative transfers. 
The proof is very similar to a proof in \cite{Ron24} and we write it for completeness.
% We write the proofs  for the sake of completeness.

% For brevity, we attribute certain inequalities to individual rationality, rather than elaborating on their derivation from the allocation rule and payment scheme satisfying individual rationality. We adopt the same approach for the no negative transfers property.
\begin{proof}[Proof of Lemma \ref{lemma-small-pay}]
For part \ref{item-1}, note that because of individual rationality:
\begin{equation}\label{eq-1}
 v_1^{one}(f(v_1^{one},v_2^{one}))-P_1(v_1^{one},v_2^{one})\ge 0   
\end{equation}
We remind that by Claim \ref{claim-mua-sm-instances} part \ref{condi-1}, the allocation rule $f$ allocates to bidder $1$ at least one item given $(v_1^{one},v_2^{one})$, so:
\begin{equation}\label{eq-2}
 v_1^{one}(f(v_1^{one},v_2^{one}))=1   
\end{equation}
Combining (\ref{eq-1}) and (\ref{eq-2}) gives part \ref{item-1}.



For part \ref{item-2}, note that given $(v_1^{ONE},v_2^{ALL})$, Claim \ref{claim-mua-sm-instances} part \ref{condi-2} implies that
player $1$ gets the empty bundle. Because of individual rationality: 
$$v_1^{ONE}(f(v_1^{ONE},v_2^{all}))-P_1(v_1^{ONE},v_2^{all})\ge 0 $$
Since $v_1^{ONE}(f(v_1^{ONE},v_2^{all}))=0$, we get that $0\ge P_1(v_1^{ONE},v_2^{all})$, and because of the no negative transfers property, we get $P_1(v_1^{ONE},v_2^{all})=0$, as needed.

For part \ref{item-3}, note that by Claim \ref{claim-mua-sm-instances} part \ref{condi-3}, player $1$ wins all of the items given $(v_1^{all},v_2^{one})$, so $v_1^{all}(f(v_1^{all},v_2^{one}))=k^2$. Combining this fact with individual rationality implies that $P_1(v_1^{all},v_2^{one})\le k^2$. Therefore:
\begin{equation*}
    v_1^{ALL}(f(v_1^{all},v_2^{one}))-P_1(v_1^{all},v_2^{one})\ge k^4 - k^2
\end{equation*}
Since the mechanism is obviously strategy-proof, it is therefore also dominant-strategy incentive compatible, we get that: $    v_1^{ALL}(f(v_1^{ALL},v_2^{one}))-P_1(v_1^{ALL},v_2^{one})\ge k^4 - k^2
$. The only valuable bundle for player $1$ given $v_1^{ALL}$ is the grand bundle, so we get that 
$f(v_1^{ALL},v_2^{one})$ allocates all items to player $1$. In addition, it implies that the payment of player $1$ given $(v_1^{ALL},v_2^{one})$ is necessarily at most $k^2$, which completes the proof. \qedhere

% Note that $f$ clearly also allocates all items to player $1$ given $({v_1}^{all},v_2^{one})$ because of its approximation guarantee. The fact that the mechanisms is dominant-strategy incentive compatible and $f$ outputs the same allocation for both $(\hat{v_1},v_2^{one})$ and  $({v_1}^{all},v_2^{one})$ implies that $P_1({v_1}^{all},v_2^{one})=P_1(\hat{v_1},v_2^{one})$, so $P_1({v_1}^{all},v_2^{one})$ is also smaller than $k^2$, which completes the proof.     
\end{proof}

\subsection{Proof of Theorem \ref{thm:decreasing-marginals}}
\label{subsec::proof--dec-mua}
\begin{proof}
First of all, the mechanism is obviously strategy-proof, and the proof of it is identical to the proof of \cref{claim-mua-sm-mechanism-osp}. 

We now proceed to prove the approximation guarantee of the mechanism, following an approach that closely resembles the proof of \cref{lem:single-minded-approx}. We remind that a bidder $i$ is \emph{critical} if allocating to her the grand bundle
gives a $1/100$-approximation to the optimal welfare. 
% As before, in the case that there is a critical bidder $i$, allocating $i$ the grand bundle necessarily gives a $1/100$-approximation to the optimal welfare.  
Thus, if there exists a critical bidder, we obtain a $1/200$-approximation by running an ascending auction for the grand bundle with probability at least $1/2$.

    In the case that there does not exist a critical bidder, we may again appeal to our sampling lemma, i.e., Lemma \ref{lem:auction-sampling}, to show that when we sample bidders, we obtain an ``accurate enough'' estimates with probability at least $1/2$. Formally, with probability $1/2$ we have that the social welfare $\text{OPT}(S)$ contained in the sampled set $S$ is between $\text{OPT}/5$ and $\text{OPT}$  and the same holds for $\text{OPT}(U)$, the welfare among the unsampled bidders. 
    % (and, further, the social welfare $\text{OPT}(U)$ of the set of unsampled bidders also falls in this range). 
    Therefore, with probability $\frac 1 2 $ we set a per-item price $p$ which is in the range $[\text{OPT}/(50m), \text{OPT}/(10m)]$. We will now further refine the case analysis by examining the number of items sold.
    % , still similarly to the proof of \cref{lem:single-minded-approx},  
    
    The \textquote{easy} case is if we sell at least $\nicefrac m 2$ items. 
      Since an unsampled bidder buying $t$ goods spends at least $\frac{t\text{OPT}}{50m}$, their value for the purchased bundle is at least $\frac{t\text{OPT}}{50m}$. Therefore, the total value of all bidders who purchase goods is at least $\frac{m}{2}\cdot\frac{\text{OPT}}{50m} = \frac{\text{OPT}}{100}$. Therefore,  we  obtain welfare of at least $\text{OPT}/100$. Altogether, since we run uniform sampling with probability $\nicefrac{1}{2}$ and the estimation is
\textquote{good} with probability $\nicefrac{1}{2}$, we obtain  a $400$-approximation to the welfare.
    

    
    Suppose, by contrast, that we sell fewer than $\nicefrac m 2$ items to the unsampled bidders.  
    To analyze this case, let $\vec{q}=(q_1,\dots,q_n)$ be the optimal allocation if the items are divided only among the bidders in $U$ (clearly, every bidder not in $U$ is allocated zero items, and the welfare of $\vec q$ is equal to $\text{OPT}(U)$). 
    Observe that the welfare of $\vec q$
is partitioned to ``low'' marginal values, i.e., marginal values less than or equal to $p$ and ``high'' marginal values, which are greater than $p$.

Observe that since Mechanism \ref{alg:single-minded} allocates less than $\frac m 2$ items, every bidder in  $U$  could have bought additional items at a price of $p$. Therefore, all bidders who were allocated in $\vec q$ were also allocated by Mechanism \ref{alg:single-minded} all the items for which they had \textquote{high} marginal values. Therefore, it remains to bound the loss coming from \textquote{low} margins. Observe that the marginal value each agent in $U$ has for receiving an additional good is no more than $\text{OPT}/(10m)$ and since there are $m$ items in total allocated in $\vec q$, the total welfare of $\vec q$ coming from \textquote{low} marginals is at most $\text{OPT}/10$. However, by assumption the total welfare of $\vec q$ is at least $\text{OPT}/5$,  so at least 
$\text{OPT}/10$ of welfare comes from \textquote{high} marginals which also contribute to the welfare of the allocation of Mechanism \ref{alg:single-minded}. As we said before, this 
% happens with probability $\frac{1}{4}$,
depends on finding a good partition of $U,S$ and running a uniform price auction which occurs in probability $\nicefrac{1}{4}$, 
so overall the expected welfare of the mechanism in this case is at least $\frac{OPT}{40}$. 
    

    
    
    
% Note that  since we sold fewer than $m/2$ items, we know that each bidder in $U$ had the opportunity to purchase items at a price of $p$. 

% additional units of the good but chose not to do so. Therefore, all the marginal values the portion of the welfare of $\vec q$  

% Because each bidder in $U$ purchased their utility maximizing bundle, we have that the marginal value each agent in $U$ has for receiving an additional good is no more than $\text{OPT}/(10m)$.  
    
%     We will now use this fact. Let $\vec{q}$ denote the optimal allocation of items to bidders in $U$ (that has welfare of $\text{OPT}(U)$ by definition). We consider the portion of total welfare of $\vec{q}$
% which comes from ``low'' marginal values (i.e., marginal values less than or equal to $\text{OPT}/(10m)$), and compare it with ``high'' marginal values, namely with values greater than $\text{OPT}/(10m)$. 
%     % and consider the portion of the  total welfare of $\vec{q}$ which comes from these ``low'' marginal values (i.e., marginal values less than or equal to $\text{OPT}/(10m)$). 
%     Since there are $m$ items in total we have that the portion of the welfare the comes from \textquote{low marginals} is at most $\text{OPT}/10$.  Since $\text{OPT}(U) \geq \text{OPT}/5$, the portion of the welfare in $\vec{q}$ coming from ``high'' marginal values is at least $\text{OPT}/10$.  But since each bidder in $U$ had the opportunity to purchase additional goods (i.e., no bidder was prevented from purchasing a good that she otherwise would have wanted to), it must be that our auction obtains all of the welfare in $\vec{q}$ coming from ``high'' marginals, i.e., at least $\text{OPT}/10$ welfare.  Hence, since we run the sample-and-price phase with probability $1/2$ and, conditioned on this, accurately sample with probability at least $1/2$, we obtain social welfare at least $\text{OPT}/400$ in the case that there is no critical bidder.

Combining all cases, we conclude that the expected welfare of Mechanism \ref{alg:single-minded} is at least $\frac{OPT}{400}$,  thereby completing the proof.
\end{proof}

% \subsection{Proof of \cref{claim-oneone-same,claim-ONE-ALL-same}} \label{subsec:mua-sm-claims-proofs}




\subsection{ Proof of  Theorem \ref{thm-lb-mua-dec}: Impossibility for 2 Items and 2 Bidders} \label{subsec-lb-proof-mua-dec}
First of all, We assume that the domain $V_i$ of each bidder consists of valuations with values in  $\{0,1,\ldots,k^4\}$ that satisfy decreasing marginal utilities, where $k$ is an arbitrarily large number. 

Assume towards a contradiction that 
there exists an obviously strategy-proof, individually rational, no negative transfers mechanism $A$ together with strategy profile $\mathcal S=(\mathcal S_1,\mathcal S_2)$
that implement an allocation rule and payment schemes  $(f,P_1,P_2):V_1\times V_2\to \allocs\times \mathbb R^2$, where $f$ 
gives an approximation strictly better than $2$ to the optimal social welfare.  
% Thus, there exists
% an allocation rule
% $f:V_1\times\cdots\times V_n\to \mathcal T$  and payment schemes  $P_1,\ldots,P_n:V_1\times \cdots \times V_n\to \mathbb{R}^n$ the payment schemes  
% Fix an allocation rule  with approximation ratio $\alpha$ such that $\alpha>\max\{\frac{1}{n},\frac{1}{m}\}$. Let $\mathcal{M}$ be a normalized mechanism 
% and strategies  $(\mathcal S_1,\ldots,\mathcal{S}_n)$  that realize $f$ together with the payment schemes $P_1,\ldots,P_n:V_1\times \cdots \times V_n\to \mathbb{R}^n$. 
For every player $i$, we define three valuations that will be of particular interest:  
$$
v_i^{ALL}(x)=\begin{cases}
k^2 &\quad x=1, \\
2k^2 &\quad x=2.
\end{cases} \\\quad 
v_i^{one}(x)= \begin{cases}
1 \quad x=1, \\
1 \quad x=2.
\end{cases}
 \quad
v_i^{ONE}(x)=\begin{cases}
4k \quad x= 1, \\
4k  \quad x=2.
\end{cases} 
$$
where $k$ is arbitrarily large. 


For the analysis of the mechanism, we define the following subsets of valuations:
$
\mathcal{V}_1=\{v_1^{one},v_1^{ONE},\allowbreak v_1^{ALL}\}\allowbreak \subseteq V_1$ and  $\mathcal{V}_2=\{v_2^{one},v_2^{ONE},v_2^{ALL}\}\subseteq V_2$.  
% We denote $\mathcal{V}_1\times\cdots\times \mathcal{V}_n$ with $\mathcal V$.  
% use the notation $\mathcal V=\mathcal{V}_1\times\cdots\times \mathcal{V}_n$.    
%$\mathcal V_1=\{v_1^{one},v_1^{ONE},v_1^{all}\}$ and $\mathcal V_2=\{v_2^{one},v_2^{ONE},v_2^{all}\}$. For every player $i\ge 3$, we define $\mathcal V_i=\{v_i^{one}\}$.
% We now analyze the mechanism given that the valuations of the players belong to $\mathcal V=\mathcal V_1\times  \cdots \times \mathcal V_n$.  
Observe that:
\begin{claim}\label{claim-mua-dec-instances}
Every deterministic mechanism that has approximation strictly better than $2$ necessarily satisfies the following  conditions simultaneously:
\begin{enumerate}
    \item Given the valuation profile $(v_1^{one},v_2^{one})$, the mechanism
    allocates one item to bidder $1$ and one item to bidder $2$. \label{condi-1-dec}
% \item Given the valuation profile $(v_1^{\text{ONE}}, v_2^{\text{one}})$, bidder $1$ wins at least one item.  \label{condi-2-dec}
    \item Given the valuation profile $(v_1^{ONE}, v_2^{ALL})$, bidder $2$ wins all items.
    \label{condi-3-dec}
\end{enumerate}
\end{claim}
The proof of \cref{claim-mua-dec-instances} is straightforward: if a deterministic mechanism does not satisfy one of conditions, then due the fact that $k$ is arbitrarily large implies that the approximation guarantee of the mechanism is at most $2$ in the worst case.  


We now focus on 
$
\mathcal{V}_1\times \mathcal{V}_2$. 
Observe that there necessarily exists a vertex $u$, and valuations $v_1,v_1' \in \mathcal{V}_1$, and  $v_2,v_2' \in \mathcal{V}_2$ such that $(\mathcal{S}_1(v_1), \mathcal{S}_2(v_2))$ diverge at vertex $u$. This follows from \cref{claim-mua-dec-instances}, which implies that the mechanism $A$ must output different allocations for different valuation profiles in $\mathcal{V}_1 \times \mathcal{V}_2$. Consequently, not all valuation profiles end up in the same leaf, meaning that divergence must occur at some point. 

% Observe that either bidder $1$ or $2$ has to send different messages for different valuations in $\mathcal V_i$ at some vertex. This is an immediate implication of \cref{claim-mua-dec-instances}, as the mechanism $A$ necessarily outputs different allocations given the valuation profiles $(v_1^{one},v_2^{one})$ and $(v_1^{ONE},v_2^{ALL})$, meaning that the behavior profiles 
 % \dnote{Missing parentheses here?} $(\mathcal S_1(v_1^{one},\mathcal S_2(v_2^{one}))$ and $(\mathcal S_1(v_1^{all},\mathcal S_2(v_2^{one}))$ reach different leaves.

Let $u$ be the first vertex in the protocol such that 
 %the behavior profiles 
 $(\mathcal{S}_1(v_1),\mathcal{S}_2(v_2))$ and $(\mathcal{S}_1(v_1'),\mathcal{S}_2(v_2'))$ diverge, i.e., dictate different messages. 
Note that by definition this implies that $u\in Path(\mathcal{S}_1(v_1),\mathcal{S}_2(v_2))\cap Path(\mathcal{S}_1(v_1'),\mathcal{S}_2(v_2'))$ and that either bidder $1$ or bidder $2$ sends different messages for the valuations in $\mathcal{V}_1$ or $\mathcal V_2$, respectively. 
Without loss of generality, we assume that bidder $1$ sends different messages, meaning that there exist $v_1,v_1'\in \mathcal{V}_1$ such that $\mathcal S_1(v_1)$ and $\mathcal S_1(v_1')$ dictate different messages at vertex $u$.  
We remind that $\mathcal{V}_1=\{v_1^{one},v_1^{ONE},v_1^{all}\}$, so the  following claims jointly imply a contradiction, completing the proof: 
 
 \begin{comment}
 Let $u$ be the first vertex in the protocol such that the behavior profiles $(\mathcal{S}_1(v_1),\mathcal{S}_2(v_2))$ and $(\mathcal{S}_1(v_1'),\mathcal{S}_2(v_2'))$ dictate different messages. 
Note that by definition $u\in Path(\mathcal{S}_1(v_1),\mathcal{S}_2(v_2))\cap Path(\mathcal{S}_1(v_1'),\mathcal{S}_2(v_2'))$. We remind that each vertex is associated with only one player that sends messages in it. 
We assume without loss of generality that player $1$ is the player that sends a message in vertex $u$, so  there exist $v_1,v_1'\in \mathcal{V}_1$ such that $\mathcal S_1(v_1)$ and $\mathcal S_1(v_1')$ dictate different messages at vertex $u$. We remind that $\mathcal{V}_1=\{v_1^{one},v_1^{ONE},v_1^{all}\}$, so 
the  following claims jointly imply a contradiction, completing the proof of Theorem \ref{thm-lb-mua-dec}:
\end{comment}
\begin{claim}\label{claim-oneone-same-dec}
    The strategy $\mathcal S_1$ dictates the same message at vertex $u$ for the valuations $v_1^{one}$ and $v_1^{ONE}$. 
\end{claim}
\begin{claim}\label{claim-one-all-same-dec}
        The strategy $\mathcal S_1$ dictates the same message at vertex $u$ for the valuations $v_1^{ONE}$ and $v_1^{ALL}$.
\end{claim}
% Observe that by construction, every $v\in \mathcal V$ satisfies that vertex $u$ is in $Path(\mathcal S(v))$. 
In the proofs of Claims \ref{claim-oneone-same-dec} and Claim \ref{claim-one-all-same-dec}, we use the following lemma:
\begin{lemma}\label{lemma-small-pay-dec}
    The allocation rule $f$ and the payment scheme $P_1$ of bidder $1$ satisfy that:
    \begin{enumerate}
        \item Given $(v_1^{one},v_{2}^{one})$ bidder $1$ wins  one item and pays at most $1$.  \label{item-1-dec}
        \item  Given $(v_1^{ONE},v_2^{ALL})$, bidder $1$ gets the empty bundle and pays zero.   \label{item-2-dec}
        \item Given $(v_1^{ALL},v_2^{one})$, bidder wins all items and pays at most $2k$. \label{item-3-dec}  
    \end{enumerate}
\end{lemma}
The lemma is a direct consequence of the properties of the mechanism. 
We use 
it to prove \cref{claim-oneone-same-dec,claim-one-all-same-dec}
and defer the proof to  \cref{sec-small-pay-proof-alltogether-dec}.  The proofs are identical to those in \cite{Ron24}, and we include them here for completeness.
% We now use Lemma \ref{lemma-small-pay} to obtain a contradiction. 
% The proof proceeds as follows. We will show that since $\mathcal S_1$ is obviously dominant, then it necessarily dictates the same message for $v_1^{one}$ and for $v_1^{ONE}$ at vertex $u$. Using similar arguments, we will also show that $\mathcal S_1$ dictates the same message for $v_1^{ONE}$ and $v_1^{all}$ at vertex $u$. Thus, the strategy $\mathcal S_1$ assigns the same message to all the valuations in $\mathcal V_1$, so we get a contradiction, which completes the proof.  
%%%PROOF-OF-PLACE%%%
\begin{proof}[Proof of Claim \ref{claim-oneone-same-dec}]
% \begin{proof}[of Claim \ref{claim-oneone-same}]
    Note that by Lemma \ref{lemma-small-pay-dec} part \ref{item-1-dec}, $f(v_1^{one},v_2^{one})$ allocates at least one item to player $1$ and $P_1(v_1^{one},v_2^{one})\le 1$. Therefore:
\begin{equation}\label{eq-good-leaf1-dec}
 v_1^{ONE}(f(v_1^{one},v_2^{one}))-P_1(v_1^{one},v_2^{one})\ge 4k-1 
\end{equation}
 In contrast, by part \ref{item-2-dec} of Lemma \ref{lemma-small-pay-dec},   $f(v_1^{ONE},v_2^{ALL})$ allocates no items to player $1$ and $P_1(v_1^{ONE},v_2^{ALL})=0$, so:
 \begin{equation}\label{eq-bad-leaf1-dec}
 v_1^{ONE}(f(v_1^{ONE},v_2^{ALL}))-P_1(v_1^{ONE},v_2^{ALL})= 0   
\end{equation}
Combining inequalities (\ref{eq-good-leaf1-dec}) and (\ref{eq-bad-leaf1-dec}) gives:
\begin{equation*}
  v_1^{ONE}(f(v_1^{ONE},v_2^{ALL}))-P_1(v_1^{ONE},v_2^{ALL})< 
  v_1^{ONE}(f(v_1^{one},v_2^{one}))-P_1(v_1^{one},v_2^{one})  
\end{equation*}
We remind that vertex $u$ belongs in $Path(\mathcal S_1(v_1^{one}),\mathcal S_2(v_2^{one}))$ and also in
$Path(\mathcal{S}_1(v_1^{ONE}),\allowbreak\mathcal{S}_2(v_2^{ALL}))$. Therefore, Lemma \ref{lemma-bad-leaf-good-leaf} gives that the strategy $\mathcal S_1$ dictates the same message for  $v_1^{one}$ and $v_1^{ONE}$ at vertex $u$.
\end{proof}

\begin{proof}[Proof of \cref{claim-one-all-same-dec}]
    Following the same approach as in the proof of Claim \ref{claim-oneone-same-dec}, note that by \cref{lemma-small-pay-dec}  \cref{item-3-dec}: 
\begin{equation}\label{break-align-dec}
v_1^{ONE}(f(v_1^{ALL},v_2^{one}))-P_1(v_1^{ALL},v_2^{one}) \ge 4k-2k=2k     
\end{equation}
Also, by \cref{lemma-small-pay-dec}  \cref{item-2}:
\begin{equation}\label{break-align2-dec}
  v_1^{ONE}(f(v_1^{ONE},v_2^{ALL}))  
-P_1(v_1^{ONE},v_2^{ALL})=0   
\end{equation}
Combining \cref{break-align-dec} and \cref{break-align2-dec}:
\begin{equation*}
    v_1^{ONE}(f(v_1^{ONE},v_2^{ALL}))  
-P_1(v_1^{ONE},v_2^{ALL})< v_1^{ONE}(f(v_1^{ALL},v_2^{one}))-P_1(v_1^{ALL},v_2^{one})
\end{equation*}
% where the first inequality is by Lemma \ref{lemma-small-pay} part \ref{item-3} and the equality is by Lemma \ref{lemma-small-pay} part \ref{item-2}.
Combining the above inequality with the fact that 
vertex $u$ belongs in $Path(\mathcal S_1(v_1^{ALL}),\mathcal S_2(v_2^{one}))$ and in
$Path(\mathcal{S}_1(v_1^{ONE}),\mathcal{S}_2(v_2^{ALL}))$ implies that by Lemma \ref{lemma-bad-leaf-good-leaf} 
% applying Lemma \ref{lemma-bad-leaf-good-leaf} gives
the obviously dominant strategy $\mathcal S_1$ dictates the same message for the valuations $v_1^{ONE}$ and $v_1^{ALL}$ at vertex $u$. 
\end{proof}


\subsection{Proof of Lemma \ref{lemma-small-pay-dec}: Observations About The Mechanism}\label{sec-small-pay-proof-alltogether-dec}
The proof is a direct consequence of the approximation guarantee of the mechanism and the fact that it is 
% the mechanisms in all of them are 
obviously strategy-proof (and thus also dominant-strategy incentive compatible) and satisfies individual rationality and no negative transfers. 
%We write the proof  for the sake of completeness.

%Throughout the proof, for brevity we say that certain inequalities hold  because of individual rationality instead of saying that they hold because the allocation rule together with the payment scheme are realized by a mechanism and strategies that satisfy individual rationality. We do the same for the no negative transfers property. 



% The following  abbreviations will be  utilized in the proofs of the following lemmas. For Lemma \ref{}
% all of them, we have
% previously defined the set $\mathcal V$ in a way that gurantees that 
%  that every valuation profile $(v_1,\ldots,\allowbreak v_n)\allowbreak\in \mathcal V_1\times \cdots \times \mathcal V_n$ satisfies that for every $i\ge 3$, $\mathcal V_i$ is a singleton, so   the valuation $v_i$ is in fact $v_i^{one}$. Therefore, we write $f(v_1,v_2)$ for $f(v_1,v_2,v_3^{one},\ldots,v_n^{one})$ and $P_1(v_1,v_2)$ for $P_1(v_1,v_2,v_3^{one},\ldots,v_n^{one})$. 
% In addition, we say for abbreviation 
% By Claim \ref{claim-ir-npt-norm}, the mechanism $\mathcal{M}$ together with the strategies $(\mathcal S_1,\ldots,\mathcal S_n)$ are normalized (because by assumption they satisfy individual rationality and no negative transfers). We will use this fact extensively in the analysis. 
% Now we can start proving the items in Lemma \ref{lemma-small-pay}. 

For part \ref{item-1}, note that because of individual rationality:
\begin{equation}\label{eq-1-dec}
 v_1^{one}(f(v_1^{one},v_2^{one}))-P_1(v_1^{one},v_2^{one})\ge 0   
\end{equation}
We remind that by \cref{claim-mua-dec-instances} \cref{condi-1-dec}, the allocation rule $f$ allocates to bidder $1$ one item given $(v_1^{one},v_2^{one})$, so:
\begin{equation}\label{eq-2-dec}
 v_1^{one}(f(v_1^{one},v_2^{one}))=1   
\end{equation}
Combining (\ref{eq-1-dec}) and (\ref{eq-2-dec}) gives part \ref{item-1-dec}.



For part \ref{item-2-dec}, note that 
by \cref{claim-mua-dec-instances} \cref{condi-3-dec}, 
given $(v_1^{ONE},v_2^{ALL})$ player $1$ gets the empty bundle. Because of individual rationality: 
$$v_1^{ONE}(f(v_1^{ONE},v_2^{ALL}))-P_1(v_1^{ONE},v_2^{ALL})\ge 0 $$
Since $v_1^{ONE}(f(v_1^{ONE},v_2^{ALL}))=0$, we get that $0\ge P_1(v_1^{ONE},v_2^{ALL})$, and because of the no negative transfers property, $P_1(v_1^{ONE},v_2^{ALL})=0$, as needed.  

To prove part \ref{item-3-dec}, we define another valuation:
$$
\hat{v}_1(x)=\begin{cases}
    k \quad &x=1, \\
    2k \quad &x=2.
\end{cases}
$$
Given $(\hat{v}_1,v_2^{one})$, player $1$ wins $2$ items because  
$f$ gives an approximation strictly better than $2$. Thus, the inequality $\hat{v}_1(f(\hat{v}_1,v_2^{one}))-P_1(\hat{v}_1,v_2^{one})\ge 0$ holds because of individual rationality, and therefore $P_1(\hat{v}_1,v_2^{one})\le 2k$. 

Note that $f$ clearly also allocates all items to player $1$ given $({v_1}^{ALL},v_2^{one})$ because of its approximation guarantee. The fact that the mechanism is dominant-strategy incentive compatible and $f$ outputs the same allocation for both $(\hat{v}_1,v_2^{one})$ and  $({v_1}^{ALL},v_2^{one})$ implies that $P_1({v_1}^{ALL},v_2^{one})=P_1(\hat{v}_1,v_2^{one})$, so $P_1({v_1}^{ALL},v_2^{one})$ is also smaller than $2k$, which completes the proof. 

\subsection{Proof of Lemma \ref{lemma:mono-mua-dec}: A  Deterministic Mechanism For 3 Items and 2 Bidders}\label{sec-impos-mua-dec}
% \begin{proof}[Proof of \cref{lemma:mono-mua-dec}]
    Consider the following mechanism: allocate one item to each bidder, and run an ascending auction on the remaining item. Assume tie breaking is in favor of player $1$, i.e., if both players have the same value for the item, then player $1$ wins it.
    This mechanism is a generalized ascending auction, so by \cref{lemma-partial} it is obviously strategy-proof. 
    
    We now analyze its approximation guarantee. 
Intuitively, the mechanism guarantees a $1.5$ approximation because the worst  case scenario is that some bidder  should  have gotten all three items in the optimal solution, but gets instead only two items. The decreasing marginal property ensures that the loss is bounded by at most $\frac{1}{3}$ of the optimal social welfare.

% for it is at most $\max\{v_1(2)-v_1(1),v_2(1)-v_2(0)\}$. 

% and his marginal value for it is at most $\frac{1}{3}$ of $\max\{v_i(2)-v_i(1),v_i(1)-v_i(0)\}$. 

% The worst 
 
%  is straightforward: one can easily see that the 
    
    Formally, 
denote with $(q_1,q_2)$ the allocation of the mechanism, and let $(o_1,o_2)$ be a welfare-maximizing allocation. We use the following notations: $$ALG=v_1(q_1)+v_2(q_2),\quad OPT=v_1(o_1)+v_2(o_2)$$

    

    Note that the algorithm necessarily outputs an allocation where one bidder gets $2$ items and the other bidder gets $1$ item. Assume without loss of generality that the ascending auction gives bidder $1$ two items and bidder $2$ one item, meaning that  $q_1=2$ and $q_2=1$. Note that by the definition of the auction:
    \begin{equation}\label{eq-opt-ge-alg}
        v_1(2)-v_1(1)\ge v_2(2)-v_2(1) 
    \end{equation}
To show that $ALG \ge \frac{2}{3}\cdot OPT$,  we proceed with the following case analysis:
\paragraph{Case I: $o_1=3$ and $o_2=0$.}
We will first show that $v_1(3)-v_1(2)\le \frac{OPT}{3}$, and then show why it implies that $ALG \ge \frac{2}{3}\cdot OPT$.
Observe that:
\begin{equation*}\label{eq-case-30}
    OPT=v_1(3)=v_1(3)-v_1(2)+v_1(2)-v_1(1)+v_1(1) \ge 3\cdot \big(v_1(3)-v_1(2)\big)
    % \le 3 \cdot v_1(1)
\end{equation*}
where the inequality holds because $v_1$ has decreasing marginal values. 
Therefore:
$$
OPT=v_1(3)=v_1(2)+v_1(3)-v_1(2)\le ALG +v_1(3)-v_1(2)\le ALG + \frac{OPT}{3}
$$
So $ALG \ge \frac{2\cdot OPT}{3}$, as needed. 


% $v_1(1)\ge \frac{OPT}{3}$. Combining this with \cref{eq-case-30} gives that:
% Since $v_1(2)-v_1(1)\ge v_1(3)-v_1(2)$

\paragraph{Case II: $o_1=2$ and $o_2=1$.} In this case, the optimal allocation is identical to the allocation of the mechanism, so $ALG=OPT$ holds trivially .

\paragraph{Case III: $o_1=1$ and $o_2=2$.} Observe that $ALG=v_1(2)+v_2(1)\ge v_1(1)+v_2(2)=OPT$, where the inequality is by \cref{eq-opt-ge-alg}.

% Observe that by construction, the fact that the auction outputs the allocation $(q_1=2,q_2=1)$ implies that:
% $
%     v_1(2)-v_1(1)\ge v_2(2)-v_2(1)$.
% Therefore,  $ALG=v_1(2)+v_2(1)\ge v_1(1)+v_2(2)=OPT$, where the inequality holds by 

\paragraph{Case IV: $o_1=0$ and $o_2=3$.} Due to the same explanation as in case III, we have that: 
\begin{equation}\label{case-03}
    ALG=v_1(2)+v_2(1)\ge v_1(1)+v_2(2)
\end{equation}
And also that:
\begin{align*}
v_1(1) &\ge v_1(2)-v_1(1) &\text{(due to decreasing margins)} \\
&\ge v_2(2)-v_2(1) &\text{(by \cref{eq-opt-ge-alg})} \\
&\ge v_2(3)-v_2(2) &\text{(due to decreasing margins)} \\ 
\end{align*}
So $v_1(1)+v_2(2)\ge v_2(3)=OPT$. Combining this with \cref{case-03} gives that $ALG \ge OPT$, as needed.  
% \end{proof}

\section{Missing Proofs from Section \ref{sec-combi}: Combinatorial Auctions} \label{app-missing-combi}
\subsection{Proof of Lemma \ref{lemma-add-osp}
% An Upper Bound For Additive Bidders
} \label{subsec::proof-add-osp}
Fix any realization of the coin flips of the mechanism.  Observe that ``sampled'' bidders receive $0$ utility regardless of their report so
reporting their valuations trtuhfully is obviously dominant for them. 
% the mechanism is OSP for them.  
Further the ``unsampled'' bidders purchase all available items for which they have  positive utility and purchase no items for which they have negative utility when being truthful, so for them as well truthfulness is an obviously dominant strategy. 
% the mechanism is clearly OSP for them as well.    
%We consider separately the items which have one uniquely highest value bidder and the items which have multiple highest value bidders.  

%First consider an arbitrary item $j$ with a uniquely highest value bidder and denote this bidder $\text{OPT}(j)$.  Our mechanism certainly allocates item $j$ to $\text{OPT}(j)$ if some second highest value bidder for item $j$ is in the sample, $\text{OPT}(j)$ is not in the sample, and if the mechanism instructs bidders to buy only items for which they have strictly positive utility (i.e., unsampled bidders select the minimum sized set in their demand set).  Since all three of these events are independent and occur with probabilities $1/2$, $1/2$, and $2/3$, respectively, our mechanism obtains an $6$-approximation to the welfare component from among items with a uniquely highest value bidder.

%Now consider an arbitrary item $j$ which has at least $2$ bidders tied for the highest value and denote these bidders $i$ and $i'$.  Our mechanism certainly allocates item $j$ to some bidder with value $v_{ij}$ if one of $i$ or $i'$ is sampled and the other is not sampled and if the mechanism instructs bidders to buy all items for which they have non-negative utility (i.e., unsampled bidders select the maximum sized set in their demand set).  Since these events are independent and occur with probabilities $1/2$ and $1/3$, respectively, our mechanism obtains a $6$-approximation to the welfare component from among items which do not have a uniquely highest value bidder.

% \subsection{Using Mechanism \ref{alg:additive} On Unit-Demand Bidders}


% \subsection{Proof of \cref{thm:ud-upper}: A Mechanism for Unit-Demand Bidders}





\subsection{Proof of Theorem \ref{thm-lb-ud}: An Impossibility for Unit-Demand Bidders} \label{lb-ud-proof-place}
% The proof follows the same structure as the proof of \cref{thm-mua-sm-lb} of describing 
% a distribution $\mathcal D$ and showing that it is hard for every deterministic mechanism. By applying Yao's lemma (\cref{lem:yaos}), we get hardness for randomized obviously strategy-proof mechanisms. 
% Similarly, we prove for the case of two bidders and two items, but the proof extends to any number of bidders and any number of items by adding bidders with the all-zero valuation and assuming that the bidders in our construction have zero values for the additional items. However, the case analysis we employ in this proof is significantly more involved.   

The proof follows the same structure as in \cref{thm-mua-sm-lb}: roughly speaking, we describe a distribution $\mathcal{D}$, show that it is “hard” for deterministic mechanisms, and then use Yao's Lemma to deduce hardness for randomized mechanisms.


The proof outline is as follows. In \cref{subsubsec-description-ud}, 
we describe the  hard distribution $\mathcal D$. 
In \cref{subsubsec-performance-ud} we analyze the allocation and payments of a deterministic \textquote{good} mechanism 
% in several scenarios, 
and explain why these properties  imply that  a \textquote{good} mechanism does not exist in  \cref{subsubsec-contradiction-ud}.
We defer technical proofs to \cref{app-claims-proofs-ud,subsubsec-prop-alloc-ud}. 

We prove for the case of two bidders and two items, but the proof extends to any number of bidders and any number of items by adding bidders with the all-zero valuation and assuming that the bidders in our construction have zero values for the additional items.


\subsubsection[Construction of a ``Hard'' Distribution D]{Construction of a ``Hard'' Distribution $\mathcal{D}$}\label{subsubsec-description-ud}
    To define the probability distribution over the valuation profiles, we name the two items $a$ and $b$, and let $k$ be an arbitrarily large number. Note that since these valuations are unit-demand, we can fully describe them by specifying their value per item.
    Consider the following valuations:  
\[
% \forall S\subseteq M, \quad
v_1^{{a,one}}(x) = 
\begin{cases}
1 & x=a,\\
0 & \text{otherwise.}
\end{cases} \quad 
v_2^{{b,one}}(x) = 
\begin{cases}
1 & x=b,\\
0 & \text{otherwise.}
\end{cases}
\]

\[
% \forall S\subseteq M, \quad
v_1^{{a,mid}}(x) = 
\begin{cases}
k^2 & x=a,\\
0 & \text{otherwise.}
\end{cases} \quad 
v_2^{{b,mid}}(x) = 
\begin{cases}
k^2 & x=b,\\
0 & \text{otherwise.}
\end{cases}
\]

\[
% \forall S\subseteq M, \quad
v_1^{{a,large}}(x) = 
\begin{cases}
k^4 & x=a,\\
0 & \text{otherwise.}
\end{cases} \quad 
v_2^{{b,large}}(x) = 
\begin{cases}
k^4 & x=b,\\
0 & \text{otherwise.}
\end{cases}
\]

\[
% \forall S\subseteq M, \quad
v_1^{{b,small}}(x) = 
\begin{cases}
k & x=b,\\
0 & \text{otherwise.}
\end{cases} \quad 
v_2^{{a,small}}(x) = 
\begin{cases}
k & x=a,\\
0 & \text{otherwise.}
\end{cases}
\]

\[
% \forall S\subseteq M, \quad
v_1^{{b,mid}}(x) = 
\begin{cases}
k^2 & x=b,\\
0 & \text{otherwise.}
\end{cases} \quad 
v_2^{{a,mid}}(x) = 
\begin{cases}
k^2 & x=a,\\
0 & \text{otherwise.}
\end{cases}
\]

\[
% \forall S\subseteq M, \quad
v_1^{{b,large}}(x) = 
\begin{cases}
k^4 & x=b,\\
0 & \text{otherwise.}
\end{cases} \quad 
v_2^{{a,large}}(x) = 
\begin{cases}
k^4 & x=a,\\
0 & \text{otherwise.}
\end{cases}
\]

\[
% \forall S\subseteq M, \quad
v_1^{{both}}(x) = 
\begin{cases}
2k+3 & x=a,\\
2k+1 & x=b,\\
0 & \text{otherwise.}
\end{cases} \quad 
v_2^{{both}}(x) = 
\begin{cases}
2k+3 & x=b,\\
2k+1 & x=a,\\
0 & \text{otherwise.}
\end{cases}
\]


Consider the following valuation profiles: 
\begin{align*}
 I_1 = (v_1^{{a,one}}&, v_2^{{b,one}}),\quad  I_2 = (v_1^{a,mid}, v_2^{{a,large}}),  \quad I_3 = (v_1^{b,large}, v_2^{b,mid}), \quad 
 I_4 = (v_1^{b,mid}, v_2^{b,large}) \\ &I_5 = (v_1^{a,large}, v_2^{a,mid}) \quad I_6 = (v_1^{b,small}, v_2^{b,one}) \quad I_7 =(v_1^{a,one}, v_2^{a,small}) 
\end{align*}
% $I_1 = (v_1^{{a,one}}, v_2^{{b,one}})$, 
% $I_2 = (v_1^{a,mid}, v_2^{{a,large}})$,
% $I_3 = (v_1^{b,large}, v_2^{b,mid})$,
% $I_4 = (v_1^{b,mid}, v_2^{b,large})$,
% $I_5 = (v_1^{a,large}, v_2^{a,mid})$,
% $I_6 = (v_1^{b,small}, v_2^{b,one})$
% and $I_7 =(v_1^{a,one}, v_2^{a,small})$.
Let $\mathcal D$ be the distribution over valuation profiles where the probability of the valuation profile $I_1$ is $\frac{1}{4}$, and the probability of all the valuation profiles $I_2,I_3,I_4,I_5,I_6$ and $I_7$ is $\frac{1}{8}$ each. 

\subsubsection[The Performance of the Deterministic Mechanism A on the "Hard" Distribution D]{The Performance of the Deterministic Mechanism $A$ on the ``Hard'' Distribution $\mathcal{D}$}\label{subsubsec-performance-ud}
We remind that our goal is to show that no deterministic mechanism that satisfies all of the desired properties extracts more than $\frac{7}{8}$ of the optimal welfare. For that, we begin by observing that:
\begin{lemma}\label{lemma-ud-instances}
Every deterministic mechanism that has approximation better than $\frac{7}{8}$ necessarily satisfies all of the following conditions:
\begin{enumerate}
    \item Given the valuation profile $I_1 = (v_1^{\text{a,one}}, v_2^{\text{b,one}})$, the mechanism
    allocates item $a$ to player $1$ and item $b$ to player $2$. \label{condi-1-ud}
    % $A$ outputs an allocation with welfare at most $1$. 
\item Given the valuation profile $I_2 = (v_1^{a,mid}, v_2^{a,large})$, the mechanism allocates item $a$ to bidder $2$. \label{condi-2-ud}

    \item Given the valuation profile $I_3 = (v_1^{b,large}, v_2^{b,mid})$, the mechanism allocates item $b$  to bidder $1$.
    \label{condi-3-ud}
\item Given the valuation profile $I_4 = (v_1^{b,mid}, v_2^{b,large})$, the mechanism allocates item $b$ to bidder $2$. \label{condi-4-ud}
\item Given the valuation profile $I_5 = (v_1^{a,large}, v_2^{a,mid})$, the mechanism allocates item $a$ to bidder $1$. \label{condi-5-ud}
\item Given the valuation profile $I_6 = (v_1^{b,small}, v_2^{b,one})$, the mechanism allocates item $b$ to bidder $1$. \label{condi-6-ud}
\item Given the valuation profile $I_7 = (v_1^{a,one}, v_2^{a,small})$, the mechanism allocates item $a$ to bidder $2$. 
\end{enumerate}
\end{lemma}
The proof of \cref{lemma-ud-instances} is straightforward: if a deterministic mechanism does not satisfy one of the conditions, then due to the fact that $k$ is arbitrarily large, its approximation guarantee is at most $\frac{7}{8}$ with respect to the distribution $\mathcal{D}$.

\subsubsection[Reaching a Contradiction: No Deterministic and OSP Mechanism Succeeds on D]{Reaching a Contradiction: No Deterministic and OSP Mechanism Succeeds on $\mathcal D$}
\label{subsubsec-contradiction-ud}
We now employ \cref{lemma-ud-instances} to prove  \cref{thm-lb-ud}. Let the domain $V_i$ of each bidder consist 
of unit-demand valuations with values in  $\{0,1,\ldots,k^4\}$, where $k$ is an arbitrarily large integer. 


Fix a deterministic mechanism $A$ and strategies
$(\mathcal S_1,\mathcal S_2)$ that are individually rational, satisfy no negative transfers for all the valuations  $V_1\times V_2$ and give approximation better than $\frac{7}{8}$
in expectation over the valuation profiles in the distribution $\mathcal D$. 
Denote with  $(f,P_1,P_2)$ be the allocation rule and the payment scheme that $A$ and $(\mathcal S_1,\mathcal S_2)$ realize, and assume towards a contradiction that $A$ and $(\mathcal S_1,\mathcal S_2)$ are obviously strategy-proof. 


For the analysis of the deterministic mechanism $A$, we focus on the following subsets of the domains of the valuations:
$
\mathcal{V}_1=\{v_1^{a,one},v_1^{a,mid},v_1^{a,large},v_1^{b,small},v_1^{b,mid},v_1^{b,large},v_1^{both}\}$ and  $\mathcal V_2=\{v_2^{b,one},v_2^{b,mid},v_2^{b,large},\allowbreak v_2^{a,small},v_2^{a,mid}, \allowbreak v_2^{a,large},v_2^{both}\}$.
% We now focus on $\mathcal{V}_1\times \mathcal{V}_2$. 
Observe that there necessarily exists a vertex $u$, and valuations $v_1,v_1' \in \mathcal{V}_1$, and  $v_2,v_2' \in \mathcal{V}_2$ such that $(\mathcal{S}_1(v_1), \mathcal{S}_2(v_2))$ diverge at vertex $u$. This follows from \cref{lemma-ud-instances}, which implies that the mechanism $A$ must output different allocations for different valuation profiles in $\mathcal{V}_1 \times \mathcal{V}_2$. Consequently, not all valuation profiles end up in the same leaf, meaning that divergence must occur at some point.

Let $u$ be the first vertex in the protocol such that 
 %the behavior profiles 
 $(\mathcal{S}_1(v_1),\mathcal{S}_2(v_2))$ and $(\mathcal{S}_1(v_1'),\mathcal{S}_2(v_2'))$ diverge, i.e., dictate different messages. 
Note that by definition this implies that $u\in Path(\mathcal{S}_1(v_1),\mathcal{S}_2(v_2))\cap Path(\mathcal{S}_1(v_1'),\mathcal{S}_2(v_2'))$ and that either bidder $1$ or bidder $2$ sends different messages for the valuations in $\mathcal{V}_1$ or $\mathcal V_2$, respectively. 
Without loss of generality, we assume that bidder $1$ sends different messages, meaning that there exist $v_1,v_1'\in \mathcal{V}_1$ such that $\mathcal S_1(v_1)$ and $\mathcal S_1(v_1')$ dictate different messages at vertex $u$.
\begin{comment}
Observe that either bidder $1$ or $2$ has to send different messages for different valuations in $\mathcal V_i$ at some vertex. 
This is an immediate implication of \cref{lemma-ud-instances}. The reason for it is that since 
the mechanism $A$ outputs different allocations for different valuation profiles in $\mathcal V_1\times \mathcal V_2$, 
these valuation profiles end up in different leaves, so they necessarily diverge at some vertex.
% given the valuation profiles $I_1=(v_1^{one},v_2^{one})$ and
%  $I_2=(v_1^{all},v_2^{one})$, meaning that the behavior profiles $(\mathcal S_1(v_1^{one},\mathcal S_2(v_2^{one}))$ and $(\mathcal S_1(v_1^{all},\mathcal S_2(v_2^{one}))$ reach different leaves, so they have to diverge at some vertex. 
 
 Let $u$ be the first vertex in the protocol of the mechanism $A$ where they diverge, meaning that  the behavior profiles $(\mathcal{S}_1(v_1),\mathcal{S}_2(v_2))$ and $(\mathcal{S}_1(v_1'),\mathcal{S}_2(v_2'))$ dictate different messages. 
Note that by definition $u\in Path(\mathcal{S}_1(v_1),\mathcal{S}_2(v_2))\cap Path(\mathcal{S}_1(v_1'),\mathcal{S}_2(v_2'))$. We remind that each vertex is associated with only one player that sends messages in it. Note that the distribution $\mathcal D$ we have defined is symmetric, so we can assume without loss of generality that player $1$ is the player that sends a message in vertex $u$. Thus, there exist $v_1,v_1'\in \mathcal{V}_1$ such that $\mathcal S_1(v_1)$ and $\mathcal S_1(v_1')$ dictate different messages at vertex $u$. 
\end{comment}
However, the following collection of claims show that since the strategy $\mathcal S_1$ is obviously dominant, it dictates the same message for all the valuations in $\mathcal{V}_1$. Thus, we get a contradiction, which completes the proof of \cref{thm-lb-ud}:
% We remind that $\mathcal{V}_1=\{v_1^{a,one},v_1^{a,mid},v_1^{a,large},v_1^{b,small},v_1^{b,mid},v_1^{b,large},v_1^{both}\}$, so
% % Thus,
% the  following claims jointly imply a contradiction, completing the proof of \cref{thm-lb-ud}:
\begin{claim}\label{claim-a-one-mid}
    The strategy $\mathcal S_1$ dictates the same message at vertex $u$ for the valuations $v_1^{a,one}$ and $v_1^{a,mid}$. 
\end{claim}
\begin{claim}\label{claim-a-mid-large}
    The strategy $\mathcal S_1$ dictates the same message at vertex $u$ for the valuations $v_1^{a,mid}$ and $v_1^{a,large}$. 
\end{claim}
\begin{claim}\label{claim-b-small-large}
    The strategy $\mathcal S_1$ dictates the same message at vertex $u$ for the valuations $v_1^{b,small}$ and $v_1^{b,mid}$. 
\end{claim}
\begin{claim}\label{claim-b-mid-large}
    The strategy $\mathcal S_1$ dictates the same message at vertex $u$ for the valuations $v_1^{b,mid}$ and $v_1^{b,large}$. 
\end{claim}
\begin{claim}\label{claim-a-both}
    The strategy $\mathcal S_1$ dictates the same message at vertex $u$ for the valuations $v_1^{both}$ and $v_1^{a,mid}$. 
\end{claim}
\begin{claim}\label{claim-b-both}
    The strategy $\mathcal S_1$ dictates the same message at vertex $u$ for the valuations $v_1^{both}$ and $v_1^{b,mid}$. 
\end{claim}
To prove these claims, we use the following collection of observations about the allocation and the payment scheme of player $1$:    
\begin{lemma}\label{lemma-small-pay-ud}
    The allocation rule $f$ and the payment scheme $P_1$ of bidder $1$ satisfy that:
    % Let $f$ be an allocation rule and let $P_1$ be the payment scheme of bidder $1$ that 
    % The allocation rule $f$ and the  payment scheme $P_1$ $(f,P_1,\ldots,P_n):V_1\times \cdots \times V_n\to \mathbb{R}^{n}$
    % are realized by a dominant-strategy, individually rational and no negative transfers mechanism. Then: 
    \begin{enumerate}
        \item Given the valuation profiles $(v_1^{a,one},v_{2}^{b,one})$ 
        % $(v_1^{a,mid},v_{2}^{b,one})$ 
        and $(v_1^{a,large},v_{2}^{b,one})$, bidder $1$ wins item $a$ and pays at most $1$.  \label{item-1-ud}
        \item  Given the valuation profile $(v_1^{a,mid},v_2^{a,large})$, bidder $1$ wins a bundle that does not contain item $a$ and pays zero.   \label{item-2-ud}
    %     \item  Given $(v_1^{a,large},v_2^{a,large})$, bidder $1$ either: 
    %     \begin{enumerate*}[label=(\alph*)]
    % \item gets a bundle not containing item  $a$ and pays zero \emph{or}
    % \item gets a bundle that contains item $a$ and pays at least $k^2$.
    % \end{enumerate*}
    %     \label{item-3-ud}
        \item Given the valuation profiles $(v_1^{b,small},v_2^{b,one})$,
        $(v_1^{b,mid},v_2^{b,one})$ and $(v_1^{b,large},v_2^{b,one})$,  
        bidder $1$ wins item $b$ and pays at most $k$. 
        \label{item-4-ud}
           \item  Given the valuation profile $(v_1^{b,mid},v_2^{b,large})$, bidder $1$ wins a bundle that does not contain item $b$ and pays zero.   \label{item-4.5-ud}
    % \item  Given $(v_1^{b,large},v_2^{b,large})$, bidder $1$ either: 
    %     \begin{enumerate*}[label=(\alph*)]
    % \item gets a bundle not containing item  $b$ and pays zero \emph{or}
    % \item gets a bundle that contains item $b$ and pays at least $k^2$.
    % \end{enumerate*}
    %     \label{item-5-ud}
\item Given the valuation profile $(v_1^{both},v_2^{b,one})$, 
bidder $1$ gets a bundle that contains item $a$ and pays at most $1$. \label{item-6-ud}
\item Given the valuation profile $(v_1^{both},v_2^{a,large})$, bidder $1$ does not win item $a$.  If bidder $1$ wins item $b$, then he pays at most $2k+1$. If he wins a bundle that contains neither item $a$ or item $b$, then he pays zero. \label{item-complicated-ud}
% \item If the allocation rule $f(v_1^{both},v_2^{a,large})$
%     outputs an allocation where bidder $1$ wins a bundle containing item $b$, then he pays at most $2k+1$. \label{item-7-ud}
% \item If the allocation rule $f(v_1^{both},v_2^{a,large})$
%     outputs an allocation where bidder $1$ wins a bundle not containing item $b$, then he pays zero. \label{item-8-ud}
    \end{enumerate}
\end{lemma}
\subsubsection[All Valuations in V\_1 Send The Same Message: Proofs of Claims \ref{claim-a-one-mid} to \ref{claim-b-both}]{All Valuations in $\mathcal V_1$ Send The Same Message: Proofs of Claims \ref{claim-a-one-mid} to \ref{claim-b-both}}\label{app-claims-proofs-ud}
We now prove Claims \ref{claim-a-one-mid} to \ref{claim-b-both}, which jointly imply a contradiction. All proofs make extensive use of \cref{lemma-ud-instances}, which analyzes the allocation and the payments of any deterministic mechanisms with the desired properties. All proofs are quite similar to each other, and we write them for completeness. The only claim that requires a more involved case analysis is \cref{claim-b-both}.

\begin{proof}[Proof of \cref{claim-a-one-mid}]
      Note that by Lemma \ref{lemma-small-pay-ud} part \ref{item-1-ud}, $f(v_1^{a,one},v_2^{b,one})$ allocates item $a$ to bidder $1$ and $P_1(v_1^{a,one},v_2^{b,one})\le 1$. Therefore:
\begin{equation}\label{eq-good-leaf1-ud}
 v_1^{a,mid}(f(v_1^{a,one},v_2^{b,one}))-P_1(v_1^{a,one},v_2^{b,one})\ge k^2-1   
\end{equation}
 In contrast, by part \ref{item-2-ud} of Lemma \ref{lemma-small-pay-ud}, the allocation rule  $f(v_1^{a,mid},v_2^{a,large})$ allocates no items to player $1$ and $P_1(v_1^{a,mid},v_2^{a,large})=0$, so:
 \begin{equation}\label{eq-bad-leaf1-ud}
 v_1^{a,mid}(f(v_1^{a,mid},v_2^{a,large}))-P_1(v_1^{a,mid},v_2^{a,large})= 0   
\end{equation}
We remind that $k$ is arbitrarily large,
so combining inequalities (\ref{eq-good-leaf1-ud}) and (\ref{eq-bad-leaf1-ud}) gives:
\begin{equation*}
 v_1^{a,mid}(f(v_1^{a,mid},v_2^{a,large}))-P_1(v_1^{a,mid},v_2^{a,large})< 
v_1^{a,mid}(f(v_1^{a,one},v_2^{b,one}))-P_1(v_1^{a,one},v_2^{b,one})  
\end{equation*}
We remind that vertex $u$ belongs in $Path(\mathcal S_1(v_1^{a,one}),\mathcal S_2(v_2^{b,one}))$ and also in
$Path(\mathcal{S}_1(v_1^{a,mid}),\allowbreak\mathcal{S}_2(v_2^{a,large}))$. Therefore, Lemma \ref{lemma-bad-leaf-good-leaf} gives that the strategy $\mathcal S_1$ dictates the same message for  $v_1^{a,one}$ and $v_1^{a,mid}$ at vertex $u$.
\end{proof}

\begin{proof}[Proof of \cref{claim-a-mid-large}]
     By Lemma \ref{lemma-small-pay-ud} part \ref{item-1-ud}, $f(v_1^{a,large},v_2^{b,one})$ allocates item $a$ to bidder $1$ and $P_1(v_1^{a,large},\allowbreak v_2^{b,one})\le 1$. Therefore:
\begin{equation}\label{eq-good-leaf2-ud}
 v_1^{a,mid}(f(v_1^{a,large},v_2^{b,one}))-P_1(v_1^{a,large},v_2^{b,one})\ge k^2-1   
\end{equation}
Whereas by part \ref{item-2-ud} of Lemma \ref{lemma-small-pay-ud}: 
% player $1$ either does not win item $a$ or pays at least $k^2$, so:
 \begin{equation}\label{eq-bad-leaf2-ud}
 v_1^{a,mid}(f(v_1^{a,mid},v_2^{a,large}))-P_1(v_1^{a,mid},v_2^{a,large})=0   
\end{equation}
Combining the inequalities (\ref{eq-good-leaf2-ud}) and (\ref{eq-bad-leaf2-ud}) gives:
\begin{equation*}
 v_1^{a,mid}(f(v_1^{a,mid},v_2^{a,large}))-P_1(v_1^{a,mid},v_2^{a,large})< 
v_1^{a,mid}(f(v_1^{a,large},v_2^{b,one}))-P_1(v_1^{a,large},v_2^{b,one})
\end{equation*}
Note that vertex $u$ belongs in $Path(\mathcal S_1(v_1^{a,large}),\mathcal S_2(v_2^{b,one}))$ and also in
$Path(\mathcal{S}_1(v_1^{a,mid}),\allowbreak\mathcal{S}_2(v_2^{a,large}))$. Therefore, Lemma \ref{lemma-bad-leaf-good-leaf} gives that the strategy $\mathcal S_1$ dictates the same message for  $v_1^{a,mid}$ and $v_1^{a,large}$ at vertex $u$.
\end{proof}
      

\begin{proof}[Proof of \cref{claim-b-small-large}]
        By Lemma \ref{lemma-small-pay-ud} part \ref{item-4-ud}, $f(v_1^{b,small},v_2^{b,one})$ allocates item $b$ to bidder $1$ and $P_1(v_1^{b,small},\allowbreak v_2^{b,one})\le k$. Therefore:
\begin{equation}\label{eq-good-leaf3-ud}
 v_1^{b,mid}(f(v_1^{b,small},v_2^{b,one}))-P_1(v_1^{b,small},v_2^{b,one})\ge k^2-k   
\end{equation}
Whereas by part \ref{item-4.5-ud} of Lemma \ref{lemma-small-pay-ud}: 
% player $1$ either does not win item $a$ or pays at least $k^2$, so:
 \begin{equation}\label{eq-bad-leaf3-ud}
 v_1^{b,mid}(f(v_1^{b,mid},v_2^{b,large}))-P_1(v_1^{b,mid},v_2^{b,large})= 0   
\end{equation}
Combining inequalities (\ref{eq-good-leaf3-ud}) and (\ref{eq-bad-leaf3-ud}) gives:
\begin{equation*}
 v_1^{b,mid}(f(v_1^{b,mid},v_2^{b,large}))-P_1(v_1^{b,mid},v_2^{b,large})< 
v_1^{b,mid}(f(v_1^{b,small},v_2^{b,one}))-P_1(v_1^{b,small},v_2^{b,one})
\end{equation*}
Note that vertex $u$ belongs in $Path(\mathcal S_1(v_1^{b,small}),\mathcal S_2(v_2^{b,one}))$ and also in
$Path(\mathcal{S}_1(v_1^{b,mid}),\allowbreak\mathcal{S}_2(v_2^{b,large}))$. Therefore, Lemma \ref{lemma-bad-leaf-good-leaf} gives that the strategy $\mathcal S_1$ dictates the same message for  $v_1^{b,mid}$ and $v_1^{b,small}$ at vertex $u$.
\end{proof}
\begin{proof}[Proof of \cref{claim-b-mid-large}]
          By Lemma \ref{lemma-small-pay-ud} part \ref{item-4-ud}, $f(v_1^{b,large},v_2^{b,one})$ allocates item $b$ to bidder $1$ and $P_1(v_1^{b,large},\allowbreak v_2^{b,one})\le k$. Therefore:
\begin{equation}\label{eq-good-leaf4-ud}
 v_1^{b,mid}(f(v_1^{b,large},v_2^{b,one}))-P_1(v_1^{b,large},v_2^{b,one})\ge k^2-k   
\end{equation}
Whereas by part \ref{item-4.5-ud} of Lemma \ref{lemma-small-pay-ud}: 
% player $1$ either does not win item $a$ or pays at least $k^2$, so:
 \begin{equation}\label{eq-bad-leaf4-ud}
 v_1^{b,mid}(f(v_1^{b,mid},v_2^{b,large}))-P_1(v_1^{b,mid},v_2^{b,large})= 0   
\end{equation}
Combining inequalities (\ref{eq-good-leaf4-ud}) and (\ref{eq-bad-leaf4-ud}) gives:
\begin{equation*}
 v_1^{b,mid}(f(v_1^{b,mid},v_2^{b,large}))-P_1(v_1^{b,mid},v_2^{b,large})< 
v_1^{b,mid}(f(v_1^{b,large},v_2^{b,one}))-P_1(v_1^{b,large},v_2^{b,one})
\end{equation*}
Note that vertex $u$ belongs in $Path(\mathcal S_1(v_1^{b,large}),\mathcal S_2(v_2^{b,one}))$ and also in
$Path(\mathcal{S}_1(v_1^{b,mid}),\allowbreak\mathcal{S}_2(v_2^{b,large}))$. Therefore, Lemma \ref{lemma-bad-leaf-good-leaf} gives that the strategy $\mathcal S_1$ dictates the same message for  $v_1^{b,mid}$ and $v_1^{b,large}$ at vertex $u$.
\end{proof}
\begin{proof}[Proof of \cref{claim-a-both}]
            By Lemma \ref{lemma-small-pay-ud} part \ref{item-6-ud}, $f(v_1^{both},v_2^{b,one})$ allocates item $a$ to bidder $1$ and $P_1(v_1^{both},\allowbreak v_2^{b,one})\le 1$. Therefore:
\begin{equation}\label{eq-good-leaf5-ud}
 v_1^{a,mid}(f(v_1^{both},v_2^{b,one}))-P_1(v_1^{both},v_2^{b,one})\ge k^2-1   
\end{equation}
Whereas by part \ref{item-2-ud} of Lemma \ref{lemma-small-pay-ud}: 
% player $1$ either does not win item $a$ or pays at least $k^2$, so:
 \begin{equation}\label{eq-bad-leaf5-ud}
 v_1^{a,mid}(f(v_1^{a,mid},v_2^{a,large}))-P_1(v_1^{a,mid},v_2^{a,large})=0   
\end{equation}
Combining the inequalities (\ref{eq-good-leaf5-ud}) and (\ref{eq-bad-leaf5-ud}) gives:
\begin{equation*}
 v_1^{a,mid}(f(v_1^{a,mid},v_2^{a,large}))-P_1(v_1^{a,mid},v_2^{a,large})< 
 v_1^{a,mid}(f(v_1^{both},v_2^{b,one}))-P_1(v_1^{both},v_2^{b,one})
\end{equation*}
Note that vertex $u$ belongs in $Path(\mathcal S_1(v_1^{both}),\mathcal S_2(v_2^{b,one}))$ and also in
$Path(\mathcal{S}_1(v_1^{a,mid}),\allowbreak\mathcal{S}_2(v_2^{a,large}))$. Therefore, Lemma \ref{lemma-bad-leaf-good-leaf} gives that the strategy $\mathcal S_1$ dictates the same message for  $v_1^{a,mid}$ and $v_1^{both}$ at vertex $u$.
\end{proof}
\begin{proof}[Proof of \cref{claim-b-both}]
    % To prove that the strategy $\mathcal S_1$ dictates the same message for $v_1^{both}$ and $v_1^{b,mid}$, {we consider at the following cases.} 
    Note that by \cref{lemma-small-pay-ud} part \ref{item-complicated-ud}, given the valuation profile $(v_1^{both},v_2^{a,large})$, bidder $1$ cannot win item $a$ so we consider the following two cases: the case where he wins item $b$ and the case where he wins neither of these items.

    Assume that bidder $1$ wins item $b$ given the valuation profile $(v_1^{both},v_2^{a,large})$. Note that in this case, \cref{lemma-small-pay-ud} part \ref{item-complicated-ud} also implies that:
    \begin{equation}\label{eq-good-leaf6-ud}
 v_1^{b,mid}(f(v_1^{both},v_2^{a,large}))-P_1(v_1^{both},v_2^{a,large})\ge k^2 -2k-1   
\end{equation}
Whereas by part \ref{item-4.5-ud} of Lemma \ref{lemma-small-pay-ud}: 
% player $1$ either does not win item $a$ or pays at least $k^2$, so:
 \begin{equation}\label{eq-bad-leaf6-ud}
 v_1^{b,mid}(f(v_1^{b,mid},v_2^{b,large}))-P_1(v_1^{b,mid},v_2^{b,large})= 0   
\end{equation}
Combining the inequalities (\ref{eq-good-leaf6-ud}) and (\ref{eq-bad-leaf6-ud}) gives:
\begin{equation*}
 v_1^{b,mid}(f(v_1^{b,mid},v_2^{b,large}))-P_1(v_1^{b,mid},v_2^{b,large})< 
 v_1^{b,mid}(f(v_1^{both},v_2^{a,large}))-P_1(v_1^{both},v_2^{a,large})
\end{equation*}
Note that vertex $u$ belongs in $Path(\mathcal S_1(v_1^{both}),\mathcal S_2(v_2^{a,large}))$ and also in
$Path(\mathcal{S}_1(v_1^{a,mid}),\allowbreak\mathcal{S}_2(v_2^{a,large}))$. Therefore, Lemma \ref{lemma-bad-leaf-good-leaf} gives that the strategy $\mathcal S_1$ dictates the same message for  $v_1^{both}$ and $v_1^{b,mid}$ at vertex $u$, which concludes this case.

For the latter case, where bidder $1$ gets a bundle that contains neither item $a$ nor item $b$ given the valuation profile $(v_1^{both},v_2^{a,large})$, by \cref{lemma-small-pay-ud} part \ref{item-complicated-ud}: 
\begin{equation}\label{eq-bad-leaf7-ud}
 v_1^{both}(f(v_1^{both},v_2^{a,large}))-P_1(v_1^{both},v_2^{a,large})=0   
\end{equation}
Whereas by part \ref{item-4-ud} of Lemma \ref{lemma-small-pay-ud}: 
% player $1$ either does not win item $a$ or pays at least $k^2$, so:
 \begin{equation}\label{eq-good-leaf7-ud}
 v_1^{both}(f(v_1^{b,mid},v_2^{b,one}))-P_1(v_1^{b,mid},v_2^{b,one}) \ge 2k+1-k=k+1   
\end{equation}
Combining (\ref{eq-bad-leaf7-ud}) and  (\ref{eq-good-leaf7-ud})  gives:
\begin{equation*}
 v_1^{both}(f(v_1^{both},v_2^{a,large}))-P_1(v_1^{both},v_2^{a,large})< 
 v_1^{both}(f(v_1^{b,mid},v_2^{b,one}))-P_1(v_1^{b,mid},v_2^{b,one})
\end{equation*}
Note that vertex $u$ belongs in $Path(\mathcal S_1(v_1^{both}),\mathcal S_2(v_2^{a,large}))$ and also in
$Path(\mathcal{S}_1(v_1^{b,mid}),\allowbreak\mathcal{S}_2(v_2^{b,one}))$. Therefore, Lemma \ref{lemma-bad-leaf-good-leaf} gives that the strategy $\mathcal S_1$ dictates the same message for  $v_1^{both}$ and $v_1^{b,mid}$ at vertex $u$, which resolves the second case and completes the proof.
\end{proof}

\subsubsection{Proof of Lemma \ref{lemma-small-pay-ud}:
Observations About The Mechanism} \label{subsubsec-prop-alloc-ud}
The proof is a direct consequence of the approximation guarantee of the mechanism and the fact that it is 
% the mechanisms in all of them are 
obviously strategy-proof (and thus also dominant-strategy incentive compatible) and satisfies individual rationality and no negative transfers. 
We write the proof  for the sake of completeness.

Throughout the proof, we say that an item $x$ is \emph{valuable} for a valuation $v$ if $v(\{x\})>0$.

\begin{proof}[Proof of \cref{lemma-small-pay-ud}]
    
For part \ref{item-1-ud}, note that because of individual rationality:
\begin{equation}\label{eq-1-ud}
 v_1^{a,one}(f(v_1^{a,one},v_2^{b,one}))-P_1(v_1^{a,one},v_2^{b,one})\ge 0   
\end{equation}
We remind that by \cref{lemma-ud-instances} part \ref{condi-1-ud}, 
the allocation rule $f$ allocates item $a$ to bidder $1$ given $(v_1^{one},v_2^{one})$, so:
\begin{equation}\label{eq-2-ud}
 v_1^{a,one}(f(v_1^{a,one},v_2^{b,one}))=1   
\end{equation}
Combining (\ref{eq-1-ud}) and (\ref{eq-2-ud}) proves the part \ref{item-1-ud} for the valuation profile $(v_1^{a,one},v_2^{b,one})$.
Observe that it also implies that: 
% \begin{equation}\label{mid-one-eq-ud-val}
%     v_1^{a,mid}(f(v_1^{a,one},v_2^{b,one}))=k^2
% \end{equation}
\begin{equation*}\label{mid-one-eq-ud}
v_1^{a,large}(f(v_1^{a,one},v_2^{b,one}))-P_1(v_1^{a,one},v_2^{b,one})\ge k^4-1
\end{equation*}
% To analyze the allocation and payment given $(v_1^{a,mid},v_2^{b,one})$, observe that since 
% $f(v_1^{a,one},v_2^{b,one})$ allocates to bidder $1$ wins item $a$ and pays at most $1$ implies that:
Combining this inequality with the fact that the allocation rule $f$ and the payment scheme $P_1$ are realized by a dominant-strategy mechanism gives that:
\begin{equation}\label{eq-part1-unit}
\begin{aligned}
    v_1^{a,large}(f(v_1^{a,large},v_2^{b,one}))-P_1(v_1^{a,large},v_2^{b,one})&\ge v_1^{a,large}(f(v_1^{a,one},v_2^{b,one}))-P_1(v_1^{a,one},v_2^{b,one}) \\
    &\ge k^4-1 
    % &\text{(by (\ref{mid-one-eq-ud}))} 
\end{aligned}  
\end{equation}
Note that the property of no negative transfers implies that $P_1(v_1^{a,large},v_2^{b,one})\ge 0$, so
$v_1^{a,large}(f(v_1^{a,large},\allowbreak v_2^{b,one}))\ge k^4-1$. Since only item $a$ is valuable for  $v_1^{a,large}$, we can deduce that player $1$ wins it given the valuation profile $(v_1^{a,large}\allowbreak,v_2^{b,one})$, which further implies that in fact:
\begin{equation}\label{eq-part1-another}
v_1^{a,large}(f(v_1^{a,large}\allowbreak,v_2^{b,one}))= k^4    
\end{equation}
Now, combining (\ref{eq-part1-unit}) and (\ref{eq-part1-another}) gives that $P_1(v_1^{a,large},v_2^{b,one})\le 1$, which completes the proof for $(v_1^{a,large},v_1^{b,one})$. 
% The proof that given $(v_1^{a,large},v_2^{b,one})$, bidder $1$ wins item $a$ and pays at most $1$ is analogous to the proof for $(v_1^{a,mid},v_1^{b,one})$.   

For part \ref{item-2-ud}, observe that by  \cref{lemma-ud-instances} part \ref{condi-2-ud}, given $(v_1^{a,mid},v_2^{a,large})$, bidder $2$ gets item $a$, so clearly bidder $1$ does not get it. Since only item $a$ is valuable for $v_1^{a,mid}$, we get that $v_1^{a,mid}(f(v_1^{a,mid},v_2^{a,large}))\allowbreak=0$, so by individual rationality $P_1(v_1^{a,mid},v_2^{a,large})\le 0$. Due to no negative transfers, we get that $P_1(v_1^{a,mid},v_2^{a,large})=0$, which completes the proof of this part. 

We now prove part \ref{item-4-ud}. We begin by proving it for the valuation profile $(v_1^{b,small},v_2^{b,one})$. Note that by \cref{lemma-ud-instances} part \ref{condi-6-ud}, given this valuation profile, bidder $1$ wins item $b$. Combining this fact with the fact that the mechanism satisfies individual rationality implies that $P_1(v_1^{b,small},v_2^{b,one})\le k$, as needed. Showing that the same goes for $(v_1^{b,mid},v_2^{b,one})$ and $(v_1^{b,large},v_2^{b,one})$ is analogous to the proof of part \ref{item-1-ud} for the valuation profile $(v_1^{a,mid},v_2^{b,one})$ above.

To prove \ref{item-4.5-ud}, note that by \cref{lemma-ud-instances} part \ref{condi-4-ud}, bidder $1$ does not win item $b$ given $(v_1^{b,mid},v_2^{b,large})$. Thus, he has to pay at most zero due to individual rationality, and the property of no negative transfers implies that $P_1(v_1^{b,mid},v_2^{b,large})=0$, as needed.  

For part \ref{item-6-ud}, we first show that bidder $1$ wins a bundle that contains item $a$ given $(v_1^{both},v_2^{b,one})$. Note that since the allocation rule $f$ and the payment scheme $P_1$ are realized by a dominant-strategy mechanism, we have that:
\begin{equation}\label{eq-part6-unit}
\begin{aligned}
    v_1^{both}(f(v_1^{both},v_2^{b,one}))-P_1(v_1^{both},v_2^{b,one})&\ge v_1^{both}(f(v_1^{a,one},v_2^{b,one}))-P_1(v_1^{a,one},v_2^{b,one}) \\
    &\ge 2k+2 &\text{(by part \ref{item-1-ud})}
\end{aligned}    
\end{equation}
Combining (\ref{eq-part6-unit}) with the property of no negative transfers implies that $v_1^{both}(f(v_1^{both},v_2^{b,one}))\ge 2k+2$. Thus, given $(v_1^{both},v_2^{b,one})$, bidder $1$ necessarily gets a bundle that contains item $a$. 


For the upper bound on the payment, note that in fact $v_1^{both}(f(v_1^{both},v_2^{b,one}))=2k+3$.  
% and that by part \ref{item-1-ud}, $P_1(v_1^{a,one},v_2^{b,one})\le 1$. 
Combining it with inequality (\ref{eq-part6-unit}) gives that $P_1(v_1^{both},v_2^{b,one})\le 1$. By that, we  complete the proof of part \ref{item-6-ud}. 

We are now finally ready to wrap up by proving part \ref{item-complicated-ud}. We begin by showing that given $(v_1^{both},v_2^{a,large})$, bidder $1$ does not win item $a$. This is due to weak monotonicity. Formally, we remind that by part \ref{item-2-ud}, given $(v_1^{a,mid},v_2^{a,large})$, bidder $1$ wins a bundle $S^{mid}$
that does not contain $a$. Thus, if $f(v_1^{both},v_2^{a,large})$ allocates to bidder $1$ 
a bundle $S^{both}$ that contains item $a$, then $v_1^{a,mid}(S^{both})-v_1^{a,mid}(S^{mid})=k^2$ whereas $v_1^{both}(S^{both})-v_1^{both}(S^{mid})=2k+3$, so $f$ is not weakly monotone. Since $f$ and $P_1$ realize a dominant-strategy mechanism, \cref{wmon-lemma} gives that $f$ has to be weakly-monotone, so we get a contradiction. Thus, $f(v_1^{both},v_2^{a,large})$ has to allocate bidder $1$ a bundle that does not contain item $a$. 

The bounds on the payments are a straightforward implication of individual rationality and no negative transfers. If $f(v_1^{both},v_2^{a,large})$ allocates item $b$ to bidder $1$ then the payment is at most $2k+1$, and if it allocates to bidder $1$ no valuable items, then the payment has to be at most zero. Due to no negative transfers, it is zero exactly.   
\end{proof}

\subsection{Proof of Theorem \ref{thm-lb-add}: An Impossibility for Additive Bidders} \label{lb-add-proof-place}
The proof follows the same structure as in \cref{thm-mua-sm-lb} and \cref{thm-lb-ud}: roughly speaking, we describe a distribution $\mathcal{D}$, show that it is “hard” for deterministic mechanisms, and then use Yao's Lemma to deduce hardness for randomized mechanisms. In particular, this proof is very similar to the proof of \cref{thm-lb-ud} in \cref{lb-ud-proof-place}, and we write both for the sake of completeness. 

The proof outline is as follows. In \cref{subsubsec-description-add}, 
we describe the  hard distribution $\mathcal D$. 
In \cref{subsubsec-performance-add} we analyze the allocation and payments of a deterministic \textquote{good} mechanism 
% in several scenarios, 
and explain why these properties  imply that  a \textquote{good} mechanism does not exist in  \cref{subsubsec-contradiction-add}.
We defer technical proofs to \cref{app-claims-proofs-add,subsubsec-prop-alloc-add}. 

We prove for the case of two bidders and two items, but the proof extends to any number of bidders and any number of items by adding bidders with the all-zero valuation and assuming that the bidders in our construction have zero values for the additional items.

\begin{comment}
The outline of the proof is as follows. 
In \cref{subsubsec-description-add},
we describe the  hard distribution $\mathcal D$. 
In \cref{subsubsec-performance-add}, we state the allocation of a deterministic mechanism 
that gives an approximation better than $\frac{7}{8}$ to the optimal welfare.  
We conclude by further analyzing the allocation and payments of a deterministic \textquote{good} mechanism in several scenarios, and explaining why these properties  imply that  a \textquote{good} mechanism does not exist (\cref{subsubsec-contradiction-add}).
We defer  technical proofs to \cref{app-claims-proofs-add} and \cref{subsubsec-prop-alloc-add}.

We prove for the case of two bidders and two items, but the proof extends to any number of bidders and any number of items by adding bidders with the all-zero valuation and assuming that the bidders in our construction have zero values for the additional items.
\end{comment}

\subsubsection[Construction of a "Hard" Distribution D]{Construction of a \textquote{Hard} Distribution $\mathcal D$} \label{subsubsec-description-add}
    To define the probability distribution over the valuation profiles, we name the two items $a$ and $b$, and let $k$ be an arbitrarily large number. Note that since these valuations are additive, we can fully describe them by specifying their value for each item.
    Consider the following valuations:  
\[
% \forall S\subseteq M, \quad
v_1^{{a,one}}(x) = 
\begin{cases}
1 & x=a,\\
0 & \text{otherwise.}
\end{cases} \quad 
v_2^{{b,one}}(x) = 
\begin{cases}
1 & x=b,\\
0 & \text{otherwise.}
\end{cases}
\]

\[
% \forall S\subseteq M, \quad
v_1^{{a,mid}}(x) = 
\begin{cases}
k^2 & x=a,\\
0 & \text{otherwise.}
\end{cases} \quad 
v_2^{{b,mid}}(x) = 
\begin{cases}
k^2 & x=b,\\
0 & \text{otherwise.}
\end{cases}
\]

\[
% \forall S\subseteq M, \quad
v_1^{{a,large}}(x) = 
\begin{cases}
k^4 & x=a,\\
0 & \text{otherwise.}
\end{cases} \quad 
v_2^{{b,large}}(x) = 
\begin{cases}
k^4 & x=b,\\
0 & \text{otherwise.}
\end{cases}
\]

\[
% \forall S\subseteq M, \quad
v_1^{{b,small}}(x) = 
\begin{cases}
k & x=b,\\
0 & \text{otherwise.}
\end{cases} \quad 
v_2^{{a,small}}(x) = 
\begin{cases}
k & x=a,\\
0 & \text{otherwise.}
\end{cases}
\]

\[
% \forall S\subseteq M, \quad
v_1^{{b,mid}}(x) = 
\begin{cases}
k^2 & x=b,\\
0 & \text{otherwise.}
\end{cases} \quad 
v_2^{{a,mid}}(x) = 
\begin{cases}
k^2 & x=a,\\
0 & \text{otherwise.}
\end{cases}
\]

\[
% \forall S\subseteq M, \quad
v_1^{{b,large}}(x) = 
\begin{cases}
k^4 & x=b,\\
0 & \text{otherwise.}
\end{cases} \quad 
v_2^{{a,large}}(x) = 
\begin{cases}
k^4 & x=a,\\
0 & \text{otherwise.}
\end{cases}
\]

\[
% \forall S\subseteq M, \quad
v_1^{{both}}(x) = 
\begin{cases}
2k+3 & x=a,\\
2k+1 & x=b,\\
0 & \text{otherwise.}
\end{cases} \quad 
v_2^{{both}}(x) = 
\begin{cases}
2k+3 & x=b,\\
2k+1 & x=a,\\
0 & \text{otherwise.}
\end{cases}
\]
Note that all the valuations except for $v_1^{both}$ and $v_2^{both}$ are unit-demand and additive simultaneously. 
Consider the following valuation profiles: 
\begin{align*}
 I_1 = (v_1^{{a,one}},& v_2^{{b,one}}), \quad  I_2 = (v_1^{a,mid}, v_2^{{a,large}}),  \quad I_3 = (v_1^{b,large}, v_2^{b,mid}), \quad
 I_4 = (v_1^{b,mid}, v_2^{b,large}) \\ & I_5 = (v_1^{a,large}, v_2^{a,mid}) \quad I_6 = (v_1^{b,small}, v_2^{b,one}) \quad I_7 =(v_1^{a,one}, v_2^{a,small}) 
\end{align*}
% $I_1 = (v_1^{{a,one}}, v_2^{{b,one}})$, 
% $I_2 = (v_1^{a,mid}, v_2^{{a,large}})$,
% $I_3 = (v_1^{b,large}, v_2^{b,mid})$,
% $I_4 = (v_1^{b,mid}, v_2^{b,large})$,
% $I_5 = (v_1^{a,large}, v_2^{a,mid})$,
% $I_6 = (v_1^{b,small}, v_2^{b,one})$
% and $I_7 =(v_1^{a,one}, v_2^{a,small})$.
Let $\mathcal D$ be the distribution over valuation profiles where the probability of the valuation profile $I_1$ is $\frac{1}{4}$, and the probability of the valuation profiles $I_2,I_3,I_4,I_5,I_6$ and $I_7$ is $\frac{1}{8}$ each. 

\subsubsection[The Performance of the Deterministic Mechanism A on the "Hard" Distribution D]{The Performance of the Deterministic Mechanism $A$ on the \textquote{Hard} Distribution $\mathcal D$}\label{subsubsec-performance-add}
We remind that our goal is to show that no deterministic mechanism that satisfies all of the desired properties extracts more than $\frac{7}{8}$ of the optimal welfare. For that, we begin by observing that:
\begin{lemma}\label{lemma-add-instances}
Every deterministic mechanism that has approximation better than $\frac{8}{7}$ necessarily satisfies all of the following conditions:
\begin{enumerate}
    \item Given the valuation profile $I_1 = (v_1^{\text{a,one}}, v_2^{\text{b,one}})$, the mechanism
    allocates item $a$ to player $1$ and item $b$ to player $2$. \label{condi-1-add}
    % $A$ outputs an allocation with welfare at most $1$. 
\item Given the valuation profile $I_2 = (v_1^{a,mid}, v_2^{a,large})$, the mechanism allocates item $a$ to bidder $2$. \label{condi-2-add}

    \item Given the valuation profile $I_3 = (v_1^{b,large}, v_2^{b,mid})$, the mechanism allocates item $b$  to bidder $1$.
    \label{condi-3-add}
\item Given the valuation profile $I_4 = (v_1^{b,mid}, v_2^{b,large})$, the mechanism allocates item $b$ to bidder $2$. \label{condi-4-add}
\item Given the valuation profile $I_5 = (v_1^{a,large}, v_2^{a,mid})$, the mechanism allocates item $a$ to bidder $1$. \label{condi-5-add}
\item Given the valuation profile $I_6 = (v_1^{b,small}, v_2^{b,one})$, the mechanism allocates item $b$ to bidder $1$. \label{condi-6-add}
\item Given the valuation profile $I_7 = (v_1^{a,one}, v_2^{a,small})$, the mechanism allocates item $a$ to bidder $2$. 
\end{enumerate}
\end{lemma}
The proof of \cref{lemma-add-instances} is straightforward and identical to the proof of \cref{lemma-ud-instances}: if a deterministic mechanism does not satisfy one of conditions, then the fact that $k$ is arbitrarily large implies that its approximation guarantee  is at most $\frac{8}{7}$ with respect to the distribution $\mathcal D$. 
\subsubsection[Reaching a Contradiction: No Deterministic and OSP Mechanism Succeeds on D]{Reaching a Contradiction: No Deterministic and OSP Mechanism Succeeds on $\mathcal D$}
\label{subsubsec-contradiction-add}
We now employ \cref{lemma-add-instances} to prove  \cref{thm-lb-add}. Let the domain $V_i$ of each bidder consist 
of additive valuations with values in  $\{0,1,\ldots,k^4\}$, where $k$ is an arbitrarily large integer. 


Fix a deterministic mechanism $A$ and strategies
$(\mathcal S_1,\mathcal S_2)$ that are individually rational, satisfy no negative transfers for all the valuations  $V_1\times V_2$ and give approximation better than $\frac{7}{8}$
in expectation over the valuation profiles in the distribution $\mathcal D$. 
Denote with  $(f,P_1,P_2)$ be the allocation rule and the payment scheme that $A$ and $(\mathcal S_1,\mathcal S_2)$ realize, and assume towards a contradiction that $A$ and $(\mathcal S_1,\mathcal S_2)$ are obviously strategy-proof. 


For the analysis of the deterministic mechanism $A$, we focus on the following subsets of the domains of the valuations:
$
\mathcal{V}_1=\{v_1^{a,one},v_1^{a,mid},v_1^{a,large},v_1^{b,small},v_1^{b,mid},v_1^{b,large},v_1^{both}\}$ and  $\mathcal V_2=\{v_2^{b,one},v_2^{b,mid},v_2^{b,large},\allowbreak v_2^{a,small},v_2^{a,mid}, \allowbreak v_2^{a,large},v_2^{both}\}$.
Observe that there necessarily exists a vertex $u$, and valuations $v_1,v_1' \in \mathcal{V}_1$, and  $v_2,v_2' \in \mathcal{V}_2$ such that $(\mathcal{S}_1(v_1), \mathcal{S}_2(v_2))$ diverge at vertex $u$. This follows from \cref{lemma-add-instances}, which implies that the mechanism $A$ must output different allocations for different valuation profiles in $\mathcal{V}_1 \times \mathcal{V}_2$. Consequently, not all valuation profiles end up in the same leaf, meaning that divergence must occur at some point.

Let $u$ be the first vertex in the protocol such that 
 %the behavior profiles 
 $(\mathcal{S}_1(v_1),\mathcal{S}_2(v_2))$ and $(\mathcal{S}_1(v_1'),\mathcal{S}_2(v_2'))$ diverge, i.e., dictate different messages. 
Note that by definition this implies that $u\in Path(\mathcal{S}_1(v_1),\mathcal{S}_2(v_2))\cap Path(\mathcal{S}_1(v_1'),\mathcal{S}_2(v_2'))$ and that either bidder $1$ or bidder $2$ sends different messages for the valuations in $\mathcal{V}_1$ or $\mathcal V_2$, respectively. 
Without loss of generality, we assume that bidder $1$ sends different messages, meaning that there exist $v_1,v_1'\in \mathcal{V}_1$ such that $\mathcal S_1(v_1)$ and $\mathcal S_1(v_1')$ dictate different messages at vertex $u$.
\begin{comment}
Observe that either bidder $1$ or $2$ has to send different messages for different valuations in $\mathcal V_i$ at some vertex. 
This is an immediate implication of \cref{lemma-add-instances}. Due to \cref{lemma-add-instances}, 
the mechanism $A$ outputs different allocations for different valuation profiles in $\mathcal V_1\times \mathcal V_2$, so they necessarily end up in different leaves, so they necessarily diverge at some vertex.
% given the valuation profiles $I_1=(v_1^{one},v_2^{one})$ and
%  $I_2=(v_1^{all},v_2^{one})$, meaning that the behavior profiles $(\mathcal S_1(v_1^{one},\mathcal S_2(v_2^{one}))$ and $(\mathcal S_1(v_1^{all},\mathcal S_2(v_2^{one}))$ reach different leaves, so they have to diverge at some vertex. 
 
 Let $u$ be the first vertex in the protocol of the mechanism $A$ where they diverge, meaning that  the behavior profiles $(\mathcal{S}_1(v_1),\mathcal{S}_2(v_2))$ and $(\mathcal{S}_1(v_1'),\mathcal{S}_2(v_2'))$ dictate different messages. 
Note that by definition $u\in Path(\mathcal{S}_1(v_1),\mathcal{S}_2(v_2))\cap Path(\mathcal{S}_1(v_1'),\mathcal{S}_2(v_2'))$. We remind that each vertex is associated with only one player that sends messages in it. Note that the distribution $\mathcal D$ we have defined is symmetric, so we can assume without loss of generality that player $1$ is the player that sends a message in vertex $u$. Thus, there exist $v_1,v_1'\in \mathcal{V}_1$ such that $\mathcal S_1(v_1)$ and $\mathcal S_1(v_1')$ dictate different messages at vertex $u$. 
\end{comment}
However, the following collection of claims show that since the strategy $\mathcal S_1$ is obviously dominant, it dictates the same message for all the valuations in $\mathcal{V}_1$. Thus, we get a contradiction, which completes the proof of \cref{thm-lb-add}:
\begin{claim}\label{claim-a-one-mid-add}
    The strategy $\mathcal S_1$ dictates the same message at vertex $u$ for the valuations $v_1^{a,one}$ and $v_1^{a,mid}$. 
\end{claim}
\begin{claim}\label{claim-a-mid-large-add}
    The strategy $\mathcal S_1$ dictates the same message at vertex $u$ for the valuations $v_1^{a,mid}$ and $v_1^{a,large}$. 
\end{claim}
\begin{claim}\label{claim-b-small-large-add}
    The strategy $\mathcal S_1$ dictates the same message at vertex $u$ for the valuations $v_1^{b,small}$ and $v_1^{b,mid}$. 
\end{claim}
\begin{claim}\label{claim-b-mid-large-add}
    The strategy $\mathcal S_1$ dictates the same message at vertex $u$ for the valuations $v_1^{b,mid}$ and $v_1^{b,large}$. 
\end{claim}
\begin{claim}\label{claim-a-both-add}
    The strategy $\mathcal S_1$ dictates the same message at vertex $u$ for the valuations $v_1^{both}$ and $v_1^{a,mid}$. 
\end{claim}
\begin{claim}\label{claim-b-both-add}
    The strategy $\mathcal S_1$ dictates the same message at vertex $u$ for the valuations $v_1^{both}$ and $v_1^{b,mid}$. 
\end{claim}
To prove these claims, we use the following observations about the allocation and the payment scheme of player $1$:    
\begin{lemma}\label{lemma-small-pay-add}
    The allocation rule $f$ and the payment scheme $P_1$ of bidder $1$ satisfy that:
    % Let $f$ be an allocation rule and let $P_1$ be the payment scheme of bidder $1$ that 
    % The allocation rule $f$ and the  payment scheme $P_1$ $(f,P_1,\ldots,P_n):V_1\times \cdots \times V_n\to \mathbb{R}^{n}$
    % are realized by a dominant-strategy, individually rational and no negative transfers mechanism. Then: 
    \begin{enumerate}
        \item Given the valuation profiles $(v_1^{a,one},v_{2}^{b,one})$ and $(v_1^{a,large},v_{2}^{b,one})$, bidder $1$ wins item $a$ and pays at most $1$.  \label{item-1-add}
        \item  Given the valuation profile $(v_1^{a,mid},v_2^{a,large})$, bidder $1$ wins a bundle that does not contain item $a$ and pays zero.   \label{item-2-add}
        \item Given the valuation profiles $(v_1^{b,small},v_2^{b,one})$,
        $(v_1^{b,mid},v_2^{b,one})$ and $(v_1^{b,large},v_2^{b,one})$,  
        bidder $1$ wins item $b$ and pays at most $k$. 
        \label{item-4-add}
           \item  Given the valuation profile $(v_1^{b,mid},v_2^{b,large})$, bidder $1$ wins a bundle that does not contain item $b$ and pays zero.   \label{item-4.5-add}
\item Given the valuation profile $(v_1^{both},v_2^{b,one})$, 
bidder $1$ gets a bundle that contains item $a$ and pays at most $4k+4$. \label{item-6-add}
\item Given the valuation profile $(v_1^{both},v_2^{a,large})$, bidder $1$ does not win item $a$.  If bidder $1$ wins item $b$, then he pays at most $2k+1$. If he wins a bundle that contains neither item $a$ or item $b$, then he pays zero. \label{item-complicated-add}
    \end{enumerate}
\end{lemma}
\subsubsection[All Valuations in V\_1 Send The Same Message: Proofs of Claims \ref{claim-a-one-mid-add} to \ref{claim-b-both-add}]{All Valuations in $\mathcal V_1$ Send The Same Message: Proofs of Claims \ref{claim-a-one-mid-add} to \ref{claim-b-both-add}}\label{app-claims-proofs-add}
We will now prove only \cref{claim-a-both-add} and \cref{claim-b-both-add}, since the proofs of the rest of the claims in fact appear in \cref{app-claims-proofs-ud}: the proof of \cref{claim-a-one-mid-add} is identical to the proof of \cref{claim-a-one-mid}, and the same goes for the proofs of  \cref{claim-a-mid-large-add} and \cref{claim-a-mid-large}, 
the proof of \cref{claim-b-small-large-add} and  \cref{claim-b-small-large} and the proof of \cref{claim-b-mid-large-add} and \cref{claim-b-mid-large}.

% We now prove claims \ref{claim-a-one-mid-add} to \ref{claim-b-both-add}, which jointly imply a contradiction. All proofs repeatedly use \cref{lemma-add-instances}, which analyzes the allocation and the payments of any deterministic mechanisms with the desired properties. 
% % Also, they are quite similar and we write them for completeness. 
% Only \cref{claim-a-both-add} and \cref{claim-a-both-add} require a proof. 



\begin{proof}[Proof of \cref{claim-a-both-add}]
            By Lemma \ref{lemma-small-pay-add} part \ref{item-6-add}, $f(v_1^{both},v_2^{b,one})$ allocates item $a$ to bidder $1$ and $P_1(v_1^{both},\allowbreak v_2^{b,one})\le 4k+4$. Therefore:
\begin{equation}\label{eq-good-leaf5-add}
 v_1^{a,mid}(f(v_1^{both},v_2^{b,one}))-P_1(v_1^{both},v_2^{b,one})\ge k^2-4k-4   
\end{equation}
Whereas by part \ref{item-2-add} of Lemma \ref{lemma-small-pay-add}: 
% player $1$ either does not win item $a$ or pays at least $k^2$, so:
 \begin{equation}\label{eq-bad-leaf5-add}
 v_1^{a,mid}(f(v_1^{a,mid},v_2^{a,large}))-P_1(v_1^{a,mid},v_2^{a,large})=0   
\end{equation}
Combining the inequalities (\ref{eq-good-leaf5-add}) and (\ref{eq-bad-leaf5-add}) gives:
\begin{equation*}
 v_1^{a,mid}(f(v_1^{a,mid},v_2^{a,large}))-P_1(v_1^{a,mid},v_2^{a,large})< 
 v_1^{a,mid}(f(v_1^{both},v_2^{b,one}))-P_1(v_1^{both},v_2^{b,one})
\end{equation*}
Note that vertex $u$ belongs in $Path(\mathcal S_1(v_1^{both}),\mathcal S_2(v_2^{b,one}))$ and also in
$Path(\mathcal{S}_1(v_1^{a,mid}),\allowbreak\mathcal{S}_2(v_2^{a,large}))$. Therefore, Lemma \ref{lemma-bad-leaf-good-leaf} gives that the strategy $\mathcal S_1$ dictates the same message for  $v_1^{a,mid}$ and $v_1^{both}$ at vertex $u$.
\end{proof}
\begin{proof}[Proof of \cref{claim-b-both-add}]
    % To prove that the strategy $\mathcal S_1$ dictates the same message for $v_1^{both}$ and $v_1^{b,mid}$, {we consider at the following cases.} 
    Note that by \cref{lemma-small-pay-add} part \ref{item-complicated-add}, given the valuation profile $(v_1^{both},v_2^{a,large})$, bidder $1$ cannot win item $a$ so we consider the following two cases: the case where he wins item $b$ and the case where he wins neither of these items.

    Assume that bidder $1$ wins item $b$ given the valuation profile $(v_1^{both},v_2^{a,large})$. Note that in this case, \cref{lemma-small-pay-add} part \ref{item-complicated-add} also implies that:
    \begin{equation}\label{eq-good-leaf6-add}
 v_1^{b,mid}(f(v_1^{both},v_2^{a,large}))-P_1(v_1^{both},v_2^{a,large})\ge k^2 -2k-1   
\end{equation}
Whereas by part \ref{item-4.5-add} of Lemma \ref{lemma-small-pay-add}: 
% player $1$ either does not win item $a$ or pays at least $k^2$, so:
 \begin{equation}\label{eq-bad-leaf6-add}
 v_1^{b,mid}(f(v_1^{b,mid},v_2^{b,large}))-P_1(v_1^{b,mid},v_2^{b,large})= 0   
\end{equation}
Combining the inequalities (\ref{eq-good-leaf6-add}) and (\ref{eq-bad-leaf6-add}) gives:
\begin{equation*}
 v_1^{b,mid}(f(v_1^{b,mid},v_2^{b,large}))-P_1(v_1^{b,mid},v_2^{b,large})< 
 v_1^{b,mid}(f(v_1^{both},v_2^{a,large}))-P_1(v_1^{both},v_2^{a,large})
\end{equation*}
Note that vertex $u$ belongs in $Path(\mathcal S_1(v_1^{both}),\mathcal S_2(v_2^{a,large}))$ and also in
$Path(\mathcal{S}_1(v_1^{a,mid}),\allowbreak\mathcal{S}_2(v_2^{a,large}))$. Therefore, Lemma \ref{lemma-bad-leaf-good-leaf} gives that the strategy $\mathcal S_1$ dictates the same message for  $v_1^{both}$ and $v_1^{b,mid}$ at vertex $u$, which completing this case.

For the latter case, where bidder $1$ gets a bundle that contains neither item $a$ nor item $b$ given the valuation profile $(v_1^{both},v_2^{a,large})$, note that for \cref{lemma-small-pay-add} part \ref{item-complicated-add} implies that:
\begin{equation}\label{eq-bad-leaf7-add}
 v_1^{both}(f(v_1^{both},v_2^{a,large}))-P_1(v_1^{both},v_2^{a,large})=0   
\end{equation}
Whereas by part \ref{item-4-add} of Lemma \ref{lemma-small-pay-add}: 
% player $1$ either does not win item $a$ or pays at least $k^2$, so:
 \begin{equation}\label{eq-good-leaf7-add}
 v_1^{both}(f(v_1^{b,mid},v_2^{b,one}))-P_1(v_1^{b,mid},v_2^{b,one}) \ge 2k+1-k=k+1 
\end{equation}
Combining the inequalities (\ref{eq-bad-leaf7-add}) and (\ref{eq-good-leaf7-add}) gives:
\begin{equation*}
 v_1^{both}(f(v_1^{both},v_2^{a,large}))-P_1(v_1^{both},v_2^{a,large})< 
 v_1^{both}(f(v_1^{b,mid},v_2^{b,one}))-P_1(v_1^{b,mid},v_2^{b,one})
\end{equation*}
Note that vertex $u$ belongs in $Path(\mathcal S_1(v_1^{both}),\mathcal S_2(v_2^{a,large}))$ and also in
$Path(\mathcal{S}_1(v_1^{b,mid}),\allowbreak\mathcal{S}_2(v_2^{b,one}))$. Therefore, Lemma \ref{lemma-bad-leaf-good-leaf} gives that the strategy $\mathcal S_1$ dictates the same message for  $v_1^{both}$ and $v_1^{b,mid}$ at vertex $u$, which solves the second case and completes the proof.
\end{proof}

\subsubsection{Proof of Lemma \ref{lemma-small-pay-add}: Observations About The Mechanism} \label{subsubsec-prop-alloc-add}
The proof of \cref{lemma-small-pay-ud} is a direct consequence of the approximation guarantee of the mechanism and the fact that it is 
% the mechanisms in all of them are 
obviously strategy-proof (and thus also dominant-strategy incentive compatible) and satisfies individual rationality and no negative transfers. 
% Most of it identical to the proof of \cref{lemma-small-pay-ud} and we write the remaining the parts below. 
% Throughout the proof, we say for abbreviation that certain inequalities hold  because of individual rationality instead of saying that they hold because the allocation rule together with the payment scheme are realized by a mechanism and strategies that satisfy individual rationality. We do the same for the no negative transfers property. Also, we say that an item $x$ is \emph{valuable} for a valuation $v$ if $v(\{x\})>0$.

\begin{proof}[Proof of \cref{lemma-small-pay-add}]
The proofs of parts \ref{item-1-add}, \ref{item-2-add}, \ref{item-4-add} and \ref{item-4.5-add} are identical to the respective parts in the proof of \cref{lemma-small-pay-ud}. 


For part \ref{item-6-add}, we first show that bidder $1$ wins a bundle that contains item $a$ given $(v_1^{both},v_2^{b,one})$. Note that since the allocation rule $f$ and the payment scheme $P_1$ are realized by a dominant-strategy mechanism, we have that:
\begin{equation}\label{eq-part6-add}
\begin{aligned}
    v_1^{both}(f(v_1^{both},v_2^{b,one}))-P_1(v_1^{both},v_2^{b,one})&\ge v_1^{both}(f(v_1^{a,one},v_2^{b,one}))-P_1(v_1^{a,one},v_2^{b,one}) \\
    &\ge 2k+2 &\text{(by part \ref{item-1-add})}
\end{aligned}    
\end{equation}
Combining (\ref{eq-part6-add}) with the property of no negative transfers implies that $v_1^{both}(f(v_1^{both},v_2^{b,one}))\ge 2k+2$. Thus, given $(v_1^{both},v_2^{b,one})$, bidder $1$ necessarily gets a bundle that contains item $a$. 
For the upper bound on the payment, note that $v_1^{both}(f(v_1^{both},v_2^{b,one}))\le 4k+4$, so by individual rationality  $P_1(v_1^{a,both},v_2^{b,one})\le 4k+4$. By that, we  complete the proof of part \ref{item-6-add}. 

For part \ref{item-complicated-add},
we begin by showing that given $(v_1^{both},v_2^{a,large})$, bidder $1$ does not win item $a$. This is due to weak monotonicity. Formally, we remind that by part \ref{item-2-add}, given $(v_1^{a,mid},v_2^{a,large})$, bidder $1$ wins a bundle $S^{mid}$
that does not contain $a$. Thus, if $f(v_1^{both},v_2^{a,large})$ allocates 
a bundle $S^{both}$ that contains item $a$ to bidder $1$, then we have that $v_1^{a,mid}(S^{both})-v_1^{a,mid}(S^{mid})=k^2$ whereas $v_1^{both}(S^{both})-v_1^{both}(S^{mid})\le 4k+4$, so $f$ is not weakly monotone. Since $f$ and $P_1$ realize a dominant-strategy mechanism, \cref{wmon-lemma} implies that $f$ has to be weakly-monotone, so we get a contradiction. Thus, $f(v_1^{both},v_2^{a,large})$ has to allocate bidder $1$ a bundle that does not contain item $a$. 

The bounds on the payments are a straightforward implication of individual rationality and no negative transfers. If $f(v_1^{both},v_2^{a,large})$ allocates item $b$ to bidder $1$ then the payment is at most $2k+1$, and if it allocates to bidder $1$ no valuable items, then the payment has to be at most zero. Due to no negative transfers, it is zero exactly.   
\end{proof}

\subsection{Proofs of Claims \ref{cl:subadditive} and \ref{cl:general}} \label{subsec::proofs-subadd-general}
\begin{proof}[Proof of \cref{cl:subadditive}]
First the bidders are assigned a an arbitrary order. 
% \shirinote{Arbitrary or fixed? it's a bit unclear I think} 
This is clearly OSP.  Then, after randomly partitioning the bidder into two groups $G_1$ and $G_2$ (which is done uniformly at random and independently of bidder valuations) with probability $1/2$ the auction runs a second-price auction for the grand bundle on bidders in $G_1$.  This is clearly OSP for these bidders.  

With the remaining probability bidders in $G_1$ are discarded and the auction learns the highest value among these bidders for the grand bundle (this is also clearly OSP since bidders in $G_1$ do not gain any positive utility under any value profile).  After learning this value in an OSP way from bidders in $G_1$ the auction runs the \textsc{Binary-Search-Mechanism} on the bidders in $G_2$.  Recall that this mechanism draws independently and uniformly at random a round $r_i$ for each $i \in G_2$ and a random final round $r^*$ (before any bidder in $G_2$ acts).  

Then in each round, the mechanism sets a price for each item, allows the bidders in that round (in the pre-specified order generated at the outset of the auction) to state their demand set for ``unclaimed'' items and ``conditionally claim'' them, and then if the current round is $r^*$, terminate the auction awarding bidders in $r^*$ their conditionally claimed items (otherwise, no bidders in the round are allocated any items).  Observe that each bidder in $G_2$ participates in exactly one round and is, thus, asked a single demand query.  If the round $r_i$ that bidder $i$ participates in is $r^*$ then she weakly maximizes her utility when she is called to act by truthfully reporting her demand set (since she is allocated exactly these items).  On the other hand, if $r_i \neq r^*$ then she obtains no utility under any possible valuation profile of all agents and, thus, truthfully reporting her demand set is a weakly obviously dominant strategy.  

Since for any fixed realization of the random outcomes of all coin flips the resulting mechanism is OSP, we have that the randomized mechanism is universally OSP.
\end{proof}
% \subsection{Proof of \cref{cl:general}}
\begin{proof}[Proof of \cref{cl:general}]
    For this auction, bidders are randomly partitioned into three groups \texttt{STAT}, \texttt{SECOND-PRICE}, and \texttt{FIXED}.  We argue that for any fixed partition the mechanism is OSP and, hence, the mechanism is universally OSP. 
    
    Bidders in \texttt{STAT} cannot possibly win any goods and the mechanism just requests that they report their values.  Since they win nothing regardless of report, it is an obviously dominant strategy for them to report their information truthfully. 
    Bidders in \texttt{SECOND-PRICE} participate in a second-price auction for the grand bundle.  Since this can be implemented as an ascending-price auction for the grand bundle,  \cref{lemma-partial} 
    implies that bidders in \texttt{SECOND-PRICE} have an obviously dominant truthful strategy.  
    Finally, bidders in \texttt{FIXED} are approached in an arbitrary order and they are allocated their utility maximizing bundle of the remaining items given a fixed vector of prices.  
    Thus, 
    % Since this step can be implemented by allowing each bidder to select their favorite bundle at a fixed vector of prices, there is an implementation of the mechanism which gives 
    each
     bidder in \texttt{FIXED} an obviously dominant truthful strategy.  
    
    As such, each bidder on any fixed realization of the random partition has an obviously dominant truthful strategy and the entire mechanism is then universally OSP.
\end{proof}




\end{document}
