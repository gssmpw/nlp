% Let $\items$ be the set of $m$ items and $\bidders$ the set of $n$ bidders.  Each bidder has a private valuation function $v_i:2^{M}\to \mathbb R^{+}$ that specifies her value for every subset of items that belongs to a domain of valuation function $V_i$ that is known to the auctioneer.
Let $\items$ be the set of $m$ items and $\bidders$ the set of $n$ bidders. Each bidder $i$ has a private valuation function $v_i: 2^{\items} \to \mathbb{R}^{+}$, which specifies her value for every subset of items. This function belongs to a domains of valuations  $V_i$ that is known to the auctioneer.  
The auctioneer's goal is to maximize social welfare, i.e., to output a partition $(S_1, \dots, S_n)$ of the items $\items$ among the bidders $\bidders$ that maximizes $\sum_i v_i(S_i)$.


% The goal of the auctioneer is to maximize the social welfare, namely to output a partition $(S_1,\ldots,S_n)$ of the items $M$ among the bidders $\bidders$ such that $\sum_i v_i(S_i)$ is optimized. 

To achieve this, the auctioneer designs mechanisms, with this work focusing specifically on randomized ones.
%i.e., communication protocols between the auctioneer and bidders.
A randomized protocol $\mathcal M$ randomly chooses, in advance, one of several deterministic protocols to follow. We denote the deterministic mechanisms in the support of the randomized mechanism with $\mathcal A$.
Throughout the paper, we use the terms protocol and mechanism interchangeably.
We defer all proofs to \cref{app:missing-proofs-prems}. 

\paragraph{Components of Deterministic Protocols} Following the random selection, all bidders face a deterministic protocol.
We represent protocols as trees where each internal node corresponds to a bidder called to "speak" or communicate a message.
% To describe deterministic protocols, we represent them as trees in which each internal node corresponds to a bidder who is called to ``speak,'' or communicate a message. 
Each such node has a set of possible messages, and the next node in the tree is determined by the message sent by the bidder. 
% We note that each node has a unique bidder speaking (i.e., the game is sequential) and the message sent at each node is observed by all bidders (i.e., the game has perfect information)\footnote{Assuming sequentiality and perfect information is without loss, see, e.g., \cite{BG17} and \cite{Ron24}.}. \shirinote{Some word juggling done here}
A leaf in the protocol specifies an allocation of items to bidders and a payment for each bidder.
%Also, each leaf in the protocol is labeled with an allocation of the items to the bidders and a payment for every bidder.  

Fixing a randomly chosen deterministic protocol $A\in \mathcal A$, let $\mathcal{N}_i$ denote the set of all nodes in which a particular bidder $i$ is called to speak.  Then, the behavior $B_i$ of player $i$ assigns a message to each node in $\mathcal{N}_i$. We denote with $\mathcal{B}_i$ be the set of all possible behaviors.  Observe that a behavior profile $B = (B_1,\dots, B_n) \in \mathcal B_1\times\cdots\times \mathcal B_n$ thus defines a root to leaf path in $A$.  We let $Path(B)$ denote all the nodes along the path defined by $B$ and $Leaf(B)$ denote the leaf
that $Path(B)$ ends with.  For every behavior profile $B$ and for every player $i$, we 
denote with $f_i(B)$ and with $p_i(B)$ respectively the allocation and the payment of player $i$ that are specified in $Leaf(B)$. For two behavior profiles $B$ and $B'$, we denote all of the nodes appearing in both $Path(B)$ and $Path(B')$ as $Path(B)\cap Path(B')$. Two behavior profiles $(B_1,\ldots,B_n)$ and $(B_1',\ldots,B_n')$ \emph{diverge at vertex $u$} if $u$ is the last shared vertex in $Path(B_1,\ldots,B_n)$ and $Path(B_1',\ldots,B_n')$.


Finally, the \emph{strategy} $\mathcal{S}_i$ of player $i$ is a function specifying a behavior of player $i$ for each possible valuation in $V_i$ and every possible deterministic protocol $A$ in the support $\mathcal A$. Formally, $\mathcal S_i:\mathcal A \times V_i\to\mathcal B_i$. We often abuse notation by referring to the strategy of player $i$ in the deterministic protocol $A$, that is, the partial function $\mathcal{S}_i(A, \cdot)$ simply as a strategy and denoting it by $\mathcal{S}_i$. 




A deterministic mechanism $A$ together with strategies $(\mathcal S_1(A,\cdot ),\ldots,\mathcal S_n(A,\cdot))$ realize allocation rule $f:V_1\times \cdots \times V_n \to \allocs$ and payment schemes $P_1,\ldots,P_n:V_1\times \cdots \times V_n \to \mathbb R^n$ if for every $(v_1,\ldots,v_n)\in V_1\times\cdots\times V_n$, it holds that $Leaf(\mathcal S_1(A,v_1),\ldots,\mathcal S_n(A,v_n))$ is labeled with the allocation $f(v_1,\ldots,v_n)$ and with the payment $P_i(v_1,\ldots,v_n)$ for every player $i$. 
% where $\forall i \in \bidders$ the behavior $B_i=\mathcal S_i(A,v_i)$. %for every player $i$. 


\paragraph{Properties of Randomized Mechanisms}
To analyze the performance of our  mechanisms, we compare against the optimal social welfare.  Let $\mathbf{T} = (T_1, \dots ,T_n)$ denote a feasible allocation of the items (i.e., each item is allocated to at most one bidder) and $\allocs$ denote the set of all feasible allocations. 
 Then, we let $\opt(I) = \max_{\mathbf{T} \in \allocs}\sum_{i \in \bidders}{v_i(T_i)}$ denote the optimal social welfare achievable on a given instance $I=(v_1,\ldots,v_n)$ and $\mathbb{E}[W(\mechanism(I))]$ denote the \emph{expected} social welfare achieved by mechanism $\mechanism$ on instance $I$ (where the expectation is taken over the random choice of which deterministic protocol is to be run by the mechanism).  We then say that mechanism $\mechanism$ obtains an $\alpha$-approximation to the optimal social welfare on a class of instances $\mathcal{I}$ if \[\sup_{I \in \mathcal{I}} \frac{\opt(I)}{\mathbb{E}[W(\mechanism(I))]} \leq \alpha.\]

In addition to the objective of welfare maximization, our goal is to design randomized mechanisms  satisfying three key desiderata: (i) \emph{ex-post no negative transfers}; (ii)  \emph{ex-post individual rationality}; and (iii) \emph{universal obvious strategy-proofness}, i.e., ex-post obvious strategy-proofness.\footnote{
A randomized mechanism satisfies a given property \emph{ex-post} if that property holds for every deterministic mechanism that has a non-zero probability of being selected.} For simplicity, in the rest of the paper, we omit the prefix ``ex-post'' when referring to these properties. 
% and similar ones, as all properties we consider hold in the ex-post sense.

%by the randomization process.}  
A mechanism $\mechanism$ with support $\mathcal A$ satisfies \emph{no negative transfers} if for every leaf in every deterministic protocol $A\in \mathcal A$, the payment of every player $i$ is at least zero.
A mechanism $\mathcal M$ with strategy profile $(\mathcal S_1,\ldots, \mathcal S_n)$ and support $\mathcal A$ satisfies 
 \emph{individual rationality} if,
for every deterministic protocol $A\in \mathcal A$,
the allocation rule $f:V_1\times \cdots \times V_n\to \allocs$ and payment schemes $P_1,\ldots,P_n$  that it realizes satisfy that 
for every  $(v_1,\ldots,v_n)\in V_1\times \cdots \times V_n$ and every player $i$:
$
v_i(f(v_1,\ldots,v_n))-P_i(v_1,\ldots,v_n)\ge 0
$.
Namely, a mechanism is ex-post individually rational if each player obtains non-negative utility for participating in the mechanism (hence, there is no incentive to avoid participation).  

We will now define obvious strategy-proofness in the universal sense. A mechanism $\mathcal M$ with support $\mathcal A$ and the strategies $(\mathcal S_1,\ldots,\mathcal S_n)$ is universally obviously strategy-proof if for every deterministic protocol $A\in \mathcal A$,
the strategies $(\mathcal S_1(A,\cdot),\ldots,\mathcal S_n(A,\cdot))$ are obviously dominant. 
It remains to define what it means for a strategy to be obviously dominant. Loosely speaking, a strategy 
$\mathcal S_i(A,\cdot)$
of bidder $i$ in a deterministic mechanism $A$ is \emph{obviously dominant} if,
each time player $i$ is called to speak, the worst-case outcome from sending the message defined by 
$\mathcal S_i(A,\cdot)$ 
is weakly better than the best-case outcome from any other strategy.
Thus, it is conceptually ``easy'' for player $i$ to find $\mathcal{S}_i$, understand its dominance and follow it. Despite their intuitive appeal, the definition of obviously
dominant strategies is quite subtle. Thus, 
below, we state only the basic properties of obviously strategy-proof mechanisms that we use to prove our results, and 
we defer the  definition to 
\cref{app-formalities}.






%%%%%%% OLD VERSION%%%%%
%Finally, a mechanism $\mechanism$ satisfies universal obvious strategy-proofness if each player $i$ has an obviously dominant strategy corresponding to her true value $v_i$ for each possible random choice of protocol. \pinksout{i.e., each deterministic protocol in the support of $\mechanism$ equips each agent with an obviously dominant truthful strategy.}  Loosely, we say that a strategy \pinksout{$\mathcal{S}_i$}
%\sre{$\mathcal S_i(A,\cdot)$}
%of bidder $i$ \sre{in a deterministic mechanism $A$} is \emph{obviously dominant} if,
%each time \sre{player} $i$ is called to speak, the worst-case outcome from sending the message defined by \pinksout{$\mathcal{S}_i$}
%\sre{$\mathcal S_i(A,\cdot)$} 
%\pinksout{over all possible strategies of the other bidders} is weakly better than the best-case outcome from \sre{any other strategy.}
%\pinksout{sending a message defined by strategy $\mathcal{S'}_i$ for all $\mathcal{S'}_i$.} 
%\sre{Thus, it is conceptually ``easy'' for player $i$ to find $\mathcal{S}_i$, understand its dominance and follow it.}
%\pinksout{As such, it is conceptually ``easy'' for the $i$ to find and follow $\mathcal{S}_i$.}  Despite their intuitive appeal, the definition of obviously strategy-proof mechanism\sre{e} is quite subtle. Thus, in the subsection below, we state only the basic properties of obviously strategy-proof mechanisms that we use to prove our results, and defer the precise definition to \cref{app-formalities}.







\paragraph{Generalized Ascending Auctions}
To prove some of our positive results, we  employ a specific form of auction, which we name generalized ascending auctions. In particular, some of our randomized mechanisms will be a randomization of such auctions. 

A \emph{generalized ascending auction} defines two possible allocations for each bidder $i$: the \emph{base} bundle $X_i^{B}$ and the \emph{potential} bundle $X_i^{P}$, where $X_i^{B} \subseteq X_i^{P}$. Each bidder $i$ initially ``holds'' the base bundle $X_i^{B}$ and is placed in the ``active'' set $\bidders^{A}$.  
Each $i \in \bidders^{A}$ faces a monotonically increasing price trajectory for receiving $X_i^{P}$ instead of $X_i^{B}$. Bidders drop out when the price for $X_i^{P}$ becomes too high, at which point they are awarded $X_i^{B}$ at a price of $0$ and removed from $\bidders^{A}$ (i.e., they become inactive). The auction terminates when it is feasible to allocate $X_i^{P}$ to all remaining active bidders $i \in \bidders^{A}$ and $X_i^{B}$ to all inactive bidders $i \notin \bidders^{A}$.  
For illustration, an auction where bidder $1$ always receives a fixed item $a$, while the remaining items $M\setminus \{a\}$ are allocated via an ascending auction among the other bidders, is a generalized ascending auction.


% A \emph{generalized ascending auction} defines two possible allocations for each bidder $i$, the \emph{base} bundle $X_i^{B}$ and the \emph{potential} bundle $X_i^{P}$ where $X_i^{B} \subseteq X_i^{P}$.  Each bidder $i$ initially ``holds'' bundle $X_i^{B}$ and is placed in the ``active'' set $\bidders^{A}$.  Each $i \in \bidders^{A}$ faces a monotonically increasing price trajectory to instead receive $X_i^{P}$ (in place of $X_i^{B}$).  Bidders then drop out when the price for receiving $X_i^{P}$ is too high at which point they are awarded $X_i^{B}$ at a price of $0$ and removed from $\bidders^{A}$ (i.e., become inactive). Finally, the auction terminates when allocating all $i \in \bidders^{A}$ their potential bundle $X_i^{P}$ and all $i \notin \bidders^{A}$ their base bundle $X_i^{B}$ is feasible.  
% % runs an ascending auction on the remaining items between the remaining bidders. 
% For illustration, an auction where bidder $1$ always gets some item $a$ and there is an ascending auction on the remaining items $M\setminus \{a\}$ is a generalized ascending auction.  
\begin{lemma}\label{lemma-partial}
    Every generalized ascending auction is obviously strategy-proof. 
\end{lemma}


\subsection{Tools For Establishing Lower Bounds}
To prove impossibility results for randomized mechanisms, we employ an adaptation of Yao's lemma \cite{Yao83}, which is formalized in \cref{lem:yaos}. This approach allows us to focus on the performance of deterministic mechanisms with obviously dominant strategies when evaluated over a distribution of valuation profiles. We then restate a property of deterministic mechanisms from \cite{Ron24}, which we will use extensively in our impossibility proofs (\cref{lemma-bad-leaf-good-leaf}).

% provide a tool to prove impossibilities 

\paragraph{From Randomized Mechanisms to Deterministic Mechanisms} 


% To prove impossibilities for randomized mechanisms, we analyze the performance
% of deterministic mechanisms over a distribution of valuations, applying an adaptation of Yao's lemma \cite{Yao83}.
% \sre{Observe that \cref{lemma-bad-leaf-good-leaf} is for deterministic mechanisms, while we show impossibilities for randomized ones. In fact, we prove impossibilities for randomized mechanisms by analyzing the performance for deterministic mechanisms over a distribution of valuations, using a straightforward adaptation of Yao's lemma \cite{Yao83}.} 
%In order to utilize \cref{lemma-bad-leaf-good-leaf}, which deals with deterministic OSP mechanisms, to show impossibility results for \emph{randomized} OSP mechanisms we apply a variant of Yao's principle, which we describe below.
%Below we describe a variant of Yao's principle, which we will use in our lower bounds. 

To state and prove \cref{lem:yaos}, we need the following notations. 
Given a deterministic obviously strategy-proof mechanism $A$ that has obviously dominant strategies $(\mathcal S_1,\allowbreak \ldots, \mathcal S_n)$ and a valuation profile $(v_1,\ldots,v_n)$, we denote with $A(v_1,\ldots,v_n)$ the welfare of the allocation that $A$ outputs given $(\mathcal S_1(v_1),\allowbreak\ldots,\mathcal S_n(v_n))$.
% \footnote{If there is more than one profile of obviously dominant strategies, we break ties by taking the one with the highest welfare.}
Given a randomized mechanism  $\mathcal M$ 
which is a distribution over such deterministic mechanisms, let $\mathcal M(v_1,\ldots,v_n)$ be the expected welfare of the mechanism given the valuation profile $(v_1,\ldots,v_n)$. 
We denote with $OPT(v_1,\ldots,v_n)$ the optimal welfare.  
\begin{lemma}\label{lem:yaos} \cite{Yao83}
Fix a set of $n$ bidders with domains of valuations $V=V_1\times \cdots \times V_n$.
    Let $\mathcal D$ be a distribution over a set of valuation profiles taken from $V$ and fix an accuracy parameter $\alpha$.
    If for every deterministic mechanism $A$ that is obviously strategy-proof and satisfies no negative transfers and individual rationality, it holds that:
    $$
    \E_{(v_1,\ldots,v_n)\sim \mathcal D}\Big[\dfrac{A(v_1,\ldots,v_n)}{OPT(v_1,\ldots,v_n)}\Big] \le  \frac{1}{\alpha}
    $$
Then, every randomized mechanism $\mathcal M$
that is obviously strategy-proof and satisfies individual rationality and no negative transfers
satisfies that its approximation ratio in the worst case does not exceed $\alpha$. 
\end{lemma}
Note the following subtlety in the statement of the lemma: we consider 
deterministic mechanisms that are obviously strategy-proof, individually rational and 
satisfy no negative transfers with respect to all the valuations in $V$, not only the valuations in the support of $\mathcal D$. Accordingly, our proof implies an impossibility for randomized mechanisms which are a probability distribution over deterministic mechanisms that satisfy all the above properties with respect to all the valuations in $V$. 
%The proof is straightforward, and follows closely the proof given in \cite{Rough16}. 
%\shirinote{I suspect that this is too much credit for Tim for citing a classic. Let me know what you think.} 
A familiar reader may anticipate fully the proof of \cref{lem:yaos}, which we write for completeness in \cref{app:missing-proofs-prems}. 
% in \cref{subsec-yaos-lem}.



\paragraph{Proving Lower Bounds For Deterministic Mechanisms} 
Having utilized \cref{lem:yaos} to transition from analyzing randomized mechanisms to analyzing deterministic ones, our next step is to establish lower bounds for the latter. To that end, we invoke \cref{lemma-bad-leaf-good-leaf}, a structural property of deterministic obviously strategy-proof mechanisms originally presented in \cite{Ron24}. While the lemma’s statement may appear technical, it is a natural and intuitive property of obviously strategy-proof mechanisms that stems directly from their definition (see Figure~2 in \cite{Ron24} for an explanatory illustration).  
% to some essence, it admits a direct and intuitive interpretation that follows naturally from the definition of obvious strategy-proofness 


% Once we have used \cref{lem:yaos} to analyze the performance of deterministic mechanisms over a distribution of valuation profiles rather than the performance of randomized mechanisms, our next step is to prove lower bounds on deterministic mechanisms. 

% For that, we use  \cref{lemma-bad-leaf-good-leaf} below: 
%  \cref{lemma-bad-leaf-good-leaf}  is a property of deterministic obviously strategy-proof mechanisms  stated in \cite{Ron24}.  While the lemma statement is somewhat technical, it has an intuitive interpretation and straightforward proof that follows directly from the definition of obvious strategy proofness (see Figure 2 in \cite{Ron24} for details).
\begin{lemma}
%[Lemma 2.4 of \cite{Ron24}]
\label{lemma-bad-leaf-good-leaf}
Fix a deterministic obviously strategy-proof mechanism $A$ with strategies $(\mathcal S_1,\ldots, \mathcal S_n)$
that realize an allocation rule and payment schemes $(f,P_1,\ldots,P_n):V_1\times\cdots \times V_n\to \allocs \times \mathbb R^n$.
Fix a player $i$, a vertex $u\in \mathcal N_i$
and
two valuation profiles $(v_i,v_{-i}),(v_i',v_{-i}')$ such that the following conditions hold simultaneously:
\begin{enumerate}
    \item $u\in Path(\mathcal S_i(v_i),\mathcal S_{-i}(v_{-i}))\cap Path(\mathcal S_i(v_i'), \mathcal S_{-i}(v_{-i}'))$. 
    \item $v_i(f(v_i,v_{-i}))-P_i(v_i,v_{-i})< 
v_i(f(v_i',v_{-i}'))-P_i(v_i',v_{-i}')$.
\end{enumerate}
Then,  the strategy  $\mathcal S_i$ dictates the same message for the valuations  $v_i$ and $v_i'$ at vertex $u$.   
\end{lemma}




