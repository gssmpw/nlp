% \subsection{Proof of \cref{lem:yaos}} \label{subsec-yaos-lem}

\begin{proof}[Proof of Lemma \ref{lemma-partial}]
    We establish the proof by describing an obviously dominant strategy for every bidder.
We argue that for each bidder $i$ the strategy which has the bidder remain in the auction if and only if her current clock price $p_i$ is less than or equal to $v_i(X_i^{P}) - v_i(X_i^{B})$, i.e., the increase in value she has for receiving $X_i^{P}$ compared to $X_i^{B}$, is obviously dominant. 
We prove it by a straightforward case analysis. In particular, we demonstrate that the worst case given the truthful strategy is always more profitable than the best case given any other strategy. 
% The proof of this fact is rather straightforward case analysis. 

Consider first the case where $p_i \leq v_i(X_i^{P}) - v_i(X_i^{B})$.  The best-case utility $i$ can receive by deviating from the strategy and exiting the auction early is $v_i(X_i^{B})$ (by receiving $X_i^{B}$ for a price of $0$), but the worst-case utility she receives by following the strategy and staying in the auction until the next price increment is also $v_i(X_i^{B})$ (since she can always exit the auction if the price then becomes too high).  

Now consider the case where $p_i > v_i(X_i^{P}) - v_i(X_i^{B})$.  The worst-case utility bidder $i$ can receive by following the strategy and exiting the auction is $v_i(X_i^{B})$. 
On the other hand, if she deviates from it by remaining an active player and the auction were to terminate she would obtain utility 
less than  $v_i(X_i^{P}) - (v_i(X_i^{P}) - v_i(X_i^B)) \leq v_i(X_i^B)$.  Thus, in this case as well, her worst case utility from following the strategy is better than the best case when deviating, which completes the proof.
\end{proof}


\begin{proof}[Proof of \cref{lem:yaos}] 
Fix a randomized mechanism $\mathcal M$ and 
denote with $\mathcal{D}_{\mathcal{M}}$ its distribution over the deterministic mechanisms in its support.
Note that by assumption every deterministic mechanism $A$ in the support of $\mathcal M$ satisfies that:
$$
    \E_{(v_1,\ldots,v_n)\sim {\mathcal{D}}}\Big[\dfrac{A(v_1,\ldots,v_n)}{OPT(v_1,\ldots,v_n)}\Big]\le 
    \frac{1}{\alpha}
$$
Since this is true for every deterministic mechanism in the support, averaging over the randomness of the mechanism $\mathcal M$:
$$
   \E_{(v_1,\ldots,v_n)\sim \mathcal{D},A\sim \mathcal {D}_M}\Big[\dfrac{A(v_1,\ldots,v_n)}{OPT(v_1,\ldots,v_n)}\Big]\le \frac{1}{\alpha} 
$$
Denote with $(v_1^\ast,\ldots,v_n^\ast)$ the valuation profile in the support of $\mathcal D$ that minimizes $\E_{A\sim \mathcal \mathcal D_M}\Big[\frac{A(v_1^\ast,\ldots,v_n^\ast)}{OPT(v_1^\ast,\ldots,v_n^\ast)}\Big]$. Since the minimum is  smaller or equal than the average:
$$
   \E_{A\sim \mathcal \mathcal D_M}\Big[\dfrac{A(v_1^\ast,\ldots,v_n^\ast)}{OPT(v_1^\ast,\ldots,v_n^\ast)}\Big]\le  \frac{1}{\alpha}
$$
Thus, the randomized mechanism $\mathcal M$ does not give an  approximation better than $\alpha$ to the optimal welfare given the valuation profile $(v_1^\ast,\ldots,v_n^\ast)$, which completes the proof.
\end{proof}




