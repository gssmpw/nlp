We now turn to settings with heterogeneous items.
We explore settings involving additive and unit-demand bidders and conclude by considering mechanisms for subadditive and general valuations.
The proofs of all theorems can be found in 
%the full version.
\cref{app-missing-combi}. 
%full-version-change-tag%
\subsection{Additive Valuations}
A valuation $v_i$ is \emph{additive} if bidder $i$ has a value $v_{ij} \geq 0$ for item $j$ and the value bidder $i$ has for receiving a set of items $A_i$ is equal to $\sum_{j \in A_i}{v_{ij}}$. 
\subsubsection{Upper Bound}
% \sre{Notes to ourselves:
% \begin{enumerate}
%     \item Shahar mentioned in a meeting that we can probably improve the constant by sampling $\frac{1}{e}$ of the bidders. This is a reminder for us to check this out at some point.
%     \item A way to solve the tie breaking without loss in the approximation factor: Do the \textquote{"Rebecca Way"}: set an arbitrary ranking on the bidders, and allocate an item to a bidder only if either her value is strictly higher than the price or her value is equal to the price and this bidder is ranked higher than the bidder who set the price. 
%     \end{enumerate}
%     }
We show that the sampling approach yields a $4$-approximation for this setting:
\begin{theorem}\label{thm-ub-add}
    There is a universally OSP mechanism for bidders with additive valuations that gives a $4$-approximation to the optimal welfare.
\end{theorem}
We now describe Mechanism \ref{alg:additive}. Simply put, Mechanism \ref{alg:additive} samples a threshold price for each item and then uses these threshold prices as a posted-price mechanism for the unsampled bidders. To handle tie-breaking, priority is given to bidders with higher indices. 
% \cref{app-missing-combi}.
% that it is universally OSP and gives a $4$-approximation to the optimal welfare. 
\begin{algorithm2e}
%\setstretch{1.1}
\SetKwInOut{Input}{Input}
\Input{A set of bidders $\bidders$ and a set of $M$ items}

 Index the bidders in some arbitrary fixed order


 $S \leftarrow \emptyset$, $U \leftarrow \emptyset$
 
Independently assign each bidder to set $S$ with probability $\nicefrac{1}{2}$ and to set $U$ with probability $\nicefrac{1}{2}$.

 ``Discard'' each bidder in $S$ and learn their value for each item

 For each $j \in M$: set a price $p_j$ on item $j$ equal to $\max_{i \in S}{v_{ij}}$ and let $n(j)$ denote the smallest index among bidders in $\arg\max_{i \in S}{v_{ij}}$
 
 For each $i \in U$ in an arbitrary order:  Let $i$ purchase all previously unsold items $j \in M$ for which either: (i) $v_{ij} > p_j$; or (ii) $v_{ij} = p_j$ and $i$ has a lower index than $n(j)$
 
 \caption{``\textsc{Additive}''}
 \label{alg:additive}
\end{algorithm2e}
The following two lemmata jointly provide the proof of \cref{thm-ub-add}.
% We show the following two lemmata below, which jointly provide the proof of \cref{thm-ub-add}. 
\begin{lemma}\label{lemma-add-osp}
Mechanism \ref{alg:additive} is universally OSP.
\end{lemma}
The proof of \cref{lemma-add-osp} is straightforward, and we write it for completeness in \cref{subsec::proof-add-osp}. 


\begin{lemma}\label{lem-add-approx}
Mechanism \ref{alg:additive} obtains a $4$-approximation to the optimal social welfare in the presence of additive bidders.
\end{lemma}
\begin{proof}
Observe that since the valuation functions are additive,  optimal solutions must allocate each item $j$ to some bidder $i \in \text{argmax}_{i \in \bidders}{v_{ij}}$.  In particular, one optimal solution allocates each item $j$ to the bidder $i^*_j \in \text{argmax}_{i \in \bidders}{v_{ij}}$ with the smallest index according to the order over the bidders that the mechanism specifies.  We argue that each item $j$ is allocated to its corresponding bidder $i^*_j$ by Mechanism \ref{alg:additive} with probability at least $\frac 1 4$. Linearity of expectation directly implies a $4$-approximation to the optimal welfare.

To see that we allocate each item $j$ to
$i^*_j$
% an optimal bidder 
with probability $\frac 1 4$, first observe that $i^*_j$ is placed in $U$ with probability $\frac 1 2$.  Moreover, let $\tilde{i}_j$ denote the \textquote{second-highest bidder}, which we define as
the bidder in $\text{argmax}_{i \in \bidders \setminus \{i^*_j\}}{v_{ij}}$
that has the smallest index.
 This bidder is placed in $S$ (independently of the placement of $i^*_j$) with probability $1/2$.  
Observe that when bidder $\tilde{i}_j$ is in $S$ and $i^*_j$ is in $U$,
then item $j$ is necessarily available for bidder $i^*_j$ who gets it indeed. 
This occurs with probability at least $\frac{1}{4}$, as desired.\qedhere
\end{proof}

% \sre{Simply put, Mechanism \ref{alg:additive} generates a threshold price for each item and then applies these thresholds as posted prices for the bidders who were not sampled. We prioritize higher-indexed bidders to tie breaking issues.}

%The correctness is based on the fact that for additive valuations, allocating each item to the highest bidder is optimal. 
%The added layer of allowing bidders to take items only if their threshold exceeds the threshold of the bidders 


% Let $m$ denote the number of items and $n$ denote the number of agents. 
%Consider the mechanism $\mathcal{M}_4$ which independently for every bidder $i$ with probability $1/2$ marks $i$ as a ``sampled'' bidder (she is marked ``unsampled'' otherwise).  The sampled bidders are then eliminated from winning the items and the mechanism learns their valuation functions (using polynomially many value queries).  $\mathcal{M}_4$ then assigns a price to each item equal to the highest value for that item among the sampled bidders.  Finally, with probability $1/3$ we allow each unsampled bidder to claim her maximum sized demand set at the prices set by the sampling phase, and with the remaining $2/3$ probability we allow each bidder, in an arbitrary order to claim her minimum sized demand set.


\subsubsection{Lower Bound}
\begin{theorem}\label{thm-lb-add}
For a combinatorial auction with $m\ge 2$ items and $n\ge 2$ additive bidders, there 
is no randomized obviously strategy-proof mechanism that satisfies individual rationality and no negative transfers and gives approximation better than $\frac{8}{7}$ to the optimal social welfare.
\end{theorem}
The proof follows the same structure as the proof of \cref{thm-mua-sm-lb} of describing 
a distribution $\mathcal D$ and showing that it is hard for every deterministic mechanism. By applying Yao's Lemma (\cref{lem:yaos}), we get hardness for randomized obviously strategy-proof mechanisms. 
The proof can be found in \cref{lb-add-proof-place}. 
We note that the construction of \cref{thm-lb-add} is similar to a construction in \cite{Ron24}.  However, the case analysis we employ is more involved as it includes additional valuation profiles. 


\subsection{Unit-Demand Valuations}
We now address bidders with \emph{unit-demand} valuations, where there is a value $v_{ij} \geq 0$ for each $i \in \bidders$ and  $j \in \items$ and the value bidder $i$ has for a set $A_i$ is equal to $\max_{j \in A_i}{v_{ij}}$. 



% Consider the following example that demonstrates it. If the number of items $m$ is equal to the number of bidders $n$, and there 
% Assume that there are two bidders that have value $1$ for all the items, and the rest of the bidders have values $1-\epsilon$ for all the items. Then, the price 
% \begin{example}
%     The following example demonstrates why  unit demand bidders are harder than additive bidders. For additive bidders, if we would have had an oracle that says to us for every item $j$, $\max_i{v_i(j)}$, i.e. the highest value for this item, then by setting the price to be $p_j=\max_i{v_i(j)}$ for every item $j$ would have give us the optimal allocation (up to tie breaking). 

%     However, this approach gives a bad approximation for unit-demand bidders. Assume that bidder $1$ has value $1$ for all the items, and the rest of the bidders have values $1-\epsilon$ for all the items. Then, setting those prices gives only an $n$-approximation.   
% \end{example}
% \shirinote{For some reason after writing it down this example felt less compelling to me, maybe I missed some details? Anyhow, hopefully it will be handy in the writeup!}

% Without loss of generality, we may assume that the optimal solution is a matching of items to bidders.  For convenience we denote the optimal matching $\mu^*$ and the item matched to bidder $i$ in the optimal matching as $\mu^*(i)$ whereas we denote the bidder matched to item $j$ in the optimal matching as $\mu^{-*}(j)$ (e.g., $\mu^*$ is a function from bidders to items and $\mu^{-*}$ is its inverse).

% \shirinote{do we want to keep \cref{claim-ud-1} and \cref{claim-ud-2}?}
% \begin{claim}\label{claim-ud-1}
% Consider an instance with unit demand bidders and its optimal solution $\mu^*$.  Consider setting a price of $\frac{v_{\mu^{-*}(j)}(j)}{10}$ for each item $j$.  If bidders are queried in an arbitrary order and purchase arbitrary items in their demand sets at these prices the resulting matching yields welfare within a constant factor of the optimum.
% \end{claim} 

% \begin{proof}
%     Consider the allocation produced by approaching the bidders in an arbitrary order and allowing them to purchase arbitrary items in their demand sets.  We consider separately the items allocated in the optimal allocation which are sold by the demand query process and those which are left unsold.  Consider an arbitrary item $j$ allocated in the optimal solution which is sold in the demand query process.  Since its price is $\frac{v_{\mu^{-*}(j)}(j)}{10}$ we have that the value of the bidder who purchases it is at least $1/10$ of the value of the bidder who receives it in the optimal solution.  Now consider an item $j$ which is allocated in the optimal solution which is left unsold in the demand query process.  Since it is unsold it was necessarily available for purchase by $\mu^{-*}(j)$ but this bidder selected a different good.  The utility that $\mu^{-*}(j)$ obtains in the demand query process is then at least $\frac{9v_{\mu^{-*}(j)}(j)}{10}$.  Finally, observe that each bidder buys at most one item in the demand query process and is allocated at most one item in the optimal solution.  As such, we may assign at most one sold item and one unsold item to each bidder and we obtain 
% \end{proof}

% \begin{claim}\label{claim-ud-2}
% Lower bound on ``accurately'' priced goods.
% \end{claim}
\subsubsection{Upper Bound}
This setting appears more complicated than the setting of additive valuations, as the approach of setting the price of each item to be the price of the second highest bid fails miserably:
\begin{example}\label{ex:ud-failure}
    Consider the instance where there are $\sqrt{n}$ \textquote{high} bidders with value $2$ for all items, and the rest of the bidders are \textquote{low}, in the sense that they value all items at $1$. Assume that the number of items, $m$, is equal to the number of bidders, $n$.
    
    Note that if we sample roughly half of the bidders and use their highest values to determine the prices for the unsampled bidders, as we
    do in Mechanism \ref{alg:additive},
    %did in the mechanism $\mathcal M_3$ 
    % used for additive valuations,
    we get $\approx\frac{1}{\sqrt n}$ of the optimal welfare. This is because at least one of the \textquote{high} bidders is in the sample with probability $1-\frac{1}{2^{\sqrt n}}$. Thus in this very likely case, the price of all items is set to be $2$ and none of the \textquote{low} bidders take any item, so the welfare obtained in expectation is at most $\sqrt{n}$. However, the  optimal welfare is $n+\sqrt{n}$. 
\end{example}
% (we provide an example that demonstrates it in the full version.)
%full-version-change-tag%
% (see \cref{ex:ud-failure} in Appendix \ref{app-missing-combi}).
Thus, to obtain a constant factor approximation for this setting,  we use the beautiful algorithm of \cite{reiffenhauser2019optimal}, originally formulated for the problem of strategy-proof online matching: 
\begin{algorithm2e}
%\setstretch{1.1}
\SetKwInOut{Input}{Input}
\Input{A set of bidders $\bidders$ and a set of  items $M$}

Ensure uniqueness of all optimal solutions for any fixed subset of bidders and items by fixing a consistent tie-breaking rule between optimal allocations

 Choose uniformly at random permutation $\sigma$ over the bidders and index the bidders in this order

 $S \leftarrow \emptyset$, $M_A \leftarrow M$
  
 ``Discard'' the first $\lfloor n/e \rfloor$ bidders and learn their value for each item and add these bidders to $S$

 For each consecutive bidder $i \in \{\lfloor n/e \rfloor + 1, \dots, n\}$:

    \quad Compute a price $p_j$ for each item $j \in M_A$ equal to $OPT(S, M_A) - OPT(S, M_A \setminus \{j\})$ (i.e., the decrease in welfare if $j$ were taken away from bidders in $S$)

    \quad Let $i$ purchase her favorite item (i.e., the item for which $v_{ij} - p_j$ is maximized and greater than $0$) at the current prices and let $j^i$ denote this item (if any). Use the tie breaking rule to determine whether player $i$ can take items for which $v_{ij}-p_j=0$

    \quad $M_A \leftarrow M_A \setminus \{j^i\}$, $S \leftarrow S \cup \{i\}$

    \quad Ask $i$ for her value of $i$ for each item
 
 \caption{``\textsc{Unit-Demand}'' (adapted from Algorithm 2 of \cite{reiffenhauser2019optimal})}
 \label{alg:unit-demand}
\end{algorithm2e}

We note that Mechanism \ref{alg:unit-demand} is Algorithm 2 of \cite{reiffenhauser2019optimal}, which we slightly adapt to our offline setting and rephrase to make the fact that the mechanism is universally obviously strategy-proof more clear. 


% By adapting the mechanism of \cite{reiffenhauser2019optimal}, we get: 
\begin{theorem}\label{thm:ud-upper}
Mechanism \ref{alg:unit-demand} is universally OSP and achieves an $e$-approximation
to the optimal social welfare in the presence of unit-demand bidders.
\end{theorem}
\begin{proof}
The approximation ratio of the mechanism directly follows from Theorem 1 of \cite{reiffenhauser2019optimal}.  To see that the mechanism is obviously strategy-proof, observe that the discarded bidders obtain no utility regardless of their report (and, thus, true value reporting is weakly obviously dominant). As for the remaining bidders, they get to select their most preferred remaining item. After that, their reported information does not affect their utility. Therefore, picking their favorite remaining item and then answer the queries afterwards truthfully is an obviously dominant strategy.   
\end{proof}

% The mechanism referred to in \cref{thm:ud-upper} is Mechanism \ref{alg:unit-demand} below.   






%We defer the proof to \cref{app-claims-proofs-ud}. 


% \begin{proof}
% Follows closely from mechanism in \citet{reiffenhauser2019optimal}. \sre{TODO - further elaborate.}
% \end{proof}
\subsubsection{Lower Bound}
\begin{theorem}\label{thm-lb-ud}
For a combinatorial auction with $m\ge 2$ items and $n\ge 2$ unit-demand bidders, there 
is no randomized obviously strategy-proof mechanism that satisfies individual rationality and no negative transfers and gives approximation better than $\frac{8}{7}$ to the optimal social welfare.
\end{theorem}
Similarly to \cref{thm-mua-sm-lb} and \cref{thm-lb-add}, the proof follows the structure of describing a distribution $\mathcal{D}$, proving that it is hard for every deterministic mechanism, and applying Yao's lemma (\cref{lem:yaos}). In fact, the proof closely resembles that of \cref{thm-lb-add} due to the fact that most valuations used in both constructions are simultaneously additive and unit-demand. The full proof is provided in \cref{lb-ud-proof-place}.



%The proof of Theorem \ref{thm-lb-ud} can be found in \cref{lb-ud-proof-place}.
% is extremely similar to the proof of \cref{thm-lb-add} and can be found in  
%We write them both for completeness. Note that the similarity holds because  most of the valuations used in both proofs are both unit-demand and additive. 
% can be found in \cref{lb-ud-proof-place}. It follows the same structure as the proof of \cref{thm-mua-sm-lb} of describing 
% a distribution $\mathcal D$ and showing that it is hard for every deterministic mechanism. \toedit{Also similarly, 
% we prove for the case of two bidders and two items, but the proof extends to any number of bidders and any number of items by adding bidders with the all-zero valuation and assume that the bidders in our construction have zero values for the additional items.} However, the case analysis we employ in this proof is significantly more involved.   

\subsection{More General Valuations}
In light of our previous results, one may wonder whether there exists a \textquote{rich enough} class of valuations for which randomized OSP mechanisms are provably unable to extract more than a constant fraction of the welfare. Perhaps surprisingly, the answer is no. However, the state-of-the-art \emph{computationally efficient} randomized mechanisms for subadditive and general valuations\footnote{A function $v$ is subadditive if for every two bundles of items $A, B \subseteq M$, it holds that $v(A \cup B) \leq v(A) + v(B)$. A valuation is general monotone if for every two bundles $A\subseteq B$, it holds that $v(B)\ge v(A)$.} are, in fact, universally OSP:
%(although neither mechanism is presented as such).  
%\subsection{Subadditive to Submodular valuations}
\begin{claim}\label{cl:subadditive}
    The $O((\log\log(m))^3)$-approximate randomized mechanism of \cite{assadi2021improved} for subadditive valuations is universally OSP.
\end{claim}
\begin{claim}\label{cl:general}
The $O(\sqrt{m})$-approximate randomized mechanism of \cite{dobzinski2012truthful} for general valuations is universally OSP.
\end{claim}
The proofs are straightforward and can be found in \cref{subsec::proofs-subadd-general}. 
% \subsection{General Valuations}
% \cite{DNS12} is obviously universally OSP, so if we don't improve it by the submission, we \sre{should} state for the class of general valuations and in particular for the class of unknown-single minded bidders we can get $\mathcal O(\sqrt{m})$ approximation.   
