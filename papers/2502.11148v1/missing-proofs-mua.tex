\subsection{Proof of Theorem \ref{claim:ascending-bundles-approx}}
% \begin{proof}
Fix a valuation profile $(v_1,d_1),\ldots,(v_n,d_n)$ where $d_i$ is the desired set and $v_i$ is the value for them and
consider a social welfare optimizing allocation $\vec O=(O_1,\ldots,O_n)$, whose  welfare we denote with $\opt$. In particular, Let $\vec O$ be the allocation that allocates the minimum number of items among the optimal ones.\footnote{Note that we slightly abuse notation here, since $v_i$ also stands for a valuation function.}   

% Without loss of generality, we may assume  that bidders who receive at least one item in this allocation receive exactly the minimum number of items which yields them positive value.  
Consider partitioning the bidders who receive at least one item in this allocation into $k+1$ disjoint subsets based on the number of items allocated to them 
% such that bidders in part $i \in \{0,1, 2, \dots, k=\lceil \log{m} \rceil\}$ receive between $2^{i}$ and $2^{i+1}-1$ items. 
such that bidders in group $i\in \{0,1, 2, \dots, k=\lceil \log{m} \rceil-1\}$ receive between $2^{i}+1$ and $2^{i+1}$ items. We denote the subset of bidders in group $i$ with $\mathcal B_i$. 

Denote with $\mathcal U[0,k]$ the uniform distribution over the integers $\{0,\ldots,k\}$. Observe that the fact there exist at most $\log m$ groups implies that: 
% Observe that the expected welfare of bidders belonging to a group $I$ sampled uniformly at random  is equal to:
\begin{equation}\label{eq-opt-lb}
\mathbb{E}_{i \sim \mathcal{U}[0, k]} \big[\sum_{i \in \mathcal B_i} v_i\big] = 
\frac{1}{\log m} \cdot \sum_i v_i(O_i)= \frac{OPT}{\log m}
\end{equation}
% Observe that there necessarily exists a group $I^\ast$ of bidders such that: 
% \begin{equation}\label{eq-opt-lb}
%  \sum_{i\in I^\ast} v_i \ge \sum_i \frac{v_i(O_i)}{\log m}= \frac{OPT}{ \log m}   
% \end{equation}

For every $i\in \{0,\ldots,k\}$, denote with $\delta_i$ the total number of bidders whose demand $d_i$ is smaller than or equal to $2^{i+1}$ (including bidders who are not allocated in the optimal allocation).  Observe that since
each bidder in the group $\mathcal B_i$ receives between $2^{i}+1$ and $2^{i+1}$ items in the optimal allocation, the number of the bidders in each subset  $\mathcal B_i$ is at most 
$\min\{\Big\lfloor \frac{m}{2^{i}+1}\Big\rfloor ,\delta_i\}$.  

Now, fix an integer $i\in \{0,1,\ldots,k\}$ and
consider the allocation of \textsc{Random-Bundles} whenever it allocates bundles of size $2^{i+1}$. Denote the bidders who are satiated in this allocation with $\mathcal A_i$. We make the following two observations regarding this set of bidders.

First, note that by construction the bidders in $\mathcal A_i$ are the bidders with the highest values who are  satiated by bundles of size  $2^{i+1}$. Since the bidders in $\mathcal B_i$ are also satiated by bundles of size $2^{i+1}$, we get that every bidder who is in $\mathcal A_i$ has a higher value than any bidder who is in $\mathcal B_i\setminus \mathcal A_i$. Formally: 
% since bidders in $I^\ast \setminus A$ are satiated by bundles of size $2^{i^\ast+1}$ and the bidders in $A$ are the bidders with highest values that are satiated by this size of bundles, we get that: 
\begin{equation}\label{eq-diff}
  \max_{i\in \mathcal B_i \setminus \mathcal A_i} v_i \le \min_{i\in \mathcal A_i} v_i  
\end{equation}
Second, since \textsc{Random-Bundles} allocates bundles of size $2^{i+1}$, it implies that $|\mathcal A_i|\ge \min\{\lfloor \frac{m}{2^{i+1}}\rfloor,\delta_i\}$ and therefore: 
\begin{equation}\label{eq-diff-count}
2|\mathcal A_i|\ge \min \{\Big\lfloor \frac{m}{2^{i}}\Big\rfloor,2\cdot \delta_i\} \ge |\mathcal B_i|
\end{equation}
where we remind that the rightmost inequality holds because $|\mathcal B_i|\le \min\{\Big\lfloor \frac{m}{2^{i}+1}\Big\rfloor ,\delta_i\}$.
Combining (\ref{eq-diff}) and (\ref{eq-diff-count}) gives that for every subset of bidders $\mathcal B_i$:
\begin{equation}\label{eq-comp}
    \sum_{i\in \mathcal B_i} v_i \le    \sum_{i\in \mathcal B_i\cap \mathcal A_i} v_i  + 
        \sum_{i\in \mathcal B_i\setminus \mathcal A_i} v_i \le \sum_{i\in \mathcal A_i} v_i + |\mathcal B_i \setminus \mathcal A_i|\cdot \min_{i\in \mathcal A_i} v_i \le 3\cdot \sum_{i\in \mathcal A_i} v_i
\end{equation}
where the rightmost inequality holds since clearly $|\mathcal B_i\setminus \mathcal A_i| \le |\mathcal B_i|$. 


Now, denote with $ALG$ the expected welfare of 
\textsc{Random-Bundles}. We remind that \textsc{Random-Bundles} samples $i\in \{0,\ldots,k\}$ uniformly at random, and therefore:
% chooses a group of bidders $I$ from $\mathcal{U}[u,k]$ uniformly at random, and therefore: 
% allocates to the bidders in $A^\ast$ with probability at least $\frac{1}{\log m}$, and therefore:
% \begin{equation}\label{eq-alg-lb}
%     ALG \ge \frac{\sum_{i\in A}}{\log m}
% \end{equation}
% Combining all of the above gives that:
\begin{align*}
     ALG &=  \mathbb{E}_{i \sim \mathcal{U}[0, k]} \Big[\sum_{i \in \mathcal A_i} v_i\Big] \\
     % \frac{\sum_{i\in A}v_i}{\log m} \\ 
%&\text{\small{(\textsc{Random-Bundles} allocates items to $A$ with probability at least $\nicefrac{1}{\log m}$})} \\
& \ge \mathbb{E}_{i \sim \mathcal{U}[0, k]} \Big[\frac{\sum_{i \in \mathcal B_i} v_i}{3}\Big] &\text{(by (\ref{eq-comp}))} \\
% \frac{\sum_{i \in \mathcal B_i}  v_i\Big]}{3}  &\text{(by (\ref{eq-opt-lb}))} \\
 &= \frac{OPT}{3\log m} &\text{(by (\ref{eq-opt-lb}))} 
\end{align*}
which completes the proof. 









\begin{comment}
Consider a social welfare optimizing allocation.  Without loss of generality, we may assume that each bidder who receives at least one item in this allocation receives exactly the minimum number of items which yields them positive value.  
Consider partitioning the bidders who receive at least one item in this allocation into $k+1$ disjoint subsets based on the number of items allocated to them 
% such that bidders in part $i \in \{0,1, 2, \dots, k=\lceil \log{m} \rceil\}$ receive between $2^{i}$ and $2^{i+1}-1$ items. 
such that bidders in part $i \in \{0,1, 2, \dots, k=\lceil \log{m} \rceil-1\}$ receive between $2^{i}+1$ and $2^{i+1}$ items.
\src{did some numbering changes here}


  
We first consider the contribution of bidders to the optimal social welfare who are in the ``large'' bundle parts of our bidder partition.  Observe that there are at most $8$ bidders in total served in any optimal allocation who demand at least $\nicefrac{m}{8}$ items.  As such there are at most $8$ bidders across parts $k-3,k-2,k-1$ and $k$.  Thus, we know that welfare contributed by bidders in these parts is no more than $8$ times the welfare \textsc{Random-Bundles} obtains if it bundles all items together and auctions this grand bundle.  


Now consider an arbitrary part $i$ in our bidder partition with $i \leq k-3$.  Note that by definition $\left \lfloor \frac{m}{2^{i+1}}\right\rfloor \geq 2$.
Observe, however, that since the bidders in part $i$ have demand between $2^{i}$ and $2^{i+1} - 1$ we know that there are at most $\left \lfloor \frac{m}{2^{i}}\right\rfloor$ bidders in part $i$ and they are all satiated by receiving $2^{i+1}$ items.  Since $\left\lfloor \frac{m}{2^{i}}\right\rfloor \leq 2\cdot \left(\left\lfloor\frac{m}{2^{i+1}}\right\rfloor + 1\right)$ for all positive integers $m$ and $i$ and since $2\cdot \left(\left\lfloor\frac{m}{2^{i+1}}\right\rfloor + 1\right) \leq 3\cdot \left\lfloor\frac{m}{2^{i+1}}\right\rfloor$ whenever $\left\lfloor\frac{m}{2^{i+1}}\right\rfloor \geq 2$, we have that the optimal welfare contributed by bidders in any part $i \leq k - 3$ is no more than $3$ times the welfare \textsc{Random-Bundles} obtains when the bundle size is $\ell = 2^{i+1}$.

As such the contribution to the social welfare from bidders in each part $i \leq k-3$ is covered within a factor $3$ by a unique auction in the support of \textsc{Random-Bundles} with bundle size $\ell = 2^{i+1}$ and the contribution to the to the social welfare from bidders in parts $i > k - 3$ is covered within a factor $8$ by the auction for the grand bundle.  Since there are $\lceil \log{m} \rceil + 1$ auctions in the support of \textsc{Random-Bundles} and each are run uniformly at random, we obtain the $O(\log{m})$-approximation as desired.
\end{comment}
% \end{proof}

\subsection{Proof of Lemma \ref{lem:auction-sampling}}\label{subsec::proof-auction-sampling}

To prove \cref{lem:auction-sampling}, we rely on the following useful and general lemma from \cite{bei2017worst}.  
\begin{lemma}[Lemma 2.1 from \cite{bei2017worst}]
Consider any subadditive function $f : A \rightarrow \mathbb{R}$.  For a given subset $S \subseteq A$ and a positive integer $k$ we assume that $f(S) \geq k \cdot f(\{i\})$ for any $i \in S$.  Further, suppose that $S$ is divided uniformly at random into two groups $T_1$ and $T_2$.  Then, with probability of at least $1/2$, we have $f(T_1) \geq \frac{k-1}{4k} \cdot  f(S)$ and $f(T_2) \geq \frac{k-1}{4k} \cdot f(S)$.
\end{lemma}

Applying this lemma to our auction setting, we can prove our desired statement.

\begin{proof}[Proof of Lemma \ref{lem:auction-sampling}]
To show the desired statement we first argue that the optimal welfare $f$ achievable in any  auction setting (either with homogeneous or heterogeneous items) is a \emph{subadditive} function \emph{over the bidders}. To this end,  consider a set of bidders $S$ and the optimal solution over these bidders.  Observe that each item is allocated to any bidder at most once.  Thus, for any two sets $T_1$ and $T_2$ such that $T_1 \cup T_2 = S$ we have that any feasible solution $x$ in $S$ comprises bidders which appear either in $T_1$ or $T_2$ (or possibly both).  Since we can then construct feasible solutions in $T_1$ and $T_2$ which capture the allocation $x$ it must be that the value of the optimal solution in $T_1$ plus the value of the optimal solution in $T_2$ is greater than or equal to the value of the optimal solution in $S$.

But then, since we have that the optimal welfare on an instance of combinatorial auctions is subadditive over the ground set of bidders and we assume that no bidder is critical, we have that the $k$ in the statement of Lemma 2.1 of \cite{bei2017worst} is $100$.  As such, we have that when we randomly partition the bidders into unsampled and sampled sets, both sets have an optimal welfare that is sets are within a factor $99/400 > 1/5$ of the optimal welfare with probability at least $1/2$.
\end{proof}

\subsection{Proof of Lemma \ref{lemma-small-pay}}\label{subsec::proof-lemma-small-pay}
% We now prove Lemma \ref{lemma-small-pay}. 
% Lemma \ref{lemma-small-pay-add} and Lemma \ref{lemma-small-pay-unit}. 
The proof is a direct consequence of the approximation guarantees of the mechanisms considered, as well 
as  the fact that they are
% the mechanisms in all of them are 
obviously strategy-proof (and thus also dominant-strategy incentive compatible) and satisfy individual rationality and no negative transfers. 
The proof is very similar to a proof in \cite{Ron24} and we write it for completeness.
% We write the proofs  for the sake of completeness.

% For brevity, we attribute certain inequalities to individual rationality, rather than elaborating on their derivation from the allocation rule and payment scheme satisfying individual rationality. We adopt the same approach for the no negative transfers property.
\begin{proof}[Proof of Lemma \ref{lemma-small-pay}]
For part \ref{item-1}, note that because of individual rationality:
\begin{equation}\label{eq-1}
 v_1^{one}(f(v_1^{one},v_2^{one}))-P_1(v_1^{one},v_2^{one})\ge 0   
\end{equation}
We remind that by Claim \ref{claim-mua-sm-instances} part \ref{condi-1}, the allocation rule $f$ allocates to bidder $1$ at least one item given $(v_1^{one},v_2^{one})$, so:
\begin{equation}\label{eq-2}
 v_1^{one}(f(v_1^{one},v_2^{one}))=1   
\end{equation}
Combining (\ref{eq-1}) and (\ref{eq-2}) gives part \ref{item-1}.



For part \ref{item-2}, note that given $(v_1^{ONE},v_2^{ALL})$, Claim \ref{claim-mua-sm-instances} part \ref{condi-2} implies that
player $1$ gets the empty bundle. Because of individual rationality: 
$$v_1^{ONE}(f(v_1^{ONE},v_2^{all}))-P_1(v_1^{ONE},v_2^{all})\ge 0 $$
Since $v_1^{ONE}(f(v_1^{ONE},v_2^{all}))=0$, we get that $0\ge P_1(v_1^{ONE},v_2^{all})$, and because of the no negative transfers property, we get $P_1(v_1^{ONE},v_2^{all})=0$, as needed.

For part \ref{item-3}, note that by Claim \ref{claim-mua-sm-instances} part \ref{condi-3}, player $1$ wins all of the items given $(v_1^{all},v_2^{one})$, so $v_1^{all}(f(v_1^{all},v_2^{one}))=k^2$. Combining this fact with individual rationality implies that $P_1(v_1^{all},v_2^{one})\le k^2$. Therefore:
\begin{equation*}
    v_1^{ALL}(f(v_1^{all},v_2^{one}))-P_1(v_1^{all},v_2^{one})\ge k^4 - k^2
\end{equation*}
Since the mechanism is obviously strategy-proof, it is therefore also dominant-strategy incentive compatible, we get that: $    v_1^{ALL}(f(v_1^{ALL},v_2^{one}))-P_1(v_1^{ALL},v_2^{one})\ge k^4 - k^2
$. The only valuable bundle for player $1$ given $v_1^{ALL}$ is the grand bundle, so we get that 
$f(v_1^{ALL},v_2^{one})$ allocates all items to player $1$. In addition, it implies that the payment of player $1$ given $(v_1^{ALL},v_2^{one})$ is necessarily at most $k^2$, which completes the proof. \qedhere

% Note that $f$ clearly also allocates all items to player $1$ given $({v_1}^{all},v_2^{one})$ because of its approximation guarantee. The fact that the mechanisms is dominant-strategy incentive compatible and $f$ outputs the same allocation for both $(\hat{v_1},v_2^{one})$ and  $({v_1}^{all},v_2^{one})$ implies that $P_1({v_1}^{all},v_2^{one})=P_1(\hat{v_1},v_2^{one})$, so $P_1({v_1}^{all},v_2^{one})$ is also smaller than $k^2$, which completes the proof.     
\end{proof}

\subsection{Proof of Theorem \ref{thm:decreasing-marginals}}
\label{subsec::proof--dec-mua}
\begin{proof}
First of all, the mechanism is obviously strategy-proof, and the proof of it is identical to the proof of \cref{claim-mua-sm-mechanism-osp}. 

We now proceed to prove the approximation guarantee of the mechanism, following an approach that closely resembles the proof of \cref{lem:single-minded-approx}. We remind that a bidder $i$ is \emph{critical} if allocating to her the grand bundle
gives a $1/100$-approximation to the optimal welfare. 
% As before, in the case that there is a critical bidder $i$, allocating $i$ the grand bundle necessarily gives a $1/100$-approximation to the optimal welfare.  
Thus, if there exists a critical bidder, we obtain a $1/200$-approximation by running an ascending auction for the grand bundle with probability at least $1/2$.

    In the case that there does not exist a critical bidder, we may again appeal to our sampling lemma, i.e., Lemma \ref{lem:auction-sampling}, to show that when we sample bidders, we obtain an ``accurate enough'' estimates with probability at least $1/2$. Formally, with probability $1/2$ we have that the social welfare $\text{OPT}(S)$ contained in the sampled set $S$ is between $\text{OPT}/5$ and $\text{OPT}$  and the same holds for $\text{OPT}(U)$, the welfare among the unsampled bidders. 
    % (and, further, the social welfare $\text{OPT}(U)$ of the set of unsampled bidders also falls in this range). 
    Therefore, with probability $\frac 1 2 $ we set a per-item price $p$ which is in the range $[\text{OPT}/(50m), \text{OPT}/(10m)]$. We will now further refine the case analysis by examining the number of items sold.
    % , still similarly to the proof of \cref{lem:single-minded-approx},  
    
    The \textquote{easy} case is if we sell at least $\nicefrac m 2$ items. 
      Since an unsampled bidder buying $t$ goods spends at least $\frac{t\text{OPT}}{50m}$, their value for the purchased bundle is at least $\frac{t\text{OPT}}{50m}$. Therefore, the total value of all bidders who purchase goods is at least $\frac{m}{2}\cdot\frac{\text{OPT}}{50m} = \frac{\text{OPT}}{100}$. Therefore,  we  obtain welfare of at least $\text{OPT}/100$. Altogether, since we run uniform sampling with probability $\nicefrac{1}{2}$ and the estimation is
\textquote{good} with probability $\nicefrac{1}{2}$, we obtain  a $400$-approximation to the welfare.
    

    
    Suppose, by contrast, that we sell fewer than $\nicefrac m 2$ items to the unsampled bidders.  
    To analyze this case, let $\vec{q}=(q_1,\dots,q_n)$ be the optimal allocation if the items are divided only among the bidders in $U$ (clearly, every bidder not in $U$ is allocated zero items, and the welfare of $\vec q$ is equal to $\text{OPT}(U)$). 
    Observe that the welfare of $\vec q$
is partitioned to ``low'' marginal values, i.e., marginal values less than or equal to $p$ and ``high'' marginal values, which are greater than $p$.

Observe that since Mechanism \ref{alg:single-minded} allocates less than $\frac m 2$ items, every bidder in  $U$  could have bought additional items at a price of $p$. Therefore, all bidders who were allocated in $\vec q$ were also allocated by Mechanism \ref{alg:single-minded} all the items for which they had \textquote{high} marginal values. Therefore, it remains to bound the loss coming from \textquote{low} margins. Observe that the marginal value each agent in $U$ has for receiving an additional good is no more than $\text{OPT}/(10m)$ and since there are $m$ items in total allocated in $\vec q$, the total welfare of $\vec q$ coming from \textquote{low} marginals is at most $\text{OPT}/10$. However, by assumption the total welfare of $\vec q$ is at least $\text{OPT}/5$,  so at least 
$\text{OPT}/10$ of welfare comes from \textquote{high} marginals which also contribute to the welfare of the allocation of Mechanism \ref{alg:single-minded}. As we said before, this 
% happens with probability $\frac{1}{4}$,
depends on finding a good partition of $U,S$ and running a uniform price auction which occurs in probability $\nicefrac{1}{4}$, 
so overall the expected welfare of the mechanism in this case is at least $\frac{OPT}{40}$. 
    

    
    
    
% Note that  since we sold fewer than $m/2$ items, we know that each bidder in $U$ had the opportunity to purchase items at a price of $p$. 

% additional units of the good but chose not to do so. Therefore, all the marginal values the portion of the welfare of $\vec q$  

% Because each bidder in $U$ purchased their utility maximizing bundle, we have that the marginal value each agent in $U$ has for receiving an additional good is no more than $\text{OPT}/(10m)$.  
    
%     We will now use this fact. Let $\vec{q}$ denote the optimal allocation of items to bidders in $U$ (that has welfare of $\text{OPT}(U)$ by definition). We consider the portion of total welfare of $\vec{q}$
% which comes from ``low'' marginal values (i.e., marginal values less than or equal to $\text{OPT}/(10m)$), and compare it with ``high'' marginal values, namely with values greater than $\text{OPT}/(10m)$. 
%     % and consider the portion of the  total welfare of $\vec{q}$ which comes from these ``low'' marginal values (i.e., marginal values less than or equal to $\text{OPT}/(10m)$). 
%     Since there are $m$ items in total we have that the portion of the welfare the comes from \textquote{low marginals} is at most $\text{OPT}/10$.  Since $\text{OPT}(U) \geq \text{OPT}/5$, the portion of the welfare in $\vec{q}$ coming from ``high'' marginal values is at least $\text{OPT}/10$.  But since each bidder in $U$ had the opportunity to purchase additional goods (i.e., no bidder was prevented from purchasing a good that she otherwise would have wanted to), it must be that our auction obtains all of the welfare in $\vec{q}$ coming from ``high'' marginals, i.e., at least $\text{OPT}/10$ welfare.  Hence, since we run the sample-and-price phase with probability $1/2$ and, conditioned on this, accurately sample with probability at least $1/2$, we obtain social welfare at least $\text{OPT}/400$ in the case that there is no critical bidder.

Combining all cases, we conclude that the expected welfare of Mechanism \ref{alg:single-minded} is at least $\frac{OPT}{400}$,  thereby completing the proof.
\end{proof}

% \subsection{Proof of \cref{claim-oneone-same,claim-ONE-ALL-same}} \label{subsec:mua-sm-claims-proofs}




\subsection{ Proof of  Theorem \ref{thm-lb-mua-dec}: Impossibility for 2 Items and 2 Bidders} \label{subsec-lb-proof-mua-dec}
First of all, We assume that the domain $V_i$ of each bidder consists of valuations with values in  $\{0,1,\ldots,k^4\}$ that satisfy decreasing marginal utilities, where $k$ is an arbitrarily large number. 

Assume towards a contradiction that 
there exists an obviously strategy-proof, individually rational, no negative transfers mechanism $A$ together with strategy profile $\mathcal S=(\mathcal S_1,\mathcal S_2)$
that implement an allocation rule and payment schemes  $(f,P_1,P_2):V_1\times V_2\to \allocs\times \mathbb R^2$, where $f$ 
gives an approximation strictly better than $2$ to the optimal social welfare.  
% Thus, there exists
% an allocation rule
% $f:V_1\times\cdots\times V_n\to \mathcal T$  and payment schemes  $P_1,\ldots,P_n:V_1\times \cdots \times V_n\to \mathbb{R}^n$ the payment schemes  
% Fix an allocation rule  with approximation ratio $\alpha$ such that $\alpha>\max\{\frac{1}{n},\frac{1}{m}\}$. Let $\mathcal{M}$ be a normalized mechanism 
% and strategies  $(\mathcal S_1,\ldots,\mathcal{S}_n)$  that realize $f$ together with the payment schemes $P_1,\ldots,P_n:V_1\times \cdots \times V_n\to \mathbb{R}^n$. 
For every player $i$, we define three valuations that will be of particular interest:  
$$
v_i^{ALL}(x)=\begin{cases}
k^2 &\quad x=1, \\
2k^2 &\quad x=2.
\end{cases} \\\quad 
v_i^{one}(x)= \begin{cases}
1 \quad x=1, \\
1 \quad x=2.
\end{cases}
 \quad
v_i^{ONE}(x)=\begin{cases}
4k \quad x= 1, \\
4k  \quad x=2.
\end{cases} 
$$
where $k$ is arbitrarily large. 


For the analysis of the mechanism, we define the following subsets of valuations:
$
\mathcal{V}_1=\{v_1^{one},v_1^{ONE},\allowbreak v_1^{ALL}\}\allowbreak \subseteq V_1$ and  $\mathcal{V}_2=\{v_2^{one},v_2^{ONE},v_2^{ALL}\}\subseteq V_2$.  
% We denote $\mathcal{V}_1\times\cdots\times \mathcal{V}_n$ with $\mathcal V$.  
% use the notation $\mathcal V=\mathcal{V}_1\times\cdots\times \mathcal{V}_n$.    
%$\mathcal V_1=\{v_1^{one},v_1^{ONE},v_1^{all}\}$ and $\mathcal V_2=\{v_2^{one},v_2^{ONE},v_2^{all}\}$. For every player $i\ge 3$, we define $\mathcal V_i=\{v_i^{one}\}$.
% We now analyze the mechanism given that the valuations of the players belong to $\mathcal V=\mathcal V_1\times  \cdots \times \mathcal V_n$.  
Observe that:
\begin{claim}\label{claim-mua-dec-instances}
Every deterministic mechanism that has approximation strictly better than $2$ necessarily satisfies the following  conditions simultaneously:
\begin{enumerate}
    \item Given the valuation profile $(v_1^{one},v_2^{one})$, the mechanism
    allocates one item to bidder $1$ and one item to bidder $2$. \label{condi-1-dec}
% \item Given the valuation profile $(v_1^{\text{ONE}}, v_2^{\text{one}})$, bidder $1$ wins at least one item.  \label{condi-2-dec}
    \item Given the valuation profile $(v_1^{ONE}, v_2^{ALL})$, bidder $2$ wins all items.
    \label{condi-3-dec}
\end{enumerate}
\end{claim}
The proof of \cref{claim-mua-dec-instances} is straightforward: if a deterministic mechanism does not satisfy one of conditions, then due the fact that $k$ is arbitrarily large implies that the approximation guarantee of the mechanism is at most $2$ in the worst case.  


We now focus on 
$
\mathcal{V}_1\times \mathcal{V}_2$. 
Observe that there necessarily exists a vertex $u$, and valuations $v_1,v_1' \in \mathcal{V}_1$, and  $v_2,v_2' \in \mathcal{V}_2$ such that $(\mathcal{S}_1(v_1), \mathcal{S}_2(v_2))$ diverge at vertex $u$. This follows from \cref{claim-mua-dec-instances}, which implies that the mechanism $A$ must output different allocations for different valuation profiles in $\mathcal{V}_1 \times \mathcal{V}_2$. Consequently, not all valuation profiles end up in the same leaf, meaning that divergence must occur at some point. 

% Observe that either bidder $1$ or $2$ has to send different messages for different valuations in $\mathcal V_i$ at some vertex. This is an immediate implication of \cref{claim-mua-dec-instances}, as the mechanism $A$ necessarily outputs different allocations given the valuation profiles $(v_1^{one},v_2^{one})$ and $(v_1^{ONE},v_2^{ALL})$, meaning that the behavior profiles 
 % \dnote{Missing parentheses here?} $(\mathcal S_1(v_1^{one},\mathcal S_2(v_2^{one}))$ and $(\mathcal S_1(v_1^{all},\mathcal S_2(v_2^{one}))$ reach different leaves.

Let $u$ be the first vertex in the protocol such that 
 %the behavior profiles 
 $(\mathcal{S}_1(v_1),\mathcal{S}_2(v_2))$ and $(\mathcal{S}_1(v_1'),\mathcal{S}_2(v_2'))$ diverge, i.e., dictate different messages. 
Note that by definition this implies that $u\in Path(\mathcal{S}_1(v_1),\mathcal{S}_2(v_2))\cap Path(\mathcal{S}_1(v_1'),\mathcal{S}_2(v_2'))$ and that either bidder $1$ or bidder $2$ sends different messages for the valuations in $\mathcal{V}_1$ or $\mathcal V_2$, respectively. 
Without loss of generality, we assume that bidder $1$ sends different messages, meaning that there exist $v_1,v_1'\in \mathcal{V}_1$ such that $\mathcal S_1(v_1)$ and $\mathcal S_1(v_1')$ dictate different messages at vertex $u$.  
We remind that $\mathcal{V}_1=\{v_1^{one},v_1^{ONE},v_1^{all}\}$, so the  following claims jointly imply a contradiction, completing the proof: 
 
 \begin{comment}
 Let $u$ be the first vertex in the protocol such that the behavior profiles $(\mathcal{S}_1(v_1),\mathcal{S}_2(v_2))$ and $(\mathcal{S}_1(v_1'),\mathcal{S}_2(v_2'))$ dictate different messages. 
Note that by definition $u\in Path(\mathcal{S}_1(v_1),\mathcal{S}_2(v_2))\cap Path(\mathcal{S}_1(v_1'),\mathcal{S}_2(v_2'))$. We remind that each vertex is associated with only one player that sends messages in it. 
We assume without loss of generality that player $1$ is the player that sends a message in vertex $u$, so  there exist $v_1,v_1'\in \mathcal{V}_1$ such that $\mathcal S_1(v_1)$ and $\mathcal S_1(v_1')$ dictate different messages at vertex $u$. We remind that $\mathcal{V}_1=\{v_1^{one},v_1^{ONE},v_1^{all}\}$, so 
the  following claims jointly imply a contradiction, completing the proof of Theorem \ref{thm-lb-mua-dec}:
\end{comment}
\begin{claim}\label{claim-oneone-same-dec}
    The strategy $\mathcal S_1$ dictates the same message at vertex $u$ for the valuations $v_1^{one}$ and $v_1^{ONE}$. 
\end{claim}
\begin{claim}\label{claim-one-all-same-dec}
        The strategy $\mathcal S_1$ dictates the same message at vertex $u$ for the valuations $v_1^{ONE}$ and $v_1^{ALL}$.
\end{claim}
% Observe that by construction, every $v\in \mathcal V$ satisfies that vertex $u$ is in $Path(\mathcal S(v))$. 
In the proofs of Claims \ref{claim-oneone-same-dec} and Claim \ref{claim-one-all-same-dec}, we use the following lemma:
\begin{lemma}\label{lemma-small-pay-dec}
    The allocation rule $f$ and the payment scheme $P_1$ of bidder $1$ satisfy that:
    \begin{enumerate}
        \item Given $(v_1^{one},v_{2}^{one})$ bidder $1$ wins  one item and pays at most $1$.  \label{item-1-dec}
        \item  Given $(v_1^{ONE},v_2^{ALL})$, bidder $1$ gets the empty bundle and pays zero.   \label{item-2-dec}
        \item Given $(v_1^{ALL},v_2^{one})$, bidder wins all items and pays at most $2k$. \label{item-3-dec}  
    \end{enumerate}
\end{lemma}
The lemma is a direct consequence of the properties of the mechanism. 
We use 
it to prove \cref{claim-oneone-same-dec,claim-one-all-same-dec}
and defer the proof to  \cref{sec-small-pay-proof-alltogether-dec}.  The proofs are identical to those in \cite{Ron24}, and we include them here for completeness.
% We now use Lemma \ref{lemma-small-pay} to obtain a contradiction. 
% The proof proceeds as follows. We will show that since $\mathcal S_1$ is obviously dominant, then it necessarily dictates the same message for $v_1^{one}$ and for $v_1^{ONE}$ at vertex $u$. Using similar arguments, we will also show that $\mathcal S_1$ dictates the same message for $v_1^{ONE}$ and $v_1^{all}$ at vertex $u$. Thus, the strategy $\mathcal S_1$ assigns the same message to all the valuations in $\mathcal V_1$, so we get a contradiction, which completes the proof.  
%%%PROOF-OF-PLACE%%%
\begin{proof}[Proof of Claim \ref{claim-oneone-same-dec}]
% \begin{proof}[of Claim \ref{claim-oneone-same}]
    Note that by Lemma \ref{lemma-small-pay-dec} part \ref{item-1-dec}, $f(v_1^{one},v_2^{one})$ allocates at least one item to player $1$ and $P_1(v_1^{one},v_2^{one})\le 1$. Therefore:
\begin{equation}\label{eq-good-leaf1-dec}
 v_1^{ONE}(f(v_1^{one},v_2^{one}))-P_1(v_1^{one},v_2^{one})\ge 4k-1 
\end{equation}
 In contrast, by part \ref{item-2-dec} of Lemma \ref{lemma-small-pay-dec},   $f(v_1^{ONE},v_2^{ALL})$ allocates no items to player $1$ and $P_1(v_1^{ONE},v_2^{ALL})=0$, so:
 \begin{equation}\label{eq-bad-leaf1-dec}
 v_1^{ONE}(f(v_1^{ONE},v_2^{ALL}))-P_1(v_1^{ONE},v_2^{ALL})= 0   
\end{equation}
Combining inequalities (\ref{eq-good-leaf1-dec}) and (\ref{eq-bad-leaf1-dec}) gives:
\begin{equation*}
  v_1^{ONE}(f(v_1^{ONE},v_2^{ALL}))-P_1(v_1^{ONE},v_2^{ALL})< 
  v_1^{ONE}(f(v_1^{one},v_2^{one}))-P_1(v_1^{one},v_2^{one})  
\end{equation*}
We remind that vertex $u$ belongs in $Path(\mathcal S_1(v_1^{one}),\mathcal S_2(v_2^{one}))$ and also in
$Path(\mathcal{S}_1(v_1^{ONE}),\allowbreak\mathcal{S}_2(v_2^{ALL}))$. Therefore, Lemma \ref{lemma-bad-leaf-good-leaf} gives that the strategy $\mathcal S_1$ dictates the same message for  $v_1^{one}$ and $v_1^{ONE}$ at vertex $u$.
\end{proof}

\begin{proof}[Proof of \cref{claim-one-all-same-dec}]
    Following the same approach as in the proof of Claim \ref{claim-oneone-same-dec}, note that by \cref{lemma-small-pay-dec}  \cref{item-3-dec}: 
\begin{equation}\label{break-align-dec}
v_1^{ONE}(f(v_1^{ALL},v_2^{one}))-P_1(v_1^{ALL},v_2^{one}) \ge 4k-2k=2k     
\end{equation}
Also, by \cref{lemma-small-pay-dec}  \cref{item-2}:
\begin{equation}\label{break-align2-dec}
  v_1^{ONE}(f(v_1^{ONE},v_2^{ALL}))  
-P_1(v_1^{ONE},v_2^{ALL})=0   
\end{equation}
Combining \cref{break-align-dec} and \cref{break-align2-dec}:
\begin{equation*}
    v_1^{ONE}(f(v_1^{ONE},v_2^{ALL}))  
-P_1(v_1^{ONE},v_2^{ALL})< v_1^{ONE}(f(v_1^{ALL},v_2^{one}))-P_1(v_1^{ALL},v_2^{one})
\end{equation*}
% where the first inequality is by Lemma \ref{lemma-small-pay} part \ref{item-3} and the equality is by Lemma \ref{lemma-small-pay} part \ref{item-2}.
Combining the above inequality with the fact that 
vertex $u$ belongs in $Path(\mathcal S_1(v_1^{ALL}),\mathcal S_2(v_2^{one}))$ and in
$Path(\mathcal{S}_1(v_1^{ONE}),\mathcal{S}_2(v_2^{ALL}))$ implies that by Lemma \ref{lemma-bad-leaf-good-leaf} 
% applying Lemma \ref{lemma-bad-leaf-good-leaf} gives
the obviously dominant strategy $\mathcal S_1$ dictates the same message for the valuations $v_1^{ONE}$ and $v_1^{ALL}$ at vertex $u$. 
\end{proof}


\subsection{Proof of Lemma \ref{lemma-small-pay-dec}: Observations About The Mechanism}\label{sec-small-pay-proof-alltogether-dec}
The proof is a direct consequence of the approximation guarantee of the mechanism and the fact that it is 
% the mechanisms in all of them are 
obviously strategy-proof (and thus also dominant-strategy incentive compatible) and satisfies individual rationality and no negative transfers. 
%We write the proof  for the sake of completeness.

%Throughout the proof, for brevity we say that certain inequalities hold  because of individual rationality instead of saying that they hold because the allocation rule together with the payment scheme are realized by a mechanism and strategies that satisfy individual rationality. We do the same for the no negative transfers property. 



% The following  abbreviations will be  utilized in the proofs of the following lemmas. For Lemma \ref{}
% all of them, we have
% previously defined the set $\mathcal V$ in a way that gurantees that 
%  that every valuation profile $(v_1,\ldots,\allowbreak v_n)\allowbreak\in \mathcal V_1\times \cdots \times \mathcal V_n$ satisfies that for every $i\ge 3$, $\mathcal V_i$ is a singleton, so   the valuation $v_i$ is in fact $v_i^{one}$. Therefore, we write $f(v_1,v_2)$ for $f(v_1,v_2,v_3^{one},\ldots,v_n^{one})$ and $P_1(v_1,v_2)$ for $P_1(v_1,v_2,v_3^{one},\ldots,v_n^{one})$. 
% In addition, we say for abbreviation 
% By Claim \ref{claim-ir-npt-norm}, the mechanism $\mathcal{M}$ together with the strategies $(\mathcal S_1,\ldots,\mathcal S_n)$ are normalized (because by assumption they satisfy individual rationality and no negative transfers). We will use this fact extensively in the analysis. 
% Now we can start proving the items in Lemma \ref{lemma-small-pay}. 

For part \ref{item-1}, note that because of individual rationality:
\begin{equation}\label{eq-1-dec}
 v_1^{one}(f(v_1^{one},v_2^{one}))-P_1(v_1^{one},v_2^{one})\ge 0   
\end{equation}
We remind that by \cref{claim-mua-dec-instances} \cref{condi-1-dec}, the allocation rule $f$ allocates to bidder $1$ one item given $(v_1^{one},v_2^{one})$, so:
\begin{equation}\label{eq-2-dec}
 v_1^{one}(f(v_1^{one},v_2^{one}))=1   
\end{equation}
Combining (\ref{eq-1-dec}) and (\ref{eq-2-dec}) gives part \ref{item-1-dec}.



For part \ref{item-2-dec}, note that 
by \cref{claim-mua-dec-instances} \cref{condi-3-dec}, 
given $(v_1^{ONE},v_2^{ALL})$ player $1$ gets the empty bundle. Because of individual rationality: 
$$v_1^{ONE}(f(v_1^{ONE},v_2^{ALL}))-P_1(v_1^{ONE},v_2^{ALL})\ge 0 $$
Since $v_1^{ONE}(f(v_1^{ONE},v_2^{ALL}))=0$, we get that $0\ge P_1(v_1^{ONE},v_2^{ALL})$, and because of the no negative transfers property, $P_1(v_1^{ONE},v_2^{ALL})=0$, as needed.  

To prove part \ref{item-3-dec}, we define another valuation:
$$
\hat{v}_1(x)=\begin{cases}
    k \quad &x=1, \\
    2k \quad &x=2.
\end{cases}
$$
Given $(\hat{v}_1,v_2^{one})$, player $1$ wins $2$ items because  
$f$ gives an approximation strictly better than $2$. Thus, the inequality $\hat{v}_1(f(\hat{v}_1,v_2^{one}))-P_1(\hat{v}_1,v_2^{one})\ge 0$ holds because of individual rationality, and therefore $P_1(\hat{v}_1,v_2^{one})\le 2k$. 

Note that $f$ clearly also allocates all items to player $1$ given $({v_1}^{ALL},v_2^{one})$ because of its approximation guarantee. The fact that the mechanism is dominant-strategy incentive compatible and $f$ outputs the same allocation for both $(\hat{v}_1,v_2^{one})$ and  $({v_1}^{ALL},v_2^{one})$ implies that $P_1({v_1}^{ALL},v_2^{one})=P_1(\hat{v}_1,v_2^{one})$, so $P_1({v_1}^{ALL},v_2^{one})$ is also smaller than $2k$, which completes the proof. 

\subsection{Proof of Lemma \ref{lemma:mono-mua-dec}: A  Deterministic Mechanism For 3 Items and 2 Bidders}\label{sec-impos-mua-dec}
% \begin{proof}[Proof of \cref{lemma:mono-mua-dec}]
    Consider the following mechanism: allocate one item to each bidder, and run an ascending auction on the remaining item. Assume tie breaking is in favor of player $1$, i.e., if both players have the same value for the item, then player $1$ wins it.
    This mechanism is a generalized ascending auction, so by \cref{lemma-partial} it is obviously strategy-proof. 
    
    We now analyze its approximation guarantee. 
Intuitively, the mechanism guarantees a $1.5$ approximation because the worst  case scenario is that some bidder  should  have gotten all three items in the optimal solution, but gets instead only two items. The decreasing marginal property ensures that the loss is bounded by at most $\frac{1}{3}$ of the optimal social welfare.

% for it is at most $\max\{v_1(2)-v_1(1),v_2(1)-v_2(0)\}$. 

% and his marginal value for it is at most $\frac{1}{3}$ of $\max\{v_i(2)-v_i(1),v_i(1)-v_i(0)\}$. 

% The worst 
 
%  is straightforward: one can easily see that the 
    
    Formally, 
denote with $(q_1,q_2)$ the allocation of the mechanism, and let $(o_1,o_2)$ be a welfare-maximizing allocation. We use the following notations: $$ALG=v_1(q_1)+v_2(q_2),\quad OPT=v_1(o_1)+v_2(o_2)$$

    

    Note that the algorithm necessarily outputs an allocation where one bidder gets $2$ items and the other bidder gets $1$ item. Assume without loss of generality that the ascending auction gives bidder $1$ two items and bidder $2$ one item, meaning that  $q_1=2$ and $q_2=1$. Note that by the definition of the auction:
    \begin{equation}\label{eq-opt-ge-alg}
        v_1(2)-v_1(1)\ge v_2(2)-v_2(1) 
    \end{equation}
To show that $ALG \ge \frac{2}{3}\cdot OPT$,  we proceed with the following case analysis:
\paragraph{Case I: $o_1=3$ and $o_2=0$.}
We will first show that $v_1(3)-v_1(2)\le \frac{OPT}{3}$, and then show why it implies that $ALG \ge \frac{2}{3}\cdot OPT$.
Observe that:
\begin{equation*}\label{eq-case-30}
    OPT=v_1(3)=v_1(3)-v_1(2)+v_1(2)-v_1(1)+v_1(1) \ge 3\cdot \big(v_1(3)-v_1(2)\big)
    % \le 3 \cdot v_1(1)
\end{equation*}
where the inequality holds because $v_1$ has decreasing marginal values. 
Therefore:
$$
OPT=v_1(3)=v_1(2)+v_1(3)-v_1(2)\le ALG +v_1(3)-v_1(2)\le ALG + \frac{OPT}{3}
$$
So $ALG \ge \frac{2\cdot OPT}{3}$, as needed. 


% $v_1(1)\ge \frac{OPT}{3}$. Combining this with \cref{eq-case-30} gives that:
% Since $v_1(2)-v_1(1)\ge v_1(3)-v_1(2)$

\paragraph{Case II: $o_1=2$ and $o_2=1$.} In this case, the optimal allocation is identical to the allocation of the mechanism, so $ALG=OPT$ holds trivially .

\paragraph{Case III: $o_1=1$ and $o_2=2$.} Observe that $ALG=v_1(2)+v_2(1)\ge v_1(1)+v_2(2)=OPT$, where the inequality is by \cref{eq-opt-ge-alg}.

% Observe that by construction, the fact that the auction outputs the allocation $(q_1=2,q_2=1)$ implies that:
% $
%     v_1(2)-v_1(1)\ge v_2(2)-v_2(1)$.
% Therefore,  $ALG=v_1(2)+v_2(1)\ge v_1(1)+v_2(2)=OPT$, where the inequality holds by 

\paragraph{Case IV: $o_1=0$ and $o_2=3$.} Due to the same explanation as in case III, we have that: 
\begin{equation}\label{case-03}
    ALG=v_1(2)+v_2(1)\ge v_1(1)+v_2(2)
\end{equation}
And also that:
\begin{align*}
v_1(1) &\ge v_1(2)-v_1(1) &\text{(due to decreasing margins)} \\
&\ge v_2(2)-v_2(1) &\text{(by \cref{eq-opt-ge-alg})} \\
&\ge v_2(3)-v_2(2) &\text{(due to decreasing margins)} \\ 
\end{align*}
So $v_1(1)+v_2(2)\ge v_2(3)=OPT$. Combining this with \cref{case-03} gives that $ALG \ge OPT$, as needed.  
% \end{proof}