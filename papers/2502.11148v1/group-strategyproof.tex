In his seminal work, \cite{li} has observed that  obviously strategy-proof mechanisms satisfy the quality of being weakly group strategy-proof\footnote{Loosely speaking, a mechanism $A$ is weakly group strategy-proof if no coalition of bidders can all simultaneously increase their utility by deviating from the dominant strategy.}, but the converse is not true: the top trading cycles mechanism is weakly group strategy-proof, but is not obviously strategy-proof (see also \cite{troyan2019obviously}). This gives rise to the question of understanding the differences between the two notions.

On one hand, weak group strategy-proofness is already
quite restrictive. For illustration, in certain settings (such as multi-unit auction with decreasing marginal valuations)  enforcing weak group strategy-proofness precludes exact welfare approximation \cite{GMR17}. 
This raises the question of how much more restrictive a notion can become beyond weak group strategy-proofness, and in particular whether obvious strategy-proofness is more restrictive than weak group strategy-proofness.  We ask:
%As such, a crisp separation between OSP and group strategy-proofness in auction settings is unclear -- 
are there settings in which OSP mechanisms must necessarily obtain worse approximation guarantees than mechanisms which are ``only'' weakly group strategy-proof? 

We answer this question affirmatively by examining the setting with heterogeneous items and unit-demand bidders.  First, as \cite{Ron24} has shown, deterministic obviously strategy-proof mechanisms that satisfy individual rationality and no negative transfers cannot obtain approximation better than $\min\{m,n\}$. Furthermore, as we show in \cref{thm-lb-ud},  one cannot achieve a $(\nicefrac 8 7-\varepsilon)$-approximation to the social welfare with any randomized OSP mechanism in this setting (similarly, assuming individual rationality and no negative transfers).  On the other hand, we will now explain why  for this setting the VCG mechanism (which is clearly optimal and deterministic) \emph{is} weakly group strategy-proof.  

This holds due to the combination of the following two well known facts.
First, the outcome corresponding to the minimum-price Walrasian equilibrium is weakly group strategy-proof in unit-demand settings even beyond quasi-linear utilities (see, e.g., \cite{morimoto2015strategy}). The second known fact is that for unit-demand settings the minimum-price Walrasian equilibrium corresponds to the VCG outcome and prices (see, e.g., Theorem 5 of \cite{gul1999walrasian}).
Therefore, we obtain a separation between the approximation ratio achievable by OSP mechanisms (which is $\min\{m,n\}$ for deterministic mechanisms \cite{Ron24}, and bounded away from $1$ for randomized ones) and the ratio of weakly group strategy-proof mechanisms (which is $1$ exactly).  


We leave open the question of understanding the power of weakly group strategy-proof mechanisms for additional auction settings. In particular, we have yet to understand the power of weakly group strategy-proof mechanisms for additive bidders.  
For this setting, we cannot apply the arguments previously made for unit-demand bidders, as the the Vickrey-Clarke-Groves (VCG) mechanism 
%\cite{Vic61,C1971,G1973}
is not weakly group strategy-proof\footnote{Take an example with two bidders and two items where the first bidder has value $10$ for both items and the second has value $9$ for both items.  By colluding and reporting that they only value distinct items, both bidders receive an item at a price of $0$ (which is preferable to the VCG outcome for both bidders).}. 

  

%To our knowledge, this is the first known separation for auction settings between the approximation guarantees achievable between OSP mechanisms and group strategy-proof mechanisms and 
% An intriguing line of future work would be to see how large this separation is for various auction settings.