In this section, we describe improved approximation guarantees for the special case of two bidders and two items. Observe that this is the simplest case for which the power of randomized obviously strategy-proof mechanisms is not resolved.\footnote{If we consider only one bidder, then a mechanism that always allocates to her all the items is optimal and obviously strategy-proof. If there is only one item, then an ascending auction on this item among the bidders is optimal and obviously strategy-proof.} We believe that understanding the power of randomized obviously strategy-proof mechanisms in this simple case is a first step towards understanding their power for the more general settings of arbitrary number of bidders and items. 

We consider three classes of valuations: multi-unit auctions with unknown single-minded bidders (\cref{subsec:22-mua-sm}), which admit a $\frac{4}{3}$-approximation, combinatorial auctions with subadditive bidders (\cref{subsec:22-ca-subadd}), which also admit a $\frac{4}{3}$-approximation, and combinatorial auctions with general bidders (\cref{subsec:22-ca-general}), which admit a $\frac{3}{2}$-approximation.

We remind the reader that for unknown single-minded bidders in a multi-unit auction, even with just two bidders and two items, no mechanism achieves an approximation better than $\frac{5}{6}$ (\cref{thm-mua-sm-lb}). Thus, there remains a gap of $\frac{1}{12}$ in our understanding of this setting.
Since unit-demand and additive valuations are subadditive, these classes also admit a $\frac{4}{3}$-approximation. As we have shown that even for two bidders and two items (\cref{thm-lb-ud,thm-lb-add}) no mechanism achieves an approximation better than $\frac{7}{8}$, there remains a gap of $\frac{1}{8}$.

In the proofs, we denote $v_i(j~|~j') = v_i(\{j,j'\}) - v(\{j'\})$ as the marginal gain in value of bidder $i$ when adding item $j$ to her bundle already containing item $j'$. We denote the optimal welfare with $\opt$. 

\subsection{Multi Unit Auction for Single-Minded Bidders with Unknown Demands} \label{subsec:22-mua-sm}
% \src{TO DO - mention that it is for unknown bundle sizes}
 % \paragraph{Single-minded, identical items, unknown bundle sizes}
Let $p = 1/2$ and consider the mechanism $\mathcal{M}_1$ which with probability $p$ runs a second-price auction for the grand bundle and with probability $(1-p)$ allocates a uniformly random agent a single item and runs a second-price auction for the second item.  
\begin{claim}\label{claim-mec-sm-mua-osp}
   The mechanism $\mathcal{M}_1$ is universally OSP.  
\end{claim}
\begin{proof}
Observe that $\mathcal{M}_1$ is a probability distribution over two deterministic mechanisms, both of which are generalized ascending auctions. By \cref{lemma-partial}, they are therefore obviously strategy-proof. Thus, $\mathcal M_1$ is obviously strategy-proof in the universal sense.
\end{proof}

\begin{claim}
The mechanism $\mathcal{M}_1$ achieves a $\nicefrac 4 3$-approximation to the optimal social welfare.
\end{claim}
\begin{proof}
    Let $v_i$ denote the private value of agent $i$ if she receives a bundle which satiates her demand. 
    
    First suppose that the optimal allocation obtains positive value from exactly one agent.  Without loss of generality, say that this is agent $1$.  We may immediately conclude that $v_1 \geq v_2$.  Consider that, by definition, agent $1$ is satiated by the grand bundle.  Thus, with probability $p$, the mechanism $\mathcal{M}_1$ satiates agent $1$ and receives a value $v_1$.  On the other hand, with probability $(1-p)/2$, the mechanism $\mathcal{M}_1$ allocates agent $1$ the first item and runs a second-price auction for the second item.  This necessarily yields total welfare at least $v_1$ since $v_1 \geq v_2$ and since agent $1$ will either have marginal value for the second item equal to $v_1$ if she is satiated only by both items or marginal value $0$ if she is satiated by a single item.  Thus, in the case where the optimal allocation serves exactly one agent, $\mathcal{M}_1$ achieves an approximation ratio of at least $(p + (1-p)/2)$.

    Now suppose that the optimal allocation obtains positive value from both agents.  Note that we may immediately conclude that each agent is satiated when she receives at least one item.  Without loss of generality, suppose that $v_1 \geq v_2$.  But then, with probability $p$ $\mathcal{M}_1$ allocates both items to bidder $1$ and obtains welfare $p\cdot v_1$.  On the other hand, in the case that $\mathcal{M}_1$ allocates a single item to a uniformly random agent and runs a second-price auction for the second item, we have that the agent who is randomly given the first item has $0$ marginal value for the second item whereas the other agent has positive marginal value for the item.  But then, in this case, which occurs with probability $(1-p)$, $\mathcal{M}_1$ obtains the optimal welfare.  As such, the expected welfare of $\mathcal{M}_1$ is $p \cdot v_1 + (1-p)\cdot (v_1 + v_2)$.  Since $v_1 \geq v_2$ we then obtain an approximation of $p/2 + (1-p)$.

    The approximation in the first case is monotonically increasing in $p$ whereas the approximation in the second case is monotonically decreasing.  Taking $1/2 + p/2 = p/2 + (1-p)$ gives $p = 1/2$ and thus setting $p=\nicefrac{1}{2}$ implies that $\mathcal M_1$ an approximation ratio of $\nicefrac 4 3$ as desired.
\end{proof}



\subsection{Combinatorial Auction with Subadditive Bidders}\label{subsec:22-ca-subadd}
Consider the mechanism $\mathcal{M}_2$ which uniformly at random selects a bidder $i$ and uniformly at random selects an item $j$ and allocates item $j$ to bidder $i$ and after this allocation occurs runs an ascending second-price auction for the remaining item.  Note that $\mathcal M_2$ is in fact a probability distribution over generalized ascending auctions, so by \cref{lem:auction-sampling} it is OSP in the universal sense.
We argue that it achieves a $\frac 4 3$-approximation for the broad class of \emph{monotone subadditive} valuations.  A function
$v:2^M\to \mathbb{R}^{\geq 0}$ is \emph{monotone} if for all bundles $T \subseteq T' \subseteq M$ we have that $v(T) \geq v(T')$.\footnote{Monotonicity is sometimes called ``free-disposal'' in the literature since a bidder is always weakly happier to receive more goods (because she can always ``dispose'' of goods she is unhappy to receive).}  A function $v:2^M\to \mathbb{R}^{\geq 0}$
is \emph{subadditive}  if for all bundles $T, T' \subseteq M$ we have that $v(T) + v(T') \geq v(T \cup T')$. 

Subadditive valuations are more general than submodular valuations and monotone subadditive valuations capture, as special cases, both additive and unit-demand valuations.  As such, $\mathcal{M}_2$ provides improved approximations for two bidders and two items in combinatorial auctions with both unit-demand and additive valuations.
\begin{claim}
The mechanism $\mathcal{M}_2$ achieves a $\nicefrac 4 3$-approximation to the optimal social welfare for monotone subadditive valuations.
\end{claim}
\begin{proof}
    Let $\{1,2\}$ denote the set of two bidders and $\{a,b\}$ denote the set of two items.  
    We consider two cases depending on whether the optimal allocation is such that one bidder receives both items or  each bidder receives an item.

    We begin with the former case and without loss of generality assume that the optimal allocation awards both items to bidder $1$ (the case where bidder $2$ obtains both items is symmetric), i.e., $\opt = v_1(\{a,b\})$.   Since $\opt = v_1(\{a,b\})$, it must be that $v_1(a~|~b) \geq v_2(\{a\})$ and $v_1(b~|~a) \geq v_2(\{b\})$. 
    
    We now consider all possible outcomes of the randomized mechanism $\mathcal M_2$. If the mechanism $\mathcal M_2$ randomly allocates either item $a$ or $b$ to bidder $1$ our auction obtains welfare equal to $v_1(\{a,b\})$.  On the other hand, if we randomly allocate $a$ to bidder $2$ then our auction obtains welfare of $v_2(\{a\}) + \max\{v_2(b~|~a),v_1(\{b\})\} \geq v_1(\{b\})$ and, similarly, if we allocate $b$ to bidder $2$ then our auction obtains $v_2(\{b\}) + \max\{v_2(a~|~b),v_1(\{a\})\} \geq v_1(\{a\})$.  Combining these cases with the probabilities they occur gives that $\mathcal M_2$ achieves expected welfare at least $\frac{1}{2}\cdot v_1(\{a,b\}) + \frac{1}{4}\cdot v_1(\{a\}) + \frac{1}{4}\cdot v_1(\{b\})$.  But then, by subadditivity, we have that the auction obtains welfare at least $\frac{3}{4}\cdot v_1(\{a,b\})$.

    Now consider the latter case where the optimal allocation awards each bidder a single item and suppose that bidder $1$ receives $a$ and bidder $2$ receives $b$ (the opposite case is symmetric), i.e., $\opt = v_1(\{a\}) + v_2(\{b\})$.  We then have that $v_1(\{a\}) \geq v_2(\{a~|~b\})$ and $v_2(\{b\}) \geq v_1(\{b ~|~ a\})$.  This means that if $\mathcal M_2$  randomly allocates item $a$ to bidder $1$ or item $b$ to bidder $2$ our auction then obtains the optimal social welfare.  On the other hand, if we allocate $b$ to $1$ our auction obtains welfare $v_1(\{b\}) + \max\{v_1(a~|~b) ,v_2(\{a\})\} \geq v_1(\{a,b\}) \geq v_1(\{a\})$ by monotonicity and similarly if we allocate $a$ to $2$ our auction obtains welfare $v_2(\{a\} + \max\{v_2(b~|~a) ,v_1(\{b\})\} \geq v_2(\{a,b\}) \geq v_2(\{b\})$.  
    
    Combining these cases with the probabilities they occur implies that the expected welfare of $\mathcal M_2$ is at least $\frac{1}{2}\left(v_1(\{a\}) + v_2(\{b\})\right) + \frac{1}{4}v_1(\{a\}) + \frac{1}{4}v_2(\{b\}) = \frac{3}{4}\left(v_1(\{a\}) + v_2(\{b\})\right)$, as desired.
\end{proof}

\subsection{Combinatorial Auction with General Monotone Bidders}\label{subsec:22-ca-general}
Let $p$ be equal to $\frac{1}{3}$. 
Consider the mechanism $\mathcal M_3$ that with probability $p$ runs an ascending second-price auction on the grand bundle and with probability $(1-p)$ allocates a uniformly random agent a uniformly random item and runs an ascending second-price auction for the remaining item. 
\begin{claim}
    The mechanism $\mathcal{M}_3$ is universally OSP.  
\end{claim}
The proof is identical to the proof of Claim \ref{claim-mec-sm-mua-osp}. 
\begin{claim}
The mechanism $\mathcal{M}_3$ achieves a $\nicefrac 3 2$-approximation to the optimal social welfare.
\end{claim}
\begin{proof}
    Let $\{1,2\}$ denote the set of two bidders and $\{a,b\}$ denote the set of the two items. 
Throughout the proof, we slightly abuse notation by writing the value of a valuation $v$ for item $a$ as  $v(a)$ instead of $v_1(\{a\})$. 

The proof goes by a case  analysis.  First, we analyze the approximation guarantee of the mechanism if the optimal allocation allocates both items to the same agent, and then we analyze the case where the optimal allocation assigns a different item to each bidder. 

We begin with the first case:  assume that the optimal allocation assigns both $a$ and $b$ to the same bidder, say bidder $1$ without loss of generality. 
% The mechanism is anonymous so we can assume without loss of generality that bidder $1$ wins both items. 
Note that if we run ascending auction on the grand bundle, then by assumption it outputs the optimal allocation.  Now, consider the case where $\mathcal M_3$ randomly allocates bidder $1$ with item $a$, and then we runs an ascending auction on item $b$.  Note that:
$v_1(a,b) \ge v_1(a)+v_2(b)$, so $v_1(a ~|~ b)\ge v_2(b)$. Therefore, the mechanism obtains welfare of $v_1(a)+\max\{v_1(b~|~a)+v_2(b)\}\ge v_1(a,b)=\opt$, i.e., $\mathcal M_3$ obtains the optimal welfare. Due to the same reasons, $\mathcal M_3$ obtains the optimal welfare also for the case where $b$ is randomly allocated to bidder $1$. 
% Therefore, in this case bidder $1$ wins item $b$ and item $a$ (or, alternatively, bidder $1$ wins item $a$ and bidder $2$ wins item $b$ but $v_1(a) + v_2(b) = v_1(a,b)$) so we optimize the welfare.
%\shirinote{I feel like this is true up to tie breaking, although in case of a tie we are also good because then if bidder $2$ wins $b$ it just means that there are two optimal allocations and we necessarily output one of them. However, I'm not sure right now how to write it both formally and elegantly.} 
% For the same reasons exactly, if we randomly allocate item $b$ to bidder $1$, then she ends up winning item $a$ as well. 
Therefore, the expected welfare is  at least $\big(p+\frac{1-p}{2}\big)\cdot \opt$. 

Consider the complementary case where the optimal allocation assigns one item to each bidder, without loss of generality 
item $a$ to  $1$ and  $b$ to $2$. 
The expected welfare of $\mathcal M_6$ is:
\begin{align*}
    &p\cdot \underbrace{\max\{v_1(a,b),v_2(a,b)\}}_{\ge OPT/2} \quad&\text{(ascending auction on $\{a,b\}$)} &\\
    +&\frac{1-p}{4}\cdot\Big(v_1(a)+\max\big\{v_2(b),v_1(a,b)-v_1(a)\big\}\Big) &(1\gets a) \\ 
    +&\frac{1-p}{4}\cdot\Big(v_1(b)+
    \max\big\{v_2(a),v_1(a,b)-v_1(b)\big\} \Big) &(1\gets b) \\
    +&\frac{1-p}{4}\cdot\Big(v_2(a)+
    \max\big\{v_1(b),v_2(a,b)-v_2(a)\big\} \Big) &(2\gets a) \\
    +&\frac{1-p}{4}\cdot\Big(v_2(b)+
    \max\big\{v_1(a),v_2(a,b)-v_2(b)\big\} \Big) &(2\gets b)
\end{align*}
where $(1\gets a)$ signifies the case where bidder $1$ is randomly allocated with $a$ and an ascending auction is run on $b$, and $(1\gets b)$, $(2\gets a)$ and $(2\gets b)$ are defined analogously. 
Now, observe that since $\opt=v_1(a)+v_2(b)$, we have that: 
 $v_2(b)\ge v_1(b~|~a))$ and also $v_1(a)\ge v_2(a~|~b))$. Therefore, we obtain the optimal welfare in cases $(1\gets a)$ and $(2\gets b)$.  
 % if either bidder $1$ is randomly allocated with item $a$ or bidder $2$ is randomly allocated with item $b$. 

 The contribution of the remaining cases,
where the bidders are allocated the \textquote{wrong} items, i.e., $(1\gets b)$ and $(2\gets a)$, to the expected welfare is: 
 \begin{multline*}
     \frac{1-p}{4}\cdot\Big(v_1(b)+
    \max\big\{v_2(a),v_1(a~|~b)\big\}+ v_2(a)+
    \max\big\{v_1(b),v_2(b~|~a)\big\}\Big)   \\  
    \ge \frac{1-p}{4}\cdot\Big( v_1(a,b)+v_2(a,b)  \Big) \ge \frac{1-p}{4}\cdot\big(v_1(a)+v_2(b)\big)=\frac{1-p}{4}\cdot \opt
 \end{multline*}
Therefore, the expected welfare is $\frac{p\cdot \opt}{2}$ + $\frac{3(1-p)\cdot \opt}{4}$.  
Since $p=\frac{1}{3}$, in both cases the expected welfare is at least $\frac{2}{3}$ of the optimum, which completes the proof. 
\end{proof}
