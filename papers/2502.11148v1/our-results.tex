Our main results are upper bounds and lower bounds for randomized OSP mechanisms.
We focus our attention on ``universally'' OSP mechanisms, i.e., mechanisms which are a distribution over deterministic OSP mechanisms.
We analyze all the settings considered by 
 \cite{Ron24} and  show that:
\begin{enumerate}
    \item For additive bidders in a combinatorial auction, there is a mechanism that obtains a $4$ approximation and
    no mechanism has approximation better than $\frac{8}{7}\approx 1.14$  (\cref{thm-ub-add} and \cref{thm-lb-add}).
\item For unit-demand bidders in a combinatorial auction, 
there is a mechanism
    that obtains an $e\approx 2.72$ approximation and
no mechanism has approximation better than
than $\frac{8}{7}\approx 1.14$ (\cref{thm:ud-upper} and \cref{thm-lb-ud}). 
    \item For single-minded bidders in a multi-unit auction with unknown demands, 
    %\sre{In particular, our results hold for bidders which are unknown single-minded, meaning that both their demand and their value for their desired set is private information,},
    there is a mechanism that obtains a $400$ approximation and no mechanism has approximation better than $1.2$ (\cref{thm-ub-mua-sm} and \cref{thm-mua-sm-lb}). 
\end{enumerate}
All the impossibilities %presented
are for mechanisms that satisfy individual rationality and no negative transfers. Likewise,
%Correspondingly,
our proposed mechanisms
% all of the mechanisms that we propose 
conform to these conditions. 


Observe that our upper bounds demonstrate the power of randomization for obviously strategy-proof mechanism design:
whilst deterministic OSP mechanisms 
can only obtain an approximation of 
$\{m,n\}$ to the optimal welfare \cite{Ron24}, 
 % whilst \cite{Ron24} shows that only an approximation of $\{m,n\}$ can be obtained for deterministic obviously strategy-proof mechanisms, 
all these classes admit a randomized OSP mechanisms with a  constant factor approximation. 
%to the optimal welfare. 
In addition, we observe that the randomized $poly(m)$-communication mechanisms that are
dominant-strategy incentive compatible and obtain the state of the art approximation guarantees for \textquote{richer} classes of valuations in combinatorial auctions are in fact  obviously strategy-proof (see \cref{cl:subadditive,cl:general}).\footnote{We also provide a $400$ approximation to the optimal welfare for
multi-unit auctions with bidders whose valuations satisfy decreasing marginal utilities (\cref{thm:decreasing-marginals}). This is the only multi-parameter domain for which the power of deterministic mechanisms is not known. In \cref{subsub::non-mono}, 
we describe a non-monotonicity effect that illustrates a barrier towards proving impossibilities for this class.}
% provide additional results in 
% % the full version \sre{TODO}
% % full-version-change-tag%

% which illustrate the difficulty of proving lower bounds for this class.}  


{Our upper bounds are motivated by the following observation:
%regarding the lower bound 
the constructions of \cite{Ron24}
% is that 
show 
%They are based on the fact 
that
every  mechanism that provides a non-trivial approximation to the welfare satisfies that the first bidder that \textquote{speaks} in the mechanism does not have an obviously dominant strategy. The underlying cause of this  phenomenon is that when querying a bidder for the first time, the mechanism fails because it has no information regarding the valuations of the other bidders. 
% does not have 
% does not have obviously dominant strategies. The underlying cause of this  phenomenon is that when querying a bidder for the first time, the mechanism has no information regarding the valuations of the other bidders. Thus, this bidder has to 
Thus, to overcome this impossibility, our proposed mechanisms are based on the classic secretary approach of sampling a sufficient fraction of the bidders and aggregating their information to determine
a price per item. Owing to the use of randomization, this can be done 
in an obviously dominant manner while maintaining a high fraction of the welfare in expectation.}  

% This motivates the sampling phase, which allows us to 

% The sampling phase, which
% can be seen as randomizing the identity of that bidder, 
% we use in all of our algorithms, is key since 
% approach, which we use in our 
% We overcome this impossibility by essentialy randomize the identity of that player, 
% is pivotal, since by  by randomizing who that first bidder is, we manage to extract high welfare.}

Our lower bounds for combinatorial auctions with unit-demand and additive bidders further emphasize 
the restrictiveness of obvious strategy-proofness compared to implementation in dominant strategies. 
Not only getting an approximation better than $\min\{m,n\}$ is impossible deterministically, but even if we allow randomization we cannot get more than $87.5\%$ of the optimal welfare. In contrast, these settings  have  dominant-strategy mechanisms that extract the optimal welfare and are also efficient both from a computation and a communication perspective.
One disadvantage of our main results is that the lower bounds and upper bounds that we provide are quite far apart. As a step to bridge this gap, we show in %full-version-change-tag%
% the full version
\cref{sec-22} 
that for two bidders and two items, all the aforementioned classes
admit mechanisms that give a $\frac{4}{3}$ approximation.  
% We leave open the question of resolving the approximation power of randomized OSP mechanisms with arbitrary number of items and bidders. 
 % We conclude by showing a separation between obviously strategy-proof mechanisms and weakly group strategy-proof mechanisms, which can be found in the full version.
%full-version-change-tag%
% (\cref{app:wgsp}). 
 % We also provide a randomized OSP mechanism that gives a $400$ approximation to a multi-unit auction with decreasing marginal utilities (\cref{app:wgsp}). 