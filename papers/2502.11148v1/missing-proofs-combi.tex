\subsection{Proof of Lemma \ref{lemma-add-osp}
% An Upper Bound For Additive Bidders
} \label{subsec::proof-add-osp}
Fix any realization of the coin flips of the mechanism.  Observe that ``sampled'' bidders receive $0$ utility regardless of their report so
reporting their valuations trtuhfully is obviously dominant for them. 
% the mechanism is OSP for them.  
Further the ``unsampled'' bidders purchase all available items for which they have  positive utility and purchase no items for which they have negative utility when being truthful, so for them as well truthfulness is an obviously dominant strategy. 
% the mechanism is clearly OSP for them as well.    
%We consider separately the items which have one uniquely highest value bidder and the items which have multiple highest value bidders.  

%First consider an arbitrary item $j$ with a uniquely highest value bidder and denote this bidder $\text{OPT}(j)$.  Our mechanism certainly allocates item $j$ to $\text{OPT}(j)$ if some second highest value bidder for item $j$ is in the sample, $\text{OPT}(j)$ is not in the sample, and if the mechanism instructs bidders to buy only items for which they have strictly positive utility (i.e., unsampled bidders select the minimum sized set in their demand set).  Since all three of these events are independent and occur with probabilities $1/2$, $1/2$, and $2/3$, respectively, our mechanism obtains an $6$-approximation to the welfare component from among items with a uniquely highest value bidder.

%Now consider an arbitrary item $j$ which has at least $2$ bidders tied for the highest value and denote these bidders $i$ and $i'$.  Our mechanism certainly allocates item $j$ to some bidder with value $v_{ij}$ if one of $i$ or $i'$ is sampled and the other is not sampled and if the mechanism instructs bidders to buy all items for which they have non-negative utility (i.e., unsampled bidders select the maximum sized set in their demand set).  Since these events are independent and occur with probabilities $1/2$ and $1/3$, respectively, our mechanism obtains a $6$-approximation to the welfare component from among items which do not have a uniquely highest value bidder.

% \subsection{Using Mechanism \ref{alg:additive} On Unit-Demand Bidders}


% \subsection{Proof of \cref{thm:ud-upper}: A Mechanism for Unit-Demand Bidders}





\subsection{Proof of Theorem \ref{thm-lb-ud}: An Impossibility for Unit-Demand Bidders} \label{lb-ud-proof-place}
% The proof follows the same structure as the proof of \cref{thm-mua-sm-lb} of describing 
% a distribution $\mathcal D$ and showing that it is hard for every deterministic mechanism. By applying Yao's lemma (\cref{lem:yaos}), we get hardness for randomized obviously strategy-proof mechanisms. 
% Similarly, we prove for the case of two bidders and two items, but the proof extends to any number of bidders and any number of items by adding bidders with the all-zero valuation and assuming that the bidders in our construction have zero values for the additional items. However, the case analysis we employ in this proof is significantly more involved.   

The proof follows the same structure as in \cref{thm-mua-sm-lb}: roughly speaking, we describe a distribution $\mathcal{D}$, show that it is “hard” for deterministic mechanisms, and then use Yao's Lemma to deduce hardness for randomized mechanisms.


The proof outline is as follows. In \cref{subsubsec-description-ud}, 
we describe the  hard distribution $\mathcal D$. 
In \cref{subsubsec-performance-ud} we analyze the allocation and payments of a deterministic \textquote{good} mechanism 
% in several scenarios, 
and explain why these properties  imply that  a \textquote{good} mechanism does not exist in  \cref{subsubsec-contradiction-ud}.
We defer technical proofs to \cref{app-claims-proofs-ud,subsubsec-prop-alloc-ud}. 

We prove for the case of two bidders and two items, but the proof extends to any number of bidders and any number of items by adding bidders with the all-zero valuation and assuming that the bidders in our construction have zero values for the additional items.


\subsubsection[Construction of a ``Hard'' Distribution D]{Construction of a ``Hard'' Distribution $\mathcal{D}$}\label{subsubsec-description-ud}
    To define the probability distribution over the valuation profiles, we name the two items $a$ and $b$, and let $k$ be an arbitrarily large number. Note that since these valuations are unit-demand, we can fully describe them by specifying their value per item.
    Consider the following valuations:  
\[
% \forall S\subseteq M, \quad
v_1^{{a,one}}(x) = 
\begin{cases}
1 & x=a,\\
0 & \text{otherwise.}
\end{cases} \quad 
v_2^{{b,one}}(x) = 
\begin{cases}
1 & x=b,\\
0 & \text{otherwise.}
\end{cases}
\]

\[
% \forall S\subseteq M, \quad
v_1^{{a,mid}}(x) = 
\begin{cases}
k^2 & x=a,\\
0 & \text{otherwise.}
\end{cases} \quad 
v_2^{{b,mid}}(x) = 
\begin{cases}
k^2 & x=b,\\
0 & \text{otherwise.}
\end{cases}
\]

\[
% \forall S\subseteq M, \quad
v_1^{{a,large}}(x) = 
\begin{cases}
k^4 & x=a,\\
0 & \text{otherwise.}
\end{cases} \quad 
v_2^{{b,large}}(x) = 
\begin{cases}
k^4 & x=b,\\
0 & \text{otherwise.}
\end{cases}
\]

\[
% \forall S\subseteq M, \quad
v_1^{{b,small}}(x) = 
\begin{cases}
k & x=b,\\
0 & \text{otherwise.}
\end{cases} \quad 
v_2^{{a,small}}(x) = 
\begin{cases}
k & x=a,\\
0 & \text{otherwise.}
\end{cases}
\]

\[
% \forall S\subseteq M, \quad
v_1^{{b,mid}}(x) = 
\begin{cases}
k^2 & x=b,\\
0 & \text{otherwise.}
\end{cases} \quad 
v_2^{{a,mid}}(x) = 
\begin{cases}
k^2 & x=a,\\
0 & \text{otherwise.}
\end{cases}
\]

\[
% \forall S\subseteq M, \quad
v_1^{{b,large}}(x) = 
\begin{cases}
k^4 & x=b,\\
0 & \text{otherwise.}
\end{cases} \quad 
v_2^{{a,large}}(x) = 
\begin{cases}
k^4 & x=a,\\
0 & \text{otherwise.}
\end{cases}
\]

\[
% \forall S\subseteq M, \quad
v_1^{{both}}(x) = 
\begin{cases}
2k+3 & x=a,\\
2k+1 & x=b,\\
0 & \text{otherwise.}
\end{cases} \quad 
v_2^{{both}}(x) = 
\begin{cases}
2k+3 & x=b,\\
2k+1 & x=a,\\
0 & \text{otherwise.}
\end{cases}
\]


Consider the following valuation profiles: 
\begin{align*}
 I_1 = (v_1^{{a,one}}&, v_2^{{b,one}}),\quad  I_2 = (v_1^{a,mid}, v_2^{{a,large}}),  \quad I_3 = (v_1^{b,large}, v_2^{b,mid}), \quad 
 I_4 = (v_1^{b,mid}, v_2^{b,large}) \\ &I_5 = (v_1^{a,large}, v_2^{a,mid}) \quad I_6 = (v_1^{b,small}, v_2^{b,one}) \quad I_7 =(v_1^{a,one}, v_2^{a,small}) 
\end{align*}
% $I_1 = (v_1^{{a,one}}, v_2^{{b,one}})$, 
% $I_2 = (v_1^{a,mid}, v_2^{{a,large}})$,
% $I_3 = (v_1^{b,large}, v_2^{b,mid})$,
% $I_4 = (v_1^{b,mid}, v_2^{b,large})$,
% $I_5 = (v_1^{a,large}, v_2^{a,mid})$,
% $I_6 = (v_1^{b,small}, v_2^{b,one})$
% and $I_7 =(v_1^{a,one}, v_2^{a,small})$.
Let $\mathcal D$ be the distribution over valuation profiles where the probability of the valuation profile $I_1$ is $\frac{1}{4}$, and the probability of all the valuation profiles $I_2,I_3,I_4,I_5,I_6$ and $I_7$ is $\frac{1}{8}$ each. 

\subsubsection[The Performance of the Deterministic Mechanism A on the "Hard" Distribution D]{The Performance of the Deterministic Mechanism $A$ on the ``Hard'' Distribution $\mathcal{D}$}\label{subsubsec-performance-ud}
We remind that our goal is to show that no deterministic mechanism that satisfies all of the desired properties extracts more than $\frac{7}{8}$ of the optimal welfare. For that, we begin by observing that:
\begin{lemma}\label{lemma-ud-instances}
Every deterministic mechanism that has approximation better than $\frac{7}{8}$ necessarily satisfies all of the following conditions:
\begin{enumerate}
    \item Given the valuation profile $I_1 = (v_1^{\text{a,one}}, v_2^{\text{b,one}})$, the mechanism
    allocates item $a$ to player $1$ and item $b$ to player $2$. \label{condi-1-ud}
    % $A$ outputs an allocation with welfare at most $1$. 
\item Given the valuation profile $I_2 = (v_1^{a,mid}, v_2^{a,large})$, the mechanism allocates item $a$ to bidder $2$. \label{condi-2-ud}

    \item Given the valuation profile $I_3 = (v_1^{b,large}, v_2^{b,mid})$, the mechanism allocates item $b$  to bidder $1$.
    \label{condi-3-ud}
\item Given the valuation profile $I_4 = (v_1^{b,mid}, v_2^{b,large})$, the mechanism allocates item $b$ to bidder $2$. \label{condi-4-ud}
\item Given the valuation profile $I_5 = (v_1^{a,large}, v_2^{a,mid})$, the mechanism allocates item $a$ to bidder $1$. \label{condi-5-ud}
\item Given the valuation profile $I_6 = (v_1^{b,small}, v_2^{b,one})$, the mechanism allocates item $b$ to bidder $1$. \label{condi-6-ud}
\item Given the valuation profile $I_7 = (v_1^{a,one}, v_2^{a,small})$, the mechanism allocates item $a$ to bidder $2$. 
\end{enumerate}
\end{lemma}
The proof of \cref{lemma-ud-instances} is straightforward: if a deterministic mechanism does not satisfy one of the conditions, then due to the fact that $k$ is arbitrarily large, its approximation guarantee is at most $\frac{7}{8}$ with respect to the distribution $\mathcal{D}$.

\subsubsection[Reaching a Contradiction: No Deterministic and OSP Mechanism Succeeds on D]{Reaching a Contradiction: No Deterministic and OSP Mechanism Succeeds on $\mathcal D$}
\label{subsubsec-contradiction-ud}
We now employ \cref{lemma-ud-instances} to prove  \cref{thm-lb-ud}. Let the domain $V_i$ of each bidder consist 
of unit-demand valuations with values in  $\{0,1,\ldots,k^4\}$, where $k$ is an arbitrarily large integer. 


Fix a deterministic mechanism $A$ and strategies
$(\mathcal S_1,\mathcal S_2)$ that are individually rational, satisfy no negative transfers for all the valuations  $V_1\times V_2$ and give approximation better than $\frac{7}{8}$
in expectation over the valuation profiles in the distribution $\mathcal D$. 
Denote with  $(f,P_1,P_2)$ be the allocation rule and the payment scheme that $A$ and $(\mathcal S_1,\mathcal S_2)$ realize, and assume towards a contradiction that $A$ and $(\mathcal S_1,\mathcal S_2)$ are obviously strategy-proof. 


For the analysis of the deterministic mechanism $A$, we focus on the following subsets of the domains of the valuations:
$
\mathcal{V}_1=\{v_1^{a,one},v_1^{a,mid},v_1^{a,large},v_1^{b,small},v_1^{b,mid},v_1^{b,large},v_1^{both}\}$ and  $\mathcal V_2=\{v_2^{b,one},v_2^{b,mid},v_2^{b,large},\allowbreak v_2^{a,small},v_2^{a,mid}, \allowbreak v_2^{a,large},v_2^{both}\}$.
% We now focus on $\mathcal{V}_1\times \mathcal{V}_2$. 
Observe that there necessarily exists a vertex $u$, and valuations $v_1,v_1' \in \mathcal{V}_1$, and  $v_2,v_2' \in \mathcal{V}_2$ such that $(\mathcal{S}_1(v_1), \mathcal{S}_2(v_2))$ diverge at vertex $u$. This follows from \cref{lemma-ud-instances}, which implies that the mechanism $A$ must output different allocations for different valuation profiles in $\mathcal{V}_1 \times \mathcal{V}_2$. Consequently, not all valuation profiles end up in the same leaf, meaning that divergence must occur at some point.

Let $u$ be the first vertex in the protocol such that 
 %the behavior profiles 
 $(\mathcal{S}_1(v_1),\mathcal{S}_2(v_2))$ and $(\mathcal{S}_1(v_1'),\mathcal{S}_2(v_2'))$ diverge, i.e., dictate different messages. 
Note that by definition this implies that $u\in Path(\mathcal{S}_1(v_1),\mathcal{S}_2(v_2))\cap Path(\mathcal{S}_1(v_1'),\mathcal{S}_2(v_2'))$ and that either bidder $1$ or bidder $2$ sends different messages for the valuations in $\mathcal{V}_1$ or $\mathcal V_2$, respectively. 
Without loss of generality, we assume that bidder $1$ sends different messages, meaning that there exist $v_1,v_1'\in \mathcal{V}_1$ such that $\mathcal S_1(v_1)$ and $\mathcal S_1(v_1')$ dictate different messages at vertex $u$.
\begin{comment}
Observe that either bidder $1$ or $2$ has to send different messages for different valuations in $\mathcal V_i$ at some vertex. 
This is an immediate implication of \cref{lemma-ud-instances}. The reason for it is that since 
the mechanism $A$ outputs different allocations for different valuation profiles in $\mathcal V_1\times \mathcal V_2$, 
these valuation profiles end up in different leaves, so they necessarily diverge at some vertex.
% given the valuation profiles $I_1=(v_1^{one},v_2^{one})$ and
%  $I_2=(v_1^{all},v_2^{one})$, meaning that the behavior profiles $(\mathcal S_1(v_1^{one},\mathcal S_2(v_2^{one}))$ and $(\mathcal S_1(v_1^{all},\mathcal S_2(v_2^{one}))$ reach different leaves, so they have to diverge at some vertex. 
 
 Let $u$ be the first vertex in the protocol of the mechanism $A$ where they diverge, meaning that  the behavior profiles $(\mathcal{S}_1(v_1),\mathcal{S}_2(v_2))$ and $(\mathcal{S}_1(v_1'),\mathcal{S}_2(v_2'))$ dictate different messages. 
Note that by definition $u\in Path(\mathcal{S}_1(v_1),\mathcal{S}_2(v_2))\cap Path(\mathcal{S}_1(v_1'),\mathcal{S}_2(v_2'))$. We remind that each vertex is associated with only one player that sends messages in it. Note that the distribution $\mathcal D$ we have defined is symmetric, so we can assume without loss of generality that player $1$ is the player that sends a message in vertex $u$. Thus, there exist $v_1,v_1'\in \mathcal{V}_1$ such that $\mathcal S_1(v_1)$ and $\mathcal S_1(v_1')$ dictate different messages at vertex $u$. 
\end{comment}
However, the following collection of claims show that since the strategy $\mathcal S_1$ is obviously dominant, it dictates the same message for all the valuations in $\mathcal{V}_1$. Thus, we get a contradiction, which completes the proof of \cref{thm-lb-ud}:
% We remind that $\mathcal{V}_1=\{v_1^{a,one},v_1^{a,mid},v_1^{a,large},v_1^{b,small},v_1^{b,mid},v_1^{b,large},v_1^{both}\}$, so
% % Thus,
% the  following claims jointly imply a contradiction, completing the proof of \cref{thm-lb-ud}:
\begin{claim}\label{claim-a-one-mid}
    The strategy $\mathcal S_1$ dictates the same message at vertex $u$ for the valuations $v_1^{a,one}$ and $v_1^{a,mid}$. 
\end{claim}
\begin{claim}\label{claim-a-mid-large}
    The strategy $\mathcal S_1$ dictates the same message at vertex $u$ for the valuations $v_1^{a,mid}$ and $v_1^{a,large}$. 
\end{claim}
\begin{claim}\label{claim-b-small-large}
    The strategy $\mathcal S_1$ dictates the same message at vertex $u$ for the valuations $v_1^{b,small}$ and $v_1^{b,mid}$. 
\end{claim}
\begin{claim}\label{claim-b-mid-large}
    The strategy $\mathcal S_1$ dictates the same message at vertex $u$ for the valuations $v_1^{b,mid}$ and $v_1^{b,large}$. 
\end{claim}
\begin{claim}\label{claim-a-both}
    The strategy $\mathcal S_1$ dictates the same message at vertex $u$ for the valuations $v_1^{both}$ and $v_1^{a,mid}$. 
\end{claim}
\begin{claim}\label{claim-b-both}
    The strategy $\mathcal S_1$ dictates the same message at vertex $u$ for the valuations $v_1^{both}$ and $v_1^{b,mid}$. 
\end{claim}
To prove these claims, we use the following collection of observations about the allocation and the payment scheme of player $1$:    
\begin{lemma}\label{lemma-small-pay-ud}
    The allocation rule $f$ and the payment scheme $P_1$ of bidder $1$ satisfy that:
    % Let $f$ be an allocation rule and let $P_1$ be the payment scheme of bidder $1$ that 
    % The allocation rule $f$ and the  payment scheme $P_1$ $(f,P_1,\ldots,P_n):V_1\times \cdots \times V_n\to \mathbb{R}^{n}$
    % are realized by a dominant-strategy, individually rational and no negative transfers mechanism. Then: 
    \begin{enumerate}
        \item Given the valuation profiles $(v_1^{a,one},v_{2}^{b,one})$ 
        % $(v_1^{a,mid},v_{2}^{b,one})$ 
        and $(v_1^{a,large},v_{2}^{b,one})$, bidder $1$ wins item $a$ and pays at most $1$.  \label{item-1-ud}
        \item  Given the valuation profile $(v_1^{a,mid},v_2^{a,large})$, bidder $1$ wins a bundle that does not contain item $a$ and pays zero.   \label{item-2-ud}
    %     \item  Given $(v_1^{a,large},v_2^{a,large})$, bidder $1$ either: 
    %     \begin{enumerate*}[label=(\alph*)]
    % \item gets a bundle not containing item  $a$ and pays zero \emph{or}
    % \item gets a bundle that contains item $a$ and pays at least $k^2$.
    % \end{enumerate*}
    %     \label{item-3-ud}
        \item Given the valuation profiles $(v_1^{b,small},v_2^{b,one})$,
        $(v_1^{b,mid},v_2^{b,one})$ and $(v_1^{b,large},v_2^{b,one})$,  
        bidder $1$ wins item $b$ and pays at most $k$. 
        \label{item-4-ud}
           \item  Given the valuation profile $(v_1^{b,mid},v_2^{b,large})$, bidder $1$ wins a bundle that does not contain item $b$ and pays zero.   \label{item-4.5-ud}
    % \item  Given $(v_1^{b,large},v_2^{b,large})$, bidder $1$ either: 
    %     \begin{enumerate*}[label=(\alph*)]
    % \item gets a bundle not containing item  $b$ and pays zero \emph{or}
    % \item gets a bundle that contains item $b$ and pays at least $k^2$.
    % \end{enumerate*}
    %     \label{item-5-ud}
\item Given the valuation profile $(v_1^{both},v_2^{b,one})$, 
bidder $1$ gets a bundle that contains item $a$ and pays at most $1$. \label{item-6-ud}
\item Given the valuation profile $(v_1^{both},v_2^{a,large})$, bidder $1$ does not win item $a$.  If bidder $1$ wins item $b$, then he pays at most $2k+1$. If he wins a bundle that contains neither item $a$ or item $b$, then he pays zero. \label{item-complicated-ud}
% \item If the allocation rule $f(v_1^{both},v_2^{a,large})$
%     outputs an allocation where bidder $1$ wins a bundle containing item $b$, then he pays at most $2k+1$. \label{item-7-ud}
% \item If the allocation rule $f(v_1^{both},v_2^{a,large})$
%     outputs an allocation where bidder $1$ wins a bundle not containing item $b$, then he pays zero. \label{item-8-ud}
    \end{enumerate}
\end{lemma}
\subsubsection[All Valuations in V\_1 Send The Same Message: Proofs of Claims \ref{claim-a-one-mid} to \ref{claim-b-both}]{All Valuations in $\mathcal V_1$ Send The Same Message: Proofs of Claims \ref{claim-a-one-mid} to \ref{claim-b-both}}\label{app-claims-proofs-ud}
We now prove Claims \ref{claim-a-one-mid} to \ref{claim-b-both}, which jointly imply a contradiction. All proofs make extensive use of \cref{lemma-ud-instances}, which analyzes the allocation and the payments of any deterministic mechanisms with the desired properties. All proofs are quite similar to each other, and we write them for completeness. The only claim that requires a more involved case analysis is \cref{claim-b-both}.

\begin{proof}[Proof of \cref{claim-a-one-mid}]
      Note that by Lemma \ref{lemma-small-pay-ud} part \ref{item-1-ud}, $f(v_1^{a,one},v_2^{b,one})$ allocates item $a$ to bidder $1$ and $P_1(v_1^{a,one},v_2^{b,one})\le 1$. Therefore:
\begin{equation}\label{eq-good-leaf1-ud}
 v_1^{a,mid}(f(v_1^{a,one},v_2^{b,one}))-P_1(v_1^{a,one},v_2^{b,one})\ge k^2-1   
\end{equation}
 In contrast, by part \ref{item-2-ud} of Lemma \ref{lemma-small-pay-ud}, the allocation rule  $f(v_1^{a,mid},v_2^{a,large})$ allocates no items to player $1$ and $P_1(v_1^{a,mid},v_2^{a,large})=0$, so:
 \begin{equation}\label{eq-bad-leaf1-ud}
 v_1^{a,mid}(f(v_1^{a,mid},v_2^{a,large}))-P_1(v_1^{a,mid},v_2^{a,large})= 0   
\end{equation}
We remind that $k$ is arbitrarily large,
so combining inequalities (\ref{eq-good-leaf1-ud}) and (\ref{eq-bad-leaf1-ud}) gives:
\begin{equation*}
 v_1^{a,mid}(f(v_1^{a,mid},v_2^{a,large}))-P_1(v_1^{a,mid},v_2^{a,large})< 
v_1^{a,mid}(f(v_1^{a,one},v_2^{b,one}))-P_1(v_1^{a,one},v_2^{b,one})  
\end{equation*}
We remind that vertex $u$ belongs in $Path(\mathcal S_1(v_1^{a,one}),\mathcal S_2(v_2^{b,one}))$ and also in
$Path(\mathcal{S}_1(v_1^{a,mid}),\allowbreak\mathcal{S}_2(v_2^{a,large}))$. Therefore, Lemma \ref{lemma-bad-leaf-good-leaf} gives that the strategy $\mathcal S_1$ dictates the same message for  $v_1^{a,one}$ and $v_1^{a,mid}$ at vertex $u$.
\end{proof}

\begin{proof}[Proof of \cref{claim-a-mid-large}]
     By Lemma \ref{lemma-small-pay-ud} part \ref{item-1-ud}, $f(v_1^{a,large},v_2^{b,one})$ allocates item $a$ to bidder $1$ and $P_1(v_1^{a,large},\allowbreak v_2^{b,one})\le 1$. Therefore:
\begin{equation}\label{eq-good-leaf2-ud}
 v_1^{a,mid}(f(v_1^{a,large},v_2^{b,one}))-P_1(v_1^{a,large},v_2^{b,one})\ge k^2-1   
\end{equation}
Whereas by part \ref{item-2-ud} of Lemma \ref{lemma-small-pay-ud}: 
% player $1$ either does not win item $a$ or pays at least $k^2$, so:
 \begin{equation}\label{eq-bad-leaf2-ud}
 v_1^{a,mid}(f(v_1^{a,mid},v_2^{a,large}))-P_1(v_1^{a,mid},v_2^{a,large})=0   
\end{equation}
Combining the inequalities (\ref{eq-good-leaf2-ud}) and (\ref{eq-bad-leaf2-ud}) gives:
\begin{equation*}
 v_1^{a,mid}(f(v_1^{a,mid},v_2^{a,large}))-P_1(v_1^{a,mid},v_2^{a,large})< 
v_1^{a,mid}(f(v_1^{a,large},v_2^{b,one}))-P_1(v_1^{a,large},v_2^{b,one})
\end{equation*}
Note that vertex $u$ belongs in $Path(\mathcal S_1(v_1^{a,large}),\mathcal S_2(v_2^{b,one}))$ and also in
$Path(\mathcal{S}_1(v_1^{a,mid}),\allowbreak\mathcal{S}_2(v_2^{a,large}))$. Therefore, Lemma \ref{lemma-bad-leaf-good-leaf} gives that the strategy $\mathcal S_1$ dictates the same message for  $v_1^{a,mid}$ and $v_1^{a,large}$ at vertex $u$.
\end{proof}
      

\begin{proof}[Proof of \cref{claim-b-small-large}]
        By Lemma \ref{lemma-small-pay-ud} part \ref{item-4-ud}, $f(v_1^{b,small},v_2^{b,one})$ allocates item $b$ to bidder $1$ and $P_1(v_1^{b,small},\allowbreak v_2^{b,one})\le k$. Therefore:
\begin{equation}\label{eq-good-leaf3-ud}
 v_1^{b,mid}(f(v_1^{b,small},v_2^{b,one}))-P_1(v_1^{b,small},v_2^{b,one})\ge k^2-k   
\end{equation}
Whereas by part \ref{item-4.5-ud} of Lemma \ref{lemma-small-pay-ud}: 
% player $1$ either does not win item $a$ or pays at least $k^2$, so:
 \begin{equation}\label{eq-bad-leaf3-ud}
 v_1^{b,mid}(f(v_1^{b,mid},v_2^{b,large}))-P_1(v_1^{b,mid},v_2^{b,large})= 0   
\end{equation}
Combining inequalities (\ref{eq-good-leaf3-ud}) and (\ref{eq-bad-leaf3-ud}) gives:
\begin{equation*}
 v_1^{b,mid}(f(v_1^{b,mid},v_2^{b,large}))-P_1(v_1^{b,mid},v_2^{b,large})< 
v_1^{b,mid}(f(v_1^{b,small},v_2^{b,one}))-P_1(v_1^{b,small},v_2^{b,one})
\end{equation*}
Note that vertex $u$ belongs in $Path(\mathcal S_1(v_1^{b,small}),\mathcal S_2(v_2^{b,one}))$ and also in
$Path(\mathcal{S}_1(v_1^{b,mid}),\allowbreak\mathcal{S}_2(v_2^{b,large}))$. Therefore, Lemma \ref{lemma-bad-leaf-good-leaf} gives that the strategy $\mathcal S_1$ dictates the same message for  $v_1^{b,mid}$ and $v_1^{b,small}$ at vertex $u$.
\end{proof}
\begin{proof}[Proof of \cref{claim-b-mid-large}]
          By Lemma \ref{lemma-small-pay-ud} part \ref{item-4-ud}, $f(v_1^{b,large},v_2^{b,one})$ allocates item $b$ to bidder $1$ and $P_1(v_1^{b,large},\allowbreak v_2^{b,one})\le k$. Therefore:
\begin{equation}\label{eq-good-leaf4-ud}
 v_1^{b,mid}(f(v_1^{b,large},v_2^{b,one}))-P_1(v_1^{b,large},v_2^{b,one})\ge k^2-k   
\end{equation}
Whereas by part \ref{item-4.5-ud} of Lemma \ref{lemma-small-pay-ud}: 
% player $1$ either does not win item $a$ or pays at least $k^2$, so:
 \begin{equation}\label{eq-bad-leaf4-ud}
 v_1^{b,mid}(f(v_1^{b,mid},v_2^{b,large}))-P_1(v_1^{b,mid},v_2^{b,large})= 0   
\end{equation}
Combining inequalities (\ref{eq-good-leaf4-ud}) and (\ref{eq-bad-leaf4-ud}) gives:
\begin{equation*}
 v_1^{b,mid}(f(v_1^{b,mid},v_2^{b,large}))-P_1(v_1^{b,mid},v_2^{b,large})< 
v_1^{b,mid}(f(v_1^{b,large},v_2^{b,one}))-P_1(v_1^{b,large},v_2^{b,one})
\end{equation*}
Note that vertex $u$ belongs in $Path(\mathcal S_1(v_1^{b,large}),\mathcal S_2(v_2^{b,one}))$ and also in
$Path(\mathcal{S}_1(v_1^{b,mid}),\allowbreak\mathcal{S}_2(v_2^{b,large}))$. Therefore, Lemma \ref{lemma-bad-leaf-good-leaf} gives that the strategy $\mathcal S_1$ dictates the same message for  $v_1^{b,mid}$ and $v_1^{b,large}$ at vertex $u$.
\end{proof}
\begin{proof}[Proof of \cref{claim-a-both}]
            By Lemma \ref{lemma-small-pay-ud} part \ref{item-6-ud}, $f(v_1^{both},v_2^{b,one})$ allocates item $a$ to bidder $1$ and $P_1(v_1^{both},\allowbreak v_2^{b,one})\le 1$. Therefore:
\begin{equation}\label{eq-good-leaf5-ud}
 v_1^{a,mid}(f(v_1^{both},v_2^{b,one}))-P_1(v_1^{both},v_2^{b,one})\ge k^2-1   
\end{equation}
Whereas by part \ref{item-2-ud} of Lemma \ref{lemma-small-pay-ud}: 
% player $1$ either does not win item $a$ or pays at least $k^2$, so:
 \begin{equation}\label{eq-bad-leaf5-ud}
 v_1^{a,mid}(f(v_1^{a,mid},v_2^{a,large}))-P_1(v_1^{a,mid},v_2^{a,large})=0   
\end{equation}
Combining the inequalities (\ref{eq-good-leaf5-ud}) and (\ref{eq-bad-leaf5-ud}) gives:
\begin{equation*}
 v_1^{a,mid}(f(v_1^{a,mid},v_2^{a,large}))-P_1(v_1^{a,mid},v_2^{a,large})< 
 v_1^{a,mid}(f(v_1^{both},v_2^{b,one}))-P_1(v_1^{both},v_2^{b,one})
\end{equation*}
Note that vertex $u$ belongs in $Path(\mathcal S_1(v_1^{both}),\mathcal S_2(v_2^{b,one}))$ and also in
$Path(\mathcal{S}_1(v_1^{a,mid}),\allowbreak\mathcal{S}_2(v_2^{a,large}))$. Therefore, Lemma \ref{lemma-bad-leaf-good-leaf} gives that the strategy $\mathcal S_1$ dictates the same message for  $v_1^{a,mid}$ and $v_1^{both}$ at vertex $u$.
\end{proof}
\begin{proof}[Proof of \cref{claim-b-both}]
    % To prove that the strategy $\mathcal S_1$ dictates the same message for $v_1^{both}$ and $v_1^{b,mid}$, {we consider at the following cases.} 
    Note that by \cref{lemma-small-pay-ud} part \ref{item-complicated-ud}, given the valuation profile $(v_1^{both},v_2^{a,large})$, bidder $1$ cannot win item $a$ so we consider the following two cases: the case where he wins item $b$ and the case where he wins neither of these items.

    Assume that bidder $1$ wins item $b$ given the valuation profile $(v_1^{both},v_2^{a,large})$. Note that in this case, \cref{lemma-small-pay-ud} part \ref{item-complicated-ud} also implies that:
    \begin{equation}\label{eq-good-leaf6-ud}
 v_1^{b,mid}(f(v_1^{both},v_2^{a,large}))-P_1(v_1^{both},v_2^{a,large})\ge k^2 -2k-1   
\end{equation}
Whereas by part \ref{item-4.5-ud} of Lemma \ref{lemma-small-pay-ud}: 
% player $1$ either does not win item $a$ or pays at least $k^2$, so:
 \begin{equation}\label{eq-bad-leaf6-ud}
 v_1^{b,mid}(f(v_1^{b,mid},v_2^{b,large}))-P_1(v_1^{b,mid},v_2^{b,large})= 0   
\end{equation}
Combining the inequalities (\ref{eq-good-leaf6-ud}) and (\ref{eq-bad-leaf6-ud}) gives:
\begin{equation*}
 v_1^{b,mid}(f(v_1^{b,mid},v_2^{b,large}))-P_1(v_1^{b,mid},v_2^{b,large})< 
 v_1^{b,mid}(f(v_1^{both},v_2^{a,large}))-P_1(v_1^{both},v_2^{a,large})
\end{equation*}
Note that vertex $u$ belongs in $Path(\mathcal S_1(v_1^{both}),\mathcal S_2(v_2^{a,large}))$ and also in
$Path(\mathcal{S}_1(v_1^{a,mid}),\allowbreak\mathcal{S}_2(v_2^{a,large}))$. Therefore, Lemma \ref{lemma-bad-leaf-good-leaf} gives that the strategy $\mathcal S_1$ dictates the same message for  $v_1^{both}$ and $v_1^{b,mid}$ at vertex $u$, which concludes this case.

For the latter case, where bidder $1$ gets a bundle that contains neither item $a$ nor item $b$ given the valuation profile $(v_1^{both},v_2^{a,large})$, by \cref{lemma-small-pay-ud} part \ref{item-complicated-ud}: 
\begin{equation}\label{eq-bad-leaf7-ud}
 v_1^{both}(f(v_1^{both},v_2^{a,large}))-P_1(v_1^{both},v_2^{a,large})=0   
\end{equation}
Whereas by part \ref{item-4-ud} of Lemma \ref{lemma-small-pay-ud}: 
% player $1$ either does not win item $a$ or pays at least $k^2$, so:
 \begin{equation}\label{eq-good-leaf7-ud}
 v_1^{both}(f(v_1^{b,mid},v_2^{b,one}))-P_1(v_1^{b,mid},v_2^{b,one}) \ge 2k+1-k=k+1   
\end{equation}
Combining (\ref{eq-bad-leaf7-ud}) and  (\ref{eq-good-leaf7-ud})  gives:
\begin{equation*}
 v_1^{both}(f(v_1^{both},v_2^{a,large}))-P_1(v_1^{both},v_2^{a,large})< 
 v_1^{both}(f(v_1^{b,mid},v_2^{b,one}))-P_1(v_1^{b,mid},v_2^{b,one})
\end{equation*}
Note that vertex $u$ belongs in $Path(\mathcal S_1(v_1^{both}),\mathcal S_2(v_2^{a,large}))$ and also in
$Path(\mathcal{S}_1(v_1^{b,mid}),\allowbreak\mathcal{S}_2(v_2^{b,one}))$. Therefore, Lemma \ref{lemma-bad-leaf-good-leaf} gives that the strategy $\mathcal S_1$ dictates the same message for  $v_1^{both}$ and $v_1^{b,mid}$ at vertex $u$, which resolves the second case and completes the proof.
\end{proof}

\subsubsection{Proof of Lemma \ref{lemma-small-pay-ud}:
Observations About The Mechanism} \label{subsubsec-prop-alloc-ud}
The proof is a direct consequence of the approximation guarantee of the mechanism and the fact that it is 
% the mechanisms in all of them are 
obviously strategy-proof (and thus also dominant-strategy incentive compatible) and satisfies individual rationality and no negative transfers. 
We write the proof  for the sake of completeness.

Throughout the proof, we say that an item $x$ is \emph{valuable} for a valuation $v$ if $v(\{x\})>0$.

\begin{proof}[Proof of \cref{lemma-small-pay-ud}]
    
For part \ref{item-1-ud}, note that because of individual rationality:
\begin{equation}\label{eq-1-ud}
 v_1^{a,one}(f(v_1^{a,one},v_2^{b,one}))-P_1(v_1^{a,one},v_2^{b,one})\ge 0   
\end{equation}
We remind that by \cref{lemma-ud-instances} part \ref{condi-1-ud}, 
the allocation rule $f$ allocates item $a$ to bidder $1$ given $(v_1^{one},v_2^{one})$, so:
\begin{equation}\label{eq-2-ud}
 v_1^{a,one}(f(v_1^{a,one},v_2^{b,one}))=1   
\end{equation}
Combining (\ref{eq-1-ud}) and (\ref{eq-2-ud}) proves the part \ref{item-1-ud} for the valuation profile $(v_1^{a,one},v_2^{b,one})$.
Observe that it also implies that: 
% \begin{equation}\label{mid-one-eq-ud-val}
%     v_1^{a,mid}(f(v_1^{a,one},v_2^{b,one}))=k^2
% \end{equation}
\begin{equation*}\label{mid-one-eq-ud}
v_1^{a,large}(f(v_1^{a,one},v_2^{b,one}))-P_1(v_1^{a,one},v_2^{b,one})\ge k^4-1
\end{equation*}
% To analyze the allocation and payment given $(v_1^{a,mid},v_2^{b,one})$, observe that since 
% $f(v_1^{a,one},v_2^{b,one})$ allocates to bidder $1$ wins item $a$ and pays at most $1$ implies that:
Combining this inequality with the fact that the allocation rule $f$ and the payment scheme $P_1$ are realized by a dominant-strategy mechanism gives that:
\begin{equation}\label{eq-part1-unit}
\begin{aligned}
    v_1^{a,large}(f(v_1^{a,large},v_2^{b,one}))-P_1(v_1^{a,large},v_2^{b,one})&\ge v_1^{a,large}(f(v_1^{a,one},v_2^{b,one}))-P_1(v_1^{a,one},v_2^{b,one}) \\
    &\ge k^4-1 
    % &\text{(by (\ref{mid-one-eq-ud}))} 
\end{aligned}  
\end{equation}
Note that the property of no negative transfers implies that $P_1(v_1^{a,large},v_2^{b,one})\ge 0$, so
$v_1^{a,large}(f(v_1^{a,large},\allowbreak v_2^{b,one}))\ge k^4-1$. Since only item $a$ is valuable for  $v_1^{a,large}$, we can deduce that player $1$ wins it given the valuation profile $(v_1^{a,large}\allowbreak,v_2^{b,one})$, which further implies that in fact:
\begin{equation}\label{eq-part1-another}
v_1^{a,large}(f(v_1^{a,large}\allowbreak,v_2^{b,one}))= k^4    
\end{equation}
Now, combining (\ref{eq-part1-unit}) and (\ref{eq-part1-another}) gives that $P_1(v_1^{a,large},v_2^{b,one})\le 1$, which completes the proof for $(v_1^{a,large},v_1^{b,one})$. 
% The proof that given $(v_1^{a,large},v_2^{b,one})$, bidder $1$ wins item $a$ and pays at most $1$ is analogous to the proof for $(v_1^{a,mid},v_1^{b,one})$.   

For part \ref{item-2-ud}, observe that by  \cref{lemma-ud-instances} part \ref{condi-2-ud}, given $(v_1^{a,mid},v_2^{a,large})$, bidder $2$ gets item $a$, so clearly bidder $1$ does not get it. Since only item $a$ is valuable for $v_1^{a,mid}$, we get that $v_1^{a,mid}(f(v_1^{a,mid},v_2^{a,large}))\allowbreak=0$, so by individual rationality $P_1(v_1^{a,mid},v_2^{a,large})\le 0$. Due to no negative transfers, we get that $P_1(v_1^{a,mid},v_2^{a,large})=0$, which completes the proof of this part. 

We now prove part \ref{item-4-ud}. We begin by proving it for the valuation profile $(v_1^{b,small},v_2^{b,one})$. Note that by \cref{lemma-ud-instances} part \ref{condi-6-ud}, given this valuation profile, bidder $1$ wins item $b$. Combining this fact with the fact that the mechanism satisfies individual rationality implies that $P_1(v_1^{b,small},v_2^{b,one})\le k$, as needed. Showing that the same goes for $(v_1^{b,mid},v_2^{b,one})$ and $(v_1^{b,large},v_2^{b,one})$ is analogous to the proof of part \ref{item-1-ud} for the valuation profile $(v_1^{a,mid},v_2^{b,one})$ above.

To prove \ref{item-4.5-ud}, note that by \cref{lemma-ud-instances} part \ref{condi-4-ud}, bidder $1$ does not win item $b$ given $(v_1^{b,mid},v_2^{b,large})$. Thus, he has to pay at most zero due to individual rationality, and the property of no negative transfers implies that $P_1(v_1^{b,mid},v_2^{b,large})=0$, as needed.  

For part \ref{item-6-ud}, we first show that bidder $1$ wins a bundle that contains item $a$ given $(v_1^{both},v_2^{b,one})$. Note that since the allocation rule $f$ and the payment scheme $P_1$ are realized by a dominant-strategy mechanism, we have that:
\begin{equation}\label{eq-part6-unit}
\begin{aligned}
    v_1^{both}(f(v_1^{both},v_2^{b,one}))-P_1(v_1^{both},v_2^{b,one})&\ge v_1^{both}(f(v_1^{a,one},v_2^{b,one}))-P_1(v_1^{a,one},v_2^{b,one}) \\
    &\ge 2k+2 &\text{(by part \ref{item-1-ud})}
\end{aligned}    
\end{equation}
Combining (\ref{eq-part6-unit}) with the property of no negative transfers implies that $v_1^{both}(f(v_1^{both},v_2^{b,one}))\ge 2k+2$. Thus, given $(v_1^{both},v_2^{b,one})$, bidder $1$ necessarily gets a bundle that contains item $a$. 


For the upper bound on the payment, note that in fact $v_1^{both}(f(v_1^{both},v_2^{b,one}))=2k+3$.  
% and that by part \ref{item-1-ud}, $P_1(v_1^{a,one},v_2^{b,one})\le 1$. 
Combining it with inequality (\ref{eq-part6-unit}) gives that $P_1(v_1^{both},v_2^{b,one})\le 1$. By that, we  complete the proof of part \ref{item-6-ud}. 

We are now finally ready to wrap up by proving part \ref{item-complicated-ud}. We begin by showing that given $(v_1^{both},v_2^{a,large})$, bidder $1$ does not win item $a$. This is due to weak monotonicity. Formally, we remind that by part \ref{item-2-ud}, given $(v_1^{a,mid},v_2^{a,large})$, bidder $1$ wins a bundle $S^{mid}$
that does not contain $a$. Thus, if $f(v_1^{both},v_2^{a,large})$ allocates to bidder $1$ 
a bundle $S^{both}$ that contains item $a$, then $v_1^{a,mid}(S^{both})-v_1^{a,mid}(S^{mid})=k^2$ whereas $v_1^{both}(S^{both})-v_1^{both}(S^{mid})=2k+3$, so $f$ is not weakly monotone. Since $f$ and $P_1$ realize a dominant-strategy mechanism, \cref{wmon-lemma} gives that $f$ has to be weakly-monotone, so we get a contradiction. Thus, $f(v_1^{both},v_2^{a,large})$ has to allocate bidder $1$ a bundle that does not contain item $a$. 

The bounds on the payments are a straightforward implication of individual rationality and no negative transfers. If $f(v_1^{both},v_2^{a,large})$ allocates item $b$ to bidder $1$ then the payment is at most $2k+1$, and if it allocates to bidder $1$ no valuable items, then the payment has to be at most zero. Due to no negative transfers, it is zero exactly.   
\end{proof}

\subsection{Proof of Theorem \ref{thm-lb-add}: An Impossibility for Additive Bidders} \label{lb-add-proof-place}
The proof follows the same structure as in \cref{thm-mua-sm-lb} and \cref{thm-lb-ud}: roughly speaking, we describe a distribution $\mathcal{D}$, show that it is “hard” for deterministic mechanisms, and then use Yao's Lemma to deduce hardness for randomized mechanisms. In particular, this proof is very similar to the proof of \cref{thm-lb-ud} in \cref{lb-ud-proof-place}, and we write both for the sake of completeness. 

The proof outline is as follows. In \cref{subsubsec-description-add}, 
we describe the  hard distribution $\mathcal D$. 
In \cref{subsubsec-performance-add} we analyze the allocation and payments of a deterministic \textquote{good} mechanism 
% in several scenarios, 
and explain why these properties  imply that  a \textquote{good} mechanism does not exist in  \cref{subsubsec-contradiction-add}.
We defer technical proofs to \cref{app-claims-proofs-add,subsubsec-prop-alloc-add}. 

We prove for the case of two bidders and two items, but the proof extends to any number of bidders and any number of items by adding bidders with the all-zero valuation and assuming that the bidders in our construction have zero values for the additional items.

\begin{comment}
The outline of the proof is as follows. 
In \cref{subsubsec-description-add},
we describe the  hard distribution $\mathcal D$. 
In \cref{subsubsec-performance-add}, we state the allocation of a deterministic mechanism 
that gives an approximation better than $\frac{7}{8}$ to the optimal welfare.  
We conclude by further analyzing the allocation and payments of a deterministic \textquote{good} mechanism in several scenarios, and explaining why these properties  imply that  a \textquote{good} mechanism does not exist (\cref{subsubsec-contradiction-add}).
We defer  technical proofs to \cref{app-claims-proofs-add} and \cref{subsubsec-prop-alloc-add}.

We prove for the case of two bidders and two items, but the proof extends to any number of bidders and any number of items by adding bidders with the all-zero valuation and assuming that the bidders in our construction have zero values for the additional items.
\end{comment}

\subsubsection[Construction of a "Hard" Distribution D]{Construction of a \textquote{Hard} Distribution $\mathcal D$} \label{subsubsec-description-add}
    To define the probability distribution over the valuation profiles, we name the two items $a$ and $b$, and let $k$ be an arbitrarily large number. Note that since these valuations are additive, we can fully describe them by specifying their value for each item.
    Consider the following valuations:  
\[
% \forall S\subseteq M, \quad
v_1^{{a,one}}(x) = 
\begin{cases}
1 & x=a,\\
0 & \text{otherwise.}
\end{cases} \quad 
v_2^{{b,one}}(x) = 
\begin{cases}
1 & x=b,\\
0 & \text{otherwise.}
\end{cases}
\]

\[
% \forall S\subseteq M, \quad
v_1^{{a,mid}}(x) = 
\begin{cases}
k^2 & x=a,\\
0 & \text{otherwise.}
\end{cases} \quad 
v_2^{{b,mid}}(x) = 
\begin{cases}
k^2 & x=b,\\
0 & \text{otherwise.}
\end{cases}
\]

\[
% \forall S\subseteq M, \quad
v_1^{{a,large}}(x) = 
\begin{cases}
k^4 & x=a,\\
0 & \text{otherwise.}
\end{cases} \quad 
v_2^{{b,large}}(x) = 
\begin{cases}
k^4 & x=b,\\
0 & \text{otherwise.}
\end{cases}
\]

\[
% \forall S\subseteq M, \quad
v_1^{{b,small}}(x) = 
\begin{cases}
k & x=b,\\
0 & \text{otherwise.}
\end{cases} \quad 
v_2^{{a,small}}(x) = 
\begin{cases}
k & x=a,\\
0 & \text{otherwise.}
\end{cases}
\]

\[
% \forall S\subseteq M, \quad
v_1^{{b,mid}}(x) = 
\begin{cases}
k^2 & x=b,\\
0 & \text{otherwise.}
\end{cases} \quad 
v_2^{{a,mid}}(x) = 
\begin{cases}
k^2 & x=a,\\
0 & \text{otherwise.}
\end{cases}
\]

\[
% \forall S\subseteq M, \quad
v_1^{{b,large}}(x) = 
\begin{cases}
k^4 & x=b,\\
0 & \text{otherwise.}
\end{cases} \quad 
v_2^{{a,large}}(x) = 
\begin{cases}
k^4 & x=a,\\
0 & \text{otherwise.}
\end{cases}
\]

\[
% \forall S\subseteq M, \quad
v_1^{{both}}(x) = 
\begin{cases}
2k+3 & x=a,\\
2k+1 & x=b,\\
0 & \text{otherwise.}
\end{cases} \quad 
v_2^{{both}}(x) = 
\begin{cases}
2k+3 & x=b,\\
2k+1 & x=a,\\
0 & \text{otherwise.}
\end{cases}
\]
Note that all the valuations except for $v_1^{both}$ and $v_2^{both}$ are unit-demand and additive simultaneously. 
Consider the following valuation profiles: 
\begin{align*}
 I_1 = (v_1^{{a,one}},& v_2^{{b,one}}), \quad  I_2 = (v_1^{a,mid}, v_2^{{a,large}}),  \quad I_3 = (v_1^{b,large}, v_2^{b,mid}), \quad
 I_4 = (v_1^{b,mid}, v_2^{b,large}) \\ & I_5 = (v_1^{a,large}, v_2^{a,mid}) \quad I_6 = (v_1^{b,small}, v_2^{b,one}) \quad I_7 =(v_1^{a,one}, v_2^{a,small}) 
\end{align*}
% $I_1 = (v_1^{{a,one}}, v_2^{{b,one}})$, 
% $I_2 = (v_1^{a,mid}, v_2^{{a,large}})$,
% $I_3 = (v_1^{b,large}, v_2^{b,mid})$,
% $I_4 = (v_1^{b,mid}, v_2^{b,large})$,
% $I_5 = (v_1^{a,large}, v_2^{a,mid})$,
% $I_6 = (v_1^{b,small}, v_2^{b,one})$
% and $I_7 =(v_1^{a,one}, v_2^{a,small})$.
Let $\mathcal D$ be the distribution over valuation profiles where the probability of the valuation profile $I_1$ is $\frac{1}{4}$, and the probability of the valuation profiles $I_2,I_3,I_4,I_5,I_6$ and $I_7$ is $\frac{1}{8}$ each. 

\subsubsection[The Performance of the Deterministic Mechanism A on the "Hard" Distribution D]{The Performance of the Deterministic Mechanism $A$ on the \textquote{Hard} Distribution $\mathcal D$}\label{subsubsec-performance-add}
We remind that our goal is to show that no deterministic mechanism that satisfies all of the desired properties extracts more than $\frac{7}{8}$ of the optimal welfare. For that, we begin by observing that:
\begin{lemma}\label{lemma-add-instances}
Every deterministic mechanism that has approximation better than $\frac{8}{7}$ necessarily satisfies all of the following conditions:
\begin{enumerate}
    \item Given the valuation profile $I_1 = (v_1^{\text{a,one}}, v_2^{\text{b,one}})$, the mechanism
    allocates item $a$ to player $1$ and item $b$ to player $2$. \label{condi-1-add}
    % $A$ outputs an allocation with welfare at most $1$. 
\item Given the valuation profile $I_2 = (v_1^{a,mid}, v_2^{a,large})$, the mechanism allocates item $a$ to bidder $2$. \label{condi-2-add}

    \item Given the valuation profile $I_3 = (v_1^{b,large}, v_2^{b,mid})$, the mechanism allocates item $b$  to bidder $1$.
    \label{condi-3-add}
\item Given the valuation profile $I_4 = (v_1^{b,mid}, v_2^{b,large})$, the mechanism allocates item $b$ to bidder $2$. \label{condi-4-add}
\item Given the valuation profile $I_5 = (v_1^{a,large}, v_2^{a,mid})$, the mechanism allocates item $a$ to bidder $1$. \label{condi-5-add}
\item Given the valuation profile $I_6 = (v_1^{b,small}, v_2^{b,one})$, the mechanism allocates item $b$ to bidder $1$. \label{condi-6-add}
\item Given the valuation profile $I_7 = (v_1^{a,one}, v_2^{a,small})$, the mechanism allocates item $a$ to bidder $2$. 
\end{enumerate}
\end{lemma}
The proof of \cref{lemma-add-instances} is straightforward and identical to the proof of \cref{lemma-ud-instances}: if a deterministic mechanism does not satisfy one of conditions, then the fact that $k$ is arbitrarily large implies that its approximation guarantee  is at most $\frac{8}{7}$ with respect to the distribution $\mathcal D$. 
\subsubsection[Reaching a Contradiction: No Deterministic and OSP Mechanism Succeeds on D]{Reaching a Contradiction: No Deterministic and OSP Mechanism Succeeds on $\mathcal D$}
\label{subsubsec-contradiction-add}
We now employ \cref{lemma-add-instances} to prove  \cref{thm-lb-add}. Let the domain $V_i$ of each bidder consist 
of additive valuations with values in  $\{0,1,\ldots,k^4\}$, where $k$ is an arbitrarily large integer. 


Fix a deterministic mechanism $A$ and strategies
$(\mathcal S_1,\mathcal S_2)$ that are individually rational, satisfy no negative transfers for all the valuations  $V_1\times V_2$ and give approximation better than $\frac{7}{8}$
in expectation over the valuation profiles in the distribution $\mathcal D$. 
Denote with  $(f,P_1,P_2)$ be the allocation rule and the payment scheme that $A$ and $(\mathcal S_1,\mathcal S_2)$ realize, and assume towards a contradiction that $A$ and $(\mathcal S_1,\mathcal S_2)$ are obviously strategy-proof. 


For the analysis of the deterministic mechanism $A$, we focus on the following subsets of the domains of the valuations:
$
\mathcal{V}_1=\{v_1^{a,one},v_1^{a,mid},v_1^{a,large},v_1^{b,small},v_1^{b,mid},v_1^{b,large},v_1^{both}\}$ and  $\mathcal V_2=\{v_2^{b,one},v_2^{b,mid},v_2^{b,large},\allowbreak v_2^{a,small},v_2^{a,mid}, \allowbreak v_2^{a,large},v_2^{both}\}$.
Observe that there necessarily exists a vertex $u$, and valuations $v_1,v_1' \in \mathcal{V}_1$, and  $v_2,v_2' \in \mathcal{V}_2$ such that $(\mathcal{S}_1(v_1), \mathcal{S}_2(v_2))$ diverge at vertex $u$. This follows from \cref{lemma-add-instances}, which implies that the mechanism $A$ must output different allocations for different valuation profiles in $\mathcal{V}_1 \times \mathcal{V}_2$. Consequently, not all valuation profiles end up in the same leaf, meaning that divergence must occur at some point.

Let $u$ be the first vertex in the protocol such that 
 %the behavior profiles 
 $(\mathcal{S}_1(v_1),\mathcal{S}_2(v_2))$ and $(\mathcal{S}_1(v_1'),\mathcal{S}_2(v_2'))$ diverge, i.e., dictate different messages. 
Note that by definition this implies that $u\in Path(\mathcal{S}_1(v_1),\mathcal{S}_2(v_2))\cap Path(\mathcal{S}_1(v_1'),\mathcal{S}_2(v_2'))$ and that either bidder $1$ or bidder $2$ sends different messages for the valuations in $\mathcal{V}_1$ or $\mathcal V_2$, respectively. 
Without loss of generality, we assume that bidder $1$ sends different messages, meaning that there exist $v_1,v_1'\in \mathcal{V}_1$ such that $\mathcal S_1(v_1)$ and $\mathcal S_1(v_1')$ dictate different messages at vertex $u$.
\begin{comment}
Observe that either bidder $1$ or $2$ has to send different messages for different valuations in $\mathcal V_i$ at some vertex. 
This is an immediate implication of \cref{lemma-add-instances}. Due to \cref{lemma-add-instances}, 
the mechanism $A$ outputs different allocations for different valuation profiles in $\mathcal V_1\times \mathcal V_2$, so they necessarily end up in different leaves, so they necessarily diverge at some vertex.
% given the valuation profiles $I_1=(v_1^{one},v_2^{one})$ and
%  $I_2=(v_1^{all},v_2^{one})$, meaning that the behavior profiles $(\mathcal S_1(v_1^{one},\mathcal S_2(v_2^{one}))$ and $(\mathcal S_1(v_1^{all},\mathcal S_2(v_2^{one}))$ reach different leaves, so they have to diverge at some vertex. 
 
 Let $u$ be the first vertex in the protocol of the mechanism $A$ where they diverge, meaning that  the behavior profiles $(\mathcal{S}_1(v_1),\mathcal{S}_2(v_2))$ and $(\mathcal{S}_1(v_1'),\mathcal{S}_2(v_2'))$ dictate different messages. 
Note that by definition $u\in Path(\mathcal{S}_1(v_1),\mathcal{S}_2(v_2))\cap Path(\mathcal{S}_1(v_1'),\mathcal{S}_2(v_2'))$. We remind that each vertex is associated with only one player that sends messages in it. Note that the distribution $\mathcal D$ we have defined is symmetric, so we can assume without loss of generality that player $1$ is the player that sends a message in vertex $u$. Thus, there exist $v_1,v_1'\in \mathcal{V}_1$ such that $\mathcal S_1(v_1)$ and $\mathcal S_1(v_1')$ dictate different messages at vertex $u$. 
\end{comment}
However, the following collection of claims show that since the strategy $\mathcal S_1$ is obviously dominant, it dictates the same message for all the valuations in $\mathcal{V}_1$. Thus, we get a contradiction, which completes the proof of \cref{thm-lb-add}:
\begin{claim}\label{claim-a-one-mid-add}
    The strategy $\mathcal S_1$ dictates the same message at vertex $u$ for the valuations $v_1^{a,one}$ and $v_1^{a,mid}$. 
\end{claim}
\begin{claim}\label{claim-a-mid-large-add}
    The strategy $\mathcal S_1$ dictates the same message at vertex $u$ for the valuations $v_1^{a,mid}$ and $v_1^{a,large}$. 
\end{claim}
\begin{claim}\label{claim-b-small-large-add}
    The strategy $\mathcal S_1$ dictates the same message at vertex $u$ for the valuations $v_1^{b,small}$ and $v_1^{b,mid}$. 
\end{claim}
\begin{claim}\label{claim-b-mid-large-add}
    The strategy $\mathcal S_1$ dictates the same message at vertex $u$ for the valuations $v_1^{b,mid}$ and $v_1^{b,large}$. 
\end{claim}
\begin{claim}\label{claim-a-both-add}
    The strategy $\mathcal S_1$ dictates the same message at vertex $u$ for the valuations $v_1^{both}$ and $v_1^{a,mid}$. 
\end{claim}
\begin{claim}\label{claim-b-both-add}
    The strategy $\mathcal S_1$ dictates the same message at vertex $u$ for the valuations $v_1^{both}$ and $v_1^{b,mid}$. 
\end{claim}
To prove these claims, we use the following observations about the allocation and the payment scheme of player $1$:    
\begin{lemma}\label{lemma-small-pay-add}
    The allocation rule $f$ and the payment scheme $P_1$ of bidder $1$ satisfy that:
    % Let $f$ be an allocation rule and let $P_1$ be the payment scheme of bidder $1$ that 
    % The allocation rule $f$ and the  payment scheme $P_1$ $(f,P_1,\ldots,P_n):V_1\times \cdots \times V_n\to \mathbb{R}^{n}$
    % are realized by a dominant-strategy, individually rational and no negative transfers mechanism. Then: 
    \begin{enumerate}
        \item Given the valuation profiles $(v_1^{a,one},v_{2}^{b,one})$ and $(v_1^{a,large},v_{2}^{b,one})$, bidder $1$ wins item $a$ and pays at most $1$.  \label{item-1-add}
        \item  Given the valuation profile $(v_1^{a,mid},v_2^{a,large})$, bidder $1$ wins a bundle that does not contain item $a$ and pays zero.   \label{item-2-add}
        \item Given the valuation profiles $(v_1^{b,small},v_2^{b,one})$,
        $(v_1^{b,mid},v_2^{b,one})$ and $(v_1^{b,large},v_2^{b,one})$,  
        bidder $1$ wins item $b$ and pays at most $k$. 
        \label{item-4-add}
           \item  Given the valuation profile $(v_1^{b,mid},v_2^{b,large})$, bidder $1$ wins a bundle that does not contain item $b$ and pays zero.   \label{item-4.5-add}
\item Given the valuation profile $(v_1^{both},v_2^{b,one})$, 
bidder $1$ gets a bundle that contains item $a$ and pays at most $4k+4$. \label{item-6-add}
\item Given the valuation profile $(v_1^{both},v_2^{a,large})$, bidder $1$ does not win item $a$.  If bidder $1$ wins item $b$, then he pays at most $2k+1$. If he wins a bundle that contains neither item $a$ or item $b$, then he pays zero. \label{item-complicated-add}
    \end{enumerate}
\end{lemma}
\subsubsection[All Valuations in V\_1 Send The Same Message: Proofs of Claims \ref{claim-a-one-mid-add} to \ref{claim-b-both-add}]{All Valuations in $\mathcal V_1$ Send The Same Message: Proofs of Claims \ref{claim-a-one-mid-add} to \ref{claim-b-both-add}}\label{app-claims-proofs-add}
We will now prove only \cref{claim-a-both-add} and \cref{claim-b-both-add}, since the proofs of the rest of the claims in fact appear in \cref{app-claims-proofs-ud}: the proof of \cref{claim-a-one-mid-add} is identical to the proof of \cref{claim-a-one-mid}, and the same goes for the proofs of  \cref{claim-a-mid-large-add} and \cref{claim-a-mid-large}, 
the proof of \cref{claim-b-small-large-add} and  \cref{claim-b-small-large} and the proof of \cref{claim-b-mid-large-add} and \cref{claim-b-mid-large}.

% We now prove claims \ref{claim-a-one-mid-add} to \ref{claim-b-both-add}, which jointly imply a contradiction. All proofs repeatedly use \cref{lemma-add-instances}, which analyzes the allocation and the payments of any deterministic mechanisms with the desired properties. 
% % Also, they are quite similar and we write them for completeness. 
% Only \cref{claim-a-both-add} and \cref{claim-a-both-add} require a proof. 



\begin{proof}[Proof of \cref{claim-a-both-add}]
            By Lemma \ref{lemma-small-pay-add} part \ref{item-6-add}, $f(v_1^{both},v_2^{b,one})$ allocates item $a$ to bidder $1$ and $P_1(v_1^{both},\allowbreak v_2^{b,one})\le 4k+4$. Therefore:
\begin{equation}\label{eq-good-leaf5-add}
 v_1^{a,mid}(f(v_1^{both},v_2^{b,one}))-P_1(v_1^{both},v_2^{b,one})\ge k^2-4k-4   
\end{equation}
Whereas by part \ref{item-2-add} of Lemma \ref{lemma-small-pay-add}: 
% player $1$ either does not win item $a$ or pays at least $k^2$, so:
 \begin{equation}\label{eq-bad-leaf5-add}
 v_1^{a,mid}(f(v_1^{a,mid},v_2^{a,large}))-P_1(v_1^{a,mid},v_2^{a,large})=0   
\end{equation}
Combining the inequalities (\ref{eq-good-leaf5-add}) and (\ref{eq-bad-leaf5-add}) gives:
\begin{equation*}
 v_1^{a,mid}(f(v_1^{a,mid},v_2^{a,large}))-P_1(v_1^{a,mid},v_2^{a,large})< 
 v_1^{a,mid}(f(v_1^{both},v_2^{b,one}))-P_1(v_1^{both},v_2^{b,one})
\end{equation*}
Note that vertex $u$ belongs in $Path(\mathcal S_1(v_1^{both}),\mathcal S_2(v_2^{b,one}))$ and also in
$Path(\mathcal{S}_1(v_1^{a,mid}),\allowbreak\mathcal{S}_2(v_2^{a,large}))$. Therefore, Lemma \ref{lemma-bad-leaf-good-leaf} gives that the strategy $\mathcal S_1$ dictates the same message for  $v_1^{a,mid}$ and $v_1^{both}$ at vertex $u$.
\end{proof}
\begin{proof}[Proof of \cref{claim-b-both-add}]
    % To prove that the strategy $\mathcal S_1$ dictates the same message for $v_1^{both}$ and $v_1^{b,mid}$, {we consider at the following cases.} 
    Note that by \cref{lemma-small-pay-add} part \ref{item-complicated-add}, given the valuation profile $(v_1^{both},v_2^{a,large})$, bidder $1$ cannot win item $a$ so we consider the following two cases: the case where he wins item $b$ and the case where he wins neither of these items.

    Assume that bidder $1$ wins item $b$ given the valuation profile $(v_1^{both},v_2^{a,large})$. Note that in this case, \cref{lemma-small-pay-add} part \ref{item-complicated-add} also implies that:
    \begin{equation}\label{eq-good-leaf6-add}
 v_1^{b,mid}(f(v_1^{both},v_2^{a,large}))-P_1(v_1^{both},v_2^{a,large})\ge k^2 -2k-1   
\end{equation}
Whereas by part \ref{item-4.5-add} of Lemma \ref{lemma-small-pay-add}: 
% player $1$ either does not win item $a$ or pays at least $k^2$, so:
 \begin{equation}\label{eq-bad-leaf6-add}
 v_1^{b,mid}(f(v_1^{b,mid},v_2^{b,large}))-P_1(v_1^{b,mid},v_2^{b,large})= 0   
\end{equation}
Combining the inequalities (\ref{eq-good-leaf6-add}) and (\ref{eq-bad-leaf6-add}) gives:
\begin{equation*}
 v_1^{b,mid}(f(v_1^{b,mid},v_2^{b,large}))-P_1(v_1^{b,mid},v_2^{b,large})< 
 v_1^{b,mid}(f(v_1^{both},v_2^{a,large}))-P_1(v_1^{both},v_2^{a,large})
\end{equation*}
Note that vertex $u$ belongs in $Path(\mathcal S_1(v_1^{both}),\mathcal S_2(v_2^{a,large}))$ and also in
$Path(\mathcal{S}_1(v_1^{a,mid}),\allowbreak\mathcal{S}_2(v_2^{a,large}))$. Therefore, Lemma \ref{lemma-bad-leaf-good-leaf} gives that the strategy $\mathcal S_1$ dictates the same message for  $v_1^{both}$ and $v_1^{b,mid}$ at vertex $u$, which completing this case.

For the latter case, where bidder $1$ gets a bundle that contains neither item $a$ nor item $b$ given the valuation profile $(v_1^{both},v_2^{a,large})$, note that for \cref{lemma-small-pay-add} part \ref{item-complicated-add} implies that:
\begin{equation}\label{eq-bad-leaf7-add}
 v_1^{both}(f(v_1^{both},v_2^{a,large}))-P_1(v_1^{both},v_2^{a,large})=0   
\end{equation}
Whereas by part \ref{item-4-add} of Lemma \ref{lemma-small-pay-add}: 
% player $1$ either does not win item $a$ or pays at least $k^2$, so:
 \begin{equation}\label{eq-good-leaf7-add}
 v_1^{both}(f(v_1^{b,mid},v_2^{b,one}))-P_1(v_1^{b,mid},v_2^{b,one}) \ge 2k+1-k=k+1 
\end{equation}
Combining the inequalities (\ref{eq-bad-leaf7-add}) and (\ref{eq-good-leaf7-add}) gives:
\begin{equation*}
 v_1^{both}(f(v_1^{both},v_2^{a,large}))-P_1(v_1^{both},v_2^{a,large})< 
 v_1^{both}(f(v_1^{b,mid},v_2^{b,one}))-P_1(v_1^{b,mid},v_2^{b,one})
\end{equation*}
Note that vertex $u$ belongs in $Path(\mathcal S_1(v_1^{both}),\mathcal S_2(v_2^{a,large}))$ and also in
$Path(\mathcal{S}_1(v_1^{b,mid}),\allowbreak\mathcal{S}_2(v_2^{b,one}))$. Therefore, Lemma \ref{lemma-bad-leaf-good-leaf} gives that the strategy $\mathcal S_1$ dictates the same message for  $v_1^{both}$ and $v_1^{b,mid}$ at vertex $u$, which solves the second case and completes the proof.
\end{proof}

\subsubsection{Proof of Lemma \ref{lemma-small-pay-add}: Observations About The Mechanism} \label{subsubsec-prop-alloc-add}
The proof of \cref{lemma-small-pay-ud} is a direct consequence of the approximation guarantee of the mechanism and the fact that it is 
% the mechanisms in all of them are 
obviously strategy-proof (and thus also dominant-strategy incentive compatible) and satisfies individual rationality and no negative transfers. 
% Most of it identical to the proof of \cref{lemma-small-pay-ud} and we write the remaining the parts below. 
% Throughout the proof, we say for abbreviation that certain inequalities hold  because of individual rationality instead of saying that they hold because the allocation rule together with the payment scheme are realized by a mechanism and strategies that satisfy individual rationality. We do the same for the no negative transfers property. Also, we say that an item $x$ is \emph{valuable} for a valuation $v$ if $v(\{x\})>0$.

\begin{proof}[Proof of \cref{lemma-small-pay-add}]
The proofs of parts \ref{item-1-add}, \ref{item-2-add}, \ref{item-4-add} and \ref{item-4.5-add} are identical to the respective parts in the proof of \cref{lemma-small-pay-ud}. 


For part \ref{item-6-add}, we first show that bidder $1$ wins a bundle that contains item $a$ given $(v_1^{both},v_2^{b,one})$. Note that since the allocation rule $f$ and the payment scheme $P_1$ are realized by a dominant-strategy mechanism, we have that:
\begin{equation}\label{eq-part6-add}
\begin{aligned}
    v_1^{both}(f(v_1^{both},v_2^{b,one}))-P_1(v_1^{both},v_2^{b,one})&\ge v_1^{both}(f(v_1^{a,one},v_2^{b,one}))-P_1(v_1^{a,one},v_2^{b,one}) \\
    &\ge 2k+2 &\text{(by part \ref{item-1-add})}
\end{aligned}    
\end{equation}
Combining (\ref{eq-part6-add}) with the property of no negative transfers implies that $v_1^{both}(f(v_1^{both},v_2^{b,one}))\ge 2k+2$. Thus, given $(v_1^{both},v_2^{b,one})$, bidder $1$ necessarily gets a bundle that contains item $a$. 
For the upper bound on the payment, note that $v_1^{both}(f(v_1^{both},v_2^{b,one}))\le 4k+4$, so by individual rationality  $P_1(v_1^{a,both},v_2^{b,one})\le 4k+4$. By that, we  complete the proof of part \ref{item-6-add}. 

For part \ref{item-complicated-add},
we begin by showing that given $(v_1^{both},v_2^{a,large})$, bidder $1$ does not win item $a$. This is due to weak monotonicity. Formally, we remind that by part \ref{item-2-add}, given $(v_1^{a,mid},v_2^{a,large})$, bidder $1$ wins a bundle $S^{mid}$
that does not contain $a$. Thus, if $f(v_1^{both},v_2^{a,large})$ allocates 
a bundle $S^{both}$ that contains item $a$ to bidder $1$, then we have that $v_1^{a,mid}(S^{both})-v_1^{a,mid}(S^{mid})=k^2$ whereas $v_1^{both}(S^{both})-v_1^{both}(S^{mid})\le 4k+4$, so $f$ is not weakly monotone. Since $f$ and $P_1$ realize a dominant-strategy mechanism, \cref{wmon-lemma} implies that $f$ has to be weakly-monotone, so we get a contradiction. Thus, $f(v_1^{both},v_2^{a,large})$ has to allocate bidder $1$ a bundle that does not contain item $a$. 

The bounds on the payments are a straightforward implication of individual rationality and no negative transfers. If $f(v_1^{both},v_2^{a,large})$ allocates item $b$ to bidder $1$ then the payment is at most $2k+1$, and if it allocates to bidder $1$ no valuable items, then the payment has to be at most zero. Due to no negative transfers, it is zero exactly.   
\end{proof}

\subsection{Proofs of Claims \ref{cl:subadditive} and \ref{cl:general}} \label{subsec::proofs-subadd-general}
\begin{proof}[Proof of \cref{cl:subadditive}]
First the bidders are assigned a an arbitrary order. 
% \shirinote{Arbitrary or fixed? it's a bit unclear I think} 
This is clearly OSP.  Then, after randomly partitioning the bidder into two groups $G_1$ and $G_2$ (which is done uniformly at random and independently of bidder valuations) with probability $1/2$ the auction runs a second-price auction for the grand bundle on bidders in $G_1$.  This is clearly OSP for these bidders.  

With the remaining probability bidders in $G_1$ are discarded and the auction learns the highest value among these bidders for the grand bundle (this is also clearly OSP since bidders in $G_1$ do not gain any positive utility under any value profile).  After learning this value in an OSP way from bidders in $G_1$ the auction runs the \textsc{Binary-Search-Mechanism} on the bidders in $G_2$.  Recall that this mechanism draws independently and uniformly at random a round $r_i$ for each $i \in G_2$ and a random final round $r^*$ (before any bidder in $G_2$ acts).  

Then in each round, the mechanism sets a price for each item, allows the bidders in that round (in the pre-specified order generated at the outset of the auction) to state their demand set for ``unclaimed'' items and ``conditionally claim'' them, and then if the current round is $r^*$, terminate the auction awarding bidders in $r^*$ their conditionally claimed items (otherwise, no bidders in the round are allocated any items).  Observe that each bidder in $G_2$ participates in exactly one round and is, thus, asked a single demand query.  If the round $r_i$ that bidder $i$ participates in is $r^*$ then she weakly maximizes her utility when she is called to act by truthfully reporting her demand set (since she is allocated exactly these items).  On the other hand, if $r_i \neq r^*$ then she obtains no utility under any possible valuation profile of all agents and, thus, truthfully reporting her demand set is a weakly obviously dominant strategy.  

Since for any fixed realization of the random outcomes of all coin flips the resulting mechanism is OSP, we have that the randomized mechanism is universally OSP.
\end{proof}
% \subsection{Proof of \cref{cl:general}}
\begin{proof}[Proof of \cref{cl:general}]
    For this auction, bidders are randomly partitioned into three groups \texttt{STAT}, \texttt{SECOND-PRICE}, and \texttt{FIXED}.  We argue that for any fixed partition the mechanism is OSP and, hence, the mechanism is universally OSP. 
    
    Bidders in \texttt{STAT} cannot possibly win any goods and the mechanism just requests that they report their values.  Since they win nothing regardless of report, it is an obviously dominant strategy for them to report their information truthfully. 
    Bidders in \texttt{SECOND-PRICE} participate in a second-price auction for the grand bundle.  Since this can be implemented as an ascending-price auction for the grand bundle,  \cref{lemma-partial} 
    implies that bidders in \texttt{SECOND-PRICE} have an obviously dominant truthful strategy.  
    Finally, bidders in \texttt{FIXED} are approached in an arbitrary order and they are allocated their utility maximizing bundle of the remaining items given a fixed vector of prices.  
    Thus, 
    % Since this step can be implemented by allowing each bidder to select their favorite bundle at a fixed vector of prices, there is an implementation of the mechanism which gives 
    each
     bidder in \texttt{FIXED} an obviously dominant truthful strategy.  
    
    As such, each bidder on any fixed realization of the random partition has an obviously dominant truthful strategy and the entire mechanism is then universally OSP.
\end{proof}

