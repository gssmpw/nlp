

{We note that there is a gap between our upper and lower bounds and we} leave open 
%as important follow-up work 
the question of understanding the exact approximation ratio of these classes of mechanisms and of randomized OSP mechanisms in general.  A particular enticing question is whether there exists an $\mathcal O(1)$-approximate randomized OSP mechanism for more general complement-free valuations (e.g., submodular or subadditive valuations) or whether one can demonstrate an $\omega(1)$-lower bound on the performance of any OSP mechanism in these settings.  

Toward the latter, one would need to circumvent a crucial limitation of our lower bound approach. The reason for it is that we  prove our lower bounds
by presenting a distribution over valuation profiles, and showing that a mechanism has to fail on at least one valuation profile.  To prove super-constant lower bounds, it is essential to find distributions whose support has super-constant number of instances and show that every deterministic mechanism 
errs on a significant fraction of them.
%Note that , when proving lower bounds for deterministic mechanisms, the set of valuations being analyzed can be arbitrarily large. To overrule the existence of a deterministic mechanism with approximation $\alpha$,
%we can come up with a set of instances which is arbitrarily large, and it suffices to show that a mechanism has to fail on one of them.
%However, when proving lower bounds for randomized mechanisms, we face a new challenge. 
%To prove super-constant lower bounds for classes with combinatorial valuations, it is essential to find distributions whose support has super-constant number of instances for which every deterministic mechanism 
%has to err 
%errs on a significant fraction of them.

There are also interesting  questions outside the realm of the auction settings.  For example, exploring other settings to determine whether there exists a separation between the performance
dominant-strategy and obviously strategy-proof mechanisms is an intriguing direction.

% \pinksout{Finally, performing further empirical investigations regarding whether the goal of ``understandability'' is achieved by randomized OSP mechanisms would be important for the purpose of various practical applications.} \src{I think that we can give up on this due to lack of space.. What do you think?}



% \begin{itemize}
%     \item Randomized mechanisms for multi-unit auction with general valuations.
%     \item Super-constant lower bound for more general valuations, climbing up the complement free hierarchy?
%     \item Improving the constants? 
%     \item Further investigate empirically whether the goal of understandability is attained by randomized OSP mechanisms. 
%     \item The power of randomized mechanisms for single-parameter domains. 
% \end{itemize}
% \begin{itemize}
%     \item Talk about computation/communication?
%     \item Optimize constants?
%     \item Separations in other settings?  Anywhere where DSIC is asymptotically easier than OSP (randomized)?
% \end{itemize}