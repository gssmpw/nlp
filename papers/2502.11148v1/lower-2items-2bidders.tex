

\begin{theorem}\label{thm-mua-sm-lb}
For a multi-unit auction with $m\ge 2$ items and $n \ge 2$ unknown single-minded bidders,
no randomized mechanism that satisfies OSP, individual rationality and no negative transfers has approximation better than $\nicefrac{6}{5}$.
\end{theorem}
We note that that the proof uses a distribution of valuations that is based on the 
construction of \cite{Ron24}.  
\begin{proof}
We assume the domain $V_i$ of each bidder consists of single-minded valuations with values in  $\{0,1,\ldots,k^4\}$, where $k$ is an arbitrarily large number. 
Our example has only two bidders,  but it can be  extended to any number of bidders by adding bidders with the all-zero valuation. 

    % For the proof, 
To  use 
    our variant of Yao's principle, 
    % (which is stated in the full version),
    %full-version-change-tag%
we define a distribution $\mathcal D$ of valuation profiles
    and show that no deterministic mechanism that satisfies
    obvious strategy-proofness, individual rationality and no negative transfers with respect to $V=V_1\times V_2$ has approximation better than $\frac{6}{5}$ in expectation over $\mathcal D$. 
    To define it, 
consider the following valuations:  
\[
\renewcommand{\arraystretch}{1}
\begin{aligned}
v_i^{\text{one}}(x) &= \begin{cases}
1 & x \geq 1,\\
0 & \text{else.}
\end{cases}
\quad
v_i^{\text{ONE}}(x) = \begin{cases}
k^2 + 1 & x \geq 1,\\
0 & \text{else.}
\end{cases}\\[4pt]
v_i^{\text{all}}(x) &= \begin{cases}
k^2 & x=m,\\
0 & \text{else.}
\end{cases}
\quad
v_i^{\text{ALL}}(x) = \begin{cases}
k^4 & x=m,\\
0 & \text{else.}
\end{cases}
\end{aligned}
\]
% \[
% v_i^{\text{one}}(x) = 
% \begin{cases}
% 1 & x \geq 1,\\
% 0 & \text{else.}
% \end{cases}
%  \hspace{0.25em}
% v_i^{\text{ONE}}(x) = 
% \begin{cases}
% k^2 + 1 & x \geq 1,\\
% 0 & \text{else.}
% \end{cases}
% \]
% \[
% v_i^{\text{all}}(x) = 
% \begin{cases}
% k^2 & x=m,\\
% 0 & \text{else.}
% \end{cases}
%  \hspace{0.25em}
% v_i^{\text{ALL}}(x) = 
% \begin{cases}
% k^4 & x=m,\\
% 0 & \text{else.}
% \end{cases}
% \]
Consider the following valuation profiles:
\[
\begin{aligned}
& I_1 = (v_1^{\text{one}}, v_2^{\text{one}}) \quad 
I_2 = (v_1^{\text{all}}, v_2^{\text{one}}) \quad 
I_3 = (v_1^{\text{ONE}}, v_2^{\text{ALL}}) \quad 
I_4 = (v_1^{\text{one}}, v_2^{\text{all}}) \quad
I_5 = (v_1^{\text{ALL}}, v_2^{\text{ONE}})
\end{aligned}
\]
% $I_1 = (v_1^{\text{one}}, v_2^{\text{one}})$, 
% $I_2 = (v_1^{\text{all}}, v_2^{\text{one}})$,
% $I_3 = (v_1^{\text{ONE}}, v_2^{\text{ALL}})$,
% $I_4 = (v_1^{\text{one}}, v_2^{\text{all}})$,
% and $I_5 =(v_1^{\text{ALL}}, v_2^{\text{ONE}})$.
Denote with $\mathcal D$ the distribution over 
%valuation 
profiles where the probability of  $I_1$ is $\frac{1}{3}$, and the probability of $I_2,I_3,I_4$ and $I_5$ is $\frac{1}{6}$ each. 
Observe that:
%Our goal is to show that no deterministic mechanism that satisfies all of the desired properties extracts more than $\frac{5}{6}$ of the optimal welfare. For that, we begin by making the following observation:
\begin{claim}\label{claim-mua-sm-instances}
Every deterministic mechanism that has approximation strictly better than $\nicefrac{6}{5}$ necessarily satisfies all of the following conditions:
\begin{enumerate}
    \item Given the valuation profile $I_1=(v_1^{one},v_2^{one})$, the mechanism
    allocates at least one item to every bidder. \label{condi-1}
    % $A$ outputs an allocation with welfare at most $1$. 
    \item Given the valuation profile $I_2 = (v_1^{\text{all}}, v_2^{\text{one}})$, the mechanism allocates all items to bidder $1$.
    \label{condi-3}
\item Given the valuation profile $I_3 = (v_1^{\text{ONE}}, v_2^{\text{ALL}})$, the mechanism allocates all items to bidder $2$. \label{condi-2}
\item Given the valuation profile $I_4 = (v_1^{\text{one}}, v_2^{\text{all}})$, the mechanism allocates all items to bidder $2$. 
\item Given the valuation profile $I_5 =(v_1^{\text{ALL}}, v_2^{\text{ONE}})$, the mechanism allocates all items to bidder $1$. \label{condi-5}
\end{enumerate}
\end{claim}
The proof of \cref{claim-mua-sm-instances} is straightforward: if a deterministic mechanism violates one of the conditions, then since $k$ is arbitrarily large, then it  extracts at most $\frac{5}{6}$ of the optimal welfare in expectation over the distribution $\mathcal D$. 


Fix a deterministic mechanism $A$ and strategies
$(\mathcal S_1,\mathcal S_2)$ that are individually rational and satisfy no negative transfers with respect to the valuations $V_1\times V_2$ 
and give approximation better than $\frac{6}{5}$
in expectation over the valuation profiles in the distribution $\mathcal D$. 
Let  $(f,P_1,P_2)$ be the allocation and payment rules that the mechanism $A$ and the strategies $(\mathcal S_1,\mathcal S_2)$ jointly realize. Assume towards a contradiction that $A$ and $(\mathcal S_1,\mathcal S_2)$ are OSP. 



To analyze the mechanism, we focus on the following subsets of the domains of the valuations:
$
\mathcal{V}_1=\{v_1^{one},v_1^{ONE},v_1^{ALL}\}$ and  $\mathcal{V}_2=\{v_2^{one},v_2^{ONE},v_2^{ALL}\}$.\footnote{The cautious reader may have noticed that $\mathcal V_i$ does not contain  $v_i^{all}$. This is intentional, and it will be clear from the remainder of the proof why including this valuation is not necessary.} 
We begin by observing that there necessarily exists a vertex $u$, and valuations $v_1,v_1' \in \mathcal{V}_1$, and  $v_2,v_2' \in \mathcal{V}_2$ such that $(\mathcal{S}_1(v_1), \mathcal{S}_2(v_2))$ diverge at vertex $u$. This follows from \cref{claim-mua-sm-instances}, which implies that the mechanism $A$ must output different allocations for different valuation profiles in $\mathcal{V}_1 \times \mathcal{V}_2$. Consequently, not all valuation profiles end up in the same leaf, meaning that divergence must occur at some point. 
% given the valuation profiles $I_1=(v_1^{one},v_2^{one})$ and
% either bidder $1$ or $2$ has to send different messages for different valuations in $\mathcal V_i$ at some vertex. This is an immediate implication of \cref{claim-mua-sm-instances}, as 
% the mechanism $A$ necessarily outputs different allocations given the valuation profiles $I_1=(v_1^{one},v_2^{one})$ and
%  $I_2=(v_1^{all},v_2^{one})$, meaning that the behaviors 
%  % \dnote{Missing parentheses here?}
%  $(\mathcal S_1(v_1^{one}),\mathcal S_2(v_2^{one}))$ and $(\mathcal S_1(v_1^{all}),\mathcal S_2(v_2^{one}))$ reach different leaves and thus have to diverge at some point.  
 
 Let $u$ be the first vertex in the protocol such that 
 %the behavior profiles 
 $(\mathcal{S}_1(v_1),\mathcal{S}_2(v_2))$ and $(\mathcal{S}_1(v_1'),\mathcal{S}_2(v_2'))$ diverge, i.e., dictate different messages. 
Note that by definition this implies that $u\in Path(\mathcal{S}_1(v_1),\mathcal{S}_2(v_2))\cap Path(\mathcal{S}_1(v_1'),\mathcal{S}_2(v_2'))$ and that either bidder $1$ or bidder $2$ sends different messages for the valuations in $\mathcal{V}_1$ or $\mathcal V_2$, respectively. 
Without loss of generality, we assume that bidder $1$ sends different messages, meaning that there exist $v_1,v_1'\in \mathcal{V}_1$ such that $\mathcal S_1(v_1)$ and $\mathcal S_1(v_1')$ dictate different messages at vertex $u$.  
We remind that $\mathcal{V}_1=\{v_1^{one},v_1^{ONE},v_1^{all}\}$, so the  following claims jointly imply a contradiction, completing the proof: 
\begin{claim}\label{claim-oneone-same}
    The strategy $\mathcal S_1$ dictates the same message at vertex $u$ for the valuations $v_1^{one}$ and $v_1^{ONE}$. 
\end{claim}
\begin{claim}\label{claim-ONE-ALL-same}
        The strategy $\mathcal S_1$ dictates the same message at vertex $u$ for the valuations $v_1^{ONE}$ and $v_1^{ALL}$.
\end{claim}
% We defer the proofs of \cref{claim-oneone-same,claim-ONE-ALL-same} to the full version.
% \cref{subsec:mua-sm-claims-proofs}. 
We include the proofs for the sake of completeness, {but note that} they are identical to the proofs provided in \cite{Ron24}.
% The proofs are are based on the properties of the mechanism: its approximation guarantee, obvious strategy-proofness, individual rationality and no-negative-transfers.  
The proofs make use of the following lemma, which is a collection of observations about the allocation and the payment scheme of player $1$:    
\begin{lemma}\label{lemma-small-pay}
    The allocation rule $f$ and the payment scheme $P_1$ of bidder $1$ satisfy that:
    % Let $f$ be an allocation rule and let $P_1$ be the payment scheme of bidder $1$ that 
    % The allocation rule $f$ and the  payment scheme $P_1$ $(f,P_1,\ldots,P_n):V_1\times \cdots \times V_n\to \mathbb{R}^{n}$
    % are realized by a dominant-strategy, individually rational and no-negative-transfers mechanism. Then: 
    \begin{enumerate}
        \item Given $(v_1^{one},v_{2}^{one})$, bidder $1$ wins at least one item and pays at most $1$.  \label{item-1}
        \item  Given $(v_1^{ONE},v_2^{ALL})$, bidder $1$ gets the empty bundle and pays zero.   \label{item-2}
        \item Given $(v_1^{ALL},v_2^{one})$, bidder $1$ wins all the items and pays at most $k^2$. \label{item-3}  
    \end{enumerate}
\end{lemma}
The lemma is a direct consequence of the approximation guarantees of the mechanism, together with the fact that it is obviously strategy-proof and satisfies individual rationality and no negative transfers.  We use \cref{lemma-small-pay} 
 now and defer the proof to \cref{subsec::proof-lemma-small-pay}.
\begin{proof}[Proof of \cref{claim-oneone-same}]
     Note that by Lemma \ref{lemma-small-pay} item \ref{item-1}, $f(v_1^{one},v_2^{one})$ allocates at least one item to player $1$ and $P_1(v_1^{one},v_2^{one})\le 1$. Therefore:
\begin{equation}\label{eq-good-leaf1}
 v_1^{ONE}(f(v_1^{one},v_2^{one}))-P_1(v_1^{one},v_2^{one})\ge k^2   
\end{equation}
 In contrast, by part \ref{item-2} of Lemma \ref{lemma-small-pay},   $f(v_1^{ONE},v_2^{ALL})$ allocates no items to player $1$ and $P_1(v_1^{ONE},v_2^{ALL})=0$, so:
 \begin{equation}\label{eq-bad-leaf1}
 v_1^{ONE}(f(v_1^{ONE},v_2^{ALL}))-P_1(v_1^{ONE},v_2^{ALL})= 0   
\end{equation}
Combining inequalities (\ref{eq-good-leaf1}) and (\ref{eq-bad-leaf1}) gives:
\begin{align*}
  v_1^{ONE}(f(v_1^{ONE},v_2^{ALL}))-P_1(v_1^{ONE},v_2^{ALL})< 
  v_1^{ONE}(f(v_1^{one},v_2^{one}))-P_1(v_1^{one},v_2^{one})  
\end{align*}
We remind that vertex $u$ belongs in $Path(\mathcal S_1(v_1^{one}),\mathcal S_2(v_2^{one}))$ and also in
$Path(\mathcal{S}_1(v_1^{ONE}),\allowbreak\mathcal{S}_2(v_2^{ALL}))$. Therefore, Lemma \ref{lemma-bad-leaf-good-leaf} gives that the strategy $\mathcal S_1$ dictates the same message for  $v_1^{one}$ and $v_1^{ONE}$ at vertex $u$. 
% See Figure \ref{subfig-1} for an illustration.
\end{proof}
\begin{proof}[Proof of \cref{claim-ONE-ALL-same}]
    Following the same approach as in the proof of Claim \ref{claim-oneone-same}, note that by \cref{lemma-small-pay} \cref{item-3}: 
\begin{equation}\label{break-align}
v_1^{ONE}(f(v_1^{ALL},v_2^{one}))-P_1(v_1^{ALL},v_2^{one}) \ge k^2+1-k^2     
\end{equation}
Also, by \cref{lemma-small-pay}  \cref{item-2}:
\begin{equation}\label{break-align2}
  v_1^{ONE}(f(v_1^{ONE},v_2^{ALL}))  
-P_1(v_1^{ONE},v_2^{ALL})=0   
\end{equation}
Combining \cref{break-align} and \cref{break-align2}:
\begin{equation*}
    v_1^{ONE}(f(v_1^{ONE},v_2^{ALL}))  
-P_1(v_1^{ONE},v_2^{ALL})<  v_1^{ONE}(f(v_1^{ALL},v_2^{one}))-P_1(v_1^{ALL},v_2^{one})
\end{equation*}
% where the first inequality is by Lemma \ref{lemma-small-pay} part \ref{item-3} and the equality is by Lemma \ref{lemma-small-pay} part \ref{item-2}.
Given the above inequality with the fact that 
vertex $u$ belongs in $Path(\mathcal S_1(v_1^{ALL}),\mathcal S_2(v_2^{one}))$ and in
$Path(\mathcal{S}_1(v_1^{ONE}),\mathcal{S}_2(v_2^{ALL}))$,
Lemma \ref{lemma-bad-leaf-good-leaf}
implies  
% applying Lemma \ref{lemma-bad-leaf-good-leaf} gives
that the strategy $\mathcal S_1$ dictates the same message for the valuations $v_1^{ONE}$ and $v_1^{ALL}$ at vertex $u$. 
\end{proof}


\end{proof}
