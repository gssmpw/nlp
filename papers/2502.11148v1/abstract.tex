We investigate the problem of designing \emph{randomized} obviously strategy-proof (OSP) mechanisms in several  canonical  auction settings. 
Obvious strategy-proofness, introduced by Li \cite{li}, strengthens the well-known concept of dominant-strategy incentive compatibility (DSIC). Loosely speaking, it ensures that even agents who struggle with contingent reasoning can identify that their dominant strategy is optimal.



Thus, one would hope to design OSP mechanisms with good approximation guarantees.
Unfortunately, 
deterministic OSP mechanisms fail to achieve an approximation better than $\min\{m,n\}$ where $m$ is the number of items and $n$ is the number of bidders, even 
for the simple settings of additive and unit-demand bidders \cite{Ron24}. 
We circumvent these impossibilities
by showing that randomized
mechanisms that are obviously strategy-proof in the universal sense obtain a \emph{constant} factor approximation for these classes. 
{We show that this phenomenon occurs also for the setting of a multi-unit auction with single-minded bidders.}
Thus, our results provide a more positive outlook on the design of OSP mechanisms and exhibit a stark separation between the power of randomized and deterministic OSP mechanisms.

%To complement the picture, we provide impossibilities for  randomized OSP mechanisms in each setting. This further demonstrates that OSP mechanisms are significantly weaker than dominant-strategy mechanisms: it is well known that the deterministic VCG mechanism outputs an optimal allocation in dominant-strategies, whereas we show that even randomized OSP mechanisms cannot obtain more than $87.5\%$ of the optimal welfare.   

To complement the picture, we provide impossibilities for randomized OSP mechanisms in each setting. While the deterministic VCG mechanism is well known to output an optimal allocation in dominant strategies, we show that even randomized OSP mechanisms cannot obtain more than $87.5\%$ of the optimal welfare. This further demonstrates that OSP mechanisms are significantly weaker than dominant-strategy mechanisms.

% In addition, we explore the power of randomized OSP mechanisms in a multi-unit auction, 

% and that the same holds a
% provide analogous results for a multi-unit auction with unknown single-minded bidders. We show that in these environments as well, a trivial ascending auction on the grand bundle is optimal.