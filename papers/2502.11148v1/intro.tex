% \sre{Heads up --- colored comments are in file!!!!!!!!!}
Economics is the science of how to allocate scarce resources to several competing parties. In particular, auctions serve as a useful playground to understand who should get what and for what price.  
We assume a {good-willed} central planner who aims to allocate the resources in a way that maximizes the social welfare of all parties involved.  To achieve that, she has to overcome the following obstacle: the information of bidders is private and they are 
 interested in maximizing their own utility. Therefore, she must carefully design the elicitation mechanism to align the incentives of the agents with her own objective.
 

%"The most obvious disadvantages of the design are the complexity of the problem it
%poses to bidders, the reluctance of bidders to reveal their values, and the strategic issues
%posed by budget constraints." -- the lovely and lonely}  

%In particular, many modern markets are orgnanized as auctions  are often 
%Auction theory concerns the allocation of scarce resources among


%\pinksout{Arguably the most fundamental question in mechanism design and multi-agent resource allocation is how one should allocate scarce resources to self-interested agents aiming to maximize efficiency.}  
For a specific example, consider the  setting where an auctioneer 
%\pinksout{(the central planner)}
wants to allocate a single item among a set $\bidders$ of  bidders.  
Each bidder $i$ has a value $v_i$ for receiving the good.  To maximize social welfare, the auctioneer should give the item to the bidder of highest value, i.e., $i^* = \text{argmax}_{i \in \bidders}{v_{i}}$.  %\pinksout{As these values are the \emph{private information} of the bidders} 
For that, the auctioneer must design a mechanism which collects
information about the value of the bidders
and decides which bidder wins
%\pinksout{will receive} 
the good and the payment $p_i \geq 0$ of each bidder.
%i \in \bidders$.  %\pinksout{Each bidder, however, wants to maximize her own \emph{utility}, i.e., the value a bidder receives from an outcome minus what she pays to the auctioneer, and can strategically report her private information to the auctioneer if it would be to her benefit.  The auctioneer must, thus, carefully design the elicitation mechanism in order to align the incentives of the bidders with her own objective.} \shirinote{I really like the last phrase, I moved it to be in the first paragraph}

The classic solution proposed for this 
%single-item 
setting is the \emph{sealed-bid} second-price auction, wherein bidders report their values directly to the auctioneer and the highest-valued bidder is awarded the good at the second highest price \cite{Vic61}.  It is well-known that this auction is \emph{dominant-strategy incentive compatible} (i.e., \emph{strategy-proof}), meaning that each bidder maximizes her utility by truthfully reporting her private value regardless of the reports of the other bidders.  Theory suggests, therefore, that bidders should never misreport their value in this auction.  However, in practice, 
%even in this extremely simple setting, 
``real-world'' bidders report bids not equal to their true value \cite{KHL1987}.  Thus, there appears to be a mismatch between the prediction of the theory of strategy-proof mechanisms and the observed outcomes. 

An alternative to the sealed-bid second-price auction for the single-item setting is the ascending price (Japanese) auction.  In this auction, a price clock gradually increases over time and bidders drop out whenever the asking price becomes too high.  The ascending price auction implements the exact same outcome as the sealed-bid auction: it awards the item to the highest-valued bidder at the second-highest value. However, bidders appear empirically  more likely to follow their optimal truthful strategy when facing an ascending-price auction compared to the sealed-bid format.

To address this discrepancy, Li introduced the notion of \emph{obvious strategy-proofness} (OSP), a strengthening of strategy-proofness \cite{li}. Loosely speaking, an OSP mechanism ensures that even agents unable to perform contingent reasoning can recognize truth-telling as the optimal strategy. OSP provides a theoretical explanation for the prevalence of ascending auctions over sealed-bid implementations, by claiming that their popularity stems from the fact that they are simpler for bidders to understand than sealed-bid implementations.

% To address this discrepancy, %in a pioneering work 
% Li has defined the notion of \emph{obvious strategy-proofness} (OSP), a strengthening of strategy-proofness \cite{li}. Loosely speaking, if a mechanism is OSP then even agents unable to perform contingent reasoning can identify that truth-telling is the optimal strategy. Obvious strategy-proofness provides a theoretical explanation for the prevalence of the ascending auction, by claiming that its popularity stems from the fact that it is simpler for bidders to understand than the sealed-bid
% format. 
%second-price auction.

Since its introduction by \cite{li}, obvious strategy-proofness has emerged as a ``gold standard'' for strategic simplicity in mechanism design and the notion has attracted a great deal of attention.  For instance, various refinements and relaxations of OSP have been proposed (e.g., \cite{pycia2023theory,nagel2023measure,FV23}) and the design of OSP mechanisms has been examined various settings, including, e.g., one-sided matching \cite{troyan2019obviously} and two-sided  matching  markets \cite{AG18,Thomas21}, scheduling \cite{ferraioli2019obviously}, voting problems \cite{BG17,arribillaga2020obvious} and allocation problems \cite{BG17}.  Another line of work has aimed to find \emph{characterizations} of OSP mechanisms in various domains.  For instance, \cite{li} showed that for all single-parameter binary allocation settings the class of OSP mechanisms coincides with the class of ``personalized clock auctions'' (essentially, a natural generalization of the Japanese auction for a single item), and further work has  established characterizations for all linear single-parameter domains \cite{carmine1,carmine2} and beyond \cite{pycia2023obviously,Mackenzie20}.  Using these characterizations, both lower bounds (impossibility results) and upper bounds (mechanisms with proven guarantees) have been proposed for various single-parameter auction settings, including binary allocation with general feasibility constraints \cite{DGR14,CGS22,FGGS22}  and procurement  settings \cite{BGGST22}.  

Beyond single-parameter settings, the picture regarding the performance of OSP mechanisms in auctions becomes somewhat pessimistic.
\cite{BG17} initially showed that, for additive bidders, no obviously strategy-proof mechanism optimizes the welfare,
and recent work of \cite{Ron24} has, essentially, ``closed the book'' on deterministic OSP mechanisms for multi-parameter combinatorial auction settings.  Even if bidders have additive or unit-demand valuation functions (which are commonly thought to be  ``easy''), the trivial OSP mechanism which 
runs an ascending-price auction for the grand bundle achieves the \emph{best possible} approximation guarantee of $\min\{m,n\}$, where $m$ is the number of items and $n$ is the number of bidders. 
To  circumvent these strong impossibilities, we turn our attention to \emph{randomized} OSP mechanisms.
 

\subsection{Our Results}
Our main results are upper bounds and lower bounds for randomized OSP mechanisms.
We focus our attention on ``universally'' OSP mechanisms, i.e., mechanisms which are a distribution over deterministic OSP mechanisms.
We analyze all the settings considered by 
 \cite{Ron24} and  show that:
\begin{enumerate}
    \item For additive bidders in a combinatorial auction, there is a mechanism that obtains a $4$ approximation and
    no mechanism has approximation better than $\frac{8}{7}\approx 1.14$  (\cref{thm-ub-add} and \cref{thm-lb-add}).
\item For unit-demand bidders in a combinatorial auction, 
there is a mechanism
    that obtains an $e\approx 2.72$ approximation and
no mechanism has approximation better than
than $\frac{8}{7}\approx 1.14$ (\cref{thm:ud-upper} and \cref{thm-lb-ud}). 
    \item For single-minded bidders in a multi-unit auction with unknown demands, 
    %\sre{In particular, our results hold for bidders which are unknown single-minded, meaning that both their demand and their value for their desired set is private information,},
    there is a mechanism that obtains a $400$ approximation and no mechanism has approximation better than $1.2$ (\cref{thm-ub-mua-sm} and \cref{thm-mua-sm-lb}). 
\end{enumerate}
All the impossibilities %presented
are for mechanisms that satisfy individual rationality and no negative transfers. Likewise,
%Correspondingly,
our proposed mechanisms
% all of the mechanisms that we propose 
conform to these conditions. 


Observe that our upper bounds demonstrate the power of randomization for obviously strategy-proof mechanism design:
whilst deterministic OSP mechanisms 
can only obtain an approximation of 
$\{m,n\}$ to the optimal welfare \cite{Ron24}, 
 % whilst \cite{Ron24} shows that only an approximation of $\{m,n\}$ can be obtained for deterministic obviously strategy-proof mechanisms, 
all these classes admit a randomized OSP mechanisms with a  constant factor approximation. 
%to the optimal welfare. 
In addition, we observe that the randomized $poly(m)$-communication mechanisms that are
dominant-strategy incentive compatible and obtain the state of the art approximation guarantees for \textquote{richer} classes of valuations in combinatorial auctions are in fact  obviously strategy-proof (see \cref{cl:subadditive,cl:general}).\footnote{We also provide a $400$ approximation to the optimal welfare for
multi-unit auctions with bidders whose valuations satisfy decreasing marginal utilities (\cref{thm:decreasing-marginals}). This is the only multi-parameter domain for which the power of deterministic mechanisms is not known. In \cref{subsub::non-mono}, 
we describe a non-monotonicity effect that illustrates a barrier towards proving impossibilities for this class.}
% provide additional results in 
% % the full version \sre{TODO}
% % full-version-change-tag%

% which illustrate the difficulty of proving lower bounds for this class.}  


{Our upper bounds are motivated by the following observation:
%regarding the lower bound 
the constructions of \cite{Ron24}
% is that 
show 
%They are based on the fact 
that
every  mechanism that provides a non-trivial approximation to the welfare satisfies that the first bidder that \textquote{speaks} in the mechanism does not have an obviously dominant strategy. The underlying cause of this  phenomenon is that when querying a bidder for the first time, the mechanism fails because it has no information regarding the valuations of the other bidders. 
% does not have 
% does not have obviously dominant strategies. The underlying cause of this  phenomenon is that when querying a bidder for the first time, the mechanism has no information regarding the valuations of the other bidders. Thus, this bidder has to 
Thus, to overcome this impossibility, our proposed mechanisms are based on the classic secretary approach of sampling a sufficient fraction of the bidders and aggregating their information to determine
a price per item. Owing to the use of randomization, this can be done 
in an obviously dominant manner while maintaining a high fraction of the welfare in expectation.}  

% This motivates the sampling phase, which allows us to 

% The sampling phase, which
% can be seen as randomizing the identity of that bidder, 
% we use in all of our algorithms, is key since 
% approach, which we use in our 
% We overcome this impossibility by essentialy randomize the identity of that player, 
% is pivotal, since by  by randomizing who that first bidder is, we manage to extract high welfare.}

Our lower bounds for combinatorial auctions with unit-demand and additive bidders further emphasize 
the restrictiveness of obvious strategy-proofness compared to implementation in dominant strategies. 
Not only getting an approximation better than $\min\{m,n\}$ is impossible deterministically, but even if we allow randomization we cannot get more than $87.5\%$ of the optimal welfare. In contrast, these settings  have  dominant-strategy mechanisms that extract the optimal welfare and are also efficient both from a computation and a communication perspective.
One disadvantage of our main results is that the lower bounds and upper bounds that we provide are quite far apart. As a step to bridge this gap, we show in %full-version-change-tag%
% the full version
\cref{sec-22} 
that for two bidders and two items, all the aforementioned classes
admit mechanisms that give a $\frac{4}{3}$ approximation.  
% We leave open the question of resolving the approximation power of randomized OSP mechanisms with arbitrary number of items and bidders. 
 % We conclude by showing a separation between obviously strategy-proof mechanisms and weakly group strategy-proof mechanisms, which can be found in the full version.
%full-version-change-tag%
% (\cref{app:wgsp}). 
 % We also provide a randomized OSP mechanism that gives a $400$ approximation to a multi-unit auction with decreasing marginal utilities (\cref{app:wgsp}). 


\subsection{Why Randomization?}
On first impression, one might argue that randomization adds impractical complexity to a mechanism, and does not align with the simplicity we aim to achieve when designing OSP mechanisms.  Indeed,  ``real-world'' agents could possibly be confused by randomization. 
%to be 
%and, 
Moreover, it can be difficult to verify that an outcome is the result of some pre-specified random process.  We emphasize, however, that the randomization at use in our work is rather ``straightforward'' in the sense that bidders do not reason about 
%expected utilities or 
expected outcomes since, on any fixed result of the random process, they face a mechanism where they have an obviously dominant strategy.
% , and, crucially, as we discussed above, bidders in our auctions need not reason about 
%expected utilities or 
% expected outcomes since, on any fixed result of the random process, they face a mechanism where they have an obviously dominant strategy. 

Moreover, randomized mechanisms are prevalent in practice, e.g., in drafts in sports for new team members, housing programs, and ``greencard'' allocation, and well-studied elsewhere in theory, in particular in fair division (see \cite{budish2013designing} among many others). 
In fact, designing fair mechanisms without randomization is challenging, as the order of player selection often plays a crucial role—bidders who are selected earlier tend to have an advantage over those selected later. Randomizing the order of choice is a natural solution, and there appears to be no alternative to this approach.
%In fact,  it is not clear  how to design fair mechanisms without randomization, as the order of selection of players often plays a crucial role. Randomizing the order of choice is then a natural solution, and it is unclear whether there are any alternatives to this approach.


The power of randomization for welfare maximization is,  despite a great deal of work, still not fully understood. If we put aside complexity considerations, then the deterministic VCG mechanism is optimal, so randomized dominant-strategy mechanisms do not \textquote{beat} their randomized counterparts.  
% randomized dominant-strategy mechanisms cannot be more powerful  there is no separation for dominant-strategy mechanisms. 

In contrast, 
 if we require communication-efficient mechanisms, i.e. mechanisms with number of bits that is polynomial in $m$ and $n$ in the worst case, then 
 %for the setting of dominant-strategy mechanisms for bidders with arbitrary monotone valuations in a combinatorial auctions,
 randomized dominant-strategy mechanisms achieve an approximation of $\mathcal O(\sqrt m)$ when the bidders have arbitrary monotone valuations, whilst every deterministic mechanism that achieves approximation better than $\mathcal O(m^{1-\epsilon})$ has exponential communication \cite{dobzinski2012truthful,DRV22}. If we settle for mechanisms that only satisfy the weaker notion of ex-post incentive compatibility\footnote{A mechanism is ex-post incentive compatible if it has strategies that form a Nash equilibrium, in contrast to dominant-strategy mechanisms where the bidders have dominant strategies.}, then
the best deterministic mechanisms lag behind their randomized counterparts \cite{QW24,AS19,assadi2021improved}, but no such separation is known. Our work, in contrast, demonstrates such a separation between deterministic and randomized OSP mechanisms.
% for welfare maximization if we strengthen the incentive property to OSP. %obvious strategy-proofness.






