\begin{table}[t]
\centering

\renewcommand{\arraystretch}{0.95} % Adjust the row height factor
    \resizebox{1.0\linewidth}{!}{
    \begin{tabular}{m{4cm} c c cc}
    \toprule
    \multirow{2}[2]{*}{Method} & 
    \multirow{2}[2]{*}{\makecell{$N$-shot Scenario}} & \multirow{2}[2]{*}{RMSE} & \multirow{2}[2]{*}{MAE} \\
    & & & \\ [2pt]
    \midrule
    \addlinespace[4pt] 
    % \midrule [2pt] -> line strength
    VPP4DC & 0  & 0.247 & 0.077\\ [4pt]
    DepthPrompting & 1  & 0.358 & 0.206\\ [2pt]
     & 10  & 0.220 & 0.101\\ [4pt]
    UniDC & 1  & 0.210 & 0.107 \\ [2pt]
     & 10  & 0.166 & 0.079 \\ [4pt]
     \midrule 
     \addlinespace[4pt] 
    Ours (+Marigold) & 0  & 0.149 & \textbf{0.059} \\[2pt]
    Ours (+DepthFM) & 0  & \textbf{0.145} & 0.077\\ [2pt]
    \bottomrule
    \end{tabular}
    }
   \vspace{-3pt}
\caption{\textbf{Quantitative comparison with depth-prior-based methods  on the NYU test set.} We compare our method with zero- and few-shot approaches leveraging various depth foundation models.
}
\label{tab:depthprior}
\vspace{-15pt}
\end{table}
