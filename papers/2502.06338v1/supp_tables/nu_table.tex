\begingroup
\setlength{\tabcolsep}{11pt}
\begin{table}[t]
\centering
\renewcommand{\arraystretch}{1.4} % Adjust the row height factor
    \resizebox{1.0\linewidth}{!}{
    \begin{tabular}{m{2.5cm} cc cc }
    \toprule
    \multirow{2}[2]{*}{Method} & \multicolumn{2}{c}{PCACC} & \multicolumn{2}{c}{ProxyTTA}\\
     \cmidrule(lr){2-3}  \cmidrule(lr){4-5} 
    & RMSE & MAE & RMSE & MAE \\
    \midrule 
     Pre-trained %\footnotemark[2]
     & 3.998 & 1.967 & 6.630 & 3.064 \\ 
     BNAdapt
     & 1.801 & 0.828 & 6.391 & 2.306\\ 
     CoTTA
     & 2.668 & 1.222 & 6.099 & 2.676\\ 
     ProxyTTA
     & 1.755 & 0.799 & 5.509 & 2.062\\ 
    \midrule
    Ours
    & 1.516 & 0.561 & 5.876 & 2.499\\
    \bottomrule
    \end{tabular}
    }
\caption{\textbf{Quantitative results on the nuScenes depth completion benchmarks.} We evaluate our method on the nuScenes dataset using both PCACC~\cite{huang2022pcacc} and ProxyTTA~\cite{park2024testtime} ground truth datasets, employing the CostDCNet~\cite{kam2022costdcnet} model pre-trained on KITTI DC. Excluding ours, other test-time adaptation methods are adapted with CostDCNet. Our method shows favorable performance on the outdoor dataset and demonstrates domain generalizability, by evaluating on the more physically accurate benchmark.}
\label{tab:nu_table}
\vspace{-5pt}
\end{table}
\endgroup
