\definecolor{tabfirst}{rgb}{1, 0.7, 0.7} % red
\definecolor{tabsecond}{rgb}{1, 0.85, 0.7} % orange
\definecolor{tabthird}{rgb}{1, 1, 0.7} % yellow


\begingroup
\begin{table}[!t]
\centering
\renewcommand{\arraystretch}{1.1} % Adjust the row height factor
    \resizebox{1.0\linewidth}{!}{
    \begin{tabular}{m{2.6cm} ccc}
    \toprule
    \multirow{2}{*}{Filtering Method} & \multirow{2}{*}{In-domain} & {Sparse} & {Semi-dense}\\
    \cmidrule(lr){3-3}  \cmidrule(lr){4-4}
    & & AUC~($\downarrow$) & AUC~($\downarrow$)\\
    % Filtering Method & AUC (RMSE) & AUC (MAE)\\
    \midrule
    None & \xmark & 1.3541 & 2.5441  \\
    Window Filter & \xmark & \cellcolor{tabsecond}0.3480 & 0.9629  \\ 
    Ours & \xmark & \cellcolor{tabfirst}{0.2959} & \cellcolor{tabsecond}0.5103  \\ 
    \midrule
    Lidar Confidence & \cmark & 1.0521  & \cellcolor{tabfirst}{0.2117} \\
    \bottomrule
    \end{tabular}
    }
\caption{\textbf{Quantitative evaluation of outlier filtering.}
Our prior-based outlier filtering method demonstrates favorable
% impressive
performance compared to existing methods.
In this table, ``None" denotes a synchronized sparse
and accumulated depth map
without any postprocessing.
Note that the ``Lidar Confidence'' method is a learning-based method trained on the same domain dataset with the evaluation dataset.
}
\label{tab:noise_filter}
\vspace{-4mm}
\end{table}
