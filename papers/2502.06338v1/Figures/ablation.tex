\begin{figure}[!ht]
\centering
\begin{minipage}{\linewidth}
    \centering
    % Table
    \renewcommand{\arraystretch}{1.3} % Adjust the row height factor
    \resizebox{1.0\linewidth}{!}{
        \begin{tabular}{ccc  cc cc}
        \toprule
        \multirow{2}[2]{*}{\makecell{Sampling \\ Method}} &\multirow{2}[2]{*}{\makecell{R-SSIM \\ Loss}} & \multirow{2}[2]{*}{\makecell{Outlier \\ Filtering}} & \multicolumn{2}{c}{KITTI DC} & \multicolumn{2}{c}{VOID} \\
        \cmidrule(lr){4-5} \cmidrule(lr){6-7} 
         & & & RMSE & MAE & RMSE & MAE\\
        \midrule
        Naïve & & & 3.514 & 1.942 & 0.199 & 0.130\\
        Guided & & & 2.113 & 0.801 & 0.210 & 0.138\\ 
        Ours & & & 1.610 & 0.406 & 0.125 & 0.046 \\
        Ours & {\cmark} & & 1.502 & 0.409 & 0.111 & 0.044 \\
        Ours & {\cmark} & {\cmark} & 1.413 & 0.397 & 0.112 & 0.045\\
        \bottomrule
        \end{tabular}
    }
    \vspace{-3pt}
    \caption{\textbf{Ablation studies.} We ablate our proposed methods including test-time alignment, R-SSIM loss, and prior-based outlier filtering, to demonstrate their effectiveness.}
    \label{tab:ablation_method}
\end{minipage}

\vspace{10pt} 

\begin{minipage}{\linewidth}
    \centering
    % Figure
    \includegraphics[width=0.95\linewidth]{Figures/r-ssim_arxiv.pdf}
    \vspace{-3pt}
    \caption{\textbf{Qualitative ablation of R-SSIM loss.} This structural regularization sharpens details in areas such as signposts and car shapes.} 
    \vspace{-10pt}
    \label{fig:r-ssim}
\end{minipage}
\end{figure}
\vspace{-5pt}