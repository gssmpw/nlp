\section{Conclusion}
% We propose a novel prior-based zero-shot depth completion method 
% % Prior를 하다는거.
% % test-time alignment.참조. 
% that aligns the sparse measurements with the prior of a pre-trained depth diffusion model at the test time.
% the first study demonstrating the importance of prior knowledge and how to leverage depth prior to achieve generalizable depth completion, addressing the challenge of domain shifts.
% Our test-time alignment method ensures that the completed depth map is consistent with sparse measurement while considering the structural and depth affinity of the scene derived from the depth prior.
% Our approach not only performs well on any single data but also enhances the performance of depth completion across various domains, capturing both fine-grained details and the global context of the scene.
% We hope this work will be regarded as a milestone in addressing the challenge of domain-generalizable depth completion and that our exploration of leveraging prior knowledge with sparse cues will inspire future research.

% We propose a novel prior-based zero-shot depth completion, the first study demonstrating the importance of prior knowledge in addressing the challenge of domain shifts.
% Our approach aligning with measurements at test time, \ie test-time alignment, ensures that the completed depth map is consistent with sparse measurement while incorporating the structural and depth affinity of the scene derived from the depth prior.
% Additionally, our proposed prior-based outlier filtering algorithm ensuring the more reliable measurements, can be applicable to methods using sparse depth supervision.
% This prior-based approach enhances the performance of depth completion across various domains, capturing both fine-grained details and the global context of the scene.
% We believe this work marks a significant step toward domain-generalizable depth completion and our exploration of leveraging prior knowledge with sparse cues will inspire future research.


We propose a novel prior-based zero-shot depth completion method, the first study demonstrating the importance of monocular depth prior knowledge in addressing the challenge of domain shifts.
Our test-time alignment approach ensures that the completed depth map remains consistent with sparse measurements while incorporating structural depth affinity of the scene derived from the depth prior.
% Our approach aligning with measurements at test time, \ie test-time alignment, ensures that the completed depth map is consistent with sparse measurement while incorporating the structural depth affinity of the scene derived from the depth prior.
% Also, 
% Additionally,
% our proposed prior-based outlier filtering algorithm ensuring more reliable measurements can be applicable to methods using sparse depth supervision.
This prior-based approach enhances the performance of depth completion across various domains, capturing the context of the scene.
We believe this work marks a significant step toward generalizable depth completion and our exploration of leveraging prior knowledge will inspire future work.


% foundational prior를 사용한 첫 work이다. 그러나 diffusion model을 사용해서 좀 느리다 ,쿠션멘트
% diffusion nature. 효과성을 확인했으니까, 다음 웍으로는 consistency model같은게 될 수 있을거같다. 우리가 확인했는데. 
%%%%%%%%%%%%%%%%%%%%%%%% Suppl : limitation %%%%%%%%%%%%%%%%%%%%%%%
\para{Limitation}
Our zero-shot depth completion is the first work to use monocular depth foundation model priors for generalizable depth completion, but it adopts the standard guided sampling approach in latent diffusion models, which may be slow to process.
% In this work, we demonstrate the effectiveness of the affine-invariant depth diffusion prior for depth completion.
% As a next step, accelerating this process through consistency models and developing effective priors using consistency models 
% could be promising directions.  
As a next step, accelerating this process building upon the recent advancements in the acceleration of diffusion model naïve~\cite{song2023consistency} and guided sampling~\cite{chung2022come} could be promising directions.
% Our test-time alignment method adopts the standard approach of guiding the sampling process of the latent diffusion model, which may lead to slower processing time.
% However, there have been recent advancements in the acceleration of diffusion model sampling~\cite{song2023consistency} and guiding sampling trajectory~\cite{chung2022come, dai2024motionlcm}, achieving real-time performance.
% Although we did not address latency in our current study, as explored in this line of work, we are excited about improving speed in a zero-shot manner for future work.
%%%%%%%%%%%%%%%%%%%%%%%%%%%%%%%%%%%%%%%%%%%%%%

% Consistency Distillation.
% Furthermore, unlike previous TTA approaches~\cite{wang2021tent, wang2022continual, park2024testtime}, which need fine-tunning that requires sufficient test-domain data and still faces the challenge in practical scenarios, our method does not necessitate fine-tuning on test domain dataset.
