\begin{figure*}[!ht]
    \centering
    \includegraphics[width=\linewidth]{figures-src/concept.pdf}
    \caption{SliderSpace decomposes the visual variation of diffusion model's knowledge corresponding to a concept. These directions can be perceived as interpretable directions of the model's hierarchical knowledge. We show the decomposed slider direction for a concept using SliderSpace and the corresponding labels generated by Claude 3.5 Sonnet.}
    \label{fig:concept}
\end{figure*}

\section{Method}
In this section, we introduce our zero-shot depth completion method, which leverages the depth prior~\cite{ke2023repurposing, gui2024depthfm}
derived from the foundation model~\cite{rombach2022highresolution}. This 
enables our method to be generalizable across any domain.
The core concept of our approach is to align the affine-invariant depth prior with sparse measurements on an absolute scale to complete the dense and well-structured depth map, as illustrated in \Fref{fig:concept}.

\subsection{Preliminary}
\label{sec:preliminary}
\para{Diffusion model and guided sampling}
Diffusion models~\cite{ho2020denoising, song2022denoising} aim to model data distribution $p(\mathbf{x})$ through iterative perturbation and restoration, known as forward and reverse processes.
This is represented by the score-based generative model~\cite{song2021scorebased}, learning the score function $\mathbf{s}_\theta$ parameterized by $\theta$ the gradient of the log probability density function with respect to the data, \ie, $\mathbf{s}_\theta(\mathbf{x}) =\nabla_{\mathbf{x}} \log p(\mathbf{x};\theta)$.
Score-based diffusion models 
estimate the score $\mathbf{s}_\theta(\mathbf{x}_t)$ at intermediate state $\mathbf{x}_t$ for timestep $t$ which defines a process.

For image generation and editing, diffusion models leverage the guidance function during the sampling process to adjust the output to the specific condition
~\cite{ho2022classifierfree, dhariwal2021diffusion}.
The guidance can be 
defined 
by any differentiable mapping output to guidance modality, as follows~\cite{bansal2024universal}: 
\begin{equation}
\label{eq:guide_sampling}
\hat{\mathbf{s}}_{\theta}(\mathbf{x}_t, t, \mathbf{y}) = \mathbf{s}_{\theta}(\mathbf{x}_t, t) + w \nabla_{\mathbf{x}_t} \mathcal{L}\left(f\left(\mathbf{x}_0\left(\mathbf{x}_t\right)\right),\mathbf{y}\right),
\end{equation}
where $w$ and $\mathbf{y}$ represent weight and guidance, respectively.
The function $f(\cdot)$ can be any differentiable function whose output can compute a loss $\mathcal{L}$ with guidance condition $\mathbf{y}$, and
$\mathbf{x}_0\left(\mathbf{x}_t\right)$ is obtained by using Tweedie's formula~\cite{efron2011tweedie}, 
which provides an approximation of the posterior mean. 
This guided sampling approach extends unconditional diffusion models to conditional ones without separate model training.

\para{Inverse problem}
The goal of an inverse problem is to determine an unknown variable from known measurement, often formulated as $\mathcal{A}(\mathbf{x})=\mathbf{y}$,
where $\mathcal{A}{:}\, \mathbb{R}^m {\rightarrow}  \mathbb{R}^n$ represents the known forward measurement operator, $\mathbf{y}\in \mathbb{R}^n$ and $\mathbf{x}\in \mathbb{R}^m$,
the measurement and the unknown variable, respectively.
When 
$m>n$, it becomes an ill-posed problem, requiring a prior to find 
solve a
Maximum A Posterior (MAP) estimation:
\begin{equation}
\label{eq:map}
    \argmax p(\mathbf{x}|\mathbf{y})\propto p(\mathbf{x}) p(\mathbf{y} | \mathbf{x}),
\end{equation}
where $p(\mathbf{x})$ represents our prior of the signal $\mathbf{x}$ and $p(\mathbf{y} | \mathbf{x})$ is likelihood measuring 
$\mathcal{A}(\mathbf{x})\approx\mathbf{y}$, \eg, $\|\mathbf{y} {-} \mathcal{A}(\mathbf{x})\|_2^2$.
By taking $-\log(\cdot)$ to \Eref{eq:map}, it can be  
formulated as an optimization problem
that regularizes the solution, ensuring that $\mathbf{x}$ follows the characteristics of the prior:
\begin{equation}
    \label{eq:inv_opt}
    \argmin_{\mathbf{x}} \left\|\mathbf{y} - \mathcal{A}\left(\mathbf{x}\right) \right\|_2^2 - \log p(\mathbf{x}).
\end{equation}
Also, given the gradient of $\log p(\mathbf{x}|\mathbf{y})$ in \Eref{eq:map} as
\begin{equation}
    \nabla_{\mathbf{x}}\log p(\mathbf{x}|\mathbf{y}) =  \nabla_{\mathbf{x}}\log p(\mathbf{x}) + \nabla_{\mathbf{x}}\log p(\mathbf{y} | \mathbf{x}),
\end{equation}
the prior term $\nabla_{\mathbf{x}}\log p(\mathbf{x})$ corresponds to the score $\mathbf{s}_{\theta}(\mathbf{x})$, which can be obtained by diffusion models.
Therefore, by simply adding the gradient of the likelihood term to the reverse sampling process, the inverse problem can be effectively solved while leveraging the diffusion prior~\cite{chung2023dps}
as follows:
\begin{equation}
    \label{eq:inv_sampling}
    \hat{\mathbf{s}}_{\theta}(\mathbf{x}_t, t, \mathbf{y}) = \mathbf{s}_{\theta}(\mathbf{x}_t, t) + w \nabla_{\mathbf{x}_t}\left\| \mathbf{y} - \mathcal{A}\left(\mathbf{x}_0\left(\mathbf{x}_t\right)\right) \right\|_2^2.
\end{equation}
This has an analogous form with \Eref{eq:guide_sampling}; thus, the inverse problem can be effectively tackled with the guided sampling.


With pre-trained image diffusion models, \eg, \citet{rombach2022highresolution}, as the score function $\mathbf{s}_{\theta}(\mathbf{x})$ and a prior, it provides 
powerful image prior across various tasks by its 
comprehensive semantic understanding and structural knowledge learned from a lot of images~\cite{wang2023exploiting, namekata2024emerdiff}.
\citet{ke2023repurposing} leverage this rich visual knowledge to achieve generalizable monocular depth estimation, resulting in high-quality outputs within an affine-invariant depth space. In our work, we exploit this depth diffusion model for computing the score as a depth prior.

\section{Method}\label{sec:method}
\begin{figure}
    \centering
    \includegraphics[width=0.85\textwidth]{imgs/heatmap_acc.pdf}
    \caption{\textbf{Visualization of the proposed periodic Bayesian flow with mean parameter $\mu$ and accumulated accuracy parameter $c$ which corresponds to the entropy/uncertainty}. For $x = 0.3, \beta(1) = 1000$ and $\alpha_i$ defined in \cref{appd:bfn_cir}, this figure plots three colored stochastic parameter trajectories for receiver mean parameter $m$ and accumulated accuracy parameter $c$, superimposed on a log-scale heatmap of the Bayesian flow distribution $p_F(m|x,\senderacc)$ and $p_F(c|x,\senderacc)$. Note the \emph{non-monotonicity} and \emph{non-additive} property of $c$ which could inform the network the entropy of the mean parameter $m$ as a condition and the \emph{periodicity} of $m$. %\jj{Shrink the figures to save space}\hanlin{Do we need to make this figure one-column?}
    }
    \label{fig:vmbf_vis}
    \vskip -0.1in
\end{figure}
% \begin{wrapfigure}{r}{0.5\textwidth}
%     \centering
%     \includegraphics[width=0.49\textwidth]{imgs/heatmap_acc.pdf}
%     \caption{\textbf{Visualization of hyper-torus Bayesian flow based on von Mises Distribution}. For $x = 0.3, \beta(1) = 1000$ and $\alpha_i$ defined in \cref{appd:bfn_cir}, this figure plots three colored stochastic parameter trajectories for receiver mean parameter $m$ and accumulated accuracy parameter $c$, superimposed on a log-scale heatmap of the Bayesian flow distribution $p_F(m|x,\senderacc)$ and $p_F(c|x,\senderacc)$. Note the \emph{non-monotonicity} and \emph{non-additive} property of $c$. \jj{Shrink the figures to save space}}
%     \label{fig:vmbf_vis}
%     \vspace{-30pt}
% \end{wrapfigure}


In this section, we explain the detailed design of CrysBFN tackling theoretical and practical challenges. First, we describe how to derive our new formulation of Bayesian Flow Networks over hyper-torus $\mathbb{T}^{D}$ from scratch. Next, we illustrate the two key differences between \modelname and the original form of BFN: $1)$ a meticulously designed novel base distribution with different Bayesian update rules; and $2)$ different properties over the accuracy scheduling resulted from the periodicity and the new Bayesian update rules. Then, we present in detail the overall framework of \modelname over each manifold of the crystal space (\textit{i.e.} fractional coordinates, lattice vectors, atom types) respecting \textit{periodic E(3) invariance}. 

% In this section, we first demonstrate how to build Bayesian flow on hyper-torus $\mathbb{T}^{D}$ by overcoming theoretical and practical problems to provide a low-noise parameter-space approach to fractional atom coordinate generation. Next, we present how \modelname models each manifold of crystal space respecting \textit{periodic E(3) invariance}. 

\subsection{Periodic Bayesian Flow on Hyper-torus \texorpdfstring{$\mathbb{T}^{D}$}{}} 
For generative modeling of fractional coordinates in crystal, we first construct a periodic Bayesian flow on \texorpdfstring{$\mathbb{T}^{D}$}{} by designing every component of the totally new Bayesian update process which we demonstrate to be distinct from the original Bayesian flow (please see \cref{fig:non_add}). 
 %:) 
 
 The fractional atom coordinate system \citep{jiao2023crystal} inherently distributes over a hyper-torus support $\mathbb{T}^{3\times N}$. Hence, the normal distribution support on $\R$ used in the original \citep{bfn} is not suitable for this scenario. 
% The key problem of generative modeling for crystal is the periodicity of Cartesian atom coordinates $\vX$ requiring:
% \begin{equation}\label{eq:periodcity}
% p(\vA,\vL,\vX)=p(\vA,\vL,\vX+\vec{LK}),\text{where}~\vec{K}=\vec{k}\vec{1}_{1\times N},\forall\vec{k}\in\mathbb{Z}^{3\times1}
% \end{equation}
% However, there does not exist such a distribution supporting on $\R$ to model such property because the integration of such distribution over $\R$ will not be finite and equal to 1. Therefore, the normal distribution used in \citet{bfn} can not meet this condition.

To tackle this problem, the circular distribution~\citep{mardia2009directional} over the finite interval $[-\pi,\pi)$ is a natural choice as the base distribution for deriving the BFN on $\mathbb{T}^D$. 
% one natural choice is to 
% we would like to consider the circular distribution over the finite interval as the base 
% we find that circular distributions \citep{mardia2009directional} defined on a finite interval with lengths of $2\pi$ can be used as the instantiation of input distribution for the BFN on $\mathbb{T}^D$.
Specifically, circular distributions enjoy desirable periodic properties: $1)$ the integration over any interval length of $2\pi$ equals 1; $2)$ the probability distribution function is periodic with period $2\pi$.  Sharing the same intrinsic with fractional coordinates, such periodic property of circular distribution makes it suitable for the instantiation of BFN's input distribution, in parameterizing the belief towards ground truth $\x$ on $\mathbb{T}^D$. 
% \yuxuan{this is very complicated from my perspective.} \hanlin{But this property is exactly beautiful and perfectly fit into the BFN.}

\textbf{von Mises Distribution and its Bayesian Update} We choose von Mises distribution \citep{mardia2009directional} from various circular distributions as the form of input distribution, based on the appealing conjugacy property required in the derivation of the BFN framework.
% to leverage the Bayesian conjugacy property of von Mises distribution which is required by the BFN framework. 
That is, the posterior of a von Mises distribution parameterized likelihood is still in the family of von Mises distributions. The probability density function of von Mises distribution with mean direction parameter $m$ and concentration parameter $c$ (describing the entropy/uncertainty of $m$) is defined as: 
\begin{equation}
f(x|m,c)=vM(x|m,c)=\frac{\exp(c\cos(x-m))}{2\pi I_0(c)}
\end{equation}
where $I_0(c)$ is zeroth order modified Bessel function of the first kind as the normalizing constant. Given the last univariate belief parameterized by von Mises distribution with parameter $\theta_{i-1}=\{m_{i-1},\ c_{i-1}\}$ and the sample $y$ from sender distribution with unknown data sample $x$ and known accuracy $\alpha$ describing the entropy/uncertainty of $y$,  Bayesian update for the receiver is deducted as:
\begin{equation}
 h(\{m_{i-1},c_{i-1}\},y,\alpha)=\{m_i,c_i \}, \text{where}
\end{equation}
\begin{equation}\label{eq:h_m}
m_i=\text{atan2}(\alpha\sin y+c_{i-1}\sin m_{i-1}, {\alpha\cos y+c_{i-1}\cos m_{i-1}})
\end{equation}
\begin{equation}\label{eq:h_c}
c_i =\sqrt{\alpha^2+c_{i-1}^2+2\alpha c_{i-1}\cos(y-m_{i-1})}
\end{equation}
The proof of the above equations can be found in \cref{apdx:bayesian_update_function}. The atan2 function refers to  2-argument arctangent. Independently conducting  Bayesian update for each dimension, we can obtain the Bayesian update distribution by marginalizing $\y$:
\begin{equation}
p_U(\vtheta'|\vtheta,\bold{x};\alpha)=\mathbb{E}_{p_S(\bold{y}|\bold{x};\alpha)}\delta(\vtheta'-h(\vtheta,\bold{y},\alpha))=\mathbb{E}_{vM(\bold{y}|\bold{x},\alpha)}\delta(\vtheta'-h(\vtheta,\bold{y},\alpha))
\end{equation} 
\begin{figure}
    \centering
    \vskip -0.15in
    \includegraphics[width=0.95\linewidth]{imgs/non_add.pdf}
    \caption{An intuitive illustration of non-additive accuracy Bayesian update on the torus. The lengths of arrows represent the uncertainty/entropy of the belief (\emph{e.g.}~$1/\sigma^2$ for Gaussian and $c$ for von Mises). The directions of the arrows represent the believed location (\emph{e.g.}~ $\mu$ for Gaussian and $m$ for von Mises).}
    \label{fig:non_add}
    \vskip -0.15in
\end{figure}
\textbf{Non-additive Accuracy} 
The additive accuracy is a nice property held with the Gaussian-formed sender distribution of the original BFN expressed as:
\begin{align}
\label{eq:standard_id}
    \update(\parsn{}'' \mid \parsn{}, \x; \alpha_a+\alpha_b) = \E_{\update(\parsn{}' \mid \parsn{}, \x; \alpha_a)} \update(\parsn{}'' \mid \parsn{}', \x; \alpha_b)
\end{align}
Such property is mainly derived based on the standard identity of Gaussian variable:
\begin{equation}
X \sim \mathcal{N}\left(\mu_X, \sigma_X^2\right), Y \sim \mathcal{N}\left(\mu_Y, \sigma_Y^2\right) \Longrightarrow X+Y \sim \mathcal{N}\left(\mu_X+\mu_Y, \sigma_X^2+\sigma_Y^2\right)
\end{equation}
The additive accuracy property makes it feasible to derive the Bayesian flow distribution $
p_F(\boldsymbol{\theta} \mid \mathbf{x} ; i)=p_U\left(\boldsymbol{\theta} \mid \boldsymbol{\theta}_0, \mathbf{x}, \sum_{k=1}^{i} \alpha_i \right)
$ for the simulation-free training of \cref{eq:loss_n}.
It should be noted that the standard identity in \cref{eq:standard_id} does not hold in the von Mises distribution. Hence there exists an important difference between the original Bayesian flow defined on Euclidean space and the Bayesian flow of circular data on $\mathbb{T}^D$ based on von Mises distribution. With prior $\btheta = \{\bold{0},\bold{0}\}$, we could formally represent the non-additive accuracy issue as:
% The additive accuracy property implies the fact that the "confidence" for the data sample after observing a series of the noisy samples with accuracy ${\alpha_1, \cdots, \alpha_i}$ could be  as the accuracy sum  which could be  
% Here we 
% Here we emphasize the specific property of BFN based on von Mises distribution.
% Note that 
% \begin{equation}
% \update(\parsn'' \mid \parsn, \x; \alpha_a+\alpha_b) \ne \E_{\update(\parsn' \mid \parsn, \x; \alpha_a)} \update(\parsn'' \mid \parsn', \x; \alpha_b)
% \end{equation}
% \oyyw{please check whether the below equation is better}
% \yuxuan{I fill somehow confusing on what is the update distribution with $\alpha$. }
% \begin{equation}
% \update(\parsn{}'' \mid \parsn{}, \x; \alpha_a+\alpha_b) \ne \E_{\update(\parsn{}' \mid \parsn{}, \x; \alpha_a)} \update(\parsn{}'' \mid \parsn{}', \x; \alpha_b)
% \end{equation}
% We give an intuitive visualization of such difference in \cref{fig:non_add}. The untenability of this property can materialize by considering the following case: with prior $\btheta = \{\bold{0},\bold{0}\}$, check the two-step Bayesian update distribution with $\alpha_a,\alpha_b$ and one-step Bayesian update with $\alpha=\alpha_a+\alpha_b$:
\begin{align}
\label{eq:nonadd}
     &\update(c'' \mid \parsn, \x; \alpha_a+\alpha_b)  = \delta(c-\alpha_a-\alpha_b)
     \ne  \mathbb{E}_{p_U(\parsn' \mid \parsn, \x; \alpha_a)}\update(c'' \mid \parsn', \x; \alpha_b) \nonumber \\&= \mathbb{E}_{vM(\bold{y}_b|\bold{x},\alpha_a)}\mathbb{E}_{vM(\bold{y}_a|\bold{x},\alpha_b)}\delta(c-||[\alpha_a \cos\y_a+\alpha_b\cos \y_b,\alpha_a \sin\y_a+\alpha_b\sin \y_b]^T||_2)
\end{align}
A more intuitive visualization could be found in \cref{fig:non_add}. This fundamental difference between periodic Bayesian flow and that of \citet{bfn} presents both theoretical and practical challenges, which we will explain and address in the following contents.

% This makes constructing Bayesian flow based on von Mises distribution intrinsically different from previous Bayesian flows (\citet{bfn}).

% Thus, we must reformulate the framework of Bayesian flow networks  accordingly. % and do necessary reformulations of BFN. 

% \yuxuan{overall I feel this part is complicated by using the language of update distribution. I would like to suggest simply use bayesian update, to provide intuitive explantion.}\hanlin{See the illustration in \cref{fig:non_add}}

% That introduces a cascade of problems, and we investigate the following issues: $(1)$ Accuracies between sender and receiver are not synchronized and need to be differentiated. $(2)$ There is no tractable Bayesian flow distribution for a one-step sample conditioned on a given time step $i$, and naively simulating the Bayesian flow results in computational overhead. $(3)$ It is difficult to control the entropy of the Bayesian flow. $(4)$ Accuracy is no longer a function of $t$ and becomes a distribution conditioned on $t$, which can be different across dimensions.
%\jj{Edited till here}

\textbf{Entropy Conditioning} As a common practice in generative models~\citep{ddpm,flowmatching,bfn}, timestep $t$ is widely used to distinguish among generation states by feeding the timestep information into the networks. However, this paper shows that for periodic Bayesian flow, the accumulated accuracy $\vc_i$ is more effective than time-based conditioning by informing the network about the entropy and certainty of the states $\parsnt{i}$. This stems from the intrinsic non-additive accuracy which makes the receiver's accumulated accuracy $c$ not bijective function of $t$, but a distribution conditioned on accumulated accuracies $\vc_i$ instead. Therefore, the entropy parameter $\vc$ is taken logarithm and fed into the network to describe the entropy of the input corrupted structure. We verify this consideration in \cref{sec:exp_ablation}. 
% \yuxuan{implement variant. traditionally, the timestep is widely used to distinguish the different states by putting the timestep embedding into the networks. citation of FM, diffusion, BFN. However, we find that conditioned on time in periodic flow could not provide extra benefits. To further boost the performance, we introduce a simple yet effective modification term entropy conditional. This is based on that the accumulated accuracy which represents the current uncertainty or entropy could be a better indicator to distinguish different states. + Describe how you do this. }



\textbf{Reformulations of BFN}. Recall the original update function with Gaussian sender distribution, after receiving noisy samples $\y_1,\y_2,\dots,\y_i$ with accuracies $\senderacc$, the accumulated accuracies of the receiver side could be analytically obtained by the additive property and it is consistent with the sender side.
% Since observing sample $\y$ with $\alpha_i$ can not result in exact accuracy increment $\alpha_i$ for receiver, the accuracies between sender and receiver are not synchronized which need to be differentiated. 
However, as previously mentioned, this does not apply to periodic Bayesian flow, and some of the notations in original BFN~\citep{bfn} need to be adjusted accordingly. We maintain the notations of sender side's one-step accuracy $\alpha$ and added accuracy $\beta$, and alter the notation of receiver's accuracy parameter as $c$, which is needed to be simulated by cascade of Bayesian updates. We emphasize that the receiver's accumulated accuracy $c$ is no longer a function of $t$ (differently from the Gaussian case), and it becomes a distribution conditioned on received accuracies $\senderacc$ from the sender. Therefore, we represent the Bayesian flow distribution of von Mises distribution as $p_F(\btheta|\x;\alpha_1,\alpha_2,\dots,\alpha_i)$. And the original simulation-free training with Bayesian flow distribution is no longer applicable in this scenario.
% Different from previous BFNs where the accumulated accuracy $\rho$ is not explicitly modeled, the accumulated accuracy parameter $c$ (visualized in \cref{fig:vmbf_vis}) needs to be explicitly modeled by feeding it to the network to avoid information loss.
% the randomaccuracy parameter $c$ (visualized in \cref{fig:vmbf_vis}) implies that there exists information in $c$ from the sender just like $m$, meaning that $c$ also should be fed into the network to avoid information loss. 
% We ablate this consideration in  \cref{sec:exp_ablation}. 

\textbf{Fast Sampling from Equivalent Bayesian Flow Distribution} Based on the above reformulations, the Bayesian flow distribution of von Mises distribution is reframed as: 
\begin{equation}\label{eq:flow_frac}
p_F(\btheta_i|\x;\alpha_1,\alpha_2,\dots,\alpha_i)=\E_{\update(\parsnt{1} \mid \parsnt{0}, \x ; \alphat{1})}\dots\E_{\update(\parsn_{i-1} \mid \parsnt{i-2}, \x; \alphat{i-1})} \update(\parsnt{i} | \parsnt{i-1},\x;\alphat{i} )
\end{equation}
Naively sampling from \cref{eq:flow_frac} requires slow auto-regressive iterated simulation, making training unaffordable. Noticing the mathematical properties of \cref{eq:h_m,eq:h_c}, we  transform \cref{eq:flow_frac} to the equivalent form:
\begin{equation}\label{eq:cirflow_equiv}
p_F(\vec{m}_i|\x;\alpha_1,\alpha_2,\dots,\alpha_i)=\E_{vM(\y_1|\x,\alpha_1)\dots vM(\y_i|\x,\alpha_i)} \delta(\vec{m}_i-\text{atan2}(\sum_{j=1}^i \alpha_j \cos \y_j,\sum_{j=1}^i \alpha_j \sin \y_j))
\end{equation}
\begin{equation}\label{eq:cirflow_equiv2}
p_F(\vec{c}_i|\x;\alpha_1,\alpha_2,\dots,\alpha_i)=\E_{vM(\y_1|\x,\alpha_1)\dots vM(\y_i|\x,\alpha_i)}  \delta(\vec{c}_i-||[\sum_{j=1}^i \alpha_j \cos \y_j,\sum_{j=1}^i \alpha_j \sin \y_j]^T||_2)
\end{equation}
which bypasses the computation of intermediate variables and allows pure tensor operations, with negligible computational overhead.
\begin{restatable}{proposition}{cirflowequiv}
The probability density function of Bayesian flow distribution defined by \cref{eq:cirflow_equiv,eq:cirflow_equiv2} is equivalent to the original definition in \cref{eq:flow_frac}. 
\end{restatable}
\textbf{Numerical Determination of Linear Entropy Sender Accuracy Schedule} ~Original BFN designs the accuracy schedule $\beta(t)$ to make the entropy of input distribution linearly decrease. As for crystal generation task, to ensure information coherence between modalities, we choose a sender accuracy schedule $\senderacc$ that makes the receiver's belief entropy $H(t_i)=H(p_I(\cdot|\vtheta_i))=H(p_I(\cdot|\vc_i))$ linearly decrease \emph{w.r.t.} time $t_i$, given the initial and final accuracy parameter $c(0)$ and $c(1)$. Due to the intractability of \cref{eq:vm_entropy}, we first use numerical binary search in $[0,c(1)]$ to determine the receiver's $c(t_i)$ for $i=1,\dots, n$ by solving the equation $H(c(t_i))=(1-t_i)H(c(0))+tH(c(1))$. Next, with $c(t_i)$, we conduct numerical binary search for each $\alpha_i$ in $[0,c(1)]$ by solving the equations $\E_{y\sim vM(x,\alpha_i)}[\sqrt{\alpha_i^2+c_{i-1}^2+2\alpha_i c_{i-1}\cos(y-m_{i-1})}]=c(t_i)$ from $i=1$ to $i=n$ for arbitrarily selected $x\in[-\pi,\pi)$.

After tackling all those issues, we have now arrived at a new BFN architecture for effectively modeling crystals. Such BFN can also be adapted to other type of data located in hyper-torus $\mathbb{T}^{D}$.

\subsection{Equivariant Bayesian Flow for Crystal}
With the above Bayesian flow designed for generative modeling of fractional coordinate $\vF$, we are able to build equivariant Bayesian flow for each modality of crystal. In this section, we first give an overview of the general training and sampling algorithm of \modelname (visualized in \cref{fig:framework}). Then, we describe the details of the Bayesian flow of every modality. The training and sampling algorithm can be found in \cref{alg:train} and \cref{alg:sampling}.

\textbf{Overview} Operating in the parameter space $\bthetaM=\{\bthetaA,\bthetaL,\bthetaF\}$, \modelname generates high-fidelity crystals through a joint BFN sampling process on the parameter of  atom type $\bthetaA$, lattice parameter $\vec{\theta}^L=\{\bmuL,\brhoL\}$, and the parameter of fractional coordinate matrix $\bthetaF=\{\bmF,\bcF\}$. We index the $n$-steps of the generation process in a discrete manner $i$, and denote the corresponding continuous notation $t_i=i/n$ from prior parameter $\thetaM_0$ to a considerably low variance parameter $\thetaM_n$ (\emph{i.e.} large $\vrho^L,\bmF$, and centered $\bthetaA$).

At training time, \modelname samples time $i\sim U\{1,n\}$ and $\bthetaM_{i-1}$ from the Bayesian flow distribution of each modality, serving as the input to the network. The network $\net$ outputs $\net(\parsnt{i-1}^\mathcal{M},t_{i-1})=\net(\parsnt{i-1}^A,\parsnt{i-1}^F,\parsnt{i-1}^L,t_{i-1})$ and conducts gradient descents on loss function \cref{eq:loss_n} for each modality. After proper training, the sender distribution $p_S$ can be approximated by the receiver distribution $p_R$. 

At inference time, from predefined $\thetaM_0$, we conduct transitions from $\thetaM_{i-1}$ to $\thetaM_{i}$ by: $(1)$ sampling $\y_i\sim p_R(\bold{y}|\thetaM_{i-1};t_i,\alpha_i)$ according to network prediction $\predM{i-1}$; and $(2)$ performing Bayesian update $h(\thetaM_{i-1},\y^\calM_{i-1},\alpha_i)$ for each dimension. 

% Alternatively, we complete this transition using the flow-back technique by sampling 
% $\thetaM_{i}$ from Bayesian flow distribution $\flow(\btheta^M_{i}|\predM{i-1};t_{i-1})$. 

% The training objective of $\net$ is to minimize the KL divergence between sender distribution and receiver distribution for every modality as defined in \cref{eq:loss_n} which is equivalent to optimizing the negative variational lower bound $\calL^{VLB}$ as discussed in \cref{sec:preliminaries}. 

%In the following part, we will present the Bayesian flow of each modality in detail.

\textbf{Bayesian Flow of Fractional Coordinate $\vF$}~The distribution of the prior parameter $\bthetaF_0$ is defined as:
\begin{equation}\label{eq:prior_frac}
    p(\bthetaF_0) \defeq \{vM(\vm_0^F|\vec{0}_{3\times N},\vec{0}_{3\times N}),\delta(\vc_0^F-\vec{0}_{3\times N})\} = \{U(\vec{0},\vec{1}),\delta(\vc_0^F-\vec{0}_{3\times N})\}
\end{equation}
Note that this prior distribution of $\vm_0^F$ is uniform over $[\vec{0},\vec{1})$, ensuring the periodic translation invariance property in \cref{De:pi}. The training objective is minimizing the KL divergence between sender and receiver distribution (deduction can be found in \cref{appd:cir_loss}): 
%\oyyw{replace $\vF$ with $\x$?} \hanlin{notations follow Preliminary?}
\begin{align}\label{loss_frac}
\calL_F = n \E_{i \sim \ui{n}, \flow(\parsn{}^F \mid \vF ; \senderacc)} \alpha_i\frac{I_1(\alpha_i)}{I_0(\alpha_i)}(1-\cos(\vF-\predF{i-1}))
\end{align}
where $I_0(x)$ and $I_1(x)$ are the zeroth and the first order of modified Bessel functions. The transition from $\bthetaF_{i-1}$ to $\bthetaF_{i}$ is the Bayesian update distribution based on network prediction:
\begin{equation}\label{eq:transi_frac}
    p(\btheta^F_{i}|\parsnt{i-1}^\calM)=\mathbb{E}_{vM(\bold{y}|\predF{i-1},\alpha_i)}\delta(\btheta^F_{i}-h(\btheta^F_{i-1},\bold{y},\alpha_i))
\end{equation}
\begin{restatable}{proposition}{fracinv}
With $\net_{F}$ as a periodic translation equivariant function namely $\net_F(\parsnt{}^A,w(\parsnt{}^F+\vt),\parsnt{}^L,t)=w(\net_F(\parsnt{}^A,\parsnt{}^F,\parsnt{}^L,t)+\vt), \forall\vt\in\R^3$, the marginal distribution of $p(\vF_n)$ defined by \cref{eq:prior_frac,eq:transi_frac} is periodic translation invariant. 
\end{restatable}
\textbf{Bayesian Flow of Lattice Parameter \texorpdfstring{$\boldsymbol{L}$}{}}   
Noting the lattice parameter $\bm{L}$ located in Euclidean space, we set prior as the parameter of a isotropic multivariate normal distribution $\btheta^L_0\defeq\{\vmu_0^L,\vrho_0^L\}=\{\bm{0}_{3\times3},\bm{1}_{3\times3}\}$
% \begin{equation}\label{eq:lattice_prior}
% \btheta^L_0\defeq\{\vmu_0^L,\vrho_0^L\}=\{\bm{0}_{3\times3},\bm{1}_{3\times3}\}
% \end{equation}
such that the prior distribution of the Markov process on $\vmu^L$ is the Dirac distribution $\delta(\vec{\mu_0}-\vec{0})$ and $\delta(\vec{\rho_0}-\vec{1})$, 
% \begin{equation}
%     p_I^L(\boldsymbol{L}|\btheta_0^L)=\mathcal{N}(\bm{L}|\bm{0},\bm{I})
% \end{equation}
which ensures O(3)-invariance of prior distribution of $\vL$. By Eq. 77 from \citet{bfn}, the Bayesian flow distribution of the lattice parameter $\bm{L}$ is: 
\begin{align}% =p_U(\bmuL|\btheta_0^L,\bm{L},\beta(t))
p_F^L(\bmuL|\bm{L};t) &=\mathcal{N}(\bmuL|\gamma(t)\bm{L},\gamma(t)(1-\gamma(t))\bm{I}) 
\end{align}
where $\gamma(t) = 1 - \sigma_1^{2t}$ and $\sigma_1$ is the predefined hyper-parameter controlling the variance of input distribution at $t=1$ under linear entropy accuracy schedule. The variance parameter $\vrho$ does not need to be modeled and fed to the network, since it is deterministic given the accuracy schedule. After sampling $\bmuL_i$ from $p_F^L$, the training objective is defined as minimizing KL divergence between sender and receiver distribution (based on Eq. 96 in \citet{bfn}):
\begin{align}
\mathcal{L}_{L} = \frac{n}{2}\left(1-\sigma_1^{2/n}\right)\E_{i \sim \ui{n}}\E_{\flow(\bmuL_{i-1} |\vL ; t_{i-1})}  \frac{\left\|\vL -\predL{i-1}\right\|^2}{\sigma_1^{2i/n}},\label{eq:lattice_loss}
\end{align}
where the prediction term $\predL{i-1}$ is the lattice parameter part of network output. After training, the generation process is defined as the Bayesian update distribution given network prediction:
\begin{equation}\label{eq:lattice_sampling}
    p(\bmuL_{i}|\parsnt{i-1}^\calM)=\update^L(\bmuL_{i}|\predL{i-1},\bmuL_{i-1};t_{i-1})
\end{equation}
    

% The final prediction of the lattice parameter is given by $\bmuL_n = \predL{n-1}$.
% \begin{equation}\label{eq:final_lattice}
%     \bmuL_n = \predL{n-1}
% \end{equation}

\begin{restatable}{proposition}{latticeinv}\label{prop:latticeinv}
With $\net_{L}$ as  O(3)-equivariant function namely $\net_L(\parsnt{}^A,\parsnt{}^F,\vQ\parsnt{}^L,t)=\vQ\net_L(\parsnt{}^A,\parsnt{}^F,\parsnt{}^L,t),\forall\vQ^T\vQ=\vI$, the marginal distribution of $p(\bmuL_n)$ defined by \cref{eq:lattice_sampling} is O(3)-invariant. 
\end{restatable}


\textbf{Bayesian Flow of Atom Types \texorpdfstring{$\boldsymbol{A}$}{}} 
Given that atom types are discrete random variables located in a simplex $\calS^K$, the prior parameter of $\boldsymbol{A}$ is the discrete uniform distribution over the vocabulary $\parsnt{0}^A \defeq \frac{1}{K}\vec{1}_{1\times N}$. 
% \begin{align}\label{eq:disc_input_prior}
% \parsnt{0}^A \defeq \frac{1}{K}\vec{1}_{1\times N}
% \end{align}
% \begin{align}
%     (\oh{j}{K})_k \defeq \delta_{j k}, \text{where }\oh{j}{K}\in \R^{K},\oh{\vA}{KD} \defeq \left(\oh{a_1}{K},\dots,\oh{a_N}{K}\right) \in \R^{K\times N}
% \end{align}
With the notation of the projection from the class index $j$ to the length $K$ one-hot vector $ (\oh{j}{K})_k \defeq \delta_{j k}, \text{where }\oh{j}{K}\in \R^{K},\oh{\vA}{KD} \defeq \left(\oh{a_1}{K},\dots,\oh{a_N}{K}\right) \in \R^{K\times N}$, the Bayesian flow distribution of atom types $\vA$ is derived in \citet{bfn}:
\begin{align}
\flow^{A}(\parsn^A \mid \vA; t) &= \E_{\N{\y \mid \beta^A(t)\left(K \oh{\vA}{K\times N} - \vec{1}_{K\times N}\right)}{\beta^A(t) K \vec{I}_{K\times N \times N}}} \delta\left(\parsn^A - \frac{e^{\y}\parsnt{0}^A}{\sum_{k=1}^K e^{\y_k}(\parsnt{0})_{k}^A}\right).
\end{align}
where $\beta^A(t)$ is the predefined accuracy schedule for atom types. Sampling $\btheta_i^A$ from $p_F^A$ as the training signal, the training objective is the $n$-step discrete-time loss for discrete variable \citep{bfn}: 
% \oyyw{can we simplify the next equation? Such as remove $K \times N, K \times N \times N$}
% \begin{align}
% &\calL_A = n\E_{i \sim U\{1,n\},\flow^A(\parsn^A \mid \vA ; t_{i-1}),\N{\y \mid \alphat{i}\left(K \oh{\vA}{KD} - \vec{1}_{K\times N}\right)}{\alphat{i} K \vec{I}_{K\times N \times N}}} \ln \N{\y \mid \alphat{i}\left(K \oh{\vA}{K\times N} - \vec{1}_{K\times N}\right)}{\alphat{i} K \vec{I}_{K\times N \times N}}\nonumber\\
% &\qquad\qquad\qquad-\sum_{d=1}^N \ln \left(\sum_{k=1}^K \out^{(d)}(k \mid \parsn^A; t_{i-1}) \N{\ydd{d} \mid \alphat{i}\left(K\oh{k}{K}- \vec{1}_{K\times N}\right)}{\alphat{i} K \vec{I}_{K\times N \times N}}\right)\label{discdisc_t_loss_exp}
% \end{align}
\begin{align}
&\calL_A = n\E_{i \sim U\{1,n\},\flow^A(\parsn^A \mid \vA ; t_{i-1}),\N{\y \mid \alphat{i}\left(K \oh{\vA}{KD} - \vec{1}\right)}{\alphat{i} K \vec{I}}} \ln \N{\y \mid \alphat{i}\left(K \oh{\vA}{K\times N} - \vec{1}\right)}{\alphat{i} K \vec{I}}\nonumber\\
&\qquad\qquad\qquad-\sum_{d=1}^N \ln \left(\sum_{k=1}^K \out^{(d)}(k \mid \parsn^A; t_{i-1}) \N{\ydd{d} \mid \alphat{i}\left(K\oh{k}{K}- \vec{1}\right)}{\alphat{i} K \vec{I}}\right)\label{discdisc_t_loss_exp}
\end{align}
where $\vec{I}\in \R^{K\times N \times N}$ and $\vec{1}\in\R^{K\times D}$. When sampling, the transition from $\bthetaA_{i-1}$ to $\bthetaA_{i}$ is derived as:
\begin{equation}
    p(\btheta^A_{i}|\parsnt{i-1}^\calM)=\update^A(\btheta^A_{i}|\btheta^A_{i-1},\predA{i-1};t_{i-1})
\end{equation}

The detailed training and sampling algorithm could be found in \cref{alg:train} and \cref{alg:sampling}.




\para{Problem formulation}
\label{sec:problem_form}
To leverage the prior knowledge, we formulate
a depth completion as an inverse problem that estimates unknown dense depth from 
observed sparse measurements.
$\mathbf{y}$ represents the observed sparse depth, 
$\mathbf{x}$ is the unknown dense depth, and 
$\mathcal{A}{:}\,\mathbb{R}^m{\rightarrow}\mathbb{R}^n$ is a binary measurement matrix of which entry  $[\mathcal{A}]_{ij}$ is $1$ if the entities $[\mathbf{y}]_i$ is measured from $[\mathbf{x}]_j$, $0$ otherwise. 
We follow \Eref{eq:inv_sampling}, where sparse depth serves as guidance.
We use the depth diffusion models \cite{ke2023repurposing, gui2024depthfm} extended from the latent diffusion model (LDM)~\cite{rombach2022highresolution} as prior, where
$\mathbf{x}$ is 
decomposed with the decoder $\mathcal{D}{:}\, \mathbf{z} \rightarrow \mathbf{x}$ as:
\begin{equation}
\label{eq:depth_guide_sampling}
\hat{\mathbf{s}}_{\theta}= \mathbf{s}_{\theta}(\mathbf{z}_t, t) + w {\nabla_{\mathbf{z}_t}}\left\| \mathbf{y} - \mathcal{A}\left(\mathcal{D}\left(\mathbf{z}_0\left(\mathbf{z}_t\right)\right)\right) \right\|_2^2,
\end{equation}
where $\mathbf{z}\in \mathbb{R}^{4\times H \times W}$ represents the latent of LDM but the decoder output $\mathbf{x}$ is treated as a flatten vector for convenience.

\subsection{Test-time Alignment with Hard Constraints}
\label{sec:opt_sampling}
Depth measurements obtained in practice are often sparse, unevenly distributed, and noisy. 
When the sparse measurements are used as guidance, the ill-posed nature of the problem, combined with the stochastic behavior of diffusion models, can lead to scores that produce undesirable solutions~\cite{kim2024regtext} and does not even guarantee that the estimation corresponds to the known sparse measurements. 
To deal with this, we propose a test-time alignment that incorporates the correction step 
to enforce the sparse measurement as harder constraints than encouraging guidance in a soft manner by \Eref{eq:depth_guide_sampling}.
This involves an optimization loop at regular intervals to enforce
measurement constraints as a correction step.
We further show the potential for uncertain solutions from the stochastic process in the supplementary material, illustrating why the alignment is necessary.

Additionally, we adopt $\mathbf{z}_0(\mathbf{z}_t)$ as optimizable variable.
Pre-trained diffusion models take input $\mathbf{z}_t$ aligend with the noise level at each timestep $t$.
However, directly optimizing $\mathbf{z}_t$  without considering input characteristics may lead to suboptimal results
\cite{chung2022improving, chung2023dps, chung2024dds}.
To address this, inspired by \citet{song2024solving}, we use $\mathbf{z}_0(\mathbf{z}_t)$ estimated from $\mathbf{z}_t$.
The optimization loop is formulated as:
\begin{equation}
    \label{eq:opt_loop}
    \hat{\mathbf{z}}_0(\mathbf{z}_t) = \argmin_{\mathbf{z}_0(\mathbf{z}_t)} \left\| \mathbf{y} - \mathcal{A}\left(\mathcal{D}\left(\mathbf{z}_0\left(\mathbf{z}_t\right)\right)\right) \right\|_2^2.
\end{equation}
Then, to ensure adherence to the correct noise level, the measurement-consistent $\hat{\mathbf{z}}_0(\mathbf{z}_t)$ is remapped to an intermediate latent $\hat{\mathbf{z}}_t$ by adding time-scheduled Gaussian noise, as expressed below:
\begin{equation}
    \label{eq:remap}
    p\left(\hat{\mathbf{z}}_{t} | \hat{\mathbf{z}}_0(\mathbf{z}_t)\right) = \mathcal{N}(\sqrt{\bar{\alpha}_{t}} ~ \hat{\mathbf{z}}_0(\mathbf{z}_t), (1 - \bar{\alpha}_{t}) I),
\end{equation}
\noindent where $\bar{\alpha}_{t} = \prod_{i=1}^t \alpha_i,$ and $\alpha_t$ is variance schedule at time $t$.

Since the score $\hat{\mathbf{s}}_{\theta}(\mathbf{z}_t, t)$ is directly added to the latent $\mathbf{z}_t$ at each step,
\Eref{eq:depth_guide_sampling} can be rewritten in terms of $\mathbf{z}_0(\mathbf{z}_t)$ with a modulated weight factor $\zeta$, as follows:
\begin{equation}
    \label{eq:depth_guide_sampling_latent}
    \hat{\mathbf{z}}_t = \mathbf{z}_t + \zeta \nabla_{\mathbf{z}_t} \left\| \mathbf{y} - \mathcal{A}\left(\mathcal{D}\left(\mathbf{z}_0(\mathbf{z}_t)\right)\right) \right\|_2^2.
\end{equation}
Here, \Eref{eq:depth_guide_sampling_latent} is replaced by the two-step process of \Eref{eq:opt_loop} and \Eref{eq:remap}, allowing our test-time alignment process to effectively achieve measurement-consistent desirable solutions.
Figure \ref{fig:opt_loop} illustrates the our test-time alignment process.
Figure~\ref{fig:depth_align} demonstrates how effectively our test-time alignment method estimates unseen depth areas by aligning sparse measurements with an affine-invariant depth prior. This result highlights the need for correction.
Examples of undesirable solutions and their corrected ones by our method are provided in the supplementary material.
\begin{figure*}[t]
\centering
   \includegraphics[width=1.0\linewidth]{Figures/depth_align_arxiv.pdf}
   \caption{\textbf{Alignment with metric depth.} 
   % \after{
   We evaluate our method's effectiveness against ground truth (GT), accumulated semi-densely.
   We use only sparse depth (a) to align with actual metric depth values in complex scenes, ensuring a desirable solution.
   The white lines in (b), (c), and the x-axis of (d) represent pixel indices with valid depth points in a row of the GT.
   } 
\label{fig:depth_align}
\vspace{-3mm}
\end{figure*}
% % \vspace{-3mm}
\begin{algorithm}[t]
\caption{Prior-based outlier filtering algorithm.}
\label{algorithm:1}
\begin{algorithmic}[1]
\setlength{\itemsep}{0.15em}
    \State \textbf{Parameters:} Number of segments $N$, Filter threshold $\tau$
    \State \textbf{Input:} 
    % RGB image $I$, 
    Estimated relative depth $D_r$, Sparse metric depth $\mathbf{y}$, Set of sparse point locations $\Omega(\mathbf{y})$.
    \State \textbf{Output:} Set of reliable sparse point locations $\Omega(\mathbf{y}^*)$.
    \State $\{\Omega(S_i)\}_{i=1 \cdots N} \gets \text{SuperPixel}\left( D_r, N 
    % \oplus I 
    \right)$ 
    % \Comment{Cluster all the points into segments $S_i$}
    \For{$i = 1$ \textbf{to} $N$}
        \State $\Omega(\mathbf{y}_i) \gets \Omega(\mathbf{y}) \cap \Omega(S_i)$
        % \Comment{$\mathbf{y}_i$ denotes sparse depth points on $\Omega(\mathbf{y}_i)$}
        \State $\hat{\mathbf{y}}_i \gets %\underset{S_i \cap S_p}
        {\text{RANSAC~Regressor}}(\mathds{1}_{\Omega(\mathbf{y}_i)} \odot D_r,~ \mathbf{y}_i)$ 
        % \Comment{Perform RANSAC Regression on points $\mathbf{y} \cap S_i$}
        \State $\Omega(\mathbf{y}_i^{*}) \gets |\hat{\mathbf{y}}_i-\mathbf{y}_i| > \tau$
        % \Comment{Filter depth points with values exceeding threshold $\tau$.}
    \EndFor
    \State \( \Omega(\mathbf{y}^*) \gets \bigcup_{i=1}^{N} \Omega(\mathbf{y}_i^{*}) \)
\end{algorithmic}
\end{algorithm}
% \vspace{-3mm}

Until now, in solving \Eref{eq:depth_guide_sampling}, we use an affine-invariant depth model for completing metric depths without special care. However, a natural question arises: ``\textit{Is the affine-invariant depth model compatible with estimating metric depths in our framework?}'' The following analysis shows that it may be sufficient.

\vspace{1mm}\noindent\textbf{Can we use an affine-invariant depth model for completing metric depths?}
Depth estimation models are often trained to estimate affine-invariant depth with scale and shift invariant loss to achieve generalizable performance~\cite{Ranftl2022midas, ke2023repurposing, eigen2014invariant}.
Thus, depth prior operates in the affine-invariant depth space, which does not directly correspond to the metric depth used in measurements.
Even though the given sparse metric depth is normalized between 0 and 1, 
their statistics including 
mean and variance
can differ, and the relationship between real metric depth and estimated affine-invariant depth is often 
non-linear (see 
the left of \Fref{fig:depth_align} (d)).
Therefore, to determine if 
\Eref{eq:depth_guide_sampling} can be used to solve this problem, we need to verify whether the normalized metric depth space lies within the data distribution generated by the diffusion model.

To confirm this, we conduct an empirical investigation through the following procedure: given $\tilde{\mathbf{x}}_0$, dense depth map estimated from the pre-trained depth completion model, we perform its reconstruction using an affine-invariant depth diffusion model.
This process involves sequentially encoding $\tilde{\mathbf{x}}_0$ to $\tilde{\mathbf{z}}_0$, 
then doing inversion by adding noise~\cite{song2022denoising}, which results in $\tilde{\mathbf{z}}_t$. 
Next, we perform reverse sampling, $\nabla_{\mathbf{z}_t}\log p(\tilde{\mathbf{z}}_t)$ with only the affine-invariant depth diffusion prior.
The reconstructed result achieves similar performance compared to the original one, $\tilde{\mathbf{x}}_0$, excluding encoding-decoding information loss. 
The details and results of the experiment are provided in the supplementary material.
This result suggests that the affine-invariant depth prior is sufficiently capable of handling the metric depth space, 
which corresponds to:
\begin{equation}
\label{eq:verify}
    \nabla_{\mathbf{z}_t}\log p(\tilde{\mathbf{z}}_t) \approx \nabla_{\tilde{\mathbf{z}}_t}\log p(\tilde{\mathbf{z}}_t).
\end{equation}
Thus, we just need to align this prior with metric depth cue validating using \Eref{eq:depth_guide_sampling} to solve ill-posed depth completion.

% \section{Criterion V: Generalization (Table \ref{tab:generalization})} \label{sec:generalization}

Generalization refers to LLM planners' ability to apply learned strategies to new, more complex out-of-domain scenarios beyond its training environment, which can be enhanced through three key approaches: \emph{fine-tuning} (described previously in Section \ref{sec:foundations}), \emph{generalized planning}, and \emph{skill storage}. Given the diverse user queries in the real-world deployments, ensuring LLM planners' generalizability is important alongside other performance.

\vspace{-0.05in}
\paragraph {Generalized Planning} Generalized planning extracts common patterns from training solutions to tackle larger, more complex tasks within the same domain \cite{srivastava2011new}. 
For example, in the Delivery dataset \cite{yang2022pg3}, models trained on small-scale deliveries (9–17 locations) can generalize to larger ones (70–100 locations) using the same core strategy. 
\citet{silver2024generalized} approached this by prompting LLMs to summarize the domain and generate a minimal, generalizable Python-based plan.  

\vspace{-0.05in}
\paragraph {Skill Storage} Skill storage focuses on learning and reusing previously acquired skills to tackle new problems. E.g., \citet{wang2023voyager} introduced a skill library that stores successfully executed skills (e.g., Combat Zombie). These skills are abstracted and generalized for reuse in similar situations (e.g., fighting spiders involves similar actions to fighting zombies). When encountering an unseen task, the LLM planning system retrieves relevant learned skills based on the task and current states, then applies them to generate an effective solution.

% \begin{table}[t!]
\centering
    \scriptsize
    \setlength{\tabcolsep}{0.0035\linewidth}
    \caption{\textbf{Computational efficiency of EvSSC across different datasets.} Memory denotes training memory usage.}
    %\vskip-1ex
\setlength{\tabcolsep}{4pt} %5pt 设定列之间的宽度
\resizebox{\columnwidth}{!}{%
\begin{tabular}{l|>{\columncolor{gray!10}}l>{\columncolor{blue!8}}l|>{\columncolor{gray!10}}l>{\columncolor{blue!8}}l}
\toprule
\textbf{ } & \textbf{VoxFormer-S} & \textbf{EvSSC (VoxFormer)} & \textbf{SGN-S} & \textbf{EvSSC (SGN)}\\
\midrule\midrule
\multicolumn{5}{c}{\textit{DSEC-SSC}} \\ \midrule 
\textbf{mIoU} &25.62 & 26.34 & 29.06 & 29.55\\ 
\textbf{IoU}  & 47.25 & 47.29 & 43.70 & 43.99\\ 
\textbf{Memory}  & 9.74G & 10.52G & 10.19G & 10.70G\\ 
\textbf{Latency} & 0.732s & 0.836s & 0.941s & 1.193s \\\midrule 
\multicolumn{5}{c}{\textit{SemanticKITTI-E}} \\ \midrule 
\textbf{mIoU} & 12.86 & 13.61 & 14.55 & 15.15\\ 
\textbf{IoU}  & 44.42 & 45.01 & 43.60 & 43.17\\ 
\textbf{Memory} & 14.87G & 15.78G & 15.29G & 17.79G \\ 
\textbf{Latency}  & 0.996s & 1.005s & 0.855s &1.005s\\ \midrule 
\multicolumn{5}{c}{\textit{SemanticKITTI-C Shot Noise}} \\ \midrule 
\textbf{mIoU} & 8.29 & 12.64 & 13.62 & 14.32\\ 
\textbf{IoU}  & 44.26 & 45.04 & 42.05 & 42.54\\ 
\textbf{Memory} & 14.87G & 15.78G & 15.29G & 17.79G \\ 
\textbf{Latency}  & 0.996s & 1.005s & 0.855s &1.005s\\
\bottomrule
\end{tabular}
}
\label{table:efficiency}
%\vskip-3ex
\end{table}


\subsection{Prior-based Outlier Filtering}
\label{sec:noise_filter}
Practical depth sensing methods often produce outliers, such as unsynchronized depth with RGB or see-through points~\cite{conti22confidence}), making sparse depth measurements unreliable.
This degrades the performance of methods relying on sparse depth supervision~\cite{wong2021unsupervised,wong2020void}.
We also use sparse depth measurement as supervision during test-time alignment, this makes the alignment process prone to divergence or slow convergence.
To address this,
we utilize data-driven depth prior~\cite{ke2023repurposing, gui2024depthfm}, which benefits from the more precise synchronization with RGB images and depth affinity.
To obtain outlier-free sparse points $\mathbf{y}^*$, we adopt a divide-and-conquer approach.
We define local segments based on depth affinity, grouping regions where relative depth values are similar within a spatially local area.
Within these segments, the depth distribution can be easily categorized into inliers and outliers, enabling us to effectively identify outliers.

Affine-invariant depth map $D_r$ is divided into local segments $S_i$, which are regions with a high probability of having similar depths with considering location. 
For this clustering 
we leverage the superpixel algorithm~\cite{achanta2012slic, li2015lsc}.
In each region, we perform linear least-square fitting to map affine-invariant depth to metric depth using sparse metric depth measurements $\mathbf{y}_i$.
However, since these sparse measurements are influenced by outliers, we use RANSAC~\cite{fischler1981ransac} to perform outlier-robust linear least-square fitting on points where noisy $\mathbf{y}$ intersects $S_i$ \ie, $\mathbf{y}_i \leftarrow S_i \cap \mathbf{y}$.
This allows us to estimate outlier-robust metric depth values $\hat{\mathbf{y}}_i$ in local regions $S_i$.
Then, points with significant deviations exceeding $\tau$ are identified as outliers and filtered out.
Our proposed filtering algorithm, based on monocular depth prior, is detailed in Algorithm~\ref{algorithm:1}.
% \red{We also evaluate our algorithm using a standard metric for assessing the reliability of outlier detection confidence, as detailed in the supplementary material.}

% \vspace{-3mm}
\begin{algorithm}[t]
\caption{Prior-based outlier filtering algorithm.}
\label{algorithm:1}
\begin{algorithmic}[1]
\setlength{\itemsep}{0.15em}
    \State \textbf{Parameters:} Number of segments $N$, Filter threshold $\tau$
    \State \textbf{Input:} 
    % RGB image $I$, 
    Estimated relative depth $D_r$, Sparse metric depth $\mathbf{y}$, Set of sparse point locations $\Omega(\mathbf{y})$.
    \State \textbf{Output:} Set of reliable sparse point locations $\Omega(\mathbf{y}^*)$.
    \State $\{\Omega(S_i)\}_{i=1 \cdots N} \gets \text{SuperPixel}\left( D_r, N 
    % \oplus I 
    \right)$ 
    % \Comment{Cluster all the points into segments $S_i$}
    \For{$i = 1$ \textbf{to} $N$}
        \State $\Omega(\mathbf{y}_i) \gets \Omega(\mathbf{y}) \cap \Omega(S_i)$
        % \Comment{$\mathbf{y}_i$ denotes sparse depth points on $\Omega(\mathbf{y}_i)$}
        \State $\hat{\mathbf{y}}_i \gets %\underset{S_i \cap S_p}
        {\text{RANSAC~Regressor}}(\mathds{1}_{\Omega(\mathbf{y}_i)} \odot D_r,~ \mathbf{y}_i)$ 
        % \Comment{Perform RANSAC Regression on points $\mathbf{y} \cap S_i$}
        \State $\Omega(\mathbf{y}_i^{*}) \gets |\hat{\mathbf{y}}_i-\mathbf{y}_i| > \tau$
        % \Comment{Filter depth points with values exceeding threshold $\tau$.}
    \EndFor
    \State \( \Omega(\mathbf{y}^*) \gets \bigcup_{i=1}^{N} \Omega(\mathbf{y}_i^{*}) \)
\end{algorithmic}
\end{algorithm}
% \vspace{-3mm}
\section{Criterion V: Generalization (Table \ref{tab:generalization})} \label{sec:generalization}

Generalization refers to LLM planners' ability to apply learned strategies to new, more complex out-of-domain scenarios beyond its training environment, which can be enhanced through three key approaches: \emph{fine-tuning} (described previously in Section \ref{sec:foundations}), \emph{generalized planning}, and \emph{skill storage}. Given the diverse user queries in the real-world deployments, ensuring LLM planners' generalizability is important alongside other performance.

\vspace{-0.05in}
\paragraph {Generalized Planning} Generalized planning extracts common patterns from training solutions to tackle larger, more complex tasks within the same domain \cite{srivastava2011new}. 
For example, in the Delivery dataset \cite{yang2022pg3}, models trained on small-scale deliveries (9–17 locations) can generalize to larger ones (70–100 locations) using the same core strategy. 
\citet{silver2024generalized} approached this by prompting LLMs to summarize the domain and generate a minimal, generalizable Python-based plan.  

\vspace{-0.05in}
\paragraph {Skill Storage} Skill storage focuses on learning and reusing previously acquired skills to tackle new problems. E.g., \citet{wang2023voyager} introduced a skill library that stores successfully executed skills (e.g., Combat Zombie). These skills are abstracted and generalized for reuse in similar situations (e.g., fighting spiders involves similar actions to fighting zombies). When encountering an unseen task, the LLM planning system retrieves relevant learned skills based on the task and current states, then applies them to generate an effective solution.

\subsection{Losses}
\label{sec:losses}
Our objective for optimization includes sparse depth consistency loss and regularization terms: a local smoothness loss to preserve depth prior and a new relative structure similarity loss to maintain structural prior inherent in depth prior.

\para{Sparse depth consistency}
Given the sparse depth measurement $y$, it ensures consistency with the metric depth.
To effectively integrate the observed measurements with affine-invariant depth prior and mitigate potential uncertainties, we employ $L_{1}$ loss as follows:
\begin{equation}
    \resizebox{0.59\hsize}{!}{$
    \mathcal{L}_{depth} = \scalebox{1.4}{$\frac{1}{|\Omega(\mathbf{y})|}$} \sum\limits_{\Omega(\mathbf{y})} |\mathbf{y}- \mathcal{A}(\hat{D})|,
    $}
\end{equation}
where $\mathcal{A}$ is the operation that Hadamard product with the zero-one mask $\mathds{1}_{\Omega(\mathbf{y})}$ and $\hat{D}$ represents completed depth.

\para{Local smoothness}
Using only sparse depth guidance risks losing the prior knowledge inherent in pre-trained depth diffusion models~\cite{ke2023repurposing, gui2024depthfm}, such as the property of depth which is locally smooth.
To mitigate this, we introduce a regularization term that enforces smoothness by applying the $L_1$ norm to gradients in both the $X$ and $Y$ directions, with reduced gradient weights near edges to prevent over-smoothing.
% To avoid over-smoothing at edges, gradient weights are set to smaller values near edges.
The loss function is defined as follows:
\begin{equation}
    \resizebox{0.89\hsize}{!}{$
    \mathcal{L}_{smooth} =\scalebox{1.5}{$\frac{1}{|\Omega|}$} \sum\limits_{c\in\Omega} \lambda_X(c) |\partial_X \hat{D}(c)| + \lambda_Y(c) |\partial_Y \hat{D}(c)|,
    $}
\end{equation}
where $\lambda_X(c)=e^{-|\partial_X I(c)|}$, $\lambda_Y(c)=e^{-|\partial_Y I(c)|}$, and $c \in \Omega$ represents the set of all pixel locations~\cite{park2024testtime}.
However, using only these loss functions may dilute the structural prior in the pre-trained depth diffusion model, which is key for detail sharpness. 

\para{Relative Structure Similarity}
To address this, we design a new structure regularization term that transfers structure from the depth estimated by an off-the-shelf model to regularize overly smooth structures.
Inspired by the structure similarity (SSIM) loss~\cite{wang2004ssim}, we propose the Relative Stucture Similarity (R-SSIM) loss, designed to transfer structure across domains.
This loss is derived from SSIM by dropping the luminance term, which relies on absolute values:
\begin{equation}
    \mathcal{L}_{r-ssim}(d_1,d_2) = 1-\frac{2\sigma_{d_1d_2} + C}{\sigma_{d_1}^2 + \sigma_{d_2}^2 + C},
\end{equation}
where $d_1$ and $d_2$ represent spatial information in different domains, $C$ is a constant, and $\sigma$ denotes the normalized standard deviation of pixel values.
Here, $d_1$ is the relative depth map, and $d_2$ is the estimated complete depth map (or vice versa).
The key point is that these domains may differ in pixel value ranges and statistics. 
% \red{Ablation studies on R-SSIM loss are in the supplementary materials.}

\vspace{1.5mm}
\noindent Our comprehensive loss function is as follows:
\begin{equation}
    \mathcal{L} = \mathcal{L}_{depth} + \lambda_{smooth} \mathcal{L}_{smooth} + \lambda_{r-ssim} \mathcal{L}_{r-ssim},
\end{equation}
where $\lambda_{smooth}$ and $\lambda_{r-ssim}$ are regularization weights.
