%\qz{this section should contain definitions not appeared in Intro. Essentially it expands the problem space description, presents the design goal, and shed light onto our solution}

\label{sec:concept}
%\begin{figure*}
%    \centering
%    \includegraphics[width=.9\textwidth]{figures/ai_bridged_cla.pdf}
%    \caption{\qz{this is a bit confusing - I thought outline is also generated by AI system on the right? and on the left - author creates a narrative space rather than narrative example even just using traditional tools... }\yw{I think the emphasis here is that the author creates concrete storylines vs. abstract outline. In both cases they are creating narrative space. "Narrative example" is a misleading term. The middle diagram is describing existing creation paradigm for AI-bridged IN, not ours.  }}
%    \label{fig:multipivot}
%    \Description{}
%\end{figure*}

% [story vs. narrative. plot]


% Describe the distinct properties of a narrative space (game plot): a graph full of events. Complexity of this graph. Two types of authorial intent: procedural intent and declarative intent. 

% [abstraction]

% Three ways to represent narrative space:

% - Narrative Instance: a sample path in the space, good to show the details and procedural intent

% - Narrative Variant: a sub-plot in the space, good to perceive the diff and variations

% - Narrative Outline: high-level fuzzy description of important events; easy to see the authorial intent; hard to perceive details etc.

% \paragraph{From Narrative Instance to Narrative Space (or Executable Game-plot)}

% \paragraph{From Narrative Variants to Narrative Space (or Executable Game-plot)}

% \paragraph{From Narrative Outline to Narrative Space (or Executable Game-plot)}


%\subsection{Current Paradigm of AI-bridged Interactive Narrative Creation}

%\yw{probably should define narrative space more explicitly}
%\yw{it's not easy to grasp the connection between ``outline'' and ``prompt''}

%The creation of interactive narratives has moved beyond traditional paradigms where authors directly write content for players. Instead, with AI represented by LLMs, creators now develop a "narrative space," the conceptual layer based on which AI can further do creations to produce the final artifact ~\cite{moreno2007documental, gmeiner2023dimensions,lee2024navigating,wu2024role,margineda2024development}. Approaches such as setting personas for characters and driving AI behavior based on these personas (e.g., AI Village ~\cite{park2023generative}), or evolving stories within predefined game settings (e.g., AI Dungeon ~\cite{ai_dungeon2}), all exemplify this paradigm of AI-bridged interactive narrative creation. 


%Interactive Narrative allows players to influence storylines through their actions, with authors creating a {\em narrative space} of possible storylines. AI-bridged interactive narrative generates the narrative space using prompts, enhancing player agency and enabling emergent narratives beyond predefined branches.

%A clear demonstration of this paradigm involves using LLMs (e.g., ChatGPT) to function as a game engine for a text adventure game based on a narrative space revolving around an existing story. Specifically, an outline prompt will be provided to describe the narrative space, such as:

%\whosays{Act as a game engine that turns the following story into a text-adventure game. [Story Outline] The narrative flow should emulate the pacing and events of the story as closely as possible, ensuring that choices do not prematurely advance the plot. Pause the game after 5 rounds. After that...~\cite{10333140}}

%In this setup, the narrative space is anchored to the provided story with this outline determining its boundary. During gameplay, the player might input actions like "go to the mountain, and explore" via text, to which the AI responds dynamically to make the entire narrative aligned with the outline. For instance, if "the deer meets with the player" is specified in the outline, the AI arranges appropriate events, such as "The deer also moves to the mountain," to facilitate this outcome. Therefore, the outline directs the AI creation, defining the narrative space by specifying key events while allowing variations.  Through these interactions, the AI and player collaboratively unfold the narrative within the space.

%To obtain an ideal setting of the narrative space, creators need to put effort into refining the outline in the prompt. To do so, a common strategy for novice creators is to start with a concrete narrative example ~\cite{tomlinson2006learning,micallef2024use}. Based on the example, they can define features of the narrative space like how the play-time narrative can deviate from the example using specifications (e.g., "keep the original pace"), or identifying key moments (e.g., "this character dies after the first battle."). With iteratively generating examples, creators extract the desirable or undesirable elements and summarize them as rules for the outline to guide the narrative space. For instance, a rule might be "killing characters happens after the first scene." After numerous trials refining prompts, creators may feel they have clearly established the boundaries of the narrative space. At this point, creators may further test the game by anticipating potential player actions in gameplay. Then, despite creators' initial efforts at refinement, exceptions may still arise and make the narrative deviate from their intent. For instance, creators may not know that LLM could use implicit action "attack" and lead to the unexpected death of characters, or the abandonment of killing action makes the narrative fail to describe a key scene. Alternatively, a player might kill an important character in the first scene, disrupting the intended storyline. This leads to further rounds of revisions until the creator is again satisfied with the defined narrative space. Even so, creators may still lack a clear understanding of potential outcomes in actual gameplay.

%This highlights the issues lying in the mutual transformation between the narrative example and outline in AI-bridged interactive narrative creation. First, creators typically generate narrative examples and use them to derive descriptions for the outline to define the narrative space——both of which are developed heuristically. As a result, the outline that bound the narrative space is obtained through a ruleless trial-and-error process, which is inherently \textbf{intractable}. Moreover, the outline is gradually concretized by AI into narrative examples, which are further influenced by players' actions during gameplay. In this process, creators often lack a clear mental model of the AI's or the player's potential behaviors, making it difficult to set effective boundaries for the narrative space. This leaves the AI-driven creation process largely \textbf{uncontrollable}.


% Currently, large language models (LLMs) used in game development usually simply use LLMs to enrich the game content. Figures 1.A and 1.B show two examples of widely used AI-powered games. In Figure 1.A (Co-generation of Game Plots), without authors' specification on a concrete story, users and the LLM model alternately generate plot content. Rather than functioning strictly as a game, this setup resembles a collaborative writing task, where both the player and the LLM jointly contribute to script writing. In such a mode, the game creator's expression makes a minimal impact. In Figure 1.B (LLM-Generated Character Behavior), LLMs are used to enhance character behavior, particularly by adding vividness and detail to the language of the characters in the game plots. However, in this case, the game narrative remains fixed, with LLMs enriching the expression of pre-defined storylines. Despite of action choices provided to players in the game, this interaction often creates only the illusion of influencing the story, as the underlying narrative remains unaffected. While players may interact with characters by, for example, typing “hello,” their actions have no meaningful impact on the broader narrative.


% However, an ideal narrative-driven game should balance authorial intent and interactivity. Authorial intent refers to the narrative the author aims to convey, while interactivity refers to the player’s ability to meaningfully influence the story. The goal is to enable players to alter the narrative in impactful ways while maintaining a coherent story with clear moral expression from the author. However, neither is the approach in Figure 1. A nor Figure 1. B is capable of fully achieving this balance. 


% In interaction mode like AI Dungeon, players' interactivity is largely freeform, but there is little to no authorial intent guiding the experience. This makes it difficult for an author to predict or control what players will experience. On the other hand, using LLMs to enhance character behavior without altering the core storyline leaves the narrative largely unchanged. 

% Both interaction modes that leverage LLMs nowadays to create narrative-driven games overlook a critical aspect of typical game creation: the goal is not to generate a specific story but rather to create a narrative space. A narrative space encompasses all the possible ways a story can evolve while maintaining the same core theme or moral. For example, in a story centered around the theme of kindness, one version might depict an ant saving a dove in a forest, while another could feature the dove saving the ant. Despite the differences in these versions, both convey the same moral—kindness. By creatively arranging story elements, game creators can unfold multiple stories within a single narrative space. Typically, such work involves extensive manual planning by game creators, who write multiple branches and versions of the narrative to explore different storylines. Without addressing this concept of a narrative space, LLM-powered systems either lock players into a fixed story or leave the narrative highly open-ended, lacking the necessary authorial guidance. The challenge lies in designing LLM systems that allow for meaningful interactivity while preserving the author’s intent within a flexible narrative space, and effectively unfolding the narrative space into game plots.

% [draft, not satisfied.]







%\subsection{Shaping Narrative Space for AI-bridged Game Plot Generation}

\label{design_goals}

Interactive Narrative allows players to influence storylines through their actions, with authors creating a narrative space of possible storylines. \revision{During the authoring process, authors predefine the range of player actions and creates multiple storylines reflecting the consequences of different player choices.} AI-bridged IN generates just-in-time narrative content that adapts to different game world states, freeing authors from enumerating storylines. 
%In this workflow, authors prompt the AI model with an abstract narrative specifications, such as: 
%\whosays{Act as a game engine that turns the following story into a text-adventure game. [Story Outline] The narrative flow should emulate the pacing and events of the story as closely as possible, ensuring that choices do not prematurely advance the plot. Pause the game after 5 rounds. After that...~\cite{playbrary}}
However, shifting from traditional IN to AI-bridged IN presents challenges for authors in expressing, perceiving, and controlling the narrative space. Authors often struggle to articulate their implicit narrative intents in high-level prompts \cite{mirowski2023co} and may underexpress their intent to AI systems \cite{kreminski2024intent}. While novice authors might start with a concrete example \cite{tomlinson2006learning,micallef2024use}, a single narrative instance can be both overly detailed and insufficient, as it includes unnecessary specifics and lacks broader context \cite{kreminski2024intent}. Therefore, neither concrete instances nor abstract specifications alone are ideal for defining a narrative space. Instead, the ability to configure the level of abstraction is necessary to support AI-bridged IN authoring. 

On the other hand, once a narrative space is defined via prompts, the author has limited insight into the player experience, as players are responded with unscripted character actions and dialogs generated by LLMs at play-time. It is difficult to identify and prevent the deviations beyond the the author's narrative intent. Therefore, it is important to provide valuable information on how different types of player could react to instances, through which authors could preview the narrative instances as they get unfolded in the player experience \cite{kim2023language}. Furthermore, transforming prompts into meaningful game plots is not trivial. It requires effective narrative planning to generate causally sound event sequences. Central to this success is modeling the logical causal progression of the game plot \cite{riedl2010narrative}. However, LLMs are not natively planners in creating causal progression and have been found to cause hallucinations without external verifier to validate the coherency and executability of the generated plan \cite{kambhampati2024llms}. 
%Building on existing research in interactive narrative, we examined the current AI-bridged creation paradigm and identified several key challenges. The first issue is the difficulty in defining appropriate boundaries for the narrative space. Currently, the process of generating narrative examples and summarizing them into higher-level abstractions is largely heuristic, relying on intuitive thinking. However, constructing a narrative space—particularly through language-based boundaries like an outline—requires thorough thinking, and involves careful organization of language and structure. 
%Another challenge lies in understanding the boundaries of AI’s role within the narrative space. Creators need a clear mental model of how AI operates within this space to guide their creation. Without this understanding, there can be a disconnection between the creator’s intent and the final artifact built by the AI. Finally, the unpredictable behaviors of players interacting with the narrative add another layer of complexity~\cite{marincioni2024effect,peng2024player,you2024dungeons}. As player choices influence the AI's ongoing narrative development, unexpected outcomes can arise, making it difficult to maintain coherence in the story. This uncertainty makes it harder for creators to ensure a controllable narrative experience. 
% : (1) the challenges of understanding and editing "narrative space" in creating AI-driven interactive narratives, even though it’s important for LLM-powered interactive narrative, and (2) how to effectively turn a narrative space into a playable game plot.
% Many people find it hard to work with narrative space because it requires thinking beyond a single, straight story, even in traditional interactive narrative creation. This becomes even more complex when LLM-driven characters are involved. Even experienced writers may struggle to break down a linear story into a flexible structure that allows for different paths and outcomes. However, narrative space is essential for creating interactive plots that go beyond one storyline. Our framework helps users by providing them with intuitive representations of the narrative space and guiding them on how to sculpt the narrative space leveraging the concept of abstraction. By abstracting from a concrete story, users can distill the core elements of the plot and build multiple interactive possibilities around them,
Motivated by the challenges unresolved in AI-bridged IN, we developed the following design goals to guide the design of system:

\textbf{DG1: Enable users to perceive the narrative space.} 
The narrative space in AI-bridged IN contains various possible storylines, which are generated at play-time based on player actions. Authors might struggle to envision what types of variations would be possible. The system should provide representations that can help authors to explore and understand the narrative space. 
%Narrative examples are intuitive but insufficient in picturing the entire narrative space. Outlines, on the other hand, offer a structured view of the narrative space but are more abstract and difficult to derive efficiently. For novice creators, relying on solely either view can not provide them with a clear vision of the narrative space. We then aim to provide both formats of narrative examples and outlines as descriptors of the narrative space for a better perception of the space. 
%important because it prevents narrative instances that deviates from the desired narrative space

% Narrative space is challenging to grasp due to it inherently indicates analytical planning of the narrative. Even experienced writers may struggle to generate into a flexible structure that allows for different paths and outcomes. To address this, we focus on providing clear, intuitive presentations of the narrative space that highlight its important characteristics. By offering visual or structural representations, users can more easily perceive and comprehend the complexity of the narrative space.


\textbf{DG2: Support configurable level of abstraction in editing narrative space.} 
Concrete instances can be overly detailed, while abstract specifications can be too vague. Supporting users to adjust the level of abstraction helps them balance between details and abstraction, which allows the narrative instances to emerge from player interactions, while still adhering to the narrative structure. 
%Narrative examples are intuitive but insufficient in picturing the entire narrative space. Outlines, on the other hand, offer a structured view of the narrative space but are more abstract and difficult to derive efficiently. For novice creators, relying on solely either view can not provide them with a clear vision of the narrative space. We then aim to provide both formats of narrative examples and outlines as descriptors of the narrative space for a better perception of the space. 


%In addition to perceiving the narrative space, users also need effective ways to edit and refine it by setting appropriate boundaries. We propose using abstraction as a simple yet powerful tool that allows users to transform their straightforward stories into boundaries of narrative space. By applying high-level concepts through abstraction, users can more easily shape and adjust the narrative space to align with their creative vision. Editing narrative space via language inherently involves analytical thinking. Thus, Despite the competence of existing LLMs, their unstructured free-form communication with laypeople does not fully unlock their potential in such a specific domain. 
%To solve this, we will provide structured tools that guide users in using both flexible and systematic methods to shape their narrative space. By integrating these tools with LLM assistance, users will gain a better understanding of how and why certain narrative spaces are formed, allowing them to make more informed edits while maintaining flexibility in their creative process.

\textbf{DG3: Generate meaningful game events that react to player actions at play-time.} An engaging player experience requires the generated plots to represent logical causal progression that follows the game mechanism.
%in the game plot.
The proposed system should support simulating casual dynamics and generate meaningful narrative content based on player actions. 


%According to classic interactive narrative design principles, two key perspectives must be considered to reach a balance. First, the narrative should include enough diverse paths to ensure variety in the plot. Second, players’ actions should meaningfully impact the narrative’s development, altering the direction of their experience. Leveraging LLMs provides a solid foundation for generating diverse plot outcomes that respond dynamically to player input. Therefore, our approach to unfolding the narrative space into game plots utilizes LLMs to creatively generate content in real-time, offering interactivity that adapts to the players' actions.
%"Narrative and interactivity must be developed concurrently". Thus, the authorial intent in narrative construction must also be preserved in plot generation. LLMs are bad at this. Thus, we leverage the idea of symbolic planning, with authorial intent to characterize the narrative space serving as planning needs. 








\section{Method}
In this section, we introduce our zero-shot depth completion method, which leverages the depth prior~\cite{ke2023repurposing, gui2024depthfm}
derived from the foundation model~\cite{rombach2022highresolution}. This 
enables our method to be generalizable across any domain.
The core concept of our approach is to align the affine-invariant depth prior with sparse measurements on an absolute scale to complete the dense and well-structured depth map, as illustrated in \Fref{fig:concept}.

\subsection{Preliminary}
\label{sec:preliminary}
\para{Diffusion model and guided sampling}
Diffusion models~\cite{ho2020denoising, song2022denoising} aim to model data distribution $p(\mathbf{x})$ through iterative perturbation and restoration, known as forward and reverse processes.
This is represented by the score-based generative model~\cite{song2021scorebased}, learning the score function $\mathbf{s}_\theta$ parameterized by $\theta$ the gradient of the log probability density function with respect to the data, \ie, $\mathbf{s}_\theta(\mathbf{x}) =\nabla_{\mathbf{x}} \log p(\mathbf{x};\theta)$.
Score-based diffusion models 
estimate the score $\mathbf{s}_\theta(\mathbf{x}_t)$ at intermediate state $\mathbf{x}_t$ for timestep $t$ which defines a process.

For image generation and editing, diffusion models leverage the guidance function during the sampling process to adjust the output to the specific condition
~\cite{ho2022classifierfree, dhariwal2021diffusion}.
The guidance can be 
defined 
by any differentiable mapping output to guidance modality, as follows~\cite{bansal2024universal}: 
\begin{equation}
\label{eq:guide_sampling}
\hat{\mathbf{s}}_{\theta}(\mathbf{x}_t, t, \mathbf{y}) = \mathbf{s}_{\theta}(\mathbf{x}_t, t) + w \nabla_{\mathbf{x}_t} \mathcal{L}\left(f\left(\mathbf{x}_0\left(\mathbf{x}_t\right)\right),\mathbf{y}\right),
\end{equation}
where $w$ and $\mathbf{y}$ represent weight and guidance, respectively.
The function $f(\cdot)$ can be any differentiable function whose output can compute a loss $\mathcal{L}$ with guidance condition $\mathbf{y}$, and
$\mathbf{x}_0\left(\mathbf{x}_t\right)$ is obtained by using Tweedie's formula~\cite{efron2011tweedie}, 
which provides an approximation of the posterior mean. 
This guided sampling approach extends unconditional diffusion models to conditional ones without separate model training.

\para{Inverse problem}
The goal of an inverse problem is to determine an unknown variable from known measurement, often formulated as $\mathcal{A}(\mathbf{x})=\mathbf{y}$,
where $\mathcal{A}{:}\, \mathbb{R}^m {\rightarrow}  \mathbb{R}^n$ represents the known forward measurement operator, $\mathbf{y}\in \mathbb{R}^n$ and $\mathbf{x}\in \mathbb{R}^m$,
the measurement and the unknown variable, respectively.
When 
$m>n$, it becomes an ill-posed problem, requiring a prior to find 
solve a
Maximum A Posterior (MAP) estimation:
\begin{equation}
\label{eq:map}
    \argmax p(\mathbf{x}|\mathbf{y})\propto p(\mathbf{x}) p(\mathbf{y} | \mathbf{x}),
\end{equation}
where $p(\mathbf{x})$ represents our prior of the signal $\mathbf{x}$ and $p(\mathbf{y} | \mathbf{x})$ is likelihood measuring 
$\mathcal{A}(\mathbf{x})\approx\mathbf{y}$, \eg, $\|\mathbf{y} {-} \mathcal{A}(\mathbf{x})\|_2^2$.
By taking $-\log(\cdot)$ to \Eref{eq:map}, it can be  
formulated as an optimization problem
that regularizes the solution, ensuring that $\mathbf{x}$ follows the characteristics of the prior:
\begin{equation}
    \label{eq:inv_opt}
    \argmin_{\mathbf{x}} \left\|\mathbf{y} - \mathcal{A}\left(\mathbf{x}\right) \right\|_2^2 - \log p(\mathbf{x}).
\end{equation}
Also, given the gradient of $\log p(\mathbf{x}|\mathbf{y})$ in \Eref{eq:map} as
\begin{equation}
    \nabla_{\mathbf{x}}\log p(\mathbf{x}|\mathbf{y}) =  \nabla_{\mathbf{x}}\log p(\mathbf{x}) + \nabla_{\mathbf{x}}\log p(\mathbf{y} | \mathbf{x}),
\end{equation}
the prior term $\nabla_{\mathbf{x}}\log p(\mathbf{x})$ corresponds to the score $\mathbf{s}_{\theta}(\mathbf{x})$, which can be obtained by diffusion models.
Therefore, by simply adding the gradient of the likelihood term to the reverse sampling process, the inverse problem can be effectively solved while leveraging the diffusion prior~\cite{chung2023dps}
as follows:
\begin{equation}
    \label{eq:inv_sampling}
    \hat{\mathbf{s}}_{\theta}(\mathbf{x}_t, t, \mathbf{y}) = \mathbf{s}_{\theta}(\mathbf{x}_t, t) + w \nabla_{\mathbf{x}_t}\left\| \mathbf{y} - \mathcal{A}\left(\mathbf{x}_0\left(\mathbf{x}_t\right)\right) \right\|_2^2.
\end{equation}
This has an analogous form with \Eref{eq:guide_sampling}; thus, the inverse problem can be effectively tackled with the guided sampling.


With pre-trained image diffusion models, \eg, \citet{rombach2022highresolution}, as the score function $\mathbf{s}_{\theta}(\mathbf{x})$ and a prior, it provides 
powerful image prior across various tasks by its 
comprehensive semantic understanding and structural knowledge learned from a lot of images~\cite{wang2023exploiting, namekata2024emerdiff}.
\citet{ke2023repurposing} leverage this rich visual knowledge to achieve generalizable monocular depth estimation, resulting in high-quality outputs within an affine-invariant depth space. In our work, we exploit this depth diffusion model for computing the score as a depth prior.

% \begin{figure}
%     \centering
%     \includegraphics[width=0.5\linewidth]{Move_teaser.pdf}
%     \caption{Comparison of different dynamic compute approaches. length of arrow indicates residual transformation per token while width indicates velocity of transformation.}
%     \label{fig:enter-label}
% \end{figure}

\section{Method}
\label{sec:method}
Residual connections play a crucial role in shaping token representations, yet their dynamics remain underexplored in the context of efficient decoding. In this work, we delve deeper into transformer residual dynamics and investigate how modulating residual transformation velocity can improve inference efficiency in token-level processing, optimizing both dense and sparse MoE transformers.


\subsection{Residual Dynamics and Motivation for Multi-rate Residuals} \label{sec:motivation}

To analyze how hidden representations evolve across different layers of a transformer architecture, it's crucial to consider the effect of residual connections. Each transformer decoder layer typically has residual connections across attention and MLP submodules. As the residual stream $h_i$ traverses from interval $E_j$ to $E_{j+1}$, it undergoes a residual transformation given by:  
% \begin{equation}
% \label{eq:slow_residual_transformation}
% H_{E_{j+1}} = H_{E_j} \prod_{i=E_j}^{E_{j+1}} \left( I + \mathcal{A}_i \right) \left( I + \mathcal{M}_i \right) \quad \text{where} \quad \mathcal{A}_i = f(c_i, h_{i}), \mathcal{M}_i = g(h_i)
% \end{equation}

\begin{equation} \label{eq:slow_residual_transformation}
h_{E_{j+1}} = h_{E_j} + \sum_{i=E_j}^{E_{j+1}-1} \left( \mathcal{A}_i(h_i) + \mathcal{M}_i(h_i + \mathcal{A}_i(h_i)) \right) \quad \text{where} \quad \mathcal{A}_i = f(c_i, h_{i}), \mathcal{M}_i = g(h_i). 
\end{equation}

Here, \( \mathcal{A}_i \) denotes the non-linear transformation introduced by the multi-head attention mechanism at layer \( i \), while \( \mathcal{M}_i \) corresponds to the non-linear transformation of the MLP block at the same layer. These transformations depend on the input residual stream \( h_i \) and, in the case of \( \mathcal{A}_i \), the previous contextual representation \( c_i \).\footnote{Normalization layers are typically applied in practice but are omitted here for simplicity of the argument.}


% For easy tokens, the magnitude and direction of this delta transformation become progressively smaller with each successive layer as shown in \cref{fig:delta_transformation}. Consequently, it is feasible to predict these tokens after only a few residual connections, whereas harder tokens necessitate more extensive processing through additional layers.

\begin{figure}[ht]
    \centering
    \begin{subfigure}{0.48\textwidth}
        \centering
        \includegraphics[width=\textwidth]{sections/figures/residual_change.pdf}
        \caption{}
        \label{fig:residual_change}
    \end{subfigure}%
    \hfill
    \begin{subfigure}{0.48\textwidth}
        \centering
        \includegraphics[width=\textwidth]{sections/figures/alignment_wrt_dedicated_model.pdf}
        \caption{}
    \label{fig:alignment_wrt_dedicated_model}
    \end{subfigure}
    \caption{(a) As residual streams propagate through the model, the directional shifts in the residuals become progressively smaller. (b) A dedicated model with $k$ layers achieves a faster rate of change in residual streams and higher alignment than base model leveraging early exit mechanisms at layer $k$.}
    \label{fig}
\end{figure}


To examine whether residual transformations can be accelerated across layers, we conducted experiments using a diverse set of prompts on a pre-trained Phi3 model~\cite{phi3_report}. As illustrated in \cref{fig:residual_change}, we measured the directional shift in residual states as \( 1 - \mathcal{C}(h_{i-1}, h_i) \), where \(\mathcal{C}\) denotes normalized cosine similarity. This shift is notably higher in the initial layers, gradually decreasing in subsequent layers. This behavior allows traditional early exit approaches to effectively accelerate decoding by enabling earlier exits for simpler tokens. However, these approaches typically rely on a distance-based approximation, where the full residual transformation of the model is approximated by the residual transformations of the initial layers. To gain deeper insights into the distance versus velocity aspects of residual transformation, we conducted a comparative study. Specifically, we trained an early exit head at layer $k$ of the Phi3 model, which consists of 32 layers, restricting the distance traveled by each token. To accelerate the residual transformation relative to number of layers, we trained a smaller model consisting of only $k$ layers, while keeping all other hyperparameters consistent. We then compared the next-token prediction accuracy of the early exit head of the base model with that of the smaller model. To ensure an equal number of trainable parameters, we inserted low-rank adapters into the smaller model and trained only these adapters, whereas, in the distance-based approach, we trained solely the early exit head. In addition, to accelerate the residual transformation in smaller model, we distilled the residual streams from the larger model by incorporating a distillation loss ~\cite{sanh2019distilbert} between the residual state at layer \(i\) of the smaller model and the residual state at layer \(4 \times i\) of the larger model. As shown in ~\cref{fig:alignment_wrt_dedicated_model} the smaller model demonstrates a significantly faster rate of change in residual streams, leading to higher next token prediction accuracy after $k$ layers compared to the base model that employs traditional early exit mechanisms after $k$ layers \cite{schuster2022confident, chen2023eellm, varshney-etal-2024-investigating}. This experimental setup, which modifies only the rate of change in residual streams while keeping other factors constant, suggests that dense transformers, trained with a fixed number of layers, may inherently possess a slow residual transformation bias.

This observation raises an intriguing question: if the rate of change in residual streams could be accelerated relative to the number of layers, is it possible to facilitate earlier alignment for a greater proportion of tokens? Earlier alignment would be beneficial to not only facilitate dynamic computation but also for generating speculative tokens efficiently with high acceptance rates in speculative decoding setups ~\cite{leviathan2023fast, chen2023accelerating}. 

%thereby enhancing the efficiency of early exiting? 
 % This bias likely constrains the effectiveness of early exiting, particularly for easier tokens. By addressing this limitation through accelerated residual transformations, we hypothesize that it is possible to substantially improve the efficiency and accuracy of early exit strategies in transformer models.

\subsection{Multi-Rate Residual Transformation} \label{m2r2_method}

To address the slow residual transformation bias described in ~\cref{sec:motivation}, we introduce \textit{accelerated residual streams} that operate at rate $R$ relative to original slow residual stream. We pair slow residual stream, $h$ with an accelerated residual stream, $p$, which has an intrinsic bias towards earlier alignment. Relative to ~\cref{eq:slow_residual_transformation}, accelerated residual transformation from interval $E_j$ to $E_{j+1}$ can be represented as: 

% \begin{equation}
% \label{eq:fast_residual_transformation}
% P_{E_{j+1}} = P_{E_j} \prod_{i=E_j}^{E_{j+1}} \left( I + \hat{\mathcal{A}_i} \right) \left( I + \hat{\mathcal{M}_i} \right) \quad \text{where} \quad \hat{\mathcal{A}_i} = \hat{f}(c_i, P_{i}), \hat{\mathcal{M}_i} = \hat{g}(P_{i})
% \end{equation}


\begin{equation} \label{eq:fast_residual_transformation}
p_{E_{j+1}} = p_{E_j} + \sum_{i=E_j}^{E_{j+1}-1} \left( \hat{\mathcal{A}_i}(p_i) + \hat{\mathcal{M}_i}(p_i + \hat{\mathcal{A}_i}(p_i)) \right) \quad \text{where} \quad \hat{\mathcal{A}_i} = \hat{f}(c_i, p_{i}), \hat{\mathcal{M}_i} = \hat{g}(h_i), 
\end{equation}



where $\hat{\mathcal{A}_i}$ and $\hat{\mathcal{M}_i}$ denote non-linear transformation added by layer $i$ to previous accelerated residual $p_{i}$. Similar to $\mathcal{A}_i$, non-linear transformation $\hat{\mathcal{A}_i}$ attends to same context $c_i$ but uses a different transformation $\hat{f}$ for accelerating $p_{E_j}$ relative to $h_{E_j}$. 

We integrate accelerated residual transformation directly into the base network using parallel accelerator adapters such that rank of accelerator adapters $R_p << d$ where $d$ denotes base model hidden dimension. This setup allows the slow residual stream $h_{E_j}$ to pass through the base model layers while the accelerated residual stream $p_{E_j}$ utilizes these parallel adapters as shown in ~\cref{fig:m2r2_main}. Both slow and accelerated residuals are processed in same forward pass via attention masking and incur negligible additional inference latency in memory bound decoding setups, while in compute bound decoding setups where FLOPs optimization is essential, accelerated residual stream utilizes a fraction of attention heads that of slow residual (see ~\cref{sec:flops_optimization}). Additionally, to maximize the utility of accelerated residual transformations without introducing dedicated KV caches, we propose a shared caching mechanism between the slow and accelerated streams which minimally impact alignment benefits of our approach while offering substantial memory savings (see ~\cref{fig:koala_alignment}). Specifically, the attention operation on the slow residuals \( \text{MHA}(h_t, h_{\leq t}, h_{\leq t}) \) is redefined for accelerated residuals as 
\[
\hat{\mathcal{A}} = MHA(p_t, h_{<t} \oplus p_t, h_{<t} \oplus p_t),
\]
where the accelerated residual at time-step $t$, \( p_t \) attends to the slow residual’s KV cache, facilitating the reuse of contextual information across both residual streams without incurring additional caching costs. Here, \(MHA(q, k, v) \) represents multi-head attention between query \( q \), key \( k \), and value \( v \).

\begin{figure}
    \centering
    \includegraphics[width=0.8\linewidth]{sections//figures/m2r2_main2.pdf}
    \caption{Multi-rate Residuals Framework: Slow residual stream of base model is accompanied by a faster stream that operates at a $2-(J+1)\times$ rate relative to the slow stream, undergoing transformations via accelerator adapters as detailed in \cref{m2r2_method}, where J denotes number of early exit intervals. Colors within the slow and fast residual streams indicate similarity, with matching colors representing the most closely aligned residual states. At the beginning of the forward pass and at each exit point, the accelerated residual state is initialized from the corresponding slow residual state to avoid gradient conflict during training (see ~\cref{sec:grad_conflict}). Early exiting decisions are informed by the Accelerated Residual Latent Attention (ARLA) mechanism, described in \cref{method_arla}, which evaluates residual dynamics across consecutive exit gates.}
    \label{fig:m2r2_main}
\end{figure}

% Furthermore. to maximize the benefits of fast residual transformations without using dedicated KV caches, we propose sharing the fast network’s cache with the slow network. Formally speaking, We modify attention operation on slow residuals $MHA(H_t, H_{<=t}, H_{<=t})$ as $MHA(P_{t}, H_{<t} \oplus P_t, H_{<t}  \oplus P_t)$ such that accelerated residuals attend to previous slow context KV cache, where $MHA(q,k,v)$ denotes multi head attention between query, $q$, key $k$ and value $v$.


\subsection{Enhanced Early Residual Alignment}
Early residual alignment is instrumental in optimizing early exiting, speculative decoding, and Mixture-of-Experts (MoE) inference mechanisms. In this section, we provide a detailed analysis of how accelerated residuals enhance these inference setups.

% By aligning the residual states of intermediate layers with the final output representations, the model can maintain high prediction accuracy even when computations are truncated at earlier layers. This enables more reliable early exiting, reducing the overall computational cost while preserving performance. Additionally, in speculative decoding, early residual alignment allows the model to make confident predictions using faster, partial computations, thereby accelerating inference without sacrificing output quality.


\subsubsection{Early Exiting} \label{method_early_exiting}

A prevalent strategy for enabling early exiting at an intermediate layer $E_{j}$ involves approximating the residual transformation between $E_{j}$ and the final layer $N-1$ using a linear, context independent mapping, $\mathcal{T}$, such that $H_{N-1} \approx \mathcal{T}(H_{E_{j}})$. This approximation has been extensively employed in conventional approaches ~\cite{schuster2022confident, chen2023eellm, varshney-etal-2024-investigating}, providing a computationally efficient means to project the output of deeper layers from intermediate states. Specifically, residual state of layer $N-1$ with this approximation can be expressed as:


% \begin{equation}
% \label{eq: vanila_ea_assumption}
% \Phi(H_{E_{j}}) \sim H_{E_{j}} \prod_{i=E_{j}}^{N}\left( I + \mathcal{A}_i \right) \left( I + \mathcal{M}_i \right) \quad \text{where} \quad \Phi \perp C
% \end{equation}

\begin{equation} \label{eq:early_exiting}
h_{E_j} + \sum_{i=E_j}^{N-1} \left( \mathcal{A}_i(h_i) + \mathcal{M}_i(h_i + \mathcal{A}_i(h_i)) \right) \sim \mathcal{T}(h_{E_{j}})  \quad \text{where} \quad \mathcal{T} \perp c. 
\end{equation}


Here, $\mathcal{A}_i$ and $\mathcal{M}_i$ represent the residual contributions of the multi-head attention and MLP layers, respectively, while $\mathcal{T}$ remains independent of $c$, the preceding context.

This approach is inherently limited by two major factors: first, the assumption of linearity between $h_{E_{j}}$ and $h_{N-1}$ may not hold uniformly for all tokens, particularly when $E_j \ll N$. Second, the linear transformation $\mathcal{T}$ disregards the influence of the context $c$ and fails to account for the latent representations of previous contextual states. In contrast, M2R2 accelerated residual states mitigate both of these challenges by approximating the slow residual transformation of all layers via a faster residual transformation of fewer layers as:
% \begin{equation}
% H_{E_j} \prod_{i=E_j}^{N}\left( I + \mathcal{A}_i \right) \left( I + \mathcal{M}_i \right) \sim P_{E_j} \prod_{i=E_j}^{E_j+1}\left( I + \hat{\mathcal{A}_i} \right) \left( I + \hat{\mathcal{M}_i} \right)
% \end{equation}


\begin{equation} \label{eq:m2r2_approximating_ea}
h_{E_j} + \sum_{i=E_j}^{N-1} \left( \mathcal{A}_i(h_i) + \mathcal{M}_i(h_i + \mathcal{A}_i(h_i)) \right) \sim p_{E_j} + \sum_{i=E_j}^{E_{j+1}-1} \left( \hat{\mathcal{A}_i}(p_i) + \hat{\mathcal{M}_i}(p_i + \hat{\mathcal{A}_i}(p_i)) \right), 
\end{equation}

% \begin{equation} \label{eq:fast_residual_transformation}
% p_{E_{j+1}} = p_{E_j} + \sum_{i=E_j}^{E_{j+1}-1} \left( \hat{\mathcal{A}_i}(p_i) + \hat{\mathcal{M}_i}(p_i + \hat{\mathcal{A}_i}(p_i)) \right) \quad \text{where} \quad \hat{\mathcal{A}_i} = \hat{f}(c_i, p_{i}), \hat{\mathcal{M}_i} = \hat{g}(h_i) 
% \end{equation}






where $p_{E_j}$ is initialized from the slow residual state $h_{E_j}$ at each early exit interval $E_j$ using an identity transformation (see ~\cref{fig:m2r2_main}). As shown in ~\cref{fig:m2r2_residual_sim}, accelerated residuals offer a smoother, more consistent shift in residual direction across layers, in contrast to the abrupt changes typically seen at early exit points in standard early exit methods. Moreover, the normalized cosine similarity between accelerated states at early exit intervals and final residual states is substantially higher compared to traditional early exit techniques, highlighting improved alignment with final layer representations. Traditional adaptive compute methods are constrained by two principal factors: the number of tokens eligible for early exit at intermediate layers and the precision of early exit decision. If residual streams fail to saturate early, the majority of tokens remain ineligible for exit, thereby diminishing potential speedups. Additionally, imprecise delineations between tokens suitable for early exit can lead to underthinking (premature exits that adversely affect accuracy) or overthinking (unnecessary processing that compromises efficiency) ~\cite{zhou2020self, dai2020dynamic}. Enhanced early alignment using ~\cref{eq:m2r2_approximating_ea} helps to address  first issue. To address the second issue we introduce Accelerated Residual Latent Attention, which dynamically assesses the saturation of the residual stream, allowing for a more precise differentiation between tokens that can exit early and those requiring further processing.

% This results in uniform change in residual direction    
% % We keep $\mathcal{A} = \hat{\mathcal{A}}$, while $\hat{\mathcal{M}}$ is accelerated by a factor of $2 - (N_{E}+1)X$ relative to the slower residual transformation $\mathcal{M}$, where $N_E$ represents number of early exiting intervals.
% Figure~\cref{fig:rate_change_comparison} illustrates the comparative rate of change between these transformation streams.



% fig:rate_change_comparison
% - grid plot x axis -> layer id (0, 8) , y axis -> layer id -> dark color cell for max similarity , lighter for lower 
% 
-------------------------------------------------------
Let's consider residual stream $h_i$ traverses through interval $E_j$ to $E_{j+1}$ and undergoes residual transformation given by 
\begin{equation}
h_{E_{j+1}} = h_{E_j} \prod_{i=E_j}^{E_{j+1}} \left( 1 + \delta_i \right)    
\end{equation}

where $\delta_i$ denotes non-linear transformation added by layer $i$. Each non-linear transformation of layer $i$ is a function of previous contextual representation, $c_i$ and input residual stream $h_i-1$ as
$\delta_i = f(c_i, h_{i-1})$ 

One way to exit early at exit $E_j+1$ is to assume that residual transformation from $E_j+1$ to final layer $N-1$ can be approximated by a linear function $\phi$ as $h_{N-1} \sim \Phi(h_{E_j+1})$ and most conventional approaches such as \todo{cite EA papers} use this approach. In other words, 

\begin{equation}
\Phi(h_{E_j+1} \sim h_{E_j+1} \prod_{i=E_j+1}^{N} \left( 1 + \delta_i \right)   
\end{equation}

This approach suffers from two primary issues, linearity assumption from $h_E_j+1$ to $H_N-1$ if often incorrect, particularly when $E_j << N$. More importantly, linear transformation $\Phi$ doesn't consider effect of context $C_i$. M2R2  effectively addresses these issues as accelerated residual stream at interval $E_j+1$ can be represented as 

\begin{equation}
r_{E_{j+1}} = r_{E_j} \prod_{i=E_j}^{E_{j+1}} \left( 1 + \gamma_i \right)    
\end{equation}

where $\gamma_i$ denotes non-linear transformation added by layer $i$ to previous accelerated residual $r_i-1$. Similar to $\delta_i$, non-linear transformation $\gamma_i$ considers context $C_i$ as 
$\gamma_i = g(c_i, r_{i-1})$. So in summary, slow residual transformation is approximated by accelerated residual as: 

\begin{equation}
h_{E_j} \prod_{i=E_j}^{N} \left( 1 + \delta_i \right) \sim h_{E_j} \prod_{i=E_j}^{E_j+1} \left( 1 + \gamma_i \right)
\end{equation}

It's worth noting that accelerated residual $r_i$ and slow residual $h_i$ are processed concurrently at layer $i$ by constructing proper attention mask such as attention of slow residual is represented as 

$MHA(H_it, H_{i<=t}, H_{i<=t}$ while attention of fast residual is computed as 

$MHA(r_it, H_{i<=t}, H_{i<=t}$ where $MHA(q,k,v$ denotes multi head attention between query, $q$, key $k$ and value $v$.


------------------------------------------------------------------

Vertical latent attention on accelerated residual is computed as 
$MHA(S_mt, S(Ej<=i<=m)t, S(Ej<=i<=m)t)$ where $Smt$ denotes query/key/value projection in latent domain at layer $m$ at time $t$. 
------------------------------------------------------------------

Gradient conflict Avoidance: 

Let's consider $w_j$ is a trainable parameter that belongs to a layer between $E_j$ and $E_j+1$. Consider early exit loss at gate $E_j+1$, $L_j+1$, gradient propagation of $w_j$ at another trainable parameter $w_j-n$ can be gives as 

$\sum_{k=E_j-n}^{E_j} \beta_k \frac{\partial L_{E_k}}{\partial w_k}$

where $\beta_j$ denotes backward transformation coefficient for weight $w_j$ to reach gate $E_j$. 
 
On the other hand, gradient propagation in proposed approach can be represented as 

\[
\frac{\partial L_{E_j}}{\partial w_j} = 
\begin{cases} 
\beta_j \frac{\partial L_{E_j}}{\partial w_j} & \text{if } E_j \leq w_j \leq E_{j+1} \\
0 & \text{otherwise}
\end{cases}
\]







% \begin{figure}[ht]
%     \centering
%     \includegraphics[width=0.8\textwidth, height=5cm]{rate_change_comparison.png}
%     \caption{Rate of change comparison between fast and slow residual streams.}
%     \label{fig:rate_change_comparison}
% \end{figure}

%vary k and and plot EA accuracy for larger and smaller models. 

% \begin{figure}[ht]
%     \centering
%     \includegraphics[width=0.5\textwidth,height=5cm]{sections/figures/alignment_comparison_dialogsum.pdf}
%     \caption{Alignment of exited tokens for different early exit layers using traditional early exiting heads, dedicated faster networks, and faster residuals.}
%     \label{fig:small_model_early_exiting}
% \end{figure}


\textbf{Accelerated Residual Latent Attention} \label{method_arla}

In the context of residual streams, we observe that the decision to exit at a given layer can be more effectively informed by analyzing the dynamics of residual stream transformations, instead of solely relying on a classification head applied at the early exit interval $E_j$. To capture the subtle dynamics of residual acceleration, we propose a \textit{Accelerated Residual Latent Attention} (ARLA) mechanism. This approach involves making the exit decision at gate $E_j$ by attending to the residuals spanning from gate $E_{j-1}$ to $E_j$, rather than considering only the residual at gate $E_j$. To minimize the computational overhead associated with exit decision-making, the attention mechanism operates within the latent domain as depicted in ~\cref{fig:arla_arch}. Formally, for each interval $[E_j, E_{j+1}]$, the accelerated residuals are projected into Query ($Q^s_{E_j}, \ldots, Q^s_{E_{j+1}}$), Key ($K^s_{E_j}, \ldots, K^s_{E_{j+1}}$), and Value ($V^s_{E_j}, \ldots, V^s_{E_{j+1}}$) vectors, with latent dimension $d^s$ for $Q^s$, $K^s$, and $V^s$ being significantly smaller than hidden dimension of $p$.\footnote{We use $d^s = 64$ for experiments described in ~\cref{sec:experiments}.} Notably, when the router is allowed to make exit decisions at gate $E_j$ based on residual change dynamics, we observe that the attention is not confined to the residual state at $E_j$ but is distributed across residual states from $E_{j-1}$ to $E_j$, %as illustrated in Figure~\ref{fig:vertical_latent_attention_dynamics}. 
This broader focus on residual dynamics significantly reduces decision ambiguity in early exits, as demonstrated in Figure~\ref{fig:roc_arla}, which contrasts routers based on the last hidden state, and the proposed ARLA router.

%show R -> S transformation. 
%show parameter and flop overhead as compared to adapter on last hidden state.

% \begin{figure}[ht]
%     \centering
%     \includegraphics[width=0.5\textwidth,height=5cm]{sections/figures/roc_arla.pdf}
%     \caption{ROC curves of early exit decision strategies: confidence-based methods (CALM/LITE), routers based on the accelerated hidden state, and latent attention routers.}
%     \label{fig:decision_making_comparison}
% \end{figure}

% \begin{figure}[ht]
%     \centering
%     \includegraphics[width=0.5\textwidth,height=5cm]{vertical_latent_attention.png}
%     \caption{Vertical latent attention mechanism for optimizing early exit decisions by considering residuals from gate \(M\) through \(M-1\).}
%     \label{fig:vertical_latent_attention}
% \end{figure}

\begin{figure}[ht]
    \centering
    \begin{subfigure}{0.52\textwidth}
        \centering
        \includegraphics[width=\textwidth, height = 4cm]{sections/figures/arla_arch.pdf}
        \caption{Accelerated Residual Latent Attention (ARLA): Accelerated residuals between early exit gates are projected into latent domain and attention over residual states within the interval is computed to capture residual dynamics and exit decision is made based on residual saturation.}
        \label{fig:arla_arch}
    \end{subfigure}%
    \hfill
    \begin{subfigure}{0.45\textwidth}
        \centering
        \includegraphics[width=\textwidth, height = 4.5cm]{sections/figures/vla_roc.pdf}
        \caption{ROC classification curves of early exit decision strategies using a linear router used on last residual state ~\cite{schuster2022confident, varshney-etal-2024-investigating, chen2023eellm}  and using ARLA approach that considers residual dynamics. }
        \label{fig:roc_arla}
    \end{subfigure}
    \caption{Effectiveness of ARLA in capturing residual dynamics for early exiting decisions.}


\end{figure}



% \begin{figure}[ht]
%     \centering
%     \includegraphics[width=1\textwidth,height=5cm]{sections/figures/arla.pdf}
%     \caption{fig that plots 32 rows 2 cols heatmap showing attention at each gate}
%     \label{fig:vertical_latent_attention_dynamics}
% \end{figure}

\subsubsection{Self Speculative Decoding} \label{method_self_speculative_decoding}

An alternative means to exploit the early alignment properties of our approach is through the use of accelerated residual states for speculative token sampling to accelerate autoregressive decoding. Speculative decoding aims to speed up memory-bound transformer inference by employing a lightweight draft model to predict candidate tokens, while verifying speculated tokens in parallel and advancing token generation by more than one token per full model invocation \cite{leviathan2023fast, chen2023accelerating, xia2023speculative, miao2023specinfer}. Despite its effectiveness in accelerating large language models (LLMs), speculative decoding introduces substantial complexity in both deployment and training. A separate draft model must be specifically trained and aligned with the target model for each application, which increases the training load and operational complexity ~\cite{chen2023accelerating}. Additionally, this approach is resource-inefficient, as it requires both the draft and target models to be simultaneously maintained in memory during inference \cite{leviathan2023fast, chen2023accelerating}. 

One strategy to address this inefficiency is to leverage the initial layers of the target model itself to generate speculative candidates, as depicted in ~\cite{Tang2024}. While this method reduces the autoregressive overhead associated with speculation, it suffers from suboptimal acceptance rates. This occurs because the linear transformation employed for translating hidden states from layer $k$ to the final layer $N$ is typically a poor approximation, as discussed in ~\cref{sec:motivation} and ~\cref{method_early_exiting}. Our approach resolves this limitation by utilizing accelerated residuals, which demonstrate higher fidelity to their slower counterparts. By utilizing accelerated residuals operating at a rate of $N/k$, where $k$ denotes the number of layers used for candidate speculation, we are able to efficiently generate speculative tokens for decoding.\footnote{We typically set $k = 4$ to balance the trade-off between autoregressive drafting overhead and acceptance rate, as discussed in~\cref{sec:experiments}.}
 This technique not only obviates the need for multiple models during inference but also improves the overall efficiency and effectiveness of speculative decoding.

\begin{figure}
    \centering    \includegraphics[width=1\linewidth]{sections/figures/m2r2_aot_loading.pdf}
    \caption{Ahead-of-Time Expert Loading: M2R2 accelerated residual stream predicts experts required for future layers, reducing reliance on on-demand lazy loading. Speculative pre-loading is efficiently overlapped with computation of multi-head attention (MHA) and MLP transformations. Only incorrectly speculated experts are loaded lazily, resulting in faster inference steps and improved computational efficiency. Here, H indicates LBM Host while D indicates HBM Device.}
    \label{fig:moe_expert_aot_loading}
\end{figure}


\subsubsection{Ahead of Time Expert Loading:} \label{method_aot_expert_loading}

Recent advancements in sparse Mixture-of-Experts (MoE) architectures ~\cite{shazeer2017outrageously, fedus2022switch, artetxe2019massively, lepikhin2020gshard, zoph2022designing} have introduced a paradigm shift in token generation by dynamically activating only a subset of experts per input, achieving superior efficiency in comparison to dense models, particularly under memory-bound constraints of autoregressive decoding \cite{fedus2022switch, zoph2022designing}. This sparse activation approach enables MoE-based language models to generate tokens more swiftly, leveraging the efficiency of selective expert usage and avoiding the overhead of full dense layer invocation. In dense transformer models, pre-loading layers is a common strategy to enhance throughput, as computations of current layer can be overlapped with pre-loading of next layer parameters ~\cite{narayanan2021efficient, shoeybi2020megatron}. However, MoE models face a unique challenge: expert selection occurs dynamically based on previous layer’s output, making it infeasible to preload next layer’s experts in parallel. This limitation results in inherent latency, as expert loading becomes a sequential, on-demand process ~\cite{lepikhin2020gshard, fedus2022switch}.

To address this inefficiency, our method introduces a mechanism with \textit{accelerated residuals}, which not only captures key characteristics of base slower residual states but also exhibit high cosine similarity with their final counterparts (as illustrated in \cref{fig:m2r2_residual_sim}). By employing accelerated residual streams, we can effectively predict the necessary experts for future layers well in advance of their actual invocation. Specifically, using a $2\times$ accelerated residual, the experts needed for layers $2i+2$ and $2i+3$ can be identified while still computing in layer $i$, thus overcoming the bottleneck of sequential, on-demand expert selection and mitigating latency in the decoding pipeline, as shown in \cref{fig:moe_expert_aot_loading}. Note that, we use fixed set of accelerator adapters for transforming accelerated residuals (as discussed in ~\cref{m2r2_method}) while slow residual is transformed via expert routing mechanism. 

Furthermore, our approach integrates a Least Recently Used (LRU) caching strategy, which enhances memory efficiency by replacing the least recently used experts with speculated experts that are anticipated to be needed in upcoming layers. This hybrid approach of preemptive expert loading with LRU caching yields substantial improvements over traditional on-demand loading or standalone caching strategies. By minimizing cache misses and efficiently managing memory, this approach addresses both compute and memory bottlenecks, leading to faster, more resource-efficient token generation in MoE architectures. A comprehensive evaluation of this strategy, in relation to state-of-the-art methods, is provided in \cref{experiments_aot}, and the compute and memory traces on an A100 GPU are detailed in \cref{fig:moe_aot_cuda_trace}.



% Recent advancements in sparse Mixture-of-Experts (MoE) architectures have introduced the concept of utilizing distinct computational paths for different tokens \cite{shazeer2017outrageously}. This approach, wherein only a subset of experts are activated per input, enables MoE-based language models to generate tokens more swiftly compared to their dense counterparts due to memory-bound nature of auto-regressive decoding. In dense models, pre-loading layers in advance is a common strategy to enhance computational efficiency. However, this technique is not applicable to MoE models, where expert selection occurs dynamically based on the outputs of previous layers, preventing parallel pre-fetching of experts.

% Our proposed method addresses this inefficiency. Accelerated residuals, which are highly similar to their slower counterparts (see \cref{fig:similarity}), can reliably predict the necessary experts ahead of time. For instance, by utilizing $2X$ accelerated residual stream, we can predict the experts needed for the layer $2i+1$ and $2i+3$ while carrying out computation in layer $i$. This enables us to commence expert loading significantly earlier, as illustrated in \cref{expert_loading}, effectively mitigating the delays observed with the naive on-demand expert loading. Additionally, our method benefits from incorporating a Least Recently Used (LRU) strategy, where speculated experts replace those that are least recently utilized, resulting in improved performance compared to using either strategy alone. For a comprehensive evaluation, refer to \cref{moe_trace}, which provides a CUDA compute and memory trace of our approach executed on <>.



% A naive solution involves using the residual state of the previous layer along with the gating function of the next layer to predict which experts need to be loaded, and initiating the expert loading process in parallel with the attention computation of the next layer. Yet, as shown in \cref{fig:MOE_attn_vs_loading_time}, the attention computation for medium to long contexts is considerably faster than the expert loading time, making this approach inefficient.




\subsection{Training} \label{method_training}
% This approach is feasible due to the absence of gradient conflicts, as discussed in \cref{sec:grad_conflict}.

To accelerate residual streams, we employ parallel accelerator adapters as described in \cref{m2r2_method}.  For the early exiting use-case outlined in \cref{method_early_exiting}, we define the training objective for these adapters using the following loss function, which combines cross-entropy loss at each exit $E_j$ with distillation loss at each layer $i$. Loss weights coefficients $\alpha_0$ and $\alpha_1$ are employed to balance contribution of corresponding losses.

\begin{align} \label{eq:mr_loss}
L_{\text{m2r2}} = \underbrace{-\alpha_0 \sum_{j=1}^{J} \sum_{t=1}^{T} \log p_{\theta} \left( \hat{y}_t^{E_j} \mid y_{<t}, x \right)}_{\text{cross-entropy loss}} 
+ \underbrace{\alpha_1\sum_{i=1}^{E_{J-1}} \sum_{t=1}^{T} \| \mathbf{p}_{t}^{i} - \mathbf{h}_{t}^{((i - E_{j(i)}) \cdot R_i) + E_{j(i)})} \|^2}_{\text{distillation loss}}.
\end{align}

where $\hat{y}_t^{E_j}$ denotes the predictions from the accelerated residual stream at layer $E_j$ and time step $t$, $y_t$ represents the corresponding ground truth tokens, and $x$ indicates previous context tokens. The distillation loss at each layer $i$ is computed by comparing accelerated residuals at layer $i$ with slow residuals at layer $(i - E_{j(i)}) \cdot R_i + E_{j(i)}$, where $R_i$ denotes the rate of accelerated residuals at layer $i$ while $E_{j(i)}$ represents the most recent gate layer index such that $E_{j(i)} <= i$. \( J \) represents the total number of early exit gates, N denotes number of hidden layers and $E_j$ denotes layer index corresponding to gate index $j$ and \( T \) denotes the sequence length. 

In dynamic compute settings, after training of accelerator adapters, we optimize the query, key, and value parameters governing the ARLA routers (see ~\cref{method_arla}) across all exits in parallel on binary cross entropy loss between predicted decision and ground truth exiting decision. The ground truth labels for the router are determined based on whether the application of the final logit head on $\hat{y}_t^{E_j}$ yields the correct next-token prediction. 


% The objective for this optimization is defined by the following loss function:


%TODO are equations required ? 
% \begin{equation} \label{eq:arla_loss_combined}\small
%     L_{\text{arla}} = -\frac{1}{N} \sum_{t=1}^{T} \left( \sum_{j=1}^{E_n} \left[ O_t^{E_j} \log(\hat{O}_t^{E_j}) + (1 - O_t^{E_j}) \log(1 - \hat{O}_t^{E_j}) \right] \right), \quad \text{where} \quad 
%     O_t^{E_j} = \begin{cases} 
%     1, & \text{if } L(\hat{y}_t^{E_j}) = y_t^{E_j} \\
%     0, & \text{otherwise}
%     \end{cases}
% \end{equation}

% where $\hat{O}_t^{E_j}$ represents the binary predicted logits produced by the vertical latent attention router, as described in \cref{sec:arla}, at gate $E_j$ and time step $t$, and $O_t^{E_j}$ denotes the corresponding ground truth labels. The ground truth labels for the router are determined based on whether the application of the logit head on $\hat{y}_t^{E_j}$ yields the correct next-token prediction. The parameters controlling vertical latent attention are trained concurrently to ensure consistency and efficient use of computational resources.

For self-speculative decoding, as described in \cref{method_self_speculative_decoding}, the training objective remains the same as \cref{eq:mr_loss}, but with the number of intervals set to $J = 1$ and the rate of residual transformation set to $R_n = N/k$, where the first $k$ layers generate speculative candidate tokens. In the context of Ahead-of-Time Expert Loading for Mixture-of-Experts (MoE) models (see \cref{method_aot_expert_loading}), setting the rate of residual transformation to $R_n = 2$ typically offers a good trade-off between the accuracy of expert speculation and AoT pre-loading of experts. 

% Thus, we set $J = 1$ and $E_1 = 16$.


~\subsection{FLOPs Optimization} \label{sec:flops_optimization}

Naively implemented, M2R2 incurs higher FLOP overhead compared to traditional speculative decoding and early exiting approaches such as ~\cite{medusa, schuster2022confident, Tang2024}. However, modern accelerators demonstrate compute bandwidth that exceeds memory access bandwidth by an order of magnitude or more~\cite{databricksLLMInference2023, jouppi2021ten}, meaning increased FLOPs do not necessarily translate to increased decoding latency. Nevertheless, to ensure fair comparison and efficiency in compute bound scenarios, we introduce targeted optimizations.

~\textbf{Attention FLOPs Optimization} For medium-to-long context lengths, attention computation dominates FLOPs in the self-attention layer, surpassing the contribution from MLP layers. Specifically, matrix multiplications involving queries, cached keys, and cached values scale with $l_{kv} * l_{q}$ where $l_{kv}$ denotes previous context length and $l_q$ denotes current query length. Since M2R2 pairs accelerated residuals with slow residuals, a naive implementation results in twice the FLOPs consumption compared to a standard attention layer. To address this, we limit the attention of accelerated residual stream to selectively attend to the top-k most relevant tokens, identified by the slow residual stream based on top attention coefficients\footnote{We set to k = 64 and attend to top 64 tokens as identified by the slow residual stream.}. This is possible since slow and accelerated residual streams are processed in same forward pass and accelerated streams have access to attention coefficients of slow stream. Note that, the faster residual stream still retains the flexibility to assign distinct attention coefficients to these tokens. Furthermore, we design the faster residual stream to employ only 8 attention heads, compared to the 32 heads used in the slow residual stream of the Phi-3 model, reducing query, key, value, and output projection FLOPs by a factor of 1/4. ~\cref{fig:m2r2_num_heads_ablation} indicates effect of using a slicker stream on alignment. As depicted, using $\hat{n}_h = 8$ offers a good trade-off between alignment and FLOPs overhead. 

~\textbf{MLP FLOPs Optimization} The accelerator adapters operating on the accelerated residual stream are intentionally designed with lower rank than their counterparts in the base model. This reduces FLOP overhead by a factor proportional to $hiddenSize / rank$. Additionally, since the faster residual stream uses only 8 attention heads (compared to 32 in the slow residual stream of Phi-3), the subsequent MLP layers process a smaller set of activations, further reducing FLOPs by another factor of 1/4.

These optimizations significantly reduce the FLOP overhead per speculative draft generation, as illustrated in ~\cref{fig:flops_optmization}. Notably, while traditional early-exiting speculative approaches such as DEED require propagating the full slow residual state through the initial layers, incurring substantial computational costs, M2R2 achieves efficient token generation via slimmer, low-rank faster residual streams. In contrast, Medusa introduces considerable FLOP overhead due to per-head computations scaling with $d^2+dv$\footnote{Here $d$ denotes hidden state dimension while $v$ denotes vocab size.}, whereas M2R2 employs low-rank layers for both MLP and language modeling heads, maintaining computational efficiency. All experiments involving the M2R2 approach, as detailed in ~\cref{sec:experiments}, are conducted using these FLOPs optimizations.









% \[
% O_t^{E_j} = 
% \begin{cases} 
% 1, & \text{if } L(\hat{y}_t^{E_j}) = y_t^{E_j} \\
% 0, & \text{otherwise}
% \end{cases}
% \]




%add distillation
% We train accelerator adapters described in \cref{m2r2_method} to accelerate residual streams on next token prediction all in parallel since there are no gradient conflict issues as described in \cref{sec:grad_conflict}.

% \begin{align} \label{eq:mr_loss}
% L_{mr} =  & -\sum_{j = 1}^{E_n} (\sum_{t=1}^{T}\log p_{\theta} (\hat{y}_t^{E_j} | \hat{y}_{<t}, x)) \nonumber
% \end{align}

% where $\hat{y_t^{E_j}}$ denotes predicted logits obtained from accelerated residual stream at gate $E_j$ and time-step $t$ while $y_t^{E_j}$ denotes corresponding truth tokens. 

% Upon training of adapters responsible for accelerating residual streams, we train query, key, value parameters responsible for vertical latent attention of all gates in parallel as

% \begin{equation} \label{eq:arla_loss}
%     L_{arla} = -\frac{1}{N} (\sum_{t=1}^{T}(1\sum_{j=1}^{E_n} \left[ O_t^{E_j} \log(\hat{O}_t^{E_j}) + (1 - o_t^{E_j}) \log(1 - \hat{o_t}_{E_j}) \right]))
% \end{equation}

% where $\hat{O_t^{E_j}}$ denotes binary predicted logits obtained from vertical latent attention router described in \cref{sec:arla} at gate $E_j$ and timestep $t$ while $O_t^{E_j}$ denotes corresponding truth label. Truth labels for router are obtained by computing whether logit head application on $\hat{y}_t^j$ results in true next token prediction. Formally speaking, 

% $O_t^{E_j} = 1 if L(\hat{y_t^{E_j}}) == y_t^{E_j} , 0 otherwise$. 

% Parameters responsible for vertical latent attention are also trained in parallel as well. 

%todo: training slow and fast residuals together and distillation can be two training mdoes. 
%Distillation can be an ablation. 




% Although transformer decoding is memory bound on most mainstream accelerators, there could be scenarios where flop savings are crucial. For instance, on on-device settings power consumption is directly correlated with flops per decoding step and reducing flops does help with overall energy consumption. Vanilla early exiting methods help with flop reduction but suffer from mismatch between training and inference due to early exited tokens. If token at decoding step $t$, $T_t$ exited at layer $E_i$, while token $T_{t+k}$ exits at layer $E_j$ such that $E_i < E_j$, hidden state $H_{t+k}l$ does not have corresponding hidden state $H_tl$ to attend to where $E_i < l <= E_j$. One solution that's often used in literature is to rely on last hidden state available, $H_t{E_j}$, however it tends to be sub-optimal and does affect generation quality \cite{ref}.  To alleviate this mismatch while reducing flops, we train router such that attention mask between token $T_{t+k}$ and token $T_{<t+k}$ is given by: 

% \begin{equation}
%     a_{T_{{t+k}{T_{<t+k}}} = 1 if  E_{T_{<t+k}} >= E{T_{t+k}}
%     else 0
% \end{equation}

% This attention mask enables router to account for exited tokens and get trained accordingly. Since attention mechanism during decoding remains exactly same as that during training, impact on generation quality tends to be minimal as noted in \cref{fig:gen_auality_with_and_without_recompute_attention_show_flops}.  Although MoD does not suffer from training and inference mismatch, we observe that it suffers from discountinuity between pre-training and super-vised fine-tuning resulting in sub-optimal perplexity. On the other hand, our method doesn't not require pre-training , doesn't suffer from discountinuity, and achieves much better perplexity in super-vised fine-tuning and instruction tuning setups as shown in \cref{fig:Mod_vs_m2r2_loss_curves}.






% Our techniques are directly applicable in such scenarios.    




%expert loading with cuda streams in experiments
\para{Problem formulation}
\label{sec:problem_form}
To leverage the prior knowledge, we formulate
a depth completion as an inverse problem that estimates unknown dense depth from 
observed sparse measurements.
$\mathbf{y}$ represents the observed sparse depth, 
$\mathbf{x}$ is the unknown dense depth, and 
$\mathcal{A}{:}\,\mathbb{R}^m{\rightarrow}\mathbb{R}^n$ is a binary measurement matrix of which entry  $[\mathcal{A}]_{ij}$ is $1$ if the entities $[\mathbf{y}]_i$ is measured from $[\mathbf{x}]_j$, $0$ otherwise. 
We follow \Eref{eq:inv_sampling}, where sparse depth serves as guidance.
We use the depth diffusion models \cite{ke2023repurposing, gui2024depthfm} extended from the latent diffusion model (LDM)~\cite{rombach2022highresolution} as prior, where
$\mathbf{x}$ is 
decomposed with the decoder $\mathcal{D}{:}\, \mathbf{z} \rightarrow \mathbf{x}$ as:
\begin{equation}
\label{eq:depth_guide_sampling}
\hat{\mathbf{s}}_{\theta}= \mathbf{s}_{\theta}(\mathbf{z}_t, t) + w {\nabla_{\mathbf{z}_t}}\left\| \mathbf{y} - \mathcal{A}\left(\mathcal{D}\left(\mathbf{z}_0\left(\mathbf{z}_t\right)\right)\right) \right\|_2^2,
\end{equation}
where $\mathbf{z}\in \mathbb{R}^{4\times H \times W}$ represents the latent of LDM but the decoder output $\mathbf{x}$ is treated as a flatten vector for convenience.

\subsection{Test-time Alignment with Hard Constraints}
\label{sec:opt_sampling}
Depth measurements obtained in practice are often sparse, unevenly distributed, and noisy. 
When the sparse measurements are used as guidance, the ill-posed nature of the problem, combined with the stochastic behavior of diffusion models, can lead to scores that produce undesirable solutions~\cite{kim2024regtext} and does not even guarantee that the estimation corresponds to the known sparse measurements. 
To deal with this, we propose a test-time alignment that incorporates the correction step 
to enforce the sparse measurement as harder constraints than encouraging guidance in a soft manner by \Eref{eq:depth_guide_sampling}.
This involves an optimization loop at regular intervals to enforce
measurement constraints as a correction step.
We further show the potential for uncertain solutions from the stochastic process in the supplementary material, illustrating why the alignment is necessary.

Additionally, we adopt $\mathbf{z}_0(\mathbf{z}_t)$ as optimizable variable.
Pre-trained diffusion models take input $\mathbf{z}_t$ aligend with the noise level at each timestep $t$.
However, directly optimizing $\mathbf{z}_t$  without considering input characteristics may lead to suboptimal results
\cite{chung2022improving, chung2023dps, chung2024dds}.
To address this, inspired by \citet{song2024solving}, we use $\mathbf{z}_0(\mathbf{z}_t)$ estimated from $\mathbf{z}_t$.
The optimization loop is formulated as:
\begin{equation}
    \label{eq:opt_loop}
    \hat{\mathbf{z}}_0(\mathbf{z}_t) = \argmin_{\mathbf{z}_0(\mathbf{z}_t)} \left\| \mathbf{y} - \mathcal{A}\left(\mathcal{D}\left(\mathbf{z}_0\left(\mathbf{z}_t\right)\right)\right) \right\|_2^2.
\end{equation}
Then, to ensure adherence to the correct noise level, the measurement-consistent $\hat{\mathbf{z}}_0(\mathbf{z}_t)$ is remapped to an intermediate latent $\hat{\mathbf{z}}_t$ by adding time-scheduled Gaussian noise, as expressed below:
\begin{equation}
    \label{eq:remap}
    p\left(\hat{\mathbf{z}}_{t} | \hat{\mathbf{z}}_0(\mathbf{z}_t)\right) = \mathcal{N}(\sqrt{\bar{\alpha}_{t}} ~ \hat{\mathbf{z}}_0(\mathbf{z}_t), (1 - \bar{\alpha}_{t}) I),
\end{equation}
\noindent where $\bar{\alpha}_{t} = \prod_{i=1}^t \alpha_i,$ and $\alpha_t$ is variance schedule at time $t$.

Since the score $\hat{\mathbf{s}}_{\theta}(\mathbf{z}_t, t)$ is directly added to the latent $\mathbf{z}_t$ at each step,
\Eref{eq:depth_guide_sampling} can be rewritten in terms of $\mathbf{z}_0(\mathbf{z}_t)$ with a modulated weight factor $\zeta$, as follows:
\begin{equation}
    \label{eq:depth_guide_sampling_latent}
    \hat{\mathbf{z}}_t = \mathbf{z}_t + \zeta \nabla_{\mathbf{z}_t} \left\| \mathbf{y} - \mathcal{A}\left(\mathcal{D}\left(\mathbf{z}_0(\mathbf{z}_t)\right)\right) \right\|_2^2.
\end{equation}
Here, \Eref{eq:depth_guide_sampling_latent} is replaced by the two-step process of \Eref{eq:opt_loop} and \Eref{eq:remap}, allowing our test-time alignment process to effectively achieve measurement-consistent desirable solutions.
Figure \ref{fig:opt_loop} illustrates the our test-time alignment process.
Figure~\ref{fig:depth_align} demonstrates how effectively our test-time alignment method estimates unseen depth areas by aligning sparse measurements with an affine-invariant depth prior. This result highlights the need for correction.
Examples of undesirable solutions and their corrected ones by our method are provided in the supplementary material.
\begin{figure*}[t]
\centering
   \includegraphics[width=1.0\linewidth]{Figures/depth_align_arxiv.pdf}
   \caption{\textbf{Alignment with metric depth.} 
   % \after{
   We evaluate our method's effectiveness against ground truth (GT), accumulated semi-densely.
   We use only sparse depth (a) to align with actual metric depth values in complex scenes, ensuring a desirable solution.
   The white lines in (b), (c), and the x-axis of (d) represent pixel indices with valid depth points in a row of the GT.
   } 
\label{fig:depth_align}
\vspace{-3mm}
\end{figure*}
% % \vspace{-3mm}
\begin{algorithm}[t]
\caption{Prior-based outlier filtering algorithm.}
\label{algorithm:1}
\begin{algorithmic}[1]
\setlength{\itemsep}{0.15em}
    \State \textbf{Parameters:} Number of segments $N$, Filter threshold $\tau$
    \State \textbf{Input:} 
    % RGB image $I$, 
    Estimated relative depth $D_r$, Sparse metric depth $\mathbf{y}$, Set of sparse point locations $\Omega(\mathbf{y})$.
    \State \textbf{Output:} Set of reliable sparse point locations $\Omega(\mathbf{y}^*)$.
    \State $\{\Omega(S_i)\}_{i=1 \cdots N} \gets \text{SuperPixel}\left( D_r, N 
    % \oplus I 
    \right)$ 
    % \Comment{Cluster all the points into segments $S_i$}
    \For{$i = 1$ \textbf{to} $N$}
        \State $\Omega(\mathbf{y}_i) \gets \Omega(\mathbf{y}) \cap \Omega(S_i)$
        % \Comment{$\mathbf{y}_i$ denotes sparse depth points on $\Omega(\mathbf{y}_i)$}
        \State $\hat{\mathbf{y}}_i \gets %\underset{S_i \cap S_p}
        {\text{RANSAC~Regressor}}(\mathds{1}_{\Omega(\mathbf{y}_i)} \odot D_r,~ \mathbf{y}_i)$ 
        % \Comment{Perform RANSAC Regression on points $\mathbf{y} \cap S_i$}
        \State $\Omega(\mathbf{y}_i^{*}) \gets |\hat{\mathbf{y}}_i-\mathbf{y}_i| > \tau$
        % \Comment{Filter depth points with values exceeding threshold $\tau$.}
    \EndFor
    \State \( \Omega(\mathbf{y}^*) \gets \bigcup_{i=1}^{N} \Omega(\mathbf{y}_i^{*}) \)
\end{algorithmic}
\end{algorithm}
% \vspace{-3mm}

Until now, in solving \Eref{eq:depth_guide_sampling}, we use an affine-invariant depth model for completing metric depths without special care. However, a natural question arises: ``\textit{Is the affine-invariant depth model compatible with estimating metric depths in our framework?}'' The following analysis shows that it may be sufficient.

\vspace{1mm}\noindent\textbf{Can we use an affine-invariant depth model for completing metric depths?}
Depth estimation models are often trained to estimate affine-invariant depth with scale and shift invariant loss to achieve generalizable performance~\cite{Ranftl2022midas, ke2023repurposing, eigen2014invariant}.
Thus, depth prior operates in the affine-invariant depth space, which does not directly correspond to the metric depth used in measurements.
Even though the given sparse metric depth is normalized between 0 and 1, 
their statistics including 
mean and variance
can differ, and the relationship between real metric depth and estimated affine-invariant depth is often 
non-linear (see 
the left of \Fref{fig:depth_align} (d)).
Therefore, to determine if 
\Eref{eq:depth_guide_sampling} can be used to solve this problem, we need to verify whether the normalized metric depth space lies within the data distribution generated by the diffusion model.

To confirm this, we conduct an empirical investigation through the following procedure: given $\tilde{\mathbf{x}}_0$, dense depth map estimated from the pre-trained depth completion model, we perform its reconstruction using an affine-invariant depth diffusion model.
This process involves sequentially encoding $\tilde{\mathbf{x}}_0$ to $\tilde{\mathbf{z}}_0$, 
then doing inversion by adding noise~\cite{song2022denoising}, which results in $\tilde{\mathbf{z}}_t$. 
Next, we perform reverse sampling, $\nabla_{\mathbf{z}_t}\log p(\tilde{\mathbf{z}}_t)$ with only the affine-invariant depth diffusion prior.
The reconstructed result achieves similar performance compared to the original one, $\tilde{\mathbf{x}}_0$, excluding encoding-decoding information loss. 
The details and results of the experiment are provided in the supplementary material.
This result suggests that the affine-invariant depth prior is sufficiently capable of handling the metric depth space, 
which corresponds to:
\begin{equation}
\label{eq:verify}
    \nabla_{\mathbf{z}_t}\log p(\tilde{\mathbf{z}}_t) \approx \nabla_{\tilde{\mathbf{z}}_t}\log p(\tilde{\mathbf{z}}_t).
\end{equation}
Thus, we just need to align this prior with metric depth cue validating using \Eref{eq:depth_guide_sampling} to solve ill-posed depth completion.

% % \setcounter{footnote}{0} 
\definecolor{tabfirst}{rgb}{1, 0.7, 0.7} % red
\definecolor{tabsecond}{rgb}{1, 0.85, 0.7} % orange
\definecolor{tabthird}{rgb}{1, 1, 0.7} % yellow
\newcommand{\mystrut}{\rule[-0.4ex]{0pt}{1.7ex}}
\newcommand{\highlight}[2]{\colorbox{#1}{\mystrut#2}}

% \newcommand{\mystrut}{\rule[-0.2ex]{0pt}{0.5ex}}
% \newcommand{\highlight}[2]{\colorbox{#1}{\rule[-0.3ex]{0pt}{0.1ex}\strut #2}}

\begingroup
\setlength{\tabcolsep}{12pt}
\begin{table*}[t]
\centering
\renewcommand{\arraystretch}{1.1} % Adjust the row height factor
    \resizebox{0.9\linewidth}{!}{
    \begin{tabular}{m{2.5cm} cc cc cc cc }
    \toprule
    \multirow{3}[3]{*}{Method} & \multicolumn{4}{c}{Indoor} & \multicolumn{4}{c}{Outdoor} \\
    \cmidrule(lr){2-5} \cmidrule(lr){6-9} 
    & \multicolumn{2}{c}{NYUv2} & \multicolumn{2}{c}{SceneNet}
    & \multicolumn{2}{c}{Waymo} & \multicolumn{2}{c}{nuScenes} \\
     \cmidrule(lr){2-3} \cmidrule(lr){4-5} \cmidrule(lr){6-7} \cmidrule(lr){8-9}
    & RMSE & MAE & RMSE & MAE & RMSE & MAE & RMSE & MAE \\
    \midrule 
    % Sparse-to-Dense~\cite{Ma2017SparseToDense} & & &\\
     Pre-trained %\footnotemark[2]
     & 0.446 & 0.189 & 0.443 & 0.173
     & 2.821 & 1.514
     & 3.998 & 1.967 \\ 
     BNAdapt
     & 0.410 & 0.189 & 0.446 & 0.176
    & 2.194 & \cellcolor{tabthird}1.122
    & 1.801 & 0.828\\ 
     CoTTA
     & 0.376 & 0.147 & 0.405 & 0.136
     & 2.652 & 1.227
     & 2.668 & 1.222 \\ 
     ProxyTTA
     & \cellcolor{tabthird}0.203 & \cellcolor{tabthird}0.095 & \cellcolor{tabthird}0.357 & \cellcolor{tabthird}0.125
    & \cellcolor{tabthird}2.178 & \cellcolor{tabfirst}0.971
    & \cellcolor{tabthird}1.755 & \cellcolor{tabthird}0.799\\ 
    \midrule
    Ours~(+Marigold)
    & \cellcolor{tabsecond}0.149 & \cellcolor{tabfirst}0.059 & \cellcolor{tabsecond}0.207 & \cellcolor{tabsecond}0.099 
    & \cellcolor{tabfirst}2.115 & \cellcolor{tabsecond}1.121
    & \cellcolor{tabfirst}1.561 & \cellcolor{tabfirst}0.561\\
    Ours~(+DepthFM)
    & \cellcolor{tabfirst}0.145 & \cellcolor{tabsecond}0.077 & \cellcolor{tabfirst}0.178 & \cellcolor{tabfirst}0.081 
    & \cellcolor{tabsecond}2.162 & 1.133
    & \cellcolor{tabsecond}1.622 & \cellcolor{tabsecond}0.618\\
    % & \textbf{1.516} & \textbf{0.561} & \TODO{} & \TODO{} & \textbf{0.149} & \textbf{0.059} & 0.413 & 0.243 \\ 
    \bottomrule
    \end{tabular}
    }
\caption{\textbf{Quantitative comparison of generalizable performance.}
We evaluate the generalizability of our method by comparing it with test-time adaptation methods across various domain datasets.
% \textbf{Bold} denotes best and \textit{Italics} second-best. 
In this table, the pre-trained depth completion model is CostDCNet~\cite{kam2022costdcnet}, trained on KITTI DC for outdoor and VOID for indoor adaptation.
It is used for each adaptation method---BNAdapt~\cite{wang2021tent}, CoTTA~\cite{wang2022continual}, and ProxyTTA~\cite{park2024testtime}---excluding ours, for adapting to each domain.
The first best is marked in \highlight{tabfirst}{red}, the second in \highlight{tabsecond}{orange}, and the third in \highlight{tabthird}{yellow}.
}
\label{tab:generalization}
\vspace{-5pt}
\end{table*}
\endgroup





% % \setcounter{footnote}{0} 
% \definecolor{tabfirst}{rgb}{1, 0.7, 0.7} % red
% \definecolor{tabsecond}{rgb}{1, 0.85, 0.7} % orange
% \definecolor{tabthird}{rgb}{1, 1, 0.7} % yellow
% \newcommand{\mystrut}{\rule[-0.4ex]{0pt}{1.7ex}}
% \newcommand{\highlight}[2]{\colorbox{#1}{\mystrut#2}}

% % \newcommand{\mystrut}{\rule[-0.2ex]{0pt}{0.5ex}}
% % \newcommand{\highlight}[2]{\colorbox{#1}{\rule[-0.3ex]{0pt}{0.1ex}\strut #2}}

% \begingroup
% \setlength{\tabcolsep}{12pt}
% \begin{table*}[t]
% \centering
% \renewcommand{\arraystretch}{1.1} % Adjust the row height factor
%     \resizebox{0.9\linewidth}{!}{
%     \begin{tabular}{m{2.5cm} cc cc cc cc }
%     \toprule
%     \multirow{3}[3]{*}{Method} & \multicolumn{4}{c}{Indoor} & \multicolumn{4}{c}{Outdoor} \\
%     \cmidrule(lr){2-5} \cmidrule(lr){6-9} 
%     & \multicolumn{2}{c}{NYUv2} & \multicolumn{2}{c}{SceneNet}
%     & \multicolumn{2}{c}{Waymo\footnotemark[2]} & \multicolumn{2}{c}{nuScenes} \\
%      \cmidrule(lr){2-3} \cmidrule(lr){4-5} \cmidrule(lr){6-7} \cmidrule(lr){8-9}
%     & RMSE & MAE & RMSE & MAE & RMSE & MAE & RMSE & MAE \\
%     \midrule 
%     % Sparse-to-Dense~\cite{Ma2017SparseToDense} & & &\\
%      Pre-trained %\footnotemark[2]
%      & 0.446 & 0.189 & 0.443 & 0.173
%      & 3.078 & 1.175
%      & 6.630 & 2.656 \\ 
%      BNAdapt
%      & 0.410 & 0.189 & 0.446 & 0.176
%     & \cellcolor{tabthird}1.877 & \cellcolor{tabthird}0.596
%     & 6.391 & 2.306\\ 
%      CoTTA
%      & \cellcolor{tabthird}0.376 & \cellcolor{tabthird}0.147 & \cellcolor{tabthird}0.405 & \cellcolor{tabthird}0.136
%      & 2.140 & 0.689
%      & \cellcolor{tabthird}6.099 & \cellcolor{tabthird}2.676 \\ 
%      ProxyTTA
%      & \cellcolor{tabsecond}0.203 & \cellcolor{tabsecond}0.095 & \cellcolor{tabsecond}0.357 & \cellcolor{tabsecond}0.125
%     & \cellcolor{tabfirst}1.580 & \cellcolor{tabfirst}0.466
%     & \cellcolor{tabfirst}5.509 & \cellcolor{tabfirst}2.062\\ 
%     \midrule
%     Ours
%     & \cellcolor{tabfirst}0.149 & \cellcolor{tabfirst}0.059 & \cellcolor{tabfirst}0.207 & \cellcolor{tabfirst}0.099 
%     & \cellcolor{tabsecond}1.873 & \cellcolor{tabsecond}0.590
%     & \cellcolor{tabsecond}5.876 & \cellcolor{tabsecond}2.499\\
%     % & \textbf{1.516} & \textbf{0.561} & \TODO{} & \TODO{} & \textbf{0.149} & \textbf{0.059} & 0.413 & 0.243 \\ 
%     \bottomrule
%     \end{tabular}
%     }
% \caption{\textbf{Quantitative comparison on generalizable performance.}
% We evaluate the generalizability of our method by comparing it with test-time adaptation methods on various domain datasets.
% % \textbf{Bold} denotes best and \textit{Italics} second-best. 
% In this table, the pre-trained depth completion model is CostDCNet~\cite{kam2022costdcnet}, trained on KITTI DC for outdoor and VOID for indoor adaptation.
% It is used for each adaptation method---BNAdapt~\cite{wang2021tent}, CoTTA~\cite{wang2022continual} and ProxyTTA~\cite{park2024testtime}---excluding ours, for adapting to each domain.
% The first best is \highlight{tabfirst}{red}, the second is \highlight{tabsecond}{orange}, and the third is \highlight{tabthird}{yellow}.
% }
% \label{tab:generalization}
% \vspace{-5pt}
% \end{table*}
% \endgroup

 % \footnotetext[2]{
 % The pre-trained model is CostDCNet~\cite{kam2022costdcnet}, trained on KITTI DC for outdoor and VOID for indoor adaptation. It is used for each adaptation method excluding ours, with reported performance using the best results from Park \etal~\cite{park2024testtime}.
 % }

% \setlength{\tabcolsep}{3pt}
\begin{table}
\centering
\caption{Efficiency of Different Methods}
\vspace{-0.1in}
\label{tab:efficiency}
% \resizebox{\textwidth}{!}{%
\scalebox{0.65}{%
\begin{tabular}{|c|ccc|ccc|ccc|}
  \hline
  \multirow{2}{*}{} & \multicolumn{3}{c|}{\textbf{memory size (MB)}} & \multicolumn{3}{c|}{\textbf{training time (s)}} & \multicolumn{3}{c|}{\textbf{matching time (s)}}\\
  % \cline{2-7}
  % {} & MByte & seconds/ep & seconds\\
  {} & \textbf{Beijing} & \textbf{Porto} & \textbf{Chengdu} & \textbf{Beijing} & \textbf{Porto} & \textbf{Chengdu} & \textbf{Beijing} & \textbf{Porto} & \textbf{Chengdu}\\
  % \cline{2-7}
  % {} & (MByte) & (minutes/ep) & (seconds/K) & (MByte) & (minutes/ep) & (seconds/K)\\
  \hline
  \textbf{MDP} & 1819MB & 2039MB & 2122MB & - & - & - & 389.14s & 361.15s & 599.51s  \\
  \textbf{HMM} & 1209MB & 1388MB & 1361MB & - & - & - & 427.97s & 380.05s & 589.08s \\
  % \hline
  \textbf{FMM} & 897MB & 931MB & 981MB & - & - & - & 1.13s & 1.02s & 1.87s \\
  % \hline
  \textbf{AMM} & 957MB & 1013MB & 1124MB & - & - & - & 3.42s & 3.05s & 5.16s \\
  % \hline
  \textbf{MTrajRec} & 9045MB & 12428MB & 11265MB & 182.4s & 2200.2s & 25672.4s & 51.22s & 42.27s & 73.68s\\
  % \hline
  \textbf{L2MM} & 9087MB & 11875MB & 10898MB & 189.1s & 2314.2s & 27032.2s & 6.71s & 5.26s & 9.10s\\
  % \hline
  \textbf{GraphMM} & 8537MB & 11752MB & 10378MB & 48.4s & 645.2s & 7311.4s & 8.06s & 6.96s & 11.18s\\
  % \hline
  \textbf{\modelName} & 2530MB & 2299MB & 2357MB & 11.9s & 126.4s & 1507.8s & 1.09s & 0.95s & 1.65s\\
  \hline
\end{tabular}}
\vspace{-0.15in}
\end{table}

\subsection{Prior-based Outlier Filtering}
\label{sec:noise_filter}
Practical depth sensing methods often produce outliers, such as unsynchronized depth with RGB or see-through points~\cite{conti22confidence}), making sparse depth measurements unreliable.
This degrades the performance of methods relying on sparse depth supervision~\cite{wong2021unsupervised,wong2020void}.
We also use sparse depth measurement as supervision during test-time alignment, this makes the alignment process prone to divergence or slow convergence.
To address this,
we utilize data-driven depth prior~\cite{ke2023repurposing, gui2024depthfm}, which benefits from the more precise synchronization with RGB images and depth affinity.
To obtain outlier-free sparse points $\mathbf{y}^*$, we adopt a divide-and-conquer approach.
We define local segments based on depth affinity, grouping regions where relative depth values are similar within a spatially local area.
Within these segments, the depth distribution can be easily categorized into inliers and outliers, enabling us to effectively identify outliers.

Affine-invariant depth map $D_r$ is divided into local segments $S_i$, which are regions with a high probability of having similar depths with considering location. 
For this clustering 
we leverage the superpixel algorithm~\cite{achanta2012slic, li2015lsc}.
In each region, we perform linear least-square fitting to map affine-invariant depth to metric depth using sparse metric depth measurements $\mathbf{y}_i$.
However, since these sparse measurements are influenced by outliers, we use RANSAC~\cite{fischler1981ransac} to perform outlier-robust linear least-square fitting on points where noisy $\mathbf{y}$ intersects $S_i$ \ie, $\mathbf{y}_i \leftarrow S_i \cap \mathbf{y}$.
This allows us to estimate outlier-robust metric depth values $\hat{\mathbf{y}}_i$ in local regions $S_i$.
Then, points with significant deviations exceeding $\tau$ are identified as outliers and filtered out.
Our proposed filtering algorithm, based on monocular depth prior, is detailed in Algorithm~\ref{algorithm:1}.
% \red{We also evaluate our algorithm using a standard metric for assessing the reliability of outlier detection confidence, as detailed in the supplementary material.}

% \vspace{-3mm}
\begin{algorithm}[t]
\caption{Prior-based outlier filtering algorithm.}
\label{algorithm:1}
\begin{algorithmic}[1]
\setlength{\itemsep}{0.15em}
    \State \textbf{Parameters:} Number of segments $N$, Filter threshold $\tau$
    \State \textbf{Input:} 
    % RGB image $I$, 
    Estimated relative depth $D_r$, Sparse metric depth $\mathbf{y}$, Set of sparse point locations $\Omega(\mathbf{y})$.
    \State \textbf{Output:} Set of reliable sparse point locations $\Omega(\mathbf{y}^*)$.
    \State $\{\Omega(S_i)\}_{i=1 \cdots N} \gets \text{SuperPixel}\left( D_r, N 
    % \oplus I 
    \right)$ 
    % \Comment{Cluster all the points into segments $S_i$}
    \For{$i = 1$ \textbf{to} $N$}
        \State $\Omega(\mathbf{y}_i) \gets \Omega(\mathbf{y}) \cap \Omega(S_i)$
        % \Comment{$\mathbf{y}_i$ denotes sparse depth points on $\Omega(\mathbf{y}_i)$}
        \State $\hat{\mathbf{y}}_i \gets %\underset{S_i \cap S_p}
        {\text{RANSAC~Regressor}}(\mathds{1}_{\Omega(\mathbf{y}_i)} \odot D_r,~ \mathbf{y}_i)$ 
        % \Comment{Perform RANSAC Regression on points $\mathbf{y} \cap S_i$}
        \State $\Omega(\mathbf{y}_i^{*}) \gets |\hat{\mathbf{y}}_i-\mathbf{y}_i| > \tau$
        % \Comment{Filter depth points with values exceeding threshold $\tau$.}
    \EndFor
    \State \( \Omega(\mathbf{y}^*) \gets \bigcup_{i=1}^{N} \Omega(\mathbf{y}_i^{*}) \)
\end{algorithmic}
\end{algorithm}
% \vspace{-3mm}
% \setcounter{footnote}{0} 
\definecolor{tabfirst}{rgb}{1, 0.7, 0.7} % red
\definecolor{tabsecond}{rgb}{1, 0.85, 0.7} % orange
\definecolor{tabthird}{rgb}{1, 1, 0.7} % yellow
\newcommand{\mystrut}{\rule[-0.4ex]{0pt}{1.7ex}}
\newcommand{\highlight}[2]{\colorbox{#1}{\mystrut#2}}

% \newcommand{\mystrut}{\rule[-0.2ex]{0pt}{0.5ex}}
% \newcommand{\highlight}[2]{\colorbox{#1}{\rule[-0.3ex]{0pt}{0.1ex}\strut #2}}

\begingroup
\setlength{\tabcolsep}{12pt}
\begin{table*}[t]
\centering
\renewcommand{\arraystretch}{1.1} % Adjust the row height factor
    \resizebox{0.9\linewidth}{!}{
    \begin{tabular}{m{2.5cm} cc cc cc cc }
    \toprule
    \multirow{3}[3]{*}{Method} & \multicolumn{4}{c}{Indoor} & \multicolumn{4}{c}{Outdoor} \\
    \cmidrule(lr){2-5} \cmidrule(lr){6-9} 
    & \multicolumn{2}{c}{NYUv2} & \multicolumn{2}{c}{SceneNet}
    & \multicolumn{2}{c}{Waymo} & \multicolumn{2}{c}{nuScenes} \\
     \cmidrule(lr){2-3} \cmidrule(lr){4-5} \cmidrule(lr){6-7} \cmidrule(lr){8-9}
    & RMSE & MAE & RMSE & MAE & RMSE & MAE & RMSE & MAE \\
    \midrule 
    % Sparse-to-Dense~\cite{Ma2017SparseToDense} & & &\\
     Pre-trained %\footnotemark[2]
     & 0.446 & 0.189 & 0.443 & 0.173
     & 2.821 & 1.514
     & 3.998 & 1.967 \\ 
     BNAdapt
     & 0.410 & 0.189 & 0.446 & 0.176
    & 2.194 & \cellcolor{tabthird}1.122
    & 1.801 & 0.828\\ 
     CoTTA
     & 0.376 & 0.147 & 0.405 & 0.136
     & 2.652 & 1.227
     & 2.668 & 1.222 \\ 
     ProxyTTA
     & \cellcolor{tabthird}0.203 & \cellcolor{tabthird}0.095 & \cellcolor{tabthird}0.357 & \cellcolor{tabthird}0.125
    & \cellcolor{tabthird}2.178 & \cellcolor{tabfirst}0.971
    & \cellcolor{tabthird}1.755 & \cellcolor{tabthird}0.799\\ 
    \midrule
    Ours~(+Marigold)
    & \cellcolor{tabsecond}0.149 & \cellcolor{tabfirst}0.059 & \cellcolor{tabsecond}0.207 & \cellcolor{tabsecond}0.099 
    & \cellcolor{tabfirst}2.115 & \cellcolor{tabsecond}1.121
    & \cellcolor{tabfirst}1.561 & \cellcolor{tabfirst}0.561\\
    Ours~(+DepthFM)
    & \cellcolor{tabfirst}0.145 & \cellcolor{tabsecond}0.077 & \cellcolor{tabfirst}0.178 & \cellcolor{tabfirst}0.081 
    & \cellcolor{tabsecond}2.162 & 1.133
    & \cellcolor{tabsecond}1.622 & \cellcolor{tabsecond}0.618\\
    % & \textbf{1.516} & \textbf{0.561} & \TODO{} & \TODO{} & \textbf{0.149} & \textbf{0.059} & 0.413 & 0.243 \\ 
    \bottomrule
    \end{tabular}
    }
\caption{\textbf{Quantitative comparison of generalizable performance.}
We evaluate the generalizability of our method by comparing it with test-time adaptation methods across various domain datasets.
% \textbf{Bold} denotes best and \textit{Italics} second-best. 
In this table, the pre-trained depth completion model is CostDCNet~\cite{kam2022costdcnet}, trained on KITTI DC for outdoor and VOID for indoor adaptation.
It is used for each adaptation method---BNAdapt~\cite{wang2021tent}, CoTTA~\cite{wang2022continual}, and ProxyTTA~\cite{park2024testtime}---excluding ours, for adapting to each domain.
The first best is marked in \highlight{tabfirst}{red}, the second in \highlight{tabsecond}{orange}, and the third in \highlight{tabthird}{yellow}.
}
\label{tab:generalization}
\vspace{-5pt}
\end{table*}
\endgroup





% % \setcounter{footnote}{0} 
% \definecolor{tabfirst}{rgb}{1, 0.7, 0.7} % red
% \definecolor{tabsecond}{rgb}{1, 0.85, 0.7} % orange
% \definecolor{tabthird}{rgb}{1, 1, 0.7} % yellow
% \newcommand{\mystrut}{\rule[-0.4ex]{0pt}{1.7ex}}
% \newcommand{\highlight}[2]{\colorbox{#1}{\mystrut#2}}

% % \newcommand{\mystrut}{\rule[-0.2ex]{0pt}{0.5ex}}
% % \newcommand{\highlight}[2]{\colorbox{#1}{\rule[-0.3ex]{0pt}{0.1ex}\strut #2}}

% \begingroup
% \setlength{\tabcolsep}{12pt}
% \begin{table*}[t]
% \centering
% \renewcommand{\arraystretch}{1.1} % Adjust the row height factor
%     \resizebox{0.9\linewidth}{!}{
%     \begin{tabular}{m{2.5cm} cc cc cc cc }
%     \toprule
%     \multirow{3}[3]{*}{Method} & \multicolumn{4}{c}{Indoor} & \multicolumn{4}{c}{Outdoor} \\
%     \cmidrule(lr){2-5} \cmidrule(lr){6-9} 
%     & \multicolumn{2}{c}{NYUv2} & \multicolumn{2}{c}{SceneNet}
%     & \multicolumn{2}{c}{Waymo\footnotemark[2]} & \multicolumn{2}{c}{nuScenes} \\
%      \cmidrule(lr){2-3} \cmidrule(lr){4-5} \cmidrule(lr){6-7} \cmidrule(lr){8-9}
%     & RMSE & MAE & RMSE & MAE & RMSE & MAE & RMSE & MAE \\
%     \midrule 
%     % Sparse-to-Dense~\cite{Ma2017SparseToDense} & & &\\
%      Pre-trained %\footnotemark[2]
%      & 0.446 & 0.189 & 0.443 & 0.173
%      & 3.078 & 1.175
%      & 6.630 & 2.656 \\ 
%      BNAdapt
%      & 0.410 & 0.189 & 0.446 & 0.176
%     & \cellcolor{tabthird}1.877 & \cellcolor{tabthird}0.596
%     & 6.391 & 2.306\\ 
%      CoTTA
%      & \cellcolor{tabthird}0.376 & \cellcolor{tabthird}0.147 & \cellcolor{tabthird}0.405 & \cellcolor{tabthird}0.136
%      & 2.140 & 0.689
%      & \cellcolor{tabthird}6.099 & \cellcolor{tabthird}2.676 \\ 
%      ProxyTTA
%      & \cellcolor{tabsecond}0.203 & \cellcolor{tabsecond}0.095 & \cellcolor{tabsecond}0.357 & \cellcolor{tabsecond}0.125
%     & \cellcolor{tabfirst}1.580 & \cellcolor{tabfirst}0.466
%     & \cellcolor{tabfirst}5.509 & \cellcolor{tabfirst}2.062\\ 
%     \midrule
%     Ours
%     & \cellcolor{tabfirst}0.149 & \cellcolor{tabfirst}0.059 & \cellcolor{tabfirst}0.207 & \cellcolor{tabfirst}0.099 
%     & \cellcolor{tabsecond}1.873 & \cellcolor{tabsecond}0.590
%     & \cellcolor{tabsecond}5.876 & \cellcolor{tabsecond}2.499\\
%     % & \textbf{1.516} & \textbf{0.561} & \TODO{} & \TODO{} & \textbf{0.149} & \textbf{0.059} & 0.413 & 0.243 \\ 
%     \bottomrule
%     \end{tabular}
%     }
% \caption{\textbf{Quantitative comparison on generalizable performance.}
% We evaluate the generalizability of our method by comparing it with test-time adaptation methods on various domain datasets.
% % \textbf{Bold} denotes best and \textit{Italics} second-best. 
% In this table, the pre-trained depth completion model is CostDCNet~\cite{kam2022costdcnet}, trained on KITTI DC for outdoor and VOID for indoor adaptation.
% It is used for each adaptation method---BNAdapt~\cite{wang2021tent}, CoTTA~\cite{wang2022continual} and ProxyTTA~\cite{park2024testtime}---excluding ours, for adapting to each domain.
% The first best is \highlight{tabfirst}{red}, the second is \highlight{tabsecond}{orange}, and the third is \highlight{tabthird}{yellow}.
% }
% \label{tab:generalization}
% \vspace{-5pt}
% \end{table*}
% \endgroup

 % \footnotetext[2]{
 % The pre-trained model is CostDCNet~\cite{kam2022costdcnet}, trained on KITTI DC for outdoor and VOID for indoor adaptation. It is used for each adaptation method excluding ours, with reported performance using the best results from Park \etal~\cite{park2024testtime}.
 % }

\subsection{Losses}
\label{sec:losses}
Our objective for optimization includes sparse depth consistency loss and regularization terms: a local smoothness loss to preserve depth prior and a new relative structure similarity loss to maintain structural prior inherent in depth prior.

\para{Sparse depth consistency}
Given the sparse depth measurement $y$, it ensures consistency with the metric depth.
To effectively integrate the observed measurements with affine-invariant depth prior and mitigate potential uncertainties, we employ $L_{1}$ loss as follows:
\begin{equation}
    \resizebox{0.59\hsize}{!}{$
    \mathcal{L}_{depth} = \scalebox{1.4}{$\frac{1}{|\Omega(\mathbf{y})|}$} \sum\limits_{\Omega(\mathbf{y})} |\mathbf{y}- \mathcal{A}(\hat{D})|,
    $}
\end{equation}
where $\mathcal{A}$ is the operation that Hadamard product with the zero-one mask $\mathds{1}_{\Omega(\mathbf{y})}$ and $\hat{D}$ represents completed depth.

\para{Local smoothness}
Using only sparse depth guidance risks losing the prior knowledge inherent in pre-trained depth diffusion models~\cite{ke2023repurposing, gui2024depthfm}, such as the property of depth which is locally smooth.
To mitigate this, we introduce a regularization term that enforces smoothness by applying the $L_1$ norm to gradients in both the $X$ and $Y$ directions, with reduced gradient weights near edges to prevent over-smoothing.
% To avoid over-smoothing at edges, gradient weights are set to smaller values near edges.
The loss function is defined as follows:
\begin{equation}
    \resizebox{0.89\hsize}{!}{$
    \mathcal{L}_{smooth} =\scalebox{1.5}{$\frac{1}{|\Omega|}$} \sum\limits_{c\in\Omega} \lambda_X(c) |\partial_X \hat{D}(c)| + \lambda_Y(c) |\partial_Y \hat{D}(c)|,
    $}
\end{equation}
where $\lambda_X(c)=e^{-|\partial_X I(c)|}$, $\lambda_Y(c)=e^{-|\partial_Y I(c)|}$, and $c \in \Omega$ represents the set of all pixel locations~\cite{park2024testtime}.
However, using only these loss functions may dilute the structural prior in the pre-trained depth diffusion model, which is key for detail sharpness. 

\para{Relative Structure Similarity}
To address this, we design a new structure regularization term that transfers structure from the depth estimated by an off-the-shelf model to regularize overly smooth structures.
Inspired by the structure similarity (SSIM) loss~\cite{wang2004ssim}, we propose the Relative Stucture Similarity (R-SSIM) loss, designed to transfer structure across domains.
This loss is derived from SSIM by dropping the luminance term, which relies on absolute values:
\begin{equation}
    \mathcal{L}_{r-ssim}(d_1,d_2) = 1-\frac{2\sigma_{d_1d_2} + C}{\sigma_{d_1}^2 + \sigma_{d_2}^2 + C},
\end{equation}
where $d_1$ and $d_2$ represent spatial information in different domains, $C$ is a constant, and $\sigma$ denotes the normalized standard deviation of pixel values.
Here, $d_1$ is the relative depth map, and $d_2$ is the estimated complete depth map (or vice versa).
The key point is that these domains may differ in pixel value ranges and statistics. 
% \red{Ablation studies on R-SSIM loss are in the supplementary materials.}

\vspace{1.5mm}
\noindent Our comprehensive loss function is as follows:
\begin{equation}
    \mathcal{L} = \mathcal{L}_{depth} + \lambda_{smooth} \mathcal{L}_{smooth} + \lambda_{r-ssim} \mathcal{L}_{r-ssim},
\end{equation}
where $\lambda_{smooth}$ and $\lambda_{r-ssim}$ are regularization weights.
