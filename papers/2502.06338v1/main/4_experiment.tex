\begin{table}[t!]
\centering
    \scriptsize
    \setlength{\tabcolsep}{0.0035\linewidth}
    \caption{\textbf{Computational efficiency of EvSSC across different datasets.} Memory denotes training memory usage.}
    %\vskip-1ex
\setlength{\tabcolsep}{4pt} %5pt 设定列之间的宽度
\resizebox{\columnwidth}{!}{%
\begin{tabular}{l|>{\columncolor{gray!10}}l>{\columncolor{blue!8}}l|>{\columncolor{gray!10}}l>{\columncolor{blue!8}}l}
\toprule
\textbf{ } & \textbf{VoxFormer-S} & \textbf{EvSSC (VoxFormer)} & \textbf{SGN-S} & \textbf{EvSSC (SGN)}\\
\midrule\midrule
\multicolumn{5}{c}{\textit{DSEC-SSC}} \\ \midrule 
\textbf{mIoU} &25.62 & 26.34 & 29.06 & 29.55\\ 
\textbf{IoU}  & 47.25 & 47.29 & 43.70 & 43.99\\ 
\textbf{Memory}  & 9.74G & 10.52G & 10.19G & 10.70G\\ 
\textbf{Latency} & 0.732s & 0.836s & 0.941s & 1.193s \\\midrule 
\multicolumn{5}{c}{\textit{SemanticKITTI-E}} \\ \midrule 
\textbf{mIoU} & 12.86 & 13.61 & 14.55 & 15.15\\ 
\textbf{IoU}  & 44.42 & 45.01 & 43.60 & 43.17\\ 
\textbf{Memory} & 14.87G & 15.78G & 15.29G & 17.79G \\ 
\textbf{Latency}  & 0.996s & 1.005s & 0.855s &1.005s\\ \midrule 
\multicolumn{5}{c}{\textit{SemanticKITTI-C Shot Noise}} \\ \midrule 
\textbf{mIoU} & 8.29 & 12.64 & 13.62 & 14.32\\ 
\textbf{IoU}  & 44.26 & 45.04 & 42.05 & 42.54\\ 
\textbf{Memory} & 14.87G & 15.78G & 15.29G & 17.79G \\ 
\textbf{Latency}  & 0.996s & 1.005s & 0.855s &1.005s\\
\bottomrule
\end{tabular}
}
\label{table:efficiency}
%\vskip-3ex
\end{table}


\section{Experiments}
In this section, we demonstrate the effectiveness of our prior-based depth completion method in indoor (NYUv2~\cite{silberman2012nyu}, SceneNet~\cite{mccormac2017scenenet}, VOID~\cite{wong2020void}) and outdoor (Waymo~\cite{sun2020waymo}, nuScenes~\cite{caesar2020nuscene}, KITTI DC~\cite{uhrig2017sparsity}) scenarios, through both quantitative and qualitative evaluations. 
For evaluation, we use the Root Mean Squared Error (RMSE) and Mean Absolute Error (MAE), both standard metrics in depth completion where lower values indicate better performance.
The results are reported in meters.
Further details are provided in the supplementary material.

\subsection{Domain Generalization}
\label{sec:exp_domain}
\Tref{tab:generalization} summarizes the domain generalization performance of our method and previous test-time adaptation methods~\cite{wang2021tent,wang2022continual, park2024testtime} on indoor (NYU, SceneNet) and outdoor (Waymo, nuScenes).
Across various datasets, our prior-based approach consistently achieves the best or second-best performance.
 Notably, unlike test-time adaptation methods relying on pre-trained depth completion models in metric depth space, our method operates in affine-invariant depth space while achieving impressive performance.
Additionally, we demonstrate the model generality of our method by applying it to two depth diffusion models, Marigold~\cite{ke2023repurposing} and DepthFM~\cite{gui2024depthfm}, as shown in \Tref{tab:generalization}.
\Tref{tab:efficiency} further presents the inference time of our method across base models and sampling steps, demonstrating its potential for improving efficiency with minimal performance.
We also observe that our method captures details on the scene, reflecting true performance and demonstrating robust domain generalization as shown in~\Fref{fig:qualitative_adapt_outdoor} and \ref{fig:qualitative_adapt_indoor}.
We provide additional qualitative results in supplementary material.
\begin{figure*}[t]
\centering
    \includegraphics[width=0.95\linewidth]{Figures/qualitative_adapt_outdoor_arxiv.pdf}
    % \setlength{\abovecaptionskip}{2pt}
   \vspace{-3pt}
   \caption{\textbf{Qualitative comparison on the nuScenes test set.}
    In outdoor scenarios, our test-time alignment method performs robustly even under extreme weather conditions, clearly identifying critical elements such as vehicles and signs.
       } 
\label{fig:qualitative_adapt_outdoor}
% \vspace{-5pt}
\vspace{-1mm}
\end{figure*}

\begin{figure*}[t]
\centering
    \includegraphics[width=0.95\linewidth]{Figures/qualitative_adapt_indoor_arxiv.pdf}
    % \setlength{\abovecaptionskip}{2pt}
   \vspace{-3pt}
   \caption{\textbf{Qualitative comparison on the NYU test set.}
   In indoor scenarios, our test-time alignment method accurately captures scene structures (\eg, chairs) compared to the existing test-time adaptation methods.
   } 
\label{fig:qualitative_adapt_indoor}
\vspace{-2mm}
\end{figure*}


In the outdoor datasets, the ground truth is obtained by accumulating LiDAR points after removing those corresponding to moving objects, which can lead to variations in the ground truth. 
For a more reliable benchmark, we use the ground truth provided by \citet{park2024testtime} for the Waymo and by \citet{huang2022pcacc} for the nuScenes.
In the supplementary material, we discuss in detail the differences in ground truth acquisition methods and their impact on the performance of depth completion methods.

\begin{table}[t]
\centering

\renewcommand{\arraystretch}{0.95} % Adjust the row height factor
    \resizebox{1.0\linewidth}{!}{
    \begin{tabular}{m{4cm} c c cc}
    \toprule
    \multirow{2}[2]{*}{Method} & 
    \multirow{2}[2]{*}{\makecell{$N$-shot Scenario}} & \multirow{2}[2]{*}{RMSE} & \multirow{2}[2]{*}{MAE} \\
    & & & \\ [2pt]
    \midrule
    \addlinespace[4pt] 
    % \midrule [2pt] -> line strength
    VPP4DC & 0  & 0.247 & 0.077\\ [4pt]
    DepthPrompting & 1  & 0.358 & 0.206\\ [2pt]
     & 10  & 0.220 & 0.101\\ [4pt]
    UniDC & 1  & 0.210 & 0.107 \\ [2pt]
     & 10  & 0.166 & 0.079 \\ [4pt]
     \midrule 
     \addlinespace[4pt] 
    Ours (+Marigold) & 0  & 0.149 & \textbf{0.059} \\[2pt]
    Ours (+DepthFM) & 0  & \textbf{0.145} & 0.077\\ [2pt]
    \bottomrule
    \end{tabular}
    }
   \vspace{-3pt}
\caption{\textbf{Quantitative comparison with depth-prior-based methods  on the NYU test set.} We compare our method with zero- and few-shot approaches leveraging various depth foundation models.
}
\label{tab:depthprior}
\vspace{-15pt}
\end{table}

\subsection{Comparison with Depth-Prior-Based Methods}
We compare our depth-prior-based method, which leverages depth diffusion models~\cite{ke2023repurposing, gui2024depthfm}, with other depth completion methods utilizing depth foundation models.
Each method relies on different depth foundation models: VPP4DC~\cite{bartolomei2024vpp4dc} employs a stereo matching network~\cite{lipson2021raft}, DepthPrompting~\cite{park2024depthprompting} utilizes ResNet34~\cite{he2016resnet} to extract depth features~\cite{lu2020depth, qiu2019deeplidar}, and UniDC~\cite{park2024unidc} leverages DepthAnything~\cite{depthanything}. 
\Tref{tab:depthprior} shows the effectiveness of our method leveraging depth diffusion models.



\subsection{Comparison with Unsupervised Methods}
\begin{table*}[t]
\centering
\renewcommand{\arraystretch}{1.1} 
    \resizebox{0.9\linewidth}{!}{
    \begin{tabular}{m{3.6cm} ccc cc cc }
    \toprule
    \multirow{2}[2]{*}{Method} & \multicolumn{3}{c}{Features} & \multicolumn{2}{c}{KITTI DC} & \multicolumn{2}{c}{VOID} \\
    \cmidrule(lr){2-4} \cmidrule(lr){5-6} \cmidrule(lr){7-8} % column name mid rule
    & \makecell{Sparse Depth \\ Supervision} & 
    \makecell{Photometric \\ Consistency Loss} &
    \makecell{In-domain \\ Training} 
    & RMSE & MAE & RMSE & MAE \\
    \midrule 
    Self-S2D & {\textcolor{green(ncs)}{\cmark}}  &
    {\textcolor{green(ncs)}{\cmark}} (two-view)
    & 
    {\textcolor{green(ncs)}{\cmark}} 
    & 1.384 & 0.358 & 0.243 & 0.178\\ 
    VOICED & {\textcolor{green(ncs)}{\cmark}}
    &
    \phantom{--}{\textcolor{green(ncs)}{\cmark}} (multi-view)
    & 
    {\textcolor{green(ncs)}{\cmark}}  
    &
    1.230 & 0.308 & 0.169 & 0.085\\ 
    ScaffNet & {\textcolor{green(ncs)}{\cmark}}

    &
    \phantom{--}{\textcolor{green(ncs)}{\cmark}} (multi-view)
    & 
    {\textcolor{green(ncs)}{\cmark}} 
    &
    1.182 & 0.286 & 0.119 & 0.059\\ 
    KBNet & {\textcolor{green(ncs)}{\cmark}} 

    &
    \phantom{--}{\textcolor{green(ncs)}{\cmark}} (multi-view)
    & 
    {\textcolor{green(ncs)}{\cmark}} 
    &
    1.126 & 0.260 & 0.095 & 0.039\\ 
    SPTR & {\textcolor{green(ncs)}{\cmark}} 
    &
    \phantom{--}{\textcolor{green(ncs)}{\cmark}} (multi-view)
    & 
    {\textcolor{green(ncs)}{\cmark}} 
    &
    1.111 & 0.254 & 0.091 & 0.040\\ 
    \midrule
    Ours w/ ~Our Filtering & & & & 1.413 & 0.397 & 0.111& 0.044\\ 
    Ours w/ ~Manual Filtering & \multirow{-2}{*}{\textcolor{green(ncs)}{\cmark}} & \multirow{-2}{*}{\phantom{-}\textcolor{red}{\xmark} (monocular)} & \multirow{-2}{*}{\textcolor{red}{\xmark}} & 1.198 & 0.287 & 0.112& 0.045\\ 
    \bottomrule

    \end{tabular}
    }
   \vspace{-3pt}
\caption{\textbf{Quantitative comparison with unsupervised methods.}
Despite weaker settings, our method performs comparably to unsupervised methods (Self-S2D~\cite{ma2018self}, VOICED~\cite{wong2020void}, ScaffNet~\cite{wong2021scaffnet}, KBNet~\cite{wong2021unsupervised}, and SPTR~\cite{sptr}) when sparse depth, \ie the supervision signal, is reliable.
 To demonstrate this, we ablate two filtering methods: our prior-based filtering and manual filtering, which is the outlier filtering method suggested by each benchmark.
 % \hs{
 In this table, our method uses Marigold~\cite{ke2023repurposing} as the base model.}
% }
\label{tab:unsup_comparison}
\vspace{-2mm}
\end{table*}


We compare our zero-shot depth completion method with unsupervised methods~\cite{wong2021unsupervised, ma2018self, wong2021scaffnet} trained on the split training dataset of each benchmark, \ie, in-domain training.
As shown in~\Tref{tab:unsup_comparison}, our method demonstrates favorable performance without dense depth data, multi-view, and in-domain training on KITTI DC and VOID. 
Additionally, our method achieves comparable performance when adopting manual filtering, that is, the outlier filtering method suggested by each benchmark.
Figure~\ref{fig:unsup_in} shows qualitative results of ours and unsupervised methods. 
Our method achieves higher-fidelity depth completion, preserving the depth affinity better than other unsupervised methods.

\subsection{Ablation Studies}
\begin{figure}[t]
\centering
   \includegraphics[width=0.95\linewidth]{Figures/unsup_in_arxiv.pdf}
   % \setlength{\abovecaptionskip}{2pt}
   \vspace{-3pt}
   \caption{\textbf{Qualitative comparison on theVOID test set.}
    Compared to the state-of-the-art unsupervised method KBNet~\cite{wong2021unsupervised}, which uses multi-view photometric consistency, our prior-based approach better preserves scene structures and details using only monocular input.
   } 
\label{fig:unsup_in}
\vspace{-4mm}
\end{figure}


% \begin{table}[t]
\centering

\renewcommand{\arraystretch}{1.3} % Adjust the row height factor
    \resizebox{1.0\linewidth}{!}{
    \begin{tabular}{ccc  cc cc}
    \toprule
    \multirow{2}[2]{*}{\makecell{Sampling \\ Method}} &\multirow{2}[2]{*}{\makecell{R-SSIM \\ Loss}} & \multirow{2}[2]{*}{\makecell{Outlier \\ Filtering}} & \multicolumn{2}{c}{KITTI DC} & \multicolumn{2}{c}{VOID} \\
    \cmidrule(lr){4-5} \cmidrule(lr){6-7} 
     & & & RMSE & MAE & RMSE & MAE\\
    \midrule
    Naïve & & & 3.514 & 1.942 & 0.199 & 0.130\\
    Guided & & & 2.113 & 0.801 & 0.210 & 0.138\\ 
    % Guided Sampling + LORA~\cite{bansal2024universal, hu2022lora} & \textbf{1.4819} & 0.4463 \\ 
    Ours & & & 1.610 & 0.406 & 0.125 & 0.046 \\
    Ours & {\cmark} & & 1.502 & 0.409 & 0.111 & 0.044 \\
    Ours & {\cmark} & {\cmark} & 1.413 & 0.397 & 0.112 & 0.045\\
    \bottomrule
    \end{tabular}
    }
% \setlength{\abovecaptionskip}{2pt}
   \vspace{-3pt}
\caption{\textbf{Ablation studies.} We ablate our proposed methods including test-time alignment, R-SSIM loss, and prior-based outlier filtering, to demonstrate their effectiveness. 
% in zero-shot depth completion.
}
\label{tab:ablation_method}
% \vspace{-15pt}
\end{table}

% \begin{figure}[!t]
\centering
   \includegraphics[width=1.0\linewidth]{Figures/r-ssim_arxiv.pdf}
   \caption{\textbf{Qualitative ablation of R-SSIM loss.}
    This structural regularization sharpens details in areas such as signposts and car shapes.
   } 
\label{fig:r-ssim}
\end{figure}
% \begin{table}[!t]
% \centering
% \scalebox{0.68}{
%     \begin{tabular}{ll cccc}
%       \toprule
%       & \multicolumn{4}{c}{\textbf{Intellipro Dataset}}\\
%       & \multicolumn{2}{c}{Rank Resume} & \multicolumn{2}{c}{Rank Job} \\
%       \cmidrule(lr){2-3} \cmidrule(lr){4-5} 
%       \textbf{Method}
%       &  Recall@100 & nDCG@100 & Recall@10 & nDCG@10 \\
%       \midrule
%       \confitold{}
%       & 71.28 &34.79 &76.50 &52.57 
%       \\
%       \cmidrule{2-5}
%       \confitsimple{}
%     & 82.53 &48.17
%        & 85.58 &64.91
     
%        \\
%        +\RunnerUpMiningShort{}
%     &85.43 &50.99 &91.38 &71.34 
%       \\
%       +\HyReShort
%         &- & -
%        &-&-\\
       
%       \bottomrule

%     \end{tabular}
%   }
% \caption{Ablation studies using Jina-v2-base as the encoder. ``\confitsimple{}'' refers using a simplified encoder architecture. \framework{} trains \confitsimple{} with \RunnerUpMiningShort{} and \HyReShort{}.}
% \label{tbl:ablation}
% \end{table}
\begin{table*}[!t]
\centering
\scalebox{0.75}{
    \begin{tabular}{l cccc cccc}
      \toprule
      & \multicolumn{4}{c}{\textbf{Recruiting Dataset}}
      & \multicolumn{4}{c}{\textbf{AliYun Dataset}}\\
      & \multicolumn{2}{c}{Rank Resume} & \multicolumn{2}{c}{Rank Job} 
      & \multicolumn{2}{c}{Rank Resume} & \multicolumn{2}{c}{Rank Job}\\
      \cmidrule(lr){2-3} \cmidrule(lr){4-5} 
      \cmidrule(lr){6-7} \cmidrule(lr){8-9} 
      \textbf{Method}
      & Recall@100 & nDCG@100 & Recall@10 & nDCG@10
      & Recall@100 & nDCG@100 & Recall@10 & nDCG@10\\
      \midrule
      \confitold{}
      & 71.28 & 34.79 & 76.50 & 52.57 
      & 87.81 & 65.06 & 72.39 & 56.12
      \\
      \cmidrule{2-9}
      \confitsimple{}
      & 82.53 & 48.17 & 85.58 & 64.91
      & 94.90&78.40 & 78.70& 65.45
       \\
      +\HyReShort{}
       &85.28 & 49.50
       &90.25 & 70.22
       & 96.62&81.99 & \textbf{81.16}& 67.63
       \\
      +\RunnerUpMiningShort{}
       % & 85.14& 49.82
       % &90.75&72.51
       & \textbf{86.13}&\textbf{51.90} & \textbf{94.25}&\textbf{73.32}
       & \textbf{97.07}&\textbf{83.11} & 80.49& \textbf{68.02}
       \\
   %     +\RunnerUpMiningShort{}
   %    & 85.43 & 50.99 & 91.38 & 71.34 
   %    & 96.24 & 82.95 & 80.12 & 66.96
   %    \\
   %    +\HyReShort{} old
   %     &85.28 & 49.50
   %     &90.25 & 70.22
   %     & 96.62&81.99 & 81.16& 67.63
   %     \\
   % +\HyReShort{} 
   %     % & 85.14& 49.82
   %     % &90.75&72.51
   %     & 86.83&51.77 &92.00 &72.04
   %     & 97.07&83.11 & 80.49& 68.02
   %     \\
      \bottomrule

    \end{tabular}
  }
\caption{\framework{} ablation studies. ``\confitsimple{}'' refers using a simplified encoder architecture. \framework{} trains \confitsimple{} with \RunnerUpMiningShort{} and \HyReShort{}. We use Jina-v2-base as the encoder due to its better performance.
}
\label{tbl:ablation}
\end{table*}
\Tref{tab:ablation_method} shows ablation studies to assess the efficacy of the test-time alignment method, R-SSIM loss, and outlier filtering algorithm.
The ablation studies are conducted on both indoor (VOID) and outdoor (KITTI DC) datasets.
Compared to other sampling methods, \ie, no guidance and the guided sampling~\cite{bansal2024universal}, the proposed test-time alignment method brings significant performance gain. The R-SSIM loss further enhances the performance and has a remarkable effect on preserving depth affinity.
The prior-based outlier filtering is more effective on the outdoor dataset 
than on the indoor dataset, as the sparse depth in the indoor dataset consists of
reliable points sampled from the ground truth.
We also qualitatively ablate the performance of the R-SSIM loss as shown in \Fref{fig:r-ssim}, highlighting how it effectively regularizes diffusion structural prior, leading to sharpen details.

