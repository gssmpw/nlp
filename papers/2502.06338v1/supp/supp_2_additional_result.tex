% \begin{figure*}[t]
\centering
   \includegraphics[width=0.9\linewidth]{supp_figures/waymo_arxiv.pdf}
   \caption{\textbf{Qualitative results on Waymo}} 
\label{fig:qual_waymo}
\end{figure*}
% \begin{figure*}[t]
\centering
   \includegraphics[width=0.85\linewidth]{supp_figures/scenenet_arxiv.pdf}
   \caption{\textbf{Qualitative results on SceneNet}} 
\label{fig:qual_scenenet}
\end{figure*}
\begin{figure*}[t]
\centering
\begin{minipage}{\textwidth}
   \centering
   \includegraphics[width=0.85\linewidth]{supp_figures/scenenet_arxiv.pdf}
   \caption{\textbf{Qualitative results on SceneNet.} Our zero-shot depth completion method outperforms the state-of-the-art test-time adaptation method by capturing the scene's structure effectively.} 
   \label{fig:qual_scenenet}
\end{minipage}

\vspace{10mm} % Adjust this space as needed

\begin{minipage}{\textwidth}
   \centering
   \includegraphics[width=0.9\linewidth]{supp_figures/waymo_arxiv.pdf}
   \caption{\textbf{Qualitative results on Waymo.} Our zero-shot depth completion method demonstrates robust performance even in extreme environments, such as rain or nighttime conditions.} 
   \label{fig:qual_waymo}
\end{minipage}
\end{figure*}


\section{Additional Qualitative Results}
In this section, we provide additional qualitative results corresponding to the experiments discussed in each subsection of the main paper.

\para{Domain generalization}
We provide additional qualitative results for dataset not covered in the main paper, such as SceneNet~\cite{mccormac2017scenenet} and Waymo~\cite{sun2020waymo}, in \Fref{fig:qual_scenenet} and \ref{fig:qual_waymo}.
Most existing pre-trained depth completion models tend to fail when faced with the difficult conditions typically encountered in real-world environments.
We also demonstrate the robust performance of our prior-based method in extreme environments, such as rain or nighttime, as shown in Fig.\ref{fig:qual_scenenet} and \ref{fig:qual_waymo}.
Additional results for these scenes will also be provided in the supplementary video.
% By mixing latents across consecutive frames, as suggested in [Reference], we can efficiently achieve temporally consistent results.
% % https://huggingface.co/docs/diffusers/using-diffusers/marigold_usage 이걸 레퍼런스로 달고싶습니다

\begin{figure*}[t]
\centering
   \includegraphics[width=1.0\linewidth]{supp_figures/unsup_out_arxiv.pdf}
   \caption{\textbf{Qualitative comparison on KITTI DC validation set.}
    Compared to the unsupervised methods~\cite{ma2018self, wong2021unsupervised} with comparable quantitative performance, our prior-based approach better preserves the scene structure and details.
   } 
\label{fig:unsup_out}
\end{figure*}


\para{Comparison with unsupervised methods on KITTI}
In \Fref{fig:unsup_out}, we provide a qualitative comparison on the KITTI DC dataset, an outdoor dataset not included in the main paper. Despite using only a monocular RGB view and sparse depth, unlike previous unsupervised methods~\cite{ma2018self, wong2021unsupervised}, we also complete a well-structured depth map
% KITTI에 대해서 추가 결과 제공. - figure

% (Optional) 3D unproject결과 NuScene / nyu에 대해서 추가 제공 - figure
