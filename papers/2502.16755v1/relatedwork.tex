\section{RELATED WORK}
\subsection{Distinct Characteristics of E-Scooter Rider Behaviour}

Although extensive research has explored cyclist behaviours and safety perceptions, the emergence of e-scooters as a novel mode of transport requires a specific investigative focus. Distinct from bicycles, e-scooters exhibit unique usage characteristics that demand specific attention in research. The \textit{demographic profile} of e-scooter users differs from that of cyclists. According to \cite{WeRideAustraliaReport}, the majority of e-scooter users are predominantly within the age groups of 18-34 and 35-44, whereas cycling participation is considerably higher among older age groups. Specifically, only 13.46\% of individuals aged between 45 and 54 engage in e-scooter usage, compared to 37\% in the same age group who participate in cycling. This trend is supported by additional research \cite{su142114303, NIKIFORIADIS2021102790}, which indicates that younger individuals are more inclined to use e-scooters, thereby the rider behaviour of cyclists and e-scooter riders can be different.

\textit{Accident statistics} reveal that incidents involving e-scooters surpass those associated with bicycles. A comparative study by \citet{james2023comparison} on injuries related to e-scooters, bicycles, and motorbikes, found that injuries from e-scooter use surged by 2.8 times over a four-year period, while bicycle-related injuries increased by only 1.2 times. This marked disparity highlights the need for further research into understanding user behaviour specific to e-scooters. Furthermore, \citet{su11205591} highlights that individuals feel less comfortable around e-scooters compared to bicycles. This might exhibit varied pedestrian reactions that influence differences in rider behaviours.

The \textit{manoeuvring patterns} of e-scooters, which are powered electrically, differ significantly from those of traditional bicycles \cite {10.1145/3313831.3376499}. For instance, research by \citet{visualAttenstionTobii} demonstrated that e-scooters are more likely to weave through pedestrian traffic compared to bicycles. \citet{dozza2022data} suggest that e-scooters offer greater manoeuvrability and comfort than bicycles, while they necessitate longer braking distances.

Unlike bicycles, \textit{regulations} governing e-scooters exhibit significant variability across localities \cite{10.1145/3313831.3376499, NIKIFORIADIS2021102790}. For example, Italian law restricts e-scooter speeds to 20 km/h, permits their use on both pedestrian and cycle lanes, and sets a minimum user age of 14 years \cite{su142114303}.  In contrast, Austrian regulations require a minimum rider age of 12 years, enforce a maximum speed limit of 25 km/h, and stipulate that e-scooters be used on bike paths or roadways where bike paths are not available \cite{CountryOverview}. This diversity is not only international but also within a country, reflecting the challenges of integrating this novel transportation mode into existing systems. Due to their recent introduction, the precise impact of e-scooters remains unclear. Therefore, some cities conduct e-scooter trials before integrating e-scooters in their transport systems \cite{VICtrial, AusTrials}.

In summary, in contrast to bicycles, which are well-established components of transportation planning, e-scooters represent unique challenges that emphasise the importance of e-scooter-focused rider behaviour analysis.



    \subsection{Traditional Methods for Capturing Rider Behaviour}
There are multiple methodologies employed to study e-scooter rider behaviour and interactions. \textit{(1) Interview} is one of the most common approaches that provides detailed insights into e-scooter usage \cite{10.1007/978-3-030-85613-7_26,10.1145/3313831.3376499, doi:10.1080/17450101.2021.1967097}. However, the presence of an interviewer might influence responses, and participants may not fully disclose their behaviours due to biases like social desirability or recall issues \cite{ burt2023scooter}. 
\textit{(2) Observations} \cite{TUNCER2020102702, 10.1145/3313831.3376499,10.1145/3544548.3581049, 10.1145/3544548.3581045, anke2023micro} on individual rider allow researchers to directly assess rider behaviour, which might be affected by observer bias. Further, the data analysis can be complex and time-consuming. \textit{(3) Survey} \cite{LAA2020102874, GIOLDASIS2021106427, anke2023micro, WEISS2024100047} provide data from a broad sample of riders but are prone to personal biases and often lack the context of traffic conditions that might influence riding behaviour \cite{10.1145/3025453.3025911}. \textit{(4) Media report analysis} \cite{GOSSLING2020102230} offers a broader public opinion on issues and trends, yet it typically focuses more on accidents rather than normal riding behaviors, which might skew understanding of everyday e-scooter use \cite{YANG2020105608, WHITE2023182}.

These traditional methods have strengths and inherent limitations in different situations. Overall, they can be influenced by personal biases, inaccuracies in self-reporting, lack of context details and can be time-consuming. In contrast, the use of sensors and cameras to capture events as they occur can provide an unbiased and detailed record of behaviours that might be unnoticeable to the human eye.

    \subsection{Using Multi-modal Data to Analyse Rider Behaviour}

The usage of different data modalities such as video \cite{hong2022evaluation, 10.1145/3313831.3376499,KOVACSOVA2018270, KAYA2021106380, WHITE2023182}, gaze movements \cite{visualAttenstionTobii, 10.1145/3544548.3581049, MANTUANO2017408,10.1145/3204493.3214307, KAYA2021106380, VONSTULPNAGEL2020222}, physiological measurements \cite{COBB2021172, PhysiologicalCyclist, distefano2020physiological},
and vehicle operation data (e.g., speed, acceleration, braking, steering angle) \cite{MORGENSTERN2020104740, doi:10.1080/15389588.2019.1643015} to understand driver or rider behaviour has been an established area of research (see Table \ref{tab:lit_gap}).


\begin{table}[h]
    \centering
    \captionsetup{justification=centering}
    \caption{Related works for rider behaviour analysis with different data modalities.}
    \label{tab:lit_gap}
    \small
    \begin{tabular}{lccccccccc}
        \toprule
         & \multicolumn{4}{c}{Road user} & & \multicolumn{3}{c}{Device} & \\
         \cmidrule(lr){2-5} \cmidrule(lr){7-9}
         Paper & Pedestrian & Cyclist & E-scooter & Driver & & Acceler & Eye & Video   & Infrastructure \\
         & & & rider & & & -ometer & -tracker & camera & comparison\\
        \midrule
        \citet{10.1145/3313831.3376499} & & & \checkmark & & & & & \checkmark & -  \\ \hline
        \citet{hong2022evaluation} & & & \checkmark & & & & & \checkmark  & -  \\ \hline
         \citet{visualAttenstionTobii} & \checkmark  & \checkmark  & \checkmark & & & \checkmark & \checkmark & & -  \\ \hline
         \citet{10.1145/3544548.3581049} & & \checkmark & & & & & \checkmark & & -  \\ \hline
         \citet{10.1145/3568444.3568451}  & & \checkmark & & & & & \checkmark & \checkmark & Yes  \\ \hline
        \citet{MANTUANO2017408} & & \checkmark & & & & & \checkmark & & Yes \\ \hline
        \citet{6595621} & & \checkmark & & & & \checkmark & & \checkmark & - \\ \hline
        \citet{PETZOLDT2017477} & & \checkmark & & & & \checkmark & & \checkmark & - \\ \hline
        \citet{7795977} & & \checkmark & & & & \checkmark & & \checkmark & - \\ \hline
        \citet{10.1145/3204493.3214307} & \checkmark & \checkmark & & & & & \checkmark & & - \\ \hline
       \citet{KAYA2021106380} & & & & \checkmark & & & \checkmark & \checkmark & -  \\ \hline
        Ours & & & \checkmark & & & \checkmark & \checkmark & \checkmark & Yes \\
        \bottomrule
    \end{tabular}
\end{table}


E-scooters being a novel transport modality, researchers have conducted qualitative video-based studies to understand the rider behaviour. \citet{10.1145/3313831.3376499} explored the riding practices of e-scooter riders, employing a video ethnographic study. They collected video recordings of 3 e-scooter riders in urban spaces, using a chest-mounted camera worn by a researcher. Their findings revealed information on negotiations with other users of public spaces, and the novel adaptation of riding behaviour in various circumstances, such as traffic lights and road crossings. \citet{hong2022evaluation} used video recordings collected from fixed cameras located at different sites (e.g., side walks, cross walks, intersections, pedestrian areas) to analyse e-scooter traffic. They reported instances of hazardous riding behaviours and the ways in which e-scooter riders modify their riding strategies in response to varying traffic conditions.

Given that eye movements indicate visual attention, eye-tracking based studies have been used to understand road users' behaviour \cite{10.1145/3607822.3614532}. \citet{visualAttenstionTobii} compared the visual attention and speed of 12 participants while they were functioning as pedestrians, cyclists, and e-scooter riders on a shared road. The results showed the distribution of visual fixations across AOIs varied between the different types of users. Specifically, cyclists and e-scooter riders were more likely to focus their attention on the road ahead as compared to pedestrians. \citet{10.1145/3544548.3581049} collected gaze movements and GPS data from 12 commuter cyclists. Their analysis showed how cyclists' gaze patterns are distributed across different parts of vehicles, traffic lights and road markings in various traffic contexts. To compare the movement patterns of cyclists and pedestrians, \citet{10.1145/3204493.3214307} used gaze movement data and accelerometer data collected from eye-tracker glasses. They found that both cyclists and pedestrians have common gaze sequences and different patterns of shoulder checks. \citet{KAYA2021106380} explored the effect of drivers’ cycling experience on their visual scanning behaviour using a vehicle instrumented with two cameras, and eye movements of the driver. Their results showed that drivers with cycling experience have a lower probability of visual scanning failures compared to other drivers.

Only a few naturalistic studies have experimented with the differences in rider behaviours across various types of transport infrastructure \cite{MANTUANO2017408, 10.1145/3568444.3568451}. \citet{MANTUANO2017408} examined the gaze behaviour of cyclists on cycle tracks, both exclusive and shared with pedestrians, involving 16 participants. They analysed the proportion of fixations and attention dispersion using eye-tracker glasses. In contrast to their approach, our study collects a more diverse range of data and considers additional traffic conditions, thereby enabling a broader set of evaluation metrics. \citet{10.1145/3568444.3568451} explored the behaviour of cyclists on self-balancing bicycles across different types of transport infrastructure, including parks, main roads, and junctions, through video footage and eye movement analysis. Different to our research, which primarily examines the natural riding behaviours of e-scooter users, their study focuses on the multitasking behaviours of participants while riding cycles.

In summary, different from previous efforts, our work compares the e-scooter rider behaviour at different transport infrastructure, in contrast to other studies that focus on rider behaviour in specific traffic situations such as left turns, traffic light-protected intersections, and cross walks. Furthermore, we employ additional devices to collect more data, allowing us to obtain multiple measurements for improved comparisons. Our research specifically targets e-scooter riders, as opposed to cyclists, pedestrians or drivers, thereby shedding light on this novel mode of transport users.