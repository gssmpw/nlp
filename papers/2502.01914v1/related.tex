\section{Related Works}


Graph-based cooperative games and the properties of core imputations within them have been extensively studied. The seminal work of \cite{Shapley1971assignment} established that in the assignment game, the core is precisely characterized by the set of optimal dual solutions. A natural extension of the assignment game is the $b$-matching game, where each vertex can be matched multiple times up to a predefined capacity.

There are two versions of the $b$-matching game. \cite{b_matching_nucleolus} distinguish them as simple and non-simple $b$-matching games. The simple $b$-matching game is the edge-constrained version where each edge is to be matched at most once. \cite{biro2018stable} demonstrated that verifying whether an imputation belongs to the core in this setting is co-NP-complete. Our work focuses on the non-simple $b$-matching game, the edge-unconstrained version, where edges can be used multiple times. These games are often referred to as \textit{transportation games} (see \cite{Transportation_games}), and we establish a similar hardness result in this setting. Notably, \cite{sotomayor1992multiple}) proved that the core of transportation games is always non-empty.

Beyond $b$-matching games, another well-studied class of cooperative games is the \textit{minimum-cost spanning tree (MST) games}. \cite{GranotHuberman1981} introduced MST games and showed that their core is always non-empty. Several methods have been developed to compute core imputations, including Bird’s rule(\cite{Bird1976}) and fairness-based approaches (\cite{Kar}, \cite{feltkamp1994irreducible}, \cite{bergantinos2007fair}, \cite{trudeau2012new}). However, verifying whether an imputation belongs to the core of an MST game is co-NP-hard (\cite{Faigle1997complexity}).

A closely related class of cooperative games is the max-flow game, introduced by \cite{Kalai1982totally}, where the core is also always non-empty. Yet, similar to MST games, testing whether an imputation belongs to the core is co-NP-hard (\cite{Fang2002computational}).

Transportation games share structural properties with these games. \cite{Transportation_games} demonstrated that any optimal dual solution of the linear programming (LP) formulation provides a core imputation. However, unlike in the assignment game, these dual-based imputations, collectively called the \textit{Owen set}, do not fully characterize the core of $b$-matching games (\cite{vazirani2023lpduality}).

The Owen set, first introduced by \cite{Owen1975} for linear production games, was later formalized in \cite{Owen.Characterization}. It consists of core imputations derived from dual solutions, where each dual variable represents an agent's shadow price. More recently, \cite{ggsv2024equitablefaircore} extended the Owen set framework to MST and max-flow games, noting that in these settings, dual variables do not directly correspond to shadow prices, requiring careful interpretation of dual-optimal solutions.

Finally, fairness concepts such as minimax, maximin, leximin, and leximax have been widely explored in cooperative game theory(\cite{minimax_group_fairness}). \cite{Vazirani-leximin} developed an efficient method for computing leximin/leximax fair core imputations for the assignment game. Later, \cite{ggsv2024equitablefaircore} established that computing such imputations for MST and max-flow games is NP-hard, although these imputations can be efficiently computed within the Owen set.


% \todo[inline]{RG: Should we add discussion about Nucleolus? Discuss with Vijay.}

