% This file is for the preliminaries section

\section{Preliminaries}


\begin{definition}
    Let $N$ be a set of agents. A cooperative game on $N$ is defined by a {\em characteristic function} $\nu: 2^N \rightarrow \mathbb{R}_+$, where for each $S \subseteq N$, $\nu(S)$ is the value that the sub-coalition $S$ can produce on its own. $N$ is also called the grand coalition. We will use $(N,\nu)$ to define a game.
    % A cooperative game involves a set \(N\) of \(n\) agents and a function \(v\), where \(v: 2^N \to \mathbb{R}_+\) assigns to each subset \(S\) of \(N\) a value \(\nu(S)\), representing what that group can achieve on its own. The entire set \(N\) is known as the grand coalition.
\end{definition}

$\nu(N)$ is also called \textit{value/worth} of the game.  

% \todo{Add definition of profit share and update the definition of imputation.}

\begin{definition}
    A \textit{profit share} is $p: N \to \mathbb{R}_+$ is an assignment of profits to the agents.
\end{definition}
An imputation is a kind of profit share that distributes the exact worth of the game among the agents, i.e.,
\begin{definition}
An \textit{imputation} is a profit share $p$ such that $\sum_{i \in N} p(i)=\nu(N)$.
\end{definition}

We extend the definition of $p$ to a set of agents, say $S$, to represent the total profit received by all agents in $S$, i.e., $p(S)=\sum_{i\in S}p(i)$.

\begin{definition}\label{defCore}
An imputation $p$ is in the {\em core} of a profit-sharing game $(N,\nu)$ if and only if for every possible sub coalition of agents $S \subseteq N$, the total profit shares of the members of $S$ is at least as large as the worth they can generate by themselves i.e., $ p(S) \geq \nu(S)$.
\end{definition}


\begin{definition}
Let $P$ be a set of imputations of a game $(N,\nu)$ and $p_1,p_2 \in P$. Let $l_1,l_2$ be the lists formed by arranging the shares of agents in $p_1,p_2$ in ascending order. $l_1$ is {\em lexicographically larger} than $l_2$ if $l_1$ has the larger value at the first index where the two lists differ. The imputation in $P$ which is lexicographically larger than all other imputations in $P$ is the {\em lexicographically minimum} or {\em leximin} imputation in $P$.  
\end{definition}


\begin{definition}
Let $P$ be a set of imputations of a game $(N,\nu)$ and $p_1,p_2 \in P$. Let $l_1,l_2$ be the lists formed by arranging the shares of agents in $p_1,p_2$ in descending order. $l_1$ is {\em lexicographically smaller} than $l_2$ if $l_1$ has the smaller value at the first index where the two lists differ. The imputation in $P$ which is lexicographically smaller than all other imputations in $P$ is the {\em lexicographically maximum} or {\em leximax} imputation in $P$.  
\end{definition}

Let $G = (U,V,E)$ be a weighted bipartite graph with edge weights $w: E \rightarrow \mathbb{R}_+$. Let $b: U\cup V \rightarrow \mathbb{Z}_+$, a vertex capacity function, specify the number of times a vertex can be matched. Any choice of edges, with multiplicity, subject to vertex capacity function, $b$, is called a $b$-matching. 

A \textit{maximum weight bipartite $b$-matching game}, \textit{$b$-matching game} for short, is a game on an instance of bipartite $b$-matching graph with the vertices as agents and the worth of a sub coalition of agents $S\subseteq U\cup V$, denoted by $\nu(S)$, is the weight of a maximum weight $b$-matching, max. wt. $b$-matching for short, in the graph G restricted to vertices in $S$ only. Where obvious, we will use $G$ to mean the set of all agents and $p(G)$ and $\nu(G)$ to mean the total profit of all agents and the worth of the game respectively.







