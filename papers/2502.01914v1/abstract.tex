% This file is for abstract

\begin{abstract}

The core is a quintessential solution concept for profit sharing in cooperative game theory. An imputation allocates the worth of the given game among its agents. The imputation lies in the core of the game if, for each sub-coalition, the amount allocated to its agents is at least the worth of this sub-coalition. Hence, under a core imputation, each of exponentially many sub-coalitions gets satisfied. 

The following computational question has received much attention: Given an imputation, does it lie in the core? Clearly, this question lies in co-NP, since a co-NP certificate for this problem would be a sub-coalition which is not satisfied under the imputation. This question is in P for the assignment game \cite{Shapley1971assignment} and has been shown to be co-NP-hard for several natural games, including max-flow \cite{Fang2002computational} and MST \cite{Faigle1997complexity}. The one natural game for which this question has remained open is the $b$-matching game when the number of times an edge can be matched is unconstrained; in case each edge can be matched at most once, it is co-NP-hard \cite{biro2018stable}.

At the outset, it was not clear which way this open question would resolve: on the one hand, for all but one game, this problem was shown co-NP-hard and on the other hand, proximity to the assignment problem and the deep structural properties of matching could lead to a positive result. In this paper, we show that the problem is indeed co-NP-hard. 

\end{abstract}
