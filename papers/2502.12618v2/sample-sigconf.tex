%%
%% This is file `sample-sigconf.tex',
%% generated with the docstrip utility.
%%
%% The original source files were:
%%
%% samples.dtx  (with options: `all,proceedings,bibtex,sigconf')
%% 
%% IMPORTANT NOTICE:
%% 
%% For the copyright see the source file.
%% 
%% Any modified versions of this file must be renamed
%% with new filenames distinct from sample-sigconf.tex.
%% 
%% For distribution of the original source see the terms
%% for copying and modification in the file samples.dtx.
%% 
%% This generated file may be distributed as long as the
%% original source files, as listed above, are part of the
%% same distribution. (The sources need not necessarily be
%% in the same archive or directory.)
%%
%%
%% Commands for TeXCount
%TC:macro \cite [option:text,text]
%TC:macro \citep [option:text,text]
%TC:macro \citet [option:text,text]
%TC:envir table 0 1
%TC:envir table* 0 1
%TC:envir tabular [ignore] word
%TC:envir displaymath 0 word
%TC:envir math 0 word
%TC:envir comment 0 0
%%
%%
%% The first command in your LaTeX source must be the \documentclass
%% command.
%%
%% For submission and review of your manuscript please change the
%% command to \documentclass[manuscript, screen, review]{acmart}.
%%
%% When submitting camera ready or to TAPS, please change the command
%% to \documentclass[sigconf]{acmart} or whichever template is required
%% for your publication.
%%
%%
% \documentclass[sigconf]{acmart}
\documentclass[sigconf]{acmart}
\pdfoutput=1
%%
%% \BibTeX command to typeset BibTeX logo in the docs
\usepackage{float}
\usepackage{stfloats}
\usepackage{amsthm}
\usepackage{hyperref}
\usepackage{url}
\usepackage{graphicx}
\usepackage{wrapfig}
\usepackage{bbm}
\usepackage{subfigure}
\usepackage{caption}
\usepackage{subcaption}
\usepackage{ragged2e} 
\usepackage{enumitem}
\usepackage{balance}
\usepackage{booktabs,makecell, multirow, tabularx}
\AtBeginDocument{%
  \providecommand\BibTeX{{%
    Bib\TeX}}}
\newtheorem{proposition}{Proposition}
\newtheorem{lemma}{Lemma}
%% Rights management information.  This information is sent to you
%% when you complete the rights form.  These commands have SAMPLE
%% values in them; it is your responsibility as an author to replace
%% the commands and values with those provided to you when you
%% complete the rights form.
% \setcopyright{acmlicensed}
% \copyrightyear{2018}
% \acmYear{2018}
% \acmDOI{XXXXXXX.XXXXXXX}

\copyrightyear{2025}
\acmYear{2025}
\setcopyright{acmlicensed}
\acmConference[WWW '25] {Proceedings of the ACM Web Conference 2025}{April 28--May 2, 2025}{Sydney, NSW, Australia.}
\acmBooktitle{Proceedings of the ACM Web Conference 2025 (WWW '25), April 28--May 2, 2025, Sydney, NSW, Australia}
\acmISBN{979-8-4007-1274-6/25/04}
\acmDOI{10.1145/3696410.3714927}
\settopmatter{printacmref=true}

\newcommand{\ie}{\emph{i.e., }}
\newcommand{\eg}{\emph{e.g., }}
\newcommand{\etal}{\emph{et al.}}
\newcommand{\etc}{\emph{etc.}}
\newcommand{\wrt}{\emph{w.r.t. }}
\newcommand{\cf}{\emph{cf. }}
\newcommand{\aka}{\emph{a.k.a. }}
\newcommand{\st}{\emph{s.t. }}


%% These commands are for a PROCEEDINGS abstract or paper.
%%
%%  Uncomment \acmBooktitle if the title of the proceedings is different
%%  from ``Proceedings of ...''!
%%
% \acmBooktitle{Woodstock '18: ACM Symposium on Neural Gaze Detection,
 % June 03--05, 2018, Woodstock, NY}


%%
%% Submission ID.
%% Use this when submitting an article to a sponsored event. You'll
%% receive a unique submission ID from the organizers
%% of the event, and this ID should be used as the parameter to this command.
%%\acmSubmissionID{123-A56-BU3}

%%
%% For managing citations, it is recommended to use bibliography
%% files in BibTeX format.
%%
%% You can then either use BibTeX with the ACM-Reference-Format style,
%% or BibLaTeX with the acmnumeric or acmauthoryear sytles, that include
%% support for advanced citation of software artefact from the
%% biblatex-software package, also separately available on CTAN.
%%
%% Look at the sample-*-biblatex.tex files for templates showcasing
%% the biblatex styles.
%%

%%
%% The majority of ACM publications use numbered citations and
%% references.  The command \citestyle{authoryear} switches to the
%% "author year" style.
%%
%% If you are preparing content for an event
%% sponsored by ACM SIGGRAPH, you must use the "author year" style of
%% citations and references.
%% Uncommenting
%% the next command will enable that style.
%%\citestyle{acmauthoryear}


%%
%% end of the preamble, start of the body of the document source.
\begin{document}

%%
%% The "title" command has an optional parameter,
%% allowing the author to define a "short title" to be used in page headers.
\title{Uncertainty-Aware Graph Structure Learning}

%%
%% The "author" command and its associated commands are used to define
%% the authors and their affiliations.
%% Of note is the shared affiliation of the first two authors, and the
%% "authornote" and "authornotemark" commands
%% used to denote shared contribution to the research.
% \author{Ben Trovato}
% \authornote{Both authors contributed equally to this research.}
% \email{trovato@corporation.com}
% \orcid{1234-5678-9012}
% \author{G.K.M. Tobin}
% \authornotemark[1]
% \email{webmaster@marysville-ohio.com}
% \affiliation{%
%   \institution{Institute for Clarity in Documentation}
%   \city{Dublin}
%   \state{Ohio}
%   \country{USA}
% }

% \author{Lars Th{\o}rv{\"a}ld}
% \affiliation{%
%   \institution{The Th{\o}rv{\"a}ld Group}
%   \city{Hekla}
%   \country{Iceland}}
% \email{larst@affiliation.org}

% \author{Valerie B\'eranger}
% \affiliation{%
%   \institution{Inria Paris-Rocquencourt}
%   \city{Rocquencourt}
%   \country{France}
% }

% \author{Aparna Patel}
% \affiliation{%
%  \institution{Rajiv Gandhi University}
%  \city{Doimukh}
%  \state{Arunachal Pradesh}
%  \country{India}}

% \author{Huifen Chan}
% \affiliation{%
%   \institution{Tsinghua University}
%   \city{Haidian Qu}
%   \state{Beijing Shi}
%   \country{China}}

% \author{Charles Palmer}
% \affiliation{%
%   \institution{Palmer Research Laboratories}
%   \city{San Antonio}
%   \state{Texas}
%   \country{USA}}
% \email{cpalmer@prl.com}

% \author{John Smith}
% \affiliation{%
%   \institution{The Th{\o}rv{\"a}ld Group}
%   \city{Hekla}
%   \country{Iceland}}
% \email{jsmith@affiliation.org}

% \author{Julius P. Kumquat}
% \affiliation{%
%   \institution{The Kumquat Consortium}
%   \city{New York}
%   \country{USA}}
% \email{jpkumquat@consortium.net}

%%
%% By default, the full list of authors will be used in the page
%% headers. Often, this list is too long, and will overlap
%% other information printed in the page headers. This command allows
%% the author to define a more concise list
%% of authors' names for this purpose.
\author{Shen Han}
\authornotemark[2]
\orcid{0000-0001-6714-5237}
\affiliation{
\department{The State Key Laboratory of Blockchain and Data Security,}
\institution{Zhejiang University}
\city{Hangzhou}
\country{China}}
\email{drhanshen@zju.edu.cn}

\author{Zhiyao Zhou}
\authornotemark[2]
\orcid{0009-0005-9291-169X}
\affiliation{
\department{The State Key Laboratory of Blockchain and Data Security,}
\institution{Zhejiang University}
\city{Hangzhou}
\country{China}}
\email{zjucszzy@zju.edu.cn}
\author{Jiawei Chen}
\authornote{Corresponding author.}
\authornote{College of Computer Science and Technology, Zhejiang University.}
\authornote{Hangzhou High-Tech Zone (Binjiang) Institute of Blockchain and Data Security.}
\orcid{0000-0002-4752-2629}
\affiliation{
\department{The State Key Laboratory of Blockchain and Data Security,}
% \institution{Hangzhou High-Tech Zone (Binjiang) Institute of Blockchain and Data Security,}
\institution{Zhejiang University}
\city{Hangzhou}
\country{China}}
\email{sleepyhunt@zju.edu.cn}
\author{Zhezheng Hao}
\orcid{0000-0001-9900-894X}
\affiliation{
\institution{Northwestern Polytechnical University}
\city{Xi'an}
\country{China}}
\email{haozhezheng@outlook.com}

\author{Sheng Zhou}
\orcid{0000-0003-3645-1041}
\affiliation{
\institution{Zhejiang University}
\city{Hangzhou}
\country{China}}
\email{zhousheng_zju@zju.edu.cn}

\author{Gang Wang}
\orcid{0000-0001-6248-1426}
\affiliation{
\institution{Bangsun Technology}
\city{Hangzhou}
\country{China}}
\email{wanggang@bsfit.com.cn}

\author{Yan Feng}
\authornotemark[2]
\orcid{0000-0002-3605-5404}
\affiliation{
\department{The State Key Laboratory of Blockchain and Data Security,}
\institution{Zhejiang University}
\city{Hangzhou}
\country{China}}
\email{fengyan@zju.edu.cn}

\author{Chun Chen}
\authornotemark[2]
\orcid{0000-0002-6198-7481}
\affiliation{
\department{The State Key Laboratory of Blockchain and Data Security,}
\institution{Zhejiang University}
\city{Hangzhou}
\country{China}}
\email{chenc@zju.edu.cn}


\author{Can Wang}
\orcid{0000-0002-5890-4307}
\authornotemark[3]
\affiliation{
\department{The State Key Laboratory of Blockchain and Data Security,}
% \institution{Hangzhou High-Tech Zone (Binjiang) Institute of Blockchain and Data Security,}
\institution{Zhejiang University}
\city{Hangzhou}
\country{China}}
\email{wcan@zju.edu.cn}
\renewcommand{\shortauthors}{Shen Han, et al.}
%%
%% The abstract is a short summary of the work to be presented in the
%% article.
\begin{abstract}
\textit{Graph Neural Networks} (GNNs) have become a prominent approach for learning from graph-structured data. However, their effectiveness can be significantly compromised when the graph structure is suboptimal.
To address this issue, \textit{Graph Structure Learning} (GSL) has emerged as a promising technique that refines node connections adaptively. Nevertheless, we identify two key limitations in existing GSL methods: 1) Most methods primarily focus on node similarity to construct relationships, while overlooking the quality of node information. Blindly connecting low-quality nodes and aggregating their ambiguous information can degrade the performance of other nodes. 2) The constructed graph structures are often constrained to be symmetric, which may limit the model's flexibility and effectiveness.

To overcome these limitations, we propose an \textbf{Uncertainty-aware Graph Structure Learning} (UnGSL) strategy. UnGSL estimates the uncertainty of node information and utilizes it to adjust the strength of directional connections, where the influence of nodes with high uncertainty is adaptively reduced. 
Importantly, UnGSL serves as a plug-in module that can be seamlessly integrated into existing GSL methods with minimal additional computational cost. In our experiments, we implement UnGSL into six representative GSL methods, demonstrating consistent performance improvements.
The code is available at \href{https://github.com/UnHans/UnGSL}{https://github.com/UnHans/UnGSL}.


% Graph structure learning (GSL) is a promising data-centric approach that generates denoised graph data to enhance the performance of graph neural networks (GNN). 
% However, previous methods neglect to model the uncertainty of nodes in the structure learning process, which potentially undermines the performance of GSL models, as uncertainty indicates how reliable a node is for establishing connections.
% %we first theoretically analyze the impact of aggregation on node uncertainty
% % In this paper, we first theoretically analyze the relationship of predictive uncertainty in the neighborhood of a node and discover latent positive correlations between the node's entropy and that of its neighbors.
% In this paper, we first theoretically analyze the impact of aggregation on node uncertainty and discover latent positive correlations between a node's entropy and that of its neighbors.
% Building on this insight, we then propose the \textbf{U}ncertainty-aware \textbf{N}eighbor \textbf{L}earning (UnGSL) strategy, a lightweight plug-in that uses node-wise thresholds to identify the uncertainty levels of neighbors and adaptively refine the graph based on these uncertainties.
% UnGSL integrates seamlessly with GSL models, significantly improving prediction performance while introducing fewer edges and optimizable parameters.
% Experiments demonstrate that UnGSL consistently improves the prediction accuracy of state-of-the-art GSL models across 5 benchmark datasets, with an average increase of $2.18\%$.
\end{abstract}
%%
%% The code below is generated by the tool at http://dl.acm.org/ccs.cfm.
%% Please copy and paste the code instead of the example below.
%%
\begin{CCSXML}
<ccs2012>
   <concept>
       <concept_id>10010147.10010257.10010293.10010294</concept_id>
       <concept_desc>Computing methodologies~Neural networks</concept_desc>
       <concept_significance>500</concept_significance>
       </concept>
   <concept>
       <concept_id>10002951.10003227.10003351</concept_id>
       <concept_desc>Information systems~Data mining</concept_desc>
       <concept_significance>300</concept_significance>
       </concept>
 </ccs2012>
\end{CCSXML}

\ccsdesc[500]{Computing methodologies~Neural networks}
\ccsdesc{Information systems~Data mining}
% \ccsdesc[300]{Information systems~Data mining}

%%
%% Keywords. The author(s) should pick words that accurately describe
%% the work being presented. Separate the keywords with commas.
\keywords{Graph Structure Learning; Graph Neural Network; Uncertainty}
%% A "teaser" image appears between the author and affiliation
%% information and the body of the document, and typically spans the
%% page.
% \received{20 February 2007}
% \received[revised]{12 March 2009}
% \received[accepted]{5 June 2009}

%%
%% This command processes the author and affiliation and title
%% information and builds the first part of the formatted document.
\maketitle

\section{Introduction}
Graph neural networks (GNNs) \cite{kipf2017semisupervised,wu2019simplifying,gilmer2017neural} have demonstrated remarkable performance in tackling graph-structured data.
To date, GNNs have evolved with increasingly sophisticated model architectures \cite{dong2021equivalence,deng2024polynormer,liu2024scalable,wang2024distributionally,chen2024sigformer,chen2024macro} to enhance their capabilities.   
However, these model-centric methods often neglect potential flaws in the underlying graph structure, which can lead to suboptimal performance.  
In practice, graph data frequently exhibit suboptimal characteristics, such as noisy connections and incomplete information, due to the inherent complexities and inconsistencies in data collection \cite{zhou2023opengsl,um2023confidencebased}.


\begin{figure}[t]
    \setlength{\abovecaptionskip}{5pt}
    \subfigure[Cora]{
       \centering
       \includegraphics[width=0.45\linewidth,trim = 55 30 -10 10]{F1line_chartCora.pdf}
    }
    \subfigure[Citeseer]{
        \centering
        \includegraphics[width=0.45\linewidth,trim = 35 20 20 10]{F1line_chartCiteseer.pdf} 
    }
    % \vspace{-12pt}

    \caption{Performance varies with the ratios of eliminated neighbors on Cora and Citeseer datasets. Here we first constructed a graph based on GRCN \cite{yu2021graph}. We then eliminate a certain ratio of neighbors with the highest entropy for each node, and evaluate the performance of GCN on such a pruned graph. 
    {For comparison, we also report the performance under random elimination.}
    }

    %添加的去掉不相似结点的实验结果
    % the neighbors with highest uncertainty (evaluated by entropy cf xx) are eliminated.  
    % We trained the GCN model on graphs optimized by the GRCN model, then computed the Shannon entropy of each node based on the softmax output. 
    % Finally, we progressively eliminated edges connected to the highest-entropy neighbors for each node according to a fixed ratio.
    \setlength{\abovecaptionskip}{-3.cm}
    \label{fig1}
    \vspace{-15pt}
    \Description{}
\end{figure}



To address these issues, Graph Structure Learning (GSL) \cite{jin2020graph,chen2020iterative,fatemi2021slaps,zhang2021hierarchical,lu2024latent}, a data-centric approach, has garnered increasing attention. 
Beyond learning node representations with GNNs, GSL learns to refine node connections and edge weights. 
This approach has been shown to effectively enhance the accuracy of GNNs on downstream tasks while improving their resilience to topological perturbations \cite{zhou2023opengsl,li2023gslb}.   
Early work on GSL directly treated the graph structure (\ie the adjacency matrix) as learnable parameters. 
However, due to the large parameter space, these strategies often incur substantial computational overhead and are difficult to train effectively \cite{franceschi2019learning,jin2020graph}. 
More recently, research has shifted towards embedding-based GSL \cite{chen2020iterative,liu2022compact,wang2023prose,in2024self},  which constructs the adjacency matrix based on the similarity of node embeddings. 
Various similarity metrics, such as cosine similarity \cite{wang2023prose,in2024self} or neural networks \cite{liu2022compact}, have been employed. 
These methods aim to increase graph homophily and typically achieve state-of-the-art performance, as nodes with similar features (or embeddings) are more likely to be connected.

Despite their success, we identify two key limitations in these embedding-based GSL methods:
\begin{itemize} [leftmargin=1.0em]
    \item \textbf{These methods mainly rely on embedding similarity for graph construction while neglecting the quality of node information.} 
    Given the critical role of edges in GNNs as conduits for information propagation, it is essential to evaluate the quality of the information being propagated.  
    Aggregating unclear or ambiguous information from neighbor nodes can disrupt the embedding learning of the target node. 
    Constructing connections based solely on node similarity, without assessing the quality of the node’s information, may lead to suboptimal performance. 
    To empirically validate this point, we conduct a simple experiment using a representative GSL method (GRCN~\cite{yu2021graph}). 
    As shown in Figure \ref{fig1}, removing a certain proportion of neighbors with the highest entropy results in a significant performance gain. 
    \item \textbf{These methods often generate symmetric graph structures, which can hinder their effectiveness.} Current embedding-based methods tend to construct symmetric relationships between nodes, implying that both nodes exert an equal and bidirectional influence during the GNN learning process.  
    This imposed symmetry can constrain the model's flexibility and capacity, particularly when the connected nodes differ in quality. For example, consider a scenario where a high-quality node is linked to a low-quality node (refer to Figure \ref{fig2}).  While the high-quality node can provide valuable information that greatly benefits the low-quality node, the reverse influence from the low-quality node may have a negative impact on the high-quality node. 
    Symmetric relationships fail to account for this disparity, leading to a dilemma in the learning process.
    This inspires us to explore asymmetric structure learning.
    By modeling directional relationships separately, we allow the low-quality node to benefit from the high-quality node’s information while reducing the adverse influence in the opposite direction, thus protecting the high-quality node from negative effects. Although a few studies \cite{song2022towards,song2024optimal,lu2024latent} have begun to explore asymmetric graph structure learning, they typically restrict the asymmetry to relations between labeled and unlabeled nodes, overlooking the richer relationships between the vast majority of unlabeled nodes.
\end{itemize}

\begin{figure}[t]
% \setlength{\abovecaptionskip}{10pt}
\centering
\includegraphics[width=0.8\linewidth, trim = 70 20 70 -10]{Fig2.pdf}
\caption{Illustration of how our UnGSL differs from existing embedding-based GSL methods. Existing GSL learns symmetric relationships, leading to a dilemma when managing connections between high-quality and low-quality nodes.
% Specifically, assigning higher edge weights can cause high-quality nodes to be contaminated by low-quality information, while assigning lower edge weights prevents low-quality nodes from fully benefiting from high-quality information. 
In contrast, UnGSL learns asymmetric relationships, allowing low-quality nodes to benefit from high-quality nodes while mitigating the negative influence of low-quality nodes.}
% Illustration of existing embedding-based graph structure learning (GSL) Vs. our uncertainty-aware graph structure learning (UnGSL). 
% Embedding-based methods may create symmetric edges between nodes of varying quality, which can negatively affect high-quality nodes.
% In contrast, quality-aware method enables low-quality nodes to benefit from the information of high-quality nodes, while mitigating the adverse effects in the reverse direction by modeling directional relationships separately.

\label{fig2}
\vspace{-15pt}
\Description{}
\end{figure}

To overcome these limitations, we propose an uncertainty-aware graph structure learning (UnGSL) method that considers nodes' information quality to learn an asymmetric graph structure. 
UnGSL directly utilizes the uncertainty (Shannon Entropy \cite{abdar2021review}) of the node in classification to indicate the node's information quality, and conducts theoretical analyses to demonstrate that blindly aggregating information from high-uncertainty nodes would lift the lower bound of uncertainty for the target node.
Building on this, UnGSL leverages a learnable node-wise threshold to differentiate low-quality neighbors from high-quality ones, and adaptively reduces directional edge weights from those low-quality neighbors. 
Notably, our UnGSL is simple and can be easily incorporated into various embedding-based GSL methods, boosting their performance with minor extra computational overhead.
% In our experiments, we implement UnGSL with six representative GSL methods, consistently achieving improvements across five conventional graph datasets, with an average performance increase of 2.18\%.  Additional experiments, including analyses across various GNN architectures and robustness evaluations, further demonstrate that integrating UnGSL enhances model generalization and increases robustness to structural perturbations.

In summary, this work makes the following contributions:
\begin{itemize} [leftmargin=1.0em]
\item We highlight the necessity of modeling a node's uncertainty in graph structure learning, and theoretically demonstrate that the uncertainty of a node after GNN layer is positively correlated with those of its neighbors.
\item We propose a simple yet novel uncertainty-aware graph structure learning strategy (UnGSL), which can be seamlessly integrated with various embedding-based GSL models to mitigate the directional impact of high-uncertainty nodes.
\item We conduct extensive experiments to demonstrate that UnGSL can consistently boost existing embedding-based GSL models across seven benchmark datasets, with an average performance increase of 2.07\%.
\end{itemize}



\section{Preliminaries}
\label{sec2}
In this section, we introduce  basic notations and  background on GNNs and GSL methods.

Consider the graph $\mathcal{G}=(\mathcal{V}, \mathcal{E}, \mathbf{A}, \mathbf{X})$, where $\mathcal{V}$ represents a set of $n$ nodes $\{v_1,...,v_n\}$ and $\mathcal{E}$ represents the set of edges. Let $\mathbf{A}$ be the initial adjacency matrix of the graph, where $\mathbf{A}_{ij}=1$ if an edge exists between node $v_i$ and $v_j$; otherwise, $\mathbf{A}_{ij}=0$. The matrix $\mathbf{X}=[x_1,...,x_n]\in \mathbb{R}^{n \times d}$ represents the node feature matrix, where each column $x_i$ corresponds to the feature vector of node $v_i$. Let 
$\mathbf{D}$ denote the diagonal degree matrix defined as $\mathbf{D}_{ii}=1+\sum_{j}\mathbf{A}_{ij}$; and 
$\hat{\mathbf{A}}$ denotes the normalized adjacency matrix with self-loop, \ie $\hat{\mathbf{A}}=\mathbf{D}^{-\frac{1}{2}} (\mathbf{A}+\mathbf{I}) \mathbf{D}^{-\frac{1}{2}}$ for symmetric normalization or $\hat{\mathbf{A}}=\mathbf{D}^{-1} (\mathbf{A}+\mathbf{I})$ for row normalization. 
% that formulating an initial noisy adjacency matrix $\mathbf{A}\in \mathbb{R}^{n\times n}$, where $\mathbf{A}_{ij}=1$ if an edge exists between node $v_i$ and $v_j$, otherwise $\mathbf{A}_{ij}=0$.
% $\mathbf{X}=[x_1,...,x_n]\in \mathbb{R}^{n \times d}$ is the initial node feature matrix with dimension $d$. 
% Specifically, we use $\mathbf{U} \in \mathbb{R}^{n \times 1}$ to represent the uncertainty matrix of all nodes. 
\subsection{Graph Neural Networks}
Graph Neural Networks (GNNs)  have become a prominent approach for learning from graph-structured data, where node representations are learned by iteratively aggregating and transforming information from neighboring nodes. 
In recent years, various designs for the aggregation and transformation processes have given rise to different GNN models\cite{kipf2017semisupervised,veličković2018graph,wu2019simplifying}. 
Among these, the Graph Convolutional Network (GCN) \cite{kipf2017semisupervised} stands out as one of the most widely adopted and influential architectures. 
The operation at the $l$-th layer in GCN can be formulated as:
\begin{equation}
    \mathbf{Z}^{(l)}=\sigma(\hat{\mathbf{A}}\mathbf{Z}^{(l-1)}\mathbf{W}^{(l)}),
\end{equation}
where $\sigma( \cdot )$ denotes the activation function, $\mathbf{W}^{(l)}$ is a learnable parameter matrix used to transform the node features. 

Given the critical role of the graph structure in GNNs, which determines the sources of information aggregation, ensuring the quality of graph structure is of paramount importance. Recent work demonstrates that suboptimal graph structures,  even with the introduction of a small percentage of noisy edges or topological perturbations (\eg 10\%), can significantly degrade the performance of GNNs (\eg 25\%) \cite{zhou2023opengsl,li2023gslb}.
% Given the critical role of the graph structure in GNNs, which determines the sources of information aggregation, ensuring the quality of graph structure is of paramount importance.    加一句话说明一下,这个重要性,这里用个例子吧,“various studies provide empirical evidences, e.g., suboptimal graph structure (e.g., 加一些扰动或者噪音)会带来显著的性能下降 (over XX\%)”。  

% Graph Neural Networks (GNNs), designed for learning features from graph-structured data, have been widely utilized in graph structure learning.
% Graph Convolutional Network (GCN) \cite{kipf2017semisupervised} is one of the most prevalent GNN model.
% Generally, the $l$-th layer in GCN can be formulated as:
% \begin{equation}
%     \mathbf{Z}^{(l)}=\text{ReLU}(\hat{\mathbf{A}}\mathbf{Z}^{(l-1)}\mathbf{W}^{(l)})
% \end{equation}
% % Generally, a two-layer GCN model with $\theta=(\mathbf{W}^{(0)},\mathbf{W}^{(1)})$ can be formulated as:
% % \begin{equation}
% %     f_{\theta}(\mathbf{A,X}) = \text{softmax}(\hat{\mathbf{A}}
% %     \ \sigma(\hat{\mathbf{A}}\mathbf{X}\mathbf{W}^{(0)})\mathbf{W}^{(1)})
% % \end{equation}
% where ReLU$(\cdot)$ is an activation function \cite{nair2010rectified}.
% $\hat{\mathbf{A}}$ is the normalized adjacency matrix with self-loop, such as the symmetric normalized adjacency  $\hat{\mathbf{A}}=\mathbf{D}^{-\frac{1}{2}} (\mathbf{A}+\mathbf{I}) \mathbf{D}^{-\frac{1}{2}}$ and the row normalized adjacency matrix $\hat{\mathbf{A}}=\mathbf{D}^{-1} (\mathbf{A}+\mathbf{I})$.
% $\mathbf{D}$ is a diagonal degree matrix defined as $\mathbf{D}_{ii}=1+\sum_{j}\mathbf{A}_{ij}.$
\subsection{Graph Structure Learning}
% Graph structure learning (GSL) aims to learn an optimal graph $\mathbf{S}$ by recovering valuable edges and removing noisy edges in the original graph $\mathbf{A}$ .
Graph Structure Learning (GSL) aims to enhance the accuracy and robustness of GNNs by learning a refined adjacency matrix $\mathbf{S}$, where $\mathbf{S}_{ij}$ denotes the edge weight between node $v_{i}$ and node $v_{j}$. 
For convenience, we also use $\mathbf{S}_{ij}$ to indicate whether there exists a edge between node $v_i$ and $v_j$.
% we also use $\mathbf{S}_{ij}$ to represent an edge $e_{ij}$. 
When $\mathbf{S}_{ij}>0$, it indicates the presence of a directed edge from $v_{j}$ to $v_{i}$
, with the edge weight equal to $\mathbf{S}_{ij}$. 
Conversely,  $\mathbf{S}_{ij}=0$ indicates that no edge exists.

The quality of the learned adjacency matrix is often examined on the downstream tasks. 
Conventionally, $\mathbf{S}$ is fed into a GNN encoder to obtain node representations, which are used to evaluate performance on downstream tasks.
The objective function of most GSL methods can be generally formulated as:
\begin{equation}
    \mathcal{L} =\mathcal{L}_{{Task}}(\mathbf{Z},\mathbf{Y})+\lambda\mathcal{L}_{{Reg}}(\mathbf{Z},  
    \mathbf{S}),
    \label{eq_summary_gsl}
\end{equation}
where $\mathbf{Z}=\text{GNN}(\mathbf{S},\mathbf{X})$ is the node embedding matrix, $\mathcal{L}_{\text{Task}}$ aims to utilize the labels $\mathbf{Y}$ as a supervised signal to optimize both the GNN encoder and the adjacency matrix $\mathbf{S}$.
The regularization term $\mathcal{L}_{\text{Reg}}$ are introduced by existing GSL methods to achieve diverse desirable properties in the learned adjacency matrix $\mathbf{S}$, such as sparsity \cite{jin2020graph}, feature smoothness \cite{song2024optimal}, and connectivity \cite{yu2021graph} in the graph structure.
For example, PROGNN \cite{jin2020graph} employs $l_{1}$ norm penalization on the learned adjacency matrix to enhance the sparsity of the graph structure.
% introduces the sparsity loss term ${||\mathbf{A}||}_{1}$,  which targets using $l_{1}$ norm to prevent the graph structure from becoming too dense.
$\lambda$ is a trade-off hyperparameter.

Traditional GSL methods \cite{franceschi2019learning,jin2020graph} that treat each edge $\mathbf{S}_{ij}$ as a learnable parameter often suffer from significant computational overhead and are challenging to train efficiently.
Recent research has focused on embedding-based GSL methods \cite{chen2020iterative,liu2022compact,wang2023prose,in2024self} , which employ an auxiliary GNN encoder to extract node embeddings from the initial graph and estimate embedding similarities as edge weights:
\begin{equation}
\label{embed}
    \mathbf{E} = \text{GNN}(\mathbf{A},\mathbf{X}),
\end{equation}
\begin{equation}
\label{sim}
    \mathbf{S}_{ij} = \mathbf{S}_{ji} = \phi(\mathbf{E}_{i},\mathbf{E}_j).
\end{equation}
Here $\mathbf{E}$ is the node embedding matrix,  which differs from $\mathbf{Z}$ as it is specifically used for constructing the graph structure.
Note that existing works generally utilize different GNN architectures and input structure compared with those used to compute the embedding $\mathbf{Z}$ when constructing the embedding $\mathbf{E}$.
$\phi(\cdot)$ is a metric function (\ie cosine similarity) used to calculate the similarity between nodes.

Although the structure modeling paradigm in Equation (\ref{sim}) is widely adopted in GSL models, it suffers from two key limitations:
\begin{itemize}[leftmargin=1.0em] 
    \item  \textbf{This paradigm relies on embedding similarity while neglecting the quality of node information.}
    {Only semantic similarities between embeddings $\mathbf{E}_i$  and $\mathbf{E}_j$ are considered in this structure modeling paradigm, while the varying uncertainties of them, which reflect their information quality, are neglected.
    This may undermine the quality of embeddings of target nodes when aggregating inferior information from low-quality neighbors (as validated by the preliminary experiment in the Introduction) .}    
    \item \textbf{This paradigm constraining the graph to be symmetric, which potentially hinder effectiveness of GSL models. }
    The pair-wise similarity  constrains the learned edge between $v_i$ and $v_j$ to be bidirectional, overlooking their unequal influence due to varying information quality.
    For example, if $\mathbf{E}_i$ contains higher-quality information than $\mathbf{E}_j$, the constructed edge $\mathbf{S}_{ji}$ can provide valuable information that greatly benefits the low-quality node $v_j$ while the edge $\mathbf{S}_{ij}$ propagates inferior information to poison the embedding of $v_i$.
    By modeling directional relationships separately, we enable the low-quality node to benefit from the high-quality node's information while mitigating the negative impact in the reverse direction.
    % \textcolor{red}{Intuitively, it is more reasonable to propagate positive information from reliable nodes to uncertain ones and meanwhile avoid negative impacts in the opposite direction.}
\end{itemize}

Although a few works \cite{song2022towards,song2024optimal,lu2024latent} have been proposed to learn asymmetric graphs by generating directed edges $\mathbf{S}_{ij}$ from the labeled node $v_j$ to unlabeled node $v_i$ and constraining $\mathbf{S}_{ji}=0$, which facilitates the propagation of label information and avoids introducing inconsistency to labeled nodes.
However, these methods completely rely  on annotated labels and fail to learn reasonable asymmetric connections between the vast majority of unlabeled nodes.
% , especially in practical scenarios with limited labels. Moreover, they cannot generalize to unsupervised GSL scenarios.

Given the flaws of existing methods, we argue for the necessity of incorporating node uncertainty into graph structure learning to learn an optimal asymmetric structure.
We propose the uncertainty-aware graph structure learning (UnGSL) method to enhance GSL models, which leverage learnable node-wise thresholds to identify high-uncertainty neighbors and adaptively reduce directional edge weights from those low-quality neighbors.
\section{Methodology}
\label{sec3}
{In this section, we first conduct theoretical and empirical analyses to demonstrate the detrimental impact of neighbors with high uncertainty levels on GNN learning (Subsection~\ref{sec3.1}). 
We then present the proposed uncertainty-aware graph structure learning method in detail (Subsection ~\ref{sec3.2}).}

\begin{figure}[t]

       \centering
       \includegraphics[width=0.8\linewidth,trim = 35 30 -10 -5]{figure_node_neighbor_ent_cora.pdf}



    \caption{Visualization of node entropy after GNN aggregation  (i.e., $u_{i}$) alongside the average entropy of its neighbors  (i.e., $\sum_{v_j \in \mathcal{N}{(v_i)}}\mathbf{\hat{A}}_{ij} u_{j}^{\prime}$) on Cora dataset.\label{Figure2}}
    \setlength{\belowcaptionskip}{-15.cm}
    \vspace{-15pt}
    \Description{}
\end{figure}

\subsection{Analyses on the Impact of Neighbor Uncertainty}
\label{sec3.1}
Aggregating unclear or ambiguous information can intuitively disrupt the learning process of target nodes, thereby negatively affecting the performance of Graph Neural Networks (GNNs). In this section, we aim to conduct both theoretical and empirical analyses to substantiate this claim. To begin, we introduce several formal concepts to facilitate these analyses.

\textbf{Semi-supervised Node Classification Task.} For convenience, we refer to recent analytical work on GNNs \cite{ma2022is} and focus our theoretical analysis on the semi-supervised node classification task, which is the most common and widely studied scenario. Nevertheless, at the end of this section, we will also discuss how our method can be adapted to the unsupervised learning scenario.  
Following the definitions in \cite{ma2022is}, we consider a $K$-class classification problem and employ a linear classification model. 
Formally, the classification logits can be expressed as:
\begin{equation}
    \mathbf{O}=\mathbf{D^{-1}(A+I)X}\mathbf{W},
\end{equation}
where $\mathbf{W} \in \mathbb{R}^{d \times K}$ is the linear classification matrix.
Assume the logit in matrix $\mathbf{O}$ are bounded by the scalar 1, \ie max $|\mathbf{O}_{ij}| < 1$.
From a local perspective for node $v_i$, its probabilities can be formulated as :
\begin{equation}
    p_{i} = \frac{\mathbf{O}_{i} +  \mathbf{1}_{K} }
    {\sum_{j=1}^{K}(\mathbf{O}_{ij} + 1)},
\label{eq3}
\end{equation}
where $\mathbf{1}_{K}$ is a $K$-dimensional all-ones vector.
Here we simply omit the nonlinear exponential function in the softmax, given that our primary focus is on the impact of aggregation on node uncertainty and the exponential function mainly serves to generate positive values.  
This simplification has been adopted in prior work \cite{zhao2020uncertainty,yu2023uncertainty}, and it can also be considered a first-order Taylor approximation.

% \begin{figure}[t]
%     \centering
%     \includegraphics[width=1.0\linewidth, trim = 10 120 10 41]{Framework.pdf}
%     \caption{\textcolor{blue}{Overview of the proposed UnGSL strategy. We begin by selecting a specific GSL method that UnGSL aims to enhance and pretrain it to obtain the classifier. Next, the chosen GSL method is re-trained, during which Equation \ref{eq10} is employed to generate the adjacency matrix. Finally, the learned graph structure is leveraged for various downstream applications.}}
%     \label{fig_Overview}
%     \vspace{-15pt}
% \end{figure}

\textbf{Uncertainty Estimation via Entropy.}
Entropy measures the information uncertainty within a probability distribution \cite{abdar2021review}.
For node classification tasks, the entropy of the classification probabilities reflects the classifier's certainty in assigning the node representation to a specific class.
A high-entropy node indicates that its representation carries significant uncertainty, making it challenging for the classifier to reach a confident decision. Aggregating information from such nodes can poison the target node's representation, hindering the generation of accurate predictions.
Given probabilities $p_i$ of node $v_i$, its entropy is defined as:
\begin{equation}
\label{eq7}
    u_{i} = - \sum_{k=1}^{K} p_{ik} \text{log}(p_{ik}).  
\end{equation}
We further discuss  the uncertainty metric for unsupervised learning scenario at the end of this section.

Formally, to demonstrate the impact of the uncertainty of neighbors along GNNs, we have the following proposition, with detailed proof provided in Appendix \ref{proof}: 
\begin{proposition}
\label{prop1}
Define the logits of the initial node feature matrix as $\mathbf{O}^{\prime} = \mathbf{X}\mathbf{W}$.
For a given node $v_i$, let $u_i$ denote its entropy after GNN aggregation. 
$\forall v_j \in \mathcal{N}{(v_i)}$, let $u_j^{\prime}$ denote the entropy of its initial features.
Then the entropy of $v_i$ and the entropy of  $v_j \in \mathcal{N}{(v_i)} $ satisfy the following inequality:
\begin{equation}
     u_i \geq \sum_{v_j \in \mathcal{N}{(v_i)}}\eta_{j} u_{j}^{\prime},
\end{equation}
where 
\begin{equation}
\eta_{j}=\frac{
\sum_{k=1}^{K}\hat{\mathbf{A}}_{ij}{(O_{jk}^{\prime}+1})}
{ \sum_{v_j \in \mathcal{N}{(v_i)}}
\sum_{k=1}^{K}\hat{\mathbf{A}}_{ij}(O_{jk}^{\prime} + 1) }.
\end{equation}
For coefficient $\eta_{j}$, we have $ 0<\eta_{j} <1$ and $\sum_{v_j \in \mathcal{N}{(v_i)}}\eta_{j} =1$.
\label{proposition 1}
\end{proposition}
 
\textbf{Discussion.}
Proposition \ref{prop1} establishes the bounded relationship between the uncertainty of a targe node after aggregation and the uncertainty of its neighbors. 
It suggests blindly aggregating information from high-uncertainty nodes would lift the lower bound of the uncertainty for the target node. 
Removing or mitigating the impact of these high-uncertainty nodes, may significantly decrease the uncertainty of the target node, leading to performance gains.

\textbf{Empirical Analyses.}
We conduct simple experiments
to empirically demonstrate that aggregating neighbors with higher uncertainty would result in a high-uncertainty node.
Specifically, we train the model with 1-layer GCN and linear classifier on given datasets. Then we visualize the entropy of a node after aggregation (\ie $u_{i}$) alongside the average entropy of its neighbors (\ie $\sum_{v_j \in \mathcal{N}{(v_i)}}\mathbf{\hat{A}}_{ij} u_{j}^{\prime}$).
As shown in Figure \ref{Figure2}, we observe a strong positive correlation between the entropy of a node after aggregation and the average entropy of its neighbors, which substantiates our proposition (results on more datasets are provided in Appendix \ref{ADDC1}).
%添加异质数据集的分布结果

The above analyses clearly illustrates the impact of node uncertainty in aggregation process in GNNs. 
Blindly connecting and aggregating information from nodes with high uncertainty may undermine the performance of the nodes themselves. 
It is therefore important to consider node uncertainty in graph structure learning to learn a reasonable asymmetric structure.
Specifically,
one can prevent a node falling into a high-uncertainty region by weakening its connections to neighbors with high uncertainty. 
Meanwhile, it can receive more stable information via strengthened connections with low-uncertainty nodes.

\subsection{Uncertainty-aware Graph Structure Learning}
\label{sec3.2}
Given the importance of considering uncertainty in graph structure learning, we propose the simple yet novel uncertainty-aware graph structure learning (UnGSL) method that leverages learnable node-wise thresholds to distinguish low-quality neighbors from high-quality ones and adaptively refines edges based on their uncertainty levels.
% Figure \ref{fig_Overview} illustrates the framework of UnGSL.
Beyond similarity, we introduce uncertainty to guide the learning of the refined adjacency matrix.
Specifically, the uncertainty estimated through entropy is transformed into confidence scores.
% we first pretrain the GSL model to get the classifier and then estimate the uncertainty using the entropy (Equation (\ref{eq7})). 
% Note that this strategy is not applicable to the unsupervised GSL methods.
% We have also discussed uncertainty estimation for these methods at the end of this section, where we suggest using the node contrastive learning loss as a proxy for uncertainty.
% This stage specifically focuses on uncertainty estimation, while both the GNNs and the graph structure are re-trained by UnGSL.
% Afterwards, UnGSL normalizes these entropy into confidence scores, which are used to construct a node confidence matrix. 
Afterwards, UnGSL applies learnable node-wise thresholds to split neighbors with different levels of confidence into two groups (\ie high-confidence and low confidence) for each node. 
Finally, it amplifies edge weights from confident neighbors while reducing edge weights from uncertain ones.
In summary, we refine the generation of adjacent matrix of existing GSL from Equation (\ref{sim}) into Equation (\ref{eq10}):
% \textcolor{blue}{
% \begin{equation}
% \label{eq10}
%     \hat{\mathbf{S}} = \mathbf{S} \odot \psi({\mathbf{C}-{\boldsymbol{\varepsilon}}^{\top} \cdot \mathbf{1}_{n} } ),
% \end{equation}
% where
% \begin{equation}
% \label{eq9}
% \mathbf{C}={\mathbf{1}_{n}^{\top}} \cdot e^\mathbf{-U}
% \end{equation}
% }
{
\begin{equation}
\label{eq10}
    \hat{\mathbf{S}}_{ij} = \mathbf{S}_{ij} \cdot \psi({e^{-u_{j}} -{{\varepsilon}_{i}}} ),
\end{equation}
}
Here $u_{j}$ is entropy of node $v_{j}$ estimated during pretraining stage, $\varepsilon_{i}$ is the learnable node-wise threshold of node $v_{i}$, 
$\psi(\cdot)$ is an activate function,
$\mathbf{S}$ is the generated adjacency matrix of the GSL model.

% Here Equation (\ref{eq9}) aims to normalize node uncertainty into confidence within the interval $(0,1)$.
Equation (\ref{eq10}) first normalizes node uncertainty into confidence within the interval $(0,1)$ , and then leverages node-wise learnable thresholds $\boldsymbol{\varepsilon}$ to distinguish low-confidence neighbors (\ie ${e}^{-u_{j}} < {\varepsilon_{i}}$) from high-confidence ones (\ie ${e}^{-u_{j}} \geq {\varepsilon_{i}}$ ) and adaptively refines the corresponding edges using the following activation function:
\begin{equation}
\label{eq11}
    \psi(x) = \left\{
    \begin{array}{lr}
         \tau \cdot s(x),  &{x \geq 0,}\\
        {\beta}, &{x < 0,}
    \end{array} 
    \right.
\end{equation}
where $s(\cdot)$ is the sigmoid function, $\tau$ is a hyperparameter amplifies the edge weights from high-confidence neighbors, $\beta$ is a hyperparameter that controls the reduction of edge weights from low-confidence neighbors.

For the training of UnGSL, we employ the same loss function (\ie Equation (\ref{eq_summary_gsl})) and the same training procedure as the original GSL model, except we modify the generation of the adjacency matrix from Equation (\ref{sim}) to Equation (\ref{eq10}), \ie using 
$\hat{\mathbf{S}}$ to replace the generated adjacency matrix $\mathbf{S}$ in existing GSL methods. 
Therefore, UnGSL can be seamlessly integrated into existing GSL methods with minimal additional computational cost.

\textbf{Implementation Details.}
We begin by selecting a specific GSL method that UnGSL aims to enhance , pretraining it to obtain the classifier and estimating uncertainty using entropy (Equation (\ref{eq7})).
Next, the chosen GSL method is re-trained, during which Equation (\ref{eq10}) is employed to generate the adjacency matrix. 
Finally, the learned graph structure is leveraged for downstream applications.
\subsection{Discussion}
Based on the above formulation, we next discuss several key advantages of UnGSL:

\textbf{Uncertainty-aware.}
UnGSL considers the information quality of nodes during the structure modeling process, facilitating the learning of an optimal graph that mitigates the negative impact of high-uncertainty neighbors in GNN learning.

\textbf{Model-agnostic.}
UnGSL refines the learned adjacency matrix $\mathbf{S}$ by utilizing uncertainty-aware weights. 
In essence, UnGSL just replaces the calculation of the adjacency matrix from Equation (\ref{sim}) to Equation (\ref{eq10}) for GSL methods. 
Therefore, UnGSL can be seamlessly integrated into existing GSL methods to further enhance their ability to learn graphs, improving the performance of GNNs on downstream tasks. 
Comprehensive experiments supporting this can be found in Subsection~\ref{RQ1} and Subsection \ref{RQ3}.

\textbf{Asymmetric Graph Structure.}
UnGSL proposes learning an
asymmetric graph, where the edge weights between nodes differ based on their uncertainty levels.
Specifically, UnGSL weakens edges from uncertain nodes to confident nodes, mitigating the impact of inferior information. Meanwhile it enhances the edge in the opposite direction to improve the representations of uncertain nodes.

Notably, several methods that construct directed edges from labeled nodes to unlabeled nodes~\cite{song2024optimal,lu2024latent} can be viewed as specific cases of UnGSL, as the embeddings of labeled nodes are directly optimized during training and therefore more likely to exhibit lower uncertainty.
In contrast, UnGSL models the asymmetric relationships among the predominantly unlabeled nodes, leading to superior performance. We further empirically validate this in Subsection~\ref{RQ1}.

\textbf{Efficiency.}
UnGSL contains only $n$ learnable parameters and refines the existing edges of the given graph without generating new ones. 
The additional operations introduced by UnGSL have a complexity of $O(n+m)$, where $n$ and $m$ denote the number of nodes and edges in the graph.
Therefore, UnGSL imposes minimal computational costs on the base GSL models.
We provide empirical evidence to demonstrate the efficiency of UnGSL in Subsection~\ref{seceffiency}.

\textbf{Adaptive to Unsupervised Scenarios.}
Despite the mainstream focus of GSL research on supervised learning, our approach can be generalized to unsupervised GSL scenarios. The main challenge lies in determining an appropriate uncertainty metric for unsupervised GSL method.
We suggest employing the self-supervised structure learning loss 
(\ie node contrastive learning loss \cite{liu2022towards}) as a proxy for uncertainty, which can be formulated as:
\begin{equation}
\label{eq13}
    u_{i} = \frac{1}{2}(l(z_{i},{\tilde{z}}_{i})+ l({\tilde{z}}_{i},z_{i}),
\end{equation}
where 
\begin{equation}
l(z_{i},{\tilde{z}}_{i}) = - \log\frac{e^{\text{sim}(\mathrm{z}_{i},\mathrm{\tilde{z}}_{i})/t}}{\sum_{k=1}^n e^{\text{sim}(\mathrm{z}_{i},\mathrm{\tilde{z}}_{k})/t}}.
\end{equation}
Here $\text{sim}(\cdot)$ is the cosine similarity function.
$\mathrm{z}_{i}$ and $\mathrm{\tilde{z}}_{i}$ are the embeddings of node $v_{i}$ learned by GNN from target graph and augmented graph, respectively. 
$\sum_{k=1}^n e^{\text{sim}(\mathrm{z}_{i},\mathrm{\tilde{z}}_{k})/t}$ is cumulative similarity between $z_{i}$ and embeddings of different nodes in the augmented graph. 
$t$ is the temperature parameter.
The GCL loss can measure the invariance of node representations to feature or structure perturbations, where this invariance can be interpreted as the uncertainty of nodes with respect to their original features and structure.


% \subsubsection{Uncertainty Besides Entropy}
% \label{more_uncertainty}
% Entropy is available in supervised GSL models, as it is derived from the probability distribution that is optimized using label information.
% To incorporate the UnGSL strategy into unsupervised graph structure learning models \cite{liu2022towards}, we suggest employing the self-supervised structure learning loss (i.e., contrastive learning loss) as uncertainty.
% The contrastive learning (CL) loss typically reflects the robustness of node embeddings to topological perturbations.
% Nodes connected to neighbors with high CL loss tend to exhibit reduced robustness to these perturbations after aggregation.
% With the UnGSL strategy, the unsupervised GSL model can identify neighbors with high CL loss in the learned graph and weaken their connections to mitigate the associated negative impact.


\section{Experiments}
\label{sec4}
In this section, we conduct experiments to evaluate the effectiveness of the proposed UnGSL strategy. 
Our experiments aim to answer the following research questions:
\begin{itemize}[leftmargin=1.0em]
    \item \textbf{RQ1:} Does the integration of UnGSL into existing GSL methods lead to performance improvements?
    \item \textbf{RQ2:} What is the impact of key configurations (\eg node-wise thresholds, asymmetric graphs, hyperparameters $\beta$ and $\tau$) on UnGSL performance? 
    \item  \textbf{RQ3:} Can UnGSL enhance the robustness of GNNs against structural noise, feature noise and lable noise? 
    \item \textbf{RQ4:} How well does UnGSL generalize across different GNN backbones? 
    \item \textbf{RQ5:} Does UnGSL introduce significant additional computational overhead? 
\end{itemize}
\begin{table*}[ht]
%\begin{minipage}[t]{\linewidth}
\renewcommand{\arraystretch}{1.20} % 可以调节, 1.2指高度是默认的1.2倍
\centering
\setlength{\abovecaptionskip}{0.cm}
\caption{Node classification accuracy±std comparsion(\%). 
Each experiment is repeated 10 times with different random seeds.
"OOM" denotes out of memory.
The top-performing results are marked in \textbf{bold}.}
\label{Tab:performance}
\setlength{\tabcolsep}{0.015\textwidth}{
% \begin{tabular}{l|lllll|lll}
{
\begin{tabular}{cccccccc}
\hline
        Model & Cora & Citeseer & Pubmed & BlogCatalog & Roman-empire & Flickr & Ogbn-arxiv \\ \hline
      
        GRCN & 84.70±0.31 & 72.49±0.77 & 78.94±0.16 & 76.17±0.23 & 44.29±0.28 & 59.55±0.31 & OOM  \\ 
        
        GRCN+UnGSL & \textbf{85.84±0.51} & \textbf{73.88±0.55} & \textbf{79.59±0.48} & 
        \textbf{76.78±0.11} & 
        \textbf{52.54±0.31} &\textbf{63.85±0.22} & OOM \\ 
        \hline

        PROGNN & 80.39±0.41 & 67.94±0.52 & OOM & 76.17±0.22 & OOM & 61.74±0.24 & OOM \\
        
        PROGNN+UnGSL & \textbf{81.86±0.55} & 
        \textbf{69.66±0.27} & 
        OOM & 
        \textbf{76.82±0.19} & 
        OOM &\textbf{61.92±0.43} & OOM \\ 
        \hline
        % SUBLIME & 82.50±0.6 & 71.56±0.17 & 80.41±0.69 & 93.39±0.24 & 63.48±0.53  \\
        
        % SUBLIME+UnGSL & \textbf{83.41±0.63} & 
        % \textbf{72.88±0.94} & 
        % \textbf{81.39±1.23} & \textbf{94.22±0.37} & 
        % \textbf{64.82±0.18}  \\ 
        %         \hline
        PROSE & {81.1±0.45} & {72.3±0.37} & 83.3±0.71 & {75.31±0.17} & 55.61±0.34  &
        59.77±1.13
        & 71.22±0.31\\ 
        
        PROSE+UnGSL & \textbf{81.90±0.31} & \textbf{73.10 ±0.32} & 
        \textbf{83.86±0.30} & \textbf{75.77±0.36} & 
        \textbf{56.17±0.31} 
        &
        \textbf{64.72±0.97}
        &
        \textbf{71.46±0.12}
        \\
        \hline
        IDGL & 84.50±0.5 & {72.49±0.67} & {82.83±0.33} & {89.66±0.28} & {46.67±0.56} & 85.77±0.16 & 70.45 ± 0.36 \\ 

        IDGL+UnGSL & \textbf{84.90±0.42} & \textbf{73.74±1.01} & \textbf{83.33±0.32} & \textbf{92.13±0.18} & \textbf{47.05±0.59} & \textbf{86.42±0.35} &\textbf{71.02±0.12}\\ 
        \hline
        
        SLAPS & 72.89±1.02 & {70.05±0.83} & {70.96±0.99} & {91.62±0.39} & {65.35±0.45} & 83.89±0.70 &  OOM \\

        SLAPS+UnGSL & \textbf{74.28±0.95} & \textbf{72.08±0.94} & \textbf{72.25±1.48} & \textbf{91.89±0.41} & \textbf{66.20±0.33} & \textbf{84.92±0.68} &OOM \\
        \hline

        SUBLIME & 83.30±1.04 & 72.36±0.68 & 80.41±0.69 & 95.20±0.21 & 63.48±0.53 & 88.68±0.23 & 71.37±0.13 \\
        
        SUBLIME+UnGSL & \textbf{84.24±0.91} & 
        \textbf{74.34±0.73} & 
        \textbf{80.84±0.92} & \textbf{96.18±0.38} & 
        \textbf{65.45±0.32} & \textbf{89.23±0.20} & \textbf{71.82±0.15}  \\ 
                \hline
        % Avg. Improve(\%) & 1.44\% & {2.34\%} & {0.87\%} & {1.24\%} & {5.00\%}  \\ 
        % \hline
\end{tabular}}}
%\end{minipage}%
\vspace{0.pt}
\end{table*}
\begin{table}[!t]
\renewcommand{\arraystretch}{1.2}
\vspace{0pt}
\centering
\setlength{\abovecaptionskip}{0.cm}
\caption{Performance comparison between the CUR decomposition and UnGSL modules.}
\label{Tab:CUR Vs UnGSL}
\setlength{\tabcolsep}{0.005\textwidth}{
% \begin{tabular}{l|lllll|lll}
{
\begin{tabular}{cccccc}
\hline
        Model & Cora & Citeseer & Roman-empire \\ \hline
      
        GRCN+CUR & 84.89±0.22 & 73.35±0.46  & 44.04±0.28 \\ 
        
        GRCN+UnGSL & \textbf{85.84±0.51} & \textbf{73.88±0.55}  & 
        \textbf{52.54±0.31}  \\ 
        \hline

        IDGL+CUR & 84.73±0.23 & 72.93±0.85  & OOM \\ 
        
        IDGL+UnGSL & \textbf{84.90±0.42} & \textbf{73.74±1.01} & 
        \textbf{47.05±0.59}  \\ 
        \hline
\vspace{-20pt}
\end{tabular}}}
\end{table}
\subsection{Experiment Settings}
\subsubsection{Datasets.}
To comprehensively evaluate UnGSL's performance on node classification, we follow previous works~\cite{zhou2023opengsl, yu2021graph} and select $7$ commonly used datasets, including four homophilous citation datasets~\cite{sen2008collective,OGB} (Cora, Citeseer, Pubmed and Ogbn-arxiv) and three heterophilous datasets (Blogcatalog~\cite{huang2017label} , Roman-Empire~\cite{platonov2023critical} and Flickr \cite{huang2017label}). 
The chosen datasets cover a wide range of homophily levels and graph sizes, allowing us to demonstrate UnGSL's effectiveness under various conditions. 
For a fair comparison, we strictly follow the data split settings used in the newly proposed benchmark for GSL~\cite{zhou2023opengsl}. 
Detailed statistics of these datasets are provided in Appendix~\ref{dataset}.

\subsubsection{Baselines.} To demonstrate UnGSL's generalizability across different GSL models, we select $6$ state-of-the-art GSL algorithms corresponding to the chosen datasets as baselines,
including supervised models (GRCN \cite{wang2021graph}, PROGNN \cite{jin2020graph}, IDGL \cite{chen2020iterative}, PROSE \cite{wang2023prose}, SLAPS \cite{fatemi2021slaps}), and a self-supervised model (SUBLIME \cite{liu2022towards}).
Detailed information of these baselines is provided in Appendix \ref{baselines}.
For all models, we report the average performance and standard deviations of 10 runs with different random seeds.

\subsubsection{Configuration.}
With regard to hyperparameter settings, only the hyperparameter $\beta$ and the learning rate of the threshold $\varepsilon$ require fine-tuning.
The threshold $\varepsilon$ is consistently initialized as a random number in the range [0, 1] across various datasets. 
The hyperparameter $\tau$ is simply fixed as a constant value of 2. 
The hyperparameter $\beta$ 
can be quickly tuned using parameter search tools like Bayesian optimization within the range of [0.001,1].
We optimized the UnGSL using Adam optimizer, with the learning rate selected from range [0.0001, 0.01].
For detailed hyperparameter settings please refer to Appendix~\ref{appx:hyper}.

{For all baselines, we strictly adhere to their original settings for hyperparameter tuning to ensure that they attain best performance. 
All GSL methods are evaluated based on the performance of GNNs on downstream tasks when using the learned structure. 
We also consider cross-architecture scenarios in Subsection~\ref{RQ4}, where GSL training and downstream tasks use different GNN architectures.}

\subsection{Main Results (RQ1)}

\label{RQ1}
\subsubsection{Comparison to Vanilla GSL Models.}
Table \ref{Tab:performance} presents the experimental results of the UnGSL module applied to various GSL models. 
We can observe that: 1) UnGSL significantly improves the node classification accuracy for all GSL models across all datasets with an average increase of 2.07$\%$, achieving new state-of-the-art performance in the GSL literature.
The improvement brought by UnGSL is particularly pronounced on GRCN \cite{yu2021graph}, achieving an average improvement of 5.12$\%$.  
These results empirically demonstrate UnGSL's effectiveness in further denoising the learned graph structure from the perspective of uncertainty, resulting in more  accurate predictions.
2) For the self-supervised GSL model SUBLIME \cite{liu2022towards}, UnGSL continues to achieve higher accuracy by utilizing contrastive loss in Equation (\ref{eq13}) as a proxy of uncertainty estimation to guide structure refinement, resulting in an average improvement of 1.40$\%$.
\begin{table}
\renewcommand{\arraystretch}{1.0}
\vspace{0pt}
\centering
\setlength{\abovecaptionskip}{0.cm}
\caption{Ablation study on the UnGSL when integrating with GRCN model \cite{yu2021graph}.}
% We report the performance of UnGSL and two variants.}
\label{Tab:Ablation}
\setlength{\tabcolsep}{0.005\textwidth}{
{
\begin{tabular}{cccccc}
\hline
        Method & Cora & Citeseer %& Pubmed & BlogCatalog %
        & Roman-empire \\ \hline
      
        Fixed $\boldsymbol{\varepsilon}$ & 85.23±0.15 & 73.2±0.96 %& 79.37±0.23 & 76.64±0.17%
        & 44.72±0.14 \\ 
        $\text{Symmetrize}\, {\hat{\mathbf{S}}}$  & {85.03±0.40} & {73.74±0.49} &{52.28±0.025}   \\ 
        UnGSL & \textbf{85.84±0.51} & \textbf{73.88±0.55} & %\textbf{79.67 ±0.31} & 
        %\textbf{76.78±0.11} & 
        \textbf{52.54±0.31}  \\ 
        \hline
\end{tabular}}}
\vspace{0.5pt}
\end{table}

\begin{figure}
    \vspace{-5pt}
    \centering    \includegraphics[width=0.720\linewidth, trim = 45 20 50 10]{BetaAnalysisGRCN.pdf}
    \caption{Comparsion of different $\beta$ on  Cora and Citeseer datasets.}
    \label{beta analysis}
    \vspace{-19pt}
    \Description{}
\end{figure}
\subsubsection{Comparison to the Label-oriented Directed GSL Module.}
\label{sec4.2.2}
% The results of SUBLIME+UnGSL demonstrate UnGSL's generalizability to self-supervised GSL models. 
% To comprehensively demonstrate UnGSL's superiority over another plug-and-play GSL module, CUR decomposition, we validate their performance on the GRCN and IDGL models.
Here we compare the performance of UnGSL with CUR decomposition \cite{lu2024latent}, another GSL module proposed recently to learn asymmetric graph structure by constructing directed edges from labeled nodes to unlabeled nodes.
Table \ref{Tab:CUR Vs UnGSL} shows the experiment results of UnGSL and CUR decomposition.
We can observe that: 1) UnGSL consistently outperforms CUR decomposition on supervised GSL methods.
2) UnGSL demonstrates better scalability with large-scale graphs compared to CUR decomposition. 
For example, IDGL+UnGSL is applicable to the Roman-Empire dataset and further enhances accuracy, while IDGL+CUR encounters an out-of-memory problem.
% The results suggest that directly modeling the structure from labeled to unlabeled nodes is less effective than capturing asymmetric relationships between nodes with varying levels of uncertainty. The latter approach enables the model to learn more nuanced and comprehensive relationships among the majority of unlabeled nodes.

% \subsubsection{Comparison of Predictive Uncertainty.}
% To demonstrate that UnGSL effectively reduces the predictive uncertainty of nodes in a learned graph, we analyze the changes in the entropy distribution.
% Specifically, we select nodes from the test sets of the Cora and Citeseer datasets, calculate their entropies, and present the results in a box plot, as shown in Fig. \ref{boxplot}.
% We observe that GSL models incorporating UnGSL generally reduce node entropy, demonstrating UnGSL's effectiveness in learning high-quality structures that prevent nodes from falling into regions of high uncertainty.
% \begin{figure*}[!h]
%     \begin{minipage}{0.40\linewidth}
%         \centering
%         \renewcommand{\arraystretch}{1.35}
%         \setlength{\abovecaptionskip}{0.16cm}
%         % \vspace{-5pt}
%         \captionof{table}{Ablation study on the UnGSL in GRCN model.
%         The top-performing results are marked in \textbf{bold}.}
%         \label{Tab:Ablation}
%         \setlength{\tabcolsep}
%         {0.008\textwidth}
%  % 调整间距
% {
%         \begin{tabular}{cccccc}
% \hline
%         Method & Cora & Citeseer %& Pubmed & BlogCatalog %
%         & Roman-empire \\ \hline
      
%         Fixed $\boldsymbol{\varepsilon}$ & 85.23±0.15 & 73.2±0.96 %& 79.37±0.23 & 76.64±0.17%
%         & 44.72±0.14 \\ 
%         $\text{Symmetrize}\, {\mathbf{A}}^{*}$  & {85.03±0.40} & {73.74±0.49} &{52.28±0.025}   \\ 
%         UnGSL & \textbf{85.47±0.40} & \textbf{73.93±0.45} & %\textbf{79.67 ±0.31} & 
%         %\textbf{76.78±0.11} & 
%         \textbf{52.46±0.32}  \\ 
%         \hline
% \end{tabular}}
% \vskip -0.1in
% % \vskip -0.08in
% \end{minipage}
% \hfill
%     \begin{minipage}{0.45\linewidth}
%         \centerline{\includegraphics[width=0.72\linewidth, trim = 100 -5 80 0]{F2line_chart.pdf}}
%         \vskip -0.17in
%         \setlength{\abovecaptionskip}{0.3cm}
%         \captionof{figure}{Effects of different $\beta$ on Cora.}
%         \label{linefig_2}
%     \end{minipage}
%     % \vskip -0.175in
% \end{figure*}

\subsection{Ablation Studies (RQ2)}

\label{RQ2}
\subsubsection{Effects of Adaptive Threshold $\boldsymbol{\varepsilon}$.}
To validate the effectiveness of the learnable threshold, we consider a variant where we fixed the $\boldsymbol{\varepsilon}$ in UnGSL during training phase.
Specifically, for each node, we select a fixed proportion of the most uncertain neighbors, as high-uncertainty neighbors. 
This proportion is consistent across all nodes. Next, we reweight edges by Equation (\ref{eq10}).
As shown in Table \ref{Tab:Ablation}, the learnable $\boldsymbol{\varepsilon}$ outperforms the fixed $\boldsymbol{\varepsilon}$ across 3 datasets.
The results indicate that the learnable threshold effectively differentiates high-uncertainty neighbors from low-uncertainty neighbors. 
% By adaptively modifying the corresponding edges, UnGSL further enhances node representations and facilitates accurate predictions.

\subsubsection{Superiority of Asymmetric Graph.}
To demonstrate the superiority of the asymmetric edges constructed by UnGSL, we symmetrize the graph refined by UnGSL during training phase, generating an symmetric graph structure. 
As shown in Table \ref{Tab:Ablation}, symmetrizing graph $\hat{\mathbf{S}}$ of UnGSL degrades its performance on node classification.
The underlying reason is that the symmetrization operation may perturb the graph by weakening edges from low-uncertainty neighbors and strengthening edge weights from high-uncertainty neighbors, which violates the core mechanism of UnGSL.

\subsubsection{Analysis on Hyperparameter $\beta$.}
To explore the role of $\beta$ in UnGSL's activation function $\psi( \cdot )$, we assign different values to  $\beta$ from the interval $[0,1]$ and evaluate the corresponding performance on the GRCN model \cite{yu2021graph}.
As shown in Figure \ref{beta analysis}, we can observe that:
(1) UnGSL enhances the accuracy of the GSL model by reducing edge weights with low-uncertainty neighbors using an appropriate $\beta$.
(2) UnGSL with positive $\beta$ consistently outperform UnGSL with $\beta=0$.
The results indicate that high-uncertainty neighbors still contain valuable information that enhances the node's representation, and removing connections from high-uncertainty nodes blindly may lead to information loss.
% Furthermore, setting $\beta=0$ in UnGSL directly reduces the average node degree, leading to increased neighbor distribution variance and consequently undermining intra-class node separability~\cite{ma2022is}. 
Results for additional GSL models are provided in Appendix \ref{appx: hyperbeta}.
\subsubsection{Analysis on Hyperparameter $\tau$.}
We  conduct additional experiments to evaluate the model's performance with different values of $\tau$
on Cora and Citeseer datasets using the base GSL model GRCN \cite{yu2021graph}.
The results are presented in Table \ref{Tab:tau}.
As can be seen, as $\tau$ increases, the performance initially improves and then declines. 
Hyperparameter $\tau$ controls influence of high-confidence neighbors. 
While fine-tuning $\tau$ could potentially enhance model performance, we find that simply setting $\tau = 2$ is sufficient to achieve good results.

\begin{table}
\renewcommand{\arraystretch}{1.2}
% \vspace{2pt}
\centering
\setlength{\abovecaptionskip}{0.cm}
\caption{Comparsion of different $\tau$ on Cora and Citeseer datasets.}
\label{Tab:tau}
\setlength{\tabcolsep}{0.005\textwidth}{
{
\begin{tabular}{cccc}
\hline
        Method & $\tau =1 $ & $\tau=2$ %& Pubmed & BlogCatalog %
        & $\tau=3$ \\ \hline
      
        Cora & 85.12±0.47 & 85.84±0.51 %& 79.37±0.23 & 76.64±0.17%
        & 84.87±0.55 \\ 
        Citeseer  & {72.77±0.90} & {73.88±0.55} &{73.09±0.42}   \\  
        \hline
\end{tabular}}}
\vspace{-5pt}
\end{table}

\begin{table}[!t]
	\centering
        \small
        \setlength{\abovecaptionskip}{0.cm}
	\caption{Robustness analysis with random structural noise injection on Cora dataset.
 }
\setlength{\tabcolsep}{0.0011\textwidth}
	\label{tab:robust}
	
{
\begin{tabular}[t!]{c|ccccc|ccccc}
        \toprule
        {} &  \multicolumn{5}{c|}{Edge Deletion Level} &  \multicolumn{5}{c}{Edge Addition Level}\\
        \midrule
        {Methods} & 0$\%$ & 
        20$\%$ & 40$\%$ & 60$\%$ & 80$\%$ & 
        0$\%$ & 
        20$\%$ & 40$\%$ & 60$\%$ & 80$\%$\\
        \midrule
GRCN    & 84.70 & 83.00 & {79.87} & {78.47} & {75.17} & {84.70} & {78.03} & {75.13} & 73.20 & 69.43 \\
GRCN+UnGSL & {85.84} & {84.80} &{81.83} & {80.20} &{76.32} & {85.84} & {79.17} & {76.27} & {74.42} & {70.60}\\
Improve(\%) & {1.34\%} & {2.16\%} & {2.45\%} & {2.21\%} & {1.53\%} & {1.34\%} & {1.46\%} & {1.65\%} & 1.66\% & 1.69\% \\ 
\midrule
IDGL & 84.50 & 81.60 &80.31 & {76.57} & {72.18} & {84.50} & {78.59} & {77.24} & 74.43 & 73.11  \\
IDGL+UnGSL & {84.90} & {82.88} & {81.41} & {77.83} & {73.24} & {84.90} & {80.02} & {78.77} & {76.25} & {75.31} \\
Improve(\%) & 0.47\% & 1.57\% & 1.38\% & 1.65\% & 1.47\% & 0.47\% & 1.82\% & 1.98\% & 2.45\% & 3.01\% \\
\bottomrule
\end{tabular}
}
\vspace{-10pt}
\end{table}
% \begin{table*}[!t]
% 	\centering
%         \small
%         \setlength{\abovecaptionskip}{0.cm}
% 	\caption{Robustness analysis with random noise injection on Cora dataset.
%     We evaluate the performance of UnGSL under structural noise, feature noise, and label noise.
%  The value in bold signifies the top-performing result.}
% \setlength{\tabcolsep}{0.004\textwidth}
% 	\label{tab:robust}
	
% {
% \begin{tabular}[t!]{c|ccccc|ccccc|ccccc|ccccc}
%         \toprule
%         {} &  \multicolumn{5}{c|}{Random Edge Deletion Level} &  \multicolumn{5}{c|}{Random Edge Addition Level} &  \multicolumn{5}{c|}{Feature Noise Level} & \multicolumn{5}{c}{Label Noise Level}\\
%         \midrule
%         {Methods} & 0$\%$ & 
%         20$\%$ & 40$\%$ & 60$\%$ & 80$\%$ & 
%         0$\%$ & 
%         20$\%$ & 40$\%$ & 60$\%$ & 80$\%$
%         &
%         0$\%$ & 
%         20$\%$ & 40$\%$ & 60$\%$ & 80$\%$
%         &
%         0$\%$ & 
%         20$\%$ & 40$\%$ & 60$\%$ & 80$\%$\\
%         \midrule
% GRCN    & 84.70 & 83.00 & {79.87} & {78.47} & {75.17} & {84.70} & {78.03} & {75.13} & 73.20 & 69.43 & 84.70 & 82.47 &{81.60} & {79.40} & {72.40} & {84.70} & {81.33} & {73.32} & 65.21 & 58.13\\
% GRCN+UnGSL & \textbf{85.84} & \textbf{84.80} &\textbf{81.83} & \textbf{80.20} & \textbf{76.32} & \textbf{85.84} & \textbf{79.17} & \textbf{76.27} & \textbf{74.42} & \textbf{70.60} & \textbf{85.84} & \textbf{83.70} & \textbf{82.80} & \textbf{80.62} & \textbf{73.70} & \textbf{85.84} & \textbf{83.43} & \textbf{74.52} & \textbf{67.94} & \textbf{62.17}\\
% Improve(\%) & {1.34\%} & {2.16\%} & {2.45\%} & {2.21\%} & {1.53\%} & {1.34\%} & {1.46\%} & {1.65\%} & 1.66\% & 1.69\% & 1.34\% & 1.49\%  &{1.47\%} & {1.53\%} & {1.80\%} & {1.34\%} & {2.58\%} & {1.63\%} & 4.18\% & 6.95\%  \\ 
% \midrule
% IDGL & 84.50 & 81.60 &80.31 & {76.57} & {72.18} & {84.50} & {78.59} & {77.24} & 74.43 & 73.11 & 84.50 & 82.10 &{79.25} & {75.23} & {65.16} & 84.50 &{77.60} & {70.80} & {54.81} & 43.65 \\
% IDGL+UnGSL & \textbf{84.90} & \textbf{82.88} & \textbf{81.41} & \textbf{77.83} & \textbf{73.24} & \textbf{84.90} & \textbf{80.02} & \textbf{78.77} & \textbf{76.25} & \textbf{75.31} &\textbf{ 84.90} & \textbf{83.29} & \textbf{80.52} & \textbf{77.43} & \textbf{66.50} & \textbf{84.90} & \textbf{79.35} & \textbf{72.03} & \textbf{56.13} & \textbf{44.51} \\
% Improve(\%) & 0.47\% & 1.57\% & 1.38\% & 1.65\% & 1.47\% & 0.47\% & 1.82\% & 1.98\% & 2.45\% & 3.01\% & 0.47\% & 1.45\% & 1.60\%  & 2.92\%  & 2.04\% & 0.47\%  & 2.26\% & {1.74\%} & {2.41\%} & {1.97\%} \\
% \bottomrule
% \end{tabular}
% }
% \end{table*}


\subsection{Robustness Analysis (RQ3)}
\label{RQ3}

To evaluate the robustness of UnGSL under structural noise, we compare the accuracy of the base GSL model and GSL+UnGSL under different noise levels and calculate the relative improvement achieved by GSL+UnGSL.
Specifically, we randomly add or remove edges from the original dataset.
% Specifically, we evaluate the robustness of UnGSL under three types of noise scenarios: structural noise, feature noise, and label noise.
% For feature noise, we randomly mask a proportion of node features by replacing their values with zeros. 
% This approach allows us to investigate the performance of UnGSL when node features are subjected to varying degrees of damage.
% For label noise, we randomly alter node labels.
% The ratio of modifications  varies from 0 to 0.8 to simulate different levels of noise.
We conduct experiments on the Cora dataset, and the results are presented in Table \ref{tab:robust}.
We can observe that:
1)  UnGSL consistently enhances the performance of GSL models across different noise levels.
2) With increasing levels of noise, UnGSL achieves greater relative improvements. 
For example, in the random edge addition scenario, IDGL+UnGSL achieves a 3.01\% improvement under 80\% edge perturbation, which is significantly higher than the 0.47\% improvement observed at the 0\% noise level.
% ; with 80\% wrong labels, GRCN+UnGSL achieves a 6.95\% improvement, which is significantly higher than the improvement under 0\% wrong labels scenario (1.34\%).
In summary, the results suggest that UnGSL enhances robustness against structural noise.
We also validate UnGSL's robustness against feature noise and label noise, and the results are presented in Appendix \ref{appx:robust}.
% Results on more datasets are provided in Appendix \ref{}.}
% \subsubsection{Robustness Against Structural Noise}
% To assess the structural robustness of UnGSL in GSL models under topological perturbations, We randomly add or remove edges from the original Cora dataset.
% The ratio of modified edges varied from 0 to 0.8 to simulate different levels of edge noise. 
% As shown in Table \ref{tab:robust}, we can observe that:
% \textcolor{red}{
% 1) all GSL models experience a performance decline as the ratio of modified edges increases, while UnGSL can consistently improves the performance of GSL models across different perturbation ratios.
% 2) With increasing levels of edge noise, UnGSL achieves greater relative improvements. 
% For example, in the random edge addition scenario, IDGL+UnGSL achieves a 3.01\% improvement under 80\% edge perturbation, which is significantly higher than the 0.47\% improvement observed at the 0\% noise level.
% In summary, the results suggest that UnGSL enhances model robustness to topological noise.
% }
% \textcolor{red}
% {
% \subsubsection{Robustness Against Feature Noise}
% To evaluate the robustness of UnGSL against feature noise, we randomly mask a proportion of node features by replacing their values with zeros. 
% This approach allows us to investigate the performance of UnGSL when node features are subjected to varying degrees of damage.
% The results are presented in Table \ref{tab:robust}, we can also observe that:
% 1) UnGSL consistently enhances the performance of GSL models across various feature perturbation levels.
% 2) As the feature noise level increases,  UnGSL consistently achieves greater accuracy improvements compared to its performance in scenarios of 0\% feature noise.
% In general, the results demonstrate UnGSL's robustness against feature noise.
% }
% \textcolor{red}
% {
% \subsubsection{Robustness Against Label Noise}
% The performance of GSL models often relies heavily on high-quality node labels, which are challenging to obtain in real-world scenarios due to unreliable sources or adversarial attacks \cite{NoisyGL}.
% To verify the robustness of UnGSL to imprecise node labels, we randomly alter node labels and compare the performance of UnGSL+GSL with the base GSL model.
% As shown in Table \ref{tab:robust}, we can observe that:
% 1) UnGSL consistently improves the accuracy of GSL as the proportion of incorrect node labels increases.
% 2) The improvement achieved by UnGSL under high levels of label noise is significantly greater than under no label noise. 
% For instance, with 80\% wrong labels, GRCN+UnGSL achieves a 6.95\% improvement, which is significantly higher than the improvement under 0\% wrong labels scenario (1.34\%) . 
% }
% The results demonstrate the robustness of UnGSL against label noise.

\begin{table}[!t]
	\centering
        \small
        \setlength{\abovecaptionskip}{0.cm}
	\caption{Generalizability of UnGSL with different backbones on Cora dataset. 
 The value in bold signifies the top-performing result.}
	\label{tab:general}
	\resizebox{\linewidth}{!}
{
\begin{tabular}[t!]{c|cccc}
        \toprule
        {Methods} & SGC & 
        APPNP & GAT & JKNet\\
        \midrule
GRCN    & 84.40±0.00 & 84.00±0.15 &{81.45±1.04} & {83.73±1.33}  \\
GRCN+UnGSL & \textbf{84.80±0.10} & \textbf{84.40±0.61} &\textbf{82.45±1.04} & \textbf{84.53±1.38}\\
\midrule
IDGL & 78.00±0.00 & 82.13±0.32 &{79.87±1.01} & {76.23±1.19}\\
IDGL+UnGSL & \textbf{78.20±0.12} & \textbf{82.50±0.61} & \textbf{80.23±0.49} & \textbf{77.10±0.50}\\
\bottomrule
\end{tabular}
}
\vspace{-5pt}
\end{table}

\begin{table}
\renewcommand{\arraystretch}{1.0}
% \vspace{15pt}
\centering
\setlength{\abovecaptionskip}{0.cm}
\caption{ Convergence time and GPU memory consumption of different models on Ogbn-arxiv dataset.}
\label{Tab:efficiencyarxiv}
\setlength{\tabcolsep}{0.005\textwidth}{
{
\begin{tabular}{cccccc}
\hline
        Model & Time(s) & Memory(MB) \\ \hline
      
        IDGL & 105 & 14,566  \\ 
        
        IDGL+UnGSL & 113 & 15,338   \\ 
        \hline

        SUBLIME & 807 & 14,586  \\ 
        
        SUBLIME+UnGSL & 886 & 14,678  \\ 
        \hline
\vspace{-15pt}
\end{tabular}}}
\end{table}

\subsection{Generalizability on GNN Models (RQ4)}
\label{RQ4}
We further consider the scenario where GSL training and downstream tasks use different GNN backbones. 
We evaluate the generalizability of the learned structures generated by the GSL+UnGSL on several other GNN models, including SGC \citep{wu2019simplifying}, APPNP \citep{gasteiger2018combining}, GAT \citep{veličković2018graph}, and JKNet \citep{xu2018representation}. 
The results are presented in Table \ref{tab:general}.
We observe that the graphs produced by GSL+UnGSL improve the prediction of various GNN models compared to those generated by vanilla GSL. 
Overall, UnGSL demonstrates its generalizability in enhancing performance across different GNN architectures.
Results for additional datasets are provided in Appendix \ref{appx: general}.

\subsection{Efficiency Analysis (RQ5)}
\label{seceffiency}
We analyze the time and memory efficiency of UnGSL on the Ogbn-Arxiv dataset, a large-scale graph with 169,343 nodes and 1,157,799 edges.
To assess time efficiency, we evaluate the algorithms by measuring the time taken to converge, \ie to reach optimal performance on the validation set. As shown in Table \ref{Tab:efficiencyarxiv}, UnGSL slightly increases convergence time (8.71\% on average) and training memory (2.97$\%$ on average) compared to the base GSL models.
The results demonstrate UnGSL's efficiency when applied to GSL models. 
% \begin{table*}[!t]
% 	\centering
%         \small
% 	\caption{Generalizability of UnGSL with different backbones on Cora and Citeseer datasets. 
%  The value in bold signifies the top-performing result.}
% 	\label{tab:general}
% 	\resizebox{\linewidth}{!}
% {
% \begin{tabular}[t!]{c|cccc|cccc}
%         \toprule
%         {} &  \multicolumn{4}{c|}{Cora} &  \multicolumn{4}{c}{Citeseer} \\
%         \midrule
%         {Methods} & SGC & 
%         APPNP & GAT & JKNet & SGC & 
%         APPNP & GAT & JKNet\\
%         \midrule
% GRCN    & 84.40±0.00 & 84.00±0.15 &{81.45±1.04} & {83.73±1.33} & {72.00±0.00} & {72.73±0.96} & {70.77±1.06} & {71.83±0.72} \\
% GRCN+UnGSL & \textbf{84.80±0.10} & \textbf{84.40±0.61} &\textbf{82.45±1.04} & \textbf{84.53±1.38} & \textbf{72.37±0.06} & \textbf{73.50±0.30} & \textbf{70.88±1.70} & \textbf{72.32±1.75}\\
% \midrule
% IDGL & 78.00±0.00 & 82.13±0.32 &{79.87±1.01} & {76.23±1.19} & {71.90±0.00} & {72.20±0.75} & {63.83±0.97} & {71.00±1.15} \\
% IDGL+UnGSL & \textbf{78.20±0.12} & \textbf{82.50±0.61} & \textbf{80.23±0.49} & \textbf{77.10±0.50} & \textbf{73.40±0.00} & \textbf{73.37±0.61} & \textbf{67.00±0.69} & \textbf{72.33±0.9} \\
% \bottomrule
% \end{tabular}
% }
% \end{table*}
\section{Related Work}
\label{sec5}
\subsection{Graph Neural Networks}
Graph neural networks (GNNs) are powerful models for learning node representations from graph data.
Existing GNNs can be categorized as spectral GNNs and spatial GNNs.
Spectral GNNs leverage eigenvectors and eigenvalues within the graph Laplacian matrix to design graph signal filters in the spectral domain \cite{bruna2013spectral,defferrard2016convolutional,li2021dimensionwise,he2022convolutional}.
Spatial GNNs \cite{kipf2017semisupervised,gilmer2017neural,veličković2018graph,wu2019simplifying,ding2022sketch,wu2021disenkgat} simplified spectral graph filter \cite{defferrard2016convolutional} using first-order approximation, which aggregates features from neighboring nodes in the spatial graph to generate node embeddings.
%\cite{kipf2017semisupervised,gilmer2017neural,veličković2018graph,wu2019simplifying,ding2022sketch}.
% \citet{kipf2017semisupervised} simplified spectral graph filter \citep{defferrard2016convolutional} using first-order approximation, which aggregates features from neighboring nodes in the spatial graph to generate node embeddings.
% Building on this foundation, numerous spatial GNN variants with specialized aggregation methods have since been proposed \citep{gilmer2017neural,veličković2018graph,wu2019simplifying,ding2022sketch},  achieving remarkable performance across various tasks.
% In recent years, GNNs have developed  sophisticated architectures to  handle complex task scenarios, such as structural distribution shifts \cite{gui2024joint}, continual graph learning \cite{zhang2022cglb}, and model explainability \cite{wang2024gnnboundary}.
However, existing GNNs assume that the input graph structure is sufficiently clean for learning, whereas real-world graphs are often noisy and incomplete, which limits GNN performance on downstream tasks \cite{geisler2021robustness,chen2023bias}.
In this paper, we propose uncertainty-aware graph structure learning, which effectively denoises the graph structure to alleviate the above limitation.



% \begin{figure}[t]
% \centering
% \includegraphics[width=1.0\linewidth]{effiency_cora.pdf}
% \caption{Time and space consumption of different methods on Cora dataset.}
% \label{fig:efficiencycora}
% \vspace{-10pt}
% \end{figure}

\subsection{Graph Structure Learning}
Graph Structure Learning (GSL) aims to learn an optimal graph that improves the accuracy and robustness of Graph Neural Networks (GNNs) on downstream tasks.
Early GSL methods \cite{franceschi2019learning,jin2020graph} directly treat the target adjacency matrix as learnable parameters, incurring substantial
computational overhead and optimization challenge.
Mainstream GSL methods \cite{chen2020iterative,yu2021graph,wang2023prose,in2024self} learn edge weights based on node-pair embedding similarities, employing various metrics such as cosine similarity \cite{wang2023prose,in2024self}, inner product \cite{yu2021graph} or neural networks \cite{liu2022compact}.
These methods aim to increase graph homophily and typically achieve state-of-the-art performance, as these metrics can capture nodes with similar semantics.
However, these embedding-based GSL methods suffer from two limitations:
First, they construct edges solely based on embedding similarities while neglecting node uncertainty, which may introduce inferior information that poison the target node's embedding. 
Although some works~\cite{liu2022compact, duan2024structural} incorporate predictive uncertainty in structure learning, they only use it for cross-view structure fusion without distinguishing nodes of varying uncertainty levels within a graph.
Second, embedding-based methods impose bidirectional edges between nodes, disregarding their unequal influence due to varying levels of uncertainty.
Recently, several methods propose constructing directed edges from labeled to unlabeled nodes to facilitate the propagation of supervision signals~\cite{song2022towards,song2024optimal,lu2024latent}.
However, these methods fail to learn reasonable
asymmetric connections between the vast majority of unlabeled nodes.
% However, existing GSL models overlook the predictive uncertainty of neighboring nodes and the symmetry constraint in the structure modeling process, which  significantly impact node representations during aggregation.
% Although a few GSL methods \cite{wang2021graph,liu2022compact}  incorporate predictive uncertainty, it is used either for adaptive multi-view node representation fusion \cite{liu2022compact} or as a measure of similarity between nodes \cite{wang2021graph,in2024self}.
Different from the above works, we propose an uncertainty-aware structure learning (UnGSL) strategy that learns node-wise thresholds to differentiate low-uncertainty from high-uncertainty neighbors and adaptively refine the corresponding edges.
% Notably, UnGSL is able to  directly applied to most GSL models where edge weights updated through gradient descent,  and it is not compatible with a few GSL models \cite{li2022reliable,zou2023se,wang2021graph} that do not meet this criterion.

\subsection{Uncertainty in GNNs}
GNNs inevitably present uncertainty towards their predictions, leading to unstable and erroneous prediction results \cite{wang2024uncertainty}.
In recent years, uncertainty in GNNs has been widely researched to adapt various tasks, including out-of-distribution (OOD) detection \cite{zhao2020uncertainty,stadler2021graph,yu2023uncertainty}, trustworthy GNN learning \cite{wang2021confident,hsu2022makes}, and GNN modeling \cite{um2023confidencebased}.
However, these uncertainty-based GNNs assume that the input graph structure is clean and primarily focus on incorporating uncertainty into the model architecture.
In contrast, our proposed uncertainty-aware graph structure learning aims to refine the edge based on node uncertainty, assisting in improving graph quality. 
% However, these GSL methods  overlook the predictive uncertainty of neighboring nodes and the resulting asymmetry in the structure.
% In this paper, we use the predictive uncertainty to differentiate between nodes in structure learning process.
% Predictive uncertainty is typically quantified using maximum softmax probability and entropy in classification tasks.
% For accurate estimation and application of uncertainty, predictive uncertainty is commonly categorized into aleatoric and epistemic uncertainty based on their distinct sources. \cite{hullermeier2021aleatoric}.
% Specifically, aleatoric uncertainty refers to data uncertainty from statistical randomness while epistemic uncertainty indicates model uncertainty due to limited knowledge in collected data \cite{zhao2020uncertainty}.
% Notably, the entropy of a node's representation is influenced by both aleatoric and epistemic uncertainty\cite{wang2024uncertainty}. 
% Therefore, we use predictive entropy to differentiate between nodes when constructing edges.
\section{Conclusion}
\label{sec6}
In this paper, we first conduct theoretical and empirical analyses to demonstrate the detrimental impact of neighbors with high uncertainty on GNN learning.
Building on this, we propose the UnGSL strategy, a lightweight plug-in module that integrates seamlessly with state-of-the-art GSL models and boosts performance with minimal extra computational overhead.
UnGSL learns node-wise thresholds to differentiate between low-uncertainty and high-uncertainty neighbors, and adaptively refines the graph based on each node's uncertainty level.
Experiments demonstrate that UnGSL consistently enhances the performance and robustness of  GSL models.
In the future, we plan to explore more effective uncertainty metrics to accurately identify uncertain nodes in graph structure learning.
% , since the entropy is calculated based on softmax output,which only reflects the total predictive uncertainty instead of the model uncertainty, leading to false confidence under the distribution shift.
%%
%% The acknowledgments section is defined using the "acks" environment
%% (and NOT an unnumbered section). This ensures the proper
%% identification of the section in the article metadata, and the
%% consistent spelling of the heading.
\begin{acks}
This work is supported by the National Natural Science Foundation of China (62476244,62372399,62476245), Zhejiang Provincial Natural Science Foundation of China (Grant No: LTGG23F030005).
\end{acks}

%%
%% The next two lines define the bibliography style to be used, and
%% the bibliography file.
\bibliographystyle{ACM-Reference-Format}
\bibliography{sample-base}
\balance
\clearpage
%%
%% If your work has an appendix, this is the place to put it.
\appendix
\section{Proof of Proposition \ref{proposition 1} }
\label{proof}
To prove Proposition \ref{proposition 1}, we first introduce the log-sum inequality \cite{csiszar2004information} below.

\begin{lemma}[Log-sum Inequality]
Let $a_1,...,a_n$ and $b_1,...,b_n$ be non-negative numbers. 
Denote the sum of all $a_i$s by a and the sum of all $b_i$s by b.
Then log-sum inequality states that
\begin{equation}
    \sum_{i=1}^{n}a_{i}\text{log}\frac{a_i}{b_i} \geq a\text{log}\frac{a}{b},
\end{equation}
\end{lemma}
\textit{Proof.}
$\forall v_i \in \mathcal{V}$, by define the  logit $o_i=(\sum_{v_j \in \mathcal{N}{(v_i)}}\hat{\mathbf{A}}_{ij}x_j)\mathbf{W}$ and logit $o_{i}^{\prime}=x_i\mathbf{W}$, where $\sum_{v_j \in \mathcal{N}{(v_i)}}\hat{\mathbf{A}}_{ij}=1$, we have:
\begin{equation}
    o_i = \sum_{v_j \in \mathcal{N}{(v_i)}}\hat{\mathbf{A}}_{ij}o_{j}^{\prime} ,
\end{equation}
then for predictive probability vector $p_i$ of $o_i$, $\forall v_j \in \mathcal{N}{(v_i)}$, let $p_j^{\prime}$ denote the classification probabilities of its initial features.
We have:
 \begin{align}
    p_i &= \frac{o_{i} +  \mathbf{1}_{K} }{\sum_{k=1}^{K}(o_{ik} + 1)}\nonumber\\
    &= \frac{\sum_{v_j \in \mathcal{N}{(v_i)}}\hat{\mathbf{A}}_{ij}(o_{j}^{\prime} +  \mathbf{1}_{K} )}
    {\sum_{k=1}^{K}(\sum_{v_j \in \mathcal{N}{(v_i)}}\hat{\mathbf{A}}_{ij}o_{jk}^{\prime} + 1)}\nonumber\\
    &=\frac{\sum_{v_j \in \mathcal{N}{(v_i)}}p_{j}^{\prime}
     (\hat{\mathbf{A}}_{ij}{\sum_{k=1}^{K}(o_{jk}^{\prime} + 1)}) }
    {\sum_{v_j \in \mathcal{N}{(v_i)}}\hat{\mathbf{A}}_{ij}\sum_{k=1}^{K}(o_{jk}^{\prime} + 1)},\nonumber\\
    &=\sum_{v_j \in \mathcal{N}{(v_i)}}{\eta_j}p_{j}^{\prime},
 \end{align}
where,$\sum_{v_j \in \mathcal{N}{(v_i)}}{\eta_j}=1$.
Now we focus the $l$-th element in the probability vector:
\begin{align}
    p_{il} & =  \sum_{v_j \in \mathcal{N}{(v_i)}}{\eta_j}p_{jl}^{\prime}\nonumber\\
    \Longrightarrow 
    p_{il}\text{log}p_{il}&=(\sum_{v_j \in \mathcal{N}{(v_i)}}{\eta_j}p_{jl}^{\prime})\text{log}(\sum_{v_j \in \mathcal{N}{(v_i)}}{\eta_j}p_{jl}^{\prime})\nonumber\\
    &=(\sum_{v_j \in \mathcal{N}{(v_i)}}{\eta_j}p_{jl}^{\prime})\text{log}(\frac{{\sum_{v_j \in \mathcal{N}{(v_i)}}\eta_j}p_{jl}^{\prime}}{\sum_{v_j \in \mathcal{N}{(v_i)}}{\eta_j}})\nonumber\\
    &\leq \sum_{v_j \in \mathcal{N}{(v_i)}}{\eta_j}p_{jl}^{\prime}\text{log}p_{jl}^{\prime}\nonumber\\
    \Longrightarrow 
    -\sum_{l=1}^{K}p_{il}\text{log}p_{il} &\geq 
    -\sum_{v_j \in \mathcal{N}{(v_i)}}{\eta_j}\sum_{l=1}^{K}p_{jl}^{\prime}\text{log}p_{jl}^{\prime}\nonumber\\
    &=\sum_{v_j \in \mathcal{N}{(v_i)}}{\eta_j} u_{j}^{\prime},
\end{align}
which completes the proof.




\section{Datasets}
\label{dataset}
Table \ref{Tab:dataset stastics} shows the statistics of 7 datasets.
% \begin{figure}[tbp]
%     \subfigure[Cora]{
%        \centering
%        \includegraphics[width=0.47\linewidth,trim = 30 10 40 10]{boxplotcora.pdf}
%     }
%     \subfigure[Citeseer]{
%         \centering
%         \includegraphics[width=0.47\linewidth,trim = 40 10 30 10]{boxplotciteseer.pdf}
%     }
%     \vspace{-10pt}
%     \caption{Boxplots of the node entropy on Cora and Citeseer datasets.\label{boxplot}}
%     % \vspace{-14pt}
% \end{figure}
\section{Baselines}
\label{baselines}
we introduce all GSL models used in the experiment in this section.
\begin{itemize}[leftmargin=1.0em]
    \item {GRCN} \cite{yu2021graph} uses a GCN to extract topological features and compute the similarity between nodes as the edge weights of the learned graph.
    \item {ProGNN} \cite{jin2020graph} treats the graph as a learnable adjacency matrix and optimizes the sparsity, low-rankness, and feature smoothness of the graph structure.
    \item {PROSE} \cite{wang2023prose} identifies influential nodes using PageRank scores and reconstructs the graph structure by connecting these influential nodes.
    \item {IDGL} \cite{chen2020iterative} iteratively learns the graph structure and node embeddings, and introduces a node-anchor message-passing paradigm to scale IDGL to large graphs.
    \item {SLAPS} \cite{fatemi2021slaps} proposes learning a denoising autoencoder to filter noisy edges in the graph structure.
    \item {CUR} \cite{lu2024latent} is a model-agnostic structure learning module, which proposes constructing unidirectional edges from unlabeled nodes to labeled nodes via CUR decomposition to facilitate the propagation of supervision signals to unlabeled nodes.
    \item {SUBLIME} \cite{liu2022towards} is an unsupervised GSL model that employs contrastive learning between the learned graph and an augmented graph to enhance the robustness of the graph structure.
\end{itemize}

\section{Additional Experimental Results}
\subsection{Visualization of Node Entropy and Average Neighbor Entropy }
\label{ADDC1}
We also conduct experiments on more datasets with different homophily characteristics.
Figure \ref{Figuread2} presents the node entropy after GNN aggregation, along with the average entropy of its neighbors, on the Citeseer and Flickr datasets.
We observe a strong positive correlation between the entropy of nodes and that of their neighbors across different datasets.

% \subsection{Visualization on Predictive Uncertainty.}
% To demonstrate that UnGSL effectively reduces the predictive uncertainty of nodes in a learned graph, we analyze the changes in the entropy distribution.
% Specifically, we select nodes from the test sets of the Cora and Citeseer datasets, calculate their entropies, and present the results in a box plot, as shown in Figure  \ref{boxplot}.
% We observe that GSL models incorporating UnGSL generally reduce node entropy, demonstrating UnGSL's effectiveness in learning high-quality structures that prevent nodes from falling into regions of high uncertainty.

\begin{figure}[ht]
    \subfigure[Citeseer dataset]{
       \centering
     \includegraphics[width=0.8\linewidth,trim = 35 20 -10 -5]{figure_node_neighbor_ent_citeseer.pdf}
     }
    \subfigure[Flickr dataset]{
        \centering
        \includegraphics[width=0.8\linewidth,trim = 35 20 -10 -5]{figure_node_neighbor_ent_flickr.pdf}
            \Description{}
    }
    \caption{Visualization of node entropy after GNN aggregation  (i.e., $u_{i}$) alongside the average entropy of its neighbors  (i.e., $\sum_{v_j \in \mathcal{N}{(v_i)}}\mathbf{\hat{A}}_{ij}u_{j}^{\prime}$) on Citeseer and Flickr datasets.\label{Figuread2}}
        \Description{}
\end{figure}

\subsection{Hyperparameter Settings}



\label{appx:hyper}
Table \ref{hyper} presents the hyperparameter settings of learning rate  and $\beta$ in UnGSL.




\subsection{Additional Results on the Hyperparameter \texorpdfstring{$\beta$}{}}
\label{appx: hyperbeta}
We assign different values to $\beta$ from the interval [0,1] and evaluate the corresponding performance on the IDGL model \cite{chen2020iterative}. 
The results are presented in Figure \ref{beta analysis IDGL}.



\begin{figure}[th]
    \centering    \includegraphics[width=0.70\linewidth, trim = 45 0 50 0]{BetaAnalysisIDGL.pdf}
    \caption{Comparsion of different $\beta$ on  Cora and Citeseer datasets.}
    \label{beta analysis IDGL}
    \vspace{-10pt}
        \Description{}
\end{figure}




% \begin{table*}[!t]
% 	\centering
%         \small
%         \setlength{\abovecaptionskip}{0.cm}
% 	\caption{Robustness analysis with random noise injection on Blogcatalog dataset.
%     We evaluate the performance of UnGSL under structural noise, feature noise, and label noise.
%  The value in bold signifies the top-performing result.}
% \setlength{\tabcolsep}{0.004\textwidth}
% 	\label{tab:robust}
	
% {
% \begin{tabular}[t!]{c|ccccc|ccccc|ccccc|ccccc}
%         \toprule
%         {} &  \multicolumn{5}{c|}{Random Edge Deletion Level} &  \multicolumn{5}{c|}{Random Edge Addition Level} &  \multicolumn{5}{c|}{Feature Noise Level} & \multicolumn{5}{c}{Label Noise Level}\\
%         \midrule
%         {Methods} & 0$\%$ & 
%         20$\%$ & 40$\%$ & 60$\%$ & 80$\%$ & 
%         0$\%$ & 
%         20$\%$ & 40$\%$ & 60$\%$ & 80$\%$
%         &
%         0$\%$ & 
%         20$\%$ & 40$\%$ & 60$\%$ & 80$\%$
%         &
%         0$\%$ & 
%         20$\%$ & 40$\%$ & 60$\%$ & 80$\%$\\
%         \midrule
% GRCN    & 76.17 & 72.55 & {71.63} & {69.22} & {64.04} & {76.17} & {75.60} & {73.86} & 73.08 & 72.38 & 76.17 & 75.59 & 75.26 & 74.67 & 71.99 & {76.17} &73.58 & 70.85 & 66.12 & 56.01\\
% GRCN+UnGSL & \textbf{76.78} & \textbf{73.54} &\textbf{72.72} & \textbf{70.29} & \textbf{64.95} & \textbf{76.78} & \textbf{76.67} & \textbf{75.0} & \textbf{74.10} & \textbf{73.62} & \textbf{76.78} & \textbf{76.28} & \textbf{75.97} & \textbf{75.26} & \textbf{72.76} & \textbf{76.78} & \textbf{74.14} & \textbf{71.56} & \textbf{67.10} & \textbf{56.61}\\
% Improve(\%) & {0.80\%} & {1.36\%} & {1.52\%} & {1.54\%} & {1.42\%} & {0.80\%} & {1.41\%} & {1.54\%} & 1.39\% & 1.71\% & 0.80\% & 0.91\%  &{0.94\%} & {0.79\%} & {1.06\%} & {0.80\%} & {0.76\%} & {1.00\%} & 1.48\% & 1.07\%  \\ 
% \midrule
% IDGL & 89.66 & 89.10 & 88.5 & 86.44 & 80.77 & {89.66} & 88.53 & 88.49 & 87.65 & 87.67 & 89.66 & 88.61 & 86.87 & 83.88 & 81.67 & 89.66 &84.55 & 80.66 & 70.44 & 49.70 \\
% IDGL+UnGSL & \textbf{92.13} & \textbf{92.02} & \textbf{91.13} & \textbf{90.53} & \textbf{87.24} & \textbf{92.13} & \textbf{91.64} & \textbf{91.24} & \textbf{91.27} & \textbf{91.00} &\textbf{ 92.13} &\textbf{91.27} & \textbf{89.16} & \textbf{86.34} & \textbf{84.39} & \textbf{92.13} & \textbf{87.68} & \textbf{83.29} & \textbf{76.00} & \textbf{52.71}\\
% Improve(\%) & 2.75\% & 3.27\% & 2.97\% & 4.73\% & 8.01\% & 2.75\% & 3.51\% & 3.11\% & 4.13\% & 3.80\% & 2.75\% & 3.00\% & 2.64\% & 2.93\% & 3.33\% & 2.75\%  & 3.70\% & 3.26\% & 7.89\% & 6.06\%\\
% \bottomrule
% \end{tabular}
% }
% \end{table*}

\subsection{Additional Results for Robustness Analysis}
\label{appx:robust}
We evaluate the robustness of UnGSL against label noise and feature noise on the Cora dataset. The results are shown in Table \ref{tab:robust1}.
For feature noise, we randomly mask a proportion of node features by replacing their values with zeros. 
This approach allows us to investigate the performance of UnGSL when node features are subjected to varying degrees of damage.
For label noise, we randomly alter node labels.
The ratio of modifications  varies from 0 to 0.8 to simulate different levels of noise.
We can observe that:
1)  UnGSL consistently enhances the performance of GSL models across different noise levels.
2) With increasing levels of noise, UnGSL achieves greater relative improvements. 
For example, with 80\% wrong labels, GRCN+UnGSL achieves a 6.95\% improvement, which is significantly higher than the improvement under 0\% wrong labels scenario (1.34\%).
In summary, the results suggest that UnGSL enhances robustness to label noise and feature noise.

\subsection{Additional Results on Generalizability of UnGSL}
\label{appx: general}
Table \ref{tab:generalciteseer} illustrates the generalizability of UnGSL using various backbones on the Citeseer dataset.

\begin{table}[t]
\begin{minipage}[t]{\linewidth}
\renewcommand{\arraystretch}{1.1} % 可以调节, 1.2指高度是默认的1.2倍
\centering
%****这里记得把caption写好
\setlength{\abovecaptionskip}{-0.05cm}
\caption{Detailed statistics of node classification datasets.}
\label{Tab:dataset stastics}
\setlength{\tabcolsep}{0.008\textwidth}{
% \begin{tabular}{l|lllll|lll}
{
\begin{tabular}{lcccccc}
\hline
        Dataset & \#Nodes & \#Edges & \#Feat. & \#Avg.degree  & \#Homophily \\ \hline
      
        Cora & 2,708 & 5,278 & 1,433 & 3.9 &  0.81 \\ 
        
        Citeseer & {3,327} & {4,552} & {3,703} &  {2.7} &   0.74  \\ 

        Pubmed & 19,717 & 44,324 & 500 & 4.5 &  0.80  \\
        
        BlogCatalog & 5,196 & 
        171,743 & 
        8,189 & 66.1 & 
         0.40  \\ 
        Roman-Empire & 22,662 & 32,927 & 300 & 2.9 &  0.05  \\ 
        Flickr & 7,575 & 239,738 & 12,047 &  63.3 & 0.24
        \\
        Ogbn-arxiv & 169,343  & 1,157,799  &  767 & 13.67 & 0.65
        \\
        \hline
\end{tabular}}}
\end{minipage}%
% \vspace{-15pt}
\end{table}


\begin{table}[t]
\centering
% \small
\setlength{\abovecaptionskip}{-0.05cm}
\caption{Hyperparameter settings of learning rate, $\beta$ and initial value in UnGSL. }
\label{hyper}
% \renewcommand{\arraystretch}{0.98}
{\small
% \resizebox{\columnwidth}{!}{
\scalebox{0.73}{
\begin{tabular}{c|c|cccccc}
% \noalign{\smallskip}{\smallskip}
\hline
  & Setting & GRCN & PROGNN & PROSE & IDGL & SLAPS & SUBLIME\\
\midrule
\multirow{3}{*}{Cora} 
& lr         & 0.0005 & 0.0001 & 0.01   & 0.001   & 0.0001          & 0.001          \\
& $\beta$    & 0.40  &0.82 & 0.01   & 0.95 & 0.75          & 0.98          \\

\midrule
\multirow{3}{*}{Citeseer} 
&     lr     & 0.006 & 0.03 & 0.001 & 0.0005 & 0.03 & 0.03 \\
& $\beta$   & 0.56 & 0.57 & 0.75  & 0.78 &0.19          & 0.89          \\


\midrule
\multirow{3}{*}{Pubmed} 
& lr         & 0.09  & - & 0.01   & 0.001 &0.02         & 0.004          \\

& $\beta$   & 0.35  & - & 0.22 & 0.16 & 0.65 & 0.46          \\

\midrule
\multirow{2}{*}{BlogCatalog} 
& lr         & 0.0056  &0.0002 & 0.06  &0.01& 0.007          & 0.0001          \\

& $\beta$   & 0.65  & 0.87 & 0.36 & 0.90   & 0.18        & 0.47         \\


\midrule
\multirow{3}{*}{Roman-Empire} 
& lr         & 0.0008
 & - & 0.001 &0.002   & 0.02          & 0.007        \\

& $\beta$   & 0.001 & - & 0.001 & 0.82   & 0.83          & 0.54       \\

\midrule
\multirow{3}{*}{Flickr} 
& lr         &0.008
 & 0.0001 & 0.03 &0.002   & 0.036          & 0.0001        \\

& $\beta$   & 0.08 & 0.96 & 0.71 & 0.70   & 0.23          & 0.82       \\

\midrule
\multirow{3}{*}{Ogbn-arxiv} 
& lr         & -
 & - & 0.001 &0.003   & -          & 0.024        \\

& $\beta$   & - & - & 0.2 & 0.59  & -          & 0.48       \\

\bottomrule

\end{tabular}
}}
\label{tab:hype}
\vspace{-5pt}
\end{table}

\begin{table}[!t]
	\centering
        \small
        \setlength{\abovecaptionskip}{0.cm}
	\caption{Robustness analysis with random noise injection on Cora dataset.
    We evaluate the performance of UnGSL under feature noise and label noise.
 }
\setlength{\tabcolsep}{0.0011\textwidth}
	\label{tab:robust1}

{
\begin{tabular}[t!]{c|ccccc|ccccc}
        \toprule
        {} &  \multicolumn{5}{c|}{Feature Noise Level} & \multicolumn{5}{c}{Label Noise Level}\\
        \midrule
        {Methods} & 0$\%$ & 
        20$\%$ & 40$\%$ & 60$\%$ & 80$\%$ & 
        0$\%$ & 
        20$\%$ & 40$\%$ & 60$\%$ & 80$\%$
       \\
        \midrule
GRCN    &  84.70 & 82.47 &{81.60} & {79.40} & {72.40} & {84.70} & {81.33} & {73.32} & 65.21 & 58.13\\
GRCN+UnGSL & {85.84} & {83.70} & {82.80} & {80.62} & {73.70} & {85.84} & {83.43} & {74.52} & {67.94} & {62.17}\\
Improve(\%) & 1.34\% & 1.49\%  &{1.47\%} & {1.53\%} & {1.80\%} & {1.34\%} & {2.58\%} & {1.63\%} & 4.18\% & 6.95\%  \\ 
\midrule
IDGL & 84.50 & 82.10 &{79.25} & {75.23} & {65.16} & 84.50 &{77.60} & {70.80} & {54.81} & 43.65 \\
IDGL+UnGSL & {84.90} & {83.29} & {80.52} & {77.43} & {66.50} & {84.90} & {79.35} & {72.03} & {56.13} & {44.51} \\
Improve(\%) & 0.47\% & 1.45\% & 1.60\%  & 2.92\%  & 2.04\% & 0.47\%  & 2.26\% & {1.74\%} & {2.41\%} & {1.97\%} \\
\bottomrule
\end{tabular}
}
\end{table}

\begin{table}[!t]
	\centering
        \small
        \setlength{\abovecaptionskip}{-0.05cm}
	\caption{Generalizability of UnGSL with different backbones on Citeseer dataset. 
 The value in bold signifies the top-performing result.}
	\label{tab:generalciteseer}
	\resizebox{\linewidth}{!}
{
\begin{tabular}[ht]{c|cccc}
        \toprule
        {Methods} & SGC & 
        APPNP & GAT & JKNet \\
        \midrule
GRCN    & {72.00±0.00} & {72.73±0.96} & {70.77±1.06} & {71.83±0.72} \\
GRCN+UnGSL & \textbf{72.37±0.06} & \textbf{73.50±0.30} & \textbf{70.88±1.70} & \textbf{72.32±1.75}\\
\midrule
IDGL  & {71.90±0.00} & {72.20±0.75} & {63.83±0.97} & {71.00±1.15} \\
IDGL+UnGSL  & \textbf{73.40±0.00} & \textbf{73.37±0.61} & \textbf{67.00±0.69} & \textbf{72.33±0.9} \\
\bottomrule
\end{tabular}
}
\end{table}

\end{document}
\endinput
%%
%% End of file `sample-sigconf.tex'.
