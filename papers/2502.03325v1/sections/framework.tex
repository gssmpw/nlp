\section{Theoretical Model}
This theoretical model shows that model performance (\( P_{\text{out}} \)) relies not only on basic voltage from model inherent capabilities ($\mathcal{E}_{\text{model}}$) but also on external factors like reasoning difficulty ($R_{CoT}$) and additional voltage from extra demonstrations in ICL ($\mathcal{E}_{ICL}$), which are described as follows in detail:

\subsection{Model Inherent Capabilities as Power Supply}
In our model, we draw an analogy by conceptualizing LLM's inherent capabilities as a power supply that drives computational processes and reasoning tasks (Figure~\ref{fig:main}). The voltage value \( \mathcal{E}_{\text{model}} \) represents the strength of the inherent capability, governing the LLM's capacity to execute a range of tasks. Just as the voltage in an electrical circuit determines the power available to drive various components, the  internal capabilities govern its ability to manage increasingly complex inputs.

\subsection{Faraday's Law of In-Context Learning}

Inspired by \citet{wang-etal-2023-label}, the process of ICL can be conceptualized as the interaction of LLM with contextual data, where it absorbs, retains, and eventually releases semantic information flow. This dynamic flow of information resembles the changing behavior of a magnetic field as its initial strength $\Phi^B_0$ decreases to zero.
Building on this analogy, we propose Faraday's Law for the ICL paradigm, where the rate of change in the ``magnetic field" further induces an extra voltage:
\begin{equation}
    \mathcal{E}_{\text{ICL}} = -\frac{d\Phi^B(t)}{dt} = \lambda\Phi_0^B,
\end{equation}
where $\lambda$ represents the uniform rate of decrease in magnetic field intensity. This perspective enables a deeper quantitative understanding of ICL through the concept of a ``semantic magnetic field," as illustrated in Figure~\ref{fig:main}.


\subsection{Ohm's Law of Chain-of-Thought}
In a parallel vein, \citet{chen2024unlocking} have discovered a combination law for LLM reasoning, which aligns with the series formula of electrical resistors after transformation (See in Appendix~\ref{sec:method}).
Inspired by this, we conceptualize the difficulty of CoT reasoning as analogous to series circuits. In this framework, each reasoning process introduces a distinct resistor ($R_{\text{CoT}}=\sum_i R_i$), compounding the task's overall difficulty ($R_i$). As illustrated in the right panel of Figure~\ref{fig:main}, Ohm's law for CoT can be expressed as:
\begin{equation}
    I_{model} = \frac{\mathcal{E}_{\text{model}} + \mathcal{E}_{\text{ICL}}}{R_{\text{CoT}} + R_0},\label{eq:ohm}
\end{equation}
where $R_0$ represents the resistance of an output resistor with static resistance value in the circuit, which reflects the difficulty of reasoning conclusion.
This formulation quantifies the cumulative effect of reasoning resistors, providing a modular quantitative perspective on cognitive task complexity.



\subsection{Model Performance as Output Power of Bulb}
Extending the conceptual model, we equate model performance to the power output of a light bulb in an electronic circuit. Task execution efficiency depends on a nonlinear interaction of internal and external factors, rather than solely on the model’s intrinsic capabilities. The output power, $P_{\text{out}}$, is expressed as:
\begin{equation}
    P_{\text{out}} = I^2_{model} R_0 =\frac{\left(\mathcal{E}_{\text{model}} + \mathcal{E}_{\text{ICL}}\right)^2 R_0}{(R_{\text{CoT}} + R_0)^2}.\label{eq:power}
\end{equation}
Here, output power represents the model's effective performance, with higher power indicating better task execution, akin to a brighter bulb. The equation highlights that maximizing performance requires enhancing the basic voltage of power supply ($\mathcal{E}_{\text{model}}$) of LLM, minimizing reasoning resistor ($R_{\text{CoT}}$) in CoT, and optimizing the extra voltage ($\mathcal{E}_{\text{ICL}}$) provided by ICL. This synergy explains and predicts the model’s performance on complex tasks quantitatively.