


\subsection{Neuron Activation Visualization Across Reasoning Strategies}
To investigate the internal mechanisms underlying reasoning processes in the LLAMA3-8B model, we conducted a series of experiments focusing on three distinct strategies: Zero-shot CoT, ICL with Semantic Magnetic Field Retrieval, and Program of Thought (PoT). By examining the neuron activation values and frequencies, with a particular emphasis on the most activated attention layer, we aim to uncover key insights into the relationship between reasoning strategies and neural dynamics. These findings further validate the theoretical framework of the Electromagnetic Logical Circuit Model (\modelname{}).
The experiments were performed using the BigGSM dataset, a standard benchmark for mathematical reasoning tasks. For Zero-shot CoT, reasoning pathways were activated without any contextual demonstrations, relying solely on the model’s inherent reasoning capability. Specifically, the phrase "Let's think step-by-step" was appended to the query to stimulate step-by-step reasoning. For ICL, demonstrations were retrieved dynamically using the Balanced General-purpose Embedding (BGE) model, adopting a 3-shot configuration. Demonstrations were selected from the GSM8K dataset to maximize their alignment with the query embedding vector, as determined by the semantic magnetic field strength. Finally, PoT decomposed tasks into intermediate steps to simplify reasoning by minimizing planning resistor.


Figure~\ref{fig:explanation} presents the neuron activation heatmaps for the most activated attention layer under the three reasoning strategies. For Zero-shot CoT, the heatmap reveals a sparse activation pattern, with limited regions of intense activation, reflecting the difficulty of reasoning without contextual support. In contrast, ICL with Semantic Magnetic Field Retrieval exhibits a more distributed activation pattern, suggesting that semantically aligned demonstrations broaden the activation spectrum and stimulate a more extensive set of neurons. For PoT, the heatmap shows highly localized activation peaks, corresponding to the decomposition of reasoning tasks into intermediate steps, which reduces planning resistor and enhances computational efficiency.

The observed differences in neuron activation patterns align closely with the theoretical predictions of the ELCM. Specifically, the sparsity in Zero-shot CoT highlights the higher logical resistor encountered in the absence of contextual demonstrations. The broader activation in ICL demonstrates the efficacy of the semantic magnetic field in reducing resistor and amplifying reasoning current. Meanwhile, the localized activation in PoT confirms the effectiveness of task decomposition in optimizing the planning process and minimizing resistor.

\begin{figure}[t]
    \centering
    \includegraphics[width=\textwidth]{figure/explanation.pdf}
    \caption{Neuron activation heatmaps for the most activated attention layer under three reasoning strategies: (a) Zero-shot CoT, (b) ICL with Semantic Magnetic Field Retrieval, and (c) Program of Thought (PoT).}\label{fig:explanation}
\end{figure}

