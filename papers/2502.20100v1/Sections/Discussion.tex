



\section{Discussion}


The results of the survey shows that the DDPM can generate highly realistic ultrasound images that are hard to distinguish, even for ultrasound experts. Although senior cardiologists could distinguish synthetic images better than random guessing, they were still correct only in 63 7\% of the cases. \newline

Generative AI can create unrealistic and anatomically incorrect images, also known as hallucinations, and the DDPM in this work is no exception. Fig.~\ref{fig: hallucination_repaint} shows an example of a problematic hallucination in which the model creates an additional mitral valve. In this case, the augmentation would add noise to the training data. For segmentation augmentations, the most crucial parts are the parts with the reference segmentation masks, which are original and real. The remaining background region is of less importance and in most cases it is not detrimental if the generated surroundings are not perfectly accurate, although there are exceptions as the example in Fig.~\ref{fig: hallucination_repaint}. Still, the comparison to the baseline augmentations where the surrounding pixels remain black shows that it is still important that the surroundings look realistic. \newline









Our study does not include segmentation of the LA because it is often not fully visible in the original scan. This is especially true for the LV-focused HUNT4 images. When using data augmentation, particularly depth augmentation, the diffusion model can generate parts of the LA that were missing in the original scan. This creates problems because the original labels only correspond to the visible parts of the LA in the original image. Therefore, we restrict ourselves to the LV and myocardium (MYO) in this work. \newline

While the segmentation models predict both LV and MYO, the experiments only evaluate on the LV. The experiments do not evaluate on the MYO because there is a notable difference between the annotation conventions between HUNT4 and CAMUS. The annotations in the CAMUS dataset consistently label the MYO notably thicker than the HUNT4. Fig.~\ref{fig: dataset_differences} shows an example of this. There are also differences in annotation conventions for the LV lumen, but these are less pronounced than for the MYO. \newline

\noindent
The clinical evaluation on HUNT4 showed the generative augmentations lead to narrower limits of agreement with the reference in terms of EF. The reduction in limits of agreement originates from a reduction in segmentation failures (outliers). Fig. \ref{fig: biggest_diff.pdf} shows visualizations of segmentation outputs for HUNT4 study participants where the segmentation models with and without augmentations lead to the largest differences in automatic EF. In the training set of the HUNT4 development set, all views are LV-focused meaning that a shorter scan depth is used so that the LV covers most of the scan sector. LV-focused views are used because this aligns with clinical guidelines, which recommend optimizing the view to ensure the left ventricle is clearly and fully visualized for accurate assessment. However, in practice it can be hard to get a standardized view. Without the generative augmentations, the model overfits on LV-focused views and thus often fails to segment the LV correctly when views are not focused on the LV. This explains why the depth augmentation are the most successful augmentation in the ablation study. \newline





The clinical evaluation shows that the bias changes depending on which dataset the segmentation model was trained on. This bias can be corrected for by measuring its magnitude on a subset of the target domain and adjusting accordingly on new, unseen data. The reason for the change of bias can be the data distribution in the training set, the annotation conventions, or the methodology of the tools used for manual measurement \cite{olaisen2024automatic}. An in-depth analysis of the bias is out of scope for this work. \newline
%The manual reference EF values are not a ground truth and there is typically large inter-observer variability in clinical measurements.


Both the ablation study and the clinical evaluation on HUNT4 show that the model trained on HUNT4 benefits the most from the generative augmentations. 
The HUNT4 dataset is more standardized, contains recordings from mostly healthy volunteers and contains less variation than the CAMUS dataset. Thus, the increase in variation from the generative augmentations mostly benefits this dataset. However, also the segmentation model trained and tested on CAMUS shows small, but statistically significant ($p<0.05$), improvement using the Wilcoxon signed-rank test \cite{woolson2007wilcoxon}. \newline

The proposed generative augmentations improve the variation in terms of acquisition, positioning and size of the LV in the image, but do not diversify in terms of shape of the heart itself. In practice, these two types of variation might be correlated, as more diversity in patients would naturally lead to more variation in scan sectors. \newline



%The clinical evaluation showed that without generative augmentations, the segmentation model trained on the CAMUS dataset results in narrower limits of agreement for EF estimation than the model trained on HUNT4. This is somewhat surprising, since the annotation conventions are different than HUNT4. This shows the importance of variation in the training dataset for deep learning in cardiac ultrasound, where typically relatively few annotations are available. By adding generative augmentations, the limits of agreement of the model trained on HUNT4 are on par with the CAMUS model. \newline

% add figure like bottom of Improved Denoising Diffusion Probabilistic Models with lots of examples





The study only looks into generative augmentations for segmentation on cardiac images, but there is nothing preventing similar generative augmentation methods to be applied to other tasks or other imaging modalities. Of course, the type of generative augmentations in this work are tailored to cardiac ultrasound, but the concept of generative augmentations itself is flexible. Any segmentation task for which enriching the positioning of the reference masks can not be achieved realistically with regular augmentations could use the proposed approach. \newline




