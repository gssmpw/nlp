\section{Materials and Methods}

\subsection{Datasets}\label{section: Datasets}


The \textbf{HUNT4Echo dataset }\cite{Olaisen2023-fy}, part of the Helse Undersøkelsen i Nord-Trøndelag study, is an ultrasound dataset of 2,211 volunteers of LV-focused apical 2-chamber (A2C) and apical 4-chamber (A4C) views, acquired using a GE Vivid E95 scanner. Each recording includes three cardiac cycles. \newline 

The \textbf{model development set} is a subset that includes single-frame segmentation annotations for both end-diastole (ED) and end-systole (ES), providing pixel-level labels for the left ventricle (LV), left atrium (LA), and myocardium (MYO). The model development set consists of 1058 annotated ED and ES frames of 529 recordings from 311 patients. \newline

The \textbf{ejection fraction set} is a subset disjunct with the model development set of 1900 patients with reference biplane LV volumes in ED and ES. The volumes were obtained following current clinical guidelines by manual tracing and using Simpson's method of discs in the clinically approved EchoPAC software from GE HealthCare on the HUNT4 recordings. \newline 
% The patients in the ejection fraction set are different from the patients in the model development set. \newline



The \textbf{CAMUS dataset} \cite{leclerc2019deep} is a publicly available dataset containing single cycle recordings from 500 patients, acquired using a GE Vivid E95 scanner (GE Vingmed Ultrasound AS, Norway). The dataset contains one A2C and one A4C recording for each patient and annotations for both the ED and ES frame in each recording, resulting in a total of 2000 image-annotation pairs. Like the HUNT4 dataset, the annotations are pixel-level LV, LA, MYO. The training, validation and test set contain 400, 50 and 50 patients respectively. \newline





\begin{figure}
     \centering
          \begin{subfigure}[b]{0.49\linewidth}
         \centering \includegraphics[trim={0cm 0cm 0cm 0cm}, clip, width=\linewidth]{figures/CAMUS_heatmap.png}
         \caption{CAMUS \newline}
     \end{subfigure}
     \begin{subfigure}[b]{0.49\linewidth}
         \centering \includegraphics[trim={0cm 0cm 0cm 0cm}, clip, width=\linewidth]{figures/HUNT4_heatmap.png}
         \caption{HUNT4 model development set}
     \end{subfigure}
     \begin{subfigure}[b]{0.49\linewidth}
         \centering \includegraphics[trim={0cm 0cm 0cm 0cm}, clip, width=\linewidth]{figures/Augmented_CAMUS_heatmap.png}
         \caption{CAMUS with generative augmentations \newline}
     \end{subfigure}
     \begin{subfigure}[b]{0.49\linewidth}
         \centering \includegraphics[trim={0cm 0cm 0cm 0cm}, clip, width=\linewidth]{figures/Augmented_HUNT4_heatmap.png}
         \caption{HUNT4 model development set with generative augmentations}
     \end{subfigure}
   \caption{Heatmaps of pixels belonging to the LV after resizing to $256\times256$. This illustrates the difference in scan depth variation and LV positioning in the two datasets, before and after generative augmentations.}
    \label{fig: heatmaps}
\end{figure}

The HUNT4 model development set and the CAMUS dataset should be combined with care due to differences in annotation conventions. For example, the myocardium is consistently annotated as significantly thicker in CAMUS compared to HUNT4. Fig.~\ref{fig: dataset_differences} illustrates this.
There is also a difference in the way the LV is annotated. This is noticeable in the reduced Dice scores in the results when HUNT4 is used for training and CAMUS is used for testing and vica versa, see Table \ref{table: ablation_study_1}).
The segmentation models in this work only segment the LV and MYO labels and the experiments only evaluate on the LV. We elaborate on this choice in the Discussion.
\newline

\begin{figure}
     \centering
     \begin{subfigure}[b]{0.49\linewidth}
         \centering \includegraphics[trim={0cm 0cm 0cm 0cm}, clip, width=\linewidth]{figures/example_camus_iso.png}
         \caption{CAMUS}
     \end{subfigure}
     \begin{subfigure}[b]{0.49\linewidth}
         \centering \includegraphics[trim={0cm 0cm 0cm 0cm}, clip, width=\linewidth]{figures/example_hunt4_iso.png}
         \caption{HUNT4}
     \end{subfigure}
   \caption{Example frames from the CAMUS and HUNT4 datasets. The CAMUS and HUNT4 dataset contain the same cardiac views, but the frames in HUNT4 are consistently LV-focused, while those in CAMUS are not. The annotations conventions are also different in both datasets, which can be seen clearly in the thickness of the annotated myocardium (blue).}
    \label{fig: dataset_differences}
\end{figure}


\begin{figure}[h]
\centering
  \centering
  \includegraphics[trim={0cm 0cm 0cm 0 cm}, clip,width = 1\linewidth]{figures/depths_hist_norm.png}
  \caption{Distribution of imaging depths in CAMUS and HUNT4.}
  \label{fig: depths_hist}
\end{figure}

\begin{figure}[h]
\centering
  \centering
  \includegraphics[trim={0cm 0cm 0cm 0 cm}, clip,width = 1\linewidth]{figures/sector_angles_hist_norm.png}
  \caption{Distribution of sector angles in CAMUS and HUNT4.}
  \label{fig: sector_angles_hist}
\end{figure}


\begin{figure}[h]
\centering
  \centering
  \includegraphics[trim={0cm 0cm 0cm 0 cm}, clip,width = 1\linewidth]{figures/ef_hist_norm.png}
  \caption{Distribution of EF in CAMUS and HUNT4.}
  \label{fig: ef_hist}
\end{figure}

The recordings in the HUNT4 study follow the clinical guidelines for quantitative measurements and  maximize the area of LV in the image by adjusting the depth \cite{lang2015recommendations}. This is not the case for CAMUS, resulting in less standardized images. The top part of Fig.~ \ref{fig: heatmaps} shows the heatmaps of pixels labeled as LV in both datasets, illustrating that the LV occupies a larger portion of the image in HUNT4 compared to CAMUS. This is a consequence of the distribution of acquisition depth and width, shown in Figs.~\ref{fig: depths_hist} and \ref{fig: sector_angles_hist}. The CAMUS dataset also has a larger variety in terms of EF, as shown in Fig.~\ref{fig: ef_hist}.










