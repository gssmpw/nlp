\section{Conclusion}

This work explores using generative augmentations for cardiac ultrasound segmentation. Our results show that diffusion models can generate highly realistic cardiac ultrasound images indistinguishable from real images by experts. We show how these generative models can be used to improve segmentation model accuracy and generalizability through generative augmentations. \newline

The proposed generative augmentations are most useful for datasets with limited variation in terns of acquisition and positioning of the left ventricle in the image. This is relevant in the medical domain, as datasets are often limited according to the acquisition protocol and preference of the personnel at a given center. \newline

Although this work only studies segmentation for cardiac ultrasound, the concept of generative augmentations could be generalized to other tasks or imaging modalities.


