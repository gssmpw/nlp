\section{Results}


\subsection{Results of segmentation ablation study}
\label{subsection: results of segmentation ablation study}

Table \ref{table: ablation_study_1} shows the results of the ablation study on the CAMUS dataset, using Dice score and Hausdorff distance as metrics.


\begin{table}
\scriptsize
  \centering
  \renewcommand{\arraystretch}{1.1} % Increase vertical spacing
  \caption{Segmentation results on the CAMUS dataset. The Dice score and Hausdorff distance are only for the LV lumen label.}
  \begin{tabular}{m{60pt}m{100pt}m{100pt}m{100pt}}
    \toprule
      Training dataset   & Augmentation & Dice score & Hausdorff distance in mm\\
    \midrule
    \multirow{7}{1.4cm}{HUNT4} & None & 0.746 $\pm$ 0.19 & 14.30 $\pm$ 11.2\\
    & Depth increase & \textbf{0.895} $\pm$ \textbf{0.06} & \textbf{3.29} $\pm$ \textbf{3.47}\\
    & Rotation & 0.800 $\pm$ 0.18 & 7.70 $\pm$ 7.89\\
    & Sector width adjustments & 0.821 $\pm$  0.14 & 10.44 $\pm$ 9.84\\
    & Translation & 0.822 $\pm$ 0.17 & 7.08 $\pm$ 7.36\\
    & Combination & 0.892 $\pm$ 0.07 & 3.39 $\pm$ 2.52\\
    & Combination w/o RePaint & 0.768 $\pm$ 0.19 & 12.86 $\pm$ 11.09\\
    \midrule
    \multirow{7}{1.4cm}{CAMUS} & None & 0.930 $\pm$ 0.06 & 1.82 $\pm$ 1.69\\
    & Depth increase & 0.936 $\pm$ 0.04 & 1.63 $\pm$ 1.15 \\
    & Rotation & 0.934 $\pm$ 0.05 & 1.76 $\pm$ 1.61\\
    & Sector width adjustments & 0.936 $\pm$ 0.04 & 1.77 $\pm$ 1.64\\
    & Translation & 0.934 $\pm$ 0.04 & 1.70 +- 1.21\\
    & Combination & & \\
    & Combination w/o RePaint &  0.920 $\pm$ 0.08 & 2.20 $\pm$ 3.41 \\
    \bottomrule
  \end{tabular}
      \label{table: ablation_study_1}
\end{table}


\subsection{Results of clinical evaluation on HUNT4}

Fig.~\ref{fig: ef} shows the Bland-Altman plots comparing the automatic EF measurements with the manual reference. The plot shows that the segmentation models trained with the generative augmentations lead to narrower limits of agreement with the reference. The bias also changes depending on which dataset the segmentation model was trained on. This can be related to the data distribution in the training set or the annotating conventions and is out of scope for this work.


\begin{figure*}
     \centering
     \begin{subfigure}{0.4\linewidth}
         \centering
         \includegraphics[trim={0.9cm 1.6cm 0.4cm 2cm},clip,width=\textwidth]{figures/bland_altman_hunt4_on_hunt4.pdf}
         \caption{HUNT4 without augmentations}
     \end{subfigure}
     \hspace{1cm}
     \begin{subfigure}[b]{0.4\textwidth}
         \centering
         \includegraphics[trim={0.9cm 1.6cm 0.4cm 2cm},clip, width=\textwidth]{figures/bland_altman_hunt4_all_on_hunt4.pdf}
         \caption{HUNT4 with augmentations}
     \end{subfigure}
    \hfill
        \caption{Bland–Altman plots comparing the manual reference with automatic EF measurements obtained via segmentation trained with and without generative augmentations. The out-of-distribution exams are the exams where at least one frame used in the calculation is out-of-distribution for HUNT4 (depth $>150$mm or sector angle $>70^\circ$).}
        \label{fig: ef}
\end{figure*}