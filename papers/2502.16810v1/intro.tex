\section{Introduction}

While large language models (LLMs) have made significant strides across various tasks, their ability to \emph{persuade} remains an underexplored frontier (see a discussion of related work in Section~\ref{sec:related}). 
This however is a particularly important capability since  persuasion-related economic activities --- a common thread in almost all voluntary transactions from advertising and lobbying to litigation and negotiation --- underpin roughly 30\% of the US GDP~\citep{antioch2013persuasion, mccloskey1995one}, hence gives rise to tremendous opportunity for applying LLMs   
 across a wide range of sectors. Meanwhile, this same potential introduces serious trustworthiness concerns. If LLMs can generate persuasive content at scale, their influence on human opinions raises risks of misinformation, manipulation and misuse, especially in sensitive domains such as political campaigns~\citep{voelkel2023artificial, goldstein2024persuasive}.

In addition to these profound economic and societal applications, the relationship between the nature of intelligence and persuasion has been a fundamental research question since the time of early Greek philosophy. Aristotle viewed persuasion as both an art and an expression of intelligence, rooted in the ability to reason and communicate effectively. Yet, the Greeks also cautioned against the sophistry that overly relies on rhetorical and emotional techniques divorced from the truth. This tension becomes especially relevant today with the rise of generative AI.
Notable figures, such as Sam Altman, have predicted that AI systems could achieve superhuman persuasion without superhuman general intelligence~\citep{altman_2023}. This dichotomy raises several crucial questions for LLM research: How can we reliably and consistently measure fact-based persuasiveness? Does greater intelligence inherently lead to stronger persuasive capabilities? And if not, what specific abilities must LLMs develop to truly master persuasion? 
Surprisingly little is known about these questions, and this is what we embark on in this paper.
\begin{quote}
    \textit{``The faculty of observing, in any given case, the available means of persuasion.''} 
    \newline
    \mbox{}\hfill --- Aristotle, \textit{Rhetoric}. 
\end{quote}
In this paper, we study language generation for \emph{grounded persuasion}  --- a particular form of persuasion, inspired by Aristotle’s philosophy, that is grounded in fact, tailored to the audience, and adapted to contextual factors. Grounded persuasion is crucial for applications in marketing and advertising, and its effectiveness can be linked directly to measurable behavioral changes (e.g., in terms of engagement and conversions) while constrained by factual accuracy.
We choose real estate marketing as our testbed for grounded persuasion and construct a realistic evaluation environment to include the process of preference elicitation and measure the persuasiveness of preference-based generation. To enable grounded persuasion in this context, we design an LLM-based agent with three key modules: a Grounding Module, which mimics human expertise in signaling critical and credible selling points; a Personalization Module, which tailors content to user preferences; and a Marketing Module, which ensures factual accuracy and integrates localized features. We use this model-based approach to back up LLMs'  generation and are able to achieve superhuman persuasion in real estate marketing. Our findings lay the groundwork for leveraging LLMs in large-scale, targeted marketing and beyond, offering a scalable solution to complex persuasion tasks in real-world applications.

\textbf{Our Contribution}\quad  Our primary research objective is to demonstrate a path towards effective design of persuasive language agents with a backbone of more principled theory. Towards this end, we choose to focus on an application of automated marketing, where we employ the economic theory of strategic communication games to guide the agentic process ranging from processing products' (factual) raw attributes, to selecting features to highlight, and ultimately to generate human-like marketing texts. We start from developing a micro-economic framework for automated marketing, and then operationalize this framework  by leveraging the capabilities of LLMs. In addition, we set up a realistic evaluation process of the persuasiveness capability. 
This includes building a large real estate dataset from Zillow, designing a dedicated survey website to mimic the house search process and test with a focused group of potential home buyers. Our experiments suggest marketing descriptions generated by our methods have a clear winning edge of 70\% over those written by expert human realtors. 

