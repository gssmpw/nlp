\section{Related Prior Work and Motivation}
\vspace{-0.1cm}
\subsection{Prior Fault Attacks on Multi-Tenant FPGAs}
\vspace{-0.1cm}

In FPGAs, PRRs are designed to maintain logical isolation to ensure secure operation for co-tenant modules \cite{4223233}.
However, recent work explores the vulnerabilities of multi-tenant FPGAs to remotely induced fault attacks; these attacks do not require physical access to the FPGA. An adversary can implement power-wasters on the FPGA to exploit the power distribution network (PDN), which supplies power to all the tenant modules that are configured on the FPGA. This results in rapid voltage fluctuations on the shared PDN.

In \cite{8056840}, authors implemented ROs for generating abrupt voltage drops and showed their effectiveness in DoS and crashing of the FPGA. Extending upon this approach, \cite{FPGAhammer} explored the potential of ROs to induce voltage drop-based fault-injections in an AES module. By selecting the appropriate RO toggling frequency, faults can be injected in AES, resulting in incorrect ciphertext generation or key recovery. \cite{Sugawara2019OscillatorCentre, provelengios2020power, Krautter2019MitigatingCloud} explored non-combinational ROs and glitch amplification circuits that evade the Amazon Web Services design rule check (DRC).
In \cite{8844478}, a runtime fault attack was demonstrated that did not require combinational loops or glitch-inducing designs. Instead, the attack exploited short circuits within the FPGA to force simultaneous writes to the same memory address in a dual-port RAM. These memory collisions caused permanent alterations in the bitstream configuration. In contrast to the aforementioned attacks that are executed at runtime, \cite{chaudhuri2024hackingfabrictargetingpartial} explores a fault attack that is carried out during the process of bitstream reconfiguration. In this approach, faults are injected directly into the bitstream, which persist even after the bitstream is configured onto the FPGA.

\vspace{-0.1cm}
\subsection{Fault Attack Countermeasures}
\vspace{-0.1cm}
Several countermeasures in the literature aim to detect fault attacks that employ power-wasters; these countermeasures are focused on identifying runtime fault attacks. In \cite{9643485}, fault attacks are detected using voltage drop-based sensors. This technique aims at disabling the interconnects of the attacker module, blocking it from FPGA configuration. Another fault-monitoring scheme, presented in \cite{10.1145/3451236}, disables the clock signal of the fault-impacted tenant to halt the activity of power-wasters. 
These methods rely on power-wasters being continuously activated for a significantly long duration to detect fault attacks. Note that all prior runtime fault attacks \cite{8056840, FPGAhammer, 7809042, 8844478} persist throughout the victim module's operation. Such prolonged attack durations increase the likelihood of detection by the techniques presented in \cite{9643485, 10.1145/3451236}. 

In contrast, FLARE is stealthier and faster, with a higher likelihood of evading known countermeasures. FLARE minimizes the attack duration by precisely activating the power-wasters only when the bitstream configuration address is being loaded into the RM. Unlike \cite{chaudhuri2024hackingfabrictargetingpartial}, which is detected by CRC, FLARE evades CRC detection successfully. 
Table \ref{comp} presents a qualitative comparison of FLARE with prior fault attacks.

% \subsection{Comparison of FLARE with State-of-the-art Fault Attacks}

% Similar to \cite{chaudhuri2024hackingfabrictargetingpartial}, FLARE activates power-wasters only during the partial reconfiguration process of bitstreams into the RM. However, FLARE targets only the configuration address of the bitstream for fault-injection, specifically the configuration address. This targeted approach allows FLARE to be significantly faster than runtime fault attacks and the method in \cite{chaudhuri2024hackingfabrictargetingpartial}, which requires the power-wasters to remain active for the duration of the entire bitstream loading process to the RM. Moreover, FLARE alters the configuration address of the user bitstream, impacting multiple co-tenant modules on the FPGA, a critical vulnerability that was not addressed in prior fault attack schemes \cite{8056840} \cite{8844478} \cite{FPGAhammer} \cite{chaudhuri2024hackingfabrictargetingpartial} \cite{provelengios2020power}.



\begin{table}[t]
\centering
% \begin{threeparttable}
%r2d2 cap attack current sensing
\fontsize{7.3}{7.3}\selectfont

\caption{Comparison of FLARE with prior fault attacks on multi-tenant FPGAs (Note: Few $\mu$s (ms) refers to a duration of a few microseconds (milliseconds)).}

    \label{compare}
    \begin{tabular}{|c| c| c| c| c|c|} 
    \hline
% \multirow{2}{*}{$K$}&
{Attack} &Target & Loop-free  &Attack &Address&Tenant \\
method&module&oscillator &duration&altered?&impacted \\
&&evaluated?&&& \\
\hline
\cite{8844478}&RAM&\xmark&Continuous&\xmark&Single \\

\cite{FPGAhammer}&AES&\xmark&Continuous&\xmark&Single \\

\cite{8056840}&FPGA&\xmark&Continuous&\xmark&Single \\

\cite{7809042}&Bitstream&\xmark&Continuous&\xmark&Single\\

\cite{chaudhuri2024hackingfabrictargetingpartial}&Bitstream&\xmark&Few ms&\xmark&Single \\

\cite{provelengios2020power}&RSA&\xmark&Continuous&\xmark&Single \\

\cite{Krautter2019MitigatingCloud} &FPGA&\checkmark&Continuous&\xmark&Single \\
\multirow{2}{*}{\textbf{FLARE}} &Bitstream, &\multirow{2}{*}{\checkmark}&\multirow{2}{*}{Few $\mu$s}&\multirow{2}{*}{\checkmark}&\multirow{2}{*}{Multiple} \\
&FPGA PRR(s)&&&& \\



\hline
\end{tabular}

\vspace{-0.4cm}

\label{comp}
\end{table}



