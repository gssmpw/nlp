\documentclass[a4paper,conference]{IEEEtran}
\IEEEoverridecommandlockouts
\usepackage{subcaption}
\usepackage{fancyhdr,setspace}
\usepackage{comment}
\usepackage{caption}
\usepackage{todonotes}
\captionsetup[figure]{font=normalsize,labelfont=normalsize}
% \usepackage{caption}
\captionsetup{skip=0pt}
\usepackage{bm}
%\usepackage{floatrow}
\renewcommand{\footnoterule}{}

% \fancyhf{}
% \fancyfoot[L]{\large Regular Paper}
% \fancyfoot[C]{\large INTERNATIONAL TEST CONFERENCE}
% \fancyfoot[C]{\sffamily\fontsize{9pt}{9pt}\selectfont\thepage}
% \fancyfoot[R]{\large \thepage}

\usepackage[compact]{titlesec} 
% Some Computer Society conferences also require the compsoc mode option,
% but others use the standard conference format.
%
% If IEEEtran.cls has not been installed into the LaTeX system files,
% manually specify the path to it like:
% \documentclass[cgggffbconference]{../sty/IEEEtran}
\usepackage{cite}
\usepackage{xcolor}
\usepackage{blindtext, graphicx, tabularx, multicol, lipsum, subfig,eqnarray,amsmath,chemformula,multirow,threeparttable}
%\usepackage{biblatex}
\usepackage{enumitem}
\usepackage[utf8]{inputenc}
\usepackage{amssymb,amsmath}

\usepackage{float}
\DeclareMathOperator{\arccosh}{arcCosh}
\usepackage[boxed]{algorithm2e}
\usepackage[noend]{algpseudocode}
\usepackage[compact]{titlesec}
\algnewcommand{\IfThenElse}[3]{% \IfThenElse{<if>}{<then>}{<else>}
  \State \algorithmicif\ #1\ \algorithmicthen\ #2\ \algorithmicelse\ #3}
\makeatletter
\newcommand{\removelatexerror}{\let\@latex@error\@gobble}
\makeatother
%footnote comand
\makeatletter
\renewcommand\footnoterule{%
  \kern-3\p@
  \hrule\@width\columnwidth
  \kern2.6\p@}
  \makeatother
%\usepackage{algorithmic}
\usepackage{letltxmacro}


% Theorems ...
\usepackage{amsthm}
\newtheorem{problem}{Problem}
\newtheorem{theorem}{Theorem}
\newtheorem{corollary}{Corollary}
\usepackage{amsmath}
%\usepackage{mathpazo}
%\usepackage[ruled,vlined]{algorithm2e}
\newtheorem{lemma}{Lemma}
\SetKwInput{KwInput}{Input}                % Set the Input
\SetKwInput{KwOutput}{Output}  
\DeclareMathOperator*{\E}{\mathbb{E}}
%]==============================================
\usepackage{tikz}
\usepackage{capt-of}
\usepackage{amssymb} \usepackage{booktabs} 
\usepackage{pifont}% http://ctan.org/pkg/pifont
\newcommand{\cmark}{\ding{51}}%
\newcommand{\xmark}{\ding{55}}%
\newcommand{\rpm}{\raisebox{.2ex}{$\scriptstyle\pm$}}

% \usepackage{algorithmic}
\DeclareMathOperator*{\argmax}{arg\,max}
\usepackage{diagbox}
\usepackage{xcolor,etoolbox}
\usepackage{dblfloatfix}
% \let\mybibitem\bibitem
% \renewcommand{\bibitem}[1]{%
% \ifstrequal{#1}{edgeTPU}{\color{red}\mybibitem{#1}}}

\let\mybibitem\bibitem
\renewcommand{\bibitem}[1]{%
\ifstrequal{#1}{edgeTPU}{\color{black}\mybibitem{#1}}
{\ifstrequal{#1}{xyz}{\color{blue}\mybibitem{#1}}
{\color{black}\mybibitem{#1}}}%
}

% Some very useful LaTeX packages include:
% (uncomment the ones you want to load)

 \pagestyle{plain}


%----------------------- This is special for ITC --------------------------




\setlength{\parskip}{0pt}

% *** GRAPHICS RELATED PACKAGES ***
%
\ifCLASSINFOpdf
 
\else

\fi


%\usepackage[ruled,vlined]{algorithm2e}
\usepackage{comment}



\begin{document}

% Woohoo!!Paper is accepted!!
%

\title{FLARE: \underline{F}ault Attack \underline{L}everaging \underline{A}ddress \underline{R}econfiguration \underline{E}xploits in Multi-Tenant FPGAs$^*$\thanks{$^*$The work of J. Chaudhuri and K. Chakrabarty was supported in part by the National Science Foundation under grant no. CNS-2011561. The work of Hassan Nassar was supported in part by the German Federal Ministry of Education and Research (BMBF) through grant 01IS23066 as part of the Software Campus Project ``HE-Trust'' and in part by the ``Helmholtz Pilot Program for Core Informatics (kikit)'' at KIT. The work of Mehdi Tahoori was supported by German Research Foundation (DFG) projects SAUBER and SecFShare.}}


% author names and affiliations
% use a multiple column layout for up to three different
% affiliations
\author{\IEEEauthorblockN{Jayeeta Chaudhuri$^\dagger{}$, Hassan Nassar$^{\ddagger}{}$, Dennis R.E. Gnad$^{\ddagger}{}$, Jörg Henkel$^{\ddagger}{}$, \\
Mehdi B. Tahoori$^{\ddagger}{}$,  and Krishnendu Chakrabarty$^\dagger{}$}
\IEEEauthorblockA{$^\dagger{}$School of Electrical, Computer, and Energy Engineering, Arizona State University, Tempe, AZ, USA \\
$^\ddagger{}$Institute of Computer Engineering, Karlsruhe Institute of Technology (KIT), Karlsruhe, Germany
\\
}
\\[-5.0ex]
}






\maketitle

\begin{abstract}

 
Modern FPGAs are increasingly supporting multi-tenancy to enable dynamic reconfiguration of user modules. While multi-tenant FPGAs improve utilization and flexibility, this paradigm introduces critical security threats. In this paper, we present \textbf{FLARE}, a fault attack that exploits vulnerabilities in the partial reconfiguration process, specifically while a user bitstream is being uploaded to the FPGA by a reconfiguration manager. Unlike traditional fault attacks that operate during module runtime, FLARE injects faults in the bitstream during its reconfiguration, altering the configuration address and redirecting it to unintended partial reconfigurable regions (PRRs). This enables the overwriting of pre-configured co-tenant modules, disrupting their functionality. FLARE leverages power-wasters that activate briefly during the reconfiguration process, making the attack stealthy and more challenging to detect with existing countermeasures. Experimental results on a Xilinx Pynq FPGA demonstrate the effectiveness of FLARE in compromising multiple user bitstreams during the reconfiguration process.
\end{abstract}


% \thispagestyle{fancy}
% \fancyhead{}
% \renewcommand{\headrulewidth}{0pt}
% \fancyhf{}
% \fancyfoot[L]{\large Regular Paper}
% \\ 978-1-7281-4823-6/19/\$31.00 \copyright2019 IEEE}
% \fancyfoot[C]{\large INTERNATIONAL TEST CONFERENCE}
% \fancyfoot[R]{\large \thepage}

%----------------------- This is special for ITC --------------------------
\thispagestyle{plain}
% For peer review papers, you can put extra information on the cover
% page as needed:
% \ifCLASSOPTIONpeerreview
% \begin{center} \bfseries EDICS Category: 3-BBND \end{center}
% \fi
%
% For peerreview papers, this IEEEtran command inserts a page break and
% creates the second title. It will be ignored for other modes.
\IEEEpeerreviewmaketitle

\section{Introduction}

Large language models (LLMs) have achieved remarkable success in automated math problem solving, particularly through code-generation capabilities integrated with proof assistants~\citep{lean,isabelle,POT,autoformalization,MATH}. Although LLMs excel at generating solution steps and correct answers in algebra and calculus~\citep{math_solving}, their unimodal nature limits performance in plane geometry, where solution depends on both diagram and text~\citep{math_solving}. 

Specialized vision-language models (VLMs) have accordingly been developed for plane geometry problem solving (PGPS)~\citep{geoqa,unigeo,intergps,pgps,GOLD,LANS,geox}. Yet, it remains unclear whether these models genuinely leverage diagrams or rely almost exclusively on textual features. This ambiguity arises because existing PGPS datasets typically embed sufficient geometric details within problem statements, potentially making the vision encoder unnecessary~\citep{GOLD}. \cref{fig:pgps_examples} illustrates example questions from GeoQA and PGPS9K, where solutions can be derived without referencing the diagrams.

\begin{figure}
    \centering
    \begin{subfigure}[t]{.49\linewidth}
        \centering
        \includegraphics[width=\linewidth]{latex/figures/images/geoqa_example.pdf}
        \caption{GeoQA}
        \label{fig:geoqa_example}
    \end{subfigure}
    \begin{subfigure}[t]{.48\linewidth}
        \centering
        \includegraphics[width=\linewidth]{latex/figures/images/pgps_example.pdf}
        \caption{PGPS9K}
        \label{fig:pgps9k_example}
    \end{subfigure}
    \caption{
    Examples of diagram-caption pairs and their solution steps written in formal languages from GeoQA and PGPS9k datasets. In the problem description, the visual geometric premises and numerical variables are highlighted in green and red, respectively. A significant difference in the style of the diagram and formal language can be observable. %, along with the differences in formal languages supported by the corresponding datasets.
    \label{fig:pgps_examples}
    }
\end{figure}



We propose a new benchmark created via a synthetic data engine, which systematically evaluates the ability of VLM vision encoders to recognize geometric premises. Our empirical findings reveal that previously suggested self-supervised learning (SSL) approaches, e.g., vector quantized variataional auto-encoder (VQ-VAE)~\citep{unimath} and masked auto-encoder (MAE)~\citep{scagps,geox}, and widely adopted encoders, e.g., OpenCLIP~\citep{clip} and DinoV2~\citep{dinov2}, struggle to detect geometric features such as perpendicularity and degrees. 

To this end, we propose \geoclip{}, a model pre-trained on a large corpus of synthetic diagram–caption pairs. By varying diagram styles (e.g., color, font size, resolution, line width), \geoclip{} learns robust geometric representations and outperforms prior SSL-based methods on our benchmark. Building on \geoclip{}, we introduce a few-shot domain adaptation technique that efficiently transfers the recognition ability to real-world diagrams. We further combine this domain-adapted GeoCLIP with an LLM, forming a domain-agnostic VLM for solving PGPS tasks in MathVerse~\citep{mathverse}. 
%To accommodate diverse diagram styles and solution formats, we unify the solution program languages across multiple PGPS datasets, ensuring comprehensive evaluation. 

In our experiments on MathVerse~\citep{mathverse}, which encompasses diverse plane geometry tasks and diagram styles, our VLM with a domain-adapted \geoclip{} consistently outperforms both task-specific PGPS models and generalist VLMs. 
% In particular, it achieves higher accuracy on tasks requiring geometric-feature recognition, even when critical numerical measurements are moved from text to diagrams. 
Ablation studies confirm the effectiveness of our domain adaptation strategy, showing improvements in optical character recognition (OCR)-based tasks and robust diagram embeddings across different styles. 
% By unifying the solution program languages of existing datasets and incorporating OCR capability, we enable a single VLM, named \geovlm{}, to handle a broad class of plane geometry problems.

% Contributions
We summarize the contributions as follows:
We propose a novel benchmark for systematically assessing how well vision encoders recognize geometric premises in plane geometry diagrams~(\cref{sec:visual_feature}); We introduce \geoclip{}, a vision encoder capable of accurately detecting visual geometric premises~(\cref{sec:geoclip}), and a few-shot domain adaptation technique that efficiently transfers this capability across different diagram styles (\cref{sec:domain_adaptation});
We show that our VLM, incorporating domain-adapted GeoCLIP, surpasses existing specialized PGPS VLMs and generalist VLMs on the MathVerse benchmark~(\cref{sec:experiments}) and effectively interprets diverse diagram styles~(\cref{sec:abl}).

\iffalse
\begin{itemize}
    \item We propose a novel benchmark for systematically assessing how well vision encoders recognize geometric premises, e.g., perpendicularity and angle measures, in plane geometry diagrams.
	\item We introduce \geoclip{}, a vision encoder capable of accurately detecting visual geometric premises, and a few-shot domain adaptation technique that efficiently transfers this capability across different diagram styles.
	\item We show that our final VLM, incorporating GeoCLIP-DA, effectively interprets diverse diagram styles and achieves state-of-the-art performance on the MathVerse benchmark, surpassing existing specialized PGPS models and generalist VLM models.
\end{itemize}
\fi

\iffalse

Large language models (LLMs) have made significant strides in automated math word problem solving. In particular, their code-generation capabilities combined with proof assistants~\citep{lean,isabelle} help minimize computational errors~\citep{POT}, improve solution precision~\citep{autoformalization}, and offer rigorous feedback and evaluation~\citep{MATH}. Although LLMs excel in generating solution steps and correct answers for algebra and calculus~\citep{math_solving}, their uni-modal nature limits performance in domains like plane geometry, where both diagrams and text are vital.

Plane geometry problem solving (PGPS) tasks typically include diagrams and textual descriptions, requiring solvers to interpret premises from both sources. To facilitate automated solutions for these problems, several studies have introduced formal languages tailored for plane geometry to represent solution steps as a program with training datasets composed of diagrams, textual descriptions, and solution programs~\citep{geoqa,unigeo,intergps,pgps}. Building on these datasets, a number of PGPS specialized vision-language models (VLMs) have been developed so far~\citep{GOLD, LANS, geox}.

Most existing VLMs, however, fail to use diagrams when solving geometry problems. Well-known PGPS datasets such as GeoQA~\citep{geoqa}, UniGeo~\citep{unigeo}, and PGPS9K~\citep{pgps}, can be solved without accessing diagrams, as their problem descriptions often contain all geometric information. \cref{fig:pgps_examples} shows an example from GeoQA and PGPS9K datasets, where one can deduce the solution steps without knowing the diagrams. 
As a result, models trained on these datasets rely almost exclusively on textual information, leaving the vision encoder under-utilized~\citep{GOLD}. 
Consequently, the VLMs trained on these datasets cannot solve the plane geometry problem when necessary geometric properties or relations are excluded from the problem statement.

Some studies seek to enhance the recognition of geometric premises from a diagram by directly predicting the premises from the diagram~\citep{GOLD, intergps} or as an auxiliary task for vision encoders~\citep{geoqa,geoqa-plus}. However, these approaches remain highly domain-specific because the labels for training are difficult to obtain, thus limiting generalization across different domains. While self-supervised learning (SSL) methods that depend exclusively on geometric diagrams, e.g., vector quantized variational auto-encoder (VQ-VAE)~\citep{unimath} and masked auto-encoder (MAE)~\citep{scagps,geox}, have also been explored, the effectiveness of the SSL approaches on recognizing geometric features has not been thoroughly investigated.

We introduce a benchmark constructed with a synthetic data engine to evaluate the effectiveness of SSL approaches in recognizing geometric premises from diagrams. Our empirical results with the proposed benchmark show that the vision encoders trained with SSL methods fail to capture visual \geofeat{}s such as perpendicularity between two lines and angle measure.
Furthermore, we find that the pre-trained vision encoders often used in general-purpose VLMs, e.g., OpenCLIP~\citep{clip} and DinoV2~\citep{dinov2}, fail to recognize geometric premises from diagrams.

To improve the vision encoder for PGPS, we propose \geoclip{}, a model trained with a massive amount of diagram-caption pairs.
Since the amount of diagram-caption pairs in existing benchmarks is often limited, we develop a plane diagram generator that can randomly sample plane geometry problems with the help of existing proof assistant~\citep{alphageometry}.
To make \geoclip{} robust against different styles, we vary the visual properties of diagrams, such as color, font size, resolution, and line width.
We show that \geoclip{} performs better than the other SSL approaches and commonly used vision encoders on the newly proposed benchmark.

Another major challenge in PGPS is developing a domain-agnostic VLM capable of handling multiple PGPS benchmarks. As shown in \cref{fig:pgps_examples}, the main difficulties arise from variations in diagram styles. 
To address the issue, we propose a few-shot domain adaptation technique for \geoclip{} which transfers its visual \geofeat{} perception from the synthetic diagrams to the real-world diagrams efficiently. 

We study the efficacy of the domain adapted \geoclip{} on PGPS when equipped with the language model. To be specific, we compare the VLM with the previous PGPS models on MathVerse~\citep{mathverse}, which is designed to evaluate both the PGPS and visual \geofeat{} perception performance on various domains.
While previous PGPS models are inapplicable to certain types of MathVerse problems, we modify the prediction target and unify the solution program languages of the existing PGPS training data to make our VLM applicable to all types of MathVerse problems.
Results on MathVerse demonstrate that our VLM more effectively integrates diagrammatic information and remains robust under conditions of various diagram styles.

\begin{itemize}
    \item We propose a benchmark to measure the visual \geofeat{} recognition performance of different vision encoders.
    % \item \sh{We introduce geometric CLIP (\geoclip{} and train the VLM equipped with \geoclip{} to predict both solution steps and the numerical measurements of the problem.}
    \item We introduce \geoclip{}, a vision encoder which can accurately recognize visual \geofeat{}s and a few-shot domain adaptation technique which can transfer such ability to different domains efficiently. 
    % \item \sh{We develop our final PGPS model, \geovlm{}, by adapting \geoclip{} to different domains and training with unified languages of solution program data.}
    % We develop a domain-agnostic VLM, namely \geovlm{}, by applying a simple yet effective domain adaptation method to \geoclip{} and training on the refined training data.
    \item We demonstrate our VLM equipped with GeoCLIP-DA effectively interprets diverse diagram styles, achieving superior performance on MathVerse compared to the existing PGPS models.
\end{itemize}

\fi 

% \textcolor{blue}{Arjun}

\section{Related Work}\label{sec:related_works}
\gls{bp} estimation from \gls{ecg} and \gls{ppg} waveforms has received significant attention due to its potential for continuous, unobtrusive monitoring. Earlier work relied on classical machine learning with handcrafted features, but deep learning methods have since emerged as more robust alternatives. Convolutional or recurrent architectures designed for \gls{ecg}/\gls{ppg} have shown strong performance, including ResUNet with self-attention~\cite{Jamil}, U-Net variants~\cite{Mahmud_2022}, and hybrid \gls{cnn}--\gls{rnn} models~\cite{Paviglianiti2021ACO}. These architectures often outperform traditional feature-engineering approaches, particularly when both \gls{ecg} and \gls{ppg} signals are used~\cite{Paviglianiti2021ACO}.

Nevertheless, many existing methods train solely on \gls{ecg}/\gls{ppg} data, which, while plentiful~\cite{mimiciii,vitaldb,ptb-xl}, often exhibit significant variability in signal quality and patient-specific characteristics. This variability poses challenges for achieving robust generalization across populations. Recent work has explored transfer learning to overcome these issues; for example, Yang \emph{et~al.}~\cite{yang2023cross} studied the transfer of \gls{eeg} knowledge to \gls{ecg} classification tasks, achieving improved performance and reduced training costs. Joshi \emph{et~al.}~\cite{joshi2021deep} also explored the transfer of \gls{eeg} knowledge using a deep knowledge distillation framework to enhance single-lead \gls{ecg}-based sleep staging. However, these studies have largely focused on within-modality or narrow domain adaptations, leaving open the broader question of whether an \gls{eeg}-based foundation model can serve as a versatile starting point for generalized biosignal analysis.

\gls{eeg} has become an attractive candidate for pre-training large models not only because of the availability of large-scale \gls{eeg} repositories~\cite{TUEG} but also due to its rich multi-channel, temporal, and spectral dynamics~\cite{jiang2024large}. While many time-series modalities (for example, voice) also exhibit rich temporal structure, \gls{eeg}, \gls{ecg}, and \gls{ppg} share common physiological origins and similar noise characteristics, which facilitate the transfer of temporal pattern recognition capabilities. In other words, our hypothesis is that the underlying statistical properties and multi-dimensional dynamics in \gls{eeg} make it particularly well-suited for learning robust representations that can be effectively adapted to \gls{ecg}/\gls{ppg} tasks. Our work is the first to validate the feasibility of fine-tuning a transformer-based model initially trained on EEG (CEReBrO~\cite{CEReBrO}) for arterial \gls{bp} estimation using \gls{ecg} and \gls{ppg} data.

Beyond accuracy, real-world deployment of \gls{bp} estimation models calls for efficient inference. Traditional deep networks can be computationally expensive, motivating recent interest in quantization and other compression techniques~\cite{nagel2021whitepaperneuralnetwork}. Few studies have combined large-scale pre-training with post-training quantization for \gls{bp} monitoring. Hence, our method integrates these two aspects: leveraging a potent \gls{eeg}-based foundation model and applying quantization for a compact, high-accuracy cuffless \gls{bp} solution.
\section{Threat Model}

We consider a multi-tenant FPGA setup, i.e., multiple users (tenants) deploy their modules independently on different regions of the FPGA. In this scenario, an attacker is a malicious third-party user who reconfigures one of the PRRs of the FPGA with a malicious power-waster. The goal of the attacker is to manipulate the configuration address of the user bitstream while it is being loaded onto the RM. The RM, which is configured on the same FPGA, loads the bitstream in its memory and subsequently configures it onto the designated PRRs of the FPGA. However, the RM is not designed with security as a focus, making it vulnerable to exploitation. 

\textbf{Attacker Knowledge:} The RM includes a status register that can be read by  applications and tenants without requiring access privileges \cite{7946114}. An attacker can monitor this status  to detect when a bitstream is being loaded into the RM, including the critical timing of loading the configuration address.

\textbf{Attacker Capability:} The attacker activates the power-wasters when the  configuration address of the bitstream is being loaded into the RM. Upon activation of the power-wasters, there are significant voltage fluctuations in the PDN, which induces faults in the bitstream.


\textbf{Attack Outcome:} The injected faults in the bitstream alter its configuration address, redirecting it to the wrong PRR. 

Note that the RM is capable of loading both encrypted and unencrypted bitstreams. However, bitstreams are typically decrypted before FPGA configuration \cite{ref_ultra}. Therefore, in our experiments, we specifically focus on decrypted bitstreams.


\section{Fault Attack Targeting Reconfiguration Address Manipulation}
\vspace{-0.1cm}
In this section, we provide a detailed analysis of FLARE, which targets configuration address manipulation. First, we implement several power-wasters that have been shown to cause voltage-based fault attacks and DoS \cite{8056840, FPGAhammer, Krautter2019MitigatingCloud, 9810438}. We evaluate the appropriate configuration parameters of these power-wasters for triggering a successful fault attack on the FPGA. Next, we describe how activation of these power-wasters during the partial reconfiguration process of bitstreams can adversely impact co-tenant modules on the FPGA. Finally, we provide a detailed description of the bitstream reconfiguration process and the attack setup on the FPGA.
\vspace{-0.1cm}
 \subsection{Fault Injection Using Power-Wasters }
\vspace{-0.1cm}
 As demonstrated in \cite{8056840, chaudhuri2024hackingfabrictargetingpartial}, ROs can be implemented as malicious power-wasters, causing high voltage fluctuations when triggered at a specific toggling frequency. Alternatively, loop-free self-clocked ROs  bypass DRC and are capable of generating clock glitches and severe voltage fluctuations on the PDN \cite{Sugawara2019OscillatorCentre}. Based on these observations, we evaluate both combinational and loop-free ROs for injecting faults into bitstreams. To induce malicious fault injections in bitstreams during partial reconfiguration, it is crucial to determine appropriate fault-injection parameters for the RO grid: (1) Number of ROs, (2) toggling frequency, and (3) duty cycle. We choose the upper bound of the number of RO and self-clocked RO instances to be 16000 and the toggling frequency range as $10^5-10^6$ Hz to be consistent with the experimental setup in prior work \cite{chaudhuri2024hackingfabrictargetingpartial, gnad2020remote}. These power-wasters are deployed on the FPGA to launch fault attacks on user bitstreams while they are being loaded into the RM for partial reconfiguration.
\vspace{-0.1cm}
\begin{figure}
\centering
\includegraphics[width=0.32\textwidth]{Figures/bs_data.pdf}

\caption{Structure of a Xilinx partial bitstream (the target region for attack is highlighted in red).}
\label{structure}
\vspace{-0.5cm}
\end{figure}
 \subsection{Proposed Fault Attack Methodology}
\vspace{-0.1cm}
 Prior fault attacks \cite{FPGAhammer, 8056840} require the power-wasters to be activated continuously throughout the runtime of user modules to ensure a successful  fault attack. This significantly increases the attack \textit{exposure window}, i.e., the duration for which the attack needs to be active, making it more susceptible to detection. Unlike previous approaches, FLARE is precisely timed to be activated only when the configuration address of the bitstream is being loaded to the RM and remains inactive otherwise. This targeted activation makes the attack stealthy and minimizes the likelihood of detection \cite{9643485, 10.1145/3451236}. By injecting faults in `select' part of the bitstream, FLARE manipulates the configuration address, redirecting the bitstream to incorrect PRR(s), which we refer to as \textit{victim} PRRs.
 
\textbf{Injecting Faults to the Reconfiguration Address:} To understand how FLARE operates, it is important to examine the structure of an FPGA bitstream. Fig. \ref{structure} shows the bitstream structure of a Xilinx 7-series FPGA. The bitstream is partitioned into several structured frames \cite{pham2017bitman}.  The first section is the synchronization header, which initializes the bitstream. The header is followed by configuration address frames, which allocate a fixed set of bits namely `select' \cite{9643485}. Finally, the footer includes CRC values. The `select' part identifies the target PRRs where the bitstream is intended to be uploaded. The precise timing of FLARE is based on the fixed and well-defined structure of the Xilinx FPGA bitstream \cite{9643485}. The configuration process of a bitstream on the FPGA operates at a specific clock frequency. Using this frequency and information about the number of frames prior to the `select' frames, an attacker can estimate the time window during which the `select' part will be configured. In this duration, the attacker activates the RO grid to inject faults in the bitstream. 
 % For instance, the header may include a NOP (no operation) command to introduce a delay before programming begins. 
 
\textbf{Analysis of Attack Impact:} We emphasize that while earlier attacks primarily focused on single-tenant vulnerabilities or localized faults within a specific module, FLARE exposes a broader vulnerability by targeting multiple modules co-located on the FPGA, a threat that was not explored in prior fault attacks \cite{chaudhuri2024hackingfabrictargetingpartial, FPGAhammer, krautter2021remote, provelengios2020power}. Importantly, we evaluate the effectiveness of the attack after deactivating the power-wasters. At this point, the faulty bitstreams are already loaded into the FPGA, and any errors in module functionality are observed post-reconfiguration. The objectives of FLARE are as follows:

\begin{itemize}[leftmargin=*,topsep=0pt]
    \item \textit{Denial-of-service}:  By manipulating the configuration address of the user bitstream, the bitstream is redirected to incorrect PRRs. This results in an incomplete configuration of the intended module, rendering it non-functional. 
    \item \textit{Faulty computation in co-tenant module(s)}: When a bitstream is redirected to a wrong configuration address, it may overwrite the existing logic of a co-tenant module. This leads to erroneous computations and incorrect functionality of the module. For instance, arithmetic modules used for data processing may generate incorrect results, potentially compromising the integrity and functionality of user modules. Note that once the faults are injected in the victim PRRs, they persist until the FPGA is fully reconfigured. 
\end{itemize}


\vspace{-0.2cm}

\subsection{Attack Setup}
\vspace{-0.1cm}
\begin{figure}[t]
\includegraphics[width=1\linewidth]{Figures/algo_ets.pdf}

\caption{Pseudocode for partial reconfiguration of bitstreams on the FPGA via CoRQ.}\label{alg}

\vspace{-0.6cm}
\end{figure}


\textbf{Dynamic Partial Reconfiguration Using CoRQ:} We utilize an open source runtime RM, namely Command-based Reconfiguration Queue (CoRQ), for loading and dynamically reconfiguring bitstreams on the FPGA \cite{7946114}. CoRQ can store multiple partial bitstreams corresponding to different modules, and reconfigure them onto specific PRRs of the FPGA based on their configuration addresses. The steps for dynamic reconfiguration of a bitstream are illustrated in Fig. \ref{alg}. First, the FPGA bitstreams are loaded into the CoRQ memory (marked as \textbf{\textit{sel\_bitstore}}). Next, they are passed through CRC evaluation (marked as \textbf{\textit{crc\_check}}). If CRC passes, the bitstreams are successfully configured on the FPGA, else they are blocked from reconfiguration.
Vivado allows users to enable or disable the CRC of bitstreams based on their requirements. In scenarios where the bitstream is encrypted, the CRC is disabled using the constraint \textit{BITSTREAM.GENERAL.CRC Disable} \cite{ref_ultra}. To assess the impact of FLARE on CRC, we deliberately enable CRC during bitstream generation in all our experiments.

CoRQ shares its status as a register that can be read by the processing system, enabling real-time guarantees that can be exploited by attackers. By monitoring when CoRQ starts to be busy, we can precisely determine when the configuration address  begins to be uploaded (explained in Section IV.B).

Note that while our experiments target a specific RM, the attack methodology is generic and applicable to various RMs in FPGAs as the reconfiguration always has the same flow.



\textbf{Implementation of Victim Modules:} We evaluate FLARE on two design modules: (1) Adders, which are commonly used in ALUs and signal processing, and (2) AES, which is needed in communication and cryptography operations \cite{krautter2021remote}.  To create a realistic attack scenario, we configure a majority of the lookup tables (LUTs) on the FPGA with these modules \cite{luo2020stealthy}. This reflects real-world multi-tenant FPGAs, where shared modules boost throughput and utilization \cite{mbongue2020architecture}.
\vspace{-0.1cm}
\subsubsection{Case Study 1: Adder}
\vspace{-0.1cm}
In our first case study, we evaluate combinational adders as their outputs are not impacted by timing violations commonly associated with fault attacks \cite{9643485}, allowing us to specifically focus on the impact of address manipulation. To identify which adders are affected by the fault-injected input bitstream, we implement two adder clusters, referred to as adder cluster \#1 and adder cluster \#2, with each cluster containing $n$ adder modules. Through experimental analysis, we observe that initializing adders separately rather than grouping them into clusters is not feasible, as it results in inefficient resource utilization.  To ensure maximum LUT usage, we set $n=500$; $n>500$ causes placement error during Vivado implementation. Each adder module computes the sum of fixed inputs. In an address manipulation-based fault attack, the input bitstream is redirected to the adder PRRs, leading to erroneous computations in the adder modules. To localize the faulty adders, we generate signals $flag_1^i$ and $flag_2^i $, $1 \leq i \leq n$, corresponding to each adder module $i$ in adder clusters \#1 and \#2, respectively. A value of `0' for $flag_1^i$ ($flag_2^i$) indicates that adder $i$ in adder cluster \#1 (adder cluster \#2) produces the correct output, while a value of `1' indicates a faulty output.

Monitoring the $flag_1$ and $flag_2$ signals directly is not practical due to their large widths of 500 bits each. To address this, we encode $flag_1$  and $flag_2$ using priority encoder modules $p_1$ and $p_2$, respectively; this enables us to localize the faulty adder modules within the respective adder clusters. The adder clusters and priority encoders contribute to 21.2\% of LUT utilization. Finally, we read the data from $flt_{sig}$, which encodes the information about the faulty adders. Specifically, the lower 10 bits of $flt_{sig}$, denoted as $flt_{sig}[9:0]$, are used by $p_1$ to indicate the faulty adder from adder cluster \#1. Similarly, the next 10 bits, $flt_{sig}[19:10]$, are used by  $p_2$ to denote the faulty adder from adder cluster \#2.

\vspace{-0.1cm}

\subsubsection{Case Study 2: AES}
\vspace{-0.1cm}
In our second case study, we evaluate FLARE on an AES-128 implementation. AES encryption and decryption involve the following operations -- \textit{sub\_bytes}, \textit{mix\_columns}, \textit{shift\_rows}, and \textit{key\_expansion}. Given the substantial overhead of each AES module in terms of the number of configured LUTs on the FPGA (each AES module utilizes 10223 LUTs), we implement only two such AES instances, namely AES \#1 and AES \#2, which contribute to 38.6\% of the FPGA LUT utilization. For our experiments, we configure both the AES modules with the same 128-bit plaintext and 128-bit key. A fault attack is successful when the fault-injected bitstream is redirected to atleast one of the AES modules, resulting in computational errors. We compare the outputs of the AES modules against the predetermined ciphertext. If the ciphertext of AES \#1 (AES \#2) matches the expected value, we set $flag_1$ ($flag_2$) as 0, else we set it as 1. The fault detection signals
$flag_1$ and $flag_2$ are encoded in the least significant bits $flt_{sig}[0]$ and $flt_{sig}[1]$, respectively. The value of $flt_{sig}$ determines one of the following:
\begin{itemize}[leftmargin=*,topsep=0pt]
    \item $flt_{sig}=0$: No fault attack is detected in the AES modules.
    \item $flt_{sig}=1$: A fault attack results in incorrect reconfiguration of a bitstream in the PRR of AES \#1. 
    \item $flt_{sig}=2$: A fault attack results in redirection of the bitstream to AES \#2. 
    \item  $flt_{sig}=3$: Both AES \#1 and AES \#2 are impacted. 
\end{itemize}
\vspace{-0.1cm}






\section{Results}
\label{sec:Results}

In this section, we present various analysis results that demonstrate the adoption of code obfuscation in Google Play.

\subsection{Overall Obfuscation Trends} 
\label{sec:obstrend}

\subsubsection{Presence of obfuscation} Out of the 548,967 Google Play Store APKs analyzed, we identified 308,782 obfuscated apps, representing approximately 56.25\% of the total. In Figure~\ref{fig:obfuscated_percentage}, we show the year-wise percentage of obfuscated apps for 2016-2023. There is an overall obfuscation increase of 13\% between 2016 and 2023, and as can be seen, the percentage of obfuscated apps has been increasing in the last few years, barring 2019 and 2020. As explained in Section~\ref{subsec:dataset}, 2019 and 2020 contain apps that are more likely to be abandoned by developers, and as such, they may not use advanced development practices.

\begin{figure}[h!]
\centering
    \includegraphics[width=\linewidth]{Figures/Only_obfuscation_trendV2.pdf}
    \caption{Percentage of obfuscated apps by year} \vspace{-4mm}
    \label{fig:obfuscated_percentage}
\end{figure}


From 2016 to 2018, the obfuscation levels were relatively stable at around 50-55\%, while from 2021 to 2023, there was a marked rise, reaching approximately 66\% in 2023. This indicates a growing focus on app protection measures among developers, likely driven by heightened security and IP concerns and the availability of advanced obfuscation tools.


\subsubsection{Obfuscation tools} Among the obfuscated APKs, our tool detector identified that 40.92\% of the apps use Proguard, 36.64\% use Allatori, 1.01\% use DashO, and 21.43\% use other (i.e., unknown) tools. We show the yearly trends in Figure~\ref{fig:ofbuscated_tool}. Note that we omit results in 2019 and 2020 ({\bf cf.} Section~\ref{subsec:dataset}).

ProGuard and Allatori are the most consistently used obfuscation tools, with ProGuard showing a slight overall increase in popularity and Allatori demonstrating variability. This inclination could be attributed to ProGuard being the default obfuscator integrated into Android Studio, a widely used development environment for Android applications. Notably, ProGuard usage increased by 13\% from 2018 to 2021, likely due to the introduction of R8 in April 2019~\cite{release_note_android}, which further simplified ProGuard integration with Android apps.

\begin{figure}[h]
\centering
    \includegraphics[width=\linewidth]{Figures/Initial_Tool_Trend_2019_dropV2.pdf} 
    \caption{Yearly obfuscation tool usage}
    \label{fig:ofbuscated_tool}
\end{figure}


DashO consistently remains low in usage, likely due to its high cost. The use of other obfuscation tools decreased until 2018 but has shown a resurgence from 2021 to 2023. This suggests that developers might be using other or custom tools, or our detector might be predicting some apps obfuscated with Proguard or Allatori as `other.' To investigate, we manually checked a sample of apps from the `other' category and confirmed they are indeed obfuscated. However, we could not determine which obfuscation tools the developers used. We discuss this potential limitation further in Section~\ref{sec:limitations}.


\subsubsection{Obfuscation techniques} We show the year-wise breakdown of obfuscation technique usage in Figure~\ref{fig:obfuscated_tech}. Among the various obfuscation techniques, Identifier Renaming emerged as the most prevalent, with 99.62\% of obfuscated apps using it alone or in combination with other methods (Categories of Only IR, IR and CF, IR and SE, or All three). Furthermore, 81.04\% of obfuscated apps used Control Flow Modification, and 62.76\% used String Encryption. The pervasive use of Identifier Renaming (IR) can be attributed to the fact that all obfuscation tools support it ({\bf cf.} Table~\ref{tab:ob_tool_cap}). Similarly, lower adoption of Control Flow Modification and String Encryption can be attributed to Proguard not supporting it. 

\begin{figure}[h]
\centering
    \includegraphics[width=\linewidth]{Figures/Initial_Tech_Trend_2019_dropV2.pdf} 
    \caption{Yearly obfuscation technique usage}
    \label{fig:obfuscated_tech}
\end{figure}



Next, we investigate the adoption of obfuscation on Google Play Store from various perspectives. Same as earlier, due to the smaller dataset size and possible bias ({\bf cf.} Section~\ref{subsec:dataset}), we exclude the APKs from 2019 and 2020 from this analyses.


\subsection{App Genre}
\label{sec:app_genre}

First, we investigate whether the obfuscation practices vary according to the App genre. Initially, we analysed all the APKs together before separating them into two snapshots.


\begin{figure*}[h]
    \centering
    \includegraphics[width=\linewidth]{Figures/AppGenreObfuscationV3.pdf}
    \caption{Obfuscated app percentage by genre (overall)}
    \label{fig:app_genre_overall}
\end{figure*}

Figure~\ref{fig:app_genre_overall} shows the genre-wise obfuscated app percentage. We note that 19 genres have more than 60\% of the apps obfuscated, and almost all the genres have more than 40\% obfuscation percentage. \textit{Casino} genre has the highest obfuscation percentage rate at 80\%, and overall, game genres tend to be more obfuscated than the other genres. The higher obfuscation usage in casino apps is logical due to their nature. These apps often simulate or involve gambling activities and handle monetary transactions and sensitive data related to in-game purchases, making them attractive targets for reverse engineering and hacking. This necessitates robust security measures to prevent fraud and protect user data. 


\begin{figure}[h]
    \centering
    \includegraphics[width=\linewidth]{Figures/AppGenre2018_2023ComparisonV3.pdf}
    \caption{Percentage of obfuscated apps by genre (2018-2023)}
    \label{fig:app_genre_comparison}
\end{figure}



\subsubsection{Genre-wise obfuscation trends in the two snapshots} To investigate the adoption of obfuscation over time, we study the two snapshots of Google Play separately, i.e., APKs from 2016-2018 as one group and APKs from 2021-2023 as another. 

Figure~\ref{fig:app_genre_comparison} illustrates the change in obfuscation levels by app genre between 2016-2018 to 2021-2023. Notably, app categories such as Education, Weather, and Parenting, which had obfuscation levels below the 2018 average, have increased to above the 2023 average by 2023. One possible reason for this in Education and Parenting apps can be the increase in online education activities during and after COVID-19 and the developers identifying the need for app hardening.

There are some genres, such as Casino and Action, for which the percentage of obfuscated apps didn't change across the two snapshots (i.e., purple and orange circles are close together in Figure~\ref{fig:app_genre_comparison}). This is because these genres are highly obfuscated from the beginning. Finally, the four genres, including Simulation and Role Playing, have a lower percentage of obfuscated apps in the 2021-2023 dataset. Our manual analysis didn't result in a conclusion as to why.


\begin{figure}[!h]
    \centering
    \includegraphics[width=\linewidth]{Figures/AppGenreTechAllV5.pdf}
    \caption{Obfuscation technique usage by genre (overall)}
    \label{fig:app_genre_all_tech}
\end{figure}


\subsubsection{Obfuscation techniques in different app genres} In Figure~\ref{fig:app_genre_all_tech}, we show the prevalence of key obfuscation techniques among various genres. As expected, almost all obfuscated apps in all genres used  Identifier Renaming. Also, it can be noted that genres with more obfuscated app percentages tend to use all three obfuscation techniques. Notably, more than 85\% of \textit{Casino} genre apps employ multiple obfuscation techniques

\subsubsection{Obfuscation tool usage in different app genres} We also investigated whether specific obfuscation tools are favoured by developers in different genres. However, apart from the expected observation that  ProGuard and Allatori being the most used tools, we didn't find any other interesting patterns. Therefore, we haven't included those measurement results.

\subsection{App Developers}
Next, we investigate individual developer-wise code obfuscation practices. From the pool of analyzed APKs, we identified the number of apps associated with each developer. Subsequently, we sorted the developers according to the number of apps they had created and selected the top 100 developers with the highest number of APKs for the 2016-2018 and 2021-2023 datasets. For the 2018 snapshot, we had 8,349 apps among the top 100 developers, while for the 2023 snapshot, we had 11,338 apps among the top 100 developers.

We then proceeded to detect whether or not these developers obfuscate their apps and, if so, what kind of tools and techniques they use. We present our results in five levels; developer obfuscating over 80\% of their apps, 60\%--80\% of apps, 40\%--60\% of apps, less than 40\%, and no obfuscation.

Figure~\ref{fig:developer_trend_my_apps_all} compares the two datasets in terms of developer obfuscation adoption. It shows that more developers have moved to obfuscate more than 80\% of their apps in the 2021-2023 dataset (76\%) compared to the 2016-2018 dataset (48\%).

We also found that among developers who obfuscate more than 80\% of their apps, 73\% in 2018 and 93\% in 2023 used the same obfuscation tool. Additionally, these top developers employ Control Flow Modification (CF) and String Encryption (SE) above the average values discussed in Section~\ref{sec:obstrend}. Specifically, in 2018, top developers used CF in 81.3\% of cases and SE in 66.7\%, while in 2023, these figures increased to 88.2\% and 78.9\%. This results in two insights: 1) Most top developers obfuscate all their apps with advanced techniques, possibly due to concerns about IP and security, and 2) Developers stick to a single tool, possibly due to specialized knowledge or because they bought a commercial licence.

\begin{figure}[]
    \centering
    \includegraphics[width=\linewidth]{Figures/Developer_Analysed_Comparison.pdf}
    \caption{Obfuscation usage (Top-100 developers)}
    \label{fig:developer_trend_my_apps_all}
\end{figure}


Finally, we investigate the obfuscation practices of developers with only one app in Table~\ref{tab:my-table}. According to the table, from those developers, 45.5\% of them obfuscated their apps in the 2016-2018 dataset and 57.2\% obfuscated their apps in the 2021-2023 dataset, showing a clear increase. However, these percentages are approximately 10\% lower than the average obfuscation rate in both cohorts discussed in Section~\ref{sec:obstrend}. This indicates that single-app developers may be less aware or concerned about code protection.


\begin{table}[]
\caption{Developers with only one app}
\label{tab:my-table}
\resizebox{\columnwidth}{!}{%
\begin{tabular}{cccccc}
\hline
\textbf{Year} & \textbf{\begin{tabular}[c]{@{}c@{}}Non\\ Obfuscated\end{tabular}} & \multicolumn{4}{c}{\textbf{Obfuscated}} \\ \hline
\multirow{3}{*}{\textbf{\begin{tabular}[c]{@{}c@{}}2018 \\ Snapshot\end{tabular}}} & \multirow{3}{*}{\begin{tabular}[c]{@{}c@{}}26,581 \\ (54.5\%)\end{tabular}} & \multicolumn{4}{c}{\begin{tabular}[c]{@{}c@{}}22,214 (45.5\%)\end{tabular}} \\ \cline{3-6} 
 &  & \textbf{ProGuard} & \textbf{Allatori} & \textbf{DashO} & \textbf{Other} \\ \cline{3-6} 
 &  & 6,131 & 8,050 & 658 & 7,375 \\ \hline
\multirow{3}{*}{\textbf{\begin{tabular}[c]{@{}c@{}}2023 \\ Snapshot\end{tabular}}} & \multirow{3}{*}{\begin{tabular}[c]{@{}c@{}}19,510 \\ (42.8\%)\end{tabular}} & \multicolumn{4}{c}{\begin{tabular}[c]{@{}c@{}}26,084 (57.2\%)\end{tabular}} \\ \cline{3-6} 
 &  & \textbf{ProGuard} & \textbf{Allatori} & \textbf{DashO} & \textbf{Other} \\ \cline{3-6} 
 &  & 12,697 & 9,672 & 234 & 3,581 \\ \hline
\end{tabular}%
}
\end{table}

\subsection{Top-k Apps}

Next, we investigate the obfuscation practices of top apps in Google Play Store. First, we rank the apps using the same criterion used by our previous work~\cite{rajasegaran2019multi, karunanayake2020multi, seneviratne2015early}. That is, we sort the apps in descending order of number of downloads, average rating, and rating count, with the intuition that top apps have high download numbers and high ratings, even when reviewed by a large number of users. Then, we investigated the percentage of obfuscated apps and obfuscation tools and technique usage as summarized in Table~\ref{tab:top_k_apps_2018_2023}.

When considering the highly ranked applications (i.e., top-1,000), the obfuscation percentage is notably higher, at around 93\%, in both datasets, which is significantly higher than the average percentage of obfuscation we observed in Section~\ref{sec:obstrend}. Top-ranked apps, likely due to their higher visibility and potential revenue, invest more in obfuscation to safeguard their intellectual property and enhance security. 

The obfuscation percentage decreases when going from the top 1,000 apps to the top 30,000 apps. Nonetheless, the obfuscation percentage in both datasets remains around similar values until the top 30,000 (e.g., $\sim$74\% for top-30,000). This indicates that the major increase in obfuscation in the 2021-2023 dataset comes from apps beyond the top 30,000.

When observing the tools used, the usage of ProGuard increases as we move from top to lower-ranked apps in both datasets. This may be because ProGuard is free and the default in Android Studio, while commercial tools like Allatori and DashO are expensive. There is a notable increase in the use of Allatori among the top apps in the 2021-2023 dataset. Regarding obfuscation techniques, the top 1,000 apps utilize all three techniques more frequently than lower-ranked apps in both snapshots. This indicates that the top 1,000 apps are more heavily protected compared to lower-ranked ones.

\begin{table*}[]
\caption{Summary of analysis results for Top-k apps in 2018 and 2023}
\label{tab:top_k_apps_2018_2023}
\resizebox{\textwidth}{!}{%
\begin{tabular}{lccccccccc}
\hline
\multicolumn{1}{c}{\begin{tabular}[c]{@{}c@{}}Top k apps - \\ Year\end{tabular}} & \begin{tabular}[c]{@{}c@{}}Total \\ Apps\end{tabular} & \begin{tabular}[c]{@{}c@{}}Obfuscation\\ Percentage\end{tabular} & \begin{tabular}[c]{@{}c@{}}ProGuard\\ Percentage\end{tabular} & \begin{tabular}[c]{@{}c@{}}Allatori\\ Percentage\end{tabular} & \begin{tabular}[c]{@{}c@{}}DashO\\ Percentage\end{tabular} & \begin{tabular}[c]{@{}c@{}}Other\\ Percentage\end{tabular} & \begin{tabular}[c]{@{}c@{}}IR\\ Percentage\end{tabular} & \begin{tabular}[c]{@{}c@{}}CF\\ Percentage\end{tabular} & \begin{tabular}[c]{@{}c@{}}SE\\ Percentage\end{tabular} \\ \hline
1k (2018) & 1,000 & 93.40 & 29.98 & 28.48 & 0.64 & 40.90 & 99.90 & 88.76 & 65.42 \\
10k (2018) & 10,000 & 85.19 & 25.55 & 35.32 & 0.47 & 38.65 & 99.90 & 88.76 & 71.91 \\
20k (2018) & 20,000 & 78.42 & 26.31 & 36.76 & 0.57 & 36.36 & 99.87 & 87.37 & 71.49 \\
30k (2018) & 30,000 & 74.40 & 27.30 & 37.71 & 0.64 & 34.36 & 99.82 & 86.75 & 71.11 \\
30k+ (2018) & 314,568 & 53.36 & 36.72 & 34.70 & 1.33 & 27.24 & 99.34 & 83.54 & 63.11 \\ \hline
1k (2023) & 1,000 & 92.50 & 24.00 & 51.89 & 1.95 & 22.16 & 100.0 & 92.54 & 83.68 \\
10k (2023) & 10,000 & 81.88 & 26.03 & 56.20 & 1.03 & 16.74 & 99.89 & 89.40 & 82.01 \\
20k (2023) & 20,000 & 76.62 & 30.48 & 52.92 & 0.96 & 15.64 & 99.93 & 85.80 & 78.01 \\
30k (2023) & 30,000 & 73.72 & 33.87 & 50.34 & 0.89 & 14.90 & 99.95 & 83.31 & 75.34 \\
30k+ (2023) & 206,216 & 61.90 & 46.56 & 38.21 & 0.64 & 14.59 & 99.97 & 77.51 & 62.50 \\ \hline
\end{tabular}%
}
\end{table*}

We present RiskHarvester, a risk-based tool to compute a security risk score based on the value of the asset and ease of attack on a database. We calculated the value of asset by identifying the sensitive data categories present in a database from the database keywords. We utilized data flow analysis, SQL, and Object Relational Mapper (ORM) parsing to identify the database keywords. To calculate the ease of attack, we utilized passive network analysis to retrieve the database host information. To evaluate RiskHarvester, we curated RiskBench, a benchmark of 1,791 database secret-asset pairs with sensitive data categories and host information manually retrieved from 188 GitHub repositories. RiskHarvester demonstrates precision of (95\%) and recall (90\%) in detecting database keywords for the value of asset and precision of (96\%) and recall (94\%) in detecting valid hosts for ease of attack. Finally, we conducted an online survey to understand whether developers prioritize secret removal based on security risk score. We found that 86\% of the developers prioritized the secrets for removal with descending security risk scores.
% \textcolor{blue}{Arjun}


\bibliographystyle{IEEEtran}
{
\hyphenpenalty=10000
\exhyphenpenalty=10000
\sloppy
\bibliography{references}
\vspace{-0.4cm}
}


\end{document}