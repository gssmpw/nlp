\documentclass[a4paper,conference]{IEEEtran}
\IEEEoverridecommandlockouts
\usepackage{subcaption}
\usepackage{fancyhdr,setspace}
\usepackage{comment}
\usepackage{caption}
\usepackage{todonotes}
\captionsetup[figure]{font=normalsize,labelfont=normalsize}
% \usepackage{caption}
\captionsetup{skip=0pt}
\usepackage{bm}
%\usepackage{floatrow}
\renewcommand{\footnoterule}{}

% \fancyhf{}
% \fancyfoot[L]{\large Regular Paper}
% \fancyfoot[C]{\large INTERNATIONAL TEST CONFERENCE}
% \fancyfoot[C]{\sffamily\fontsize{9pt}{9pt}\selectfont\thepage}
% \fancyfoot[R]{\large \thepage}

\usepackage[compact]{titlesec} 
% Some Computer Society conferences also require the compsoc mode option,
% but others use the standard conference format.
%
% If IEEEtran.cls has not been installed into the LaTeX system files,
% manually specify the path to it like:
% \documentclass[cgggffbconference]{../sty/IEEEtran}
\usepackage{cite}
\usepackage{xcolor}
\usepackage{blindtext, graphicx, tabularx, multicol, lipsum, subfig,eqnarray,amsmath,chemformula,multirow,threeparttable}
%\usepackage{biblatex}
\usepackage{enumitem}
\usepackage[utf8]{inputenc}
\usepackage{amssymb,amsmath}

\usepackage{float}
\DeclareMathOperator{\arccosh}{arcCosh}
\usepackage[boxed]{algorithm2e}
\usepackage[noend]{algpseudocode}
\usepackage[compact]{titlesec}
\algnewcommand{\IfThenElse}[3]{% \IfThenElse{<if>}{<then>}{<else>}
  \State \algorithmicif\ #1\ \algorithmicthen\ #2\ \algorithmicelse\ #3}
\makeatletter
\newcommand{\removelatexerror}{\let\@latex@error\@gobble}
\makeatother
%footnote comand
\makeatletter
\renewcommand\footnoterule{%
  \kern-3\p@
  \hrule\@width\columnwidth
  \kern2.6\p@}
  \makeatother
%\usepackage{algorithmic}
\usepackage{letltxmacro}


% Theorems ...
\usepackage{amsthm}
\newtheorem{problem}{Problem}
\newtheorem{theorem}{Theorem}
\newtheorem{corollary}{Corollary}
\usepackage{amsmath}
%\usepackage{mathpazo}
%\usepackage[ruled,vlined]{algorithm2e}
\newtheorem{lemma}{Lemma}
\SetKwInput{KwInput}{Input}                % Set the Input
\SetKwInput{KwOutput}{Output}  
\DeclareMathOperator*{\E}{\mathbb{E}}
%]==============================================
\usepackage{tikz}
\usepackage{capt-of}
\usepackage{amssymb} \usepackage{booktabs} 
\usepackage{pifont}% http://ctan.org/pkg/pifont
\newcommand{\cmark}{\ding{51}}%
\newcommand{\xmark}{\ding{55}}%
\newcommand{\rpm}{\raisebox{.2ex}{$\scriptstyle\pm$}}

% \usepackage{algorithmic}
\DeclareMathOperator*{\argmax}{arg\,max}
\usepackage{diagbox}
\usepackage{xcolor,etoolbox}
\usepackage{dblfloatfix}
% \let\mybibitem\bibitem
% \renewcommand{\bibitem}[1]{%
% \ifstrequal{#1}{edgeTPU}{\color{red}\mybibitem{#1}}}

\let\mybibitem\bibitem
\renewcommand{\bibitem}[1]{%
\ifstrequal{#1}{edgeTPU}{\color{black}\mybibitem{#1}}
{\ifstrequal{#1}{xyz}{\color{blue}\mybibitem{#1}}
{\color{black}\mybibitem{#1}}}%
}

% Some very useful LaTeX packages include:
% (uncomment the ones you want to load)

 \pagestyle{plain}


%----------------------- This is special for ITC --------------------------




\setlength{\parskip}{0pt}

% *** GRAPHICS RELATED PACKAGES ***
%
\ifCLASSINFOpdf
 
\else

\fi


%\usepackage[ruled,vlined]{algorithm2e}
\usepackage{comment}



\begin{document}

% Woohoo!!Paper is accepted!!
%

\title{FLARE: \underline{F}ault Attack \underline{L}everaging \underline{A}ddress \underline{R}econfiguration \underline{E}xploits in Multi-Tenant FPGAs$^*$\thanks{$^*$The work of J. Chaudhuri and K. Chakrabarty was supported in part by the National Science Foundation under grant no. CNS-2011561. The work of Hassan Nassar was supported in part by the German Federal Ministry of Education and Research (BMBF) through grant 01IS23066 as part of the Software Campus Project ``HE-Trust'' and in part by the ``Helmholtz Pilot Program for Core Informatics (kikit)'' at KIT. The work of Mehdi Tahoori was supported by German Research Foundation (DFG) projects SAUBER and SecFShare.}}


% author names and affiliations
% use a multiple column layout for up to three different
% affiliations
\author{\IEEEauthorblockN{Jayeeta Chaudhuri$^\dagger{}$, Hassan Nassar$^{\ddagger}{}$, Dennis R.E. Gnad$^{\ddagger}{}$, Jörg Henkel$^{\ddagger}{}$, \\
Mehdi B. Tahoori$^{\ddagger}{}$,  and Krishnendu Chakrabarty$^\dagger{}$}
\IEEEauthorblockA{$^\dagger{}$School of Electrical, Computer, and Energy Engineering, Arizona State University, Tempe, AZ, USA \\
$^\ddagger{}$Institute of Computer Engineering, Karlsruhe Institute of Technology (KIT), Karlsruhe, Germany
\\
}
\\[-5.0ex]
}






\maketitle

\begin{abstract}

 
Modern FPGAs are increasingly supporting multi-tenancy to enable dynamic reconfiguration of user modules. While multi-tenant FPGAs improve utilization and flexibility, this paradigm introduces critical security threats. In this paper, we present \textbf{FLARE}, a fault attack that exploits vulnerabilities in the partial reconfiguration process, specifically while a user bitstream is being uploaded to the FPGA by a reconfiguration manager. Unlike traditional fault attacks that operate during module runtime, FLARE injects faults in the bitstream during its reconfiguration, altering the configuration address and redirecting it to unintended partial reconfigurable regions (PRRs). This enables the overwriting of pre-configured co-tenant modules, disrupting their functionality. FLARE leverages power-wasters that activate briefly during the reconfiguration process, making the attack stealthy and more challenging to detect with existing countermeasures. Experimental results on a Xilinx Pynq FPGA demonstrate the effectiveness of FLARE in compromising multiple user bitstreams during the reconfiguration process.
\end{abstract}


% \thispagestyle{fancy}
% \fancyhead{}
% \renewcommand{\headrulewidth}{0pt}
% \fancyhf{}
% \fancyfoot[L]{\large Regular Paper}
% \\ 978-1-7281-4823-6/19/\$31.00 \copyright2019 IEEE}
% \fancyfoot[C]{\large INTERNATIONAL TEST CONFERENCE}
% \fancyfoot[R]{\large \thepage}

%----------------------- This is special for ITC --------------------------
\thispagestyle{plain}
% For peer review papers, you can put extra information on the cover
% page as needed:
% \ifCLASSOPTIONpeerreview
% \begin{center} \bfseries EDICS Category: 3-BBND \end{center}
% \fi
%
% For peerreview papers, this IEEEtran command inserts a page break and
% creates the second title. It will be ignored for other modes.
\IEEEpeerreviewmaketitle

% 
% 
The widespread integration of communication networks and smart devices in modern control systems has increased the vulnerability of industrial systems to online cyber-attacks, e.g., Industroyer, Blackenergy, etc \citep{osti_1505628}.
% Modern control systems have seen a large push to include communication networks and smart devices to increase performance, made possible by improvements in communication device cost and energy consumption. This trend has been coupled with the usage of open-standard communication protocols among industrial control systems, making them vulnerable to online cyber-attacks such as Industroyer, Blackenergy, etc \citep{osti_1505628}. 
To counter this, methods have been developed to improve security by achieving attack detection, mitigation, and monitoring, among others \citep{sandberg2022secure}. This paper focuses on active attack diagnosis to mitigate stealthy attacks. 
%
%\subsection{Literature review}

Active diagnosis techniques rely on the inclusion of additional moduli to control systems
% inclusion within the control system of additional moduli 
to alter the behavior of the system compared to information known by the attacker. 
For instance, the concept of additive watermarking was introduced in \cite{mo2015physical}, where noise signals of known mean and variance are added at the plant and compensated for it at the controller. 
This compensation, however, is not exact, causing some performance degradation. Thus, trade-offs between performance and detectability  are necessary \citep{zhu2023detection}.
% A later work \citep{zhu2023detection} designs the watermark signal by trading performance for detection. Thus, although additive watermarking serves as a good detection scheme, they endure performance losses even in the nominal case. 

In encrypted control \citep{darup2021encrypted}, the sensor data is encrypted, sent to the controller, and then operated on directly. Encrypted input signals are sent back to the plant for decryption. Although encryption is widespread in IT security, in control systems it presents some concerns, such as the introduction of time delays \citep{stabile2024verifiable}, while it may present inherent weaknesses \citep{alisic2023model}.
% they are not preferred as they introduce time delays \citep{stabile2024verifiable} which can cause instability, and some encryption schemes can be very weak  \citep{alisic2023model}. 

In moving target defense \citep{griffioen2020moving}, the plant is augmented with fictitious dynamics, known to the controller. The plant output is transmitted to the controller along with the fictitious states over a network under attack. 
The additional measurements then aide in the detection of attacks. 
This comes at the cost of higher communication bandwidth needs, which increases rapidly with the dimension of the augmented systems.
% Since the dynamics of the fictitious dynamics are exactly known to the controller, the attack is detected easily. However, when the scale of the system increases, the communication bandwidth used by moving the target defense approach increases rapidly. 

Other recently proposed works include two-way coding \citep{fang2019two}, a weak encryuption technique, and dynamic masking \citep{abdalmoaty2023privacy}, which enhances privacy as well as security, have been shown to be effective against zero-dynamics attacks.
% Two-way coding \citep{fang2019two} and dynamic masking \citep{abdalmoaty2023privacy} are other recently proposed approaches. Two-way coding is another form of weak encryption technique whilst dynamic masking proposes an architecture that enhances both privacy and security. These schemes are shown to be effective against zero dynamics attacks but remain to be studied for other classes of attacks. 
% Recent extensions include \citep{mukherjee2021secure,ramos2024privacy}.
% Some other works which are related are \citep{mukherjee2021secure}, an extension of \cite{fang2019two}. The work \citep{ramos2024privacy} is an extension of moving target defense for multi-agent systems. 
Furthermore, filtering techniques for attack detection are proposed by \cite{murguia2020security,hashemi2022codesign,escudero2023safety}, while not focusing on stealthy attacks.
% The works \citep{murguia2020security,hashemi2022codesign,escudero2023safety} develop filtering techniques to guarantee safety, without being focused on stealthy covert attacks.

Multiplicative watermarking (mWM) has been proposed by the authors as a diagnosis technique \citep{ferrari2020switching}. mWM consists of a pair of filters on each communication channel between the plant and its controller; the scheme is affine to weak encryption, whereby ``encoding'' and ``decoding'' are done by changing signals' dynamic characteristics through inverse pairs of filters. This enables original signals to be recovered exactly, and thus does not lead to performance degradation.
% A multiplicative watermark is an affine to a weak encryption technique, through which the signal is ``encoded'' by a filter, changing its dynamic behavior. The use of inverse pairs means that the original signal can be recovered, through ``decoding'' via an inverse filter. As such, differently to techniques based on additive watermarking, no performance is lost due to the injection of noise, and there are no bandwidth limitations.

%\subsection{Contributions}
One of the critical features of multiplicative watermarking is that to detect stealthy attacks, the mWM filter parameters must be switched over time. In this paper, an algorithm to optimally design the mWM parameters after a switching event is presented, enhancing detection performance, without changing the switching time.
% This is done without changing the switching time, which is taken as given.

\textcolor{black}{
To formalize the filter design problem, we suppose the defender is interested in optimal performance against adversaries injecting covert attacks with matched system parameters \citep{smith2015covert}, including the mWM parameters prior to the switch. This scenario represents a worst case where malicious agents can take full control of the system while remaining undetected.
Thus, the attack strategy is explicitly included within the formulation of the closed-loop system, and the mWM filters are chosen by solving an optimization problem minimizing the attack-energy-constrained output-to-output gain (AEC-OOG) \citep{anand2023risk}, a variation of the output-to-output gain proposed in  \cite{teixeira2015strategic}.
}
The main contributions of this paper are:
% We consider an adversary injecting a covert attack with matched system parameters \citep{smith2015covert}, i.e., an attacker with full knowledge of the control system parameters, including those of the mWM filters before the switch. This scenario is taken as a worst case, as it has been shown that this class of attacks can be made stealthy. To quantitatively define a cost, the output-to-output gain (OOG) \citep{teixeira2015strategic} is leveraged,
% a metric introduced to evaluate the impact of an additive attack in a control system. %Specifically, OOG evaluates the worst-case performance loss that an attacker injecting an undetectable attack can obtain. 
% Here, the maximum performance loss caused by a stealthy adversary with limited energy is taken, the attack-energy-constrained OOG (AEC-OOG) \citep{anand2023risk}. The main contributions of this paper are:
\begin{enumerate}
%[label=\alph*.]
\item The problem of optimally designing the switching mWM filters is formulated as an optimization problem, with the AEC-OOG is taken as the objective;%where the AEC-OOG is taken as the impact metric; 
\item The worst-case scenario of a covert attack with exact knowledge of plant and mWM filter parameters is embedded within the design problem;
% The optimization problem is defined to incorporate the worst-case scenario of a covert attack with exact knowledge of plant and mWM filter parameters;
\item The feasibility of the optimization problem is shown to be dependent only on stability conditions; 
\item A solution scheme is proposed to promote randomization of the mWM filter parameters such that an eavesdropping adversary cannot remain stealthy.
\end{enumerate} 

This builds on the results of \cite{ferrari2020switching}, where the focus was on the design of the switching protocols, rather than the parameters themselves.
Compared to previous work \citep{gallo2021design}, this paper introduces an optimization problem which is always feasible (thanks to the use of AEC-OOG in the objective), while also considering a more sophisticated class of covert attacks, where the presence of watermark is known to the adversary. 
Moreover, this paper poses a different objective than \citep{zhang2023hybrid}; indeed, while \citep{zhang2023hybrid} provided a design strategy to ensure certain privacy properties, in this paper we address the problem of optimal parameter design following a switching event.


%\subsection{Organization}
The rest of the paper is organized as follows. 
After formulating the problem in Section~\ref{sec:PF}, we propose our design algorithm in Section~\ref{sec:main}, and analyze its properties. It is then evaluated through a numerical example in Section~\ref{sec:NE}, and concluding remarks are given Section~\ref{sec:Con}.
% We provide the problem background in Section~\ref{sec:PF}. We formulate the design problem in Section~\ref{sec:main}, together with an analysis of its properties. The proposed algorithm is evaluated through a numerical example in Section \ref{sec:NE}. Concluding remarks are offered in Section \ref{sec:Con}.
% \textcolor{blue}{Arjun}

\section{Mobile Networks Powered by \glspl{LLM}}
\label{sec:LLM_enabled_MNs}
\begin{figure*}[t!]
\centering
\includegraphics[width=.99\textwidth]{Fig1.eps}
    \caption{Possible architectural designs for integrated \gls{LLM} and \gls{MNO} ecosystem.}
    \label{fig:LLM_possible_architectures}
\end{figure*}
The historical data of the \gls{MNO}, archived over years of expertise, constitutes a solid foundation for training the \gls{LLM} using structured and unstructured multi-modal inputs (as illustrated in Fig.~\ref{fig:LLM_possible_architectures}a) such as user intents, network logs, alarm descriptions, trouble tickets, \gls{PCAP} files (e.g. from Wireshark or tcpdump), dashboard screenshots, audio recordings (e.g. from \gls{IVR} systems), video feeds (e.g. from infrastructure surveillance), and \gls{NWDAF} analytics. To this end, a separate collection framework aggregates data from various sources into a centralized repository, and extracts most informative features such as warnings, error codes, timestamps, and user/gNB/session/bearer/\gls{QoS} flow/slice IDs. The extracted features are then converted into unified embeddings that are combined into a common vector space with suitable metadata (e.g. to differentiate data formats). The resulting vector store is used to fine-tune the \gls{LLM} to deeply internalize \gls{MNO}-specific knowledge \cite{Bariah2023understanding}. This allows the \gls{LLM} to learn patterns, sequences, and deviations that correlate with normal or faulty network operations. This is made possible using a timestamp-based cross-referencing to link different entries from several data sources, allowing detailed description and context for each flagged event as well as the resolution workflow for the spotted anomalies.

In live mobile networks, fresh multi-modal data is continuously fed into the \gls{LLM}, either uploaded in batches or streamed in real-time. The \gls{LLM} analyzes this data and identifies potential anomalous behaviors in light of its accumulated learning. In case of new anomalies not covered during the fine-tuning stage, the \gls{LLM} can rely on clustering techniques to group similar patterns and flag outliers as suspected behaviors. The \gls{LLM} is also capable of using \gls{RAG}-enabled external knowledge databases such as \gls{3GPP} documents \cite{Said2024instruct}, \gls{IEEE} standards, \gls{IETF} RFCs and vendors documentation \cite{soman2023observations} to compare the actual network behavior with the expected one to identify misconfigurations and spot unusual trends in protocols and communication flows. Well-crafted prompts, on the other hand, can guide the \gls{LLM} responses to provide focused solutions. Paradigms such as the \gls{CoT} reasoning can be used to break down the \gls{LLM} insights into a series of simplified and actionable sub-tasks. It can be extended by the \gls{ToT} technique to explore different reasoning paths and identify the most optimal solution. The \gls{LLM} can naturally produce stepwise reasoning if datasets used for fine-tuning contain \gls{CoT} and \gls{ToT} examples, or through creative prompting \cite{Zhou2024survey}. In parallel, \gls{NOC} engineers can intervene to confirm, guide or reject the \gls{LLM} findings, if needed, e.g. using its intuitive conversational interface. Through continuous self-learning, the \gls{LLM} will dynamically adapt to evolving network conditions, optimizing its performance over time \cite{Chaparadza2023optimization}.

%For instance, when a network experiences latency issues, the \gls{LLM} seamlessly analyze multi-modal information from diverse origins to identify the root cause, e.g. overloaded \gls{UPF} due to insufficient capacity, and then suggest a solution, e.g. step-by-step instructions including suitable code scripts for the involved \glspl{NF} to autonomously reroute traffic or modify policies. Conventional 5G networks can only alert about anomalies using suitable \gls{NWDAF} analytics that track the violated thresholds and notify the \gls{OAM} center to display the details on complex dashboards.

By incorporating \glspl{LLM} (e.g. as \glspl{NF}) into upcoming 6G networks, expected to be designed with \gls{SbD} principles \cite{Khaloopour2024Resilience}, \glspl{LLM} will naturally inherit the same built-in security safeguards rather than adding them as an afterthought. This design-driven approach focuses on proactive threat management, end-to-end encryption, authentication, network slicing isolation, \gls{AI}-driven threat detection with automated reactions, and stateless designs, fostering a resilient \gls{LLM}.
%The design-driven security in 5G and upcoming 6G networks ensures that security is natively integrated at every layer of the architecture rather than added as an afterthought. This approach focuses on proactive threat management, end-to-end encryption, authentication, network slicing, and \gls{AI}-driven threat detection and automated reactions to counter evolving cyber threats.



\section{Threat Model}

We consider a multi-tenant FPGA setup, i.e., multiple users (tenants) deploy their modules independently on different regions of the FPGA. In this scenario, an attacker is a malicious third-party user who reconfigures one of the PRRs of the FPGA with a malicious power-waster. The goal of the attacker is to manipulate the configuration address of the user bitstream while it is being loaded onto the RM. The RM, which is configured on the same FPGA, loads the bitstream in its memory and subsequently configures it onto the designated PRRs of the FPGA. However, the RM is not designed with security as a focus, making it vulnerable to exploitation. 

\textbf{Attacker Knowledge:} The RM includes a status register that can be read by  applications and tenants without requiring access privileges \cite{7946114}. An attacker can monitor this status  to detect when a bitstream is being loaded into the RM, including the critical timing of loading the configuration address.

\textbf{Attacker Capability:} The attacker activates the power-wasters when the  configuration address of the bitstream is being loaded into the RM. Upon activation of the power-wasters, there are significant voltage fluctuations in the PDN, which induces faults in the bitstream.


\textbf{Attack Outcome:} The injected faults in the bitstream alter its configuration address, redirecting it to the wrong PRR. 

Note that the RM is capable of loading both encrypted and unencrypted bitstreams. However, bitstreams are typically decrypted before FPGA configuration \cite{ref_ultra}. Therefore, in our experiments, we specifically focus on decrypted bitstreams.


\section{Fault Attack Targeting Reconfiguration Address Manipulation}
\vspace{-0.1cm}
In this section, we provide a detailed analysis of FLARE, which targets configuration address manipulation. First, we implement several power-wasters that have been shown to cause voltage-based fault attacks and DoS \cite{8056840, FPGAhammer, Krautter2019MitigatingCloud, 9810438}. We evaluate the appropriate configuration parameters of these power-wasters for triggering a successful fault attack on the FPGA. Next, we describe how activation of these power-wasters during the partial reconfiguration process of bitstreams can adversely impact co-tenant modules on the FPGA. Finally, we provide a detailed description of the bitstream reconfiguration process and the attack setup on the FPGA.
\vspace{-0.1cm}
 \subsection{Fault Injection Using Power-Wasters }
\vspace{-0.1cm}
 As demonstrated in \cite{8056840, chaudhuri2024hackingfabrictargetingpartial}, ROs can be implemented as malicious power-wasters, causing high voltage fluctuations when triggered at a specific toggling frequency. Alternatively, loop-free self-clocked ROs  bypass DRC and are capable of generating clock glitches and severe voltage fluctuations on the PDN \cite{Sugawara2019OscillatorCentre}. Based on these observations, we evaluate both combinational and loop-free ROs for injecting faults into bitstreams. To induce malicious fault injections in bitstreams during partial reconfiguration, it is crucial to determine appropriate fault-injection parameters for the RO grid: (1) Number of ROs, (2) toggling frequency, and (3) duty cycle. We choose the upper bound of the number of RO and self-clocked RO instances to be 16000 and the toggling frequency range as $10^5-10^6$ Hz to be consistent with the experimental setup in prior work \cite{chaudhuri2024hackingfabrictargetingpartial, gnad2020remote}. These power-wasters are deployed on the FPGA to launch fault attacks on user bitstreams while they are being loaded into the RM for partial reconfiguration.
\vspace{-0.1cm}
\begin{figure}
\centering
\includegraphics[width=0.32\textwidth]{Figures/bs_data.pdf}

\caption{Structure of a Xilinx partial bitstream (the target region for attack is highlighted in red).}
\label{structure}
\vspace{-0.5cm}
\end{figure}
 \subsection{Proposed Fault Attack Methodology}
\vspace{-0.1cm}
 Prior fault attacks \cite{FPGAhammer, 8056840} require the power-wasters to be activated continuously throughout the runtime of user modules to ensure a successful  fault attack. This significantly increases the attack \textit{exposure window}, i.e., the duration for which the attack needs to be active, making it more susceptible to detection. Unlike previous approaches, FLARE is precisely timed to be activated only when the configuration address of the bitstream is being loaded to the RM and remains inactive otherwise. This targeted activation makes the attack stealthy and minimizes the likelihood of detection \cite{9643485, 10.1145/3451236}. By injecting faults in `select' part of the bitstream, FLARE manipulates the configuration address, redirecting the bitstream to incorrect PRR(s), which we refer to as \textit{victim} PRRs.
 
\textbf{Injecting Faults to the Reconfiguration Address:} To understand how FLARE operates, it is important to examine the structure of an FPGA bitstream. Fig. \ref{structure} shows the bitstream structure of a Xilinx 7-series FPGA. The bitstream is partitioned into several structured frames \cite{pham2017bitman}.  The first section is the synchronization header, which initializes the bitstream. The header is followed by configuration address frames, which allocate a fixed set of bits namely `select' \cite{9643485}. Finally, the footer includes CRC values. The `select' part identifies the target PRRs where the bitstream is intended to be uploaded. The precise timing of FLARE is based on the fixed and well-defined structure of the Xilinx FPGA bitstream \cite{9643485}. The configuration process of a bitstream on the FPGA operates at a specific clock frequency. Using this frequency and information about the number of frames prior to the `select' frames, an attacker can estimate the time window during which the `select' part will be configured. In this duration, the attacker activates the RO grid to inject faults in the bitstream. 
 % For instance, the header may include a NOP (no operation) command to introduce a delay before programming begins. 
 
\textbf{Analysis of Attack Impact:} We emphasize that while earlier attacks primarily focused on single-tenant vulnerabilities or localized faults within a specific module, FLARE exposes a broader vulnerability by targeting multiple modules co-located on the FPGA, a threat that was not explored in prior fault attacks \cite{chaudhuri2024hackingfabrictargetingpartial, FPGAhammer, krautter2021remote, provelengios2020power}. Importantly, we evaluate the effectiveness of the attack after deactivating the power-wasters. At this point, the faulty bitstreams are already loaded into the FPGA, and any errors in module functionality are observed post-reconfiguration. The objectives of FLARE are as follows:

\begin{itemize}[leftmargin=*,topsep=0pt]
    \item \textit{Denial-of-service}:  By manipulating the configuration address of the user bitstream, the bitstream is redirected to incorrect PRRs. This results in an incomplete configuration of the intended module, rendering it non-functional. 
    \item \textit{Faulty computation in co-tenant module(s)}: When a bitstream is redirected to a wrong configuration address, it may overwrite the existing logic of a co-tenant module. This leads to erroneous computations and incorrect functionality of the module. For instance, arithmetic modules used for data processing may generate incorrect results, potentially compromising the integrity and functionality of user modules. Note that once the faults are injected in the victim PRRs, they persist until the FPGA is fully reconfigured. 
\end{itemize}


\vspace{-0.2cm}

\subsection{Attack Setup}
\vspace{-0.1cm}
\begin{figure}[t]
\includegraphics[width=1\linewidth]{Figures/algo_ets.pdf}

\caption{Pseudocode for partial reconfiguration of bitstreams on the FPGA via CoRQ.}\label{alg}

\vspace{-0.6cm}
\end{figure}


\textbf{Dynamic Partial Reconfiguration Using CoRQ:} We utilize an open source runtime RM, namely Command-based Reconfiguration Queue (CoRQ), for loading and dynamically reconfiguring bitstreams on the FPGA \cite{7946114}. CoRQ can store multiple partial bitstreams corresponding to different modules, and reconfigure them onto specific PRRs of the FPGA based on their configuration addresses. The steps for dynamic reconfiguration of a bitstream are illustrated in Fig. \ref{alg}. First, the FPGA bitstreams are loaded into the CoRQ memory (marked as \textbf{\textit{sel\_bitstore}}). Next, they are passed through CRC evaluation (marked as \textbf{\textit{crc\_check}}). If CRC passes, the bitstreams are successfully configured on the FPGA, else they are blocked from reconfiguration.
Vivado allows users to enable or disable the CRC of bitstreams based on their requirements. In scenarios where the bitstream is encrypted, the CRC is disabled using the constraint \textit{BITSTREAM.GENERAL.CRC Disable} \cite{ref_ultra}. To assess the impact of FLARE on CRC, we deliberately enable CRC during bitstream generation in all our experiments.

CoRQ shares its status as a register that can be read by the processing system, enabling real-time guarantees that can be exploited by attackers. By monitoring when CoRQ starts to be busy, we can precisely determine when the configuration address  begins to be uploaded (explained in Section IV.B).

Note that while our experiments target a specific RM, the attack methodology is generic and applicable to various RMs in FPGAs as the reconfiguration always has the same flow.



\textbf{Implementation of Victim Modules:} We evaluate FLARE on two design modules: (1) Adders, which are commonly used in ALUs and signal processing, and (2) AES, which is needed in communication and cryptography operations \cite{krautter2021remote}.  To create a realistic attack scenario, we configure a majority of the lookup tables (LUTs) on the FPGA with these modules \cite{luo2020stealthy}. This reflects real-world multi-tenant FPGAs, where shared modules boost throughput and utilization \cite{mbongue2020architecture}.
\vspace{-0.1cm}
\subsubsection{Case Study 1: Adder}
\vspace{-0.1cm}
In our first case study, we evaluate combinational adders as their outputs are not impacted by timing violations commonly associated with fault attacks \cite{9643485}, allowing us to specifically focus on the impact of address manipulation. To identify which adders are affected by the fault-injected input bitstream, we implement two adder clusters, referred to as adder cluster \#1 and adder cluster \#2, with each cluster containing $n$ adder modules. Through experimental analysis, we observe that initializing adders separately rather than grouping them into clusters is not feasible, as it results in inefficient resource utilization.  To ensure maximum LUT usage, we set $n=500$; $n>500$ causes placement error during Vivado implementation. Each adder module computes the sum of fixed inputs. In an address manipulation-based fault attack, the input bitstream is redirected to the adder PRRs, leading to erroneous computations in the adder modules. To localize the faulty adders, we generate signals $flag_1^i$ and $flag_2^i $, $1 \leq i \leq n$, corresponding to each adder module $i$ in adder clusters \#1 and \#2, respectively. A value of `0' for $flag_1^i$ ($flag_2^i$) indicates that adder $i$ in adder cluster \#1 (adder cluster \#2) produces the correct output, while a value of `1' indicates a faulty output.

Monitoring the $flag_1$ and $flag_2$ signals directly is not practical due to their large widths of 500 bits each. To address this, we encode $flag_1$  and $flag_2$ using priority encoder modules $p_1$ and $p_2$, respectively; this enables us to localize the faulty adder modules within the respective adder clusters. The adder clusters and priority encoders contribute to 21.2\% of LUT utilization. Finally, we read the data from $flt_{sig}$, which encodes the information about the faulty adders. Specifically, the lower 10 bits of $flt_{sig}$, denoted as $flt_{sig}[9:0]$, are used by $p_1$ to indicate the faulty adder from adder cluster \#1. Similarly, the next 10 bits, $flt_{sig}[19:10]$, are used by  $p_2$ to denote the faulty adder from adder cluster \#2.

\vspace{-0.1cm}

\subsubsection{Case Study 2: AES}
\vspace{-0.1cm}
In our second case study, we evaluate FLARE on an AES-128 implementation. AES encryption and decryption involve the following operations -- \textit{sub\_bytes}, \textit{mix\_columns}, \textit{shift\_rows}, and \textit{key\_expansion}. Given the substantial overhead of each AES module in terms of the number of configured LUTs on the FPGA (each AES module utilizes 10223 LUTs), we implement only two such AES instances, namely AES \#1 and AES \#2, which contribute to 38.6\% of the FPGA LUT utilization. For our experiments, we configure both the AES modules with the same 128-bit plaintext and 128-bit key. A fault attack is successful when the fault-injected bitstream is redirected to atleast one of the AES modules, resulting in computational errors. We compare the outputs of the AES modules against the predetermined ciphertext. If the ciphertext of AES \#1 (AES \#2) matches the expected value, we set $flag_1$ ($flag_2$) as 0, else we set it as 1. The fault detection signals
$flag_1$ and $flag_2$ are encoded in the least significant bits $flt_{sig}[0]$ and $flt_{sig}[1]$, respectively. The value of $flt_{sig}$ determines one of the following:
\begin{itemize}[leftmargin=*,topsep=0pt]
    \item $flt_{sig}=0$: No fault attack is detected in the AES modules.
    \item $flt_{sig}=1$: A fault attack results in incorrect reconfiguration of a bitstream in the PRR of AES \#1. 
    \item $flt_{sig}=2$: A fault attack results in redirection of the bitstream to AES \#2. 
    \item  $flt_{sig}=3$: Both AES \#1 and AES \#2 are impacted. 
\end{itemize}
\vspace{-0.1cm}







% \begin{figure*}[htpb!]
% \label{}
% \centering

%     {{\label{ROCIowaCedar} \includegraphics[width=\textwidth/3]{figures/IowaCedar_roc.png}}}%
%     \qquad
%     {{\label{ROCIowaDesMoines} \includegraphics[width=\textwidth/3]{figures/IowaDesMoines_roc.png} }%
%   \captionsetup{justification=centering}
%   \caption{\Acf{ROC} curves for \acf{RW} Iowa (CR) and  \acf{RW} Iowa (DM) dataset. Dummy model here represents a model whose output is solely a ``no Flood'' for all pixels.}
%   \label{fig:RW_ROC_Curves}%
% \end{figure*}



\section{Results and Discussions}
\label{sec:Results}

In this section, we aim to answer three main questions. First, we want to validate our hypothesis that \ac{SYN} data is a viable proxy for \ac{RW} data when training ML models for downscaling. Secondly, we seek to assess how much more skillful ML-based downscaling is compared to classical, non-data-driven techniques, such as our baseline methods, \textit{i.e.}, thresholded bicubic and Lanczos interpolation. Finally, we would like to appraise the extent to which data-driven models like ours are transferable (in terms of usefulness) to other regions without major performance degradations.  
To assess the quality of the models, we conduct a multiple comparison test --namely the Holm-Bonferroni procedure \cite{HolmBonferroni1979} -- that is designed to control the \ac{FWER}. We notice that, with a \ac{FWER} of $10^{-3}$, all the differences in model performance are significant. The only exception to this trend was observed in \ac{RW}-GH for whom the pairwise differences between \ac{RCAN} and \ac{ESRT}, Lanczos and Bicubic were not significant with the aforementioned \ac{FWER}. 

%Finally, we aim to find out the factors influencing the transferability of our models from one region to another.

\subsection{Potential of using SYN Data for RW downscaling}

In order to evaluate the utility of synthetic data for training, we compare performances of our candidate models on both \ac{SYN} and \ac{RW} Iowa data whose results are presented in Table \ref{tab:IowaResults}. We notice that 
\textbf{(i)} For the Iowa datasets, there is a drop in performance of all the models when going from \ac{SYN} to \ac{RW} datasets, 
\textbf{(ii)} for the \ac{RW}-IA (CR) as well as \ac{RW}-IA (DM) datasets, both bicubic and Lanczos interpolation have accuracies and MCC up to 70.89\% and 0.42 respectively while the deep learning models have accuracies and MCC up to 73.34\% and 0.46 respectively, 
\textbf{(iii)} There is a roughly 6\% accuracy improvement for the \ac{SYN} data for the deep learning models compared to the bicubic and lanczos models and this improvement drops to about 3\% for \ac{RW} data,  
\textbf{(iv)} the performance of all the models remain consistent across both \ac{RW}-IA datasets and \textbf{(v)} in \figref{fig:RW_ROC_Curves}, we observe that there is a high degree of overlap among the \ac{ROC} curves for the data-driven models.

From (i) and (iv) we can conclude that \ac{SYN} data is more intricate than \ac{RW} data. This implies that the benefits yielded by training with \ac{SYN} dataset, while significant, is not as prominent in the \ac{RW} Iowa datasets. 
% This may be due to sensor noise prevalent in the \ac{RW} Landsat-8 data that can be harder to reproduce in the synthetically generated examples. 
(i), (iii) and (v) implies that while \ac{SYN} data is not an exact replacement for \ac{RW} data, it provides a rather significant edge, which is all the more important when there is insufficient \ac{RW} for training. From (ii) we can conclude that the three proposed data driven models outperform classical super-resolution techniques such as bicubic and lanczos, conclusion supported by the \ac{ROC} curves in Figure \ref{fig:RW_ROC_Curves} for whom the data-driven models, in general, lie above the non-data-driven alternatives. Observation (iv) shows that  for the climatically similar \ac{RW}-Iowa(CR) and \ac{RW}-Iowa(DM) regions, training on \ac{SYN} Iowa data does indeed provide an edge. 

% have similar climate. 

\begin{figure*}[t!]
    \centering
    \begin{subfigure}[t]{0.5\textwidth}
        \centering
        \includegraphics[width=\textwidth/2]{figures/IowaCedar_roc.png}
        \caption{}
    \end{subfigure}%
    ~ 
    \begin{subfigure}[t]{0.5\textwidth}
        \centering
        \includegraphics[width=\textwidth/2]{figures/IowaDesMoines_roc.png}
        \caption{}
    \end{subfigure}
    \vspace*{0.5cm}
    \caption{    \label{fig:RW_ROC_Curves} \Acf{ROC} curves for (a) RW-IA (CR) and (b) RW-IA (DM) dataset. Na\"ive model here represents a model whose output is solely a ``no Flood'' for all pixels. Star here represents the pixel-wise classifier with a threshold of 0.5.}
\end{figure*}


\subsection{Effectiveness of data-driven approaches}

In order to evaluate the effectiveness of ML models in the downscaling task, we compare performances of our candidate models to Lanczos and bicubic interpolation methods by looking at figures of some sample predictions from Iowa (Figure \ref{fig:RWIowaDesMoines}), performance comparison in the region of Iowa in Table \ref{tab:IowaResults} and the ROC curves in Figure \ref{fig:RW_ROC_Curves} for \ac{RW} data. We notice that 
\textbf{(vi)} For RW-IA (DM) samples, the deep learning models maintain a higher degree of spatial continuity in the predicted \ac{FIM}, 
\textbf{(vii)} We observe that  bicubic and Lanczos interpolation produces over-smoothed \ac{FIM} reconstructions, while the plain \ac{RDN}, \ac{RCAN} and \ac{ESRT} models are more detail-inclusive. Similar conclusions can be drawn upon inspecting the \ac{ROC} curves in Figure \ref{fig:RW_ROC_Curves} and 
\textbf{(viii)} For RW-IA (CR), the ML models show a performance improvement of 3.06\% when comparing the best ML model and non-data-driven method and, while for RW-IA (DM) there is a performance improvement of 2.45\%.


Figures \ref{fig:EUSamples} and \ref{fig:RWIowaDesMoines} show the spatial disparity among the models whose details are often obscured in aggregated metrics such as accuracy. (vi) This implies that these data-driven models are better are recognizing an underlying stream network geometry than the classical methods. However, when it comes to narrow river streams, all the models struggle capturing the nuances of the \ac{FIM} resultant from localized high elevation features such as small islands within rivers or man-made structures. (vii) shows a clear advantage of our data-driven approaches over the non-data-driven alternatives. (viii) indicates the benefits of the data-driven models when evaluated over Iowa. 



\subsection{Applicability of our models to external regions}

To evaluate how transferable our models are, we draw conclusions from figures of the sample predictions from Western Europe (Figure \ref{fig:EUSamples}) and Ghana (Figure \ref{fig:GhanaSamples}) as well as the performance comparison in Table \ref{tab:ExternalResults}. We notice that 
\textbf{(ix)} for Ghana all of the models fail to adequately inundate the pixels over separated areas on account of several disconnected regions of inundation in the chosen area,
\textbf{(x)} the ML models outperform non-data driven methods for RW-EU, 
\textbf{(xi)} for the RW-EU dataset, there is an improvement of 4.89\% when comparing the accuracy of the best data- and non-data-driven methods, 
\textbf{(xii)} For RW-RR and RW-GH, there is marginal improvement (up to 0.77\% in accuracy) of the ML methods over the non-data driven methods and 
\textbf{(xiii)} For RW-EU, we notice that the ML models produce more connected streams over the non-data-driven models. 

(x) and (xi) implies that the models are transferable when considering hydroclimaticalogically similar regions since Iowa and the Meuse river in Europe lie within mid temperate zones. Similar to the observation (vi) for RW-IA (DM), (xiii) implies that the benefits of the ML model in identifying underlying network streams is also transferable to hydroclimatologically similar regions. In contrast, (xii) and (ix) both imply that the trained ML models struggle to generalize to RW-RR \& RW-GH. We speculate that this may be due to the significant differences in geography and climate when compared to Iowa. 

% More specifically, we notice that Ghana has a lot of disconnected regions when compared to Iowa and Western Europe, possibly indicating a geomorphological dissimilarity. Additionally, in the case of Red River and Ghana, we also speculate that they include drivers to flood inundation that are different from Iowa and Western Europe, which lie within mild temperate zones. Ghana on the other hand has a tropical (dry and hot) climate.

Our study directly implies that good quality synthetic data can be useful surrogates for downscaling low-resolution \acp{WFM} to high-resolution \acp{FIM} in regions, where such data are hard to come by, even when downscaling by a factor of 10. We noticed that such models were readily transferable to climatically similar regions as the region of training. However, Such derived ML models did not feature significantly different transferability when evaluated over hydroclimatologically dissimilar regions, which we attribute to different flood inundation characteristics, primarily at finer scales. A possible avenue to circumvent such issues is to explore additional training approaches that fall under the general area of domain adaptation. Nevertheless, data-driven models are still advantageous (and, hence, preferable) over non-data-driven alternatives in transfer scenarios like the one we considered here. 


%%%%%%%%%%%%%%%%%%%%%%%%%%%%%%% unused text %%%%%%%%%%%%%%%%%%%%%%%%%%%%%%%%%%%%%%%



% \tabref{tab:AccuracyResults} depicts test accuracies obtained by our models on both \ac{SYN} and \ac{RW} data. For Iowan floods, a comparison of \ac{SYN} and \ac{RW} results shows \textbf{(i)} bicubic and Lanczos interpolations remarkably gaining about $3\%$ in accuracy, as well as \textbf{(ii)} \ac{RDN} and \ac{RCAN} remaining relatively stable, while \textbf{(iii)} topography-aware models loosing $2.7\%$ in performance. From (i) one can conclude that \ac{SYN} data are morphologically slightly more intricate than \ac{RW} data. Also, (i) and (ii) likely imply that \ac{SYN} data, excluding topography, can serve as satisfactory surrogates of \ac{RW} data. However, as implied by (iii), our topography-dependent models seems to be particularly sensitive to distributional shifts of their combined inputs (\acp{WFM} and topographic features). More specifically, the topography-informed models' performance edge, while still statistically significant, is extremely marginal, even when compared to our non-data-driven approaches. Next, when comparing results between the cases of Iowan and Ghanaian \ac{RW} data, one observes that \textbf{(iv)} the accuracy of bicubic and Lanczos interpolations drops by almost $5\%$ due to over-smoothing. This may imply that Ghanaian \acp{FIM} bare a more complex morphology, when compared to Iowan \acp{FIM}. Also, \textbf{(v)} our topography-agnostic, data-driven models' performance degrades more gracefully (by about $2\%$), while \textbf{(vi)} our topography-aware models perform, virtually, as bad as our non-data-driven approaches. Hence, the differences in the data populations of the two regions we considered are significant enough to render our topography-dependent models noncompetitive. 



Software development is increasingly conceived as a collaboration activity between developers and AIs. Indeed, IDEs already implement features to enable interactive development, with AI suggesting implementations that are reused by developers.

Although multiple studies show this interaction can be successful, there is still limited understanding of how the models must be configured and used in the context of code generation tasks. This study addresses this gap, systematically investigating the impact of several key parameters, including the repeated submission of a prompt to accommodate for the non-deterministic nature of the models.

Our study reveals several key findings about the usage of ChatGPT. In particular, we discovered how creativity, although up to a limited extent, is useful to increase the range of methods whose code can be generated correctly. A major role is played by parameter top-p, which is commonly underrated, and instead has a major impact on the correctness of the results, with lower values producing better results. Finally, prompts should be submitted multiple times, with $5$ repetitions combined with a temperature of $1.2$ resulting in an effective configuration in our experiments.  

Future work concerns two main research directions. One is about replicating this experiment with other AI assistants, to validate our findings in multiple contexts. The second research direction concerns finding strategies to deal with the need to submit the same prompt multiple times to obtain a useful result, and thus developing approaches able to select or merge multiple responses automatically. 
% \textcolor{blue}{Arjun}


\bibliographystyle{IEEEtran}
{
\hyphenpenalty=10000
\exhyphenpenalty=10000
\sloppy
\bibliography{references}
\vspace{-0.4cm}
}


\end{document}