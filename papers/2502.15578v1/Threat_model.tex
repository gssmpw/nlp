\section{Threat Model}

We consider a multi-tenant FPGA setup, i.e., multiple users (tenants) deploy their modules independently on different regions of the FPGA. In this scenario, an attacker is a malicious third-party user who reconfigures one of the PRRs of the FPGA with a malicious power-waster. The goal of the attacker is to manipulate the configuration address of the user bitstream while it is being loaded onto the RM. The RM, which is configured on the same FPGA, loads the bitstream in its memory and subsequently configures it onto the designated PRRs of the FPGA. However, the RM is not designed with security as a focus, making it vulnerable to exploitation. 

\textbf{Attacker Knowledge:} The RM includes a status register that can be read by  applications and tenants without requiring access privileges \cite{7946114}. An attacker can monitor this status  to detect when a bitstream is being loaded into the RM, including the critical timing of loading the configuration address.

\textbf{Attacker Capability:} The attacker activates the power-wasters when the  configuration address of the bitstream is being loaded into the RM. Upon activation of the power-wasters, there are significant voltage fluctuations in the PDN, which induces faults in the bitstream.


\textbf{Attack Outcome:} The injected faults in the bitstream alter its configuration address, redirecting it to the wrong PRR. 

Note that the RM is capable of loading both encrypted and unencrypted bitstreams. However, bitstreams are typically decrypted before FPGA configuration \cite{ref_ultra}. Therefore, in our experiments, we specifically focus on decrypted bitstreams.

