\section{Introduction}

Field Programmable Gate Arrays (FPGAs) are increasingly being used as customized accelerator platforms to support high-performance computing and accelerating compute-intensive workloads. Major cloud service providers (CSPs) such as Amazon, Alibaba Cloud, and Microsoft offer FPGA-accelerated instances, which allow users to deploy their custom modules with high flexibility and efficiency, particularly for artificial intelligence and machine learning applications~\cite{ref_17_amazon, firestone2018azure}. With their scalability across several compute-intensive tasks, FPGAs are being increasingly adopted for virtualization and multi-tenancy applications. The multi-tenant approach allows CSPs to maximize resource utilization by allowing users access to simultaneously reconfigure their applications on separate modules of the same FPGA.  

However, the multi-tenancy approach introduces several security threats, particularly physical and remote attacks. Physical attacks occur when an attacker gains direct access to the FPGA system, enabling them to measure power consumption, gather timing information, or measure electrical current~\cite{moradi2011vulnerability, standaert2004power, standaert2006updates}. However, emerging threats demonstrate that fault attacks can be activated remotely via untrusted users using on-chip sensors that impact the functionality of co-tenant modules, without requiring physical access to the FPGA~\cite{trimberger2017security}. In~\cite{8056840}, ring oscillator (RO)-based circuits have been used to create excessive voltage fluctuations, that lead to a permanent crash of the FPGA. Building on this, \cite{FPGAhammer, 7809042, 8844478} showed that by repeatedly toggling a set of ROs, attackers can induce fault attacks on an AES module. In~\cite{chaudhuri2024hackingfabrictargetingpartial}, certain toggling frequencies of an RO grid were shown to induce fault attacks during partial reconfiguration on an FPGA. Faults are injected into the bitstream by targeting the reconfiguration manager (RM), which manages the upload and configuration of bitstreams in the partial reconfigurable regions (PRRs) of the FPGA. Unlike~\cite{8056840, FPGAhammer, 7809042, 8844478}, these faults persist even after the attack ceases, resulting in computational errors after FPGA deployment.

Prior runtime fault attacks~\cite{8056840, FPGAhammer, 7809042, 8844478} require extended attack durations, as power-wasters need to remain activated throughout the operation of the tenant modules. This requirement increases the likelihood of attack detection by voltage-monitoring schemes~\cite{9643485, 10.1145/3451236}. In~\cite{chaudhuri2024hackingfabrictargetingpartial}, the attack duration is reduced by activating power-wasters only during partial reconfiguration. However,~\cite{chaudhuri2024hackingfabrictargetingpartial} is successful only when Cyclic Redundancy Check (CRC) of a bitstream is disabled, which makes the attack more detectable. Moreover, all the aforementioned attacks are limited to disrupting only a single tenant module. 

In this paper, we introduce FLARE, a novel fault attack that adversely impacts multiple tenants by altering the reconfiguration address of a user bitstream rather than the bitstream data. To achieve this, we employ malicious power-wasters that induce faults in the bitstream while it is being configured onto the FPGA through the RM. These faults alter the `select' part of the bitstream, which specifies the configuration address that determines the exact PRR where the bitstream is to be loaded \cite{9643485}. By injecting faults into the configuration address, the bitstream is incorrectly redirected, resulting in the overwriting of PRRs of other co-tenants. This leads to erroneous computation and potential Denial-of-Service (DoS) for the user bitstream. The `select' part of the bitstream is much smaller than the full bitstream, consisting of only a few configuration words. By precisely targeting fault injection in the `select' part, the attack duration is significantly minimized, making it \textit{stealthy} and harder to detect by the schemes described in \cite{10.1145/3451236, 9643485}. In our paper, a stealthy attack is defined as the attacker's ability to evade detection by countermeasures that rely on monitoring extended fault injection activity. 


We demonstrate the effectiveness of FLARE using benign design modules that are configured within a multi-tenant FPGA setup. Fig. \ref{flare} illustrates the proposed fault attack setup.


The key contributions of this paper are as follows.
\begin{itemize}[leftmargin=*,topsep=0pt]
\item We determine the appropriate configuration parameters of several power-wasters to induce a successful fault attack.
    \item We demonstrate the first address manipulation fault attack that targets the process of FPGA partial reconfiguration.
    \item We show that targeted fault injections in the configuration address section of the bitstream can disrupt the functionality of other modules sharing the FPGA fabric.
    \item We evaluate FLARE on several modules, including an AES cryptography implementation and adder instances.
    \item We implement FLARE on hardware, targeting the Xilinx Pynq-Z1 FPGA board.
    
\end{itemize}

The remainder of the paper is organized as follows. Section II provides an overview of prior attacks on multi-tenant FPGAs and establishes the motivation of our attack. The threat model is described in Section III. Section IV presents the attack framework and the experimental setup. Section V provides experimental results. Finally, Section VI concludes the paper.
\vspace{-0.5cm}


\begin{figure}

\includegraphics[width=0.48\textwidth]{Figures/Threat_model.pdf}

\caption{Proposed attack setup of FLARE (Adder$_0$: benign module configured in address \textit{PRR-tenant$_2$}. Due to fault-injection, an user bitstream is redirected to \textit{PRR-tenant$_2$} instead of its intended address \textit{PRR-tenant$_1$}).}
\label{flare}
\vspace{-0.6cm}

\end{figure}
