
\documentclass[a4paper,notitlepage]{article}

\usepackage[a4paper,top=2cm,bottom=2cm,left=2.5cm,right=2.5cm,marginparwidth=1.75cm]{geometry}

\usepackage{authblk}

\usepackage{hyperref}
\hypersetup{
  colorlinks   = true, %Colours links instead of ugly boxes
  %urlcolor     = black!50!red, %Colour for external hyperlinks
  linkcolor    = black!50!brown, %Colour of internal links
  citecolor   = black!50!brown, %Colour of citations
  %bookmarks = false
}
% \usepackage{algorithm}

\RequirePackage{algorithm}
\RequirePackage{algorithmic}

% \usepackage{algorithmic}
% \usepackage{algpseudocode}

\usepackage{bm}
\usepackage{graphicx,xcolor}

\usepackage{mathtools,tikz}
\usepackage{hhline}
\usepackage{multirow}
\usepackage{enumitem}  
\usepackage{caption}
\usepackage{subcaption}

\usepackage{appendix}



\usepackage{cite}
\usepackage{amsmath,amssymb,amsfonts,amsthm,xspace}
\theoremstyle{definition}
\newtheorem{example}{Example}
\newtheorem{remark}{Remark}
\newtheorem{theorem}{Theorem}
\newtheorem{defn}{Definition}

\newtheorem*{defn*}{Definition}
\newtheorem*{lemma*}{Lemma}

\newtheorem{lemma}[theorem]{Lemma}  
\newtheorem{corollary}[theorem]{Corollary}
\newtheorem{prop}[theorem]{Proposition}
\newtheorem{ques}[theorem]{Question}


\newcommand{\w}[1]{\ensuremath{\mathsf{w}_{\mathsf{H}}(#1)}} 
\newcommand{\h}[2]{\ensuremath{\mathsf{d}_{\mathsf{H}}(#1, #2)}} 
\newcommand{\ind}[2]{\ensuremath{\mathsf{index}_{\mathsf{H}}(#1, #2)}} 
\newcommand{\ipr}[2]{\ensuremath{\langle #1, #2\rangle} }

\newcommand{\Expth}{\mbox{$\hat{{\mathbf E}}$} }
\newcommand{\Prob}{\ensuremath{{\mathbb P}}}
\newcommand{\expect}[1]{\Expt \left[ #1 \right]}
\newcommand{\prob}[1]{\Prob \left\{ #1 \right\}}
\newcommand{\defineqq}{\ensuremath{\stackrel{\textup{\tiny def}}{=}}}

\newcommand{\norm}[1]{\left\lVert#1\right\rVert_2}
\newcommand{\normi}[1]{\left\lVert#1\right\rVert}


\foreach \x in {A,...,Z}{%
\expandafter\xdef\csname vec\x \endcsname{\noexpand\ensuremath{\noexpand\mathbf{\x}}}
}

% define calligaraphic versions of all uppercase letters \cA etc
\foreach \x in {A,...,Z}{%
\expandafter\xdef\csname c\x \endcsname{\noexpand\ensuremath{\noexpand\mathcal{\x}}}
}

% define mathbb versions of all uppercase letters \bbA etc
\foreach \x in {A,...,Z}{%
\expandafter\xdef\csname bb\x \endcsname{\noexpand\ensuremath{\noexpand\mathbb{\x}}}
}

\DeclareMathOperator*{\argmax}{arg\,max}


\newcommand\reals{{\mathbb R}}
\newcommand{\wh}[1]{\ensuremath{{\left|#1\right|}_{\mathsf{H}}}} 
\newcommand{\ina}[1]{\left<#1\right>}
\newcommand{\inb}[1]{\left\{#1\right\}}
\newcommand{\inp}[1]{\left(#1\right)}
\newcommand{\insq}[1]{\left[#1\right]}
\newcommand{\inl}[1]{\left|#1\right|}
\newcommand{\bcs}{\ensuremath{\mathsf{1bCSbinary}}}
\newcommand{\logreg}{\ensuremath{\mathsf{LogisticRegression}}}
\newcommand{\spl}{\ensuremath{\mathsf{SparseLinearReg}}}
\newcommand{\topk}{\ensuremath{\mathsf{TopK-Correlations}}}
\newcommand{\bx}{\mathbf{x}}
\newcommand{\ba}{\mathbf{a}}
\newcommand{\by}{\mathbf{y}}
\newcommand{\bz}{\mathbf{z}}
\newcommand{\bs}{\mathbf{s}}
\newcommand{\bw}{\mathbf{w}}
\newcommand{\bb}{\mathbf{b}}
\newcommand{\be}{\mathbf{e}}
\newcommand{\sign}[1]{\mathsf{sign}(#1)}
\newcommand{\sorted}[1]{\mathsf{sorted}(#1)}


\newcommand{\red}[1]{{\textcolor{red}{#1}}}
\newcommand{\blue}[1]{{\textcolor{blue}{#1}}}
\newcommand{\olive}[1]{{\textcolor{olive}{#1}}}
\newcommand{\aqua}[1]{{\textcolor{cyan}{#1}}}


\iffalse
\usepackage{url}
\usepackage{microtype}






% \usepackage{amsmath}
% \usepackage{amssymb}
% \usepackage{amsthm}
\usepackage{verbatim}

% \usepackage{mdwtab}

% \usepackage{pdfpages}

\DeclareCaptionFormat{myformat}{\fontsize{8}{9}\selectfont#1#2#3}
\captionsetup{format=myformat}

% \usepackage{graphicx}
% \usepackage{textcomp}











% If you use natbib package, activate the following three lines:
% \usepackage[round]{natbib}
\usepackage[square]{natbib}
\renewcommand{\bibname}{References}
\renewcommand{\bibsection}{\subsubsection*{\bibname}}

% If you use BibTeX in apalike style, activate the following line:
%\bibliographystyle{apalike}


\fi


\title{Exact Recovery of Sparse Binary Vectors from \\ Generalized Linear Measurements}
\author[1]{Arya Mazumdar}
\author[1]{Neha Sangwan}

\affil[1]{\small Halicioglu Data Science Institute, University of California San Diego, La Jolla, United States}
% \affil[2]{\small Halicioglu Data Science Institute, University of California San Diego, La Jolla, United States}

\begin{document}
\maketitle


\begin{abstract}
 We consider the problem of {\em exact} recovery of a $k$-sparse binary vector  from generalized linear measurements (such as {\em logistic regression}). We analyze the {\em linear estimation} algorithm (Plan, Vershynin, Yudovina, 2017), and also show information theoretic lower bounds on the number of required measurements. As a consequence of our results, for  noisy one bit quantized linear measurements (\bcs), we obtain a sample complexity of $O((k+\sigma^2)\log{n})$, where $\sigma^2$ is the noise variance. This is shown to be optimal  due to the information theoretic lower bound. 
{We also obtain tight sample complexity characterization for logistic regression.}

  Since \bcs\ is a strictly harder problem than noisy linear measurements (\spl) because of added quantization, the same sample complexity is achievable for \spl. 
  While this sample complexity  can be obtained via the popular lasso algorithm, linear estimation is  computationally more efficient. 
  Our lower bound  holds for any set of measurements for \spl\, (similar bound was known for Gaussian measurement matrices) and is closely matched by the maximum-likelihood upper bound. 
  For \spl,  it was conjectured in Gamarnik and Zadik, 2017 that there is a statistical-computational gap and the number of measurements should be at least $(2k+\sigma^2)\log{n}$ for efficient algorithms to exist. It is worth noting that our results imply that there is 
   no such statistical-computational gap for \bcs\ and logistic regression.
\end{abstract}


\section{Introduction}
\label{sec:introduction}
The business processes of organizations are experiencing ever-increasing complexity due to the large amount of data, high number of users, and high-tech devices involved \cite{martin2021pmopportunitieschallenges, beerepoot2023biggestbpmproblems}. This complexity may cause business processes to deviate from normal control flow due to unforeseen and disruptive anomalies \cite{adams2023proceddsriftdetection}. These control-flow anomalies manifest as unknown, skipped, and wrongly-ordered activities in the traces of event logs monitored from the execution of business processes \cite{ko2023adsystematicreview}. For the sake of clarity, let us consider an illustrative example of such anomalies. Figure \ref{FP_ANOMALIES} shows a so-called event log footprint, which captures the control flow relations of four activities of a hypothetical event log. In particular, this footprint captures the control-flow relations between activities \texttt{a}, \texttt{b}, \texttt{c} and \texttt{d}. These are the causal ($\rightarrow$) relation, concurrent ($\parallel$) relation, and other ($\#$) relations such as exclusivity or non-local dependency \cite{aalst2022pmhandbook}. In addition, on the right are six traces, of which five exhibit skipped, wrongly-ordered and unknown control-flow anomalies. For example, $\langle$\texttt{a b d}$\rangle$ has a skipped activity, which is \texttt{c}. Because of this skipped activity, the control-flow relation \texttt{b}$\,\#\,$\texttt{d} is violated, since \texttt{d} directly follows \texttt{b} in the anomalous trace.
\begin{figure}[!t]
\centering
\includegraphics[width=0.9\columnwidth]{images/FP_ANOMALIES.png}
\caption{An example event log footprint with six traces, of which five exhibit control-flow anomalies.}
\label{FP_ANOMALIES}
\end{figure}

\subsection{Control-flow anomaly detection}
Control-flow anomaly detection techniques aim to characterize the normal control flow from event logs and verify whether these deviations occur in new event logs \cite{ko2023adsystematicreview}. To develop control-flow anomaly detection techniques, \revision{process mining} has seen widespread adoption owing to process discovery and \revision{conformance checking}. On the one hand, process discovery is a set of algorithms that encode control-flow relations as a set of model elements and constraints according to a given modeling formalism \cite{aalst2022pmhandbook}; hereafter, we refer to the Petri net, a widespread modeling formalism. On the other hand, \revision{conformance checking} is an explainable set of algorithms that allows linking any deviations with the reference Petri net and providing the fitness measure, namely a measure of how much the Petri net fits the new event log \cite{aalst2022pmhandbook}. Many control-flow anomaly detection techniques based on \revision{conformance checking} (hereafter, \revision{conformance checking}-based techniques) use the fitness measure to determine whether an event log is anomalous \cite{bezerra2009pmad, bezerra2013adlogspais, myers2018icsadpm, pecchia2020applicationfailuresanalysispm}. 

The scientific literature also includes many \revision{conformance checking}-independent techniques for control-flow anomaly detection that combine specific types of trace encodings with machine/deep learning \cite{ko2023adsystematicreview, tavares2023pmtraceencoding}. Whereas these techniques are very effective, their explainability is challenging due to both the type of trace encoding employed and the machine/deep learning model used \cite{rawal2022trustworthyaiadvances,li2023explainablead}. Hence, in the following, we focus on the shortcomings of \revision{conformance checking}-based techniques to investigate whether it is possible to support the development of competitive control-flow anomaly detection techniques while maintaining the explainable nature of \revision{conformance checking}.
\begin{figure}[!t]
\centering
\includegraphics[width=\columnwidth]{images/HIGH_LEVEL_VIEW.png}
\caption{A high-level view of the proposed framework for combining \revision{process mining}-based feature extraction with dimensionality reduction for control-flow anomaly detection.}
\label{HIGH_LEVEL_VIEW}
\end{figure}

\subsection{Shortcomings of \revision{conformance checking}-based techniques}
Unfortunately, the detection effectiveness of \revision{conformance checking}-based techniques is affected by noisy data and low-quality Petri nets, which may be due to human errors in the modeling process or representational bias of process discovery algorithms \cite{bezerra2013adlogspais, pecchia2020applicationfailuresanalysispm, aalst2016pm}. Specifically, on the one hand, noisy data may introduce infrequent and deceptive control-flow relations that may result in inconsistent fitness measures, whereas, on the other hand, checking event logs against a low-quality Petri net could lead to an unreliable distribution of fitness measures. Nonetheless, such Petri nets can still be used as references to obtain insightful information for \revision{process mining}-based feature extraction, supporting the development of competitive and explainable \revision{conformance checking}-based techniques for control-flow anomaly detection despite the problems above. For example, a few works outline that token-based \revision{conformance checking} can be used for \revision{process mining}-based feature extraction to build tabular data and develop effective \revision{conformance checking}-based techniques for control-flow anomaly detection \cite{singh2022lapmsh, debenedictis2023dtadiiot}. However, to the best of our knowledge, the scientific literature lacks a structured proposal for \revision{process mining}-based feature extraction using the state-of-the-art \revision{conformance checking} variant, namely alignment-based \revision{conformance checking}.

\subsection{Contributions}
We propose a novel \revision{process mining}-based feature extraction approach with alignment-based \revision{conformance checking}. This variant aligns the deviating control flow with a reference Petri net; the resulting alignment can be inspected to extract additional statistics such as the number of times a given activity caused mismatches \cite{aalst2022pmhandbook}. We integrate this approach into a flexible and explainable framework for developing techniques for control-flow anomaly detection. The framework combines \revision{process mining}-based feature extraction and dimensionality reduction to handle high-dimensional feature sets, achieve detection effectiveness, and support explainability. Notably, in addition to our proposed \revision{process mining}-based feature extraction approach, the framework allows employing other approaches, enabling a fair comparison of multiple \revision{conformance checking}-based and \revision{conformance checking}-independent techniques for control-flow anomaly detection. Figure \ref{HIGH_LEVEL_VIEW} shows a high-level view of the framework. Business processes are monitored, and event logs obtained from the database of information systems. Subsequently, \revision{process mining}-based feature extraction is applied to these event logs and tabular data input to dimensionality reduction to identify control-flow anomalies. We apply several \revision{conformance checking}-based and \revision{conformance checking}-independent framework techniques to publicly available datasets, simulated data of a case study from railways, and real-world data of a case study from healthcare. We show that the framework techniques implementing our approach outperform the baseline \revision{conformance checking}-based techniques while maintaining the explainable nature of \revision{conformance checking}.

In summary, the contributions of this paper are as follows.
\begin{itemize}
    \item{
        A novel \revision{process mining}-based feature extraction approach to support the development of competitive and explainable \revision{conformance checking}-based techniques for control-flow anomaly detection.
    }
    \item{
        A flexible and explainable framework for developing techniques for control-flow anomaly detection using \revision{process mining}-based feature extraction and dimensionality reduction.
    }
    \item{
        Application to synthetic and real-world datasets of several \revision{conformance checking}-based and \revision{conformance checking}-independent framework techniques, evaluating their detection effectiveness and explainability.
    }
\end{itemize}

The rest of the paper is organized as follows.
\begin{itemize}
    \item Section \ref{sec:related_work} reviews the existing techniques for control-flow anomaly detection, categorizing them into \revision{conformance checking}-based and \revision{conformance checking}-independent techniques.
    \item Section \ref{sec:abccfe} provides the preliminaries of \revision{process mining} to establish the notation used throughout the paper, and delves into the details of the proposed \revision{process mining}-based feature extraction approach with alignment-based \revision{conformance checking}.
    \item Section \ref{sec:framework} describes the framework for developing \revision{conformance checking}-based and \revision{conformance checking}-independent techniques for control-flow anomaly detection that combine \revision{process mining}-based feature extraction and dimensionality reduction.
    \item Section \ref{sec:evaluation} presents the experiments conducted with multiple framework and baseline techniques using data from publicly available datasets and case studies.
    \item Section \ref{sec:conclusions} draws the conclusions and presents future work.
\end{itemize}
%!TeX root=paper.tex
\section{Main results}


\subsection{Algorithm}\label{sec:alg}
We analyze the simple linear estimation based algorithm from \cite{vershyninPlan} for generalized linear measurements, specializing it for binary vectors.
The algorithm (Algorithm~\ref{alg:1}) takes the sensing matrix $\vecA$ and the output vector $\by$ as the inputs. 
% The output $\by$ is set to $\by$ (see \eqref{eq:spl}) for \spl\ and to $\sign{\by}$  for \bcs. 
For each column $\vecA_i,\, i\in [1:n]$ of the sensing matrix, the algorithm computes $l_i = \ipr{\by}{\vecA_i} = \sum_{j = 1}^{m}y_jA_{j,i}$ where $A_{j,i}$ is the entry at $j^{\text{th}}$ row and $i^{\text{th}}$ column.

The vector $\mathbf{l} = \inp{l_1, \ldots, l_n}$ is then sorted in decreasing order. The output of the algorithm is a set containing the indices of the top-$k$ elements of the sorted vector. That is, if the sorted vector is $\inp{l_{\alpha_1}, l_{\alpha_2}, \ldots, l_{\alpha_n}}$ where $l_{\alpha_i}\geq l_{\alpha_j}$ for $i\leq j$, then the output of the algorithm is  $\cS = \inb{\alpha_1, \ldots, \alpha_k}$.



\begin{algorithm}[tbh!]
   \caption{Top-$k$ correlated indices}
   \label{alg:1}
\begin{algorithmic}
   \STATE {\bfseries Input:} Sensing matrix $\vecA\in \bbR^{m\times n}$ and output $\mathbf{y}\in \bbR^{m}$ 
   \STATE {\bfseries Output:} a $k$-sized subset of $[1:n]$
   \STATE $\mathbf{l} \gets (0, \ldots, 0)$,\, $\mathbf{l}\in \reals^n$
        % \STATE $\mathbf{l} \gets (0, \ldots, 0)$,\, $\mathbf{l}\in \reals^n$
    \FOR{each $i\in [1:n]$}
      \STATE $l_i \gets \sum_{j = 1}^{m}y_jA_{j,i}$ 
    \ENDFOR
    \STATE Sort $\mathbf{l}$ in decreasing order and let $\cS$ be the top $k$ indices.\\
    \STATE {\bfseries Return:} $\mathcal{\cS}$ 
   % \REPEAT
   % \STATE Initialize $noChange = true$.
   % \FOR{$i=1$ {\bfseries to} $m-1$}
   % \IF{$x_i > x_{i+1}$}
   % \STATE Swap $x_i$ and $x_{i+1}$
   % \STATE $noChange = false$
   % \ENDIF
   % \ENDFOR
   % \UNTIL{$noChange$ is $true$}
\end{algorithmic}
\end{algorithm}


% \begin{algorithm}
%   \caption{Top-$k$ correlated indices}\label{alg:1}
%    \hspace*{\algorithmicindent} \textbf{Input:} Sensing matrix $\vecA\in \bbR^{m\times n}$ and output $\mathbf{w}\in \bbR^{m}$ \\
% \hspace*{\algorithmicindent} \textbf{Output:} a $k$-sized subset of $[1:n]$ \\
% \vspace{-0.4cm}
% \begin{algorithmic}[1]
%     \State $\mathbf{l} \gets (0, \ldots, 0)$,\, $\mathbf{l}\in \reals^n$
%     \For{each $i\in [1:n]$}
%       \State $l_i \gets \sum_{j = 1}^{m}A_{j,i}y_j$ 
%     \EndFor
%     \State Sort $\mathbf{l}$ in decreasing order and let $\cS$ be the top $k$ indices.\\
%     \Return $\mathcal{\cS}$ 
% \end{algorithmic}
% \end{algorithm}

The convergence and sample complexity guarantees for the algorithm are shown for the case when each entry of $\vecA$ is chosen iid $\cN(0,1)$. Note that such a matrix satisfies the power constraint in \eqref{eq:power_constraint}. As we argued in Section~\ref{sec:intro}, for the unknown signal $\bx$,  the output $\by = \vecA{\bx}+\bz$ is correlated with each column $\vecA_i$ for $i\in \cS_{\bx}$ and uncorrelated with $\vecA_j$ for $j\notin \cS_{\bx}$. In particular, for large number of samples, when $i\in \cS_{\bx}$, the inner product $\ipr{\by}{\vecA_i}$ is close to $\bbE\insq{\ipr{\by}{\vecA_i}} = m$ (for linear regression) with high probability. On the other hand, $\ipr{\by}{\vecA_j}$ is close to $0$ for $j\notin \cS_{\bx}$.  Thus, $l_i$ for $i\in \cS_{\bx}$ will dominate over $l_j$ for $j\notin \cS_{\bx}$. This line of argument also works when the output is binary, though in this case $\bbE\insq{\ipr{{\by}}{\vecA_i}}$ for $i\in \cS_{\bx}$ is different. 
This is the main idea of Algorithm~\ref{alg:1}. We first present Theorem~\ref{thm:alg_general} for generalized linear measurements.
% Theorem~\ref{thm:alg_bcs} and Theorem~\ref{thm:alg_spl} formalize this intuition for \bcs\ and \spl\ respectively. 
\begin{theorem}[Sample Complexity of Algorithm~\ref{alg:1} for GLMs]\label{thm:alg_general}
Suppose the GLM is such that for each $i\in [m]$, $y_i$ is a subgaussian random variable with subgaussian norm given by $\normi{{y_i}}_{\psi_2}$. For any $\bx$, suppose for some $L$, $\bbE\insq{g'(\vecA_i^T\bx)}\geq L\cdot
\normi{{y_i}}_{\psi_2}$   for all $i\in [m]$. Algorithm~\ref{alg:1} recovers the unknown signal with high probability if 
\begin{align}
m \geq \frac{C}{{\min\inb{L, L^2}}}(\log\inp{k}+\log\inp{n-k})\label{eq: alg_bound}
\end{align} where $C$ is some constant.
\end{theorem}
When $y_j$ is subgaussian, $y_j\vecA_{i,j}$ for any $i,j$ is a sub-exponential random variable. This observation allows us to use a concentration result for sub-exponential random variables to analyse the sample complexity. See Section~\ref{sec:proofs} for a detailed proof. 

As corollaries to Theorem~\ref{thm:alg_general}, we obtain the following sample complexity bounds for \bcs\ and \spl. These corollaries are proved in Appendix~\ref{proof:sec:alg}.
\begin{corollary}[Sample Complexity of Algorithm~\ref{alg:1} for \bcs]\label{thm:alg_bcs}
Algorithm~\ref{alg:1} recovers the unknown signal for \bcs\ with high probability if $m=O\inp{\inp{k+\sigma^2}(\log\inp{k}+\log\inp{n-k})}$.
\end{corollary}
\begin{corollary}[Sample Complexity of Algorithm~\ref{alg:1} for \spl]\label{thm:alg_spl}
Algorithm~\ref{alg:1} recovers the unknown signal for \spl\ if $m=O\inp{\inp{k+\sigma^2}(\log\inp{k}+\log\inp{n-k})}$.
\end{corollary}
 Interestingly, the sample complexity for both \bcs\ and \spl\ is the same. This can be explained by similar values of $L$, which result in similar rates of concentration of $l_i$'s around their expectation in both the cases. 
This also implies that in the regime where $m = O((k+\sigma^2)\log(n-k))$, having access to $\vecA_i^T \bx+z_i$ instead of $\sign{\vecA_i^T \bx+z_i}$, does not improve the sample complexity beyond constants.

Using Theorem~\ref{thm:alg_general}, we obtain the following corollary for logistic regression (see proof in Appendix~\ref{proof:sec:alg}).
\begin{corollary}[Sample Complexity of Algorithm~\ref{alg:1} for \logreg]\label{thm:alg_logreg}
Algorithm~\ref{alg:1} recovers the unknown signal for \logreg\ if $m=O\inp{\inp{k+1/\beta^2}\inp{\log{k}+\log\inp{n-k}}}$.
\end{corollary}
Comparing the sample complexity bounds of \bcs\ and \logreg, we notice that the sample complexity is similar except that   the noise variance $\sigma^2$ is replaced by $1/\beta^2$. This relationship is not surprising as a similar relationship was also present in the sample complexity bounds in \cite{hsu2024sample} (for logistic regression) and \cite{kuchelmeister2024finite} (for probit model). Note that, in the noiseless case, when $\beta\rightarrow \infty$ (or $\sigma = 0$ for \bcs), the sample complexity is $O(k\log{n})$, which is close to the simple counting lower bound of $k\log{n/k}$. On the other hand, when $\beta = 0$ (or $\sigma\rightarrow \infty$ for \bcs), $m\rightarrow \infty$, which makes intuitive sense as very high levels of noise render the output useless.


To compute the time complexity of the algorithm, notice that the for loop in step 2 takes $O(n\times m)$ time and step 4 takes $O(n\log{n})$ time. Thus, the computational complexity of the algorithm is $O(nm+n\log{n})$, which is $O((k+\sigma^2)n\log{n})$ for $m = O((k+\sigma^2)\log{n}$.
To compute the time complexity of the algorithm, notice that the for loop in step 2 takes $O(n\times m)$ time and step 4 takes $O(n\log{n})$ time. Thus, the computational complexity of the algorithm is $O(nm+n\log{n})$, which is $O((k+\sigma^2)n\log{n})$ for $m = O((k+\sigma^2)\log{n}$.




\subsection{Lower bounds on sample complexity}\label{sec:sample_compexity}
We establish a lower bound for generalized linear measurements using standard information-theoretic arguments based on Fano's inequality. While the upper bound in Theorem~\ref{thm:alg_general} is derived for the maximum probability of error over all  $k$-sparse vectors, the lower bound applies even in the weaker setting of the average probability of error, where 
$\bx$ is chosen uniformly at random.
\begin{theorem}[Lower bound for GLMs]\label{thm: lower_bdglm} Consider any  sensing matrix $\vecA$.
For a uniformly chosen $k$-sparse vector $\bx$, an algorithm $\phi$ satisfies $$\bbP\inp{\phi(\vecA, \by) \neq \bx}\leq \delta$$   only if the number of measurements $$m\geq \frac{k\log\inp{\frac{n}{k}}}{I}\inp{1 - \frac{h_2(\delta) + \delta k\log{n}}{k\log{n/k}}}$$ for some $I$ such that $I\geq {I(y_i; \bx|\vecA)}, \, i\in [m]$. In particular, when $y\in \inb{-1, 1}$, we have $\bbE\insq{\inp{g(\vecA_i^T\bx)}^2} \geq I(y_i, \bx|\vecA)$ where the expectation is over the randomness of $\vecA$ and $\bx$.
\end{theorem}
The lower bound can be interpreted in terms of a communication problem, where the input message $\bx$ is encoded to $\vecA\bx$. The decoding function takes in as input the encoding map $\vecA$ and the output vector $\by$ in order to recover $\bx$ with high probability. For optimal recovery, one needs at least $\frac{\text{message entropy}}{\text{capacity}}$ number of measurements (follows from noisy channel coding theorem~\cite{thomas2006elements}). In Theorem~\ref{thm: lower_bdglm}, the entropy of the message set $\log{n \choose k}\approx k\log{n/k}$ and the proxy for capacity is the upper bound on mutual information $I$. We provide a detailed proof of the theorem in  Section~\ref{sec:proofs}.


We first present lower bounds for \bcs\  and \logreg. The lower bound for \bcs\ is given for any sensing matrix $\vecA$ which satisfies the power constraint given by \eqref{eq:power_constraint}, whereas the one for \logreg\ is only for the special case when each entry of the sensing matrix is iid $\cN(0,1)$. Recall that \eqref{eq:power_constraint} holds in this case.  For \bcs\ (and \logreg\ respectively), we can use the upper bound of $\bbE\insq{\inp{g(\vecA_i^T\bx)}^2}$ on the mutual information term. The dependence of $\sigma^2$ (and $1/\beta^2$ respectively) requires careful bounding of this term, which is done in the formal proofs in Appendix~\ref{proof:sec:lower_bd}.


As mentioned earlier, we need at least $k\log\inp{n/k}$ measurements for \bcs and \logreg. This is because the entropy of a randomly chosen $k$-sparse vector is approximately $k\log\inp{n/k}$ and we learn at most one bit with each measurement. However, due to corruption with noise, we learn less than a bit of information about the unknown signal with each measurement. The information gain gets worse as the noise level increases. 
Our lower bounds make this reasoning explicit.  
\begin{corollary}[\bcs\ lower bound]\label{thm: lower_bd_bcs} Suppose, each row $\vecA_i, \, i\in [1:m]$ of the sensing matrix $\vecA$ satisfies the power constraint~\eqref{eq:power_constraint}.
For a uniformly chosen $k$-sparse vector $\bx$, an algorithm $\phi$ satisfies $$\bbP\inp{\phi(\vecA, {\by}) \neq \bx}\leq \delta$$ for the problem of $\bcs$ only if the number of measurements $$m\geq \frac{k+\sigma^2}{2}\log\inp{\frac{n}{k}}\inp{1 - \frac{h_2(\delta) + \delta k\log{n}}{k\log{n/k}}}.$$ 
\end{corollary}

\begin{corollary}[\logreg\ lower bound]\label{thm: lower_bd_log_reg} Consider a Gaussian  sensing matrix $\vecA$ where each entry is chosen iid $N(0,1)$.
For a uniformly chosen $k$-sparse vector $\bx$, an algorithm $\phi$ satisfies $$\bbP\inp{\phi(\vecA, \bw) \neq \bx}\leq \delta$$ for the problem of $\logreg$ only if the number of measurements $$m\geq \frac{1}{2}\inp{k+\frac{1}{\beta^2}}\log\inp{\frac{n}{k}}\inp{1 - \frac{h_2(\delta) + \delta k\log{n}}{k\log{n/k}}}.$$ 
\end{corollary}



Theorem~\ref{thm: lower_bdglm} also implies an information theoretic lower bound for \spl, which is presented below and proved in Appendix~\ref{proof:sec:lower_bd}. Note that the denominator term in the bound $\frac{1}{2}\log\inp{1+\frac{k}{\sigma^2}}$ is the capacity of a Gaussian channel with power constraint $k$ and noise variance $\sigma^2$. 
\begin{corollary}[\spl\ lower bound]\label{thm: spl_lower_bd_1}
Under the average power constraint \eqref{eq:power_constraint} on  $\vecA$, for a uniformly chosen $k$-sparse vector $\bx$, an algorithm $\phi$ satisfies $$\bbP\inp{\phi(\vecA, {\by}) \neq \bx}\leq \delta$$ only if the number of measurements
$$m\geq \frac{k\log\inp{\frac{n}{k}}-\inp{h_2(\delta) + \delta k\log{n}}}{\frac{1}{2}\log\inp{1+\frac{k}{\sigma^2}}}.$$
\end{corollary} 

\subsection{Tighter upper and lower bounds for \spl}\label{sec:tighter_bounds_spl}
We present information theoretic upper and lower bounds for \spl\ in this section. Similar to Section~\ref{sec:alg}, our upper bound is for the maximum probability of error, while the lower bounds hold even for the weaker criterion of average probability of error.

We first present an upper bound based on the maximum likelihood estimator (MLE) where  we  decode to $\hat{\bx}$ if, on output $\by$, 
\begin{align*}
\hat{\bx} = \argmax_{\stackrel{\bx\in \inb{0,1}^n}{\wh{\bx} = k}}\,\, p(\by|{\bx})
\end{align*} where $p(\by|{\bx})$ denotes the probability density function of $\by$ on input $\bx$.
\begin{theorem}[MLE upper bound for \spl]\label{thm:upper_bd_mle} Suppose  entries of the measurement matrix $\vecA$ are i.i.d. $\cN(0,1).$
The MLE  is correct with high probability if 
\begin{align}m\geq \max_{l\in[1:k]}  \frac{nN(l)}{\frac{1}{2}\log\inp{\frac{ l}{2\sigma^2}+1}}\label{eq:upper_bd_mle}
\end{align}where  $N(l):=  \frac{k}{n} h_2\inp{\frac{l}{k}} + (1-\frac{k}{n})h_2\inp{\frac{l}{n-k}}$. 
\end{theorem}
We prove the theorem in Appendix~\ref{proof:MLE}. The main proof idea involves analysing the probability that the output of the MLE is $2l$ Hamming distance away from the unknown signal $\bx$ for different values of $l\in [1:k]$ (assuming $k\leq n/2$). This depends on the number of such vectors (approximately $2^{nN(l)}$) and the probability that the MLE outputs a vector which is $2l$ Hamming distance away from $\bx$. 

Note that when $l = k\inp{1-\frac{k}{n}}$, $nN(l) = nh_2(k/n)\approx k\log{\frac{n}{k}}$ and $\log\inp{\frac{k\inp{1-k/n}}{2\sigma^2}+1}\leq \log\inp{\frac{k}{2\sigma^2}+1}$.
Thus, $m$ is at least $\frac{2k\log{n/k}}{\log\inp{\frac{k}{2\sigma^2}+1}}$ (see the bound for Corollary~\ref{thm: spl_lower_bd_1}). It is not immediately clear if this value of $l= k\inp{1-\frac{k}{n}}$ is the optimizer. However, for large $n$, this appears to be the case numerically as shown in Plot~\ref{plot:1}.

\begin{figure}[t]
\includegraphics[width=7cm]{Unknown2.png}
\centering
\caption{The figure shows the plot of the MLE upper bound \eqref{eq:upper_bd_mle} (given by m1) for different values of $k$. This is displayed in blue color. A plot of $\frac{2nN(l)}{\log\inp{\frac{ l}{2\sigma^2}+1}}$ is also presented for $l = k\inp{1-\frac{k}{n}}$ in orange color, given by m2. A part of the plot is zoomed in to emphasize the closeness between the lines. In these plots,  $\sigma^2$ is set to 1,  $n$ is 50000 and $k$ ranges from 1000 to 25000 $(n/2)$. }\label{plot:1}
\end{figure}


Inspired by the MLE analysis, we derive a lower bound with the same structure as \eqref{eq:upper_bd_mle}. We generate the unknown signal $\bx$ using the following distribution: A vector $\tilde{\bx}$ is chosen uniformly at random from the set of all $k$-sparse vectors. Given $\tilde{\bx}$, the unknown input signal $\bx$ is chosen uniformly from the set of all $k$-sparse vector which are at a Hamming distance $2l$ from $\bx$. 
The lower bound is then obtained by computing upper and lower bounds on $I(\vecA, \by;\bx|\tilde{\bx})$.
We show this lower bound only for random matrices where each entry is chosen iid $\cN(0,1)$.
\begin{theorem}[\spl\ lower bound]\label{thm:lower_bd_spl}
If each entry of $\vecA$ is chosen iid $\cN(0,1)$, then for a uniformly chosen $k$-sparse vector $\bx$, an algorithm $\phi$ satisfies 
\begin{align}
    \bbP\inp{\phi(\vecA, {\by}) \neq \bx}\leq \delta\label{eq:spl_lower_bd_l}
\end{align}  only if the number of measurements $$m\geq \max_l\frac{nN(l) - 2\log{n}- h_2(\delta) - \delta k\log{n}}{\frac{1}{2}\log\inp{1+\frac{l}{\sigma^2}\inp{2-\frac{l}{k}}}} .$$
\end{theorem} The proof of Theorem~\ref{thm:lower_bd_spl} is given in Appendix~\ref{proof:MLE}.

If we choose $l = k\inp{1-\frac{k}{n}}$ in Theorem~\ref{thm:lower_bd_spl}, we recover corollary~\ref{thm: spl_lower_bd_1} for the special case of Gaussian design.
% \begin{corollary}\label{corollary2:lower_bd_spl}
% If  each entry of $\vecA$ is chosen iid $\cN(0,1)$, then for a uniformly chosen $k$-sparse vector $\bx$, an algorithm $\phi$ satisfies 
% $$\bbP\inp{\phi(\vecA, {\by}) \neq \bx}\leq \delta$$
% only if the number of measurements 
% $$m\geq \frac{k\log\inp{\frac{n}{k}} - 2\log{n}- h_2(\delta) - \delta k\log{n}}{\log\inp{1+\frac{k}{\sigma^2}}} .$$
% \end{corollary}

% Corollary~\ref{corollary2:lower_bd_spl} can also be proved directly for any sensing matrix $\vecA$ which satisfies \eqref{eq:power_constraint} (non-necessarily a Gaussian design). 


% \begin{figure}[t]
% \includegraphics[width=8cm]{plot.png}
% \centering
% \caption{The figure shows the plot of the MLE upper bound \eqref{eq:upper_bd_mle} (given by m1) for different values of $n$. This is displayed in blue color. A plot of $\frac{2nN(l)}{\log\inp{\frac{ l}{2\sigma^2}+1}}$ is also presented for $l = k\inp{1-\frac{k}{n}}$ in orange color, given by m2. In these plots,  $\sigma^2$ is set to 1 and $k$ is $0.2n$. }\label{plot:1}
% \end{figure}


\section{Proofs}\label{sec:proofs}
\begin{proof}[Proof of Theorem~\ref{thm:alg_general}]
Consider any input $\bx$ and a sensing matrix $\vecA$ where each entry is chosen iid $\cN(0,1)$. Suppose ${\bx}$ is supported on $\cS\subseteq[1:n]$ where $|\cS|=k$. Let $\by = (y_1, \ldots, y_m)$. Consider the event
\begin{align*}
    \cF = \inb{\sum_{i=1}^m{ {y_i A_{i,j}}}> \sum_{i=1}^m{ {y_i A_{i,j'}}}\text{ for all } j \in \cS,  j'\in\cS^c}
\end{align*}
It is clear that under $\cF$, the algorithm is correct.
We will compute the probability of $\cF^c$.
\begin{align}
\bbP\inp{\cF^c} &= \bbP\inp{\bigcup_{j\in\cS}\bigcup_{j'\in \cS^c}\inb{\sum_{i=1}^m{ {y_i A_{i,j'}}}\geq \sum_{i=1}^m{ {y_i A_{i,j}}}}}\nonumber\\
&\leq \sum_{j\in\cS}\sum_{j'\in \cS^c}\bbP\inp{\sum_{i=1}^m{ {y_i A_{i,j'}}}\geq \sum_{i=1}^m{ {y_i A_{i,j}}}}\nonumber\\
&=\sum_{j\in\cS}\sum_{j'\in \cS^c}\bbP\inp{\sum_{i=1}^m{ \inp{y_i (A_{i,j'}-A_{i,j})}}\geq 0}\label{eq:log_prob_f^c1}
\end{align}

For any $i\in[1:m]$, $j\in\cS$ and $j'\in \cS^c$, we first compute $\bbE\insq{y_i (A_{i,j}-A_{i,j'})}$.
\begin{align}
\bbE\insq{y_i (A_{i,j}-A_{i,j'})} &= \bbE\insq{y_i A_{i,j}}-\bbE\insq{y_iA_{i,j'}}\nonumber\\
& \stackrel{(a)}{=} \bbE\insq{y_i A_{i,j}} \label{eq:log_expt11}\\
&\stackrel{(b)}{=}\frac{\bbE\insq{y_iA_{i,\cS}}}{k}\nonumber\\
& = \frac{\bbE\insq{y_i\vecA_{i}^T\bx}}{k}\nonumber\\
&= \frac{\bbE\insq{A_{i}^T\bx\,\bbE\insq{y_1|\vecA_i^T\bx}}}{k}\nonumber\\
& \stackrel{(c)}{=} \frac{\bbE\insq{A_{i}^T\bx g\inp{\vecA_i^T\bx}}}{k}\nonumber\\
& \stackrel{(d)}{=} {\bbE\insq{ g'\inp{\vecA_i^T\bx}}}:= E\label{eq:log_expt1_avg11}
\end{align} where $(a)$ follows from the fact that $y_i$ and  $A_{i,j'}$ are zero mean, independent random variables and $(b)$ follows by defining $A_{i,\cS} = \sum_{j\in \cS}A_{i,j}$ and noticing that the random variables $y_i A_{i,j}$ are identically distributed for all $j\in \cS$, $(c)$ follows from \eqref{eq:glm} and $(d)$ follows from Stein's lemma. 
\begin{align*}
\bbP&\inp{\sum_{i=1}^m{ \inp{y_i (A_{i,j'}-A_{i,j})}}\geq 0}\\
&=\bbP\inp{\sum_{i=1}^m{ \inp{y_i (A_{i,j}-A_{i,j'})}}\leq 0}\\
& = \bbP\inp{\sum_{i=1}^m{ \inp{y_i (A_{i,j}-A_{i,j'})}} - mE\leq -mE}\\
& \leq \bbP\inp{\left|\sum_{i=1}^m{ \inp{y_i (A_{i,j}-A_{i,j'})}} - mE\right|\geq mE}\\
\end{align*}
\vspace{-0.001cm}
To compute this, note that for all $i\in [1:m]$, $y_i$  is a subgaussian random variable and ${y_i}\inp{A_{i, j}-A_{i, j'}}$ being product of two subgaussian random variables is a subexponential random variable (see [Lemma 2.7.7]\cite{vershynin}). Note that $\bbE\insq{\sum_{i=1}^m{ \inp{y_i (A_{i,j}-A_{i,j'})}}} = mE$ where $E$ was defined in \eqref{eq:log_expt1_avg11}.
Also, 
\begin{align*}
&\hspace{-0.3cm}\normi{{y_i}\inp{A_{i, j}-A_{i, j'}} - 1}_{\psi_1}\\
&\stackrel{(a)}{\leq} C\normi{{y_i}\inp{A_{i, j}-A_{i, j'}}}_{\psi_1}\\
&\stackrel{(b)}{\leq} C\normi{{y_i}}_{\psi_2}\normi{\inp{A_{i, j}-A_{i, j'}}}_{\psi_2}\\
&\stackrel{(c)}{\leq} C\normi{{y_i}}_{\psi_2}2C'\\
& = C_1\normi{{y_i}}_{\psi_2} \qquad\qquad\text{ for some constant $C_1$}.
\end{align*} Here, $(a)$ follows from [Exercise 2.7.10]\cite{vershynin}, $(b)$ from [Lemma 2.7.7]\cite{vershynin} and $(c)$ from [Example 2.5.8]\cite{vershynin}.
With this
\begin{align*}
\bbP&\inp{\left|\sum_{i=1}^m{ \inp{y_i (A_{i,j}-A_{i,j'})}} - mE\right|\geq mE}\\
&\stackrel{(a)}{\leq} 2\exp\inp{-c\min\inp{\frac{m^2E^2}{mC_1^2\normi{{y_i}}_{\psi_2}^2}, \frac{mE}{C_1\normi{{y_i}}_{\psi_2}} }}\\
&\stackrel{(b)}{\leq} 2\exp\inp{-cm\min\inp{\frac{mL^2}{C_1^2}, \frac{mL}{C_1 }}}
\end{align*} where $(a)$ follows from [Theorem 2.8.1]\cite{vershynin} and $(b)$ follows from the assumption in the lemma that $\frac{E}{\normi{{y_i}}_{\psi_2}} = \frac{\bbE\insq{g'(\vecA_i^T\bx)}}{\normi{{y_i}}_{\psi_2}}\geq L$.
Thus, from \eqref{eq:log_prob_f^c1}, 
\begin{align*}
\bbP\inp{\cF^c}&\leq k(n-k)2\exp\inp{-C_2 m\min\inp{L^2, L}}\\
&\rightarrow 0 \text{ if }m\geq C_2\inp{\log{k}+\log\inp{n-k}}\frac{1}{\min\inp{L^2, L}}
\end{align*} for some constant $C_2$.
\end{proof}


\iffalse
% \begin{proof}[Proof of Theorem~\ref{thm:alg_bcs}]
% Consider any input $\bx$ and a sensing matrix $\vecA$ where each entry is chosen iid $\cN(0,1)$. Suppose ${\bx}$ is supported on $\cS\subseteq[1:n]$ where $|\cS|=k$. Let $\bs = (s_1, \ldots, s_m)$ denote $\sign{\by}$. Consider the event
% \begin{align*}
%     \cF = \inb{\sum_{i=1}^m{ {s_i A_{i,j}}}> \sum_{i=1}^m{ {s_i A_{i,j'}}}\text{ for all } j \in \cS,  j'\in\cS^c}
% \end{align*}
% It is clear that under $\cF$, the algorithm is correct.
% We will compute the probability of $\cF^c$.
% % Note that
% % \begin{align*}
% %     \cF^c = \bigcup_{j\in\cS}\bigcup_{j'\in \cS^c}\inb{\sum_{i=1}^m{ {s_i A_{i,j'}}}\geq \sum_{i=1}^m{ {s_i A_{i,j}}}}.
% % \end{align*}
% % Thus,
% \begin{align}
% \bbP\inp{\cF^c} &= \bbP\inp{\bigcup_{j\in\cS}\bigcup_{j'\in \cS^c}\inb{\sum_{i=1}^m{ {s_i A_{i,j'}}}\geq \sum_{i=1}^m{ {s_i A_{i,j}}}}}\nonumber\\
% &\leq \sum_{j\in\cS}\sum_{j'\in \cS^c}\bbP\inp{\sum_{i=1}^m{ {s_i A_{i,j'}}}\geq \sum_{i=1}^m{ {s_i A_{i,j}}}}\nonumber\\
% &=\sum_{j\in\cS}\sum_{j'\in \cS^c}\bbP\inp{\sum_{i=1}^m{ \inp{s_i (A_{i,j'}-A_{i,j})}}\geq 0}\label{eq:prob_f^c}
% \end{align}

% For any $i\in[1:m]$, $j\in\cS$ and $j'\in \cS^c$, we first compute $\bbE\insq{s_i (A_{i,j}-A_{i,j'})}$.
% \begin{align}
% \bbE\insq{s_i (A_{i,j}-A_{i,j'})} &= \bbE\insq{s_i A_{i,j}}-\bbE\insq{s_iA_{i,j'}}\nonumber\\
% & \stackrel{(a)}{=} \bbE\insq{ A_{i,j}\bbE\insq{s_i| A_{i,j}}} - 0\label{eq:expt1}
% \end{align} where $(a)$ follows from the fact that $s_i$ and  $A_{i,j'}$ are zero mean, independent random variables. 

% For any $\cU\subseteq[1:n]$, we denote $\sum_{l \in \cU}A_{i,l}$ by $A_{i,\cU}$.  For any $A_{i,j} = a$, 
% \begin{align*}
% \bbP&\inp{s_i = 1|A_{i,j} = a} \\
% &= \bbP\inp{A_{i,\cS\setminus\inb{j}}+z_i\geq -a} \\
% & = \bbP\inp{\frac{A_{i,\cS\setminus\inb{j}}+z_i}{\sqrt{k-1+\sigma^2}}\geq -\frac{a}{\sqrt{k-1+\sigma^2}}}\\
% & = 1-\Phi\inp{-\frac{a}{\sqrt{k-1+\sigma^2}}}
% \end{align*} where $\Phi(x) = \frac{1}{2\pi}\int_{-\infty}^{x}e^{-\frac{t^2}{2}}dt$ is the cumulative distribution function of the standard Gaussian distribution. Thus, 
% $\bbP\inp{s_i = -1|A_{i,j} = a}  = \Phi\inp{-\frac{a}{\sqrt{k-1+\sigma^2}}}$ and 
% $$\bbE\insq{s_i| A_{i,j}=a} = 1-2\Phi\inp{-\frac{a}{\sqrt{k-1+\sigma^2}}}.$$ We are now ready to compute $\bbE\insq{ A_{i,j}\bbE\insq{s_i| A_{i,j}}}$.
% \begin{align}
% \bbE&\insq{ A_{i,j}\bbE\insq{s_i| A_{i,j}}} \nonumber\\
% &= \bbE\insq{ A_{i,j}\inp{1-2\Phi\inp{-\frac{A_{i,j}}{\sqrt{k-1+\sigma^2}}}}}\nonumber\\
% &= \bbE\insq{ A_{i,j}}-2\bbE\insq{A_{i,j}\Phi\inp{-\frac{A_{i,j}}{\sqrt{k-1+\sigma^2}}}}\nonumber\\
% &= 0-2\bbE\insq{A_{i,j}\Phi\inp{-\frac{A_{i,j}}{\sqrt{k-1+\sigma^2}}}}\label{eq:expt2}
% \end{align}
% \begin{align}
% \bbE&\insq{A_{i,j}\Phi\inp{-\frac{A_{i,j}}{\sqrt{k-1+\sigma^2}}}}\nonumber\\
% & = \int_{-\infty}^{\infty}a\frac{1}{\sqrt{2\pi}}e^{-\frac{a^2}{2}}\inp{\frac{1}{\sqrt{2\pi}}\int_{-\infty}^{-\frac{a}{\sqrt{k-1+\sigma^2}}}e^{-\frac{t^2}{2}}dt}da\nonumber\\
% & = \frac{1}{2\pi}\int_{-\infty}^{\infty}\int_{-\infty}^{-\frac{a}{\sqrt{k-1+\sigma^2}}}ae^{-\frac{a^2}{2}}e^{-\frac{t^2}{2}}dt\,da\nonumber\\
% & \stackrel{(a)}{=} \frac{1}{2\pi}\int_{-\infty}^{\infty}\int_{-\infty}^{-{t}{\sqrt{k-1+\sigma^2}}}ae^{-\frac{a^2}{2}}e^{-\frac{t^2}{2}}da\,dt\nonumber\\
% & =\frac{1}{2\pi}\int_{-\infty}^{\infty}\inp{\int_{-\infty}^{-{t}{\sqrt{k-1+\sigma^2}}}ae^{-\frac{a^2}{2}}da}e^{-\frac{t^2}{2}}dt\nonumber\\
% & = \frac{1}{2\pi}\int_{-\infty}^{\infty}\inp{-e^{-\frac{t^2(k-1+\sigma^2)}{2}}}e^{-\frac{t^2}{2}}dt\nonumber\\
% & = -\frac{1}{\sqrt{2\pi\inp{k+\sigma^2}}}\int_{-\infty}^{\infty}\frac{\sqrt{k+\sigma^2}}{\sqrt{2\pi}}e^{-\frac{t^2(k+\sigma^2)}{2}}dt\nonumber\\
% & = -\frac{1}{\sqrt{2\pi\inp{k+\sigma^2}}}\label{eq:expt3}
% \end{align} where $(a)$ follows for change of variable formula for integration. From \eqref{eq:expt1},\eqref{eq:expt2} and \eqref{eq:expt3}, we have 
% \begin{align}
% \bbE\insq{s_i (A_{i,j}-A_{i,j'})} = \sqrt{\frac{2}{\pi}}\times\frac{1}{\sqrt{\inp{k+\sigma^2}}}.\label{eq:expt4}
% \end{align}
% Define $E := \sqrt{\frac{2}{\pi}}\times\frac{1}{\sqrt{\inp{k+\sigma^2}}}$. Then,
% \begin{align*}
% \bbP&\inp{\sum_{i=1}^m{ \inp{s_i (A_{i,j'}-A_{i,j})}}\geq 0}\\
% &=\bbP\inp{\sum_{i=1}^m{ \inp{s_i (A_{i,j}-A_{i,j'})}}\leq 0}\\
% & = \bbP\inp{\sum_{i=1}^m{ \inp{s_i (A_{i,j}-A_{i,j'})}} - mE\leq -mE}\\
% & \leq \bbP\inp{\left|\sum_{i=1}^m{ \inp{s_i (A_{i,j}-A_{i,j'})}} - mE\right|\geq mE}\\
% \end{align*}
% \vspace{-0.001cm}
% To compute this, notice that for all $i\in [1:m]$, $s_i$  is a subgaussian random variable (see \cite[Example 2.5.8]{vershynin}) and ${s_i}\inp{A_{i, j}-A_{i, j'}}$ being product of two subgaussian random variables is a subexponential random variable (see \cite[
% Lemma 2.7.7]{vershynin}). Note that $\bbE\insq{\sum_{i=1}^m{ \inp{s_i (A_{i,j}-A_{i,j'})}}} = mE$.
% Also, 
% \begin{align*}
% &\hspace{-0.3cm}\normi{{s_i}\inp{A_{i, j}-A_{i, j'}} - 1}_{\psi_1}\\
% &\stackrel{(a)}{\leq} C\normi{{s_i}\inp{A_{i, j}-A_{i, j'}}}_{\psi_1}\\
% &\stackrel{(b)}{\leq} C\normi{{s_i}}_{\psi_2}\normi{\inp{A_{i, j}-A_{i, j'}}}_{\psi_2}\\
% &\stackrel{(c)}{\leq} C\frac{1}{\ln{2}}2\\
% & = C_1 \qquad\qquad\text{ for some constant $C_1$}.
% \end{align*} Here, $(a)$ follows from \cite[Exercise 2.7.10]{vershynin}, $(b)$ from \cite[
% Lemma 2.7.7]{vershynin} and $(c)$ from \cite[Example 2.5.8]{vershynin}.
% With this
% \begin{align*}
% \bbP&\inp{\left|\sum_{i=1}^m{ \inp{s_i (A_{i,j}-A_{i,j'})}} - mE\right|\geq mE}\\
% &\stackrel{(a)}{\leq} 2\exp\inp{-c\min\inp{\frac{m^2E^2}{mC_1^2}, \frac{mE}{C_1} }}\\
% % &= 2\exp\inp{-c\min\inp{\frac{2m}{\pi C_1^2\inp{k+\sigma^2}}, \frac{m\sqrt{2}}{\sqrt{\pi}C_1\sqrt{\inp{k+\sigma^2}}} }}\\
% &\stackrel{(b)}{\leq} 2\exp\inp{-c\inp{\frac{2m}{\pi C_1^2\inp{k+\sigma^2}}}}
% \end{align*} where $(a)$ follows from \cite[Theorem 2.8.1]{vershynin} and $(b)$ follows by substituting value of $E$ and noting that  $ x\geq \sqrt{x}$ when $x\geq 1$.
% From \eqref{eq:prob_f^c}, 
% Thus,
% \begin{align*}
% \bbP\inp{\cF^c}&\leq k(n-k)2\exp\inp{-c\inp{\frac{2m}{\pi C_1^2\inp{k+\sigma^2}}}}\\
% &\rightarrow 0 \text{ if }m\geq C_2\inp{\log{k}+\log\inp{n-k}}\inp{k+\sigma^2}.
% \end{align*} for some constant $C_2$.


% \end{proof}

\fi


\begin{proof}[Proof of Theorem~\ref{thm: lower_bdglm}]
Suppose $\bx$ is distributed uniformly on the set of all $k$-sparse binary vectors. Then,
\begin{align}
I(\vecA, \by; \bx)  &= H(\bx) - H(\bx|\vecA, \by)\nonumber\\
&\stackrel{(a)}{\geq} \log{n \choose k } - h_2(\delta) - \delta\log\inp{{n \choose k } + 1}\nonumber\\
&\geq k\log{n/k}- h_2(\delta)- \delta k\log\inp{n}\label{eq:log_lower_bd_bcy_1}
\end{align} where $(a)$ follows from Fano's inequality [Theorem~2.10.1]\cite{thomas2006elements}.
We also note that
\begin{align*}
 I(\vecA, \by; \bx) &= I(\vecA ; \bx)    +  I(\by; \bx|\vecA)\\
 &\stackrel{(a)}{ = }0 + I(\by; \bx|\vecA).
\end{align*}where $(a)$ holds because $\vecA$ and $\bx$ are independent. Let $y_{j\in[1:i-1]}$ denote $\inp{y_1, \ldots, y_{i-1}}$. 
\begin{align}
I&(\by; \bx|\vecA) = \sum_{i = 1}^{m}I(y_i; \bx|\vecA, y_{j\in[1:i-1]})\nonumber\\
& = \sum_{i = 1}^{m}\Big(H(y_i|\vecA, y_{j\in[1:i-1]})\nonumber\\
&\qquad- H(y_i| \bx,\vecA, y_{j\in[1:i-1]})\Big)\nonumber\\
& \stackrel{(a)}{\leq}\sum_{i = 1}^{m}\inp{H(y_i|\vecA)- H(y_i| \bx,\vecA)}\nonumber\\
& = \sum_{i = 1}^{m}I(y_i;\bx|\vecA)\nonumber\\
&\stackrel{(b)}{\leq} mI \label{eq:log_lower_bd_bg}
\end{align}where $(a)$ follows from $H(y_i|\vecA, y_{j\in[1:i-1]})\leq H(y_i|\vecA)$ and $H(y_i| \bx,\vecA, y_{j\in[1:i-1]}) = H(y_i| \bx,\vecA)$ as $y_i$ is conditionally independent of $y_{j\in[1:i-1]}$ conditioned on $\bx$ and $\vecA$ and $(b)$ follows from the assumption in the Theorem. Thus, from \eqref{eq:log_lower_bd_bcy_1} and \eqref{eq:log_lower_bd_bg},
\begin{align*}
% m\frac{2k}{\pi\sigma^2} &\geq k\log{n/k}- h_2(\delta)- \delta k\log\inp{n}\\
mI &\geq  k\log\inp{n/k}\inp{1-\frac{h_2(\delta)+ \delta k\log\inp{n}}{k\log{n/k}}}
\end{align*}
This gives us the desired bound.

We can further simplify $I(y_i;\bx|\vecA)$ when $y_i\in \inb{-1,1}$, 
\begin{align*}
I(y_i;\bx|\vecA) &= H(y_i|\vecA)- H(y_i|\bx, \vecA)\\
&\stackrel{(a)}{\leq} 1- H(y_i|\bx, \vecA_i).
\end{align*}where $(a)$ holds because $H(y_i|\vecA)\leq H(y_i) =1$ and $y_i$ is conditionally independent of $(\vecA_1\ldots, \vecA_{i-1}, \vecA_{i+1}, \ldots, \vecA_{m})$ conditioned on $\vecA_i$ and $\bx$. Here $\vecA_i$, $i\in [1:m]$ denotes the $i^{\text{th}}$ row of the sensing matrix $\vecA$.

Suppose $\bx$ is fixed and $\bbP\inp{y_i = 1} = \frac{1}{2} + t$ for some $t\in [-1,1]$.  Then $\bbE\insq{y_i|\vecA_i} = 2t = g(\vecA_i^T\bx)$.
\begin{align*}
H(y_i|\vecA_i, \bx) &\stackrel{(a)}{=} \bbE\insq{h_2\inp{\frac{1}{2} + t}}\\
& \stackrel{(b)}{\geq} \bbE_{\bx}\insq{\bbE_{\vecA}\insq{4\inp{\frac{1}{2} + t}\inp{\frac{1}{2} - t}\Big|\bx}}\\
& = 1- \bbE_{\bx}\insq{\bbE\insq{\inp{2t}^2\Big|\bx}}\\
& = 1- \bbE_{\bx}\insq{\bbE\insq{\inp{g(\vecA_i^T\bx)}^2\Big|\bx}}\\
& = 1- \bbE_{\vecA, \bx}\insq{\inp{g(\vecA_i^T\bx)}^2}
\end{align*}
where in $(a)$, the expectation is over $\vecA$ and $\bx$. The inequality $(b)$ follows from [Theorem 1.2]\cite{topsoe2001bounds}. With this $I(y_i;\bx|\vecA)\leq \bbE\insq{\inp{g(\vecA_i^T\bx)}^2}$.
\end{proof}










\section{Conclusion and open problems}\label{sec:conclusion}
We analyze a simple algorithm (the ``average algorithm'' from \cite{vershyninPlan}  followed by `top-k' selection) for recovering sparse binary vectors from generalized linear measurements; along with an information theoretic lower bound. This gives optimal sample complexity characterization for \bcs\ and \logreg. On the other hand, the required number of measurements for the noisy linear case (\spl), which is $O((k+\sigma^2)\log{n})$, is as good as the sample complexity of any other known efficient algorithm for this problem, up to constants.  An interesting open problem is to find a design matrix and an efficient algorithm which requires less than $(k+\sigma^2)\log{n}$ samples for \spl. When the noise variance is zero, we show such an algorithm in  Remark~\ref{remark:noNoise}. 


We also present almost matching information theoretic upper and lower bounds for \spl\ given by  \eqref{eq:upper_bd_mle} and \eqref{eq:spl_lower_bd_l} respectively. The bounds are in the form of an optimization problem. While we present numerical evidence which suggests that \eqref{eq:upper_bd_mle} is optimized by $l = k\inp{1-\frac{k}{n}}$, a formal proof is still missing. The bounds in   \eqref{eq:upper_bd_mle} and \eqref{eq:spl_lower_bd_l} also differ slightly by constants in the denominator, which seems to be a persistent gap in this problem.


\paragraph{Acknowledgment}
This work is supported in part by NSF awards 2217058 and 2112665. The authors would like to thank Krishna Narayanan who introduced them to the binary linear regression problem at the Simons Institute program on Error-correcting codes.



\bibliography{example_paper}
\bibliographystyle{alpha}


%%%%%%%%%%%%%%%%%%%%%%%%%%%%%%%%%%%%%%%%%%%%%%%%%%%%%%%%%%%%%%%%%%%%%%%%%%%%%%%
%%%%%%%%%%%%%%%%%%%%%%%%%%%%%%%%%%%%%%%%%%%%%%%%%%%%%%%%%%%%%%%%%%%%%%%%%%%%%%%
% APPENDIX
%%%%%%%%%%%%%%%%%%%%%%%%%%%%%%%%%%%%%%%%%%%%%%%%%%%%%%%%%%%%%%%%%%%%%%%%%%%%%%%
%%%%%%%%%%%%%%%%%%%%%%%%%%%%%%%%%%%%%%%%%%%%%%%%%%%%%%%%%%%%%%%%%%%%%%%%%%%%%%%
\newpage
\appendix
\onecolumn
\section{Proofs}\label{appendix:proofs}
\subsection{Missing proofs from Section~\ref{sec:alg}}\label{proof:sec:alg}
\begin{proof}[Proof of Corollary~\ref{thm:alg_bcs}]
We need to compute a lower bound $L$ on  $\frac{\bbE\insq{g'(\vecA_i^T\bx)}}{\normi{{y_i}}_{\psi_2}}$.  Instead of computing $\bbE\insq{g'(\vecA_i^T\bx)}$, we will compute $\bbE\insq{y_iA_{i,j}}$ for  any $j$ in the support of $\bx$. From \eqref{eq:log_expt11} and \eqref{eq:log_expt1_avg11}, we note that $\bbE\insq{y_iA_{i,j}} = \bbE\insq{g'(\vecA_i^T\bx)}$.
Also note that
$\bbE\insq{y_i A_{i,j}}= \bbE\insq{ A_{i,j}\bbE\insq{y_i| A_{i,j}}}$.


For any $\cU\subseteq[1:n]$, we denote $\sum_{l \in \cU}A_{i,l}$ by $A_{i,\cU}$.  For any $A_{i,j} = a$, 
\begin{align*}
\bbP&\inp{y_i = 1|A_{i,j} = a} \\
&= \bbP\inp{A_{i,\cS\setminus\inb{j}}+z_i\geq -a} \\
& = \bbP\inp{\frac{A_{i,\cS\setminus\inb{j}}+z_i}{\sqrt{k-1+\sigma^2}}\geq -\frac{a}{\sqrt{k-1+\sigma^2}}}\\
& = 1-\Phi\inp{-\frac{a}{\sqrt{k-1+\sigma^2}}}
\end{align*} where $\Phi(x) = \frac{1}{2\pi}\int_{-\infty}^{x}e^{-\frac{t^2}{2}}dt$ is the cumulative distribution function of the standard Gaussian distribution. Thus, 
$\bbP\inp{y_i = -1|A_{i,j} = a}  = \Phi\inp{-\frac{a}{\sqrt{k-1+\sigma^2}}}$ and 
$$\bbE\insq{y_i| A_{i,j}=a} = 1-2\Phi\inp{-\frac{a}{\sqrt{k-1+\sigma^2}}}.$$ We are now ready to compute $\bbE\insq{ A_{i,j}\bbE\insq{y_i| A_{i,j}}}$.
\begin{align}
\bbE&\insq{ A_{i,j}\bbE\insq{y_i| A_{i,j}}} \nonumber\\
&= \bbE\insq{ A_{i,j}\inp{1-2\Phi\inp{-\frac{A_{i,j}}{\sqrt{k-1+\sigma^2}}}}}\nonumber\\
&= \bbE\insq{ A_{i,j}}-2\bbE\insq{A_{i,j}\Phi\inp{-\frac{A_{i,j}}{\sqrt{k-1+\sigma^2}}}}\nonumber\\
&= 0-2\bbE\insq{A_{i,j}\Phi\inp{-\frac{A_{i,j}}{\sqrt{k-1+\sigma^2}}}}\label{eq:expt2}
\end{align}
\begin{align}
\bbE&\insq{A_{i,j}\Phi\inp{-\frac{A_{i,j}}{\sqrt{k-1+\sigma^2}}}}\nonumber\\
& = \int_{-\infty}^{\infty}a\frac{1}{\sqrt{2\pi}}e^{-\frac{a^2}{2}}\inp{\frac{1}{\sqrt{2\pi}}\int_{-\infty}^{-\frac{a}{\sqrt{k-1+\sigma^2}}}e^{-\frac{t^2}{2}}dt}da\nonumber\\
& = \frac{1}{2\pi}\int_{-\infty}^{\infty}\int_{-\infty}^{-\frac{a}{\sqrt{k-1+\sigma^2}}}ae^{-\frac{a^2}{2}}e^{-\frac{t^2}{2}}dt\,da\nonumber\\
& \stackrel{(a)}{=} \frac{1}{2\pi}\int_{-\infty}^{\infty}\int_{-\infty}^{-{t}{\sqrt{k-1+\sigma^2}}}ae^{-\frac{a^2}{2}}e^{-\frac{t^2}{2}}da\,dt\nonumber\\
& =\frac{1}{2\pi}\int_{-\infty}^{\infty}\inp{\int_{-\infty}^{-{t}{\sqrt{k-1+\sigma^2}}}ae^{-\frac{a^2}{2}}da}e^{-\frac{t^2}{2}}dt\nonumber\\
& = \frac{1}{2\pi}\int_{-\infty}^{\infty}\inp{-e^{-\frac{t^2(k-1+\sigma^2)}{2}}}e^{-\frac{t^2}{2}}dt\nonumber\\
& = -\frac{1}{\sqrt{2\pi\inp{k+\sigma^2}}}\int_{-\infty}^{\infty}\frac{\sqrt{k+\sigma^2}}{\sqrt{2\pi}}e^{-\frac{t^2(k+\sigma^2)}{2}}dt\nonumber\\
& = -\frac{1}{\sqrt{2\pi\inp{k+\sigma^2}}}\label{eq:expt3}
\end{align} where $(a)$ follows for change of variable formula for integration. From \eqref{eq:expt2} and \eqref{eq:expt3}, we have 
\begin{align}
\bbE\insq{y_i A_{i,j}} = \sqrt{\frac{2}{\pi}}\times\frac{1}{\sqrt{\inp{k+\sigma^2}}}.\label{eq:expt4}
\end{align} From [Example 2.5.8]\cite{vershynin}, we also note that $\normi{{y_i}}_{\psi_2} = 1$. Thus, $L = \sqrt{\frac{2}{\pi}}\times\frac{1}{\sqrt{\inp{k+\sigma^2}}}$ and $\min\inb{L, L^2} = L^2$, which when substituted in \eqref{eq: alg_bound}  gives the desired bound.
\end{proof}

\begin{proof}[Proof of Corollary~\ref{thm:alg_spl}]
We first note that  $\frac{\bbE\insq{g'(\vecA_i^T\bx)}}{\normi{{y_i}}_{\psi_2}} = \frac{\bbE\insq{y_iA_{i,j}}}{\normi{{y_i}}_{\psi_2}}$ for  any $j$ in the support of $\bx$. This follows from \eqref{eq:log_expt11} and \eqref{eq:log_expt1_avg11}. We first compute $\bbE\insq{y_iA_{i,j}}$, which is the same as $\bbE\insq{\inp{\vecA_i^T\bx+z_i}A_{i,j}}$ for \spl. 
Note that $\bbE\insq{\inp{\vecA_i^T\bx+z_i}\inp{A_{i, j}}} = \bbE\insq{A_{i, j}^2} = 1$.
Also, from [Example 2.5.8]\cite{vershynin} 
\begin{align*}
\normi{\inp{\vecA_i^T\bx+z_i}}_{\psi_2}\leq C\sqrt{k+\sigma^2}\\
\end{align*} for some constants $C$. With this,
\begin{align*}
\frac{\bbE\insq{g'(\vecA_i^T\bx)}}{\normi{{y_i}}_{\psi_2}}\geq \frac{1}{C\sqrt{k+\sigma^2}}: =L.
\end{align*}Note that $\min\inb{L, L^2} = L^2$, which when substituted in \eqref{eq: alg_bound} gives the desired bound.
\end{proof}

\begin{proof}[Proof of Corollary~\ref{thm:alg_logreg}]
We will first compute $g(\vecA_i^T\bx) = \bbE\insq{y_i|\vecA_i^T\bx}$ for \logreg.
\begin{align*}
g(\vecA_i^T\bx)&=\bbE\insq{y_i|\vecA_i^T\bx}\\
&= \frac{1}{1+e^{-\beta \vecA_i^T\bx}}-\frac{e^{-\beta \vecA_i^T\bx}}{1+e^{-\beta \vecA_i^T\bx}}\\
&=\frac{1-e^{-\beta \vecA_i^T\bx}}{1+e^{-\beta \vecA_i^T\bx}}\\
& \stackrel{(a)}{=} {\tanh\inp{\frac{\beta \vecA_i^T\bx}{2}}}
\end{align*}where $(a)$ uses the definition of $\tanh$. Then
\begin{align*}
\bbE\insq{g'(\vecA_i^T\bx)} &= \frac{\beta}{2}\bbE\insq{\frac{1}{{\text{cosh}}^2\inp{\frac{\beta \vecA_i^T\bx}{2}}}}\\
& \stackrel{(c)}{\geq}\frac{\beta}{2}\bbE\insq{e^{-\frac{\inp{\beta\vecA_i^T\bx}^2}{4}}}
\end{align*} where $(c)$ follows from the inequality $\text{cosh}(t)\leq e^{t^2/2}$ (see [Exercise 2.2.3]\cite{vershynin}). 

% \begin{align*}
% \bbE\insq{y_i\vecA_i^T\bx} &= \bbE\insq{\vecA_i^T\bx\bbE\insq{y_i|\vecA_i^T\bx}}\\
% & = \bbE\insq{\vecA_i^T\bx\inp{\frac{1}{1+e^{-\beta \vecA_i^T\bx}}-\frac{e^{-\beta \vecA_i^T\bx}}{1+e^{-\beta \vecA_i^T\bx}}}}\\
% & = \bbE\insq{\vecA_i^T\bx\inp{\frac{1-e^{-\beta \vecA_i^T\bx}}{1+e^{-\beta \vecA_i^T\bx}}}}\\
% & = \bbE\insq{\vecA_i^T\bx\inp{\frac{e^{\beta \vecA_i^T\bx}-1}{e^{\beta \vecA_i^T\bx}+1}}}\\
% & \stackrel{(a)}{=} \bbE\insq{\vecA_i^T\bx\inp{\tanh\inp{\frac{\beta \vecA_i^T\bx}{2}}}}\\
% & \stackrel{(b)}{=} \frac{k\beta}{2}\bbE\insq{\frac{1}{{\text{cosh}}^2\inp{\frac{\beta \vecA_i^T\bx}{2}}}}\\
% & \stackrel{(c)}{\geq}\frac{k\beta}{2}\bbE\insq{e^{-\frac{\beta^2\vecA_i^T\bx^2}{2}}}\\
% \end{align*} , $(b)$ follows by noting that $\vecA_i^T\bx\sim N(0, k)$ and using Stein's lemma and $(c)$ follows from the inequality $\text{cosh}(t)\leq e^{t^2/2}$ (see \cite[Exercise 2.2.3]{vershynin}). 

Now, we need to compute $\bbE\insq{e^{-\frac{\inp{\beta\vecA_i^T\bx}^2}{4}}}$ where $\vecA_i^T\bx\sim N(0,k)$. Let $\sigma_1 := \frac{1}{\frac{\beta^2}{2}+\frac{1}{k}}$. Then

\begin{align}
\bbE\insq{e^{-\frac{\inp{\beta\vecA_i^T\bx}^2}{4}}} &= \int_{-\infty}^{\infty}\frac{1}{\sqrt{2\pi k}}e^{-\beta^2a^2/4}e^{-a^2/2k} da\nonumber\\
& = \sqrt{\frac{\sigma_1}{k}}\int_{-\infty}^{\infty}\frac{1}{\sqrt{2\pi \sigma_1}}e^{-x^2/2\sigma_1} da\nonumber\\
& = \sqrt{\frac{\sigma_1}{k}}\nonumber\\
& = \sqrt{\frac{2}{2+\beta^2 k}}\label{eq:expectation_log}
\end{align}

Thus,
\begin{align*}
\bbE\insq{g'(\vecA_i^T\bx)}&\geq \frac{\beta}{2}\sqrt{\frac{2}{2+\beta^2 k}}\\
& = \frac{1}{2}\sqrt{\frac{2}{2/\beta^2+ k}}.
\end{align*}
From [Example 2.5.8]\cite{vershynin}, we also note that $\normi{{y_i}}_{\psi_2} = 1$. Thus, $L = \frac{1}{2}\sqrt{\frac{2}{2/\beta^2+ k}}$ and $\min\inp{L, L^2} = L^2$, which gives the desired bound.

\end{proof}


\subsection{Missing proofs from Section~\ref{sec:sample_compexity}}\label{proof:sec:lower_bd}
\begin{proof}[Proof of Corollary~\ref{thm: lower_bd_bcs}]
Consider a sensing matrix $\vecA$ which satisfies the power constraint \eqref{eq:power_constraint}. 

% Suppose $\bx$ is distributed uniformly on the set of all $k$-sparse binary vectors and $\bs = \sign{\vecA\bx+\bz}$. Then,
% \begin{align}
% I(\vecA, \bs; \bx)  &= H(\bx) - H(\bx|\vecA, \bs)\nonumber\\
% &\stackrel{(a)}{\geq} \log{n \choose k } - h_2(\delta) - \delta\log\inp{{n \choose k } + 1}\nonumber\\
% &\geq k\log{n/k}- h_2(\delta)- \delta k\log\inp{n}\label{eq:lower_bd_bcs_1}
% \end{align} where $(a)$ follows from Fano's inequality \cite[Theorem~2.10.1]{thomas2006elements}.
% We also note that
% \begin{align*}
%  I(\vecA, \bs; \bx) &= I(\vecA ; \bx)    +  I(\bs; \bx|\vecA)\\
%  &\stackrel{(a)}{ = }0 + I(\bs; \bx|\vecA).
% \end{align*}where $(a)$ holds because $\vecA$ and $\bx$ are independent. Let $s_{j\in[1:i-1]}$ denote $\inp{s_1, \ldots, s_{i-1}}$. 
% \begin{align}
% I&(\bs; \bx|\vecA) = \sum_{i = 1}^{m}I(s_i; \bx|\vecA, s_{j\in[1:i-1]})\nonumber\\
% & = \sum_{i = 1}^{m}\Big(H(s_i|\vecA, s_{j\in[1:i-1]})\nonumber\\
% &\qquad- H(s_i| \bx,\vecA, s_{j\in[1:i-1]})\Big)\nonumber\\
% & \stackrel{(a)}{\leq}\sum_{i = 1}^{m}\inp{H(s_i|\vecA)- H(s_i| \bx,\vecA)}\nonumber\\
% & = \sum_{i = 1}^{m}I(s_i;\bx|\vecA)\label{eq:lower_bd_bcs}
% \end{align}where $(a)$ follows from $H(s_i|\vecA, s_{j\in[1:i-1]})\leq H(s_i|\vecA)$ and $H(s_i| \bx,\vecA, s_{j\in[1:i-1]}) = H(s_i| \bx,\vecA)$ as $s_i$ is conditionally independent of $s_{j\in[1:i-1]}$ conditioned on $\bx$ and $\vecA$.
% For any $i$, 
% \begin{align*}
% I(s_i;\bx|\vecA) &= H(s_i|\vecA)- H(s_i|\bx, \vecA)\\
% &\stackrel{(a)}{\leq} 1- H(s_i|\vecA_i^T\bx).
% \end{align*}where $(a)$ holds because $H(s_i|\vecA)\leq H(s_i) =1$ and $s_i$ is conditionally independent of $(\vecA_1\ldots, \vecA_{i-1}, A_{i+1}, \ldots, A_{m})$ and $\bx$ conditioned on $\vecA_i^T\bx$. 
Here $\vecA_i$, $i\in [1:m]$ denotes the $i^{\text{th}}$ row of the sensing matrix $\vecA$.
For any realization $b\in \bbR$ of $\vecA_i^T\bx$, 
\begin{align*}
\bbP(y_i = 1|\vecA_i^T\bx = b) &= \bbP(z_i\geq -b)= \bbP\inp{\frac{z_i}{\sigma}\geq \frac{-b}{\sigma}}\\
& = \frac{1- \sign{b}}{2} + \sign{b}Q\inp{\frac{|b|}{\sigma}}.
\end{align*} 

For $a>0$, let $R(a):=\frac{1}{\sqrt{2\pi}}\int_{0}^{a}e^{-u^2/2}du$. Then $Q(a) = \frac{1}{2}-R(a)$. Suppose $\bx$ is fixed. Then,
\begin{align*}
g(\vecA_i^T\bx) & =\bbE\insq{y_i|\vecA}  = \bbE\insq{y_i|\vecA_i^T\bx}\\
& = \frac{1- \sign{\vecA_i^T\bx}}{2} + \sign{\vecA_i^T\bx}Q\inp{\frac{|\vecA_i^T\bx|}{\sigma}} - \inp{1-\inp{\frac{1- \sign{\vecA_i^T\bx}}{2} + \sign{\vecA_i^T\bx}Q\inp{\frac{|\vecA_i^T\bx|}{\sigma}}}}\\
& = \sign{\vecA_i^T\bx}\inp{1-2Q\inp{\frac{|\vecA_i^T\bx|}{\sigma}}}\\
& = \sign{\vecA_i^T\bx}\inp{2R\inp{\frac{|\vecA_i^T\bx|}{\sigma}}}
\end{align*}
% This means that $H(s_i|\vecA_i^T\bx = b) = h_2\inp{Q(|b/\sigma|)}$ where $h_2(a)$ denotes the binary entropy of the distribution $(a, 1-a)$. We will lower bound $H(s_i|\vecA_i^T\bx = b)$ using a lower bound on the binary entropy function.
% We know that
% \begin{align*}
% H(s_i|\vecA_i^T\bx) &\stackrel{(a)}{=} \bbE\insq{h_2\inp{Q\inp{\frac{\inl{\vecA_i^T\bx}}{\sigma}}}}\\
% &\stackrel{(a)}{\geq} \bbE\insq{4{Q\inp{\frac{\inl{\vecA_i^T\bx}}{\sigma}}}\inp{1-Q\inp{\frac{\inl{\vecA_i^T\bx}}{\sigma}}}}
% \end{align*}
% where in $(a)$, the expectation is over $\vecA_i^T\bx$. The inequality $(b)$ follows from \cite[Theorem 1.2]{topsoe2001bounds}. 
% For $a>0$, let $R(a):=\frac{1}{\sqrt{2\pi}}\int_{0}^{a}e^{-u^2/2}du$. Then $Q(a) = \frac{1}{2}-R(a)$. Using this,
% \begin{align*}
% \bbE&\insq{h_2\inp{Q\inp{\frac{\inl{\vecA_i^T\bx}}{\sigma}}}}\\
% & \geq \bbE\insq{4\inp{\frac{1}{2} - R\inp{\frac{\inl{\vecA_i^T\bx}}{\sigma}}}\inp{\frac{1}{2} + R\inp{\frac{\inl{\vecA_i^T\bx}}{\sigma}}}}\\
% & = 1-4\bbE\insq{\inp{R\inp{\frac{\inl{\vecA_i^T\bx}}{\sigma}}}^2}.
% \end{align*}
For any $a>0$,
\begin{align*}
R\inp{a} &= \frac{1}{\sqrt{2\pi}}\int_{0}^{a}e^{-u^2/2}du\\
& \leq \frac{1}{\sqrt{2\pi}}\int_{0}^{a}1du = \frac{a}{\sqrt{2\pi}}.
\end{align*}
Thus, 
\begin{align*}
\bbE\insq{\inp{g\inp{\vecA_i^T\bx}^2}} &= \bbE\insq{\inp{2R\inp{\frac{|\vecA_i^T\bx|}{\sigma}}}^2}\\
&\leq \bbE\insq{4\inp{\frac{\vecA_i^T\bx}{\sqrt{2\pi}\sigma}}^2}\\
&\stackrel{(a)}{\leq}\frac{2k}{\pi\sigma^2} 
\end{align*}where $(a)$ follows from the power constraint $\bbE\insq{\inp{\vecA_i^T\bx}^2}\leq k$ (see \eqref{eq:power_constraint}). This holds for any $\bx$, including a randomly chosen sparse vector.
% \begin{align*}
%  \bbE\insq{h_2\inp{Q\inp{\frac{\vecA_i^T\bx}{\sigma}}}}&\geq 1-4\bbE\insq{\frac{\inp{\vecA_i^T\bx}^2}{{2\pi\sigma^2}}}\\
% & \stackrel{(a)}{\geq} 1-\frac{2k}{\pi\sigma^2}.
% \end{align*}Here, $(a)$ follows from the power constraint $\bbE\insq{\inp{\vecA_i^T\bx}^2}\leq k$ (see \eqref{eq:power_constraint}). 
% This implies that
% \begin{align*}
% I(s_i;\bx|\vecA)&\leq 1-\inp{1-\frac{2k}{\pi\sigma^2}} = \frac{2k}{\pi\sigma^2} 
% \end{align*}
% and from \eqref{eq:lower_bd_bcs_1} and \eqref{eq:lower_bd_bcs},
Thus,
\begin{align}
% m\frac{2k}{\pi\sigma^2} &\geq k\log{n/k}- h_2(\delta)- \delta k\log\inp{n}\\
m &\geq  \frac{\pi\sigma^2}{2k}k\log\inp{n/k}\inp{1-\frac{h_2(\delta)+ \delta k\log\inp{n}}{k\log{n/k}}}\nonumber\\
& \geq \sigma^2\log\inp{n/k}\inp{1-\frac{h_2(\delta)+ \delta k\log\inp{n}}{k\log{n/k}}}\label{eq:final_bound_bcs1}
\end{align}
% Thus,
% \begin{align}
% % m&\geq \frac{\pi\sigma^2}{2}\log\inp{n/k}\inp{1-\frac{h_2(\delta)+ \delta k\log\inp{n}}{k\log{n/k}}}\nonumber\\
% m&\geq \sigma^2\log\inp{n/k}\inp{1-\frac{h_2(\delta)+ \delta k\log\inp{n}}{k\log{n/k}}}\label{eq:final_bound_bcs1}
% \end{align}
On the other hand, $I(y_i;\bx|\vecA)\leq 1$. Thus,
\begin{align}
k\log&\inp{n/k}\inp{1-\frac{h_2(\delta)+ \delta k\log\inp{n}}{k\log{n/k}}}\nonumber\\
&\leq \sum_{i = 1}^{m}I(y_i;\bx|\vecA)\leq \sum_{i = 1}^{m}H(y_i|\vecA)\nonumber\\
&\leq m.\label{eq:final_bound_bcs2}
\end{align}
Combining \eqref{eq:final_bound_bcs1} and \eqref{eq:final_bound_bcs2}, we get the desired bound.
\end{proof}

\begin{proof}[Proof of Corollary~\ref{thm: lower_bd_log_reg}]
Consider a Gaussian sensing matrix $\vecA$. Suppose $\bx$ is distributed uniformly on the set of all $k$-sparse binary vectors. 
% Then,
% \begin{align}
% I(\vecA, \by; \bx)  &= H(\bx) - H(\bx|\vecA, \by)\nonumber\\
% &\stackrel{(a)}{\geq} \log{n \choose k } - h_2(\delta) - \delta\log\inp{{n \choose k } + 1}\nonumber\\
% &\geq k\log{n/k}- h_2(\delta)- \delta k\log\inp{n}\label{eq:log_lower_bd_bcw_1}
% \end{align} where $(a)$ follows from Fano's inequality [Theorem~2.10.1]\cite{thomas2006elements}.
% We also note that
% \begin{align*}
%  I(\vecA, \by; \bx) &= I(\vecA ; \bx)    +  I(\by; \bx|\vecA)\\
%  &\stackrel{(a)}{ = }0 + I(\by; \bx|\vecA).
% \end{align*}where $(a)$ holds because $\vecA$ and $\bx$ are independent. Let $y_{j\in[1:i-1]}$ denote $\inp{y_1, \ldots, y_{i-1}}$. 
% \begin{align}
% I&(\by; \bx|\vecA) = \sum_{i = 1}^{m}I(y_i; \bx|\vecA, y_{j\in[1:i-1]})\nonumber\\
% & = \sum_{i = 1}^{m}\Big(H(y_i|\vecA, y_{j\in[1:i-1]})\nonumber\\
% &\qquad- H(y_i| \bx,\vecA, y_{j\in[1:i-1]})\Big)\nonumber\\
% & \stackrel{(a)}{\leq}\sum_{i = 1}^{m}\inp{H(y_i|\vecA)- H(y_i| \bx,\vecA)}\nonumber\\
% & = \sum_{i = 1}^{m}I(y_i;\bx|\vecA)\label{eq:log_lower_bd_bcs}
% \end{align}where $(a)$ follows from $H(y_i|\vecA, y_{j\in[1:i-1]})\leq H(y_i|\vecA)$ and $H(y_i| \bx,\vecA, y_{j\in[1:i-1]}) = H(y_i| \bx,\vecA)$ as $y_i$ is conditionally independent of $y_{j\in[1:i-1]}$ conditioned on $\bx$ and $\vecA$.
% For any $i$, 
% \begin{align*}
% I(y_i;\bx|\vecA) &= H(y_i|\vecA)- H(y_i|\bx, \vecA)\\
% &\stackrel{(a)}{\leq} 1- H(y_i|\vecA_i^T\bx).
% \end{align*}where $(a)$ holds because $H(y_i|\vecA)\leq H(y_i) =1$ and $y_i$ is conditionally independent of $(\vecA_1\ldots, \vecA_{i-1}, A_{i+1}, \ldots, A_{m})$ and $\bx$ conditioned on $\vecA_i^T\bx$. Here $\vecA_i$, $i\in [1:m]$ denotes the $i^{\text{th}}$ row of the sensing matrix $\vecA$.

Suppose $t = \frac{1}{2}\tanh{\frac{\beta\vecA_i^T\bx}{2}} \inp{=\frac{(1-e^{-\beta \vecA_i^T\bx}}{2\inp{1+e^{-\beta \vecA_i^T\bx}}}}$. Then, 
\begin{align*}
\frac{1}{1+e^{-\beta \vecA_i^T\bx}} &= \frac{1}{2}+t\text{ and }\\
1-\frac{1}{1+e^{-\beta \vecA_i^T\bx}} &= \frac{1}{2}-t
\end{align*}

With this,
\begin{align*}
\bbE\insq{\inp{g\inp{\vecA_i^T\bx}^2}} &= \bbE\insq{4t^2}\\
& = \bbE\insq{\inp{\tanh{\frac{\beta\vecA_i^T\bx}{2}}}^2}
\end{align*}

% \begin{align*}
% H(y_i|\vecA_i^T\bx) &\stackrel{(a)}{=} \bbE\insq{h_2\inp{\frac{1}{1+e^{-\beta \vecA_i^T\bx}}}}\\
% &\stackrel{(b)}{\geq} \bbE\insq{4{\frac{1}{1+e^{-\beta \vecA_i^T\bx}}}\inp{1-\frac{1}{1+e^{-\beta \vecA_i^T\bx}}}}\\
% & = \bbE\insq{4\inp{\frac{1}{2} - t}\inp{\frac{1}{2} + t}}\\
% \end{align*}
% where in $(a)$, the expectation is over $\vecA_i^T\bx$. The inequality $(b)$ follows from [Theorem 1.2]\cite{topsoe2001bounds}. Thus,
% \begin{align*}
% \bbE&\insq{h_2\inp{\frac{1}{1+e^{-\beta \vecA_i^T\bx}}}}\\
% & \geq  1-4\bbE\insq{\inp{\frac{1}{2}\tanh{\frac{\beta\vecA_i^T\bx}{2}}}^2}\\
% & = 1 - \bbE\insq{\inp{\tanh{\frac{\beta\vecA_i^T\bx}{2}}}^2}.
% \end{align*}
Note that,
\begin{align*}
\bbE\insq{\inp{\tanh{\frac{\beta\vecA_i^T\bx}{2}}}^2} = 1-\bbE\insq{\inp{\text{sech}{\frac{\beta\vecA_i^T\bx}{2}}}^2}
\end{align*} and
\begin{align*}
\bbE\insq{\inp{\text{sech}{\frac{\beta\vecA_i^T\bx}{2}}}^2} &= \bbE\insq{\frac{1}{\inp{\text{cosh}{\frac{\beta\vecA_i^T\bx}{2}}}^2}}\\
&\stackrel{(a)}{\geq} \bbE\insq{e^{-\inp{\beta\vecA_i^T\bx/2}^2}}\\
&\stackrel{(b)}{=}\sqrt{\frac{1}{1+\beta^2 k/2}}\\
&\stackrel{(c)}{\geq}1-\frac{\beta^2k}{2}
\end{align*} where $(a)$ follows from the inequality $\text{cosh}(t)\leq e^{t^2/2}$ (see [Exercise 2.2.3]\cite{vershynin}),  $(b)$ follows from \eqref{eq:expectation_log} and $(c)$ holds because $1-\frac{x}{2}\leq \frac{1}{\sqrt{1+x}}$ for any $x\geq 0$.
Thus, 
\begin{align*}
\bbE\insq{\inp{g\inp{\vecA_i^T\bx}^2}} \leq \frac{\beta^2k}{2}.
\end{align*}
% \begin{align*}
% H(y_i|\vecA_i^T\bx)&\geq 1-\frac{\beta^2 k}{4}.
% \end{align*}
This implies that
% \begin{align*}
% I(y_i;\bx|\vecA)&\leq 1-\inp{1-\frac{\beta^2k}{4}} =\frac{\beta^2k}{4}
% \end{align*}
% and from \eqref{eq:log_lower_bd_bcw_1} and \eqref{eq:log_lower_bd_bcs},
\begin{align*}
% m\frac{2k}{\pi\sigma^2} &\geq k\log{n/k}- h_2(\delta)- \delta k\log\inp{n}\\
m\frac{\beta^2k}{2} &\geq  k\log\inp{n/k}\inp{1-\frac{h_2(\delta)+ \delta k\log\inp{n}}{k\log{n/k}}}
\end{align*}
Thus,
\begin{align}
m&\geq \frac{2}{\beta^2}\log\inp{n/k}\inp{1-\frac{h_2(\delta)+ \delta k\log\inp{n}}{k\log{n/k}}}\nonumber\\
&\geq \frac{1}{\beta^2}\log\inp{n/k}\inp{1-\frac{h_2(\delta)+ \delta k\log\inp{n}}{k\log{n/k}}}\label{eq:log_final_bound_bcs1}
\end{align}
We also  know that for any $i$, $I(y_i;\bx|\vecA)\leq H(y_i|\vecA)\leq 1$. Thus, we also obtain that
\begin{align}
m\geq k\log\inp{n/k}\inp{1-\frac{h_2(\delta)+ \delta k\log\inp{n}}{k\log{n/k}}}\label{eq:log_final_bound_bcs2}
\end{align}
Combining \eqref{eq:log_final_bound_bcs1} and \eqref{eq:log_final_bound_bcs2}, we get the desired bound.
\end{proof}



% \subsection{Proof of Theorem~\ref{thm: spl_lower_bd_1}}\label{proof:thm: spl_lower_bd_1}
\begin{proof}[Proof of Corollary~\ref{thm: spl_lower_bd_1}]
Suppose $\bx$ is generated uniformly at random from the set of all $k$-sparse vectors and $\vecA$ is any sensing matrix which satisfies the power constraint given by \eqref{eq:power_constraint}. Then,
% \begin{align}
% I(\vecA, \by;\bx) &= H(\bx)-H(\bx|\by, \vecA)\nonumber\\
%  &\stackrel{(a)}{\geq} \log{n \choose k } - h_2(\delta) - \delta\log\inp{{n \choose k } + 1}\nonumber\\
%  &\geq k\log\inp{\frac{n}{k}}- h_2(\delta) - \delta k\log{n}.\label{eq:new1}
% \end{align} where $(a)$ follows from the Fano's inequality.
% We also see that
% \begin{align*}
% I(\vecA,\by;\bx) &= I(\vecA, \bx)+I(\by;\bx|\vecA)\\
% &\stackrel{(a)}{=}0 +I(\by;\bx|\vecA)
% \end{align*}where $(a)$ follows from noting that $\vecA$ and $\bx$ are independent. Suppose $y_{j\in [1:i-1]}$ denote $(y_1, \ldots, y_{i-1})$. Then,
\begin{align*}
I&(y_i;\bx| \vecA) = h(y_i|\vecA)-h(y_i|\bx,\vecA)\\
&\leq \inp{h(y_i)-h(\mathbf{A}_i^T\bx + z_i|\bx,  \vecA)}\\
& = h(y_i) - h(z_i)\\
&\leq \inp{h(w_i)-h(z_i)}
\end{align*} where in the last inequality, $w_i\sim\cN\inp{0, \sigma^2_w}$ where $\mathsf{Var}(y_i)\leq \sigma^2_w$. We will now compute an upper bound on $\mathsf{Var}(y_i)$.
\begin{align*}
\mathsf{Var}(y_i) &\leq \bbE\insq{\inp{\vecA_i^T\bx+z_i}^2} = \bbE\insq{\inp{\vecA_i^T\bx}^2} + \sigma^2\\
&\leq k + \sigma^2
\end{align*} 
Thus, we have
\begin{align}
\inp{h(w_i)-h(z_i)}=\frac{1}{2}\log\inp{\frac{{k}}{\sigma^2}+1}.\label{eq:new2}
\end{align}
With this, we conclude that
% \begin{align*}
% \frac{m}{2}\log\inp{\frac{{k}}{\sigma^2}+1}
% \geq k\log\inp{\frac{n}{k}}- h_2(\delta) - \delta k\log{n}
% \end{align*} and 
\begin{align*}
m&\geq \frac{2k\log\inp{\frac{n}{k}}- h_2(\delta) - \delta k\log{n}}{\frac{1}{2}\log\inp{\frac{{k}}{\sigma^2}+1}}.
\end{align*} 



\end{proof}


% \input{AppendixSPL}
% ICCV 2025 Paper Template; see https://github.com/cvpr-org/author-kit

\documentclass[10pt,twocolumn,letterpaper]{article}

%%%%%%%%% PAPER TYPE  - PLEASE UPDATE FOR FINAL VERSION
% \usepackage{iccv}              % To produce the CAMERA-READY version
\usepackage[review]{iccv}      % To produce the REVIEW version
% \usepackage[pagenumbers]{iccv} % To force page numbers, e.g. for an arXiv version

% Import additional packages in the preamble file, before hyperref
%
% --- inline annotations
%
\newcommand{\red}[1]{{\color{red}#1}}
\newcommand{\todo}[1]{{\color{red}#1}}
\newcommand{\TODO}[1]{\textbf{\color{red}[TODO: #1]}}
% --- disable by uncommenting  
% \renewcommand{\TODO}[1]{}
% \renewcommand{\todo}[1]{#1}



\newcommand{\VLM}{LVLM\xspace} 
\newcommand{\ours}{PeKit\xspace}
\newcommand{\yollava}{Yo’LLaVA\xspace}

\newcommand{\thisismy}{This-Is-My-Img\xspace}
\newcommand{\myparagraph}[1]{\noindent\textbf{#1}}
\newcommand{\vdoro}[1]{{\color[rgb]{0.4, 0.18, 0.78} {[V] #1}}}
% --- disable by uncommenting  
% \renewcommand{\TODO}[1]{}
% \renewcommand{\todo}[1]{#1}
\usepackage{slashbox}
% Vectors
\newcommand{\bB}{\mathcal{B}}
\newcommand{\bw}{\mathbf{w}}
\newcommand{\bs}{\mathbf{s}}
\newcommand{\bo}{\mathbf{o}}
\newcommand{\bn}{\mathbf{n}}
\newcommand{\bc}{\mathbf{c}}
\newcommand{\bp}{\mathbf{p}}
\newcommand{\bS}{\mathbf{S}}
\newcommand{\bk}{\mathbf{k}}
\newcommand{\bmu}{\boldsymbol{\mu}}
\newcommand{\bx}{\mathbf{x}}
\newcommand{\bg}{\mathbf{g}}
\newcommand{\be}{\mathbf{e}}
\newcommand{\bX}{\mathbf{X}}
\newcommand{\by}{\mathbf{y}}
\newcommand{\bv}{\mathbf{v}}
\newcommand{\bz}{\mathbf{z}}
\newcommand{\bq}{\mathbf{q}}
\newcommand{\bff}{\mathbf{f}}
\newcommand{\bu}{\mathbf{u}}
\newcommand{\bh}{\mathbf{h}}
\newcommand{\bb}{\mathbf{b}}

\newcommand{\rone}{\textcolor{green}{R1}}
\newcommand{\rtwo}{\textcolor{orange}{R2}}
\newcommand{\rthree}{\textcolor{red}{R3}}
\usepackage{amsmath}
%\usepackage{arydshln}
\DeclareMathOperator{\similarity}{sim}
\DeclareMathOperator{\AvgPool}{AvgPool}

\newcommand{\argmax}{\mathop{\mathrm{argmax}}}     



% It is strongly recommended to use hyperref, especially for the review version.
% hyperref with option pagebackref eases the reviewers' job.
% Please disable hyperref *only* if you encounter grave issues, 
% e.g. with the file validation for the camera-ready version.
%
% If you comment hyperref and then uncomment it, you should delete *.aux before re-running LaTeX.
% (Or just hit 'q' on the first LaTeX run, let it finish, and you should be clear).
\definecolor{iccvblue}{rgb}{0.21,0.49,0.74}
\usepackage[pagebackref,breaklinks,colorlinks,allcolors=iccvblue]{hyperref}
\usepackage{overpic}
\usepackage{makecell}
\usepackage{adjustbox} 
%\usepackage{float}
%\usepackage{arydshln}
\usepackage{multirow}
\usepackage{float}
\usepackage{bbding}
\usepackage{times}
\usepackage{epsfig}
\usepackage{graphicx}
\usepackage{amsmath}
\usepackage{amsthm}
\usepackage{amssymb}
\usepackage{booktabs}
\usepackage{xcolor}
\usepackage{enumitem}
\usepackage{lipsum}
\usepackage{diagbox}

\usepackage{algorithmic} % 引入 algorithmic 包
\usepackage{algpseudocode} % 引入 algpseudocode 包(如果需要更现代的控制结构)

\newcommand\blfootnote[1]{%
  \begingroup
  \renewcommand\thefootnote{}\footnote{#1}%
  \addtocounter{footnote}{-1}%
  \endgroup
}

\usepackage[ruled,vlined,linesnumbered]{algorithm2e}
\makeatletter
\newcommand{\algorithmfootnote}[2][\footnotesize]{%
  \let\old@algocf@finish\@algocf@finish% Store algorithm finish macro
  \def\@algocf@finish{\old@algocf@finish% Update finish macro to insert "footnote"
    \leavevmode\rlap{\begin{minipage}{\linewidth}
    #1#2
    \end{minipage}}%
  }%
}
\makeatother
%%%%%%%%% PAPER ID  - PLEASE UPDATE
\def\paperID{5291} % *** Enter the Paper ID here
\def\confName{ICCV}
\def\confYear{2025}
\newtheorem{theorem}{Theorem}[section]
\newtheorem{lemma}[theorem]{Lemma}
%%%%%%%%% TITLE - PLEASE UPDATE
\title{Q-PETR: Quant-aware Position Embedding Transformation for Multi-View 3D Object Detection\\------------ Supplementary Material ------------}

%%%%%%%%% AUTHORS - PLEASE UPDATE
\author{First Author\\
Institution1\\
Institution1 address\\
{\tt\small firstauthor@i1.org}
% For a paper whose authors are all at the same institution,
% omit the following lines up until the closing ``}''.
% Additional authors and addresses can be added with ``\and'',
% just like the second author.
% To save space, use either the email address or home page, not both
\and
Second Author\\
Institution2\\
First line of institution2 address\\
{\tt\small secondauthor@i2.org}
}

\begin{document}
\maketitle

\section{Preliminaries}
\label{sec:petr_preliminaries}

\paragraph{PETR} enhances 2D image features with 3D position-aware properties using camera-ray positional encoding (PE), enabling refined query updates for 3D bounding box prediction. Specifically, surround-view images $\mathbf{I}$ pass through a backbone to generate 2D features $\mathbf{f}_{2D}$, while camera-ray PE $\mathbf{p}_c$ is computed using camera intrinsics and extrinsics. The learnable query embeddings $q$ serve as the initial queries $\mathbf{Q}$ for the decoder. Here, $\mathbf{f}_{2D}$ serves as the values $\mathbf{V}$, and adding $\mathbf{p}_c$ to $\mathbf{f}_{2D}$ element-wise forms the 3D position-aware keys $\mathbf{K}$.

The decoder updates the queries using these key-value pairs through self-attention, cross-attention, and feed-forward network (FFN) modules. The updated query vectors are passed through an MLP to predict 3D bounding box categories and attributes, repeating for $L$ cycles. The entire PETR process is summarized in Algorithm~\ref{algo:algorithm_petr}.
\begin{algorithm}[htb]
\SetAlgoLined
\DontPrintSemicolon
\SetNoFillComment
\SetInd{1em}{1em} % <-- 关键:取消缩进
\footnotesize

\KwData{Surround-view images $\mathbf{I}$, camera intrinsics and extrinsics}
\KwResult{3D bounding boxes $\mathbf{b}^l$, categories $\mathbf{c}^l$ for $l = 1$ to $L$}

Compute image features: $\mathbf{f}_{2D} = \text{Backbone}(\mathbf{I})$\\
Compute camera-ray PE $\mathbf{p}_c$ using camera intrinsics and extrinsics\\
Form 3D position-aware keys: $\mathbf{K} = \mathbf{f}_{2D} + \mathbf{p}_c$ \tcp{Element-wise addition}
Set values: $\mathbf{V} = \mathbf{f}_{2D}$ \\
Initialize queries: $\mathbf{Q} = q$ (For simplicity, omit \(\mathbf{Q}\)'s encoding.)\\
\For{$l = 1$ to $L$}{
  $\mathbf{Q} \gets \texttt{QProj}(\mathbf{Q})$; 
  $\mathbf{K} \gets \texttt{KProj}(\mathbf{K})$; 
  $\mathbf{V} \gets \texttt{VProj}(\mathbf{V})$ \\ 
  $\mathbf{A}_s = \texttt{MultiHeadAtt}(\mathbf{Q}, \mathbf{Q}, \mathbf{Q})$ \tcp{Self-Attn}
  $\mathbf{A}_c = \texttt{MultiHeadAtt}(\mathbf{A}_s, \mathbf{K}, \mathbf{V})$ \tcp{Cross-Attn}
  $\mathbf{Q} \gets \texttt{FFN}(\mathbf{Q} + \mathbf{A}_c)$ \\
  $\mathbf{b}^{l} \gets \texttt{MLP}(\mathbf{Q})$; 
  $\mathbf{c}^{l} \gets \texttt{MLP}(\mathbf{Q})$ \\
}
\Return{$(\mathbf{b}^l, \mathbf{c}^l)$ for $l = 1$ to $L$}

\caption{Pseudo-code of PETR.}
\label{algo:algorithm_petr}
\end{algorithm}



\iffalse
\section{Quantization Failure of PETR}
\label{sec:quantization_failure_of_petr}

We evaluate the performance of several PETR configurations~\cite{liu2022petr} using the official code. Under standard 8-bit symmetric per-tensor post-training quantization (PTQ), PETR suffers significant performance degradation, with an average drop of 58.2\% in mAP and 36.9\% in NDS on the nuScenes validation dataset (see Table~\ref{tab:performance_drop_for_ptq_on_raw_petr}). 

\begin{table}[htb] %{0.45\linewidth}
    %\tiny
    %\scriptsize
    \footnotesize
    \setlength{\tabcolsep}{1.4mm}
    %\small
    %\setlength{\tabcolsep}{2.5mm}
    %\normalsize
    %\large
    \centering
    \begin{tabular}{l|c|c|c|c|c|c}
    \toprule[1.5pt]
    \multirow{2}{*}{Bac} & \multirow{2}{*}{Size} & \multirow{2}{*}{Feat} & \multicolumn{2}{c|}{FP32 Acc}                               & \multicolumn{2}{c}{INT8 Acc}   \\ \cline{4-7}
     & & & \multicolumn{1}{c|}{mAP} & \multicolumn{1}{c|}{NDS} & \multicolumn{1}{c|}{mAP} & \multicolumn{1}{c}{NDS} \\ 
    \midrule
    \textcolor{white}{0}R50\textcolor{white}{0} & 1408$\times$512 & c5 & 30.5 & 35.0 & 18.4(12.1$\downarrow$) & 27.3(\textcolor{white}{0}7.7$\downarrow$) \\
    \textcolor{white}{0}R50\textcolor{white}{0} & 1408$\times$512 & p4 & 31.7 & 36.7 & 15.7(16.0$\downarrow$) & 26.1(10.6$\downarrow$) \\
    V2-99 & \textcolor{white}{0}800$\times$320 & p4 & 37.8 & 42.6 & 10.9(26.9$\downarrow$) & 23.6(19.0$\downarrow$) \\
    V2-99 & 1600$\times$640 & p4 & 40.4 & 45.5 & 11.3(29.1$\downarrow$) & 23.9(21.6$\downarrow$) \\
    \bottomrule[1.5pt]
    \end{tabular}
    \vspace{-0.3cm}
    \caption{
    PETR's performance of 3D object detection on nuSences val set, directly utilizing the pre-trained parameters from the official repository.
    }
    %\vspace{-0.5cm}
    \label{tab:performance_drop_for_ptq_on_raw_petr}
\end{table}

\iffalse
Enlightened by the existing state-of-the-art post-training quantization methods~\cite{dong2019hawq1,dong2020hawq2,hubara2021adaq,li2021brecq,liu2021ptqvit,nagel2020up,yao2021hawq3}, the adaptive rounding proxy objective is introduced to measure the quantization performance degradation:
\begin{equation}
\begin{aligned}
    \label{eqnAdaptiveEll}
    \textstyle
    \ell(\widetilde W)
    &=\mathbf{E}_x\left[ \Vert{ (W- \widetilde W) x \Vert}^2 \right] \\
    &=tr\left( (W - \widetilde W) H (W - \widetilde W)^T \right)
\end{aligned}
\end{equation}
Where $W$ is the float weight tensor of a learnable operator, $\widetilde W$ are the quantized weight, 
$x$ is an input tensor sampled randomly from a calibration set, 
and $H$ is the second moment matrix of these vectors, interpreted as a proxy Hessian.
\fi

\paragraph{Layer-wise Quantization Error Analysis.} Quantizing a pre-trained network introduces output noise, degrading performance. To identify the root causes of quantization failure, we employ the signal-to-quantization-noise ratio (SQNR), inspired by recent PTQ advancements~\cite{pandey2023practical, yang2023efficient, pagliari2023plinio}:

\begin{equation}\label{eq:sqnr}
SQNR_{q,b} = 10\log_{10} \left( \frac{ \sum_{i=1}^N \mathbb{E}[{\mathcal{F}}{\theta}(x_{i})^{2} ] }{ \sum_{i=1}^N \mathbb{E}[ e(x_{i})^{2} ] } \right)
\end{equation}

Here, $N$ is the number of calibration data points; $\mathcal{F}{\theta}$ denotes the full-precision network; the quantization error is $e(x_i) = \mathcal{F}{\theta}(x_i) - \mathcal{Q}{q,b}(\mathcal{F}{\theta}(x_i))$; and $\mathcal{Q}_{q,b}(\mathcal{F}_\theta)$ denotes the network output when only the target layer is quantized to $b$ bits, with all other layers kept at full precision.

Since 8-bit weight quantization results in only a minor loss of precision, we focus on quantization errors arising from operator inputs. Using the PETR configuration from the first row of Table~\ref{tab:performance_drop_for_ptq_on_raw_petr}, we obtain layer-wise SQNRs, depicted in Fig.~\ref{fig:ASQNR}. From these results, we identify three main factors contributing to quantization errors:

\paragraph{Observation 1: Position Encoding Design Flaws Lead to Quantization Difficulties.}
Our in-depth analysis reveals that the quantization issues in PETR fundamentally stem from a design flaw in the positional encoding module, which manifests in two interrelated aspects. \textbf{(a) The inverse-sigmoid operator disrupts feature distribution balance.} As indicated by the red arrow in Fig.\ref{fig:ASQNR}, quantization difficulties stem from PETR's positional encoding module. Analyzing its construction (Fig.\ref{fig:pe_compare}(a)), we find that the inverse-sigmoid operation induces an imbalanced feature distribution. Specifically, Fig.~\ref{fig:distribution_before_and_after_insigmoid} shows that before applying inverse-sigmoid, the feature distribution is balanced and quantization-friendly, whereas afterward, it exhibits significant outliers. \textbf{(b) Magnitude Disparity between Camera-ray PE and Image Features.} As highlighted by the purple arrow in Fig.~\ref{fig:ASQNR}, applying 8-bit symmetric linear quantization to the 3D position-aware key $\mathbf{K}$ leads to significant performance degradation. To investigate this phenomenon, we conduct a statistical analysis of the magnitude distributions between image features and camera-ray positional encodings (PE). As illustrated in Fig.~\ref{fig:pe_img_compare}, both token-wise and channel-wise comparisons reveal that camera-ray PE exhibits an order-of-magnitude larger dynamic range (typically within $\pm$120) compared to image features (confined to $\pm$3). This severe imbalance creates a critical issue during quantization. When applying symmetric linear quantization with an 8-bit integer range (-128 to 127), the scaling factor \( s \) in Eq~\ref{e:eq2} becomes dominated by the extreme values of PE. Consequently, image features---occupying only ~2.5\% of the total dynamic range—are compressed into merely 7 discrete quantization bins (-3 to +3). As visualized in Fig.~\ref{fig:magnitude_distributions_of_image_feature_and_camera_ray_PE}, over 95\% of the original image feature variations collapse into the zero-centered bins, resulting in catastrophic information loss. This quantization artifact directly explains the observed performance drop in Table~\ref{tab:performance_drop_for_ptq_on_raw_petr}. To mitigate this issue, we propose two essential modifications: 1) eliminating the inverse-sigmoid operation that exacerbates outlier magnitudes, and 2) redesigning the positional encoding architecture to align its magnitude distribution with that of image features. These adaptations ensure balanced quantization resolution allocation, preserving critical information in both PE and image features. To mitigate this issue, we propose two essential modifications: 1) eliminating the inverse-sigmoid operation that exacerbates outlier magnitudes, and 2) redesigning the positional encoding architecture to align its magnitude distribution with that of image features. These adaptations ensure balanced distribution for quantization, preserving critical information in both PE and image features.
% To address these challenges, it is imperative to avoid the inverse-sigmoid operation and to redesign the positional encoding so that its magnitude distribution aligns better with that of the image features, thereby enhancing overall quantization-friendliness.


\begin{figure}[htb]
\centering
	\includegraphics[width=0.8\linewidth]{./figs/coor3d_before_and_after_inversesigmoid.pdf}
    %\vspace{-0.5cm}
	\caption{Feature distribution before and after the inverse-sigmoid operator. Red arrows highlight outliers.}
	\label{fig:distribution_before_and_after_insigmoid}
\end{figure}
\vspace{-0.5cm}
\begin{figure}[htb]
\centering
	\includegraphics[width=0.75\linewidth]{./figs/pe_img_compare.pdf}
    %\vspace{-0.3cm}
	\caption{Magnitude Distribution of Image Features and Positional Encodings: A Token-wise and Channel-wise Comparison}
	\label{fig:pe_img_compare}
\end{figure}


\begin{figure}[htb]
\centering
	\includegraphics[width=0.8\linewidth]{./figs/img_feat_and_pe_distribution.pdf}
    \vspace{-0.3cm}
	\caption{The distributions of image features and camera-ray position encodings after symmetric quantization using the quantization parameters derived from the 3D position-aware $\mathbf{K}$.}
	\label{fig:magnitude_distributions_of_image_feature_and_camera_ray_PE}
\end{figure}
\vspace{-0.5cm}


\paragraph{Observation 2: Dual-Dimensional Heterogeneity in Cross-Attention Leads to Quantization Bottlenecks.}

As evidenced by the green arrow in Fig.~\ref{fig:ASQNR} and further clarified in Fig.~\ref{fig:scaled_dot_product}, the scaled dot-product in cross-attention exhibits pronounced heterogeneity on two levels. First, the inter-head variance spans 2–3 orders of magnitude, while within each head, the value distribution is extremely broad (e.g., ranging beyond [$-10^3$, $10^3$]). We merge the head and query dimensions to directly reveal the row-wise feature distribution. The results show that regardless of whether quantization is performed per head, per token, or on the entire tensor, the excessively large softmax inputs result in significant quantization errors. This confirms that existing quantization paradigms are fundamentally inadequate for handling the severe amplitude disparities in the cross-attention mechanism.



\begin{figure}[htb]
\centering
	\includegraphics[width=0.8\linewidth]{./figs/softmax_input.pdf}
    %\vspace{-0.3cm}
	\caption{The distributions of scaled dot-product in cross-attention. There are significant amplitude fluctuations along the head dimension.}
	\label{fig:scaled_dot_product}
\end{figure}


\fi

\section{Experimental Setup}\label{sec:exp_setup}
\textbf{Benchmark.}
We use the nuScenes dataset, a comprehensive autonomous driving dataset covering object detection, tracking, and LiDAR segmentation. The vehicle is equipped with one LiDAR, five radars, and six cameras providing a 360-degree view. The dataset comprises 1,000 driving scenes split into training (700 scenes), validation (150 scenes), and testing (150 scenes) subsets. Each scene lasts 20 seconds, annotated at 2 Hz.

\textbf{Metrics.}
Following the official evaluation protocol, we report the nuScenes Score (NDS), mean Average Precision (mAP), and five true positive metrics: mean Average Translation Error (mATE), Scale Error (mASE), Orientation Error (mAOE), Velocity Error (mAVE), and Attribute Error (mAAE).


\textbf{Experimental Details.}
Our experiments encompass both floating-point training and quantization configurations. For floating-point training, we follow PETR series settings, using PETR with an R50dcn backbone unless specified, and utilize the C5 feature (1/32 resolution output) as the 2D feature. Input images are at $1408 \times 512$ resolution. Both the lidar-ray PE and QD-aware lidar-ray PE use a pixel-wise depth of 30m with three anchor embeddings per axis. The 3D perception space is defined as $[-61.2, 61.2]$m along the X and Y axes, and $[-10, 10]$m along the Z axis. We also compare these positional encodings on StreamPETR, using a V2-99 backbone and input images of $800 \times 320$ resolution.

Training uses the AdamW optimizer (weight decay 0.01) with an initial learning rate of $2.0 \times 10^{-4}$, decayed via a cosine annealing schedule. We train for 24 epochs with a batch size of 8 on four NVIDIA RTX 4090 GPUs. No test-time augmentation is applied.

For quantization, we adopt 8-bit symmetric per-tensor post-training quantization, using 32 randomly selected training images for calibration. When quantizing the scaled dot-product in cross-attention, we define a candidate set of 20 scaling factors.


\section{Theoretical Analysis of Magnitude Bounds in Position Encodings}
\label{sec:mag_analysis}

\subsection{Normalization Framework and Input Conditioning}
\label{subsec:normalization}
To establish a unified analytical framework, we first formalize the spatial normalization process for various ray-based position encodings. Let $\mathbf{p} = (x, y, z)$ denote the 3D coordinates within the perception range $x, y \in [-51.2, 51.2]$ meters and $z \in [-5, 3]$ meters. The normalized coordinates $\mathbf{v} \in [0, 1]^3$ are computed as:

\begin{equation}
    \mathbf{v} = \left( \frac{x + 51.2}{102.4}, \frac{y + 51.2}{102.4}, \frac{z + 5.0}{8.0} \right)
    \label{eq:normalization}
\end{equation}

Noting that $\mathbf{v}$ is clamped to $\mathbf{v}_c$ within the range $[0, 1]$, the distribution ranges of the normalized sampled points in positional encodings are characterized as follows:
\begin{itemize}
    \item For the sampled point of Camera-Ray PE, denoted as $\mathbf{v}_c^{CR}$, the distribution spans the unit cube, i.e., $[0, 1] \times [0, 1] \times [0, 1]$.
    \item For the sampled points of LiDAR-Ray PE and QDPE, denoted as $\mathbf{v}_c^{LR}$ and $\mathbf{v}_c^{QD}$ respectively, the distributions are constrained to $[0, 0.79] \times [0, 0.79] \times [0, 1]$.
\end{itemize}
Here, the value $0.79$ is derived from the ratio $30/51.2$, where $30$ corresponds to the fixed depth setting in the encoding process. This distinction highlights the inherent differences in spatial coverage and normalization strategies employed by these positional encodings.

\subsection{Magnitude Propagation Analysis}
\label{subsec:magnitude_propagation}

\subsubsection{Camera-Ray Position Encoding}
\label{subsubsec:camera_pe}
As illustrated in Fig.~\ref{fig:pe_compare} (a), the encoding pipeline consists of two critical stages:

\textbf{Stage 1: Inverse Sigmoid Transformation}
\begin{equation}
    \hat{\mathbf{v}}^{CR} = \ln\left(\frac{\mathbf{v}_c^{CR} + \epsilon}{1 - (\mathbf{v}_c^{CR} + \epsilon)}\right), \quad \epsilon = 10^{-5}
    \label{eq:logit_transform_cr}
\end{equation}
Empirical analysis reveals a maximum magnitude $\eta_{\text{max}} = \max(\|\hat{\mathbf{v}}^{CR}\|_\infty) \approx 11.5$.

\textbf{Stage 2: MLP Projection} (Through Two Fully-Connected Layers)
\begin{equation}
    \text{PE}_{\text{CR}} = \mathbf{W}_2 \sigma(\mathbf{W}_1 \hat{\mathbf{v}}^{CR} + \mathbf{b}_1) + \mathbf{b}_2
    \label{eq:mlp_transform_cr}
\end{equation}
where $\sigma$ denotes the ReLU activation function. Let $\Gamma = \max(\|\mathbf{W}_1\|_{\max}, \|\mathbf{W}_2\|_{\max})$ be the maximum weight magnitude. We derive the upper bound:
\begin{equation}
    \| \text{PE}_{\text{CR}} \|_\infty \leq 256 \cdot 192 \cdot \Gamma^2 \cdot 11.5
    \label{eq:cr_bound}
\end{equation}
where $192$ and $256$ denote the input tensor channels for $\mathbf{W}_1$ and $\mathbf{W}_2$, respectively.

\subsubsection{LiDAR-Ray Position Encoding}
\label{subsubsec:lidar_pe}
Unlike Camera-Ray PE, the encoding process of LiDAR-Ray PE introduces sinusoidal modulation between the inverse sigmoid transformation and MLP projection, as shown in Fig.~\ref{fig:pe_compare} (b). The magnitude propagation for LiDAR-Ray PE is as follows:

\textbf{Stage 1: Inverse Sigmoid Transformation}
\begin{equation}
    \hat{\mathbf{v}}^{LR} = \ln\left(\frac{\mathbf{v}_c^{LR} + \epsilon}{1 - (\mathbf{v}_c^{LR} + \epsilon)}\right), \quad \epsilon = 10^{-5}
    \label{eq:logit_transform_lr}
\end{equation}
Empirical analysis reveals a maximum magnitude $\eta_{\text{max}} = \max(\|\hat{\mathbf{v}}^{LR}\|_\infty) \approx 1.8$.

\textbf{Stage 2: Spectral Embedding}
\begin{equation}
    \phi(\hat{\mathbf{v}}^{LR}) = \bigoplus_{k=1}^{32} \left[\sin(\omega_k \hat{\mathbf{v}}^{LR}), \cos(\omega_k \hat{\mathbf{v}}^{LR})\right]
    \label{eq:sinusoidal}
\end{equation}
where $\bigoplus$ denotes concatenation. This ensures:
\begin{equation}
    \| \phi(\hat{\mathbf{v}}^{LR}) \|_\infty \leq 1.0
    \label{eq:sin_bound}
\end{equation}

\textbf{Stage 3: MLP Projection} (Following setting in Camera-Ray PE)
\begin{equation}
    \| \text{PE}_{\text{LR}} \|_\infty \leq 256 \cdot 192 \cdot \Gamma^2 \cdot 1.0
    \label{eq:lr_bound}
\end{equation}

\subsubsection{Ours QD-PE}
\label{subsubsec:qd_pe}
The proposed encoding introduces anchor-based constraints, as depicted in Fig.~\ref{fig:pe_compare} (c):

\textbf{Stage 1: Anchor Interpolation} (For Each Axis $\alpha \in \{x, y, z\}$)
\begin{equation}
    \mathbf{e}_\alpha = \frac{p_\alpha - L_\alpha^i}{\Delta L_\alpha} \mathbf{E}_\alpha^{i+1} + \frac{L_\alpha^{i+1} - p_\alpha}{\Delta L_\alpha} \mathbf{E}_\alpha^i
    \label{eq:anchor_interp}
\end{equation}
where $\mathbf{E}_\alpha^i$ denotes learnable anchor embeddings.
Via Theorem~\ref{thm:anchor}, the magnitude is constrain to:
\begin{equation}
    \| \mathbf{e}_\alpha \|_\infty \leq \gamma \quad 
    \label{eq:anchor_bound}
\end{equation}

\textbf{Stage 2: MLP Projection}
\begin{equation}
    \| \text{PE}_{\text{QD}} \|_\infty \leq 256 \cdot 192 \cdot \Gamma^2 \cdot 0.8
    \label{eq:qd_bound}
\end{equation}

\subsection{Comparative Magnitude Analysis}
\label{subsec:comparative}
The derived bounds reveal fundamental differences in magnitude scaling:
\begin{align}
    \frac{\| \text{PE}_{\text{CR}} \|}{\| \text{PE}_{\text{LR}} \|} &\approx \frac{11.5}{1.0} = 11.5 
    \label{eq:ratio_lidar} \\
    \frac{\| \text{PE}_{\text{CR}} \|}{\| \text{PE}_{\text{QD}} \|} &\approx \frac{11.5}{0.8} = 14.3
    \label{eq:ratio_qd}
\end{align}
This analysis demonstrates that QD-PE requires $14\times$ less quantization range than Camera-Ray PE.

\subsection{Theoretical Guarantee of Magnitude Constraints}
\label{subsec:theorem}
\begin{theorem}[Anchor Embedding Magnitude Bound]
\label{thm:anchor}
Let $\mathbf{E}_\alpha^i, \mathbf{E}_\alpha^{i+1}$ be adjacent anchor embeddings with $\|\mathbf{E}_\alpha^i\|_\infty \leq \gamma$. For any point $p_\alpha \in [L_\alpha^i, L_\alpha^{i+1}]$, its interpolated embedding satisfies:
\begin{equation}
    \| \mathbf{e}_\alpha \|_\infty \leq \gamma
\end{equation}
\end{theorem}

\begin{proof}
Let $\lambda = \frac{p_\alpha - L_\alpha^i}{\Delta L_\alpha} \in [0, 1]$. The interpolated embedding becomes:
\begin{equation}
    \mathbf{e}_\alpha = \lambda \mathbf{E}_\alpha^{i+1} + (1 - \lambda) \mathbf{E}_\alpha^i
\end{equation}
For any component $k$:
\begin{equation}
    |e_{\alpha,k}| \leq \lambda |E_{\alpha,k}^{i+1}| + (1 - \lambda) |E_{\alpha,k}^i| \leq \lambda \gamma + (1 - \lambda) \gamma = \gamma
\end{equation}
Thus, $\|\mathbf{e}_\alpha\|_\infty \leq \gamma$ holds for all dimensions.
\end{proof}
Through the application of regularization (e.g., L2 constraint) on the anchor embeddings $\mathbf{E}_\alpha^i$
during training, the magnitude of $\gamma$ can be explicitly controlled.
Empirically, we find that this value converges to approximately 0.8 in our experiments.


\section{More Ablation Study}\label{sec:more_ablation}

\begin{figure*}[htb]
\centering
	\includegraphics[width=0.8\linewidth]{./figs/pe visual.pdf}
    %\vspace{-0.8cm}
	\caption{Qualitative comparison of the local similarity.}
	\label{fig:pe_visual_compare}
\end{figure*}

\subsection{Local Similarity of Position Encoding Features}
%\textbf{Local Similarity of Position Encoding Features.} 
Fig.~\ref{fig:pe_visual_compare} shows that QD-PE significantly outperforms 3D point PE and cameraray PE in local similarity of position encoding. Its similarity distribution appears more compact and concentrated, validating the method's superiority in local spatial information modeling and its capability to precisely capture neighborhood spatial relationships around target pixels.
% is a qualitative comparison of QD-PE, 3D point PE, and cameraray PE from the back perspective of a 3D surround-view system, focusing on the local similarity of position encoding features. The red box in the top row marks a selected pixel, and the similarity maps illustrate how each method encodes spatial relationships around that pixel. A more concentrated similarity distribution indicates a stronger position encoding. As shown, our method achieves a more compact and focused similarity, highlighting its effectiveness in capturing local spatial information compared to other approaches.

\section{Limitations}
Although our method incurs almost no quantization accuracy loss, 
users need to replace the camera-ray in the original PETR series with our proposed QDPE. 
The only drawback is that this requires retraining. 
However, from the perspective of quantization deployment,
this retraining is beneficial, 
and the floating-point precision can even be improved.


\end{document}

\section{Comparison with \cite{vershyninPlan}}\label{sec:comparison_PV}
Algorithm~\ref{alg:1} is similar to the two step estimation procedure outlined in \cite{vershyninPlan} which was given to estimate the unknown signal within a two norm guarantee. 
Computing the vector $\mathbf{l} = \inp{l_1, \ldots, l_n}$ is the same as the first step of the procedure in [Section~1.2]\cite{vershyninPlan} where a linear estimator is computed. The second step of our algorithm (sorting and keeping the top-$k$ indices) can be thought of as a projection on a feasible set [Section~1.3]\cite{vershyninPlan}. However, this requires the estimation error to be small enough for the exact recovery of a binary vector.

The setup in \cite{vershyninPlan} is for the recovery of an unknown signal with small two-norm error, whereas our problem of exact recovery of a sparse binary vector is more suited for recovery under  infinity norm. This results in weak bounds ($m\approx O(k^2)$) when we specialize various results in \cite{vershyninPlan} to our case. We first note that we require $\bbE\norm{\frac{\hat{x}}{\norm{\hat{x}}}-\bar{x}}< \sqrt{\frac{{2}}{{k}}}$ for exact recovery. Otherwise, there exist two binary $k$-sparse vectors which have hamming distance at least two. 

We first consider the 1-bit compressed sensing result in Section 3.5 (page 13). Setting the LHS to $\sqrt{\frac{{2}}{{k}}}$, we get
\begin{align*}
\sqrt{\frac{{2}}{{k}}}\leq C\sqrt{\frac{k\log\inp{2n/k}}{m}}.
\end{align*}
This implies that $m\approx C_1 k^2\log\inp{2n/k}$ for some constant $C_1$.

Next, we consider [Theorem 9.1]\cite{vershyninPlan}. Note that for 1-bit compressed sensing $\eta^2 = 1$ and 
\begin{align*}
\mu &= \bbE\insq{s_1\ipr{a_1}{\bar{x}}}\\
& =\bbE\insq{s_1\ipr{a_1}{{x}}}\\
& \stackrel{(a)}{=} \frac{1}{\sqrt{k}}\sqrt{\frac{2}{\pi}}\times\frac{k}{\sqrt{\inp{k+\sigma^2}}}\\
&= \sqrt{\frac{2}{\pi}}\times\frac{\sqrt{k}}{\sqrt{\inp{k+\sigma^2}}}.
\end{align*} where $(a)$ follows from \eqref{eq:expt4}. 
Then, 
\begin{align*}
\norm{x-\mu\bar{x}} &= \norm{x-\sqrt{\frac{2}{\pi}}\times\frac{\sqrt{k}}{\sqrt{\inp{k+\sigma^2}}}\frac{x}{\sqrt{k}}}\\
& = \norm{x-\sqrt{\frac{2}{\pi}}\times\frac{x}{\sqrt{\inp{k+\sigma^2}}}}
\end{align*} We require $\norm{x-\mu\bar{x}}<\frac{2}{\sqrt{\pi\inp{k+\sigma^2}}}$ in order to exactly recover the unknown signal $x$. 



We assume that $K$ is also a closed cone in $\bbR^n$. Then, by [Section~2.4]\cite{vershyninPlan}, $w_t(K) = tw_1(K) \leq t C\sqrt{k\log\inp{2n/k}}$ ([Section~2.4]\cite{vershyninPlan}). We choose $s = w_1(K)$. Substituting the bound for LHS and taking the limit $t\rightarrow 0$, we get
\begin{align*}
\frac{2}{\sqrt{\pi\inp{k+\sigma^2}}}\leq \frac{8 C \sqrt{k\log\inp{2n/k}}}{\sqrt{m}}.
\end{align*} Thus, $m\approx 4C(k+ \sigma^2)k\log\inp{2n/k}$. 








% \section{Introduction}
\label{sec:introduction}
The business processes of organizations are experiencing ever-increasing complexity due to the large amount of data, high number of users, and high-tech devices involved \cite{martin2021pmopportunitieschallenges, beerepoot2023biggestbpmproblems}. This complexity may cause business processes to deviate from normal control flow due to unforeseen and disruptive anomalies \cite{adams2023proceddsriftdetection}. These control-flow anomalies manifest as unknown, skipped, and wrongly-ordered activities in the traces of event logs monitored from the execution of business processes \cite{ko2023adsystematicreview}. For the sake of clarity, let us consider an illustrative example of such anomalies. Figure \ref{FP_ANOMALIES} shows a so-called event log footprint, which captures the control flow relations of four activities of a hypothetical event log. In particular, this footprint captures the control-flow relations between activities \texttt{a}, \texttt{b}, \texttt{c} and \texttt{d}. These are the causal ($\rightarrow$) relation, concurrent ($\parallel$) relation, and other ($\#$) relations such as exclusivity or non-local dependency \cite{aalst2022pmhandbook}. In addition, on the right are six traces, of which five exhibit skipped, wrongly-ordered and unknown control-flow anomalies. For example, $\langle$\texttt{a b d}$\rangle$ has a skipped activity, which is \texttt{c}. Because of this skipped activity, the control-flow relation \texttt{b}$\,\#\,$\texttt{d} is violated, since \texttt{d} directly follows \texttt{b} in the anomalous trace.
\begin{figure}[!t]
\centering
\includegraphics[width=0.9\columnwidth]{images/FP_ANOMALIES.png}
\caption{An example event log footprint with six traces, of which five exhibit control-flow anomalies.}
\label{FP_ANOMALIES}
\end{figure}

\subsection{Control-flow anomaly detection}
Control-flow anomaly detection techniques aim to characterize the normal control flow from event logs and verify whether these deviations occur in new event logs \cite{ko2023adsystematicreview}. To develop control-flow anomaly detection techniques, \revision{process mining} has seen widespread adoption owing to process discovery and \revision{conformance checking}. On the one hand, process discovery is a set of algorithms that encode control-flow relations as a set of model elements and constraints according to a given modeling formalism \cite{aalst2022pmhandbook}; hereafter, we refer to the Petri net, a widespread modeling formalism. On the other hand, \revision{conformance checking} is an explainable set of algorithms that allows linking any deviations with the reference Petri net and providing the fitness measure, namely a measure of how much the Petri net fits the new event log \cite{aalst2022pmhandbook}. Many control-flow anomaly detection techniques based on \revision{conformance checking} (hereafter, \revision{conformance checking}-based techniques) use the fitness measure to determine whether an event log is anomalous \cite{bezerra2009pmad, bezerra2013adlogspais, myers2018icsadpm, pecchia2020applicationfailuresanalysispm}. 

The scientific literature also includes many \revision{conformance checking}-independent techniques for control-flow anomaly detection that combine specific types of trace encodings with machine/deep learning \cite{ko2023adsystematicreview, tavares2023pmtraceencoding}. Whereas these techniques are very effective, their explainability is challenging due to both the type of trace encoding employed and the machine/deep learning model used \cite{rawal2022trustworthyaiadvances,li2023explainablead}. Hence, in the following, we focus on the shortcomings of \revision{conformance checking}-based techniques to investigate whether it is possible to support the development of competitive control-flow anomaly detection techniques while maintaining the explainable nature of \revision{conformance checking}.
\begin{figure}[!t]
\centering
\includegraphics[width=\columnwidth]{images/HIGH_LEVEL_VIEW.png}
\caption{A high-level view of the proposed framework for combining \revision{process mining}-based feature extraction with dimensionality reduction for control-flow anomaly detection.}
\label{HIGH_LEVEL_VIEW}
\end{figure}

\subsection{Shortcomings of \revision{conformance checking}-based techniques}
Unfortunately, the detection effectiveness of \revision{conformance checking}-based techniques is affected by noisy data and low-quality Petri nets, which may be due to human errors in the modeling process or representational bias of process discovery algorithms \cite{bezerra2013adlogspais, pecchia2020applicationfailuresanalysispm, aalst2016pm}. Specifically, on the one hand, noisy data may introduce infrequent and deceptive control-flow relations that may result in inconsistent fitness measures, whereas, on the other hand, checking event logs against a low-quality Petri net could lead to an unreliable distribution of fitness measures. Nonetheless, such Petri nets can still be used as references to obtain insightful information for \revision{process mining}-based feature extraction, supporting the development of competitive and explainable \revision{conformance checking}-based techniques for control-flow anomaly detection despite the problems above. For example, a few works outline that token-based \revision{conformance checking} can be used for \revision{process mining}-based feature extraction to build tabular data and develop effective \revision{conformance checking}-based techniques for control-flow anomaly detection \cite{singh2022lapmsh, debenedictis2023dtadiiot}. However, to the best of our knowledge, the scientific literature lacks a structured proposal for \revision{process mining}-based feature extraction using the state-of-the-art \revision{conformance checking} variant, namely alignment-based \revision{conformance checking}.

\subsection{Contributions}
We propose a novel \revision{process mining}-based feature extraction approach with alignment-based \revision{conformance checking}. This variant aligns the deviating control flow with a reference Petri net; the resulting alignment can be inspected to extract additional statistics such as the number of times a given activity caused mismatches \cite{aalst2022pmhandbook}. We integrate this approach into a flexible and explainable framework for developing techniques for control-flow anomaly detection. The framework combines \revision{process mining}-based feature extraction and dimensionality reduction to handle high-dimensional feature sets, achieve detection effectiveness, and support explainability. Notably, in addition to our proposed \revision{process mining}-based feature extraction approach, the framework allows employing other approaches, enabling a fair comparison of multiple \revision{conformance checking}-based and \revision{conformance checking}-independent techniques for control-flow anomaly detection. Figure \ref{HIGH_LEVEL_VIEW} shows a high-level view of the framework. Business processes are monitored, and event logs obtained from the database of information systems. Subsequently, \revision{process mining}-based feature extraction is applied to these event logs and tabular data input to dimensionality reduction to identify control-flow anomalies. We apply several \revision{conformance checking}-based and \revision{conformance checking}-independent framework techniques to publicly available datasets, simulated data of a case study from railways, and real-world data of a case study from healthcare. We show that the framework techniques implementing our approach outperform the baseline \revision{conformance checking}-based techniques while maintaining the explainable nature of \revision{conformance checking}.

In summary, the contributions of this paper are as follows.
\begin{itemize}
    \item{
        A novel \revision{process mining}-based feature extraction approach to support the development of competitive and explainable \revision{conformance checking}-based techniques for control-flow anomaly detection.
    }
    \item{
        A flexible and explainable framework for developing techniques for control-flow anomaly detection using \revision{process mining}-based feature extraction and dimensionality reduction.
    }
    \item{
        Application to synthetic and real-world datasets of several \revision{conformance checking}-based and \revision{conformance checking}-independent framework techniques, evaluating their detection effectiveness and explainability.
    }
\end{itemize}

The rest of the paper is organized as follows.
\begin{itemize}
    \item Section \ref{sec:related_work} reviews the existing techniques for control-flow anomaly detection, categorizing them into \revision{conformance checking}-based and \revision{conformance checking}-independent techniques.
    \item Section \ref{sec:abccfe} provides the preliminaries of \revision{process mining} to establish the notation used throughout the paper, and delves into the details of the proposed \revision{process mining}-based feature extraction approach with alignment-based \revision{conformance checking}.
    \item Section \ref{sec:framework} describes the framework for developing \revision{conformance checking}-based and \revision{conformance checking}-independent techniques for control-flow anomaly detection that combine \revision{process mining}-based feature extraction and dimensionality reduction.
    \item Section \ref{sec:evaluation} presents the experiments conducted with multiple framework and baseline techniques using data from publicly available datasets and case studies.
    \item Section \ref{sec:conclusions} draws the conclusions and presents future work.
\end{itemize}
% %!TeX root=paper.tex
\section{Main results}


\subsection{Algorithm}\label{sec:alg}
We analyze the simple linear estimation based algorithm from \cite{vershyninPlan} for generalized linear measurements, specializing it for binary vectors.
The algorithm (Algorithm~\ref{alg:1}) takes the sensing matrix $\vecA$ and the output vector $\by$ as the inputs. 
% The output $\by$ is set to $\by$ (see \eqref{eq:spl}) for \spl\ and to $\sign{\by}$  for \bcs. 
For each column $\vecA_i,\, i\in [1:n]$ of the sensing matrix, the algorithm computes $l_i = \ipr{\by}{\vecA_i} = \sum_{j = 1}^{m}y_jA_{j,i}$ where $A_{j,i}$ is the entry at $j^{\text{th}}$ row and $i^{\text{th}}$ column.

The vector $\mathbf{l} = \inp{l_1, \ldots, l_n}$ is then sorted in decreasing order. The output of the algorithm is a set containing the indices of the top-$k$ elements of the sorted vector. That is, if the sorted vector is $\inp{l_{\alpha_1}, l_{\alpha_2}, \ldots, l_{\alpha_n}}$ where $l_{\alpha_i}\geq l_{\alpha_j}$ for $i\leq j$, then the output of the algorithm is  $\cS = \inb{\alpha_1, \ldots, \alpha_k}$.



\begin{algorithm}[tbh!]
   \caption{Top-$k$ correlated indices}
   \label{alg:1}
\begin{algorithmic}
   \STATE {\bfseries Input:} Sensing matrix $\vecA\in \bbR^{m\times n}$ and output $\mathbf{y}\in \bbR^{m}$ 
   \STATE {\bfseries Output:} a $k$-sized subset of $[1:n]$
   \STATE $\mathbf{l} \gets (0, \ldots, 0)$,\, $\mathbf{l}\in \reals^n$
        % \STATE $\mathbf{l} \gets (0, \ldots, 0)$,\, $\mathbf{l}\in \reals^n$
    \FOR{each $i\in [1:n]$}
      \STATE $l_i \gets \sum_{j = 1}^{m}y_jA_{j,i}$ 
    \ENDFOR
    \STATE Sort $\mathbf{l}$ in decreasing order and let $\cS$ be the top $k$ indices.\\
    \STATE {\bfseries Return:} $\mathcal{\cS}$ 
   % \REPEAT
   % \STATE Initialize $noChange = true$.
   % \FOR{$i=1$ {\bfseries to} $m-1$}
   % \IF{$x_i > x_{i+1}$}
   % \STATE Swap $x_i$ and $x_{i+1}$
   % \STATE $noChange = false$
   % \ENDIF
   % \ENDFOR
   % \UNTIL{$noChange$ is $true$}
\end{algorithmic}
\end{algorithm}


% \begin{algorithm}
%   \caption{Top-$k$ correlated indices}\label{alg:1}
%    \hspace*{\algorithmicindent} \textbf{Input:} Sensing matrix $\vecA\in \bbR^{m\times n}$ and output $\mathbf{w}\in \bbR^{m}$ \\
% \hspace*{\algorithmicindent} \textbf{Output:} a $k$-sized subset of $[1:n]$ \\
% \vspace{-0.4cm}
% \begin{algorithmic}[1]
%     \State $\mathbf{l} \gets (0, \ldots, 0)$,\, $\mathbf{l}\in \reals^n$
%     \For{each $i\in [1:n]$}
%       \State $l_i \gets \sum_{j = 1}^{m}A_{j,i}y_j$ 
%     \EndFor
%     \State Sort $\mathbf{l}$ in decreasing order and let $\cS$ be the top $k$ indices.\\
%     \Return $\mathcal{\cS}$ 
% \end{algorithmic}
% \end{algorithm}

The convergence and sample complexity guarantees for the algorithm are shown for the case when each entry of $\vecA$ is chosen iid $\cN(0,1)$. Note that such a matrix satisfies the power constraint in \eqref{eq:power_constraint}. As we argued in Section~\ref{sec:intro}, for the unknown signal $\bx$,  the output $\by = \vecA{\bx}+\bz$ is correlated with each column $\vecA_i$ for $i\in \cS_{\bx}$ and uncorrelated with $\vecA_j$ for $j\notin \cS_{\bx}$. In particular, for large number of samples, when $i\in \cS_{\bx}$, the inner product $\ipr{\by}{\vecA_i}$ is close to $\bbE\insq{\ipr{\by}{\vecA_i}} = m$ (for linear regression) with high probability. On the other hand, $\ipr{\by}{\vecA_j}$ is close to $0$ for $j\notin \cS_{\bx}$.  Thus, $l_i$ for $i\in \cS_{\bx}$ will dominate over $l_j$ for $j\notin \cS_{\bx}$. This line of argument also works when the output is binary, though in this case $\bbE\insq{\ipr{{\by}}{\vecA_i}}$ for $i\in \cS_{\bx}$ is different. 
This is the main idea of Algorithm~\ref{alg:1}. We first present Theorem~\ref{thm:alg_general} for generalized linear measurements.
% Theorem~\ref{thm:alg_bcs} and Theorem~\ref{thm:alg_spl} formalize this intuition for \bcs\ and \spl\ respectively. 
\begin{theorem}[Sample Complexity of Algorithm~\ref{alg:1} for GLMs]\label{thm:alg_general}
Suppose the GLM is such that for each $i\in [m]$, $y_i$ is a subgaussian random variable with subgaussian norm given by $\normi{{y_i}}_{\psi_2}$. For any $\bx$, suppose for some $L$, $\bbE\insq{g'(\vecA_i^T\bx)}\geq L\cdot
\normi{{y_i}}_{\psi_2}$   for all $i\in [m]$. Algorithm~\ref{alg:1} recovers the unknown signal with high probability if 
\begin{align}
m \geq \frac{C}{{\min\inb{L, L^2}}}(\log\inp{k}+\log\inp{n-k})\label{eq: alg_bound}
\end{align} where $C$ is some constant.
\end{theorem}
When $y_j$ is subgaussian, $y_j\vecA_{i,j}$ for any $i,j$ is a sub-exponential random variable. This observation allows us to use a concentration result for sub-exponential random variables to analyse the sample complexity. See Section~\ref{sec:proofs} for a detailed proof. 

As corollaries to Theorem~\ref{thm:alg_general}, we obtain the following sample complexity bounds for \bcs\ and \spl. These corollaries are proved in Appendix~\ref{proof:sec:alg}.
\begin{corollary}[Sample Complexity of Algorithm~\ref{alg:1} for \bcs]\label{thm:alg_bcs}
Algorithm~\ref{alg:1} recovers the unknown signal for \bcs\ with high probability if $m=O\inp{\inp{k+\sigma^2}(\log\inp{k}+\log\inp{n-k})}$.
\end{corollary}
\begin{corollary}[Sample Complexity of Algorithm~\ref{alg:1} for \spl]\label{thm:alg_spl}
Algorithm~\ref{alg:1} recovers the unknown signal for \spl\ if $m=O\inp{\inp{k+\sigma^2}(\log\inp{k}+\log\inp{n-k})}$.
\end{corollary}
 Interestingly, the sample complexity for both \bcs\ and \spl\ is the same. This can be explained by similar values of $L$, which result in similar rates of concentration of $l_i$'s around their expectation in both the cases. 
This also implies that in the regime where $m = O((k+\sigma^2)\log(n-k))$, having access to $\vecA_i^T \bx+z_i$ instead of $\sign{\vecA_i^T \bx+z_i}$, does not improve the sample complexity beyond constants.

Using Theorem~\ref{thm:alg_general}, we obtain the following corollary for logistic regression (see proof in Appendix~\ref{proof:sec:alg}).
\begin{corollary}[Sample Complexity of Algorithm~\ref{alg:1} for \logreg]\label{thm:alg_logreg}
Algorithm~\ref{alg:1} recovers the unknown signal for \logreg\ if $m=O\inp{\inp{k+1/\beta^2}\inp{\log{k}+\log\inp{n-k}}}$.
\end{corollary}
Comparing the sample complexity bounds of \bcs\ and \logreg, we notice that the sample complexity is similar except that   the noise variance $\sigma^2$ is replaced by $1/\beta^2$. This relationship is not surprising as a similar relationship was also present in the sample complexity bounds in \cite{hsu2024sample} (for logistic regression) and \cite{kuchelmeister2024finite} (for probit model). Note that, in the noiseless case, when $\beta\rightarrow \infty$ (or $\sigma = 0$ for \bcs), the sample complexity is $O(k\log{n})$, which is close to the simple counting lower bound of $k\log{n/k}$. On the other hand, when $\beta = 0$ (or $\sigma\rightarrow \infty$ for \bcs), $m\rightarrow \infty$, which makes intuitive sense as very high levels of noise render the output useless.


To compute the time complexity of the algorithm, notice that the for loop in step 2 takes $O(n\times m)$ time and step 4 takes $O(n\log{n})$ time. Thus, the computational complexity of the algorithm is $O(nm+n\log{n})$, which is $O((k+\sigma^2)n\log{n})$ for $m = O((k+\sigma^2)\log{n}$.
To compute the time complexity of the algorithm, notice that the for loop in step 2 takes $O(n\times m)$ time and step 4 takes $O(n\log{n})$ time. Thus, the computational complexity of the algorithm is $O(nm+n\log{n})$, which is $O((k+\sigma^2)n\log{n})$ for $m = O((k+\sigma^2)\log{n}$.




% \subsection{Does metric learning provably help in learning a task?}

%Several works have stipulated sample complexity for generalization bounds and excess risk for metric and similarity learning for classification tasks~\cite{Guo2013GuaranteedCV,Bellet2012SimilarityLF}
Understanding a question of the form: "if metric learning provably helps in classification" would require analyzing whether metric learning helps achieve 
 better bounds asymptotically. Ideally, this would require either providing a sharper bound for classification tasks compared to the known algorithms w/ a priori learning a metric or showing that it can be a method to beat lower bounds for learning specific tasks. Note, metric learning generally comes up with a computational overhead, for it is designed as a convex/non-convex objective. Now, even if there is an oracle that solves that objective for `free', beating a lower bound could be tricky for the general technique for achieving the lower bounds is showing statistical bottlenecks in terms of VC-dim of the hypothesis class or information theoretic arguments. 

 \subsection{Sample complexity of learning a metric}

Sample complexity for generalization bound has been established for both general metric learning problem and metric learning for classification: $k$NN in the imbalanced data setting~\cite{viola20,Gautheron2020MetricLF}, and linear classification~\cite{Bellet2012SimilarityLF}; Mahalanobis distance metric~\cite{Verma2015SampleCO}, non-linear metric learning in sparse and bounded amplification regimes (embeddings corresponding to neural networks )~\cite{Kozdoba2021DimensionFG}.
More recently, learning a Mahalanobis metric has been further explored under various settings, e.g. low dimensional metric learning~\cite{Mason2017LearningLM}, multitask metric learning~\cite{Wang2019MultitaskML}, under noise labels~\cite{Alishahi2023LinearDM}.

\subsubsection{Learning a metric for perceptron in the noisy setting}

\begin{shaded}
\noindent Setting: Given a sample space $\cX \in \reals^d$, with a distribution $P_X$. Label set $\cY = \curly{-1,1}$ with noise $P_{Y|X}$. $S^1$ and $S^{-1}$ denotes examples with two labels.\\

\noindent\textit{at time} $t = 1,2,\ldots$
\begin{enumerate}
    \item environment provides a labeled input $(x_t,y_t) \sim P_{X,Y}$
    \item learner updates $S^1_t:= S^1_{t-1} \cup \curly{x_t}$ or $S^{-1}_t:= S^{-1}_{t-1}  \cup \{x_t\}$ 
    \item learner updates $M_t \leftarrow \textsf{Gradient-Update}(M_{t-1}, S^1_t,S^{-1}_t)$ 
    \item if $w_{t-1}$ misclassifies $M_t^{1/2}x_t$ then learner updates
    \begin{enumerate}
        \item $w_t \leftarrow w_{t-1} + y_t(M_t^{1/2}x_t)$
    \end{enumerate}
    %learner updates the teaching set $\mathcal{T}_t := \mathcal{T}_{t-1} \cup \{(x_i^t,x_j^t,x_k^t)\}$
    %\item  learner makes updates $M_t \leftarrow M_{t-1} + \eta\cdot \frac{1}{t} \sum_{i=1}^t (x_i - x_k)(x_i - x_k)^{\top} - (x_i - x_k)(x_i - x_j)^{\top}$
    %\item learner updates $\hat{M_t} \leftarrow \textsf{EigenDecompose}(M_t)$
    %\item teacher receives $\ell(M_t, \mathcal{T}_t)$
\end{enumerate}
\end{shaded}
Here, for a given labeled example $(x,y)$,
\begin{align*}
\textsf{Gradient-Update}(M_{t-1}, S^1_t,S^{-1}_t) = M_{t-1} + \eta y\cdot\paren{ \sum_{x' \in S^1_t} (x - x')(x-x')^{\top} - \sum_{x' \in S^{+1}_t} (x - x')(x-x')^{\top}} 
\end{align*}
One problem of interest is if this approach achieves a better bound on the excess risk for some loss function $\ell$:
\begin{align*}
    \expctover{(x,y)\sim P_{X,Y}}{\ell(w_t(x),y)} - \expctover{(x,y)\sim P_{X,Y}}{\ell(f^*(x),y)}
\end{align*}


% \section{Proofs}
\label{sec:appendix}


% \appendix
% \section{Comparison with \cite{vershyninPlan}}\label{sec:comparison_PV}
Algorithm~\ref{alg:1} is similar to the two step estimation procedure outlined in \cite{vershyninPlan} which was given to estimate the unknown signal within a two norm guarantee. 
Computing the vector $\mathbf{l} = \inp{l_1, \ldots, l_n}$ is the same as the first step of the procedure in [Section~1.2]\cite{vershyninPlan} where a linear estimator is computed. The second step of our algorithm (sorting and keeping the top-$k$ indices) can be thought of as a projection on a feasible set [Section~1.3]\cite{vershyninPlan}. However, this requires the estimation error to be small enough for the exact recovery of a binary vector.

The setup in \cite{vershyninPlan} is for the recovery of an unknown signal with small two-norm error, whereas our problem of exact recovery of a sparse binary vector is more suited for recovery under  infinity norm. This results in weak bounds ($m\approx O(k^2)$) when we specialize various results in \cite{vershyninPlan} to our case. We first note that we require $\bbE\norm{\frac{\hat{x}}{\norm{\hat{x}}}-\bar{x}}< \sqrt{\frac{{2}}{{k}}}$ for exact recovery. Otherwise, there exist two binary $k$-sparse vectors which have hamming distance at least two. 

We first consider the 1-bit compressed sensing result in Section 3.5 (page 13). Setting the LHS to $\sqrt{\frac{{2}}{{k}}}$, we get
\begin{align*}
\sqrt{\frac{{2}}{{k}}}\leq C\sqrt{\frac{k\log\inp{2n/k}}{m}}.
\end{align*}
This implies that $m\approx C_1 k^2\log\inp{2n/k}$ for some constant $C_1$.

Next, we consider [Theorem 9.1]\cite{vershyninPlan}. Note that for 1-bit compressed sensing $\eta^2 = 1$ and 
\begin{align*}
\mu &= \bbE\insq{s_1\ipr{a_1}{\bar{x}}}\\
& =\bbE\insq{s_1\ipr{a_1}{{x}}}\\
& \stackrel{(a)}{=} \frac{1}{\sqrt{k}}\sqrt{\frac{2}{\pi}}\times\frac{k}{\sqrt{\inp{k+\sigma^2}}}\\
&= \sqrt{\frac{2}{\pi}}\times\frac{\sqrt{k}}{\sqrt{\inp{k+\sigma^2}}}.
\end{align*} where $(a)$ follows from \eqref{eq:expt4}. 
Then, 
\begin{align*}
\norm{x-\mu\bar{x}} &= \norm{x-\sqrt{\frac{2}{\pi}}\times\frac{\sqrt{k}}{\sqrt{\inp{k+\sigma^2}}}\frac{x}{\sqrt{k}}}\\
& = \norm{x-\sqrt{\frac{2}{\pi}}\times\frac{x}{\sqrt{\inp{k+\sigma^2}}}}
\end{align*} We require $\norm{x-\mu\bar{x}}<\frac{2}{\sqrt{\pi\inp{k+\sigma^2}}}$ in order to exactly recover the unknown signal $x$. 



We assume that $K$ is also a closed cone in $\bbR^n$. Then, by [Section~2.4]\cite{vershyninPlan}, $w_t(K) = tw_1(K) \leq t C\sqrt{k\log\inp{2n/k}}$ ([Section~2.4]\cite{vershyninPlan}). We choose $s = w_1(K)$. Substituting the bound for LHS and taking the limit $t\rightarrow 0$, we get
\begin{align*}
\frac{2}{\sqrt{\pi\inp{k+\sigma^2}}}\leq \frac{8 C \sqrt{k\log\inp{2n/k}}}{\sqrt{m}}.
\end{align*} Thus, $m\approx 4C(k+ \sigma^2)k\log\inp{2n/k}$. 


% % ICCV 2025 Paper Template; see https://github.com/cvpr-org/author-kit

\documentclass[10pt,twocolumn,letterpaper]{article}

%%%%%%%%% PAPER TYPE  - PLEASE UPDATE FOR FINAL VERSION
% \usepackage{iccv}              % To produce the CAMERA-READY version
\usepackage[review]{iccv}      % To produce the REVIEW version
% \usepackage[pagenumbers]{iccv} % To force page numbers, e.g. for an arXiv version

% Import additional packages in the preamble file, before hyperref
%
% --- inline annotations
%
\newcommand{\red}[1]{{\color{red}#1}}
\newcommand{\todo}[1]{{\color{red}#1}}
\newcommand{\TODO}[1]{\textbf{\color{red}[TODO: #1]}}
% --- disable by uncommenting  
% \renewcommand{\TODO}[1]{}
% \renewcommand{\todo}[1]{#1}



\newcommand{\VLM}{LVLM\xspace} 
\newcommand{\ours}{PeKit\xspace}
\newcommand{\yollava}{Yo’LLaVA\xspace}

\newcommand{\thisismy}{This-Is-My-Img\xspace}
\newcommand{\myparagraph}[1]{\noindent\textbf{#1}}
\newcommand{\vdoro}[1]{{\color[rgb]{0.4, 0.18, 0.78} {[V] #1}}}
% --- disable by uncommenting  
% \renewcommand{\TODO}[1]{}
% \renewcommand{\todo}[1]{#1}
\usepackage{slashbox}
% Vectors
\newcommand{\bB}{\mathcal{B}}
\newcommand{\bw}{\mathbf{w}}
\newcommand{\bs}{\mathbf{s}}
\newcommand{\bo}{\mathbf{o}}
\newcommand{\bn}{\mathbf{n}}
\newcommand{\bc}{\mathbf{c}}
\newcommand{\bp}{\mathbf{p}}
\newcommand{\bS}{\mathbf{S}}
\newcommand{\bk}{\mathbf{k}}
\newcommand{\bmu}{\boldsymbol{\mu}}
\newcommand{\bx}{\mathbf{x}}
\newcommand{\bg}{\mathbf{g}}
\newcommand{\be}{\mathbf{e}}
\newcommand{\bX}{\mathbf{X}}
\newcommand{\by}{\mathbf{y}}
\newcommand{\bv}{\mathbf{v}}
\newcommand{\bz}{\mathbf{z}}
\newcommand{\bq}{\mathbf{q}}
\newcommand{\bff}{\mathbf{f}}
\newcommand{\bu}{\mathbf{u}}
\newcommand{\bh}{\mathbf{h}}
\newcommand{\bb}{\mathbf{b}}

\newcommand{\rone}{\textcolor{green}{R1}}
\newcommand{\rtwo}{\textcolor{orange}{R2}}
\newcommand{\rthree}{\textcolor{red}{R3}}
\usepackage{amsmath}
%\usepackage{arydshln}
\DeclareMathOperator{\similarity}{sim}
\DeclareMathOperator{\AvgPool}{AvgPool}

\newcommand{\argmax}{\mathop{\mathrm{argmax}}}     



% It is strongly recommended to use hyperref, especially for the review version.
% hyperref with option pagebackref eases the reviewers' job.
% Please disable hyperref *only* if you encounter grave issues, 
% e.g. with the file validation for the camera-ready version.
%
% If you comment hyperref and then uncomment it, you should delete *.aux before re-running LaTeX.
% (Or just hit 'q' on the first LaTeX run, let it finish, and you should be clear).
\definecolor{iccvblue}{rgb}{0.21,0.49,0.74}
\usepackage[pagebackref,breaklinks,colorlinks,allcolors=iccvblue]{hyperref}
\usepackage{overpic}
\usepackage{makecell}
\usepackage{adjustbox} 
%\usepackage{float}
%\usepackage{arydshln}
\usepackage{multirow}
\usepackage{float}
\usepackage{bbding}
\usepackage{times}
\usepackage{epsfig}
\usepackage{graphicx}
\usepackage{amsmath}
\usepackage{amsthm}
\usepackage{amssymb}
\usepackage{booktabs}
\usepackage{xcolor}
\usepackage{enumitem}
\usepackage{lipsum}
\usepackage{diagbox}

\usepackage{algorithmic} % 引入 algorithmic 包
\usepackage{algpseudocode} % 引入 algpseudocode 包(如果需要更现代的控制结构)

\newcommand\blfootnote[1]{%
  \begingroup
  \renewcommand\thefootnote{}\footnote{#1}%
  \addtocounter{footnote}{-1}%
  \endgroup
}

\usepackage[ruled,vlined,linesnumbered]{algorithm2e}
\makeatletter
\newcommand{\algorithmfootnote}[2][\footnotesize]{%
  \let\old@algocf@finish\@algocf@finish% Store algorithm finish macro
  \def\@algocf@finish{\old@algocf@finish% Update finish macro to insert "footnote"
    \leavevmode\rlap{\begin{minipage}{\linewidth}
    #1#2
    \end{minipage}}%
  }%
}
\makeatother
%%%%%%%%% PAPER ID  - PLEASE UPDATE
\def\paperID{5291} % *** Enter the Paper ID here
\def\confName{ICCV}
\def\confYear{2025}
\newtheorem{theorem}{Theorem}[section]
\newtheorem{lemma}[theorem]{Lemma}
%%%%%%%%% TITLE - PLEASE UPDATE
\title{Q-PETR: Quant-aware Position Embedding Transformation for Multi-View 3D Object Detection\\------------ Supplementary Material ------------}

%%%%%%%%% AUTHORS - PLEASE UPDATE
\author{First Author\\
Institution1\\
Institution1 address\\
{\tt\small firstauthor@i1.org}
% For a paper whose authors are all at the same institution,
% omit the following lines up until the closing ``}''.
% Additional authors and addresses can be added with ``\and'',
% just like the second author.
% To save space, use either the email address or home page, not both
\and
Second Author\\
Institution2\\
First line of institution2 address\\
{\tt\small secondauthor@i2.org}
}

\begin{document}
\maketitle

\section{Preliminaries}
\label{sec:petr_preliminaries}

\paragraph{PETR} enhances 2D image features with 3D position-aware properties using camera-ray positional encoding (PE), enabling refined query updates for 3D bounding box prediction. Specifically, surround-view images $\mathbf{I}$ pass through a backbone to generate 2D features $\mathbf{f}_{2D}$, while camera-ray PE $\mathbf{p}_c$ is computed using camera intrinsics and extrinsics. The learnable query embeddings $q$ serve as the initial queries $\mathbf{Q}$ for the decoder. Here, $\mathbf{f}_{2D}$ serves as the values $\mathbf{V}$, and adding $\mathbf{p}_c$ to $\mathbf{f}_{2D}$ element-wise forms the 3D position-aware keys $\mathbf{K}$.

The decoder updates the queries using these key-value pairs through self-attention, cross-attention, and feed-forward network (FFN) modules. The updated query vectors are passed through an MLP to predict 3D bounding box categories and attributes, repeating for $L$ cycles. The entire PETR process is summarized in Algorithm~\ref{algo:algorithm_petr}.
\begin{algorithm}[htb]
\SetAlgoLined
\DontPrintSemicolon
\SetNoFillComment
\SetInd{1em}{1em} % <-- 关键:取消缩进
\footnotesize

\KwData{Surround-view images $\mathbf{I}$, camera intrinsics and extrinsics}
\KwResult{3D bounding boxes $\mathbf{b}^l$, categories $\mathbf{c}^l$ for $l = 1$ to $L$}

Compute image features: $\mathbf{f}_{2D} = \text{Backbone}(\mathbf{I})$\\
Compute camera-ray PE $\mathbf{p}_c$ using camera intrinsics and extrinsics\\
Form 3D position-aware keys: $\mathbf{K} = \mathbf{f}_{2D} + \mathbf{p}_c$ \tcp{Element-wise addition}
Set values: $\mathbf{V} = \mathbf{f}_{2D}$ \\
Initialize queries: $\mathbf{Q} = q$ (For simplicity, omit \(\mathbf{Q}\)'s encoding.)\\
\For{$l = 1$ to $L$}{
  $\mathbf{Q} \gets \texttt{QProj}(\mathbf{Q})$; 
  $\mathbf{K} \gets \texttt{KProj}(\mathbf{K})$; 
  $\mathbf{V} \gets \texttt{VProj}(\mathbf{V})$ \\ 
  $\mathbf{A}_s = \texttt{MultiHeadAtt}(\mathbf{Q}, \mathbf{Q}, \mathbf{Q})$ \tcp{Self-Attn}
  $\mathbf{A}_c = \texttt{MultiHeadAtt}(\mathbf{A}_s, \mathbf{K}, \mathbf{V})$ \tcp{Cross-Attn}
  $\mathbf{Q} \gets \texttt{FFN}(\mathbf{Q} + \mathbf{A}_c)$ \\
  $\mathbf{b}^{l} \gets \texttt{MLP}(\mathbf{Q})$; 
  $\mathbf{c}^{l} \gets \texttt{MLP}(\mathbf{Q})$ \\
}
\Return{$(\mathbf{b}^l, \mathbf{c}^l)$ for $l = 1$ to $L$}

\caption{Pseudo-code of PETR.}
\label{algo:algorithm_petr}
\end{algorithm}



\iffalse
\section{Quantization Failure of PETR}
\label{sec:quantization_failure_of_petr}

We evaluate the performance of several PETR configurations~\cite{liu2022petr} using the official code. Under standard 8-bit symmetric per-tensor post-training quantization (PTQ), PETR suffers significant performance degradation, with an average drop of 58.2\% in mAP and 36.9\% in NDS on the nuScenes validation dataset (see Table~\ref{tab:performance_drop_for_ptq_on_raw_petr}). 

\begin{table}[htb] %{0.45\linewidth}
    %\tiny
    %\scriptsize
    \footnotesize
    \setlength{\tabcolsep}{1.4mm}
    %\small
    %\setlength{\tabcolsep}{2.5mm}
    %\normalsize
    %\large
    \centering
    \begin{tabular}{l|c|c|c|c|c|c}
    \toprule[1.5pt]
    \multirow{2}{*}{Bac} & \multirow{2}{*}{Size} & \multirow{2}{*}{Feat} & \multicolumn{2}{c|}{FP32 Acc}                               & \multicolumn{2}{c}{INT8 Acc}   \\ \cline{4-7}
     & & & \multicolumn{1}{c|}{mAP} & \multicolumn{1}{c|}{NDS} & \multicolumn{1}{c|}{mAP} & \multicolumn{1}{c}{NDS} \\ 
    \midrule
    \textcolor{white}{0}R50\textcolor{white}{0} & 1408$\times$512 & c5 & 30.5 & 35.0 & 18.4(12.1$\downarrow$) & 27.3(\textcolor{white}{0}7.7$\downarrow$) \\
    \textcolor{white}{0}R50\textcolor{white}{0} & 1408$\times$512 & p4 & 31.7 & 36.7 & 15.7(16.0$\downarrow$) & 26.1(10.6$\downarrow$) \\
    V2-99 & \textcolor{white}{0}800$\times$320 & p4 & 37.8 & 42.6 & 10.9(26.9$\downarrow$) & 23.6(19.0$\downarrow$) \\
    V2-99 & 1600$\times$640 & p4 & 40.4 & 45.5 & 11.3(29.1$\downarrow$) & 23.9(21.6$\downarrow$) \\
    \bottomrule[1.5pt]
    \end{tabular}
    \vspace{-0.3cm}
    \caption{
    PETR's performance of 3D object detection on nuSences val set, directly utilizing the pre-trained parameters from the official repository.
    }
    %\vspace{-0.5cm}
    \label{tab:performance_drop_for_ptq_on_raw_petr}
\end{table}

\iffalse
Enlightened by the existing state-of-the-art post-training quantization methods~\cite{dong2019hawq1,dong2020hawq2,hubara2021adaq,li2021brecq,liu2021ptqvit,nagel2020up,yao2021hawq3}, the adaptive rounding proxy objective is introduced to measure the quantization performance degradation:
\begin{equation}
\begin{aligned}
    \label{eqnAdaptiveEll}
    \textstyle
    \ell(\widetilde W)
    &=\mathbf{E}_x\left[ \Vert{ (W- \widetilde W) x \Vert}^2 \right] \\
    &=tr\left( (W - \widetilde W) H (W - \widetilde W)^T \right)
\end{aligned}
\end{equation}
Where $W$ is the float weight tensor of a learnable operator, $\widetilde W$ are the quantized weight, 
$x$ is an input tensor sampled randomly from a calibration set, 
and $H$ is the second moment matrix of these vectors, interpreted as a proxy Hessian.
\fi

\paragraph{Layer-wise Quantization Error Analysis.} Quantizing a pre-trained network introduces output noise, degrading performance. To identify the root causes of quantization failure, we employ the signal-to-quantization-noise ratio (SQNR), inspired by recent PTQ advancements~\cite{pandey2023practical, yang2023efficient, pagliari2023plinio}:

\begin{equation}\label{eq:sqnr}
SQNR_{q,b} = 10\log_{10} \left( \frac{ \sum_{i=1}^N \mathbb{E}[{\mathcal{F}}{\theta}(x_{i})^{2} ] }{ \sum_{i=1}^N \mathbb{E}[ e(x_{i})^{2} ] } \right)
\end{equation}

Here, $N$ is the number of calibration data points; $\mathcal{F}{\theta}$ denotes the full-precision network; the quantization error is $e(x_i) = \mathcal{F}{\theta}(x_i) - \mathcal{Q}{q,b}(\mathcal{F}{\theta}(x_i))$; and $\mathcal{Q}_{q,b}(\mathcal{F}_\theta)$ denotes the network output when only the target layer is quantized to $b$ bits, with all other layers kept at full precision.

Since 8-bit weight quantization results in only a minor loss of precision, we focus on quantization errors arising from operator inputs. Using the PETR configuration from the first row of Table~\ref{tab:performance_drop_for_ptq_on_raw_petr}, we obtain layer-wise SQNRs, depicted in Fig.~\ref{fig:ASQNR}. From these results, we identify three main factors contributing to quantization errors:

\paragraph{Observation 1: Position Encoding Design Flaws Lead to Quantization Difficulties.}
Our in-depth analysis reveals that the quantization issues in PETR fundamentally stem from a design flaw in the positional encoding module, which manifests in two interrelated aspects. \textbf{(a) The inverse-sigmoid operator disrupts feature distribution balance.} As indicated by the red arrow in Fig.\ref{fig:ASQNR}, quantization difficulties stem from PETR's positional encoding module. Analyzing its construction (Fig.\ref{fig:pe_compare}(a)), we find that the inverse-sigmoid operation induces an imbalanced feature distribution. Specifically, Fig.~\ref{fig:distribution_before_and_after_insigmoid} shows that before applying inverse-sigmoid, the feature distribution is balanced and quantization-friendly, whereas afterward, it exhibits significant outliers. \textbf{(b) Magnitude Disparity between Camera-ray PE and Image Features.} As highlighted by the purple arrow in Fig.~\ref{fig:ASQNR}, applying 8-bit symmetric linear quantization to the 3D position-aware key $\mathbf{K}$ leads to significant performance degradation. To investigate this phenomenon, we conduct a statistical analysis of the magnitude distributions between image features and camera-ray positional encodings (PE). As illustrated in Fig.~\ref{fig:pe_img_compare}, both token-wise and channel-wise comparisons reveal that camera-ray PE exhibits an order-of-magnitude larger dynamic range (typically within $\pm$120) compared to image features (confined to $\pm$3). This severe imbalance creates a critical issue during quantization. When applying symmetric linear quantization with an 8-bit integer range (-128 to 127), the scaling factor \( s \) in Eq~\ref{e:eq2} becomes dominated by the extreme values of PE. Consequently, image features---occupying only ~2.5\% of the total dynamic range—are compressed into merely 7 discrete quantization bins (-3 to +3). As visualized in Fig.~\ref{fig:magnitude_distributions_of_image_feature_and_camera_ray_PE}, over 95\% of the original image feature variations collapse into the zero-centered bins, resulting in catastrophic information loss. This quantization artifact directly explains the observed performance drop in Table~\ref{tab:performance_drop_for_ptq_on_raw_petr}. To mitigate this issue, we propose two essential modifications: 1) eliminating the inverse-sigmoid operation that exacerbates outlier magnitudes, and 2) redesigning the positional encoding architecture to align its magnitude distribution with that of image features. These adaptations ensure balanced quantization resolution allocation, preserving critical information in both PE and image features. To mitigate this issue, we propose two essential modifications: 1) eliminating the inverse-sigmoid operation that exacerbates outlier magnitudes, and 2) redesigning the positional encoding architecture to align its magnitude distribution with that of image features. These adaptations ensure balanced distribution for quantization, preserving critical information in both PE and image features.
% To address these challenges, it is imperative to avoid the inverse-sigmoid operation and to redesign the positional encoding so that its magnitude distribution aligns better with that of the image features, thereby enhancing overall quantization-friendliness.


\begin{figure}[htb]
\centering
	\includegraphics[width=0.8\linewidth]{./figs/coor3d_before_and_after_inversesigmoid.pdf}
    %\vspace{-0.5cm}
	\caption{Feature distribution before and after the inverse-sigmoid operator. Red arrows highlight outliers.}
	\label{fig:distribution_before_and_after_insigmoid}
\end{figure}
\vspace{-0.5cm}
\begin{figure}[htb]
\centering
	\includegraphics[width=0.75\linewidth]{./figs/pe_img_compare.pdf}
    %\vspace{-0.3cm}
	\caption{Magnitude Distribution of Image Features and Positional Encodings: A Token-wise and Channel-wise Comparison}
	\label{fig:pe_img_compare}
\end{figure}


\begin{figure}[htb]
\centering
	\includegraphics[width=0.8\linewidth]{./figs/img_feat_and_pe_distribution.pdf}
    \vspace{-0.3cm}
	\caption{The distributions of image features and camera-ray position encodings after symmetric quantization using the quantization parameters derived from the 3D position-aware $\mathbf{K}$.}
	\label{fig:magnitude_distributions_of_image_feature_and_camera_ray_PE}
\end{figure}
\vspace{-0.5cm}


\paragraph{Observation 2: Dual-Dimensional Heterogeneity in Cross-Attention Leads to Quantization Bottlenecks.}

As evidenced by the green arrow in Fig.~\ref{fig:ASQNR} and further clarified in Fig.~\ref{fig:scaled_dot_product}, the scaled dot-product in cross-attention exhibits pronounced heterogeneity on two levels. First, the inter-head variance spans 2–3 orders of magnitude, while within each head, the value distribution is extremely broad (e.g., ranging beyond [$-10^3$, $10^3$]). We merge the head and query dimensions to directly reveal the row-wise feature distribution. The results show that regardless of whether quantization is performed per head, per token, or on the entire tensor, the excessively large softmax inputs result in significant quantization errors. This confirms that existing quantization paradigms are fundamentally inadequate for handling the severe amplitude disparities in the cross-attention mechanism.



\begin{figure}[htb]
\centering
	\includegraphics[width=0.8\linewidth]{./figs/softmax_input.pdf}
    %\vspace{-0.3cm}
	\caption{The distributions of scaled dot-product in cross-attention. There are significant amplitude fluctuations along the head dimension.}
	\label{fig:scaled_dot_product}
\end{figure}


\fi

\section{Experimental Setup}\label{sec:exp_setup}
\textbf{Benchmark.}
We use the nuScenes dataset, a comprehensive autonomous driving dataset covering object detection, tracking, and LiDAR segmentation. The vehicle is equipped with one LiDAR, five radars, and six cameras providing a 360-degree view. The dataset comprises 1,000 driving scenes split into training (700 scenes), validation (150 scenes), and testing (150 scenes) subsets. Each scene lasts 20 seconds, annotated at 2 Hz.

\textbf{Metrics.}
Following the official evaluation protocol, we report the nuScenes Score (NDS), mean Average Precision (mAP), and five true positive metrics: mean Average Translation Error (mATE), Scale Error (mASE), Orientation Error (mAOE), Velocity Error (mAVE), and Attribute Error (mAAE).


\textbf{Experimental Details.}
Our experiments encompass both floating-point training and quantization configurations. For floating-point training, we follow PETR series settings, using PETR with an R50dcn backbone unless specified, and utilize the C5 feature (1/32 resolution output) as the 2D feature. Input images are at $1408 \times 512$ resolution. Both the lidar-ray PE and QD-aware lidar-ray PE use a pixel-wise depth of 30m with three anchor embeddings per axis. The 3D perception space is defined as $[-61.2, 61.2]$m along the X and Y axes, and $[-10, 10]$m along the Z axis. We also compare these positional encodings on StreamPETR, using a V2-99 backbone and input images of $800 \times 320$ resolution.

Training uses the AdamW optimizer (weight decay 0.01) with an initial learning rate of $2.0 \times 10^{-4}$, decayed via a cosine annealing schedule. We train for 24 epochs with a batch size of 8 on four NVIDIA RTX 4090 GPUs. No test-time augmentation is applied.

For quantization, we adopt 8-bit symmetric per-tensor post-training quantization, using 32 randomly selected training images for calibration. When quantizing the scaled dot-product in cross-attention, we define a candidate set of 20 scaling factors.


\section{Theoretical Analysis of Magnitude Bounds in Position Encodings}
\label{sec:mag_analysis}

\subsection{Normalization Framework and Input Conditioning}
\label{subsec:normalization}
To establish a unified analytical framework, we first formalize the spatial normalization process for various ray-based position encodings. Let $\mathbf{p} = (x, y, z)$ denote the 3D coordinates within the perception range $x, y \in [-51.2, 51.2]$ meters and $z \in [-5, 3]$ meters. The normalized coordinates $\mathbf{v} \in [0, 1]^3$ are computed as:

\begin{equation}
    \mathbf{v} = \left( \frac{x + 51.2}{102.4}, \frac{y + 51.2}{102.4}, \frac{z + 5.0}{8.0} \right)
    \label{eq:normalization}
\end{equation}

Noting that $\mathbf{v}$ is clamped to $\mathbf{v}_c$ within the range $[0, 1]$, the distribution ranges of the normalized sampled points in positional encodings are characterized as follows:
\begin{itemize}
    \item For the sampled point of Camera-Ray PE, denoted as $\mathbf{v}_c^{CR}$, the distribution spans the unit cube, i.e., $[0, 1] \times [0, 1] \times [0, 1]$.
    \item For the sampled points of LiDAR-Ray PE and QDPE, denoted as $\mathbf{v}_c^{LR}$ and $\mathbf{v}_c^{QD}$ respectively, the distributions are constrained to $[0, 0.79] \times [0, 0.79] \times [0, 1]$.
\end{itemize}
Here, the value $0.79$ is derived from the ratio $30/51.2$, where $30$ corresponds to the fixed depth setting in the encoding process. This distinction highlights the inherent differences in spatial coverage and normalization strategies employed by these positional encodings.

\subsection{Magnitude Propagation Analysis}
\label{subsec:magnitude_propagation}

\subsubsection{Camera-Ray Position Encoding}
\label{subsubsec:camera_pe}
As illustrated in Fig.~\ref{fig:pe_compare} (a), the encoding pipeline consists of two critical stages:

\textbf{Stage 1: Inverse Sigmoid Transformation}
\begin{equation}
    \hat{\mathbf{v}}^{CR} = \ln\left(\frac{\mathbf{v}_c^{CR} + \epsilon}{1 - (\mathbf{v}_c^{CR} + \epsilon)}\right), \quad \epsilon = 10^{-5}
    \label{eq:logit_transform_cr}
\end{equation}
Empirical analysis reveals a maximum magnitude $\eta_{\text{max}} = \max(\|\hat{\mathbf{v}}^{CR}\|_\infty) \approx 11.5$.

\textbf{Stage 2: MLP Projection} (Through Two Fully-Connected Layers)
\begin{equation}
    \text{PE}_{\text{CR}} = \mathbf{W}_2 \sigma(\mathbf{W}_1 \hat{\mathbf{v}}^{CR} + \mathbf{b}_1) + \mathbf{b}_2
    \label{eq:mlp_transform_cr}
\end{equation}
where $\sigma$ denotes the ReLU activation function. Let $\Gamma = \max(\|\mathbf{W}_1\|_{\max}, \|\mathbf{W}_2\|_{\max})$ be the maximum weight magnitude. We derive the upper bound:
\begin{equation}
    \| \text{PE}_{\text{CR}} \|_\infty \leq 256 \cdot 192 \cdot \Gamma^2 \cdot 11.5
    \label{eq:cr_bound}
\end{equation}
where $192$ and $256$ denote the input tensor channels for $\mathbf{W}_1$ and $\mathbf{W}_2$, respectively.

\subsubsection{LiDAR-Ray Position Encoding}
\label{subsubsec:lidar_pe}
Unlike Camera-Ray PE, the encoding process of LiDAR-Ray PE introduces sinusoidal modulation between the inverse sigmoid transformation and MLP projection, as shown in Fig.~\ref{fig:pe_compare} (b). The magnitude propagation for LiDAR-Ray PE is as follows:

\textbf{Stage 1: Inverse Sigmoid Transformation}
\begin{equation}
    \hat{\mathbf{v}}^{LR} = \ln\left(\frac{\mathbf{v}_c^{LR} + \epsilon}{1 - (\mathbf{v}_c^{LR} + \epsilon)}\right), \quad \epsilon = 10^{-5}
    \label{eq:logit_transform_lr}
\end{equation}
Empirical analysis reveals a maximum magnitude $\eta_{\text{max}} = \max(\|\hat{\mathbf{v}}^{LR}\|_\infty) \approx 1.8$.

\textbf{Stage 2: Spectral Embedding}
\begin{equation}
    \phi(\hat{\mathbf{v}}^{LR}) = \bigoplus_{k=1}^{32} \left[\sin(\omega_k \hat{\mathbf{v}}^{LR}), \cos(\omega_k \hat{\mathbf{v}}^{LR})\right]
    \label{eq:sinusoidal}
\end{equation}
where $\bigoplus$ denotes concatenation. This ensures:
\begin{equation}
    \| \phi(\hat{\mathbf{v}}^{LR}) \|_\infty \leq 1.0
    \label{eq:sin_bound}
\end{equation}

\textbf{Stage 3: MLP Projection} (Following setting in Camera-Ray PE)
\begin{equation}
    \| \text{PE}_{\text{LR}} \|_\infty \leq 256 \cdot 192 \cdot \Gamma^2 \cdot 1.0
    \label{eq:lr_bound}
\end{equation}

\subsubsection{Ours QD-PE}
\label{subsubsec:qd_pe}
The proposed encoding introduces anchor-based constraints, as depicted in Fig.~\ref{fig:pe_compare} (c):

\textbf{Stage 1: Anchor Interpolation} (For Each Axis $\alpha \in \{x, y, z\}$)
\begin{equation}
    \mathbf{e}_\alpha = \frac{p_\alpha - L_\alpha^i}{\Delta L_\alpha} \mathbf{E}_\alpha^{i+1} + \frac{L_\alpha^{i+1} - p_\alpha}{\Delta L_\alpha} \mathbf{E}_\alpha^i
    \label{eq:anchor_interp}
\end{equation}
where $\mathbf{E}_\alpha^i$ denotes learnable anchor embeddings.
Via Theorem~\ref{thm:anchor}, the magnitude is constrain to:
\begin{equation}
    \| \mathbf{e}_\alpha \|_\infty \leq \gamma \quad 
    \label{eq:anchor_bound}
\end{equation}

\textbf{Stage 2: MLP Projection}
\begin{equation}
    \| \text{PE}_{\text{QD}} \|_\infty \leq 256 \cdot 192 \cdot \Gamma^2 \cdot 0.8
    \label{eq:qd_bound}
\end{equation}

\subsection{Comparative Magnitude Analysis}
\label{subsec:comparative}
The derived bounds reveal fundamental differences in magnitude scaling:
\begin{align}
    \frac{\| \text{PE}_{\text{CR}} \|}{\| \text{PE}_{\text{LR}} \|} &\approx \frac{11.5}{1.0} = 11.5 
    \label{eq:ratio_lidar} \\
    \frac{\| \text{PE}_{\text{CR}} \|}{\| \text{PE}_{\text{QD}} \|} &\approx \frac{11.5}{0.8} = 14.3
    \label{eq:ratio_qd}
\end{align}
This analysis demonstrates that QD-PE requires $14\times$ less quantization range than Camera-Ray PE.

\subsection{Theoretical Guarantee of Magnitude Constraints}
\label{subsec:theorem}
\begin{theorem}[Anchor Embedding Magnitude Bound]
\label{thm:anchor}
Let $\mathbf{E}_\alpha^i, \mathbf{E}_\alpha^{i+1}$ be adjacent anchor embeddings with $\|\mathbf{E}_\alpha^i\|_\infty \leq \gamma$. For any point $p_\alpha \in [L_\alpha^i, L_\alpha^{i+1}]$, its interpolated embedding satisfies:
\begin{equation}
    \| \mathbf{e}_\alpha \|_\infty \leq \gamma
\end{equation}
\end{theorem}

\begin{proof}
Let $\lambda = \frac{p_\alpha - L_\alpha^i}{\Delta L_\alpha} \in [0, 1]$. The interpolated embedding becomes:
\begin{equation}
    \mathbf{e}_\alpha = \lambda \mathbf{E}_\alpha^{i+1} + (1 - \lambda) \mathbf{E}_\alpha^i
\end{equation}
For any component $k$:
\begin{equation}
    |e_{\alpha,k}| \leq \lambda |E_{\alpha,k}^{i+1}| + (1 - \lambda) |E_{\alpha,k}^i| \leq \lambda \gamma + (1 - \lambda) \gamma = \gamma
\end{equation}
Thus, $\|\mathbf{e}_\alpha\|_\infty \leq \gamma$ holds for all dimensions.
\end{proof}
Through the application of regularization (e.g., L2 constraint) on the anchor embeddings $\mathbf{E}_\alpha^i$
during training, the magnitude of $\gamma$ can be explicitly controlled.
Empirically, we find that this value converges to approximately 0.8 in our experiments.


\section{More Ablation Study}\label{sec:more_ablation}

\begin{figure*}[htb]
\centering
	\includegraphics[width=0.8\linewidth]{./figs/pe visual.pdf}
    %\vspace{-0.8cm}
	\caption{Qualitative comparison of the local similarity.}
	\label{fig:pe_visual_compare}
\end{figure*}

\subsection{Local Similarity of Position Encoding Features}
%\textbf{Local Similarity of Position Encoding Features.} 
Fig.~\ref{fig:pe_visual_compare} shows that QD-PE significantly outperforms 3D point PE and cameraray PE in local similarity of position encoding. Its similarity distribution appears more compact and concentrated, validating the method's superiority in local spatial information modeling and its capability to precisely capture neighborhood spatial relationships around target pixels.
% is a qualitative comparison of QD-PE, 3D point PE, and cameraray PE from the back perspective of a 3D surround-view system, focusing on the local similarity of position encoding features. The red box in the top row marks a selected pixel, and the similarity maps illustrate how each method encodes spatial relationships around that pixel. A more concentrated similarity distribution indicates a stronger position encoding. As shown, our method achieves a more compact and focused similarity, highlighting its effectiveness in capturing local spatial information compared to other approaches.

\section{Limitations}
Although our method incurs almost no quantization accuracy loss, 
users need to replace the camera-ray in the original PETR series with our proposed QDPE. 
The only drawback is that this requires retraining. 
However, from the perspective of quantization deployment,
this retraining is beneficial, 
and the floating-point precision can even be improved.


\end{document}




% \input{arxiv/logistic_regression}
% \input{arxiv/general_case}
% \newpage

% \bibliographystyle{alpha}
% \bibliography{arxiv/refs}
\end{document}