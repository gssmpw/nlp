
\documentclass[a4paper,notitlepage]{article}

\usepackage[a4paper,top=2cm,bottom=2cm,left=2.5cm,right=2.5cm,marginparwidth=1.75cm]{geometry}

\usepackage{authblk}

\usepackage{hyperref}
\hypersetup{
  colorlinks   = true, %Colours links instead of ugly boxes
  %urlcolor     = black!50!red, %Colour for external hyperlinks
  linkcolor    = black!50!brown, %Colour of internal links
  citecolor   = black!50!brown, %Colour of citations
  %bookmarks = false
}
% \usepackage{algorithm}

\RequirePackage{algorithm}
\RequirePackage{algorithmic}

% \usepackage{algorithmic}
% \usepackage{algpseudocode}

\usepackage{bm}
\usepackage{graphicx,xcolor}

\usepackage{mathtools,tikz}
\usepackage{hhline}
\usepackage{multirow}
\usepackage{enumitem}  
\usepackage{caption}
\usepackage{subcaption}

\usepackage{appendix}



\usepackage{cite}
\usepackage{amsmath,amssymb,amsfonts,amsthm,xspace}
\theoremstyle{definition}
\newtheorem{example}{Example}
\newtheorem{remark}{Remark}
\newtheorem{theorem}{Theorem}
\newtheorem{defn}{Definition}

\newtheorem*{defn*}{Definition}
\newtheorem*{lemma*}{Lemma}

\newtheorem{lemma}[theorem]{Lemma}  
\newtheorem{corollary}[theorem]{Corollary}
\newtheorem{prop}[theorem]{Proposition}
\newtheorem{ques}[theorem]{Question}


\newcommand{\w}[1]{\ensuremath{\mathsf{w}_{\mathsf{H}}(#1)}} 
\newcommand{\h}[2]{\ensuremath{\mathsf{d}_{\mathsf{H}}(#1, #2)}} 
\newcommand{\ind}[2]{\ensuremath{\mathsf{index}_{\mathsf{H}}(#1, #2)}} 
\newcommand{\ipr}[2]{\ensuremath{\langle #1, #2\rangle} }

\newcommand{\Expth}{\mbox{$\hat{{\mathbf E}}$} }
\newcommand{\Prob}{\ensuremath{{\mathbb P}}}
\newcommand{\expect}[1]{\Expt \left[ #1 \right]}
\newcommand{\prob}[1]{\Prob \left\{ #1 \right\}}
\newcommand{\defineqq}{\ensuremath{\stackrel{\textup{\tiny def}}{=}}}

\newcommand{\norm}[1]{\left\lVert#1\right\rVert_2}
\newcommand{\normi}[1]{\left\lVert#1\right\rVert}


\foreach \x in {A,...,Z}{%
\expandafter\xdef\csname vec\x \endcsname{\noexpand\ensuremath{\noexpand\mathbf{\x}}}
}

% define calligaraphic versions of all uppercase letters \cA etc
\foreach \x in {A,...,Z}{%
\expandafter\xdef\csname c\x \endcsname{\noexpand\ensuremath{\noexpand\mathcal{\x}}}
}

% define mathbb versions of all uppercase letters \bbA etc
\foreach \x in {A,...,Z}{%
\expandafter\xdef\csname bb\x \endcsname{\noexpand\ensuremath{\noexpand\mathbb{\x}}}
}

\DeclareMathOperator*{\argmax}{arg\,max}


\newcommand\reals{{\mathbb R}}
\newcommand{\wh}[1]{\ensuremath{{\left|#1\right|}_{\mathsf{H}}}} 
\newcommand{\ina}[1]{\left<#1\right>}
\newcommand{\inb}[1]{\left\{#1\right\}}
\newcommand{\inp}[1]{\left(#1\right)}
\newcommand{\insq}[1]{\left[#1\right]}
\newcommand{\inl}[1]{\left|#1\right|}
\newcommand{\bcs}{\ensuremath{\mathsf{1bCSbinary}}}
\newcommand{\logreg}{\ensuremath{\mathsf{LogisticRegression}}}
\newcommand{\spl}{\ensuremath{\mathsf{SparseLinearReg}}}
\newcommand{\topk}{\ensuremath{\mathsf{TopK-Correlations}}}
\newcommand{\bx}{\mathbf{x}}
\newcommand{\ba}{\mathbf{a}}
\newcommand{\by}{\mathbf{y}}
\newcommand{\bz}{\mathbf{z}}
\newcommand{\bs}{\mathbf{s}}
\newcommand{\bw}{\mathbf{w}}
\newcommand{\bb}{\mathbf{b}}
\newcommand{\be}{\mathbf{e}}
\newcommand{\sign}[1]{\mathsf{sign}(#1)}
\newcommand{\sorted}[1]{\mathsf{sorted}(#1)}


\newcommand{\red}[1]{{\textcolor{red}{#1}}}
\newcommand{\blue}[1]{{\textcolor{blue}{#1}}}
\newcommand{\olive}[1]{{\textcolor{olive}{#1}}}
\newcommand{\aqua}[1]{{\textcolor{cyan}{#1}}}


\iffalse
\usepackage{url}
\usepackage{microtype}






% \usepackage{amsmath}
% \usepackage{amssymb}
% \usepackage{amsthm}
\usepackage{verbatim}

% \usepackage{mdwtab}

% \usepackage{pdfpages}

\DeclareCaptionFormat{myformat}{\fontsize{8}{9}\selectfont#1#2#3}
\captionsetup{format=myformat}

% \usepackage{graphicx}
% \usepackage{textcomp}











% If you use natbib package, activate the following three lines:
% \usepackage[round]{natbib}
\usepackage[square]{natbib}
\renewcommand{\bibname}{References}
\renewcommand{\bibsection}{\subsubsection*{\bibname}}

% If you use BibTeX in apalike style, activate the following line:
%\bibliographystyle{apalike}


\fi


\title{Exact Recovery of Sparse Binary Vectors from \\ Generalized Linear Measurements}
\author[1]{Arya Mazumdar}
\author[1]{Neha Sangwan}

\affil[1]{\small Halicioglu Data Science Institute, University of California San Diego, La Jolla, United States}
% \affil[2]{\small Halicioglu Data Science Institute, University of California San Diego, La Jolla, United States}

\begin{document}
\maketitle


\begin{abstract}
 We consider the problem of {\em exact} recovery of a $k$-sparse binary vector  from generalized linear measurements (such as {\em logistic regression}). We analyze the {\em linear estimation} algorithm (Plan, Vershynin, Yudovina, 2017), and also show information theoretic lower bounds on the number of required measurements. As a consequence of our results, for  noisy one bit quantized linear measurements (\bcs), we obtain a sample complexity of $O((k+\sigma^2)\log{n})$, where $\sigma^2$ is the noise variance. This is shown to be optimal  due to the information theoretic lower bound. 
{We also obtain tight sample complexity characterization for logistic regression.}

  Since \bcs\ is a strictly harder problem than noisy linear measurements (\spl) because of added quantization, the same sample complexity is achievable for \spl. 
  While this sample complexity  can be obtained via the popular lasso algorithm, linear estimation is  computationally more efficient. 
  Our lower bound  holds for any set of measurements for \spl\, (similar bound was known for Gaussian measurement matrices) and is closely matched by the maximum-likelihood upper bound. 
  For \spl,  it was conjectured in Gamarnik and Zadik, 2017 that there is a statistical-computational gap and the number of measurements should be at least $(2k+\sigma^2)\log{n}$ for efficient algorithms to exist. It is worth noting that our results imply that there is 
   no such statistical-computational gap for \bcs\ and logistic regression.
\end{abstract}


\documentclass[../main.tex]{subfiles}
\graphicspath{{../images/}}
\makeatletter
\def\input@path{{../images/}}
\makeatother
\begin{document}
\section{Introduction}
\begin{figure}
\centering
\begin{tikzpicture}
\node[inner sep=0pt] (ws) at (0, 0) {
\includegraphics[height=.4\textwidth, trim={10cm 0 10cm 0},clip]{world_space.png}};
\node[inner sep=0pt] (cs) at (6,0) {\includegraphics[height=.4\textwidth, trim={10cm 1cm 10cm 4cm},clip]{conf_space.png}};
\end{tikzpicture}
\vspace{-5pt}
\label{fig:pbrm_intro}
\caption{\textbf{Left}: Shows world space obstacles as grey spheres. Robots start and goal configuration is colored red and green, respectively. Configurations along the computed path are colored transparent blue. \textbf{Right:} Mapped world space scenario to configuration space. Obstacle region is the grey mesh. Red spheres are collision-free regions computed by the neural SCDF. The optimized shortest path in the convex corridor is the blue curve.}
\vspace{-25pt}
\end{figure}
Motion planning is the problem of finding a collision-free trajectory that connects a given start and goal configuration. The planning takes place in the configuration space of the robot. For single body robots, like mobile robots or drones, the configuration space and the world space are usually the same. This simplifies the planning, since explicit obstacle representations are available which enables geometrical tools like separating hyperplanes, smallest distance to obstacles etc., to be used when designing motion planning algorithms. For multi-body robots like manipulators, the situation is completely different. The world space obstacles are usually mapped to non-convex regions, and to make the problem even harder, the mapping is usually not known. Forming explicit representations of the obstacle region in the configuration space is usually too expensive or intractable. Despite all of this, sampling based planners are used with great success, which mainly is due to their use of implicit representations of the obstacle region. The basic idea is to construct a graph in the configuration space that covers and connects the collision-free region. From this graph, a path can be extracted that connects a given start and goal configuration. The approach is computationally expensive, since the graph is constructed with the smallest geometrical building block available, points, which represents a collision-check. Furthermore, the extracted paths from the graph are non-smooth and jagged due to the stochastic nature of the approach. This adds an additional post-processing step to the process, where the paths are shortcutted and smoothened, before the path can be used for tracking. Clearly a lot of time is invested to form this graph and produce smooth paths. Thus, if the obstacles start to move, then all of this work is done in no use, since all points that make up this graph need to be re-verified, which is simply too time consuming to be done in real time.
\\\\
In this work, we want to address the existing drawbacks of the sampling based planners. Our main contribution is an improved motion planner where each vertex in the graph covers a collision-free region in the form of a sphere instead of a point and where the edges are formed with neighboring intersecting spheres. This representation has the advantage of instead of returning piecewise linear paths, returning a sequence of overlapping spheres, i.e. a convex corridor, that connects a given start and goal configuration, illustrated in Figure \ref{fig:pbrm_intro}. This convex corridor allows us to use convex optimization to produce smooth trajectories, instead of computationally expensive post-processing methods. The representation further allows us to estimate the coverage of the collision-free space, which gives us awareness and feedback in the offline roadmap construction phase. Finally, our representation is simple to adapt to moving obstacles, simply requery for the new radii and recheck for intersections. 
\\\\
The spherical collision-free regions are formed using a signed distance function (SDF), which is a function that returns the smallest distance from an arbitrary point to the boundary of an obstacle. As the name implies, the distance is signed, thus if the point is inside the obstacle it is negative otherwise positive. If the distance is positive, a sphere with radius equal to the distance is guaranteed to cover a collision-free region. Using an SDF in motion planning is not new, but what is novel about our approach is that we express the distance in the configuration space instead of the world space and by doing so allows us to form these convex collision-free regions. We refer to the resulting SDF as a signed configuration distance function (SCDF). Computing an SCDF analytically is non-trivial, our approach is therefore to parameterize the SCDF with a deep neural network and learn the mapping by supervised learning. Our resulting neural SCDF can compute distances for different parameter values of obstacle shapes and we also show how multiple distances can be combined, thus making our approach flexible.
\section{Related work}
Motion planning algorithms can roughly be divided into three families, grid-based, sampling based and optimization based methods. Grid-based methods (GBM) discretize the planning space from which a graph is then compiled. A standard search method is A$^\star$ \citep{a_star}, which is classified as an \textit{informed} search method, since it employs a heuristic function to speed up the search. A$^\star$ guarantees to return an optimal path at the level of discretization used. GBMs usually discretize the planning space by a regular lattice and this limits the GBMs to problems with low dimensionality due to the curse of dimensionality. Thus, GBMs are usually limited to single-body robots where the degrees of freedom (DOF) are low. To overcome the inherent scaling problem with the GBMs, stochastic methods are usually used for multi-body robots. These methods are termed as sampling-based methods (SBM) and core members within this family are the rapidly-exploring random trees (RRT) \citep{rrt} and the probabilistic roadmap (PRM) \citep{prm}. RRT grows a tree from the start configuration and explores the collision-free region in a rapid way until it is able to connect to the goal region. RRT is usually improved by bi-directional planning \citep{rrt_connect}, i.e. an additional tree is grown from the goal configuration and the trees are tested for connection after any tree has been expanded. RRT is a single-query method, thus it searches for a path from scratch each time it is queried. Contrary to this, PRM is a multi-query method, which solves for multiple queries without starting from scratch. PRM does this by creating a roadmap (graph) that covers the collision-free space as an offline step. The graph is then used to solve for multiple queries. PRMs are used in cases where the environment does not change since the extra offline step is too computationally costly and needs to be re-done if the environment is changed. In our work, we address this inherent issue by using a different roadmap representation. Our vertices in the graph cover a collision-free region in the form of spheres and we form the edges by checking for intersecting spheres. If something in the environment changes, we recompute the spheres radii and recheck the intersections, without relying on collision detection. We use a trained neural network to compute the sphere radius, therefore querying for the radius can be done fast, hence our representation enables the PRM for dynamic environments.
\\\\
In the recent decades, optimization based methods (OBM) \citep{chomp, schulman, itomp, stomp} have been introduced as an alternative to SBM for multi-body robots. Like the SBM, the OBMs scale well to higher dimensional problems and produce smoother motion. It is common to use a SDF in the optimization since it is a smooth function, thus enabling gradient-based methods. However, the standard way of expressing the SDF is in world space. The distance therefore needs to be mapped to the configuration space by the forward kinematics. This mapping makes the optimization problem a non-linear program (NLP), which is computationally expensive to solve. Recently, a different approach has been proposed. In \cite{mp_gcs} motion planning is formulated as a convex optimization problem by using the graph of convex sets framework \citep{gcs}. The underlying idea is to decompose the collision-free space into intersecting convex sets from which a convex optimization problem is formulated. In cases where an explicit representation of the obstacles in the configuration space exists, like for single-body robots, creating collision-free convex regions can be done fast \citep{iris}. For multi-body robots, this is non-trivial. Existing work does this successfully \citep{iris_nlp, iris_c} by an optimization based approach, but the methods are still too time consuming to be used in the presence of moving obstacles. Our approach is instead to use deep learning to learn an SDF expressed in the configuration space. With this, we can query for shortest distances to the collision boundary, which allows us to expand spherical regions which are collision-free. Our approach is fast and therefore enables our suggested roadmap planner to be used in dynamic environments.
\\\\
Recent research has focused on learning collision detection \citep{fk_kernel_distance, diffco, graphdistnet} by predicting the signed distance between the robot links and the surrounding obstacles in the world space. The learned SDF is used in trajectory optimization but since the distance is expressed in the world space, the problem becomes an NLP and therefore takes a long time to solve. We take a novel approach and suggest to instead express the signed distance in the configuration space. This allows us to improve the PRM at the same time as it enables convex optimization for trajectory optimization, which runs faster and is more reliable than NLP solvers. In \cite{cspf} a learned signed distance function in the configuration space is proposed similar to our approach. However, their approach is restricted to point cloud representations, while we propose to represent the obstacles as parameterized geometric shapes, e.g. spheres. Furthermore, we also show how to use our learned SCDF to improve an existing roadmap planner.
\section{Problem formulation}
A robot is located in the world space, $\W \subset \R^3 $. The unique location of the robot is given by its configuration $\q \in \C$, where $\C$ is the configuration space. The set of points covered by the robots bodies at a certain configuration is expressed as $\B(\q) \subset \W$. The robot is surrounded by $\NrObst$ obstacles $\O = \bigcup_{i=1}^{\NrObst} \O_i$, where  $\O_i \subset \W$. The representation of the obstacle in the configuration space is the set $\C\O_i = \{\q \in \C \: |\: \B(\q) \cap \O_i \neq \emptyset \}$. The obstacle space is formed as $\Co = \bigcup_{i=1}^{\NrObst} \C \O_i$. The complement is referred to as the free space, $\Cf = \C \setminus \Co$. The path planning problem is a tuple, ($\Cf$, $\qStart$, $\qGoal$), where we want to connect a query pair, consisting of a start, $\qStart$, and goal configuration, $\qGoal$, with a geometric path, $\q(s): [0, 1] \mapsto \Cf$, such that $\q(0)=\qStart$ and $\q(1)=\qGoal$, or report correctly when such a path does not exist.
\end{document}

%!TeX root=paper.tex
\section{Main results}


\subsection{Algorithm}\label{sec:alg}
We analyze the simple linear estimation based algorithm from \cite{vershyninPlan} for generalized linear measurements, specializing it for binary vectors.
The algorithm (Algorithm~\ref{alg:1}) takes the sensing matrix $\vecA$ and the output vector $\by$ as the inputs. 
% The output $\by$ is set to $\by$ (see \eqref{eq:spl}) for \spl\ and to $\sign{\by}$  for \bcs. 
For each column $\vecA_i,\, i\in [1:n]$ of the sensing matrix, the algorithm computes $l_i = \ipr{\by}{\vecA_i} = \sum_{j = 1}^{m}y_jA_{j,i}$ where $A_{j,i}$ is the entry at $j^{\text{th}}$ row and $i^{\text{th}}$ column.

The vector $\mathbf{l} = \inp{l_1, \ldots, l_n}$ is then sorted in decreasing order. The output of the algorithm is a set containing the indices of the top-$k$ elements of the sorted vector. That is, if the sorted vector is $\inp{l_{\alpha_1}, l_{\alpha_2}, \ldots, l_{\alpha_n}}$ where $l_{\alpha_i}\geq l_{\alpha_j}$ for $i\leq j$, then the output of the algorithm is  $\cS = \inb{\alpha_1, \ldots, \alpha_k}$.



\begin{algorithm}[tbh!]
   \caption{Top-$k$ correlated indices}
   \label{alg:1}
\begin{algorithmic}
   \STATE {\bfseries Input:} Sensing matrix $\vecA\in \bbR^{m\times n}$ and output $\mathbf{y}\in \bbR^{m}$ 
   \STATE {\bfseries Output:} a $k$-sized subset of $[1:n]$
   \STATE $\mathbf{l} \gets (0, \ldots, 0)$,\, $\mathbf{l}\in \reals^n$
        % \STATE $\mathbf{l} \gets (0, \ldots, 0)$,\, $\mathbf{l}\in \reals^n$
    \FOR{each $i\in [1:n]$}
      \STATE $l_i \gets \sum_{j = 1}^{m}y_jA_{j,i}$ 
    \ENDFOR
    \STATE Sort $\mathbf{l}$ in decreasing order and let $\cS$ be the top $k$ indices.\\
    \STATE {\bfseries Return:} $\mathcal{\cS}$ 
   % \REPEAT
   % \STATE Initialize $noChange = true$.
   % \FOR{$i=1$ {\bfseries to} $m-1$}
   % \IF{$x_i > x_{i+1}$}
   % \STATE Swap $x_i$ and $x_{i+1}$
   % \STATE $noChange = false$
   % \ENDIF
   % \ENDFOR
   % \UNTIL{$noChange$ is $true$}
\end{algorithmic}
\end{algorithm}


% \begin{algorithm}
%   \caption{Top-$k$ correlated indices}\label{alg:1}
%    \hspace*{\algorithmicindent} \textbf{Input:} Sensing matrix $\vecA\in \bbR^{m\times n}$ and output $\mathbf{w}\in \bbR^{m}$ \\
% \hspace*{\algorithmicindent} \textbf{Output:} a $k$-sized subset of $[1:n]$ \\
% \vspace{-0.4cm}
% \begin{algorithmic}[1]
%     \State $\mathbf{l} \gets (0, \ldots, 0)$,\, $\mathbf{l}\in \reals^n$
%     \For{each $i\in [1:n]$}
%       \State $l_i \gets \sum_{j = 1}^{m}A_{j,i}y_j$ 
%     \EndFor
%     \State Sort $\mathbf{l}$ in decreasing order and let $\cS$ be the top $k$ indices.\\
%     \Return $\mathcal{\cS}$ 
% \end{algorithmic}
% \end{algorithm}

The convergence and sample complexity guarantees for the algorithm are shown for the case when each entry of $\vecA$ is chosen iid $\cN(0,1)$. Note that such a matrix satisfies the power constraint in \eqref{eq:power_constraint}. As we argued in Section~\ref{sec:intro}, for the unknown signal $\bx$,  the output $\by = \vecA{\bx}+\bz$ is correlated with each column $\vecA_i$ for $i\in \cS_{\bx}$ and uncorrelated with $\vecA_j$ for $j\notin \cS_{\bx}$. In particular, for large number of samples, when $i\in \cS_{\bx}$, the inner product $\ipr{\by}{\vecA_i}$ is close to $\bbE\insq{\ipr{\by}{\vecA_i}} = m$ (for linear regression) with high probability. On the other hand, $\ipr{\by}{\vecA_j}$ is close to $0$ for $j\notin \cS_{\bx}$.  Thus, $l_i$ for $i\in \cS_{\bx}$ will dominate over $l_j$ for $j\notin \cS_{\bx}$. This line of argument also works when the output is binary, though in this case $\bbE\insq{\ipr{{\by}}{\vecA_i}}$ for $i\in \cS_{\bx}$ is different. 
This is the main idea of Algorithm~\ref{alg:1}. We first present Theorem~\ref{thm:alg_general} for generalized linear measurements.
% Theorem~\ref{thm:alg_bcs} and Theorem~\ref{thm:alg_spl} formalize this intuition for \bcs\ and \spl\ respectively. 
\begin{theorem}[Sample Complexity of Algorithm~\ref{alg:1} for GLMs]\label{thm:alg_general}
Suppose the GLM is such that for each $i\in [m]$, $y_i$ is a subgaussian random variable with subgaussian norm given by $\normi{{y_i}}_{\psi_2}$. For any $\bx$, suppose for some $L$, $\bbE\insq{g'(\vecA_i^T\bx)}\geq L\cdot
\normi{{y_i}}_{\psi_2}$   for all $i\in [m]$. Algorithm~\ref{alg:1} recovers the unknown signal with high probability if 
\begin{align}
m \geq \frac{C}{{\min\inb{L, L^2}}}(\log\inp{k}+\log\inp{n-k})\label{eq: alg_bound}
\end{align} where $C$ is some constant.
\end{theorem}
When $y_j$ is subgaussian, $y_j\vecA_{i,j}$ for any $i,j$ is a sub-exponential random variable. This observation allows us to use a concentration result for sub-exponential random variables to analyse the sample complexity. See Section~\ref{sec:proofs} for a detailed proof. 

As corollaries to Theorem~\ref{thm:alg_general}, we obtain the following sample complexity bounds for \bcs\ and \spl. These corollaries are proved in Appendix~\ref{proof:sec:alg}.
\begin{corollary}[Sample Complexity of Algorithm~\ref{alg:1} for \bcs]\label{thm:alg_bcs}
Algorithm~\ref{alg:1} recovers the unknown signal for \bcs\ with high probability if $m=O\inp{\inp{k+\sigma^2}(\log\inp{k}+\log\inp{n-k})}$.
\end{corollary}
\begin{corollary}[Sample Complexity of Algorithm~\ref{alg:1} for \spl]\label{thm:alg_spl}
Algorithm~\ref{alg:1} recovers the unknown signal for \spl\ if $m=O\inp{\inp{k+\sigma^2}(\log\inp{k}+\log\inp{n-k})}$.
\end{corollary}
 Interestingly, the sample complexity for both \bcs\ and \spl\ is the same. This can be explained by similar values of $L$, which result in similar rates of concentration of $l_i$'s around their expectation in both the cases. 
This also implies that in the regime where $m = O((k+\sigma^2)\log(n-k))$, having access to $\vecA_i^T \bx+z_i$ instead of $\sign{\vecA_i^T \bx+z_i}$, does not improve the sample complexity beyond constants.

Using Theorem~\ref{thm:alg_general}, we obtain the following corollary for logistic regression (see proof in Appendix~\ref{proof:sec:alg}).
\begin{corollary}[Sample Complexity of Algorithm~\ref{alg:1} for \logreg]\label{thm:alg_logreg}
Algorithm~\ref{alg:1} recovers the unknown signal for \logreg\ if $m=O\inp{\inp{k+1/\beta^2}\inp{\log{k}+\log\inp{n-k}}}$.
\end{corollary}
Comparing the sample complexity bounds of \bcs\ and \logreg, we notice that the sample complexity is similar except that   the noise variance $\sigma^2$ is replaced by $1/\beta^2$. This relationship is not surprising as a similar relationship was also present in the sample complexity bounds in \cite{hsu2024sample} (for logistic regression) and \cite{kuchelmeister2024finite} (for probit model). Note that, in the noiseless case, when $\beta\rightarrow \infty$ (or $\sigma = 0$ for \bcs), the sample complexity is $O(k\log{n})$, which is close to the simple counting lower bound of $k\log{n/k}$. On the other hand, when $\beta = 0$ (or $\sigma\rightarrow \infty$ for \bcs), $m\rightarrow \infty$, which makes intuitive sense as very high levels of noise render the output useless.


To compute the time complexity of the algorithm, notice that the for loop in step 2 takes $O(n\times m)$ time and step 4 takes $O(n\log{n})$ time. Thus, the computational complexity of the algorithm is $O(nm+n\log{n})$, which is $O((k+\sigma^2)n\log{n})$ for $m = O((k+\sigma^2)\log{n}$.
To compute the time complexity of the algorithm, notice that the for loop in step 2 takes $O(n\times m)$ time and step 4 takes $O(n\log{n})$ time. Thus, the computational complexity of the algorithm is $O(nm+n\log{n})$, which is $O((k+\sigma^2)n\log{n})$ for $m = O((k+\sigma^2)\log{n}$.




\subsection{Lower bounds on sample complexity}\label{sec:sample_compexity}
We establish a lower bound for generalized linear measurements using standard information-theoretic arguments based on Fano's inequality. While the upper bound in Theorem~\ref{thm:alg_general} is derived for the maximum probability of error over all  $k$-sparse vectors, the lower bound applies even in the weaker setting of the average probability of error, where 
$\bx$ is chosen uniformly at random.
\begin{theorem}[Lower bound for GLMs]\label{thm: lower_bdglm} Consider any  sensing matrix $\vecA$.
For a uniformly chosen $k$-sparse vector $\bx$, an algorithm $\phi$ satisfies $$\bbP\inp{\phi(\vecA, \by) \neq \bx}\leq \delta$$   only if the number of measurements $$m\geq \frac{k\log\inp{\frac{n}{k}}}{I}\inp{1 - \frac{h_2(\delta) + \delta k\log{n}}{k\log{n/k}}}$$ for some $I$ such that $I\geq {I(y_i; \bx|\vecA)}, \, i\in [m]$. In particular, when $y\in \inb{-1, 1}$, we have $\bbE\insq{\inp{g(\vecA_i^T\bx)}^2} \geq I(y_i, \bx|\vecA)$ where the expectation is over the randomness of $\vecA$ and $\bx$.
\end{theorem}
The lower bound can be interpreted in terms of a communication problem, where the input message $\bx$ is encoded to $\vecA\bx$. The decoding function takes in as input the encoding map $\vecA$ and the output vector $\by$ in order to recover $\bx$ with high probability. For optimal recovery, one needs at least $\frac{\text{message entropy}}{\text{capacity}}$ number of measurements (follows from noisy channel coding theorem~\cite{thomas2006elements}). In Theorem~\ref{thm: lower_bdglm}, the entropy of the message set $\log{n \choose k}\approx k\log{n/k}$ and the proxy for capacity is the upper bound on mutual information $I$. We provide a detailed proof of the theorem in  Section~\ref{sec:proofs}.


We first present lower bounds for \bcs\  and \logreg. The lower bound for \bcs\ is given for any sensing matrix $\vecA$ which satisfies the power constraint given by \eqref{eq:power_constraint}, whereas the one for \logreg\ is only for the special case when each entry of the sensing matrix is iid $\cN(0,1)$. Recall that \eqref{eq:power_constraint} holds in this case.  For \bcs\ (and \logreg\ respectively), we can use the upper bound of $\bbE\insq{\inp{g(\vecA_i^T\bx)}^2}$ on the mutual information term. The dependence of $\sigma^2$ (and $1/\beta^2$ respectively) requires careful bounding of this term, which is done in the formal proofs in Appendix~\ref{proof:sec:lower_bd}.


As mentioned earlier, we need at least $k\log\inp{n/k}$ measurements for \bcs and \logreg. This is because the entropy of a randomly chosen $k$-sparse vector is approximately $k\log\inp{n/k}$ and we learn at most one bit with each measurement. However, due to corruption with noise, we learn less than a bit of information about the unknown signal with each measurement. The information gain gets worse as the noise level increases. 
Our lower bounds make this reasoning explicit.  
\begin{corollary}[\bcs\ lower bound]\label{thm: lower_bd_bcs} Suppose, each row $\vecA_i, \, i\in [1:m]$ of the sensing matrix $\vecA$ satisfies the power constraint~\eqref{eq:power_constraint}.
For a uniformly chosen $k$-sparse vector $\bx$, an algorithm $\phi$ satisfies $$\bbP\inp{\phi(\vecA, {\by}) \neq \bx}\leq \delta$$ for the problem of $\bcs$ only if the number of measurements $$m\geq \frac{k+\sigma^2}{2}\log\inp{\frac{n}{k}}\inp{1 - \frac{h_2(\delta) + \delta k\log{n}}{k\log{n/k}}}.$$ 
\end{corollary}

\begin{corollary}[\logreg\ lower bound]\label{thm: lower_bd_log_reg} Consider a Gaussian  sensing matrix $\vecA$ where each entry is chosen iid $N(0,1)$.
For a uniformly chosen $k$-sparse vector $\bx$, an algorithm $\phi$ satisfies $$\bbP\inp{\phi(\vecA, \bw) \neq \bx}\leq \delta$$ for the problem of $\logreg$ only if the number of measurements $$m\geq \frac{1}{2}\inp{k+\frac{1}{\beta^2}}\log\inp{\frac{n}{k}}\inp{1 - \frac{h_2(\delta) + \delta k\log{n}}{k\log{n/k}}}.$$ 
\end{corollary}



Theorem~\ref{thm: lower_bdglm} also implies an information theoretic lower bound for \spl, which is presented below and proved in Appendix~\ref{proof:sec:lower_bd}. Note that the denominator term in the bound $\frac{1}{2}\log\inp{1+\frac{k}{\sigma^2}}$ is the capacity of a Gaussian channel with power constraint $k$ and noise variance $\sigma^2$. 
\begin{corollary}[\spl\ lower bound]\label{thm: spl_lower_bd_1}
Under the average power constraint \eqref{eq:power_constraint} on  $\vecA$, for a uniformly chosen $k$-sparse vector $\bx$, an algorithm $\phi$ satisfies $$\bbP\inp{\phi(\vecA, {\by}) \neq \bx}\leq \delta$$ only if the number of measurements
$$m\geq \frac{k\log\inp{\frac{n}{k}}-\inp{h_2(\delta) + \delta k\log{n}}}{\frac{1}{2}\log\inp{1+\frac{k}{\sigma^2}}}.$$
\end{corollary} 

\subsection{Tighter upper and lower bounds for \spl}\label{sec:tighter_bounds_spl}
We present information theoretic upper and lower bounds for \spl\ in this section. Similar to Section~\ref{sec:alg}, our upper bound is for the maximum probability of error, while the lower bounds hold even for the weaker criterion of average probability of error.

We first present an upper bound based on the maximum likelihood estimator (MLE) where  we  decode to $\hat{\bx}$ if, on output $\by$, 
\begin{align*}
\hat{\bx} = \argmax_{\stackrel{\bx\in \inb{0,1}^n}{\wh{\bx} = k}}\,\, p(\by|{\bx})
\end{align*} where $p(\by|{\bx})$ denotes the probability density function of $\by$ on input $\bx$.
\begin{theorem}[MLE upper bound for \spl]\label{thm:upper_bd_mle} Suppose  entries of the measurement matrix $\vecA$ are i.i.d. $\cN(0,1).$
The MLE  is correct with high probability if 
\begin{align}m\geq \max_{l\in[1:k]}  \frac{nN(l)}{\frac{1}{2}\log\inp{\frac{ l}{2\sigma^2}+1}}\label{eq:upper_bd_mle}
\end{align}where  $N(l):=  \frac{k}{n} h_2\inp{\frac{l}{k}} + (1-\frac{k}{n})h_2\inp{\frac{l}{n-k}}$. 
\end{theorem}
We prove the theorem in Appendix~\ref{proof:MLE}. The main proof idea involves analysing the probability that the output of the MLE is $2l$ Hamming distance away from the unknown signal $\bx$ for different values of $l\in [1:k]$ (assuming $k\leq n/2$). This depends on the number of such vectors (approximately $2^{nN(l)}$) and the probability that the MLE outputs a vector which is $2l$ Hamming distance away from $\bx$. 

Note that when $l = k\inp{1-\frac{k}{n}}$, $nN(l) = nh_2(k/n)\approx k\log{\frac{n}{k}}$ and $\log\inp{\frac{k\inp{1-k/n}}{2\sigma^2}+1}\leq \log\inp{\frac{k}{2\sigma^2}+1}$.
Thus, $m$ is at least $\frac{2k\log{n/k}}{\log\inp{\frac{k}{2\sigma^2}+1}}$ (see the bound for Corollary~\ref{thm: spl_lower_bd_1}). It is not immediately clear if this value of $l= k\inp{1-\frac{k}{n}}$ is the optimizer. However, for large $n$, this appears to be the case numerically as shown in Plot~\ref{plot:1}.

\begin{figure}[t]
\includegraphics[width=7cm]{Unknown2.png}
\centering
\caption{The figure shows the plot of the MLE upper bound \eqref{eq:upper_bd_mle} (given by m1) for different values of $k$. This is displayed in blue color. A plot of $\frac{2nN(l)}{\log\inp{\frac{ l}{2\sigma^2}+1}}$ is also presented for $l = k\inp{1-\frac{k}{n}}$ in orange color, given by m2. A part of the plot is zoomed in to emphasize the closeness between the lines. In these plots,  $\sigma^2$ is set to 1,  $n$ is 50000 and $k$ ranges from 1000 to 25000 $(n/2)$. }\label{plot:1}
\end{figure}


Inspired by the MLE analysis, we derive a lower bound with the same structure as \eqref{eq:upper_bd_mle}. We generate the unknown signal $\bx$ using the following distribution: A vector $\tilde{\bx}$ is chosen uniformly at random from the set of all $k$-sparse vectors. Given $\tilde{\bx}$, the unknown input signal $\bx$ is chosen uniformly from the set of all $k$-sparse vector which are at a Hamming distance $2l$ from $\bx$. 
The lower bound is then obtained by computing upper and lower bounds on $I(\vecA, \by;\bx|\tilde{\bx})$.
We show this lower bound only for random matrices where each entry is chosen iid $\cN(0,1)$.
\begin{theorem}[\spl\ lower bound]\label{thm:lower_bd_spl}
If each entry of $\vecA$ is chosen iid $\cN(0,1)$, then for a uniformly chosen $k$-sparse vector $\bx$, an algorithm $\phi$ satisfies 
\begin{align}
    \bbP\inp{\phi(\vecA, {\by}) \neq \bx}\leq \delta\label{eq:spl_lower_bd_l}
\end{align}  only if the number of measurements $$m\geq \max_l\frac{nN(l) - 2\log{n}- h_2(\delta) - \delta k\log{n}}{\frac{1}{2}\log\inp{1+\frac{l}{\sigma^2}\inp{2-\frac{l}{k}}}} .$$
\end{theorem} The proof of Theorem~\ref{thm:lower_bd_spl} is given in Appendix~\ref{proof:MLE}.

If we choose $l = k\inp{1-\frac{k}{n}}$ in Theorem~\ref{thm:lower_bd_spl}, we recover corollary~\ref{thm: spl_lower_bd_1} for the special case of Gaussian design.
% \begin{corollary}\label{corollary2:lower_bd_spl}
% If  each entry of $\vecA$ is chosen iid $\cN(0,1)$, then for a uniformly chosen $k$-sparse vector $\bx$, an algorithm $\phi$ satisfies 
% $$\bbP\inp{\phi(\vecA, {\by}) \neq \bx}\leq \delta$$
% only if the number of measurements 
% $$m\geq \frac{k\log\inp{\frac{n}{k}} - 2\log{n}- h_2(\delta) - \delta k\log{n}}{\log\inp{1+\frac{k}{\sigma^2}}} .$$
% \end{corollary}

% Corollary~\ref{corollary2:lower_bd_spl} can also be proved directly for any sensing matrix $\vecA$ which satisfies \eqref{eq:power_constraint} (non-necessarily a Gaussian design). 


% \begin{figure}[t]
% \includegraphics[width=8cm]{plot.png}
% \centering
% \caption{The figure shows the plot of the MLE upper bound \eqref{eq:upper_bd_mle} (given by m1) for different values of $n$. This is displayed in blue color. A plot of $\frac{2nN(l)}{\log\inp{\frac{ l}{2\sigma^2}+1}}$ is also presented for $l = k\inp{1-\frac{k}{n}}$ in orange color, given by m2. In these plots,  $\sigma^2$ is set to 1 and $k$ is $0.2n$. }\label{plot:1}
% \end{figure}


\section{Proofs}\label{sec:proofs}
\begin{proof}[Proof of Theorem~\ref{thm:alg_general}]
Consider any input $\bx$ and a sensing matrix $\vecA$ where each entry is chosen iid $\cN(0,1)$. Suppose ${\bx}$ is supported on $\cS\subseteq[1:n]$ where $|\cS|=k$. Let $\by = (y_1, \ldots, y_m)$. Consider the event
\begin{align*}
    \cF = \inb{\sum_{i=1}^m{ {y_i A_{i,j}}}> \sum_{i=1}^m{ {y_i A_{i,j'}}}\text{ for all } j \in \cS,  j'\in\cS^c}
\end{align*}
It is clear that under $\cF$, the algorithm is correct.
We will compute the probability of $\cF^c$.
\begin{align}
\bbP\inp{\cF^c} &= \bbP\inp{\bigcup_{j\in\cS}\bigcup_{j'\in \cS^c}\inb{\sum_{i=1}^m{ {y_i A_{i,j'}}}\geq \sum_{i=1}^m{ {y_i A_{i,j}}}}}\nonumber\\
&\leq \sum_{j\in\cS}\sum_{j'\in \cS^c}\bbP\inp{\sum_{i=1}^m{ {y_i A_{i,j'}}}\geq \sum_{i=1}^m{ {y_i A_{i,j}}}}\nonumber\\
&=\sum_{j\in\cS}\sum_{j'\in \cS^c}\bbP\inp{\sum_{i=1}^m{ \inp{y_i (A_{i,j'}-A_{i,j})}}\geq 0}\label{eq:log_prob_f^c1}
\end{align}

For any $i\in[1:m]$, $j\in\cS$ and $j'\in \cS^c$, we first compute $\bbE\insq{y_i (A_{i,j}-A_{i,j'})}$.
\begin{align}
\bbE\insq{y_i (A_{i,j}-A_{i,j'})} &= \bbE\insq{y_i A_{i,j}}-\bbE\insq{y_iA_{i,j'}}\nonumber\\
& \stackrel{(a)}{=} \bbE\insq{y_i A_{i,j}} \label{eq:log_expt11}\\
&\stackrel{(b)}{=}\frac{\bbE\insq{y_iA_{i,\cS}}}{k}\nonumber\\
& = \frac{\bbE\insq{y_i\vecA_{i}^T\bx}}{k}\nonumber\\
&= \frac{\bbE\insq{A_{i}^T\bx\,\bbE\insq{y_1|\vecA_i^T\bx}}}{k}\nonumber\\
& \stackrel{(c)}{=} \frac{\bbE\insq{A_{i}^T\bx g\inp{\vecA_i^T\bx}}}{k}\nonumber\\
& \stackrel{(d)}{=} {\bbE\insq{ g'\inp{\vecA_i^T\bx}}}:= E\label{eq:log_expt1_avg11}
\end{align} where $(a)$ follows from the fact that $y_i$ and  $A_{i,j'}$ are zero mean, independent random variables and $(b)$ follows by defining $A_{i,\cS} = \sum_{j\in \cS}A_{i,j}$ and noticing that the random variables $y_i A_{i,j}$ are identically distributed for all $j\in \cS$, $(c)$ follows from \eqref{eq:glm} and $(d)$ follows from Stein's lemma. 
\begin{align*}
\bbP&\inp{\sum_{i=1}^m{ \inp{y_i (A_{i,j'}-A_{i,j})}}\geq 0}\\
&=\bbP\inp{\sum_{i=1}^m{ \inp{y_i (A_{i,j}-A_{i,j'})}}\leq 0}\\
& = \bbP\inp{\sum_{i=1}^m{ \inp{y_i (A_{i,j}-A_{i,j'})}} - mE\leq -mE}\\
& \leq \bbP\inp{\left|\sum_{i=1}^m{ \inp{y_i (A_{i,j}-A_{i,j'})}} - mE\right|\geq mE}\\
\end{align*}
\vspace{-0.001cm}
To compute this, note that for all $i\in [1:m]$, $y_i$  is a subgaussian random variable and ${y_i}\inp{A_{i, j}-A_{i, j'}}$ being product of two subgaussian random variables is a subexponential random variable (see [Lemma 2.7.7]\cite{vershynin}). Note that $\bbE\insq{\sum_{i=1}^m{ \inp{y_i (A_{i,j}-A_{i,j'})}}} = mE$ where $E$ was defined in \eqref{eq:log_expt1_avg11}.
Also, 
\begin{align*}
&\hspace{-0.3cm}\normi{{y_i}\inp{A_{i, j}-A_{i, j'}} - 1}_{\psi_1}\\
&\stackrel{(a)}{\leq} C\normi{{y_i}\inp{A_{i, j}-A_{i, j'}}}_{\psi_1}\\
&\stackrel{(b)}{\leq} C\normi{{y_i}}_{\psi_2}\normi{\inp{A_{i, j}-A_{i, j'}}}_{\psi_2}\\
&\stackrel{(c)}{\leq} C\normi{{y_i}}_{\psi_2}2C'\\
& = C_1\normi{{y_i}}_{\psi_2} \qquad\qquad\text{ for some constant $C_1$}.
\end{align*} Here, $(a)$ follows from [Exercise 2.7.10]\cite{vershynin}, $(b)$ from [Lemma 2.7.7]\cite{vershynin} and $(c)$ from [Example 2.5.8]\cite{vershynin}.
With this
\begin{align*}
\bbP&\inp{\left|\sum_{i=1}^m{ \inp{y_i (A_{i,j}-A_{i,j'})}} - mE\right|\geq mE}\\
&\stackrel{(a)}{\leq} 2\exp\inp{-c\min\inp{\frac{m^2E^2}{mC_1^2\normi{{y_i}}_{\psi_2}^2}, \frac{mE}{C_1\normi{{y_i}}_{\psi_2}} }}\\
&\stackrel{(b)}{\leq} 2\exp\inp{-cm\min\inp{\frac{mL^2}{C_1^2}, \frac{mL}{C_1 }}}
\end{align*} where $(a)$ follows from [Theorem 2.8.1]\cite{vershynin} and $(b)$ follows from the assumption in the lemma that $\frac{E}{\normi{{y_i}}_{\psi_2}} = \frac{\bbE\insq{g'(\vecA_i^T\bx)}}{\normi{{y_i}}_{\psi_2}}\geq L$.
Thus, from \eqref{eq:log_prob_f^c1}, 
\begin{align*}
\bbP\inp{\cF^c}&\leq k(n-k)2\exp\inp{-C_2 m\min\inp{L^2, L}}\\
&\rightarrow 0 \text{ if }m\geq C_2\inp{\log{k}+\log\inp{n-k}}\frac{1}{\min\inp{L^2, L}}
\end{align*} for some constant $C_2$.
\end{proof}


\iffalse
% \begin{proof}[Proof of Theorem~\ref{thm:alg_bcs}]
% Consider any input $\bx$ and a sensing matrix $\vecA$ where each entry is chosen iid $\cN(0,1)$. Suppose ${\bx}$ is supported on $\cS\subseteq[1:n]$ where $|\cS|=k$. Let $\bs = (s_1, \ldots, s_m)$ denote $\sign{\by}$. Consider the event
% \begin{align*}
%     \cF = \inb{\sum_{i=1}^m{ {s_i A_{i,j}}}> \sum_{i=1}^m{ {s_i A_{i,j'}}}\text{ for all } j \in \cS,  j'\in\cS^c}
% \end{align*}
% It is clear that under $\cF$, the algorithm is correct.
% We will compute the probability of $\cF^c$.
% % Note that
% % \begin{align*}
% %     \cF^c = \bigcup_{j\in\cS}\bigcup_{j'\in \cS^c}\inb{\sum_{i=1}^m{ {s_i A_{i,j'}}}\geq \sum_{i=1}^m{ {s_i A_{i,j}}}}.
% % \end{align*}
% % Thus,
% \begin{align}
% \bbP\inp{\cF^c} &= \bbP\inp{\bigcup_{j\in\cS}\bigcup_{j'\in \cS^c}\inb{\sum_{i=1}^m{ {s_i A_{i,j'}}}\geq \sum_{i=1}^m{ {s_i A_{i,j}}}}}\nonumber\\
% &\leq \sum_{j\in\cS}\sum_{j'\in \cS^c}\bbP\inp{\sum_{i=1}^m{ {s_i A_{i,j'}}}\geq \sum_{i=1}^m{ {s_i A_{i,j}}}}\nonumber\\
% &=\sum_{j\in\cS}\sum_{j'\in \cS^c}\bbP\inp{\sum_{i=1}^m{ \inp{s_i (A_{i,j'}-A_{i,j})}}\geq 0}\label{eq:prob_f^c}
% \end{align}

% For any $i\in[1:m]$, $j\in\cS$ and $j'\in \cS^c$, we first compute $\bbE\insq{s_i (A_{i,j}-A_{i,j'})}$.
% \begin{align}
% \bbE\insq{s_i (A_{i,j}-A_{i,j'})} &= \bbE\insq{s_i A_{i,j}}-\bbE\insq{s_iA_{i,j'}}\nonumber\\
% & \stackrel{(a)}{=} \bbE\insq{ A_{i,j}\bbE\insq{s_i| A_{i,j}}} - 0\label{eq:expt1}
% \end{align} where $(a)$ follows from the fact that $s_i$ and  $A_{i,j'}$ are zero mean, independent random variables. 

% For any $\cU\subseteq[1:n]$, we denote $\sum_{l \in \cU}A_{i,l}$ by $A_{i,\cU}$.  For any $A_{i,j} = a$, 
% \begin{align*}
% \bbP&\inp{s_i = 1|A_{i,j} = a} \\
% &= \bbP\inp{A_{i,\cS\setminus\inb{j}}+z_i\geq -a} \\
% & = \bbP\inp{\frac{A_{i,\cS\setminus\inb{j}}+z_i}{\sqrt{k-1+\sigma^2}}\geq -\frac{a}{\sqrt{k-1+\sigma^2}}}\\
% & = 1-\Phi\inp{-\frac{a}{\sqrt{k-1+\sigma^2}}}
% \end{align*} where $\Phi(x) = \frac{1}{2\pi}\int_{-\infty}^{x}e^{-\frac{t^2}{2}}dt$ is the cumulative distribution function of the standard Gaussian distribution. Thus, 
% $\bbP\inp{s_i = -1|A_{i,j} = a}  = \Phi\inp{-\frac{a}{\sqrt{k-1+\sigma^2}}}$ and 
% $$\bbE\insq{s_i| A_{i,j}=a} = 1-2\Phi\inp{-\frac{a}{\sqrt{k-1+\sigma^2}}}.$$ We are now ready to compute $\bbE\insq{ A_{i,j}\bbE\insq{s_i| A_{i,j}}}$.
% \begin{align}
% \bbE&\insq{ A_{i,j}\bbE\insq{s_i| A_{i,j}}} \nonumber\\
% &= \bbE\insq{ A_{i,j}\inp{1-2\Phi\inp{-\frac{A_{i,j}}{\sqrt{k-1+\sigma^2}}}}}\nonumber\\
% &= \bbE\insq{ A_{i,j}}-2\bbE\insq{A_{i,j}\Phi\inp{-\frac{A_{i,j}}{\sqrt{k-1+\sigma^2}}}}\nonumber\\
% &= 0-2\bbE\insq{A_{i,j}\Phi\inp{-\frac{A_{i,j}}{\sqrt{k-1+\sigma^2}}}}\label{eq:expt2}
% \end{align}
% \begin{align}
% \bbE&\insq{A_{i,j}\Phi\inp{-\frac{A_{i,j}}{\sqrt{k-1+\sigma^2}}}}\nonumber\\
% & = \int_{-\infty}^{\infty}a\frac{1}{\sqrt{2\pi}}e^{-\frac{a^2}{2}}\inp{\frac{1}{\sqrt{2\pi}}\int_{-\infty}^{-\frac{a}{\sqrt{k-1+\sigma^2}}}e^{-\frac{t^2}{2}}dt}da\nonumber\\
% & = \frac{1}{2\pi}\int_{-\infty}^{\infty}\int_{-\infty}^{-\frac{a}{\sqrt{k-1+\sigma^2}}}ae^{-\frac{a^2}{2}}e^{-\frac{t^2}{2}}dt\,da\nonumber\\
% & \stackrel{(a)}{=} \frac{1}{2\pi}\int_{-\infty}^{\infty}\int_{-\infty}^{-{t}{\sqrt{k-1+\sigma^2}}}ae^{-\frac{a^2}{2}}e^{-\frac{t^2}{2}}da\,dt\nonumber\\
% & =\frac{1}{2\pi}\int_{-\infty}^{\infty}\inp{\int_{-\infty}^{-{t}{\sqrt{k-1+\sigma^2}}}ae^{-\frac{a^2}{2}}da}e^{-\frac{t^2}{2}}dt\nonumber\\
% & = \frac{1}{2\pi}\int_{-\infty}^{\infty}\inp{-e^{-\frac{t^2(k-1+\sigma^2)}{2}}}e^{-\frac{t^2}{2}}dt\nonumber\\
% & = -\frac{1}{\sqrt{2\pi\inp{k+\sigma^2}}}\int_{-\infty}^{\infty}\frac{\sqrt{k+\sigma^2}}{\sqrt{2\pi}}e^{-\frac{t^2(k+\sigma^2)}{2}}dt\nonumber\\
% & = -\frac{1}{\sqrt{2\pi\inp{k+\sigma^2}}}\label{eq:expt3}
% \end{align} where $(a)$ follows for change of variable formula for integration. From \eqref{eq:expt1},\eqref{eq:expt2} and \eqref{eq:expt3}, we have 
% \begin{align}
% \bbE\insq{s_i (A_{i,j}-A_{i,j'})} = \sqrt{\frac{2}{\pi}}\times\frac{1}{\sqrt{\inp{k+\sigma^2}}}.\label{eq:expt4}
% \end{align}
% Define $E := \sqrt{\frac{2}{\pi}}\times\frac{1}{\sqrt{\inp{k+\sigma^2}}}$. Then,
% \begin{align*}
% \bbP&\inp{\sum_{i=1}^m{ \inp{s_i (A_{i,j'}-A_{i,j})}}\geq 0}\\
% &=\bbP\inp{\sum_{i=1}^m{ \inp{s_i (A_{i,j}-A_{i,j'})}}\leq 0}\\
% & = \bbP\inp{\sum_{i=1}^m{ \inp{s_i (A_{i,j}-A_{i,j'})}} - mE\leq -mE}\\
% & \leq \bbP\inp{\left|\sum_{i=1}^m{ \inp{s_i (A_{i,j}-A_{i,j'})}} - mE\right|\geq mE}\\
% \end{align*}
% \vspace{-0.001cm}
% To compute this, notice that for all $i\in [1:m]$, $s_i$  is a subgaussian random variable (see \cite[Example 2.5.8]{vershynin}) and ${s_i}\inp{A_{i, j}-A_{i, j'}}$ being product of two subgaussian random variables is a subexponential random variable (see \cite[
% Lemma 2.7.7]{vershynin}). Note that $\bbE\insq{\sum_{i=1}^m{ \inp{s_i (A_{i,j}-A_{i,j'})}}} = mE$.
% Also, 
% \begin{align*}
% &\hspace{-0.3cm}\normi{{s_i}\inp{A_{i, j}-A_{i, j'}} - 1}_{\psi_1}\\
% &\stackrel{(a)}{\leq} C\normi{{s_i}\inp{A_{i, j}-A_{i, j'}}}_{\psi_1}\\
% &\stackrel{(b)}{\leq} C\normi{{s_i}}_{\psi_2}\normi{\inp{A_{i, j}-A_{i, j'}}}_{\psi_2}\\
% &\stackrel{(c)}{\leq} C\frac{1}{\ln{2}}2\\
% & = C_1 \qquad\qquad\text{ for some constant $C_1$}.
% \end{align*} Here, $(a)$ follows from \cite[Exercise 2.7.10]{vershynin}, $(b)$ from \cite[
% Lemma 2.7.7]{vershynin} and $(c)$ from \cite[Example 2.5.8]{vershynin}.
% With this
% \begin{align*}
% \bbP&\inp{\left|\sum_{i=1}^m{ \inp{s_i (A_{i,j}-A_{i,j'})}} - mE\right|\geq mE}\\
% &\stackrel{(a)}{\leq} 2\exp\inp{-c\min\inp{\frac{m^2E^2}{mC_1^2}, \frac{mE}{C_1} }}\\
% % &= 2\exp\inp{-c\min\inp{\frac{2m}{\pi C_1^2\inp{k+\sigma^2}}, \frac{m\sqrt{2}}{\sqrt{\pi}C_1\sqrt{\inp{k+\sigma^2}}} }}\\
% &\stackrel{(b)}{\leq} 2\exp\inp{-c\inp{\frac{2m}{\pi C_1^2\inp{k+\sigma^2}}}}
% \end{align*} where $(a)$ follows from \cite[Theorem 2.8.1]{vershynin} and $(b)$ follows by substituting value of $E$ and noting that  $ x\geq \sqrt{x}$ when $x\geq 1$.
% From \eqref{eq:prob_f^c}, 
% Thus,
% \begin{align*}
% \bbP\inp{\cF^c}&\leq k(n-k)2\exp\inp{-c\inp{\frac{2m}{\pi C_1^2\inp{k+\sigma^2}}}}\\
% &\rightarrow 0 \text{ if }m\geq C_2\inp{\log{k}+\log\inp{n-k}}\inp{k+\sigma^2}.
% \end{align*} for some constant $C_2$.


% \end{proof}

\fi


\begin{proof}[Proof of Theorem~\ref{thm: lower_bdglm}]
Suppose $\bx$ is distributed uniformly on the set of all $k$-sparse binary vectors. Then,
\begin{align}
I(\vecA, \by; \bx)  &= H(\bx) - H(\bx|\vecA, \by)\nonumber\\
&\stackrel{(a)}{\geq} \log{n \choose k } - h_2(\delta) - \delta\log\inp{{n \choose k } + 1}\nonumber\\
&\geq k\log{n/k}- h_2(\delta)- \delta k\log\inp{n}\label{eq:log_lower_bd_bcy_1}
\end{align} where $(a)$ follows from Fano's inequality [Theorem~2.10.1]\cite{thomas2006elements}.
We also note that
\begin{align*}
 I(\vecA, \by; \bx) &= I(\vecA ; \bx)    +  I(\by; \bx|\vecA)\\
 &\stackrel{(a)}{ = }0 + I(\by; \bx|\vecA).
\end{align*}where $(a)$ holds because $\vecA$ and $\bx$ are independent. Let $y_{j\in[1:i-1]}$ denote $\inp{y_1, \ldots, y_{i-1}}$. 
\begin{align}
I&(\by; \bx|\vecA) = \sum_{i = 1}^{m}I(y_i; \bx|\vecA, y_{j\in[1:i-1]})\nonumber\\
& = \sum_{i = 1}^{m}\Big(H(y_i|\vecA, y_{j\in[1:i-1]})\nonumber\\
&\qquad- H(y_i| \bx,\vecA, y_{j\in[1:i-1]})\Big)\nonumber\\
& \stackrel{(a)}{\leq}\sum_{i = 1}^{m}\inp{H(y_i|\vecA)- H(y_i| \bx,\vecA)}\nonumber\\
& = \sum_{i = 1}^{m}I(y_i;\bx|\vecA)\nonumber\\
&\stackrel{(b)}{\leq} mI \label{eq:log_lower_bd_bg}
\end{align}where $(a)$ follows from $H(y_i|\vecA, y_{j\in[1:i-1]})\leq H(y_i|\vecA)$ and $H(y_i| \bx,\vecA, y_{j\in[1:i-1]}) = H(y_i| \bx,\vecA)$ as $y_i$ is conditionally independent of $y_{j\in[1:i-1]}$ conditioned on $\bx$ and $\vecA$ and $(b)$ follows from the assumption in the Theorem. Thus, from \eqref{eq:log_lower_bd_bcy_1} and \eqref{eq:log_lower_bd_bg},
\begin{align*}
% m\frac{2k}{\pi\sigma^2} &\geq k\log{n/k}- h_2(\delta)- \delta k\log\inp{n}\\
mI &\geq  k\log\inp{n/k}\inp{1-\frac{h_2(\delta)+ \delta k\log\inp{n}}{k\log{n/k}}}
\end{align*}
This gives us the desired bound.

We can further simplify $I(y_i;\bx|\vecA)$ when $y_i\in \inb{-1,1}$, 
\begin{align*}
I(y_i;\bx|\vecA) &= H(y_i|\vecA)- H(y_i|\bx, \vecA)\\
&\stackrel{(a)}{\leq} 1- H(y_i|\bx, \vecA_i).
\end{align*}where $(a)$ holds because $H(y_i|\vecA)\leq H(y_i) =1$ and $y_i$ is conditionally independent of $(\vecA_1\ldots, \vecA_{i-1}, \vecA_{i+1}, \ldots, \vecA_{m})$ conditioned on $\vecA_i$ and $\bx$. Here $\vecA_i$, $i\in [1:m]$ denotes the $i^{\text{th}}$ row of the sensing matrix $\vecA$.

Suppose $\bx$ is fixed and $\bbP\inp{y_i = 1} = \frac{1}{2} + t$ for some $t\in [-1,1]$.  Then $\bbE\insq{y_i|\vecA_i} = 2t = g(\vecA_i^T\bx)$.
\begin{align*}
H(y_i|\vecA_i, \bx) &\stackrel{(a)}{=} \bbE\insq{h_2\inp{\frac{1}{2} + t}}\\
& \stackrel{(b)}{\geq} \bbE_{\bx}\insq{\bbE_{\vecA}\insq{4\inp{\frac{1}{2} + t}\inp{\frac{1}{2} - t}\Big|\bx}}\\
& = 1- \bbE_{\bx}\insq{\bbE\insq{\inp{2t}^2\Big|\bx}}\\
& = 1- \bbE_{\bx}\insq{\bbE\insq{\inp{g(\vecA_i^T\bx)}^2\Big|\bx}}\\
& = 1- \bbE_{\vecA, \bx}\insq{\inp{g(\vecA_i^T\bx)}^2}
\end{align*}
where in $(a)$, the expectation is over $\vecA$ and $\bx$. The inequality $(b)$ follows from [Theorem 1.2]\cite{topsoe2001bounds}. With this $I(y_i;\bx|\vecA)\leq \bbE\insq{\inp{g(\vecA_i^T\bx)}^2}$.
\end{proof}










\section{Conclusion and open problems}\label{sec:conclusion}
We analyze a simple algorithm (the ``average algorithm'' from \cite{vershyninPlan}  followed by `top-k' selection) for recovering sparse binary vectors from generalized linear measurements; along with an information theoretic lower bound. This gives optimal sample complexity characterization for \bcs\ and \logreg. On the other hand, the required number of measurements for the noisy linear case (\spl), which is $O((k+\sigma^2)\log{n})$, is as good as the sample complexity of any other known efficient algorithm for this problem, up to constants.  An interesting open problem is to find a design matrix and an efficient algorithm which requires less than $(k+\sigma^2)\log{n}$ samples for \spl. When the noise variance is zero, we show such an algorithm in  Remark~\ref{remark:noNoise}. 


We also present almost matching information theoretic upper and lower bounds for \spl\ given by  \eqref{eq:upper_bd_mle} and \eqref{eq:spl_lower_bd_l} respectively. The bounds are in the form of an optimization problem. While we present numerical evidence which suggests that \eqref{eq:upper_bd_mle} is optimized by $l = k\inp{1-\frac{k}{n}}$, a formal proof is still missing. The bounds in   \eqref{eq:upper_bd_mle} and \eqref{eq:spl_lower_bd_l} also differ slightly by constants in the denominator, which seems to be a persistent gap in this problem.


\paragraph{Acknowledgment}
This work is supported in part by NSF awards 2217058 and 2112665. The authors would like to thank Krishna Narayanan who introduced them to the binary linear regression problem at the Simons Institute program on Error-correcting codes.



\bibliography{example_paper}
\bibliographystyle{alpha}


%%%%%%%%%%%%%%%%%%%%%%%%%%%%%%%%%%%%%%%%%%%%%%%%%%%%%%%%%%%%%%%%%%%%%%%%%%%%%%%
%%%%%%%%%%%%%%%%%%%%%%%%%%%%%%%%%%%%%%%%%%%%%%%%%%%%%%%%%%%%%%%%%%%%%%%%%%%%%%%
% APPENDIX
%%%%%%%%%%%%%%%%%%%%%%%%%%%%%%%%%%%%%%%%%%%%%%%%%%%%%%%%%%%%%%%%%%%%%%%%%%%%%%%
%%%%%%%%%%%%%%%%%%%%%%%%%%%%%%%%%%%%%%%%%%%%%%%%%%%%%%%%%%%%%%%%%%%%%%%%%%%%%%%
\newpage
\appendix
\onecolumn
\section{Proofs}\label{appendix:proofs}
\subsection{Missing proofs from Section~\ref{sec:alg}}\label{proof:sec:alg}
\begin{proof}[Proof of Corollary~\ref{thm:alg_bcs}]
We need to compute a lower bound $L$ on  $\frac{\bbE\insq{g'(\vecA_i^T\bx)}}{\normi{{y_i}}_{\psi_2}}$.  Instead of computing $\bbE\insq{g'(\vecA_i^T\bx)}$, we will compute $\bbE\insq{y_iA_{i,j}}$ for  any $j$ in the support of $\bx$. From \eqref{eq:log_expt11} and \eqref{eq:log_expt1_avg11}, we note that $\bbE\insq{y_iA_{i,j}} = \bbE\insq{g'(\vecA_i^T\bx)}$.
Also note that
$\bbE\insq{y_i A_{i,j}}= \bbE\insq{ A_{i,j}\bbE\insq{y_i| A_{i,j}}}$.


For any $\cU\subseteq[1:n]$, we denote $\sum_{l \in \cU}A_{i,l}$ by $A_{i,\cU}$.  For any $A_{i,j} = a$, 
\begin{align*}
\bbP&\inp{y_i = 1|A_{i,j} = a} \\
&= \bbP\inp{A_{i,\cS\setminus\inb{j}}+z_i\geq -a} \\
& = \bbP\inp{\frac{A_{i,\cS\setminus\inb{j}}+z_i}{\sqrt{k-1+\sigma^2}}\geq -\frac{a}{\sqrt{k-1+\sigma^2}}}\\
& = 1-\Phi\inp{-\frac{a}{\sqrt{k-1+\sigma^2}}}
\end{align*} where $\Phi(x) = \frac{1}{2\pi}\int_{-\infty}^{x}e^{-\frac{t^2}{2}}dt$ is the cumulative distribution function of the standard Gaussian distribution. Thus, 
$\bbP\inp{y_i = -1|A_{i,j} = a}  = \Phi\inp{-\frac{a}{\sqrt{k-1+\sigma^2}}}$ and 
$$\bbE\insq{y_i| A_{i,j}=a} = 1-2\Phi\inp{-\frac{a}{\sqrt{k-1+\sigma^2}}}.$$ We are now ready to compute $\bbE\insq{ A_{i,j}\bbE\insq{y_i| A_{i,j}}}$.
\begin{align}
\bbE&\insq{ A_{i,j}\bbE\insq{y_i| A_{i,j}}} \nonumber\\
&= \bbE\insq{ A_{i,j}\inp{1-2\Phi\inp{-\frac{A_{i,j}}{\sqrt{k-1+\sigma^2}}}}}\nonumber\\
&= \bbE\insq{ A_{i,j}}-2\bbE\insq{A_{i,j}\Phi\inp{-\frac{A_{i,j}}{\sqrt{k-1+\sigma^2}}}}\nonumber\\
&= 0-2\bbE\insq{A_{i,j}\Phi\inp{-\frac{A_{i,j}}{\sqrt{k-1+\sigma^2}}}}\label{eq:expt2}
\end{align}
\begin{align}
\bbE&\insq{A_{i,j}\Phi\inp{-\frac{A_{i,j}}{\sqrt{k-1+\sigma^2}}}}\nonumber\\
& = \int_{-\infty}^{\infty}a\frac{1}{\sqrt{2\pi}}e^{-\frac{a^2}{2}}\inp{\frac{1}{\sqrt{2\pi}}\int_{-\infty}^{-\frac{a}{\sqrt{k-1+\sigma^2}}}e^{-\frac{t^2}{2}}dt}da\nonumber\\
& = \frac{1}{2\pi}\int_{-\infty}^{\infty}\int_{-\infty}^{-\frac{a}{\sqrt{k-1+\sigma^2}}}ae^{-\frac{a^2}{2}}e^{-\frac{t^2}{2}}dt\,da\nonumber\\
& \stackrel{(a)}{=} \frac{1}{2\pi}\int_{-\infty}^{\infty}\int_{-\infty}^{-{t}{\sqrt{k-1+\sigma^2}}}ae^{-\frac{a^2}{2}}e^{-\frac{t^2}{2}}da\,dt\nonumber\\
& =\frac{1}{2\pi}\int_{-\infty}^{\infty}\inp{\int_{-\infty}^{-{t}{\sqrt{k-1+\sigma^2}}}ae^{-\frac{a^2}{2}}da}e^{-\frac{t^2}{2}}dt\nonumber\\
& = \frac{1}{2\pi}\int_{-\infty}^{\infty}\inp{-e^{-\frac{t^2(k-1+\sigma^2)}{2}}}e^{-\frac{t^2}{2}}dt\nonumber\\
& = -\frac{1}{\sqrt{2\pi\inp{k+\sigma^2}}}\int_{-\infty}^{\infty}\frac{\sqrt{k+\sigma^2}}{\sqrt{2\pi}}e^{-\frac{t^2(k+\sigma^2)}{2}}dt\nonumber\\
& = -\frac{1}{\sqrt{2\pi\inp{k+\sigma^2}}}\label{eq:expt3}
\end{align} where $(a)$ follows for change of variable formula for integration. From \eqref{eq:expt2} and \eqref{eq:expt3}, we have 
\begin{align}
\bbE\insq{y_i A_{i,j}} = \sqrt{\frac{2}{\pi}}\times\frac{1}{\sqrt{\inp{k+\sigma^2}}}.\label{eq:expt4}
\end{align} From [Example 2.5.8]\cite{vershynin}, we also note that $\normi{{y_i}}_{\psi_2} = 1$. Thus, $L = \sqrt{\frac{2}{\pi}}\times\frac{1}{\sqrt{\inp{k+\sigma^2}}}$ and $\min\inb{L, L^2} = L^2$, which when substituted in \eqref{eq: alg_bound}  gives the desired bound.
\end{proof}

\begin{proof}[Proof of Corollary~\ref{thm:alg_spl}]
We first note that  $\frac{\bbE\insq{g'(\vecA_i^T\bx)}}{\normi{{y_i}}_{\psi_2}} = \frac{\bbE\insq{y_iA_{i,j}}}{\normi{{y_i}}_{\psi_2}}$ for  any $j$ in the support of $\bx$. This follows from \eqref{eq:log_expt11} and \eqref{eq:log_expt1_avg11}. We first compute $\bbE\insq{y_iA_{i,j}}$, which is the same as $\bbE\insq{\inp{\vecA_i^T\bx+z_i}A_{i,j}}$ for \spl. 
Note that $\bbE\insq{\inp{\vecA_i^T\bx+z_i}\inp{A_{i, j}}} = \bbE\insq{A_{i, j}^2} = 1$.
Also, from [Example 2.5.8]\cite{vershynin} 
\begin{align*}
\normi{\inp{\vecA_i^T\bx+z_i}}_{\psi_2}\leq C\sqrt{k+\sigma^2}\\
\end{align*} for some constants $C$. With this,
\begin{align*}
\frac{\bbE\insq{g'(\vecA_i^T\bx)}}{\normi{{y_i}}_{\psi_2}}\geq \frac{1}{C\sqrt{k+\sigma^2}}: =L.
\end{align*}Note that $\min\inb{L, L^2} = L^2$, which when substituted in \eqref{eq: alg_bound} gives the desired bound.
\end{proof}

\begin{proof}[Proof of Corollary~\ref{thm:alg_logreg}]
We will first compute $g(\vecA_i^T\bx) = \bbE\insq{y_i|\vecA_i^T\bx}$ for \logreg.
\begin{align*}
g(\vecA_i^T\bx)&=\bbE\insq{y_i|\vecA_i^T\bx}\\
&= \frac{1}{1+e^{-\beta \vecA_i^T\bx}}-\frac{e^{-\beta \vecA_i^T\bx}}{1+e^{-\beta \vecA_i^T\bx}}\\
&=\frac{1-e^{-\beta \vecA_i^T\bx}}{1+e^{-\beta \vecA_i^T\bx}}\\
& \stackrel{(a)}{=} {\tanh\inp{\frac{\beta \vecA_i^T\bx}{2}}}
\end{align*}where $(a)$ uses the definition of $\tanh$. Then
\begin{align*}
\bbE\insq{g'(\vecA_i^T\bx)} &= \frac{\beta}{2}\bbE\insq{\frac{1}{{\text{cosh}}^2\inp{\frac{\beta \vecA_i^T\bx}{2}}}}\\
& \stackrel{(c)}{\geq}\frac{\beta}{2}\bbE\insq{e^{-\frac{\inp{\beta\vecA_i^T\bx}^2}{4}}}
\end{align*} where $(c)$ follows from the inequality $\text{cosh}(t)\leq e^{t^2/2}$ (see [Exercise 2.2.3]\cite{vershynin}). 

% \begin{align*}
% \bbE\insq{y_i\vecA_i^T\bx} &= \bbE\insq{\vecA_i^T\bx\bbE\insq{y_i|\vecA_i^T\bx}}\\
% & = \bbE\insq{\vecA_i^T\bx\inp{\frac{1}{1+e^{-\beta \vecA_i^T\bx}}-\frac{e^{-\beta \vecA_i^T\bx}}{1+e^{-\beta \vecA_i^T\bx}}}}\\
% & = \bbE\insq{\vecA_i^T\bx\inp{\frac{1-e^{-\beta \vecA_i^T\bx}}{1+e^{-\beta \vecA_i^T\bx}}}}\\
% & = \bbE\insq{\vecA_i^T\bx\inp{\frac{e^{\beta \vecA_i^T\bx}-1}{e^{\beta \vecA_i^T\bx}+1}}}\\
% & \stackrel{(a)}{=} \bbE\insq{\vecA_i^T\bx\inp{\tanh\inp{\frac{\beta \vecA_i^T\bx}{2}}}}\\
% & \stackrel{(b)}{=} \frac{k\beta}{2}\bbE\insq{\frac{1}{{\text{cosh}}^2\inp{\frac{\beta \vecA_i^T\bx}{2}}}}\\
% & \stackrel{(c)}{\geq}\frac{k\beta}{2}\bbE\insq{e^{-\frac{\beta^2\vecA_i^T\bx^2}{2}}}\\
% \end{align*} , $(b)$ follows by noting that $\vecA_i^T\bx\sim N(0, k)$ and using Stein's lemma and $(c)$ follows from the inequality $\text{cosh}(t)\leq e^{t^2/2}$ (see \cite[Exercise 2.2.3]{vershynin}). 

Now, we need to compute $\bbE\insq{e^{-\frac{\inp{\beta\vecA_i^T\bx}^2}{4}}}$ where $\vecA_i^T\bx\sim N(0,k)$. Let $\sigma_1 := \frac{1}{\frac{\beta^2}{2}+\frac{1}{k}}$. Then

\begin{align}
\bbE\insq{e^{-\frac{\inp{\beta\vecA_i^T\bx}^2}{4}}} &= \int_{-\infty}^{\infty}\frac{1}{\sqrt{2\pi k}}e^{-\beta^2a^2/4}e^{-a^2/2k} da\nonumber\\
& = \sqrt{\frac{\sigma_1}{k}}\int_{-\infty}^{\infty}\frac{1}{\sqrt{2\pi \sigma_1}}e^{-x^2/2\sigma_1} da\nonumber\\
& = \sqrt{\frac{\sigma_1}{k}}\nonumber\\
& = \sqrt{\frac{2}{2+\beta^2 k}}\label{eq:expectation_log}
\end{align}

Thus,
\begin{align*}
\bbE\insq{g'(\vecA_i^T\bx)}&\geq \frac{\beta}{2}\sqrt{\frac{2}{2+\beta^2 k}}\\
& = \frac{1}{2}\sqrt{\frac{2}{2/\beta^2+ k}}.
\end{align*}
From [Example 2.5.8]\cite{vershynin}, we also note that $\normi{{y_i}}_{\psi_2} = 1$. Thus, $L = \frac{1}{2}\sqrt{\frac{2}{2/\beta^2+ k}}$ and $\min\inp{L, L^2} = L^2$, which gives the desired bound.

\end{proof}


\subsection{Missing proofs from Section~\ref{sec:sample_compexity}}\label{proof:sec:lower_bd}
\begin{proof}[Proof of Corollary~\ref{thm: lower_bd_bcs}]
Consider a sensing matrix $\vecA$ which satisfies the power constraint \eqref{eq:power_constraint}. 

% Suppose $\bx$ is distributed uniformly on the set of all $k$-sparse binary vectors and $\bs = \sign{\vecA\bx+\bz}$. Then,
% \begin{align}
% I(\vecA, \bs; \bx)  &= H(\bx) - H(\bx|\vecA, \bs)\nonumber\\
% &\stackrel{(a)}{\geq} \log{n \choose k } - h_2(\delta) - \delta\log\inp{{n \choose k } + 1}\nonumber\\
% &\geq k\log{n/k}- h_2(\delta)- \delta k\log\inp{n}\label{eq:lower_bd_bcs_1}
% \end{align} where $(a)$ follows from Fano's inequality \cite[Theorem~2.10.1]{thomas2006elements}.
% We also note that
% \begin{align*}
%  I(\vecA, \bs; \bx) &= I(\vecA ; \bx)    +  I(\bs; \bx|\vecA)\\
%  &\stackrel{(a)}{ = }0 + I(\bs; \bx|\vecA).
% \end{align*}where $(a)$ holds because $\vecA$ and $\bx$ are independent. Let $s_{j\in[1:i-1]}$ denote $\inp{s_1, \ldots, s_{i-1}}$. 
% \begin{align}
% I&(\bs; \bx|\vecA) = \sum_{i = 1}^{m}I(s_i; \bx|\vecA, s_{j\in[1:i-1]})\nonumber\\
% & = \sum_{i = 1}^{m}\Big(H(s_i|\vecA, s_{j\in[1:i-1]})\nonumber\\
% &\qquad- H(s_i| \bx,\vecA, s_{j\in[1:i-1]})\Big)\nonumber\\
% & \stackrel{(a)}{\leq}\sum_{i = 1}^{m}\inp{H(s_i|\vecA)- H(s_i| \bx,\vecA)}\nonumber\\
% & = \sum_{i = 1}^{m}I(s_i;\bx|\vecA)\label{eq:lower_bd_bcs}
% \end{align}where $(a)$ follows from $H(s_i|\vecA, s_{j\in[1:i-1]})\leq H(s_i|\vecA)$ and $H(s_i| \bx,\vecA, s_{j\in[1:i-1]}) = H(s_i| \bx,\vecA)$ as $s_i$ is conditionally independent of $s_{j\in[1:i-1]}$ conditioned on $\bx$ and $\vecA$.
% For any $i$, 
% \begin{align*}
% I(s_i;\bx|\vecA) &= H(s_i|\vecA)- H(s_i|\bx, \vecA)\\
% &\stackrel{(a)}{\leq} 1- H(s_i|\vecA_i^T\bx).
% \end{align*}where $(a)$ holds because $H(s_i|\vecA)\leq H(s_i) =1$ and $s_i$ is conditionally independent of $(\vecA_1\ldots, \vecA_{i-1}, A_{i+1}, \ldots, A_{m})$ and $\bx$ conditioned on $\vecA_i^T\bx$. 
Here $\vecA_i$, $i\in [1:m]$ denotes the $i^{\text{th}}$ row of the sensing matrix $\vecA$.
For any realization $b\in \bbR$ of $\vecA_i^T\bx$, 
\begin{align*}
\bbP(y_i = 1|\vecA_i^T\bx = b) &= \bbP(z_i\geq -b)= \bbP\inp{\frac{z_i}{\sigma}\geq \frac{-b}{\sigma}}\\
& = \frac{1- \sign{b}}{2} + \sign{b}Q\inp{\frac{|b|}{\sigma}}.
\end{align*} 

For $a>0$, let $R(a):=\frac{1}{\sqrt{2\pi}}\int_{0}^{a}e^{-u^2/2}du$. Then $Q(a) = \frac{1}{2}-R(a)$. Suppose $\bx$ is fixed. Then,
\begin{align*}
g(\vecA_i^T\bx) & =\bbE\insq{y_i|\vecA}  = \bbE\insq{y_i|\vecA_i^T\bx}\\
& = \frac{1- \sign{\vecA_i^T\bx}}{2} + \sign{\vecA_i^T\bx}Q\inp{\frac{|\vecA_i^T\bx|}{\sigma}} - \inp{1-\inp{\frac{1- \sign{\vecA_i^T\bx}}{2} + \sign{\vecA_i^T\bx}Q\inp{\frac{|\vecA_i^T\bx|}{\sigma}}}}\\
& = \sign{\vecA_i^T\bx}\inp{1-2Q\inp{\frac{|\vecA_i^T\bx|}{\sigma}}}\\
& = \sign{\vecA_i^T\bx}\inp{2R\inp{\frac{|\vecA_i^T\bx|}{\sigma}}}
\end{align*}
% This means that $H(s_i|\vecA_i^T\bx = b) = h_2\inp{Q(|b/\sigma|)}$ where $h_2(a)$ denotes the binary entropy of the distribution $(a, 1-a)$. We will lower bound $H(s_i|\vecA_i^T\bx = b)$ using a lower bound on the binary entropy function.
% We know that
% \begin{align*}
% H(s_i|\vecA_i^T\bx) &\stackrel{(a)}{=} \bbE\insq{h_2\inp{Q\inp{\frac{\inl{\vecA_i^T\bx}}{\sigma}}}}\\
% &\stackrel{(a)}{\geq} \bbE\insq{4{Q\inp{\frac{\inl{\vecA_i^T\bx}}{\sigma}}}\inp{1-Q\inp{\frac{\inl{\vecA_i^T\bx}}{\sigma}}}}
% \end{align*}
% where in $(a)$, the expectation is over $\vecA_i^T\bx$. The inequality $(b)$ follows from \cite[Theorem 1.2]{topsoe2001bounds}. 
% For $a>0$, let $R(a):=\frac{1}{\sqrt{2\pi}}\int_{0}^{a}e^{-u^2/2}du$. Then $Q(a) = \frac{1}{2}-R(a)$. Using this,
% \begin{align*}
% \bbE&\insq{h_2\inp{Q\inp{\frac{\inl{\vecA_i^T\bx}}{\sigma}}}}\\
% & \geq \bbE\insq{4\inp{\frac{1}{2} - R\inp{\frac{\inl{\vecA_i^T\bx}}{\sigma}}}\inp{\frac{1}{2} + R\inp{\frac{\inl{\vecA_i^T\bx}}{\sigma}}}}\\
% & = 1-4\bbE\insq{\inp{R\inp{\frac{\inl{\vecA_i^T\bx}}{\sigma}}}^2}.
% \end{align*}
For any $a>0$,
\begin{align*}
R\inp{a} &= \frac{1}{\sqrt{2\pi}}\int_{0}^{a}e^{-u^2/2}du\\
& \leq \frac{1}{\sqrt{2\pi}}\int_{0}^{a}1du = \frac{a}{\sqrt{2\pi}}.
\end{align*}
Thus, 
\begin{align*}
\bbE\insq{\inp{g\inp{\vecA_i^T\bx}^2}} &= \bbE\insq{\inp{2R\inp{\frac{|\vecA_i^T\bx|}{\sigma}}}^2}\\
&\leq \bbE\insq{4\inp{\frac{\vecA_i^T\bx}{\sqrt{2\pi}\sigma}}^2}\\
&\stackrel{(a)}{\leq}\frac{2k}{\pi\sigma^2} 
\end{align*}where $(a)$ follows from the power constraint $\bbE\insq{\inp{\vecA_i^T\bx}^2}\leq k$ (see \eqref{eq:power_constraint}). This holds for any $\bx$, including a randomly chosen sparse vector.
% \begin{align*}
%  \bbE\insq{h_2\inp{Q\inp{\frac{\vecA_i^T\bx}{\sigma}}}}&\geq 1-4\bbE\insq{\frac{\inp{\vecA_i^T\bx}^2}{{2\pi\sigma^2}}}\\
% & \stackrel{(a)}{\geq} 1-\frac{2k}{\pi\sigma^2}.
% \end{align*}Here, $(a)$ follows from the power constraint $\bbE\insq{\inp{\vecA_i^T\bx}^2}\leq k$ (see \eqref{eq:power_constraint}). 
% This implies that
% \begin{align*}
% I(s_i;\bx|\vecA)&\leq 1-\inp{1-\frac{2k}{\pi\sigma^2}} = \frac{2k}{\pi\sigma^2} 
% \end{align*}
% and from \eqref{eq:lower_bd_bcs_1} and \eqref{eq:lower_bd_bcs},
Thus,
\begin{align}
% m\frac{2k}{\pi\sigma^2} &\geq k\log{n/k}- h_2(\delta)- \delta k\log\inp{n}\\
m &\geq  \frac{\pi\sigma^2}{2k}k\log\inp{n/k}\inp{1-\frac{h_2(\delta)+ \delta k\log\inp{n}}{k\log{n/k}}}\nonumber\\
& \geq \sigma^2\log\inp{n/k}\inp{1-\frac{h_2(\delta)+ \delta k\log\inp{n}}{k\log{n/k}}}\label{eq:final_bound_bcs1}
\end{align}
% Thus,
% \begin{align}
% % m&\geq \frac{\pi\sigma^2}{2}\log\inp{n/k}\inp{1-\frac{h_2(\delta)+ \delta k\log\inp{n}}{k\log{n/k}}}\nonumber\\
% m&\geq \sigma^2\log\inp{n/k}\inp{1-\frac{h_2(\delta)+ \delta k\log\inp{n}}{k\log{n/k}}}\label{eq:final_bound_bcs1}
% \end{align}
On the other hand, $I(y_i;\bx|\vecA)\leq 1$. Thus,
\begin{align}
k\log&\inp{n/k}\inp{1-\frac{h_2(\delta)+ \delta k\log\inp{n}}{k\log{n/k}}}\nonumber\\
&\leq \sum_{i = 1}^{m}I(y_i;\bx|\vecA)\leq \sum_{i = 1}^{m}H(y_i|\vecA)\nonumber\\
&\leq m.\label{eq:final_bound_bcs2}
\end{align}
Combining \eqref{eq:final_bound_bcs1} and \eqref{eq:final_bound_bcs2}, we get the desired bound.
\end{proof}

\begin{proof}[Proof of Corollary~\ref{thm: lower_bd_log_reg}]
Consider a Gaussian sensing matrix $\vecA$. Suppose $\bx$ is distributed uniformly on the set of all $k$-sparse binary vectors. 
% Then,
% \begin{align}
% I(\vecA, \by; \bx)  &= H(\bx) - H(\bx|\vecA, \by)\nonumber\\
% &\stackrel{(a)}{\geq} \log{n \choose k } - h_2(\delta) - \delta\log\inp{{n \choose k } + 1}\nonumber\\
% &\geq k\log{n/k}- h_2(\delta)- \delta k\log\inp{n}\label{eq:log_lower_bd_bcw_1}
% \end{align} where $(a)$ follows from Fano's inequality [Theorem~2.10.1]\cite{thomas2006elements}.
% We also note that
% \begin{align*}
%  I(\vecA, \by; \bx) &= I(\vecA ; \bx)    +  I(\by; \bx|\vecA)\\
%  &\stackrel{(a)}{ = }0 + I(\by; \bx|\vecA).
% \end{align*}where $(a)$ holds because $\vecA$ and $\bx$ are independent. Let $y_{j\in[1:i-1]}$ denote $\inp{y_1, \ldots, y_{i-1}}$. 
% \begin{align}
% I&(\by; \bx|\vecA) = \sum_{i = 1}^{m}I(y_i; \bx|\vecA, y_{j\in[1:i-1]})\nonumber\\
% & = \sum_{i = 1}^{m}\Big(H(y_i|\vecA, y_{j\in[1:i-1]})\nonumber\\
% &\qquad- H(y_i| \bx,\vecA, y_{j\in[1:i-1]})\Big)\nonumber\\
% & \stackrel{(a)}{\leq}\sum_{i = 1}^{m}\inp{H(y_i|\vecA)- H(y_i| \bx,\vecA)}\nonumber\\
% & = \sum_{i = 1}^{m}I(y_i;\bx|\vecA)\label{eq:log_lower_bd_bcs}
% \end{align}where $(a)$ follows from $H(y_i|\vecA, y_{j\in[1:i-1]})\leq H(y_i|\vecA)$ and $H(y_i| \bx,\vecA, y_{j\in[1:i-1]}) = H(y_i| \bx,\vecA)$ as $y_i$ is conditionally independent of $y_{j\in[1:i-1]}$ conditioned on $\bx$ and $\vecA$.
% For any $i$, 
% \begin{align*}
% I(y_i;\bx|\vecA) &= H(y_i|\vecA)- H(y_i|\bx, \vecA)\\
% &\stackrel{(a)}{\leq} 1- H(y_i|\vecA_i^T\bx).
% \end{align*}where $(a)$ holds because $H(y_i|\vecA)\leq H(y_i) =1$ and $y_i$ is conditionally independent of $(\vecA_1\ldots, \vecA_{i-1}, A_{i+1}, \ldots, A_{m})$ and $\bx$ conditioned on $\vecA_i^T\bx$. Here $\vecA_i$, $i\in [1:m]$ denotes the $i^{\text{th}}$ row of the sensing matrix $\vecA$.

Suppose $t = \frac{1}{2}\tanh{\frac{\beta\vecA_i^T\bx}{2}} \inp{=\frac{(1-e^{-\beta \vecA_i^T\bx}}{2\inp{1+e^{-\beta \vecA_i^T\bx}}}}$. Then, 
\begin{align*}
\frac{1}{1+e^{-\beta \vecA_i^T\bx}} &= \frac{1}{2}+t\text{ and }\\
1-\frac{1}{1+e^{-\beta \vecA_i^T\bx}} &= \frac{1}{2}-t
\end{align*}

With this,
\begin{align*}
\bbE\insq{\inp{g\inp{\vecA_i^T\bx}^2}} &= \bbE\insq{4t^2}\\
& = \bbE\insq{\inp{\tanh{\frac{\beta\vecA_i^T\bx}{2}}}^2}
\end{align*}

% \begin{align*}
% H(y_i|\vecA_i^T\bx) &\stackrel{(a)}{=} \bbE\insq{h_2\inp{\frac{1}{1+e^{-\beta \vecA_i^T\bx}}}}\\
% &\stackrel{(b)}{\geq} \bbE\insq{4{\frac{1}{1+e^{-\beta \vecA_i^T\bx}}}\inp{1-\frac{1}{1+e^{-\beta \vecA_i^T\bx}}}}\\
% & = \bbE\insq{4\inp{\frac{1}{2} - t}\inp{\frac{1}{2} + t}}\\
% \end{align*}
% where in $(a)$, the expectation is over $\vecA_i^T\bx$. The inequality $(b)$ follows from [Theorem 1.2]\cite{topsoe2001bounds}. Thus,
% \begin{align*}
% \bbE&\insq{h_2\inp{\frac{1}{1+e^{-\beta \vecA_i^T\bx}}}}\\
% & \geq  1-4\bbE\insq{\inp{\frac{1}{2}\tanh{\frac{\beta\vecA_i^T\bx}{2}}}^2}\\
% & = 1 - \bbE\insq{\inp{\tanh{\frac{\beta\vecA_i^T\bx}{2}}}^2}.
% \end{align*}
Note that,
\begin{align*}
\bbE\insq{\inp{\tanh{\frac{\beta\vecA_i^T\bx}{2}}}^2} = 1-\bbE\insq{\inp{\text{sech}{\frac{\beta\vecA_i^T\bx}{2}}}^2}
\end{align*} and
\begin{align*}
\bbE\insq{\inp{\text{sech}{\frac{\beta\vecA_i^T\bx}{2}}}^2} &= \bbE\insq{\frac{1}{\inp{\text{cosh}{\frac{\beta\vecA_i^T\bx}{2}}}^2}}\\
&\stackrel{(a)}{\geq} \bbE\insq{e^{-\inp{\beta\vecA_i^T\bx/2}^2}}\\
&\stackrel{(b)}{=}\sqrt{\frac{1}{1+\beta^2 k/2}}\\
&\stackrel{(c)}{\geq}1-\frac{\beta^2k}{2}
\end{align*} where $(a)$ follows from the inequality $\text{cosh}(t)\leq e^{t^2/2}$ (see [Exercise 2.2.3]\cite{vershynin}),  $(b)$ follows from \eqref{eq:expectation_log} and $(c)$ holds because $1-\frac{x}{2}\leq \frac{1}{\sqrt{1+x}}$ for any $x\geq 0$.
Thus, 
\begin{align*}
\bbE\insq{\inp{g\inp{\vecA_i^T\bx}^2}} \leq \frac{\beta^2k}{2}.
\end{align*}
% \begin{align*}
% H(y_i|\vecA_i^T\bx)&\geq 1-\frac{\beta^2 k}{4}.
% \end{align*}
This implies that
% \begin{align*}
% I(y_i;\bx|\vecA)&\leq 1-\inp{1-\frac{\beta^2k}{4}} =\frac{\beta^2k}{4}
% \end{align*}
% and from \eqref{eq:log_lower_bd_bcw_1} and \eqref{eq:log_lower_bd_bcs},
\begin{align*}
% m\frac{2k}{\pi\sigma^2} &\geq k\log{n/k}- h_2(\delta)- \delta k\log\inp{n}\\
m\frac{\beta^2k}{2} &\geq  k\log\inp{n/k}\inp{1-\frac{h_2(\delta)+ \delta k\log\inp{n}}{k\log{n/k}}}
\end{align*}
Thus,
\begin{align}
m&\geq \frac{2}{\beta^2}\log\inp{n/k}\inp{1-\frac{h_2(\delta)+ \delta k\log\inp{n}}{k\log{n/k}}}\nonumber\\
&\geq \frac{1}{\beta^2}\log\inp{n/k}\inp{1-\frac{h_2(\delta)+ \delta k\log\inp{n}}{k\log{n/k}}}\label{eq:log_final_bound_bcs1}
\end{align}
We also  know that for any $i$, $I(y_i;\bx|\vecA)\leq H(y_i|\vecA)\leq 1$. Thus, we also obtain that
\begin{align}
m\geq k\log\inp{n/k}\inp{1-\frac{h_2(\delta)+ \delta k\log\inp{n}}{k\log{n/k}}}\label{eq:log_final_bound_bcs2}
\end{align}
Combining \eqref{eq:log_final_bound_bcs1} and \eqref{eq:log_final_bound_bcs2}, we get the desired bound.
\end{proof}



% \subsection{Proof of Theorem~\ref{thm: spl_lower_bd_1}}\label{proof:thm: spl_lower_bd_1}
\begin{proof}[Proof of Corollary~\ref{thm: spl_lower_bd_1}]
Suppose $\bx$ is generated uniformly at random from the set of all $k$-sparse vectors and $\vecA$ is any sensing matrix which satisfies the power constraint given by \eqref{eq:power_constraint}. Then,
% \begin{align}
% I(\vecA, \by;\bx) &= H(\bx)-H(\bx|\by, \vecA)\nonumber\\
%  &\stackrel{(a)}{\geq} \log{n \choose k } - h_2(\delta) - \delta\log\inp{{n \choose k } + 1}\nonumber\\
%  &\geq k\log\inp{\frac{n}{k}}- h_2(\delta) - \delta k\log{n}.\label{eq:new1}
% \end{align} where $(a)$ follows from the Fano's inequality.
% We also see that
% \begin{align*}
% I(\vecA,\by;\bx) &= I(\vecA, \bx)+I(\by;\bx|\vecA)\\
% &\stackrel{(a)}{=}0 +I(\by;\bx|\vecA)
% \end{align*}where $(a)$ follows from noting that $\vecA$ and $\bx$ are independent. Suppose $y_{j\in [1:i-1]}$ denote $(y_1, \ldots, y_{i-1})$. Then,
\begin{align*}
I&(y_i;\bx| \vecA) = h(y_i|\vecA)-h(y_i|\bx,\vecA)\\
&\leq \inp{h(y_i)-h(\mathbf{A}_i^T\bx + z_i|\bx,  \vecA)}\\
& = h(y_i) - h(z_i)\\
&\leq \inp{h(w_i)-h(z_i)}
\end{align*} where in the last inequality, $w_i\sim\cN\inp{0, \sigma^2_w}$ where $\mathsf{Var}(y_i)\leq \sigma^2_w$. We will now compute an upper bound on $\mathsf{Var}(y_i)$.
\begin{align*}
\mathsf{Var}(y_i) &\leq \bbE\insq{\inp{\vecA_i^T\bx+z_i}^2} = \bbE\insq{\inp{\vecA_i^T\bx}^2} + \sigma^2\\
&\leq k + \sigma^2
\end{align*} 
Thus, we have
\begin{align}
\inp{h(w_i)-h(z_i)}=\frac{1}{2}\log\inp{\frac{{k}}{\sigma^2}+1}.\label{eq:new2}
\end{align}
With this, we conclude that
% \begin{align*}
% \frac{m}{2}\log\inp{\frac{{k}}{\sigma^2}+1}
% \geq k\log\inp{\frac{n}{k}}- h_2(\delta) - \delta k\log{n}
% \end{align*} and 
\begin{align*}
m&\geq \frac{2k\log\inp{\frac{n}{k}}- h_2(\delta) - \delta k\log{n}}{\frac{1}{2}\log\inp{\frac{{k}}{\sigma^2}+1}}.
\end{align*} 



\end{proof}


% \input{AppendixSPL}
% \clearpage
% \setcounter{page}{1}
% \maketitlesupplementary

\appendix

\section*{Appendix Overview}
This technical appendix consists of the following sections:
\begin{itemize}
% \item
% In Section~\ref{method}, we provide more method design.
\item
In Section~\ref{ablation}, we provide more ablation experiments to verify the effectiveness of FlowScene.
\item
In Section~\ref{quantitative}, we provide quantitative results from more experiments.
\item
In Section~\ref{visualizations}, we present more visual qualitative results of the SemanticKITTI val set.
\item
In Section~\ref{sec:limit},  we analyze the shortcomings of our method and directions for future work.
\end{itemize} 

\section{More Ablation Studies}
\label{ablation}
\subsection{Ablation Study for Optical Flow Networks}
Table~\ref{tab:flow} presents the performance of different optical flow networks. We compare several state-of-the-art methods, including PWC-Net~\cite{sun2018pwc}, RAFT~\cite{teed2020raft}, and FlowFormer~\cite{huang2022flowformer}, along with our setting, GMFlow~\cite{xu2022gmflow}, which is highlighted in the last row.
Our setting achieves the highest IoU of 45.01\% and mIoU of 18.13\%, outperforming all other methods in both metrics. These results suggest that GMFlow effectively captures motion cues and integrates them into the semantic scene completion task, providing superior performance over the other optical flow networks tested, with significantly fewer parameters.

\subsection{Ablation Study for Backbone Networks}
Table~\ref{tab:backbone} examines the impact of different backbone networks on the performance of FlowScene. The study compares EfficientNetB7~\cite{tan2019efficientnet}, ResNet50~\cite{he2016resnet}, and RepVit-M2.3~\cite{wang2023repvit}(our setting). Our method, using RepVit-M2.3, achieves the highest IoU of 45.01\% and mIoU of 18.13\%, surpassing both EfficientNetB7 (44.31\% IoU, 17.63\% mIoU) and ResNet50 (44.12\% IoU, 16.98\% mIoU). RepVit-M2.3, though achieving the best performance, maintains a relatively low parameter count of 22.4M. In comparison, EfficientNetB7 has a much higher parameter count of 63.8M, while ResNet50 is more parameter-efficient at 25.6M. RepVit-M2.3 offers a good balance between performance and parameter count, making it an ideal choice for our backbone network.
%--------------------------------
\begin{table}[t]
  \centering\small
\setlength{\tabcolsep}{4pt}{
  \begin{tabular}{l|ccc}
    \toprule
    {\textbf{Method}}& {\textbf{IoU}(\%)}&{\textbf{mIoU}(\%)}&\textbf{Params(M)} \\
    \midrule
    PWC-Net+~\cite{sun2018pwc}& 43.31&17.13&8.8\\
    RAFT~\cite{teed2020raft}&44.12&17.56&5.3 \\
    FlowFormer~\cite{huang2022flowformer}&44.33&17.74&18.2 \\
    \rowcolor{gray!20}{GMFlow}~\cite{xu2022gmflow}&{\textbf{45.01}}&\textbf{18.13}&\textbf{4.7} \\
    \bottomrule
  \end{tabular}
  }
      \caption{Ablation study for optical flow networks.}
  \label{tab:flow}
  % \vspace{-5mm}
\end{table}
%-------------------------------------------------------------
%--------------------------------
\begin{table}[t]
  \centering\small
\setlength{\tabcolsep}{4pt}{
  \begin{tabular}{l|ccc}
    \toprule
    {\textbf{Method}}& {\textbf{IoU}(\%)}&{\textbf{mIoU}(\%)}&\textbf{Params(M)} \\
    \midrule
    EfficientNetB7~\cite{tan2019efficientnet}& 44.31&17.63&63.8\\
    ResNet50~\cite{he2016resnet}&44.12&16.98&25.6 \\
    % RepVit-M1.5~\cite{wang2023repvit}&43.25&17.21&14.0 \\
    \rowcolor{gray!20}{RepVit-M2.3}~\cite{wang2023repvit}&{\textbf{45.01}}&\textbf{18.13}&\textbf{22.4} \\
    \bottomrule
  \end{tabular}
  }
      \caption{Ablation study for backbone networks.}
  \label{tab:backbone}
  % \vspace{-5mm}
\end{table}
%-------------------------------------------------------------
\section{More Quantitative Results}
\label{quantitative}

To provide a more thorough comparison, we provide additional quantitative results of semantic scene completion on the SemanticKITTI validation set in Table~\ref{tab:val}. The results further demonstrate the effectiveness of our approach in enhancing 3D scene perception performance.
Compared with the previous state-of-the-art methods, FlowScene is superior to other HTCL~\cite{li2024htcl} in semantic scene understanding, with a 1.00\% increase in mIoU. In addition, compared with Symphonize~\cite{jiang2024symphonize}, huge improvements are made in both occupancy and semantics. IoU and mIoU enhancement are of great significance for practical applications. It proves that we are not simply reducing a certain metric to achieve semantic scene completion.
%-------------------------------------------------------------------------

\begin{table*}[ht]
  \setlength{\tabcolsep}{2pt}
  \renewcommand\arraystretch{1.2}
  
  \resizebox{\textwidth}{!}{
  \begin{tabular}{l|l|c|r|rrrrrrrrrrrrrrrrrrr|r}
    \toprule
    \textbf{Methods} &\textbf{Published}&\textbf{Inputs}&\textbf{IoU}   
    & \rotatebox{90}{\vcenteredbox{\colorbox[RGB]{255,0,255}{\textcolor[RGB]{255,0,255}{\rule{1px}{1px}}}} \textbf{road} (15.30$\%$)} 
    & \rotatebox{90}{\vcenteredbox{\colorbox[RGB]{75,0,75}{\textcolor[RGB]{75,0,75}{\rule{1px}{1px}}}} \textbf{sidewalk} (11.13$\%$)} 
    & \rotatebox{90}{\vcenteredbox{\colorbox[RGB]{255,150,255}{\textcolor[RGB]{255,150,255}{\rule{1px}{1px}}}} \textbf{parking} (1.12$\%$)} 
    & \rotatebox{90}{\vcenteredbox{\colorbox[RGB]{175,0,75}{\textcolor[RGB]{175,0,75}{\rule{1px}{1px}}}} \textbf{other-grnd} (0.56$\%$)} 
    &  \rotatebox{90}{\vcenteredbox{\colorbox[RGB]{255,200,0}{\textcolor[RGB]{255,200,0}{\rule{1px}{1px}}}} \textbf{building} (14.10$\%$)} 
    &  \rotatebox{90}{\vcenteredbox{\colorbox[RGB]{100,150,245}{\textcolor[RGB]{100,150,245}{\rule{1px}{1px}}}} \textbf{car} (3.92$\%$)} 
    & \rotatebox{90}{\vcenteredbox{\colorbox[RGB]{80,30,180}{\textcolor[RGB]{80,30,180}{\rule{1px}{1px}}}} \textbf{truck} (0.16$\%$)} 
    & \rotatebox{90}{\vcenteredbox{\colorbox[RGB]{100,230,245}{\textcolor[RGB]{100,230,245}{\rule{1px}{1px}}}} \textbf{bicycle} (0.03$\%$)}  
    & \rotatebox{90}{\vcenteredbox{\colorbox[RGB]{30,60,150}{\textcolor[RGB]{30,60,150}{\rule{1px}{1px}}}} \textbf{motocycle} (0.03$\%$)} 
    & \rotatebox{90}{\vcenteredbox{\colorbox[RGB]{0,0,255}{\textcolor[RGB]{0,0,255}{\rule{1px}{1px}}}} \textbf{other-vehicle} (0.20$\%$)} 
    & \rotatebox{90}{\vcenteredbox{\colorbox[RGB]{0,175,0}{\textcolor[RGB]{0,175,0}{\rule{1px}{1px}}}} \textbf{vegetation} (39.3$\%$)} 
    & \rotatebox{90}{\vcenteredbox{\colorbox[RGB]{135,60,0}{\textcolor[RGB]{135,60,0}{\rule{1px}{1px}}}}  \textbf{trunk} (0.51$\%$)} 
    & \rotatebox{90}{\vcenteredbox{\colorbox[RGB]{150,240,80}{\textcolor[RGB]{150,240,80}{\rule{1px}{1px}}}} \textbf{terrain} (9.17$\%$)} 
    & \rotatebox{90}{\vcenteredbox{\colorbox[RGB]{255,30,30}{\textcolor[RGB]{255,30,30}{\rule{1px}{1px}}}} \textbf{person} (0.07$\%$)} 
    & \rotatebox{90}{\vcenteredbox{\colorbox[RGB]{255,40,200}{\textcolor[RGB]{255,40,200}{\rule{1px}{1px}}}} \textbf{bicylist} (0.07$\%$)} 
    & \rotatebox{90}{\vcenteredbox{\colorbox[RGB]{150,30,90}{\textcolor[RGB]{150,30,90}{\rule{1px}{1px}}}}  \textbf{motorcyclist} (0.05$\%$)} 
    &  \rotatebox{90}{\vcenteredbox{\colorbox[RGB]{255,120,50}{\textcolor[RGB]{255,120,50}{\rule{1px}{1px}}}} \textbf{fence} (3.90$\%$)} 
    & \rotatebox{90}{\vcenteredbox{\colorbox[RGB]{255,240,150}{\textcolor[RGB]{255,240,150}{\rule{1px}{1px}}}} \textbf{pole} (0.29$\%$)} 
    & \rotatebox{90}{\vcenteredbox{\colorbox[RGB]{255,0,0}{\textcolor[RGB]{255,0,0}{\rule{1px}{1px}}}} \textbf{traf.-sign} (0.08$\%$)}& \textbf{mIoU}  \\
    
    \midrule
    % LMSCNet$^\dagger$\cite{roldao2020lmscnet}&3DV'2020  & 28.61&40.68&18.22&4.38&0.00&10.31&18.33&0.00&0.00&0.00&0.00&13.66&0.02&20.54&0.00&0.00&0.00&1.21&0.00&0.00&6.70   \\
    % AICNet$^\dagger$\cite{li2020aicnet}&CVPR'2020   &29.59&43.55&20.55&11.97&0.07&12.94&14.71&4.53&0.00&0.00&0.00&15.37&2.90&28.71&0.00&0.00&0.00&2.52&0.06&0.00&8.31 \\
    % JS3C-Net$^\dagger$ \cite{yan2021sparse}&AAAI'2021   &38.98&50.49&23.74&11.94&0.07&15.03&24.65&4.41&0.00&0.00&6.15&18.11&4.33&26.86&0.67&0.27&0.00&3.94&3.77&1.45&10.31\\
    MonoScene \cite{cao2022monoscene}&CVPR'2022 &S& 36.86&56.52&26.72&14.27&0.46&14.09&23.26&6.98&0.61&0.45&1.48&17.89&2.81&29.64&1.86&1.20&0.00&5.84&4.14&2.25&11.08 \\
    
    TPVFormer \cite{huang2023tri}& CVPR'2023&S& 35.61&56.50& 25.87& 20.60& 0.85& 13.88& 23.81& 8.08& 0.36& 0.05 &4.35 &16.92& 2.26& 30.38& 0.51& 0.89& 0.00& 5.94& 3.14& 1.52&11.36 \\
    
    OccFormer\cite{zhang2023occformer}& ICCV'2023 &S& 36.50  & 58.85& 26.88& 19.61& 0.31& 14.40& 25.09& 25.53& 0.81& 1.19& 8.52& 19.63& 3.93& 32.62& 2.78& 2.82& 0.00& 5.61& 4.26& 2.86 & 13.46\\
    % VoxFormer-S\cite{li2023voxformer}& CVPR'2023&44.02&54.76&26.35&15.50&0.70&17.65&25.79&5.63&0.59&0.51&3.77&24.39&5.08&29.96&1.78&3.32&0.00&7.64&7.11&4.18&12.35\\
    Symphonize~\cite{jiang2024symphonize}&CVPR'2024&S&41.92&56.37&27.58&15.28&0.95&21.64&28.68&20.44&2.54&2.82&13.89&25.72&6.60&30.87&3.52&2.24&0.00&8.40&9.57&5.76&14.89\\
        
    VoxFormer-T\cite{li2023voxformer}& CVPR'2023&T& 44.15 &53.57&26.52&19.69&0.42&19.54&26.54&7.26&1.28&0.56&7.81&26.10&6.10&33.06&1.93&1.97&0.00&7.31&9.15&4.94 & 13.35\\
    H2GFormer~\cite{wang2024h2gformer} & AAAI'2024&T&44.69&57.00&29.37&21.74&0.34&20.51&28.21&6.80&0.95&0.91&9.32&27.44&7.80&36.26&1.15&0.10&0.00&7.98&9.88&5.81&14.29 \\
    HASSC~\cite{wang2024HASSC}&CVPR'2024&T&44.58&55.30&29.60&25.90&11.30&23.10&23.00&2.90&1.90&1.50&4.90&24.80&9.80&26.50&1.40&3.00&0.00&14.30&7.00&7.10&14.74\\
    HTCL~\cite{li2024htcl}&ECCV'2024&T&45.51&63.70&32.48&23.27&0.14&24.13&34.30&20.72&3.99&2.80&11.99&26.96&8.79&37.73&2.56&2.70&0.00&11.22&11.49&6.95&17.13\\
    \midrule
    \rowcolor{gray!20}\textbf{Ours}&  &T& {\underline{45.01}} &{\textbf{63.72}}&{\underline{32.10}}& 22.20&\underline{1.31}& {\textbf{25.63}}& {\underline{33.33}}&\textbf{33.47}&2.36& {\textbf{5.09}}&\textbf{16.99}& {\underline{26.35}}& {8.68}& {\underline{36.73}}&\textbf{3.79}& {{1.92}}& {\textbf{0.00}}& {\underline{12.05}}& {\textbf{11.65}}& {\underline{7.05}}&  {\textbf{18.13}}\\
    \bottomrule
  \end{tabular}}
  \caption{Quantitative results on the SemanticKITTI validation set. The best results are in {\textbf{Bold}}.}
  \label{tab:val}
\end{table*}
%-------------------------------------------------------------------------
\section{More Visualizations Results}
\label{visualizations}
We show visualization examples on the Semantickitti validation set, as shown in Figure~\ref{fig:sup1}. From left to right are the input image, the corresponding optical flow and occlusion mask, the front view SSC, and the top view SSC. Due to the motion information brought by the optical flow, the location information of the scene objects is more accurate and the layout is more reasonable.
We report the performance of more visual comparison results on the SemanticKITTI validation set in Figure~\ref{fig:sup2}. We compare with VoxFormer~\cite{li2023voxformer} and BRGScene~\cite{li2023stereoscene}. In general, our method performs more fine-grained segmentation of the scene and maintains clear segmentation boundaries. For example, in the segmentation completion result of cars, we predict clear separation of each car. In contrast, other methods show continuous semantic errors for occluded cars. In addition, our flow can effectively deal with the problem of mutual occlusion between different objects. Finally, we provide a video in the appendix to show the performance more intuitively.
% -------------------------------------------------
\begin{figure*}[t]
\centering
  \includegraphics[width=\textwidth]{supp2.pdf}
  \caption{Qualitative results on the SemanticKITTI validation set.}
  \label{fig:sup1}
\end{figure*}
% -------------------------------------------------

\section{Discussions}
\label{sec:limit}
Flowscene shows strong performance on the benchmark with an improved number of parameters. This is beneficial for deploying real-world autonomous driving applications. But the inference time of the model needs to be improved. Semantic scene completion in multi-camera settings is also worth attention, which is our future work. Meanwhile, the legal challenges of autonomous driving as well as privacy and data security risks are still topics of debate. Finally, the robustness of semantic scene completion is also an issue worth exploring.

% \clearpage
% -------------------------------------------------
\begin{figure*}[t]
\centering
  \includegraphics[width=\textwidth]{supp1.pdf}
  \caption{Qualitative results on the SemanticKITTI validation set.}
  \label{fig:sup2}
\end{figure*}
% -------------------------------------------------
% \clearpage
\section{Comparison with \cite{vershyninPlan}}\label{sec:comparison_PV}
Algorithm~\ref{alg:1} is similar to the two step estimation procedure outlined in \cite{vershyninPlan} which was given to estimate the unknown signal within a two norm guarantee. 
Computing the vector $\mathbf{l} = \inp{l_1, \ldots, l_n}$ is the same as the first step of the procedure in [Section~1.2]\cite{vershyninPlan} where a linear estimator is computed. The second step of our algorithm (sorting and keeping the top-$k$ indices) can be thought of as a projection on a feasible set [Section~1.3]\cite{vershyninPlan}. However, this requires the estimation error to be small enough for the exact recovery of a binary vector.

The setup in \cite{vershyninPlan} is for the recovery of an unknown signal with small two-norm error, whereas our problem of exact recovery of a sparse binary vector is more suited for recovery under  infinity norm. This results in weak bounds ($m\approx O(k^2)$) when we specialize various results in \cite{vershyninPlan} to our case. We first note that we require $\bbE\norm{\frac{\hat{x}}{\norm{\hat{x}}}-\bar{x}}< \sqrt{\frac{{2}}{{k}}}$ for exact recovery. Otherwise, there exist two binary $k$-sparse vectors which have hamming distance at least two. 

We first consider the 1-bit compressed sensing result in Section 3.5 (page 13). Setting the LHS to $\sqrt{\frac{{2}}{{k}}}$, we get
\begin{align*}
\sqrt{\frac{{2}}{{k}}}\leq C\sqrt{\frac{k\log\inp{2n/k}}{m}}.
\end{align*}
This implies that $m\approx C_1 k^2\log\inp{2n/k}$ for some constant $C_1$.

Next, we consider [Theorem 9.1]\cite{vershyninPlan}. Note that for 1-bit compressed sensing $\eta^2 = 1$ and 
\begin{align*}
\mu &= \bbE\insq{s_1\ipr{a_1}{\bar{x}}}\\
& =\bbE\insq{s_1\ipr{a_1}{{x}}}\\
& \stackrel{(a)}{=} \frac{1}{\sqrt{k}}\sqrt{\frac{2}{\pi}}\times\frac{k}{\sqrt{\inp{k+\sigma^2}}}\\
&= \sqrt{\frac{2}{\pi}}\times\frac{\sqrt{k}}{\sqrt{\inp{k+\sigma^2}}}.
\end{align*} where $(a)$ follows from \eqref{eq:expt4}. 
Then, 
\begin{align*}
\norm{x-\mu\bar{x}} &= \norm{x-\sqrt{\frac{2}{\pi}}\times\frac{\sqrt{k}}{\sqrt{\inp{k+\sigma^2}}}\frac{x}{\sqrt{k}}}\\
& = \norm{x-\sqrt{\frac{2}{\pi}}\times\frac{x}{\sqrt{\inp{k+\sigma^2}}}}
\end{align*} We require $\norm{x-\mu\bar{x}}<\frac{2}{\sqrt{\pi\inp{k+\sigma^2}}}$ in order to exactly recover the unknown signal $x$. 



We assume that $K$ is also a closed cone in $\bbR^n$. Then, by [Section~2.4]\cite{vershyninPlan}, $w_t(K) = tw_1(K) \leq t C\sqrt{k\log\inp{2n/k}}$ ([Section~2.4]\cite{vershyninPlan}). We choose $s = w_1(K)$. Substituting the bound for LHS and taking the limit $t\rightarrow 0$, we get
\begin{align*}
\frac{2}{\sqrt{\pi\inp{k+\sigma^2}}}\leq \frac{8 C \sqrt{k\log\inp{2n/k}}}{\sqrt{m}}.
\end{align*} Thus, $m\approx 4C(k+ \sigma^2)k\log\inp{2n/k}$. 








% \documentclass[../main.tex]{subfiles}
\graphicspath{{../images/}}
\makeatletter
\def\input@path{{../images/}}
\makeatother
\begin{document}
\section{Introduction}
\begin{figure}
\centering
\begin{tikzpicture}
\node[inner sep=0pt] (ws) at (0, 0) {
\includegraphics[height=.4\textwidth, trim={10cm 0 10cm 0},clip]{world_space.png}};
\node[inner sep=0pt] (cs) at (6,0) {\includegraphics[height=.4\textwidth, trim={10cm 1cm 10cm 4cm},clip]{conf_space.png}};
\end{tikzpicture}
\vspace{-5pt}
\label{fig:pbrm_intro}
\caption{\textbf{Left}: Shows world space obstacles as grey spheres. Robots start and goal configuration is colored red and green, respectively. Configurations along the computed path are colored transparent blue. \textbf{Right:} Mapped world space scenario to configuration space. Obstacle region is the grey mesh. Red spheres are collision-free regions computed by the neural SCDF. The optimized shortest path in the convex corridor is the blue curve.}
\vspace{-25pt}
\end{figure}
Motion planning is the problem of finding a collision-free trajectory that connects a given start and goal configuration. The planning takes place in the configuration space of the robot. For single body robots, like mobile robots or drones, the configuration space and the world space are usually the same. This simplifies the planning, since explicit obstacle representations are available which enables geometrical tools like separating hyperplanes, smallest distance to obstacles etc., to be used when designing motion planning algorithms. For multi-body robots like manipulators, the situation is completely different. The world space obstacles are usually mapped to non-convex regions, and to make the problem even harder, the mapping is usually not known. Forming explicit representations of the obstacle region in the configuration space is usually too expensive or intractable. Despite all of this, sampling based planners are used with great success, which mainly is due to their use of implicit representations of the obstacle region. The basic idea is to construct a graph in the configuration space that covers and connects the collision-free region. From this graph, a path can be extracted that connects a given start and goal configuration. The approach is computationally expensive, since the graph is constructed with the smallest geometrical building block available, points, which represents a collision-check. Furthermore, the extracted paths from the graph are non-smooth and jagged due to the stochastic nature of the approach. This adds an additional post-processing step to the process, where the paths are shortcutted and smoothened, before the path can be used for tracking. Clearly a lot of time is invested to form this graph and produce smooth paths. Thus, if the obstacles start to move, then all of this work is done in no use, since all points that make up this graph need to be re-verified, which is simply too time consuming to be done in real time.
\\\\
In this work, we want to address the existing drawbacks of the sampling based planners. Our main contribution is an improved motion planner where each vertex in the graph covers a collision-free region in the form of a sphere instead of a point and where the edges are formed with neighboring intersecting spheres. This representation has the advantage of instead of returning piecewise linear paths, returning a sequence of overlapping spheres, i.e. a convex corridor, that connects a given start and goal configuration, illustrated in Figure \ref{fig:pbrm_intro}. This convex corridor allows us to use convex optimization to produce smooth trajectories, instead of computationally expensive post-processing methods. The representation further allows us to estimate the coverage of the collision-free space, which gives us awareness and feedback in the offline roadmap construction phase. Finally, our representation is simple to adapt to moving obstacles, simply requery for the new radii and recheck for intersections. 
\\\\
The spherical collision-free regions are formed using a signed distance function (SDF), which is a function that returns the smallest distance from an arbitrary point to the boundary of an obstacle. As the name implies, the distance is signed, thus if the point is inside the obstacle it is negative otherwise positive. If the distance is positive, a sphere with radius equal to the distance is guaranteed to cover a collision-free region. Using an SDF in motion planning is not new, but what is novel about our approach is that we express the distance in the configuration space instead of the world space and by doing so allows us to form these convex collision-free regions. We refer to the resulting SDF as a signed configuration distance function (SCDF). Computing an SCDF analytically is non-trivial, our approach is therefore to parameterize the SCDF with a deep neural network and learn the mapping by supervised learning. Our resulting neural SCDF can compute distances for different parameter values of obstacle shapes and we also show how multiple distances can be combined, thus making our approach flexible.
\section{Related work}
Motion planning algorithms can roughly be divided into three families, grid-based, sampling based and optimization based methods. Grid-based methods (GBM) discretize the planning space from which a graph is then compiled. A standard search method is A$^\star$ \citep{a_star}, which is classified as an \textit{informed} search method, since it employs a heuristic function to speed up the search. A$^\star$ guarantees to return an optimal path at the level of discretization used. GBMs usually discretize the planning space by a regular lattice and this limits the GBMs to problems with low dimensionality due to the curse of dimensionality. Thus, GBMs are usually limited to single-body robots where the degrees of freedom (DOF) are low. To overcome the inherent scaling problem with the GBMs, stochastic methods are usually used for multi-body robots. These methods are termed as sampling-based methods (SBM) and core members within this family are the rapidly-exploring random trees (RRT) \citep{rrt} and the probabilistic roadmap (PRM) \citep{prm}. RRT grows a tree from the start configuration and explores the collision-free region in a rapid way until it is able to connect to the goal region. RRT is usually improved by bi-directional planning \citep{rrt_connect}, i.e. an additional tree is grown from the goal configuration and the trees are tested for connection after any tree has been expanded. RRT is a single-query method, thus it searches for a path from scratch each time it is queried. Contrary to this, PRM is a multi-query method, which solves for multiple queries without starting from scratch. PRM does this by creating a roadmap (graph) that covers the collision-free space as an offline step. The graph is then used to solve for multiple queries. PRMs are used in cases where the environment does not change since the extra offline step is too computationally costly and needs to be re-done if the environment is changed. In our work, we address this inherent issue by using a different roadmap representation. Our vertices in the graph cover a collision-free region in the form of spheres and we form the edges by checking for intersecting spheres. If something in the environment changes, we recompute the spheres radii and recheck the intersections, without relying on collision detection. We use a trained neural network to compute the sphere radius, therefore querying for the radius can be done fast, hence our representation enables the PRM for dynamic environments.
\\\\
In the recent decades, optimization based methods (OBM) \citep{chomp, schulman, itomp, stomp} have been introduced as an alternative to SBM for multi-body robots. Like the SBM, the OBMs scale well to higher dimensional problems and produce smoother motion. It is common to use a SDF in the optimization since it is a smooth function, thus enabling gradient-based methods. However, the standard way of expressing the SDF is in world space. The distance therefore needs to be mapped to the configuration space by the forward kinematics. This mapping makes the optimization problem a non-linear program (NLP), which is computationally expensive to solve. Recently, a different approach has been proposed. In \cite{mp_gcs} motion planning is formulated as a convex optimization problem by using the graph of convex sets framework \citep{gcs}. The underlying idea is to decompose the collision-free space into intersecting convex sets from which a convex optimization problem is formulated. In cases where an explicit representation of the obstacles in the configuration space exists, like for single-body robots, creating collision-free convex regions can be done fast \citep{iris}. For multi-body robots, this is non-trivial. Existing work does this successfully \citep{iris_nlp, iris_c} by an optimization based approach, but the methods are still too time consuming to be used in the presence of moving obstacles. Our approach is instead to use deep learning to learn an SDF expressed in the configuration space. With this, we can query for shortest distances to the collision boundary, which allows us to expand spherical regions which are collision-free. Our approach is fast and therefore enables our suggested roadmap planner to be used in dynamic environments.
\\\\
Recent research has focused on learning collision detection \citep{fk_kernel_distance, diffco, graphdistnet} by predicting the signed distance between the robot links and the surrounding obstacles in the world space. The learned SDF is used in trajectory optimization but since the distance is expressed in the world space, the problem becomes an NLP and therefore takes a long time to solve. We take a novel approach and suggest to instead express the signed distance in the configuration space. This allows us to improve the PRM at the same time as it enables convex optimization for trajectory optimization, which runs faster and is more reliable than NLP solvers. In \cite{cspf} a learned signed distance function in the configuration space is proposed similar to our approach. However, their approach is restricted to point cloud representations, while we propose to represent the obstacles as parameterized geometric shapes, e.g. spheres. Furthermore, we also show how to use our learned SCDF to improve an existing roadmap planner.
\section{Problem formulation}
A robot is located in the world space, $\W \subset \R^3 $. The unique location of the robot is given by its configuration $\q \in \C$, where $\C$ is the configuration space. The set of points covered by the robots bodies at a certain configuration is expressed as $\B(\q) \subset \W$. The robot is surrounded by $\NrObst$ obstacles $\O = \bigcup_{i=1}^{\NrObst} \O_i$, where  $\O_i \subset \W$. The representation of the obstacle in the configuration space is the set $\C\O_i = \{\q \in \C \: |\: \B(\q) \cap \O_i \neq \emptyset \}$. The obstacle space is formed as $\Co = \bigcup_{i=1}^{\NrObst} \C \O_i$. The complement is referred to as the free space, $\Cf = \C \setminus \Co$. The path planning problem is a tuple, ($\Cf$, $\qStart$, $\qGoal$), where we want to connect a query pair, consisting of a start, $\qStart$, and goal configuration, $\qGoal$, with a geometric path, $\q(s): [0, 1] \mapsto \Cf$, such that $\q(0)=\qStart$ and $\q(1)=\qGoal$, or report correctly when such a path does not exist.
\end{document}

% %!TeX root=paper.tex
\section{Main results}


\subsection{Algorithm}\label{sec:alg}
We analyze the simple linear estimation based algorithm from \cite{vershyninPlan} for generalized linear measurements, specializing it for binary vectors.
The algorithm (Algorithm~\ref{alg:1}) takes the sensing matrix $\vecA$ and the output vector $\by$ as the inputs. 
% The output $\by$ is set to $\by$ (see \eqref{eq:spl}) for \spl\ and to $\sign{\by}$  for \bcs. 
For each column $\vecA_i,\, i\in [1:n]$ of the sensing matrix, the algorithm computes $l_i = \ipr{\by}{\vecA_i} = \sum_{j = 1}^{m}y_jA_{j,i}$ where $A_{j,i}$ is the entry at $j^{\text{th}}$ row and $i^{\text{th}}$ column.

The vector $\mathbf{l} = \inp{l_1, \ldots, l_n}$ is then sorted in decreasing order. The output of the algorithm is a set containing the indices of the top-$k$ elements of the sorted vector. That is, if the sorted vector is $\inp{l_{\alpha_1}, l_{\alpha_2}, \ldots, l_{\alpha_n}}$ where $l_{\alpha_i}\geq l_{\alpha_j}$ for $i\leq j$, then the output of the algorithm is  $\cS = \inb{\alpha_1, \ldots, \alpha_k}$.



\begin{algorithm}[tbh!]
   \caption{Top-$k$ correlated indices}
   \label{alg:1}
\begin{algorithmic}
   \STATE {\bfseries Input:} Sensing matrix $\vecA\in \bbR^{m\times n}$ and output $\mathbf{y}\in \bbR^{m}$ 
   \STATE {\bfseries Output:} a $k$-sized subset of $[1:n]$
   \STATE $\mathbf{l} \gets (0, \ldots, 0)$,\, $\mathbf{l}\in \reals^n$
        % \STATE $\mathbf{l} \gets (0, \ldots, 0)$,\, $\mathbf{l}\in \reals^n$
    \FOR{each $i\in [1:n]$}
      \STATE $l_i \gets \sum_{j = 1}^{m}y_jA_{j,i}$ 
    \ENDFOR
    \STATE Sort $\mathbf{l}$ in decreasing order and let $\cS$ be the top $k$ indices.\\
    \STATE {\bfseries Return:} $\mathcal{\cS}$ 
   % \REPEAT
   % \STATE Initialize $noChange = true$.
   % \FOR{$i=1$ {\bfseries to} $m-1$}
   % \IF{$x_i > x_{i+1}$}
   % \STATE Swap $x_i$ and $x_{i+1}$
   % \STATE $noChange = false$
   % \ENDIF
   % \ENDFOR
   % \UNTIL{$noChange$ is $true$}
\end{algorithmic}
\end{algorithm}


% \begin{algorithm}
%   \caption{Top-$k$ correlated indices}\label{alg:1}
%    \hspace*{\algorithmicindent} \textbf{Input:} Sensing matrix $\vecA\in \bbR^{m\times n}$ and output $\mathbf{w}\in \bbR^{m}$ \\
% \hspace*{\algorithmicindent} \textbf{Output:} a $k$-sized subset of $[1:n]$ \\
% \vspace{-0.4cm}
% \begin{algorithmic}[1]
%     \State $\mathbf{l} \gets (0, \ldots, 0)$,\, $\mathbf{l}\in \reals^n$
%     \For{each $i\in [1:n]$}
%       \State $l_i \gets \sum_{j = 1}^{m}A_{j,i}y_j$ 
%     \EndFor
%     \State Sort $\mathbf{l}$ in decreasing order and let $\cS$ be the top $k$ indices.\\
%     \Return $\mathcal{\cS}$ 
% \end{algorithmic}
% \end{algorithm}

The convergence and sample complexity guarantees for the algorithm are shown for the case when each entry of $\vecA$ is chosen iid $\cN(0,1)$. Note that such a matrix satisfies the power constraint in \eqref{eq:power_constraint}. As we argued in Section~\ref{sec:intro}, for the unknown signal $\bx$,  the output $\by = \vecA{\bx}+\bz$ is correlated with each column $\vecA_i$ for $i\in \cS_{\bx}$ and uncorrelated with $\vecA_j$ for $j\notin \cS_{\bx}$. In particular, for large number of samples, when $i\in \cS_{\bx}$, the inner product $\ipr{\by}{\vecA_i}$ is close to $\bbE\insq{\ipr{\by}{\vecA_i}} = m$ (for linear regression) with high probability. On the other hand, $\ipr{\by}{\vecA_j}$ is close to $0$ for $j\notin \cS_{\bx}$.  Thus, $l_i$ for $i\in \cS_{\bx}$ will dominate over $l_j$ for $j\notin \cS_{\bx}$. This line of argument also works when the output is binary, though in this case $\bbE\insq{\ipr{{\by}}{\vecA_i}}$ for $i\in \cS_{\bx}$ is different. 
This is the main idea of Algorithm~\ref{alg:1}. We first present Theorem~\ref{thm:alg_general} for generalized linear measurements.
% Theorem~\ref{thm:alg_bcs} and Theorem~\ref{thm:alg_spl} formalize this intuition for \bcs\ and \spl\ respectively. 
\begin{theorem}[Sample Complexity of Algorithm~\ref{alg:1} for GLMs]\label{thm:alg_general}
Suppose the GLM is such that for each $i\in [m]$, $y_i$ is a subgaussian random variable with subgaussian norm given by $\normi{{y_i}}_{\psi_2}$. For any $\bx$, suppose for some $L$, $\bbE\insq{g'(\vecA_i^T\bx)}\geq L\cdot
\normi{{y_i}}_{\psi_2}$   for all $i\in [m]$. Algorithm~\ref{alg:1} recovers the unknown signal with high probability if 
\begin{align}
m \geq \frac{C}{{\min\inb{L, L^2}}}(\log\inp{k}+\log\inp{n-k})\label{eq: alg_bound}
\end{align} where $C$ is some constant.
\end{theorem}
When $y_j$ is subgaussian, $y_j\vecA_{i,j}$ for any $i,j$ is a sub-exponential random variable. This observation allows us to use a concentration result for sub-exponential random variables to analyse the sample complexity. See Section~\ref{sec:proofs} for a detailed proof. 

As corollaries to Theorem~\ref{thm:alg_general}, we obtain the following sample complexity bounds for \bcs\ and \spl. These corollaries are proved in Appendix~\ref{proof:sec:alg}.
\begin{corollary}[Sample Complexity of Algorithm~\ref{alg:1} for \bcs]\label{thm:alg_bcs}
Algorithm~\ref{alg:1} recovers the unknown signal for \bcs\ with high probability if $m=O\inp{\inp{k+\sigma^2}(\log\inp{k}+\log\inp{n-k})}$.
\end{corollary}
\begin{corollary}[Sample Complexity of Algorithm~\ref{alg:1} for \spl]\label{thm:alg_spl}
Algorithm~\ref{alg:1} recovers the unknown signal for \spl\ if $m=O\inp{\inp{k+\sigma^2}(\log\inp{k}+\log\inp{n-k})}$.
\end{corollary}
 Interestingly, the sample complexity for both \bcs\ and \spl\ is the same. This can be explained by similar values of $L$, which result in similar rates of concentration of $l_i$'s around their expectation in both the cases. 
This also implies that in the regime where $m = O((k+\sigma^2)\log(n-k))$, having access to $\vecA_i^T \bx+z_i$ instead of $\sign{\vecA_i^T \bx+z_i}$, does not improve the sample complexity beyond constants.

Using Theorem~\ref{thm:alg_general}, we obtain the following corollary for logistic regression (see proof in Appendix~\ref{proof:sec:alg}).
\begin{corollary}[Sample Complexity of Algorithm~\ref{alg:1} for \logreg]\label{thm:alg_logreg}
Algorithm~\ref{alg:1} recovers the unknown signal for \logreg\ if $m=O\inp{\inp{k+1/\beta^2}\inp{\log{k}+\log\inp{n-k}}}$.
\end{corollary}
Comparing the sample complexity bounds of \bcs\ and \logreg, we notice that the sample complexity is similar except that   the noise variance $\sigma^2$ is replaced by $1/\beta^2$. This relationship is not surprising as a similar relationship was also present in the sample complexity bounds in \cite{hsu2024sample} (for logistic regression) and \cite{kuchelmeister2024finite} (for probit model). Note that, in the noiseless case, when $\beta\rightarrow \infty$ (or $\sigma = 0$ for \bcs), the sample complexity is $O(k\log{n})$, which is close to the simple counting lower bound of $k\log{n/k}$. On the other hand, when $\beta = 0$ (or $\sigma\rightarrow \infty$ for \bcs), $m\rightarrow \infty$, which makes intuitive sense as very high levels of noise render the output useless.


To compute the time complexity of the algorithm, notice that the for loop in step 2 takes $O(n\times m)$ time and step 4 takes $O(n\log{n})$ time. Thus, the computational complexity of the algorithm is $O(nm+n\log{n})$, which is $O((k+\sigma^2)n\log{n})$ for $m = O((k+\sigma^2)\log{n}$.
To compute the time complexity of the algorithm, notice that the for loop in step 2 takes $O(n\times m)$ time and step 4 takes $O(n\log{n})$ time. Thus, the computational complexity of the algorithm is $O(nm+n\log{n})$, which is $O((k+\sigma^2)n\log{n})$ for $m = O((k+\sigma^2)\log{n}$.




% \subsection{Does metric learning provably help in learning a task?}

%Several works have stipulated sample complexity for generalization bounds and excess risk for metric and similarity learning for classification tasks~\cite{Guo2013GuaranteedCV,Bellet2012SimilarityLF}
Understanding a question of the form: "if metric learning provably helps in classification" would require analyzing whether metric learning helps achieve 
 better bounds asymptotically. Ideally, this would require either providing a sharper bound for classification tasks compared to the known algorithms w/ a priori learning a metric or showing that it can be a method to beat lower bounds for learning specific tasks. Note, metric learning generally comes up with a computational overhead, for it is designed as a convex/non-convex objective. Now, even if there is an oracle that solves that objective for `free', beating a lower bound could be tricky for the general technique for achieving the lower bounds is showing statistical bottlenecks in terms of VC-dim of the hypothesis class or information theoretic arguments. 

 \subsection{Sample complexity of learning a metric}

Sample complexity for generalization bound has been established for both general metric learning problem and metric learning for classification: $k$NN in the imbalanced data setting~\cite{viola20,Gautheron2020MetricLF}, and linear classification~\cite{Bellet2012SimilarityLF}; Mahalanobis distance metric~\cite{Verma2015SampleCO}, non-linear metric learning in sparse and bounded amplification regimes (embeddings corresponding to neural networks )~\cite{Kozdoba2021DimensionFG}.
More recently, learning a Mahalanobis metric has been further explored under various settings, e.g. low dimensional metric learning~\cite{Mason2017LearningLM}, multitask metric learning~\cite{Wang2019MultitaskML}, under noise labels~\cite{Alishahi2023LinearDM}.

\subsubsection{Learning a metric for perceptron in the noisy setting}

\begin{shaded}
\noindent Setting: Given a sample space $\cX \in \reals^d$, with a distribution $P_X$. Label set $\cY = \curly{-1,1}$ with noise $P_{Y|X}$. $S^1$ and $S^{-1}$ denotes examples with two labels.\\

\noindent\textit{at time} $t = 1,2,\ldots$
\begin{enumerate}
    \item environment provides a labeled input $(x_t,y_t) \sim P_{X,Y}$
    \item learner updates $S^1_t:= S^1_{t-1} \cup \curly{x_t}$ or $S^{-1}_t:= S^{-1}_{t-1}  \cup \{x_t\}$ 
    \item learner updates $M_t \leftarrow \textsf{Gradient-Update}(M_{t-1}, S^1_t,S^{-1}_t)$ 
    \item if $w_{t-1}$ misclassifies $M_t^{1/2}x_t$ then learner updates
    \begin{enumerate}
        \item $w_t \leftarrow w_{t-1} + y_t(M_t^{1/2}x_t)$
    \end{enumerate}
    %learner updates the teaching set $\mathcal{T}_t := \mathcal{T}_{t-1} \cup \{(x_i^t,x_j^t,x_k^t)\}$
    %\item  learner makes updates $M_t \leftarrow M_{t-1} + \eta\cdot \frac{1}{t} \sum_{i=1}^t (x_i - x_k)(x_i - x_k)^{\top} - (x_i - x_k)(x_i - x_j)^{\top}$
    %\item learner updates $\hat{M_t} \leftarrow \textsf{EigenDecompose}(M_t)$
    %\item teacher receives $\ell(M_t, \mathcal{T}_t)$
\end{enumerate}
\end{shaded}
Here, for a given labeled example $(x,y)$,
\begin{align*}
\textsf{Gradient-Update}(M_{t-1}, S^1_t,S^{-1}_t) = M_{t-1} + \eta y\cdot\paren{ \sum_{x' \in S^1_t} (x - x')(x-x')^{\top} - \sum_{x' \in S^{+1}_t} (x - x')(x-x')^{\top}} 
\end{align*}
One problem of interest is if this approach achieves a better bound on the excess risk for some loss function $\ell$:
\begin{align*}
    \expctover{(x,y)\sim P_{X,Y}}{\ell(w_t(x),y)} - \expctover{(x,y)\sim P_{X,Y}}{\ell(f^*(x),y)}
\end{align*}


% \section{Proofs for Deterministic Safety Algorithms}\label{sec:proofs-det}

In this section we prove the correctness of our algorithm in \Cref{sec:warmup}.
We first prove that the \textsc{Round} procedure in \Cref{alg:aac-byz} satisfies the properties below, and then prove that \Cref{alg:skeleton} solves consensus under Byzantine faults.
\begin{description}
    \item[Strong Validity] If all correct processes propose the same value $v$ and a correct process returns a pair $\langle \textsc{Grade}, v' \rangle$, then $\textsc{Grade} = \textsc{Commit}$ and $v' = v$.
    \item[Consistency] If any correct process returns  $\langle \textsc{Commit}, v \rangle$, then no correct process returns $(\cdot, v' \ne v)$.
    \item[Termination] If all correct processes propose, then every correct process eventually returns.
\end{description}

In our proofs we rely on the following properties of Byzantine Reliable Broadcast (BRB)~\cite{book}:
\begin{description}
    \item[BRB-Validity] If a correct process $p$ broadcasts a message $m$, then every correct process eventually delivers $m$.
    \item[BRB-No-duplication] Every correct process delivers at most one message.
    \item[BRB-Integrity] If some correct process delivers a message $m$ with sender $p$ and process $p$ is correct, then $m$ was previously broadcast by $p$.
    \item[BRB-Consistency] If some correct process delivers a message $m$ and another correct process delivers a message $m$, then $m = m$.
    \item[BRB-Totality] If some message is delivered by any correct process, every correct process eventually delivers a message.
\end{description}

\begin{lemma}\label{lem:byz-round-validity}
    With Byzantine faults and $n=3f+1$, \Cref{alg:aac-byz} satisfies strong validity.
\end{lemma}
\begin{proof}
    If all correct processes propose the same value $v$, then at least $2f+1$ processes BRB-broadcast an \textsc{Init} message for $v$, and therefore at most $f$ processes BRB-broadcast an \textsc{Init} message for $1-v$. Thus $v$ will be the majority value among all \textsc{Init} messages delivered in phase 1, at all correct processes. Thus all correct processes will BRB-broadcast an \textsc{Echo} message for $v$. Furthermore, no Byzantine process can produce a valid \textsc{Echo} message for $1-v$, since to do so would require a set of $2f+1$ \textsc{Init} message with a majority value of $1-v$. This is impossible due to the properties of BRB and the fact that at most $f$ processes have BRB-broadcast an \textsc{Init} message for $1-v$.
    So, all valid \textsc{Echo} messages received by correct processes will be for $v$, so all correct processes will commit $v$ at line~\ref{line:fac-byz-commit}.
\end{proof}

\begin{lemma}
    With Byzantine faults and $n=3f+1$, \Cref{alg:aac-byz} satisfies consistency.
\end{lemma}
\begin{proof}
    If a correct process $p_1$ commits $v$ at line~\ref{line:fac-byz-commit}, then it must have delivered a set $S_1$ of $2f+1$ \textsc{Echo} messages for $v$ at line~\ref{line:fac-byz-wait-echo}. Take now another process $p_2$ and consider the set $S_2$ of $2f+1$ \textsc{Echo} messages it delivers at line~\ref{line:fac-byz-wait-echo}. By quorum intersection, $S_1$ and $S_2$ must intersect in at least $f+1$ messages. By the BRB-Consistency property, these $f+1$ messages must be identical at $p_1$ and $p_2$. Thus $p_2$ delivers at least $f+1$ \textsc{Echo} messages for $v$, which constitutes a majority of the $2f+1$ \textsc{Echo} messages it delivers overall. So if $p_2$ commits a value at line~\ref{line:fac-byz-commit}, then it must commit $v$, and if $p_2$ adopts a value at line~\ref{line:fac-byz-adopt}, then it must adopt $v$.
\end{proof}

\begin{lemma}
    With Byzantine faults and $n=3f+1$, \Cref{alg:aac-byz} satisfies termination.
\end{lemma}
\begin{proof}
    Follows immediately from the algorithm and from the properties of Byzantine Reliable Broadcast. Processes perform two phases; the only blocking step of each phase is waiting for $n-f$ messages (lines~\ref{line:fac-byz-wait-init} and~\ref{line:fac-byz-wait-echo}). This waiting eventually terminates, by the BRB-Validity property and the fact that there are at least $n-f$ correct processes.
\end{proof}

\begin{theorem}\label{thm:validity-byz}
    With Byzantine faults and $n=3f+1$, \Cref{alg:skeleton} satisfies strong validity.
\end{theorem}
\begin{proof}
    This follows from the strong validity property of the \textsc{Round} procedure (\Cref{lem:byz-round-validity}): if all correct processes propose $v$ to consensus, then all correct processes propose $v$ to \textsc{Round} in the first round, where by \Cref{lem:byz-round-validity}, all correct processes commit $v$, and thus all correct processes decide $v$ at line~\ref{line:skeleton-decide}.
\end{proof}

\begin{theorem}\label{thm:agreement-byz}
    With Byzantine faults and $n=3f+1$, \Cref{alg:skeleton} satisfies agreement.
\end{theorem}
\begin{proof}
    Let $r$ be the earliest round at which some process decides and let $p$ be a process that decides $v$ at round $r$. We will show that any other process $p'$ that decides, must decide $v$. 
    
    For $p$ to decide $v$ at round $r$, \textsc{Round} must output $(\textsc{Commit}, v)$ in that round. Thus, by the consistency property of \textsc{Round}, $\textsc{Round}(r,\cdot)$ must output $(\cdot, v)$ at all correct processes. If $\textsc{Round}(r,\cdot)$ outputs $(\textsc{Commit}, v)$ for $p'$, then $p'$ decides $v$ at round $r$ (line~\ref{line:skeleton-decide}). Otherwise, all correct processes input $v$ to $\textsc{Round}(r+1,\cdot)$, and by the strong validity property, all processes (including $p'$) will output $(\textsc{Commit}, v)$ and decide $v$ at round $r+1$.
\end{proof}

\begin{theorem}\label{thm:termination-byz}
    With Byzantine faults and $n=3f+1$, \Cref{alg:skeleton} satisfies termination.
\end{theorem}
\begin{proof}
     We can describe the execution of the protocol as a Markov chain with states $0,\ldots,n-f=2f+1$; the system is at state $i$ if $i$ correct processes have estimate ($est_i$ variable) equal to $0$ before invoking $\textsc{Round}$. Due to the strong validity property of the $\textsc{Round}$ procedure, states $0$ and $2f+1$ are absorbing states. There is a non-zero transition probability from each state (including $0$ and $2f+1$), to state $0$ or $2f+1$, or both (we show this below). Therefore, with probability $1$, the system will eventually reach one of the two absorbing states and remain there. Once this happens (i.e., once all processes have the same $est_i$ variable), the strong validity property of $\textsc{Round}$ ensures that all processes (who have not decided yet) will decide within a round.
    
    It only remains to show that there is a non-zero transition probability from each state to at least one of the absorbing states $0$ and $2f+1$. Consider a state $i \notin \{0,2f+1\}$; there is a schedule $S$ with non-zero probability which leads the system from $i$ to $0$ or $2f+1$ in one invocation of \textsc{Round}. We consider two cases:
    \begin{itemize}
        \item $i < f+1$: in this case $0$ is the minority value among correct processes. In schedule $S$, the $n-f$ \textsc{Init} messages delivered by correct process at line~\ref{line:fac-byz-wait-init} are all from correct processes. Thus, every correct process sees $i$ $0$s and $2f+1-i$ $1$; $1$ is the majority value, so all correct processes adopt it for phase 2. In phase 2, $S$ again ensures that the $n-f$ \textsc{Echo} messages delivered by correct process at line~\ref{line:fac-byz-wait-echo} are all from correct processes. Thus, all correct processes see $2f+1$ \textsc{Echo} messages for $1$ and commit $1$, bringing the system to state $0$.
        \item $i \geq f+1$: in this case $0$ is the majority value among correct processes. This case is symmetrical with respect to the previous one: the only difference is that all correct processes adopt $0$ (the majority value) at the end of phase 1, and all correct processes deliver $2f+1$ \textsc{Echo} messages for $0$, thus committing $0$ and bringing the system to state $2f+1$.
    \end{itemize}
\end{proof}
% \appendix
% \section{Comparison with \cite{vershyninPlan}}\label{sec:comparison_PV}
Algorithm~\ref{alg:1} is similar to the two step estimation procedure outlined in \cite{vershyninPlan} which was given to estimate the unknown signal within a two norm guarantee. 
Computing the vector $\mathbf{l} = \inp{l_1, \ldots, l_n}$ is the same as the first step of the procedure in [Section~1.2]\cite{vershyninPlan} where a linear estimator is computed. The second step of our algorithm (sorting and keeping the top-$k$ indices) can be thought of as a projection on a feasible set [Section~1.3]\cite{vershyninPlan}. However, this requires the estimation error to be small enough for the exact recovery of a binary vector.

The setup in \cite{vershyninPlan} is for the recovery of an unknown signal with small two-norm error, whereas our problem of exact recovery of a sparse binary vector is more suited for recovery under  infinity norm. This results in weak bounds ($m\approx O(k^2)$) when we specialize various results in \cite{vershyninPlan} to our case. We first note that we require $\bbE\norm{\frac{\hat{x}}{\norm{\hat{x}}}-\bar{x}}< \sqrt{\frac{{2}}{{k}}}$ for exact recovery. Otherwise, there exist two binary $k$-sparse vectors which have hamming distance at least two. 

We first consider the 1-bit compressed sensing result in Section 3.5 (page 13). Setting the LHS to $\sqrt{\frac{{2}}{{k}}}$, we get
\begin{align*}
\sqrt{\frac{{2}}{{k}}}\leq C\sqrt{\frac{k\log\inp{2n/k}}{m}}.
\end{align*}
This implies that $m\approx C_1 k^2\log\inp{2n/k}$ for some constant $C_1$.

Next, we consider [Theorem 9.1]\cite{vershyninPlan}. Note that for 1-bit compressed sensing $\eta^2 = 1$ and 
\begin{align*}
\mu &= \bbE\insq{s_1\ipr{a_1}{\bar{x}}}\\
& =\bbE\insq{s_1\ipr{a_1}{{x}}}\\
& \stackrel{(a)}{=} \frac{1}{\sqrt{k}}\sqrt{\frac{2}{\pi}}\times\frac{k}{\sqrt{\inp{k+\sigma^2}}}\\
&= \sqrt{\frac{2}{\pi}}\times\frac{\sqrt{k}}{\sqrt{\inp{k+\sigma^2}}}.
\end{align*} where $(a)$ follows from \eqref{eq:expt4}. 
Then, 
\begin{align*}
\norm{x-\mu\bar{x}} &= \norm{x-\sqrt{\frac{2}{\pi}}\times\frac{\sqrt{k}}{\sqrt{\inp{k+\sigma^2}}}\frac{x}{\sqrt{k}}}\\
& = \norm{x-\sqrt{\frac{2}{\pi}}\times\frac{x}{\sqrt{\inp{k+\sigma^2}}}}
\end{align*} We require $\norm{x-\mu\bar{x}}<\frac{2}{\sqrt{\pi\inp{k+\sigma^2}}}$ in order to exactly recover the unknown signal $x$. 



We assume that $K$ is also a closed cone in $\bbR^n$. Then, by [Section~2.4]\cite{vershyninPlan}, $w_t(K) = tw_1(K) \leq t C\sqrt{k\log\inp{2n/k}}$ ([Section~2.4]\cite{vershyninPlan}). We choose $s = w_1(K)$. Substituting the bound for LHS and taking the limit $t\rightarrow 0$, we get
\begin{align*}
\frac{2}{\sqrt{\pi\inp{k+\sigma^2}}}\leq \frac{8 C \sqrt{k\log\inp{2n/k}}}{\sqrt{m}}.
\end{align*} Thus, $m\approx 4C(k+ \sigma^2)k\log\inp{2n/k}$. 


% % \clearpage
% \setcounter{page}{1}
% \maketitlesupplementary

\appendix

\section*{Appendix Overview}
This technical appendix consists of the following sections:
\begin{itemize}
% \item
% In Section~\ref{method}, we provide more method design.
\item
In Section~\ref{ablation}, we provide more ablation experiments to verify the effectiveness of FlowScene.
\item
In Section~\ref{quantitative}, we provide quantitative results from more experiments.
\item
In Section~\ref{visualizations}, we present more visual qualitative results of the SemanticKITTI val set.
\item
In Section~\ref{sec:limit},  we analyze the shortcomings of our method and directions for future work.
\end{itemize} 

\section{More Ablation Studies}
\label{ablation}
\subsection{Ablation Study for Optical Flow Networks}
Table~\ref{tab:flow} presents the performance of different optical flow networks. We compare several state-of-the-art methods, including PWC-Net~\cite{sun2018pwc}, RAFT~\cite{teed2020raft}, and FlowFormer~\cite{huang2022flowformer}, along with our setting, GMFlow~\cite{xu2022gmflow}, which is highlighted in the last row.
Our setting achieves the highest IoU of 45.01\% and mIoU of 18.13\%, outperforming all other methods in both metrics. These results suggest that GMFlow effectively captures motion cues and integrates them into the semantic scene completion task, providing superior performance over the other optical flow networks tested, with significantly fewer parameters.

\subsection{Ablation Study for Backbone Networks}
Table~\ref{tab:backbone} examines the impact of different backbone networks on the performance of FlowScene. The study compares EfficientNetB7~\cite{tan2019efficientnet}, ResNet50~\cite{he2016resnet}, and RepVit-M2.3~\cite{wang2023repvit}(our setting). Our method, using RepVit-M2.3, achieves the highest IoU of 45.01\% and mIoU of 18.13\%, surpassing both EfficientNetB7 (44.31\% IoU, 17.63\% mIoU) and ResNet50 (44.12\% IoU, 16.98\% mIoU). RepVit-M2.3, though achieving the best performance, maintains a relatively low parameter count of 22.4M. In comparison, EfficientNetB7 has a much higher parameter count of 63.8M, while ResNet50 is more parameter-efficient at 25.6M. RepVit-M2.3 offers a good balance between performance and parameter count, making it an ideal choice for our backbone network.
%--------------------------------
\begin{table}[t]
  \centering\small
\setlength{\tabcolsep}{4pt}{
  \begin{tabular}{l|ccc}
    \toprule
    {\textbf{Method}}& {\textbf{IoU}(\%)}&{\textbf{mIoU}(\%)}&\textbf{Params(M)} \\
    \midrule
    PWC-Net+~\cite{sun2018pwc}& 43.31&17.13&8.8\\
    RAFT~\cite{teed2020raft}&44.12&17.56&5.3 \\
    FlowFormer~\cite{huang2022flowformer}&44.33&17.74&18.2 \\
    \rowcolor{gray!20}{GMFlow}~\cite{xu2022gmflow}&{\textbf{45.01}}&\textbf{18.13}&\textbf{4.7} \\
    \bottomrule
  \end{tabular}
  }
      \caption{Ablation study for optical flow networks.}
  \label{tab:flow}
  % \vspace{-5mm}
\end{table}
%-------------------------------------------------------------
%--------------------------------
\begin{table}[t]
  \centering\small
\setlength{\tabcolsep}{4pt}{
  \begin{tabular}{l|ccc}
    \toprule
    {\textbf{Method}}& {\textbf{IoU}(\%)}&{\textbf{mIoU}(\%)}&\textbf{Params(M)} \\
    \midrule
    EfficientNetB7~\cite{tan2019efficientnet}& 44.31&17.63&63.8\\
    ResNet50~\cite{he2016resnet}&44.12&16.98&25.6 \\
    % RepVit-M1.5~\cite{wang2023repvit}&43.25&17.21&14.0 \\
    \rowcolor{gray!20}{RepVit-M2.3}~\cite{wang2023repvit}&{\textbf{45.01}}&\textbf{18.13}&\textbf{22.4} \\
    \bottomrule
  \end{tabular}
  }
      \caption{Ablation study for backbone networks.}
  \label{tab:backbone}
  % \vspace{-5mm}
\end{table}
%-------------------------------------------------------------
\section{More Quantitative Results}
\label{quantitative}

To provide a more thorough comparison, we provide additional quantitative results of semantic scene completion on the SemanticKITTI validation set in Table~\ref{tab:val}. The results further demonstrate the effectiveness of our approach in enhancing 3D scene perception performance.
Compared with the previous state-of-the-art methods, FlowScene is superior to other HTCL~\cite{li2024htcl} in semantic scene understanding, with a 1.00\% increase in mIoU. In addition, compared with Symphonize~\cite{jiang2024symphonize}, huge improvements are made in both occupancy and semantics. IoU and mIoU enhancement are of great significance for practical applications. It proves that we are not simply reducing a certain metric to achieve semantic scene completion.
%-------------------------------------------------------------------------

\begin{table*}[ht]
  \setlength{\tabcolsep}{2pt}
  \renewcommand\arraystretch{1.2}
  
  \resizebox{\textwidth}{!}{
  \begin{tabular}{l|l|c|r|rrrrrrrrrrrrrrrrrrr|r}
    \toprule
    \textbf{Methods} &\textbf{Published}&\textbf{Inputs}&\textbf{IoU}   
    & \rotatebox{90}{\vcenteredbox{\colorbox[RGB]{255,0,255}{\textcolor[RGB]{255,0,255}{\rule{1px}{1px}}}} \textbf{road} (15.30$\%$)} 
    & \rotatebox{90}{\vcenteredbox{\colorbox[RGB]{75,0,75}{\textcolor[RGB]{75,0,75}{\rule{1px}{1px}}}} \textbf{sidewalk} (11.13$\%$)} 
    & \rotatebox{90}{\vcenteredbox{\colorbox[RGB]{255,150,255}{\textcolor[RGB]{255,150,255}{\rule{1px}{1px}}}} \textbf{parking} (1.12$\%$)} 
    & \rotatebox{90}{\vcenteredbox{\colorbox[RGB]{175,0,75}{\textcolor[RGB]{175,0,75}{\rule{1px}{1px}}}} \textbf{other-grnd} (0.56$\%$)} 
    &  \rotatebox{90}{\vcenteredbox{\colorbox[RGB]{255,200,0}{\textcolor[RGB]{255,200,0}{\rule{1px}{1px}}}} \textbf{building} (14.10$\%$)} 
    &  \rotatebox{90}{\vcenteredbox{\colorbox[RGB]{100,150,245}{\textcolor[RGB]{100,150,245}{\rule{1px}{1px}}}} \textbf{car} (3.92$\%$)} 
    & \rotatebox{90}{\vcenteredbox{\colorbox[RGB]{80,30,180}{\textcolor[RGB]{80,30,180}{\rule{1px}{1px}}}} \textbf{truck} (0.16$\%$)} 
    & \rotatebox{90}{\vcenteredbox{\colorbox[RGB]{100,230,245}{\textcolor[RGB]{100,230,245}{\rule{1px}{1px}}}} \textbf{bicycle} (0.03$\%$)}  
    & \rotatebox{90}{\vcenteredbox{\colorbox[RGB]{30,60,150}{\textcolor[RGB]{30,60,150}{\rule{1px}{1px}}}} \textbf{motocycle} (0.03$\%$)} 
    & \rotatebox{90}{\vcenteredbox{\colorbox[RGB]{0,0,255}{\textcolor[RGB]{0,0,255}{\rule{1px}{1px}}}} \textbf{other-vehicle} (0.20$\%$)} 
    & \rotatebox{90}{\vcenteredbox{\colorbox[RGB]{0,175,0}{\textcolor[RGB]{0,175,0}{\rule{1px}{1px}}}} \textbf{vegetation} (39.3$\%$)} 
    & \rotatebox{90}{\vcenteredbox{\colorbox[RGB]{135,60,0}{\textcolor[RGB]{135,60,0}{\rule{1px}{1px}}}}  \textbf{trunk} (0.51$\%$)} 
    & \rotatebox{90}{\vcenteredbox{\colorbox[RGB]{150,240,80}{\textcolor[RGB]{150,240,80}{\rule{1px}{1px}}}} \textbf{terrain} (9.17$\%$)} 
    & \rotatebox{90}{\vcenteredbox{\colorbox[RGB]{255,30,30}{\textcolor[RGB]{255,30,30}{\rule{1px}{1px}}}} \textbf{person} (0.07$\%$)} 
    & \rotatebox{90}{\vcenteredbox{\colorbox[RGB]{255,40,200}{\textcolor[RGB]{255,40,200}{\rule{1px}{1px}}}} \textbf{bicylist} (0.07$\%$)} 
    & \rotatebox{90}{\vcenteredbox{\colorbox[RGB]{150,30,90}{\textcolor[RGB]{150,30,90}{\rule{1px}{1px}}}}  \textbf{motorcyclist} (0.05$\%$)} 
    &  \rotatebox{90}{\vcenteredbox{\colorbox[RGB]{255,120,50}{\textcolor[RGB]{255,120,50}{\rule{1px}{1px}}}} \textbf{fence} (3.90$\%$)} 
    & \rotatebox{90}{\vcenteredbox{\colorbox[RGB]{255,240,150}{\textcolor[RGB]{255,240,150}{\rule{1px}{1px}}}} \textbf{pole} (0.29$\%$)} 
    & \rotatebox{90}{\vcenteredbox{\colorbox[RGB]{255,0,0}{\textcolor[RGB]{255,0,0}{\rule{1px}{1px}}}} \textbf{traf.-sign} (0.08$\%$)}& \textbf{mIoU}  \\
    
    \midrule
    % LMSCNet$^\dagger$\cite{roldao2020lmscnet}&3DV'2020  & 28.61&40.68&18.22&4.38&0.00&10.31&18.33&0.00&0.00&0.00&0.00&13.66&0.02&20.54&0.00&0.00&0.00&1.21&0.00&0.00&6.70   \\
    % AICNet$^\dagger$\cite{li2020aicnet}&CVPR'2020   &29.59&43.55&20.55&11.97&0.07&12.94&14.71&4.53&0.00&0.00&0.00&15.37&2.90&28.71&0.00&0.00&0.00&2.52&0.06&0.00&8.31 \\
    % JS3C-Net$^\dagger$ \cite{yan2021sparse}&AAAI'2021   &38.98&50.49&23.74&11.94&0.07&15.03&24.65&4.41&0.00&0.00&6.15&18.11&4.33&26.86&0.67&0.27&0.00&3.94&3.77&1.45&10.31\\
    MonoScene \cite{cao2022monoscene}&CVPR'2022 &S& 36.86&56.52&26.72&14.27&0.46&14.09&23.26&6.98&0.61&0.45&1.48&17.89&2.81&29.64&1.86&1.20&0.00&5.84&4.14&2.25&11.08 \\
    
    TPVFormer \cite{huang2023tri}& CVPR'2023&S& 35.61&56.50& 25.87& 20.60& 0.85& 13.88& 23.81& 8.08& 0.36& 0.05 &4.35 &16.92& 2.26& 30.38& 0.51& 0.89& 0.00& 5.94& 3.14& 1.52&11.36 \\
    
    OccFormer\cite{zhang2023occformer}& ICCV'2023 &S& 36.50  & 58.85& 26.88& 19.61& 0.31& 14.40& 25.09& 25.53& 0.81& 1.19& 8.52& 19.63& 3.93& 32.62& 2.78& 2.82& 0.00& 5.61& 4.26& 2.86 & 13.46\\
    % VoxFormer-S\cite{li2023voxformer}& CVPR'2023&44.02&54.76&26.35&15.50&0.70&17.65&25.79&5.63&0.59&0.51&3.77&24.39&5.08&29.96&1.78&3.32&0.00&7.64&7.11&4.18&12.35\\
    Symphonize~\cite{jiang2024symphonize}&CVPR'2024&S&41.92&56.37&27.58&15.28&0.95&21.64&28.68&20.44&2.54&2.82&13.89&25.72&6.60&30.87&3.52&2.24&0.00&8.40&9.57&5.76&14.89\\
        
    VoxFormer-T\cite{li2023voxformer}& CVPR'2023&T& 44.15 &53.57&26.52&19.69&0.42&19.54&26.54&7.26&1.28&0.56&7.81&26.10&6.10&33.06&1.93&1.97&0.00&7.31&9.15&4.94 & 13.35\\
    H2GFormer~\cite{wang2024h2gformer} & AAAI'2024&T&44.69&57.00&29.37&21.74&0.34&20.51&28.21&6.80&0.95&0.91&9.32&27.44&7.80&36.26&1.15&0.10&0.00&7.98&9.88&5.81&14.29 \\
    HASSC~\cite{wang2024HASSC}&CVPR'2024&T&44.58&55.30&29.60&25.90&11.30&23.10&23.00&2.90&1.90&1.50&4.90&24.80&9.80&26.50&1.40&3.00&0.00&14.30&7.00&7.10&14.74\\
    HTCL~\cite{li2024htcl}&ECCV'2024&T&45.51&63.70&32.48&23.27&0.14&24.13&34.30&20.72&3.99&2.80&11.99&26.96&8.79&37.73&2.56&2.70&0.00&11.22&11.49&6.95&17.13\\
    \midrule
    \rowcolor{gray!20}\textbf{Ours}&  &T& {\underline{45.01}} &{\textbf{63.72}}&{\underline{32.10}}& 22.20&\underline{1.31}& {\textbf{25.63}}& {\underline{33.33}}&\textbf{33.47}&2.36& {\textbf{5.09}}&\textbf{16.99}& {\underline{26.35}}& {8.68}& {\underline{36.73}}&\textbf{3.79}& {{1.92}}& {\textbf{0.00}}& {\underline{12.05}}& {\textbf{11.65}}& {\underline{7.05}}&  {\textbf{18.13}}\\
    \bottomrule
  \end{tabular}}
  \caption{Quantitative results on the SemanticKITTI validation set. The best results are in {\textbf{Bold}}.}
  \label{tab:val}
\end{table*}
%-------------------------------------------------------------------------
\section{More Visualizations Results}
\label{visualizations}
We show visualization examples on the Semantickitti validation set, as shown in Figure~\ref{fig:sup1}. From left to right are the input image, the corresponding optical flow and occlusion mask, the front view SSC, and the top view SSC. Due to the motion information brought by the optical flow, the location information of the scene objects is more accurate and the layout is more reasonable.
We report the performance of more visual comparison results on the SemanticKITTI validation set in Figure~\ref{fig:sup2}. We compare with VoxFormer~\cite{li2023voxformer} and BRGScene~\cite{li2023stereoscene}. In general, our method performs more fine-grained segmentation of the scene and maintains clear segmentation boundaries. For example, in the segmentation completion result of cars, we predict clear separation of each car. In contrast, other methods show continuous semantic errors for occluded cars. In addition, our flow can effectively deal with the problem of mutual occlusion between different objects. Finally, we provide a video in the appendix to show the performance more intuitively.
% -------------------------------------------------
\begin{figure*}[t]
\centering
  \includegraphics[width=\textwidth]{supp2.pdf}
  \caption{Qualitative results on the SemanticKITTI validation set.}
  \label{fig:sup1}
\end{figure*}
% -------------------------------------------------

\section{Discussions}
\label{sec:limit}
Flowscene shows strong performance on the benchmark with an improved number of parameters. This is beneficial for deploying real-world autonomous driving applications. But the inference time of the model needs to be improved. Semantic scene completion in multi-camera settings is also worth attention, which is our future work. Meanwhile, the legal challenges of autonomous driving as well as privacy and data security risks are still topics of debate. Finally, the robustness of semantic scene completion is also an issue worth exploring.

% \clearpage
% -------------------------------------------------
\begin{figure*}[t]
\centering
  \includegraphics[width=\textwidth]{supp1.pdf}
  \caption{Qualitative results on the SemanticKITTI validation set.}
  \label{fig:sup2}
\end{figure*}
% -------------------------------------------------
% \clearpage



% \input{arxiv/logistic_regression}
% \input{arxiv/general_case}
% \newpage

% \bibliographystyle{alpha}
% \bibliography{arxiv/refs}
\end{document}