% === Dynamics section ===

% {\color{blue}
% OUTLINE:
% \begin{itemize}
%     \item Give equations of motion
%     \item Describe via equations why vertical transport is difficult compared to horizontal transport for the same 1D part motion, give necessary conditions for vertical transport being possible, explain why this leads to high surface acceleration.
%     \item Define partives for optimal transport, constraints ($a_{max}$)
%     \item \textit{some optimal waveform example}
%     \item Infer guidelines for waveform
% \end{itemize}
% }

\subsection{Conditions Necessary To Achieve Upward Transport}

\begin{figure}[t]
    \centering
    \includegraphics[width=0.45\textwidth]{figures/dynamics/kinematics_and_forces.jpg}
    \caption{Kinematics and dynamics.
    %
    The part, $P$, and the surface, $S$ are constrained to move purely along the $\Vec{e_z}$ direction.
    %
    The rollers on the left side of $P$ indicate sliding on a frictionless surface.
    %
    % The normal and tangential forces at the contact surface between $S$ and $P$ are shifted to ensure moment equilibrium.
    }
    \label{figure: kinematics and dynamics}
    \vspace{\shift}
\end{figure}

Our notations follow the conventions in Umbanhowar~\cite{umbanhowar2008optimal}. The part $P$ of mass $m_P$ is subject to its weight $-m_P g \Vec{e_z}$ and is held between a vertical vibrating surface $S$ with normal force $F_n \Vec{e_x}$ and tangential friction force $F_f \Vec{e_z}$ on one side, and a frictionless vertical support on the other side.
%
We assume dry Coulomb friction between $P$ and $S$ with coefficients of static and kinetic friction  $\mu_s \geq \mu_k$, respectively.
%
Let $z_S$ be the vertical position of the vibrating surface with respect to a stationary point $O$ and $z_P$ that of the part (Fig. \ref{figure: kinematics and dynamics}). We seek mechanical conditions and surface motion which will transport the part upward over time.

The equation of motion of the part is: $m_P \ddot{z}_P = F_f - m_P g$. If the part is initially stationary with respect to the surface, the part sticks to the surface ($\dot{z}_P = \dot{z}_S$) as long as $|F_f| = m_P|\ddot{z}_S + g| \leq \mu_s F_n$.
%
Therefore, to prevent the part from slipping down when the surface is stationary ($\dot{z}_S = 0$), we need:
\begin{equation}\label{eq: F_n lower bound mu_s}
    F_n > m_P g / \mu_s
\end{equation}
%
At the onset of slipping when $m_P|\ddot{z}_S + g| > \mu_s F_n$, the friction force is $F_f = \mu_k F_n \text{sgn}(\ddot{z}_S + g)$.
If the part is already slipping $(\dot{z}_P \neq \dot{z}_S)$, then $F_f = \mu_k F_n \text{sgn}(\dot{z_S} - \dot{z}_P)$ until the conditions $\dot{z_P} = \dot{z}_S$ and $|F_f| \leq \mu_s F_n$ are reached.
%
In summary, the equations of motion are:
%
\begin{flalign}
    \label{eq:EOM sticking}
    \quad \textit{sticking:} \quad & \dot{z}_P = \dot{z}_S, \quad -\frac{\mu_s F_n}{m_P} -g \leq \ddot{z}_S \leq \frac{\mu_s F_n}{m_P} - g&&\\
    \label{eq:EOM slipping}
    \quad \textit{slipping:} \quad & \ddot{z}_P = \frac{\mu_k F_n}{m_P} \text{sgn}(\dot{z}_S - \dot{z}_P) - g, \quad \dot{z}_P \neq \dot{z}_S&&
\end{flalign}
 To prevent the part from falling down indefinitely during slipping, \eqref{eq:EOM slipping} yields the requirement:
\begin{equation}\label{eq: F_n lower bound mu_k}
    F_n > m_P g / \mu_k
\end{equation}

Let the surface oscillate vertically with period $T$. %: $z_S(t) = z_S(t+T)$. 
%
We assume we can directly prescribe the surface motion up to a maximum acceleration $|\ddot{z}_S| \leq a_{max}$ corresponding to the physical limits of the surface actuator.
%
As a periodic function of time, $\dot{z}_S$ is bounded, thus the part velocity is bounded during sticking. 
%
% During slipping, from \eqref{eq:EOM slipping} and \eqref{eq: F_n lower bound mu_k} the part is always decelerating, therefore its velocity is bounded at all times.
%
During slipping, from \eqref{eq:EOM slipping}, \eqref{eq: F_n lower bound mu_k}, and periodic $z_S$ the part has finite acceleration/deceleration phases, bounding $\dot{z}_P$ at all times.
%
Therefore, excluding edge cases, starting from rest the part reaches a steady state where its velocity is $T$-periodic with a constant average velocity: $v_{ave} = (z_P(t+T) - z_P(t))/T$. 
%
Because the surface has no net travel, upward part motion ($v_{ave} > 0$) requires the surface to slip down with respect to the part, so \eqref{eq: F_n lower bound mu_s} and \eqref{eq:EOM sticking} yield:
%
\begin{equation}\label{eq: a_max lower bound}
    a_{max} > \mu_s F_n/m_P + g > 2g
\end{equation}
%

Equations \eqref{eq: F_n lower bound mu_s} to \eqref{eq: a_max lower bound} quantify the challenges which upward vertical transport presents compared to horizontal transport.
%
In most practical cases, we have $\mu_k < \mu_s < 1$, so \eqref{eq: F_n lower bound mu_s} and \eqref{eq: F_n lower bound mu_k} mean that the part $S$ must be squeezed with a force $F_n$ exceeding its own weight.
%
Equation \eqref{eq:EOM sticking} shows that gravity reduces the maximum upward part acceleration during sticking, and that overcoming this limitation requires squeezing the part harder still.
%
However, from \eqref{eq: a_max lower bound}, squeezing with higher normal forces requires more powerful actuators to reach higher surface accelerations, which already need to exceed $2g$ (compared to $a_{max} > \mu_s g$ for the horizontal case).
%
Finally, equation \eqref{eq:EOM slipping} shows that during slipping, the part accelerates faster down than up.

Note that replacing the frictionless support with a second, synchronized vibrating surface doubles $F_f$, yielding the same equations of motion but instead with $m_P / 2$: for a given $F_n$, parts twice as heavy can be lifted. 
%
% Note that replacing the frictionless support with a second, synchronized vibrating surface doubles $F_f$ and will yield the same equations of motion \eqref{eq:EOM sticking} and \eqref{eq:EOM slipping} when $m_p$ is also doubled, i.e., for a given $F_n$, parts twice as heavy can be lifted. 
%
This can be visualized by mirroring the part, surface, and rollers in Fig.~\ref{figure: kinematics and dynamics}B about the vertical wall to the left of the rollers; there is $2m_p$ between the two synchronized surfaces and the inside surface of each part experiences $F_n$ transmitted through the wall.
%
While we assume completely in-phase surfaces in this work, out-of-phase surfaces could be a topic of future investigation.
%
Furthermore, the limit where $F_n \gg m_P g$ while still satisfying \eqref{eq: a_max lower bound} is equivalent to horizontal transport of a part of mass $m_P$ with acceleration of gravity equal to $F_n/m_P$. 

\subsection{Optimal Upward Stick-Slip Transport}\label{section: optimal upward}

Our analysis adapts the optimization theory for horizontal stick-slip transport presented by Umbanhowar and Lynch~\cite{umbanhowar2008optimal}. For completeness, we reproduce their logic here as applied to our case; namely, we add gravity in the plane of motion and make the normal force $F_n$ an extra parameter.

We would like to find the periodic surface acceleration $\ddot{z}_S$ which maximises the average part velocity $v_{ave}$. Formally, given a period $T$ and a maximum surface acceleration $a_{max}$, find $\ddot{z}_S \colon [0,T] \rightarrow \left[a_{max}, a_{max}\right]$ satisfying the continuity conditions $z_S(0) = z_S(T)$ and $\dot{z}_S(0) = \dot{z}_S(T)$ which maximises:
%
\begin{equation}\label{eq: maximize L}
    \int_0^T L(t) dt = \int_0^T (\dot{z}_P(t)-\dot{z}_S(t)) dt
\end{equation}

Umbanhowar and Lynch make the two following observations, which still hold in our case.

\textit{Observation 1:} ``During any periodic steady-state motion, there
exists a $t \in \left[0,T\right]$ such that $\dot{z}_S(t) = \dot{z}_P(t)$.'' 
%
Indeed, if this were not the case, then the part would always be slipping and since the velocities are continuous functions, the slipping would occur in a fixed direction. 
%
From \eqref{eq:EOM slipping}, $\ddot{z}_P$ would therefore have a constant non-zero value, which is incompatible with steady-state velocity.

\textit{Observation 2:} ``For any optimal solution, $\dot{z}_S(t) \leq \dot{z}_P(t)$ for all $t \in \left[0,T\right]$.'' 
%
This observation is more subtle than the previous one and follows from $\mu_s > \mu_k$. 
%
Indeed, \textit{Observation 1} guarantees that for any interval where the part is slipping down relative to the surface ($\dot{z}_S(t) > \dot{z}_P(t)$), there exists an onset $t_0$ just before which the part was sticking to the surface. 
%
For $t \geq t_0$ and throughout that slipping period, the same $\dot{z}_P(t)$ can be achieved through sticking with $\dot{z}_S(t) = \dot{z}_P(t)$ while keeping $\dot{z}_S$ continuous, which leads to a larger value of $L$ and a more optimal solution.

\begin{figure}[t]
    \centering
    \includegraphics[width=0.45\textwidth]{figures/dynamics/optimal_motion.jpg}
    \caption{Nondimensional accelerations, velocities, and positions of the surface (blue) and part (red) given $f_n = 5$ and $a_{max} = 10g$. Despite the surface achieving $10g$ of acceleration, which is used to slip below and catch up to the part, the part's velocity does go negative unlike in previous work~\cite{umbanhowar2008optimal}.}
    \label{figure: optimal motion}
    \vspace{\shift}
\end{figure}

These observations show that for an optimal surface motion, the part either sticks to the surface or slips up relative to it. 
%
Therefore, the only static friction limit which is crossed in \eqref{eq:EOM sticking} is $\ddot{z}_S < -\mu_s F_n/m_P -g$ and the only value attained by $\ddot{z}_P$ in \eqref{eq:EOM slipping} is $\ddot{z}_P = -\mu_k F_n/m_P - g$.
%
So, although gravity makes the equations of motion \eqref{eq:EOM sticking} and \eqref{eq:EOM slipping} asymmetric in $z \leftrightarrow -z$, this dynamical asymmetry is never encountered during optimal surface motion.
%
Crucially, this means that the optimization results in Umbanhowar and Lynch~\cite{umbanhowar2008optimal} for horizontal equations of motion, which are symmetric in $x \leftrightarrow -x$, can also be applied to vertical motion despite the asymmetry in $z$.
%
In fact, as long as we don't allow the part to slip down relative to the surface (\textit{Observation 2}), the equations of motion for upward vertical transport given $\mu_s$ and $\mu_k$ are the same as the equations of motion for horizontal transport given fictional friction coefficients $\widetilde{\mu_s} = \mu_s F_n/(m_P g) - 1$ and $\widetilde{\mu_k} = \mu_k F_n/(m_P g) + 1$.
%
Note the peculiarity $\widetilde{\mu_s} < \widetilde{\mu_k}$, which theoretically means having the part slip down relative to the surface would allow greater upward part accelerations than sticking to the surface and therefore be preferable. However, slipping down is forbidden because these coefficients are only valid for slipping up and sticking, which have already been established to be optimal.

\begin{figure}[t]
    \centering
    \includegraphics[width=0.45\textwidth]{figures/dynamics/v_ave.jpg}
    \caption{Average normalized velocity versus the normal force per part weight ($f_n$), when $\mu_s = 0.7$ and $\mu_k = 0.6$. 
    %
    Average normalized velocity curves for a given $a_{max}$ are denoted by the dotted lines and several callouts; lighter colors indicate curves with higher maximum accelerations. 
    %
    The red line indicates the value of $f_n$ that maximizes $\frac{v_{ave}}{gT}$ for a given $a_{max}$. 
    %
    Note that this optimal $f_n$ can be an order of magnitude or greater than the force required to statically hold the part.
    %
    The normal force at which a given $a_{max}$ curve intersects the dotted line given by $\frac{v_{ave}}{gT} = 0$ is $f_{n,max}$.}
    \label{figure: v_ave}
    \vspace{\shift}
\end{figure}

It follows from Umbanhowar and Lynch~\cite{umbanhowar2008optimal} that, with $f_n = F_n / (m_P g)$ the normal force per part weight, the optimal surface motion for upward vertical transport is $\ddot{z}_S \colon \left[0,T\right] \rightarrow \left[-a_{max}, a_{max}\right]$ given by:
%
\begin{equation}
    \ddot{z_S}\colon t \mapsto
    \left\{\begin{aligned} 
        &(\mu_k f_n -1) g, &0 \leq t < T_1 &\quad \text{or} \quad t = T\\
        &-a_{max}, &T_1 \leq t < T_2\\
        &a_{max}, &T_2 \leq t < T
    \end{aligned} \right.
\end{equation}
%
where
%
\begin{equation}
\left\{\begin{aligned} 
    T_1 &= T \frac{(\mu_k f_n + 1)}{(\mu_s+\mu_k)f_n}\\
    T_2 &= T_1 + T \frac{(\frac{a_{max}}{g} + \mu_k f_n + 1)(\mu_s f_n -1)}{2\frac{a_{max}}{g}(\mu_s+\mu_k)f_n}
\end{aligned} \right.
\end{equation}
%
The velocity $\dot{z}_S$ is the integral of $\ddot{z}_S$ with zero mean. The surface motion $z_S$ scales with $T^2$, so higher frequency leads to a smaller surface range of motion.
%

An example plot of the optimal motion is given in Fig. \ref{figure: optimal motion}.
%
The three phases are: a sticking phase where the surface and part move upwards, a slipping phase where the surface slips below the part, and another slipping phase where the surface catches up to the part.
%
In practice, this part motion can be difficult to achieve because it requires the transition at $\frac{t}{T}\in \mathbb{N}$ to happen exactly when the part and surface velocities match up. Operating a little below the slipping limit for $t \in [0, T_1[$ can relax this strict timing requirement.

The corresponding average part velocity~\cite{umbanhowar2008optimal}, normalized by $gT$, is:
%
\begin{equation}\label{eq: v_ave optimal}
    \frac{v_{ave}}{gT} = \frac{(\mu_s f_n -1)^2((\frac{a_{max}}{g})^2 - (\mu_k f_n + 1)^2)}{4\frac{a_{max}}{g}(\mu_s+\mu_k)^2f_n^2}
\end{equation}
%
From \eqref{eq: v_ave optimal} we deduce the maximum normal force beyond which upward motion is impossible:
%
\begin{equation}\label{eq: f_n,supp}
    f_{n,max} = \frac{F_{n,max}}{m_P g} = \frac{1}{\mu_k}\left(\frac{a_{max}}{g} - 1\right)
\end{equation}
%
This force is the limit where the resulting kinetic friction becomes so large that the surface can no longer slip below the part for $t \in [T_2, T[$: $\ddot{z}_P = \mu_k F_{n,max} / m_P - g = -a_{max}$. 
%
Hence there is a compromise between squeezing harder to stick more on the way up and not squeezing too hard when dragging the surface back down. 
%
The optimal force between $1/\mu_s$ and $f_{n,max}$ is the root of a cubic equation given by $\partial v_{ave} / \partial f_n = 0$ and can be numerically estimated.
%
A graph of the average velocity normalized by $gT$ is given in Fig. \ref{figure: v_ave}, showing that the optimal normal force is large and scales quadratically with $a_{max}$. 
%
As can be seen in the plot, the optimal normal force can be multiple times that required for static equilibrium~\eqref{eq: F_n lower bound mu_s}, further necessitating high accelerations~\eqref{eq: a_max lower bound}.
%
This realization led us to use impact-induced accelerations to drive the surface, which can achieve the high accelerations necessary for vertical vibratory transport.
%
Additionally, impacts can help satisfy the implicit assumption in the optimal acceleration profile that desired accelerations can be instantly reached, in other words, impacts can rapidly achieve the desired accelerations.
%
Note that we are not directly impacting the part in order to achieve manipulation, as was done by Huang~\cite{huang1997vibratory}.
%
To make the part go down, one could simply reduce $f_n$ until slipping occurs. Alternatively, applying the same optimization to $-v_{ave}$ yields the surface motion which maximizes the part's downward velocity without changing $f_n$.