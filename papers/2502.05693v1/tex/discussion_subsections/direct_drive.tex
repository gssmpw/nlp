The actuator used was able to quickly supply sufficiently large accelerations in a small form factor, however, we were also curious about a direct drive solution.
%
Such a device could allow for easier waveform shaping, and possibly deliver even higher accelerations without needing impacts.
% 
To investigate this, we deconstructed an ND91-4$\Omega$ speaker into its moving coil and permanent magnet and attached a high friction surface to the coil.
%
Despite its higher acceleration capability compared with the Carlton motor, we found it to be quite difficult to achieve transport across  various waveforms and normal forces.
%
The conditions for achieving transport were significantly more forgiving with the Carlton motor.
%%%%%%%%%%%%%%%%%%%%%%%%%%%%%%%%%%%%%%%%%%%%%%%%%%%%

%%%%%%%%%%%%%%%%%%%%%%%%%%%%%%%%%%%%%%%%%%%%%%%%%%%%
Given this, we decided to add impact plates, i.e., hardstops, to contact a protrusion fixed to the moving coil to rapidly decelerate it at the end of the sticking phase.
%
While this did achieve transport in specific cases, agreement between the model's fit and the experimental data was poor. 
%
We hypothesize that this disagreement is due to inconsistent normal forces.
%
The larger vertical range of motion of the platform also results in more horizontal travel because of the flexure constraint.
%
This causes the springed plate to experience to be compressed / relaxed to a larger extent, violating the constant normal force assumption in our modeling.