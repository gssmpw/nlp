As noted by Umbanhowar and Lynch~\cite{umbanhowar2008optimal}, the addition of out-of-plane vibrations can greatly increase the optimal part velocity.
%
While we chose to assume idealized 1D in-plane vibrations for the model, our flexures do not prescribe completely linear motion.
%
The tracking data does show some horizontal motion of the surface, part, and springed plate, however, it is an order of magnitude smaller than that of the vertical motion.
%
Given that, any additional compression of the spring is taken to be a result of horizontal inertial forces from the surface and part, implying a constant normal force.
%%%%%%%%%%%%%%%%%%%%%%%%%%%%%%%%%%%%%%%%%%%%%%%%%%%%

%%%%%%%%%%%%%%%%%%%%%%%%%%%%%%%%%%%%%%%%%%%%%%%%%%%%
Finding the ideal driving waveform was nontrivial with our setup; the dynamics of the impact motor are difficult to predict and therefore specific acceleration profiles were determined empirically.
%
Even then, we were unable to achieve the precise sawtooth velocity profile discussed by Quaid~\cite{quaid1999feeder}, let alone verify the optimal surface motion discussed in Section~\ref{section: dynamics}.
%
Additionally, the plane-on-plane contact between the part, surface, and springed plate enforced a difficult alignment constraint for achieving evenly distributed friction forces.
%
A roller with point contact could have been used as in Nahum~\cite{nahum2022robotic}, as well as compliance in the flexure mount similar to the hand by Cai~\cite{cai2023hand}.
%
Misalignment, as well as differences in the electrical characteristics of each motor and tilting of the gripper could also have caused the occasional non-synchronized motion of the surfaces observable in the video attachment.
%
Regardless, these imprecisions did not seem to significantly impact the performance of the device against the springed plate or when used in the gripper.
