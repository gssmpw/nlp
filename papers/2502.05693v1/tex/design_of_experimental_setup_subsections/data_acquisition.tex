% {\color{blue}
% OUTLINE:
% \begin{itemize}
%     \item Describe the experimental setup (with a diagram)
%     \item Detail the time scale, the fps, the motion tracking, analysis software
% \end{itemize}
% }

\begin{figure}[tb]
    \centering
    \includegraphics[width=0.45\textwidth]{figures/design_of_experimental_setup/recording_setup.jpg}
    \caption{Motion capture setup. 
    %
    (A) The phone camera looks through the lens of a microscope, and observes the motion of the system.
    %
    White packaging labels are marked with dots using a mechanical pencil and affixed to the surface, part, and springed plate; these labels can be seen on the screen of the phone.
    %
    Each of the tracked components are labeled in the physical setup as well as in the view of the phone.
    %
    Movement of the pencil marks are recorded at 960 fps and fed into a motion tracking software to extract position data.
    %
    (B) Close-up of the labels as seen on the phone screen.
    %
    From left to right we have the springed plate, part, high-friction surface, and vibratory platform.}
    \label{figure: recording setup}
    \vspace{\shift}
\end{figure}
%

We used the setup shown in Fig. \ref{figure: recording setup} to get detailed motion data of the surface and part. 
%
A phone was secured in a tripod with its camera aligned along the optical axis of a microscope. 
%
We cut strips off of a white shipping label and affixed them to the surface, part, and springed plate in view of the camera.
%
A mechanical pencil was then used to mark several dots on the labels. 
%
Each vertical transport experiment was recorded using the Super Slow-Motion mode of the phone, which can capture 960 fps for 0.4 seconds. 
%
Videos were then exported to Tracker~\cite{brown2014tracker}, where the points were marked using a combination of autotracking and by hand depending on the degree of motion blur. 
%
Position data, along with velocity and acceleration data calculated using finite differences, were exported to be used in model validation.