% {\color{blue}
% OUTLINE:
% \begin{itemize}
%     \item Talk about driver papers (Katherine's paper + the extension from Inroq)
% \end{itemize}
% }

In our analysis, we assume we can prescribe surface motion to match a particular acceleration waveform.
%
In practice, we can only directly prescribe the driving force acting on the surface, whereas the resulting surface motion is coupled with the motion of the part being picked up, the normal force applied to the part, potential external mechanical stops and springs, and the electrodynamic response of the driving mechanism.
%
This coupling can be resolved either by using a theoretical model to drive the surface in open-loop, by adding sensors to achieve closed-loop control, or by empirically determining which force waveform to apply.
%%%%%%%%%%%%%%%%%%%%%%%%%%%%%%%%%%%%%%%%%%%%%%%%%%%%

%%%%%%%%%%%%%%%%%%%%%%%%%%%%%%%%%%%%%%%%%%%%%%%%%%%%
Additionally, the optimal waveform previously discussed in Section~\ref{section: dynamics} can be difficult to achieve as it requires knowledge of the part mass, the coefficients of friction, and precise timing. 
%
We instead attempted to achieve the two-stage waveform from Quaid~\cite{quaid1999feeder}, which, although it may result in a slower part velocity, is easier to tune in practice.
%
We used a voltage-controlled current source to drive the motor and empirically determined a sawtooth current waveform closely achieved the desired low-acceleration upward and large-acceleration downward motion profile.
%
The drive circuitry was reproduced from previous works~\cite{choi2017grabity, mcmahan2014dynamic}, with the addition of several potentiometers to control the sawtooth waveform's frequency, amplitude, and offset.
%
A human operator adjusted the knobs based on visual and auditory feedback, and the two-phase stick-slip behavior of the part was later validated using motion tracking data.