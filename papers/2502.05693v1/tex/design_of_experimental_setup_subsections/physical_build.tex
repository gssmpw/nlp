\begin{figure}[tb]
    \centering
    % \includegraphics[width=0.5\textwidth]{figures/design_of_experimental_setup/experimental_setup.png}
    \includegraphics[width=0.42\textwidth]{figures/design_of_experimental_setup/experimental_setup_with_motor.png}
    \caption{Schematic of experimental setup is shown in (A), with the constructed setup shown in (B) and a section view of the impact motor in (C). 
    %
    (A, B) The vibrating surface $S$ is shown on the right and the constant force springed plate is shown on the left. 
    %
    Movement of the ram inside the motor causes the surface to move. 
    %
    Flexure stiffness and actuator-platform assembly mass were minimized to avoid filtering movement of the ram. 
    %
    A screw and linear spring are used to adjust the normal force applied to $P$.
    %
    Note that springed plate assembly in (B) is viewed from above for clarity, as the shoulder bolts and spring lie in the same plane.
    %
    (C) The magnetic suspension holds the ram at an equilibrium position. 
    %
    The coil can move the ram towards the suspension, which repels the ram, or away from the suspension possibly generating a collision with the impact wall.}
    \label{figure: experimental_setup}
    \vspace{\shift}
\end{figure}

The experimental setup consists of two main components: the oscillating high-friction surface, and constant force low-friction springed plate (Fig. \ref{figure: experimental_setup}A 
and \ref{figure: experimental_setup}B). 
%
The motor is a Carlton voice-coil haptic actuator from Titan Haptics (section view shown in Fig. \ref{figure: experimental_setup}C and it was selected because of its high bandwidth and ability to rapidly achieve high accelerations in a small form factor.
%
Electromagnetic forces send the 4 gram magnetic ram either towards the magnetic suspension, which holds the ram at an equilibrium position in the center of the actuator, or the plastic end cap (impact wall).
%
With sufficient current, the suspension force can be overcome and impacts can be achieved in either direction.
%
Forces acting on the ram from the magnetic suspension, coil, and actuator structure (i.e., impacts) are applied equally and in the opposite direction to the surface causing it to move.
%%%%%%%%%%%%%%%%%%%%%%%%%%%%%%%%%%%%%%%%%%%%%%%%%%%%

%%%%%%%%%%%%%%%%%%%%%%%%%%%%%%%%%%%%%%%%%%%%%%%%%%%%

%
The surface is connected to ground via 3D-printed PLA flexures.
%
To minimize attenuation of the ram forces on the surface motion, we sought to balance the desired low stiffness of the flexures while ensuring they were strong enough support the actuator-platform assembly, as well as reduce the mass of the entire assembly.
%
We printed several iterations, varying the length, $l_f$, and thickness, $t_f$, of the flexible portion in addition to the length of the rigid portion, $l_r$.
% 
The selected flexures have the following properties: $l_f = 8 \text{ mm}$, $t_f = 0.3 \text{ mm}$, and $l_r = 9.92 \text{ mm}$.
%%%%%%%%%%%%%%%%%%%%%%%%%%%%%%%%%%%%%%%%%%%%%%%%%%%%

%%%%%%%%%%%%%%%%%%%%%%%%%%%%%%%%%%%%%%%%%%%%%%%%%%%%
The low-friction springed plate adjusts the applied normal force by means of a screw and a grounded hex nut. 
%
Shoulder bolts and bushings were used to minimize motion of the plate along or about axes orthogonal to the screw's axis. 
%
The spring used was measured to have a stiffness of 0.57 N/mm.
%
The low-friction surface was smoothed PLA from the bed-facing side of a 3D printed part. 
%
The coefficient of kinetic friction between the acrylic sheet used in our experiments and this smoothed PLA surface was calculated to be 0.18 and is assumed to be negligible in our model.
%%%%%%%%%%%%%%%%%%%%%%%%%%%%%%%%%%%%%%%%%%%%%%%%%%%%

%%%%%%%%%%%%%%%%%%%%%%%%%%%%%%%%%%%%%%%%%%%%%%%%%%%%
