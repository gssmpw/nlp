% {\color{blue}
% OUTLINE:
% \begin{itemize}
%     \item Aside from verifying model, we also wanted to see what parts it could pick up
%     \item Describe parallel mechanism design
% \end{itemize}
% }

A proof-of-concept parallel jaw gripper was designed where each jaw was outfitted with a vibrating surface as shown in Fig. \ref{figure: cover photo}. 
%
We tested the device's vertical transport ability across a range of different parts as shown in Fig. \ref{figure: grasped parts}.
%
\begin{figure}[t]
    \centering
    \includegraphics[width=0.45\textwidth]{figures/gripper/grasped_parts.jpg}
    \caption{Parts transported by the gripper. From left to right the parts are: acrylic plate, tweezers, bundled braided cable, PLA triangular prism, 80/20 corner bracket and plate, Post-Its, pliers, scissors, bearing, wrench, Dynamixel, hex keys, bottle of fluid, computer mouse, soft cover book, box cutter. The parts ranged in mass from 17 - 169 grams (from left to right, and top to bottom).}
    \label{figure: grasped parts}
    \vspace{\shift}
\end{figure}
%
The gripper was able to transport each part, with operator-tuned sawtooth waveform parameters.
%
We refer the readers to the accompanying video for transportation of these parts.
%%%%%%%%%%%%%%%%%%%%%%%%%%%%%%%%%%%%%%%%%%%%%%%%%%%%

%%%%%%%%%%%%%%%%%%%%%%%%%%%%%%%%%%%%%%%%%%%%%%%%%%%%
From our testing we found that compliant parts as well as light parts were difficult to transport. 
%
For the former, the surface could neither accelerate up nor slip behind the part. 
% 
Regarding the latter, slight misalignment between the two vibrating surfaces needed to be corrected by applied normal forces, and since we know from Fig. \ref{figure: v_ave} that excessively large normal forces prevent any part motion, we hypothesize that the normal forces required to correct alignment easily exceeded the motion threshold because of their small mass.