% Imbuing robots with general in-hand manipulation capabilities is one of many challenges keeping robots from further integrating into our day-to-day environments. 
% %
% There have been numerous efforts over the years in both the hardware and manipulation strategies used, ranging from finger-gaiting with anthropomorphic hands~\cite{andrychowicz2020learning} to extrinsic manipulation with parallel grippers~\cite{dafle2014extrinsic, chavan2020planar, vina2015slip}.
%%%%%%%%%%%%%%%%%%%%%%%%%%%%%%%%%%%%%%%%%%%%%%%%%%%%

%%%%%%%%%%%%%%%%%%%%%%%%%%%%%%%%%%%%%%%%%%%%%%%%%%%%
% A particular subtopic of interest concerns placing additional degrees of freedom (DoFs) at the fingertips to augment the capability of simpler grippers (i.e., parallel grippers). 
% %
% Simpler embodiments with cleverly placed DoFs can balance functionality with control complexity. 
% %
% Such examples include rolling fingers or fingers with conveyors~\cite{yuan2020design, cai2023hand}, brakes with passive motion mechanisms~\cite{taylor2020pnugrip}, as well as vibrating fingers~\cite{dafle2014extrinsic, nahum2022robotic}. 
%%%%%%%%%%%%%%%%%%%%%%%%%%%%%%%%%%%%%%%%%%%%%%%%%%%%

%%%%%%%%%%%%%%%%%%%%%%%%%%%%%%%%%%%%%%%%%%%%%%%%%%%%
Using vibrations to manipulate parts is not a new concept and has been used for non-prehensile motion for decades. 
%
Popular implementations include vibratory part feeders, where parts experience sequential sticking, slipping, non-contact ``hopping'', and landing phases caused by a combination of in and out-of-plane vibrations of the drive surface~\cite{lim1997conveying}. 
%
However, this ``hopping'' phase may be unsuitable for precise in-hand manipulation tasks. 
%
This limitation led Reznik and Canny to develop their ``Coulomb Pump'' based on only 1D vibrations and an always sliding assumption~\cite{reznik1998coulomb}. 
%
Parts moved due to \textit{time-asymmetry}, where the drive surface spent more time moving in the desired transport direction across one oscillation cycle.
%
Quaid built upon the ``Coulomb Pump'' by removing the always sliding assumption and noting that part transport could be achieved by a simple two-phase cycle: a slow \textit{sticking} phase in which the surface and part moved in the desired motion direction, followed by a high backward acceleration \textit{slipping} phase where the surface slips behind the part~\cite{quaid1999feeder}.
%
Umbanhowar and Lynch took Quaid's work further by deriving acceleration profiles that maximized a part's average velocity over one cycle~\cite{umbanhowar2008optimal}.
%
Instead of Quaid's two phases, Umbanhowar and Lynch added a third phase where the surface caught up to the part before its velocity became negative.
% 
They also described how part speed could be improved by simultaneously oscillating in the vertical direction with the constraint that contact be maintained.
%%%%%%%%%%%%%%%%%%%%%%%%%%%%%%%%%%%%%%%%%%%%%%%%%%%%

%
\begin{figure}[t]
    \centering
    \includegraphics[width=0.44\textwidth]{figures/introduction/cover_photo.jpg}
    \caption{Vibrating surfaces lift a part against gravity.}
    \label{figure: cover photo}
    \vspace{\shift}
\end{figure}
%

%%%%%%%%%%%%%%%%%%%%%%%%%%%%%%%%%%%%%%%%%%%%%%%%%%%%
A recent paper by Nahum used this idea by cleverly leveraging the smoothly changing in/out-of-plane vibration direction of an eccentric rotating mass (ERM) motor in conjunction with a grasp on the part to enforce contact~\cite{nahum2022robotic}. 
%
Thus, the rotating vibration direction changed the normal force, and hence the friction force, making it easier to transition from sticking to slipping as well as stay in the sticking phase. 
%
The authors noted that their device could work against gravity up to 60$^\circ$ from the horizontal, and that a stronger motor would be needed above this inclination.
%%%%%%%%%%%%%%%%%%%%%%%%%%%%%%%%%%%%%%%%%%%%%%%%%%%%

\subsection{Upward Vertical Vibratory Transport is Difficult}

%%%%%%%%%%%%%%%%%%%%%%%%%%%%%%%%%%%%%%%%%%%%%%%%%%%%
For simplicity's sake, this paper will focus on moving a part vertically against gravity using a 1D vibratory surface, with the vibrations in the direction of part movement.
%
Given a desired part motion, vertical transport is inherently more difficult than its horizontal counterpart as gravity biases the friction cone such that it is easier for the part to slip downwards rather than upwards. 
%
Take the two phases described by Quaid, the \textit{sticking} phase and the \textit{slipping} phase. 
%
This skewed friction cone limits the peak upwards surface acceleration during sticking and increases the necessary downward surface acceleration to transition to slipping.
%
Gravity compounds the part's downward acceleration during slipping.
%%%%%%%%%%%%%%%%%%%%%%%%%%%%%%%%%%%%%%%%%%%%%%%%%%%%

%%%%%%%%%%%%%%%%%%%%%%%%%%%%%%%%%%%%%%%%%%%%%%%%%%%%
Despite these difficulties, we show that it is possible to achieve purely vertical transport, i.e., against gravity, using only 1D vibrations (Fig. \ref{figure: cover photo}). 
%
This paper is organized as follows. 
%
In Section \ref{section: dynamics} we give the equations of motion for vertical vibratory transport, we show why upward motion is difficult to achieve, which conditions need to be met and why we chose to use rigid body impacts to create high accelerations.
%
We also derive the optimal vibration waveform by showing an equivalence with horizontal transport.
%
Section \ref{section: design of experimental setup} describes the design of a device capable of achieving vertical transport.
%
Section \ref{section: model validation} describes the experimental setup used to validate our dynamical model.
%
A gripper design is presented and preliminary grasping transport tests are reported in Section \ref{section: gripper}.
%
We then conclude with a comparison of our device  with direct drive devices and limitations in Section \ref{section: discussion}.
%%%%%%%%%%%%%%%%%%%%%%%%%%%%%%%%%%%%%%%%%%%%%%%%%%%%