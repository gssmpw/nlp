We validated our model by comparing the simulated part position with the experimental part position data. 
%
Velocity profiles of the surface were computed from the tracked position data using finite differences, and then used as the simulated surface velocity.
%
We found that the finite difference calculated velocity in Tracker was smoother than the position-differentiated velocity in Simulink/MATLAB, which was much more discretized.
%
This was important for stabilizing the sticking-slipping transitions that are dependent on the relative velocities.
%%%%%%%%%%%%%%%%%%%%%%%%%%%%%%%%%%%%%%%%%%%%%%%%%%%%

%%%%%%%%%%%%%%%%%%%%%%%%%%%%%%%%%%%%%%%%%%%%%%%%%%%%
Ten different transport experiments on a 9 gram acrylic plate were performed across a range of drive frequencies (9.7 to 50.7 Hz), amplitudes (4.7 to 10.6 $V_{p\text{-}p}$), offsets (-2.5 to 2.6 $V$), and normal forces (48 to 206 grams, equivalent to roughly 4 - 16 times $\mu_s m_p g$).
%
Values were chosen based on empirically observed limits of when transport would and would not occur and limits of the drive circuitry. This allowed us to test the model's capability to reproduce both successful and unsuccessful transport cases.
% 
We used two unknown parameters, $\mu_k$ between the surface and the part, and $F_n$, to fine tune the model both by hand, to narrow in on an acceptable range of a given parameter, and by using a population-based stochastic optimizer (MATLAB's \texttt{particleswarm}) for more precision. 
%
The optimizer sought to minimize the sum of the per trial mean part position error between the experiment and the model.

\begin{figure}[b]
    \vspace{\shift}
    \centering
    \includegraphics[width=0.5\textwidth]{figures/model_validation/part_position_comparison_with_surface_updated.jpg}
    \caption{Experimentally measured part position is shown as the black dotted line, with the simulated part position shown in red. The surface motion is shown in blue.}
    \label{figure: model validation}
\end{figure}
%
The average part position error was 0.16 mm, or $12\%$ when normalizing by each trial's maximum experimentally measured part position. 
%
The fit normal force deviated from its measured value by a mean of 34\%. 
%
A majority of the fit normal forces were higher than their experimentally measured counterparts, which we believe is due to the unmodeled friction forces from the springed plate on the part.
%
We also find the kinetic coefficient of friction between the surface and part to be $\mu_k=0.62$ (measured $\mu_k = 0.59$); a measured $\mu_s=0.72$ was used in the model.
%
A sample trial is shown in Fig. \ref{figure: model validation}.
%
The surface position is similar in shape to the quadratic curve in Quaid~\cite{quaid1999feeder}.
%
Given the strong agreement between the model and experimental data across all trials, we believe that Coulomb friction and the presented model are sufficient to capture the system behavior.