%%
%% This is file `sample-sigconf-authordraft.tex',
%% generated with the docstrip utility.
%%
%% The original source files were:
%%
%% samples.dtx  (with options: `all,proceedings,bibtex,authordraft')
%% 
%% IMPORTANT NOTICE:
%% 
%% For the copyright see the source file.
%% 
%% Any modified versions of this file must be renamed
%% with new filenames distinct from sample-sigconf-authordraft.tex.
%% 
%% For distribution of the original source see the terms
%% for copying and modification in the file samples.dtx.
%% 
%% This generated file may be distributed as long as the
%% original source files, as listed above, are part of the
%% same distribution. (The sources need not necessarily be
%% in the same archive or directory.)
%%
%%
%% Commands for TeXCount
%TC:macro \cite [option:text,text]
%TC:macro \citep [option:text,text]
%TC:macro \citet [option:text,text]
%TC:envir table 0 1
%TC:envir table* 0 1
%TC:envir tabular [ignore] word
%TC:envir displaymath 0 word
%TC:envir math 0 word
%TC:envir comment 0 0
%%
%%
%% The first command in your LaTeX source must be the \documentclass
%% command.
%%
%% For submission and review of your manuscript please change the
%% command to \documentclass[manuscript, screen, review]{acmart}.
%%
%% When submitting camera ready or to TAPS, please change the command
%% to \documentclass[sigconf]{acmart} or whichever template is required
%% for your publication.
%%
%%


\documentclass[sigconf]{acmart}

\copyrightyear{2025}
\acmYear{2025}
\setcopyright{acmlicensed}
\acmConference[WWW '25] {Proceedings of the ACM Web Conference 2025}{April 28--May 2, 2025}{Sydney, NSW, Australia.}
\acmBooktitle{Proceedings of the ACM Web Conference 2025 (WWW '25), April 28--May 2, 2025, Sydney, NSW, Australia}
\acmISBN{979-8-4007-1274-6/25/04}
\acmDOI{10.1145/3696410.3714617}
% 1 Authors, replace the red X's with your assigned DOI string during the rightsreview eform process.
% 2 Your DOI link will become active when the proceedings appears in the DL.
% 3 Retain the DOI string between the curly braces for uploading your presentation video.

\settopmatter{printacmref=true}




\usepackage{graphicx}
\usepackage{enumitem}
\usepackage{multirow} 
\usepackage{fancyhdr}
\usepackage{caption}
\pagestyle{empty}
\usepackage{array}
\usepackage{xcolor}
\newcolumntype{B}{>{\columncolor{blue!10}\bfseries\color{blue}}c}
\usepackage{bm}
\usepackage{colortbl}
\usepackage{threeparttable}
\usepackage{afterpage}
\usepackage{pgfplots}  
\usepackage{tablefootnote}
\usepackage{amsmath}
\usepackage{tikz}
\usepackage{pgfplots}
\usepackage{graphicx}
\usepackage{subcaption}
\usepackage{subcaption}
\usepackage{pgfplotstable}  
\usepackage{graphicx}  
\usepackage{adjustbox}
\usepackage{setspace} 
% 定义颜色映射,这里使用红色渐变  
\pgfplotsset{  
    colormap={gray}{  
        color(0)=(white);  
        color(1)=(black)  
    }  
}  
\captionsetup[subfigure]{justification=centering}
\usetikzlibrary{shadows}
% 推荐使用的包
% \usepackage{scalefnt} %用于设置字体
% \usepgfplotslibrary{fillbetween}
% \usepackage{tgtermes}

\definecolor{lightblue}{rgb}{0.94, 0.94, 1}
\definecolor{lightorange}{rgb}{1, 0.99, 0.93}
\definecolor{lightpink}{rgb}{1, 0.93, 0.93}
\definecolor{lightgreen}{rgb}{0.92, 0.98, 0.92}

\newcommand{\colorrect}[1]{\textcolor{#1}{\ding{110}}}


% 预先定义一些colors
\definecolor{color1}{RGB}{120,159,124}
\definecolor{color2}{RGB}{199,115,100}
\definecolor{color3}{RGB}{252,104,58}
\definecolor{color4}{RGB}{250,168,68}
\definecolor{color5}{RGB}{121,137,184}
\definecolor{color6}{RGB}{167,98,236}
\definecolor{color8}{RGB}{137,137,137}%%灰色
\definecolor{color9}{RGB}{90,82,252}
%%
%% end of the preamble, start of the body of the document source.
\begin{document}

%%
%% The "title" command has an optional parameter,
%% allowing the author to define a "short title" to be used in page headers.
% \title{Sherlock: Towards Multi-scene Video Abnormal Event Extraction and Localization via a Global-local Spatial-senstive LLM}



\title[Sherlock]{\texorpdfstring{\begin{minipage}[b]{0.01\textwidth}
  \raisebox{1mm}{\includegraphics[scale=0.145]{image/logo.png}}
\end{minipage}\hspace{10mm}%
\begin{minipage}[b]{0.89\textwidth}
\begin{center}
   Sherlock: Towards Multi-scene Video Abnormal Event Extraction and Localization via a Global-local Spatial-sensitive LLM
\end{center}
\end{minipage}}{}}

%%
%% The "author" command and its associated commands are used to define
%% the authors and their affiliations.
%% Of note is the shared affiliation of the first two authors, and the
%% "authornote" and "authornotemark" commands
%% used to denote shared contribution to the research.

\author{Junxiao Ma}
\email{jxma0711@stu.suda.edu.cn}
\affiliation{
  \institution{School of Computer Science and Technology, Soochow University}
  \city{Suzhou}
  \country{China}
}

\author{Jingjing Wang}
\authornote{Corresponding Author: Jingjing Wang.}
\email{djingwang@suda.edu.cn}
\affiliation{
  \institution{School of Computer Science and Technology, Soochow University}
  \city{Suzhou}
  \country{China}
}


\author{Jiamin Luo}
\email{20204027003@stu.suda.edu.cn}
\affiliation{
  \institution{School of Computer Science and Technology, Soochow University}
  \city{Suzhou}
  \country{China}
}

\author{Peiying Yu}
% \authornote{equal contribution}
\email{20244227007@stu.suda.edu.cn}
\affiliation{
  \institution{School of Computer Science and Technology, Soochow University}
  \city{Suzhou}
  \country{China}
}

\author{Guodong Zhou}
\email{gdzhou@suda.edu.cn}
\affiliation{
  \institution{School of Computer Science and Technology, Soochow University}
  \city{Suzhou}
  \country{China}
}
%%
%% By default, the full list of authors will be used in the page
%% headers. Often, this list is too long, and will overlap
%% other information printed in the page headers. This command allows
%% the author to define a more concise list
%% of authors' names for this purpose.
\renewcommand{\shortauthors}{Junxiao Ma, Jingjing Wang, Jiamin Luo, Peiying Yu \& Guodong Zhou}

%%
%% By default, the full list of authors will be used in the page
%% headers. Often, this list is too long, and will overlap
%% other information printed in the page headers. This command allows
%% the author to define a more concise list
%% of authors' names for this purpose.

%%
%% The abstract is a short summary of the work to be presented in the
%% article.
\begin{abstract}  
Test time scaling is currently one of the most active research areas that shows promise after training time scaling has reached its limits.
Deep-thinking (DT) models are a class of recurrent models that can perform easy-to-hard generalization by assigning more compute to harder test samples.
However, due to their inability to determine the complexity of a test sample, DT models have to use a large amount of computation for both easy and hard test samples.
Excessive test time computation is wasteful and can cause the ``overthinking'' problem where more test time computation leads to worse results.
In this paper, we introduce a test time training method for determining the optimal amount of computation needed for each sample during test time.
We also propose Conv-LiGRU, a novel recurrent architecture for efficient and robust visual reasoning. 
Extensive experiments demonstrate that Conv-LiGRU is more stable than DT, effectively mitigates the ``overthinking'' phenomenon, and achieves superior accuracy.
\end{abstract}  

%%
%% The code below is generated by the tool at http://dl.acm.org/ccs.cfm.
%% Please copy and paste the code instead of the example below.
%%

\begin{CCSXML}
<ccs2012>
   <concept>
       <concept_id>10010147.10010178</concept_id>
       <concept_desc>Computing methodologies~Artificial intelligence</concept_desc>
       <concept_significance>500</concept_significance>
       </concept>
 </ccs2012>
\end{CCSXML}

\ccsdesc[500]{Computing methodologies~Artificial intelligence}
%%
%% Keywords. The author(s) should pick words that accurately describe
%% the work being presented. Separate the keywords with commas.

\keywords{Multi-scene Video, Video Abnormal Event, Spatial-sensitive LLM}
%% A "teaser" image appears between the author and affiliation
%% information and the body of the document, and typically spans the
%% page.
\begin{teaserfigure}
\vspace{-0.3 cm}
\setlength{\abovecaptionskip}{0.5 ex}
% \setlength{\belowcaptionskip}{-1 ex}
  \includegraphics[width=\textwidth]{image/abstract.pdf}
  \caption{(a) and (b) illustrate two surveillance video examples for our M-VAE task and Sherlock model in two scenes (Street and Residence). Sherlock precisely generates the abnormal event quadruples and their corresponding timestamps. (c) presents a circular ratio diagram illustrating different spatial information. From (c), we observe that the global spatial information and the local spatial information (i.e., action, object relation, and background) in our M-VAE dataset are imbalanced.}
  \label{fig:abstract}
\end{teaserfigure}

% \received{20 February 2007}
% \received[revised]{12 March 2009}
% \received[accepted]{5 June 2009}

%% This command processes the author and affiliation and title
%% information and builds the first part of the formatted document.
\maketitle
\section{Introduction}
\label{sec:introduction}
The business processes of organizations are experiencing ever-increasing complexity due to the large amount of data, high number of users, and high-tech devices involved \cite{martin2021pmopportunitieschallenges, beerepoot2023biggestbpmproblems}. This complexity may cause business processes to deviate from normal control flow due to unforeseen and disruptive anomalies \cite{adams2023proceddsriftdetection}. These control-flow anomalies manifest as unknown, skipped, and wrongly-ordered activities in the traces of event logs monitored from the execution of business processes \cite{ko2023adsystematicreview}. For the sake of clarity, let us consider an illustrative example of such anomalies. Figure \ref{FP_ANOMALIES} shows a so-called event log footprint, which captures the control flow relations of four activities of a hypothetical event log. In particular, this footprint captures the control-flow relations between activities \texttt{a}, \texttt{b}, \texttt{c} and \texttt{d}. These are the causal ($\rightarrow$) relation, concurrent ($\parallel$) relation, and other ($\#$) relations such as exclusivity or non-local dependency \cite{aalst2022pmhandbook}. In addition, on the right are six traces, of which five exhibit skipped, wrongly-ordered and unknown control-flow anomalies. For example, $\langle$\texttt{a b d}$\rangle$ has a skipped activity, which is \texttt{c}. Because of this skipped activity, the control-flow relation \texttt{b}$\,\#\,$\texttt{d} is violated, since \texttt{d} directly follows \texttt{b} in the anomalous trace.
\begin{figure}[!t]
\centering
\includegraphics[width=0.9\columnwidth]{images/FP_ANOMALIES.png}
\caption{An example event log footprint with six traces, of which five exhibit control-flow anomalies.}
\label{FP_ANOMALIES}
\end{figure}

\subsection{Control-flow anomaly detection}
Control-flow anomaly detection techniques aim to characterize the normal control flow from event logs and verify whether these deviations occur in new event logs \cite{ko2023adsystematicreview}. To develop control-flow anomaly detection techniques, \revision{process mining} has seen widespread adoption owing to process discovery and \revision{conformance checking}. On the one hand, process discovery is a set of algorithms that encode control-flow relations as a set of model elements and constraints according to a given modeling formalism \cite{aalst2022pmhandbook}; hereafter, we refer to the Petri net, a widespread modeling formalism. On the other hand, \revision{conformance checking} is an explainable set of algorithms that allows linking any deviations with the reference Petri net and providing the fitness measure, namely a measure of how much the Petri net fits the new event log \cite{aalst2022pmhandbook}. Many control-flow anomaly detection techniques based on \revision{conformance checking} (hereafter, \revision{conformance checking}-based techniques) use the fitness measure to determine whether an event log is anomalous \cite{bezerra2009pmad, bezerra2013adlogspais, myers2018icsadpm, pecchia2020applicationfailuresanalysispm}. 

The scientific literature also includes many \revision{conformance checking}-independent techniques for control-flow anomaly detection that combine specific types of trace encodings with machine/deep learning \cite{ko2023adsystematicreview, tavares2023pmtraceencoding}. Whereas these techniques are very effective, their explainability is challenging due to both the type of trace encoding employed and the machine/deep learning model used \cite{rawal2022trustworthyaiadvances,li2023explainablead}. Hence, in the following, we focus on the shortcomings of \revision{conformance checking}-based techniques to investigate whether it is possible to support the development of competitive control-flow anomaly detection techniques while maintaining the explainable nature of \revision{conformance checking}.
\begin{figure}[!t]
\centering
\includegraphics[width=\columnwidth]{images/HIGH_LEVEL_VIEW.png}
\caption{A high-level view of the proposed framework for combining \revision{process mining}-based feature extraction with dimensionality reduction for control-flow anomaly detection.}
\label{HIGH_LEVEL_VIEW}
\end{figure}

\subsection{Shortcomings of \revision{conformance checking}-based techniques}
Unfortunately, the detection effectiveness of \revision{conformance checking}-based techniques is affected by noisy data and low-quality Petri nets, which may be due to human errors in the modeling process or representational bias of process discovery algorithms \cite{bezerra2013adlogspais, pecchia2020applicationfailuresanalysispm, aalst2016pm}. Specifically, on the one hand, noisy data may introduce infrequent and deceptive control-flow relations that may result in inconsistent fitness measures, whereas, on the other hand, checking event logs against a low-quality Petri net could lead to an unreliable distribution of fitness measures. Nonetheless, such Petri nets can still be used as references to obtain insightful information for \revision{process mining}-based feature extraction, supporting the development of competitive and explainable \revision{conformance checking}-based techniques for control-flow anomaly detection despite the problems above. For example, a few works outline that token-based \revision{conformance checking} can be used for \revision{process mining}-based feature extraction to build tabular data and develop effective \revision{conformance checking}-based techniques for control-flow anomaly detection \cite{singh2022lapmsh, debenedictis2023dtadiiot}. However, to the best of our knowledge, the scientific literature lacks a structured proposal for \revision{process mining}-based feature extraction using the state-of-the-art \revision{conformance checking} variant, namely alignment-based \revision{conformance checking}.

\subsection{Contributions}
We propose a novel \revision{process mining}-based feature extraction approach with alignment-based \revision{conformance checking}. This variant aligns the deviating control flow with a reference Petri net; the resulting alignment can be inspected to extract additional statistics such as the number of times a given activity caused mismatches \cite{aalst2022pmhandbook}. We integrate this approach into a flexible and explainable framework for developing techniques for control-flow anomaly detection. The framework combines \revision{process mining}-based feature extraction and dimensionality reduction to handle high-dimensional feature sets, achieve detection effectiveness, and support explainability. Notably, in addition to our proposed \revision{process mining}-based feature extraction approach, the framework allows employing other approaches, enabling a fair comparison of multiple \revision{conformance checking}-based and \revision{conformance checking}-independent techniques for control-flow anomaly detection. Figure \ref{HIGH_LEVEL_VIEW} shows a high-level view of the framework. Business processes are monitored, and event logs obtained from the database of information systems. Subsequently, \revision{process mining}-based feature extraction is applied to these event logs and tabular data input to dimensionality reduction to identify control-flow anomalies. We apply several \revision{conformance checking}-based and \revision{conformance checking}-independent framework techniques to publicly available datasets, simulated data of a case study from railways, and real-world data of a case study from healthcare. We show that the framework techniques implementing our approach outperform the baseline \revision{conformance checking}-based techniques while maintaining the explainable nature of \revision{conformance checking}.

In summary, the contributions of this paper are as follows.
\begin{itemize}
    \item{
        A novel \revision{process mining}-based feature extraction approach to support the development of competitive and explainable \revision{conformance checking}-based techniques for control-flow anomaly detection.
    }
    \item{
        A flexible and explainable framework for developing techniques for control-flow anomaly detection using \revision{process mining}-based feature extraction and dimensionality reduction.
    }
    \item{
        Application to synthetic and real-world datasets of several \revision{conformance checking}-based and \revision{conformance checking}-independent framework techniques, evaluating their detection effectiveness and explainability.
    }
\end{itemize}

The rest of the paper is organized as follows.
\begin{itemize}
    \item Section \ref{sec:related_work} reviews the existing techniques for control-flow anomaly detection, categorizing them into \revision{conformance checking}-based and \revision{conformance checking}-independent techniques.
    \item Section \ref{sec:abccfe} provides the preliminaries of \revision{process mining} to establish the notation used throughout the paper, and delves into the details of the proposed \revision{process mining}-based feature extraction approach with alignment-based \revision{conformance checking}.
    \item Section \ref{sec:framework} describes the framework for developing \revision{conformance checking}-based and \revision{conformance checking}-independent techniques for control-flow anomaly detection that combine \revision{process mining}-based feature extraction and dimensionality reduction.
    \item Section \ref{sec:evaluation} presents the experiments conducted with multiple framework and baseline techniques using data from publicly available datasets and case studies.
    \item Section \ref{sec:conclusions} draws the conclusions and presents future work.
\end{itemize}
\begin{figure*}[t]
\vspace{-0.3cm}
\setlength{\abovecaptionskip}{0.5 ex}
\setlength{\belowcaptionskip}{-3 ex}
  \centering
  \includegraphics[width=\textwidth]{image/model.pdf}
  \centering
  \caption{The overall framework of Sherlock. It consists of a Global-local Spatial-enhanced MoE (GSM) Module and a Spatial Imbalance Regulator (SIR). The SIR exerts a direct influence on the output weights of the expert gate.}
  \Description{our method xxx}
  \label{fig:model}
\end{figure*}
\section{RELATED WORK}
\label{sec:relatedwork}
In this section, we describe the previous works related to our proposal, which are divided into two parts. In Section~\ref{sec:relatedwork_exoplanet}, we present a review of approaches based on machine learning techniques for the detection of planetary transit signals. Section~\ref{sec:relatedwork_attention} provides an account of the approaches based on attention mechanisms applied in Astronomy.\par

\subsection{Exoplanet detection}
\label{sec:relatedwork_exoplanet}
Machine learning methods have achieved great performance for the automatic selection of exoplanet transit signals. One of the earliest applications of machine learning is a model named Autovetter \citep{MCcauliff}, which is a random forest (RF) model based on characteristics derived from Kepler pipeline statistics to classify exoplanet and false positive signals. Then, other studies emerged that also used supervised learning. \cite{mislis2016sidra} also used a RF, but unlike the work by \citet{MCcauliff}, they used simulated light curves and a box least square \citep[BLS;][]{kovacs2002box}-based periodogram to search for transiting exoplanets. \citet{thompson2015machine} proposed a k-nearest neighbors model for Kepler data to determine if a given signal has similarity to known transits. Unsupervised learning techniques were also applied, such as self-organizing maps (SOM), proposed \citet{armstrong2016transit}; which implements an architecture to segment similar light curves. In the same way, \citet{armstrong2018automatic} developed a combination of supervised and unsupervised learning, including RF and SOM models. In general, these approaches require a previous phase of feature engineering for each light curve. \par

%DL is a modern data-driven technology that automatically extracts characteristics, and that has been successful in classification problems from a variety of application domains. The architecture relies on several layers of NNs of simple interconnected units and uses layers to build increasingly complex and useful features by means of linear and non-linear transformation. This family of models is capable of generating increasingly high-level representations \citep{lecun2015deep}.

The application of DL for exoplanetary signal detection has evolved rapidly in recent years and has become very popular in planetary science.  \citet{pearson2018} and \citet{zucker2018shallow} developed CNN-based algorithms that learn from synthetic data to search for exoplanets. Perhaps one of the most successful applications of the DL models in transit detection was that of \citet{Shallue_2018}; who, in collaboration with Google, proposed a CNN named AstroNet that recognizes exoplanet signals in real data from Kepler. AstroNet uses the training set of labelled TCEs from the Autovetter planet candidate catalog of Q1–Q17 data release 24 (DR24) of the Kepler mission \citep{catanzarite2015autovetter}. AstroNet analyses the data in two views: a ``global view'', and ``local view'' \citep{Shallue_2018}. \par


% The global view shows the characteristics of the light curve over an orbital period, and a local view shows the moment at occurring the transit in detail

%different = space-based

Based on AstroNet, researchers have modified the original AstroNet model to rank candidates from different surveys, specifically for Kepler and TESS missions. \citet{ansdell2018scientific} developed a CNN trained on Kepler data, and included for the first time the information on the centroids, showing that the model improves performance considerably. Then, \citet{osborn2020rapid} and \citet{yu2019identifying} also included the centroids information, but in addition, \citet{osborn2020rapid} included information of the stellar and transit parameters. Finally, \citet{rao2021nigraha} proposed a pipeline that includes a new ``half-phase'' view of the transit signal. This half-phase view represents a transit view with a different time and phase. The purpose of this view is to recover any possible secondary eclipse (the object hiding behind the disk of the primary star).


%last pipeline applies a procedure after the prediction of the model to obtain new candidates, this process is carried out through a series of steps that include the evaluation with Discovery and Validation of Exoplanets (DAVE) \citet{kostov2019discovery} that was adapted for the TESS telescope.\par
%



\subsection{Attention mechanisms in astronomy}
\label{sec:relatedwork_attention}
Despite the remarkable success of attention mechanisms in sequential data, few papers have exploited their advantages in astronomy. In particular, there are no models based on attention mechanisms for detecting planets. Below we present a summary of the main applications of this modeling approach to astronomy, based on two points of view; performance and interpretability of the model.\par
%Attention mechanisms have not yet been explored in all sub-areas of astronomy. However, recent works show a successful application of the mechanism.
%performance

The application of attention mechanisms has shown improvements in the performance of some regression and classification tasks compared to previous approaches. One of the first implementations of the attention mechanism was to find gravitational lenses proposed by \citet{thuruthipilly2021finding}. They designed 21 self-attention-based encoder models, where each model was trained separately with 18,000 simulated images, demonstrating that the model based on the Transformer has a better performance and uses fewer trainable parameters compared to CNN. A novel application was proposed by \citet{lin2021galaxy} for the morphological classification of galaxies, who used an architecture derived from the Transformer, named Vision Transformer (VIT) \citep{dosovitskiy2020image}. \citet{lin2021galaxy} demonstrated competitive results compared to CNNs. Another application with successful results was proposed by \citet{zerveas2021transformer}; which first proposed a transformer-based framework for learning unsupervised representations of multivariate time series. Their methodology takes advantage of unlabeled data to train an encoder and extract dense vector representations of time series. Subsequently, they evaluate the model for regression and classification tasks, demonstrating better performance than other state-of-the-art supervised methods, even with data sets with limited samples.

%interpretation
Regarding the interpretability of the model, a recent contribution that analyses the attention maps was presented by \citet{bowles20212}, which explored the use of group-equivariant self-attention for radio astronomy classification. Compared to other approaches, this model analysed the attention maps of the predictions and showed that the mechanism extracts the brightest spots and jets of the radio source more clearly. This indicates that attention maps for prediction interpretation could help experts see patterns that the human eye often misses. \par

In the field of variable stars, \citet{allam2021paying} employed the mechanism for classifying multivariate time series in variable stars. And additionally, \citet{allam2021paying} showed that the activation weights are accommodated according to the variation in brightness of the star, achieving a more interpretable model. And finally, related to the TESS telescope, \citet{morvan2022don} proposed a model that removes the noise from the light curves through the distribution of attention weights. \citet{morvan2022don} showed that the use of the attention mechanism is excellent for removing noise and outliers in time series datasets compared with other approaches. In addition, the use of attention maps allowed them to show the representations learned from the model. \par

Recent attention mechanism approaches in astronomy demonstrate comparable results with earlier approaches, such as CNNs. At the same time, they offer interpretability of their results, which allows a post-prediction analysis. \par


\section{Temporal Representation Alignment}
\label{sec:approach}

When training a series of short-horizon goal-reaching and instruction-following tasks, our goal is to learn a representation space such that our policy can generalize to a new (long-horizon) task that can be viewed as a sequence of known subtasks.
We propose to structure this representation space by aligning the representations of states, goals, and language in a way that is more amenable to compositional generalization.

\paragraph{Notation.}
We take the setting of a goal- and language-conditioned MDP $\cM$ with state space $\cS$, continuous action space $\cA \subseteq (0,1)^{d_{\cA}}$, initial state distribution $p_0$, dynamics $\p(s'\mid s,a)$, discount factor $\gamma$, and language task distribution $p_{\ell}$.
A policy $\pi(a\mid s)$ maps states to a distribution over actions. We inductively define the $k$-step (action-conditioned) policy visitation distribution as:
\begin{align*}
    p^{\pi}_{1}(s_{1} \mid s_1, a_{1})
    &\triangleq p(s_1 \mid s_1, a_1),\\
    p^{\pi}_{k+1}(s_{k+1} \mid s_1, a_1)
    &\triangleq \nonumber\\*
      &\mspace{-120mu} \int_{\cA}\int_{\cS} p(s_{k+1} \mid s,a) \dd p^{\pi}_{k}(s \mid s_{1},a_1) \dd
        \pi(a \mid s)\\
    p^{\pi}_{k+t}(s_{k+t} \mid s_t,a_t)
    &\triangleq p^{\pi}(s_{k} \mid s_1, a_1) . \eqmark
        \label{eq:successor_distribution}
\end{align*}
Then, the discounted state visitation distribution can be defined as the distribution over $s^{+}$\llap, the state reached after $K\sim \operatorname{Geom}(1-\gamma)$ steps:
\begin{equation}
    p^{\pi}_{\gamma}(s^{+}  \mid  s,a) \triangleq \sum_{k=0}^{\infty} \gamma^{k} p^{\pi}_{k}(s^{+} \mid s,a).
    \label{eq:discounted_state_visitation}
\end{equation}

We assume access to a dataset of expert demonstrations $\cD = \{\tau_{i},\ell_i\}_{i=1}^{K}$, where each trajectory
\begin{equation}
    \tau_{i} = \{s_{t,i},a_{t,i}\}_{t=1}^{H} \in \cS \times \cA
    \label{eq:trajectory}
\end{equation}
is gathered by an expert policy $\expert$, and is then annotated with $p_{\ell}(\ell_{i} \mid s_{1,i}, s_{H,i})$.
Our aim is to learn a policy $\pi$ that can select actions conditioned on a new language instruction $\ell$.
As in prior work~\citep{walke2023bridgedata}, we handle the continuous action space by representing both our policy and the expert policy as an isotropic Gaussian with fixed variance; we will equivalently write $\pi(a\mid s, \varphi)$ or denote the mode as $\hat{a} = \pi(s,\varphi)$ for a task $\varphi$.

\begin{rebuttal}
    \subsection{Representations for Reaching Distant Goals}
    \label{sec:reaching_goals}

    We learn a goal-conditioned policy $\pi(a\mid s,g)$ that selects actions to reach a goal $g$ from expert demonstrations with behavioral cloning.
    Suppose we directly selected actions to imitate the expert on two trajectories in $\cD$:
    
    \begin{equation}
        \mspace{-100mu}\begin{tikzcd}[remember picture,sep=small]
            s_1 \rar & s_2 \rar  & \ldots \rar & s_{H} \rar & w      \quad \\
            w \rar   & s_1' \rar & \ldots \rar & s_{H}' \rar & g\quad
        \end{tikzcd}
        \begin{tikzpicture}[remember picture,overlay] \coordinate (a) at (\tikzcdmatrixname-1-5.north east);
            \coordinate (b) at (\tikzcdmatrixname-2-5.south east);
            \coordinate (c) at (a|-b);
            \draw[decorate,line width=1.5pt,decoration={brace,raise=3pt,amplitude=5pt}]
        (a) -- node[right=1.5em] {$\tau_{i}\in \cD$} (c); \end{tikzpicture}
        \label{eq:trajectory_diagram}
    \end{equation}
    When conditioned with the composed goal $g$, we would be unable to imitate effectively
        as the composed state-goal $(s,g)$ is jointly out of the training distribution.

    What \emph{would} work for reaching $g$ is to first condition the policy on the intermediate waypoint $w$, then upon reaching $w$, condition on the goal $g$, as the state-goal pairs $(s_{i},w)$, $(w,g)$, and $(s_{i}',g)$ are all in the training distribution.
    If we condition the policy on some intermediate waypoint distribution $p(w)$ (or sufficient statistics thereof) that captures all of these cases, we can stitch together the expert behaviors to reach the goal $g$.

    Our approach is to learn a representation space that captures this ability, so that a GCBC objective used in this space can effectively imitate the expert on the composed task.
     We begin with the goal-conditioned behavioral cloning~\citep{kaelbling1993learning}
        loss $\cL_{\textsc{bc}}^{\phi,\psi,\xi}$ conditioned with waypoints $w$.
    \begin{equation}
        \cL_{\textsc{bc}}\bigl(\{s_{i},a_{i},s_{i}^{+},g_{i}\}_{i=1}^{K}\bigr) = \sum_{{i=1}}^{K} \log \pi\bigl(a_{i} \mid s_{i},\psi(g_{i})\bigr).
        \label{eq:goal_conditioned_bc}
    \end{equation}
    Enforcing the invariance needed to stitch \cref{eq:trajectory_diagram} then reduces to aligning \mbox{$\psi(g) \leftrightarrow \psi(w).$}
    The temporal alignment objective $\phi(s)\leftrightarrow \phi(s^{+})$ accomplishes this indirectly by aligning both $\psi(w)$ and $\psi(g)$ to the shared waypoint representation $\phi(w)$:

    \csuse{color indices}
    \begin{align}
        &\cL_{\textsc{nce}}\bigl(\{s_{i},s_{i}^{+}\}_{i=1}^{K};\phi,\psi\bigr) =
        \log \biggl( {\frac{e^{\phi(s^+_{\i})^{T}\psi(s_{\i})}}{\sum_{{\j=1}}^{K}
                e^{\phi(s^+_{\i})^{T}\psi(s_{\j})}}} \biggr)  \nonumber\\*
                &\mspace{100mu} +
        \sum_{{\j=1}}^{K} \log \biggl( {\frac{e^{\phi(s^+_{\i})^{T}\psi(s_{\i})}}{\sum_{{\i=1}}^{K}
                e^{\phi(s^+_{\i})^{T}\psi(s_{\j})}}} \biggr)
        \label{eq:goal_alignment}
    \end{align}

        
\end{rebuttal}
\subsection{Interfacing with Language Instructions}
\label{sec:language_instructions}

To extend the representations from \cref{sec:reaching_goals} to compositional instruction following with language tasks, we need some way to ground language into the $\psi$ (future state)
representation space.
We use a similar approach to GRIF~\citep{myers2023goal}, which uses an additional CLIP-style \citep{radford2021learning} contrastive alignment loss with an additional pretrained language encoder $\xi$:
\csuse{no color indices}
\begin{align}
    &\cL_{\textsc{nce}}\bigl(\{g_{i},\ell_{i}\}_{i=1}^{K};\psi,\xi\bigr)
    = \sum_{{i=1}}^{K} \log \biggl( {\frac{e^{\psi(g_{\i})^{T}\xi(\ell_{\i})}}{\sum_{{\j=1}}^{K}
            e^{\psi(g_{\i})^{T}\xi(\ell_{\j})}}} \biggr)  \nonumber\\*
            &\mspace{100mu} +
    \sum_{{\j=1}}^{K} \log \biggl( {\frac{e^{\psi(g_{\i})^{T}\xi(\ell_{\i})}}{\sum_{{\i=1}}^{K}
            e^{\psi(g_{\i})^{T}\xi(\ell_{\j})}}} \biggr)
    \label{eq:task_alignment}
\end{align}

\subsection{Temporal Alignment}
\label{sec:temporal_alignment}

Putting together the objectives from \cref{sec:reaching_goals,sec:language_instructions} yields the Temporal Representation Alignment (\Method) approach.
\Method{} structures the representation space of goals and language instructions to better enable compositional generalization.
We learn encoders $\phi, \psi ,$ and $\xi$ to map states, goals, and language instructions to a shared representation space.

\csuse{color indices}
\begin{align}
    \cL_{\textsc{nce}} \label{eq:NCE}
    &(\{x_{i}, y_{i}\}_{i=1}^{K};f,h) =
        \sum_{{\i=1}}^{K} \log \biggl( {\frac{e^{f(y_{\i})^{T}h(x_{\i})}}{\sum_{{\j=1}}^{K}
        e^{f(y_{\i})^{T}h(x_{\j})}}} \biggr) \nonumber\\*
      &\mspace{100mu} +
        \sum_{{\j=1}}^{K} \log \biggl( {\frac{e^{f(y_{\i})^{T}h(x_{\i})}}{\sum_{{\i=1}}^{K}
        e^{f(y_{\i})^{T}h(x_{\j})}}} \biggr) \\
    \cL_{\textsc{bc}} \label{eq:BC}
    &\bigl(\{s_{i},a_{i},s^{+}_{i},\ell_{i}\}_{i=1}^{K};\pi,\psi,\xi\bigr) = \nonumber\\*
      &\mspace{-10mu} \sum_{{i=1}}^{K} \log
        \pi\bigl(a_{i} \mid s_{i},\xi(\ell_{i})\bigr) + \log \pi\bigl(a_{i} \mid
        s_{i},\psi(s^{+}_{i})\bigr) \\
    \cL_{\textsc{tra}}
    &\label{eq:TRA} \bigl( \{s_{i},a_{i},s_{i}^{+},g_{i},\ell_{i}\}_{i=1}^{K}; \pi,\phi,\psi,\xi\bigr)
        \\
    &= \underbrace{\cL_{\textsc{bc}}\bigl(\{s_{i},a_{i},s_{i}^{+},\ell_{i}\}_{i=1}^{K};\pi,\psi,\xi\bigr)}_{\text{behavioral
    cloning}} \nonumber\\*
    &+
        \underbrace{\cL_{\textsc{nce}}\bigl(\{s_{i},s_{i}^{+}\}_{i=1}^{K};\phi,\psi\bigr)}_{\text{temporal alignment}}
        + \underbrace{\cL_{\textsc{nce}}\bigl(\{g_{i},\ell_{i}\}_{i=1}^{K};\psi,\xi\bigr)}_{\text{task alignment}} \nonumber
\end{align}Note that the NCE alignment loss uses a CLIP-style symmetric contrastive objective~\citep{radford2021learning,eysenbach2024inference} \-- we highlight the indices in the NCE alignment loss~\eqref{eq:NCE} for clarity.

Our overall objective is to minimize \cref{eq:TRA} across states, actions, future states, goals, and language tasks within the training data:
\begin{align}
    &\min_{\pi,\phi,\psi,\xi} \mathbb{E}_{\substack{(s_{1,i},a_{1,i},\ldots,s_{H,i},a_{H,i},\ell) \sim \mathcal{D} \\
    i\sim\operatorname{Unif}(1\ldots H) \\
    k\sim\operatorname{Geom}(1-\gamma)}} \\*
    &\mspace{10mu}
    \Bigl[\cL_{\text{TRA}}\bigl(\{s_{t,i},a_{t,i},s_{\min(t+k,H),i},s_{H,i},\ell\}_{i=1}^{K};\pi,\phi,\psi,\xi\bigr)\Bigr].
    \label{eq:overall_objective}
\end{align}

\begin{algorithm}
    \caption{Temporal Representation Alignment}
    \label{alg:tra}
    \begin{algorithmic}[1]
        \State \textbf{input:} dataset $\mathcal{D} = (\{s_{t,i},a_{t,i}\}_{t=1}^{H},\ell_i)_{i=1}^N$
        \State initialize networks $\Theta \triangleq (\pi,\phi,\psi,\xi)$
        \While{training}
        \State sample batch $\bigl\{(s_{t,i},a_{t,i},s_{t+k,i},\ell_i)\bigr\}_{i=1}^K\sim\mathcal{D}$ \\
        \hspace*{2ex} for $k\sim\operatorname{Geom}(1-\gamma)$
        \State $\Theta \gets \Theta - \alpha \nabla_{\Theta} \cL_{\text{TRA}}\bigl(\{s_{t,i},a_{t,i},s_{t+k,i},\ell_i\}_{i=1}^K; \Theta\bigr)$
        \EndWhile
        \smallskip
        \State \textbf{output:} \parbox[t]{\linewidth}{language-conditioned policy $\pi(a_{t} | s_{t}, \xi(\ell))$ \\
            goal-conditioned policy $\pi(a_{t} | s_{t}, \psi(g))$
        }
    \end{algorithmic}
\end{algorithm}

\subsection{Implementation}
\label{sec:implementation}

A summary of our approach is shown in \cref{alg:tra}.
In essence, TRA learns three encoders: $\phi$, which encodes states, $\psi$ which encodes future goals, and $\xi$ which encodes language instructions.
Contrastive losses are used to align state representations $\phi(s_{t})$ with future goal representations $\psi(s_{t+k})$, which are in turn aligned with equivalent language task specifications $\xi(\ell)$ when available.
We then learn a behavior cloning policy $\pi$ that can be conditioned on either the goal or language instruction through the representation $\psi(g)$ or $\xi(\ell)$, respectively.

\begin{rebuttal}
    \subsection{Temporal Alignment and Compositionality}
    \label{sec:compositionality}

    We will formalize the intuition from \cref{sec:reaching_goals} that \Method{} enables compositional generalization by considering the error on a ``compositional'' version of $\cD,$ denoted $\cD^{*}$.
    Using the notation from \cref{eq:trajectory}, we can say $\cD$ is distributed according to:
    \begin{align}
        &\cD \triangleq \cD^{H} \sim \prod_{i=1}^{K} p_0(s_{1,i}) p_{\ell}(\ell_{i} \mid s_{1,i}, s_{H,i})
            \nonumber\\*
          &\mspace{60mu} \prod_{t=1}^{H} \expert(a_{t,i} \mid s_{t,i}) \p(s_{t+1,i} \mid s_{t,i}, a_{t,i}) ,
            \label{eq:dataset_distribution}
    \end{align}
    or equivalently
    \begin{align}
            &\cD^{H} \sim \prod_{i=1}^{K} p_0(s_{1,i}) p_{\ell}(\ell_{i} \mid s_{1,i}, s_{H,i}) \nonumber\\*
            &\mspace{60mu} \prod_{t=1}^{H}
            e^{\sigma^2\|\expert(s_{t,i}) - a_{t,i}\|^2}\p(s_{t+1,i} \mid s_{t,i}, a_{t,i}) ,
            \label{eq:dataset_distribution_2}
    \end{align}
    by the isotropic Gaussian assumption.
    We will define $\cD^{*} \triangleq \cD^{H'}$ to be a longer-horizon version of $\cD$ extending the behaviors gathered under $\expert$ across a horizon $\alpha H \ge H' \ge H$ that additionally satisfies a ``time-isotropy'' property: the marginal distribution of the states is uniform across the horizon, i.e., $p_0(s_{1,i}) = p_0(s_{t,i})$ for all $t \in \{1\ldots H'\}$.

    We will relate the in-distribution imitation error $\textsc{Err}(\bullet; \cD)$ to the compositional out-of-distribution imitation error $\textsc{Err}(\bullet;\cD^{*})$.
    We define
    \begin{align}
        \textsc{Err}(\hat{\pi}; \tilde{\cD})
        &= \E_{\tilde{\cD}}\Bigl[\frac{1}{H}\sum_{t=1}^{H} \mathbb{E}_{\hat{\pi}}\left[\|\tilde{a}_{t,i} -
        \hat{\pi}(\tilde{s}_{t,i}, \tilde{s}_{H, i})\|^{2}/d_{\cA}\right]\Bigr] \nonumber\\
        &\quad \text{for} \quad \{\tilde{s}_{t,i},\tilde{a}_{t,i},\tilde{\ell}_{i}\}_{t=1}^{H} \sim
            \tilde{\cD}.
            \label{eq:imitation_error}
    \end{align}
    On the training dataset this is equivalent to the expected behavioral cloning loss from \cref{eq:BC}.

                            
    \begin{assumption}
        \label{asm:policy_factorization}
        The policy factorizes through inferred waypoints as:
\begin{align}
    &\textrm{goals: }\pi(a \mid s, g)
        = \nonumber\\*
      &\mspace{50mu} \int \pi(a\mid s, w) \p(s_{t}=w \mid s_{t+k}=g) \dd{w}
        \label{eq:goal-conditioned} \\
    &\textrm{language: } \pi(a \mid s, \ell)
        = \int \pi(a\mid s, w) \nonumber\\*
      &\mspace{20mu} \p(s_{t}=w \mid s_{t+k}=g) \p(s_{t+k}=g \mid \ell) \dd{w} \dd{g} ,
        \label{eq:language-conditioned}
        \end{align}
        where denote by $\pi(s,g)$ the MLE estimate of the action $a$.

    \end{assumption}

    \makerestatable
    \begin{theorem}
        \label{thm:compositionality}
        Suppose $\cD$ is distributed according to \cref{eq:dataset_distribution} and $\cD^{*}$ is distributed according to \cref{eq:dataset_distribution}.
        When $\gamma > 1-1/H$ and $\alpha > 1$, for optimal features $\phi$ and $\psi$ under \cref{eq:overall_objective}, we have
        \begin{gather}
            \textsc{Err}(\pi; \cD^{*}) \le \textsc{Err}(\pi; \cD) +  \frac{\alpha -1}{2 \alpha }+\Bigl(\frac{ \alpha - 2 }{2\alpha}\Bigr) \1 \{\alpha >2\}  .
            \label{eq:compositionality}
        \end{gather}
    \end{theorem}

    We can also define a notion of the language-conditioned compositional generalization error:
    \begin{equation*}
        \errl(\pi; \cD^{*}) \triangleq \E_{\cD^{*}}\Bigl[\frac{1}{H}\sum_{t=1}^{H}
            \mathbb{E}_{\pi}\bigl[\|\tilde{a}_{t,i} - \pi(\tilde{s}_{t,i}, \tilde{\ell}_{i})\|^{2}\bigr]\Bigr].
            \label{eq:language_error}
    \end{equation*}

    \makerestatable
    \begin{corollary}
        \label{thm:language}
        Under the same conditions as \cref{thm:compositionality},
        \begin{equation*}
            \errl(\pi; \cD^{*}) \le \errl(\pi; \cD) +  \frac{\alpha -1}{2 \alpha }+\Bigl(\frac{ \alpha - 2 }{2\alpha}\Bigr) \1 \{\alpha >2\}  .
            \label{eq:compositionality_language}
        \end{equation*}

    \end{corollary}

    The proofs as well as a visualization of the bound are in \cref{app:compositionality}. Policy implementation details can be found in \cref{app:tra_impl}

    
                
        
    \end{rebuttal}

\section{Experimental Setup}
\label{sec:setup}

In this study, we investigate the effectiveness of fine-tuning a transformer-based model for the Portuguese LVI task. We employ an iterative methodology to identify the optimal strategy for combining training corpora from various domains into a unified training process. Our primary objective is to evaluate cross-domain effectiveness and the generalization capabilities of our models.

\subsection{Models \& Baselines}

For the transformer-based model, we use BERTimbau with 334 million parameters~\cite{souza2020bertimbau}. BERTimbau is the result from fine-tuning the original BERT model~\cite{devlin2019bertpretrainingdeepbidirectional} on a Portuguese corpus.

To establish a baseline for comparison with the BERT model, we employ N-grams combined with Naive Bayes classifiers. This choice is motivated by the proven effectiveness of such models in previous LVI studies across various Indo-European languages, including Portuguese~\cite{zampieri2012automatic}.


\subsection{Cross-Domain Training Protocol}
To ensure that our model generalizes effectively across different domains, we define a two-step training protocol. Step one is used to find the best hyperparameters to train the model so ensure the generalization capability of the model. In this step,  the model is trained on a single domain from the PtBrVid corpus and validated on the remaining domains (excluding the one used for training). The hyperparameters yielding the best performance in this cross-domain validation are then used in step two to train the model across all domains combined.

Delexicalization of the corpus is treated as a hyperparameter in our approach. We adjust the probabilities of replacing tokens found by Named Entity Recognition (NER) and Part-of-Speech (POS) tagging with the generic label (such as \texttt{LOCATION} or \texttt{NOUN}), varying these probabilities incrementally from 0\% to 100\% in 20\% steps. It is important to note that delexicalization is applied exclusively to the training set. The validation set remains unaltered, simulating a real-world scenario where the input text is not modified. We leave the study of the impact of delexicalizing the validation set on the effectiveness of the model for future research.


\subsection{Train \& Validation Data}
As referred above the PtBrVId dataset is used to train the models. However, before using for the training, we leave 1,000 documents of each domain for the validation of the model, 500 of each label. 

In the step one of our training protocol, we use 8,000 documents from each domain (4,000 from each label) to train the models. We found this sample size to be enough for the models to converge and ensure fast iteration in the training process.

For step two of our training protocol, we compile all the documents from the PtBrVid corpus including the ones used for validation in step one. To avoid the training being dominated by the more represented domains, we undersample the dataset so that all labels from all domains are equally represented. At this step, the manually annotated set from PtBrVId set is used to keep track of the generalization loss.


\subsection{Benchmarks}

In our evaluation, we use two benchmarks: the DSL-TL and FRMT datasets. As mentioned above, the DSL-TL dataset is the standard benchmark for distinguishing between EP and BP, annotated with three labels: ``EP'', ``BP'', and ``Both''. For our purposes, we exclude documents labeled ``Both'' since our training corpus does not contain that label. This results in a test set comprising 588 documents for BP and 269 for EP. The FRMT dataset~\cite{riley2022frmt} has been manually annotated to evaluate variety-specific translation systems and includes translations in both EP and BP. We adapt this corpus for the VID task, resulting in a dataset containing 5,226 documents, with 2,614 labeled as EP and 2,612 as BP.


\section{Implementation Details}
\label{app:hyper}

NER and POS tags were identified using spaCy\footnote{\url{https://spacy.io/models/pt}}. The BERT model was trained with the \texttt{transformers}\footnote{\url{https://huggingface.co/docs/transformers/}} and \texttt{pytorch}\footnote{\url{https://pytorch.org}} libraries, for a maximum of 30 epochs, using early stopping with a patience of three epochs, binary cross-entropy loss, and the AdamW optimizer. The learning rate was set to $2 \times 10^{-5}$. In addition, a learning rate scheduler was used to reduce the learning rate by a factor of 0.1 if the training loss did not improve for two consecutive epochs. N-gram models were trained using the \texttt{scikit-learn}\footnote{\url{https://scikit-learn.org/}} library. The following hyperparameters were taken into account in the grid search we performed"

\begin{itemize}
    \item \textbf{TF-IDF Max Features:} The number of maximum features extracted using TF-IDF was tested with the following values: 100, 500, 1,000, 5,000, 10,000, 50,000, and 100,000.
    \item \textbf{TF-IDF N-Grams Range:} The range of n-grams used in the TF-IDF was explored with the following configurations: (1,1), (1,2), (1,3), (1,4), (1,5), and (1,10).
    \item \textbf{TF-IDF Lower Case:} The effect of case sensitivity was tested, with the lowercasing of text being either \texttt{True} or \texttt{False}.
    \item \textbf{TF-IDF Analyzer:} The type of analyzer applied in the TF-IDF process was either \texttt{Word} or \texttt{Char}.
\end{itemize}


Regarding computational resources, this study relied on Google Cloud N1 Compute Engines to perform the tuning and training of both the baseline and the BERT architecture. For the baseline, an N1 instance with 192 CPU cores and 1024 GB of RAM was used. For BERT, we used an instance with 16 CPU cores, 30 GB of RAM, and 4x Tesla T4 GPUs. The grid search on N-grams takes approximately three hours under these conditions, while for BERT, it takes approximately 52 hours to complete. The final training took three hours for N-grams and approximately ten hours for BERT.

We have made our codebase open-source\footnote{\url{https://github.com/LIAAD/portuguese_vid}} to promote reproducibility of our results and to encourage further research in this area. % Besides that, more details regarding the training of models can be found on the appendix of the paper.
\section{Results and Discussions}



%%分三部分,抽取,定位,分类
\subsection{Experimental Results}

Table~\ref{tab:main_results} and Table!\ref{tab:plm} shows the performance comparison of different models on our M-VAE task, and we can see that: \textbf{For extraction performance}, our \textbf{Sherlock} model outperforms all baselines, with an average improvement of 10.85 ($p$-value < 0.05) over the second performance. Specifically, our \textbf{Sherlock} model surpasses the second performance by an average of 9.9 ($p$-value < 0.05), 8.59 ($p$-value < 0.05), and 9.52 ($p$-value < 0.05) in average Single, Pair, and Quadruple metrics, justifying the effectiveness of \textbf{Sherlock} on extraction task. 
\textbf{For localization performance},
our \textbf{Sherlock} model exceeds the second performance by 11.42 ($p$-value < 0.01) in average mAP@tIoU metric, justifying the effectiveness of \textbf{Sherlock} on localization task.  Furthermore, \textbf{for classification performance}, in FNRs and F2 metric, \textbf{Sherlock} surpasses the second performance in 18.38 ($p$-value < 0.01) and 14.43 ($p$-value < 0.01). This implies the importance of our global and local information and justifies the effectiveness of our \textbf{Sherlock} model on our task.




\begin{figure}[t]
\setlength{\abovecaptionskip}{1 ex}
\setlength{\belowcaptionskip}{-1 ex}
  \centering
  \includegraphics[width=\columnwidth]{image/bar6943.pdf}
  \caption{(a) is the visual comparison of our SIR and (b) is the comparison of the average inference time for a one-minute video between Sherlock and other Video-LLMs.}
  \Description{our model xxx}
  \label{fig:infertime}
  
\end{figure}

\begin{table}[]
\setlength{\belowcaptionskip}{-1 ex}
\caption{Comparison of localization and anomaly classification task with several well-performing non-LLM models.}
\resizebox{\linewidth}{!}{
\begin{tabular}{c|cccc|ccc}
\hline
                           & \multicolumn{4}{c|}{\textbf{Anomaly Location}}                                                                                                                           & \multicolumn{2}{c}{\textbf{Anomaly Cls.}}                                   \\ \cline{2-7} 
                           &                                       & \multicolumn{2}{c}{mAP@tIoU}                                                  &                                       &                                       &                                       \\ \cline{3-4}
\multirow{-3}{*}{\textbf{Models}} & 0.1                                   & 0.2                                   & 0.3                                   & \multirow{-2}{*}{Average}             & \multirow{-2}{*}{FNRs}                & \multirow{-2}{*}{F2}                  \\ \hline
BiConvLSTM\cite{tab45}                 & 52.74                                 & 37.31                                 & 31.12                                 & 40.39                                 & 68.05                                 & 44.48                                 \\
SPIL\cite{tab44}                       & 53.28                                 & 38.89                                 & 32.91                                 & 41.69                                 & 67.84                                 & 46.87                                 \\
FlowGatedNet\cite{tab43}               & 53.64                                 & 39.64                                 & 33.18                                 & 42.15                                 & 67.24                                 & 46.55                                 \\
X3D\cite{tab42}                        & 54.52                                 & 40.05                                 & 34.96                                 & 43.17                                 & 65.08                                 & 48.65                                 \\
HSCD\cite{tab41}                      & 56.14                                 & 42.87                                 & 35.28                                 & 44.76                                 & 60.36                                 & 52.28                                 \\ \hline
\rowcolor{lightpink} \textbf{Sherlock}          & {\color[HTML]{3166FF} \textbf{94.03}} & {\color[HTML]{3166FF} \textbf{82.59}} & {\color[HTML]{3166FF} \textbf{76.12}} & {\color[HTML]{3166FF} \textbf{84.24}} & {\color[HTML]{3166FF} \textbf{17.24}} & {\color[HTML]{3166FF} \textbf{83.59}} \\ \hline
\end{tabular}}
\label{tab:plm}
\end{table}

% \textbf{3)} Influenced by the evaluation model proposed by Video-Bench~\cite{video-bench}, we used T5-based and GPT-based evaluation metrics to evaluate the extraction ability of \textbf{Sherlock} and other models. The average improvement of other models is \textbf{23.09}, while we are \textbf{17.84} ($p$-value < 0.01). This indicates that \textbf{Sherlock} can follow the instructions well and extract quadruples that meet our requirements.

\begin{figure*}[t]
\vspace{-0.3cm}
\setlength{\abovecaptionskip}{0.5 ex}
\setlength{\belowcaptionskip}{-3 ex}
  \centering
  \includegraphics[width=\textwidth]{image/case.pdf}
  \caption{Two Visualized samples to compare Sherlock with other Video-LLMs.}
  \Description{our model xxx}
  \label{fig:casestudy}
\end{figure*}

\subsection{Contributions of Each Key Component}
In order to further investigate the contributions of different modules of \textbf{Sherlock}, we conduct an ablation study on our \textbf{Sherlock} model. As shown in Table~\ref{tab:main_results}, w/o AE, w/o ORE, w/o BE, w/o GE, w/o EG, and w/o pre-tuning represent without four Spatial Experts, Expert Gate, and pre-tuning stage in sec~\ref{3.3} respectively.








\textbf{Effectiveness Study of Global and Local Spatial Expert}. From Table~\ref{tab:main_results}, we can see that: The performance of \textbf{w/o AE}, \textbf{w/o ORE}, \textbf{w/o BE} and \textbf{w/o GE} degrades in all metrics, with an average decrease of 7.54 ($p$-value < 0.01), 7.57 ($p$-value < 0.01), 4.37 ($p$-value < 0.01), and 5.68 ($p$-value < 0.01) in FNRs, F2, average map@tIoU, and average event extraction metrics. This confirms the importance of global and local information in extracting and localizing abnormal events, and \textbf{Sherlock} can better model those information well.



\textbf{Effectiveness Study of Spatial Imbalance Regulator.} From Table~\ref{tab:main_results}, we can see that: \textbf{1)} Compared with \textbf{Sherlock}, \textbf{w/o EG} shows poorer performance in all metrics, with a decrease of FNRs, F2, average map@tIoU, and average extraction performance by 15.34 ($p$-value < 0.01), 16.52 ($p$-value < 0.01), 8.62 ($p$-value < 0.05) and 10.36 ($p$-value < 0.01), respectively. This demonstrates the effectiveness of GSM in global-local spatial modeling and encourages us to consider handling heterogeneity issues between spatial information in the manner of MoE. 
\textbf{2)} From Table~\ref{tab:main_results}, we can see that compared to performance of \textbf{w/o SIR}, the performance of \textbf{w/o MG} is poorer, with FNRs, F2, average map@tIoU, and average event extraction metrics decreasing by 1.94 ($p$-value < 0.05), 3.9 ($p$-value < 0.05), 1.13 ($p$-value < 0.05) and 4.84 ($p$-value < 0.05), respectively. This further demonstrates the effectiveness of $\mathcal{L}_\text{gate}$ in global-local spatial balancing and encourages us to consider using SIR to better balance spatial information. 
\textbf{3)} In addition, we record the weights of four spatial experts after training in Figure~\ref{fig:expert} and Figure~\ref{fig:infertime} (a). We can see that the weights of all experts have been relatively balanced, and each expert has demonstrated outstanding professional abilities when facing different types of abnormal videos.




% \begin{table}[]
% \setlength{\belowcaptionskip}{-1 ex}
% \caption{Comparison of localization and anomaly classification task with several well-performing non-LLM models which is conducted on publicly available datasets.}
% \resizebox{\linewidth}{!}{
% \begin{tabular}{c|cccc|ccc}
% \hline
%                            & \multicolumn{4}{c|}{\textbf{Anomaly Location}}                                                                                                                           & \multicolumn{2}{c}{\textbf{Anomaly Cls.}}                                   \\ \cline{2-7} 
%                            &                                       & \multicolumn{2}{c}{mAP@tIoU}                                                  &                                       &                                       &                                       \\ \cline{3-4}
% \multirow{-3}{*}{\textbf{Models}} & 0.1                                   & 0.2                                   & 0.3                                   & \multirow{-2}{*}{Average}             & \multirow{-2}{*}{FNRs}                & \multirow{-2}{*}{F2}                  \\ \hline
% BiConvLSTM\cite{tab45}                 & 52.74                                 & 37.31                                 & 31.12                                 & 40.39                                 & 68.05                                 & 44.48                                 \\
% SPIL\cite{tab44}                       & 53.28                                 & 38.89                                 & 32.91                                 & 41.69                                 & 67.84                                 & 46.87                                 \\
% FlowGatedNet\cite{tab43}               & 53.64                                 & 39.64                                 & 33.18                                 & 42.15                                 & 67.24                                 & 46.55                                 \\
% X3D\cite{tab42}                        & 54.52                                 & 40.05                                 & 34.96                                 & 43.17                                 & 65.08                                 & 48.65                                 \\
% HSCD\cite{tab41}                      & 56.14                                 & 42.87                                 & 35.28                                 & 44.76                                 & 60.36                                 & 52.28                                 \\ \hline
% \rowcolor{lightpink} \textbf{Sherlock}          & {\color[HTML]{3166FF} \textbf{94.03}} & {\color[HTML]{3166FF} \textbf{82.59}} & {\color[HTML]{3166FF} \textbf{76.12}} & {\color[HTML]{3166FF} \textbf{84.24}} & {\color[HTML]{3166FF} \textbf{17.24}} & {\color[HTML]{3166FF} \textbf{83.59}} \\ \hline
% \end{tabular}}
% \label{tab:plm}
% \end{table}



\textbf{Effectiveness Study of Pre-tuning}. From Table~\ref{tab:main_results}, we can see that \textbf{w/o pre-tuning}, the performance is inferior to \textbf{Sherlock}. FNRs, F2, average map@tIoU, and average event extraction metrics have decreased by 17.63 ($p$-value < 0.01), 16.95 ($p$-value < 0.01), 10.92 ($p$-value < 0.01) and 11.48 ($p$-value < 0.01), respectively. This further justifies the effectiveness of pre-tuning, as well as encourages us to use more high-quality datasets to enhance the spatial understanding ability of Video-LLMs before instruction-tuning. 





\subsection{Convergence Analysis and Practical Assessment for Sherlock}
In order to analyze the convergence of Sherlock, we record the loss of baseline Video-LLMs, Sherlock, and its variant without specific components over various training steps. The results are shown in Figure~\ref{fig:train} and we can see that: \textbf{1)} \textbf{Sherlock} demonstrates the fastest convergence compared to other Video-LLMs. At the convergence point, the loss of Sherlock is 1.05, while Video-LLaVA is 2.06. This underscores the high efficiency of Sherlock over other advanced Video-LLMs. \textbf{2)} \textbf{Sherlock} demonstrates the fastest convergence compared to its variant without specific components in Figure~\ref{fig:train}. This justifies that the spatial information along with GSM and SIR can accelerate the convergence process, which further encourages us to consider the spatial information in the M-VAE task. 

To assess practicality, we analyze the FNRs of Sherlock for each scene. As shown in Table~\ref{tab:scene}, we can observe that in every scene, Sherlock outperforms other Video-LLMs. This indicates that the possibility of misclassifying abnormal events as normal events is minimized, thereby demonstrating the importance of global and local spatial modeling of Sherlock. We also analyze the average inference time in seconds for a one-minute video. As shown in Figure~\ref{fig:infertime} (b), Sherlock does not
perform much differently from the other models in terms of inference time. This is reasonable, as some studies confirm that the MoE architecture can improve efficiency [11, 28]. This suggests that introducing more information along with a MoE module for the M-VAE task does not increase the inference time and Sherlock can maintain good inference efficiency.

% \subsection{Compared with Advanced Non-LLM Models on Public Dataset}
% In order to more comprehensively evaluate the effectiveness of Sherlock, we compare our \textbf{Sherlock} model with other advanced non-LLM models~\cite{tab41,tab42,tab43,tab44,tab45} on traditional anomaly localization and anomaly classification task based on publicly available CUVA datasets~\cite{cuva}. Specifically, we need Sherlock to determine whether each second of the video is abnormal or not without generating quadruples. As shown in Table~\ref{tab:plm}, non-LLM models not only underperform relative to other Video-LLMs presented in Table~\ref{tab:plm} but also significantly inferior to our Sherlock model. This
% further demonstrates the importance of the global and local spatial information we proposed for the M-VAE task.


\subsection{Qualitative Analysis for Sherlock}
As shown in Figure~\ref{fig:casestudy}, we visualize and compare \textbf{Sherlock} with other Video-LLMs. We randomly select two samples from our dataset and ask these models to \emph{Analyze the following video and localize the timestamp and extract the quadruple of the abnormal events}. From the figure, we can see that: \textbf{1)} Accurately localizing abnormal events and extracting correct quadruples is a huge challenge. For instance, example 2 captures a segment from 9s to 15s, where identifying the collision of the truck at road is challenging, \textbf{2)} Compared with other advanced Video-LLMs, \textbf{Sherlock} shows excellent performance in localizing abnormal events. In example 1, \textbf{Sherlock} outperforms other models in terms of accuracy. In example 2, it outperforms PandaGPT in terms of accuracy and can generate a correct quadruple. This further demonstrates the effectiveness of \textbf{Sherlock} in precisely extracting and localizing abnormal events.
We present RiskHarvester, a risk-based tool to compute a security risk score based on the value of the asset and ease of attack on a database. We calculated the value of asset by identifying the sensitive data categories present in a database from the database keywords. We utilized data flow analysis, SQL, and Object Relational Mapper (ORM) parsing to identify the database keywords. To calculate the ease of attack, we utilized passive network analysis to retrieve the database host information. To evaluate RiskHarvester, we curated RiskBench, a benchmark of 1,791 database secret-asset pairs with sensitive data categories and host information manually retrieved from 188 GitHub repositories. RiskHarvester demonstrates precision of (95\%) and recall (90\%) in detecting database keywords for the value of asset and precision of (96\%) and recall (94\%) in detecting valid hosts for ease of attack. Finally, we conducted an online survey to understand whether developers prioritize secret removal based on security risk score. We found that 86\% of the developers prioritized the secrets for removal with descending security risk scores.

\begin{acks}
We thank our anonymous reviewers for their helpful comments. This work was supported by three NSFC grants, i.e., No.62006166, No.62376178 and No.62076175. This work was also supported by a Project Funded by the Priority Academic Program Development of Jiangsu Higher Education Institutions (PAPD).
\end{acks}

\bibliographystyle{ACM-Reference-Format}
\bibliography{sample-base}




\end{document}
\endinput
%%
%% End of file `sample-sigconf-authordraft.tex'.
