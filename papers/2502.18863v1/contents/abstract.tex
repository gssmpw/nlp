\begin{abstract}
Prior studies on Video Anomaly Detection (VAD) mainly focus on detecting whether each video frame is abnormal or not in the video, which largely ignore the structured video semantic information (i.e., what, when, and where does the abnormal event happen). With this in mind, we propose a new chat-paradigm \textbf{M}ulti-scene \textbf{V}ideo \textbf{A}bnormal \textbf{E}vent Extraction and Localization (M-VAE) task, aiming to extract the abnormal event quadruples (i.e., subject, event type, object, scene) and localize such event. Further, this paper believes that this new task faces two key challenges, i.e., global-local spatial modeling and global-local spatial balancing. To this end, this paper proposes a Global-local Spatial-sensitive Large Language Model (LLM) named Sherlock, i.e., acting like \emph{Sherlock Holmes} to track down the criminal events, for this M-VAE task. Specifically, this model designs a Global-local Spatial-enhanced MoE (GSM) module and a Spatial Imbalance Regulator (SIR) to address the two challenges respectively. Extensive experiments on our M-VAE instruction dataset show the significant advantages of Sherlock over several advanced Video-LLMs. This justifies the importance of global-local spatial information for the M-VAE task and the effectiveness of Sherlock in capturing such information. 
% is demonstrated through a specially constructed dataset, showing superior performance compared to existing models.
\end{abstract}