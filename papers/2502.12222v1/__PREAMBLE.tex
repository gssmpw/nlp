\usepackage{amssymb}
\usepackage{amsmath}
\usepackage{lipsum}
%\usepackage{xcolor}
\usepackage[dvipsnames]{xcolor}
\usepackage{cleveref}
\usepackage{multirow}
\usepackage{graphicx}
\usepackage{subcaption}
\usepackage{url}
\usepackage{algorithmicx}
\usepackage{algpseudocode}
\usepackage[linesnumbered,ruled,vlined]{algorithm2e}
% per algoritmi
%\newcommand{\nonl}{\renewcommand{\nl}{\let\nl\oldnl}}% Remove line number for one line

%\usepackage{natbib}
\captionsetup{margin=10pt,font=small,labelfont=bf}

%% For including figures, graphicx.sty has been loaded in
%% elsarticle.cls. If you prefer to use the old commands
%% please give \usepackage{epsfig}

%% The amssymb package provides various useful mathematical symbols

%% The amsthm package provides extended theorem environments
%% \usepackage{amsthm}

%% The lineno packages adds line numbers. Start line numbering with
%% \begin{linenumbers}, end it with \end{linenumbers}. Or switch it on
%% for the whole article with \linenumbers.
%% \usepackage{lineno}
\usepackage{cleveref}
%% The lineno packages adds line numbers. Start line numbering with
%% \begin{linenumbers}, end it with \end{linenumbers}. Or switch it on
%% for the whole article with \linenumbers.
%% \usepackage{lineno}

%% You might want to define your own abbreviated commands for common used terms, e.g.:
\newcommand{\kms}{km\,s$^{-1}$}
\newcommand{\msun}{$M_\odot$}
\newcommand{\roberto}[1]{\textcolor{blue}{\textit{[Rob says: #1]}}}
\newcommand{\francesco}[1]{\textcolor{red}{{\it [Francesco: #1]}}}
\newcommand{\andrea}[1]{\textcolor{magenta}{{\it [Andrea: #1]}}}
\newcommand{\sal}[1]{\textcolor{orange}{{\it [Sal: #1]}}}
\renewcommand{\vec}[1]{\mathbf{#1}}


