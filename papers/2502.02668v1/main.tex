\documentclass[times,sort&compress,3p]{elsarticle}
\journal{Journal of Multivariate Analysis}
\usepackage[labelfont=bf]{caption}
\renewcommand{\figurename}{Fig.}

\usepackage[utf8]{inputenc}
\usepackage{graphicx}
\usepackage{amsmath}
\usepackage{amsfonts}
\usepackage{amsthm}
\usepackage{amssymb}
\usepackage[]{todonotes}
\usepackage{float}
\usepackage{relsize}
\usepackage{bbm}
\usepackage{subcaption}
\usepackage{algpseudocode}
\usepackage{algorithm} 
\usepackage[hidelinks]{hyperref}
\usepackage{bbold}
\usepackage{tikz}
\usepackage{circuitikz}


\newtheorem{theorem}{Theorem}
\newtheorem{corollary}{Corollary}
\newtheorem*{conjecture}{Conjecture}
\newtheorem{lemma}{Lemma}
\newtheorem{definition}{Definition}
\newtheorem{problem}{Problem}
\newtheorem{assumption}{Assumption}
\newtheorem{remark}{Remark}
\newtheorem{proposition}{Proposition}


\newcommand{\conjectureRef}{\hyperlink{hopkinsConjecture}{Low Degree Conjecture} }

\newcommand{\lemmaautorefname}{Lemma}
\newcommand{\definitionautorefname}{Definition}
\newcommand{\corollaryautorefname}{Corollary}
\newcommand{\problemautorefname}{Problem}
\renewcommand{\sectionautorefname}{Section}
\renewcommand{\subsectionautorefname}{Subection}
\newcommand{\algorithmautorefname}{Algorithm}
\newcommand{\assumptionautorefname}{Assumption}
\newcommand{\remarkautorefname}{Remark}
\newcommand{\propositionautorefname}{Proposition}
\newcommand{\conjectureautorefname}{Conjecture}
\newcommand{\ldconjectureautorefname}{Low-Degree Conjecture}

\renewcommand{\itemautorefname}{Precondition}

\renewcommand\appendixautorefname[1]{} 


\newcommand{\X}{\mathbf{X}}

\newcommand{\bA}{\mathbf{A}}
\newcommand{\bB}{\mathbf{B}}
\newcommand{\bU}{\mathbf{U}}
\newcommand{\bM}{\mathbf{M}}
\newcommand{\bN}{\mathbf{N}}
\newcommand{\bS}{\mathbf{S}}
\newcommand{\bE}{\mathbf{E}}
\newcommand{\bI}{I}\newcommand{\bW}{\mathbf{W}}
\newcommand{\bX}{\mathbf{X}}
\newcommand{\bY}{\mathbf{Y}}
\newcommand{\bZ}{\mathbf{Z}}

\newcommand{\ba}{\mathbf{a}}
\newcommand{\bb}{\mathbf{b}}
\newcommand{\bu}{\mathbf{u}}
\newcommand{\bv}{\mathbf{v}}
\newcommand{\bg}{\mathbf{g}}
\newcommand{\bw}{\mathbf{w}}
\newcommand{\be}{\mathbf{e}}
\newcommand{\bx}{\mathbf{x}}
\newcommand{\bc}{\mathbf{c}}
\newcommand{\bz}{\mathbf{z}}
\newcommand{\by}{\mathbf{y}}
\newcommand{\br}{\mathbf{r}}
\newcommand{\bnu}{\mathbf{\nu}}
\newcommand{\bmu}{\mathbf{\mu}}

\newcommand{\cQ}{\mathcal{H}_0}
\newcommand{\cP}{\mathcal{H}_1}
\newcommand{\cU}{\mathcal{U}}
\newcommand{\cO}{\mathcal{O}}
\newcommand{\cN}{\mathcal{N}}
\newcommand{\cB}{\mathcal{B}}
\newcommand{\BR}{\mathcal{BR}}
\newcommand{\cD}{\mathcal{D}}
\newcommand{\cF}{\mathcal{F}}

\newcommand{\bbE}{\mathbb{E}}
\newcommand{\bbP}{\mathbb{P}}
\newcommand{\bbR}{\mathbb{R}}
\newcommand{\bbI}{I}

\newcommand{\Natural}{\mathbb{N}}
\newcommand{\Real}{\mathbb{R}}
\newcommand{\bbS}{\mathbb{S}}

\newcommand{\ainner}{\langle \bu, \bu^* \rangle}
\newcommand{\ainnerdash}{\langle \bu', \bu^* \rangle}
\newcommand{\ainnerhat}{\langle \hat{\bu}, \bu^* \rangle}
\newcommand{\aorth}{\sqrt{1 - \ainner^2}}
\newcommand{\Data}{\mathcal{D}}
\newcommand{\Normal}{\mathcal{N}}
\newcommand{\expectation}{\mathbb{E}}
\newcommand{\probability}{\mathbb{P}}
\newcommand{\projectustar}[1]{\left\langle #1, \bu^* \right\rangle}
\newcommand{\projectu}[1]{\left\langle #1, \bu \right\rangle}
\newcommand{\sigmainner}{\sqrt{1 - \ainner^2}}
\newcommand{\inner}[1]{\left\langle #1 \right\rangle}
\newcommand{\average}[1]{\sumin \frac{#1}{n}}
\newcommand{\erfc}{\operatorname{erfc}}

\DeclareMathOperator*{\argmax}{arg\,max}
\DeclareMathOperator*{\argmin}{arg\,min}

\newcommand{\trace}{\mathrm{trace}}

\newcommand{\correction}{(\bI_d - \bu^* {\bu^*}^\top - \be_2 \be_2^\top)}
\newcommand{\correctiontwo}{(\bI_d - \bu^* {\bu^*}^\top)}
\newcommand{\mynorm}[1]{\left\|#1\right\|_2}
\newcommand{\myonenorm}[1]{\left\|#1\right\|_1}
\newcommand{\opnorm}[1]{\left\|#1\right\|_{op}}
\newcommand{\myabs}[1]{\left|#1\right|}

\newcommand{\oneOrclizNorm}[1]{\left\|#1\right\|_{\psi_1}}
\newcommand{\twoOrclizNorm}[1]{\left\|#1\right\|_{\psi_2}}
\newcommand{\epsilonNet}{\mathcal{N}_{\epsilon, d}}
\newcommand{\sumin}{\sum_{i = 1}^n}

\newcommand*\diff{\mathop{}\!\mathrm{d}}

\newcommand{\polylog}{\text{polylog}}

\newcommand{\alimit}{b}
\usepackage{dsfont}
\newcommand{\indicator}{\mathds{1}}

\newcommand{\bigO}[1]{\mathcal{O}\left( #1 \right)}
\newcommand{\smallO}[1]{o\left( #1 \right)}
\newcommand{\bigOs}[1]{\widetilde{\mathcal{O}}\left( #1 \right)}

\newcommand{\bigTheta}[1]{\Theta\left( #1 \right)}
\newcommand{\bigThetas}[1]{\widetilde{\Theta}\left( #1 \right)}

\newcommand{\bigOmega}[1]{\Omega\left( #1 \right)}
\newcommand{\bigOmegas}[1]{\widetilde{\Omega}\left( #1 \right)}

\newcommand{\relutwo}[1]{\max\left\{0, #1\right\}^2}


\newcommand{\ldlr}{\mynorm{L_{d}^{\leq D}}}

\newcommand{\ubar}{\Bar{\bu}}
\newcommand{\uhat}{\hat{\bu}}
\newcommand{\ustar}{\bu^*}
 \begin{document}

\begin{frontmatter}

\title{
Recovering Imbalanced Clusters via Gradient-Based Projection Pursuit
}

\author[1]{Martin Eppert \corref{mycorrespondingauthor}}
\author[2]{Satyaki Mukherjee}
\author[1]{Debarghya Ghoshdastidar}

\address[1]{
Technical University of Munich
School of Computation, Information and Technology - I7
Boltzmannstr. 3 
85748 Garching b. München 
Germany}
\address[2]{
National University of Singapore
Level 4, Block S17
10 Lower Kent Ridge Road
Singapore 119076 
}
\cortext[mycorrespondingauthor]{Corresponding author. Email address:\url{martin.eppert@tum.de}}
\begin{abstract}
    Projection Pursuit is a classic exploratory technique for finding interesting projections of a dataset.
    We propose a method for recovering projections containing either Imbalanced Clusters or a Bernoulli-Rademacher distribution using a gradient-based technique to optimize the projection index.
    As sample complexity is a major limiting factor in Projection Pursuit, we analyze our algorithm's sample complexity within a Planted Vector setting where we can observe that Imbalanced Clusters can be recovered more easily than balanced ones.
    Additionally, we give a generalized result that works for a variety of data distributions and projection indices.
    We compare these results to computational lower bounds in the Low-Degree-Polynomial Framework.
    Finally, we experimentally evaluate our method's applicability to real-world data using FashionMNIST and the Human Activity Recognition Dataset, where our algorithm outperforms others when only a few samples are available.
\end{abstract}
\begin{keyword}
Gradient-Based Methods \sep
Projection Pursuit \sep
Optimization \sep
Statistical Computational Gap
\MSC[2020] Primary 62H12 \sep
Secondary 62F12
\end{keyword}
\end{frontmatter}
\section{Introduction}
Projection Pursuit was introduced in \citet{PPFriedmanTukey} as a method for finding maximally interesting projections, as determined by the histogram of the projected data. 
A function known as the projection index is used to assess the interestingness of the data projected onto a subspace.
A projection is then obtained that optimizes the projection index to reveal the structure of the data.
The methodology of projection pursuit has proven itself useful as a sub-procedure in many statistical analyses.
One area of application for projection pursuit is in clustering algorithms, where finding projections that minimize kurtosis or maximize the $\ell_1$-norm can reveal cluster structures in many settings~\cite{Pena2000, davis2021clustering, Pena2001a}.
Projection pursuit is also used for outlier detection~\cite{Pena2001b, Pena2007, nicolaKurtosis}.
Here, finding projections that maximize kurtosis can reveal small clusters that may correspond to outliers.
In other applications, projection pursuit is used to discover non-Gaussian independent components from a linear mixture~\cite{fastICA, fastICAgb}.
This is done by finding non-Gaussian projections of the data, which can be done by maximizing excess kurtosis or using other suitable measures of non-Gaussianity.
Another related application of projection pursuit is the recovery of sparsely used dictionaries.
Here, it is generally assumed that an orthonormal basis exists, and the data will be sparse if represented in this basis.
Finding projections that minimize $\ell_1$-norm~\cite{bai2019subgradient} or maximize kurtosis~\cite{DictionaryLearningKurtosis} can be used to reveal individual vectors in the basis.

There exists a plethora of approaches to optimizing projection indices.
Iterative methods such as gradient descent are commonly used in projection pursuit~\cite{bai2019subgradient, DictionaryLearningKurtosis, QuSW14}.
Originating from Independent Component Analysis, the iterative FastICA algorithm is commonly used to optimize projection indices~\cite{fastICA,fastICAgb}.
The initialization of iterative methods is also essential to optimize their performance.
Methods for initializing projection pursuit algorithms include random directions~\cite{ica_random_init}, normalized samples form the data~\cite{spielman2012exact}, eigenvalues of kurtosis matrices~\cite{KurtosisLoperfido18, Pena2010} and skewness matrices~\cite{nicolaSkewness}.
\citet{Arevalillo2021} also discusses these initializations and their sampling properties.
%
A different approach to optimizing projection indices, compared to iterative methods, is using properties of higher-order moment matrices to optimize projection indices.
This is typically done by utilizing the eigenvector corresponding to the largest or smallest eigenvalue of the moment matrix. 
There are many different formulations using eigen-/singular-values of third~\cite{nicolaSkewness, LOPERFIDO2015202, Kollo2008} and fourth~\cite{Pena2010, nicolaKurtosis, mao2022optimal} order moment matrices.
%
Computational complexity is a key consideration from an implementation perspective.
Frequently, Gradient-based methods are favored due to their low computational cost.
In contrast, approaches that rely on the eigenvalues or singular values of moment matrices require the construction of large matrices.
The use of methods like quasi-Newton needs the construction of the Hessian~\cite{Friedman01031987}.
The materialization of these matrices becomes prohibitively expensive in large-dimensional settings.
Another factor increasing the computational complexity of iterative methods is the number of iterations to convergence~\cite{QuSW14}.

Although finding an optimum of the projection index is in many cases theoretically possible, even with a small number of samples (e.g., via brute-force enumeration of projections), it is generally computationally intractable~\cite{mao2022optimal, davis2021clustering}.
This is commonly called a statistical-to-computational gap when a statistical problem is solvable, but all algorithms are computationally intractable.
Assessing the optimal sample complexity that efficient algorithms can achieve in a specific setting is crucial in guiding algorithm development.
Although there is currently no way to show lower bounds for all efficient algorithms, there exist lower bounds for specific classes, such as Low-Degree Polynomials, of algorithms that are conjectured optimal for specific classes of problems~\cite{hopkins2018statistical, kunisky2019notes}.

The \emph{planted vector setting}~\cite{mao2022optimal, dudeja2024statisticalcomputational, hopkins2016fast} is a well-suited setting for studying the sampling properties of projection pursuit.
The planted vector problem is a specific instance within the broader class of planted problems concerned with recovering a hidden (planted) signal embedded in random noise~\cite{alon1998, hopkins2016fast}.
In this setting, the objective is to identify the planted vector from noisy observations, leveraging the structural properties of the signal and the noise.
In the planted vector setting, projecting the data onto a specific direction —referred to as the signal direction— reveals the hidden signal, while projections onto orthogonal directions yield data that follows a Gaussian distribution.
This setting is widely studied because it models a range of classical statistical problems, such as clustering in Gaussian mixture models, which can be framed as instances of planted vector problems.
Statistical procedures often start by whitening the data to eliminate spurious correlations, a step naturally captured in the planted vector framework, where the data is typically assumed to have unit covariance.

Due to its restrictive nature, the planted vector setting lends itself well to studying lower bounds on the sample complexity of algorithms~\cite{davis2021clustering, QuSW14, mao2022optimal, hopkins2016fast}.
\citet{mao2022optimal} show that recovering a $p$-sparse planted vector in a $d$-dimensional space requires at least $\bigOmegas{d^2 p^2}$ samples for Low-Degree Polynomials to succeed.
\citet{dudeja2024statisticalcomputational} considers the more general case of recovering a planted vector where the first $(k-1)$-moments match a Gaussian distribution.
In this setting, at least $\bigOmegas{d^{k/2}}$ samples are required for Low-Degree Polynomials to succeed.
\citet{diakonikolas2024sumofsquareslowerboundsnongaussian} extend these results to the Sum-Of-Squares framework.
\subsection{Our Contributions}
Although gradient-based algorithms are commonly used in practice, their sampling properties are still difficult to analyze compared to the analysis of spectral methods.
For example, \citet{QuSW14} note in their paper on recovery of planted vectors with gradient-based methods that the sample complexity of their method is significantly lower than their presented bounds.
This motivates us to design a gradient-based projection pursuit algorithm that is easy to analyze in the planted vector setting.
We provide lower bounds on the required sample complexity for this algorithm, which closely match our simulation results and are also competitive with known results in the planted vector setting.
%
The computational complexity of projection pursuit algorithms is also a major roadblock to their widespread adoption.
The computational complexity of our algorithm scales linearly with $\bigO{n d}$, which is significantly lower than other methods.
This is also significant since we do not need to run more than one gradient step on each sample, avoiding the computational overhead of running many iterations.

We further demonstrate the experimental efficacy of this algorithm in two different settings:
First, we demonstrate the efficacy of the presented algorithm in recovering a $d$-dimensional planted vector containing two Imbalanced Clusters with $p$ denoting the cluster probability.
In this setting, use the projection index ${\phi(x) = \max\{0,x\}^2}$ (ReLU2) for which we prove that $\bigOs{d^2 p^2}$ samples are sufficient for gradient ascent to recover the signal direction.
As cluster imbalance increases, the smaller cluster moves further from the origin while the larger cluster approaches the origin.
We also apply the algorithm to recover a Bernoulli-Rademacher planted vector using kurtosis as a projection index for which $n = \bigOmegas{d^3 p^4}$ are sufficient.

We employ multiple techniques to improve our algorithm's sample complexity and allow a tight analysis of its sample complexity.
First, we propose using fresh mini-batches in each iteration of gradient ascent~\cite{bertsekas1996incremental}.
Suppose we reuse the same dataset for each step of gradient ascent.
In that case, more samples are necessary to ensure that the optimization problem is smooth enough so that gradient ascent does not get stuck in local maxima.
Using mini-batches mitigates this problem but comes at a significant cost, as we require new samples in each iteration.
However, we prove that only a few steps are necessary to converge, and thus, the impact of resampling on the sample complexity is small.

Additionally, we suggest initializing gradient ascent with normalized samples from the dataset, similar to a technique used in \citet{spielman2012exact}. 
Many planted vectors, such as Imbalanced Clusters, have longer tails than the standard Gaussian distribution.
Exploiting this, we demonstrate that with high probability, it is possible to find an initialization closer to the signal direction than what could be found using a random initialization.
As estimating the gradient for a direction closer to the signal directions generally becomes easier, we require fewer samples to estimate the gradient accurately.

To further study the setting of planted Imbalanced Clusters, we study lower bounds on the sample complexity of recovering the planted vector using any polynomial time algorithm.
Our study of the setting where the planted vector contains two clusters, with one being significantly larger than the other, differs from commonly studied planted vector settings in which the planted vector is symmetric, such as the Bernoulli-Rademacher in \citet{mao2022optimal} and \citet{QuSW14}.
We prove a computational lower bound close to the sample complexity required for the gradient-based algorithm.
For this lower bound in the setting of Imbalanced Clusters, we extend \citet{mao2022optimal}, which uses the framework of Low-Degree-Polynomials~\cite{kunisky2019notes, hopkins2018statistical}.
Here, we obtain a lower bound on the sample complexity of $n = \bigOmegas{d^{1.5} p}$.
\citet{dudeja2024statisticalcomputational} prove a similar result, which is not applicable in the setting where $p$ decreases with respect to the problem's dimensionality. 
In contrast, our result captures the setting where $p$ decreases proportionally to the dimension.
While the number of samples required by our method is not optimal, it is sufficiently close.

Finally, to motivate the pursuit of imbalanced projections, we show that by applying our algorithm to the FashionMNIST dataset~\cite{xiao2017fashionmnist} and the Human Activity Recognition dataset~\cite{human_activity_recognition_using_smartphones_240}, we obtain directions that reveal clusters in the data that correspond to their labels.
The utility of these projections is measured by how much information they provide on the class labels.
Especially with the projection index $\phi(x) = \max\{0,x\}^2$, we observe that even with a small number of samples, it is possible to find projections that reveal the class structure of the dataset by separating one class from the others.
\subsection{Document Structure}
In \autoref{setup}, we define the planted vector setting.
\autoref{gbAlgo} describes the proposed algorithm.
\autoref{scBounds} describes a general analysis of the gradient-based algorithm.
\autoref{Relu2Section} and \autoref{KurtosisSection} demonstrate results on Imbalanced Clusters and Bernoulli-Rademacher planted vectors.
\autoref{ldplb} discusses Low-Degree lower bounds for Imbalanced Clusters planted vectors.
\autoref{experiments} contains simulations and experiments on real data. 
Further experiments on synthetic data can be found in the \autoref{supplement}.
\section{Main Results}
We use standard asymptotic notation $o(\cdot)$, $\mathcal{O}(\cdot)$, $\Theta(\cdot)$, $\Omega(\cdot)$ and $\widetilde{\mathcal{O}}(\cdot)$, $\widetilde{\Omega}(\cdot)$ which hides logarithmic factors.
$\bbS_{d-1}$ denotes the $d$-dimensional unit sphere.
$\cU(\cdot)$ denotes the uniform distribution and $\cN(\mu, \sigma^2)$ denotes the standard normal distribution with mean $\mu$ and standard deviation $\sigma$.
\subsection{Setup}
\label{setup}
We follow the literature on recovery of planted vectors \cite{mao2022optimal, dudeja2024statisticalcomputational, hopkins2016fast}, which gives a simplified formulation of the data assumption in Projection Pursuit.
Throughout the paper, we use $n$ to indicate the number of samples present and $d$ to indicate the dimensionality of the data.
In the Planted Vector Setting, the model is constructed as follows:
\begin{definition}[Planted Vector Setting]
    \label{planted_vector_setting}
    We say $\bx\sim\cD_\cF$ if, 
    $\bx \sim \mathcal{N}(\bnu \, \ustar, \bI_d - \ustar{\ustar}^\top)$.
    Given the random variable $\nu \sim \cF$ for some distribution $\cF$
    and a fixed but unknown direction $\ustar$.
\end{definition}
The distribution of $\nu$ is generally defined to follow a non-Gaussian distribution with unit variance.
Later, we will consider the setting where $\nu$ follows a distribution containing either two Imbalanced Clusters or a sparse distribution.
If $d$ is large, then if the data is projected in a random direction, it will be approximately Gaussian, but if projected in the direction $\bu^*$, the structure of $\nu$ can be observed.
\subsection{Gradient-Based Algorithm}
\label{gbAlgo}
\vspace{-0.1cm}
This section describes \autoref{twoStepAlgorithm}, a gradient-based algorithm for optimizing differentiable projection indices.
Here, we assume that we are given access to a dataset containing a planted vector as in \autoref{planted_vector_setting}.
The recovery of the signal direction ($\bu^*$) is done by finding an (approximate) solution to the following optimization problem where $\psi$ is the projection index.
\vspace{-0.2cm}
$$\hat{\bu} = \max_{\bu \in \bbS_{d-1}} \sumin \frac{\psi(\inner{\bX_i, \bu})}{n}$$
\vspace{-0.2cm}\\
This will be done by performing gradient ascent using a different projection index $\phi$ using multiple initializations.
Then $\psi$ is used to pick the best direction $\hat{\bu}$.

Two key ideas are used in the algorithm design.
The first idea is to use multiple initializations from the dataset by using $\bu_i = \frac{\bX_i}{\mynorm{\bX}}$ as initialization inspired by \citet{spielman2012exact, QuSW14}
Intuitively, initializing closer to the planted vector allows for a more accurate estimation of the gradient, which allows a decrease in sample complexity.
If the planted vector's distribution has heavier tails than the normal distribution, using normalized samples from the distribution can provide initializations closer to the signal direction than uniformly random samples.
For example if $\bbP\left[\nu = \sqrt{1/p}\right] = p$ then with probability $p$ we have an initialization for which $\ainner \approx \sqrt{1/ (d p)}$.
Instead of choosing $\bu$ uniformly at random, we only have $\ainner \approx \sqrt{1/d}$, which can be much worse in the case when $p$ is sufficiently small.
This method requires using many initializations simultaneously without knowing which initializations are close to the signal direction.
Successfully converged projections are then detected and returned.
This initialization scheme is shown in \autoref{twoStepAlgorithm}.

The second idea is to use minibatches to avoid being stuck in local optima.
As previously noted, this comes at the cost of needing new samples for each gradient ascent step.
Thus, speeding up convergence is necessary to decrease the algorithm's sample complexity.
We do this by using the Riemannian gradient.
$
    \left( \bI_d - \bu \bu^{\top}\right)
    \frac{\partial }{\partial \bu}
    \left(
        \sum_{i = 1}^{n} \frac{\phi\left(\inner{\bX_i, \bu}\right)}{n}
    \right)
$
instead of the gradient itself, which allows us to decrease the number of steps needed to converge.

Using the Riemannian gradient itself also has a downside.
If $\ainner$ becomes sufficiently large, we cannot guarantee that a gradient step does not degrade the current estimate of the signal direction.
Thus, we use a schedule for the learning rate $\eta$, decreasing $\eta$ once we are close to convergence.
The algorithm is presented in \autoref{twoStepAlgorithm}, which calls a subroutine described in \autoref{the_alg}.
\autoref{the_alg} runs Riemannian gradient ascent with a large learning rate $\eta_1$ and then extracts the solution with the largest value for the projection index, ensuring that a good solution is found.
A second run of \autoref{the_alg} with a lower learning rate $\eta_2$ is used to fine-tune the projection to find a close estimate of the signal direction $\bu^*$.
As we use multiple initializations, we end up with multiple estimates of the signal direction. 
Additionally, we are unsure if an estimate may have diverged during gradient ascent.
Thus, we use the second projection index $\psi(\cdot)$ to pick the best estimate of the signal direction.
\vspace{-0.5cm}
\\
\noindent
\begin{minipage}[t]{.5\textwidth}
\begin{algorithm}[H]
    \caption{Two-Step Gradient Ascent Algorithm}
    \label{twoStepAlgorithm}
    \begin{algorithmic}[1]\Function{two\_step\_gradient\_ascent}{$\bX, n, n_{init}, s, \eta_1, \eta_2$} 
        \For{$j = 1 ... n_{init}$}
            \State $\bu_{j} \gets \frac{ \bX_{j} }{ \mynorm{\bX_{j}} }$
        \EndFor
        \State $\hat{\bu} \gets 
        \Call{gradient\_ascent}{
            \{\bX_i\}_{i = n_{init} + 1}^{n_{init} + n s}, 
            \bu, n, 
            \eta_1, 
            s
        }$
        \vspace{0.1cm}
        \State $\hat{\bu} \gets 
        \Call{gradient\_ascent}{
            \{\bX_i\}_{i = n_{init} + n s + 1}^{n_{init} + 2 n s}, 
            \hat{\bu}, n, 
            \eta_2, 
            s
        }$
        \vspace{0.1cm}
        \State \Return $\argmax_{\hat{\bu} \in \{\hat{\bu}_{j} | j \in [n_{init}]\}} \sum_{k = 1}^{n} \frac{\psi(\inner{\bX_k, \hat{\bu}})}{n}$
    \EndFunction
    \end{algorithmic}
\end{algorithm}
\end{minipage}
\begin{minipage}[t]{.5\textwidth}
\begin{algorithm}[H]
    \caption{Gradient Ascent}
    \label{the_alg}
    \begin{algorithmic}[1]
        \Function{gradient\_ascent}{$\bX, \bu, n, \eta, s$} 
            \label{subroutine}
            \For{$i = 0 ... (s-1)$}
                \State Choose $\bar{\bX} \gets \{\bX_k\}_{k = n i}^{n (i + 1)}$
                \For{$j = 1 ... n_{init}$}
                    \State Calculate 
                    \State \hspace{0.25cm} $\bg \gets
                    \left( \bI_d - \bu_{i, j} \bu_{i, j}^{\top}\right)
                    \frac{\partial }{\partial \bu_{i, j}}
                    \left(
                        \sum_{k = 1}^{n} \frac{\phi\left(\inner{\bar{\bX}_k, \bu_{i, j}}\right)}{n}
                    \right)
                    $
                    \State Update $\ubar_{i, j} \gets \bu_{i, j} + \eta \bg$
                    \State Renormalize $\bu_{i+1, j}\gets \frac{ \ubar_{i+1, j} }{ \|\ubar_{i+1, j}\|_2 }$
                \EndFor
            \EndFor
            \For{$j = 1 ... n_{init}$}
                \State $\hat{i} \gets 
                    \argmax_{i \in [s]}
                    \sum_{k = 1}^{n} \frac{\psi(\inner{\bX_k, \bu_{i, j}})}{n}
                $
                \State $\hat{\bu}_j \gets \bu_{\hat{i}, j}$
            \EndFor
            \State \Return $\hat{\bu}$
        \EndFunction
    \end{algorithmic}
\end{algorithm}
\end{minipage} 
\subsection{Sample Complexity Bounds}
\label{scBounds}
Next, we will highlight a method of studying the sample complexity of the Gradient Ascent Subroutine (\autoref{the_alg}) when specified towards a planted vector distribution and a projection index.
We state three assumptions that must be fulfilled by the setting and the projection index to demonstrate convergence.
This analysis can then be applied to both uses of \autoref{the_alg} to complete the analysis of \autoref{twoStepAlgorithm}.
Later, in \autoref{Relu2Section} and \autoref{KurtosisSection}, we show that \autoref{twoStepAlgorithm} can recover planted vectors with close to optimal sample complexity.

\autoref{mainTheorem} gives a convergence result for arbitrary $\phi(\cdot), \psi(\cdot)$ and a planted vector distribution.
To apply \autoref{mainTheorem}, we have to demonstrate the following preconditions hold.

\autoref{assInitialization} of \autoref{mainTheorem} guarantees that at least one initialization is close enough to the signal direction, providing a good starting point for our algorithm.
\autoref{assGradient} ensures that the gradient estimates are sufficiently accurate for each step.
This also ensures that renormalization does not decrease $\ainner$.
Finally, \autoref{assTestability} ensures that the projection index $\psi(\cdot)$ can be used to (sample-)efficiently test if an initialization has converged.
In most cases, choosing $\psi = \phi$ is entirely sufficient, but selecting a convenient $\psi$ can oftentimes drastically ease the analysis.
This is necessary in the last step of the algorithm to select a converged estimate.
In the following we will use $g_{\bu}(\bx) := (\bI - \bu \bu^\top) \frac{\partial \phi(\inner{\bx, \bu})}{\partial \bu}$ for simplicity.
\begin{lemma}
\label{mainTheorem}
Let $\bX \sim \cD^n$.
For $s = \bigOmega{\log(d)}$ steps, $\delta \geq 0$, $1 > b > b-\delta > a > 0$, $\delta > 0$ and $n > 0$, if
\begin{enumerate}
    \item \label{assInitialization}
    Given $\bu_{0, i}$ for $i \in [n_{init}]$ 
    then with probability at least $1 - \smallO{1}$
    $$
        \max_i \inner{\bu_i, \bu^*} \geq a$$
    
    \item \label{assGradient}
    For an arbitrary constant $c_0 > 0$, if $\ainner \in (a, b)$ then
    $$
    \frac{
        \ainner + \eta \inner{\frac{\sumin g_\bu(\bX_i)}{n}, \bu^*}
    }{
        \sqrt{1 + \eta^2 \frac{\sumin g_\bu(\bX_i)}{n}}
    }
    \geq
    (1 + c_0)
    \ainner
    $$
    with probability at least $1 - \bigO{\frac{1}{s}}$.

    \item \label{assTestability}
    There exists a threshold $t$ where for all $\bu \in \bbS_{d-1}$
    if $\inner{\bu^*, \bu} \geq b$ then $\sumin \frac{\psi(\inner{\bX_i, \bu})}{n} \geq t$ 
    and
    if $\ainner \leq b - \delta$ then $\sumin \frac{\psi(\inner{\bX_i, \bu})}{n} \leq t$, with probability at least $1 - \smallO{1}$.
\end{enumerate}

Then, \autoref{the_alg} returns $\hat{\bu}$ such that $\max_{i \in [n_{init}]} \inner{\hat{\bu}_i, \bu^*} \geq b - \delta$ with a total of $\widetilde{\mathcal{O}}(n)$ samples steps with probability at least $1 - \smallO{1}$.
\end{lemma}

The proof of \autoref{mainTheorem} can be found in \autoref{proofMainTheorem}.
Thus, to analyze the performance of \autoref{twoStepAlgorithm}, we apply \autoref{mainTheorem} once for each execution of \autoref{the_alg}. \subsection{Application of \autoref{mainTheorem} to Imbalanced Clusters}
\label{Relu2Section}
Here, we will focus on a $\cB(p)$ containing two imbalanced clusters with an imbalance parameter $p$.

\begin{definition}[Imbalanced Clusters]
    \label{icDefinition}
    We say $\nu \sim \cB(p)$, with $p \in (0,1)$, if 
    $
        \nu = 
        \begin{cases}
          \sqrt{(1-p) / p}, & \text{with probability } p \\
          - \sqrt{p / (1-p)}, & \text{with probability } (1-p)
        \end{cases}
    $
\end{definition}
The cluster centers are chosen so that the mean is zero and the variance is one.
Due to the data having unit variance, methods such as PCA cannot recover the planted vector and need Projection Pursuit methods.
Note that for smaller $p$, the first cluster moves further away from the origin, and the second cluster shifts closer to the origin.
This behaves similarly to the Bernoulli-Rademacher distribution.
Here, we will be interested in the parameter $p \in (\frac{1}{\sqrt{d}}, \frac{1}{2})$.
For larger $p > \frac{1}{2}$, the same results follow by symmetry with the notable exception of $p = \frac{1}{2}$ where the clusters are perfectly balanced.
We choose to use the projection index $\phi(x) = \max\{0, x\}^2$.
In \autoref{relu2_sc} we demonstrate bounds on the sample complexity of \autoref{twoStepAlgorithm} using $\phi(x) = \psi(x) = \max\{0, x\}^2$.

\begin{theorem}
    \label{relu2_sc}
    For arbitrary $\beta > 0$ and $p \in \left( \frac{1}{2}, \frac{1}{\sqrt{d}}\right)$ there exist 
$\eta_1 = \bigOmega{\sqrt{d} p}$, $\eta_2 = \bigTheta{1}$, $s = \Theta(\log(d))$, $n_{init} = \Omega\left( 1 / p \right)$ and $n = \bigThetas{d^2 p^2}$ such that for sufficiently large $d$ and sufficiently small $p$, Projection Pursuit using \autoref{twoStepAlgorithm} 
    with $\bX \sim \cD_{\cB(p)}^n$ and a Projection Index $\phi(x) = \max\{0, x\}^2$
    will output $\hat{\bu}$ such that $\inner{\hat{\bu}, \bu^*} \geq 1 - \beta$ 
    with probability at least $1 - \smallO{1}$ utilizing a total of $\bigThetas{d^2 p^2}$ samples.
\end{theorem}
The proof of \autoref{relu2_sc} can be found in \autoref{relu2_sc_proof}.
\subsection{Application of \autoref{mainTheorem} to Bernoulli-Rademacher Planted Vectors}
\label{KurtosisSection}
Other commonly studied settings are the Bernoulli-Rademacher and Bernoulli-Gaussian settings~\cite{mao2022optimal, hopkins2016fast, DictionaryLearningKurtosis}.
These are both sparse distributions, i.e., are $0$ with probability $1-p$, thus are of particular interest in compressed sensing~\cite{spielman2012exact, DictionaryLearningKurtosis}.
We will prove that a Bernoulli-Rademacher planted vector can be recovered using gradient-based techniques.

\begin{definition}[Bernoulli-Rademacher \cite{spielman2012exact}]
    \label{brDefinition}
    We say $\nu \sim \BR(p)$, with $p \in (0,1)$, if
    $
        \nu = 
        \begin{cases}
          \sqrt{1 / p}, & \text{with probability } p / 2 \\
          - \sqrt{1 / p}, & \text{with probability } p / 2 \\
          0, & \text{with probability } (1-p)
        \end{cases}
    $
\end{definition}

We demonstrate, that \autoref{twoStepAlgorithm} using the projection index $\phi(x) = x^4$ can recover the planted vector using $n = \widetilde{\mathcal{O}}(d^3 p^4)$ samples.

\begin{theorem}
    \label{kurtosis_sc}
    For arbitrary $\beta > 0$ there exist 
$\eta_1 = \bigOmega{d p^2}, \eta_2 = \bigTheta{1}, s = \bigOmega{\log(d)}, n_{init} = \Theta\left( 1 / p \right)$ and ${n = \bigThetas{d^3 p^4}}$ such that for sufficiently large $d > 0$ and sufficiently small $\frac{1}{3} > p > 0$, Projection Pursuit using \autoref{twoStepAlgorithm} 
    with $\bX \sim \cD_{\BR(p)}^n$, $\phi(x) = x^4$ and $\psi(x) = -|x|$
    will output $\hat{\bu}$ such that $\inner{\hat{\bu}, \bu^*} \geq 1 - \beta$ 
    with probability at least $1 - \smallO{1}$ utilizing a total of $\bigThetas{d^3 p^4}$ samples.
\end{theorem}
The proof can be found in \autoref{kurtosisProofs}. 
\section{Statistical Computational Lower Bounds of the Planted Vector Setting}
\label{ldplb}
Here, we study whether gradient-based methods are optimal in the sense of matching computational lower bounds.
For this, we compare the sample complexity of gradient-based methods to computational lower bounds, which assess the minimum sample complexity required for any \emph{computationally efficient} algorithm (i.e., computable in polynomial time) to recover the planted vector, as defined in \autoref{planted_vector_setting}.
As this is not tractable for such a general class of algorithm, there have been rigorous results in more limited settings such as lower bounds for the statistical query model~\cite{kearnsStatisticalQuery1993}, sum of squares hierarchies~\cite{sumofsqares} and the Low Degree Polynomial Framework~\cite{kunisky2019notes, hopkins2018statistical}.
Here, we will focus on the framework of Low Degree Polynomials.

Generally, the Low-Degree Polynomial Framework uses Low-Degree Polynomials as a surrogate for efficiently computable algorithms to determine whether an efficient algorithm exists for deciding a hypothesis testing problem.
We will utilize this to obtain lower bounds on the sample complexity of efficiently computable tests.
Bounds can be obtained using the optimality of the likelihood ratio test~\cite{kunisky2019notes}.
Let $L := \frac{d\cP}{d\cQ}$ be the likelihood ratio.
\citet{Neyman1992} shows that thresholding the likelihood ratio $L$ is an optimal test, thus allowing reasoning about computational lower bounds.
The Low Degree Polynomial Framework focuses on the degree-$D$ likelihood ratio $L_d^{\leq D}$ which is defined as the likelihood ratio projected onto the subspace of polynomials of degree at most $D$, where $D$ is low, i.e., logarithmic in the size of the problem.
By demonstrating that a likelihood ratio test using $L_d^{\leq D}$ fails, we can prove that no polynomial of degree $\leq D$ can be used to construct a test to distinguish $\cP$ and $\cQ$.
This is summarized in the following conjecture.
\begin{conjecture}[Low Degree Conjecture \cite{hopkins2018statistical}]
    \hypertarget{hopkinsConjecture}
    For "sufficiently nice" sequences of probability measures $\cQ$ and $\cP$, if there exists $\epsilon > 0$ and degree $D \geq \log(d)^{1+\epsilon}$ for which $\|L_d^{\leq D}\|$ remains bounded as $d \to \infty$, then there is no polynomial-time algorithm $f$ for which if $\bX \sim \cQ$ then $f(\bX) = \cQ$ and if $\bX \sim \cP$ then $f(\bX) = \cP$ with high probability. 
\end{conjecture}
\subsection{Planted Vectors with Imbalanced Clusters}
To our knowledge, computational lower bounds have not been studied for the planted vector in the Imbalanced Clusters setting.
To obtain lower bounds on the sample complexity needed to recover a close estimate of the planted vector in polynomial time, we follow a three-step procedure following the method used in \citet{mao2022optimal}.
First, we formulate a hypothesis testing problem in \autoref{testing}, which tests between a Gaussian distribution and the Planted Vector distribution as defined in \autoref{planted_vector_setting}.
Then, we demonstrate computational lower bounds on \autoref{testing} in the Low Degree Polynomial Framework. 
Finally, we extend the computational lower bounds to the estimation problem of finding a direction $\hat{\bu}$ close to the signal direction $\bu^*$ such that $\ainnerhat \geq 1-\beta$.
This is done by reducing the estimation problem to \autoref{testing}.

\begin{problem}
\label{testing}
Let $\nu$ be a distribution over $\Real$.
Define the following null and planted distributions: 
\begin{itemize}
\item 
Under $\cQ$, observe i.i.d.\ samples $\bX_1,\ldots, \bX_n \sim \cN(0,\bI_d)^n$. 
\item 
Under $\cP$, first draw $\bu^*$ uniformly from $\bbS_{d-1}$ and i.i.d. $\bnu_1 , \cdots, \bnu_n$. Conditional on $\bu^*$ and $\{\bx_i\}$, draw independent samples $\bX_1,\ldots,\bX_n \in \Real^d$ where $\bX_i \sim \cN(\bnu_i \bu^*, \bI_n - \bu^*{\bu^*}^\top)$.
Note that this is equivalent to $\cD_{\cF}$ in \autoref{planted_vector_setting}.
\end{itemize}
Suppose that we observe the matrix $\bX \in \Real^{n \times d}$ with rows $\bX_1^\top, \dots, \bX_n^\top$. We aim to test between the hypotheses $\cQ$ and $\cP$. 
\end{problem}

In the following, we will show evidence of a computational statistical gap in the Planted Vector Setting \autoref{planted_vector_setting}.
Specifically, we will demonstrate that for $n = \widetilde{\cO}(d^{1.5} p)$ the Low Degree Likelihood ratio stays bounded and thus, according to the \conjectureRef\hspace{-0.1cm}, no polynomial time algorithm can test \autoref{testing}.
\begin{theorem}
\label{thmLDPLB}
For an instance of \autoref{testing} with $\bnu \sim \cB(p)^n$ and $n = \widetilde{\cO}(d^{1.5} p)$ the Low Degree Likelihood Ratio stays bounded for degree $D = \log(d)^{1+\epsilon}$ for $\epsilon > 0$.
$${\ldlr^2} \leq 2$$
\end{theorem}

Finally, we reduce the estimation problem to \autoref{testing}.
To do this, we have to demonstrate that it is possible to construct a test for \autoref{testing} if we have access to an estimate $\hat{\bu}$ for $\bu^*$ such that $\inner{\hat{\bu}, \bu^*} \geq 1 - \beta$.
\autoref{reduction} shows it is possible to construct such a test for \autoref{testing} if $n = \Omega(d)$.
\begin{corollary}
    \label{reduction}
    For all $\hat{\bu}$ for which $\inner{\hat{\bu}, \bu^*} \geq 1 - \beta$ for sufficiently small $\beta > 0$ and $\frac{1}{\sqrt{d}} \leq p < \frac{1}{2}$.
    Define the test $\Psi$:
    \vspace{-0.1cm}
    \begin{align*}
        \Psi :=
        \begin{cases}
            \cQ & \, \text{if} \, \sumin \frac{\phi(\inner{\bX_i, \hat{\bu}})}{n} < t \\
            \cP &\, \text{if} \, \sumin \frac{\phi(\inner{\bX_i, \hat{\bu}})}{n} \geq t
        \end{cases}
    \end{align*}
    \vspace{-0.1cm}
    With $\phi(x) = \max\{0, x\}^2$.
    There exists a threshold $t$ such that
    \vspace{-0.1cm}
    $$
        \bbP_{\cQ}\left\{ \Psi = \cP \right\}
        +
        \bbP_{\cP}\left\{ \Psi = \cQ \right\}
        \leq 
        \exp \left( - \bigTheta{\frac{n}{d}} \right)
    $$
\end{corollary}

In the regime where $n < d+1$, we refer to \citet{zadik2022latticebased}, demonstrating the statistical impossibility of estimation in this regime.
Thus, combining \autoref{reduction} and \autoref{thmLDPLB} yields the result that if the \conjectureRef is true, no polynomial time can estimate the planted vector.
To our knowledge, currently, no efficient algorithm exists that can recover the signal direction with $n = o(d^2 p^2)$ samples.
\begin{remark}
    \citet{dudeja2024statisticalcomputational} gives results on the failure of Low-Degree Polynomials in the Planted Vector Setting.
    If for $i \in {1, \cdots, k-1}$ the moments of $\nu$ and the standard normal distribution match a bound of $n \ll d^{k/2} \lambda^{-2}$, where $\lambda$ is the signal to noise ratio defined as 
    $
        \left|\bbE[\nu^k] - \bbE[Z^k] \right| = \lambda \quad Z \sim \cN(0,1)
    $.
    In the case of $\nu \sim \cB(p)$ for $k = 3$ we obtain $\lambda \geq \frac{2}{\sqrt{p}}$.
    Here, we note that the bound is not applicable to our setting as by choosing $p$ sufficiently small, Assumption 2 cannot be fulfilled anymore.
    Thus, for completeness, we give the same bound of $n = \widetilde{\cO}(d^{1.5} p)$ which is valid for $p \in \left(\frac{1}{\sqrt{p}}, \frac{1}{2} \right)$.
\end{remark}
\subsection{Bernoulli Rademacher Planted Vectors}
The Bernoulli Rademacher setting has been thoroughly studied in \citet{mao2022optimal}.
Here, the failure of Low Degree Polynomials when $n = \widetilde{O}(d^{2} p^2)$ is demonstrated.
This lower bound is known to be tight, as a spectral algorithm can recover a Bernoulli-Rademacher planted vector with $n = \widetilde{\Theta}(d^{2} p^2)$ samples, which is tight, as there exist spectral methods which can recover the planted vector when $n = \widetilde{\Omega}(d^2 p^2)$ samples~\cite{hopkins2016fast, mao2022optimal}.
Our gradient-based method has a sample complexity of $n = \widetilde{\cO}(d^3 p^4)$, which is larger than what can be achieved using spectral methods.
In the case where $p = \frac{1}{\sqrt{d}}$, this matches the bounds obtained using spectral methods.
\begin{remark}
\citet{zadik2022latticebased} gives an algorithm to recover planted vectors using LLL-basis reduction.
This algorithm does not exhibit the statistical-to-computational gap.
This algorithm is only applicable in a very restrictive setting, as it is required that the planted vector can only take on a set of discrete values.
This can be avoided by considering a setting with a small amount of noise.
E.g., considering a hierarchical setting where $\cB(p')$ with $p' \sim \cU([\frac{p}{2}, p])$ is a simple counterexample, in which our analysis still works and where the algorithm discussed in \citet{zadik2022latticebased} fails due to the lack of robustness to noise.
\end{remark}  
\section{Experiments}
\subsection{Experiments with Synthetic Data}
\label{simulations}
In the following, we will validate the findings in \autoref{relu2_sc} and \autoref{kurtosis_sc} by running \autoref{twoStepAlgorithm} on synthetic datasets.
In the Imbalanced Clusters setting, we will be choosing the dimension $d \in \{16, ..., 512\}$ and the cluster imbalance for $p \in \{d^{-0.5}, d^{-0.3}, 0.3\}$ with $n \in \{16, ..., 2048\}$.
The algorithm is executed for $s = 2 \log_2 d$ steps and $\eta_1 = \sqrt{d} p$ and $\eta_2 = 0.5$.
The results are plotted in \autoref{asymptotics_experiment}.
In the Bernoulli-Rademacher setting we will be choosing the dimension $d \in \{16, ..., 256]$ and the cluster imbalance as either $p \in \{d^{-0.5}, d^{-0.5}, 0.3\}$ with $n \in \{16, ..., 4096\}$.
The algorithm is executed for $s = 2 \log_2 d$ steps and $\eta_1 = \sqrt{d} p$ and $\eta_2 = 0.5$.
The results are plotted in \autoref{asymptotics_experiment_kurtosis}.
To evaluate, we plot the average value of $\inner{\hat{\bu}, \bu^*}$ over $30$ independently sampled datasets for each $d, n, p$.
It can be observed that for sufficiently large $d$, the asymptotics closely match the experimental sample complexity.
\begin{figure}[H]
    \vspace{-0.3cm}
    \centering
    \begin{subfigure}[b]{0.8\textwidth}
        \centering
        \vspace{-0.3cm}
        \includegraphics[scale = 0.4]{relu2.eps}
        \vspace{-0.7cm}
        \caption{
            Projection pursuit of Imbalanced Clusters using \autoref{twoStepAlgorithm} with $\phi(x) = \psi(x) = \max\{0, x\}^2$.
            The red lines of slopes $(2, 1.3, 1)$ roughly highlight the phase transition.
        }
        \label{asymptotics_experiment}
    \end{subfigure}
    \begin{subfigure}[b]{0.8\textwidth}
        \centering
        \vspace{-0.1cm}
        \includegraphics[scale = 0.4]{kurtosis.eps}
        \vspace{-0.7cm}
        \caption{
            Projection pursuit of a sparse Bernoulli-Rademacher distribution using \autoref{twoStepAlgorithm} with $\phi(x) = x^4$ and $\psi(x) = -|x|$.
            The red lines of slopes $(3, 1.8, 1)$ roughly highlight the phase transition.
            }
        \label{asymptotics_experiment_kurtosis}
    \end{subfigure}
    \caption{
        Phase transitions in Projection Pursuit using gradient-based algorithms.
        The horizontal and vertical axes correspond to $\log_2 d$ and $\log_2 n$.
        Each pixel shows the average value of the absolute inner product between the predicted and signal directions, where white corresponds to 1 and black to 0.
    }
\end{figure}
\subsection{Comparison to Other Methods}
\label{discussion}
Here, we will apply \autoref{twoStepAlgorithm} to the planted vector problem with other projection indices.
Specifically, we will consider a list of projection pursuit methods as presented in \autoref{fig:pp_inidces}.
\begin{figure}[H]
    \begin{minipage}[t]{.67\textwidth}
    \renewcommand{\arraystretch}{1.3}
    \begin{tabular}{ r | l | l}
        \textbf{Abbreviation} & $\sumin \frac{\phi(\bX_i)}{n}$ & $\sumin \frac{\psi(\bX_i)}{n}$ \\
        \hline & \\[-2ex]
        \textbf{ReLU2} & $\average{\max\{0, \bX_i\}^2}$ & $\average{\max\{0, \bX_i\}^2}$\\
        \textbf{Kurtosis} & $\average{\bX_i^4}$ & $\average{-|\bX_i|}$\\
        \textit{Abs} & $\average{-|\bX_i|}$ & $\average{-|\bX_i|}$\\
        \textit{AbsMax} & $\average{|\bX_i|}$ & $\average{|\bX_i|}$\\
        \textit{Skewness} & $\average{\bX_i^3}$ & $\average{\bX_i^3}$\\
        \textit{ApproxEntropy} & $(\average{\bX_i^3})^2  + (\average{\bX_i^4} - 3)^2$ & $(\average{\bX_i^3})^2 + (\average{\bX_i^4} - 3)^2$\\
    \end{tabular}
    \end{minipage}
    \begin{minipage}[t]{.33\textwidth}
    \begin{tabular}{ r | l }
        \textbf{Abbreviation} & Method \\
        \hline & \\[-2ex]
        \textit{Cov4max} & $\mathbf{v}_{\lambda_{\max}}$ $(\sumin \mynorm{\bX_i}^2 \bX_i \bX_i^\top)$ \\
        \textit{Cov4min} & $\mathbf{v}_{\lambda_{\min}}(\sumin \mynorm{\bX_i}^2 \bX_i \bX_i^\top)$ \\
        \textit{3TensorDecomp} & $\mathbf{v}_{\sigma_{\min}}(\sumin \bX_i \otimes \bX_i \bX_i^\top)$ \\
    \end{tabular}
    \end{minipage}
    \caption{
        Methods used in the empirical comparison.
        On the left: Methods using the gradient-based algorithm.
        On the right: Spectral methods.
    }
    \label{fig:pp_inidces}
\end{figure}
\textbf{ReLU2} and \textbf{Kurtosis} correspond to the projection indices we study in \autoref{relu2_sc} and \autoref{kurtosis_sc}.
Additionally, we test the projection indices \textit{Abs}, which corresponds to the projection index used in \citet{QuSW14}, \textit{AbsMax}, which corresponds to the objective in \citet{davis2021clustering} and \textit{Skewness} as proposed in \citet{skewness2004paajarvi}.
\textit{ApproxEntropy} uses the entropy approximation as introduced by \citet{NIPS1997_6d9c547c}.
We choose this selection of projection indices as a representative sample of projection indices studied in the literature. 
We choose projection indices that consist of basic functions as compared to more complex projection indices (e.g., \citet{PPFriedmanTukey}) to ensure that they work well with gradient-based optimization.

Additionally, we compare the gradient-based methods to three spectral methods.
\citet{nicolaSkewness} introduces a method called MaxSkew for finding directions of large skewness(abbreviated by \textit{3TensorDecomp}).
\textit{Cov4max} and \textit{Cov4min} denote the eigenvectors corresponding to the largest and smallest eigenvalue of a kurtosis matrix as introduced in \citet{Pena2010} and \citet{Kollo2008}.
This kurtosis matrix is also studied in the context of sample efficient recovery of planted vectors \cite{mao2022optimal} and clustering of Gaussian mixtures \cite{davis2021clustering}.

Fully resampling mini-batches seems to be unnecessary if the dimension is sufficiently low.
Thus, it tends to be beneficial to subsample the dataset with replacement, which we will do for the following experiments.
Here, we compare the previously mentioned methods in the planted vector setting with a Bernoulli Rademacher planted vector in \autoref{br_comp} and with an Imbalanced Clusters planted vector in \autoref{ic_comp} with $d = 300$, $p = 0.1$.
For \autoref{twoStepAlgorithm} we use $n_{init} = 400$ initializations.
In \autoref{fig:spectral_methods_comparison}, we can observe that most algorithms only perform well on one of both settings. 
In \autoref{br_comp}, we observe that \textit{Cov4max} performs best closely followed gradient-based projection pursuit using \textbf{Kurtosis}.
This behavior is as expected by bounds on the sample complexity in the Bernoulli Rademacher setting, as \textit{Cov4max} only needs $\bigOmegas{d^2 p^2}$ samples to recover the signal direction~\cite{mao2022optimal}.
In the planted vector setting with Imbalanced Clusters \textbf{ReLU2} performs best.
\begin{figure}[H]
    \vspace{-0.3cm}
    \centering
    \begin{subfigure}[b]{0.49\textwidth}
        \includegraphics[scale = 0.4]{comparison_br_mean_large.eps}
        \caption{Bernoulli-Rademacher}
        \label{br_comp}
    \end{subfigure}
    \hfill
    \begin{subfigure}[b]{0.49\textwidth}
        \includegraphics[scale = 0.4]{comparison_ic_mean_large.eps}
        \caption{Imbalanced Clusters}
        \label{ic_comp}
    \end{subfigure}
    \caption{
        Comparison of different methods in the planted vector setting.
        We plot the average inner product between the signal direction and the recovered direction by each algorithm over 30 datasets.
    }
    \label{fig:spectral_methods_comparison}
\end{figure}
\subsection{Experiments with Real Data}
\label{experiments}
We compare the algorithms by measuring their performance on two different datasets: FashionMNIST~\cite{xiao2017fashionmnist} and the Human Activity Recognition Dataset~\cite{human_activity_recognition_using_smartphones_240}.
The FashionMNIST dataset contains $28 \times 28$ gray-scale images of fashion articles with articles labeled into $10$ categories denoting the article type (T-shirt/top, Trouser, Pullover, Dress, Coat, Sandal, Shirt, Sneaker, Bag, Ankle boot).
The Human Activity Recognition Dataset~\cite{human_activity_recognition_using_smartphones_240} has been collected from 30 subjects performing daily tasks while carrying a smartphone for inertial measurements.
Each sample consists of 561 summary statistics of the inertial measurements and is labeled by the performed activity (Walking, Walking Upstairs, Walking Downstairs, Sitting, Standing, Lying) by the subject.

For the experiments, we reduce the dimension to $100$ using PCA.
We use $n_{init} = 500$ initializations for the gradient-based algorithm and choose the $30$ directions with the largest value for the projection index.
For spectral methods, we run the spectral method $30$ times while removing the recovered directions from the dataset.
To compare the performance of the different methods, we evaluate how well a single projection can help predict the labels of the images.
This will be evaluated using the Information Gain $\mathrm{IG}(Y,A)=\mathrm {H}(Y)-\mathrm {H}(Y|A)$.
Where $Y$ are the labels assigned to the instances in the respective datasets.
Here, we choose the indicator $A = \bbI_{\inner{\bu, \bx} > t}$, where $t$ is an automatically chosen threshold for each index to maximize information gain on the training dataset.
The final information gain is evaluated on a holdout dataset.
We evaluate the projection pursuit methods on both a very small and a very large training set to determine which methods perform well with only a small number of samples and which need a large number of samples.

In \autoref{fig:InformationGain}, we plot the results for the FashionMNIST dataset, and in \autoref{fig:InformationGain_HAR}, we plot the results for the Human Activity Recognition Dataset.
Here, we can observe that the \textbf{ReLU2} projection index still performs well even if only a few samples are present, outperforming all other methods.
Other methods such as \textit{AbsMax} can only be optimized with significantly more samples but perform significantly better if optimization is successful than other projection pursuit methods.
Additionally, we note that using \textit{Kurtosis} seems to perform very similarly to \textit{Abs}, which is to be expected due to both algorithms using the same projection index to select projections.
We hypothesize that this is because the gradient of \textit{Kurtosis} is less stable than other projection indices such as \textit{Abs} offsetting the faster convergence of \textit{Kurtosis}, which seems to be less important in lower dimensional data.
\begin{figure}[H]
    \vspace{-0.3cm}
    \centering
        \begin{subfigure}[b]{0.3\textwidth}
            \centering
            \includegraphics[width=\textwidth]{small_fashion_mnist_information_gain.eps}
            \caption{$n = 600$}
        \end{subfigure}
        \begin{subfigure}[b]{0.3\textwidth}
            \centering
            \includegraphics[width=\textwidth]{large_fashion_mnist_information_gain.eps}
            \caption{$n = 60000$}
        \end{subfigure}
    \caption{
        A comparison of projection pursuit approaches on fashionMNIST.
        We plot the achieved information gain using projections produced by different projection indices.
        The box plots are generated with the 30 candidate projections generated by the algorithms.
    }
    \label{fig:InformationGain}
\end{figure}
\begin{figure}
    \vspace{-0.3cm}
    \centering
        \begin{subfigure}[b]{0.3\textwidth}
            \centering
            \includegraphics[width=\textwidth]{small_har_information_gain.eps}
            \caption{$n = 300$}
        \end{subfigure}
        \begin{subfigure}[b]{0.3\textwidth}
            \centering
            \includegraphics[width=\textwidth]{large_har_information_gain.eps}
            \caption{$n = 7352$}
        \end{subfigure}
    \caption{
        A comparison of projection pursuit approaches on the Human Activity Recognition using Smartphones Benchmark.
        We plot the achieved information gain using projections produced by different projection indices.
        The box plots are generated with the 30 candidate projections generated by the algorithms.
    }
    \label{fig:InformationGain_HAR}
\end{figure}
For demonstration purposes, we also show histograms of the data projected onto the recovered projections in \autoref{fig:histograms_fmnist_r2} and \autoref{fig:histograms_fmnist_kurtosis}.
In \autoref{fig:histograms_fmnist_r2}, it can be observed that the found projections reveal two Imbalanced Clusters where the smaller cluster contains samples of mostly one class. 
\autoref{fig:histograms_fmnist_kurtosis} shows that the Kurtosis projection index recovers projections for which many samples are projected close to zero while a few are projected away from zero, similar to a Bernoulli-Rademacher distribution.
\begin{figure}
    \vspace{-0.3cm}
    \centering
    \begin{subfigure}[b]{\textwidth}
    \centering
        \includegraphics[scale = 0.4]{hist_ReLU2.eps}
        \caption{4 recovered projection using the ReLU2 projection index}
        \label{fig:histograms_fmnist_r2}
    \end{subfigure}
    \begin{subfigure}[b]{\textwidth}
        \centering
        \includegraphics[scale = 0.4]{hist_Kurtosis.eps}
        \caption{4 recovered projection using the Kurtosis projection index}
        \label{fig:histograms_fmnist_kurtosis}
    \end{subfigure}
    \caption{
        Histograms of projections recovered by \autoref{twoStepAlgorithm} on FashionMNIST using $n = 600$ samples.
        Classes are plotted using different colors.
    }
    \label{hists}
\end{figure}
\autoref{fig:histograms_fmnist_r2} provides insight into why discovering imbalanced projections effectively recovers label-separating projections in classification tasks such as Fashion-MNIST. 
We can reasonably assume that the data of a classification task follows a cluster structure.
If the data is projected in the direction of one cluster center, we expect that the samples from the other clusters will collapse close to $0$, while the samples from the chosen cluster will move out to one side.
This effect can also be observed in \autoref{fig:histograms_fmnist_r2}.
Consequently, projecting the data along the direction of a particular cluster center naturally yields an imbalanced histogram, with samples from the corresponding cluster positioned towards one extreme.
\section{Conclusion}
We consider the performance of gradient-based algorithms for projection pursuit in the planted vector setting, where we study their sample complexity.
Specifically, we consider the setting where the planted vector follows a distribution containing two clusters of imbalanced size or a Bernoulli-Rademacher distribution.
In the former setting, Low-Degree Polynomials give a lower bound of $n = \widetilde{\Theta}(d^{1.5} p)$ and gradient-based methods can recover the signal direction provably with $n = \widetilde{\Omega}(d^2 p^2)$ thus presenting a gap of a factor of $p \sqrt{d}$ which increases as $p$ increases.
It is currently unknown whether an algorithm exists that matches the sample complexity of the lower bound in the Low-Degree Polynomial Framework.
In the latter setting, $n = \widetilde{\Omega}(d^3 p^4)$ samples are sufficient, and there exist spectral algorithms matching the computational lower bounds of $n = \widetilde{\Omega}(d^2 p^2)$ samples.
Although there still exists a gap between gradient-based methods and computational lower bounds, we can observe that in both settings, if the distribution is very imbalanced/sparse, gradient-based methods match computational lower bounds closely.
Finally, we demonstrate the favorable performance of our algorithm if the number of samples is severely limited.
\section{Acknowledgements}
This work has been supported by the German Research Foundation (DFG) through DFG-ANR PRCI ``ASCAI" (GH 257/3-1).
\appendix
\bibliographystyle{myjmva}
\bibliography{bib}
\section{Proof of \autoref{mainTheorem}}
\begin{proof}   
\label{proofMainTheorem}
Start by choosing an initialization $j \in [n_{init}]$ for which $\inner{\bu_{0, j}, \bu^*} \geq a$ which can be assumed to exist with probability at least $1 - \smallO{1}$.
By \autoref{assGradient} we know that for each step $\inner{\bu_{t+1, j}, \bu^*} \geq (1 + c_0) \inner{\bu_{t, j}, \bu^*}$ if $\inner{\bu_{t, j}, \bu^*} < b$ with proabability at least $1 - \smallO{\frac{1}{s}}$.
By applying the union bound over all steps, we can obtain that
$s = \log_{1 + c_0}\left( \frac{b}{a} \right) = \cO(\log(d))$ steps are sufficient, such that at least one direction $\bu_{i,j}$ is encountered for which $\inner{\bu_{i,j}, \bu^*} \geq b$.
By \autoref{assTestability} we can observe that $\hat{\bu} = \argmax_{\bu \in \{\bu_{i,j} | i \in [s], j \in [n_{init}]\}} \sum_{k = 1}^{n} \frac{\phi(\inner{\bX_k, \bu})}{n}$ has $\inner{\hat{\bu}, \bu^*} \geq b - \delta$ finishing the proof.
\end{proof}
\vspace{-0.1cm} 
\section{Proofs in \autoref{Relu2Section}}
\label{relu2_sc_proof}
In this section use $\phi(x) = \psi(x) = \max\{0, x\}^2$.
For simplicity we also define $\mu_1 = \sqrt{\frac{1-p}{p}}$ and $\mu_2 = - \sqrt{\frac{p}{1-p}}$.
\begin{lemma}
    \label{relu2Init}
    Given $\bu \sim \Data_{\cB(p)}$ for $p \in (1/\sqrt{d}, 1/2)$
    with probability at least $\bigTheta{p}$
    $$
    \frac{\ainner}{\mynorm{\bu}} \geq \bigTheta{\frac{1}{\sqrt{p d}}}
    $$
\end{lemma}
\begin{proof}
    We know that $\ainner = \sqrt{\frac{1-p}{p}} \geq \sqrt{\frac{1}{2 p}}$ with probability $p$.
    Conditioned on $\inner{\bu^*, \bu} = \sqrt{\frac{1-p}{p}}$ we have $\bu \sim \mathcal{N}\left( \sqrt{\frac{1-p}{p}} \bu^*, \bI_n - \bu^*{\bu^*}^\top \right)$.
    By Markov's inequality, we have
    $\mynorm{\correctiontwo \bu} \leq \bigTheta{\sqrt{d}}$ 
    with constant probability.
    The lemma follows from the combination of the previous two results.
\end{proof}
\iffalse
    Thus $
        \bbE \left[ \frac{\ainner}{\mynorm{\bu}} \right]
        \geq 
        \sqrt{\frac{1-p}{p}} \frac{1}{\sqrt{\frac{1-p}{p} + d - 1}}
        \geq
        \sqrt{\frac{1-p}{p}} \frac{1}{2\sqrt{d}}
        \geq
        \frac{1}{4\sqrt{pd}}
    $.
\fi 
\begin{lemma}
    \label{relu2Concentration}
    For any $\delta \in (0, 1)$ and $\bX \sim \cD^n_{\cB(p)}$ have
    \begin{flalign*}
    \mathbb{P}
    \left[
    \myabs{
    \inner{\sumin \frac{g_\bu(\bX_i)}{n}, \bu^*}
    - 
    \inner{\expectation\left[g_\bu(\bX)\right], \bu^*}
    }
    \geq \delta
    \right]
    \leq
    \exp
    \left(
    - \bigTheta{n \delta^2 \min\left\{ 1, \frac{p}{\ainner^2} \right\}}
    \right)
    \end{flalign*}
\end{lemma}
\begin{proof}
For simplicity we will split $\inner{g_\bu(\bx), \bu^*}$ into terms $t_1, t_2$ such that
$$
    \inner{g_\bu(\bx), \bu^*}
    = 
    \underbrace{\max \{0, \inner{\bx, \bu}\} \inner{\bx, \bu^*}}_{= t_1(\bx)}
    - 
    \underbrace{\max \{0, \inner{\bx, \bu}\} \ainner \inner{\bx, \bu}}_{= t_2(\bx)}
$$
Here we use the Sub-Gaussian norm $\| \cdot \|_{\psi_2}$ and Sub-Exponential norm $\| \cdot \|_{\psi_1}$ as defined in \cite{vershynin2018high}.

Next we will choose a fixed sample $\hat{\nu} \sim \cB(p)$ such that $\inner{\bX_i, \bu^*} = \hat{\nu}_i$ and define the \emph{empirical} balance as $\hat{p} = \sumin \frac{\indicator_{\hat{\nu} = \mu_1}}{n}$.
For now we will assume $(1 - \bar{\delta}) p \leq \hat{p} \leq (1 + \bar{\delta}) p$.
\paragraph{Bounding $\sumin t_1(\bX_i)$}
First note that if the random variable $Z$ is Sub-Gaussian, then there exists a constant $C_{centering}$ such that $\twoOrclizNorm{Z - \expectation[Z]} \leq C_{centering} \twoOrclizNorm{Z}$.
Also note that $\|\max\{ c, Z \}\|_{\psi_2} \leq \twoOrclizNorm{Z}$.

Thus 
$
\sum_{i = 1}^n 
\twoOrclizNorm{
    \frac{t_1(\bX_i)}{n}
- 
\expectation
\left[
    \frac{t_1(\bX_i)}{n}
\right]
}^2
\leq 
\hat{p} \frac{\mu_1^2}{n}
+ 
(1 - \hat{p}) \frac{\mu_2^2}{n}
\stackrel{(1)}{\leq}
\bigTheta{\frac{1}{n}}
$.
With (1) following from $\hat{p} \leq (1 + \bar{\delta}) p$.
Thus 
$$
\mathbb{P}\left[ 
\myabs{
    \sum_{i = 1}^n \frac{t_1(\bX_i)}{n}
- 
\expectation
\left[
    \sum_{i = 1}^n \frac{t_1(\bX_i)}{n}
\right]
} \leq \delta \right] 
\geq 1 - \exp\left( - \bigTheta{n \delta^2} \right)
$$
\paragraph{Bounding $\sumin t_2(\bX_i)$}
Note that if the random variable $Z$ is Sub-Exponential then there exists a constant $C_{centering}$ such that $\|Z - \expectation[Z]\|_{\psi_1} \leq C_{centering} \| Z \|_{\psi_1}$.
Additionally note if  $\|Z - \expectation[Z] \|_{\psi_2} = K$ and $\expectation[Z] < 0$ then $\|\max\{0, Z\}\|_{\psi_2} \leq K$.

Thus
$\sum_{i = 1}^n \oneOrclizNorm{
\frac{t_2(\bX_i)}{n}
- 
\expectation
\left[
    \sum_{i = 1}^n \frac{t_2(\bX_i)}{n}
\right]
}^2 
\leq 
\hat{p} \frac{\mu_1^4}{n^3}
+ 
(1 - \hat{p}) \frac{\mu_2^4}{n^3}
\leq
\bigTheta{\frac{1}{n^2}}
$ 
for $n > d$
and $\max_{i \in [n]} \| t_2(\bX_i) \|_{\psi_1} \leq \bigTheta{\frac{1}{n}}$.
Thus, we can show the following bound by applying Bernstein's inequality for sufficiently large $n \geq \frac{1}{\delta}$
\begin{flalign*}  
\mathbb{P}
\left[
\myabs{
    \sum_{i = 1}^n \frac{t_2(\bX_i)}{n}
- 
\expectation
\left[
    \sum_{i = 1}^n \frac{t_2(\bX_i)}{n}
\right]
}
\leq \delta
\right]
& \geq 
1 - \exp\left(
- \bigTheta{
    \min\left\{
    \frac{\delta^2}{\sum_{i = 1}^n \| t_2(\bX_i) \|_{\psi_1}^2},
    \frac{\delta}{\max_{i \in [n]} \| t_2(\bX_i) \|_{\psi_1}}
    \right\} 
}
\right) && \\
& \geq 
1 - \exp\left\{- \bigTheta{n \delta} \right\} &&
\end{flalign*}

Now, what is left to do is to bound $\projectustar{g_\bu(\bx)}$ with a change in $\hat{p}$.
$$
    \myabs{
    \expectation \left[ \sumin \inner{g_\bu(\bx), \bu^*} \right] - 
    \expectation \left[ \sumin \inner{g_\bu(\bx), \bu^*} \bigg| \hat{p} = (1 - \bar{\delta}) p\right]} 
    \leq \bigTheta{\ainner \bar{\delta}}
$$
By applying the Chernoff bound for the Binomial distribution, we obtain
$$\mathbb{P}\big[(1 - \bar{\delta}) p \leq \hat{p} \leq (1 + \bar{\delta}) p\big] \geq 1 - 2 \exp\left( - \frac{\bar{\delta}^2 n p}{3} \right)$$
for all $\bar{\delta} \in (0,1)$.
The lemma follows by choosing $\bar{\delta} = \bigTheta{\frac{\delta}{\ainner}}$ and applying the union bound.
\end{proof} 

\begin{lemma}
    \label{relu2Expectation}
    For arbitrary $\beta > 0$
    there exist constants $t, p_0 > 0$ such that with $\bX \sim \cD_{\cB(p)}$ for $p < p_0$ we have\\
    for $\ainner (\mu_1 - \mu_2) < t$
    $$
        \expectation [\inner{g_{\bu}(\bX), \bu^*}] 
        \geq 
        \bigTheta{\frac{\ainner^2}{\sqrt{p}}}
    $$
    and for $\frac{t}{ (\mu_1 - \mu_2)} \leq \ainner \leq 1 - \beta$ we obtain
    $$
        \expectation [\inner{g_{\bu}(\bX_i), \bu^*}] \geq \bigTheta{\ainner}
    $$
\end{lemma}

\begin{proof}
\bgroup
\renewcommand{\ainner}{a_1}
\renewcommand{\sigmainner}{a_2}
For simplicity abbreviate $\ainner = \inner{\bu, \bu^*}$ and $\sigmainner = \sqrt{1 - \inner{\bu, \bu^*}^2}$.
Define $\be_2 = \frac{\bu - \ainner \bu^*}{\sigmainner}$.
Since $\inner{\bx, \be_2} \sim \mathcal{N}(0, 1)$ we know
$\expectation[\max \{0, \inner{\bx, \be_2}\}^2] = \frac{1}{2}$
and 
$\expectation[\max \{0, \inner{\bx, \be_2}\}] = \sqrt{\frac{1}{2 \pi}}$ by applying the expectation of the half normal distribution.
and 
$f_{\inner{\bx, \be_2}}(y) \geq \underline{p}$ for $y \in [-t_0, t_0]$.

Abbreviate $f_i(x) = f_{\inner{\bx, \be_2}}\left( \frac{x - \mu_i \ainner}{\sigmainner} \right) = f_{\mathcal{N}(0,1)}\left( \frac{x - \mu_i \ainner}{\sigmainner} \right)$.
By applying \autoref{lemProjectedGrad} have

\vspace{-0.1cm}
\begin{align*}
& \expectation_{\bx \sim \Data_p} [\inner{g_\bu(\bx), \bu^*}]
 = 
\expectation_{\bx \sim \Data_p} (\inner{\bx, \bu} \inner{\bx, \bu^*} - \inner{\bx, \bu}^2 \ainner)
\\
& =
\int\limits_0^\infty
p (x \mu_1 - x^2 \ainner) f_1(x) 
+
(1-p) (x \mu_2 - x^2 \ainner) f_2(x)
dx
\\
& =
\int\limits_0^{(\mu_1 - \mu_2) \ainner}
p \mu_1 x f_1(x)
dx
+ 
\sqrt{(1-p) p}
\int\limits_0^\infty
\left( x + (\mu_1 - \mu_2) \ainner - x \right) f_2(x)
dx
 - 
\ainner
\int\limits_0^\infty
x^2
(p f_1(x) + (1-p) f_2(x))
dx
\end{align*}

We continue by bounding each term.

\paragraph{Case 1} If $(\mu_1 - \mu_2) \ainner \leq t_0$.
\begin{align*}
\int\limits_0^{(\mu_1 - \mu_2) \ainner}
p \mu_1 x f_1(x)
dx &
\geq 
\frac{\underline{p} p \mu_1}{2}
\left(
(\mu_1 - \mu_2) \ainner
\right)^2
\geq \frac{\underline{p}\ainner^2}{2}\left(\frac{(1-p)^{1.5}}{\sqrt{p}} + \frac{p^{1.5}}{\sqrt{1-p}} - 2 \sqrt{p (1-p)} \right)
 \geq
\bigTheta{\frac{\ainner^2}{\sqrt{p}}} 
\end{align*}

\paragraph{Case 2} If $(\mu_1 - \mu_2) \ainner > t_0$.
\vspace{-0.2cm}
\begin{align*}
p \mu_1
\int\limits_0^{(\mu_1 - \mu_2) \ainner}
x f_1(x) dx &
\geq 
\sqrt{(1-p) p}
\int\limits_{(\mu_1 - \mu_2) \ainner - t_0}^{(\mu_1 - \mu_2) \ainner} x f_1(x)
dx 
\\ & \geq 
\underline{p}
\sqrt{(1-p) p}
\left(
(\mu_1 - \mu_2) \ainner
\right)^2
- 
\left(
(\mu_1 - \mu_2) \ainner
- t_0
\right)^2
\\ & \geq 
\underline{p}
\sqrt{(1-p) p}
\left(
2
t_0
(\mu_1 - \mu_2) \ainner
- 
t_0^2
\right)
\\ & \geq 
\underline{p}
t_0
\sqrt{(1-p) p}
(\mu_1 - \mu_2) \ainner
\geq 
\underline{p}
t_0
\ainner
\end{align*}

For $p < p_0$, the following terms can be bounded as such
\vspace{-0.1cm}
\begin{align*}
\sqrt{(1-p) p}
\int\limits_0^\infty
\left( (\mu_1 - \mu_2) \ainner \right) f_2(x)
dx 
= 
\ainner
\int\limits_0^\infty f_2(x) dx
\geq
\ainner
\max
\left\{
0,
\frac{1}{2}
- 
\frac{2}{\sqrt{\pi}} \frac{\sqrt{\frac{p}{1-p}}\ainner}{\sigmainner}
\right\}
\end{align*}

and 
\begin{align*}
& \ainner
\int\limits_0^\infty
x^2
(p f_1(x) + (1-p) f_2(x))
dx 
\\
\leq &
\ainner
\biggl(
p
\int\limits_0^{
\mu_1 \ainner
}
x^2
f_1(x)
dx
+
p
\int\limits_0^\infty
\left(
x + \mu_1 \ainner
\right)^2
f_{\inner{\bx, \be_2}}\left(\frac{x}{\sigmainner}\right)
dx
+ 
\frac{(1-p) (1- \ainner^2)}{2}
\biggr)
\\
= &
\ainner
\biggl(
p
\int\limits_0^{
\mu_1 \ainner
}
x^2
f_1(x)
dx
+
p
\int\limits_0^\infty
x^2
f_{\inner{\bx, \be_2}}\left(\frac{x}{\sigmainner}\right)
dx
+
p
\int\limits_0^\infty
2 x \mu_1 \ainner
f_{\inner{\bx, \be_2}}\left(\frac{x}{\sigmainner}\right)
dx
+
p
\left(\mu_1 \ainner\right)^2
\int\limits_0^\infty
f_{\inner{\bx, \be_2}}\left(\frac{x}{\sigmainner}\right)
dx
+ 
\frac{(1-p) (1- \ainner^2)}{2}
\biggr)
\\
\leq &
\ainner
\Biggl[
p\biggl(
\frac{2}{3} \left(\mu_1 \ainner\right)^3
+
\frac{ (1 - \ainner^2)}{2}
+
2 \sqrt{\frac{1}{2 \pi}} \mu_1 \ainner
+
\left(\mu_1 \ainner\right)^2
\frac{1}{2}
\biggr)
+ 
\frac{(1-p) (1- \ainner^2)}{2}
\Biggr]
\leq 
\frac{\ainner \sigmainner^2}{2} + \bigTheta{\sqrt{p}\ainner^2}
\end{align*}


Thus in Case 1 have.
\begin{align*}
\expectation [\inner{g_\bu(\bx), \bu^*}] 
\geq
\bigTheta{\frac{\ainner^2}{\sqrt{p}}}
+ 
\frac{\ainner}{\sigmainner}
\biggl( \frac{1}{2} - 2 \frac{\sqrt{\frac{p}{1-p}}\ainner}{\sigmainner}
\biggr)
- 
\frac{\ainner \sigmainner^2}{2} - \bigTheta{\sqrt{p}\ainner^2}
\geq 
\bigTheta{\frac{\ainner^2}{\sqrt{p}}}
\end{align*}
Analogously in Case 2: 
$\expectation [\inner{g_\bu(\bx), \bu^*}] \geq \bigTheta{\ainner}$
\egroup
\end{proof} 
\begin{lemma}
    \label{relu2ScoreConcentration}
    For $n = \Omega(d)$ have
    $$
    \probability\left[
    \max_{\hat{\bu} \in \bbS_{d-1}} \myabs{\sumin \frac{\psi(\inner{\bX_i, \hat{\bu}})}{n} 
    -
    \expectation[\psi(\inner{\bX, \hat{\bu}})]
    }
    \geq \delta
    \right]
    \leq
    \exp \left( - \bigTheta{\frac{n \delta^2}{d \log(1/\delta)}} \right)
    $$
\end{lemma}

\begin{proof}
First make the distinction into $\bY_j \sim 
\Data_{p,j}^{n_j}$ for $j\in \{1,2\}$, where $\sum_{j\in\{1,2\}} n_j = n$
We will first note that $\bA_j(\bu) = \psi(\bY_j \bu) - \expectation[\psi(\bY_j \bu)]$ is subexponential and thus 
$
    \oneOrclizNorm{
    \sum\limits_{j\in\{1,2\}}
    \sum\limits_{i = 1}^{n_j}
    \frac{\bA_{j,i}(\bu)}{n_j}} \leq \bigTheta{\frac{1}{n}}
$.
Thus, we can obtain
$$
\probability\left[ 
    \myabs{
    \sum\limits_{j\in\{1,2\}}
    \sum\limits^{n_j}_{i = 1} 
    \frac{\bA_{j,i}(\bu)}{n}} 
    \geq \delta_c
\right]
\leq
\exp
\left(
- \bigTheta{
\delta_c^2 n
}
\right)
$$
Let $\epsilonNet$ be the minimum size $\epsilon$-Net of the $d$-dimensional unit sphere.
Thus we know
$\myabs{\epsilonNet} \leq \left(\frac{3}{\epsilon} \right)^d$.
$$
\probability\left[ 
    \max_{\bu \in \epsilonNet}
    \myabs{
    \sum\limits_{j\in\{1,2\}}
    \sum\limits^{n_j}_{i = 1}
    \frac{\bA_{j,i}(\bu)}{n}} 
    \geq \delta_c
\right]
\leq
2 \exp
\left(
d \log\left(\frac{3}{\epsilon}\right)
- 
\bigTheta{\delta_c^2 n}
\right)
$$
Next we bound the maximum deviation for $\bu$ for $\mynorm{\bu' - \bu} < \epsilon$
\begin{flalign*}
    \myabs{
        \sumin \frac{\psi(\inner{\bX_i, \bu})}{n}
        -
        \sumin \frac{\psi(\inner{\bX_i, \bu'})}{n}
    }
    &
    \leq 
    \left(
        \sqrt{\sumin\frac{\psi(\inner{\bX_i, \bu})}{n}}
        +
        \sqrt{\sumin \frac{\psi(\inner{\bX_i, \bu'})}{n}}
    \right)
    \myabs{
        \sqrt{\sumin \frac{\psi(\inner{\bX_i, \bu})}{n}}
        -
        \sqrt{\sumin \frac{\psi(\inner{\bX_i, \bu'})}{n}}
    }
    \\
    &
    \leq 
    \frac{2 \opnorm{\bX} \mynorm{\bX (\bu - \bu')}}{n}
    \leq 
    \frac{2 \epsilon \|\bX\|_{op}^2}{n}
\end{flalign*}
We continue by bounding for all $\bu \in \bbS_{d-1}$ by applying the triangle inequality
\begin{flalign*}
    \myabs{
    \sum\limits_{j\in\{1,2\}}
    \sum\limits^{n_j}_{i = 1}
    \frac{\bA_{j,i}(\bu)}{n}}
    &
    \leq
    \max_{\bu' \in \epsilonNet}
    \myabs{
    \sum\limits_{j\in\{1,2\}}
    \sum\limits^{n_j}_{i = 1}
    \frac{\bA_{j,i}(\bu)}{n}}
    +
    \frac{2 \epsilon \|\bX\|_{op}^2}{n}
\end{flalign*}

Finally we notice that $\bX$ can be decomposed into the union of $\bY_1, \bY_2$ with $n_1 + n_2 = n$, where $n_1 \sim \mathop{\mathrm{Binomial}}(n, p)$.
Thus, by the Chernoff bound for the Binomial distribution, we have
$$
\probability\left[\myabs{
    \expectation[\psi(\inner{\bX, \bu})] - \sum\limits_{j\in\{1,2\}} 
    \frac{n_j}{n} \expectation\left[\psi\left(\inner{\bY_j, \bu}\right)\right]
}
\geq \delta_p
\right]
\leq 2 \exp\left( - \bigTheta{\delta_p^2 n p} \right)
$$
Thus, we can prove the Theorem by choosing $\epsilon = \bigTheta{\delta} $ for $n > d > 1$ and $\delta < 1$.
The second to last step follows bounding $\opnorm{\bX} \leq \opnorm{\bX (\bu^* {\bu^*}^\top)} + \opnorm{\bX (\bI - \bu^* {\bu^*}^\top)} \leq \bigTheta{\sqrt{n} + t}$ with probability $1 - 2 \exp(- t^2)$~\cite{vershynin2018high}.
\begin{align*}
    &\probability\left[
    \max_{\bu \in \bbS_{d-1}} \myabs{\sumin \frac{\psi(\bX_i \bu)}{n} -\expectation[\psi(\bX \bu)]} \geq \delta
    \right]
\\
    \leq & 
    \probability\left[ 
    \max_{\bu \in \epsilonNet}
    \myabs{
    \sum\limits_{j\in\{1,2\}}
    \sum\limits^{n_j}_{i = 1}
    \frac{\bA_{j,i}}{n}} 
    \geq \frac{\delta}{3}
    \right]
    +
    \probability\left[
    \frac{2 \epsilon \opnorm{\bX}}{n}
    \geq \frac{\delta}{3}
    \right]
    +
    \probability\left[\myabs{
    \expectation[\psi(\inner{\bX, \bu})] - \sum\limits_{j\in\{1,2\}} \expectation\left[\psi\left(\inner{\bY_j, \bu}\right)\right]}
    \geq \frac{\delta}{3}
    \right]
\\
    \leq & 
    \exp
    \left(
    d \log\left(\frac{3}{\epsilon}\right) - 
    \bigTheta{\delta^2 n}
    \right)
    + 
    2 \exp(- n)
    +
    \exp\left( - \bigTheta{\delta^2 n p} \right)
\leq
    \exp \left( - \bigTheta{\frac{n \delta^2}{d \log(1/\delta)}}\right)
\end{align*}
\end{proof} 
\begin{lemma}[Testing]
    \label{relu2Testing}
    For any $\ainner > \ainnerdash \geq a_{min} > 0$ there exists $\bar{p} > 0$ such that there exists a threshold $t$ such that with probability at least $1 - \exp \left( - \bigTheta{\frac{n \Delta^2}{d \log(1/\Delta)}} \right)$ where $\Delta = \ainner - \ainnerdash$ we have
    $$
    \sumin \frac{\psi(\inner{\bX_i, \bu})}{n} \geq t
    \,
    \text{and}
    \,
    \sumin \frac{\psi(\inner{\bX_i, \bu'})}{n} \leq t
    $$
\end{lemma}
\begin{proof}
We begin by bounding the expectation of the score function.
\begin{align*}
    \expectation[\phi(\inner{\bX, \bu})]
    =
    & 
    p
    \int_0^\infty
    \left(
    x^2
    f_{\mathcal{N}(0,1)}\left(
        \frac{
            x - \ainner \sqrt{(1-p)/p}
        }{
            \sqrt{1 - \ainner^2}
        }
    \right)
    dx
    \right)
    +
    (1-p)
    \int_0^\infty
    \left(
    x^2
    f_{\mathcal{N}(0,1)}\left(
        \frac{
            x - \ainner \sqrt{p/(1-p)}
        }{
            \sqrt{1 - \ainner^2}
        }
    \right)
    dx
    \right)
    \\
    =
    &
    (1-p)   \ainner^2
    +
    p (1 - \ainner^2)
    - 
    p \int_{-\infty}^0\left(
    x^2
    f_{\mathcal{N}(0,1)}\left(
        \frac{
            x - \ainner \sqrt{(1-p)/p}
        }{
            \sqrt{1 - \ainner^2}
        }
    \right)
    dx
    \right)
    \\
    &
    +
(1-p)
\int_0^{\infty}
    \left(
    x^2
    f_{\mathcal{N}(0,1)}\left(
        \frac{
            \ainner \sqrt{p/(1-p)}
        }{
            \sqrt{1 - \ainner^2}
        }
    \right)
    dx
    \right)
\end{align*}

Thus for each $c_1 > 0$ there exists a $\bar{p} > 0$ such that for all $p \leq \bar{p}$ and $\ainner \geq l$
\begin{align*}
    \ainner^2 (1-p) + 
    (1-\ainner^2)
    \left(
    \frac{1+p}{2}
    - c_{1}
    \right)
    \leq
    \expectation[\phi(\inner{\bX, \bu})]
    \leq
    \ainner^2 (1-p) + (1-\ainner^2) \frac{1+p}{2}
\end{align*}
and for $\ainner \leq 0$ have $\expectation[\phi(\inner{\bX, \hat{\bu}})] \leq \frac{1}{2}$.


Thus, we can bound the difference in expectation for $\bu$ and $\bu'$
\begin{align*}
    \expectation[\phi(\inner{\bX, \bu})] - \expectation[\phi(\inner{\bX, \bu'})]
    \geq
    (\ainner^2 - \ainnerdash^2) \frac{1-3p}{2} - c_1 (1 - \ainner^2)
\end{align*}
Thus $\bu$ and $\bu'$ there exists a $\bar{p} > 0$ such that $\expectation[\phi(\inner{\bX, \bu})] - \expectation[\phi(\inner{\bX, \bu'})] > 0$
By \autoref{relu2ScoreConcentration} we know that 
$\myabs{\sumin \frac{\phi(\inner{\bX_i, \bu})}{n} -\expectation[\phi(\inner{\bX, \bu})]} \leq \delta$ with probability at least $1 - \exp \left( - \bigTheta{\frac{n \delta^2}{d \log(1/\delta)}} \right)$.
The lemma follows by choosing $\delta = \frac{ \myabs{ \expectation[\phi(\inner{\bX, \bu})] - \expectation[\phi(\inner{\bX, \bu'})]}}{2}$ and $t$ accordingly.
\end{proof}
\subsection{Proof of \autoref{relu2_sc}}
As previously discussed, we will split the proof of \autoref{relu2_sc} into parts for each execution of \autoref{the_alg}.
\begin{proof}
Combining \autoref{relu2Expectation} and \autoref{relu2Concentration} and choosing $\delta = \bigTheta{\min \left\{ \frac{\ainner^2}{\sqrt{p}} , \ainner\right\}}$ we obtain 
$$
    \sumin \inner{\frac{g_{\bu}(\bX_i)}{n}, \bu^*}
    \geq 
    \bigTheta{
        \min \left\{ \frac{\ainner^2}{\sqrt{p}} , \ainner\right\}
    }
$$
with probability at least $1 - \bigTheta{\frac{1}{s}}$ if $\ainner \geq a = \bigTheta{\frac{1}{\sqrt{d p}}}$ and $n = \bigOmegas{d^2 p^2}$.
Next, notice that with probability at least $1 - \bigTheta{\frac{1}{s}}$.
$$
    \mynorm{\frac{\phi'(\bX \bu)}{n}} 
    \leq 
    \mynorm{\frac{\bX \bu}{n}} 
    \leq
    \opnorm{\frac{(\bbI_d - \bu^* {\bu^*}^\top) \bX}{n}}
    +
    \mynorm{\bX \bu^*}
    \leq
    \bigTheta{
        \frac{1}{\sqrt{n}}
    }
$$
By utilizing that $(\bbI_d - \bu^* {\bu^*}^\top) \bX$ is a gaussian random matrix and that $\mynorm{\bX \bu^*}^2$ follows a binomial(scaled) distribution.
Thus, we can bound using \autoref{orthogonalGradientNorm}
$$
    \mynorm{\frac{\phi'(\bX \bu)^\top \bX \correction}{n}} 
    \leq
    \bigTheta{
        \sqrt{\frac{d}{n}}
    }
$$
Thus by applying \autoref{gradientNormRewrite} we obtain  probability $1 - \bigTheta{\frac{1}{s}}$ 
\begin{align}
    \mynorm{\frac{\sumin g_\bu(\bX_i)}{n}} 
    &\leq
    \sqrt{
        \inner{ \frac{\sumin g_\bu(\bX_i)}{n}, \bu^* }^2 \left(1 + \frac{\ainner^2}{1 - \ainner^2}\right)
        + \bigTheta{\frac{d}{n}}
    }
    \label{normBound}
\end{align}

\medskip
\textbf{First Execution of \autoref{the_alg}}
\begin{it}
There exist parameters $t_1$ and $\eta_1 = \Omega(\sqrt{d} p)$ such that for $a_1 = \bigTheta{\frac{1}{\sqrt{p d}}}$ there exists a constant $b_1 \in (0, 1)$ such that
the first execution of \autoref{the_alg} fulfills the criteria of \autoref{mainTheorem} for $n = \widetilde{\cO}(d^2 p^2)$.
\end{it}
\medskip

By \autoref{relu2Init} we have $\ainner \geq \bigTheta{\frac{1}{\sqrt{p d}}}$ and thus fulfilling \autoref{assInitialization}.

Using \eqref{normBound} we can bound
\begin{align}
\mynorm{\frac{\sumin g_\bu(\bX_i)}{n}} 
& \leq
\max \left\{
    2
    \sqrt{\inner{ \frac{\sumin g_\bu(\bX_i)}{n}, \bu^* }^2 \left(1 + \frac{\ainner^2}{1 - \ainner^2}\right)}
,
    \bigTheta{\sqrt{\frac{d}{n}}}
\right\}
\label{toVerify}
\end{align}
Thus for all $\ainner \in \left(a_1, b_1\right)$ and $n = \bigOmegas{d^2 p^2}$ we have
\begin{align*}
    \ainner^{-1} \sumin \inner{\frac{g_{\bu}(\bX_i)}{n}, \bu^*}
    \geq 
    \mynorm{\frac{\sumin g_\bu(\bX_i)}{n}} 
\end{align*}
This can be verified by using the bound \eqref{toVerify}.
By applying \autoref{renormalization_helper1}, we can verify that \autoref{assGradient} is fulfilled.
\iffalse
$$
    \ainner^{-1} \sumin \inner{\frac{g_{\bu}(\bX_i)}{n}, \bu^*}
    \geq \left(1 + \frac{\epsilon}{3}\right) \sqrt{\inner{ \frac{\sumin g_\bu(\bX_i)}{n}, \bu^* }^2 \left(1 + \frac{\ainner^2}{1 - \ainner^2}\right)}
$$
$$
    \ainner^{-1} 
    \sumin \inner{\frac{g_{\bu}(\bX_i)}{n}, \bu^*}
    \geq 
    c_2 \min \left\{ \frac{\ainner}{\sqrt{p}} , \frac{1}{\sigmainner} \right\}
    \geq
    \left( 1 - \frac{3}{\epsilon} \right) \sqrt{c_3 c_5 \frac{d}{n}}
$$
\fi
Finally, using \autoref{relu2Testing} shows that \autoref{assTestability} is fulfilled.

\medskip
\textbf{Second Execution of \autoref{the_alg}}
\begin{it}
There exist parameters $t_2$ and $\eta_2$ such that for $a_2 = b_1 - \delta > 0$ and $b_2 = 1 - \beta$ for some $\epsilon > 0$ and $\delta > 0$
the second execution of \autoref{the_alg} fulfills the criteria of \autoref{mainTheorem} for $n = \widetilde{\cO}(d^2 p^2)$.
\end{it}
\medskip

As a result of the first execution of \autoref{the_alg}, there exists a $\bu$ such that $\ainner \geq b_1 - \delta$ and thus \autoref{assInitialization} is fulfilled.
For all $\bu$ for which $\ainner \in (a,b)$ 
there exists a constant choice for $\eta_2 > 0$ such that $\mynorm{\eta_2 \frac{\sumin g_\bu(\bX_i)}{n}}^2 \leq \eta_2 \ainner^{-1} \inner{ \frac{\sumin g_\bu(\bX_i)}{n}, \bu^*}$.
By applying \autoref{renormalization_helper2}, we can show \autoref{assGradient} is fulfilled.
Finally, \autoref{assTestability} can be fulfilled by \autoref{relu2Testing}. 
\end{proof}
\section{Proofs in \autoref{KurtosisSection}}
In this section use $\phi(x) = x^4$ and $\psi(x) = -|x|$.

\label{kurtosisProofs}
\begin{lemma}
\label{kurtosisExpectation}
Have $p < \frac{1}{3}$.
Then
$$
\expectation [\inner{g_{\bu}(\bX), \bu^*}] \geq \bigOmega{\frac{\ainner^3}{p}}
$$
\end{lemma}
\begin{proof}
Define $\mu_0 = 0$ and $\mu_i = \frac{i}{\sqrt{p}}$ and $p_0 = 1-p$ and $p_i = \frac{p}{2}$ for $i \in \{1,-1\}$.
Thus
\begin{align*}
\expectation [\inner{g_{\bu}(\bX), \bu^*}] & =
    4
    \sum_{i \in \{ -1, 0, 1\}}
    p_i
    \left(
    3 \ainner (1 - \ainner^2)^2 (1 - \mu_i^2) 
    + \mu_i^2 \ainner^3 (1 - \ainner^2)(\mu_i^2 - 3)
    \right)
\\
& =
    4
    (1 - \ainner^2) \ainner^3 \left( \frac{1}{p} - 3 \right)
    \geq \bigOmega{\frac{\ainner^3}{p}}
\end{align*}
\end{proof} \begin{lemma}
    \label{kurtosisConcentration}
    For $\delta \in (0, 1)$ have
    \begin{flalign*}
    \mathbb{P}
    \left[
    \myabs{
    \inner{\sumin \frac{g_\bu(\bX_i)}{n}, \bu^*}
    - 
    \inner{\expectation\left[g_\bu(\bX)\right], \bu^*}
    }
    \geq \delta
    \right]
    \leq
    \exp
    \left(
    -
    \bigTheta{
        \frac{
            n \delta^2
        }
        {
            \log(n \log(s))^2
            \left( 1 + \max\left\{\ainner^4 p^{-1}, \ainner^8 p^{-2}\right\} \right)
        }
    }
    \right)
    \end{flalign*}
\end{lemma}
\begin{proof}
Choose an ortho-normal basis $\bE = (\bu^*, \be_2, ... , \be_d) \in \mathbb{R}^{d \times d}$ such that $\bu = \ainner \bu^* + \sqrt{1 - \ainner^2} \be_2$.
Next we will condition on $\hat{\nu}_i = \inner{\bX_i, \bu^*}$ for all $i$ where $\hat{\nu} \in \{-\sqrt{1/p}, 0, \sqrt{1/p}\}$.
For convenience we define $\hat{p} = \sumin \frac{\bbI_{\hat{\nu}_i \neq 0}}{n}$ as well as $n_j = \sumin \bbI_{\hat{\nu}_i = j \sqrt{1/p}}$ with $j \in \{ -1, 0, 1 \}$.
Additionally define $\bY_j = \left( \bX \bigl| \inner{\bX, \bu^*} = j \sqrt{\frac{1}{p}} \right)$.

Next observe that $\max_{i \in [n]}\myabs{\inner{\bX_i, \be_2}} \leq \bigTheta{\log(n \log(s))}$ with probability at least $1 - \bigTheta{\frac{1}{s}}$ by applying the union bound.
Thus for all $j$ we have with probability at least $1 - \bigTheta{\frac{1}{s}}$
$$
    \myabs{\expectation \left[\inner{g_{\bu}(\bY_j), \bu^*}\right] - \inner{g_{\bu}(\bY_j), \bu^*}}
    \leq 
    \bigTheta{\max\left\{ 1, \ainner^2 \left(j \frac{1}{\sqrt{d}}\right)^2, \ainner^4 \left(j \frac{1}{\sqrt{d}}\right)^3 \right\} \log(n \log(s))}
$$
by 
By Hoeffdings inequality we have
\begin{align*}
\bbP 
\left(
    \myabs{
    \sumin \frac{ \inner{g_\bu(\bX_{i}), \bu^*}}{n}
    -
    \sum_{j \in \{-1, 0, 1\}}
    \frac{n_j \bbE\left[ \inner{g_\bu(\bY_j), \bu^*} \right]}{n}
    }
    \geq \delta
\right)
&
\leq
\exp
\left(
-
\bigTheta{
    \frac{\delta^2}
    {
    \log(n \log(s))^2
    \sumin
    \frac{\max \left\{ 1, \ainner^4 \nu_i^4, \ainner^8 \nu_i^6 \right\}}{n^2}
    }
}
\right)
\\
&
\leq
\exp
\left(
-
\bigTheta{
    \frac{n \delta^2}
    {
    \log(n \log(s))^2
    \left( 1 + \hat{p} \max\left\{\ainner^4 p^{-2}, \ainner^8 p^{-3}\right\} \right)
    }
}
\right)
\end{align*}
\iffalse
and if $\sqrt{\frac{1}{p}} \ainner > 1$
$$
\bbP 
\left(
    \myabs{
    \sum_{j \in \{-1, 0, 1\}}
    \sumin \frac{ \inner{g_\bu(\bY_{ji}), \bu^*}}{n}
    -
    \frac{n_j \bbE\left[ \inner{g_\bu(\bY_j), \bu^*} \right]}{n}
    }
    \geq \delta
\right)
\leq
2
\exp
\left(
-
\frac{n \delta^2}
{\left(1 + \frac{\hat{p} }{p^{3}} \right) c_{20}^2 \log(n \log(s))^2}
\right)
$$
\fi

By applying the Chernoff bound for the Binomial distribution, we obtain
$$
\bbP
\left[
\myabs{
    \sum_{j \in \{-1, 0, 1\}}
    \frac{n_j \bbE\left[ \inner{g_\bu(\bY_j), \bu^*} \right]}{n}
    -
    \expectation \left[ \sumin \inner{g_\bu(\bX_i), \bu^*} \right]
}
\geq 
\delta
\right]
\leq 
\exp\left( - \bigTheta{\frac{n p \delta^2}{\max\left\{ \ainner, \frac{\ainner^3}{p} \right\}^2}} \right)
$$

Finally, we can combine all bounds to obtain the full statement.
For any constant $\bar{\delta} \in (0, 1)$ have
\begin{align*}
&\bbP 
\left(
    \myabs{
    \sumin \frac{ \inner{g_\bu(\bX_i), \bu^*}}{n}
    -
    \bbE[\inner{g_\bu(\bX), \bu^*}]
    }
    \geq \delta
\right)\\
&\leq
\bbP 
\left(
    \myabs{
    \sumin \frac{ \inner{g_\bu(\bX_{i}), \bu^*}}{n}
    -
    \sum_{j \in \{-1, 0, 1\}}
    \frac{n_j \bbE\left[ \inner{g_\bu(\bY_j), \bu^*} \right]}{n}
    }
    \geq \frac{\delta}{2}
    \Biggl|
    \myabs{\hat{p} - p} \leq p \bar{\delta}
\right)
+
\bbP 
\left(
    \myabs{\hat{p} - p} \geq p \bar{\delta}
\right)
\\
& 
\hspace{2cm}
+
\bbP
\left[
\myabs{
    \sum_{j \in \{-1, 0, 1\}}
    \frac{n_j \bbE\left[ \inner{g_\bu(\bY_j), \bu^*} \right]}{n}
    -
    \expectation \left[ \inner{g_\bu(\bX), \bu^*} \right]
}
\geq 
\frac{\delta}{2}
\right]
\\ &
\leq
\exp
\left(
-
\bigTheta{
    \frac{
        n \delta^2
    }
    {
        \log(n \log(s))^2
        \left( 1 + p \max\left\{\ainner^4 p^{-2}, \ainner^8 p^{-3}\right\} \right)
    }
}
\right)
\end{align*}
\end{proof} \begin{lemma}
    \label{absScoreConcentration}
    For $n = \Omega(d)$ have
    $$
    \probability\left[
    \max_{\hat{\bu} \in \bbS_{d-1}} \myabs{\sumin \frac{\psi(\inner{\bX_i, \hat{\bu}})}{n} 
    -
    \expectation[\psi(\inner{\bX, \hat{\bu}})]
    }
    \geq \delta
    \right]
    \leq
    \exp \left( - \bigTheta{\frac{n \delta^2}{d}} \right)
    $$
\end{lemma}

\begin{proof}
First, make the distinction into $\bY_j \sim 
\cN\left(\frac{j \bu^*}{\sqrt{p}}, \bI_d - \bu^* {\bu^*}^\top \right)^{n_j}$ for $j \in \{-1, 0, 1\}$, where $\sum_{j\in\{-1, 0, 1\}} n_j = n$.
We will first note that $\bA_j(\bu) = \psi(\bY_j \bu) - \expectation[\psi(\bY_j \bu)]$ is subgaussian and thus
$$
\probability\left[ 
    \myabs{
    \sum\limits_{j\in\{-1, 0, 1\}}
    \sum\limits^{n_j}_{i = 1} 
    \frac{\bA_{j,i}(\bu)}{n}} 
    \geq \delta_c
\right]
\leq
\exp
\left(
- \bigTheta{\delta_c^2 n}
\right)
$$
Let $\epsilonNet$ be the minimum size $\epsilon$-Net of the $d$-dimensional unit sphere.
Thus we know
$\myabs{\epsilonNet} \leq \left(\frac{3}{\epsilon} \right)^d$.
$$
\probability\left[ 
    \max_{\bu \in \epsilonNet}
    \myabs{
    \sum\limits_{j\in\{-1, 0, 1\}}
    \sum\limits^{n_j}_{i = 1}
    \frac{\bA_{j,i}(\bu)}{n}} 
    \geq \delta_c
\right]
\leq
\exp
\left(
d \log\left(\frac{3}{\epsilon}\right)
- 
\bigTheta{\delta_c^2 n}
\right)
$$

First we bound the maximum deviation for $\bu'$ for $\mynorm{\bu' - \bu} < \epsilon$
\begin{flalign*}
    \myabs{
        \sumin \frac{\psi(\inner{\bX_i, \bu})}{n}
        -
        \sumin \frac{\psi(\inner{\bX_i, \bu'})}{n}
    }
    &
    \leq \frac{\myonenorm{\bX (\bu - \bu')}}{n}
    \leq \frac{\mynorm{\bX (\bu - \bu')}}{\sqrt{n}}
    \\ & 
    \leq \frac{\epsilon \opnorm{X}}{\sqrt{n}}
\end{flalign*}
We continue by bounding for all $\bu \in \bbS_{d-1}$ by applying the triangle inequality
\begin{flalign*}
    \myabs{
    \sum\limits_{j\in\{-1, 0, 1\}}
    \sum\limits^{n_j}_{i = 1}
    \frac{\bA_{j,i}(\bu)}{n}}
    &
    \leq
    \max_{\bu' \in \epsilonNet}
    \myabs{
    \sum\limits_{j\in\{-1, 0, 1\}}
    \sum\limits^{n_j}_{i = 1}
    \frac{\bA_{j,i}(\bu')}{n}} 
    +
    \frac{\epsilon \opnorm{X}}{\sqrt{n}}
\end{flalign*}


Finally we notice that $\bX$ can be decomposed into the union of $\bY_{-1}, \bY_0, \bY_1$.
Thus, by applying the Chernoff bound for the Binomial Distribution we obtain
$$
\probability\left[\myabs{
    \expectation[\psi(\inner{\bX, \bu})] - \sum\limits_{j\in\{-1, 0, 1\}} \expectation\left[\psi\left(\inner{\bY_j, \bu}\right)\right]
}
\geq \frac{\delta_p}{\sqrt{p}}
\right]
\leq 2 \exp\left( - \bigTheta{\delta_p^2 n p} \right)
$$
Thus, we can prove the Theorem by choosing $\epsilon = \bigTheta{\delta}$ for $n > d > 1$ and $\delta < 1$.
The second to last step follows bounding $\opnorm{\bX} \leq \opnorm{\bX (\bu^* {\bu^*}^\top)} + \opnorm{\bX (\bI - \bu^* {\bu^*}^\top)} \leq \bigTheta{\sqrt{n} + t}$ with probability $1 - 2 \exp(- t^2)$~\cite{vershynin2018high}.
\begin{align*}
    &\probability\left[
    \max_{\bu \in \bbS_{d-1}} \myabs{\sumin \frac{\psi(\bX_i \bu)}{n} -\expectation[\psi(\bX \bu)]} \geq \delta
    \right]
\\
    \leq & 
    \probability\left[ 
    \max_{\bu \in \epsilonNet}
    \myabs{
    \sum\limits_{j\in\{-1, 0, 1\}}
    \sum\limits^{n_j}_{i = 1}
    \frac{\bA_{j,i}}{n}} 
    \geq \frac{\delta}{3}
    \right]
    +
    \probability\left[
    \frac{\epsilon \opnorm{X}}{\sqrt{n}}
    \geq \frac{\delta}{3}
    \right]
    +
    \probability\left[\myabs{
    \expectation[\psi(\inner{\bX, \bu})] - \sum\limits_{j\in\{-1, 0, 1\}} \expectation\left[\psi\left(\inner{\bY_j, \bu}\right)\right]}
    \geq \frac{\delta}{3}
    \right]
\\
    \leq & 
    \exp
    \left(
    d \log\left(\frac{3}{\epsilon}\right) - 
    \bigTheta{\delta^2 n}
    \right)
    + 
    2 \exp(- n)
    +
    \exp\left( - \bigTheta{\delta^2 n p^2} \right)
\leq 
    2 \exp \left( - \bigTheta{\frac{n \delta^2}{d}} \right)
\end{align*}
\end{proof} \begin{lemma}
\label{absTesting}
For all $\ainner > \inner{\bu', \bu^*}$ and $\delta > 0$.
\iffalse
If
$$
    \sqrt{\frac{2}{\pi}} (1-p)
    \left(
    \sqrt{1-\langle\bu', \bu^*\rangle^2}
    - 
    \sqrt{1-\ainner^2}
    \right)
    +
    (1 - \ainner^2) \sqrt{p}
    >
    2 \delta
$$
then 
\fi
there exists a threshold $t$ such that 
$$
    \sumin \frac{\psi(\inner{\bX_i, \bu})}{n} \geq t
    \,
    \text{and}
    \,
    \sumin \frac{\psi(\inner{\bX_i, \bu'})}{n} \leq t
$$
with a probability of at least
$1 - 2 \exp \left( - \bigTheta{\frac{n \delta^2}{d}} \right)$.
\end{lemma}
\begin{proof}
By applying the expectation of the half-normal distribution, we obtain the following upper and lower bounds.
\begin{align*}
    -
    \sqrt{\frac{2}{\pi}}
    \sqrt{1 - \ainner^2} (1-p) 
    - \sqrt{p} \ainner^2
    \geq
    \expectation[\psi(\inner{\bX, \bu})] 
    \geq 
    -
    \sqrt{\frac{2}{\pi}}
    \sqrt{1 - \ainner^2} (1-p)
    - \sqrt{p}
\end{align*}

Thus, if for all $\bu$
$$
    \myabs{
    \sumin \frac{\psi(\inner{\bX_i, \bu})}{n}
    - 
    \expectation[\psi(\inner{\bX, \bu})] 
    }
    \leq 
    \delta
$$
then 
$$
    -
    \sqrt{\frac{2}{\pi}}
    \sqrt{1 - \ainner^2} (1-p)
    - \sqrt{p}
    - \delta
    >
    -
    \sqrt{\frac{2}{\pi}}
    \sqrt{1 - \inner{\bu', \bu^*}^2} (1-p) 
    - \sqrt{p} \inner{\bu', \bu^*}^2
    + \delta
$$
indicates the existence of the threshold.
The Lemma follows by applying \autoref{absScoreConcentration} and reordering terms.
\end{proof} \begin{lemma}
\label{KurtosisGN}
With probability at least $1 -\bigTheta{\frac{1}{s}} - \exp\left(- \bigTheta{\frac{n p}{\log(n s)}}\right)$
have
$$
    \mynorm{\frac{\phi'(\bX \bu)}{n}} 
    \leq
    \bigTheta{
        \frac{1 + \frac{\ainner^3}{p}}{\sqrt{n}}
    }
$$
\end{lemma}
\begin{proof}
Choose an ortho-normal basis $\bE = (\bu^*, \be_2, ... , \be_d) \in \mathbb{R}^{d \times d}$ such that $\bu = \ainner \bu^* + \sqrt{1 - \ainner^2} \be_2$.
\begin{align*}
    \bbE[\phi'(\inner{\bX, \bu})^2]
    & = \bbE[\inner{\bX, \bu}^6]
    = \bbE\left[ \left(\ainner \inner{\bu^*, \bX} + \sqrt{1 - \ainner^2} \inner{\be_2 , \bX}\right)^6 \right]
    \\
    & \leq \bigTheta{1 + \bbE\left[\ainner \inner{\bu^*, \bX})^6\right]}
\leq \bigTheta{1 + \frac{\ainner^6}{p^2}}
\end{align*}


For convenience we define $\hat{p} = \sumin \frac{\bbI_{\inner{\bX_i, \bu^*} \neq 0}}{n}$ as well as $n_j = \sumin \bbI_{\inner{\bX_i, \bu^*} = j \sqrt{1/p}}$ with $j \in \{ -1, 0, 1 \}$.
Additionally define $\bY_j = \left( \bX \bigl| \inner{\bX, \bu^*} = j \sqrt{\frac{1}{p}} \right)$.
For $\bar{\delta} \in (0,1)$
\begin{align}
& \bbP
\left[
    \myabs{
        \sumin \frac{\phi'(\inner{\bX_i, \bu})^2}{n}
        - 
        \bbE
        \left[
            \phi'(\inner{\bX, \bu})^2
        \right]
    }
    \geq
    \delta
\right] \nonumber
\\
\leq \,
& 
\bbP
\left[
    \myabs{
        \sumin \frac{\phi'(\inner{\bX_i, \bu})^2}{n}
        -
        \sum_{j \in \{-1, 0, 1\}}
        \frac{n_j}{n} \bbE \left[ \phi'\left(\inner{\bY_{j}, \bu}\right)^2 \right]
    }
    \geq
    \frac{\delta}{2}
    \,
    \bigg\vert
    \,
    \myabs{\hat{p} - p} \leq p \bar{\delta}
    \land
    \max_{i \in [n]} \inner{\bX_i, \be_2} \leq \bigTheta{\sqrt{\log(n s)}}
\right]
+
\bbP 
\left[
    \myabs{\hat{p} - p} \geq p \bar{\delta}
\right] \nonumber
\\
& 
\hspace{1cm}
+
\bbP 
\left[
    \max_{i \in [n]} \inner{\bX_i, \be_2} \geq \bigTheta{\sqrt{\log(n s)}}
\right]
+
\bbP
\left[
\myabs{
    \left(
    \sum_{j \in \{-1, 0, 1\}}
    \frac{n_j}{n} 
    \bbE \left[ \phi'\left(\inner{\bY_{j}, \bu}\right)^2 \right]
    \right)
    -
    \bbE \left[ \phi'(\inner{\bX, \bu})^2 \right]
}
\geq 
\frac{\delta}{2}
\right] \nonumber
\\
\leq \, &
\exp\left( - \bigTheta{\frac{n \delta^2}{\log(n s) \left(1 + \frac{\ainner^6}{ p^2}\right)^2}}\right) +
2 \exp\left( - \frac{\bar{\delta}^2 n p}{3} \right)
+
2 \exp\left( \log(n) - \bigTheta{\log(ns)} \right)
+
\exp\left( - \bigTheta{\frac{\delta^2 n p}{\left(
1 + \frac{\ainner^6}{p^2}\right)^2}} \right) \label{concentration_ineqs1}
\\
\leq \, &
\bigTheta{\frac{1}{s}}
+
\exp\left( - \bigTheta{\frac{\delta^2 n p}{\log(n s)\left(1+ \frac{\ainner^6}{p^2}\right)^2}} \right) \nonumber
\end{align}
Here \eqref{concentration_ineqs1} follows by applying Hoeffdings inequality and the Chernoff bound for the Binomial distribution.
The last term in \eqref{concentration_ineqs1} follows by bounding the difference $
    \bbE \left[ \phi'\left(\inner{\bY_{1}, \bu}\right)^2 \right] - \bbE \left[ \phi'\left(\inner{\bY_{0}, \bu}\right)^2 \right]
    =
    \bbE \left[ \phi'\left(\inner{\bY_{-1}, \bu}\right)^2 \right] - \bbE \left[ \phi'\left(\inner{\bY_{0}, \bu}\right)^2 \right] 
    \leq \bigTheta{1 + \frac{\ainner^6}{p^2}}
$ and also applying the Chernoff bound.Choosing $\delta = \bigTheta{1 + \frac{\ainner^6}{p^2}}$ yields 
$$
    \bbE[\phi'(\inner{\bX, \bu})^2]
    \leq \bigTheta{1 + \frac{\ainner^6}{p^2}}
$$
With probability at least $1 - \bigTheta{\frac{1}{s}} - \exp\left(- \bigTheta{\frac{n p}{\log(n s)}}\right)$.
The lemma follows by computing the norm.
\end{proof} 
\subsection{Proof of \autoref{kurtosis_sc}}
\begin{proof}
As discussed, we will split the proof of \autoref{kurtosis_sc} into parts for each execution of \autoref{the_alg}.

\medskip
\textbf{First Execution of \autoref{the_alg}}
\begin{it}
    There exist parameters $t_1$ and $\eta_1 = \Omega(d p^2)$ such that for $a_1 = \bigTheta{\frac{1}{\sqrt{p d}}}$ and some constant $b_1 \in (0, 1)$
    the first execution of \autoref{the_alg} fulfills the criteria of \autoref{mainTheorem} for $n = \widetilde{\cO}(d^3 p^4)$.
\end{it}
\medskip

By the same argument as in \autoref{relu2Init}, we have $\ainner \geq \bigTheta{\frac{1}{\sqrt{p d}}}$ with sufficiently large probability, thus fulfilling \autoref{assInitialization}.

Using \autoref{kurtosisExpectation} and applying the concentration result of \autoref{kurtosisConcentration}, we obtain that
\begin{align}
    \sumin \inner{\frac{g_{\bu}(\bX_i)}{n}, \bu^*}
    =
    \bigOmega{
        \frac{\ainner^3}{p}
    }
    \label{kurtosisGradientFormula}
\end{align}
with probability of at least
$ 1 - \exp
\left(
-
\bigTheta{
    \frac{
        n (\ainner^3 p^{-1})^2
    }
    {
        \log(n \log(s))^2
        \left( 1 + \max\left\{\ainner^4 p^{-1}, \ainner^8 p^{-2}\right\} \right)
    }
}
\right)
$.

Next bound with probability at least 
$1 -\bigTheta{\frac{1}{s}} - \exp\left(- \bigTheta{\frac{n p}{\log(n s)}}\right)$
\begin{align}
\mynorm{\frac{\sumin g_\bu(\bX_i)}{n}} 
&\leq
\sqrt{
    \inner{ \frac{\sumin g_\bu(\bX_i)}{n}, \bu^* }^2 \left(1 + \frac{\ainner^2}{1 - \ainner^2}\right)
    + \mynorm{\frac{\phi'(\bX \bu)^\top \bX \correction}{n}}^2 
}
\label{grad_decomp}
\\
&\leq
\sqrt{
    \inner{ \frac{\sumin g_\bu(\bX_i)}{n}, \bu^* }^2 \left(1 + \frac{\ainner^2}{1 - \ainner^2}\right)
    + 
    \bigTheta{
        \left(1 + \frac{\ainner^6}{p^2}\right)\frac{d}{n}
    }
}
\label{ApplylemGradientBound}
\\
& \leq
\max \left\{
    2
    \sqrt{\inner{ \frac{\sumin g_\bu(\bX_i)}{n}, \bu^* }^2 \left(1 + \frac{\ainner^2}{1 - \ainner^2}\right)}
,
    2
    \sqrt{
        \bigTheta{\left(1 + \frac{\ainner^6}{p^2}\right)\frac{d}{n}}
    }
\right\}
\label{c5}
\end{align}
Here, \eqref{grad_decomp} follows by by decomposing the norm.
\eqref{ApplylemGradientBound} follows by applying \autoref{KurtosisGN} and  \autoref{orthogonalGradientNorm}.
Finally using $\eta_1 = \bigOmega{p^2 d}$ we apply \autoref{renormalization_helper1} to demonstrate that \autoref{assGradient} is fulfilled if $n = \bigOmegas{d^3 p^4}$.

Finally, \autoref{absTesting} shows that \autoref{assTestability} is fulfilled.

\medskip
\textbf{Second Execution of \autoref{the_alg}}
\begin{it}
    There exist parameters $t_2$ and $\eta_2 = \Omega(d p^2)$ such that for $a_2 = b_1 - \delta$ and $b_2 = 1-\beta$ for some $3 > \epsilon > 0$
    the first execution of \autoref{the_alg} fulfills the criteria of \autoref{mainTheorem} for $n = \bigOmegas{d^3 p^4}$.
\end{it}
\medskip

\autoref{assInitialization} is fulfilled by the first execution of \autoref{the_alg}.

Using \eqref{c5} have $\mynorm{\frac{\sumin g_\bu(\bX_i)}{n}} = \bigO{\frac{1}{p}}$ if $\myabs{\ainner} \leq 1-\beta$.
Choosing $\eta_2 = \bigTheta{1} > 0$ yields that
$\eta_2 \ainner^{-1}\inner{\sumin \frac{g_\bu(\bX_i)}{n}, \bu^*} \geq \mynorm{\eta_2 \frac{\sumin g_\bu(\bX_i)}{n}}^2$.
Thus applying \autoref{renormalization_helper2} we obtain that \autoref{assGradient} is satisfied if  $n = \bigOmegas{d^3 p^4}$.

Finally, \autoref{absTesting} shows that \autoref{assTestability} is fulfilled.

\end{proof}   
\section{Proofs in \autoref{ldplb}}
\subsection{Proof of \autoref{reduction}}
\begin{proof} 
First note we have $\bbE_{\cQ}[\relutwo{\bX}] = \bbE_{\cP}[\relutwo{\bX}$ if $\inner{\hat{\bu}, \bu^*} = 0$.
Thus, we can apply \autoref{relu2Testing} to obtain the result.
\end{proof} \autoref{akm} is nearly equlvalent to Lemma 6.7 in \citet{mao2022optimal}.
\begin{lemma}
\label{akm}
For $\alpha \in \mathbb{N}^n$, let $|\alpha| = \sum_{i=1}^n \alpha_i = \|\alpha\|_1$, and let $\|\alpha\|_0$ be the size of the support of $\alpha$. 
For $m \in [d]$, define a set 
\begin{equation}
\mathcal{A}(k, m) : = \left\{ \alpha \in \mathbb{N}^n : |\alpha| = k , \, \|\alpha\|_0 = m, \, \alpha_i \in \{0\} \cup \{ 3, 4, \dots \} \text{ for all } i \in [n] \right\} .
\end{equation}
Then we have 
$\left| \mathcal{A}(k, m) \right| \leq n^m k^k$.
\end{lemma}
\begin{proof}
$\left| \mathcal{A}(k, m) \right| \leq \binom{n}{k} k^k$.
\end{proof} \begin{lemma}
\label{herm_ic}
Have the Imbalanced Clusters RV $\bX \sim \Data_\nu(p)$
\begin{center}
    $\expectation[\hat{h}_0(\bX)] = 1$,
    $\expectation[\hat{h}_1(\bX)] = 0$,
    $\expectation[\hat{h}_2(\bX)] = 0$
\end{center}
And given $p < 0.5$ have for $k \geq 3$ 
$$|\expectation[\hat{h}_k(\bX)]| \leq k^{k/2} p^{1 - k/2}$$
\end{lemma}
\begin{proof}
    For $p < 0.5$ have $\expectation[\bX^k] \leq 2 p^{1 - k/2}$.
    Thus, we can bound 
    $$|\expectation[h_k(\bX)]| = \frac{1}{\sqrt{k!}} \left( \sum\limits_{i=0}^k c_i \expectation[\bX^i]\right) \leq |\expectation[\bX^i]| \sqrt{k!}$$. 
    With the last inequality following from $ \sum\limits_{i=0}^k |c_i| \leq k!$.
\end{proof} \begin{lemma}[\citet{mao2022optimal}]
\label{formula}
Consider the distribution $\cP$ in \autoref{testing} and suppose the first $D$ moments of $\nu$ are finite. For $\alpha \in \cN^N$, let $|\alpha| := \sum_{i=1}^N \alpha_i$. Then
\begin{equation}\label{eq:ld-formula}
\ldlr^2 = \sum_{d=0}^D \expectation[\langle \bu, \bu' \rangle^d] \sum_{\substack{\alpha \in \cN^N \\ |\alpha| = d}} \prod_{i=1}^N \left(\expectation_{x \sim \nu}[h_{\alpha_i}(x)]\right)^2
\end{equation}
where $\bu$ and $\bu'$ are drawn independently from $\mathcal{U}$.
\end{lemma} \begin{lemma}[\citet{mao2022optimal}]
\label{inner-product}
Let $\bu$ and $\bu'$ be independent uniform random vectors on the unit sphere in $\mathbb{R}^n$. For $k \in \mathbb{N}$, if $k$ is odd, then $\expectation[\langle \bu, \bu' \rangle^d] = 0$, and if $k$ is even, then 
$$ 
\expectation[\langle \bu, \bu' \rangle^k] \le (k/n)^{k/2}
$$
\end{lemma} \subsection{Proof of \autoref{thmLDPLB}}
The proof of \autoref{thmLDPLB} closely follows the proof of Theorem 4.5 in \citet{mao2022optimal}.
\begin{proof}
    We will use $\mathcal{A}(k, m)$ as defined in \autoref{akm} and note that 
    for $\alpha \in \mathcal{A}(k, m)$, we obtain that $\alpha \geq 3$ and thus $m \leq \lfloor k/3 \rfloor$ using \autoref{herm_ic}
    \begin{align*}
    \sum_{\substack{\alpha \in \Natural^n \\ |\alpha| = k}} \prod_{i=1}^n \left(\expectation[h_{\alpha_i}(x)]\right)^2 
    = \sum_{m=1}^{\lfloor k/3 \rfloor} \sum_{\substack{\alpha \in \mathcal{A}(k, m)}} \prod_{i=1}^n \left(\expectation[h_{\alpha_i}(x)]\right)^2
    \le \sum_{m=1}^{\lfloor k/3 \rfloor} \left| \mathcal{A}(k, m) \right| \prod_{i \in [n], \, \alpha_i \ne 0} \alpha_i^{2 \alpha_i} p^{2-\alpha_i}
    \le \sum_{m=1}^{\lfloor k/3 \rfloor} n^m k^{3k} p^{2 m - k} 
    \end{align*}
    By applying the closed form of the geometric series, we obtain
    \begin{align*}
    \sum_{\substack{\alpha \in \Natural^n \\ |\alpha| = k}} \prod_{i=1}^n \left(\expectation[h_{\alpha_i}(x)]\right)^2 
    \le k^{3k} n p^{2-k} \frac{(n p^2)^{\lfloor k/3 \rfloor} - 1}{n p^2 - 1}
    \le k^{3k} n p^{2-k} \frac{(n p^2)^{k/3}}{\frac{1}{2} n p^2}
    = 2 k^{3k} n^{k/3} p^{-k/3} .
    \end{align*}
    This combined with \autoref{inner-product} gives 
    \begin{equation*} 
    \expectation[\langle u,u' \rangle^k] \sum_{\substack{\alpha \in \Natural^n \\ |\alpha| = k}} \prod_{i=1}^n \left(\expectation[h_{\alpha_i}(x)]\right)^2 
    \le (k/d)^{k/2} \cdot 2 k^{3k} n^{k/3} p^{-k/3}
    = 2 \left( \frac{ k^{10.5} n }{ d^{3/2} p } \right)^{k/3} . 
    \end{equation*} 
    Finally, combining this with \autoref{formula}, we obtain  
    \begin{equation*} 
    \ldlr^2 = \sum_{k=0}^D \expectation[\langle u,u' \rangle^k] \sum_{\substack{\alpha \in \Natural^n \\ |\alpha| = k}} \prod_{i=1}^n \left(\expectation[h_{\alpha_i}(x)]\right)^2 
    \le 1 + 2 \sum_{k=3}^D \left( \frac{ k^{10.5} n }{ d^{3/2} p } \right)^{k/3} . 
    \end{equation*}
    This is true if $d^{1.5} p > n D^{c_{2}}$ for a sufficiently large constant $c_{2} > 0$ such that $\frac{k^{10.5} n }{ d^{3/2} p } < 1/4$. 
\end{proof}
\section{Additional Proofs}
\begin{lemma}
    \label{gradientNormRewrite}
    Choosing the ortho-normal basis $\bE = (\bu^*, \be_2, ... , \be_d) \in \mathbb{R}^{d \times d}$ such that $\bu = \ainner \bu^* + \sqrt{1 - \ainner^2} \be_2$.
    \begin{flalign*}
        \mynorm{\frac{\sumin g_\bu(\bX_i)}{n}} =
        \sqrt{
            \inner{ \frac{\sumin g_\bu(\bX_i)}{n}, \bu^* }^2
            \left(1 + \frac{\ainner^2}{1 - \ainner^2}\right)
            + \mynorm{\frac{\phi'(\bX \bu)^\top \bX \correction}{n}}^2
       } 
    \end{flalign*}
\end{lemma}
\begin{proof}
    $$
        \mynorm{\frac{\sumin g_\bu(\bX_i)}{n}} = 
        \sqrt{
            \inner{ \frac{\sumin g_\bu(\bX_i)}{n}, \bu^* }^2 + \inner{ \frac{\sumin g_\bu(\bX_i)}{n}, \be_2 }^2 + \sum_{i=3}^d \inner{ \frac{\sumin g_\bu(\bX_i)}{n}, \be_i }^2
        }
    $$
    By $\ainner \inner{ \frac{\sumin g_\bu(\bX_i)}{n}, \bu^* } + \sqrt{1-\ainner^2} \inner{ \frac{\sumin g_\bu(\bX_i)}{n}, \be_2 } = 0$ we can expand to
    \begin{flalign*}
        \mynorm{\frac{\sumin g_\bu(\bX_i)}{n} } & =
        \sqrt{
            \inner{ \frac{\sumin g_\bu(\bX_i)}{n}, \bu^* }^2
            \left(1 + \frac{\ainner^2}{1 - \ainner^2}\right)
            + \sum_{i=3}^d \inner{ \frac{\sumin g_\bu(\bX_i)}{n}, e_i }^2
       } 
    \end{flalign*}
\end{proof} \begin{lemma}
    \label{orthogonalGradientNorm}
    Choose a ortho-normal basis $\bE = (\bu^*, \be_2, ... , \be_d) \in \mathbb{R}^{d \times d}$ such that $\bu = \ainner \bu^* + \sqrt{1 - \ainner^2} \be_2$.
    Thus
    $$
        \mynorm{\frac{\phi'(\bX \bu)^\top \bX \correction}{n}} 
        =
        \bigO{
            \sqrt{d}
            \mynorm{\frac{\phi'(\bX \bu)}{n}}
        }
    $$
    with a probability of at least 
    $1 - \bigO{\frac{1}{s}}$.
\end{lemma}

\begin{proof}
First notice that $\bX \correction \sim \cN(0,\correction)^n$ and thus that for some vector $\bv$
\begin{align*}
\bbP
\left[
    \mynorm{\sumin \frac{\bv_i}{\mynorm{\bv}} \X_i \correction}
    \geq 
    \sqrt{d} + t
\right]
&
\leq
2 \exp\left( - \bigTheta{t^2} \right)
\label{bound_norm2}
\end{align*}
The lemma follows by using $t = \sqrt{\log(s)}$ and choosing $\bv = \frac{\phi'(\bX \bu)}{n}$
\end{proof} \begin{lemma}
    \label{renormalization_helper1}
    For any $c_3 \in (0,1)$ such that 
    if 
        $
        \mynorm{\frac{\sumin g_\bu(\bX_i)}{n}} \leq 
        (1 - c_3)
        \frac{\inner{ \frac{\sumin g_\bu(\bX_i)}{n}, \bu^* }}{\ainner}$ 
    and 
        $\frac{\eta \inner{ \frac{\sumin g_\bu(\bX_i)}{n}, \bu^* }}{\ainner} \geq 1$
    we have
    \begin{align*}
        \frac{\langle \bu^*, \bu + \eta \inner{\sumin \frac{g_\bu(\bX_i)}{n}, \bu^*}}{\mynorm{\bu + \inner{\sumin \frac{g_\bu(\bX_i)}{n}, \bu^*}}}
        \geq
        \ainner \left(1 + \frac{c_6}{2}\right)
    \end{align*}
\end{lemma}

\begin{proof}
\begin{flalign*}  
\inner{ 
\bu^*, 
\frac{
    \bu + \eta \sumin \frac{g_\bu(\bX_i)}{n} 
}{
    \mynorm{\bu + \eta \frac{\sumin g_\bu(\bX_i)}{n}}
}
}
& \geq
\ainner
\frac{
    1 + \eta \ainner^{-1}\inner{\sumin \frac{g_\bu(\bX_i)}{n}, \bu^*}
}{
    \sqrt{1 + \eta^2 \mynorm{\frac{\sumin g_\bu(\bX_i)}{n}}^2}
}
\\ &
\geq
\ainner
\frac{
1 + \eta \ainner^{-1} \inner{\sumin \frac{ g_\bu(\bX_i)}{n}, \bu^*}
}{
1 + (1 - c_3) \eta \ainner^{-1} \inner{\sumin \frac{ g_\bu(\bX_i)}{n}, \bu^*}
}
\geq
\ainner \left(1 + \frac{c_3}{2}\right)
\end{flalign*}
\end{proof}
%
 \begin{lemma}
    \label{renormalization_helper2}
    If $\eta \ainner^{-1}\inner{\sumin \frac{g_\bu(\bX_i)}{n}, \bu^*} \geq \mynorm{\eta \frac{\sumin g_\bu(\bX_i)}{n}}^2$
    \begin{align*}
        \frac{
            \langle \bu^*, \bu + \eta \inner{\sumin \frac{g_\bu(\bX_i)}{n}, \bu^*}
        }{
            \mynorm{\bu + \eta \sumin \frac{g_\bu(\bX_i)}{n}}
        }
        \geq
        \ainner
        \min \left\{
            1 + \eta \inner{\sumin \frac{g_\bu(\bX_i)}{n}, \bu^*}
            , 2
        \right\}
    \end{align*}
\end{lemma}

\begin{proof}
\begin{flalign*}  
\inner{ 
\bu^*, 
\frac{
    \bu + \eta \sumin \frac{g_\bu(\bX_i)}{n} 
}{
    \mynorm{\bu + \eta \frac{\sumin g_\bu(\bX_i)}{n}}
}
}
& \geq
\ainner
\frac{
    1 + \eta \ainner^{-1}\inner{\sumin \frac{g_\bu(\bX_i)}{n}, \bu^*}
}{
    \sqrt{
        1 + \mynorm{\eta \frac{\sumin g_\bu(\bX_i)}{n}}^2
    }
}
\\ &
\geq
\ainner
\sqrt{
    1 + \eta \ainner^{-1}\inner{\sumin \frac{g_\bu(\bX_i)}{n}, \bu^*}
}
\geq
\ainner 
\min
\left\{
    1 + \eta \inner{\sumin \frac{g_\bu(\bX_i)}{n}, \bu^*}
    , 2
\right\}
\end{flalign*}
\end{proof} \begin{lemma}
    \label{lemProjectedGrad}
    For any $\phi(\cdot)$ and $\bx, \bu, \bu^* \in \mathbb{R}^d$ have
    $$
        \inner{ g_\bu(\bx), \bu^* }
        =
        \frac{\partial \phi(\inner{ \bx, \bu })}{\partial \inner{ \bx, \bu }} 
        (\inner{ \bx, \bu^* } - \inner{ \bx, \bu } \inner{ \bu, \bu^* })
    $$
\end{lemma}

\begin{proof}
First recall the definition $g_\bu(\bx) = (\bI_d - \bu \bu^T) \frac{\partial \phi(\inner{\bu, \bx}}{\partial u}$.
Let us choose a new ortho-normal basis $\bE = (\bu^*, \be_2, ... , \be_d) \in \mathbb{R}^{d \times d}$.
By $\inner{ \bx, \bu } = \inner{ \bx, \bu^* } \inner{ \bu, \bu^* } + \left( \sum_{i = 2}^d \inner{ \bx, e_i } \inner{ \bu, e_i } \right)$

\begin{align*}
\inner{ g_\bu(\bx), \bu^* }
& = \inner{ (I - \bu\bu^T) \frac{\partial \phi(\inner{ \bx, \bu })}{\partial \bu}, \bu^* }
= \frac{\partial \phi(\inner{ \bx, \bu })}{\partial \inner{ \bx, \bu }} \left(
\left(1 - \inner{ \bu, \bu^* }^2\right) \inner{ \bx, \bu^* }
- \inner{ \bu, \bu^* } \left( \sum_{i = 2}^d \inner{ \bu, \be_i } \inner{ \bx, \be_i } \right)
\right)
\\ & =
\frac{\partial \phi(\inner{ \bx, \bu })}{\partial \inner{ \bx, \bu }} 
\left(
(1 - \inner{ \bu, \bu^* }^2) \inner{ \bx, \bu^* }
- \inner{ \bu, \bu^* }
(\inner{ \bx, \bu } - \inner{ \bu, \bu^* } \inner{ \bx, \bu^* })
\right)
=
\frac{\partial \phi(\inner{ \bx, \bu })}{\partial \inner{ \bx, \bu }} 
(\inner{ \bx, \bu^* } - \inner{ \bx, \bu } \inner{ \bu, \bu^* })
\end{align*}

\end{proof}

% \documentclass[twoside]{article}

% \usepackage{aistats2025}
% If your paper is accepted, change the options for the package
% aistats2025 as follows:
%
%\usepackage[accepted]{aistats2025}
%
% This option will print headings for the title of your paper and
% headings for the authors names, plus a copyright note at the end of
% the first column of the first page.

% If you set papersize explicitly, activate the following three lines:
%\special{papersize = 8.5in, 11in}
%\setlength{\pdfpageheight}{11in}
%\setlength{\pdfpagewidth}{8.5in}

% If you use natbib package, activate the following three lines:
%\usepackage[round]{natbib}
%\renewcommand{\bibname}{References}
%\renewcommand{\bibsection}{\subsubsection*{\bibname}}

% If you use BibTeX in apalike style, activate the following line:
%\bibliographystyle{apalike}

% \begin{document}

% If your paper is accepted and the title of your paper is very long,
% the style will print as headings an error message. Use the following
% command to supply a shorter title of your paper so that it can be
% used as headings.
%
%\runningtitle{I use this title instead because the last one was very long}

% If your paper is accepted and the number of authors is large, the
% style will print as headings an error message. Use the following
% command to supply a shorter version of the authors names so that
% they can be used as headings (for example, use only the surnames)
%
%\runningauthor{Surname 1, Surname 2, Surname 3, ...., Surname n}

% Supplementary material: To improve readability, you must use a single-column format for the supplementary material.
\onecolumn
\appendix
\aistatstitle{From Deep Additive Kernel Learning to Last-Layer \\ Bayesian Neural Networks via Induced Prior Approximation: \\
Supplementary Materials}

\section{SPARSE CHOLESKY DECOMPOSITION}
\label{sec:sparse chol decompose}
In this section, we present the algorithm for constructing the induced grids $\mathbf{U}$ as defined in \cref{eq:GPlayer} by using sorted dyadic points, and obtaining the sparse Choleksy decomposition of the Laplace kernel in one dimension, as proposed in \citep{ding2024sparse}.

A set of one-dimensional level-$L$ dyadic points $\Xv_L$ in increasing order over the interval $[0,1]$ is defined as:
\begin{align}
    \Xv_{L}:= \left\{ \frac{1}{2^{L}}, \frac{2}{2^{L}}, \frac{3}{2^{L}}, \ldots, \frac{2^{L}-1}{2^{L}} \right\}.
\end{align}
However, this increasing order does not yield a sparse representation of the Markov kernel $k(\cdot,\cdot)$ on the points $\Xv_L$, i.e., Cholesky decomposition of the covariance matrix $k(\Xv_L, \Xv_L)$ is not sparse. To achieve a sparse hierarchical expansion, we first sort the dyadic points $\Xv_L$ according to their levels.

\paragraph{Sorted Dyadic Points}
For level-$\ell$ dyadic points $\Xv_{\ell}$ where $ \ell=1,\ldots,L$, we first define the set $\rho(\ell)$ consisting of odd numbers as follows:
\begin{align}
    \rho(\ell) = \left\{ 1,3,5,\ldots,2^{\ell}-1 \right\}.
\end{align}
Next, we define the sorted incremental set $\Dv_{\ell}$ (with $\Xv_{0}:= \varnothing$) as:
\begin{align}
    \Dv_{\ell} = 
    \left\{ \frac{i}{2^{\ell}}: i\in \rho(\ell) \right\} = \Xv_{\ell} - \Xv_{\ell-1}, \quad  \ell=1,\ldots L.
\end{align}
Thus, the level-$L$ dyadic points $\Xv_L$ can be decomposed into disjoint incremental sets $\{ \Dv_{\ell} \}_{\ell=1}^{L}$:
\begin{align}
    \Xv_{L} = \cup_{\ell=1}^{L} \Dv_{\ell}, \quad \Dv_{i} \cap \Dv_{j} = \varnothing \text{ for $i\neq j$}.
\end{align}
Therefore, we can define the sorted level-$L$ dyadic points using these incremental sets as:
\begin{align}\label{eq:sorted dyadic}
    \Xv_{L}^{\text{sort}}:= \left\{ \Dv_1,\Dv_2, \ldots, \Dv_{L} \right\} 
    = \left\{ \frac{i \in \rho(\ell) }{2^{\ell}}, \ell=1,\ldots,L \right\}.
\end{align}
For example, the sorted level-3 dyadic points are given by:
\begin{align}
    \Xv_{3}^{\text{sort}} 
    = \bigg\{ 
    \begingroup
        \color{blue}
        \underbracket{
            \color{black}
            \frac{1}{2^1}
        }_{\color{blue}
            \Dv_1
        }
    \endgroup
    , 
    \begingroup
        \color{blue}
        \underbracket{
            \color{black}
            \frac{1}{2^2}, \frac{3}{2^2}
        }_{\color{blue}
            \Dv_2
        }
    \endgroup
    ,
    \begingroup
        \color{blue}
        \underbracket{
            \color{black}
            \frac{1}{2^3}, \frac{3}{2^3}, \frac{5}{2^3}, \frac{7}{2^3}
        }_{\color{blue}
            \Dv_3
        }
    \endgroup
     \bigg\}.
\end{align}

\paragraph{Algorithm}
We now present the algorithm for computing the inverse of the upper triangular Cholesky factor $[ \Lv_{\Xv_{L}^{\text{sort}}}^{\top} ]^{-1}$ of the covariance matrix $k(\Xv_{L}^{\text{sort}}, \Xv_{L}^{\text{sort}})$ in \Cref{alg:cholesky}, where $\Lv_{\Xv_{L}^{\text{sort}}} \Lv_{\Xv_{L}^{\text{sort}}}^{\top} = k(\Xv_{L}^{\text{sort}}, \Xv_{L}^{\text{sort}})$.. The corresponding proof can be found in \citep{ding2024sparse}. The output of \Cref{alg:cholesky} is a sparse matrix with $\Oc(3 \cdot (2^{L}-1))$ nonzero entries. Since each iteration of the for-loop only requires solving a $3 \times 3$ linear system, which costs $\Oc(3^3)$ time, the total computational complexity of \Cref{alg:cholesky} is $\Oc(2^L-1)$. This implies that the complexity of computing $\left[ \Lv_{\Uv}^{\top} \right]^{-1}$ in \cref{eq:GPlayer} is $\Oc(M)$ when $\Uv$, the induced grid of size $M$, consists of sorted dyadic points as defined in \cref{eq:sorted dyadic}.

\begin{algorithm}[hbt!]
\caption{Computation of the inverse Cholesky factor for the Markov kernel $k(\cdot, \cdot)$ on sorted one-dimensional level-$L$ dyadic points $\Xv_L^{\text{sort}}$.}
\label{alg:cholesky}
\setstretch{0.99} % set the line spacing to 0.99
\begin{algorithmic}[1]
    \STATE {\bfseries Input:} Markov kernel $k(\cdot,\cdot)$, sorted level-$L$ dyadic points $\Xv_{L}^{\text{sort}}$
    \STATE {\bfseries Output:} inverse of the upper triangular Cholesky factor $\Rv:= [ \Lv_{\Xv_{L}^{\text{sort}}}^{\top} ]^{-1}$, s.t. $\Lv_{\Xv_{L}^{\text{sort}}} \Lv_{\Xv_{L}^{\text{sort}}}^{\top} = k(\Xv_{L}^{\text{sort}}, \Xv_{L}^{\text{sort}})$
    \STATE Initialize $\Rv \leftarrow \text{zeros($2^L-1$,$2^L-1$)}$;
    \STATE Define $k(\pm \infty, \cdot) = k(\cdot, \pm \infty) = 0$;
    \FOR{$\ell=1$ {\bfseries to} $L$}
        \FOR{$i \in \rho(\ell)=\{1,3,\ldots,2^{\ell}-1\}$}
            \STATE $x_{\text{mid}} := \frac{i}{2^{\ell}}$;\quad
            $x_{\text{left}}:=\frac{i-1}{2^{\ell}}$ {\bfseries if} $i>1$ {\bfseries else} $-\infty$;\quad
            $x_{\text{right}}:=\frac{i+1}{2^{\ell}}$ {\bfseries if} $i<2^{\ell}-1$ {\bfseries else} $+\infty$;
            \STATE Get $i_{\text{mid}}$, $i_{\text{left}}$, $i_{\text{right}}$, the indices of the points $x_{\text{mid}}$, $x_{\text{left}}$, $x_{\text{right}}$ in the sorted set $\Xv_{L}^{\text{sort}}$ respectively;
            \STATE Get the coefficients $c_1$, $c_2$, $c_3$ by solving the following linear system:
            \begin{align}
                \begin{bmatrix}
                     & k(x_{\text{left}}, x_{\text{left}})
                     & k(x_{\text{left}}, x_{\text{mid}})
                     & k(x_{\text{left}}, x_{\text{right}}) \\
                     & k(x_{\text{mid}}, x_{\text{left}})
                     & k(x_{\text{mid}}, x_{\text{mid}})
                     & k(x_{\text{mid}}, x_{\text{right}}) \\
                     & k(x_{\text{right}}, x_{\text{left}})
                     & k(x_{\text{right}}, x_{\text{mid}})
                    &k(x_{\text{right}}, x_{\text{right}})
                \end{bmatrix}
                \begin{bmatrix}
                    c1\\
                    c2\\
                    c3
                \end{bmatrix}=
                \begin{bmatrix}
                    0\\
                    1\\
                    0
                \end{bmatrix}.
            \end{align}
            \STATE $[c_1,c_2,c_3] := [c_1,c_2,c_3] / \sqrt{c_2}$;
            \STATE {\bfseries if} $x_{\text{left}} \neq - \infty$, 
            {\bfseries then} $\Rv[i_{\text{left}} ,i_{\text{mid}}] = c_1$; \quad
            {\bfseries if} $x_{\text{right}} \neq + \infty$, 
            {\bfseries then} $\Rv[i_{\text{right}} ,i_{\text{mid}}] = c_3$;
            \STATE $\Rv[i_{\text{mid}} ,i_{\text{mid}}] = c_2$;
        \ENDFOR
    \ENDFOR
\end{algorithmic}
\end{algorithm}


\section{REPARAMETERIZATION OF KERNEL LENGTHSCALES}
\label{sec:theo}
Considering the additive Laplace kernel with fixed lengthscale $\tilde{\theta}$ for all base kernels, applying linear projections $\left\{ \wv_{p}^{\top}\xv \right\}_{p=1}^{P}$ on inputs $\xv\in \Rb^D$ will give:
\begin{align}
    &\sum_{p=1}^{P}\sigma^2_p k_p\left( \wv^{\top}_{p}\xv,\wv^{\top}_{p}\xv^{\prime} \right)\nonumber \\
    = & \sum_{p=1}^{P} \sigma^2_p\exp \left( -  \frac{\sum_{d=1}^{D} \left| w_{p,d}\left( x_{d}-x_{d}^{\prime} \right) \right|}{\tilde{\theta}} \right)\nonumber \\
    = & \sum_{p=1}^{P} \prod_{d=1}^{D} \sigma^2_p\exp \left( - \frac{\left| x_{d}-x_{d}^{\prime} \right|}{\tilde{\theta} / \left| w_{p,d}\right| } \right)\nonumber \\
    = & \sum_{p=1}^{P} \prod_{d=1}^{D} \sigma^2_p\exp \left( - \frac{\left| x_{d}-x_{d}^{\prime} \right|}{\theta_{p,d}} \right),
\end{align}
This still leads to an additive Laplace kernel but with adaptive lengthscale $\theta_{p,d}$ for base kernels. The resulting kernel also retains \emph{sparse} Cholesky decomposition by the properties of Markov kernels so that the complexity of inference is $\Oc(M)$.

\section{INFERENCE OF PREDICTIVE DISTRIBUTION}
\label{sec:uq of inference}
Given an input $\xv \in \Rb^D$, the prediction of the DAK model can be written in the following equation according to \cref{eq:DAK prediction}: 
\begin{align}
    \tilde{f}_{\xv}
    &= \sum_{p=1}^{P}
    \sigma_p \Big(
        \phi(h_{\psi}^{[p]}(\xv)) \zv_p
    \Big) + \mu \nonumber\\
    &= \sum_{p=1}^{P}
    \sigma_p \Big(
        \bm{\phi}_{p}^{\top} \zv_p
    \Big) + \mu,
\end{align}
where $\bm{\phi}_{p}^{\top}:=\phi(h_{\psi}^{[p]}(\xv)) \in \Rb^{1 \times M}$
% , $\mu_p:=\mu_p(h_{\psi}^{[p]}(\xv)) \in \Rb$
. We assume the variational distribution over the independent Gaussian weights $\zv_p \sim \Nc(\bm{m}_{\zv_p}, \Sv_{\zv_p})$ and the bias $\mu \sim \Nc(m_{\mu}, \sigma_{\mu}^2)$. Then it's straighforward to deduce that
\begin{align}
    \bm{\phi}_{p}^{\top} \zv_p + \mu 
    &\sim
    \Nc\left(
    \bm{\phi}_{p}^{\top} \bm{m}_{\zv_p} + m_{\mu},\hspace{0.2em}
    \bm{\phi}_{p}^{\top} \Sv_{\zv_p} \bm{\phi}_{p} + \sigma_{\mu}^2
    \right), \\
    \sigma_p \left(
    \bm{\phi}_{p}^{\top} \zv_p 
    \right) + \mu
    & \sim
    \Nc\left(
    \sigma_p ( \bm{\phi}_{p}^{\top} \bm{m}_{\zv_p} )+ m_{\mu} ,\hspace{0.2em}
    \sigma_p^2( \bm{\phi}_{p}^{\top} \Sv_{\zv_p} \bm{\phi}_{p}) + \sigma_{\mu}^2
    \right), \\
    \tilde{f}_{\xv} = 
    \sum_{p=1}^{P}
    \sigma_p \left(
    \bm{\phi}_{p}^{\top} \zv_p
    \right) + \mu
    & \sim
    \Nc\left(
    \sum_{p=1}^{P}
    \sigma_p ( \bm{\phi}_{p}^{\top} \bm{m}_{\zv_p}) + m_{\mu} ,\hspace{0.2em}
    \sum_{p=1}^{P}
    \sigma_p^2( \bm{\phi}_{p}^{\top} \Sv_{\zv_p} \bm{\phi}_{p} ) + \sigma_{\mu}^2
    \right).
\end{align}
Therefore, we obtain the predictive distribution of the $\tilde{f}(\xv)$ at the point $\xv \in \Rb^D$ and its mean and variance are given by:
\begin{subequations}
\label{eq:dak inference closed form}
\begin{align}
    \Eb\left[ \tilde{f}_{\xv} \right]
        = \sum_{p=1}^{P}
        \sigma_p ( \bm{\phi}_{p}^{\top} \bm{m}_{\zv_p}) + m_{\mu},
\end{align}
\begin{align}
    \text{Var}\left[ \tilde{f}_{\xv} \right]
        =\sum_{p=1}^{P}
        \sigma_p^2( \bm{\phi}_{p}^{\top} \Sv_{\zv_p} \bm{\phi}_{p}) + \sigma_{\mu}^2.
\end{align}
\end{subequations}
% \begin{subequations}
% \label{eq:dak inference closed form}
%     \begin{align}
%         \Eb\left[ \tilde{f}(\xv) \right]
%         = \sum_{p=1}^{P}
%         \sigma_p ( \bm{\phi}_{p}^{\top} \bm{m}_{\zv_p} + m_{\mu_p} ),
%     \end{align}
%     \begin{align}
%         \text{Var}\left[ \tilde{f}(\xv) \right]
%         =\sum_{p=1}^{P}
%         \sigma_p^2( \bm{\phi}_{p}^{\top} \Sigma_{\zv_p} \bm{\phi}_{p} + \sigma_{\mu_p}^2).
%     \end{align}
% \end{subequations}


\section{TRAINING OF VARIATIONAL INFERENCE}
\label{sec:training}
Given the dataset $\mathcal{D}=\{ \Xv, \yv \}$ where $\Xv:=\{ \xv_i \}_{i=1}^N$, $\yv=(y_1,\ldots,y_N)^{\top}$, $\xv_i \in \Rb^D$, $y_i\in\Rb$, the prediction $\tilde{f}_{\Xv}\in \Rb^N$ of DAK is given by all the parameters $\bm{\theta}=\left\{ \psi, \bm{\sigma} \right\}$, $\bm{\eta}=\left\{ \{ \mv_{\zv_{p}},\Sv_{\zv_{p}}\}_{p=1}^{P} , \{m_{\mu},\sigma_{\mu} \} \right\}$ according to \cref{eq:DAK prediction}:
\begin{align}
    \tilde{f}_{\Xv}:= \tilde{f}(\Xv; \bm{\theta}, \bm{\eta})
    = \sum_{p=1}^{P}
    \sigma_p \Big(
        \phi(h_{\psi}^{[p]}(\Xv)) \zv_p
    \Big) + \mu,
\end{align}
where $\zv_{p} \sim \mathcal{N} (\bm{m}_{\zv_p} ,\Sv_{\zv_p})$, $p=1,\ldots,P$, and $\mu \sim \mathcal{N} ( m_{\mu},\sigma^2_{\mu} )$ are variational variables $\Theta_{\text{var}}$ parameterized by $\bm{\eta}$. The variational distribution is denoted by $q_{\bm{\eta}}(\Theta_{\text{var}})= q(\mu)\prod_{p=1}^{P} q(\zv_{p}) = \Nc ( m_{\mu} ,\sigma_{\mu}^2 )\prod_{p=1}^{P} 
\Nc ( \bm{m}_{\zv_p} ,\Sv_{\zv_p} )$, and the variational prior is denoted by $p(\Theta_{\text{var}})$.

We consider the KL divergence between $q_{\bm{\eta}}(\Theta_{\text{var}})$ and the true posterior $p(\Theta_{\text{var}}\vert \yv, \Xv, \bm{\theta})$:
\begin{align}
& \qquad \text{KL} \left[ q_{\bm{\eta}}(\Theta_{\text{var}}) \| p(\Theta_{\text{var}} \vert \yv,\Xv, \bm{\theta} ) \right] \nonumber \\
= & \int q_{\bm{\eta}}(\Theta_{\text{var}} )\log \frac{q_{\bm{\eta}}(\Theta_{\text{var}} )}{p(\Theta_{\text{var}} \vert \yv,\Xv,\bm{\theta} )} d\Theta_{\text{var}} \nonumber \\
= & \int q_{\bm{\eta}}(\Theta_{\text{var}} )\log \frac{q_{\bm{\eta}}(\Theta_{\text{var}} )p(\yv \vert \Xv,\bm{\theta})}{p(\yv \vert \Xv,\bm{\theta} ,\Theta_{\text{var}} )p(\Theta_{\text{var}} )} d\Theta_{\text{var}} \nonumber \\
= & \int q_{\bm{\eta}}(\Theta_{\text{var}} )\log \frac{q_{\bm{\eta}}(\Theta_{\text{var}} )}{p(\Theta_{\text{var}} )} d\Theta_{\text{var}} -\int q_{\bm{\eta}}(\Theta_{\text{var}} )\log p(\yv \vert \tilde{f}_{\Xv} )d\Theta_{\text{var}} +\log p(\yv\vert \Xv,\bm{\theta}).
\end{align}
Using the fact that $\text{KL}[\cdot \| \cdot] \geq 0$, we have
\begin{align}
\label{eq:variational lower bound}
    \log p(\yv\vert \Xv,\bm{\theta}) & \geq \int q_{\bm{\eta}}(\Theta_{\text{var}} )\log p(\yv \vert \tilde{f}_{\Xv} )d\Theta_{\text{var}} - \text{KL} \left[ q_{\bm{\eta}}(\Theta_{\text{var}} ) \| p(\Theta_{\text{var}}) \right] \nonumber \\
    & = \Eb_{q_{\bm{\eta}}(\Theta_{\text{var}} )} \left[ \log p(\yv \vert \tilde{f}_{\Xv} ) \right] - \text{KL} \left[ q_{\bm{\eta}}(\Theta_{\text{var}} ) \| p(\Theta_{\text{var}}) \right].
\end{align}

\paragraph{Full-training.}
Firstly, we present the joint training of $\bm{\theta}$ and $\bm{\eta}$. The most common approach optimizes the marginal log-likelihood (the left-hand side of \cref{eq:variational lower bound}):
\begin{align}
    \bm{\theta}^{\ast} &=\argmax_{\bm{\theta}} \log p(\yv\vert \Xv,\bm{\theta} ) \\
    &= \argmax_{\bm{\theta}} \log \int p\left( y\vert X,\bm{\theta},\Theta_{\text{var}} \right) p(\Theta_{\text{var}})d\Theta_{\text{var}},
\end{align}
which involves intractable integral in some tasks such as classification. Instead, we optimize the variational lower bound (the right-hand side of \cref{eq:variational lower bound}):
\begin{align}
    \Theta^{\ast} := \argmax_{\bm{\theta},\bm{\eta}} \mathcal{L}(\bm{\theta},\bm{\eta}) =\argmax_{\bm{\theta},\bm{\eta}}\left\{ E_{q_{\bm{\eta}}(\Theta_{\text{var}} )}\left[ \log p(\yv|\tilde{f}_{\Xv} ) \right] -\text{KL} \left[ q_{\bm{\eta}}(\Theta_{\text{var}} )\| p(\Theta_{\text{var}} ) \right] \right\}.
\end{align}

\paragraph{Fine-tuning.}
An alternative training approach is to firstly pre-train the deterministic parameters of feature extractor by standard neural network training, with mean squared error for regression or cross-entropy for classification as the loss function, and then fine-tune the last layer additive GP with fixed features. The objective function is identical to \cref{eq:elbo}, but $\bm{\theta}$ is learned during the pre-training step and is no longer optimized during fine-tuning.


\section{ELBO}%{DERIVATION OF ELBO}
\label{sec:elbo}
\subsection{Assumptions}
Consider the model $y_i = \tilde{f}(\xv_i) + \epsilon_i$ with the i.i.d. noise $\epsilon_i \overset{\text{i.i.d.}}{\sim} \Nc(0, \sigma_{f}^2)$ and $\tilde{f} : \Rb^D \rightarrow \Rb$ is defined in \cref{eq:DAK prediction}. The training dataset is $\mathcal{D} = \{ \Xv, \yv \}$ where $\Xv:=\{ \xv_i \}_{i=1}^N$, $\yv=(y_1,\ldots,y_N)^{\top}$, $\xv_i \in \Rb^D$, $y_i\in\Rb$. $\Theta_{\text{var}}:= \{ \mu ,\{ \zv_{p}\}_{p=1}^{P} \}$ are the variational random variables consisting of Gaussian weights and bias of $P$ units, $\psi$ are the parameters of the NN, $\bm{\sigma}:=(\sigma_1, \ldots, \sigma_p)^{\top}$ are the scale parameters of base GP layers. The variational distributions are $q(\mu)=\Nc(m_{\mu}, \sigma_{\mu}^2)$, $q(\zv_p)=\Nc(\bm{m}_{\zv_p}, \Sv_{\zv_p})$ and the variational priors are $p(\mu)=\Nc(\check{m}_{\mu} ,\check{\sigma}^2_{\mu})$, $p(\zv_p)=\Nc(\check{\bm{m}}_{\zv_p} ,\check{\Sv}_{\zv_p})$. Note that $\Sv_{\zv_p}\in\Rb^{M \times M}$ is a diagonal covariance matrix due to the independence of $\zv_p$, $M$ is the number of inducing points $\Uv$ defined in \cref{eq:GPlayer}, and $\bm{m}_{\zv_p} \in \Rb^M$, $m_{\mu} \in \Rb$, $\sigma_{\mu}^2 \in \Rb$. We derive the ELBO in VI to learn the preditive posterior over the variational variables $\Theta_{\text{var}}:= \{ \mu ,\{ \zv_{p}\}_{p=1}^{P} \}$ parameterized by $\bm{\eta}:=\left\{ \{ \mv_{\zv_{p}},\Sv_{\zv_{p}}\}_{p=1}^{P} , \{m_{\mu},\sigma_{\mu} \} \right\}$, and optimize the deterministic parameters $\bm{\theta}:=\{\psi, \bm{\sigma}\}$.

\subsection{Expected Log Likelihood}
\paragraph{Closed Form}
The \emph{expected log likelihood}, which is the first term in ELBO defined in \cref{eq:elbo}, is given by 
\begin{align}
    {\Eb}_{q_{\bm{\eta}}(\Theta_{\text{var}})} \left[ \log \text{Pr} (\yv \vert \tilde{f}_{\Xv} ) \right]
    &= {\Eb}_{q_{\bm{\eta}}(\Theta_{\text{var}})} \left[ 
    \log \prod_{i=1}^{N} 
    p (y_i \vert \tilde{f}_{\xv_i} )
    \right] \nonumber\\
    &= \sum_{i=1}^{N} 
    {\Eb}_{q_{\bm{\eta}}(\Theta_{\text{var}})} \left[ 
    \log
    p (y_i \vert \tilde{f}_{\xv_i} )
    \right] \nonumber\\
    &= \sum_{i=1}^{N} 
    {\Eb}_{q_{\bm{\eta}}(\Theta_{\text{var}})} \left[ 
    \log
    \Nc( \tilde{f}_i,\hspace{0.2em} \sigma_{f}^2 )
    \right] \nonumber\\
    &= \sum_{i=1}^{N} 
    {\Eb}_{q_{\bm{\eta}}(\Theta_{\text{var}})} \left[ 
    \log \left(
    (2\pi \sigma_{f}^2)^{-\frac{1}{2}}
    \exp\left\{  
        -\frac{ (y_i - \tilde{f}_i)^2 }{2 \sigma_{f}^2}
    \right\}
    \right)
    \right] \nonumber\\
    &= \sum_{i=1}^{N} 
    {\Eb}_{q_{\bm{\eta}}(\Theta_{\text{var}})} \left[
    -\frac{1}{2} \log(2\pi) 
    - \frac{1}{2}\log(\sigma_{f}^2)
    - \frac{1}{2 \sigma_{f}^2}
    (y_i - \tilde{f}_i)^2
    \right] \nonumber\\
    &= - \frac{N}{2} \log(2\pi)
    - \frac{N}{2} \log(\sigma_{f}^2)
    - \frac{1}{2 \sigma_{f}^2}
    \sum_{i=1}^{N}
    {\Eb}_{q_{\bm{\eta}}(\Theta_{\text{var}})} \left[
    (y_i - \tilde{f}_i)^2
    \right] \nonumber\\
    &= - \frac{N}{2} \log(2\pi)
    - \frac{N}{2} \log(\sigma_{f}^2)
    - \frac{1}{2 \sigma_{f}^2}
    \sum_{i=1}^{N} \left(
    \left({\Eb}_{q(\Theta_{\text{var}})} \left[
    (y_i - \tilde{f}_i)
    \right] \right)^2
    + \text{Var}_{q(\Theta_{\text{var}})} \left[
    (y_i - \tilde{f}_i)
    \right]
    \right) \label{eq:evidence halfway},
\end{align}
where
\begin{align}
    \tilde{f}_i
    % \mu_{\tilde{f}_i} &:= \tilde{f}(\xv_i;\Theta_{\text{var}}, \Theta_{\text{det}} ) \nonumber\\
    &= \sum_{p=1}^{P} \sigma_p \Big(
    \begingroup
        \color{blue}
        \underbracket{
            \color{black}
            \phi(h_{\psi}^{[p]}(\xv_i))
        }_{\color{blue}
            :=\bm{\phi}_{i,p}^{\top} \in \Rb^{1 \times M}
        }
    \endgroup
    \zv_p
    \Big)
    + \mu
    % \begingroup
    %     \color{blue}
    %     \underbracket{
    %         \color{black}
    %         \mu_{p}(h_{\psi}^{[p]}(\xv_i))
    %     }_{\color{blue}
    %         :=\mu_{i,p} \in \Rb
    %     }
    % \endgroup 
    \nonumber\\
    &= \sum_{p=1}^{P} \sigma_p \left(
    \bm{\phi}_{i,p}^{\top} \zv_p 
    \right) + \mu.
\end{align}
Recall that the variational assumptions $q(\zv_p)=\Nc(\bm{m}_{\zv_p}, \Sv_{\zv_p})$ and $q(\mu)=\Nc(m_{\mu}, \sigma_{\mu}^2)$, we can infer that
\begin{align}
    \bm{\phi}_{i,p}^{\top} \zv_p + \mu 
    &\sim
    \Nc\left(
    \bm{\phi}_{i,p}^{\top} \bm{m}_{\zv_p} + m_{\mu},\hspace{0.2em}
    \bm{\phi}_{i,p}^{\top} \Sv_{\zv_p} \bm{\phi}_{i,p} + \sigma_{\mu}^2
    \right), \\
    \sigma_p \left(
    \bm{\phi}_{i,p}^{\top} \zv_p 
    \right) + \mu
    & \sim
    \Nc\left(
    \sigma_p ( \bm{\phi}_{i,p}^{\top} \bm{m}_{\zv_p} ) + m_{\mu},\hspace{0.2em}
    \sigma_p^2( \bm{\phi}_{i,p}^{\top} \Sv_{\zv_p} \bm{\phi}_{i,p} ) + \sigma_{\mu}^2
    \right), \\
    \tilde{f}_i = 
    \sum_{p=1}^{P}
    \sigma_p \left(
    \bm{\phi}_{i,p}^{\top} \zv_p 
    \right)+ \mu
    & \sim
    \Nc\left(
    \sum_{p=1}^{P}
    \sigma_p ( \bm{\phi}_{i,p}^{\top} \bm{m}_{\zv_p} )+ m_{\mu},\hspace{0.2em}
    \sum_{p=1}^{P}
    \sigma_p^2( \bm{\phi}_{i,p}^{\top} \Sv_{\zv_p} \bm{\phi}_{i,p} ) + \sigma_{\mu}^2
    \right), \\
    y_i - \tilde{f}_i
    & \sim 
    \Nc\left(
    y_i - 
    \sum_{p=1}^{P}
    \sigma_p ( \bm{\phi}_{i,p}^{\top} \bm{m}_{\zv_p} ) -m_{\mu},\hspace{0.2em}
    \sum_{p=1}^{P}
    \sigma_p^2( \bm{\phi}_{i,p}^{\top} \Sv_{\zv_p} \bm{\phi}_{i,p} ) + \sigma_{\mu}^2
    \right).
\end{align}
Therefore, 
\begin{subequations}\label{eq:exp and var in evidence}
    \begin{align}
        \left({\Eb}_{q(\Theta_{\text{var}})} \left[
        (y_i - \tilde{f}_i)
        \right] \right)^2
        = \left(
         y_i - 
        \sum_{p=1}^{P}
        \sigma_p ( \bm{\phi}_{i,p}^{\top} \bm{m}_{\zv_p} ) -m_{\mu}
        \right)^2,
    \end{align}
    \begin{align}
        \text{Var}_{q(\Theta_{\text{var}})}
        \left[
        (y_i - \tilde{f}_i)
        \right]
        = \sum_{p=1}^{P}
        \sigma_p^2( \bm{\phi}_{i,p}^{\top} \Sv_{\zv_p} \bm{\phi}_{i,p} ) + \sigma_{\mu}^2.
    \end{align}
\end{subequations}
By applying \cref{eq:exp and var in evidence} to \cref{eq:evidence halfway}, we derive the analytical formula for the expected evidence, expressed as
\begin{align}
    {\Eb}_{q_{\bm{\eta}}(\Theta_{\text{var}})} \left[ \log \text{Pr} (\yv \vert \tilde{f}_{\Xv} ) \right]
    &= - \frac{N}{2} \log(2\pi)
    - \frac{N}{2} \log(\sigma_{f}^2) \nonumber\\
    &- \frac{1}{2 \sigma_{f}^2}
    \sum_{i=1}^{N} \left(
        \Big(
         y_i - 
        \sum_{p=1}^{P}
        \sigma_p ( \bm{\phi}_{i,p}^{\top} \bm{m}_{\zv_p} ) -m_{\mu}
        \Big)^2
        + \sum_{p=1}^{P}
        \sigma_p^2( \bm{\phi}_{i,p}^{\top} \Sv_{\zv_p} \bm{\phi}_{i,p} )+ \sigma_{\mu}^2
    \right). \label{eq:evidence final}
\end{align}

\paragraph{Monte Carlo Approximation}
For comparison, we provide the equation for computing the Monte Carlo estimate of the ELBO in the paragraph that follows.
\begin{align}
    {\Eb}_{q_{\bm{\eta}}(\Theta_{\text{var}})} \left[ \log \text{Pr} (\yv \vert \tilde{f}_{\Xv} ) \right]
    % &= {\Eb}_{q(\Theta)} \left[ 
    % \log \prod_{i=1}^{N} 
    % p (y_i \vert \xv_i,\Theta, \psi, \bm{\sigma})
    % \right] \nonumber\\
    &= \sum_{i=1}^{N} 
    {\Eb}_{q_{\bm{\eta}}(\Theta_{\text{var}} )} \left[ 
    \log
    p (y_i \vert \tilde{f}_{\xv_i} )
    \right] \nonumber\\
    & \approx \sum_{i=1}^{N}
    \frac{1}{S}
     \sum_{s=1}^{S}
    \log
    p (y_i \vert \xv_i,\tilde{\Theta}^{(s)}_{\text{var}}, \bm{\theta} ) \nonumber\\
    &= \frac{1}{S} \sum_{i=1}^{N} 
    \sum_{s=1}^{S} 
    \log
    \Nc(y_i \left\vert\right. \tilde{f}_{i}^{(s)},\hspace{0.2em} \sigma_{f}^2 )
    \nonumber\\
    &= \frac{1}{S} \sum_{i=1}^{N} 
    \sum_{s=1}^{S} 
    \log \left(
    (2\pi \sigma_{f}^2)^{-\frac{1}{2}}
    \exp\left\{  
        -\frac{ (y_i - \tilde{f}_{i}^{(s)})^2 }{2 \sigma_{f}^2}
    \right\}
    \right)
    \nonumber\\
    &= \frac{1}{S} \sum_{i=1}^{N} 
    \sum_{s=1}^{S} \left(
    -\frac{1}{2} \log(2\pi) 
    - \frac{1}{2}\log(\sigma_{f}^2)
    - \frac{1}{2 \sigma_{f}^2}
    (y_i - \tilde{f}_{i}^{(s)})^2
    \right) \nonumber\\
    &= - \frac{N}{2} \log(2\pi)
    - \frac{N}{2} \log(\sigma_{f}^2)
    - \frac{1}{2 \sigma_{f}^2}
    \sum_{i=1}^{N}
    \frac{1}{S} \sum_{s=1}^{S}
    (y_i - \tilde{f}_{i}^{(s)})^2, \label{eq:evidence halfway mc approx}
\end{align}
where $S$ is the number of Monte Carlo samples, $\{  \tilde{\mu}^{(s)} ,\{ \tilde{\zv}_{p}^{(s)} \}_{p=1}^{P} \} := \tilde{\Theta}^{(s)}_{\text{var}}$ are the $s$-th Monte Carlo samplings over the variational parameters $\Theta_{\text{var}}$ and $\tilde{\Theta}^{(s)}_{\text{var}} \sim q_{\bm{\eta}}(\Theta_{\text{var}})$, $\tilde{f}_{i}^{(s)}$ is given as follows:
\begin{align}
    \tilde{f}_{i}^{(s)} &:= \tilde{f}(\xv_i;\tilde{\Theta}^{(s)}_{\text{var}},\bm{\theta} ) \nonumber\\
    &= \sum_{p=1}^{P} \sigma_p \Big(
    \begingroup
        \color{blue}
        \underbracket{
            \color{black}
            \phi(h_{\psi}^{[p]}(\xv_i))
        }_{\color{blue}
            :=\bm{\phi}_{i,p}^{\top} \in \Rb^{1 \times M}
        }
    \endgroup
    \tilde{\zv}_p^{(s)} 
    \Big) + \tilde{\mu}^{(s)} \nonumber\\
    &= \sum_{p=1}^{P} \sigma_p \left(
    \bm{\phi}_{i,p}^{\top} \tilde{\zv}_p^{(s)} 
    \right)+ \tilde{\mu}^{(s)}. \label{eq:mc approx mean}
\end{align}
Therefore, we plug \cref{eq:mc approx mean} into \cref{eq:evidence halfway mc approx} and get the the Monte Carlo estimate of the ELBO written in the following formula:
\begin{align}
    {\Eb}_{q_{\bm{\eta}}(\Theta_{\text{var}})} \left[ \log \text{Pr} (\yv \vert \tilde{f}_{\Xv} ) \right]
    &\approx
    - \frac{N}{2} \log(2\pi)
    - \frac{N}{2} \log(\sigma_{f}^2)
    - \frac{1}{2 \sigma_{f}^2}
    \sum_{i=1}^{N}
    \frac{1}{S} \sum_{s=1}^{S}
    \Big(y_i - 
    \sum_{p=1}^{P} \sigma_p \left(
    \bm{\phi}_{i,p}^{\top} \tilde{\zv}_p^{(s)} 
    \Big)- \tilde{\mu}^{(s)}
    \right)^2, \label{eq:evidence final mc approx} \\
    \tilde{\zv}_p^{(s)} &\sim \Nc(\bm{m}_{\zv_p}, \Sv_{\zv_p}),\qquad
    \tilde{\mu}^{(s)} \sim \Nc(m_{\mu}, \sigma_{\mu}^2).
\end{align}


\subsection{KL Divergence}
Since we place Gaussian assumptions over the variational parameters $\Theta_{\text{var}}$,  the \emph{KL divergence}, which is the second term in ELBO defined in \cref{eq:elbo}, is then given by
\begin{align}
    \text{KL} \left[ q(\Theta_{\text{var}} ) \| p(\Theta_{\text{var}}) \right]
    &= \text{KL} \left[ q( \mu ,\{ \zv_{p}\}_{p=1}^{P} ) \Vert p( \mu ,\{ \zv_{p}\}_{p=1}^{P}) \right] \nonumber\\
    & =  
    \text{KL} \left[ q(\mu) \Vert p(\mu) \right] 
    + \sum_{p=1}^{P} 
    \text{KL} \left[ q(\zv_{p}) \Vert p(\zv_{p}) \right],
\end{align}

\begin{align}
     \text{KL} \left[ q(\mu) \Vert p(\mu) \right]
     = \frac{1}{2} \left(
     \frac{\sigma_{\mu}^2}{\check{\sigma}_{\mu}^2} 
     + \frac{(m_{\mu} - \check{m}_{\mu})^2}{\check{\sigma}_{\mu}^2} 
     -\log\left( \frac{\sigma_{\mu}^2}{\check{\sigma}_{\mu}^2} \right)
     -1
     \right),
\end{align}

\begin{align}
    \text{KL} \left[ q(\zv_{p}) \Vert p(\zv_{p}) \right]
    = \frac{1}{2} \sum_{i=1}^{M} \left(
     \frac{[\Sv_{\zv_p}]_{ii}}{[\check{\Sv}_{\zv_p}]_{ii}} 
     + \frac{([\bm{m}_{\zv_p}]_{i} - [\check{\bm{m}}_{\zv_p}]_i)^2}{[\check{\Sv}_{\zv_p}]_{ii}}
     -\log\left( 
     \frac{[\Sv_{\zv_p}]_{ii}}{[\check{\Sv}_{\zv_p}]_{ii}}  
     \right)
     -1
     \right),
\end{align}
where $[\Sv_{\zv_p}]_{ii}$ is the $(i,i)$-th element of the diagonal covariance matrix $\Sv_{\zv_p} \in \Rb^{M \times M}$, $[\bm{m}_{\zv_p}]_{i}$ is the $i$-th element of the mean vector $\bm{m}_{\zv_p} \in \Rb^M$, the approximated posteriors are $q(\mu)=\Nc(m_{\mu}, \sigma_{\mu}^2)$, $q(\zv_p)=\Nc(\bm{m}_{\zv_p}, \Sv_{\zv_p})$ and the priors are $p(\mu)=\Nc(\check{m}_{\mu} ,\check{\sigma}^2_{\mu})$, $p(\zv_p)=\Nc(\check{\bm{m}}_{\zv_p} ,\check{\Sv}_{\zv_p})$.

% \subsection{Performance Comparison}
% \label{sec:toy exp compare}
% We compare the perforamce of computing the ELBO in \cref{eq:elbo} by using closed form in \cref{eq:evidence final} and using Monte Carlo approximation in \cref{eq:evidence final mc approx} in a toy example.
% \textcolor{red}{Table or Figure to add if time available}


\subsection{Limitations of the Closed-Form ELBO}

The closed-form ELBO is only applicable to regression problems. In classification, applying the softmax function to $\tilde{f}(\xv;\bm{\theta}, \bm{\eta})$ results in a non-analytic predictive distribution, meaning the ELBO must still be computed via Monte Carlo sampling during training. Similarly, the closed-form expressions for the predictive mean and variance, as provided in \cref{eq:dak inference closed form} in \Cref{sec:uq of inference}, are not applicable to classification but only apply to regression problems.


\section{COMPUTATIONAL COMPLEXITY}
\label{sec:complexity}
In this section, we discuss the computational complexity of various DKL models compared to the proposed DAK method, focusing on the GP layer as the most computationally demanding component. \Cref{tab:complexity supp} shows the computational complexity of our model compared to other state-of-the-art GP and DKL methods.

\begin{table}[ht]
    \caption{Computational complexity of the DKL models for $N$ training points. The reported training complexity is for one iteration. $\hat{M}$ is the number of inducing points in SVGP and KISS-GP, while $M$ is the size of induced grids in DAK, $M < \hat{M}$. $S$ is the number of Monte Carlo samples, $B$ is the size of mini-batch, $D_w$ is the dimension of the NN outputs in DKL, $P$ is the dimension of the outputs after applying linear transformations to the NN outputs in the proposed DAK model. DAK-MC refers to the DAK model using Monte Carlo approximation, while DAK-CF refers to the DAK model using closed-form inference and ELBO.}
    \centering
    \begin{tabular}{lcc}
    \toprule[1pt]
                  & \textbf{Inference}       & \textbf{Training} (per iteration) \\
    \midrule[0.5pt]
    NN + SVGP     & $\Oc(\hat{M}^2 N)$    & $\Oc( S D_w MB + \hat{M}^3)$ \\
    NN + KISS-GP  & $\Oc(D_w \hat{M}^{1+\frac{1}{D_w}})$  & $\Oc(S D_w MB + D_w \hat{M}^{\frac{3}{D_w}})$ \\
    DAK-MC (ours) & $\Oc(SM)$       & $\Oc(SPMB + PM)$   \\
    DAK-CF (ours) & $\Oc(M)$        & $\Oc(PMB + PM)$    \\
    \bottomrule[1pt]
    \end{tabular}
    \label{tab:complexity supp}
\end{table}

\paragraph{Inference Complexity.}
In inference based on induced approximation, computing the multiplication of the inverse of the covariance matrix $k(\Uv, \Uv)$ and a vector takes $\Oc(\hat{M}^2N)$ time for $\hat{M}$ inducing points $\Uv$ and $N$ training points when using SVGP. This cost is reduced by KISS-GP to $\Oc(D \hat{M}^{1+\frac{1}{D}})$ by decomposing the covariance matrix into a Kronecker product of $D$ one-dimensional covariance matrices of the inducing points: $k(\Uv, \Uv) = \bigotimes_{d=1}^{D} k(\Uv^{[d]}, \Uv^{[d]})$. Despite the significant reduction on complexity, it requires inducing points $\Uv$ arranged on a Cartesian grid of size $\hat{M} = \prod_{d=1}^{D} \hat{M}_d$, where $\hat{M}_d$ is the number of inducing points in the $d$-th dimension. In high-dimensional spaces, fixing $\hat{M}$ leads to very small $\hat{M}_d$ per dimension, which can degrade model performance. To address this, we propose the DAK model via sparse finite-rank approximation, which employs an additive Laplace kernel for GPs. The inverse Cholesky factor $\Lv_{\Uv}^{\top}$ for one-dimensional induced grids $\Uv$ of size $M$, where $M < \hat{M}$, as defined in \cref{eq:GPlayer}, is sparse and can be computed in $\Oc(M)$ time.

\paragraph{Training Complexity.}
In training, VI requires computing the ELBO as described in \cref{eq:elbo}, which consists of two terms: the \emph{expected log likelihood} and the \emph{KL divergence} between the variational distributions and priors. 

1) The \emph{expected log likelihood} is usually approximated via Monte Carlo sampling at a cost of $\Oc(S N_{\Theta} N)$, where $S$ is the number of Monte Carlo samples, $N_{\Theta}$ is the total number of variational parameters $\Theta_{\text{var}}$, and $N$ is the number of training points. This complexity can be reduced to $\Oc(S N_{\Theta} B)$ by applying stochastic variational inference with a mini-batch of size $B \ll N$. For DKL models using SVGP and KISS-GP, $\Theta_{\text{var}}$ are inducing variables, and the expectation does not have a closed form, requiring Monte Carlo sampling. In contrast, in the proposed DAK model, $\Theta_{\text{var}}= \{ \{ \zv_{p}\}_{p=1}^{P}, \mu \}$ consists of independent Gaussian weights $\zv_p\in \Rb^M$ and bias $\mu$. This allows us to derive an analytical form for this term, as shown in \cref{eq:evidence final} in \Cref{sec:elbo}, reducing the computational cost to $\Oc(N_{\Theta} B) = \Oc(PM B)$ when using a mini-batch of size $B$.

2) The \emph{KL divergence} between two Gaussian distributions can be computed in closed form. This leads to a linear time complexity of $\Oc(N_{\Theta})$ if the parameters $\Theta_{\text{var}}$ are independent, or cubic time $\Oc(N_{\Theta}^3)$ if they are fully correlated. In SVGP and KISS-GP, $\Theta_{\text{var}}$ represents fully correlated Gaussian distributed inducing variables, so computing the KL divergence takes $\Oc(\hat{M}^3)$ for SVGP. In KISS-GP, this can be reduced to $\Oc(D \hat{M}^{\frac{3}{D}})$ using fast eigendecomposition of Kronecker matrices. In the DAK model, the weights $\{\zv_p\}_{p=1}^{P}$ as defined in \cref{eq:GPlayer} are independent Gaussian random variables, allowing the KL divergence to be computed in $\Oc(N_{\Theta}) = \Oc(PM)$ time, where $P$ is the number of base GP layers.


\section{ADDITIONAL DISCUSSIONS}

Although interpretability is one advantage of additive models, the main motivation for replacing a GP layer with an additive GP layer in our work is to handle high-dimensional data. When the input dimension is low, it is reasonable that GPs are superior to additive GPs since the additive kernel is an approximated and restrictive kernel. However, when the input dimension increases, the computational complexity grows considerably even in GPs with sparse approximation. For example, in DKL, the output dimension of NN encoder is usually chosen as small as 2, while in pixel data experiments, DKL cannot handle the computation associated with the dimensionality when the output dimension of ResNet is 512 or more. Although DKL is superior in low-dimensional and simple cases, we view additive structure as a necessary component to achieve scalability and good performance with high-dimensional data.

\subsection{Why choosing the induced grids instead of learning the inducing points?}

From an approximation accuracy point of view, there are two separate strategies to increase the accuracy. The first one is to learn the inducing point locations. The second one, however, is to simply increase the number of inducing points on a pre-specified finer grid. The second method is much easier to implement and has a theoretical guarantee by the GP regression theory: as the inducing points become dense in the input region, the approximation will become exact. In contrast, the first approach does not have such a favorable theoretical guarantee. 

The second approach would become difficult to use for many existing methodologies as in general the computational cost would scale as $\mathcal{O}(M^3)$ with $M$ inducing points, which is particularly problematic in high dimensions. 
% The first approach can be viewed as a compromise in those situations, and that is why many existing methods chose to learn the locations of the inducing points instead.
This difficulty is resolved by additive GPs, since approximating an additive GP boils down to approximating one dimensional GPs, which can be accomplished by using a set of pre-specified inducing points on a fine grid in 1-D. One major benefit of the proposed methodology is that the computation now scales at $\mathcal{O}(M)$, enabled by the Markov kernel and the additive kernel. Therefore, a large number of inducing points can be used in an efficient way. 

The proposed method also has several additional benefits: 1) It can decouple to some extent the neural network component and GP component by avoiding learning the inducing points, which may help reduce overfitting/overconfidence; 2) The equivalence to BNN holds exactly with the fixed inducing points, whereas for learned inducing points, this BNN equivalence breaks down, and the proposed computation/training framework would not be possible to carry through; 3) It can simplify the overall optimization since there is no need to learn the inducing points.

\subsection{Limitations and future directions}

Generally, a finer grid will lead to better approximations, but the number of parameters to be trained will also increase. Therefore, there is a trade-off between the accuracy and the computational cost that we can afford. This current work is using a specific Laplace kernel, which can utilize sparse Cholesky decomposition. More general kernels may result in more computational complexity but better representation power of the model. In addition, the current variational family is restricted under mean-field assumptions. A more general variational family, e.g. full/low-rank covariance, may lead to superior performance in some applications. 


\section{EXPERIMENTAL DETAILS}
\label{sec:expdetail}
In this section, we provide additional details regarding the experiments.

\subsection{Benchmarks for Regression}
\label{subsec:regression supp}
\paragraph{Experiment Setup}
For all models, the NN architecture is a fully connected NN with rectified linear unit (ReLU) activation function \citep{nair2010rectified} and two hidden layers containing 64 and 32 neurons, respectively, structured as $D \rightarrow 64 \rightarrow 32 \rightarrow D_w$, where $D$ is the input feature size (also the size of input $\Xv$) and $D_w$ is the output feature size. The models are evaluated with $D_w=16$, 64, and 256, respectively. The number of Monte Carlo samples is set to 8 during training and 20 during inference.

The NN is a deterministic model, and we use the negative Gaussian log-likelihood as the loss function to quantify the uncertainty of the NN outputs and compute the NLPD.

For NN+SVGP, the inducing points are set to the size of 64 in $D_w$ dimension. We implement the \texttt{ApproximateGP} model in GPyTorch \citep{gardner2018gpytorch}, defining the inducing variables as variational parameters, and use \texttt{VariationalELBO} in GPyTorch to perform variational inference and compute the loss.

SV-DKL is originally designed for classification, so for a fair comparison in regression tasks, we modify it by first applying a linear embedding layer $\Wv: \Rb^{D_w} \rightarrow \Rb^P$ with $P=16$ and normalizing the outputs to the interval $[0,1]$ for each base GP, similar to the DAK model. To adapt the additive GP layer for regression, we remove the softmax function from the model in eq. (1) of \citep{wilson2016stochastic}. Given training data $\{ \xv_i, \yv_i \}_{i=1}^{N}$, the model is modified as follows:
\begin{align}
    p(\yv_i \vert \fv_i, A) = \mathcal{A}(\fv_i)^{\top} \yv_i
\end{align}
where $\fv_i \in \Rb^P$ is a vector of independent GPs followed by a linear mixing layer $\mathcal{A}(\fv_i) = A \fv_i$, with $A \in \Rb^{C \times P}$ as the transformation matrix. Here, $C=1$ for single-task regression. For each $p$-th GP ($1 \leq p \leq P$) in the additive GP layer, the corresponding inducing variables $\uv_p$ are set to the size of 64 and treated as variational parameters for training. We use the \texttt{GridInterpolationVariationalStrategy} model with \texttt{LMCVariationalStrategy} in GPyTorch to perform KISS-GP with variational inducing variables, augmented by a linear mixing layer.

For AV-DKL, the inducing points are set to size of 64 in $D_{w}$ dimension. We implement the AV-DKL model based on the source code~\cite{matias2024amortized}.

Both DAK-MC and DAK-CF use the same additive GP layer size as SV-DKL, with $P=16$, and employ fixed induced grids $\Uv = \{1/8, 2/8, \ldots, 7/8\}$ of size 7 for each base GP, which is much smaller than that of SV-DKL.

\paragraph{Metrics}
Let $\{\xv_t, y_t\}_{t=1}^{T}$ represent a test dataset of size $T$, where $\mu_t$ and $\sigma_t^2$ are the predictive mean and variance. We evaluate model performance using two common metrics: Root Mean Squared Error (RMSE) and Negative Log Predictive Density (NLPD).

RMSE is widely used to assess the accuracy of predictions, measuring how far predictions deviate from the true target values. It is calculated as:
\begin{align}
    \text{RMSE} = \sqrt{ \frac{1}{T} \sum_{t=1}^{T}(y_t - \mu_t)^2 }.
\end{align}

NLPD is a standard probabilistic metric for evaluating the quality of a model's uncertainty quantification. It represents the negative log likelihood of the test data given the predictive distribution. For GPs, NLPD is calculated as:
\begin{align}
    \text{NLPD}
    &= - \sum_{t=1}^{T} \log p(y_t = \mu_t \vert \xv_t) \\
    &= \frac{1}{T}
    \sum_{t=1}^{T} \Big[
    \frac{(y_t - \mu_t)^2}{2\sigma_t^2} + \frac{1}{2} \log(2\pi \sigma_t^2)
    \Big].
\end{align}
Both RMSE and NLPD are widely used in the GP regression literature, where smaller values indicate better model performance.

\paragraph{Computing Infrastructure}
The experiments for regression were run on Macbook Pro M1 with 8 cores and 16GB RAM.

\subsection{Benchmarks for Classification}
\label{subsec:classification supp}
We use PyTorch \citep{paszke2019pytorch} baseline of NN models, GPyTorch \citep{gardner2018gpytorch} baseline of SVGP and SV-DKL models. In classification tasks, we apply a softmax likelihood to normalize the output digits to probability distributions. The NN is a deterministic model trained via negative log-likelihood loss, while DKL and DAK models are trained via ELBO loss. The setting of all training tasks are described in \Cref{tab:model classification} and \Cref{tab:optimizer classification}.

SVGP is originally designed for single-output regression. To make it fit for multi-output classification, we used \texttt{IndependentMultitaskVariationalStrategy} in GPyTorch to implement the multi-task \texttt{ApproximateGP} model, and use \texttt{VariationalELBO} with \texttt{SoftmaxLikelihood} in GPyTorch to perform variational inference and compute the loss. 

For SV-DKL, we employed the same \texttt{VariationalELBO} with \texttt{SoftmaxLikelihood} as the variational loss objective. \texttt{GridInterpolationVariationalStrategy} is applied within \texttt{IndependentMultitaskVariationalStrategy} to perform additive KISS-GP approximation. For each KISS-GP unit, we used $64$ variational inducing points initialized on a grid of size $[-1,1]$. 

For DAK, we implemented DAK-MC using Monte Carlo estimation given the intractable softmax likelihood. We employed fixed induced grids $\Uv=\{ -31/32, -30/32, \ldots, 30/32, 31/32 \}$ of size 63 for each base GP component.

\begin{table}[ht]
\caption{Model architectures for image classification on MNIST, CIFAR-10 and CIFAR-100.}
\centering
\resizebox{0.7\linewidth}{!}{
\begin{tabular}{l|l|ccc}
\toprule[1pt]
Model                   & Hyper-parameter          & MNIST       & CIFAR-10    & CIFAR-100   \\
\midrule[0.5pt]
\multirow{4}{*}{NN+SVGP}   & Feature extractor        & CNN         & ResNet-18   & ResNet-34   \\
                        & NN out features $D_w$         & 128         & 512         & 512         \\
                        & Embedding features $P$               & 16          & 64          & 128         \\
                        & \# inducing points $\hat{M}$      & 512         & 512         & 512         \\
                        & \# epochs       & 20         & 200         & 200         \\
                        & Training strategy      & Full-training         & Full-training         & Fine-tuning         \\
\midrule[0.5pt]
\multirow{5}{*}{SV-DKL} & Feature extractor        & CNN         & ResNet-18   & ResNet-34   \\
                        & NN out features $D_w$         & 128         & 512         & 512         \\
                        & Embedding features $P$               & 16          & 64          & 128         \\
                        & \# inducing points $\hat{M}$      & 64          & 64          & 64          \\
                        & Grid bounds              & {[}-1,1{]} & {[}-1,1{]} & {[}-1,1{]} \\
                        & \# epochs       & 20         & 200         & 200         \\
                        & Training strategy       & Full-training         & Full-training         & Fine-tuning         \\
\midrule[0.5pt]
\multirow{4}{*}{DAK}    & Feature extractor        & CNN         & ResNet-18   & ResNet-34   \\
                        & NN out features $D_w$         & 128         & 512         & 512         \\
                        & Embedding features $P$               & 16          & 64          & 128         \\
                        & \# induced interpolation $M$ & 63          & 63          & 63         \\
                        & \# epochs       & 20         & 200         & 200         \\
                        & Training strategy      & Full-training         & Full-training         & Full-training         \\
\bottomrule[1pt]
\end{tabular}

}
\label{tab:model classification}
\end{table}

\paragraph{MNIST} We used a CNN implemented in PyTorch as the feature extractor: \texttt{Conv2d}(1,32,3) $\rightarrow$ \texttt{Conv2d}(32,64,3) $\rightarrow$ \texttt{MaxPool2d}(2) $\rightarrow$ \texttt{Dropout}(0.25) $\rightarrow$ \texttt{Linear}(9216,128) $\rightarrow$ \texttt{Dropout}(0.5). To make a fair comparison, for both SV-DKL and DAK, we applied an embedding module through a linear layer that transform $128$ output features into $P=16$ base GP channels. 

\paragraph{CIFAR-10} We used a ResNet-18 as the feature extractor followed by a linear embedding layer that compressed the $512$ output features into $P=64$ base GP channels. 

\paragraph{CIFAR-100} We used a pretrained ResNet-34 as the feature extractor for SV-DKL and fine-tuned GP output layers since SV-DKL struggled to fit using full-training. For proposed DAK, we used full-training. The number of base GP channels is selected as $P=128$. 

\begin{table}[ht]
\caption{Details of training optimizer for image classification on MNIST, CIFAR-10 and CIFAR-100.}
\centering
\resizebox{0.7\linewidth}{!}{

\begin{tabular}{l|ccc}
\toprule[1pt]
Optimization      & MNIST                                                             & CIFAR-10                                                                                                  & CIFAR-100                                                                                                 \\
\midrule[0.5pt]
Optimizer         & Adadelta                                                          & SGD                                                                                                       & SGD                                                                                                       \\
Initial lr.       & 1.0                                                               & 0.1                                                                                                       & 0.1                                                                                                       \\
Weight decay      & 0.0001                                                            & 0.0001                                                                                                    & 0.0001                                                                                                    \\
Scheduler         & StepLR                                                            & CosineAnnealingLR                                                                                         & CosineAnnealingLR                                                                                         \\
\midrule[0.5pt]
Data Augmentation & MNIST                                                             & CIFAR-10                                                                                                  & CIFAR-100                                                                                                 \\
\midrule[0.5pt]
RandomCrop        & -                                                                 & size=32, padding=4                                                                                        & size=32, padding=4                                                                                        \\
HorizontalFlip    & -                                                                 & p=0.5                                                                                                     & p=0.5                                                                                                     \\
% Normalization     & \begin{tabular}[c]{@{}l@{}}mean=0.1307,\\ std=0.3081\end{tabular} & \begin{tabular}[c]{@{}l@{}}mean={[}0.4914,0.4822,0.4465{]},\\ std={[}0.2023,0.1994,0.2010{]}\end{tabular} & \begin{tabular}[c]{@{}l@{}}mean={[}0.5071,0.4867,0.4408{]},\\ std={[}0.2675,0.2565,0.2761{]}\end{tabular} \\
\bottomrule[1pt]
\end{tabular}
}
\label{tab:optimizer classification}
\end{table}

\paragraph{Additional Benchmark.}  \citet{matias2024amortized} proposed Amortized Variational DKL (AV-DKL), which is a variant SV-DKL using amortization network to compute the inducing locations and variational parameters, thus attenuating the overcorrelation of NN extracted features. AV-DKL is included as the additional benchmark for classification tasks in \Cref{tab:img avdkl}. The training recipe is the same with SV-DKL. 


\begin{table*}[ht]
\caption{\small{Accuracy, NLL, ECE for AV-DKL, SV-DKL, DAK-MC on CIFAR-10/100 averaged over 3 runs. CIFAR-10 uses ResNet-18 with 64 features extracted; CIFAR-100 uses ResNet-34 with 512 features. The best results are highlighted in \textbf{bold}; the second best results are highlighted by \underline{underline}.}}
\centering
\vspace{-0.1cm}
\resizebox{\linewidth}{!}{%
\begin{tabular}{rccclccc}
\toprule[1pt]
\multicolumn{1}{l}{} & \multicolumn{3}{c}{Batch size: 128}  &  & \multicolumn{3}{c}{Batch size: 1024} \\ \cline{2-4} \cline{6-8} \vspace{-8pt} \\
\multicolumn{1}{l}{} & AV-DKL & SV-DKL & \cellcolor{Gray} DAK-MC &   & AV-DKL  & SV-DKL & \cellcolor{Gray} DAK-MC \\ 
\midrule[1pt]
CIFAR-10 - Acc. (\%) $\uparrow$    & \underline{94.23 $\pm$ 0.65}  & 93.44 $\pm$ 0.28    &  \cellcolor{Gray} \textbf{94.81 $\pm$ 0.13}   &     &  \textbf{93.32} $\pm$ \textbf{0.13}        & 90.22 $\pm$ 1.42       & \cellcolor{Gray} \underline{93.02 $\pm$ 0.18}        \\
NLL $\downarrow$     & 0.352 $\pm$ 0.084    & \underline{0.312 $\pm$ 0.033}       &  \cellcolor{Gray} \textbf{0.256} $\pm$ \textbf{0.014}     &      & \underline{0.439 $\pm$ 0.022}         & 0.485 $\pm$ 0.061       & \cellcolor{Gray} \textbf{0.345 $\pm$ 0.001}    \\
ECE $\downarrow$      & 0.048 $\pm$ 0.006    & \underline{0.046 $\pm$ 0.003}       &  \cellcolor{Gray} \textbf{0.039 $\pm$ 0.002}          &     & \underline{0.054 $\pm$ 0.001}       & 0.060 $\pm$ 0.004       & \cellcolor{Gray} \textbf{0.052 $\pm$ 0.001}           \\
\midrule[1pt]
CIFAR-100 -  Acc. (\%) $\uparrow$    & \textbf{77.47 $\pm$ 0.19}  & 74.52 $\pm$ 0.13       & \cellcolor{Gray}  \underline{76.75 $\pm$ 0.18}     &     &  \textbf{77.07 $\pm$ 0.10}        & 66.54 $\pm$ 0.74       & \cellcolor{Gray} \underline{70.38 $\pm$ 1.25}        \\
NLL $\downarrow$     & 1.787 $\pm$ 0.011    & \underline{1.041 $\pm$ 0.007}       & \cellcolor{Gray}  \textbf{1.001 $\pm$ 0.027}     &      & 2.326 $\pm$ 0.030    & \underline{1.738 $\pm$  0.058}      & \cellcolor{Gray} \textbf{1.203 $\pm$ 0.040}        \\
ECE $\downarrow$      & 0.166 $\pm$ 0.002    & \underline{0.049 $\pm$ 0.002}       & \cellcolor{Gray}  \textbf{0.041 $\pm$ 0.004}        &     & 0.175 $\pm$ 0.001         & \underline{0.148 $\pm$ 0.007}       &\cellcolor{Gray}  \textbf{0.056 $\pm$ 0.006}           \\
\bottomrule[1pt]
\end{tabular}
}
\vspace{-0.2cm}
\label{tab:img avdkl}
\end{table*}

\paragraph{Metrics} 
We evaluate model performance using four common metrics: Top-1 accuracy, ELBO, Negative Log Likelihood (NLL), and Expected Calibration Error (ECE). 

ECE is a metric used to quantify the degree of ``calibration'' of a probabilistic model in UQ, specifically for classification problems. It is defined as the weighted average of the absolute difference between the model's predicted probability (confidence) and the actual outcome (accuracy) over several bins of predicted probability. Mathematically, ECE is given by:
\begin{align}
    \text{ECE} =\sum_{m=1}^{M} \frac{\left| B_{m} \right|}{n} \left| \text{acc} (B_{m})-\text{conf} (B_{m}) \right|,
\end{align}
where $M$ is the number of bins into which the confidence values are partitioned, $B_m$ is the set of indices of samples whose predicted confidence falls into the $m$-th bin, $n$ is the total number of samples.

\paragraph{Computing Infrastructure}
The experiments for classification were run on a Linux machine with NVIDIA RTX4080 GPU, and 32GB of RAM.




\subsection{Additional Tables and Figures}
\label{sec:additional exp results}

\paragraph{Choices of learning rates.}
We evaluate the choices of learning rates on 1D regression examples. DKL requires a separate tuning of the learning rate of the GP covariance parameters, which differs from the learning rate of the NN feature extractor. In \Cref{fig:dkl lr}, we choose the learning rate of the NN feature extractor as $0.01$, while the learning rate of the GP covariance is set to different values. (a)-(c) show that different learning rates of covariance in DKL result in different predictive posterior. In particular, although the training losses for DKL in both (a) and (b) are minimal, the regressions do not fit well. On the other hand, DAK does not need a distinct recipe for tuning GP covariances because of the BNN interpretation. Furthermore, the poor posterior is indicated by the higher training loss, as illustrated in (d)-(f).

\begin{figure}[ht]
\centering
\subfloat[$\begin{gathered}\text{DKL: last-layer lr} =0.01.\\ \text{Training loss:} -0.21.\end{gathered}$]{\includegraphics[width=.3\textwidth]{toy_dkl_lr_01.pdf}}
\subfloat[$\begin{gathered}\text{DKL: last-layer lr} =0.001.\\ \text{Training loss: } -0.07.\end{gathered}$]{\includegraphics[width=.3\textwidth]{toy_dkl_lr_001.pdf}}
\subfloat[$\begin{gathered}\text{DKL: last-layer lr} =0.0001.\\ \text{Training loss: } 0.22.\end{gathered}$]{\includegraphics[width=.3\textwidth]{toy_dkl_lr_0001.pdf}}

\subfloat[$\begin{gathered}\text{DAK: last-layer lr} =0.1.\\ \text{Training loss: } 0.10.\end{gathered}$]{\includegraphics[width=.3\textwidth]{toy_dak_lr_1.pdf}}
\subfloat[$\begin{gathered}\text{DAK: last-layer lr} =0.01.\\ \text{Training loss: } 0.10.\end{gathered}$]{\includegraphics[width=.3\textwidth]{toy_dak_lr_01.pdf}}
\subfloat[$\begin{gathered}\text{DAK: last-layer lr} =0.001.\\ \text{Training loss: } 0.22.\end{gathered}$]{\includegraphics[width=.3\textwidth]{toy_dak_lr_001.pdf}}

\caption{Results on 1D regression with different last-layer learning rates. The learning rate of NN feature extractor is set as $0.01$. (a)--(f) shows the regression fits and corresponding training losses. DAK fits for the same learning rate strategy with NN feature extractor (lr=0.01), while DKL requires a separate tuning for last-layer learning rate of GPs. Additionally, a better training loss does not necessarily prevent overfitting for DKL.}
\label{fig:dkl lr}
\end{figure}


\paragraph{Learning curves.} We plot the learning curves of CIFAR-10/100 in \Cref{fig:cifar10 curves} and \ref{fig:cifar100 curves}. The learning curves of SVDKL in \Cref{fig:cifar10 curves} is more unstable, with many significant spikes, and the convergence is slower than DAK. Futhermore, SVDKL struggles to fit with full-training in CIFAR-100, and a pretrained feature extractor is used in CIFAR-100. Therefore, the learning curves of SVDKL look smoothing, but DAK fits well with full-training in CIFAR-100.


\begin{figure}[ht]
\centering
\subfloat[Test Error (\%).]{\includegraphics[width=.3\textwidth]{CIFAR_10_test_error.pdf}}
\subfloat[Test NLL.]{\includegraphics[width=.3\textwidth]{CIFAR_10_nll.pdf}}
\subfloat[ELBO.]{\includegraphics[width=.3\textwidth]{CIFAR_10_elbo.pdf}}
\caption{Test errors, test NLLs, ELBOs of NN, SVDKL, and DAK curves with batch size of 128/1024 for CIFAR-10 averaged on 3 runs. DAK outperforms SVDKL on both test error and NLL along the training epochs. Additionally, SVDKL degrades more and struggles to fit when the batch size becomes larger.}
\label{fig:cifar10 curves}
\end{figure}

\begin{figure}[ht]
\centering
\subfloat[Test Error (\%).]{\includegraphics[width=.3\textwidth]{CIFAR_100_test_error.pdf}}
\subfloat[Test NLL.]{\includegraphics[width=.3\textwidth]{CIFAR_100_nll.pdf}}
\subfloat[ELBO.]{\includegraphics[width=.3\textwidth]{CIFAR_100_elbo.pdf}}
\caption{Test errors, test NLLs, ELBOs of NN, SVDKL, and DAK curves with batch size of 128/1024 for CIFAR-100 averaged on 3 runs. DAK trained NN and last-layer additive GPs jointly, while SVDKL used the pre-trained NN and fine-tuned the last-layer GP since SVDKL struggles to fit using full-training. DAK outperforms SVDKL on both test error and NLL along the training epochs. SVDKL struggled to fit in high-dimensional multitask cases, indicating the necessity of pre-training in SVDKL. However, DAK fitted well with high dimensionality and large batch sizes.}
\label{fig:cifar100 curves}
\end{figure}







% \end{document}


\end{document}