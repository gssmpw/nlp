\subsection{Health Economics, Policy, and Reimbursement}

\paragraph{Subtopic:} 
What is the appropriate role for AI/ML powered health economic models in decision making? How do we enable and encourage these positive applications?

\paragraph{Chairs:} \textit{Ian Cromwell, Shuvom Sadhuka, and Ross Duncan}

\paragraph{Background:}
“Health Economics” is a discipline that applies economic theory to understanding the production and consumption of health and healthcare. Health Economics most prominently focuses on 1) determining value (or comparisons of value: efficiency, effectiveness) of healthcare goods and services consumed/provided, 2) how to optimize allocation of scarce resources across patients, and 3) understanding the incentives and expected behavior of various actors within the market for health. The minimum cast includes patients (consumers), providers (producers), and payers (often insurance and so distinct from consumer). It is distinct from standard economics as “health” is not appropriately modelled as a typical consumer good and the market for health structurally differs from requirements for standard economic theory.  For example, demand for life-saving care is inelastic – that is, the patient-consumer is willing to pay whatever it takes to receive care. Health is also not a good that can be transferred or purchased directly, it requires services delivered by experts with knowledge that consumers are often not able to independently assess. Finally, due in part to the large upfront costs required for intensive care episodes, the payer is not always – or even often – the same as the consumer of care. These, among other structural factors, form a special case that “Health Economics” studies – often with an eye to understanding how to optimally allocate scarce resources. Contemporary understanding of “health” is of a kind of personal capital, as outlined in Grossman \citep{grossman} How do we determine which healthcare services and technologies are best suited to deliver the most benefit for resources required \citep{cadth}?

\paragraph{Discussion:} Many participants did not have any direct health economics experience or background, and so a basic outline of health economics and its role was provided by the senior chair. Key areas of distinction where to explain the role of “quality-adjusted life years” (QALY) and “disability-adjusted life years” (DALY) often used as an outcome measure in health economic work. These are both psychometric instruments that try to capture both the length of lifetime gained by an intervention, and the quality (or capabilities) of that time \citep{Howren2013}. Methodological debates still occur over those instruments, but health economics will also use direct outcome measures and derived utility to measure benefits. Secondly, that health economics is an aid to decision-making at the policy and population level – it does not focus on specific, individual instances of care. That is, a health economist may assess whether it is worth deploying AI-assistants in hospitals, but not whether the assistant should be used in any specific instance of care.

\subsubsection{Where do health economic standards come from?}

Standards for health economic evaluation tend to be set by agencies tasked with assessing new healthcare technologies, drugs, and other interventions. The clearest example of this is the United Kindgom’s National Institute for Health and Care Excellence (NICE), an organization which determines which (typically novel) healthcare drugs or technologies should be provided by the National Health Service, and also wrote the manual on economic evaluation \citep{national2022nice}. A similar Canadian organization with a more advisory role is Canada’s Drug Agency (CDA-AMC, formerly CADTH), which has and continues to produce guidelines for health technology assessment and health economic evaluation \citep{cadth}. CDA-AMC led the adoption of open source practices in health economics, which has been adopted as the Canadian standard. The centrality of these organizations in determining standard practice is of consequence when determining how to best deploy ML/AI in health economics but are perhaps not ideally situated to lead by example where more regulatory intervention may be necessary. Certainly there is a need to standardize which models are used and why, as competing practices continue to proliferate. Vithlani et al’s 2023 systematic review of best practices in conduct and reporting may be useful for shaping future discussion \citep{vithlani2023economic}.

\subsubsection{What can AI/ML do for Health Economics?}

From a decision making perspective, private for-profit insurers in the United States have already adopted models trained to determine when to approve or deny care. Such models can be tuned to hit targeted rates and maximize billing. This does not constitute health economics as typically practiced – as this involves automation of individual care decisions, but the question of whether – from a societal perspective – the cost savings from denying “unnecessary” care outweigh the instances in which care is inappropriately denied. When such errors inevitably do occur, it is unclear who would be responsible and held accountable: the model creators? The provider? The insurers who deployed the model? 
Discussion then turned to less controversial applications of AI/ML in healthcare settings, such as using LLMs to help write authorization letters and similar documentation to expedite administrative burden. Sufficiently advanced LLM models may also be able to assist patients with health concerns – as patients are already engaging with tools like ChatGPT to learn about their condition. This may save time and money associated with concierge physician visits. 
In the literature, the International Society for Pharmacoeconomics and Outcomes Research (ISPOR) has produced the PALISADE Checklist of ML methods in Health Economics. The review suggests that ML could enhance 1) specificity of cohort selection, 2) identification of independent predictors and covariates of health outcomes, 3) predictive analytics of health outcomes, 4) causal inference, and 5) reduction of structural, parameter, and sampling uncertainty \citep{padula2022machine}. 

\subsubsection{What regulation is needed to guide AI/ML in Health Economics?}

The example of AI in health insurance prompted discussion about what sort of oversight may be needed, and what functions were appropriate for such models. Some participants raised concerns about accountability for decisions made by models, but others felt that the limitations of AI must be considered in comparison to the available alternatives. Physicians themselves are capable or error, and if it is the difference between no care and AI-moderated care then the latter would still presumably be preferable. Regulation could also stymie innovation and so some suggested that a “Consumer Reports” type transparency around AI/ML models may be preferable. 
The suggestion that AI/ML models for use in healthcare should go through the Food and Drug Association (FDA) also arose – those software is not typically held to those standards the unique role of health may urge particular caution before bringing to market. 
There was an emerging consensus that AI/Models deployed in healthcare decision making should be monitored, and that the monitoring should have regulatory enforcement. However, how to ensure scalable oversight is still an open question.
