\subsection{Health AI in Low- and Middle-income Countries}

\paragraph{Subtopic:} What are the unique challenges and opportunities of AI in low- and middle-income countries? How can we help these countries prepare and prevent epidemics and pandemics? How can AI be adapted to meet such challenges?\citep{mollura2020artificial}

\paragraph{Chairs:} \textit{Megan Coffee, Wenqian Ye, and Brighton Nuwagira}

\paragraph{Background:} 

Developing and deploying AI systems in low- and middle-income countries presents unique challenges and opportunities. These countries often face limitations in healthcare infrastructure, fragmentation in health systems and data management, and obstacles in accessing healthcare. These hurdles require innovative and context-specific AI solutions. For instance, AI has demonstrated promise in addressing infectious diseases such as COVID-19~\citep{jiang2020towards}, Mpox~\citep{cao2024mpoxvlm}, and Tuberculosis~\citep{ahmed2023topo} by improving diagnostics, clinical decision support, forecasting outbreaks, and optimizing resource management. However, using small or biased datasets in these settings increases the risk of spurious correlations~\citep{ye2024spurious}, undermining model robustness and fairness. In addition, data collection efforts in these regions often encounter significant obstacles, as standardized electronic health records are often not available. There are other infrastructural limitations, lack of standardization, and the need for effective collaboration between clinics and across geographies. International organizations like the World Health Organization (WHO), and non-profits like the International Rescue Committee (IRC), as well as Ti Kay, which has worked in Haiti, play a crucial role in facilitating these partnerships and ensuring ethical, equitable, and effective use of AI, in addition to direct partnerships with local universities and clinics. Building trust in AI systems among healthcare practitioners requires focusing on interpretability, cultural relevance, and practical utility. Engaging with local stakeholders and tailoring solutions to specific healthcare challenges are critical steps to ensure the successful integration of AI into clinical workflows in these settings.
 

% This research roundtable focuses on four topics:

% \begin{itemize}
%     \item AI for infectious diseases (COVID~\cite{jiang2020towards}, Mpox~\citep{cao2024mpoxvlm}, Tuberculosis) in low-and middle-income countries: existing methods and perspectives.
%     \item Spurious correlations in limited-resourced settings.
%     \item Data Collection and International Collaboration in low- and middle-income countries. 
%     \item Trustworthiness and Engagement for clinical practitioners, .... 
%     Efforts from recognized organizations (e.g. WHO, IRC, ...).
% \end{itemize}

\paragraph{Discussion:}
The discussion began with an exploration of the challenges and opportunities in leveraging Health AI for low- and middle-income countries, focusing on the interplay between data availability, infrastructure constraints, and the potential for innovative solutions. A key question emerged: \textit{Should Health AI in these regions prioritize a data-centric or algorithm-centric approach, particularly given the limited access to high-quality data?} 
The discussions highlighted several opportunities and strategies to overcome these challenges.

\subsubsection{Synthetic Data}
Synthetic data has been recognized as a transformative tool for mitigating data scarcity in low- and middle-income countries. These datasets, generated using advanced generative AI techniques, such as Stable Diffusion~\citep{rombach2022high}, simulate realistic and anonymized data that maintain critical statistical and clinical properties. Additionally, efforts are underway to leverage 3D generative models to create synthetic skin textures, 3D anatomical models, and digitally rendered medical images~\citep{kim2024s}. They offer a means to train models effectively where real-world data is scarce, while also addressing privacy and ethical concerns in data-sharing practices. However, participants underscored the necessity of implementing rigorous validation protocols to ensure that synthetic data aligns with the specific healthcare needs and contexts of these regions, while also accounting for the limitations in the generalizability of these models. It is important that models for low- and middle-income countries are not reliant on synthetic data as gap fillers, but instead substantive effort is made to collect representative data.

\subsubsection{Spurious Correlations, Model Robustness, and Generalization}

Participants highlighted the critical challenge posed by spurious correlations, which arise when models rely on superficial patterns in training data that fail to generalize to new or diverse distributions. This issue is particularly pronounced when AI models developed in high-resource settings are deployed in low-resource environments, where differences in demographics, healthcare practices, and disease prevalence often render such models ineffective. To overcome these limitations, the discussion emphasized the need for diverse and representative datasets that capture these variations. Additionally, employing techniques such as domain adaptation and transfer learning was identified as essential to enhance model robustness and ensure reliable performance across heterogeneous contexts.

\subsubsection{Stakeholder Engagement}
The importance of engaging all relevant stakeholders in data collection and model deployment processes was a recurring theme. Collaboration with local healthcare professionals, policymakers and community leaders ensures that AI solutions are contextually relevant, practically feasible, and aligned with local needs. Co-developing AI systems with local stakeholders foster trust and accelerates adoption in clinical workflows. These partnerships also promote long-term sustainability by integrating AI into healthcare systems in a meaningful way.

\subsubsection{Partnerships Between Developed and Low-Resource Regions}
Robust partnerships between researchers in developed countries and those in low- and middle-income regions were recognized as vital. Such collaborations facilitate knowledge transfer, financial investment, and infrastructure support, enabling the development of state-of-the-art AI models that address the specific challenges of low-resource settings. Importantly, these partnerships are mutually beneficial, driving innovation by exposing researchers to diverse healthcare contexts and fostering solutions that address the needs of both high- and low-resource regions.


\subsubsection{Data-Centric vs. Algorithm-Centric Approaches}
The discussion weighed the relative importance of data-centric versus algorithm-centric approaches. While sophisticated algorithms are crucial, there was consensus that improving the quality, representativeness, and availability of data should be the priority in low-resource settings. Without adequate and reliable training data, even the most advanced algorithms are unlikely to perform effectively.

\subsubsection{Ethics and Trustworthiness}
Building trust in AI systems was identified as a critical priority. This requires ensuring that models are interpretable, transparent, and ethically sound. Efforts by international organizations like the WHO and IRC, along with collaborations with other smaller non-profits organizations and local academics and clinicians, play an essential role in promoting ethical AI adoption. Participants also noted the importance of addressing biases and spurious correlations to enhance the fairness and reliability of AI solutions.


The discussion concluded that advancing Health AI in low- and middle-income countries demands a holistic and collaborative approach. Combining data collection, stakeholder engagement, international partnerships, and robust model development can create impactful and equitable AI solutions tailored to these regions' unique needs. By prioritizing data quality and fostering trust, the global health AI community can make meaningful strides toward improving healthcare outcomes in these settings.

% Participants discussed several solutions on the subtopics above:

% \textbf{a} 

% \textbf{b}

% \textbf{c}
