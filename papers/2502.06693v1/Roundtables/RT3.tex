\subsection{Challenges of Interdisciplinary Research}

\paragraph{Subtopic:} Navigating the complexities of interdisciplinary collaboration in health AI poses significant challenges, such as integrating diverse expertise and overcoming generational technology gaps. How can we effectively overcome these barriers to enhance data accessibility and collaborative efforts?

\paragraph{Chairs:}
\textit{Dennis Shung, Vasiliki (Vicky) Bikia, and Nicole Zhang}

\paragraph{Background:}
Combining expertise from computer science and clinical fields to effectively leverage health data is fraught with challenges \citep{kelly2019key}. Accessing comprehensive datasets, such as digital mammography data, often highlights barriers due to the need for clinician partnerships and extensive data curation. The process often involves significant time and effort to link or anonymize data, making it a lengthy endeavor to obtain usable datasets. Additionally, collaboration across disciplines often encounters communication barriers and technological reluctance. Engineers unfamiliar with medical intricacies and clinicians unaccustomed to machine learning paradigms often struggle to find a common language, while generational gaps in technology adoption can further complicate collaboration. Integrating and generalizing AI models within clinical workflows remains a significant challenge due to the diversity of data modalities and the complexity of healthcare settings. The need for model-agnostic approaches is critical to ensure that AI tools are adaptable and effective across different healthcare environments.

\paragraph{Discussion:}
Addressing these interdisciplinary challenges requires strategic partnerships and dedicated educational initiatives. The formation of robust collaborations with clinicians and the investment in continuous interdisciplinary education are essential to bridge knowledge gaps. 

Engaging in direct conversations with clinicians not only helps computer scientists understand the intricacies of medical practice but also allows clinicians to grasp the potential and limitations of AI technologies. 
Additionally, proactive measures such as attending interdisciplinary conferences provide valuable opportunities for networking and learning from leading experts in both fields \citep{patel2024crucial}. Conferences often feature workshops and symposiums that specifically focus on integrating AI into healthcare, offering practical insights and fostering collaborations. Similarly, reading interdisciplinary journals and books can enhance understanding. Exposure to each other's work environments can further enhance collaboration. Visits to clinical settings can help computer scientists see firsthand the challenges clinicians face, while visits to tech labs can show clinicians the complexities of developing and training AI models. This mutual exposure not only deepens understanding but also builds trust and respect between the disciplines, paving the way for more effective and empathetic collaborations.

Assembling interdisciplinary teams that include members proficient in both health and computer science is crucial for advancing AI integration into healthcare. These teams benefit immensely from having \textit{bilinguals} members who can translate between the technical language of computer science and the clinical language of healthcare. These bilinguals make complex concepts accessible and actionable, serving as vital bridges within the team.

Additionally, emerging systems such as virtual labs and AI agents can potentially enhance efficiency and productivity in these collaborations \citep{swanson2024virtual}. Virtual labs provide advanced simulation platforms for virtual meetings and collaborative sessions, mimicking real-world interactions and experiments. LLM agents are designed to simulate real meeting scenarios by undertaking specific roles within a team, such as note-taking, summarizing discussions, generating code, or even proposing solutions based on the dialogue. These agents not only automate routine tasks but also provide models of communication and problem-solving that human team members can learn from.

Furthermore, developing adaptive and inclusive communication strategies that cater to varying expertise levels is essential \citep{sankaran2021practical}. Coupled with the adoption of modern collaborative tools, these strategies help bridge the significant gaps between different generational and disciplinary perspectives. This approach ensures that all team members, regardless of their primary discipline or familiarity with new technologies, can fully engage and contribute to the project’s success.

Integrating AI into healthcare through interdisciplinary research presents formidable challenges but also opens significant opportunities for innovation. By fostering understanding and cooperation between fields, leveraging AI tools for enhanced collaboration, and addressing the persistent issues of data access and usability, we can pave the way for more effective and transformative healthcare innovation.



    % - computer vision research
    % - dataset of digital mammography (20k images)
    %     - but only 1/5 of dataset
    %     - common struggle for non MD
    %         - MD can access any data (status with the health system)
    %         - Need MD partner
    %         - Vicky’s PI partner with clinician (can help you navigate the data)
    %         - Other 80% may have issues or incorrect context
    %     - Stanford also share EHR through a platform retrospectively with IRB
    %         - Internal process (new data collection or recruit new patient)
    %     - Even with partners, the assumption is dataset but need a long time to access data
    %         - linked or anonymize the dataset
    %     - Someone to navigate (partnership)

    
    %         - Intersection b/w health and CS (bring people who understand both realm)
    %         - and the task to be addressed
    %     - Key is having people at intersections
    % - Partner: sometimes more ML (public dataset often)
    %     - has radiologist on the team
    %     - clinician’s idea of dataset is different
    %     - after curation can only use a small part
    %         - what exactly you care about
    %         - lucky to have someone on team who understood ML
    %         - translator
    %     - Difficult to not have people with DL ML expertise (only come with experience and being on project multi-disciplinary)
    %         - Very complicated (
    %         - need to address to patient/clinicians
    %         - will improve with time (like learning a foreign language)
    %     - Other interdisciplinary are user researchers
    %         - very useful to collaborate with
    %     - Some also work with engineers
    %         - need to communicate one more
    %         - engineers are not familiar with medical problems
    %         - Not a common communication tool (amongst each other)
    %             - ex. keywords look up
    %         - another problem: some people are older not eager to new technologies
    %     - Generational gap in medicine but also in ML
    %         - LM are new - not accepted as well
    %         - Used to have few courses in university (outdated methods described in class)
    % - How to collaborate with people who are not as enthusiastic
    %     - well established group with shared grant
    %     - They don’t feel the urge to learn deep learning and understand them
    %     - some people kept giving input because they don’t understand ml objectives (ex. supervised or self-supervised)
    %     - Need abstraction of the problem & can be useful if taking certain specifics of data - need someone to verify the idea
    % - When trying to solve clinical task - start from clinician (important part, for iterations)
    % - Some of the solutions
    %     - Spend time with them & show up to meetings
    %     - Being enthusiastic about their world and views.
    %     - Google, with LLM it is nicer (chatgpt)
    %         - easier to process paper & follow links
    %     - Domain specific books (ex. survival guide)
    %     - Face to face meeting
    %     - exposure
    % - James Zou virtual labs (LLM agent)
    %     - some super structure (subtask to different agent)
    %         - agent generate code and come back
    %     - With this tool may get how people discuss and make meetings more high yield
    %     - Need to decrease the friction of the environment
    %     - All prompt from GPT4 (showed proof of concepts)
    %         - design nanobody validated in in vitro assay
    %     - Context to agents:
    %         - Guide very clearly to the problem
    %         - Had a critic agent in every meeting (problem with LLM is very agreeable)
    %     - DS Pie?
    %         - prompt engineering
    %         - Guide LLM better (sensitivity to prompt)
    %         - to better improve outcome
    % - Trained data are often outdated
    %     - How do we handle ood data
    %     - human physiology don’t change that much
    %     - think about manifold - true in biology (not 2 subspace at the same time)
    %         - There are physiological ranges (data still valuable)
    %         - if causal (new treatment) - ex. ozempic
    %             - may not be able to use data from before
    %             - **Natural history of disease**
    %                 - allow understand as medical professional - is one of them freak accident, another person with same symptom of end stage disease
    %                 - or third patient unusual (age etc to have this) - underlying process differ
    %                 - unusual need urgent attention
    %                 - acutely - will recover quickly without the insult
    %                 - the data can be used for that - integrate new data modalities
    %     - Summary:
    %         - Data is a challenge (outdated, complete, rich)
    %         - Communication
    %             - systems to improve our environment for collaboration
    %         - Can use data modalities