\section{Related Works}
\subsection{Opinion dynamics model}

Opinion dynamics investigates the formation process of agent opinions in social systems over time by setting up fusion rules. The format of opinions, fusion rules, and underlying structures are pivotal components of opinion dynamics ____. This paper primarily concentrates on dynamic models grounded in continuous opinion forms, which offer a closer alignment with the real-world social systems____.

Different opinion dynamics models assume different fusion rules and study the phenomena of opinion formation based on these rules ____. Based on the assumption of opinion assimilation, the DeGroot model ____ sets the fusion rule as a simple averaging form. Friedkin-Johnsen (FJ) model ____ introduces agent stubbornness to model situations where agent persist in their opinions in real-world scenarios. The Deffuant-Weisbuch (DW) ____ model and Hegselmann-Krause (HK) ____ model consider the phenomenon of bias assimilation ____, where agents tend to accept opinions they are inclined to believe, introducing confidence thresholds where agents can only interact with others whose opinions are within a certain range. Subsequent research delves into more intricate fusion rules, including randomness in opinion fusion ____, deceptive interactions in social networks ____, group influence ____, and so forth.

In recent years, many studies have increasingly focused on the influence of underlying topology on opinion dynamics ____. Unlike face-to-face offline social systems, agents in online social systems often only interact with a subset of agents. Through simple transformations, the above theoretical models can be effectively transferred to graph structures. These models utilize synthetic graphs and real-world graph data to study the impact of topologies on opinion dynamics.

These theoretical models have played a significant role in analyzing social systems. However, most opinion dynamics models can only capture certain aspects of opinion fusion in real-world scenarios and fail to model the complexity of opinion evolution. Integrating insights from opinion dynamics into data-driven models can effectively combine perspectives from both theory and empirical data.

\subsection{Data-driven methods for modeling opinion dynamics}

Benefiting from the data-driven paradigm, many researchers have studied learning the formation patterns of opinions from data. Some machine learning methods have recognized the importance of graph topology. The work ____ uses a linear model to capture transient changes in opinions to adapt to complex opinion scenarios. The work ____ extends the k-means method to social graphs, investigating the formation of opinion equilibrium from a community clustering perspective. AsLM ____ proposed a linear influence model that explores how to estimate the influence strength of links in social graphs by observing the opinions evolving over time at the nodes. Additionally, this work considers the limitations of the interaction range of nodes, indirectly introducing the underlying topology. SLANT ____ describes users' latent opinions as continuous-time stochastic processes, where each expression of an opinion by a user is a noisy estimate of their current latent opinion. SLANT+ ____ is a point-process-based framework for capturing the nonlinear dynamics of opinion fusion. It processes users' opinions and message timings as a temporal point process, influenced by the opinions and message timings of their neighbors. Some methods are inspired by the fusion rules in opinion dynamics. SINN ____ incorporates the idea of ____ and uses opinion dynamics to assist model training.

These approaches partially integrate knowledge from opinion dynamics models but fail to fully incorporate this knowledge into the internal mechanisms of the model. Additionally, these methods do not explicitly incorporate graph topology into the opinion fusion process, which is inconsistent with the interaction patterns observed in social media. A unified opinion dynamics model is proposed to integrate interaction rules from various classical opinion dynamics models. Furthermore, this work delves into learning opinion dynamics on graphs using graph neural networks and applies them to real-world data, further bridging the gap between theoretical models of opinion dynamics and real-world applications.