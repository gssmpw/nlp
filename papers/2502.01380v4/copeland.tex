
\subsection{Normalized Bias and the Distortion of Copeland's Rule}
\label{sec:copeland}
We now define the social choice rule to aggregate the outcome of deliberations. Note that each deliberation is a probabilistic mapping of a pair of alternatives to a winning alternative. This mapping depends on the underlying metric space as well as the $k$ randomly sampled voters. %As we remark later, for a finite, yet large number of voters, the randomness in choosing $k$ voters can be replaced by considering all multisets of $k$ voters.

Given a deliberation model $\A$ and metric space $\sigma = (d, \vec{\rho})$, let $p_k(W,X)$ denote the probability that the outcome of deliberation among a set $S$ of $k$ randomly chosen participants between alternatives $W$ and $X$, outputs $W$ instead of $X$. This probability is over both the randomness in the choice of $S$ and the randomness in the outcome of the deliberation given $S$ (as in the random choice model). To keep our analysis simple, we assume the probability can be exactly estimated for any $(W,X)$; in other words, we assume infinitely many groups of size $k$ can be drawn from the population, and small group deliberations  run  between $W$ and $X$.  In \cref{thm:sample2,thm:sample1}, we will remove this assumption and analyze the number of groups of size $k$ that need to be sampled to approximately achieve the same distortion.

\paragraph{Copeland Rule and Distortion.} The Copeland rule dates back to Ramon Llull in the $13^{th}$ century~\cite{voting-book}. To define this rule, we say that $W \succ X$ if $p_k(W,X) \ge 1/2$.  %We now create a tournament graph on the set of outcomes of the deliberations, where we place an edge from an alternative $W$ to an alternative $X$ if $W \succ X$. 
For every pair $(X,Y)$, the Copeland rule gives $1$ point to $X$ (resp. $Y$) if $p_k(X,Y) > 1/2$ (resp. $p_k(Y,X) > 1/2$), and half a point to each alternative if $p_k(X,Y) = 1/2$. It then chooses the alternative $W$ with most points.  Such an alternative $W$ belongs to the {\em uncovered set}~\cite{landau}: For any other alternative $X$, either $W \succ X$, or there is an alternative $Y$ such that $W \succ Y$ and $Y \succ X$. We assume an arbitrary alternative in the uncovered set is output. Note that the Copeland rule operates over the ordinal outcomes of the deliberations, without knowing the metric space $\sigma$.

%\paragraph{Distortion of Copeland.}
To define distortion, %we fix an underlying metric space $\sigma$ and a deliberation model $\A$. This yields a tournament graph that we aggregate using the Copeland rule.
let $\alg(\A,\sigma)$ denote the outcome of applying Copeland to the tournament graph on deliberations using model $\A$. As before, let $c^*(\sigma) = \mbox{argmin}_{c \in C} SC(c,\sigma)$ denote the social optimum. Then the distortion of the deliberation rule $\A$ under the Copeland rule is:

$$\mbox{Distortion}(\A) = \max_{\sigma}  \frac{SC(\alg(\A,\sigma), \sigma)}{SC(c^*(\sigma),\sigma)}.$$

If the outcome $\alg(\A,\sigma)$ is probabilistic, we replace the numerator in the above expression by its expectation over the randomness in $\A$.

\paragraph{The Averaging Model is Deterministic.} As an aside, suppose there is a finite number $n$ of voters, where $n$ is suitably large. Then, in the Averaging model, the randomness in the choice of set $S$ can be replaced by considering {\em all} subsets $S$ of $k$ voters, and computing $r_k(W,X)$ as the fraction of these sets where deliberation between alternatives $W$ and $X$ leads to $W$ being chosen.  It is easy to check that as $n \rightarrow \infty$, the value $r_k(W,X)$ converges to $p_k(W,X)$. Note that the Averaging model itself is a deterministic process given $S$; further, the Copeland rule applied to the $\{q_k(W,X)\}$ is deterministic. We therefore compare the distortion of the Averaging model with that of {\em deterministic} social choice rules without deliberation. In the subsequent discussion, we will go back to assuming there is a continuum of voters, and $S$ is a randomly chosen set of $k$ rankings.

\paragraph{Mathematical Program for Distortion.} We now show a simple mathematical program whose optimal solution yields an upper bound on distortion of the Copeland rule.

Fix a metric $(d ,\vec{\rho})$, and consider two alternatives $W$ and $X$. Let $\phi_i = \B_i(W,X)$ be the normalized bias of voters at location $i$, and let $D$ denote the distribution of $\phi$, so that $\phi = \phi_i$ with probability $\rho_i$. Since we fix the metric space, we omit it from the notation $SC(W)$, etc. We have the following lemma, which formulates the analysis of the Copeland rule in~\cite{AnshelevichBEPS18} as the solution to a mathematical program.

\begin{lemma} Let $\gamma = \E[D]$. Then,
$$ \frac{SC(W)}{SC(X)} \le \frac{1+\gamma}{1-\gamma}.$$
\end{lemma}
\begin{proof}
Assume $d(W,X) = 1$. We have 
$$SC(W) - SC(X) = \sum_i \rho_i \left(d(i,W) - d(i,X) \right) = \sum_i \rho_i \phi_i = \E[D] = \gamma. $$
Next, by triangle inequality, $d(i,X) \ge d(X,W) - d(i,W)$, so that
$$ SC(X) = \sum_i \rho_i d(i,X) \ge \frac{1}{2} \sum_i \rho_i \left(d(X,W) - (d(i,W) - d(i,X) \right) = \frac{1}{2} \left(1 - \gamma \right). $$
Manipulating the above two inequalities completes the proof.
\end{proof}

We now find the worst case of $\gamma$ over all metrics $\sigma = (d, \vec{\rho})$. For this, let $q(W,X)$ denote the solution to the following optimization problem: Find a distribution $D = \{\rho_i\}$ over $\{\B_i(W,X)\}$ which optimizes:
\begin{equation} 
\label{eq:opt}
\max_D \E[D] \qquad \mbox{s.t.} \qquad W \succ X,
\end{equation}
Since $\B_i(W,X) \in [-1,1]$, the above can be reformulated as optimizing over all distributions $D$ supported on $[-1,1]$.  We next show that upper-bounding this optimum  suffices to upper-bound the distortion. 

\begin{theorem}
\label{thm:distort1}
For deliberation model $\A$, if $q(W,X) \le \theta$ for all pairs $W,X$ of alternatives, then the distortion of the Copeland rule applied on the outcomes of $\A$ is at most $ \left( \frac{1 + \theta}{1-\theta}\right)^2$.
\end{theorem}
\begin{proof}
    Fix some metric $(d,\vec{\rho})$. Suppose the Copeland rule finds outcome $W$, and suppose $X$ is the social optimum. In the case where $W \succ X$, by the previous lemma, the distortion is 
    $$\frac{SC(W)}{SC(X)} \le \frac{1 + \theta}{1-\theta}.$$ 
    Otherwise, there exists $Y$ such that $W \succ Y$ and $Y \succ X$. Then, the distortion is 
    $$ \frac{SC(W)}{SC(X)} = \frac{SC(W)}{SC(Y)} \cdot \frac{SC(Y)}{SC(X)} \le  \left( \frac{1 + \theta}{1-\theta}\right)^2. $$
    Since these bounds hold for all metrics $(d,\vec{\rho})$, the proof is complete.
\end{proof}

Given the above theorem, for any deliberation model $\A$,  our goal is to upper bound the optimum $q(W,X)$ in \cref{eq:opt} for any pair of alternatives. This will imply an upper bound on distortion. Since this upper bound is independent of the alternatives themselves (as it simply optimizes over all distributions supported on $[-1,1]$), we only need to show it for an arbitrary pair of outcomes. We will do this in the next sections for the averaging and the random choice deliberation models.

Before proceeding further, we note that solving \cref{eq:opt} also provided a lower bound for distortion.

\begin{theorem}
\label{thm:lb_main}
For $k \ge 2$, if \cref{eq:opt} has value $\theta_k$, then the distortion of any deterministic social choice rule is at least $\min\left(3,\frac{1+\theta_k}{1-\theta_k}\right)$, and that of any randomized social choice rule is at least $\min\left(2,\frac{1}{1-\theta_k}\right)$.
\end{theorem}
\begin{proof}
Consider an instance with two alternatives $W$ and $X$ that are distance $1$ apart on a line. Note that $W \succ X$ so that $p_k(W,X) = \eta \ge 1/2$. 

First consider deterministic rules. Suppose such a rule outputs $X$. Then we place the $\eta$ mass of voters preferring $W$ to $X$ at  $W$ and the $1-\eta$ mass of voters preferring $X$ to $W$ at distance $1/2$ from both alternatives.  Then $\frac{SC(X)}{SC(W)} = \frac{1+\eta}{1-\eta} \ge 3$. We therefore assume the rule outputs $W$. In this case, consider the distribution $D$ that yields the optimum solution to \cref{eq:opt}. Suppose $\Pr[D = a] = p_a$ for $a \in [-1,1]$. We place voters of mass $p_a$ at distance $\frac{1+a}{2}$ from $W$ and $\frac{1-a}{2}$ from $X$. Note that $SC(X) = \frac{1 - \E[D]}{2}$, and $SC(W) = \frac{1 + \E[D]}{2}$, which implies the distortion is at least  $\frac{1+\theta_k}{1-\theta_k}$.

Similarly, for randomized rules, let $\alpha$ denote the probability the rule outputs $W$. If $\alpha \le 1/2$, then consider the instance that places $\eta$ mass of voters preferring $W$ to $X$ at  $W$ and the $1-\eta$ mass of voters preferring $X$ to $W$ at distance $1/2$ from both alternatives. The distortion is  $\frac{\alpha \cdot SC(W) + (1-\alpha) \cdot SC(X)}{SC(W)} \ge \frac{1}{1-\eta} \ge 2$. We therefore assume $\alpha \ge 1/2$. Again consider the worst-case distribution $D$ and the instance that places voters of mass $p_a$ at distance $\frac{1+a}{2}$ from $W$ and $\frac{1-a}{2}$ from $X$. The social optimum is $X$ and it is easy to check that the  expected distortion is minimized when $\alpha = 1/2$, and is at least  $\frac{1}{2} + \frac{1}{2} \cdot \frac{1+\theta_k}{1-\theta_k} = \frac{1}{1-\theta_k}.$
\end{proof}




