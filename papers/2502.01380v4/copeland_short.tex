
\subsection{Normalized Bias and the Distortion of Copeland's Rule}
\label{sec:copeland}
We now define the social choice rule to aggregate the outcome of deliberations. Note that each deliberation is a probabilistic mapping of a pair of alternatives to a winning alternative. This mapping depends on the underlying metric space as well as the $k$ randomly sampled voters.

Given a deliberation model $\A$ and metric space $\sigma = (d, \vec{\rho})$, let $p_k(W,X)$ denote the probability that the outcome of deliberation among a set $S$ of $k$ randomly chosen participants between alternatives $W$ and $X$, outputs $W$ instead of $X$. This probability is over both the randomness in the choice of $S$ and the randomness in the outcome of the deliberation given $S$ (as in the random choice model). To keep our analysis simple, we assume the probability can be exactly estimated for any $(W,X)$; in other words, we assume infinitely many groups of size $k$ can be drawn from the population, and small group deliberations  run  between $W$ and $X$.  In \cref{thm:sample2,thm:sample1}, we will remove this assumption and analyze the number of groups of size $k$ that need to be sampled to approximately achieve the same distortion.

\paragraph{Copeland Rule and Distortion.} The Copeland rule dates back to Ramon Llull in the $13^{th}$ century~\cite{voting-book}. To define this rule, we say that $W \succ X$ if $p_k(W,X) \ge 1/2$.  We now create a tournament graph on the set of outcomes of the deliberations, where we place an edge from an alternative $W$ to an alternative $X$ if $W \succ X$. For every pair $(X,Y)$, the Copeland rule gives $1$ point to $X$ (resp. $Y$) if $p_k(X,Y) > 1/2$ (resp. $p_k(Y,X) > 1/2$), and half a point to each alternative if $p_k(X,Y) = 1/2$. It then chooses the alternative $W$ with most points.  Such an alternative $W$ belongs to the {\em uncovered set}~\cite{landau}: For any other alternative $X$, either $W \succ X$, or there is an alternative $Y$ such that $W \succ Y$ and $Y \succ X$. We assume an arbitrary alternative in the uncovered set is output. Note that the Copeland rule operates over the ordinal outcomes of the deliberations, without knowing the metric space $\sigma$.

%\paragraph{Distortion of Copeland.}
To define distortion, %we fix an underlying metric space $\sigma$ and a deliberation model $\A$. This yields a tournament graph that we aggregate using the Copeland rule.
let $c(\A,\sigma)$ denote the outcome of applying Copeland to the tournament graph on deliberations using model $\A$. As before, let $c^*(\sigma) = \mbox{argmin}_{c \in C} SC(c,\sigma)$ denote the social optimum. Then the distortion of the deliberation rule $\A$ under the Copeland rule is:

$$\mbox{Distortion}(\A) = \max_{\sigma}  \frac{SC(c(\A,\sigma), \sigma)}{SC(c^*(\sigma),\sigma)}.$$

If the outcome $c(\A,\sigma)$ is probabilistic, we replace the numerator in the above expression by its expectation over the randomness in $\A$.

\paragraph{Mathematical Program for Distortion.} We now show a simple mathematical program whose optimal solution yields an upper bound on distortion of the Copeland rule.

Fix a metric $(d ,\vec{\rho})$, and consider two alternatives $W$ and $X$. Let $\phi_i = \B_i(W,X)$ be the normalized bias of voters at location $i$, and let $D$ denote the distribution of $\phi$, so that $\phi = \phi_i$ with probability $\rho_i$. Since we fix the metric space, we omit it from the notation $SC(W)$, etc. We have the following lemma, which formulates the analysis of the Copeland rule in~\cite{AnshelevichBEPS18} as the solution to a mathematical program.

\begin{lemma} [Proved in \cref{app:omitted}]
\label{lem1}
Let $\gamma = \E[D]$. Then,
$$ \frac{SC(W)}{SC(X)} \le \frac{1+\gamma}{1-\gamma}.$$
\end{lemma}


We now find the worst case of $\gamma$ over all metrics $\sigma = (d, \vec{\rho})$. For this, let $q(W,X)$ denote the solution to the following optimization problem: Find a distribution $D = \{\rho_i\}$ over $\{\B_i(W,X)\}$ which optimizes:
\begin{equation} 
\label{eq:opt}
\max_D \E[D] \qquad \mbox{s.t.} \qquad W \succ X,
\end{equation}
Since $\B_i(W,X) \in [-1,1]$, the above can be reformulated as optimizing over all distributions $D$ supported on $[-1,1]$.  We next show that upper bounding this optimum  suffices to upper bound distortion. 

\begin{theorem} [Proved in \cref{app:omitted}]
\label{thm:distort1}
For deliberation model $\A$, if $q(W,X), q(X,W) \le \theta$ for all pairs $W,X$ of alternatives, then the distortion of the Copeland rule applied on the outcomes of $\A$ is at most $ \left( \frac{1 + \theta}{1-\theta}\right)^2$.
\end{theorem}


Given the above theorem, for any deliberation model $\A$,  our goal is to upper bound the optimum $q(W,X)$ in \cref{eq:opt} for any pair of alternatives. This will imply an upper bound on distortion. Since this upper bound is independent of the alternatives themselves (as it simply optimizes over all distributions supported on $[-1,1]$), we only need to show it for an arbitrary pair of outcomes. We will do this in the next sections for the averaging and the random choice deliberation models.

Before proceeding further, we note that solving \cref{eq:opt} also provided a lower bound for distortion.

\begin{theorem} [Proved in \cref{app:omitted}]
\label{thm:lb_main}
For $k \ge 2$, if \cref{eq:opt} has value $\theta_k$, then the distortion of any anonymous deterministic social choice rule is at least $\frac{1+\theta_k}{1-\theta_k}$, and that of any anonymous randomized social choice rule is at least $\frac{1}{1-\theta_k}$.
\end{theorem}

