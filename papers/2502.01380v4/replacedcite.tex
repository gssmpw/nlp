\section{Related Work}
Our model relates to three strands of recent research and we compare with them below.

\paragraph{Metric Distortion of Voting.} Building on the work of____, the work of____ initiated the study of metric distortion. Given a social choice rule with ordinal voter preferences that are consistent with an underlying metric over voters and candidates, the distortion is the worst-case (over all metrics) ratio of the total distance to the chosen alternative to that for the $1$-median alternative had the metric been known. They showed that the Copeland rule has distortion $5$, and no deterministic voting rule has distortion better than $3$.  The upper bound was subsequently improved via weighted tournament rules to $2 + \sqrt{5}$ by____, and later to $3$ by____.  For randomized voting rules, the work of____ showed a lower bound of $2$. This lower bound was subsequently improved to $2.11$ by____. An upper bound of $3$ follows from random dictatorship____, and this was improved via novel voting rules to $2.74$ in____. We refer the reader to____ for a survey.

Our work shows that simple models of deliberation in very small groups ($k=3,4$), where the output of the group is ordinal, yet based on their collective cardinal preferences, suffices to allow the Copeland rule to break the metric distortion lower bounds, and the distortion approaches $1$ as the group size increases. In the absence of deliberation, ``optimal'' rules such as Plurality Veto____ have better distortion than Copeland. However, these rules always output the favorite candidate of some voter. If such a rule is used to aggregate the outcome of deliberations, the distortion will be $3$ for any constant group size (see \cref{eg1}). This showcases a nice property of the Copeland rule, along with its relative ease of analysis.

\paragraph{Voting with Cardinal Information.} Our work shows that if groups of voters output ordinal preferences over alternatives via aggregating their cardinal metric information using deliberation, the resulting ordinal information can be aggregated in a way that achieves distortion bounds arbitrarily close to $1$. In our models, the outcome of deliberation favors voters with large bias towards one outcome, which is also where the median in the metric space between these voters will likely lie. 

The work of____ is similar in spirit in that it tags voters' preferences between pairs of alternatives with ``strong'' and ``weak'', counting each strong preference by a fixed larger amount than a weak preference.  However, in their model, the threshold between strong and weak preference is an arbitrary threshold on the ratio between distances to the two alternatives, and the score bump is also an arbitrary number. Even with an optimal setting of these thresholds, they are unable to break the metric distortion lower bound of $3$. This is because the voting rule only gets the original ordinal information had all voters had strong or weak preferences.  Our model has two advantages. First, it does not have arbitrary thresholds. Second, in our model, voters within a group use cardinal information optimally via deliberation  to find an ordinal ranking between two outcomes (say using $1$-median). This is not only more natural, but also squeezes more information about the metric space from the voters, so that the distortion asymptotically approaches $1$.  We also refer the reader to____ for a related distributed model for small groups. . %able to achieve distortion less than $3$ even with very small $k$. because voters can collectively decide ordinal outcomes based on their underlying cardinal information, as opposed to revealing that information individually. Indeed, it is unclear how voters would reveal such information in practice.

\paragraph{Sampling and Sortition.} Several works____ have considered distortion of voting rules when voters are randomly sampled from the population. The random dictatorship mechanism samples one voter, and achieves distortion $3$. The work of____ shows that if deliberation between two voters is modeled as bargaining with a disagreement alternative, then there is a protocol that achieves low distortion on special types of metrics called {\em median spaces}. The work of____ extends this model to three sampled voters, and bounds the second moment of distortion (and not just the expectation), which for random dictatorship, is unbounded. See also____ for a different protocol and analysis for three voters. The work of____ shows that sampling $k$ voters leads to bounded the $k^{th}$ moment of the distortion. However, with the exception of____, these works assume voters reason with their ordinal preferences, and none of these works improve on the distortion bound of $3$. In contrast, we model deliberation in a general metric space as a natural process of reasoning with cardinal information, which leads better distortion bounds with equally small group sizes.

Motivated by citizen assemblies and sortition____, the work of____ considers a model where a large random sample of voters is chosen to deliberate and find a socially optimal outcome, or $1$-median. %See also____ for a related model where voters incorporate social utility into their rankings. These two works are closely related to our results, with notable differences worth highlighting. 
Their work makes a great case for sortition: for example, most citizen assemblies have the number of participants in the several hundreds____, and the bounds of $O((\log m)/\eps^2)$ on group size provided by____ are quite salient in this setting. However, as discussed before, research in psychology and organizational science shows that such large group sizes lead to sub-optimal deliberation. This is a substantial difficulty in applying the results of____ to design an actual deliberation process. As \cref{eg1} shows, the distortion remains lower bounded by $3$ unless the group size becomes logarithmic in the number of outcomes, so this difficulty is not just a matter of improving their analysis. Our distortion bound of $1 + O(1/k)$ becomes more salient in this setting since it allows us to use groups of size $O(1/\eps)$ to get distortion bounds of $1+\eps$. Given the impossibility result mentioned above, the process has to change in some way to obtain our bound; hence, breaking down the deliberation process so that each participant takes part in multiple small deliberations as opposed to one large one is key to our result. Thus, our result is not merely a technical improvement over____, but is qualitatively different. In fact, many existing online platforms for synchronous deliberation (e.g.____) divide the participants into much smaller groups of 3 to 15, a range where our results are salient. Additionally, our results can be composed with those of____ to simultaneously provide a bound of $O((\log m)/\eps^2)$ on the total number of participants and of $O(1/\eps)$ on the size of each group in a single deliberation. 

Similarly, the work of____ posits a deliberation model where voters incorporate social utility into their rankings. We note that this work also does not posit an interaction mechanism in a small group. Additionally, it focuses on showing that the distortion under social welfare approaches a constant or linear for natural rules ({\em i.e.}, the social welfare approach starts to achieve bounds similar to those known for metric distortion), whereas we show improved metric distortion bounds. Thus, our results are both different from and complementary to this work as well as that of____.

%In contrast, our motivation for multiple small-group  deliberations comes from both the existence of such online platforms, and research in psychology on biases introduced by large group dynamics. 
Our main contribution is thus an analytical justification for such a model of interaction with very small groups. We note that the overall number of sampled voters needed for distortion close to $1$ in the averaging model remains logarithmic in the number of alternatives, hence being comparable to the bound for a single sortition. Our approach therefore presents an alternate view of how sortition can be implemented via multiple small groups, with smoothly improving distortion bounds in the group size.