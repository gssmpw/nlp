\section{Model and Preliminaries}
\label{sec:prelim}
We now formally present the model and framework for small group interaction. There is a set $C$ of $m$ candidates (or alternatives or outcomes), and $m!$ possible rankings of these alternatives. We assume a continuum of voters, each of whose preference follows one of the rankings.\footnote{We make the continuum assumption on voters for analytic convenience, noting that the relevant lower bounds for the finite voter metric distortion problem hold in the continuum voter setting as well. (See \cref{thm:lb_main}.)} The preferences of a fraction $\rho_{q}$ of the voters follow ranking $q$. %From now on, by ``voter'', we will mean a ranking $i$, and the mass $\rho_{i}$ of voters whose preferences match that ranking. 

In the metric distortion framework, we assume the rankings with $\rho_q > 0$ and the candidates are embedded in a latent metric space with distance function $d(\cdot, \cdot)$. We assume the metric space has discrete (and finite number of) locations, corresponding to the candidates and the rankings with $\rho_q > 0$. The voters whose preferences follow ranking $q$ are placed at a set of locations $S_q$, where the voter mass at location $i \in S_q$ is $\rho_{iq}$, with $\sum_{i \in S_q} \rho_{iq} = \rho_q$.  We use ``metric space'' to mean both the distance function, as well as the voter mass at each location. We denote the combination $(d, \vec{\rho})$ as $\sigma$. Let $Q$ denote the set of all locations of voters. Given a location $i \in Q$, the ranking corresponding to $i$ is denoted $\sigma_i$, and the voter mass there is denoted $\rho_i$.

\paragraph{Classic Distortion.} In the absence of deliberation, we assume the metric space is ``consistent'' with the rankings. By this, we mean that given location $i$ with $\rho_i > 0$, and two candidates $c_1$ and $c_2$,  the ranking $q = \sigma_i$ places $c_1$ higher than $c_2$  if $d(i,c_1) < d(i,c_2)$. If $d(i,c_1) = d(i,c_2)$, then either of the two candidates can be ranked higher than the other.  Note that a consistent metric always exists by placing all rankings with $\rho_q > 0$ at location $a$ and all candidates at location $b$.

A social choice rule $\mathcal{S}$ takes the rankings with $\rho_q > 0$ (and the corresponding $\rho_q$) as input, and outputs a single candidate as the ``winner''.   Note that the social choice rule only sees the rankings and the $\rho_q$, but not the underlying metric space. %is hidden from the social choice rule, but each voter can reason about its location in the metric space relative to other voters and the candidates and output a ranking over candidates consistent with the metric space, which the social choice rule uses.  
We also assume the social choice rule is {\em anonymous} meaning that its output does not depend on the identities of the alternatives -- if the alternatives are permuted by $\pi$, the social choice rule outputs $\pi(c)$ if it was originally outputting $c$. 

To measure the quality of the social choice rule, for any candidate $c$, let $SC(c,\sigma) = \sum_{i \in Q} \rho_i d(i,c)$ denote the social cost of $c$ under the metric $\sigma= (d, \vec{\rho})$.   Let $c^*(\sigma) = \mbox{argmin}_{c \in C} SC(c,\sigma)$ be the {\em social optimum} or the median outcome had the metric space $\sigma$ been known. This outcome need not be unique, in which case an arbitrary such outcome is output.  Let $\alg(\mathcal{S}, \vec{\rho})$ denote the output of the social choice rule $\mathcal{S}$ given the rankings $q$ and their masses $\rho_q$. Then the distortion of the rule is 
$$\mbox{Distortion } = \max_{\sigma } \frac{SC(\alg(\mathcal{S}, \vec{\rho}),\sigma)}{SC(c^*(\sigma),\sigma)},$$ 
%where the maximum is over all voter profiles $\sigma$ and all metric spaces $\sigma$ consistent with $\sigma$. 
The goal is to find a social choice rule with small (preferably constant) distortion.

\subsection{Deliberation Models} 
In our deliberative framework, there is a group size parameter $k \ge 2$. A single deliberation is a function that operates over two alternatives $c_1$ and $c_2$, and $k$ randomly chosen voters from the continuum of voters, and outputs one of the alternatives. By ``randomly chosen'', we mean $k$ rankings are chosen independently with replacement from the set of rankings, with ranking $q$ being chosen at each step with probability $\rho_q$. Given the continuum assumption on voters, these $k$ rankings (some of which are possibly repeated) correspond to $k$ distinct voters.

In order to define the deliberation function, we first need to define normalized bias. Given a voter at location $i \in Q$, define the {\em normalized bias} as
$$ \B_i(c_1,c_2) = \frac{d(i,c_1) - d(i,c_2)}{d(c_1,c_2)}.$$
The numerator captures the extent of the bias towards one of the two alternatives, with a positive number denoting bias towards $c_2$. The denominator normalizes this bias by the distance between the two alternatives. 

By the triangle inequality $\B(c_1,c_2) \in [-1,1]$. Note that $\B_i(c_1,c_2) = -1$ implies $d(i,c_2) = d(i,c_1) + d(c_1,c_2)$, so that given $d(i,c_1)$ and $d(c_1,c_2)$, $d(i,c_2)$ is largest possible. This captures the maximum bias of voters towards $c_1$. Similarly $\B_i(c_1,c_2) = 1$ makes $d(i,c_2)$ as small as possible, and captures strong bias towards $c_2$. Note that when $\B(c_1,c_2) = 0$, then $d(i,c_1) = d(i,c_2)$, so the voter is indifferent between the two outcomes.

The alternative that is output by the deliberation is a function of these normalized biases for the $k$ voters. We consider the following two models. In both these models, we let $S$ be the multi-set of locations of the voters who participate in the deliberation, so that $|S| = k$.

\begin{description}
\item[Averaging Model.] (\cref{sec:avg}.) Let $\tau = \sum_{i \in S} \B_i(c_1,c_2)$. If $\tau < 0$, the outcome is $c_1$, else it is $c_2$. This corresponds to voters choosing the median, $\mbox{argmin}_{c = c_1,c_2} \sum_{i \in S} d(i,c)$. This model is equivalent to the model where the group deliberates over all $m$ alternatives, and outputs the ranking consistent with the total distance, so that $c_1$ is ahead of $c_2$ in the ranking if $ \sum_{i \in S} d(i,c_1) \le \sum_{i \in S} d(i,c_2)$. As before, in the case of a tie, we assume an arbitrary tie-breaking rule between the alternatives.
\item[Random Choice Model.] (\cref{sec:random}.) In this model, let $S_1 \subseteq S$ denote the set of voters who perfer $c_1$ to $c_2$, and let $S_2 \subseteq S$ denote those that prefer $c_2$ to $c_1$. Let $A = \sum_{i \in S_1} |\B_i(c_1,c_2)|$ denote the total normalized bias of voters preferring $c_1$, and let $B = \sum_{i \in S_2} |\B_i(c_1,c_2)|$ be that for $c_2$.\footnote{Since $S$ is a multiset, if a location $i$ is repeated $\ell$ times, its contribution to the summation is $\ell \cdot |\B_i(c_1,c_2)|$.} Then the output is $c_1$ with probability $\frac{A}{A+B}$, and $c_2$ otherwise. % a voter at location $i \in S$ is chosen with probability proportional to their normalized bias, $|\B_i(c_1,c_2)|$.\footnote{Since $S$ is a multiset, if a location $i$ is repeated $\ell$ times, that location is chosen proportional to $\ell \cdot |\B_i(c_1,c_2)|$.}  If $\B_i(c_1,c_2) < 0$, alternative $c_1$ is the final outcome, else alternative $c_2$ is the final outcome. 
We assume the metric is perturbed slightly so that at least one $\B_i(c_1,c_2) \neq 0$; alternatively, if all $\B_i(c_1,c_2) = 0$, we assume $c_1$ is chosen.

In \cref{sec:random}, we also generalize the model so that there is a concave, non-decreasing function $g(x)$ with $g(0) = 0$ and $g(1) = 1$. We use $A = \sum_{i \in S_1} g(|\B_i(c_1,c_2)|)$ and $B = \sum_{i \in S_2} g(|\B_i(c_1,c_2)|)$ in randomly choosing the outcome.
\end{description}

The random choice model captures the intuition that more biased voters are likely to drive the discussion in their direction, though that outcome is far from certain. This model can also be interpreted as opinion change as follows: During deliberation, suppose every voter in $S_1$ independently changes their opinion to $c_2$ with probability $\frac{B}{A+B}$, and similarly, every voter in $S_2$ changes their opinion to $c_1$ with probability $\frac{A}{A+B}$. This process corresponds to the DeGroot model of opinion change \cite{degroot1974reaching}. If subsequently, the opinion of a random voter is implemented, this exactly corresponds to the random choice model. A final, somewhat non-deliberative interpretation is that the process chooses a voter proportional to their bias (which might be proportional to how much they speak up), and outputs that voter's preference. 

\paragraph{Remark.} For $k = 1$, both models reduce to a single voter choosing the candidate among $c_1, c_2$ that is higher in their ranking. Note also that in both these models, the outcome of deliberation is simply an ordinal preference between the two alternatives. The voters arrive at this alternative by reasoning about the underlying metric space, but this reasoning is hidden from the social choice rule that we describe next. 
