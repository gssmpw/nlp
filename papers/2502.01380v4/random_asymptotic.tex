\subsection{Asymptotic Behavior of Distortion in Group Size $k$}
We next show that the group size needed for Copeland to achieve a distortion of $1+\epsilon$ is $\tilde{O}(1/\epsilon^2)$, independent of the number of alternatives.

\begin{theorem}
\label{thm:asymp1}
For any $\epsilon > 0$, with a group size of $k = O\left(\frac{1}{\epsilon^2} \log\frac{1}{\epsilon}\right)$, the distortion of the Copeland rule in the random choice deliberation model is at most $1+\epsilon$.
\end{theorem}
\begin{proof}
  Consider the optimization problem in \cref{eq:opt2}. We can view it as finding a distribution $D$ that is $-\omega$ with probability $\alpha$ and $1$ with probability $1-\alpha$, from which $k$ samples are drawn. If $\alpha < 1/2$, the constraint cannot be satisfied for any $\omega \in [0,1]$, so that in the optimal solution $\alpha \ge 1/2$. For any $\delta > 0$, by Chernoff bounds, 
  $$ \Pr[ \ell \ge (1+\delta) k \alpha] \le e^{-k \alpha \delta^2/2} \le e^{-k  \delta^2/4}. $$
  If this event does not happen, then for $\ell^* = (1+\delta) k \alpha$, we have
  $$ \frac{\ell \cdot \omega}{\ell \cdot \omega + k-\ell } \le \frac{\ell^* \cdot \omega}{\ell^* \cdot \omega + k-\ell^* }, $$
 since the LHS is monotonically increasing in $\ell$. Letting $\beta = (1+\delta) \alpha$, by the law of total probability, the constraint therefore implies
  $$ \frac{\beta \cdot \omega}{\beta \cdot \omega + 1-\beta } + e^{-k \delta^2/4} \ge \frac{1}{2}. $$
  Let $k = \frac{4}{\delta^2} \log (2/\delta) $. Then, we have
  $$ \frac{\beta \cdot \omega}{\beta \cdot \omega + 1-\beta } \ge \frac{1 - \delta}{2} \ \  \Rightarrow \ \  \frac{(1+\delta)^2}{1-\delta} \cdot \alpha \cdot \omega \ge 1 - \alpha - \alpha \cdot \delta \ \ \Rightarrow \ \  1 - \alpha - \alpha \cdot \omega = O(\delta). $$
  Choosing $\epsilon = c \cdot \delta$ for a suitable constant $c$, this shows $\zeta_k = O(\epsilon)$. Plugging this into \cref{thm:distort1} completes the proof.
\end{proof}

