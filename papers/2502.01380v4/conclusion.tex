\section{Open Questions}
We now list some open questions. At the technical level, the main question is to close the gap between the lower and upper bounds in \cref{tab:average,tab:random}. An intriguing question for $k=2$ is the following: If every pair of voters deliberates and outputs a ranking over alternatives that is consistent with the sum of their distances to the alternatives, is there a deterministic social choice rule over these rankings that beats distortion $3$? The lower bound example in \cref{thm:theta2} can be extended to weighted tournament rules~\cite{MunagalaW19,Kempe_2020}, and shows these rules are insufficient. Further, rules like plurality veto~\cite{Kizilkaya022} that output the favorite alternative of some voter are insufficient by \cref{eg1}. Finding a rule for $k=2$  that admits to tractable analysis, yet breaks the bound of $3$ is an interesting question.

More generally, tightening our bounds will require analysis of rules beyond tournaments, which poses analytic challenges. Even for tournament rules like Copeland, it would be interesting to bypass the use of non-convex optimization tools in deriving the results in \cref{sec:avg}, which will enable extensions to larger group sizes. Indeed, our conjecture is that  for all $k \ge 2$, the optimal solution to the non-convex program in \cref{sec:avg} has the same binary support as that of the lower bound instances in \cref{thm:lb1}. It would also be interesting to develop a tight distortion bound for Copeland for $k=3$, improving \cref{thm:theta3}. 

At the modeling level, we can consider richer models of how voters randomize between outcomes. For instance, the work of~\cite{goyal2025metricdistortionprobabilisticvoting}  proposes a randomized voting model where $\Pr[W \succ_i X] = g(r)$, where $r = d(i,W)/d(i,X)$, and $g(r)$ is a bounded, increasing function, such as $g(r) = \frac{r^2}{1+r^2}$. In this model, they show improved distortion bounds without deliberation, and it would be interesting to analyze extensions to deliberation. Further, in our model, we have assumed the groups are chosen by random sampling.  In practice, the sampling could be stratified based on voter features, or their prior ranking of the alternatives. It would be interesting to extend the model in \cref{sec:random} to handle such generalizations, and find the optimal stratification. Further, individuals often enter into deliberation with goals beyond simply persuading others, for instance, they may seek to understand different viewpoints, or strive for the common good~\cite{ashkinaze2024}. It would be interesting to model  these aspects formally. 

Finally, there are aspects of real-world deliberation that we have not modeled. For instance, party affiliation and signaling significantly impacts opinion formation and social learning (see~\cite{partisan1} and citations therein), since some people take their cues from a higher authority like party, or could defer to the partisan group, especially when they know little about a topic. %Modeling party affiliation into the ranking of alternatives and into the deliberation model is a challenging open question. 
Similarly, some individuals are better at persuasion than others (regardless of the opinions they hold), and it would be interesting to add  individual capability into the model.

%Our work presents new models for small group deliberations, whose analysis has an interesting connection to small deviation bounds. In addition to tightening the bounds 