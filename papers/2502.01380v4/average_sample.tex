\subsection{Sample Complexity}
So far, our bounds have assumed $p_k(W,X)$ --- the probability that a randomly sampled group of size $k$ leads to $W$ as the outcome of deliberation --- can be estimated exactly. Via a standard argument, this can be converted to a sampling bound on the number of sampled groups needed to estimate $p_k(W,X)$ for all pairs of alternatives.

For each sampled group of size $k$, we ask the voters to rank all alternatives. Assuming that $c_1$ is ranked higher than $c_2$ if $\sum_{i \in S} d(i,c_1) \le \sum_{i \in S} d(i,c_2)$, it is easy to check that for each pair $(W,X)$, the outcome of deliberation just among this pair of alternatives will be consistent with the ranking. By a standard application of Chernoff bounds~\cite{Boucheron}, if we sample $O\left(\frac{\log (m/\delta)}{\epsilon^2}\right)$ groups of size $k$, then we can approximate each $p_k(W,X)$ to within an additive $\epsilon$ with probability $1-\delta$.  Suppose we output the Copeland winner of the tournament graph on the samples, then with high probability, for the winner $W$, and for any other alternative $X$, we either have $p_k(W,X) \ge 1/2 - \epsilon$, or there exists $Y$ such that $p_k(W,Y) \ge 1/2 - \epsilon$ and $p_k(Y,X) \ge 1/2 - \epsilon$. Since $\theta_k$ in \cref{eq:opt_avg} is smooth in the RHS of the constraint, this means $\theta_k$ is within $O(\epsilon)$ of the bounds computed above.  Plugging this into \cref{thm:distort1} yields the following theorem.

\begin{theorem}
\label{thm:sample2}
Let $d_k$ denote the distortion of the Copeland rule with groups of size $k$ if we could compute $p_k(W,X)$ exactly. Then, for any $\epsilon > 0$ and $\delta \in (0,1)$, we have that $O\left(\frac{1}{\epsilon^2} \cdot \log \frac{m}{\delta}\right)$ randomly sampled groups of size $k$ suffice to achieve distortion $d_k + \epsilon$ with probability $1-\delta$.
\end{theorem}

As shown in \cref{eg1},  if the group size $k$ is a constant and we want to beat the distortion bound of $3$, then this crucially requires the deliberating group to rank alternatives {\em beyond} their favorite (or median) alternative (among all $m$ alternatives). %This is because there are instances where even if all subsets of $k$ voters output their median alternative, the distortion cannot be better than $3$~\cite{FainGMS17,CaragiannisM024}; achieving a better bound with only favorite alternatives requires a group of size $\omega(1)$.
This makes our sampling bound above and in \cref{thm:sample1} with constant size groups different from core-set type sampling bounds for the $1$-median problem, for instance, the bounds in~\cite{CaragiannisM024}. 