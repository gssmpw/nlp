
\subsection{Distortion Bound for $k=3$}
We show that groups of size $3$ suffice for the distortion to go below the deterministic metric distortion lower bound of $3$. (As noted before, the Averaging rule for a finite but large number of voters can be viewed as a deterministic social choice rule.) Note that for $k=3$, the proof of \cref{thm:lb1} implies $\theta_3 \ge 0.25$, which also implies a lower bound of $5/3 \approx 1.667$ on the distortion of any deterministic social choice rule via \cref{thm:lb_main}. We show that the bound on $\theta_3$ is nearly tight.

\begin{theorem}
\label{thm:theta3}
For the optimization problem \cref{eq:opt_avg} with group size $k=3$, we have $0.25 \le \theta_3 \le 0.2522$. By \cref{thm:distort1}, this implies the distortion of the Copeland rule with $k = 3$ is at most $2.81$.
\end{theorem}

 

The rest of this subsection is devoted to proving $\theta_3 \le 0.2522$. We write a compact non-linear programming relaxation for the optimum of \cref{eq:opt_avg}. Note that the optimal distribution has expectation $\theta_3$. 

The key to our relaxation is the following lemma, which is a corollary of Lemma 1 in~\cite{Feige}. We present a proof for completeness.

\begin{lemma}[\cite{Feige}]  There exist independent distributions $D_1, D_2, D_3$ each supported on two values in $[-1,1]$ (these values could be different for each $D_i$), such that $\E[D_1] = \E[D_2] = \E[D_3] = \theta_3$, and 
\begin{equation}
\label{eq:feige}
\Pr_{a_1  \sim D_1, a_2 \sim D_2, a_3 \sim D_3} \left[ a_1 + a_2 + a_3 \le 0 \right] \ge 1/2. 
\end{equation}
\end{lemma}
\begin{proof}
We start with the distribution $D^*$ that optimizes \cref{eq:opt_avg}. Let $Y_1,Y_2,Y_3$ denote $i.i.d.$ random variables with distribution $D^*$ so that $a_1 \sim Y_1$, $a_2 \sim Y_2$, and $a_3 \sim Y_3$. First consider $Y_1$, and assume w.l.o.g. that it is finely discretized as $\{v_1,v_2, \ldots, v_n\}$, and let $p_j = \Pr[Y_1 = v_j]$. Let $q_j = \Pr[Y_2 + Y_3 + v_j \le 0]$. Then
$$ \sum_{j=1}^n p_j = 1 \qquad \mbox{and} \qquad \E[D^*] = \sum_{j=1}^n p_j v_j \qquad \mbox{and} \qquad \Pr[Y_1 + Y_2 + Y_3 \le 0] = \sum_{j=1}^n p_j q_j \ge 1/2.$$
Now treat the $p_j$ as non-negative variables, and maximize the LHS of the third constraint subject to the first two constraints. The optimum has at most two non-zero variables, and satisfies the third constraint (since the LHS cannot decrease). This yields a new distribution $D_1$ with $\E[D_1] = \theta_3$, and with support two. Repeat this process for $Y_2$ and then $Y_3$, completing the proof.
\end{proof}

We will find the optimal $(D_1,D_2, D_3)$ satisfying the above lemma, and this will upper bound $\theta_3$. For $i = 1,2,3$, let $D_i = a_i$ with probability $1-p_i$ and $b_i$ with probability $p_i$, where for all $i = 1,2,3$. Therefore, we have the constraints:
\begin{equation}
\label{eq:3.1}
    b_i \le a_i; \qquad a_i, b_i \in [-1,1],  \qquad  c_i = a_i - b_i, \qquad p_i \in [0,1] \qquad \forall i = 1,2,3. 
\end{equation} 
The $\{a_i,b_i,c_i,p_i\}_{i=1}^3$ and $\theta_3$ will be the variables in our program. We also have the constraints:
\begin{equation} 
\label{eq:3.2}
\theta_3 = b_i p_i + a_i (1-p_i) \qquad \forall i = 1,2,3. 
\end{equation}
Our goal is to maximize $\theta_3$. W.l.o.g., enforce the constraint that 
\begin{equation}
\label{eq:3.3}
c_3 \ge c_2 \ge c_1 \ge 0.
\end{equation}
Write $D_i = a_i - Z_i$ where $Z_i = c_i$ with probability $p_i$ and $0$ otherwise. Then the constraint in \cref{eq:feige} translates to:
\begin{equation}
\label{eq:zee}
\Pr_{z_1  \sim Z_1, z_2 \sim Z_2, z_3 \sim Z_3} \left[ \sum_{i=1}^3 z_i\ge \sum_{i=1}^3 a_i \right] \ge 1/2. 
\end{equation}

\begin{table*}[htbp]
\centering
\rowcolors{2}{gray!25}{white}
\begin{tabular}{|c|l|l|}
\hline 
Case &  Constraint on range of $\sum_{i=1}^3 a_i$ & Constraint \cref{eq:zee} \\
\hline 
1 & $ c_3 + c_2 + c_1 \ge a_3 + a_2 + a_1 \ge c_3 + c_2$ & $ p_1 p_2 p_3 \ge 1/2$ \\
2 &  $ c_3 + c_2  \ge a_3 + a_2 + a_1 \ge c_3 + c_1$ & $p_3 p_2 \ge 1/2$ \\
3 & $ c_3 + c_1  \ge a_3 + a_2 + a_1 \ge \max\{c_3, c_2 + c_1\}$  & $p_3 p_2 + p_3 p_1 (1-p_2) \ge 1/2.$ \\
4 &  $c_3   \ge a_3 + a_2 + a_1 \ge c_2 + c_1$ & $ p_3  \ge 1/2.$ \\
5 &  $ c_2+c_1   \ge a_3 + a_2 + a_1 \ge c_3 $ & $p_3  p_2   + p_3  (1-p_2)  p_1 + (1-p_3)  p_2 p_1 \ge 1/2$ \\
6 & $ \min\{c_2+c_1,c_3\}   \ge a_3 + a_2 + a_1 \ge c_2 $ & $ 1 - (1-p_3)  (1 - p_1 p_2) \ge 1/2$ \\
7 & $ c_2   \ge a_3 + a_2 + a_1 \ge c_1 $ & $ 1 - (1-p_3)  (1-p_2) \ge 1/2$ \\
8 &  $c_1 \ge a_3 + a_2 + a_1 $ & $ 1 - (1-p_3)  (1-p_2) (1-p_1) \ge 1/2$ \\
\hline
\end{tabular}
\caption{\label{tab:cases} The possible ranges for $\sum_{i=1}^3 a_i$ and the encoded constraints for each case.}
\end{table*}

In sorted order, $\sum_{i=1}^3 z_i$ can take values 
$$ c_3 + c_2 + c_1 \ge c_3 + c_2 \ge c_3 + c_1 \ge \{c_2 + c_1, c_3 \} \ge c_2 \ge c_1 \ge 0.$$
We enforce the constraint that the quantity $\sum_{i=1}^3 a_i$ lies between two of the values; this yields eight possible programs. In each of these programs, we have the constraints \cref{eq:3.1,eq:3.2,eq:3.3} and the objective is to maximize $\theta_3$. The cases differ in how \cref{eq:zee} is encoded. These cases are shown in \cref{tab:cases}, where the constraint in the third column encodes \cref{eq:zee} when $\sum_{i=1}^3 a_i$ lies in the range encoded by the constraint in the second column. 

As an example, for the first row, we have $\sum_{i=1}^3 a_i \in [c_3 + c_2, c_3 + c_2 + c_1]$. But this means for the event in \cref{eq:zee} to be satisfied, we must have $z_3 = c_3, z_2 = c_2, z_1 = c_1$, which happens with probability $p_3 p_2 p_1$. The constraint \cref{eq:zee} therefore implies $p_1 p_2 p_3 \ge 1/2$. We encode the constraints in the second and third columns of the row corresponding to Case (1) into the program for Case (1), and solve it along with constraints \cref{eq:3.1,eq:3.2,eq:3.3}, and the objective of maximizing $\theta_3$.



For each of the eight cases, we globally maximize $\theta_3$ using BARON~\cite{Sahinidis1996,KS18}. We observe that in all cases except Case (3), the global optimum is found to be at most $0.25$. In Case (3), the solver can only bound it in the range $[0.25, 0.2522]$. Nevertheless, $\theta_3 \le 0.2522$ in all cases. Plugging this into \cref{thm:distort1} completes the proof of \cref{thm:theta3}. 

We remark that BARON is a state of the art non-linear program solver, and it struggles to find the global optimum even for $k=3$ (Case 3 above). Given this, we believe an entirely different relaxation will be needed to solve the $k=4$ case.


