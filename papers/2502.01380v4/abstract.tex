\begin{abstract}
We consider models for social choice where voters rank a set of choices (or alternatives) by deliberating in small groups of size at most $k$, and these outcomes are aggregated by a social choice rule to find the winning alternative. We ground these models in the metric distortion framework, where the voters and alternatives are embedded in a latent metric space, with closer alternative being more desirable for a voter. We posit that the outcome of a small-group interaction optimally uses the voters' collective knowledge of the metric, either deterministically or probabilistically. 

We characterize the distortion of our deliberation models for small $k$, showing that groups of size $k=3$ suffice to drive the distortion bound below the deterministic metric distortion lower bound of $3$, and  groups of size $4$ suffice to break the randomized lower bound of $2.11$. We also show nearly tight asymptotic distortion bounds in the group size, showing that for any constant $\epsilon > 0$, achieving a distortion of $1+\epsilon$ needs group size that only depends on $1/\epsilon$, and not the number of alternatives. We obtain these results via formulating a basic optimization problem in small deviations of the sum of $i.i.d.$ random variables, which we solve to global optimality via non-convex optimization. The resulting bounds may be of independent interest in probability theory.
%Conceptually, our protocols and associated distortion bounds show that multiple small groups of people deliberating about specific alternatives suffice to improve on straightforward voting over all alternatives.
\end{abstract}