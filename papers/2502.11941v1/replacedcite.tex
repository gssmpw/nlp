\section{Related Works}
\label{related_work}
\subsection{Chemical modeling for air quality analysis}
In recent decades, China has experienced rapid economic growth and urbanization. This has come with challenges, one of the biggest of which is air quality, and there has been a focus on researching and understanding air quality issues in Chinese cities ____. One of the most commonly measured air pollutants is PM\(_{2.5}\)(particulate matter less than \SI{2.5}{\micro\meter} in diameter). When inhaled, airborne particulates can penetrate deep into the lung and enter the bloodstream, leading to cardiovascular diseases, strokes, and respiratory illness____. During haze episodes, which are prevalent in Beijing and the North China Plain in winter months, PM$_{2.5}$ can reach very unhealthy and even dangerous levels. Haze episodes have serious health and economic consequences, resulting in hospitalizations, loss of working time, and premature deaths____.

% !!! ***I'm not sure how to add the website, WHO2024 and OECD2016*** !!!

% Zeng et al., 2019: https://doi.org/10.1016/j.scitotenv.2019.01.262 
% Wang and Hao, 2012:  https://doi.org/10.1016/S1001-0742(11)60724-9
% Fang et al., 2009: https://doi.org/10.1016/j.atmosenv.2008.09.064
% WHO, 2024: Air quality, energy and health: Health impacts. https://www.who.int/teams/environment-climate-change-and-health/air-quality-energy-and-health/health-impacts.
% Luo et al., 2021: https://doi.org/10.3390/atmos12030323
% Gao et al., 2015: https://doi.org/10.1016/j.scitotenv.2015.01.005
% Ji et al., 2012: https://doi.org/10.1016/j.atmosenv.2011.11.053
% Jiang et al., 2015: https://doi.org/10.4209/aaqr.2014.04.0070
% OECD, 2016: https://doi.org/10.1787/9789264257474-en 
% Xie et al., 2019: https://doi.org/10.1016/j.envint.2019.05.075

Traditional approaches to air quality reanalysis often rely on chemical transport models (CTMs), which simulate the chemical and physical mechanisms and processes in the atmosphere. CTMs are usually coupled with a numerical weather prediction (NWP) model, and they are used for generating short- and mid-range reanalysis of air quality parameters____. Examples of CTMs for air quality analysis include WRF-Chem____, CMAQ____, Enviro-HIRAM____, and SILAM____. While these models are useful reanalysis tools and provide valuable insights, they are computationally expensive and highly sensitive to the precision of input data, such as uncertainties in anthropogenic emission inventories or meteorological parameters in the NWP datasets ____. These models often underestimate pollutant concentrations during severe haze episodes in China ____. Moreover, CTMs are limited to current scientific knowledge, and they still lack some important chemical mechanisms. For example, many models are missing heterogeneous oxidation of SO$_2$, which is an important mechanism for sulfate particle formation ____. Autoxidation of aromatics is also an important process, especially in China, which leads to PM$_{2.5}$ formation and is unaccounted for in most current models ____.

\subsection{Temporal domain prediction via deep learning}
Recent advancements in deep learning have significantly improved air quality reanalysis by leveraging time-series data. Recurrent Neural Networks (RNNs) and their variants, such as Long Short-Term Memory (LSTM) ____ and Gated Recurrent Units (GRUs) ____, have been widely used for sequential modeling. These models capture long-term dependencies in air pollution data and have shown promising results in predicting pollutant concentrations based on historical trends ____.

One of the main challenges of RNN-based models is the vanishing gradient problem, which can affect long-range dependencies. LSTM and GRU networks mitigate this issue through gating mechanisms that selectively retain relevant information over time. Several studies have demonstrated the effectiveness of LSTMs in air quality reanalysis ____. However, these models primarily focus on temporal dependencies and often overlook the spatial correlations between different monitoring stations ____. An attention mechanisms ____ have been integrated into LSTM-based architectures to enhance performance by selectively weighting important time steps ____. Attention-based LSTMs have demonstrated superior prediction accuracy compared to standard LSTMs, particularly in complex urban environments where pollution levels fluctuate dynamically ____. Despite their success, these methods still lack spatial adaptability, making them ineffective for predicting PM\(_{2.5}\) concentrations in unobserved locations. 

\subsection{Deep learning for spatiotemporal domain analysis}
To address these limitations, hybrid approaches have been developed to incorporate both spatial and temporal dependencies in air quality reanalysis. Traditional machine learning models, such as XGBoost ____, leverage historical time series and meteorological data to predict PM\(_{2.5}\) concentrations. Although these models are computationally efficient, they often fail to capture the complex nonlinear relationships present in spatiotemporal data ____.

To integrate spatial dependencies, 
this approach is related to image/video super-resolution techniques. Besides conventional Convolutional Networks____, Graph Neural Networks (GNNs) have been used for spatiotemporal analysis. Dynamic Graph Convolutional Recurrent Neural Networks (DCRNN) ____ and Spatio-Temporal Graph Convolutional Networks (ST-GCN) ____ leverage graph structures to model the relationships between monitoring stations, enabling more accurate reanalysis. Additionally, attention mechanisms have been employed to enhance the efficiency of spatio-temporal models. The Transformer model ____, initially designed for natural language processing, has been adapted to environmental modeling due to its ability to capture long-range dependencies ____. Hybrid models that integrate LSTMs with attention mechanisms, such as STTN (Spatio-Temporal Transformer Networks) ____, have demonstrated strong performance in estimating air pollution levels.