\pdfoutput=1

\documentclass[runningheads]{llncs}
%
\usepackage[T1]{fontenc}
% T1 fonts will be used to generate the final print and online PDFs,
% so please use T1 fonts in your manuscript whenever possible.
% Other font encondings may result in incorrect characters.
%
\let\proof\relax
\let\endproof\relax
\usepackage{subcaption}
\usepackage{amssymb}
% Use the SI units package.
\usepackage{siunitx}
%% The amsthm package provides extended theorem environments
\usepackage{graphicx}
\usepackage{multirow}
\usepackage{amsmath,amssymb,amsfonts}
\usepackage{amsthm}
\usepackage{mathrsfs}
\usepackage[title]{appendix}
\usepackage{xcolor}
\usepackage{colortbl}
\usepackage{textcomp}
\usepackage{manyfoot}
\usepackage{booktabs}
\usepackage{algorithmicx}
\usepackage{listings}
\definecolor{mistyrose}{rgb}{1.0, 0.89, 0.88}
\newcommand{\liu}[1]{\textcolor{blue}{#1}}
% Used for displaying a sample figure. If possible, figure files should
% be included in EPS format.
%
% If you use the hyperref package, please uncomment the following two lines
% to display URLs in blue roman font according to Springer's eBook style:
%\usepackage{color}
%\renewcommand\UrlFont{\color{blue}\rmfamily}
%\urlstyle{rm}
%
\usepackage{graphicx} 
\usepackage{float}     
\begin{document}
\title{Deep Spatio-Temporal Neural Network for Air Quality Reanalysis}

\titlerunning{Deep Spatio-Temporal NN for Air Quality}
% If the paper title is too long for the running head, you can set
% an abbreviated paper title here

\author{Ammar Kheder$^{1,2}$ \and
Benjamin Foreback$^{2,3}$ \and
Lili Wang$^{4}$ \and
Zhi-Song Liu$^{1,2}$ \and 
Michael Boy$^{1,2,3}$}

\authorrunning{A. Kheder et al.}
% First names are abbreviated in the running head.
% If there are more than two authors, 'et al.' is used.

\institute{Lappeenranta–Lahti University of Technology LUT 
\and
Atmospheric Modelling Centre Lahti, Lahti University Campus
\and
University of Helsinki
\and
Chinese Academy of Sciences
}
\maketitle              % typeset the header of the contribution
%
\begin{abstract}
Air quality prediction is key to mitigating health impacts and guiding decisions, yet existing models tend to focus on temporal trends while overlooking spatial generalization. We propose AQ-Net, a spatiotemporal reanalysis model for both observed and unobserved stations in the near future. AQ-Net utilizes the LSTM and multi-head attention for the temporal regression. We also propose a cyclic encoding technique to ensure continuous time representation. To learn fine-grained spatial air quality estimation, we incorporate AQ-Net with the neural kNN to explore feature-based interpolation, such that we can fill the spatial gaps given coarse observation stations. To demonstrate the efficiency of our model for spatiotemporal reanalysis, we use data from 2013--2017 collected in northern China for PM\(_{2.5}\) analysis.  Extensive experiments show that AQ-Net excels in air quality reanalysis, highlighting the potential of hybrid spatio-temporal models to better capture environmental dynamics---especially in urban areas where both spatial and temporal variability are critical.


\end{abstract}

\keywords{Air Quality Reanalysis  \and Spatiotemporal Analysis \and Deep Learning\and Attention \and kNN.}



\section{Introduction}

\begin{figure*}
    \centering
    \includegraphics[width=\textwidth]{figures/Introduction.pdf}
    \caption{Showing the novel problem statement applied to traffic prediction use case. Multiple unstructured observations from the past are used to reconstruct a hidden traffic state from which a full traffic state is forecast with a set of query locations. }
    \label{fig:intro}
\end{figure*}

% Was sagen denn die anderen warum Traffic Prediction gut ist? 
Forecasting the traffic in the near future is an important task for city management.
Data from the near past is used to predict future traffic states with spatio-temporal Graph Neural Networks \cite{bui22}.
Accurate prediction provides the opportunity to optimize traffic flow, reduce traffic jams and increase air quality \cite{Po19}.

% Wieso ist Sparsity in allen Dimensionen wichtig.
While traffic prediction relies on the availability of data from traffic sensors, there exists a plethora of reasons why sensors may stop working temporarily, such as simple errors, energy saving, or overloaded communication systems.
Considering small- or medium-sized cities, the coverage of sensors may be low because the sensors are too expensive or not available.
Also, the sensors are typically static and do not adapt to changes in the traffic flow (e.g. caused by a construction site), which motivates moving sensors that for example could be mounted on cars. 
However, both missing and moving sensors introduce sparsity, since measurements may not be available for all locations at all times.
This sparsity must be explicitly addressed in traffic prediction for a realistic application scenario, which is illustrated in figure \ref{fig:intro}.
From one hour of data on Sunday morning, only few observations of the traffic state are available at each timestep.
The number of observations may differ throughout the observed time and the observation itself can be distributed arbitrarily in the city. 
We assume a relatively low number of sensors to account for resource saving and sensor failure in our proposed framework SUSTeR.
The task is to predict the dense traffic state one timestep after the observations at all possible sensor locations.
We study this problem on the traffic dataset Metr-LA and PEMS-BAY to test our assumption that only a fraction of the sensor values would be enough for good predictions.
By modifying an existing traffic dataset, we are able to compare our results from very sparse observations to the bottom line with all information available.
A successful study will provide insights in how sensors in new cities can be reduced before installing them and further mobile sensors would save more resources and are able to adapt to new traffic situations.
We argue that in order to be adaptable to other cities and changes in traffic flows, prior information like the road network should be neglected and just the sparse observations considered.
This comes with the added benefit of making our solution applicable in regions where no openly available road network is maintained or pathways change frequently (e.g. flood areas, animal observations). 


The aforementioned problem is novel and more challenging than the commonly considered traffic prediction problem, since there exist very few observations in each input sample.
Current works for the traffic prediction problem do not consider any missing values. \cite{Li2021, Shao22}
A common method among state of the art approaches is the usage of Graph Neural Networks on graphs that model the sensor network \cite{bui22}.
The values of a sensor are applied to the same graph node for each timestep which prohibits any non-stationary sensors . 
With fixed sensor locations, the resulting sensor network is highly correlated with the road network.
Streets connecting two intersections with sensors should be also an interesting point for correlations in the sensor network.
However, variable observations and high temporal sparsity rates can not be modeled adequately in a static network.
We show in our experiments that the road network has only a small influence on the traffic predictions.

Besides the traffic prediction for future timesteps, some works explore the field of traffic speed imputation \cite{Cini22, Cuza22} where missing sensor values are predicted.
But the amount of missing values is assumed to be at most 80\%, which on average are still over 40 given sensors in each timestep in the Metr-LA dataset with a total of 207 sensors.
We consider up to 99.9\% missing values which are on average 2.4 observations in each timestep that are used as input.
Such high sparsity rates drastically decrease the chance that multiple values are present in one input sample from the same sensor location, which makes it challenging to recognize and learn temporal correlations for each location on its own.

High sparsity rates (>95\%) result in few sensor values, but if a reconstruction of the traffic state would be possible, we question if spatio-temporal graphs require nodes for each sensor.
In SUSTeR we utilize only a small amount of graph nodes for the encoding of information and do not relate such nodes to the sensor network.
We call this the hidden graph (see figure \ref{fig:intro}), which is still able to reconstruct the complete traffic state.
Due to the reduced number of nodes SUSTeR achieves faster runtimes, as shown in the experiments.
This hidden graph is not embedded directly in the spatial domain, which is why the assignment of observations, as well as the querying of the future traffic, is done with an encoder and a decoder, implemented as neural networks.
The decoding from the hidden graph to future values depends on a set of query locations.
Figure \ref{fig:intro} shows the query locations as given from outside and in combination with the reconstructed traffic state the future values are predicted.

To construct the hidden graph we encode observations from each timestep into from multiple graphs, one for each timestep. 
The graphs are created in a residual style and information is added to the node embeddings from the previous timesteps.
We choose this method to incorporate all timesteps equally into the hidden state because the redundant information along the past is non-existing for high sparsity rates.
From the sequence of graphs where our framework inserted the observations step by step we apply STGCN \cite{Yu18}, an algorithm for traffic prediction to find and learn the spatio-temporal correlations on our small number of graph nodes.
The first future timestep of the STGCN is our hidden graph in which the traffic state is reconstructed. 

% Recent work has an implicit embedding of the graph nodes into the spatial domain as the assignment from the sensor to graph node is fixed one by one.
% Because the graph has the same structure as the road network spatio-temporal correlations can be learned between those sensors.
% We reduce the number of nodes and use a non-linear assignment learned data-driven from the observations.

We find in the experiments that SUSTeR outperforms the plain STGCN and modern traffic prediction frameworks like D2STGNN for high sparsity rates $(\geq 99\%)$.
This is equivalent to only $0.2$ to $2.4$ observation for each timestep on average.
SUSTeR uses fewer parameters than the baselines and can train faster and with less training data.
Our main contributions can be summarized as follows:
\begin{itemize}
    \item We introduce a sparse and unstructured variant of the traffic prediction problem with sparsity in all dimensions. The sensors report only a fraction of their values and are arbitrarily distributed in the spatial domain.
    \item We propose SUSTeR, a framework around the STGCN architecture, which maps sparse observations onto a dense hidden graph to reconstruct the complete traffic state.
    Our code is available at github.\footnote{https://github.com/ywoelker/SUSTeR}
    \item We conducts experiments that show that SUSTeR outperforms the baselines in very sparse situations ($\geq 95\%$) and has a competitive performance in low sparsity rates.
    % \item SUSTeR trains a third faster than the next competitor.
\end{itemize}

\section{Related Work}
\label{sec:related_work}

\subsection{Robustness of Audio-Visual Speech Recognition} 

The robustness of AVSR systems has significantly advanced by integrating auditory and visual cues to improve speech recognition, especially in noisy environments. Conventional ASR methods have evolved from relying solely on audio signals \cite{schneider2019wav2vec, gulati2020conformer, baevski2020wav2vec, hsu2021hubert, chen2022wavlm, chiu2022self, radford2023robust} to incorporating visual data from speech videos \citep{makino2019recurrent}.
The multimodal AVSR methods \citep{pan2022leveraging, shi2022learning, seo2023avformer, ma2023auto} have enhanced robustness under audio-corrupted conditions, leveraging visual details like speaker's face or lip movements as well as acoustic features of speech. These advancements have been driven by various approaches, including end-to-end learning frameworks \citep{dupont2000audio, ma2021end, hong2022visual, burchi2023audio} and self-supervised pretraining \citep{ma2021lira, qu2022lipsound2, seo2023avformer, zhu2023vatlm, kim2025multitask}, which focus on audio-visual alignment and the joint training of modalities~\citep{zhang2023self, lian2023av, haliassos2022jointly, haliassos2024braven}.


Furthermore, recent advancements in AVSR highlight the importance of visual understanding alongside audio \citep{dai2024study, kim2024learning}. While initial research primarily targeted audio disturbances \citep{shi2022robust, hu2023hearing, hu2023cross, chen2023leveraging}, latest studies increasingly focus on the visual robustness to address challenges such as real-world audio-visual corruptions~\citep{hong2023watch, wang2024restoring, kim2025multitask} or modality asynchrony~\citep{zhang2024visual, fu2024boosting, li2024unified}. These efforts remark a shift towards a more balanced use of audio and visual modalities. Yet, there has been limited exploration in scaling model capacity or introducing innovative architectural designs, leaving room for further developments in AVSR system that can meticulously balance audio and visual modalities.



\subsection{MoE for Language, Vision, and Speech Models}

Mixture-of-Experts (MoE), first introduced by \citet{jacobs1991adaptive}, is a hybrid structure incorporating multiple sub-models, \ie experts, within a unified framework. The essence of sparsely-gated MoE \cite{shazeer2017outrageously, lepikhin2021gshard, dai2022stablemoe} lies in its routing mechanism where a learned router activates only a subset of experts for processing each token, significantly enhancing computational efficiency. Initially applied within LLMs using Transformer blocks, this structure has enabled unprecedented scalability \cite{fedus2022switch, zoph2022st, jiang2024mixtral, guo2025deepseek} and has been progressively adopted in multimodal models, especially in large vision-language models (LVLMs) \cite{mustafa2022multimodal, lin2024moellava, mckinzie2025mm1}.
Among these multimodal MoEs, \citet{zhu2022uni, shen2023scaling, li2023pace, li2024uni} and \citet{lee2025moai} share the similar philosophy to ours, assigning specific roles to each expert and decoupling them based on distinct modalities or tasks. These models design an expert to focus on specialized segments of input and enhance the targeted processing.

Beyond its applications in LLMs and LVLMs, the MoE framework has also been applied for speech processing \cite{you2021speechmoe, you2022speechmoe2, hu2023mixture, wang2023language}, where it has shown remarkable effectiveness in multilingual and code-switching ASR tasks. In addition, MoE has been employed in audio-visual models \cite{cheng2024mixtures, wu2024robust}, although they primarily focus on general video processing and not specifically on human speech videos. These approaches leverage MoE to model interactions between audio and visual tokens without directly processing multimodal tokens.
Our research advances the application of the MoE framework to AVSR by designing a modality-aware hierarchical gating mechanism, which categorizes experts into audio and visual groups and effectively dispatches multimodal tokens to each expert group. 
This tailored design enhances the adaptability in managing audio-visual speech inputs, which often vary in complexity due to diverse noise conditions.

\section{Approach}

\begin{figure}[t]
\centering
\includegraphics[width=\textwidth]{figures/archi.pdf}
\caption{\textbf{Overview of the proposed AQ-Net.} 
The input includes historical pollutant concentrations,  and visible station coordinates. An LSTM extracts temporal dependencies, enhanced by Multi-Head Attention to highlight critical time steps. After temporal pooling, a neural kNN module performs spatial interpolation for unobserved stations (red markers).}
\label{fig:overall_architecture}
\end{figure}

The air quality data used as input is from a network of real-time air quality sensors throughout China, which are managed by the Ministry of Environmental Protection (MEP) and published hourly by the China Environmental Monitoring Centre (CEMC) \cite{SONG2017334}. This network has been collecting measurements since 2013, with the goal to study and predict air quality issues throughout China. By 2014, there were 944 air quality monitoring stations in 190 cities, and currently, there are over 2100 stations throughout China. The air quality parameters measured by this network are PM$_{2.5}$ and PM$_{10}$, NO$_\mathrm{x}$, SO$_2$, O$_3$ and CO. The dataset has been quality controlled with the algorithm described by Wu et al.\cite{Wu2018}. This study uses data from 584 stations in the metropolitan area of northern China from 2013 to 2017.

\subsection{Overall Architecture}
Our model predicts PM\(_{2.5}\) by combining temporal and spatial dependencies. AQ-Net comprises three core components: an LSTM-MHA module, combining LSTM and multi-head attention for temporal feature extraction, a neural kNN module for spatial interpolation, and a Cyclic Encoding (CE) layer for time embedding. We use hourly measurements of PM\(_{2.5}\), PM\textsubscript{10}, CO, NO\textsubscript{2}, SO\textsubscript{2}, and O\textsubscript{3} from monitoring stations, and estimate PM\(_{2.5}\) for the coming hours and days. The LSTM captures long-range pollutant fluctuations, while multi-head attention highlights critical time steps. A temporal pooling step condenses the latent sequence into a single feature vector, which the neural kNN module uses for spatial interpolation at unobserved stations based on their nearest neighbors. This integrated architecture generates 168-hour (7-day) PM\(_{2.5}\) estimation for both observed and unobserved locations, leveraging key temporal patterns and spatial relationships.

\subsection{Proposed Modules}
Air quality reanalysis is formulated as a spatiotemporal reanalysis problem. The input data consists of historical pollutant concentrations along with their time steps and geospatial information from monitoring stations. 

\noindent\textbf{Cyclic Encoding (CE) for Temporal Features}
To preserve the periodic nature of time-related features, we apply a cyclic encoding technique to the time step $t$ using sine and cosine transformations. This approach ensures a continuous representation, preventing discontinuities between values such as 23:00 and 00:00. The encoding is defined as $x_{\sin} = \sin \left( \frac{2\pi t}{\text{cycle}} \right), x_{\cos} = \cos \left( \frac{2\pi t}{\text{cycle}} \right)$, where \( t \) represents a temporal feature (e.g., hour, day, or month), and cycle corresponds to its periodicity (e.g., 24 for hours, 7 for days, and 12 for months).

\noindent\textbf{Long Short-Term Memory (LSTM):} To capture temporal dependencies, an LSTM network processes the time series data of pollutant concentrations. Given an input sequence $X \in \mathbb{R}^{C \times T \times N}$, where C is the number of features, T is the sequence length (number of time steps), and N is the number of stations, the LSTM generates a temporal representation $Z \in \mathbb{R}^{\text{dim} \times T \times N}$, where \(\text{dim}\) represents the hidden state dimension. The LSTM captures long-term dependencies, allowing the model to learn pollutant trends over time.

\noindent\textbf{Multi-Head Attention (MHA):} Although LSTM effectively captures sequential dependencies, it treats all past observations equally at each time step. To improve performance, we integrate a Multi-Head Attention (MHA) mechanism that enhances temporal dependencies by selectively weighting relevant time steps. The attention mechanism is computed as follows:

\begin{small}
\begin{equation}
Z' = \text{softmax} \left( \frac{Q_zK_z^T}{\sqrt{d_k}} \right) V_z + Z
\label{eq:fm_1}
\end{equation}
\end{small}

\noindent where $Q_z, K_z, V_z$ are linear transformations of $Z$ and $d_k$ is a scaling factor. This mechanism enables the model to focus on important time intervals, improving its ability to recognize patterns in pollutant fluctuations.

\noindent\textbf{Spatial Interpolation via kNN}
While the LSTM-MHA module captures temporal trends, it does not account for spatial correlations between monitoring stations. To estimate PM\(_{2.5}\) values at unobserved stations, we employ a kNN-based interpolation method. Instead of using the raw PM\(_{2.5}\) values from observed stations, we first extract a ``station-wise feature vector'' from the refined temporal representation $Z'$ using temporal pooling $Z' \in \mathbb{R}^{C\times \text{dim} \times N}$. This feature vector encapsulates the learned temporal patterns of each station, rather than just the raw measurements. Given a set of observed stations with known PM\(_{2.5}\) values, missing values at unobserved stations are estimated as, $Y = h(Z', p, k)$, where $p$ represents the geospatial coordinates of the stations, and $h(\cdot)$ applies kNN-weighted interpolation. The number of neighborhoods $k$ is defined on the fly such that we obtain multiple estimations. To speed up the computation of the station-to-station distance, we utilize GPU-enabled kNN query to ensure gradient backpropagation and fast searching. The kNN module finds the nearest stations in the learned feature space and interpolates the missing values accordingly, ensuring that spatial dependencies are taken into account.

\section{Experiments}

\subsection{Dataset and model architectures}
We use real-world data from 584  monitoring stations collected between 2013 and 2017. The dataset contains hourly measurements of CO, NO\textsubscript{2}, O\textsubscript{3}, PM\textsubscript{10}, PM\textsubscript{2.5}, and SO\textsubscript{2}. Stations with incomplete time series are removed, and all features are normalized into [0, 1]. We train AQ-Net using AdamW optimizer with the learning rate of $1\times10^{-3}$. We use the MSE as the loss function to estimate the network output and ground truth values for backpropagation. The batch size is set to 32 and AQ-Net is trained for {468K iterations (about 2 hours) on the CSC server\footnote{https://csc.fi/} with one NVIDIA A100 GPU using PyTorch deep learning platform.} The k value for Neural kNN is 20.
The code can be found at \liu{\url{https://github.com/AmmarKheder/AQ-Net}}.

\subsection{Overall Performance Comparison}
To evaluate the accuracy of the reanalysis, three key metrics are considered: Mean Absolute Error (MAE), Root Mean Squared Error (RMSE), and the coefficient of determination ($R^2$). Each of these metrics provides valuable insight into the performance of the models:

\begin{small}
\begin{equation}
\text{MAE} = \frac{1}{n} \sum_{i=1}^{n} |y_i - \hat{y}_i|, \quad
        \text{RMSE} = \sqrt{\frac{1}{n} \sum_{i=1}^{n} (y_i - \hat{y}_i)^2}, \quad
        R^2 = 1 - \frac{\sum_{i=1}^{n} (y_i - \hat{y}_i)^2}{\sum_{i=1}^{n} (y_i - \bar{y})^2}
\label{eq:eval}
\end{equation}
\end{small}

\noindent We compare our proposed AQ-Net with three approaches: PatchTST~\cite{nie2022patchtst} (a Transformer tailored for time series processing), Linear Regression, and LSTM~\cite{shi2015convolutional}. Our model can be used for temporal prediction at the same monitoring stations, it can also provide spatiotemporal prediction at unseen monitor stations. 

\subsubsection{Short-term temporal reanalysis (24-Hour Input Window)}
Table~\ref{tab:short_term} shows short-term estimated PM\textsubscript{2.5} concentrations in Beijing over the next few hours based on a 24-hour historical input. These reanalysis are critical for real-time air quality monitoring, health alerts, and short-term pollution control measures.

\begin{table}[t]
\centering
\caption{\textbf{The evaluation of short-term PM\textsubscript{2.5} reanalysis.} The table presents PM\textsubscript{2.5} reanalysis performance based on $R^2$, MAE, and RMSE over 6, 12, and 24 hours using a 24-hour historical input.}
\label{tab:short_term_results}
\renewcommand{\arraystretch}{1.2} % Better spacing for readability
\begin{tabular}{c|ccc|ccc|ccc}
\hline
\multirow{2}{*}{Model} & \multicolumn{3}{c|}{6h reanalysis} & \multicolumn{3}{c|}{12h reanalysis} & \multicolumn{3}{c}{24h reanalysis} \\
                        & $R^2$$\uparrow$ & MAE$\downarrow$  & RMSE$\downarrow$  & $R^2$ & MAE$\downarrow$  & RMSE$\downarrow$  & $R^2$ & MAE$\downarrow$  & RMSE$\downarrow$  \\ \hline
\cellcolor{mistyrose}{AQ-Net}                  & \cellcolor{mistyrose}{\textbf{0.5103}} & \cellcolor{mistyrose}{\textbf{18.71}} & \cellcolor{mistyrose}{\textbf{22.87}}  & \cellcolor{mistyrose}{\textbf{0.4118}} & \cellcolor{mistyrose}{\textbf{22.04}} & \cellcolor{mistyrose}{\textbf{29.10}}  & \cellcolor{mistyrose}{\textbf{0.1894}} & \cellcolor{mistyrose}{\textbf{26.18}} & \cellcolor{mistyrose}{\textbf{33.34}}  \\
AQ-Net wo CE          &        0.4031    &      21.21      &  29.23          &  0.2312          &   25.32         &       30.23  &0.1231 & 27.21 & 34.48   \\
PatchTST                & 0.4421 & 21.65 & 27.52  & 0.3319 & 23.57 & 31.50  & 0.1601 & 27.65 & 34.08  \\
LSTM                    & 0.4648 & 20.05 & 26.44  & 0.2336 & 25.40 & 32.44  & 0.1001 & 28.38 & 35.13  \\
Linear Regression       & 0.4500 & 20.80 & 27.00  & 0.2100 & 26.00 & 33.00  & 0.0800 & 29.00 & 35.80  \\ \hline
\end{tabular}
\label{tab:short_term}
\end{table}

Table~\ref{tab:short_term_results} presents the evaluation of the performance of four models (AQ-Net, PatchTST, LSTM, and Linear Regression) for predicting PM\textsubscript{2.5} concentrations in Beijing over 6, 12, and 24 hours using a 24-hour historical input. We also have AQ-Net wo CE to represent our approach without using the proposed cyclic encoding approach. For the 6-hour estimation, AQ-Net achieves the best result with an \(R^2\) of 0.51, an MAE of 18.71, and an RMSE of 22.87, demonstrating its ability to effectively capture rapid fluctuations in pollution. Although PatchTST employs a self-attention mechanism, it underperforms slightly with an $R^2$ of 0.44, an MAE of 21.65, and an RMSE of 27.52, while the LSTM and Linear Regression models show comparable results with $R^2$ values of 0.46 and 0.45, respectively, and marginally higher error metrics. As the prediction horizon extends to 12 hours, the performance of all models deteriorates; however, AQ-Net maintains a significant lead with an $R^2$ of 0.41, whereas the other models drop to $R^2$ values of 0.33 for PatchTST, 0.23 for LSTM, and 0.21 for Linear Regression. This trend continues for the 24-hour estimation, where AQ-Net achieves an \(R^2\) of 0.19 compared to 0.16, 0.10, and 0.08 for PatchTST, LSTM, and Linear Regression, respectively. These results indicate that AQ-Net is particularly robust and effective for short-term estimation, while the competing models, especially PatchTST, LSTM, and Linear Regression, struggle to maintain their accuracy as the reanalysis horizon increases. Comparing AQ-Net and AQ-Net wo CE, we can also see that using Cyclic encoding can improve $R^2$ by 0.06$\sim$0.17 in 6$\sim$24 h reanalysis, which demonstrates its efficiency.

\subsubsection{Long-Term temporal reanalysis (336-Hour Input Window)}
Table~\ref{tab:long_term_results} shows long-term reanalysis in Beijing and analyzes how well models estimate PM\textsubscript{2.5} levels over extended periods based on a 2-week (336-hour) historical window. 
\begin{table}[t]
\centering
\caption{\textbf{The evaluation of long-term PM\textsubscript{2.5} reanalysis.} The table presents long-term reanalysis performance on MAE and RMSE over 2-day, 4-day, and 1-week horizons, using a 2-week (336-hour) historical input.}
\resizebox{\textwidth}{!}{%
  \begin{tabular}{c|cc|cc|cc}
  \hline
  \multirow{2}{*}{Model} & \multicolumn{2}{c|}{2-Day reanalysis} & \multicolumn{2}{c|}{4-Day reanalysis} & \multicolumn{2}{c}{1-Week reanalysis} \\
                       & MAE$\downarrow$  & RMSE$\downarrow$  & MAE$\downarrow$  & RMSE$\downarrow$  & MAE$\downarrow$  & RMSE$\downarrow$  \\ \hline
  \cellcolor{mistyrose}{AQ-Net}            & \cellcolor{mistyrose}{\textbf{13.57}}  & \cellcolor{mistyrose}{\textbf{16.80}}  & \cellcolor{mistyrose}{\textbf{17.44}}  & \cellcolor{mistyrose}{\textbf{21.29}}  & \cellcolor{mistyrose}{\textbf{21.29}}  & \cellcolor{mistyrose}{\textbf{25.17}}  \\
  AQ-Net wo CE          &   17.12         &     21.77       &        18.12   &       24.63     &       24.23    &       28.37     \\
  PatchTST          & 41.42           & 55.64           & 35.31           & 39.22           & 28.01           & 34.70           \\
  LSTM              & 24.04           & 28.37           & 25.11           & 31.21           & 22.87           & 28.81           \\
  Linear Regression & 25.00           & 29.00           & 26.00           & 32.00           & 23.50           & 29.50           \\ \hline
  \end{tabular}
}
\label{tab:long_term_results}
\end{table}

\begin{figure}[t]
    \centering
    \begin{minipage}{0.48\textwidth}
        \centering
        \includegraphics[width=\textwidth]{figures/7days.pdf}
        \caption{\textbf{Comparison of PM\textsubscript{2.5} reanalysis for different time slots over seven days.} The 4PM-7PM period exhibits greater variability, suggesting increased pollution activity during the late afternoon.}
        \label{fig:pm25_timeslot_comparison}
    \end{minipage}
    \hfill
    \begin{minipage}{0.48\textwidth}
        \centering
        \includegraphics[width=\textwidth]{figures/attention_evolution_head.pdf}
        \caption{\textbf{Visualization of the evolution of attention weights for selected two heads.} Head 2 reacts to short-term variations, while Head 1 maintains stable attention, capturing long-term patterns.}
        \label{fig:attention_heads_selected}
    \end{minipage}
\end{figure}

Table~\ref{tab:long_term_results} presents the results for prediction horizons of 2, 4, and 7 days. Unlike short-term reanalysis, where models estimate PM\textsubscript{2.5} concentrations step by step for each hour, long-term evaluations are conducted on a daily basis. Instead of predicting every hourly value, the goal is to assess whether the model can accurately estimate the overall pollution level for an entire day. This approach is more practical for extended reanalysis, as hourly fluctuations are less relevant when planning long-term air quality strategies. Therefore, the evaluation metrics in Table~\ref{tab:long_term_results} reflect the aggregated daily errors rather than step-by-step hourly deviations. For long-term estimation, AQ-Net retains the lowest RMSE across all horizons, effectively modeling extended dependencies. PatchTST performs strongly during the short-term periods, but suffers a sharp drop beyond two days, underscoring pure self-attention’s limitations for long-range reanalysis. Linear regression has the highest RMSE, reaffirming its inability to capture complex spatio-temporal dependencies. We can also observe that using Cyclic Encoding (CE) can improve the overall performance in all metrics.

Figure~\ref{fig:pm25_timeslot_comparison} illustrates the predicted and actual PM\textsubscript{2.5} levels over a one-week period in Beijing for two time slots: 9$\sim$12 PM and 4$\sim$7 PM. Our model effectively captures the overall temporal trends of PM\textsubscript{2.5} concentrations, with reanalysis generally following the fluctuations observed in real measurements. However, certain discrepancies are noticeable, particularly on Days 2 and 7, where morning predictions underestimate actual values, while on Day 4, afternoon predictions are slightly overestimated. These deviations suggest that while the model learns daily pollution patterns well, external factors such as meteorological changes or localized emission sources might not be fully accounted for. Notably, the model performs more consistently in the morning than in the afternoon, where greater variability is observed. 

\begin{figure}[t]
    \centering
    \begin{minipage}{0.48\textwidth}
        \centering
        \includegraphics[width=\textwidth]{figures/Attention.pdf}
        \caption{\textbf{Visualization of the attention heatmap across reanalysis and training days.} A diagonal trend suggests the model prioritizes recent observations, while deviations indicate potential long-term dependencies.}
        \label{fig:attention_heatmap}
    \end{minipage}
    \hfill
    \begin{minipage}{0.48\textwidth}
        \centering
        \includegraphics[width=\textwidth]{figures/BJ_v2.pdf}
        \caption{\textbf{Spatial interpolation of PM\textsubscript{2.5} in Beijing.}
        The PM\textsubscript{2.5} ranges from low to high (purple to yellow). $\bigcirc$ indicates stations used as input, while $\triangle$ represent predicted stations. 
    }
        \label{fig:pm25_spatial_distribution}
    \end{minipage}
\end{figure}

\subsection{Analysis of Temporal Attention Patterns}
We examine the model's temporal dependencies by analyzing MHA weights across different heads. In Figure~\ref{fig:attention_heads_selected}, Head 2 shows strong responsiveness to short-term fluctuations, while Head 1 maintains more stable weights, suggesting a focus on long-term trends. To identify and select these heads, we performed a PCA-based clustering of all attention heads, revealing that these two belong to distinct cluster: one emphasizing immediate variations (short-term) and the other capturing broader temporal structures (long-term). 

The global attention heatmap (Figure~\ref{fig:attention_heatmap}) shows how the model distributes attention between training days when reanalysis is performed. The x-axis represents input (historical) days (oldest $\rightarrow$ most recent), and the y-axis corresponds to output (reanalysis) days. The strong diagonal pattern indicates that the model prioritizes recent data, while some off-diagonal values suggest that it also captures longer-term dependencies. Darker areas attract more attention, highlighting the importance of recent pollution levels for accurate reanalysis.

\begin{figure}[t]
    \centering
    \includegraphics[width=\textwidth]{figures/mapB.pdf}
    \caption{\textbf{Daily mean PM\textsubscript{2.5} reanalysis over northern China.} Higher PM\textsubscript{2.5} is in yellow color. It highlights pollution hotspots in specific provinces. Overlapped markers indicate that multiple stations are located in very close proximity.}
    \label{fig:mapB}
\end{figure}

\subsection{Spatiotemporal reanalysis in Northern China}
Our proposed AQ-Net is able to interpolate the geographical trajectory given pollution data at known stations. Following the previous experiments in Beijing, Figure~\ref{fig:pm25_spatial_distribution} show that AQ-Net can use known stations ($\bigcirc$) to not only estimate the unknown stations ($\triangle$), but it can also estimate the global air pollution map for reanalysis. 

To illustrate the efficiency of our proposed AQ-Net on the large-scale dataset, we show spatiotemporal reanalysis in the entire northern China. As shown in Figure~\ref{fig:mapB}, utilizing the proposed neural kNN, we are able to estimate the complete spatial interpolation, capturing both observed and unobserved areas. Notably, pollution hotspots around northern and central Beijing are consistent with known urban emission sources. These results highlight the model’s ability to generalize beyond monitored stations, which is crucial for accurate city-wide air quality assessments.

\begin{figure}[t]
    \centering
    \includegraphics[width=\textwidth]{figures/rmsemae.pdf}
    \caption{\textbf{Visualization of prediction errors for hidden stations.} The bubbles indicate the RMSE or MAE. Both the color and size of the bubbles are proportional to the magnitude of the error: higher error values appear in warmer colors (yellow) and with larger circles.}
    \label{fig:rmsemae}
\end{figure}

Quantitative estimation on the spatiotemporal reanalysis is shown in Figure~\ref{fig:rmsemae}. Each circle represents the average estimation errors of predicted hidden stations in one city. We can see that our model can uniformly produce low MAE and RMSE on spatiotemporal interpolation. We also find that there are a few regions, like central China, are not well estimated. One of the reasons is that we do not have dense monitoring stations in those areas and the complex geographic and meteorological factors could have significant impacts.

\section{Conclusion}
 In this brief, we presented a 16nm reliable, time-predictable heterogeneous RISC-V SoC for \gls{ai}-enhanced mixed-criticality applications. To the best of our knowledge, this is the first SoC that combines safety features for \glspl{mcs} with hardware IPs for time-predictable execution of \glspl{mct} and leading-edge domain-specialized programmable accelerators within the same heterogeneous SoC. With a peak performance of 304.9 GOPS at 1.6TOPS/W and 260.7 GOPS/mm$^2$, and 121.8 GFLOPS at 1.1TFLOPS/W and 107GFLOPS/mm$^2$, the proposed SoC offers a comprehensive solution for reliable and deterministic execution of \gls{ai}/\gls{dsp}-enhanced \gls{mc} edge applications, achieving \gls{soa} energy efficiency under \looseness=-1  1.2~W power envelope. %The hardware description of the design is released open-source to foster future research~\footnote{\texttt{\url{https://github.com/pulp-platform/carfield}}}.
%
%To the best of our knowledge, this is the first SoC to combine safety-critical features for MCS with HW IPs for time-predictable execution of MCTs and leading edge domain-specialized programmable processor clusters into the same heterogeneous SoC, providing a comprehensive solution for reliable and deterministic execution of ML-enhanced mixed-criticality edge applications, at state-of-the-art energy efficiency and less than 1.5W power envelope. 

%\textcolor{red}{The platform and all the non-technological IPs integrated in Carfield are released open source under a liberal licence to foster future software and hardware research on reliable, time-predictable, accelerated heterogeneous computing devices for mixed-criticality applications.}



\begin{credits}

\subsubsection{Disclosure of Interests}
The authors have no competing interests to declare that are relevant to the content of this article.
% \subsubsection{\discintname}
% It is now necessary to declare any competing interests or to specifically
% state that the authors have no competing interests. Please place the
% statement with a bold run-in heading in small font size beneath the
% (optional) acknowledgments\footnote{If EquinOCS, our proceedings submission
% for example: The authors have no competing interests to declare that are
% relevant to the content of this article. Or: Author A has received research
% grants from Company W. Author B has received a speaker honorarium from
% Company X and owns stock in Company Y. Author C is a member of committee Z.
\end{credits}

%% The file named.bst is a bibliography style file for BibTeX 0.99c
\bibliographystyle{splncs04}
\bibliography{scia25}  % Assure-toi que le nom est EXACTEMENT celui de ton fichier .bib % ou un autre style comme IEEE, unsrt, etc.
\end{document}

