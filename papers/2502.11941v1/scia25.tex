\pdfoutput=1

\documentclass[runningheads]{llncs}
%
\usepackage[T1]{fontenc}
% T1 fonts will be used to generate the final print and online PDFs,
% so please use T1 fonts in your manuscript whenever possible.
% Other font encondings may result in incorrect characters.
%
\let\proof\relax
\let\endproof\relax
\usepackage{subcaption}
\usepackage{amssymb}
% Use the SI units package.
\usepackage{siunitx}
%% The amsthm package provides extended theorem environments
\usepackage{graphicx}
\usepackage{multirow}
\usepackage{amsmath,amssymb,amsfonts}
\usepackage{amsthm}
\usepackage{mathrsfs}
\usepackage[title]{appendix}
\usepackage{xcolor}
\usepackage{colortbl}
\usepackage{textcomp}
\usepackage{manyfoot}
\usepackage{booktabs}
\usepackage{algorithmicx}
\usepackage{listings}
\definecolor{mistyrose}{rgb}{1.0, 0.89, 0.88}
\newcommand{\liu}[1]{\textcolor{blue}{#1}}
% Used for displaying a sample figure. If possible, figure files should
% be included in EPS format.
%
% If you use the hyperref package, please uncomment the following two lines
% to display URLs in blue roman font according to Springer's eBook style:
%\usepackage{color}
%\renewcommand\UrlFont{\color{blue}\rmfamily}
%\urlstyle{rm}
%
\usepackage{graphicx} 
\usepackage{float}     
\begin{document}
\title{Deep Spatio-Temporal Neural Network for Air Quality Reanalysis}

\titlerunning{Deep Spatio-Temporal NN for Air Quality}
% If the paper title is too long for the running head, you can set
% an abbreviated paper title here

\author{Ammar Kheder$^{1,2}$ \and
Benjamin Foreback$^{2,3}$ \and
Lili Wang$^{4}$ \and
Zhi-Song Liu$^{1,2}$ \and 
Michael Boy$^{1,2,3}$}

\authorrunning{A. Kheder et al.}
% First names are abbreviated in the running head.
% If there are more than two authors, 'et al.' is used.

\institute{Lappeenranta–Lahti University of Technology LUT 
\and
Atmospheric Modelling Centre Lahti, Lahti University Campus
\and
University of Helsinki
\and
Chinese Academy of Sciences
}
\maketitle              % typeset the header of the contribution
%
\begin{abstract}
Air quality prediction is key to mitigating health impacts and guiding decisions, yet existing models tend to focus on temporal trends while overlooking spatial generalization. We propose AQ-Net, a spatiotemporal reanalysis model for both observed and unobserved stations in the near future. AQ-Net utilizes the LSTM and multi-head attention for the temporal regression. We also propose a cyclic encoding technique to ensure continuous time representation. To learn fine-grained spatial air quality estimation, we incorporate AQ-Net with the neural kNN to explore feature-based interpolation, such that we can fill the spatial gaps given coarse observation stations. To demonstrate the efficiency of our model for spatiotemporal reanalysis, we use data from 2013--2017 collected in northern China for PM\(_{2.5}\) analysis.  Extensive experiments show that AQ-Net excels in air quality reanalysis, highlighting the potential of hybrid spatio-temporal models to better capture environmental dynamics---especially in urban areas where both spatial and temporal variability are critical.


\end{abstract}

\keywords{Air Quality Reanalysis  \and Spatiotemporal Analysis \and Deep Learning\and Attention \and kNN.}


\section{Introduction}

% \textcolor{red}{Still on working}

% \textcolor{red}{add label for each section}


Robot learning relies on diverse and high-quality data to learn complex behaviors \cite{aldaco2024aloha, wang2024dexcap}.
Recent studies highlight that models trained on datasets with greater complexity and variation in the domain tend to generalize more effectively across broader scenarios \cite{mann2020language, radford2021learning, gao2024efficient}.
% However, creating such diverse datasets in the real world presents significant challenges.
% Modifying physical environments and adjusting robot hardware settings require considerable time, effort, and financial resources.
% In contrast, simulation environments offer a flexible and efficient alternative.
% Simulations allow for the creation and modification of digital environments with a wide range of object shapes, weights, materials, lighting, textures, friction coefficients, and so on to incorporate domain randomization,
% which helps improve the robustness of models when deployed in real-world conditions.
% These environments can be easily adjusted and reset, enabling faster iterations and data collection.
% Additionally, simulations provide the ability to consistently reproduce scenarios, which is essential for benchmarking and model evaluation.
% Another advantage of simulations is their flexibility in sensor integration. Sensors such as cameras, LiDARs, and tactile sensors can be added or repositioned without the physical limitations present in real-world setups. Simulations also eliminate the risk of damaging expensive hardware during edge-case experiments, making them an ideal platform for testing rare or dangerous scenarios that are impractical to explore in real life.
By leveraging immersive perspectives and interactions, Extended Reality\footnote{Extended Reality is an umbrella term to refer to Augmented Reality, Mixed Reality, and Virtual Reality \cite{wikipediaExtendedReality}}
(XR)
is a promising candidate for efficient and intuitive large scale data collection \cite{jiang2024comprehensive, arcade}
% With the demand for collecting data, XR provides a promising approach for humans to teach robots by offering users an immersive experience.
in simulation \cite{jiang2024comprehensive, arcade, dexhub-park} and real-world scenarios \cite{openteach, opentelevision}.
However, reusing and reproducing current XR approaches for robot data collection for new settings and scenarios is complicated and requires significant effort.
% are difficult to reuse and reproduce system makes it hard to reuse and reproduce in another data collection pipeline.
This bottleneck arises from three main limitations of current XR data collection and interaction frameworks: \textit{asset limitation}, \textit{simulator limitation}, and \textit{device limitation}.
% \textcolor{red}{ASSIGN THESE CITATION PROPERLY:}
% \textcolor{red}{list them by time order???}
% of collecting data by using XR have three main limitations.
Current approaches suffering from \textit{asset limitation} \cite{arclfd, jiang2024comprehensive, arcade, george2025openvr, vicarios}
% Firstly, recent works \cite{jiang2024comprehensive, arcade, dexhub-park}
can only use predefined robot models and task scenes. Configuring new tasks requires significant effort, since each new object or model must be specifically integrated into the XR application.
% and it takes too much effort to configure new tasks in their systems since they cannot spawn arbitrary models in the XR application.
The vast majority of application are developed for specific simulators or real-world scenarios. This \textit{simulator limitation} \cite{mosbach2022accelerating, lipton2017baxter, dexhub-park, arcade}
% Secondly, existing systems are limited to a single simulation platform or real-world scenarios.
significantly reduces reusability and makes adaptation to new simulation platforms challenging.
Additionally, most current XR frameworks are designed for a specific version of a single XR headset, leading to a \textit{device limitation} 
\cite{lipton2017baxter, armada, openteach, meng2023virtual}.
% and there is no work working on the extendability of transferring to a new headsets as far as we know.
To the best of our knowledge, no existing work has explored the extensibility or transferability of their framework to different headsets.
These limitations hamper reproducibility and broader contributions of XR based data collection and interaction to the research community.
% as each research group typically has its own data collection pipeline.
% In addition to these main limitations, existing XR systems are not well suited for managing multiple robot systems,
% as they are often designed for single-operator use.

In addition to these main limitations, existing XR systems are often designed for single-operator use, prohibiting collaborative data collection.
At the same time, controlling multiple robots at once can be very difficult for a single operator,
making data collection in multi-robot scenarios particularly challenging \cite{orun2019effect}.
Although there are some works using collaborative data collection in the context of tele-operation \cite{tung2021learning, Qin2023AnyTeleopAG},
there is no XR-based data collection system supporting collaborative data collection.
This limitation highlights the need for more advanced XR solutions that can better support multi-robot and multi-user scenarios.
% \textcolor{red}{more papers about collaborative data collection}

To address all of these issues, we propose \textbf{IRIS},
an \textbf{I}mmersive \textbf{R}obot \textbf{I}nteraction \textbf{S}ystem.
This general system supports various simulators, benchmarks and real-world scenarios.
It is easily extensible to new simulators and XR headsets.
IRIS achieves generalization across six dimensions:
% \begin{itemize}
%     \item \textit{Cross-scene} : diverse object models;
%     \item \textit{Cross-embodiment}: diverse robot models;
%     \item \textit{Cross-simulator}: 
%     \item \textit{Cross-reality}: fd
%     \item \textit{Cross-platform}: fd
%     \item \textit{Cross-users}: fd
% \end{itemize}
\textbf{Cross-Scene}, \textbf{Cross-Embodiment}, \textbf{Cross-Simulator}, \textbf{Cross-Reality}, \textbf{Cross-Platform}, and \textbf{Cross-User}.

\textbf{Cross-Scene} and \textbf{Cross-Embodiment} allow the system to handle arbitrary objects and robots in the simulation,
eliminating restrictions about predefined models in XR applications.
IRIS achieves these generalizations by introducing a unified scene specification, representing all objects,
including robots, as data structures with meshes, materials, and textures.
The unified scene specification is transmitted to the XR application to create and visualize an identical scene.
By treating robots as standard objects, the system simplifies XR integration,
allowing researchers to work with various robots without special robot-specific configurations.
\textbf{Cross-Simulator} ensures compatibility with various simulation engines.
IRIS simplifies adaptation by parsing simulated scenes into the unified scene specification, eliminating the need for XR application modifications when switching simulators.
New simulators can be integrated by creating a parser to convert their scenes into the unified format.
This flexibility is demonstrated by IRIS’ support for Mujoco \cite{todorov2012mujoco}, IsaacSim \cite{mittal2023orbit}, CoppeliaSim \cite{coppeliaSim}, and even the recent Genesis \cite{Genesis} simulator.
\textbf{Cross-Reality} enables the system to function seamlessly in both virtual simulations and real-world applications.
IRIS enables real-world data collection through camera-based point cloud visualization.
\textbf{Cross-Platform} allows for compatibility across various XR devices.
Since XR device APIs differ significantly, making a single codebase impractical, IRIS XR application decouples its modules to maximize code reuse.
This application, developed by Unity \cite{unity3dUnityManual}, separates scene visualization and interaction, allowing developers to integrate new headsets by reusing the visualization code and only implementing input handling for hand, head, and motion controller tracking.
IRIS provides an implementation of the XR application in the Unity framework, allowing for a straightforward deployment to any device that supports Unity. 
So far, IRIS was successfully deployed to the Meta Quest 3 and HoloLens 2.
Finally, the \textbf{Cross-User} ability allows multiple users to interact within a shared scene.
IRIS achieves this ability by introducing a protocol to establish the communication between multiple XR headsets and the simulation or real-world scenarios.
Additionally, IRIS leverages spatial anchors to support the alignment of virtual scenes from all deployed XR headsets.
% To make an seamless user experience for robot learning data collection,
% IRIS also tested in three different robot control interface
% Furthermore, to demonstrate the extensibility of our approach, we have implemented a robot-world pipeline for real robot data collection, ensuring that the system can be used in both simulated and real-world environments.
The Immersive Robot Interaction System makes the following contributions\\
\textbf{(1) A unified scene specification} that is compatible with multiple robot simulators. It enables various XR headsets to visualize and interact with simulated objects and robots, providing an immersive experience while ensuring straightforward reusability and reproducibility.\\
\textbf{(2) A collaborative data collection framework} designed for XR environments. The framework facilitates enhanced robot data acquisition.\\
\textbf{(3) A user study} demonstrating that IRIS significantly improves data collection efficiency and intuitiveness compared to the LIBERO baseline.

% \begin{table*}[t]
%     \centering
%     \begin{tabular}{lccccccc}
%         \toprule
%         & \makecell{Physical\\Interaction}
%         & \makecell{XR\\Enabled}
%         & \makecell{Free\\View}
%         & \makecell{Multiple\\Robots}
%         & \makecell{Robot\\Control}
%         % Force Feedback???
%         & \makecell{Soft Object\\Supported}
%         & \makecell{Collaborative\\Data} \\
%         \midrule
%         ARC-LfD \cite{arclfd}                              & Real        & \cmark & \xmark & \xmark & Joint              & \xmark & \xmark \\
%         DART \cite{dexhub-park}                            & Sim         & \cmark & \cmark & \cmark & Cartesian          & \xmark & \xmark \\
%         \citet{jiang2024comprehensive}                     & Sim         & \cmark & \xmark & \xmark & Joint \& Cartesian & \xmark & \xmark \\
%         \citet{mosbach2022accelerating}                    & Sim         & \cmark & \cmark & \xmark & Cartesian          & \xmark & \xmark \\
%         ARCADE \cite{arcade}                               & Real        & \cmark & \cmark & \xmark & Cartesian          & \xmark & \xmark \\
%         Holo-Dex \cite{holodex}                            & Real        & \cmark & \xmark & \cmark & Cartesian          & \cmark & \xmark \\
%         ARMADA \cite{armada}                               & Real        & \cmark & \xmark & \cmark & Cartesian          & \cmark & \xmark \\
%         Open-TeleVision \cite{opentelevision}              & Real        & \cmark & \cmark & \cmark & Cartesian          & \cmark & \xmark \\
%         OPEN TEACH \cite{openteach}                        & Real        & \cmark & \xmark & \cmark & Cartesian          & \cmark & \cmark \\
%         GELLO \cite{wu2023gello}                           & Real        & \xmark & \cmark & \cmark & Joint              & \cmark & \xmark \\
%         DexCap \cite{wang2024dexcap}                       & Real        & \xmark & \cmark & \xmark & Cartesian          & \cmark & \xmark \\
%         AnyTeleop \cite{Qin2023AnyTeleopAG}                & Real        & \xmark & \xmark & \cmark & Cartesian          & \cmark & \cmark \\
%         Vicarios \cite{vicarios}                           & Real        & \cmark & \xmark & \xmark & Cartesian          & \cmark & \xmark \\     
%         Augmented Visual Cues \cite{augmentedvisualcues}   & Real        & \cmark & \cmark & \xmark & Cartesian          & \xmark & \xmark \\ 
%         \citet{wang2024robotic}                            & Real        & \cmark & \cmark & \xmark & Cartesian          & \cmark & \xmark \\
%         Bunny-VisionPro \cite{bunnyvisionpro}              & Real        & \cmark & \cmark & \cmark & Cartesian          & \cmark & \xmark \\
%         IMMERTWIN \cite{immertwin}                         & Real        & \cmark & \cmark & \cmark & Cartesian          & \xmark & \xmark \\
%         \citet{meng2023virtual}                            & Sim \& Real & \cmark & \cmark & \xmark & Cartesian          & \xmark & \xmark \\
%         Shared Control Framework \cite{sharedctlframework} & Real        & \cmark & \cmark & \cmark & Cartesian          & \xmark & \xmark \\
%         OpenVR \cite{openvr}                               & Real        & \cmark & \cmark & \xmark & Cartesian          & \xmark & \xmark \\
%         \citet{digitaltwinmr}                              & Real        & \cmark & \cmark & \xmark & Cartesian          & \cmark & \xmark \\
        
%         \midrule
%         \textbf{Ours} & Sim \& Real & \cmark & \cmark & \cmark & Joint \& Cartesian  & \cmark & \cmark \\
%         \bottomrule
%     \end{tabular}
%     \caption{This is a cross-column table with automatic line breaking.}
%     \label{tab:cross-column}
% \end{table*}

% \begin{table*}[t]
%     \centering
%     \begin{tabular}{lccccccc}
%         \toprule
%         & \makecell{Cross-Embodiment}
%         & \makecell{Cross-Scene}
%         & \makecell{Cross-Simulator}
%         & \makecell{Cross-Reality}
%         & \makecell{Cross-Platform}
%         & \makecell{Cross-User} \\
%         \midrule
%         ARC-LfD \cite{arclfd}                              & \xmark & \xmark & \xmark & \xmark & \xmark & \xmark \\
%         DART \cite{dexhub-park}                            & \cmark & \cmark & \xmark & \xmark & \xmark & \xmark \\
%         \citet{jiang2024comprehensive}                     & \xmark & \cmark & \xmark & \xmark & \xmark & \xmark \\
%         \citet{mosbach2022accelerating}                    & \xmark & \cmark & \xmark & \xmark & \xmark & \xmark \\
%         ARCADE \cite{arcade}                               & \xmark & \xmark & \xmark & \xmark & \xmark & \xmark \\
%         Holo-Dex \cite{holodex}                            & \cmark & \xmark & \xmark & \xmark & \xmark & \xmark \\
%         ARMADA \cite{armada}                               & \cmark & \xmark & \xmark & \xmark & \xmark & \xmark \\
%         Open-TeleVision \cite{opentelevision}              & \cmark & \xmark & \xmark & \xmark & \cmark & \xmark \\
%         OPEN TEACH \cite{openteach}                        & \cmark & \xmark & \xmark & \xmark & \xmark & \cmark \\
%         GELLO \cite{wu2023gello}                           & \cmark & \xmark & \xmark & \xmark & \xmark & \xmark \\
%         DexCap \cite{wang2024dexcap}                       & \xmark & \xmark & \xmark & \xmark & \xmark & \xmark \\
%         AnyTeleop \cite{Qin2023AnyTeleopAG}                & \cmark & \cmark & \cmark & \cmark & \xmark & \cmark \\
%         Vicarios \cite{vicarios}                           & \xmark & \xmark & \xmark & \xmark & \xmark & \xmark \\     
%         Augmented Visual Cues \cite{augmentedvisualcues}   & \xmark & \xmark & \xmark & \xmark & \xmark & \xmark \\ 
%         \citet{wang2024robotic}                            & \xmark & \xmark & \xmark & \xmark & \xmark & \xmark \\
%         Bunny-VisionPro \cite{bunnyvisionpro}              & \cmark & \xmark & \xmark & \xmark & \xmark & \xmark \\
%         IMMERTWIN \cite{immertwin}                         & \cmark & \xmark & \xmark & \xmark & \xmark & \xmark \\
%         \citet{meng2023virtual}                            & \xmark & \cmark & \xmark & \cmark & \xmark & \xmark \\
%         \citet{sharedctlframework}                         & \cmark & \xmark & \xmark & \xmark & \xmark & \xmark \\
%         OpenVR \cite{george2025openvr}                               & \xmark & \xmark & \xmark & \xmark & \xmark & \xmark \\
%         \citet{digitaltwinmr}                              & \xmark & \xmark & \xmark & \xmark & \xmark & \xmark \\
        
%         \midrule
%         \textbf{Ours} & \cmark & \cmark & \cmark & \cmark & \cmark & \cmark \\
%         \bottomrule
%     \end{tabular}
%     \caption{This is a cross-column table with automatic line breaking.}
% \end{table*}

% \begin{table*}[t]
%     \centering
%     \begin{tabular}{lccccccc}
%         \toprule
%         & \makecell{Cross-Scene}
%         & \makecell{Cross-Embodiment}
%         & \makecell{Cross-Simulator}
%         & \makecell{Cross-Reality}
%         & \makecell{Cross-Platform}
%         & \makecell{Cross-User}
%         & \makecell{Control Space} \\
%         \midrule
%         % Vicarios \cite{vicarios}                           & \xmark & \xmark & \xmark & \xmark & \xmark & \xmark \\     
%         % Augmented Visual Cues \cite{augmentedvisualcues}   & \xmark & \xmark & \xmark & \xmark & \xmark & \xmark \\ 
%         % OpenVR \cite{george2025openvr}                     & \xmark & \xmark & \xmark & \xmark & \xmark & \xmark \\
%         \citet{digitaltwinmr}                              & \xmark & \xmark & \xmark & \xmark & \xmark & \xmark &  \\
%         ARC-LfD \cite{arclfd}                              & \xmark & \xmark & \xmark & \xmark & \xmark & \xmark &  \\
%         \citet{sharedctlframework}                         & \cmark & \xmark & \xmark & \xmark & \xmark & \xmark &  \\
%         \citet{jiang2024comprehensive}                     & \cmark & \xmark & \xmark & \xmark & \xmark & \xmark &  \\
%         \citet{mosbach2022accelerating}                    & \cmark & \xmark & \xmark & \xmark & \xmark & \xmark & \\
%         Holo-Dex \cite{holodex}                            & \cmark & \xmark & \xmark & \xmark & \xmark & \xmark & \\
%         ARCADE \cite{arcade}                               & \cmark & \cmark & \xmark & \xmark & \xmark & \xmark & \\
%         DART \cite{dexhub-park}                            & Limited & Limited & Mujoco & Sim & Vision Pro & \xmark &  Cartesian\\
%         ARMADA \cite{armada}                               & \cmark & \cmark & \xmark & \xmark & \xmark & \xmark & \\
%         \citet{meng2023virtual}                            & \cmark & \cmark & \xmark & \cmark & \xmark & \xmark & \\
%         % GELLO \cite{wu2023gello}                           & \cmark & \xmark & \xmark & \xmark & \xmark & \xmark \\
%         % DexCap \cite{wang2024dexcap}                       & \xmark & \xmark & \xmark & \xmark & \xmark & \xmark \\
%         % AnyTeleop \cite{Qin2023AnyTeleopAG}                & \cmark & \cmark & \cmark & \cmark & \xmark & \cmark \\
%         % \citet{wang2024robotic}                            & \xmark & \xmark & \xmark & \xmark & \xmark & \xmark \\
%         Bunny-VisionPro \cite{bunnyvisionpro}              & \cmark & \cmark & \xmark & \xmark & \xmark & \xmark & \\
%         IMMERTWIN \cite{immertwin}                         & \cmark & \cmark & \xmark & \xmark & \xmark & \xmark & \\
%         Open-TeleVision \cite{opentelevision}              & \cmark & \cmark & \xmark & \xmark & \cmark & \xmark & \\
%         \citet{szczurek2023multimodal}                     & \xmark & \xmark & \xmark & Real & \xmark & \cmark & \\
%         OPEN TEACH \cite{openteach}                        & \cmark & \cmark & \xmark & \xmark & \xmark & \cmark & \\
%         \midrule
%         \textbf{Ours} & \cmark & \cmark & \cmark & \cmark & \cmark & \cmark \\
%         \bottomrule
%     \end{tabular}
%     \caption{TODO, Bruce: this table can be further optimized.}
% \end{table*}

\definecolor{goodgreen}{HTML}{228833}
\definecolor{goodred}{HTML}{EE6677}
\definecolor{goodgray}{HTML}{BBBBBB}

\begin{table*}[t]
    \centering
    \begin{adjustbox}{max width=\textwidth}
    \renewcommand{\arraystretch}{1.2}    
    \begin{tabular}{lccccccc}
        \toprule
        & \makecell{Cross-Scene}
        & \makecell{Cross-Embodiment}
        & \makecell{Cross-Simulator}
        & \makecell{Cross-Reality}
        & \makecell{Cross-Platform}
        & \makecell{Cross-User}
        & \makecell{Control Space} \\
        \midrule
        % Vicarios \cite{vicarios}                           & \xmark & \xmark & \xmark & \xmark & \xmark & \xmark \\     
        % Augmented Visual Cues \cite{augmentedvisualcues}   & \xmark & \xmark & \xmark & \xmark & \xmark & \xmark \\ 
        % OpenVR \cite{george2025openvr}                     & \xmark & \xmark & \xmark & \xmark & \xmark & \xmark \\
        \citet{digitaltwinmr}                              & \textcolor{goodred}{Limited}     & \textcolor{goodred}{Single Robot} & \textcolor{goodred}{Unity}    & \textcolor{goodred}{Real}          & \textcolor{goodred}{Meta Quest 2} & \textcolor{goodgray}{N/A} & \textcolor{goodred}{Cartesian} \\
        ARC-LfD \cite{arclfd}                              & \textcolor{goodgray}{N/A}        & \textcolor{goodred}{Single Robot} & \textcolor{goodgray}{N/A}     & \textcolor{goodred}{Real}          & \textcolor{goodred}{HoloLens}     & \textcolor{goodgray}{N/A} & \textcolor{goodred}{Cartesian} \\
        \citet{sharedctlframework}                         & \textcolor{goodred}{Limited}     & \textcolor{goodred}{Single Robot} & \textcolor{goodgray}{N/A}     & \textcolor{goodred}{Real}          & \textcolor{goodred}{HTC Vive Pro} & \textcolor{goodgray}{N/A} & \textcolor{goodred}{Cartesian} \\
        \citet{jiang2024comprehensive}                     & \textcolor{goodred}{Limited}     & \textcolor{goodred}{Single Robot} & \textcolor{goodgray}{N/A}     & \textcolor{goodred}{Real}          & \textcolor{goodred}{HoloLens 2}   & \textcolor{goodgray}{N/A} & \textcolor{goodgreen}{Joint \& Cartesian} \\
        \citet{mosbach2022accelerating}                    & \textcolor{goodgreen}{Available} & \textcolor{goodred}{Single Robot} & \textcolor{goodred}{IsaacGym} & \textcolor{goodred}{Sim}           & \textcolor{goodred}{Vive}         & \textcolor{goodgray}{N/A} & \textcolor{goodgreen}{Joint \& Cartesian} \\
        Holo-Dex \cite{holodex}                            & \textcolor{goodgray}{N/A}        & \textcolor{goodred}{Single Robot} & \textcolor{goodgray}{N/A}     & \textcolor{goodred}{Real}          & \textcolor{goodred}{Meta Quest 2} & \textcolor{goodgray}{N/A} & \textcolor{goodred}{Joint} \\
        ARCADE \cite{arcade}                               & \textcolor{goodgray}{N/A}        & \textcolor{goodred}{Single Robot} & \textcolor{goodgray}{N/A}     & \textcolor{goodred}{Real}          & \textcolor{goodred}{HoloLens 2}   & \textcolor{goodgray}{N/A} & \textcolor{goodred}{Cartesian} \\
        DART \cite{dexhub-park}                            & \textcolor{goodred}{Limited}     & \textcolor{goodred}{Limited}      & \textcolor{goodred}{Mujoco}   & \textcolor{goodred}{Sim}           & \textcolor{goodred}{Vision Pro}   & \textcolor{goodgray}{N/A} & \textcolor{goodred}{Cartesian} \\
        ARMADA \cite{armada}                               & \textcolor{goodgray}{N/A}        & \textcolor{goodred}{Limited}      & \textcolor{goodgray}{N/A}     & \textcolor{goodred}{Real}          & \textcolor{goodred}{Vision Pro}   & \textcolor{goodgray}{N/A} & \textcolor{goodred}{Cartesian} \\
        \citet{meng2023virtual}                            & \textcolor{goodred}{Limited}     & \textcolor{goodred}{Single Robot} & \textcolor{goodred}{PhysX}   & \textcolor{goodgreen}{Sim \& Real} & \textcolor{goodred}{HoloLens 2}   & \textcolor{goodgray}{N/A} & \textcolor{goodred}{Cartesian} \\
        % GELLO \cite{wu2023gello}                           & \cmark & \xmark & \xmark & \xmark & \xmark & \xmark \\
        % DexCap \cite{wang2024dexcap}                       & \xmark & \xmark & \xmark & \xmark & \xmark & \xmark \\
        % AnyTeleop \cite{Qin2023AnyTeleopAG}                & \cmark & \cmark & \cmark & \cmark & \xmark & \cmark \\
        % \citet{wang2024robotic}                            & \xmark & \xmark & \xmark & \xmark & \xmark & \xmark \\
        Bunny-VisionPro \cite{bunnyvisionpro}              & \textcolor{goodgray}{N/A}        & \textcolor{goodred}{Single Robot} & \textcolor{goodgray}{N/A}     & \textcolor{goodred}{Real}          & \textcolor{goodred}{Vision Pro}   & \textcolor{goodgray}{N/A} & \textcolor{goodred}{Cartesian} \\
        IMMERTWIN \cite{immertwin}                         & \textcolor{goodgray}{N/A}        & \textcolor{goodred}{Limited}      & \textcolor{goodgray}{N/A}     & \textcolor{goodred}{Real}          & \textcolor{goodred}{HTC Vive}     & \textcolor{goodgray}{N/A} & \textcolor{goodred}{Cartesian} \\
        Open-TeleVision \cite{opentelevision}              & \textcolor{goodgray}{N/A}        & \textcolor{goodred}{Limited}      & \textcolor{goodgray}{N/A}     & \textcolor{goodred}{Real}          & \textcolor{goodgreen}{Meta Quest, Vision Pro} & \textcolor{goodgray}{N/A} & \textcolor{goodred}{Cartesian} \\
        \citet{szczurek2023multimodal}                     & \textcolor{goodgray}{N/A}        & \textcolor{goodred}{Limited}      & \textcolor{goodgray}{N/A}     & \textcolor{goodred}{Real}          & \textcolor{goodred}{HoloLens 2}   & \textcolor{goodgreen}{Available} & \textcolor{goodred}{Joint \& Cartesian} \\
        OPEN TEACH \cite{openteach}                        & \textcolor{goodgray}{N/A}        & \textcolor{goodgreen}{Available}  & \textcolor{goodgray}{N/A}     & \textcolor{goodred}{Real}          & \textcolor{goodred}{Meta Quest 3} & \textcolor{goodred}{N/A} & \textcolor{goodgreen}{Joint \& Cartesian} \\
        \midrule
        \textbf{Ours}                                      & \textcolor{goodgreen}{Available} & \textcolor{goodgreen}{Available}  & \textcolor{goodgreen}{Mujoco, CoppeliaSim, IsaacSim} & \textcolor{goodgreen}{Sim \& Real} & \textcolor{goodgreen}{Meta Quest 3, HoloLens 2} & \textcolor{goodgreen}{Available} & \textcolor{goodgreen}{Joint \& Cartesian} \\
        \bottomrule
        \end{tabular}
    \end{adjustbox}
    \caption{Comparison of XR-based system for robots. IRIS is compared with related works in different dimensions.}
\end{table*}


\section{Related Work}

In this section, we review research related to the importance and barriers to parental involvement; parental use of learning technologies; and the use of generative AI and robot in educational and parenting scenarios.

\subsection{Importance and Barriers to Parental Involvement}\label{sec-rw-2.1}

% 79 words
Early childhood is a critical period for predicting future success and well-being, with early education investments resulting in higher returns than later interventions \cite{duncan2007school, doyle2009investing}. Effective parental involvement fosters cognitive and social skills, especially in younger children \cite{blevins2016early, peck1992parent}. Parents are encouraged to prioritize home-based involvement to maximize their influence \cite{ma2016meta}, as their involvement has a greater impact on children's learning outcomes \cite{hoffner2002parents, fehrmann1987home, hill2004parent} within the family setting than partnerships with schools or communities \cite{ma2016meta, harris2008parents, fantuzzo2004multiple, sui1996effects}.

However, parents' involvement in their children's education is often constrained by practical challenges related to parents' \textit{skills}, \textit{time}, and \textit{energy}. The Hoover-Dempsey and Sandler (HDS) framework \cite{green2007parents} and the CAM framework \cite{ho2024s} both highlight these factors-- parents' perceived \textit{skills and knowledge} (capability), \textit{time} (availability), and \textit{motivation} (energy)--influence the extent of their engagement. For instance, a parent confident in math may choose to engage more in math-related tasks, while those facing inflexible schedules may participate less \cite{green2007parents}. Unlike teachers, parents often lack formal pedagogical training and may underestimate their role in supporting children's learning, particularly as young children struggle to articulate their needs \cite{hara1998parent}. The CAM framework similarly suggests parents delegate tasks to a robot when they feel less capable, have limited time, or are unmotivated. These factors reflect parents' life contexts, shaped by demographic backgrounds, occupations, and parenting responsibilities \cite{grolnick1997predictors}, highlighting the need to help parents overcome barriers to effective involvement in early education within their life contexts.

\subsection{Parental Use of Learning Technologies}\label{sec-rw-2.2}
% 207 words
Technology encourages parental involvement by facilitating parent-child engagement in learning activities while introducing risks that require active parental mediation \cite{gonzalez2022parental}. On the positive side, technology offers novel opportunities for parental engagement and enhances children's learning outcomes. For example, e-books promote interactive behaviors between parents and children better than print books \cite{korat2010new}. In addition, having access to computers at home significantly boost academic achievement of young children when parents actively mediate their use \cite{hofferth2010home, espinosa2006technology}. However, the effectiveness of these tools often depends on parents' familiarity with and attitude toward technology. Mobile applications, for instance, can improve learning outcomes but require parents to possess sufficient technology efficacy to guide their use \cite{papadakis2019parental}.

On the negative side, technology introduces risks such as excessive screen time, exposure to inappropriate content, and misinformation, which necessitate parental intervention \cite{oswald2020psychological, howard2021digital}. According to parental mediation theory, parents mitigate these risks through restrictive mediation (e.g., setting limits), active mediation (e.g., discussing content), and co-use (e.g., shared use of technology) \cite{valkenburg1999developing}. Modern technologies like video games, location-based games (\textit{e.g.,} Pokemon Go), and conversational agents (\textit{e.g.,} Alexa) also require parents to adapt their mediation strategies to ensure responsible use \cite{valkenburg1999developing, nikken2006parental, sobel2017wasn, beneteau2020parenting, yu2024parent}. Overall, parents seek to leverage technology to support their children's learning due to ite effectivenss but are also mindful of its risks. Their involvement is therefore driven by both opportunities and concerns, highlighting the need to design tools that effectively involve parents to balance benefits and risks.

\subsection{Generative AI and Companion Robots for Parenting and Education}
Generative AI and companion robots offer human-like affordances, with AI simulating human intelligence and robots providing physical human-like features. Compared to conventional models (\textit{e.g.,} machine learning) and devices (\textit{e.g.,} laptops), these emerging technologies enable natural and social interactions, creating opportunities for novel paradigms to enhance parental involvement and children's learning while introducing their unique challenges.

\subsubsection{Generative AI}
GAI offers promising support for parents by enhancing their ability to educate and engage with their children. Prior work suggested that AI-driven systems can support parenting education \cite{petsolari2024socio} and provide evidence-based advice through applications and chatbots, delivering micro-interventions such as teaching parents how to praise their children effectively \cite{davis2017parent, entenberg2023user} or offering strategies to teach complex concepts \cite{mogavi2024chatgpt, su2023unlocking}. Many parents also prefer using GAI to create educational materials tailored to their children's needs, rather than granting children direct access to these tools \cite{han2023design}. Beyond educational support, AI-based storytelling tools address practical challenges (\textit{e.g.,} time constraints) by alleviating physical labor while fostering parent-child interactions \cite{sun2024exploring}. Furthermore, GAI offers advantages to children's learning directly. It can help create personalized learning experiences by providing timely feedback and tailoring content \cite{su2023unlocking, mogavi2024chatgpt, han2024teachers}, enhancing positive learning experiences \cite{jauhiainen2023generative}. For example, a LLM-driven conversational system can teach children mathematical concepts through co-creative storytelling, achieving learning outcomes similar to human-led instructions \cite{zhang2024mathemyths}.

Despite these benefits, several concerns persist regarding the use of GAI in education. Prior work highlighted the limitations of GAI, such as its limited effectiveness in more complex learning tasks,the limited quality of the training data, and its inability to offer comprehensive educational support \cite{su2023unlocking}. There is also a significant risk of GAI producing inaccurate or biased information, discouraging independent thought among children, and threatening user privacy \cite{su2023unlocking, han2023design, han2024teachers}. Many parents are skeptical about the role of AI in their children's academic processes, concerned about the accuracy of AI-generated content, and worry that over-reliance on AI could stifle independent thinking \cite{han2023design}.

%\todo{might need to add some structural transition here}
\subsubsection{Social companion robots}
Social companion robots have proven potential to assist parents in home education settings through studies in \textit{parent-child-robot} interactions. \citet{gvirsman2020patricc} showed that the robotic system, \texttt{Patricc}, fostered more triadic interaction between parents and toddlers than a tablet, and \citet{gvirsman2024effect} found that, in a parent-toddler-robot interaction, parents tend to decrease their scaffolding affectively when the robot increases its scaffolding behavior. Similarly, \citet{chen2022designing} found that social robots enhanced parent-child co-reading activities, while \citet{chan2017wakey} demonstrated that the WAKEY robot improved morning routines and reduced parental frustration. Beyond educational support, \citet{ho2024s} uncovered that parents envisioned robots as their \textit{collaborators} to support their children's learning at home and that their collaboration patterns can be determined by the parents' capability, availability, and motivation. Although parents generally have positive attitudes toward incorporating robots into their children's learning, they remain concerned about the risk of disrupting school-based learning and potential teacher replacement \cite{tolksdorf2020parents, lee2008elementary, louie2021desire}.

In addition to parental support, social companion robots also support children in education directly through \textit{child-robot interactions}. Physically embodied robots provide adaptive assistance and verbal interaction similar to virtual or conversational agents \cite{ramachandran2019personalized, leyzberg2014personalizing, schodde2017adaptive, brown2014positive}, yet they foster greater engagement with the physical environment and encourage more advanced social behaviors during learning \cite{belpaeme2018social}, leading to improved learning outcomes \cite{leyzberg2012physical}. Prior work demonstrated that companion robots can effectively support both school-based learning (\textit{e.g.,} math \cite{lopez2018robotic}, literacy \cite{kennedy2016social, gordon2016affective}, and science \cite{davison2020working}) and home-based learning activities (\textit{e.g.,} reading \cite{michaelis2018reading, michaelis2019supporting}, number board games \cite{ho2021robomath}, and math-oriented conversations with parents \cite{ho2023designing}). For example, \citet{kennedy2016social} suggested that children can learn elements of a second language from a robot in short-term interactions, and \citet{tanaka2009use} found that children who took on the role of teaching the robot gained confidence and improved learning outcomes.

%\todo{may need to make this a separate section and explain why we propose AI-assisted robots}

% \subsubsection{Research Gap}
Parental involvement in early education is crucial and AI-assisted robots can offer promising support by helping parents overcome practical barriers (\textit{i.e.,} time, energy, and skills) and addressing concerns about technological risks. Yet, limited research has examined how technology design can simultaneously alleviate these barriers and concerns. Though \citet{zhang2022storybuddy} emphasized the importance of flexible parental involvement during reading through a system called \textit{Storybuddy}, yet they focused on a virtual chatbot rather than a physical robot, and how the flexible modes may be used in different scenarios remain unknown. Similarly, \textit{ContextQ} \cite{dietz2024contextq} presented auto-generated dialogic questions to caregivers for dialogic reading, but primarily considered situations where parents are actively involved, not scenarios where parents cannot participate fully.

In this work, we address these gaps by exploring parental involvement contexts, understanding parents' perceptions of AI-generated content, and examining how parents collaborate with AI and robots under different scenarios. In the following sections, we describe our development of  \texttt{SET}, a card-based activity, to understand parental involvement contexts (Section~\ref{sec-card}), the design of the \texttt{PAiREd} system to enable parents to co-create learning activities with an LLM (Section~\ref{sec-system}), and user study aimed to discover use patterns and understand user perceptions of the system (Section~\ref{sec-study}).
\section{Approach}

\begin{figure}[t]
\centering
\includegraphics[width=\textwidth]{figures/archi.pdf}
\caption{\textbf{Overview of the proposed AQ-Net.} 
The input includes historical pollutant concentrations,  and visible station coordinates. An LSTM extracts temporal dependencies, enhanced by Multi-Head Attention to highlight critical time steps. After temporal pooling, a neural kNN module performs spatial interpolation for unobserved stations (red markers).}
\label{fig:overall_architecture}
\end{figure}

The air quality data used as input is from a network of real-time air quality sensors throughout China, which are managed by the Ministry of Environmental Protection (MEP) and published hourly by the China Environmental Monitoring Centre (CEMC) \cite{SONG2017334}. This network has been collecting measurements since 2013, with the goal to study and predict air quality issues throughout China. By 2014, there were 944 air quality monitoring stations in 190 cities, and currently, there are over 2100 stations throughout China. The air quality parameters measured by this network are PM$_{2.5}$ and PM$_{10}$, NO$_\mathrm{x}$, SO$_2$, O$_3$ and CO. The dataset has been quality controlled with the algorithm described by Wu et al.\cite{Wu2018}. This study uses data from 584 stations in the metropolitan area of northern China from 2013 to 2017.

\subsection{Overall Architecture}
Our model predicts PM\(_{2.5}\) by combining temporal and spatial dependencies. AQ-Net comprises three core components: an LSTM-MHA module, combining LSTM and multi-head attention for temporal feature extraction, a neural kNN module for spatial interpolation, and a Cyclic Encoding (CE) layer for time embedding. We use hourly measurements of PM\(_{2.5}\), PM\textsubscript{10}, CO, NO\textsubscript{2}, SO\textsubscript{2}, and O\textsubscript{3} from monitoring stations, and estimate PM\(_{2.5}\) for the coming hours and days. The LSTM captures long-range pollutant fluctuations, while multi-head attention highlights critical time steps. A temporal pooling step condenses the latent sequence into a single feature vector, which the neural kNN module uses for spatial interpolation at unobserved stations based on their nearest neighbors. This integrated architecture generates 168-hour (7-day) PM\(_{2.5}\) estimation for both observed and unobserved locations, leveraging key temporal patterns and spatial relationships.

\subsection{Proposed Modules}
Air quality reanalysis is formulated as a spatiotemporal reanalysis problem. The input data consists of historical pollutant concentrations along with their time steps and geospatial information from monitoring stations. 

\noindent\textbf{Cyclic Encoding (CE) for Temporal Features}
To preserve the periodic nature of time-related features, we apply a cyclic encoding technique to the time step $t$ using sine and cosine transformations. This approach ensures a continuous representation, preventing discontinuities between values such as 23:00 and 00:00. The encoding is defined as $x_{\sin} = \sin \left( \frac{2\pi t}{\text{cycle}} \right), x_{\cos} = \cos \left( \frac{2\pi t}{\text{cycle}} \right)$, where \( t \) represents a temporal feature (e.g., hour, day, or month), and cycle corresponds to its periodicity (e.g., 24 for hours, 7 for days, and 12 for months).

\noindent\textbf{Long Short-Term Memory (LSTM):} To capture temporal dependencies, an LSTM network processes the time series data of pollutant concentrations. Given an input sequence $X \in \mathbb{R}^{C \times T \times N}$, where C is the number of features, T is the sequence length (number of time steps), and N is the number of stations, the LSTM generates a temporal representation $Z \in \mathbb{R}^{\text{dim} \times T \times N}$, where \(\text{dim}\) represents the hidden state dimension. The LSTM captures long-term dependencies, allowing the model to learn pollutant trends over time.

\noindent\textbf{Multi-Head Attention (MHA):} Although LSTM effectively captures sequential dependencies, it treats all past observations equally at each time step. To improve performance, we integrate a Multi-Head Attention (MHA) mechanism that enhances temporal dependencies by selectively weighting relevant time steps. The attention mechanism is computed as follows:

\begin{small}
\begin{equation}
Z' = \text{softmax} \left( \frac{Q_zK_z^T}{\sqrt{d_k}} \right) V_z + Z
\label{eq:fm_1}
\end{equation}
\end{small}

\noindent where $Q_z, K_z, V_z$ are linear transformations of $Z$ and $d_k$ is a scaling factor. This mechanism enables the model to focus on important time intervals, improving its ability to recognize patterns in pollutant fluctuations.

\noindent\textbf{Spatial Interpolation via kNN}
While the LSTM-MHA module captures temporal trends, it does not account for spatial correlations between monitoring stations. To estimate PM\(_{2.5}\) values at unobserved stations, we employ a kNN-based interpolation method. Instead of using the raw PM\(_{2.5}\) values from observed stations, we first extract a ``station-wise feature vector'' from the refined temporal representation $Z'$ using temporal pooling $Z' \in \mathbb{R}^{C\times \text{dim} \times N}$. This feature vector encapsulates the learned temporal patterns of each station, rather than just the raw measurements. Given a set of observed stations with known PM\(_{2.5}\) values, missing values at unobserved stations are estimated as, $Y = h(Z', p, k)$, where $p$ represents the geospatial coordinates of the stations, and $h(\cdot)$ applies kNN-weighted interpolation. The number of neighborhoods $k$ is defined on the fly such that we obtain multiple estimations. To speed up the computation of the station-to-station distance, we utilize GPU-enabled kNN query to ensure gradient backpropagation and fast searching. The kNN module finds the nearest stations in the learned feature space and interpolates the missing values accordingly, ensuring that spatial dependencies are taken into account.

\section{Experiments}

\subsection{Dataset and model architectures}
We use real-world data from 584  monitoring stations collected between 2013 and 2017. The dataset contains hourly measurements of CO, NO\textsubscript{2}, O\textsubscript{3}, PM\textsubscript{10}, PM\textsubscript{2.5}, and SO\textsubscript{2}. Stations with incomplete time series are removed, and all features are normalized into [0, 1]. We train AQ-Net using AdamW optimizer with the learning rate of $1\times10^{-3}$. We use the MSE as the loss function to estimate the network output and ground truth values for backpropagation. The batch size is set to 32 and AQ-Net is trained for {468K iterations (about 2 hours) on the CSC server\footnote{https://csc.fi/} with one NVIDIA A100 GPU using PyTorch deep learning platform.} The k value for Neural kNN is 20.
The code can be found at \liu{\url{https://github.com/AmmarKheder/AQ-Net}}.

\subsection{Overall Performance Comparison}
To evaluate the accuracy of the reanalysis, three key metrics are considered: Mean Absolute Error (MAE), Root Mean Squared Error (RMSE), and the coefficient of determination ($R^2$). Each of these metrics provides valuable insight into the performance of the models:

\begin{small}
\begin{equation}
\text{MAE} = \frac{1}{n} \sum_{i=1}^{n} |y_i - \hat{y}_i|, \quad
        \text{RMSE} = \sqrt{\frac{1}{n} \sum_{i=1}^{n} (y_i - \hat{y}_i)^2}, \quad
        R^2 = 1 - \frac{\sum_{i=1}^{n} (y_i - \hat{y}_i)^2}{\sum_{i=1}^{n} (y_i - \bar{y})^2}
\label{eq:eval}
\end{equation}
\end{small}

\noindent We compare our proposed AQ-Net with three approaches: PatchTST~\cite{nie2022patchtst} (a Transformer tailored for time series processing), Linear Regression, and LSTM~\cite{shi2015convolutional}. Our model can be used for temporal prediction at the same monitoring stations, it can also provide spatiotemporal prediction at unseen monitor stations. 

\subsubsection{Short-term temporal reanalysis (24-Hour Input Window)}
Table~\ref{tab:short_term} shows short-term estimated PM\textsubscript{2.5} concentrations in Beijing over the next few hours based on a 24-hour historical input. These reanalysis are critical for real-time air quality monitoring, health alerts, and short-term pollution control measures.

\begin{table}[t]
\centering
\caption{\textbf{The evaluation of short-term PM\textsubscript{2.5} reanalysis.} The table presents PM\textsubscript{2.5} reanalysis performance based on $R^2$, MAE, and RMSE over 6, 12, and 24 hours using a 24-hour historical input.}
\label{tab:short_term_results}
\renewcommand{\arraystretch}{1.2} % Better spacing for readability
\begin{tabular}{c|ccc|ccc|ccc}
\hline
\multirow{2}{*}{Model} & \multicolumn{3}{c|}{6h reanalysis} & \multicolumn{3}{c|}{12h reanalysis} & \multicolumn{3}{c}{24h reanalysis} \\
                        & $R^2$$\uparrow$ & MAE$\downarrow$  & RMSE$\downarrow$  & $R^2$ & MAE$\downarrow$  & RMSE$\downarrow$  & $R^2$ & MAE$\downarrow$  & RMSE$\downarrow$  \\ \hline
\cellcolor{mistyrose}{AQ-Net}                  & \cellcolor{mistyrose}{\textbf{0.5103}} & \cellcolor{mistyrose}{\textbf{18.71}} & \cellcolor{mistyrose}{\textbf{22.87}}  & \cellcolor{mistyrose}{\textbf{0.4118}} & \cellcolor{mistyrose}{\textbf{22.04}} & \cellcolor{mistyrose}{\textbf{29.10}}  & \cellcolor{mistyrose}{\textbf{0.1894}} & \cellcolor{mistyrose}{\textbf{26.18}} & \cellcolor{mistyrose}{\textbf{33.34}}  \\
AQ-Net wo CE          &        0.4031    &      21.21      &  29.23          &  0.2312          &   25.32         &       30.23  &0.1231 & 27.21 & 34.48   \\
PatchTST                & 0.4421 & 21.65 & 27.52  & 0.3319 & 23.57 & 31.50  & 0.1601 & 27.65 & 34.08  \\
LSTM                    & 0.4648 & 20.05 & 26.44  & 0.2336 & 25.40 & 32.44  & 0.1001 & 28.38 & 35.13  \\
Linear Regression       & 0.4500 & 20.80 & 27.00  & 0.2100 & 26.00 & 33.00  & 0.0800 & 29.00 & 35.80  \\ \hline
\end{tabular}
\label{tab:short_term}
\end{table}

Table~\ref{tab:short_term_results} presents the evaluation of the performance of four models (AQ-Net, PatchTST, LSTM, and Linear Regression) for predicting PM\textsubscript{2.5} concentrations in Beijing over 6, 12, and 24 hours using a 24-hour historical input. We also have AQ-Net wo CE to represent our approach without using the proposed cyclic encoding approach. For the 6-hour estimation, AQ-Net achieves the best result with an \(R^2\) of 0.51, an MAE of 18.71, and an RMSE of 22.87, demonstrating its ability to effectively capture rapid fluctuations in pollution. Although PatchTST employs a self-attention mechanism, it underperforms slightly with an $R^2$ of 0.44, an MAE of 21.65, and an RMSE of 27.52, while the LSTM and Linear Regression models show comparable results with $R^2$ values of 0.46 and 0.45, respectively, and marginally higher error metrics. As the prediction horizon extends to 12 hours, the performance of all models deteriorates; however, AQ-Net maintains a significant lead with an $R^2$ of 0.41, whereas the other models drop to $R^2$ values of 0.33 for PatchTST, 0.23 for LSTM, and 0.21 for Linear Regression. This trend continues for the 24-hour estimation, where AQ-Net achieves an \(R^2\) of 0.19 compared to 0.16, 0.10, and 0.08 for PatchTST, LSTM, and Linear Regression, respectively. These results indicate that AQ-Net is particularly robust and effective for short-term estimation, while the competing models, especially PatchTST, LSTM, and Linear Regression, struggle to maintain their accuracy as the reanalysis horizon increases. Comparing AQ-Net and AQ-Net wo CE, we can also see that using Cyclic encoding can improve $R^2$ by 0.06$\sim$0.17 in 6$\sim$24 h reanalysis, which demonstrates its efficiency.

\subsubsection{Long-Term temporal reanalysis (336-Hour Input Window)}
Table~\ref{tab:long_term_results} shows long-term reanalysis in Beijing and analyzes how well models estimate PM\textsubscript{2.5} levels over extended periods based on a 2-week (336-hour) historical window. 
\begin{table}[t]
\centering
\caption{\textbf{The evaluation of long-term PM\textsubscript{2.5} reanalysis.} The table presents long-term reanalysis performance on MAE and RMSE over 2-day, 4-day, and 1-week horizons, using a 2-week (336-hour) historical input.}
\resizebox{\textwidth}{!}{%
  \begin{tabular}{c|cc|cc|cc}
  \hline
  \multirow{2}{*}{Model} & \multicolumn{2}{c|}{2-Day reanalysis} & \multicolumn{2}{c|}{4-Day reanalysis} & \multicolumn{2}{c}{1-Week reanalysis} \\
                       & MAE$\downarrow$  & RMSE$\downarrow$  & MAE$\downarrow$  & RMSE$\downarrow$  & MAE$\downarrow$  & RMSE$\downarrow$  \\ \hline
  \cellcolor{mistyrose}{AQ-Net}            & \cellcolor{mistyrose}{\textbf{13.57}}  & \cellcolor{mistyrose}{\textbf{16.80}}  & \cellcolor{mistyrose}{\textbf{17.44}}  & \cellcolor{mistyrose}{\textbf{21.29}}  & \cellcolor{mistyrose}{\textbf{21.29}}  & \cellcolor{mistyrose}{\textbf{25.17}}  \\
  AQ-Net wo CE          &   17.12         &     21.77       &        18.12   &       24.63     &       24.23    &       28.37     \\
  PatchTST          & 41.42           & 55.64           & 35.31           & 39.22           & 28.01           & 34.70           \\
  LSTM              & 24.04           & 28.37           & 25.11           & 31.21           & 22.87           & 28.81           \\
  Linear Regression & 25.00           & 29.00           & 26.00           & 32.00           & 23.50           & 29.50           \\ \hline
  \end{tabular}
}
\label{tab:long_term_results}
\end{table}

\begin{figure}[t]
    \centering
    \begin{minipage}{0.48\textwidth}
        \centering
        \includegraphics[width=\textwidth]{figures/7days.pdf}
        \caption{\textbf{Comparison of PM\textsubscript{2.5} reanalysis for different time slots over seven days.} The 4PM-7PM period exhibits greater variability, suggesting increased pollution activity during the late afternoon.}
        \label{fig:pm25_timeslot_comparison}
    \end{minipage}
    \hfill
    \begin{minipage}{0.48\textwidth}
        \centering
        \includegraphics[width=\textwidth]{figures/attention_evolution_head.pdf}
        \caption{\textbf{Visualization of the evolution of attention weights for selected two heads.} Head 2 reacts to short-term variations, while Head 1 maintains stable attention, capturing long-term patterns.}
        \label{fig:attention_heads_selected}
    \end{minipage}
\end{figure}

Table~\ref{tab:long_term_results} presents the results for prediction horizons of 2, 4, and 7 days. Unlike short-term reanalysis, where models estimate PM\textsubscript{2.5} concentrations step by step for each hour, long-term evaluations are conducted on a daily basis. Instead of predicting every hourly value, the goal is to assess whether the model can accurately estimate the overall pollution level for an entire day. This approach is more practical for extended reanalysis, as hourly fluctuations are less relevant when planning long-term air quality strategies. Therefore, the evaluation metrics in Table~\ref{tab:long_term_results} reflect the aggregated daily errors rather than step-by-step hourly deviations. For long-term estimation, AQ-Net retains the lowest RMSE across all horizons, effectively modeling extended dependencies. PatchTST performs strongly during the short-term periods, but suffers a sharp drop beyond two days, underscoring pure self-attention’s limitations for long-range reanalysis. Linear regression has the highest RMSE, reaffirming its inability to capture complex spatio-temporal dependencies. We can also observe that using Cyclic Encoding (CE) can improve the overall performance in all metrics.

Figure~\ref{fig:pm25_timeslot_comparison} illustrates the predicted and actual PM\textsubscript{2.5} levels over a one-week period in Beijing for two time slots: 9$\sim$12 PM and 4$\sim$7 PM. Our model effectively captures the overall temporal trends of PM\textsubscript{2.5} concentrations, with reanalysis generally following the fluctuations observed in real measurements. However, certain discrepancies are noticeable, particularly on Days 2 and 7, where morning predictions underestimate actual values, while on Day 4, afternoon predictions are slightly overestimated. These deviations suggest that while the model learns daily pollution patterns well, external factors such as meteorological changes or localized emission sources might not be fully accounted for. Notably, the model performs more consistently in the morning than in the afternoon, where greater variability is observed. 

\begin{figure}[t]
    \centering
    \begin{minipage}{0.48\textwidth}
        \centering
        \includegraphics[width=\textwidth]{figures/Attention.pdf}
        \caption{\textbf{Visualization of the attention heatmap across reanalysis and training days.} A diagonal trend suggests the model prioritizes recent observations, while deviations indicate potential long-term dependencies.}
        \label{fig:attention_heatmap}
    \end{minipage}
    \hfill
    \begin{minipage}{0.48\textwidth}
        \centering
        \includegraphics[width=\textwidth]{figures/BJ_v2.pdf}
        \caption{\textbf{Spatial interpolation of PM\textsubscript{2.5} in Beijing.}
        The PM\textsubscript{2.5} ranges from low to high (purple to yellow). $\bigcirc$ indicates stations used as input, while $\triangle$ represent predicted stations. 
    }
        \label{fig:pm25_spatial_distribution}
    \end{minipage}
\end{figure}

\subsection{Analysis of Temporal Attention Patterns}
We examine the model's temporal dependencies by analyzing MHA weights across different heads. In Figure~\ref{fig:attention_heads_selected}, Head 2 shows strong responsiveness to short-term fluctuations, while Head 1 maintains more stable weights, suggesting a focus on long-term trends. To identify and select these heads, we performed a PCA-based clustering of all attention heads, revealing that these two belong to distinct cluster: one emphasizing immediate variations (short-term) and the other capturing broader temporal structures (long-term). 

The global attention heatmap (Figure~\ref{fig:attention_heatmap}) shows how the model distributes attention between training days when reanalysis is performed. The x-axis represents input (historical) days (oldest $\rightarrow$ most recent), and the y-axis corresponds to output (reanalysis) days. The strong diagonal pattern indicates that the model prioritizes recent data, while some off-diagonal values suggest that it also captures longer-term dependencies. Darker areas attract more attention, highlighting the importance of recent pollution levels for accurate reanalysis.

\begin{figure}[t]
    \centering
    \includegraphics[width=\textwidth]{figures/mapB.pdf}
    \caption{\textbf{Daily mean PM\textsubscript{2.5} reanalysis over northern China.} Higher PM\textsubscript{2.5} is in yellow color. It highlights pollution hotspots in specific provinces. Overlapped markers indicate that multiple stations are located in very close proximity.}
    \label{fig:mapB}
\end{figure}

\subsection{Spatiotemporal reanalysis in Northern China}
Our proposed AQ-Net is able to interpolate the geographical trajectory given pollution data at known stations. Following the previous experiments in Beijing, Figure~\ref{fig:pm25_spatial_distribution} show that AQ-Net can use known stations ($\bigcirc$) to not only estimate the unknown stations ($\triangle$), but it can also estimate the global air pollution map for reanalysis. 

To illustrate the efficiency of our proposed AQ-Net on the large-scale dataset, we show spatiotemporal reanalysis in the entire northern China. As shown in Figure~\ref{fig:mapB}, utilizing the proposed neural kNN, we are able to estimate the complete spatial interpolation, capturing both observed and unobserved areas. Notably, pollution hotspots around northern and central Beijing are consistent with known urban emission sources. These results highlight the model’s ability to generalize beyond monitored stations, which is crucial for accurate city-wide air quality assessments.

\begin{figure}[t]
    \centering
    \includegraphics[width=\textwidth]{figures/rmsemae.pdf}
    \caption{\textbf{Visualization of prediction errors for hidden stations.} The bubbles indicate the RMSE or MAE. Both the color and size of the bubbles are proportional to the magnitude of the error: higher error values appear in warmer colors (yellow) and with larger circles.}
    \label{fig:rmsemae}
\end{figure}

Quantitative estimation on the spatiotemporal reanalysis is shown in Figure~\ref{fig:rmsemae}. Each circle represents the average estimation errors of predicted hidden stations in one city. We can see that our model can uniformly produce low MAE and RMSE on spatiotemporal interpolation. We also find that there are a few regions, like central China, are not well estimated. One of the reasons is that we do not have dense monitoring stations in those areas and the complex geographic and meteorological factors could have significant impacts.

\section{Conclusion}
We introduce the Difficulty and Uncertainty-Aware Lightweight (DUAL) score, a novel scoring metric designed for cost-effective pruning. The DUAL score is the first metric to integrate both difficulty and uncertainty into a single measure, and its effectiveness in identifying the most informative samples early in training is further supported by theoretical analysis. Additionally, we propose pruning-ratio-adaptive sampling to account for sample diversity, particularly when the pruning ratio is extremely high. Our proposed pruning methods, DUAL score and DUAL score combined with Beta sampling demonstrate remarkable performance, particularly in realistic scenarios involving label noise and image corruption, by effectively distinguishing noisy samples.

Data pruning research has been evolving in a direction that contradicts its primary objective of reducing computational and storage costs while improving training efficiency. This is mainly because the computational cost of pruning often exceeds that of full training. By introducing our DUAL method, we take a crucial step toward overcoming this challenge by significantly reducing the computation cost associated with data pruning, making it feasible for practical scenarios. Ultimately, we believe this will help minimize resource waste and enhance training efficiency.


\begin{credits}

\subsubsection{Disclosure of Interests}
The authors have no competing interests to declare that are relevant to the content of this article.
% \subsubsection{\discintname}
% It is now necessary to declare any competing interests or to specifically
% state that the authors have no competing interests. Please place the
% statement with a bold run-in heading in small font size beneath the
% (optional) acknowledgments\footnote{If EquinOCS, our proceedings submission
% for example: The authors have no competing interests to declare that are
% relevant to the content of this article. Or: Author A has received research
% grants from Company W. Author B has received a speaker honorarium from
% Company X and owns stock in Company Y. Author C is a member of committee Z.
\end{credits}

%% The file named.bst is a bibliography style file for BibTeX 0.99c
\bibliographystyle{splncs04}
\bibliography{scia25}  % Assure-toi que le nom est EXACTEMENT celui de ton fichier .bib % ou un autre style comme IEEE, unsrt, etc.
\end{document}

