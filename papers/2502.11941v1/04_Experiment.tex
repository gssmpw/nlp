\section{Experiments}

\subsection{Dataset and model architectures}
We use real-world data from 584  monitoring stations collected between 2013 and 2017. The dataset contains hourly measurements of CO, NO\textsubscript{2}, O\textsubscript{3}, PM\textsubscript{10}, PM\textsubscript{2.5}, and SO\textsubscript{2}. Stations with incomplete time series are removed, and all features are normalized into [0, 1]. We train AQ-Net using AdamW optimizer with the learning rate of $1\times10^{-3}$. We use the MSE as the loss function to estimate the network output and ground truth values for backpropagation. The batch size is set to 32 and AQ-Net is trained for {468K iterations (about 2 hours) on the CSC server\footnote{https://csc.fi/} with one NVIDIA A100 GPU using PyTorch deep learning platform.} The k value for Neural kNN is 20.
The code can be found at \liu{\url{https://github.com/AmmarKheder/AQ-Net}}.

\subsection{Overall Performance Comparison}
To evaluate the accuracy of the reanalysis, three key metrics are considered: Mean Absolute Error (MAE), Root Mean Squared Error (RMSE), and the coefficient of determination ($R^2$). Each of these metrics provides valuable insight into the performance of the models:

\begin{small}
\begin{equation}
\text{MAE} = \frac{1}{n} \sum_{i=1}^{n} |y_i - \hat{y}_i|, \quad
        \text{RMSE} = \sqrt{\frac{1}{n} \sum_{i=1}^{n} (y_i - \hat{y}_i)^2}, \quad
        R^2 = 1 - \frac{\sum_{i=1}^{n} (y_i - \hat{y}_i)^2}{\sum_{i=1}^{n} (y_i - \bar{y})^2}
\label{eq:eval}
\end{equation}
\end{small}

\noindent We compare our proposed AQ-Net with three approaches: PatchTST~\cite{nie2022patchtst} (a Transformer tailored for time series processing), Linear Regression, and LSTM~\cite{shi2015convolutional}. Our model can be used for temporal prediction at the same monitoring stations, it can also provide spatiotemporal prediction at unseen monitor stations. 

\subsubsection{Short-term temporal reanalysis (24-Hour Input Window)}
Table~\ref{tab:short_term} shows short-term estimated PM\textsubscript{2.5} concentrations in Beijing over the next few hours based on a 24-hour historical input. These reanalysis are critical for real-time air quality monitoring, health alerts, and short-term pollution control measures.

\begin{table}[t]
\centering
\caption{\textbf{The evaluation of short-term PM\textsubscript{2.5} reanalysis.} The table presents PM\textsubscript{2.5} reanalysis performance based on $R^2$, MAE, and RMSE over 6, 12, and 24 hours using a 24-hour historical input.}
\label{tab:short_term_results}
\renewcommand{\arraystretch}{1.2} % Better spacing for readability
\begin{tabular}{c|ccc|ccc|ccc}
\hline
\multirow{2}{*}{Model} & \multicolumn{3}{c|}{6h reanalysis} & \multicolumn{3}{c|}{12h reanalysis} & \multicolumn{3}{c}{24h reanalysis} \\
                        & $R^2$$\uparrow$ & MAE$\downarrow$  & RMSE$\downarrow$  & $R^2$ & MAE$\downarrow$  & RMSE$\downarrow$  & $R^2$ & MAE$\downarrow$  & RMSE$\downarrow$  \\ \hline
\cellcolor{mistyrose}{AQ-Net}                  & \cellcolor{mistyrose}{\textbf{0.5103}} & \cellcolor{mistyrose}{\textbf{18.71}} & \cellcolor{mistyrose}{\textbf{22.87}}  & \cellcolor{mistyrose}{\textbf{0.4118}} & \cellcolor{mistyrose}{\textbf{22.04}} & \cellcolor{mistyrose}{\textbf{29.10}}  & \cellcolor{mistyrose}{\textbf{0.1894}} & \cellcolor{mistyrose}{\textbf{26.18}} & \cellcolor{mistyrose}{\textbf{33.34}}  \\
AQ-Net wo CE          &        0.4031    &      21.21      &  29.23          &  0.2312          &   25.32         &       30.23  &0.1231 & 27.21 & 34.48   \\
PatchTST                & 0.4421 & 21.65 & 27.52  & 0.3319 & 23.57 & 31.50  & 0.1601 & 27.65 & 34.08  \\
LSTM                    & 0.4648 & 20.05 & 26.44  & 0.2336 & 25.40 & 32.44  & 0.1001 & 28.38 & 35.13  \\
Linear Regression       & 0.4500 & 20.80 & 27.00  & 0.2100 & 26.00 & 33.00  & 0.0800 & 29.00 & 35.80  \\ \hline
\end{tabular}
\label{tab:short_term}
\end{table}

Table~\ref{tab:short_term_results} presents the evaluation of the performance of four models (AQ-Net, PatchTST, LSTM, and Linear Regression) for predicting PM\textsubscript{2.5} concentrations in Beijing over 6, 12, and 24 hours using a 24-hour historical input. We also have AQ-Net wo CE to represent our approach without using the proposed cyclic encoding approach. For the 6-hour estimation, AQ-Net achieves the best result with an \(R^2\) of 0.51, an MAE of 18.71, and an RMSE of 22.87, demonstrating its ability to effectively capture rapid fluctuations in pollution. Although PatchTST employs a self-attention mechanism, it underperforms slightly with an $R^2$ of 0.44, an MAE of 21.65, and an RMSE of 27.52, while the LSTM and Linear Regression models show comparable results with $R^2$ values of 0.46 and 0.45, respectively, and marginally higher error metrics. As the prediction horizon extends to 12 hours, the performance of all models deteriorates; however, AQ-Net maintains a significant lead with an $R^2$ of 0.41, whereas the other models drop to $R^2$ values of 0.33 for PatchTST, 0.23 for LSTM, and 0.21 for Linear Regression. This trend continues for the 24-hour estimation, where AQ-Net achieves an \(R^2\) of 0.19 compared to 0.16, 0.10, and 0.08 for PatchTST, LSTM, and Linear Regression, respectively. These results indicate that AQ-Net is particularly robust and effective for short-term estimation, while the competing models, especially PatchTST, LSTM, and Linear Regression, struggle to maintain their accuracy as the reanalysis horizon increases. Comparing AQ-Net and AQ-Net wo CE, we can also see that using Cyclic encoding can improve $R^2$ by 0.06$\sim$0.17 in 6$\sim$24 h reanalysis, which demonstrates its efficiency.

\subsubsection{Long-Term temporal reanalysis (336-Hour Input Window)}
Table~\ref{tab:long_term_results} shows long-term reanalysis in Beijing and analyzes how well models estimate PM\textsubscript{2.5} levels over extended periods based on a 2-week (336-hour) historical window. 
\begin{table}[t]
\centering
\caption{\textbf{The evaluation of long-term PM\textsubscript{2.5} reanalysis.} The table presents long-term reanalysis performance on MAE and RMSE over 2-day, 4-day, and 1-week horizons, using a 2-week (336-hour) historical input.}
\resizebox{\textwidth}{!}{%
  \begin{tabular}{c|cc|cc|cc}
  \hline
  \multirow{2}{*}{Model} & \multicolumn{2}{c|}{2-Day reanalysis} & \multicolumn{2}{c|}{4-Day reanalysis} & \multicolumn{2}{c}{1-Week reanalysis} \\
                       & MAE$\downarrow$  & RMSE$\downarrow$  & MAE$\downarrow$  & RMSE$\downarrow$  & MAE$\downarrow$  & RMSE$\downarrow$  \\ \hline
  \cellcolor{mistyrose}{AQ-Net}            & \cellcolor{mistyrose}{\textbf{13.57}}  & \cellcolor{mistyrose}{\textbf{16.80}}  & \cellcolor{mistyrose}{\textbf{17.44}}  & \cellcolor{mistyrose}{\textbf{21.29}}  & \cellcolor{mistyrose}{\textbf{21.29}}  & \cellcolor{mistyrose}{\textbf{25.17}}  \\
  AQ-Net wo CE          &   17.12         &     21.77       &        18.12   &       24.63     &       24.23    &       28.37     \\
  PatchTST          & 41.42           & 55.64           & 35.31           & 39.22           & 28.01           & 34.70           \\
  LSTM              & 24.04           & 28.37           & 25.11           & 31.21           & 22.87           & 28.81           \\
  Linear Regression & 25.00           & 29.00           & 26.00           & 32.00           & 23.50           & 29.50           \\ \hline
  \end{tabular}
}
\label{tab:long_term_results}
\end{table}

\begin{figure}[t]
    \centering
    \begin{minipage}{0.48\textwidth}
        \centering
        \includegraphics[width=\textwidth]{figures/7days.pdf}
        \caption{\textbf{Comparison of PM\textsubscript{2.5} reanalysis for different time slots over seven days.} The 4PM-7PM period exhibits greater variability, suggesting increased pollution activity during the late afternoon.}
        \label{fig:pm25_timeslot_comparison}
    \end{minipage}
    \hfill
    \begin{minipage}{0.48\textwidth}
        \centering
        \includegraphics[width=\textwidth]{figures/attention_evolution_head.pdf}
        \caption{\textbf{Visualization of the evolution of attention weights for selected two heads.} Head 2 reacts to short-term variations, while Head 1 maintains stable attention, capturing long-term patterns.}
        \label{fig:attention_heads_selected}
    \end{minipage}
\end{figure}

Table~\ref{tab:long_term_results} presents the results for prediction horizons of 2, 4, and 7 days. Unlike short-term reanalysis, where models estimate PM\textsubscript{2.5} concentrations step by step for each hour, long-term evaluations are conducted on a daily basis. Instead of predicting every hourly value, the goal is to assess whether the model can accurately estimate the overall pollution level for an entire day. This approach is more practical for extended reanalysis, as hourly fluctuations are less relevant when planning long-term air quality strategies. Therefore, the evaluation metrics in Table~\ref{tab:long_term_results} reflect the aggregated daily errors rather than step-by-step hourly deviations. For long-term estimation, AQ-Net retains the lowest RMSE across all horizons, effectively modeling extended dependencies. PatchTST performs strongly during the short-term periods, but suffers a sharp drop beyond two days, underscoring pure self-attention’s limitations for long-range reanalysis. Linear regression has the highest RMSE, reaffirming its inability to capture complex spatio-temporal dependencies. We can also observe that using Cyclic Encoding (CE) can improve the overall performance in all metrics.

Figure~\ref{fig:pm25_timeslot_comparison} illustrates the predicted and actual PM\textsubscript{2.5} levels over a one-week period in Beijing for two time slots: 9$\sim$12 PM and 4$\sim$7 PM. Our model effectively captures the overall temporal trends of PM\textsubscript{2.5} concentrations, with reanalysis generally following the fluctuations observed in real measurements. However, certain discrepancies are noticeable, particularly on Days 2 and 7, where morning predictions underestimate actual values, while on Day 4, afternoon predictions are slightly overestimated. These deviations suggest that while the model learns daily pollution patterns well, external factors such as meteorological changes or localized emission sources might not be fully accounted for. Notably, the model performs more consistently in the morning than in the afternoon, where greater variability is observed. 

\begin{figure}[t]
    \centering
    \begin{minipage}{0.48\textwidth}
        \centering
        \includegraphics[width=\textwidth]{figures/Attention.pdf}
        \caption{\textbf{Visualization of the attention heatmap across reanalysis and training days.} A diagonal trend suggests the model prioritizes recent observations, while deviations indicate potential long-term dependencies.}
        \label{fig:attention_heatmap}
    \end{minipage}
    \hfill
    \begin{minipage}{0.48\textwidth}
        \centering
        \includegraphics[width=\textwidth]{figures/BJ_v2.pdf}
        \caption{\textbf{Spatial interpolation of PM\textsubscript{2.5} in Beijing.}
        The PM\textsubscript{2.5} ranges from low to high (purple to yellow). $\bigcirc$ indicates stations used as input, while $\triangle$ represent predicted stations. 
    }
        \label{fig:pm25_spatial_distribution}
    \end{minipage}
\end{figure}

\subsection{Analysis of Temporal Attention Patterns}
We examine the model's temporal dependencies by analyzing MHA weights across different heads. In Figure~\ref{fig:attention_heads_selected}, Head 2 shows strong responsiveness to short-term fluctuations, while Head 1 maintains more stable weights, suggesting a focus on long-term trends. To identify and select these heads, we performed a PCA-based clustering of all attention heads, revealing that these two belong to distinct cluster: one emphasizing immediate variations (short-term) and the other capturing broader temporal structures (long-term). 

The global attention heatmap (Figure~\ref{fig:attention_heatmap}) shows how the model distributes attention between training days when reanalysis is performed. The x-axis represents input (historical) days (oldest $\rightarrow$ most recent), and the y-axis corresponds to output (reanalysis) days. The strong diagonal pattern indicates that the model prioritizes recent data, while some off-diagonal values suggest that it also captures longer-term dependencies. Darker areas attract more attention, highlighting the importance of recent pollution levels for accurate reanalysis.

\begin{figure}[t]
    \centering
    \includegraphics[width=\textwidth]{figures/mapB.pdf}
    \caption{\textbf{Daily mean PM\textsubscript{2.5} reanalysis over northern China.} Higher PM\textsubscript{2.5} is in yellow color. It highlights pollution hotspots in specific provinces. Overlapped markers indicate that multiple stations are located in very close proximity.}
    \label{fig:mapB}
\end{figure}

\subsection{Spatiotemporal reanalysis in Northern China}
Our proposed AQ-Net is able to interpolate the geographical trajectory given pollution data at known stations. Following the previous experiments in Beijing, Figure~\ref{fig:pm25_spatial_distribution} show that AQ-Net can use known stations ($\bigcirc$) to not only estimate the unknown stations ($\triangle$), but it can also estimate the global air pollution map for reanalysis. 

To illustrate the efficiency of our proposed AQ-Net on the large-scale dataset, we show spatiotemporal reanalysis in the entire northern China. As shown in Figure~\ref{fig:mapB}, utilizing the proposed neural kNN, we are able to estimate the complete spatial interpolation, capturing both observed and unobserved areas. Notably, pollution hotspots around northern and central Beijing are consistent with known urban emission sources. These results highlight the model’s ability to generalize beyond monitored stations, which is crucial for accurate city-wide air quality assessments.

\begin{figure}[t]
    \centering
    \includegraphics[width=\textwidth]{figures/rmsemae.pdf}
    \caption{\textbf{Visualization of prediction errors for hidden stations.} The bubbles indicate the RMSE or MAE. Both the color and size of the bubbles are proportional to the magnitude of the error: higher error values appear in warmer colors (yellow) and with larger circles.}
    \label{fig:rmsemae}
\end{figure}

Quantitative estimation on the spatiotemporal reanalysis is shown in Figure~\ref{fig:rmsemae}. Each circle represents the average estimation errors of predicted hidden stations in one city. We can see that our model can uniformly produce low MAE and RMSE on spatiotemporal interpolation. We also find that there are a few regions, like central China, are not well estimated. One of the reasons is that we do not have dense monitoring stations in those areas and the complex geographic and meteorological factors could have significant impacts.
