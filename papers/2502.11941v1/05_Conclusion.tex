\section{Conclusion}
\label{conclusion}
In this work, we addressed the challenge of reanlyzing air quality in complex urban environments, focusing on Northern China as a test bed. Our proposed AQ-Net model combines LSTM and multi-head attention to explore time series correlations, and a neural kNN to handle spatial interpolation for unobserved stations. This hybrid structure effectively captures both spatial and temporal pollution dynamics, as evidenced by strong performance in 6-to-24-hour horizons and in extended reanalysis up to seven days. Comparisons with established baselines reveal that AQ-Net outperforms these models in terms of RMSE and MAE. Notably, the attention mechanism highlights key temporal dependencies without requiring excessive complexity. Meanwhile, the neural kNN module ensures that spatial relationships among stations are preserved, enabling fine-resolution predictions even in regions lacking direct sensor measurements. AQ-Net provides valuable insights for health alerts and policy decisions. Beyond this specific application, the methodology can be extended to other pollutants and urban contexts where data coverage is uneven. Future enhancements could explore adaptive approaches to handle changing emission patterns or incorporate real-time data feeds to further refine long-term forecasts.