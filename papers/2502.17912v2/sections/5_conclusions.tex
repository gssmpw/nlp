\vspace{-3mm}
\section{Related Work}
\vspace{-3mm}

\spara{EBMs on Graph} Some previous works~\citep{hataya2021graph,liu2021graphebm} apply EBMs on graphs that are trained with MLE have been proposed for graph generation. However, their works are performed on small graphs, where the over-computational issue of the sampling adjacency matrix is neglected, making them non-scalable. In contrast, our proposed CLEBM focuses on OOD detection, decomposing the EBM networks into informative representation extraction and output energy score based on given latent. The design enables us to move MCMC to latent space, which does not suffer from sampling adjacency matrix and therefore has excellent scalability.

\spara{Graph OOD Detection} 
OOD detection for non-graph data by neural networks has garnered considerable attention in the literature~\citep{hendrycks2016baseline,maxlogits-2019,ood-dis-2018,liu2020energy,mohseni2020self,ren2019likelihood}. However, these methods typically assume that instances (such as images) are i.i.d., overlooking scenarios with inter-dependent data that are common in many real-world applications. In contrast, Graph OOD Detection that inherently includes inter-dependent structures has not been explored well. 
Recently, some works~\citep{ligraphde,bazhenov2022towards} focus on Graph OOD Detection on \textit{graph-level}, i.e., detecting OOD graphs. 
These works treat each graph as an independent instance, while OOD detection on \textit{node-level} presents unique challenges given the non-negligible inter-dependence between instances. 
To this end, Bayesian GNN models have been proposed that can detect OOD nodes within a graph by incorporating the inherent uncertainty in such inter-dependent data~\citep{GKDE-2020, GPN-2021}.  
OODGAT~\citep{song2022learning} emphasizes the importance of node connection patterns for outlier detection, explicitly modeling node interactions and separating inliers from outliers during feature propagation.
Energy-based Detection on graphs has been explored in GNNSafe~\citep{wu2023gnnsafe}, by directly combing GNNs and Energy-based Detection on Images~\citep{liu2020energy}. 
However, their energy score is directly construed by classification, which is less effective. Moreover, they use energy propagation to enhance performance, which highly relies on the homophily assumption. 
Additionally, some methods train the model with \textit{OOD Exposure}, i.e., training with both a known ID dataset and a known OOD dataset~\citep{hendrycks2018deep,liu2020energy,wu2023gnnsafe}. In contrast, we decompose EBM learning into representation learning and energy learning, delivering better detection capability without OOD exposure (see \cref{sec:main_result}). Additionally, benefits from powerful energy construction and effective graph encoder, we do not require energy propagation, thus keeping high performance across homophilic and heterophilic graphs.

\vspace{-1mm}
\section{Conclusion}
\vspace{-1mm}
We introduce a novel approach, \textbf{\shortname}, for graph OOD node detection, overcoming the heterophily issue and the computational challenges associated with MCMC sampling in large graphs. By decoupling the learning process into a GNN-based graph encoder and an energy head, we managed to leverage the GCL algorithm and classification loss to learn robust node representations and perform efficient MCMC sampling in the latent space, circumventing the need to directly sample the adjacency matrix. 
The design of \textbf{\shortname}, featuring a Multi-Hop Graph encoder and a Recurrent Update mechanism, facilitates the incorporation of topological information into node representations, which is crucial for OOD detection in graph-structured data. 
Extensive experimental evaluations have validated the effectiveness of \textbf{\shortname}, which not only exhibits superior performance on both homophilic and heterophilic graphs compared to baselines with/without OOD exposure, but also outperforms methods trained with OOD exposure in a label-insufficient scenario.

\clearpage
\thispagestyle{empty}