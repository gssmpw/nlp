%% LaTeX Template for ISIT 2025
%%
%% by Stefan M. Moser, October 2017
%% (with minor modifications by Tobias Koch, November 2023 and Michèle Wigger, November 2024)
%% 
%% derived from bare_conf.tex, V1.4a, 2014/09/17, by Michael Shell
%% for use with IEEEtran.cls version 1.8b or later
%%
%% Support sites for IEEEtran.cls:
%%
%% http://www.michaelshell.org/tex/ieeetran/
%% http://moser-isi.ethz.ch/manuals.html#eqlatex
%% http://www.ctan.org/tex-archive/macros/latex/contrib/IEEEtran/
%%

\documentclass[onecolumn, draftclsnofoot,11pt]{IEEEtran}

% \documentclass[conference,letterpaper]{IEEEtran}

%% depending on your installation, you may wish to adjust the top margin:
\addtolength{\topmargin}{9mm}

%%%%%%
%% Packages:
%% Some useful packages (and compatibility issues with the IEEE format)
%% are pointed out at the very end of this template source file (they are 
%% taken verbatim out of bare_conf.tex by Michael Shell).
%
% *** Do not adjust lengths that control margins, column widths, etc. ***
% *** Do not use packages that alter fonts (such as pslatex).         ***
%

\usepackage[utf8]{inputenc} 
\usepackage[T1]{fontenc}
\usepackage{url}
\usepackage{ifthen}
\usepackage{cite}
\usepackage[cmex10]{amsmath} % Use the [cmex10] option to ensure complicance
                             % with IEEE Xplore (see bare_conf.tex)

%% Please note that the amsthm package must not be loaded with
%% IEEEtran.cls because IEEEtran provides its own versions of
%% theorems. Also note that IEEEXplore does not accepts submissions
%% with hyperlinks, i.e., hyperref cannot be used.

\interdisplaylinepenalty=2500 % As explained in bare_conf.tex



%%%%%%
% correct bad hyphenation here
\hyphenation{op-tical net-works semi-conduc-tor}

\usepackage{times}
\usepackage{latexsym}

\usepackage[T1]{fontenc}

\usepackage[utf8]{inputenc}

\usepackage{microtype}

\usepackage{inconsolata}

\usepackage{graphicx}


\usepackage{amsmath}
\usepackage{amssymb}
\usepackage{multirow}
\usepackage{booktabs}
\usepackage{catchfile}

\usepackage[boxed]{algorithm}
\usepackage{varwidth}
\usepackage[noEnd=true,indLines=false]{algpseudocodex}
\usepackage{cleveref}
\makeatletter
\@addtoreset{ALG@line}{algorithm}
\renewcommand{\ALG@beginalgorithmic}{\small}
\algrenewcommand\alglinenumber[1]{\small #1:}
\makeatother

\usepackage[normalem]{ulem}
\usepackage{todonotes}

\usepackage{lipsum}    %
\usepackage{comment}   %
\usepackage{graphicx}  %
\usepackage{pifont}    %

\usepackage[font=small,labelfont=bf]{caption}
\usepackage{float}     %
\usepackage{booktabs}  %
\usepackage{subcaption}  %

\usepackage{listings}

\usepackage{amsthm}  %

\usepackage{hyperref}
\usepackage{refstyle}
\usepackage{amsmath}
\usepackage{cleveref}

\usepackage{booktabs}
\usepackage{multirow} %
\usepackage{soul}%
\usepackage{tabularx}
\usepackage{enumitem}

\usepackage{amssymb}


\DeclareMathOperator*{\argmax}{argmax} %
\usepackage{pifont}
\newcommand{\cmark}{\ding{51}}%
\newcommand{\xmark}{\ding{55}}%

\usepackage{graphicx}

\interfootnotelinepenalty=10000

\crefformat{section}{\S#2#1#3}
\crefformat{subsection}{\S#2#1#3}
\crefformat{subsubsection}{\S#2#1#3}
\crefrangeformat{section}{\S#3#1#4 to~\S#5#2#6}
\crefmultiformat{section}{\S#2#1#3}{ and~\S#2#1#3}{, #2#1#3}{ and~#2#1#3}
\Crefformat{figure}{#2Fig.~#1#3}
\Crefmultiformat{figure}{Figs.~#2#1#3}{ and~#2#1#3}{, #2#1#3}{ and~#2#1#3}
\Crefformat{table}{#2Tab.~#1#3}
\Crefmultiformat{table}{Tabs.~#2#1#3}{ and~#2#1#3}{, #2#1#3}{ and~#2#1#3}
\Crefformat{appendix}{#2Appx.~\S#1#3}
\crefformat{algorithm}{Alg.~#2#1#3}
\Crefformat{equation}{#2Eq.~#1#3}

\newcommand{\todo}[1]{{\color{red}[{TODO:} #1]}}

\newcommand{\task}{\textsc{1MKR}}

% ------------------------------------------------------------
\begin{document}
\title{When Wyner and Ziv Met Bayes in Quantum-Classical Realm} 

% %%% Single author, or several authors with same affiliation:
% \author{%
%  \IEEEauthorblockN{Author 1 and Author 2}
% \IEEEauthorblockA{Department of Statistics and Data Science\\
%                    University 1\\
 %                   City 1\\
  %                  Email: author1@university1.edu}% }


%%% Several authors with up to three affiliations:
\author{\IEEEauthorblockN{Mohammad Aamir Sohail\IEEEauthorrefmark{1}, Touheed Anwar Atif\IEEEauthorrefmark{2}, and
		S. Sandeep Pradhan\IEEEauthorrefmark{1}
% 		, and Arun Padakandla\IEEEauthorrefmark{2} 
		\\}
	\IEEEauthorblockA
		\IEEEauthorrefmark{1}{Department of EECS, University of Michigan, Ann Arbor, USA.\\
		\IEEEauthorrefmark{2}Los Alamos National Laboratory, USA\\
		Email: \tt mdaamir@umich.edu, tatifd@lanl.gov,  pradhanv@umich.edu
% 		,
% 		arunpr@utk.edu,
		}
}




\maketitle

%%%%%%
%% Abstract: 
%% If your paper is eligible for the student paper award, please add
%% the comment "THIS PAPER IS ELIGIBLE FOR THE STUDENT PAPER
%% AWARD." as a first line in the abstract. 
%% For the final version of the accepted paper, please do not forget
%% to remove this comment!
%%

\begin{abstract}
   In this work, we address the lossy quantum-classical source coding with the quantum side-information (QC-QSI) problem. The task is to compress the classical information about a quantum source, obtained after performing a measurement while incurring a bounded reconstruction error. Here, the decoder is allowed to use the side information to recover the classical data obtained from measurements on the source states.
   We introduce a new formulation based on a backward (posterior) channel, replacing the single-letter distortion observable with a single-letter posterior channel to capture reconstruction error. Unlike the rate-distortion framework, this formulation imposes a block error constraint. An analogous formulation is developed for lossy classical source coding with classical side information (C-CSI) problem. We derive an inner bound on the asymptotic performance limit in terms of single-letter quantum and classical mutual information quantities of the given posterior channel for QC-QSI and C-CSI cases, respectively. Furthermore, we establish a connection between rate-distortion and rate-channel theory, showing that a rate-channel compression protocol attains the optimal rate-distortion function for a specific distortion measure and level.
    %  We propose a new formulation for the QC-QSI source coding problem based on the notion of a backward (posterior) channel.
    %  We employ a single-letter posterior channel to capture the reconstruction error in place of the single-letter distortion observable. The formulation requires the reconstruction of the compressed quantum source to satisfy a block error constraint as opposed to the average single-letter distortion criterion in the rate-distortion setting. 
    %  We also develop an analogous formulation for the classical case, namely the classical source coding with classical side information (C-CSI) problem,
    %  with respect to a corresponding posterior channel.
    % We provide a sufficient condition, i.e., an inner bound to the asymptotic performance limit, in terms of single-letter quantum mutual information and mutual information quantities of the given posterior channel, for C-CSI and QC-QSI cases, respectively.  
    % Furthermore, we establish a connection between rate-distortion and rate-channel theory, showing that a rate-channel compression protocol achieves the optimal rate-distortion function for a specific distortion measure and level. 
\end{abstract}

\section{Introduction}
\label{sec:intro}
\section{Introduction}

Large language models (LLMs) have achieved remarkable success in automated math problem solving, particularly through code-generation capabilities integrated with proof assistants~\citep{lean,isabelle,POT,autoformalization,MATH}. Although LLMs excel at generating solution steps and correct answers in algebra and calculus~\citep{math_solving}, their unimodal nature limits performance in plane geometry, where solution depends on both diagram and text~\citep{math_solving}. 

Specialized vision-language models (VLMs) have accordingly been developed for plane geometry problem solving (PGPS)~\citep{geoqa,unigeo,intergps,pgps,GOLD,LANS,geox}. Yet, it remains unclear whether these models genuinely leverage diagrams or rely almost exclusively on textual features. This ambiguity arises because existing PGPS datasets typically embed sufficient geometric details within problem statements, potentially making the vision encoder unnecessary~\citep{GOLD}. \cref{fig:pgps_examples} illustrates example questions from GeoQA and PGPS9K, where solutions can be derived without referencing the diagrams.

\begin{figure}
    \centering
    \begin{subfigure}[t]{.49\linewidth}
        \centering
        \includegraphics[width=\linewidth]{latex/figures/images/geoqa_example.pdf}
        \caption{GeoQA}
        \label{fig:geoqa_example}
    \end{subfigure}
    \begin{subfigure}[t]{.48\linewidth}
        \centering
        \includegraphics[width=\linewidth]{latex/figures/images/pgps_example.pdf}
        \caption{PGPS9K}
        \label{fig:pgps9k_example}
    \end{subfigure}
    \caption{
    Examples of diagram-caption pairs and their solution steps written in formal languages from GeoQA and PGPS9k datasets. In the problem description, the visual geometric premises and numerical variables are highlighted in green and red, respectively. A significant difference in the style of the diagram and formal language can be observable. %, along with the differences in formal languages supported by the corresponding datasets.
    \label{fig:pgps_examples}
    }
\end{figure}



We propose a new benchmark created via a synthetic data engine, which systematically evaluates the ability of VLM vision encoders to recognize geometric premises. Our empirical findings reveal that previously suggested self-supervised learning (SSL) approaches, e.g., vector quantized variataional auto-encoder (VQ-VAE)~\citep{unimath} and masked auto-encoder (MAE)~\citep{scagps,geox}, and widely adopted encoders, e.g., OpenCLIP~\citep{clip} and DinoV2~\citep{dinov2}, struggle to detect geometric features such as perpendicularity and degrees. 

To this end, we propose \geoclip{}, a model pre-trained on a large corpus of synthetic diagram–caption pairs. By varying diagram styles (e.g., color, font size, resolution, line width), \geoclip{} learns robust geometric representations and outperforms prior SSL-based methods on our benchmark. Building on \geoclip{}, we introduce a few-shot domain adaptation technique that efficiently transfers the recognition ability to real-world diagrams. We further combine this domain-adapted GeoCLIP with an LLM, forming a domain-agnostic VLM for solving PGPS tasks in MathVerse~\citep{mathverse}. 
%To accommodate diverse diagram styles and solution formats, we unify the solution program languages across multiple PGPS datasets, ensuring comprehensive evaluation. 

In our experiments on MathVerse~\citep{mathverse}, which encompasses diverse plane geometry tasks and diagram styles, our VLM with a domain-adapted \geoclip{} consistently outperforms both task-specific PGPS models and generalist VLMs. 
% In particular, it achieves higher accuracy on tasks requiring geometric-feature recognition, even when critical numerical measurements are moved from text to diagrams. 
Ablation studies confirm the effectiveness of our domain adaptation strategy, showing improvements in optical character recognition (OCR)-based tasks and robust diagram embeddings across different styles. 
% By unifying the solution program languages of existing datasets and incorporating OCR capability, we enable a single VLM, named \geovlm{}, to handle a broad class of plane geometry problems.

% Contributions
We summarize the contributions as follows:
We propose a novel benchmark for systematically assessing how well vision encoders recognize geometric premises in plane geometry diagrams~(\cref{sec:visual_feature}); We introduce \geoclip{}, a vision encoder capable of accurately detecting visual geometric premises~(\cref{sec:geoclip}), and a few-shot domain adaptation technique that efficiently transfers this capability across different diagram styles (\cref{sec:domain_adaptation});
We show that our VLM, incorporating domain-adapted GeoCLIP, surpasses existing specialized PGPS VLMs and generalist VLMs on the MathVerse benchmark~(\cref{sec:experiments}) and effectively interprets diverse diagram styles~(\cref{sec:abl}).

\iffalse
\begin{itemize}
    \item We propose a novel benchmark for systematically assessing how well vision encoders recognize geometric premises, e.g., perpendicularity and angle measures, in plane geometry diagrams.
	\item We introduce \geoclip{}, a vision encoder capable of accurately detecting visual geometric premises, and a few-shot domain adaptation technique that efficiently transfers this capability across different diagram styles.
	\item We show that our final VLM, incorporating GeoCLIP-DA, effectively interprets diverse diagram styles and achieves state-of-the-art performance on the MathVerse benchmark, surpassing existing specialized PGPS models and generalist VLM models.
\end{itemize}
\fi

\iffalse

Large language models (LLMs) have made significant strides in automated math word problem solving. In particular, their code-generation capabilities combined with proof assistants~\citep{lean,isabelle} help minimize computational errors~\citep{POT}, improve solution precision~\citep{autoformalization}, and offer rigorous feedback and evaluation~\citep{MATH}. Although LLMs excel in generating solution steps and correct answers for algebra and calculus~\citep{math_solving}, their uni-modal nature limits performance in domains like plane geometry, where both diagrams and text are vital.

Plane geometry problem solving (PGPS) tasks typically include diagrams and textual descriptions, requiring solvers to interpret premises from both sources. To facilitate automated solutions for these problems, several studies have introduced formal languages tailored for plane geometry to represent solution steps as a program with training datasets composed of diagrams, textual descriptions, and solution programs~\citep{geoqa,unigeo,intergps,pgps}. Building on these datasets, a number of PGPS specialized vision-language models (VLMs) have been developed so far~\citep{GOLD, LANS, geox}.

Most existing VLMs, however, fail to use diagrams when solving geometry problems. Well-known PGPS datasets such as GeoQA~\citep{geoqa}, UniGeo~\citep{unigeo}, and PGPS9K~\citep{pgps}, can be solved without accessing diagrams, as their problem descriptions often contain all geometric information. \cref{fig:pgps_examples} shows an example from GeoQA and PGPS9K datasets, where one can deduce the solution steps without knowing the diagrams. 
As a result, models trained on these datasets rely almost exclusively on textual information, leaving the vision encoder under-utilized~\citep{GOLD}. 
Consequently, the VLMs trained on these datasets cannot solve the plane geometry problem when necessary geometric properties or relations are excluded from the problem statement.

Some studies seek to enhance the recognition of geometric premises from a diagram by directly predicting the premises from the diagram~\citep{GOLD, intergps} or as an auxiliary task for vision encoders~\citep{geoqa,geoqa-plus}. However, these approaches remain highly domain-specific because the labels for training are difficult to obtain, thus limiting generalization across different domains. While self-supervised learning (SSL) methods that depend exclusively on geometric diagrams, e.g., vector quantized variational auto-encoder (VQ-VAE)~\citep{unimath} and masked auto-encoder (MAE)~\citep{scagps,geox}, have also been explored, the effectiveness of the SSL approaches on recognizing geometric features has not been thoroughly investigated.

We introduce a benchmark constructed with a synthetic data engine to evaluate the effectiveness of SSL approaches in recognizing geometric premises from diagrams. Our empirical results with the proposed benchmark show that the vision encoders trained with SSL methods fail to capture visual \geofeat{}s such as perpendicularity between two lines and angle measure.
Furthermore, we find that the pre-trained vision encoders often used in general-purpose VLMs, e.g., OpenCLIP~\citep{clip} and DinoV2~\citep{dinov2}, fail to recognize geometric premises from diagrams.

To improve the vision encoder for PGPS, we propose \geoclip{}, a model trained with a massive amount of diagram-caption pairs.
Since the amount of diagram-caption pairs in existing benchmarks is often limited, we develop a plane diagram generator that can randomly sample plane geometry problems with the help of existing proof assistant~\citep{alphageometry}.
To make \geoclip{} robust against different styles, we vary the visual properties of diagrams, such as color, font size, resolution, and line width.
We show that \geoclip{} performs better than the other SSL approaches and commonly used vision encoders on the newly proposed benchmark.

Another major challenge in PGPS is developing a domain-agnostic VLM capable of handling multiple PGPS benchmarks. As shown in \cref{fig:pgps_examples}, the main difficulties arise from variations in diagram styles. 
To address the issue, we propose a few-shot domain adaptation technique for \geoclip{} which transfers its visual \geofeat{} perception from the synthetic diagrams to the real-world diagrams efficiently. 

We study the efficacy of the domain adapted \geoclip{} on PGPS when equipped with the language model. To be specific, we compare the VLM with the previous PGPS models on MathVerse~\citep{mathverse}, which is designed to evaluate both the PGPS and visual \geofeat{} perception performance on various domains.
While previous PGPS models are inapplicable to certain types of MathVerse problems, we modify the prediction target and unify the solution program languages of the existing PGPS training data to make our VLM applicable to all types of MathVerse problems.
Results on MathVerse demonstrate that our VLM more effectively integrates diagrammatic information and remains robust under conditions of various diagram styles.

\begin{itemize}
    \item We propose a benchmark to measure the visual \geofeat{} recognition performance of different vision encoders.
    % \item \sh{We introduce geometric CLIP (\geoclip{} and train the VLM equipped with \geoclip{} to predict both solution steps and the numerical measurements of the problem.}
    \item We introduce \geoclip{}, a vision encoder which can accurately recognize visual \geofeat{}s and a few-shot domain adaptation technique which can transfer such ability to different domains efficiently. 
    % \item \sh{We develop our final PGPS model, \geovlm{}, by adapting \geoclip{} to different domains and training with unified languages of solution program data.}
    % We develop a domain-agnostic VLM, namely \geovlm{}, by applying a simple yet effective domain adaptation method to \geoclip{} and training on the refined training data.
    \item We demonstrate our VLM equipped with GeoCLIP-DA effectively interprets diverse diagram styles, achieving superior performance on MathVerse compared to the existing PGPS models.
\end{itemize}

\fi 


\section{Main Results}
\label{sec:mainresults}
\textbf{Notations.} The set of density operators on Hilbert space $\calH_A$ is denoted by $\calD(\calH_A)$.
We denote the finite alphabet of a source as $\sfX$, and  
the set of probability distributions on the finite alphabet $\sfX$ as $\calP(\calX)$. Let $[\Theta] \deq \{1,2,\cdots,\Theta\}$. 

\begin{definition}\label{def:CQchannel}(Classical-Quantum (CQ) Channel)
Given a finite set $\sfX$ and a probability distribution $P_X$, a CQ channel $\mathcal{W}$ is   
specified by an ensemble of density of operators $\{(P_X(x),\calW_{x})\}_{ x \in \sfX}$. The corresponding average density operator is given as $\calW(P):= \sum_{x\in\sfX}P_X(x)\calW_x$.
\end{definition}
% \begin{definition}\label{def:QCchannel}(Quantum-Classical (QC) Channel or Measurement Channel)
% Given a finite set $\sfX$ and a POVM $\{\Lambda_x\}_{x\in\sfX}$, a QC channel $\calM_{\Lambda}$ acting on an input density operator $\rho$ is   
% specified by an ensemble of pure states $\{(P_X(x),|x\>)\}_{x\in \sfX}$, where $P_X(x) = \Tr\{\Lambda_x \rho\}$. The corresponding density operator is given as $\calM_\Lambda(\rho):= \sum_{x\in\sfX}\Tr\{\Lambda_x \rho\}|x\>\<x|$. 
% \end{definition}

\subsection{Lossy Quantum-Classical Source Coding with Quantum Side Information}
Consider a quantum source $\rho^{AB}\in \calD(\calH_{A}\tensor \calH_{B})$ shared between a sender $A$ and a receiver $B$.
\begin{definition}(QC-QSI Source Coding Setup) A QC-QSI source coding setup is characterized by a triple $(\rho^{AB},\sfX,\calW_{\!\ \sfX\rightarrow \refe B})$, where $\rho^{AB}$ is the bipartite density operator of the source and its side information, $\sfX$ is the reconstruction alphabet, and $\calW:\sfX\rightarrow\calD(\calH_{\refe } \tensor \calH_B)$ is a single-letter posterior CQ channel.
\end{definition}
% \vspace{-10pt}
\begin{figure}[!htb]
    \centering
    \includegraphics[scale=0.85]{QC-QSI.png}
    \vspace{-5pt}
    \caption{Illustration of Lossy QC-QSI Compression Protocol.}
    \label{fig:QC-QSI}
    \vspace{-5pt}
\end{figure}
\begin{definition}
    (Lossy QC-QSI Compression Protocol) For a given bipartite density operator $\rho^{AB}$ and a reconstruction alphabet $\sfX$, a $(n,\Theta)$ lossy QC-QSI compression protocol is characterized by $(i)$ an encoding POVM $\Gamma^{(n)}_{\sfA} \deq \{\epovm_{m}\}_{m=1}^{\Theta}$ acting on $A^n$, $(ii)$ For each $m\in[\Theta]$, a decoding POVM $\Gamma^{(n,m)}_{\sfB} \deq \{\dpovm^{(m)}_{k}\}_{k=1}^{\bTheta}$ acting on $B^n$, and  $(iii)$ a decoding map $f:[\Theta] \times [\bTheta] \rightarrow \sfX^n$, as shown in Figure \ref{fig:QC-QSI}. 
    % Let $\calR_{\text{QC-QSI}}(\rho^{AB},\sfX,\calW_{\sfX\rightarrow AB})$ denote the set of achievable rates.
\end{definition}
\begin{definition}\label{def:qc_qsi_achievability}
    (Achievability) For a given QC-QSI source coding setup $(\rho^{AB},\sfX,\calW_{\sfX\rightarrow AB})$, a rate $R$ is said to be achievable if for all $\epsilon > 0$ and all sufficiently large $n$, there exists an $(n,\Theta)$ QC-QSI lossy compression protocol such that 
$\frac{1}{n}\log \Theta \leq R + \epsilon$, and $\Xi(\Gamma^{(n)}_\sfA,\Gamma^{(n)}_\sfB,f) \leq \epsilon$, where 
$\Xi(\Gamma^{(n)}_\sfA,\Gamma^{(n)}_\sfB,f) \deq$ 
% \begin{align*}
%     \left\|({I^{\refe B}}^{\tensor n}\tensor f)({I^{R}}^{\tensor n}\tensor \calM_{\Gamma_\sfA\tensor \Gamma_\sfB})({\phi^{\refe A B}}^{\tensor n}) - \Tr\{\calM_{\Gamma_\sfA\tensor \Gamma_\sfB}({\rho^{A B}}^{\tensor n})\}\calW_{}\right\|
% \end{align*}
\begin{align*}
    \Big\| \sum_{m,k} |\xn(m,k)\>\<\xn(m,k)| \tensor \tau^{\refe B}_{\xn(m,k)}   - \sum_{m,k}  Q_{\Xn}(\xn(m,k))|\xn(m,k)\>\<\xn(m,k)| \tensor \calW_{\xn(m,k)}^{RB}\Big\|_1,
\end{align*}
% for all $\xn \!\in \!\sfX^n,$ 
where $\calW_{\xn}^{RB} \deq \bigotimes_{i=1}^n \calW_{x_i}^{RB}$,  $\tau^{R^nB^n}_{\xn}\!\deq \!\Tr_{A^n}\{({I^{\refe}}^{\tensor n}\!\tensor \epovm_{m} \tensor \dpovm^{(m)}_k) (\phi^{RAB}_\rho)^{\tensor n}({I^{\refe A}}^{\tensor n}\! \tensor\!\dpovm^{(m)}_k)\}$ is the unnormalized system-induced density operator on systems $\refe^nB^n$, $, f(m,k) = \xn(m,k), \phi_{\rho}^{RAB}$ is the canonical purification of the state $\rho^{AB}$, and $Q_{\Xn}(\xn(m,k)) \deq \Tr\{(\Omega_{m} \tensor \Xi^{(m)}_k) (\rho^{AB})^{\tensor n}\}$ is the probability of observing the sequence $\xn(m,k)$.
\end{definition}
\begin{theorem}\label{thm:QC-QSI}For a given $(\rho_{AB},\sfX,\calW_{\sfX\rightarrow \refe B})$ QC-QSI source coding setup, 
% such that $\calA(\sourcedo,\calW)$ is non-empty
a rate $R$ is achievable if $\calA(\rho^{AB},\calW_{X\rightarrow \refe B})$ is non-empty and
    $$R\geq I(X;\refe  B)_\sigma-I(X;B)_\sigma = I(X;\refe |B)_\sigma,$$
    where the quantum mutual information is computed with respect to the following quantum-classical state, $$\sigma^{XRB} \deq \sum_x P_X(x) |x\>\<x|^X \tensor \calW_x^{RB} \eqand  P_X(x) \in \calA(\rho^{AB},\calW_{X\rightarrow \refe B}),$$
    $\calA$ is the set of reconstruction distributions defined as 
$$\calA(\rho^{AB},\calW_{X\rightarrow \refe B}) \deq \{P_X\in\calP(\sfX):\sum_{x}P_X(x)\calW^{\refe B}_x = \Tr_{A}\{\phi^{\refe AB}\}\},$$ 
    $\phi^{\refe AB}  \text{ is a purification of $\rho^{AB}$, and } \{|x\>\}_{\{x\in \sfX\}}$ is an orthonormal basis 
    for the Hilbert space $\calH_X$ with $\dim{(\calH_X)}=|\calX|$.
\end{theorem}
\begin{proof}
    The proof is provided in Section \ref{sec:proof:QC-QSI}.
\end{proof}

\subsection{Lossy Classical Source Coding with Classical Side Information}
\begin{definition}(C-CSI Source Coding Setup) A C-CSI source coding setup is characterized by a triple $(\pxz,\sfY,W_{X|YZ})$, where $\pxz$ is the joint source and side-information distribution over a finite alphabet $\sfX\times\sfZ$, $\sfY$ is the reconstruction alphabet, and $W_{X|YZ}:\sfY\times\sfZ \rightarrow\sfX$ is the posterior (backward) channel, i.e., the single-letter conditional distribution of source given the reconstruction and side-information.
\end{definition}
\begin{definition}(Lossy C-CSI Source Compression Protocol) For a given $\pxz$ and reconstruction alphabet $\sfY$, an $(n,\Theta)$ lossy C-CSI source compression protocol consists of $(i)$ a randomized encoding map $\calE^{(n)}:\sfX^n\rightarrow[\Theta]$ and $(ii)$ a randomized decoding map $\calD^{(n)}:\sfZ^n \times[\Theta] \rightarrow\sfY^n,$ as shown in Figure \ref{fig:C-CSI}.  
\end{definition}
\vspace{-3pt}
\begin{figure}[!htb]
    \centering
    \includegraphics[scale=0.95]{C-CSI.png}
    \vspace{-5pt}
    \caption{Illustration of Lossy C-CSI Compression Protocol.}
    \label{fig:C-CSI}
    \vspace{-3pt}
\end{figure}
\begin{definition}(Achievability) Given C-CSI source coding setup $(\pxz,\sfY,W_{X|YZ})$, a rate R is said to be achievable if for all $\epsilon >0$ and all sufficiently large $n$, there exists an $(n,\Theta)$ lossy compression protocol such that $\frac{1}{n}\log\Theta \leq R+\epsilon$ and $\Xi(\calE^{(n)},\calD^{(n)})\leq \epsilon$, where
\begin{align}
\Xi(\calE^{(n)},\calD^{(n)}) \deq \left\|P_{X^nY^nZ^n} \!-\! P_{Y^nZ^n}W_{X|YZ}^n\right\|_{\text{TV}},\nonumber
\vspace{-10pt}
\end{align}
    $P_{X^nY^nZ^n}(\xn\!,\!\yn\!,\!\zn) \! =\!  {\pxz^n(\xn\!,\!\zn)} \sum_{m \in [\Theta]}\encodern(m|x^n)$ $\decodern(y^n|m,\!\zn),\ \forall(\xn\!,\!\yn\!,\!\zn)\in \sfX^n \times \sfY^n \times \sfZ^n$, is the system-induced distribution,
 $P_{Y^nZ^n}W_{X|YZ}^n$ is the approximating distribution, and $W_{X|YZ}^n(\xn|\yn,\zn) \!\deq\! \prod_{i=1}^nW_{X|YZ}(x_i|y_i,z_i)$. Let $\calR_{\text{C-CSI}}(\pxz,\sfY,W_{X|YZ})$ denote the set of achievable rates.
\end{definition}
\begin{theorem}\label{thm:C-CSI}(Lossy C-CSI Compression Inner Bound)
    $R\in \calR_{\text{C-CSI}}$ if $\calA(P_{XZ},W_{X|YZ})$ is non-empty and there exists a PMF $P_{UXYZ} \in \calA$ such that 
    \begin{itemize}
        \item $P_{XYZ} = \sum_{u\in \sfU}P_{UXYZ}(u,\!x,\!y,\!z)$ for all $(x,\!y,\!z)$
        \item $Z-X-U, \ X-(U,Z)-Y,\text{ and } \ X-(Y,Z)-U$ are Markov chains
        \item $R\geq I(X;U)-I(U;Z),$
    \end{itemize}  
    where $\calA$ is the set of reconstruction distributions defined as 
    $$\calA(P_{XZ},W_{X|YZ})\deq \Big\{P_{U|X}, P_{Y|UZ}:P_{X|Z} \frac{\sum_{u\in\calU}P_{U|X}P_{Y|UZ}}{\sum_{u\in\sfU}P_{U|X}P_{Y|UZ}} = W_{X|YZ} \eqand X-(Y,Z)-U \Big\}.$$
\end{theorem}
\begin{proof}
    The proof is provided in Section \ref{sec:proof:C-CSI}.
\end{proof}


\subsection{Connection Between Rate-Channel Theory and Rate-Distortion Theory}
In \cite{sohail2023unique}, we have developed a new formulation of the lossy source coding problem called \textit{rate-channel theory} which is described below. 
\begin{definition}
    [Achievability]\label{def:clserror_constraint} Given a source coding setup $(\px,\sfY,W_{X|Y})$,
a rate $R$ is said to be achievable if for all $\epsilon >0$ and all sufficiently large $n$, there exists an 
$(n,\Theta)$ lossy source compression protocol consists of  $(i)$ a randomized encoding map $\encodern:\sfX^n \longrightarrow [\Theta]$ and 
$(ii)$ a randomized decoding map $\decodern:[\Theta] \longrightarrow {\sfY}^n$ such that $\frac{1}{n}\log \Theta \leq R + \epsilon$, and $\Xi(\encodern,\decodern) \leq \epsilon$, where 
\begin{equation}\label{def:error_constraint}
  \Xi(\encodern,\decodern)\deq \frac{1}{2}\sum_{\xn \yn}\|{P_{X^n\Yn}(x^n,\yn) -
    P_{\Yn}(\yn)
    % \prevTC^n(x^n|\yn)}
    \\
    \Pi_{i=1}^n\prevTC(x_i|\hat{x}_i)}\|,
\end{equation}
and
$
P_{X^n\Yn}(\xn,\yn) =  {\px^n(x^n)} \sum_{m \in [\Theta]} \encodern(m|x^n) \decodern(\yn|m), 
% P{\curly{\calD(\calE(x^n)) = \Yn}} 
\text{  for all  } (x^n, \yn)\in \sfX^n \times {\sfY}^n,$ is the system-induced distribution,
 and $P_{\Yn}W_{X|Y}^n$ is the approximating distribution.
\end{definition}

\begin{theorem}\label{thm:Csourcecoding}(Rate-Channel Theory \cite[Theorem 2]{sohail2023unique}). For a $(\px,\sfY,W_{X|Y})$ source coding setup, a rate $R$ is said to be achievable if and only if $\calA(\px,W_{X|Y})$ is non-empty, and \begin{equation}\label{eqn:clsratedistortion}R \geq \min_{P_Y \in \calA(\px,W_{X|Y})} I(X;Y), \vspace{-5pt}\end{equation}
where $\calA(\px,W_{X|Y}) \deq \{P_Y \in \calP(\sfY): \!\! \text{for all }  x \in \sfX$, $ 
\sum_{y} P_{Y}(y) W_{X|Y}(x|y) = \px(x)\},$ is the set of reconstruction distributions.
\end{theorem}
Let us consider the case when the side information $Z$ is trivial. 
Given a lossy source coding setup $(\px,\sfY,W_{X|Y})$, 
consider a sequence of 
$(n,\Theta)$
lossy compression protocols 
that achieves the asymptotic performance  $R^\star \deq \min_{P_Y \in \calA(P_X,W_{X|Y})} I(X;Y)$
as given below in Theorem \ref{thm:Csourcecoding}.
Let 
$P_{X^nY^n}$ be the induced $n$-letter joint distribution on the source and the reconstruction vectors. 
Then, we see that 
$\lim_{n \rightarrow \infty} \frac{1}{n} \log \Theta =I(X;Y)$,
$$\lim_{n\rightarrow\infty}\|P_{X^nY^n} -
    P_{Y^n}W_{X|Y}^n\|_{\normalfont \text{TV}} = 0.$$

% and,  by Lemmas \ref{lem:averageSingleletter} and \ref{lemma:sols_linearEqn_close} (stated below and detailed proof provided in \cite{sohail2025WZ}.),
% we have
% \[
% \lim_{n \rightarrow \infty} \| P_{X_QY_Q} - P_{Y}W_{X|Y} \|_{\normalfont \text{TV}} =0,
% \]
% for some distribution $\pxhat$  
% in $\calA(\px,W_{X|Y})$ that achieves the optimality in Theorem \ref{thm:Csourcecoding},
% where $P_{X_QY_Q}=\tfrac{1}{n} \sum_{i=1}^n P_{X_iY_i}.$
% Let $c>0$ and $b(x)$ be an arbitrary constant and a function, respectively.


\begin{theorem}\label{thm:connection}
    Let $c>0$ and $b(x)$ be an arbitrary constant and a function, respectively. Consider a single-letter distortion function given as $d(x,y)= -c\log_2 W_{X|Y}(x|y)+b(x)$,
and distortion level $D={\mathbb{E}[d(X,Y)]}$ with respect to distribution $P_{Y}W_{X|Y}$, where $P_{Y} \in \calA(P_X,W_{X|Y})$ achieves the optimality in Theorem \ref{thm:Csourcecoding}. 
Then, the same sequence of protocols achieves the Shannon rate-distortion function at distortion level $D$ for the source $P_X$, and distortion function $d$, i.e., 
\vspace{-5pt}
\begin{align*}
    \lim_{n \rightarrow \infty} \!\mathbb{E} \bigg[\!\frac{1}{n} \!\sum_{i=1}^n d(X_i,Y_i)\!\bigg] \!=D,
\end{align*}
% \vspace{-5pt}
where the expectation is with respect to the distribution induced by the protocol $P_{X^nY^n}.$
\end{theorem}
\begin{proof}
    The proof is provided in Section \ref{sec:proof:connection}.
\end{proof}











\section{Proof of Theorem \ref{thm:QC-QSI}}
\label{sec:proof:QC-QSI}
% \textit{Overview of the strategy.} Sender simulates the measurement on the 

For a given $(\rho^{AB},\sfX,\calW_{\!\ \sfX \rightarrow AB})$ QC-QSI source coding setup, we choose a $P_X(x) \in \calA(\rho^{AB},\calW_{\!\ \sfX \rightarrow AB})$. From now on, we let $\Theta := 2^{nR}$  and $\Theta := 2^{n\Rbar}$.

\noindent \textbf{Codebook Design}: We generate a codebook $\calC$ consisting of $n$-length codewords by randomly and independently selecting $\Theta\times\bTheta$ sequences $\calC\deq \{\Xn(m,k): m\in [\Theta]\eqand k\in [\bTheta]\}$ according to the following pruned distribution:
 \begin{align}\label{def:qc_distribution}
     &\codeDistribution(\Xn(m) = \xn) = \left\{\!\!\!\!\begin{array}{cc}
          \dfrac{P_X^n(\xn)}{(1-\varepsilon)}  & \mbox{for} \; \xn \in \Txqc\\
           0 &  \mbox{otherwise,}
     \end{array} \right. \!\!
 \end{align} 
  where $ P_X^n(\xn) = \prod_{i=1}^n P_X(x_i)$, $\Txqc$ is the $\delta$-typical set corresponding to the distribution $P_X$ on the set $\sfX$, and $\varepsilon(\delta,n) \triangleq \sum_{\xn \not \in \Txqc} P_X^n(\xn)$. Note that $\varepsilon(\delta,n) \searrow 0$ as $n \rightarrow \infty$ and for all sufficiently small $\delta > 0$. 
  
\vspace{2pt}
\noindent\textbf{Construction of Encoding POVM}:
Let $\Pi_{\rho}^{\refe B}$ and $\Pi_{\xn}^{\refe B}$ denote the $\delta$-typical and conditional $\delta$-typical projectors defined as in \cite[Def. 15.1.3]{wilde_arxivBook} and \cite[Def. 15.2.4]{wilde_arxivBook}, with respect to $\calW^{\refe B} \deq \sum_{x\in\sfX} P_X(x)\calW^{\refe B}_x$ and $\calW_{x}^{\refe B}$, respectively.
For all $\xn \in \Txqc$, define  
\begin{align*}
    \rhotilde_{\xn}^{\refe B} \deq \hat{\Pi} \Pi_{\rho}\Pi_{\xn}\calW_{\xn}^{\refe B}\Pi_{\xn}\Pi_{\rho}\hat{\Pi} \ \! \eqand \ \! \rhotilde^{\refe B} \deq \EE_{\PP}[\rhotilde_{\Xn}^{\refe B}],
\end{align*}
and $\rhotilde_{\xn}^{\refe B} = 0$ for $\xn \notin \Txqc$, where $\hat{\Pi}$ is the cut-off 
 projector onto the subspace spanned by the eigenbasis of $\EE[\Pi_{\rho}\Pi_{\Xn}\calW_{\Xn}^{\refe B}\Pi_{\Xn}\Pi_{\rho}]$ with eigenvalues greater than $\epsilon d$, where $d \!\deq\! 2^{-n(S(\calW^{\refe B})+\delta_1)}$ and $\delta_1$ will be specified later.  
 Using the Average Gentle Measurement Lemma \cite[Lemma 9.4.3]{wilde_arxivBook}, for any given $\epsilon \in (0,1)$, and all sufficiently large $n$ and all sufficiently small $\delta$, we have 
\begin{align} \label{eq:closeness_ref_SI}
    \EE_{\PP}[\|\rhotilde_{\Xn}^{\refe B}  - \calW_{\Xn}^{\refe B} \|_1] \leq \epsilon.
\end{align}
The proof follows from the derivation of \cite[Eq. 35]{wilde_e}. Using the above definitions, for all $\xn \in \sfX^n$, we construct the operators,
\vspace{-0.065in}
\[\epovm_{\xn}^{\refe B} \deq \gamma_{\xn}\ {(\calW^{{\refe B}^{\tensor n}})}^{-1/2} \rhotilde_{\xn}^{\refe B} {(\calW^{{\refe B}^{\tensor n}})}^{-1/2} ,\text{ where }\gamma_{\xn} \!\deq\! \gamma \cdot |\{(m,k)\!:\!\Xn(m,k)\! =\! \xn\}|,\]
 $\gamma \deq (\Theta\bTheta)^{-1}  \frac{(1-\varepsilon)}{(1+\eta)}$ and $\eta \in (0,1)$ is a parameter that determines the probability of not obtaining a sub-POVM. Note that in the above definition operator  $\epovm_{\xn}^{\refe B}$ acts on $(\calH_{\refe^n} \tensor \calH_{\Bn})$, however, we define $\epovm^{A}_{\xn} \in \calL(\calH_{\An})$. To obtain this, we transform $\epovm_{\xn}^{\refe B}$ as 
\[\epovm_{\xn}^{A} = \sum_{\an \abarn} \<\an|\epovm_{\xn}^{\refe B} \ |\abarn\>_{\refe B} |\an\>\<\abarn|_{A},\]
where $\{|a\>_A\}$ is an eigenbasis of $\rho^A$, $\{|a\>_{RB}\}$ is an eigenbasis of $\rho^{\refe B} := \Tr_A\{|\phi_{\refe AB}\>\<\phi_{\refe AB}|\}$, and $|\phi_{\refe AB}\>$ is the canonical purification of $\rho^{AB}$. Furthermore, by using the equivalence of purification \cite[Thm. 5.1.1]{wilde_arxivBook}, it can be easily shown that $\Tr\{\epovm_{\xn}^A\rho^{A^{\tensor n}}\} = \Tr\{\epovm_{\xn}^{\refe B} \ \calW^{\refe B ^{\tensor n}}\}.$

Let $\I_{\{\mbox{sP}\}}$ denote the indicator random variable corresponding to the event that  $\{\epovm_{\xn}^{A} \colon \xn \in  \Txqc\}$ forms a  sub-POVM. We now provide a proposition from \cite{winter}, which will be helpful later in the analysis.
\begin{prop} \label{prop:enc_subpovm}For all $\epsilon, \eta \in (0,1)$, for all sufficiently small $\delta > 0$, and for all sufficiently large $n$, we have
$\EE[{\I_{\{\mbox{\normalfont sP}\}}}] \geq 1-\epsilon$, if $\frac{1}{n}(\log(\Theta)+\log(\bTheta)) > \chi(\{P_X(x),\calW^{\refe B}_x\})$.
\end{prop}
If $\I_{\{\mbox{sP}\}} = 1$, then construct sub-POVM $\Gamma^{(n)}_{\sfA}$ as follows: $\Gamma^{(n)}_{\sfA} \deq \big\{\sum_{k\in[\bTheta]}\epovm_{\xn(m,k)}^{A}\big\}_{m\in[\Theta]}.$ 
We then add an additional operator $\epovm_{0}^A \deq (I\!-\!\sum_{m\in[\Theta]}\sum_{k\in [\bTheta]}\epovm_{\xn(m,k)}^A)$, associated with an arbitrary sequence $\xn_0 \in \sfX^n \backslash\Txqc$, to form a valid POVM $[\Gamma^{(n)}_\sfA]$ with at most $(\Theta \times \bTheta+1)$ elements. The extra element $\epovm_{\xn_0}^A$ corresponds to a failed encoding.
% If $\I_{\{\mbox{sP}\}} = 0$, then we define $\Gamma^{(n)} = \{I\}$ and associate it with $\xn_0$. 

\vspace{2pt}
\noindent\textbf{Construction of Decoding POVM}:
For the ensemble $\{P_X(x),\calW^B_x\}$, we construct a collection of $n$-letter Bob's POVMs, one for each $m \in [\Theta]$, capable of decoding the message $k\in [\bTheta]$. 
Upon receiving the message $m$, Bob performs a sequence of binary measurements $\{\Pi_{\xn(m,k)},(I-\Pi_{\xn(m,k)})\}$ for all sequence $\xn(m,k) \in \calC$, where $\Pi_{\xn(m,k)}$ is a conditional typical projector for the tensor-product state $\calW^B_{\xn(m,k)}$.
% (Bob's system after reconstructing $\xn(m,k')$ and applying the tensor-product CQ channel $\calW_{X_i\rightarrow RB}^{\tensor n}$). 
% Let $a^{(m)}_{1}, a^{(m)}_{2}, \cdots, a^{(m)}_{\Rbar}$ enumerate all the codewords for a fixed $ m\in [2^{nR}]$ and let $a_j^{(m)}$ denote the correct codeword $\xn(m,k)$ produced from the outcomes of Alice's POVM.
% Define the decoding POVM as $$\dpovm_{k}^{(m)} \deq \Pibar_{a^{(m)}_{1}} \cdots \Pibar_{a^{(m)}_{j-1}} \Pi_{a^{(m)}_{j}} \Pibar_{a^{(m)}_{j-1}} \cdots   \Pibar_{a^{(m)}_{1}},$$
Define the decoding POVM element as
$$\dpovm_{k}^{(m)} \deq \Pibar^{(m)}_{1} \cdots \Pibar^{(m)}_{k-1} \
\Pi^{(m)}_{k} \ \Pibar^{(m)}_{k-1} \cdots \Pibar^{(m)}_{1},$$
where $ \Pibar^{(m)}_{k}$ and $ \Pi^{(m)}_{k}$ are the shorthand notation for $(I-\Pi_{\xn(m,k)})$ and $\Pi_{\xn(m,k)}$, respectively.
% Let $\Bar{\rho}_{m,l}^{\refe B} \deq (\Tr\{\epovm_{m,l}^{A^n} \rho_A^{\tensor n}\})^{-1}\Tr_{A^n}\{({I^{\refe B}}^{\tensor n}\!\tensor \epovm_{m,l}^{A^n})({\Phi^{\refe AB}}^{\tensor n})\}$ be the normalized post-measurement state from the Bob's encoding,  where $\epovm_{m,l}^{A^n}$ is used as a shorthand notation for $\epovm_{\xn(m,l)}^{A^n}$.
The following proposition demonstrates that for Bob's POVMs, we can make the average probability of error arbitrarily small by using the non-commutative union bound \cite{sen2012achieving}.
\begin{prop}\label{prop:qc_packing}
    Given the ensemble $\{P_X(x),\calW^B_x\}$ and the collection of POVMs $\{\Xi_{k}^{(m)}\}_{k\in[\bTheta]}$, for any $\epsilon \in (0,1)$
$$\EE_\PP\left[\frac{1}{\bTheta}\sum_{k\in[\bTheta]} \Tr\left\{(I-\dpovm_{k}^{(m)})\calW_{m,k}^B\right\}\right] \leq \epsilon,$$
for sufficiently small $\delta>0$ and for all sufficiently large $n$, and for all $m \in [\Theta]$, if $\frac{1}{n}\log(\bTheta) < \chi(\{P_X(x),\calW^B_x\})$. 
\end{prop}
\begin{proof}
    The proof follows from packing lemma using sequential decoding \cite[Sec. 16.6]{wilde_arxivBook}, while making the following identification. For each $m\in\calM$, identify $\calM$ as $[\bTheta]$, $\calX$ as $\Txqc$, $\{\sigma_{C_m}\}_m$ with $\{\calW_{m,k}^{\refe B}\}_k$, $\Pi_x$ with $\Pi_{k}^{(m)}$, $d$ with $2^{n(S(X|B)_{\tau}-\Bar{\delta})}$, and $D$ with $2^{n(S(B)_\tau-\Bar{\delta})}$, where $\tau^{XB} := \sum_x P_X(x) |x\>\<x|\tensor \calW^B_x$ and $\Bar{\delta} \searrow 0$ as $\delta\searrow 0$. 
    % For more details, please refer to \cite{wilde2013sequential,wilde_e}. 
\end{proof}
In general, the decoding POVM elements satisfy the condition $\sum_{k \in [\bTheta]} \dpovm_{k}^{(m)} \leq I$ for all $m \in [\Theta].$
Under the condition $\{\I_{\{\mbox{sP}\}} = 1\}$,
construct sub-POVM $\Gamma_\sfB^{(n)}$ as follows: $\{\dpovm_{k}^{(m)}\}_{k\in [\bTheta]}$ for each $m\in[\Theta]$. This sub-POVM is completed by adding an additional operator $\dpovm_{0}^{(m)}\deq(I\!-\!\sum_{k\in[\bTheta]}\dpovm_k^{(m)})$ to form a valid POVM $[\Gamma_\sfB^{(n}]$, for each $m\in [\Theta]$.

% %%%%%%%%%%%%%%%%%%%%%%%%%%%%%%%%%%%%%%%%%%%%%
% \noindent\textbf{Decoding Unitary:} Observe that after applying the binary projectors Bob recovers the $(m,k)$, and consequently the sequence $\xn(m,k)$. The post-measured state after the sequential decoding can be expressed as
% $$\frac{1}{\Tr\{\dpovm_k^{(m)}\Bar{\rho}^{B}_{m,k}\}}({I^{\refe}}^{\tensor n}\tensor \Pi_{k}^{(m)}\Pibar_{k-1}^{(m)}\cdots \Pibar_{1}^{(m)})\Bar{\rho}^{\refe B}_{m,k} ({I^{\refe}}^{\tensor n}\tensor \Pibar_{1}^{(m)} \cdots \Pibar_{k-1}^{(m)}\Pi_{k}^{(m)}),$$ where $$\Bar{\rho}_{m,k}^{\refe B} \deq \frac{1}{\Tr\{\epovm_{m,k}^{A^n} \ {\rho^A}^{\tensor n}\}}\Tr_{A^n}\{({I^{\refe B}}^{\tensor n}\!\tensor \epovm_{m,k}^{A^n})({\Phi^{\refe AB}}^{\tensor n})\}$$ is the normalized post-measurement state from Alice's encoding. Here, $\epovm_{m,k}^{A^n}$ is used as a shorthand notation for $\epovm_{\xn(m,k)}^{A^n}$, $\Phi^{\refe A B}:= |\phi\>\<\phi|^{\refe AB}$, and $|\phi\>^{\refe AB} $ is the canonical purification of $\rho^{AB}$.
% Note that the projectors $\Pi_{k}^{(m)}\cdots \Pi_{1}^{(m)}$ and the POVM element ${\Lambda_{k}^{(m)}}$ are related by a polar decomposition, given as
% \begin{align*}
%     \sqrt{\Lambda_{k}^{(m)}} = U_{m,k} \ \Pi_{k}^{(m)}\Pibar_{k-1}^{(m)}\cdots \Pibar_{1}^{(m)},
% \end{align*}
% for some unitary $U_{m,k}$. We make the use of this relation to design the decoding unitary as follows: Bob applies the binary projectors and recovers $\xn(m,k)$. Following this, Bob applies the unitary $U_{m,k}$ and the state becomes as follows:
% \begin{equation}\label{eqn:final_enc_dec}
%     \omega_{m,k}^{\refe B}\deq \frac{1}{\Tr\{\Lambda_{k}^{(m)}\Bar{\rho}^B_{m,k}\}}\left({I^{\refe}}^{\tensor n}\tensor \sqrt{\Lambda_{k}^{(m)}}\right)\Bar{\rho}^{\refe B}_{m,k}\left({I^{\refe}}^{\tensor n}\tensor \sqrt{\Lambda_{k}^{(m)}}\right).
% \end{equation}
% Finally, 

\noindent\textbf{Error Analysis}: We begin by defining the following code-dependent random variables \(E_1\), \(E_2\), and \(E_3\), which will be useful in the error analysis, given as:
\begin{align*}
    E_1 \deq \sum_{m\in [\Theta]}\sum_{k\in [\bTheta]} {(\Theta\bTheta)}^{-1}\ &\Tr\{\rhotilde_{m,k}^{\refe B}\}\eqand E_2 \deq \sum_{m\in [\Theta]}\sum_{k\in [\bTheta]}{(\Theta\bTheta)}^{-1}\ \|\rhotilde_{m,k}^{\refe B} - \calW^{\refe B}_{m,k}\|_1,
    % \\
    % \text{and }E_3&:=\sum_{m\in [\Theta]} \sum_{k\in[\bTheta]}{(\Theta\bTheta)}^{-1} \ \Tr\big\{(I-\dpovm_{k}^{(m)})\calW_{m,k}^B\big\},
\end{align*}
where $\rhotilde_{m,k}^{\refe B}$ and $\calW_{m,k}^{\refe B}$ are the shorthand notation for $\rhotilde_{\xn(m,k)}^{\refe B}$ and $\calW_{\xn(m,k)}^{\refe B},$ respectively. We provide the following proposition that bound these terms under the condition $\I_{\curly{\mbox{\normalfont sP}}} = 1$.
\begin{prop}\label{prop:code_dependent_RV}
For all $\epsilon\in(0,1)$, for all sufficiently small $\eta, \delta>0$, and for all sufficiently large $n$, we have $\EE_\PP[E_1]\geq (1-\epsilon) \eqand  \EE_\PP[E_2]\leq \epsilon$.
\end{prop}
\begin{proof}
    The proof is provided in Appendix \ref{app:prop:code_dependent_RV}.
\end{proof}
Now, Observe that after applying the binary projectors Bob generates $(m,k')$, and consequently the sequence $\xn(m,k')$ using the decoding map $f$. The (unnormalized) post-measured state after the sequential decoding can be expressed as
$$({I^{\refe}}^{\tensor n}\tensor \Pi_{k}^{(m)}\Pibar_{k-1}^{(m)}\cdots \Pibar_{1}^{(m)})\omega^{\refe B}_{m,k} ({I^{\refe}}^{\tensor n}\tensor \Pibar_{1}^{(m)} \cdots \Pibar_{k-1}^{(m)}\Pi_{k}^{(m)}),$$ where 
\vspace{-10pt}\begin{equation}\label{eqn:omegamk}
    \omega_{m,k}^{\refe B} \deq \Tr_{A^n}\{({I^{\refe B}}^{\tensor n}\!\tensor \epovm_{m,k}^{A})({\phi^{\refe AB}}^{\tensor n})\} 
\end{equation}
is the unnormalized post-measured state from Alice's encoding. Here, $\omega_{m,k}^{\refe B} \eqand  \epovm_{m,k}^{A}$ are used as a shorthand notation for $\omega_{\xn(m,k)}^{\refe B} \eqand \epovm_{\xn(m,k)}^{A}$, respectively, and  $\phi^{\refe A B}:= |\phi\>\<\phi|^{\refe AB}$ and $|\phi\>^{\refe AB} $ is the canonical purification of $\rho^{AB}$. Furthermore, note that the projectors $\Pi_{k}^{(m)}\cdots \Pi_{1}^{(m)}$ and the POVM element ${\Lambda_{k}^{(m)}}$ are related by a polar decomposition, given as
\begin{align*}
    \sqrt{\Lambda_{k}^{(m)}} = U_{m,k} \ \Pi_{k}^{(m)}\Pibar_{k-1}^{(m)}\cdots \Pibar_{1}^{(m)},
\end{align*}
for some unitary $U_{m,k}$. Therefore, Bob first applies the binary projectors and constructs $\xn(m,k')$. Following this, Bob applies the unitary $U_{m,k}$ and the (unnormalized) state becomes as follows:
\begin{equation}\label{eqn:lambdamk}
    \lambda_{(m,k),k'}^{\refe B}\deq\Big({I^{\refe}}^{\tensor n}\tensor \sqrt{\Lambda_{k'}^{(m)}}\Big)\omega^{\refe B}_{m,k}\Big({I^{\refe}}^{\tensor n}\tensor \sqrt{\Lambda_{k'}^{(m)}}\Big).
\end{equation}
% \begin{align}
% \widetilde{\Pi}_{k}^{(m)}&\deq\Pi_{k}^{(m)}\Pibar_{k-1}^{(m)}\cdots \Pibar_{1}^{(m)}\\
%     \omega_{m,k}^{\refe B} &\deq \frac{1}{\Tr\{\epovm_{m,k}^{A^n} \ {\rho^A}^{\tensor n}\}}\Tr_{A^n}\{({I^{\refe B}}^{\tensor n}\!\tensor \epovm_{m,k}^{A^n})({\Phi^{\refe AB}}^{\tensor n})\} \\
%     % \lambda_{m,k}^{\refe B} &= \frac{1}{\Tr\{\dpovm_k^{(m)}\omega^{B}_{m,k}\}}({I^{\refe}}^{\tensor n}\tensor \widetilde{\Pi}_{k}^{(m)})\omega^{\refe B}_{m,k} ({I^{\refe}}^{\tensor n}\tensor \widetilde{\Pi}_{k}^{(m)})^{\dagger}\\
%     \lambda_{(m,k),k'}^{\refe B} &= \frac{1}{\Tr\{\dpovm_{k'}^{(m)}\omega^{B}_{m,k}\}}({I^{\refe}}^{\tensor n}\tensor \widetilde{\Pi}_{k'}^{(m)})\omega^{\refe B}_{m,k} ({I^{\refe}}^{\tensor n}\tensor \widetilde{\Pi}_{k'}^{(m)})^{\dagger}, 
%     % \quad \text{for $k\neq k'$},
% \end{align}
% \begin{align}
% \widetilde{\Pi}_{k}^{(m)}&\deq\Pi_{k}^{(m)}\Pibar_{k-1}^{(m)}\cdots \Pibar_{1}^{(m)}\label{eqn:pitilde}\\
%     \omega_{m,k}^{\refe B} &\deq \Tr_{A^n}\{({I^{\refe B}}^{\tensor n}\!\tensor \epovm_{m,k}^{A})({\phi^{\refe AB}}^{\tensor n})\} \label{eqn:omegamk}\\
%     % \lambda_{m,k}^{\refe B} &= \frac{1}{\Tr\{\dpovm_k^{(m)}\omega^{B}_{m,k}\}}({I^{\refe}}^{\tensor n}\tensor \widetilde{\Pi}_{k}^{(m)})\omega^{\refe B}_{m,k} ({I^{\refe}}^{\tensor n}\tensor \widetilde{\Pi}_{k}^{(m)})^{\dagger}\\
%     \lambda_{(m,k),k'}^{\refe B} &= ({I^{\refe}}^{\tensor n}\tensor \widetilde{\Pi}_{k'}^{(m)})\omega^{\refe B}_{m,k} ({I^{\refe}}^{\tensor n}\tensor \widetilde{\Pi}_{k'}^{(m)})^{\dagger}, \label{eqn:lambdamk}
%     % \quad \text{for $k\neq k'$},
% \end{align}
% where $\omega_{m,k}^{\refe B}, \lambda_{(m,k),k'}^{\refe B} \eqand  \epovm_{m,k}^{A}$ are used as a shorthand notation for $\omega_{\xn(m,k)}^{\refe B}, \lambda_{\xn(m,k')}^{\refe B}$, and $\epovm_{\xn(m,k)}^{A}$, respectively, and  $\phi^{\refe A B}:= |\phi\>\<\phi|^{\refe AB}$ and $|\phi\>^{\refe AB} $ is the canonical purification of $\rho^{AB}$.
% Here, $\omega_{m,k}^{\refe B}$ is the unnormalized post-measurement state from Alice's encoding, and $\lambda_{(m,k),k'}^{\refe B}$ is the unnormalized start after the sequential decoding. 
If $k=k'$ (indicating correct decoding), then $\lambda_{(m,k),k'}^{\refe B} = \lambda_{m,k}^{\refe B}$, i.e., Bob successfully recovers the sequence $\xn(m,k)$.
Now, following Definition \ref{def:qc_qsi_achievability}, our objective is to show that the following term 
\begin{align*}
    \EE_\PP[&\error] \\&= \EE_\PP\Bigg[ \Big\|\sum_{m \in [\Theta]\cup \{0\}}\sum_{\substack{k\in [\bTheta]\cup \{0\}}}\sum_{k'\in [\bTheta]\cup \{0\}} \!\!|\xn(m,k')\>\<\xn(m,k')|\tensor \big(\lambda_{(m,k),k'}^{\refe B} -  \Tr\{\lambda_{(m,k),k'}^{\refe B}\}\calW_{m,k'}^{\refe B}\big)\Big\|_1\Bigg]
\end{align*}
can be made arbitrarily small for sufficiently large $n$ for the code $\codebook$, where $\calW_{m,k'}^{\refe B}\deq \calW_{\xn(m,k')}^{\refe B}.$ 

First, we split the error $ \error $ into two terms using the indicator function $\I_{\curly{\mbox{\normalfont sP}}}$ as 
\begin{align}
\error&=\I_{\curly{\mbox{\normalfont sP}}} 
\error + \round{1- \I_{\curly{\mbox{\normalfont sP}}}}\error\\
&\leq \I_{\curly{\mbox{\normalfont sP}}} 
\error  + 2\round{1- \I_{\curly{\mbox{\normalfont sP}}}} \label{eqn:qcerrorsubpovm},
\end{align}
% Consider the un-normalized post-measured states on $\Bn$: $\rhotilde^B_{(m,k)} = \Tr_{\An}\{(\epovm_{(m,k)}^A\tensor I)(\rho^{{AB}^{\tensor n}})\}.$ Define, $\lambda_{(m,k)} := \Tr\{(\epovm_{(m,k)}^A\tensor I)(\rho^{{AB}^{\tensor n}})\}$, the probability of observing $(m,k)$ for Alice.
% The probability of correct decoding for Bob is $$\mu_{(m,k)}\! \deq \!\Tr\{\Pi_{a^{(m)}_{j}} \Pibar_{a^{(m)}_{j-1}}\!\cdots \!\Pibar_{a^{(m)}_{1}} \rhotilde^B_{a^{(m)}_{j}}\Pibar_{a^{(m)}_{1}} \!\cdots \!\Pibar_{a^{(m)}_{j-1}} \Pi_{a^{(m)}_{j}}\}$$
where \eqref{eqn:qcerrorsubpovm} follows from upper bounding the trace distance between two density operators by its maximum value of two. 
Under the condition $\I_{\curly{\mbox{\normalfont sP}}} = 1$, 
 \begin{align*}
     \error &\overset{}{\leq} \underbrace{\sum_{m \in [\Theta]}\sum_{\substack{k\in [\bTheta]}}\|\lambda_{m,k}^{\refe B} -  \Tr\{\lambda_{m,k}^{\refe B}\}\calW_{m,k}^{\refe B}\|_1}_{\zeta_{\text{CP}}} 
     + \ 2 \!\!\!\!\underbrace{\sum_{\substack{k'\in [\bTheta]\cup \{0\}}}\!\!\!\! \Tr\{\lambda_{(0,0),k'}^{\refe B}\}}_{\zeta_{\text{NC}}}
     \\
     &\hspace{50pt}+2\!\!\underbrace{\sum_{m \in [\Theta]}\sum_{\substack{k \neq k' \in [\bTheta]}}  \Tr\{\lambda_{(m,k),k'}^{\refe B}\}}_{\zeta_{\text{NP}_1}}  +   \ 2\underbrace{\sum_{m \in [\Theta]}\sum_{\substack{k\in [\bTheta]}} \Tr\{\lambda_{(m,k),0}^{\refe B}\}}_{\zeta_{\text{NP}_2}},
 \end{align*}
 where the inequality follows from the fact that $\|\calW_{\xn}\|_1=1$ for all $\xn\in \sfX^n$ and  using triangle inequality.


\noindent \textbf{Step 1. Bounding the error induced by not covering, i.e., encoding error.} 

\noindent The error term $\zeta_{\text{NC}}$ captures the error induced by not covering the $n$-tensored posterior reference channel. We provide the following proposition that bounds this term.  
\begin{prop}\label{prop:qc_NC}
    For all $\epsilon\in(0,1)$, for all sufficiently small $\eta, \delta>0$, and for all sufficiently large $n$, we have $\EE_\PP[\I_{\curly{\mbox{\normalfont sP}}}\zeta_{\text{NC}}]\leq \epsilon$.
\end{prop}
\begin{proof}
The proof is provided in Appendix \ref{app:prop:proof:qc_NC}.
\end{proof}
% \textbf{EXTRA}: $1-\frac{(1-\varepsilon)}{(1+\eta)}\sum_{\xn \in \Txqc}\frac{P_\Xn(\xn)}{(1-\varepsilon)}\Tr\{\rhotilde_{\xn}^{\refe B}\}]$
%%%%%%%%%%%%%%%%%%%%%%%%%%%%%
\noindent \textbf{Step 2. Bounding the error induced by not packing, i.e., decoding error.} 

\noindent The error term $\zeta_{\text{NP}} \deq \zeta_{\text{NP}_1} + \zeta_{\text{NP}_2}$ captures the error induced by not packing, i.e., incorrect decoding. Therefore, $\zeta_{\zeta_{\text{NP}}}$ can rewritten as the $(1-$ probability of correct decoding), given as
\begin{align}
\zeta_{\text{NP}} &= \sum_{m \in [\Theta]}\sum_{\substack{k \neq k' \in [\bTheta]}}  \Tr\{\lambda_{(m,k),k'}^{\refe B}\} +  \sum_{m \in [\Theta]}\sum_{\substack{k\in [\bTheta]}} \Tr\{\lambda_{(m,k),0}^{\refe B}\}\nonumber\\
&= \sum_{m \in [\Theta]}\sum_{\substack{k\in [\bTheta]}} \Big(\Tr\big\{\big({I^{\refe}}^{\tensor n} \!\tensor\!\!\!\!\sum_{\substack{k' \neq k \in [\bTheta]}} \!\!\!\!\dpovm_{k'}^{(m)} \big)\omega^{RB}_{m,k}\big\}   +   \Tr\big\{\big({I^{\refe}}^{\tensor n} \!\!\tensor \dpovm_{0}^{(m)} \big)\omega^{RB}_{m,k}\big\}\}\Big)\nonumber\\
& = \sum_{m \in [\Theta]}\sum_{\substack{k\in [\bTheta]}}
\Tr\big\{\big(I-\dpovm_{k}^{(m)} \big)\omega^{B}_{m,k}\big\}\}\label{eqn:zetaNP},
\end{align}
where the second equality follows from \eqref{eqn:lambdamk} and $\omega^B_{m,k} = \Tr_{\refe^n}\{\omega^{\refe B}_{m,k}\}$. We provide the following proposition that bounds $\zeta_{\text{NP}}$.
\begin{prop}\label{prop:qc_NP}
    For all $\epsilon\in(0,1)$, for all sufficiently small $\eta, \delta>0$, and for all sufficiently large $n$, we have $\EE_\PP[\I_{\curly{\mbox{\normalfont sP}}}\zeta_{\text{NP}}]\leq 2\epsilon$.
\end{prop}
\begin{proof}
The proof is provided in Appendix \ref{app:prop:proof:qc_NP}.
\end{proof}
%%%%%%%%%%%%%%%%%%%%%%%%%%%%%
\noindent \textbf{Step 2. Bounding the error induced by covering and packing.} 

\noindent The error term $\zeta_{\text{CP}}$ captures the error induced by covering and packing. Consider the following inequalities:
\begin{align*}
    \zeta_{\text{CP}} 
    % &\leq \sum_{m \in [\Theta]}\sum_{\substack{k\in [\bTheta]}}\|\lambda_{m,k}^{\refe B} -  \Tr\{\lambda_{m,k}^{\refe B}\}\omega^{\refe B}_{m,k}\|_1 + \sum_{m \in [\Theta]}\sum_{\substack{k\in [\bTheta]}}\Tr\{\lambda_{m,k}^{\refe B}\}\|\omega_{m,k}^{\refe B} -  \calW_{m,k}^{\refe B}\|_1\\
    &\overset{a}{\leq} \sum_{m \in [\Theta]}\sum_{\substack{k\in [\bTheta]}} \Tr\{\omega^{\refe B}_{m,k}\}\|(\Tr\{\omega^{\refe B}_{m,k}\})^{-1}\lambda_{m,k}^{\refe B} -  \omegabar^{\refe B}_{m,k}\|_1 \\
    &\hspace{75pt}+ \sum_{m \in [\Theta]}\sum_{\substack{k\in [\bTheta]}} \Tr\{\omega^{\refe B}_{m,k}\} \|\omegabar^{\refe B}_{m,k} - (\Tr\{\omega^{\refe B}_{m,k}\})^{-1}\Tr\{\lambda_{m,k}^{\refe B}\}\omegabar^{\refe B}_{m,k}\|_1
    \\
    &\hspace{150pt}+ \sum_{m \in [\Theta]}\sum_{\substack{k\in [\bTheta]}}\Tr\{\lambda_{m,k}^{\refe B}\}\|\omegabar_{m,k}^{\refe B} -  \calW_{m,k}^{\refe B}\|_1\\
    %%%%%%%%%%%%
    &\overset{b}{=} \sum_{m \in [\Theta]}\sum_{\substack{k\in [\bTheta]}}\Tr\{\omega^{\refe B}_{m,k}\} \Big\|\Big({I^{\refe}}^{\tensor n} \!\tensor \sqrt{\Lambda_{k}^{(m)}}\Big)\omegabar^{\refe B}_{m,k}\Big({I^{\refe}}^{\tensor n} \!\tensor \sqrt{\Lambda_{k}^{(m)}}\Big) -  \omegabar^{\refe B}_{m,k}\Big\|_1 
    \\
    &\hspace{75pt} + \sum_{m \in [\Theta]}\sum_{\substack{k\in [\bTheta]}}\big(\Tr\{\omega^{\refe B}_{m,k}\}-\Tr\{({I^{\refe}}^{\tensor n} \!\tensor \Lambda_{k}^{(m)})\omega_{m,k}^{\refe B}\}\big) \|\omegabar^{\refe B}_{m,k}\|_1 + \zeta_{\text{C}}\\
    % &\hspace{150pt}+ \sum_{m \in [\Theta]}\sum_{\substack{k\in [\bTheta]}}\Tr\{\lambda_{m,k}^{\refe B}\}\|\omegabar_{m,k}^{\refe B} -  \calW_{m,k}^{\refe B}\|_1\\
    %%%%%%%%%%%%%%%%%%%%%
    &\overset{c}{\leq}
    \sum_{m \in [\Theta]}\sum_{\substack{k\in [\bTheta]}} 2 \ \Tr\{\omega^{\refe B}_{m,k}\} \sqrt{\Tr\Big\{\Big( I-\big({I^{\refe}}^{\tensor n} \!\tensor \Lambda_{k}^{(m)}\big)\Big)\omegabar^{\refe B}_{m,k}\Big\}} 
    \\&\hspace{75pt}+\sum_{m \in [\Theta]}\sum_{\substack{k\in [\bTheta]}}\Tr\Big\{\Big( I-\big({I^{\refe}}^{\tensor n} \!\tensor \Lambda_{k}^{(m)}\big)\Big) \omega^{\refe B}_{m,k}\Big\}
+ \zeta_{\text{C}}\\
    % &\hspace{150pt}+ \sum_{m \in [\Theta]}\sum_{\substack{k\in [\bTheta]}}\Tr\{\lambda_{m,k}^{\refe B}\}\|\omegabar_{m,k}^{\refe B} -  \calW_{m,k}^{\refe B}\|_1\\
     %%%%%%%%%%%%%%%%%%%%%
    &\overset{d}{\leq}
    2 \sqrt{\sum_{m \in [\Theta]}\sum_{\substack{k\in [\bTheta]}}  \Tr\{\omega^{\refe B}_{m,k}\} \Tr\Big\{\Big( I-\big({I^{\refe}}^{\tensor n} \!\tensor \Lambda_{k}^{(m)}\big)\Big)\omegabar^{\refe B}_{m,k}\Big\}} + \zeta_{\text{NP}} + \zeta_{\text{C}}\\
    % \\&\hspace{100pt}+\sum_{m \in [\Theta]}\sum_{\substack{k\in [\bTheta]}}\Tr\Big\{\Big( I-\big({I^{\refe}}^{\tensor n} \!\tensor \Lambda_{k}^{(m)}\big)\Big) \omega^{\refe B}_{m,k}\Big\}
    % \\
    % &\hspace{150pt}+ \sum_{m \in [\Theta]}\sum_{\substack{k\in [\bTheta]}}\Tr\{\lambda_{m,k}^{\refe B}\}\|\omegabar_{m,k}^{\refe B} -  \calW_{m,k}^{\refe B}\|_1
    &\overset{e}{\leq} 3\sqrt{\zeta_{\text{NP}}}+ \zeta_{\text{C}},
\end{align*}
where $(a)$ follows from adding and subtracting appropriate terms, defining $\omegabar^{\refe B}_{m,k}\deq (\Tr\{\omega^{\refe B}_{m,k}\})^{-1} \omega^{\refe B}_{m,k}$, and applying triangle inequality, $(b)$ follows from \eqref{eqn:lambdamk} and the definition $$\zeta_{\text{C}}\deq \sum_{m \in [\Theta]}\sum_{\substack{k\in [\bTheta]}}\Tr\{\lambda_{m,k}^{\refe B}\}\|\omegabar_{m,k}^{\refe B} -  \calW_{m,k}^{\refe B}\|_1,$$ $(c)$ follows from Gentle Operator Lemma \cite[Lemma 9.4.2]{wilde_arxivBook}, $(d)$ follows from \eqref{eqn:zetaNP} and applying Jensen's inequality for concave functions, and $(e)$ is based on the fact that $x \leq \sqrt{x}$ for all $x\in [0,1]$.
We provide the following proposition that bounds $\zeta_{\text{C}}$.
\begin{prop}\label{prop:qc_cov}
    For all $\epsilon\in(0,1)$, for all sufficiently small $\eta, \delta>0$, and for all sufficiently large $n$, we have $\EE_\PP[\I_{\curly{\mbox{\normalfont sP}}}\zeta_{\text{C}}]\leq 2\epsilon$.
\end{prop} 
\begin{proof}
The proof is provided in Appendix \ref{app:prop:proof:qc_cov}.
\end{proof}
Using Propositions \ref{prop:qc_NP} and \ref{prop:qc_cov}, we now bound the error term $\zeta_{\text{CP}}$. For all $\epsilon\in (0,1)$
\begin{align}
\EE_\PP[&\I_{\curly{\mbox{\normalfont sP}}}\zeta_{\text{CP}}] = \EE_\PP[3\sqrt{\zeta_{\text{NP}}}+\zeta_{\text{C}}] \leq 3\sqrt{\EE_\PP[\I_{\curly{\mbox{\normalfont sP}}} \zeta_{\text{NP}}}]+\EE_{\PP}[\I_{\curly{\mbox{\normalfont sP}}}\zeta_{\text{C}}] \leq 3\sqrt{2\epsilon}+2\epsilon\label{eqn:zetaCP}
\end{align}
where the first inequality follows from Jensen's inequality for concave functions. Finally, using Propositions \ref{prop:qc_NC} and  \ref{prop:qc_NP}, and \eqref{eqn:zetaCP}, we bound $\error$, for all $\epsilon\in(0,1),$
\begin{align*}
    \EE_\PP[\error]&\leq \EE_\PP[\I_{\curly{\mbox{\normalfont sP}}} 
\error  + 2(1- \I_{\curly{\mbox{\normalfont sP}}})] \\&\leq \EE_\PP[\I_{\curly{\mbox{\normalfont sP}}} 
\error] +2\epsilon \leq 3\sqrt{2\epsilon}+10\epsilon.
\end{align*}
Since $\EE_\PP[\error]\leq 3\sqrt{2\epsilon}+10\epsilon$, there exists a codebook $\codebook$ and the associated POVMs $\Gamma^{(n)}_\sfA$ and $\Gamma^{(n)}_\sfB$ such that $\error\leq 3\sqrt{2\epsilon}+10\epsilon$. This completes the proof of Theorem \ref{thm:QC-QSI}.

\section{Proof of Theorem \ref{thm:C-CSI}}
\label{sec:proof:C-CSI}
\noindent For a given $(\pxz,\sfY,W_{X|YZ})$ C-CSI source coding setup, we choose the distributions $(P_{U|X},P_{Y|UZ}) \in \calA(P_{XZ},W_{X|YZ})$. 
% From now on, let $\Theta \deq [2^{nR}]$ and $\bTheta \deq [2^{n\Rbar}]$

\noindent \textbf{Codebook Construction}:
We generate a codebook $\codebook$ consisting of $n$-length codewords by randomly and independently selecting $2^{n\Rbar}$ sequences $\{\Un(1),\Un(2),\cdots, \Un(2^{n\Rbar})\}$ according to the following pruned distribution:
 \begin{align}\label{def:c_distribution}
     &\codeDistribution(\Un(m) = \un) = \left\{\!\!\!\!\begin{array}{cc}
          \dfrac{\targetpu^n(\un)}{(1-\varepsilon)}  & \mbox{for} \; \un \in \Tuqc\\
           0 &  \mbox{otherwise,}
     \end{array} \right. \!\!
 \end{align} 
  where $ \targetpu^n(\un) = \prod_{i=1}^n \targetpu(u_i)$, $\Tuqc$ is the $\delta$-typical set corresponding to the distribution $\pu$ on the set $\calU$, and $\varepsilon(\delta,n) \triangleq \sum_{\un \not \in \Tuqc} \targetpu^n(\un)$. Note that $\varepsilon(\delta,n) \searrow 0$ as $n \rightarrow \infty$ and for all sufficiently small $\delta > 0$. 
 The generated codebook $\calC$ is revealed to the encoder and decoder before the C-CSI protocol begins.


\vspace{5pt}
\noindent \textbf{Encoder Description}:
For an observed source sequence $\xn$, construct a randomized encoder that chooses an index $l \in [2^{n\Rbar}]$ according to a sub-PMF $E_{L|X^n}(l|x^n)$
% \footnote{A non-negative function $q_X(x)$ over a finite alphabet $\calX$ is said to be a sub-PMF if $\sum_{x\in \calX} q_{X}(x) \leq 1$.}
, which is analogous to the likelihood encoders used in  \cite{cuff2013distributed, atif2022source,sohail2023unique}. We now specify $E_{L|X^n}(l|x^n)$ for $x^n\in \Tx$ and $l\in[2^{n\Rbar}]$, where $\hat{\delta} = \delta(|\calX| + |\calU|)$. For a  $\eta \in (0,1)$ (to be specified later), and $\delta>0$, define
\begin{align}
 E_{L|X^n}&(l|x^n) \deq \sum_{\un}   \frac{1}{2^{n\Rbar}}\frac{(1-\varepsilon)}{(1+\eta)}\frac{P^n_{X|U}(x^n|u^n)}{\px^n(x^n)}\I_{\{\xn\in \Tx\}} 
  \I_{\{\un \in \Tucond\}}
  \I_{\{\Un(l) = \un\}},\nonumber
\end{align}
% where $\prevTC^n(x^n|\hat{x}^n)=\prod_{i=1}^n \prevTC(x_i|\hat{x}_i)$ and $\px^n(x^n) = \prod_{i=1}^n\px(x_i)$.
% where $\px^n(x^n) \!\deq \!\prod_{i=1}^n\px(x_i)$. Similar to the encoder specification in \cite{atif2022source}, we also have relaxed the constraint that $E_{M|X^n}(\cdot|x^n)$ is strictly a PMF, i.e, $\sum_{m = 1}^{2^{nR}} E_{M|X^n}(m|x^n) = 1$. 
Let $\Ipmf$ denotes the indicator random variable corresponding to the event 
that $\{E_{L|\Xn}(l|\xn)\}_{l \in [\Theta]}$ forms a sub-PMF for all $\xn\in\Tx $. Once the index $l$ is chosen, it gets mapped to an index $m \in [2^{nR}]$. The mapping is done using a binning map $\calB:[2^{n\Rbar}] \rightarrow [2^{nR}]$. To summarize, on observing $\xn$, the encoder chooses $L \in [2^{n\Rbar}]$ stochastically according the PMF $E_{L|\Xn}(\cdot|\xn)$, and communicate the index $\calB(L)$ to the decoder. 

\vspace{3pt}
\noindent After specifying the PMF $E_{L|\Xn}(\cdot|\xn)$, we now characterize $P_{M|\Xn}$. If $\Ipmf \!\! = \!\! 1$, then construct the sub-PMF $P_{M|\Xn}(m|\xn) \!\deq\! \sum_{l \in [2^{n\Rbar}]}E_{L|\Xn}(l|\xn) \I_{\set{\calB(l) = m}}$, for all $\xn\in\Tx \eqand l\in [2^{n\Rbar}].$ We then append an additional PMF element $P_{L|\Xn}(0|\xn) = E_{L|\Xn}(0|\xn) \deq 1-\sum_{l\in[2^{n\Rbar}]} E_{L|X^n}(l|x^n)$ for all $\xn \in \Tx$, associated with $m=0$, to form a valid PMF $P_{M|\Xn}(m|\xn)$ for all $\xn \in \Tx$ and $ m\in\set{0} \cup [2^{nR}]$. If $\xn \not \in \Tx$, then we define $P_{M|\Xn}(m|\xn) = \I_{\set{m=0}}$. Finally, if $\Ipmf = 0, \text{ then } P_{M|\Xn}(m|\xn) = \I_{\{m=0\}}$, for all $\xn \in \calX^n$. This concludes the encoder description. We provide a proposition from \cite{atif2023lossy}, which will be helpful later in the analysis.

\vspace{3pt}
\begin{proposition}\label{prop:clssubPMF}
    For all $\epsilon,\eta \in (0,1)$, for all sufficiently small $\delta > 0$, and sufficiently large $n$, we have $\EE[{\Ipmf}] \geq 1-\epsilon$,
if $\Rbar > I(X;U)$.
\end{proposition}
% \begin{proof}
%     Proof arguments follow from \cite{atif2022source}. 
%     % For completeness, we provide the proof in Appendix \ref{app:proof:prop:clssubPMF}.
% \end{proof}
% In other words, the above proposition states that if $R > I(X;\hat{X}) + \tilde{\delta}(\delta)$, then $E_{M|X^n}(\cdot|x^n):[2^{nR}] \rightarrow \RR$ is a PMF for each $x^n \in \Tx$ with high probability, where $\tilde{\delta}(\delta) \searrow 0$ as  $\delta\searrow0$. 
% \noindent We now summarize $P_{M|X^n}$ for $m\in \{0\}\cup [2^{nR}]$ and under the condition that $\I_{\curly{\mbox{\normalfont sPMF}}} = 1$,
% \begin{equation*}
%     P_{M|X^n}(m|x^n) \deq \begin{cases}
%     \I_{\{m=0\}} & \text{if } x^n \not \in \Tx,\\
%      E_{M|X^n}(m|x^n) & \text{if } x^n \in \Tx.
%     %  , \\
%     % E_{M|X^n}(0|x^n) &\text{otherwise.}
%     \end{cases}
% \end{equation*}
% In other words, for $x^n \not \in \Tx$ the encoder outputs $0$ with probability 1. For $x^n \in \Tx$, the encoder output an index  from the set $ \calI \deq \{m \in [2^{nR}]:(x^n,\hat{X}^n(m)) \in \Txxhat\}$ according to the distribution $E_{M|X^n}(\cdot|x^n)$, and if $|\calI| = 0$, then encoder outputs $0$ with probability $ (1-\sum_{m'=1}^{2^{nR}}E_{M|X^n}(m'|x^n))$. 
% It can be easily verified that $P_{M|X^n}$ is a valid PMF. 
% \vspace{2pt}

\vspace{3pt}
\noindent \textbf{Decoder Description}: 
For an observed sequence $m \in \set{0} \cup [2^{nR}]$ communicated by the encoder and the sequence $\zn \in \calZ^n$, the decoder constructs the following set:
$\sfL(m,\zn) \deq \set{l \in [2^{n\Rbar}] : \calB(l) = m \eqand (\Un(l),\zn) \in \Tuz}.$
After this, the decoder outputs $\calD(m,\zn) = \Un(l)$ if $\sfL(m,\zn) = \set{l} \eqand m\neq 0$. Otherwise, the decoder outputs a fixed $\un_0 \in \calU^n \backslash \Tu$. At the end, the decoder chooses $\yn$ according to PMF $P^n_{Y|UZ}(\yn|\calD(m,\zn),\zn)$. This implies the PMF $P_{Y^n|MZ^n}(\cdot|m,\zn)$ can be expressed as: 
\[P_{Y^n|MZ^n}(\cdot|m,\zn) = P^n_{Y|UZ}(\cdot|\calD(m,\zn),\zn).\]

\vspace{3pt}
\noindent \textbf{Error Analysis}: 
We show that for the above-mentioned encoder and decoder, $P_{\Xn\Zn\Yn}$ is close to the approximating distribution $P_{\Yn\Zn} W^n_{X|YZ}$. These PMFs can be further expressed as follows:
\begin{align*}
    P_{\Xn\Yn\Zn}(\xn,\yn,\zn) &= \sum_{m \in [\theta] \cup \set{0}} P_{XZ}^n(\xn,\zn) P_{M|\Xn}(m|\xn)P_{\Yn|M\Zn}(\yn|m,\zn)\\
    P_{\Yn\Zn}(\yn,\zn) W^n_{X|YZ}(\xn
|\yn,\zn) &= \sum_{\sfxn}\sum_{\substack{m \in [\theta] \cup \set{0}}}  P^n_{XZ}(\sfxn,\zn)P_{M|\Xn}(m|\sfxn) P_{\Yn|M\Zn}(\yn|m,\zn)\\
& \hspace{2.5in}\times W^n_{X|YZ}(\xn
|\yn,\zn),
\end{align*}
where $P_{M|\Xn}(m|\sfxn) \eqand P_{\Yn|M\Zn}(\yn|m,\zn)$ are PMFs induced by encoder and decoder, respectively, and for convenience, we denote $\theta \deq [2^{nR}]$ and  $\bar{\theta} \deq [2^{n\Rbar}]$, for the remaining of the paper. We begin by splitting the error $ \Xi(\encodern,\decodern)$ into two terms using $\Ipmf$ as 
\begin{align}
    \Xi(\encodern,\decodern) 
    &= \Ipmf 
\Xi(\encodern,\decodern) + (1- \Ipmf)\Xi(\encodern,\decodern),\nonumber \\
&\leq \Ipmf 
\Xi(\encodern,\decodern) + ({1- \Ipmf}),\label{eqn:clserrorsubpmf}
\end{align}
where \eqref{eqn:clserrorsubpmf} follows from upper bounding the total variation between two PMFs by one, i.e., the maximum value of the 
total variation between two PMFs. 
% \vspace{10pt}
% \noindent\textbf{Step 1: Isolating the error term induced by not covering}
Using the triangle inequality, we now expand $\Xi(\encodern,\decodern)$. Under the condition $\Ipmf = 1$, as follows:
\begin{align*}
    2 \ \Xi(&\encodern,\decodern) =\sum_{\xn, \yn, \zn} \Big| \sum_{m \in [\theta] \cup \set{0}} P_{XZ}^n(\xn,\zn) P_{M|\Xn}(m|\xn)P_{\Yn|M\Zn}(\yn|m,\zn) \\
    & \hspace{60pt}-\sum_{\sfxn}\sum_{\substack{m \in [\theta] \cup \set{0} }}  P^n_{XZ}(\sfxn,\zn)P_{M|\Xn}(m|\sfxn) P_{\Yn|M\Zn}(\yn|m,\zn) W^n_{X|YZ}(\xn
    |\yn,\zn)\Big| \nonumber \\
    %%%%%%%%%%%%%%%%%%%%%%%%%%%%%%%%%%%%%%%%%%%%%%%%%%%%
    %%%%%%%%%%%% not typical 
    &=\sum_{\substack{\xn \notin \Tx \\ \yn,\ \zn}}
    \Big|P_{XZ}^n(\xn,\zn) P^n_{Y|UZ}(\yn|u_0^n,\zn) \\
    & \hspace{50pt}
    - \sum_{\sfxn}\sum_{\substack{m \in [\theta] \cup \set{0} }}   P^n_{XZ}(\sfxn,\zn)P_{M|\Xn}(m|\sfxn) P_{\Yn|M\Zn}(\yn|m,\zn) W^n_{X|YZ}(\xn
    |\yn,\zn)\Big|\\
    %%%%%%%%%%%% typical
     &+\sum_{\substack{\xn \in \Tx \\ \yn,\ \zn}}
    \Big|P_{XZ}^n(\xn,\zn) \sum_{m\in [\theta]}P_{M|\Xn}(m|\xn) P_{\Yn|m\Zn}(\yn|m,\zn) \\
    & \hspace{50pt}
    + P_{XZ}^n(\xn,\zn)P_{M|\Xn}(0|\xn)P^n_{Y|UZ}(\yn|u_0^n,\zn)\\
    & \hspace{50pt}
    - \sum_{\sfxn}\sum_{\substack{m \in [\theta] \cup \set{0}}}  P^n_{XZ}(\sfxn,\zn)P_{M|\Xn}(m|\sfxn) P_{\Yn|M\Zn}(\yn|m,\zn) W^n_{X|YZ}(\xn
    |\yn,\zn)
    \Big|\\
    %%%%%%%%%%%%%%%%%%%%%%%%%%%%%%%%%%%%%%%%%%%%%%%%%%%%
    % using the triangle inequality
    & \overset{a}{\leq} \cpe + \notce +  \er + 3 \!\!\!\!\!\!\sum_{\xn \notin \Tx}P^n_X{(\xn)} \\
    & \hspace{50pt} + \sum_{\substack{\sfxn \in \Tx \\ \xn,\ \yn,\ \zn}} P^n_{XZ}(\sfxn,\zn)P_{M|\Xn}(0|\sfxn) P_{\Yn|\Un\Zn}(\yn|u_0^n,\zn) W^n_{X|YZ}(\xn
    |\yn,\zn)\\
    %%%%%%%%%%%%%%%%%%%%%%%%%%%%%%%%%%%%%%%%%%%%%%%%%%%%
    & \overset{b}{\leq} \cpe + 2\notce  + \er + 4 \!\!\!\!\!\!\sum_{\xn \notin \Tx}P^n_X{(\xn)} 
    \overset{c}{\leq}\cpe + 2\notce + \er +  4\epsilon,
\end{align*}
for all sufficiently large $n$ and all $\delta>0$, where $(a)$ follows from triangle inequality and by defining terms $\cpe ,\notce,\eqand \er$ as follows:
\begin{align*}
    \cpe &\deq \sum_{\substack{\xn \in \Tx \\ \yn, \zn}}
    \Big|P_{XZ}^n(\xn,\zn) \sum_{m\in [\theta]}P_{M|\Xn}(m|\xn) P^n_{Y|UZ}(\yn|\calD(m,\zn),\zn) \\
    & \hspace{50pt} - \sum_{\substack{m \in [\theta] \\ \sfxn \in \Tx}}  P^n_{XZ}(\sfxn,\zn)P_{M|\Xn}(m|\sfxn) P^n_{Y|UZ}(\yn|\calD(m,\zn),\zn) W^n_{X|YZ}(\xn
    |\yn,\zn)
    \Big|,\\
    %%%%%%%%%%%%%%%%%%%%%%%%%%%%%%%%%%%%%%%%%%%%%%%
    \notce &\deq \sum_{\xn \in \Tx} P^n_{X}(\xn) \Big(1 - \sum_{l\in[\btheta]}E_{l|\Xn}(l|\xn)\Big),  \\
    %%%%%%%%%%%%%%%%%%%%%%%%%%%%%%%%%%%%%%%%%%%%%%%
    % \ee &\deq \sum_{\substack{\sfxn \in \Tx \\ \xn \notin \Tx}} \sum_{\substack{m \in [\theta] \\\yn \\ \zn }}P^n_{XZ}(\sfxn,\zn)P_{M|\Xn}(m|\sfxn) P^n_{Y|UZ}(\yn|\un,\zn) W^n_{X|YZ}(\xn|\yn,\zn), \eqand\\
    %%%%%%%%%%%%%%%%%%%%%%%%%%%%%%%%%%%%%%%%%%%%%%%
    \er &\deq \sum_{\substack{\sfxn \in \Tx \\ \xn \notin \Tx}} \sum_{\substack{m \in [\theta] \\\yn \\ \zn }}
     P^n_{XZ}(\sfxn,\zn)P_{M|\Xn}(m|\sfxn) P^n_{Y|UZ}(\yn|\calD(m,\zn),\zn) W^n_{X|YZ}(\xn
    |\yn,\zn),
\end{align*}
and $(b)$ follows by writing $P_M(0) = \sum_{\sfxn \notin \Tx} P^n_X(\sfxn) + \sum_{\xn \in \Tx} P^n_X(\xn)E_{l|\Xn}(0|\xn),$ and $(c)$ follows from the standard typically argument for all sufficiently large $n$. 

\noindent \textbf{Step 1: Bounding the error induced by not covering}

\noindent Note that the error term $\notce$ captures the error induced by not covering the $n$-product side-information assisted posterior channel. We bound this term by utilizing the following proposition.
\begin{proposition} For all $\epsilon \in (0,1)$, for all sufficiently small $\eta,\delta >0$, and for all sufficiently large $n$, we have $\EE[\Ipmf \notce] \leq \epsilon$, if $\Rbar > I(X:U)$.
\end{proposition}
\begin{proof}
    The proof follows from the \cite[Proposition 8]{atif2023lossy}.
\end{proof}
Next, we move on to isolating the error component of $\cpe$ caused by binning (packing). 

\noindent \textbf{Step 2: Isolating the term induced by binning}

\noindent We consider the term corresponding to $\cpe$. By adding and subtracting an appropriate term inside the modulus of $\cpe$ and using triangle inequality, we get $\cpe \leq \ce + \peone + \petwo$, where 
\begin{align*}
    \ce &\deq \!\!\!\sum_{\substack{\xn \in \Tx \\ \un \in \Tu }} \sum_{\substack{\yn \\ \zn}} \sum_{\substack{ l \in [\btheta]\\m\in [\theta] }} 
    \frac{1}{\btheta} \frac{(1-\varepsilon)}{(1+\eta)} 
    \I_{\set{\Un(l) = \un}} \I_{\set{\calB(l) = m}}
 \Big|P_{Z|X}^n(\zn|\xn) P_{X|U}^n(\xn|\un) P^n_{Y|UZ}(\yn|\un,\zn) \\
    & \hspace{25pt} \times \I_{\set{\xn \in \Txcond}} - \!\!\!\!\!\!\!\sum_{\substack{\sfxn \in \Txcond}} \!\!\!\!\!\!P_{Z|X}^n(\zn|\sfxn) P_{X|U}^n(\sfxn|\un) P^n_{Y|UZ}(\yn|\un,\zn) W^n_{X|YZ}(\xn
    |\yn,\zn)
    \Big|,\\
%%%%%%%%%%%%
    \peone &\deq \sum_{\substack{\xn \in \Tx \\ \un \in \Tu }} \sum_{\substack{\yn \\ \zn}} \sum_{\substack{ l \in [\btheta]\\m\in [\theta] }} 
    \frac{1}{\btheta} \frac{(1-\varepsilon)}{(1+\eta)} 
    \I_{\set{\Un(l) = \un}} \I_{\set{\calB(l) = m}} \I_{\set{\xn \in \Txcond}}
    P_{Z|X}^n(\zn|\xn) P_{X|U}^n(\xn|\un) \\
    & \hspace{100pt}  \times \Big| P^n_{Y|UZ}(\yn|\un,\zn) - P^n_{Y|UZ}(\yn|\calD(m,\zn),\zn)\Big|,\\
%%%%%%%%%%%%
    \petwo &\deq \sum_{\substack{\xn \in \Tx \\ \un \in \Tu \\ \sfxn \in \Tx}} \sum_{\substack{\yn \\ \zn}} \sum_{\substack{ l \in [\btheta]\\m\in [\theta] }} 
    \frac{1}{\btheta} \frac{(1-\varepsilon)}{(1+\eta)} 
    \I_{\set{\Un(l) = \un}} \I_{\set{\calB(l) = m}} 
    \I_{\set{\sfxn \in \Txcond}}
    % \sum_{\sfxn \in \Tx}
    P_{Z|X}^n(\zn|\sfxn) P_{X|U}^n(\sfxn|\un) \\
    & \hspace{100pt}  \times \Big| P^n_{Y|UZ}(\yn|\un,\zn) - P^n_{Y|UZ}(\yn|\calD(m,\zn),\zn)\Big| W^n_{X|YZ}(\xn|\yn,\zn).
\end{align*}
Here, $\peone$ and $\petwo$ captures the error induced by binning. Observe that, we can establish the bound $\petwo \leq \peone$ because $\sum_{\xn \in \Tx} W^n_{X|YZ}(\xn|\yn,\zn) \leq 1$. Consequently, we have $\cpe \leq \ce + 2\peone$. To bound the term $\peone$, we provide the following proposition.
\begin{proposition}\label{prop:binning} For all $\eta,\epsilon \in (0,1),$ for all sufficiently small $\delta >0$, and sufficiently large $n$, we have $\EE[\Ipmf \peone] \leq 3\epsilon$, if $(\Rbar-R) < I(U;Z)$.
\end{proposition}
\begin{proof}
The proof is provided in Appendix \ref{app:proof:prop:binning}.
\end{proof} 
\noindent Next, we proceed to analyze the error term induced by covering. 

%%%%%%%%%%%%%%%%%%%%%%%%%%%%%%%%%%%%%%%%%%%%%%%%%%%%%%%
\noindent\textbf{Step 3: Bounding the covering error}

\noindent Using the Markov chain $U-X-Z$, $X-(U,Z)-Y$, and $X-(Y,Z)-U$ which $P_{UXYZ}$ satisfies, we can rewrite the term $ P_{X|U} P_{Z|UX} P_{Y|UXZ}$ as follows:
\begin{equation}\label{eqn:mc_prob}
    P_{Z|X} P_{X|U} P_{Y|UZ} = P_{Z|UX} P_{X|U} P_{Y|UXZ} = P_{YZ|U} W_{X|YZ}.
\end{equation}
Using \eqref{eqn:mc_prob}, we can simplify the terms inside the modulus of $\ce$ as:
\begin{align*}
    \ce &\deq \sum_{\substack{\xn \in \Tx \\ \un \in \Tu }} \sum_{\substack{\yn \\ \zn}} \sum_{\substack{ l \in [\btheta]\\m\in [\theta] }} 
    \frac{1}{\btheta} \frac{(1-\varepsilon)}{(1+\eta)} 
    \I_{\set{\Un(l) = \un}} \I_{\set{\calB(l) = m}} P_{YZ|U}^n(\yn,\zn|\un) W^n_{X|YZ}(\xn|\yn,\zn) \\
    & \hspace{2in} \times \Big|\I_{\set{\xn \in \Txcond}} - \sum_{\substack{\sfxn \in \Txcond}} W^n_{X|YZ}(\sfxn
    |\yn,\zn)
    \Big|.
\end{align*}
To bound the above-simplified term, we provide the following proposition. 
\begin{proposition}\label{prop:covering}
    For all $\eta, \epsilon \in (0,1)$, for all sufficiently small $\delta >0$, and sufficiently large $n$, we have $\EE[\Ipmf \ce] \leq \epsilon$.
\end{proposition}
\begin{proof}
The proof is provided in Appendix \ref{app:proof:prop:covering}.
\end{proof}
Following Propositions \ref{prop:binning} and \ref{prop:covering}, for all $\epsilon \in (0,1)$, for all sufficiently large $n$ and sufficiently small $\delta, \eta> 0$, we obtain, $\EE[\Ipmf \cpe] \leq \EE[\Ipmf \ce] + 2\EE[\Ipmf \peone] \leq 7\epsilon$.  Finally, we are left with the analysis of the error term $\er$.

\noindent \textbf{Step 4: Bounding the error term $\er$}

\noindent By incorporating the addition and subtraction of a suitable term and applying the triangle inequality, we further upper bound the expression of the error term $\er$ as $\er \leq \ee + \epe,$ where
\begin{align*}
     \epe &\deq \sum_{\substack{\xn \notin \Tx \\ \un \in \Tu }} \sum_{\substack{\yn \\ \zn}} \sum_{\substack{ l \in [\btheta]\\m\in [\theta] }} 
    \frac{1}{\btheta} \frac{(1-\varepsilon)}{(1+\eta)} 
    \I_{\set{\Un(l) = \un}} \I_{\set{\calB(l) = m}} 
    \sum_{\sfxn \in \Tx}P_{Z|X}^n(\zn|\sfxn) P_{X|U}^n(\sfxn|\un)\\
    & \hspace{0.5in}  \times  \I_{\set{\sfxn \in \Txcond}} \Big| P^n_{Y|UZ}(\yn|\un,\zn) - P^n_{Y|UZ}(\yn|\calD(m,\zn),\zn)\Big| W^n_{X|YZ}(\xn|\yn,\zn)\\
    %%%%%%%%%%%%%%%%%%%
    \eqand \ee &\deq\sum_{\substack{\xn \notin \Tx \\ \un \in \Tu }} \sum_{\substack{\yn \\ \zn}} \sum_{\substack{ l \in [\btheta]\\m\in [\theta] }} 
    \frac{1}{\btheta} \frac{(1-\varepsilon)}{(1+\eta)} 
    \I_{\set{\Un(l) = \un}} \I_{\set{\calB(l) = m}} 
    \sum_{\sfxn \in \Tx}P_{Z|X}^n(\zn|\sfxn) P_{X|U}^n(\sfxn|\un) \\
    & \hspace{0.8in}  \times  \I_{\set{\sfxn \in \Txcond}} P^n_{Y|UZ}(\yn|\un,\zn)W^n_{X|YZ}(\xn|\yn,\zn)
\end{align*}

\noindent Similar to the error $\petwo$, note that the error term $\epe$ can be upper bounded as $\epe \leq \peone$ because $\sum_{\xn \notin \Tx} W^n_{X|YZ}(\xn|\yn,\zn) \leq 1$. Thus, for all $\epsilon \in (0,1)$, for all sufficiently large $n$ and sufficiently small $\delta, \eta > 0$, we have $\EE[\Ipmf \epe] \leq 3\epsilon$. To bound the remaining error term $\ee$, we present a proposition below.
\noindent \begin{proposition}\label{prop:encoding_error}
    For all $\epsilon \in \set{0,1}$, for all sufficiently small $\eta,\delta >0$, and sufficiently large $n$, we have $\EE[\Ipmf \ee] \leq \epsilon$.
\end{proposition}
\begin{proof}
The proof follows from the argument similar to Proposition \ref{prop:covering}. However, for completeness, we provide the proof in Appendix \ref{app:proof:prop:encoding_error}.
\end{proof}
Eventually, using Propositions \ref{prop:binning}, \ref{prop:covering}, and \ref{prop:encoding_error}, we bound the $\EE[\encodern, \decodern].$ For all $\epsilon \in (0,1)$,
\begin{equation*}
    \EE_{\codebook}[\Xi(\encodern, \decodern)] \leq \EE_{\codebook}[\Ipmf \Xi(\encodern,\decodern) + (1-\Ipmf)] \leq 17\epsilon/2,
\end{equation*}
for all sufficiently large $n$. Since $\EE_{\codebook}[\Xi(\encodern, \decodern)] \leq 17\epsilon/2$, there must exists a code $\codebook$ such that the associated error $\Xi(\encodern, \decodern) \leq 17\epsilon/2$. This completes the achievability proof.



\section{Proof of Theorem \ref{thm:connection}}
\label{sec:proof:connection}
Given 
$P_{X^nY^n}$ is the induced $n$-letter joint distribution on the source and the reconstruction vectors. 
Then, by Lemmas \ref{lem:averageSingleletter} and \ref{lemma:sols_linearEqn_close} (stated below and proof provided in \cite{sohail2025WZ}), we have
$\lim_{n \rightarrow \infty} \| P_{X_QY_Q} - P_{Y}W_{X|Y} \|_{\normalfont \text{TV}} =0,$
for some distribution $P_{Y}$  
in $\calA(\px,W_{X|Y})$ that achieves the optimality in Theorem \ref{thm:Csourcecoding},
where $P_{X_QY_Q}=\tfrac{1}{n} \sum_{i=1}^n P_{X_iY_i}.$
Then it is well known \cite[Problem 8.3]{csiszar2011information} that if one chooses, the following single-letter bounded distortion function 
$d(x,y)= -c\log_2 \prevTC(x|\xhat)+b(x)$,
and distortion level $D={\mathbb{E}[d(X,Y)]}$, where the expectation is with respect to the distribution $P_{Y} \prevTC$,
then the backward channel $\prevTC$ attains optimality in the standard rate-distortion function of the source 
$(\px,\sfX,\sfXhat,d)$
at the distortion level $D$.
Using the above arguments, we 
infer that the same protocol also achieves the 
distortion value of $D$, i.e., $\lim_{n \rightarrow \infty} \!\mathbb{E} [\tfrac{1}{n} \!\sum_{i=1}^n d(X_i,{Y_i})] \!=\!\lim_{n \rightarrow \infty} \sum_{x,y} \!P_{X_QY_Q}\!(x,y) d(x,y)\! =\!
D.$
In summary, the protocol also achieves the optimal rate-distortion function at the distortion level $D$. 
This is along anticipated lines as the current formulation uses a stricter global error criterion than the standard approach that uses a local one.

\begin{lemma} \cite[Lemma 8]{atif2023lossy}\label{lem:averageSingleletter}
% (Total variation: average single-letter distribution) 
The distributions $P_{X^nY^n}$ and $P_{Y^n}\prevTC^n$ satisfy
$$ \|P_X - \sum_{\xhat} P_{Y_Q}(\xhat)\prevTC(\cdot|\xhat) \|_{\normalfont \text{TV}} \leq 
\|P_{X_QY_Q} - P_{Y_Q}\prevTC \|_{\normalfont \text{TV}}  \leq 
\|P_{X^nY^n} -
    P_{Y^n}\prevTC^n\|_{\normalfont \text{TV}}.$$
\end{lemma}
\begin{lemma} \label{lemma:sols_linearEqn_close}
Let \( \sfA \in \mathbb{R}^{\alpha \times \beta} \) be a matrix and \( \sfb \in \mathbb{R}^{\alpha} \) a vector. Suppose \( x_0 \in \mathbb{R}^{\beta}_{\geq 0} \) is a non-negative vector satisfying the conditions \( \|\sfA x_0 - \sfb\|_1 \leq \delta \) and \( \I^\texttt{T} x_0 = 1 \) for some \( \delta > 0 \), where \( \I^\texttt{T} \) denotes a row vector of ones.
Then, there exists a \( x \in \mathbb{R}^{\beta}_{\geq 0} \) such that \( A x = b \) and \( \I^\top x = 1 \), with the additional property that
$\|x - x_0\|_1 \leq f(\delta),$
where \( f(\delta) \to 0 \) as \( \delta \to 0 \).
\end{lemma}
\begin{proof}
    The proof is provided in Appendix \ref{app:proof:lemma:sols_linearEqn_close}.
\end{proof}

\bibliographystyle{ieeetr}
\bibliography{reference}

% \newpage
\appendix
\subsection{Lloyd-Max Algorithm}
\label{subsec:Lloyd-Max}
For a given quantization bitwidth $B$ and an operand $\bm{X}$, the Lloyd-Max algorithm finds $2^B$ quantization levels $\{\hat{x}_i\}_{i=1}^{2^B}$ such that quantizing $\bm{X}$ by rounding each scalar in $\bm{X}$ to the nearest quantization level minimizes the quantization MSE. 

The algorithm starts with an initial guess of quantization levels and then iteratively computes quantization thresholds $\{\tau_i\}_{i=1}^{2^B-1}$ and updates quantization levels $\{\hat{x}_i\}_{i=1}^{2^B}$. Specifically, at iteration $n$, thresholds are set to the midpoints of the previous iteration's levels:
\begin{align*}
    \tau_i^{(n)}=\frac{\hat{x}_i^{(n-1)}+\hat{x}_{i+1}^{(n-1)}}2 \text{ for } i=1\ldots 2^B-1
\end{align*}
Subsequently, the quantization levels are re-computed as conditional means of the data regions defined by the new thresholds:
\begin{align*}
    \hat{x}_i^{(n)}=\mathbb{E}\left[ \bm{X} \big| \bm{X}\in [\tau_{i-1}^{(n)},\tau_i^{(n)}] \right] \text{ for } i=1\ldots 2^B
\end{align*}
where to satisfy boundary conditions we have $\tau_0=-\infty$ and $\tau_{2^B}=\infty$. The algorithm iterates the above steps until convergence.

Figure \ref{fig:lm_quant} compares the quantization levels of a $7$-bit floating point (E3M3) quantizer (left) to a $7$-bit Lloyd-Max quantizer (right) when quantizing a layer of weights from the GPT3-126M model at a per-tensor granularity. As shown, the Lloyd-Max quantizer achieves substantially lower quantization MSE. Further, Table \ref{tab:FP7_vs_LM7} shows the superior perplexity achieved by Lloyd-Max quantizers for bitwidths of $7$, $6$ and $5$. The difference between the quantizers is clear at 5 bits, where per-tensor FP quantization incurs a drastic and unacceptable increase in perplexity, while Lloyd-Max quantization incurs a much smaller increase. Nevertheless, we note that even the optimal Lloyd-Max quantizer incurs a notable ($\sim 1.5$) increase in perplexity due to the coarse granularity of quantization. 

\begin{figure}[h]
  \centering
  \includegraphics[width=0.7\linewidth]{sections/figures/LM7_FP7.pdf}
  \caption{\small Quantization levels and the corresponding quantization MSE of Floating Point (left) vs Lloyd-Max (right) Quantizers for a layer of weights in the GPT3-126M model.}
  \label{fig:lm_quant}
\end{figure}

\begin{table}[h]\scriptsize
\begin{center}
\caption{\label{tab:FP7_vs_LM7} \small Comparing perplexity (lower is better) achieved by floating point quantizers and Lloyd-Max quantizers on a GPT3-126M model for the Wikitext-103 dataset.}
\begin{tabular}{c|cc|c}
\hline
 \multirow{2}{*}{\textbf{Bitwidth}} & \multicolumn{2}{|c|}{\textbf{Floating-Point Quantizer}} & \textbf{Lloyd-Max Quantizer} \\
 & Best Format & Wikitext-103 Perplexity & Wikitext-103 Perplexity \\
\hline
7 & E3M3 & 18.32 & 18.27 \\
6 & E3M2 & 19.07 & 18.51 \\
5 & E4M0 & 43.89 & 19.71 \\
\hline
\end{tabular}
\end{center}
\end{table}

\subsection{Proof of Local Optimality of LO-BCQ}
\label{subsec:lobcq_opt_proof}
For a given block $\bm{b}_j$, the quantization MSE during LO-BCQ can be empirically evaluated as $\frac{1}{L_b}\lVert \bm{b}_j- \bm{\hat{b}}_j\rVert^2_2$ where $\bm{\hat{b}}_j$ is computed from equation (\ref{eq:clustered_quantization_definition}) as $C_{f(\bm{b}_j)}(\bm{b}_j)$. Further, for a given block cluster $\mathcal{B}_i$, we compute the quantization MSE as $\frac{1}{|\mathcal{B}_{i}|}\sum_{\bm{b} \in \mathcal{B}_{i}} \frac{1}{L_b}\lVert \bm{b}- C_i^{(n)}(\bm{b})\rVert^2_2$. Therefore, at the end of iteration $n$, we evaluate the overall quantization MSE $J^{(n)}$ for a given operand $\bm{X}$ composed of $N_c$ block clusters as:
\begin{align*}
    \label{eq:mse_iter_n}
    J^{(n)} = \frac{1}{N_c} \sum_{i=1}^{N_c} \frac{1}{|\mathcal{B}_{i}^{(n)}|}\sum_{\bm{v} \in \mathcal{B}_{i}^{(n)}} \frac{1}{L_b}\lVert \bm{b}- B_i^{(n)}(\bm{b})\rVert^2_2
\end{align*}

At the end of iteration $n$, the codebooks are updated from $\mathcal{C}^{(n-1)}$ to $\mathcal{C}^{(n)}$. However, the mapping of a given vector $\bm{b}_j$ to quantizers $\mathcal{C}^{(n)}$ remains as  $f^{(n)}(\bm{b}_j)$. At the next iteration, during the vector clustering step, $f^{(n+1)}(\bm{b}_j)$ finds new mapping of $\bm{b}_j$ to updated codebooks $\mathcal{C}^{(n)}$ such that the quantization MSE over the candidate codebooks is minimized. Therefore, we obtain the following result for $\bm{b}_j$:
\begin{align*}
\frac{1}{L_b}\lVert \bm{b}_j - C_{f^{(n+1)}(\bm{b}_j)}^{(n)}(\bm{b}_j)\rVert^2_2 \le \frac{1}{L_b}\lVert \bm{b}_j - C_{f^{(n)}(\bm{b}_j)}^{(n)}(\bm{b}_j)\rVert^2_2
\end{align*}

That is, quantizing $\bm{b}_j$ at the end of the block clustering step of iteration $n+1$ results in lower quantization MSE compared to quantizing at the end of iteration $n$. Since this is true for all $\bm{b} \in \bm{X}$, we assert the following:
\begin{equation}
\begin{split}
\label{eq:mse_ineq_1}
    \tilde{J}^{(n+1)} &= \frac{1}{N_c} \sum_{i=1}^{N_c} \frac{1}{|\mathcal{B}_{i}^{(n+1)}|}\sum_{\bm{b} \in \mathcal{B}_{i}^{(n+1)}} \frac{1}{L_b}\lVert \bm{b} - C_i^{(n)}(b)\rVert^2_2 \le J^{(n)}
\end{split}
\end{equation}
where $\tilde{J}^{(n+1)}$ is the the quantization MSE after the vector clustering step at iteration $n+1$.

Next, during the codebook update step (\ref{eq:quantizers_update}) at iteration $n+1$, the per-cluster codebooks $\mathcal{C}^{(n)}$ are updated to $\mathcal{C}^{(n+1)}$ by invoking the Lloyd-Max algorithm \citep{Lloyd}. We know that for any given value distribution, the Lloyd-Max algorithm minimizes the quantization MSE. Therefore, for a given vector cluster $\mathcal{B}_i$ we obtain the following result:

\begin{equation}
    \frac{1}{|\mathcal{B}_{i}^{(n+1)}|}\sum_{\bm{b} \in \mathcal{B}_{i}^{(n+1)}} \frac{1}{L_b}\lVert \bm{b}- C_i^{(n+1)}(\bm{b})\rVert^2_2 \le \frac{1}{|\mathcal{B}_{i}^{(n+1)}|}\sum_{\bm{b} \in \mathcal{B}_{i}^{(n+1)}} \frac{1}{L_b}\lVert \bm{b}- C_i^{(n)}(\bm{b})\rVert^2_2
\end{equation}

The above equation states that quantizing the given block cluster $\mathcal{B}_i$ after updating the associated codebook from $C_i^{(n)}$ to $C_i^{(n+1)}$ results in lower quantization MSE. Since this is true for all the block clusters, we derive the following result: 
\begin{equation}
\begin{split}
\label{eq:mse_ineq_2}
     J^{(n+1)} &= \frac{1}{N_c} \sum_{i=1}^{N_c} \frac{1}{|\mathcal{B}_{i}^{(n+1)}|}\sum_{\bm{b} \in \mathcal{B}_{i}^{(n+1)}} \frac{1}{L_b}\lVert \bm{b}- C_i^{(n+1)}(\bm{b})\rVert^2_2  \le \tilde{J}^{(n+1)}   
\end{split}
\end{equation}

Following (\ref{eq:mse_ineq_1}) and (\ref{eq:mse_ineq_2}), we find that the quantization MSE is non-increasing for each iteration, that is, $J^{(1)} \ge J^{(2)} \ge J^{(3)} \ge \ldots \ge J^{(M)}$ where $M$ is the maximum number of iterations. 
%Therefore, we can say that if the algorithm converges, then it must be that it has converged to a local minimum. 
\hfill $\blacksquare$


\begin{figure}
    \begin{center}
    \includegraphics[width=0.5\textwidth]{sections//figures/mse_vs_iter.pdf}
    \end{center}
    \caption{\small NMSE vs iterations during LO-BCQ compared to other block quantization proposals}
    \label{fig:nmse_vs_iter}
\end{figure}

Figure \ref{fig:nmse_vs_iter} shows the empirical convergence of LO-BCQ across several block lengths and number of codebooks. Also, the MSE achieved by LO-BCQ is compared to baselines such as MXFP and VSQ. As shown, LO-BCQ converges to a lower MSE than the baselines. Further, we achieve better convergence for larger number of codebooks ($N_c$) and for a smaller block length ($L_b$), both of which increase the bitwidth of BCQ (see Eq \ref{eq:bitwidth_bcq}).


\subsection{Additional Accuracy Results}
%Table \ref{tab:lobcq_config} lists the various LOBCQ configurations and their corresponding bitwidths.
\begin{table}
\setlength{\tabcolsep}{4.75pt}
\begin{center}
\caption{\label{tab:lobcq_config} Various LO-BCQ configurations and their bitwidths.}
\begin{tabular}{|c||c|c|c|c||c|c||c|} 
\hline
 & \multicolumn{4}{|c||}{$L_b=8$} & \multicolumn{2}{|c||}{$L_b=4$} & $L_b=2$ \\
 \hline
 \backslashbox{$L_A$\kern-1em}{\kern-1em$N_c$} & 2 & 4 & 8 & 16 & 2 & 4 & 2 \\
 \hline
 64 & 4.25 & 4.375 & 4.5 & 4.625 & 4.375 & 4.625 & 4.625\\
 \hline
 32 & 4.375 & 4.5 & 4.625& 4.75 & 4.5 & 4.75 & 4.75 \\
 \hline
 16 & 4.625 & 4.75& 4.875 & 5 & 4.75 & 5 & 5 \\
 \hline
\end{tabular}
\end{center}
\end{table}

%\subsection{Perplexity achieved by various LO-BCQ configurations on Wikitext-103 dataset}

\begin{table} \centering
\begin{tabular}{|c||c|c|c|c||c|c||c|} 
\hline
 $L_b \rightarrow$& \multicolumn{4}{c||}{8} & \multicolumn{2}{c||}{4} & 2\\
 \hline
 \backslashbox{$L_A$\kern-1em}{\kern-1em$N_c$} & 2 & 4 & 8 & 16 & 2 & 4 & 2  \\
 %$N_c \rightarrow$ & 2 & 4 & 8 & 16 & 2 & 4 & 2 \\
 \hline
 \hline
 \multicolumn{8}{c}{GPT3-1.3B (FP32 PPL = 9.98)} \\ 
 \hline
 \hline
 64 & 10.40 & 10.23 & 10.17 & 10.15 &  10.28 & 10.18 & 10.19 \\
 \hline
 32 & 10.25 & 10.20 & 10.15 & 10.12 &  10.23 & 10.17 & 10.17 \\
 \hline
 16 & 10.22 & 10.16 & 10.10 & 10.09 &  10.21 & 10.14 & 10.16 \\
 \hline
  \hline
 \multicolumn{8}{c}{GPT3-8B (FP32 PPL = 7.38)} \\ 
 \hline
 \hline
 64 & 7.61 & 7.52 & 7.48 &  7.47 &  7.55 &  7.49 & 7.50 \\
 \hline
 32 & 7.52 & 7.50 & 7.46 &  7.45 &  7.52 &  7.48 & 7.48  \\
 \hline
 16 & 7.51 & 7.48 & 7.44 &  7.44 &  7.51 &  7.49 & 7.47  \\
 \hline
\end{tabular}
\caption{\label{tab:ppl_gpt3_abalation} Wikitext-103 perplexity across GPT3-1.3B and 8B models.}
\end{table}

\begin{table} \centering
\begin{tabular}{|c||c|c|c|c||} 
\hline
 $L_b \rightarrow$& \multicolumn{4}{c||}{8}\\
 \hline
 \backslashbox{$L_A$\kern-1em}{\kern-1em$N_c$} & 2 & 4 & 8 & 16 \\
 %$N_c \rightarrow$ & 2 & 4 & 8 & 16 & 2 & 4 & 2 \\
 \hline
 \hline
 \multicolumn{5}{|c|}{Llama2-7B (FP32 PPL = 5.06)} \\ 
 \hline
 \hline
 64 & 5.31 & 5.26 & 5.19 & 5.18  \\
 \hline
 32 & 5.23 & 5.25 & 5.18 & 5.15  \\
 \hline
 16 & 5.23 & 5.19 & 5.16 & 5.14  \\
 \hline
 \multicolumn{5}{|c|}{Nemotron4-15B (FP32 PPL = 5.87)} \\ 
 \hline
 \hline
 64  & 6.3 & 6.20 & 6.13 & 6.08  \\
 \hline
 32  & 6.24 & 6.12 & 6.07 & 6.03  \\
 \hline
 16  & 6.12 & 6.14 & 6.04 & 6.02  \\
 \hline
 \multicolumn{5}{|c|}{Nemotron4-340B (FP32 PPL = 3.48)} \\ 
 \hline
 \hline
 64 & 3.67 & 3.62 & 3.60 & 3.59 \\
 \hline
 32 & 3.63 & 3.61 & 3.59 & 3.56 \\
 \hline
 16 & 3.61 & 3.58 & 3.57 & 3.55 \\
 \hline
\end{tabular}
\caption{\label{tab:ppl_llama7B_nemo15B} Wikitext-103 perplexity compared to FP32 baseline in Llama2-7B and Nemotron4-15B, 340B models}
\end{table}

%\subsection{Perplexity achieved by various LO-BCQ configurations on MMLU dataset}


\begin{table} \centering
\begin{tabular}{|c||c|c|c|c||c|c|c|c|} 
\hline
 $L_b \rightarrow$& \multicolumn{4}{c||}{8} & \multicolumn{4}{c||}{8}\\
 \hline
 \backslashbox{$L_A$\kern-1em}{\kern-1em$N_c$} & 2 & 4 & 8 & 16 & 2 & 4 & 8 & 16  \\
 %$N_c \rightarrow$ & 2 & 4 & 8 & 16 & 2 & 4 & 2 \\
 \hline
 \hline
 \multicolumn{5}{|c|}{Llama2-7B (FP32 Accuracy = 45.8\%)} & \multicolumn{4}{|c|}{Llama2-70B (FP32 Accuracy = 69.12\%)} \\ 
 \hline
 \hline
 64 & 43.9 & 43.4 & 43.9 & 44.9 & 68.07 & 68.27 & 68.17 & 68.75 \\
 \hline
 32 & 44.5 & 43.8 & 44.9 & 44.5 & 68.37 & 68.51 & 68.35 & 68.27  \\
 \hline
 16 & 43.9 & 42.7 & 44.9 & 45 & 68.12 & 68.77 & 68.31 & 68.59  \\
 \hline
 \hline
 \multicolumn{5}{|c|}{GPT3-22B (FP32 Accuracy = 38.75\%)} & \multicolumn{4}{|c|}{Nemotron4-15B (FP32 Accuracy = 64.3\%)} \\ 
 \hline
 \hline
 64 & 36.71 & 38.85 & 38.13 & 38.92 & 63.17 & 62.36 & 63.72 & 64.09 \\
 \hline
 32 & 37.95 & 38.69 & 39.45 & 38.34 & 64.05 & 62.30 & 63.8 & 64.33  \\
 \hline
 16 & 38.88 & 38.80 & 38.31 & 38.92 & 63.22 & 63.51 & 63.93 & 64.43  \\
 \hline
\end{tabular}
\caption{\label{tab:mmlu_abalation} Accuracy on MMLU dataset across GPT3-22B, Llama2-7B, 70B and Nemotron4-15B models.}
\end{table}


%\subsection{Perplexity achieved by various LO-BCQ configurations on LM evaluation harness}

\begin{table} \centering
\begin{tabular}{|c||c|c|c|c||c|c|c|c|} 
\hline
 $L_b \rightarrow$& \multicolumn{4}{c||}{8} & \multicolumn{4}{c||}{8}\\
 \hline
 \backslashbox{$L_A$\kern-1em}{\kern-1em$N_c$} & 2 & 4 & 8 & 16 & 2 & 4 & 8 & 16  \\
 %$N_c \rightarrow$ & 2 & 4 & 8 & 16 & 2 & 4 & 2 \\
 \hline
 \hline
 \multicolumn{5}{|c|}{Race (FP32 Accuracy = 37.51\%)} & \multicolumn{4}{|c|}{Boolq (FP32 Accuracy = 64.62\%)} \\ 
 \hline
 \hline
 64 & 36.94 & 37.13 & 36.27 & 37.13 & 63.73 & 62.26 & 63.49 & 63.36 \\
 \hline
 32 & 37.03 & 36.36 & 36.08 & 37.03 & 62.54 & 63.51 & 63.49 & 63.55  \\
 \hline
 16 & 37.03 & 37.03 & 36.46 & 37.03 & 61.1 & 63.79 & 63.58 & 63.33  \\
 \hline
 \hline
 \multicolumn{5}{|c|}{Winogrande (FP32 Accuracy = 58.01\%)} & \multicolumn{4}{|c|}{Piqa (FP32 Accuracy = 74.21\%)} \\ 
 \hline
 \hline
 64 & 58.17 & 57.22 & 57.85 & 58.33 & 73.01 & 73.07 & 73.07 & 72.80 \\
 \hline
 32 & 59.12 & 58.09 & 57.85 & 58.41 & 73.01 & 73.94 & 72.74 & 73.18  \\
 \hline
 16 & 57.93 & 58.88 & 57.93 & 58.56 & 73.94 & 72.80 & 73.01 & 73.94  \\
 \hline
\end{tabular}
\caption{\label{tab:mmlu_abalation} Accuracy on LM evaluation harness tasks on GPT3-1.3B model.}
\end{table}

\begin{table} \centering
\begin{tabular}{|c||c|c|c|c||c|c|c|c|} 
\hline
 $L_b \rightarrow$& \multicolumn{4}{c||}{8} & \multicolumn{4}{c||}{8}\\
 \hline
 \backslashbox{$L_A$\kern-1em}{\kern-1em$N_c$} & 2 & 4 & 8 & 16 & 2 & 4 & 8 & 16  \\
 %$N_c \rightarrow$ & 2 & 4 & 8 & 16 & 2 & 4 & 2 \\
 \hline
 \hline
 \multicolumn{5}{|c|}{Race (FP32 Accuracy = 41.34\%)} & \multicolumn{4}{|c|}{Boolq (FP32 Accuracy = 68.32\%)} \\ 
 \hline
 \hline
 64 & 40.48 & 40.10 & 39.43 & 39.90 & 69.20 & 68.41 & 69.45 & 68.56 \\
 \hline
 32 & 39.52 & 39.52 & 40.77 & 39.62 & 68.32 & 67.43 & 68.17 & 69.30  \\
 \hline
 16 & 39.81 & 39.71 & 39.90 & 40.38 & 68.10 & 66.33 & 69.51 & 69.42  \\
 \hline
 \hline
 \multicolumn{5}{|c|}{Winogrande (FP32 Accuracy = 67.88\%)} & \multicolumn{4}{|c|}{Piqa (FP32 Accuracy = 78.78\%)} \\ 
 \hline
 \hline
 64 & 66.85 & 66.61 & 67.72 & 67.88 & 77.31 & 77.42 & 77.75 & 77.64 \\
 \hline
 32 & 67.25 & 67.72 & 67.72 & 67.00 & 77.31 & 77.04 & 77.80 & 77.37  \\
 \hline
 16 & 68.11 & 68.90 & 67.88 & 67.48 & 77.37 & 78.13 & 78.13 & 77.69  \\
 \hline
\end{tabular}
\caption{\label{tab:mmlu_abalation} Accuracy on LM evaluation harness tasks on GPT3-8B model.}
\end{table}

\begin{table} \centering
\begin{tabular}{|c||c|c|c|c||c|c|c|c|} 
\hline
 $L_b \rightarrow$& \multicolumn{4}{c||}{8} & \multicolumn{4}{c||}{8}\\
 \hline
 \backslashbox{$L_A$\kern-1em}{\kern-1em$N_c$} & 2 & 4 & 8 & 16 & 2 & 4 & 8 & 16  \\
 %$N_c \rightarrow$ & 2 & 4 & 8 & 16 & 2 & 4 & 2 \\
 \hline
 \hline
 \multicolumn{5}{|c|}{Race (FP32 Accuracy = 40.67\%)} & \multicolumn{4}{|c|}{Boolq (FP32 Accuracy = 76.54\%)} \\ 
 \hline
 \hline
 64 & 40.48 & 40.10 & 39.43 & 39.90 & 75.41 & 75.11 & 77.09 & 75.66 \\
 \hline
 32 & 39.52 & 39.52 & 40.77 & 39.62 & 76.02 & 76.02 & 75.96 & 75.35  \\
 \hline
 16 & 39.81 & 39.71 & 39.90 & 40.38 & 75.05 & 73.82 & 75.72 & 76.09  \\
 \hline
 \hline
 \multicolumn{5}{|c|}{Winogrande (FP32 Accuracy = 70.64\%)} & \multicolumn{4}{|c|}{Piqa (FP32 Accuracy = 79.16\%)} \\ 
 \hline
 \hline
 64 & 69.14 & 70.17 & 70.17 & 70.56 & 78.24 & 79.00 & 78.62 & 78.73 \\
 \hline
 32 & 70.96 & 69.69 & 71.27 & 69.30 & 78.56 & 79.49 & 79.16 & 78.89  \\
 \hline
 16 & 71.03 & 69.53 & 69.69 & 70.40 & 78.13 & 79.16 & 79.00 & 79.00  \\
 \hline
\end{tabular}
\caption{\label{tab:mmlu_abalation} Accuracy on LM evaluation harness tasks on GPT3-22B model.}
\end{table}

\begin{table} \centering
\begin{tabular}{|c||c|c|c|c||c|c|c|c|} 
\hline
 $L_b \rightarrow$& \multicolumn{4}{c||}{8} & \multicolumn{4}{c||}{8}\\
 \hline
 \backslashbox{$L_A$\kern-1em}{\kern-1em$N_c$} & 2 & 4 & 8 & 16 & 2 & 4 & 8 & 16  \\
 %$N_c \rightarrow$ & 2 & 4 & 8 & 16 & 2 & 4 & 2 \\
 \hline
 \hline
 \multicolumn{5}{|c|}{Race (FP32 Accuracy = 44.4\%)} & \multicolumn{4}{|c|}{Boolq (FP32 Accuracy = 79.29\%)} \\ 
 \hline
 \hline
 64 & 42.49 & 42.51 & 42.58 & 43.45 & 77.58 & 77.37 & 77.43 & 78.1 \\
 \hline
 32 & 43.35 & 42.49 & 43.64 & 43.73 & 77.86 & 75.32 & 77.28 & 77.86  \\
 \hline
 16 & 44.21 & 44.21 & 43.64 & 42.97 & 78.65 & 77 & 76.94 & 77.98  \\
 \hline
 \hline
 \multicolumn{5}{|c|}{Winogrande (FP32 Accuracy = 69.38\%)} & \multicolumn{4}{|c|}{Piqa (FP32 Accuracy = 78.07\%)} \\ 
 \hline
 \hline
 64 & 68.9 & 68.43 & 69.77 & 68.19 & 77.09 & 76.82 & 77.09 & 77.86 \\
 \hline
 32 & 69.38 & 68.51 & 68.82 & 68.90 & 78.07 & 76.71 & 78.07 & 77.86  \\
 \hline
 16 & 69.53 & 67.09 & 69.38 & 68.90 & 77.37 & 77.8 & 77.91 & 77.69  \\
 \hline
\end{tabular}
\caption{\label{tab:mmlu_abalation} Accuracy on LM evaluation harness tasks on Llama2-7B model.}
\end{table}

\begin{table} \centering
\begin{tabular}{|c||c|c|c|c||c|c|c|c|} 
\hline
 $L_b \rightarrow$& \multicolumn{4}{c||}{8} & \multicolumn{4}{c||}{8}\\
 \hline
 \backslashbox{$L_A$\kern-1em}{\kern-1em$N_c$} & 2 & 4 & 8 & 16 & 2 & 4 & 8 & 16  \\
 %$N_c \rightarrow$ & 2 & 4 & 8 & 16 & 2 & 4 & 2 \\
 \hline
 \hline
 \multicolumn{5}{|c|}{Race (FP32 Accuracy = 48.8\%)} & \multicolumn{4}{|c|}{Boolq (FP32 Accuracy = 85.23\%)} \\ 
 \hline
 \hline
 64 & 49.00 & 49.00 & 49.28 & 48.71 & 82.82 & 84.28 & 84.03 & 84.25 \\
 \hline
 32 & 49.57 & 48.52 & 48.33 & 49.28 & 83.85 & 84.46 & 84.31 & 84.93  \\
 \hline
 16 & 49.85 & 49.09 & 49.28 & 48.99 & 85.11 & 84.46 & 84.61 & 83.94  \\
 \hline
 \hline
 \multicolumn{5}{|c|}{Winogrande (FP32 Accuracy = 79.95\%)} & \multicolumn{4}{|c|}{Piqa (FP32 Accuracy = 81.56\%)} \\ 
 \hline
 \hline
 64 & 78.77 & 78.45 & 78.37 & 79.16 & 81.45 & 80.69 & 81.45 & 81.5 \\
 \hline
 32 & 78.45 & 79.01 & 78.69 & 80.66 & 81.56 & 80.58 & 81.18 & 81.34  \\
 \hline
 16 & 79.95 & 79.56 & 79.79 & 79.72 & 81.28 & 81.66 & 81.28 & 80.96  \\
 \hline
\end{tabular}
\caption{\label{tab:mmlu_abalation} Accuracy on LM evaluation harness tasks on Llama2-70B model.}
\end{table}

%\section{MSE Studies}
%\textcolor{red}{TODO}


\subsection{Number Formats and Quantization Method}
\label{subsec:numFormats_quantMethod}
\subsubsection{Integer Format}
An $n$-bit signed integer (INT) is typically represented with a 2s-complement format \citep{yao2022zeroquant,xiao2023smoothquant,dai2021vsq}, where the most significant bit denotes the sign.

\subsubsection{Floating Point Format}
An $n$-bit signed floating point (FP) number $x$ comprises of a 1-bit sign ($x_{\mathrm{sign}}$), $B_m$-bit mantissa ($x_{\mathrm{mant}}$) and $B_e$-bit exponent ($x_{\mathrm{exp}}$) such that $B_m+B_e=n-1$. The associated constant exponent bias ($E_{\mathrm{bias}}$) is computed as $(2^{{B_e}-1}-1)$. We denote this format as $E_{B_e}M_{B_m}$.  

\subsubsection{Quantization Scheme}
\label{subsec:quant_method}
A quantization scheme dictates how a given unquantized tensor is converted to its quantized representation. We consider FP formats for the purpose of illustration. Given an unquantized tensor $\bm{X}$ and an FP format $E_{B_e}M_{B_m}$, we first, we compute the quantization scale factor $s_X$ that maps the maximum absolute value of $\bm{X}$ to the maximum quantization level of the $E_{B_e}M_{B_m}$ format as follows:
\begin{align}
\label{eq:sf}
    s_X = \frac{\mathrm{max}(|\bm{X}|)}{\mathrm{max}(E_{B_e}M_{B_m})}
\end{align}
In the above equation, $|\cdot|$ denotes the absolute value function.

Next, we scale $\bm{X}$ by $s_X$ and quantize it to $\hat{\bm{X}}$ by rounding it to the nearest quantization level of $E_{B_e}M_{B_m}$ as:

\begin{align}
\label{eq:tensor_quant}
    \hat{\bm{X}} = \text{round-to-nearest}\left(\frac{\bm{X}}{s_X}, E_{B_e}M_{B_m}\right)
\end{align}

We perform dynamic max-scaled quantization \citep{wu2020integer}, where the scale factor $s$ for activations is dynamically computed during runtime.

\subsection{Vector Scaled Quantization}
\begin{wrapfigure}{r}{0.35\linewidth}
  \centering
  \includegraphics[width=\linewidth]{sections/figures/vsquant.jpg}
  \caption{\small Vectorwise decomposition for per-vector scaled quantization (VSQ \citep{dai2021vsq}).}
  \label{fig:vsquant}
\end{wrapfigure}
During VSQ \citep{dai2021vsq}, the operand tensors are decomposed into 1D vectors in a hardware friendly manner as shown in Figure \ref{fig:vsquant}. Since the decomposed tensors are used as operands in matrix multiplications during inference, it is beneficial to perform this decomposition along the reduction dimension of the multiplication. The vectorwise quantization is performed similar to tensorwise quantization described in Equations \ref{eq:sf} and \ref{eq:tensor_quant}, where a scale factor $s_v$ is required for each vector $\bm{v}$ that maps the maximum absolute value of that vector to the maximum quantization level. While smaller vector lengths can lead to larger accuracy gains, the associated memory and computational overheads due to the per-vector scale factors increases. To alleviate these overheads, VSQ \citep{dai2021vsq} proposed a second level quantization of the per-vector scale factors to unsigned integers, while MX \citep{rouhani2023shared} quantizes them to integer powers of 2 (denoted as $2^{INT}$).

\subsubsection{MX Format}
The MX format proposed in \citep{rouhani2023microscaling} introduces the concept of sub-block shifting. For every two scalar elements of $b$-bits each, there is a shared exponent bit. The value of this exponent bit is determined through an empirical analysis that targets minimizing quantization MSE. We note that the FP format $E_{1}M_{b}$ is strictly better than MX from an accuracy perspective since it allocates a dedicated exponent bit to each scalar as opposed to sharing it across two scalars. Therefore, we conservatively bound the accuracy of a $b+2$-bit signed MX format with that of a $E_{1}M_{b}$ format in our comparisons. For instance, we use E1M2 format as a proxy for MX4.

\begin{figure}
    \centering
    \includegraphics[width=1\linewidth]{sections//figures/BlockFormats.pdf}
    \caption{\small Comparing LO-BCQ to MX format.}
    \label{fig:block_formats}
\end{figure}

Figure \ref{fig:block_formats} compares our $4$-bit LO-BCQ block format to MX \citep{rouhani2023microscaling}. As shown, both LO-BCQ and MX decompose a given operand tensor into block arrays and each block array into blocks. Similar to MX, we find that per-block quantization ($L_b < L_A$) leads to better accuracy due to increased flexibility. While MX achieves this through per-block $1$-bit micro-scales, we associate a dedicated codebook to each block through a per-block codebook selector. Further, MX quantizes the per-block array scale-factor to E8M0 format without per-tensor scaling. In contrast during LO-BCQ, we find that per-tensor scaling combined with quantization of per-block array scale-factor to E4M3 format results in superior inference accuracy across models. 


% \begin{thebibliography}{9}
% \bibitem{Laport:LaTeX}
% L.~Lamport,
%   \emph{\LaTeX: A Document Preparation System,} 
%   Addison-Wesley, Reading, Massachusetts, USA, 2nd~ed., 1994. 

% \bibitem{GMS:LaTeXComp}
% F.~Mittelbach, M,~Goossens, J.~Braams, D.~Carlisle, and
% C.~Rowley, \emph{The {\LaTeX} Companion,} Addison-Wesley,
% Reading, Massachusetts, USA, 2nd~ed., 2004.

% \bibitem{oetiker_latex}
% T.~Oetiker, H.~Partl, I.~Hyna, and E.~Schlegl, \emph{The Not So Short
%   Introduction to {\LaTeX2e}}, version 5.06, Jun.~20, 2016. [Online].
%   Available: \url{https://tobi.oetiker.ch/lshort/}

% \bibitem{typesetmoser}
% S.~M. Moser, \emph{How to Typeset Equations in {\LaTeX}}, version 4.6,
%   Sep. 29, 2017. [Online]. Available:
%   \url{http://moser-isi.ethz.ch/manuals.html#eqlatex}

% \bibitem{IEEE:pdfsettings}
% IEEE, \emph{Preparing Conference Content for the IEEE Xplore Digital
%   Library.} [Online]. Available:
%   \url{http://www.ieee.org/conferences_events/conferences/organizers/pubs/preparing_content.html}

% \bibitem{IEEE:AuthorToolbox}
% IEEE, \emph{Author Digital Toolbox.} [Online.] Available:
%   \url{http://www.ieee.org/publications_standards/publications/authors/authors_journals.html}

% \end{thebibliography}


\end{document}


%%%%%%
%% Some comments about useful packages
%% (extract from bare_conf.tex by Michael Shell)
%%

% *** MISC UTILITY PACKAGES ***
%
%\usepackage{ifpdf}
% Heiko Oberdiek's ifpdf.sty is very useful if you need conditional
% compilation based on whether the output is pdf or dvi.
% usage:
% \ifpdf
%   % pdf code
% \else
%   % dvi code
% \fi
% The latest version of ifpdf.sty can be obtained from:
% http://www.ctan.org/pkg/ifpdf
% Also, note that IEEEtran.cls V1.7 and later provides a builtin
% \ifCLASSINFOpdf conditional that works the same way.
% When switching from latex to pdflatex and vice-versa, the compiler may
% have to be run twice to clear warning/error messages.


% *** CITATION PACKAGES ***
%
%\usepackage{cite}
% cite.sty was written by Donald Arseneau
% V1.6 and later of IEEEtran pre-defines the format of the cite.sty package
% \cite{} output to follow that of the IEEE. Loading the cite package will
% result in citation numbers being automatically sorted and properly
% "compressed/ranged". e.g., [1], [9], [2], [7], [5], [6] without using
% cite.sty will become [1], [2], [5]--[7], [9] using cite.sty. cite.sty's
% \cite will automatically add leading space, if needed. Use cite.sty's
% noadjust option (cite.sty V3.8 and later) if you want to turn this off
% such as if a citation ever needs to be enclosed in parenthesis.
% cite.sty is already installed on most LaTeX systems. Be sure and use
% version 5.0 (2009-03-20) and later if using hyperref.sty.
% The latest version can be obtained at:
% http://www.ctan.org/pkg/cite
% The documentation is contained in the cite.sty file itself.


% *** GRAPHICS RELATED PACKAGES ***
%
\ifCLASSINFOpdf
  % \usepackage[pdftex]{graphicx}
  % declare the path(s) where your graphic files are
  % \graphicspath{{../pdf/}{../jpeg/}}
  % and their extensions so you won't have to specify these with
  % every instance of \includegraphics
  % \DeclareGraphicsExtensions{.pdf,.jpeg,.png}
\else
  % or other class option (dvipsone, dvipdf, if not using dvips). graphicx
  % will default to the driver specified in the system graphics.cfg if no
  % driver is specified.
  % \usepackage[dvips]{graphicx}
  % declare the path(s) where your graphic files are
  % \graphicspath{{../eps/}}
  % and their extensions so you won't have to specify these with
  % every instance of \includegraphics
  % \DeclareGraphicsExtensions{.eps}
\fi
% graphicx was written by David Carlisle and Sebastian Rahtz. It is
% required if you want graphics, photos, etc. graphicx.sty is already
% installed on most LaTeX systems. The latest version and documentation
% can be obtained at: 
% http://www.ctan.org/pkg/graphicx
% Another good source of documentation is "Using Imported Graphics in
% LaTeX2e" by Keith Reckdahl which can be found at:
% http://www.ctan.org/pkg/epslatex
%
% latex, and pdflatex in dvi mode, support graphics in encapsulated
% postscript (.eps) format. pdflatex in pdf mode supports graphics
% in .pdf, .jpeg, .png and .mps (metapost) formats. Users should ensure
% that all non-photo figures use a vector format (.eps, .pdf, .mps) and
% not a bitmapped formats (.jpeg, .png). The IEEE frowns on bitmapped formats
% which can result in "jaggedy"/blurry rendering of lines and letters as
% well as large increases in file sizes.
%
% You can find documentation about the pdfTeX application at:
% http://www.tug.org/applications/pdftex


% *** MATH PACKAGES ***
%
%\usepackage{amsmath}
% A popular package from the American Mathematical Society that provides
% many useful and powerful commands for dealing with mathematics.
%
% Note that the amsmath package sets \interdisplaylinepenalty to 10000
% thus preventing page breaks from occurring within multiline equations. Use:
%\interdisplaylinepenalty=2500
% after loading amsmath to restore such page breaks as IEEEtran.cls normally
% does. amsmath.sty is already installed on most LaTeX systems. The latest
% version and documentation can be obtained at:
% http://www.ctan.org/pkg/amsmath


% *** SPECIALIZED LIST PACKAGES ***
%
%\usepackage{algorithmic}
% algorithmic.sty was written by Peter Williams and Rogerio Brito.
% This package provides an algorithmic environment for describing algorithms.
% You can use the algorithmic environment in-text or within a figure
% environment to provide for a floating algorithm. Do NOT use the algorithm
% floating environment provided by algorithm.sty (by the same authors) or
% algorithm2e.sty (by Christophe Fiorio) as the IEEE does not use dedicated
% algorithm float types and packages that provide these will not provide
% correct IEEE style captions. The latest version and documentation of
% algorithmic.sty can be obtained at:
% http://www.ctan.org/pkg/algorithms
% Also of interest may be the (relatively newer and more customizable)
% algorithmicx.sty package by Szasz Janos:
% http://www.ctan.org/pkg/algorithmicx


% *** ALIGNMENT PACKAGES ***
%
%\usepackage{array}
% Frank Mittelbach's and David Carlisle's array.sty patches and improves
% the standard LaTeX2e array and tabular environments to provide better
% appearance and additional user controls. As the default LaTeX2e table
% generation code is lacking to the point of almost being broken with
% respect to the quality of the end results, all users are strongly
% advised to use an enhanced (at the very least that provided by array.sty)
% set of table tools. array.sty is already installed on most systems. The
% latest version and documentation can be obtained at:
% http://www.ctan.org/pkg/array

% IEEEtran contains the IEEEeqnarray family of commands that can be used to
% generate multiline equations as well as matrices, tables, etc., of high
% quality.


% *** SUBFIGURE PACKAGES ***
%\ifCLASSOPTIONcompsoc
%  \usepackage[caption=false,font=normalsize,labelfont=sf,textfont=sf]{subfig}
%\else
%  \usepackage[caption=false,font=footnotesize]{subfig}
%\fi
% subfig.sty, written by Steven Douglas Cochran, is the modern replacement
% for subfigure.sty, the latter of which is no longer maintained and is
% incompatible with some LaTeX packages including fixltx2e. However,
% subfig.sty requires and automatically loads Axel Sommerfeldt's caption.sty
% which will override IEEEtran.cls' handling of captions and this will result
% in non-IEEE style figure/table captions. To prevent this problem, be sure
% and invoke subfig.sty's "caption=false" package option (available since
% subfig.sty version 1.3, 2005/06/28) as this is will preserve IEEEtran.cls
% handling of captions.
% Note that the Computer Society format requires a larger sans serif font
% than the serif footnote size font used in traditional IEEE formatting
% and thus the need to invoke different subfig.sty package options depending
% on whether compsoc mode has been enabled.
%
% The latest version and documentation of subfig.sty can be obtained at:
% http://www.ctan.org/pkg/subfig


% *** FLOAT PACKAGES ***
%
%\usepackage{fixltx2e}
% fixltx2e, the successor to the earlier fix2col.sty, was written by
% Frank Mittelbach and David Carlisle. This package corrects a few problems
% in the LaTeX2e kernel, the most notable of which is that in current
% LaTeX2e releases, the ordering of single and double column floats is not
% guaranteed to be preserved. Thus, an unpatched LaTeX2e can allow a
% single column figure to be placed prior to an earlier double column
% figure.
% Be aware that LaTeX2e kernels dated 2015 and later have fixltx2e.sty's
% corrections already built into the system in which case a warning will
% be issued if an attempt is made to load fixltx2e.sty as it is no longer
% needed.
% The latest version and documentation can be found at:
% http://www.ctan.org/pkg/fixltx2e


%\usepackage{stfloats}
% stfloats.sty was written by Sigitas Tolusis. This package gives LaTeX2e
% the ability to do double column floats at the bottom of the page as well
% as the top. (e.g., "\begin{figure*}[!b]" is not normally possible in
% LaTeX2e). It also provides a command:
%\fnbelowfloat
% to enable the placement of footnotes below bottom floats (the standard
% LaTeX2e kernel puts them above bottom floats). This is an invasive package
% which rewrites many portions of the LaTeX2e float routines. It may not work
% with other packages that modify the LaTeX2e float routines. The latest
% version and documentation can be obtained at:
% http://www.ctan.org/pkg/stfloats
% Do not use the stfloats baselinefloat ability as the IEEE does not allow
% \baselineskip to stretch. Authors submitting work to the IEEE should note
% that the IEEE rarely uses double column equations and that authors should try
% to avoid such use. Do not be tempted to use the cuted.sty or midfloat.sty
% packages (also by Sigitas Tolusis) as the IEEE does not format its papers in
% such ways.
% Do not attempt to use stfloats with fixltx2e as they are incompatible.
% Instead, use Morten Hogholm's dblfloatfix which combines the features
% of both fixltx2e and stfloats:
%
% \usepackage{dblfloatfix}
% The latest version can be found at:
% http://www.ctan.org/pkg/dblfloatfix


% *** PDF and URL PACKAGES ***
%
%\usepackage{url}
% url.sty was written by Donald Arseneau. It provides better support for
% handling and breaking URLs. url.sty is already installed on most LaTeX
% systems. The latest version and documentation can be obtained at:
% http://www.ctan.org/pkg/url
% Basically, \url{my_url_here}.



% *** Do not adjust lengths that control margins, column widths, etc. ***
% *** Do not use packages that alter fonts (such as pslatex).         ***
%%%%%%


%%% Local Variables:
%%% mode: latex
%%% TeX-master: t
%%% End:
