\section{Literature review}
\label{sec:liter}
Most of the existing studies on NTFE methods are validated in specific cities, the aforementioned trade-off challenge has not been well addressed. Specifically, early studies relied on the city-specific supplementary data sources for NTFE \citep{daganzo1997fundamentals, pendyala2000multi, lam2001activity, vuchic2007urban}. For example, \cite{pendyala2000multi} conducted multi-day and multi-period surveys to collect OD data, land-use patterns and household numbers for estimating the network-wide flow using the traditional four-step model. Despite their insights, these methods require significant effort and expense in data collection, and the reliance on strong model assumptions often fails to reflect real-world conditions \citep{google_flow,transfer_flow_3}, which limits their applications. With the development of sensor technologies, various ITS facilities have introduced numerous supplementary data, such as LPR data, video data and trajectory data, to tackle the NTFE problem \citep{cellphone_flow, machine_flow_1, camera_flow}. For example, \citet{camera_flow} employed convolutional neural networks (CNNs) to estimate the traffic flow based on the data collected from video surveillance cameras. However, these emerging data sources are only available in a few cities and therefore the associated methods suffer from the generality issue.

Recently, numerous studies have utilized FCD or observed traffic flow, available across multiple cities, for NTFE. Although these studies demonstrate the potential of applying NTFE across cities, their accuracy may be constrained due to the inherent limitations of the information provided by FCD or observed flow alone. For example, some studies formulated NTFE as a data imputation problem and solved the problem with various methods, such as autoregressive integrated moving average (ARIMA) \citep{arima_flow} and Principle Component Analysis (PCA) \citep{BPCA_flow, PPCA_flow, KPPCA_flow}. The spatial-temporal relationships of traffic data have also been exploited and constructed to enhance the NTFE through matrix/tensor decomposition \citep{tensor_decomposition_flow_2,tensor_decomposition_flow_3,tensor_decomposition_4_flow,tensor_decomposition_5_flow,tensor_decomposition_6_flow} and machine-learning-based methods \citep{machine_flow_1,machine_flow_2}. However, most of these methods relied on the historical records of traffic flow on roads, which may degrade the estimation accuracy for those unseen links in the training set. There are also studies that have estimated unobserved traffic flow by uncovering the latent relationship between traffic speed from FCD and observed flow. These efforts include not only estimating the link-based fundamental diagram \citep{ross1988traffic, kerner2009introduction, anuar2015estimating, google_flow} and MFD \citep{est_MFD_1,est_MFD_2,est_MFD_3} but also applying advanced deep learning techniques \citep{tgc_rnn_flow,sgmc_flow, GNN_new_flow}, such as GNNs, to exploit the spatiotemporal dependencies between flow and speed. Although neural networks can effectively capture these dependencies, their accuracy remains unsatisfactory when applied to unseen roads. Essentially, this is because the limited information provided by FCD alone is insufficient to comprehensively model traffic flow dynamics on unobserved roads.

There is also an emerging trend focusing on estimating traffic flow on unseen roads by leveraging multi-source data. The multi-source data includes but is not limited to geographical, demographical, sociological, and meteorological data, and these kinds of data can be acquired in various manners. For example, the global geographical data (\textit{e.g.,} road topology, POI, and land cover) can be obtained through several map providers such as OpenStreetMap, Google Maps, and Overture Maps Foundation. The local demographical and sociological data can be obtained through census \citep{demo_census} in several countries, while the global demographical data can be estimated through satellite images and statistical population model \citep{pop_map_get01,pop_map_get02}. The global meteorological data can be derived from the Global Historical Climatology Network hourly (GHCNh) dataset \citep{GHCNh}.
In the NTFE topic, \citet{multiple_regression_flow} proposed a multiple regression method based on network topology, Annual Average Daily Traffic (AADT), transport data, and FCD. The estimated travel time from Google Maps can also benefit traffic flow estimation \citep{google_flow}. \citet{open_data_flow_estimation} utilized traffic speed from Uber Movement and road static attributes from OSM for network flow inference. Several studies also have incorporated the POI, infrastructure, weather, road topology, and socioeconomic factors into the NTFE \citep{machine_flow_1, ssl_flow, data_driven_flow, search_flow_estimation}. For example, \citet{vision_flow} included visual information of probe vehicles for NTFE. \citet{flow_topo_pop} used topography information and population statistics. Other similar ideas of using geographical data to estimate traffic flow have also been implemented \citep{transfer_flow_1,gis_flow_1,gis_flow_2}. Nonetheless, most, if not all, of these studies validate their methods within specific cities because the adopted data are not unified and globally available. Consequently, they continue to face difficulties in overcoming the trade-off challenges.