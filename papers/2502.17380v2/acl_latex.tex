% This must be in the first 5 lines to tell arXiv to use pdfLaTeX, which is strongly recommended.
\pdfoutput=1
% In particular, the hyperref package requires pdfLaTeX in order to break URLs across lines.

\documentclass[11pt]{article}

% Change "review" to "final" to generate the final (sometimes called camera-ready) version.
% Change to "preprint" to generate a non-anonymous version with page numbers.
\usepackage[final]{acl}

% Standard package includes
\usepackage{times}
\usepackage{latexsym}

% For proper rendering and hyphenation of words containing Latin characters (including in bib files)
\usepackage[T1]{fontenc}
% For Vietnamese characters
% \usepackage[T5]{fontenc}
% See https://www.latex-project.org/help/documentation/encguide.pdf for other character sets

% This assumes your files are encoded as UTF8
\usepackage[utf8]{inputenc}

% This is not strictly necessary, and may be commented out,
% but it will improve the layout of the manuscript,
% and will typically save some space.
\usepackage{microtype}
\usepackage{enumitem}

% This is also not strictly necessary, and may be commented out.
% However, it will improve the aesthetics of text in
% the typewriter font.
\usepackage{inconsolata}

%Including images in your LaTeX document requires adding
%additional package(s)
\usepackage{graphicx}

% Custom
\usepackage{adjustbox}
\usepackage{tabularx}
\usepackage{multirow}
\usepackage{booktabs}
\usepackage{hyperref}
\usepackage{tcolorbox}
\usepackage{acl}
\usepackage{amssymb}
\usepackage{amsmath}
\usepackage{subcaption}


% If the title and author information does not fit in the area allocated, uncomment the following
%
%\setlength\titlebox{<dim>}
%
% and set <dim> to something 5cm or larger.

% \title{Towards Unified End-to-End Multi-Lingual Speech Recognition and Translation via Model Merging}

\title{Low-Rank and Sparse Model Merging for Multi-Lingual\\ Speech Recognition and Translation}

% Multi-task Multi-lingual Speech Model Merging


\author{
\textbf{Qiuming Zhao\textsuperscript{1} \qquad Guangzhi Sun\textsuperscript{2}} \\ 
\textbf{Chao Zhang\textsuperscript{1} \qquad Mingxing Xu\textsuperscript{1} \qquad Thomas Fang Zheng\textsuperscript{1}\thanks{Correspondence}} \\
$^1$Tsinghua University, China \\
$^2$University of Cambridge, United Kingdom \\
\texttt{zqm23@mails.tsinghua.edu.cn, gs534@cam.ac.uk,} \\
\texttt{\{cz277,xumx,fzheng\}@tsinghua.edu.cn}
}

% Author information can be set in various styles:
% For several authors from the same institution:
% \author{Author 1 \and ... \and Author n \\
%         Address line \\ ... \\ Address line}
% if the names do not fit well on one line use
%         Author 1 \\ {\bf Author 2} \\ ... \\ {\bf Author n} \\
% For authors from different institutions:
% \author{Author 1 \\ Address line \\  ... \\ Address line
%         \And  ... \And
%         Author n \\ Address line \\ ... \\ Address line}
% To start a separate ``row'' of authors use \AND, as in
% \author{Author 1 \\ Address line \\  ... \\ Address line
%         \AND
%         Author 2 \\ Address line \\ ... \\ Address line \And
%         Author 3 \\ Address line \\ ... \\ Address line}

% \author{Qiuming Zhao \\
%   Affiliation / Address line 1 \\
%   \texttt{email@domain} \\\And
%   Second Author \\
%   Affiliation / Address line 1 \\
%   \texttt{email@domain} \\}

% \author{
%  \textbf{First Author\textsuperscript{1}},
%  \textbf{Second Author\textsuperscript{1,2}},
%  \textbf{Third T. Author\textsuperscript{1}},
%  \textbf{Fourth Author\textsuperscript{1}},
% \\
%  \textbf{Fifth Author\textsuperscript{1,2}},
%  \textbf{Sixth Author\textsuperscript{1}},
%  \textbf{Seventh Author\textsuperscript{1}},
%  \textbf{Eighth Author \textsuperscript{1,2,3,4}},
% \\
%  \textbf{Ninth Author\textsuperscript{1}},
%  \textbf{Tenth Author\textsuperscript{1}},
%  \textbf{Eleventh E. Author\textsuperscript{1,2,3,4,5}},
%  \textbf{Twelfth Author\textsuperscript{1}},
% \\
%  \textbf{Thirteenth Author\textsuperscript{3}},
%  \textbf{Fourteenth F. Author\textsuperscript{2,4}},
%  \textbf{Fifteenth Author\textsuperscript{1}},
%  \textbf{Sixteenth Author\textsuperscript{1}},
% \\
%  \textbf{Seventeenth S. Author\textsuperscript{4,5}},
%  \textbf{Eighteenth Author\textsuperscript{3,4}},
%  \textbf{Nineteenth N. Author\textsuperscript{2,5}},
%  \textbf{Twentieth Author\textsuperscript{1}}
% \\
% \\
%  \textsuperscript{1}Affiliation 1,
%  \textsuperscript{2}Affiliation 2,
%  \textsuperscript{3}Affiliation 3,
%  \textsuperscript{4}Affiliation 4,
%  \textsuperscript{5}Affiliation 5
% \\
%  \small{
%    \textbf{Correspondence:} \href{mailto:email@domain}{email@domain}
%  }
% }

\begin{document}
\maketitle
\begin{abstract}
Language diversity presents a significant challenge in speech-to-text (S2T) tasks, such as automatic speech recognition and translation. Traditional multi-task training approaches aim to address this by jointly optimizing multiple speech recognition and translation tasks across various languages. While models like Whisper, built on these strategies, demonstrate strong performance, they still face issues of high computational cost, language interference, suboptimal training configurations, and limited extensibility. To overcome these challenges, we introduce LoRS-Merging (low-rank and sparse model merging), a novel technique designed to efficiently integrate models trained on different languages or tasks while preserving performance and reducing computational overhead. LoRS-Merging combines low-rank and sparse pruning to retain essential structures while eliminating redundant parameters, mitigating language and task interference, and enhancing extensibility. Experimental results across a range of languages demonstrate that LoRS-Merging reduces the word error rate by 10\% and improves BLEU scores by 4\% compared to conventional multi-lingual multi-task training baselines. Our findings suggest that model merging, particularly LoRS-Merging, is a scalable and effective complement to traditional multi-lingual training strategies for S2T applications\footnote{The detailed data and code will be released at [URL]}.

% Experimental results across a range of languages demonstrate that LoRS-Merging significantly outperforms conventional multi-lingual multi-task training baselines.

%Experimental results across a range of languages demonstrate that LoRS-Merging reduces the word error rate by 10\% and improves BLEU scores by 4\% compared to conventional multi-lingual multi-task training baselines.

%Language diversity presents a significant challenge in speech-to-text (S2T) tasks such as automatic speech recognition and translation. Traditional multi-task training strategies attempt to achieve this by jointly optimizing multiple speech recognition and translation tasks across different languages. While yielding strong pre-trained models such as Whisper, these methods still suffer from high computational cost, language interference, suboptimal training configuration and a lack of extensibility. To address these limitations, we propose a novel model merging technique, LoRS-Merging (low-rank and sparse model merging), to efficiently integrate models trained on different languages or tasks while preserving performance and reducing computational overhead. LoRS-Merging leverages a combination of low-rank and sparse pruning to retain essential structures and eliminate redundant parameters, mitigating task and language interference and improving adaptability. Experimental results across a range of languages show that LoRS-Merging achieves a 10\% relative word error rate (WER) reduction and a 4\% relative BLEU score improvement compared to multi-lingual multi-task training baselines. Our analysis suggests that model merging, particularly LoRS-Merging, is a scalable and effective complementary approach to conventional multi-lingual training strategies for S2T applications\footnote{The detailed data and code will be released at [URL]}.

%Language diversity presents a significant challenge in speech-to-text (S2T) tasks such as automatic speech recognition (ASR) and speech translation (ST). Traditional multi-task training strategies attempt to achieve this by jointly optimizing multiple speech recognition and translation tasks across different languages. While yielding strong pre-trained models such as Whisper, these methods still suffer from high computational cost, language interference, suboptimal training configuration and lack of flexibility. To address these limitations, we propose a novel model merging technique, LoRS-Merging (Low-Rank and Sparse Model Merging), to efficiently integrate models trained on different languages or tasks while preserving performance and reducing computational overhead. LoRS-Merging leverages a combination of low-rank and sparse pruning to retain essential structures and eliminate redundant parameters, mitigating task and language interference and improving adaptability. Experimental results across a range of languages show that LoRS-Merging achieves a 10\% relative word error rate (WER) reduction and a 4\% relative BLEU score improvement compared to multi-lingual multi-task training baselines. Our analysis suggests that model merging, particularly LoRS-Merging, is a scalable and effective complementary approach to conventional multi-lingual training strategies for S2T applications\footnote{The detailed data and code will be released at [URL]}.
\end{abstract}

\section{Introduction}
Language diversity poses a significant challenge in speech-to-text (S2T) tasks, such as automatic speech recognition (ASR) \cite{prabhavalkar2023end} and speech translation (ST) \cite{xu2023recent}. 
With over 7,000 languages spoken worldwide, developing robust S2T systems that generalise across varied linguistic structures remains a fundamental research goal \cite{liu2024recent,cheng2023mu2slam,sun2023towards,saif2024m2asr,wang2021voxpopuli,le2021lightweight}.
% Traditional approaches relied on language-specific models, often requiring extensive labeled data and complex pipelines \cite{young2002htk,povey2011kaldi} including feature extraction, acoustic modeling, language modeling, and decoding. 
% However, 
The advent of end-to-end (E2E) models \cite{chan2016listen,gulati2020conformer,barrault2023seamlessm4t} has marked a paradigm shift in S2T tasks, enabling direct mapping from speech to text across multiple languages within a unified framework.
A prominent example is Whisper \cite{radford2023robust}, an advanced multi-lingual speech model trained on a large-scale, diverse dataset covering multiple languages and tasks.
% Whisper leverages a transformer-based architecture to jointly perform ASR and ST, demonstrating impressive zero-shot and few-shot generalisation capabilities.
Despite these advances, existing multi-lingual models still encounter significant challenges in scalability, efficiency, and performance trade-offs.

To address these challenges, multi-lingual training strategies \cite{saif2024m2asr,xiao2021adversarial,bai2018source} have been adopted, aiming to enhance model generalisation across languages.
These approaches typically rely on joint optimisation of diverse S2T tasks across multiple languages, leveraging shared representations to improve performance.
Nevertheless, multi-lingual training is subject to inherent limitations, including substantial training costs, complex model configurations, and limited access to training data across multiple languages and tasks.
Moreover, when handling new languages, the training methods typically require training from scratch.

To mitigate these issues, this paper proposes to use model merging \cite{ilharcoediting,yang2024model,khan2024sok} to integrate models trained on different languages or tasks while maintaining performance and reducing computational overhead.
Model merging merges the parameters of multiple separate models with different capabilities to build a universal model.
% Model merging offers a simple and efficient way to integrate models trained on different languages or tasks while maintaining performance and reducing computational overhead.
With its high flexibility, model merging enables the seamless incorporation of new languages or tasks without the need for retraining the entire model.
Additionally, since model merging allows models for different languages or tasks to be trained independently, it can effectively alleviate negative transfer issues \cite{wang2019characterizing,zhang2022survey,wang2020negative} commonly observed in multi-lingual training.
This training independence also enables the use of optimal training configurations for each language or task instead of the unified settings required in multi-lingual training.

Moreover, we propose \textbf{Lo}w-\textbf{R}ank and \textbf{S}parse model \textbf{Merging} (LoRS-Merging), which uses a low-rank component to capture the compact structure and a sparse component to capture the scattered details in the weights.
LoRS-Merging retains effective parts of structure and details while reducing redundant parts to reduce task interference.
Specifically, coarse-grained singular value pruning is used to retain the low-rank structure, while fine-grained magnitude pruning is used to remove redundant details.
The main contribution of this paper can be summarised as follows.
\begin{itemize}[itemsep=-1pt, leftmargin=*]
% \setlength\itemsep{0em}
\item We propose LoRS-Merging, a low-rank and sparse model merging method for multi-lingual ASR and speech translation. To the best of our knowledge, LoRS-Merging is the first work that explores model merging for speech models.
\item LoRS-Merging exploits the combination of low-rank structure and sparsity of language-specific and task-specific weights in model merging, minimising the parameter redundancy and conflicts as well as providing an efficient way to incorporate new knowledge from a task or language-specialised model.
% \item Experiments are performed across 10 different languages where LoRS-Merging achieves a significant improvement compared to the multi-lingual multi-task training baseline. Moreover, we show that negative interference largely exists in multi-lingual training and LoRS-Merging alleviates this issue.
\item Experiments are performed across 10 different languages where LoRS-Merging achieves \textbf{10}\% relative WER reduction and \textbf{4}\% relative BLEU increase compared to the multi-lingual multi-task training baseline. Moreover, we show that negative interference largely exists in multi-lingual training and LoRS-Merging alleviates this issue.
\end{itemize}


\section{Related Work}

\subsection{Multi-Lingual ASR and ST}
Multi-lingual speech models inherently face a trade-off between knowledge sharing and negative interference.
% Existing work on multi-lingual learning can be categorised into two categories: language-specific modeling and language-invariant modeling.
% The key question of language-specific modeling is to determine the optimal allocation between shared components and language-specific components.
Early studies adopted hand-picked sub-network sharing strategies, such as language-specific decoders \cite{dong2015multi}, attention heads \cite{zhu2020multilingual}, and layer norm/linear transformation \cite{zhang2020improving}.
Recent research has shifted toward approaches such as mixture-of-experts \cite{kwon2023mole,wang2023language}, adapters \cite{le2021lightweight,kannan2019large}, and pruning \cite{lu2022language,lai2021parp}.
% Language-invariant modeling involves using a unified model for multiple languages, aiming to learn shared representations across languages.
To enhance multi-lingual representation learning, language tokens \cite{johnson2017google}, embeddings \cite{di2019one} or output factorisations \cite{zhang2023umluniversalmonolingualoutput} are introduced to encode language identity, helping the model distinguish between languages.

The more effective approach is to adopt multi-lingual training strategies, such as multi-objective optimisation \cite{saif2024m2asr,zhang2022streamingendtoendmultilingualspeech}, adversarial learning \cite{xiao2021adversarial}, meta learning \cite{hsu2020meta}, and reinforcement learning \cite{bai2018source}.
% Beyond these training strategies, large-scale pretraining has emerged as a fundamental approach for multi-lingual speech modeling.
Moreover, large-scale pre-training by leveraging massive amounts of multi-lingual and multi-task data enables models to learn robust and transferable representations across languages, e.g. Whisper \cite{radford2023robust}, SeamlessM4T \cite{barrault2023seamlessm4t}, and AudioPaLM \cite{rubenstein2023audiopalm}. LoRS-Merging, as an efficient post-training method proposed in this paper, further advances multi-lingual ASR and ST based on pre-trained speech models.
% which demonstrate state-of-the-art performance in multi-lingual ASR and ST.


\subsection{Model Merging}
Model merging \cite{yang2024model,khan2024sok} is an efficient post-training technique that integrates knowledge from models trained on different domains.
One stream of research focuses on the loss landscape geometry \cite{khan2024sok} and studies the linear mode connectivity (LMC) \cite{frankle2020linear,draxler2018essentially} property that demonstrates the existence of a linearly connected path between local minima within the same loss basin.
% Recent studies primarily focus on two perspectives: loss landscapes and model spaces.
% Loss landscape geometry \cite{khan2024sok} includes major aspects such as mode convexity, determinism, directedness, and connectivity.
% Among these, the linear mode connectivity (LMC) \cite{frankle2020linear,draxler2018essentially} property demonstrates that there exists a linearly connected path between multiple local minima within the same loss basin, along which the loss remains nearly constant.
Many studies \cite{nagarajan2019uniform,izmailov2018averaging,frankle2020linear} indicate that if two neural networks share part of their optimisation trajectory, such as different finetuned models from the same pretrained model, they typically satisfy LMC, allowing interpolation without sacrificing accuracy and forming the basis of our model merging method. 
For local minima in different loss basins, inspired by the permutation invariance \cite{entezarirole} of neural networks, neuron alignment techniques \cite{ainsworthgit,singh2020model,tatro2020optimizing} can be used to place them into the same basin, thereby reducing merging loss.

Another stream considers the model spaces, including activation spaces and weight spaces.
Research on activation spaces seeks to align the output representations or loss of the merged model with those of each single model as closely as possible \cite{yangrepresentation,wei2025modeling,xiong2024multi}. Studies based on weight spaces aim to remove redundant parameters or localise effective parameters to resolve task interference.
TIES-Merging \cite{yadav2024ties} and DARE \cite{yu2024language} perform magnitude or random pruning on each single model to significantly remove redundant parameters.
TALL-masks \cite{wanglocalizing} and Localise-and-Stitch \cite{he2024localize} optimise binary masks to localise sparse and effective task-specific parameters. In contrast, LoRS-Merging explores weight space merging by considering not only the detailed parameter redundancy as well as maintaining the effective structure of the weight space via low-rank pruning.
% Additionally, some studies use sophisticated rules to determine merging coefficients, such as RegMean \cite{regmean}, Fisher-Merging \cite{matena2022merging}, and AdaMerging \cite{yangadamerging}.

\section{Methodology}
\begin{figure*}[t]
  \includegraphics[width=\linewidth]{images/LoRS-Merging-v4.pdf}
  \caption{Model merging process with the proposed LoRS-Merging for speech models on multi-lingual ASR and ST tasks.
  % Step 2 applies low-rank adaptation on each language and one task, 
  In step 1, a suitable pre-trained speech model is selected.
  In step 2, the pre-trained model is finetuned with the task-language-specific data.
  In step 3, apply LoRS to the delta parameters to reduce model redundancy.
  In step 4, merge the delta parameters to get a multi-lingual and multi-task merged model.}
  \vspace{-0.5cm}
  \label{fig:overview}
\end{figure*}

\subsection{Preliminaries}
\subsubsection{Task Arithmetic}
Among diverse model merging methods, Task Arithmetic (TA) \cite{ilharcoediting} has become a fundamental technique in this field due to its simplicity and effectiveness.
TA introduces the concept of "task vector", defined as the delta parameter derived by subtracting pretrained weights from finetuned weights.
By performing simple arithmetic operations on task vectors, TA enables task learning, forgetting, and analogising.

Assume that $\theta = \{W_l\}_{l=1}^{L}$ represents the parameters of the model, where $W_l$ is the weight of $l$-th layer, and $L$ is the total number of layers.
Given a pretrained model $\theta_0$ and a model $\theta_i$ finetuned on task $t_i$, the task vector is computed as $\tau_i = \theta_i - \theta_0$.
Multiple task vectors can be summed to form a multi-task model, expressed as $\theta_\text{merged} = \theta_0 + \lambda \sum_{i=1}^{n} \tau_i$,  where $\lambda$ is a scaling coefficient for the task vectors.


\subsubsection{Pruning}
\label{sec:pruning}
Given that neural networks are typically over-parameterised and exhibit high redundancy, a considerable number of neurons or connections can be pruned without affecting accuracy \cite{lecun1989optimal}.
In model merging, pruning methods can reduce redundant parameters to mitigate task interference, thereby improving the merging performance.

% \cite{han2015learning} 
\textbf{Magnitude Pruning} (MP) is an unstructured pruning method that prunes connections based on the magnitude of parameters as a measure of importance.
Specifically, MP prunes the parameters according to a specific ratio $p$, as follows.
\begin{equation}
M_{ij} = 
\begin{cases} 
1 & \text{if } |w_{ij}| \in \text{top } p\% \\
0 & \text{o.w.}
\end{cases}
\end{equation}
% The pruned weight matrix can be represented as:
\begin{equation}
W_\text{pruned} = M \odot W
\end{equation}
where $W, M \in \mathbb{R}^{d \times k}$, and $\odot$ denotes the element-wise multiplication.
However, MP only focuses on the redundancy at the parameter level, overlooking the crucial structural information, which may lead to the disruption of the weight structure.

\textbf{Singular Value Pruning} (SVP) is a structured pruning method that removes smaller singular values and their corresponding singular vectors.
In particular, SVP retains only the top $r$ singular values while discarding the others.
\begin{equation}
W = U \Sigma V^T
\end{equation}
\begin{equation}
W_\text{pruned} = U_r \Sigma_r V_r^T
\end{equation}
where $U \in \mathbb{R}^{d \times d}$ and $V \in \mathbb{R}^{k \times k}$ are the left and right singular vector matrices of $W$, and $U_r$, $V_r$ denote their first $r$ columns.
Although SVP preserves a compact weight structure, its coarse pruning granularity makes it challenging to reduce redundancy at a fine-grained parameter level.


\subsection{Model Merging for Speech Models}

The model merging process for speech model on S2T tasks with LoRS-Merging as an example is shown in Fig. \ref{fig:overview}, which comprises four steps. In step 1, a suitable pre-trained speech model is selected. In step 2, for each target language and target task combination, e.g. Catalan ASR, the pre-trained model is finetuned with the task-language-specific data and the delta weight is obtained. In step 3, weight pruning is applied to remove redundant and conflicting delta parameters. In step 4, task arithmetic is applied to combine pruned delta weights into each single merged matrix and hence obtain the merged model.

Model merging allows new language or task knowledge to be integrated into the model in a flexible post-training manner. When a new set of data for a specific language is obtained, model merging incorporates such knowledge by fine-tuning with the new data alone with data-specific configuration, which also releases the burden of requiring other data to avoid catastrophic forgetting. This benefit is thoroughly demonstrated in our experiments.

\subsection{Low-Rank and Sparse Model Merging}
\begin{figure}[t]
  \includegraphics[width=\linewidth]{images/LoRS-v2.pdf}
  \caption{Illustration of LoRS-Merging method in detail. SVD stands for singular value decomposition and SVP for singular value pruning. MP is magnitude pruning operating on residual of the original weight matrix and the low-rank matrix.}
  \label{fig:lorsmerge}
  \vspace{-0.4cm}
\end{figure}

The weights of neural networks contain information on both structure and details. 
Structural information is coherent, compact, and coarse-grained, whereas detail information is incoherent, scattered, and fine-grained.
Both structural and detail information include effective and redundant parts.
To reduce redundant parts in both the structure and detail aspects of the weights while retaining effective parts, the LoRS-Merging method is introduced as shown in detail in Fig. \ref{fig:lorsmerge}, which exploits the combination of low-rank structure by SVP and sparsity by MP.
SVP performs coarse-grained pruning at the structure level, while MP enables fine-grained pruning at the detail level.

In the implementation, we approximate the original weights as the sum of a low-rank component and a sparse component, where the low-rank component captures the compact structure, and the sparse component captures the scattered details, as shown in Eqn. \eqref{eq:sum}.
\begin{equation}
W \approx L + S
\label{eq:sum}
\end{equation}
where $L$ represents the low-rank component, and $S$ represents the sparse component.
Specifically, \( L \) is the low-rank matrix obtained by retaining the top \( r \) singular values and their corresponding singular vectors from \( W \):
\begin{equation}
L = U_r \Sigma_r V_r^T
\end{equation}
and $S$ is the sparse matrix obtained by performing MP on the residual of $W$ and $L$:
\begin{equation}
S = M \odot (W - L)
\end{equation}
To simplify the description, we refer to this entire process as $\text{LoRS}(\cdot)$. In this manner, SVP decouples the structure and details of the weight, preserving a compact structure while allowing fine-grained MP to remove redundant parts in the details.

For each model finetuned on single specific language or task data, we apply $\text{LoRS}(\cdot)$ to its task vector as a preprocessing step to reduce language or task interference in model merging.
A multi-lingual or multi-task model can be achieved through simple merging, expressed as: 
\begin{equation}
\theta_\text{merged} = \theta_0 + \lambda \sum_{i=1}^{n} \text{LoRS}(\tau_i)
\end{equation}
Compared to multi-lingual or multi-task training methods, model merging is a simpler and more efficient approach, enabling the seamless incorporation of new languages or tasks without the need for retraining. Additionally, due to its training independence, it mitigates negative transfer and provides optimal training configurations for each language or task to improve performance.


\section{Experimental Setup}
\subsection{Data}
\textbf{CoVoST-2} \cite{wang2020covost} is a large-scale multi-lingual ST corpus based on Common Voice.
It covers translations from English into 15 languages and from 21 languages into English, with a total of 2,880 hours of speech from 78k speakers.
We selected 5 high-resource languages and 5 low-resource languages as two language sets to investigate their ASR tasks and the from X to English ST tasks.
The high-resource language set includes Catalan (ca), German (de), Spanish (es), French (fr), and Italian (it), while the low-resource language set includes Indonesian (id), Dutch (nl), Portuguese (pt), Russian (ru), and Swedish (sv).
Due to the more abundant data in the high-resource language set, our main experimental results are obtained on the high-resource language set, while the low-resource language set serves as an auxiliary validation set.
To balance the amount of data across different languages, we fixed the duration of traning data for each language, with 5 hours for the high-resource language set and 1 hour for the low-resource language set.
The dev and test sets of both language sets are 1 hour in duration.


\subsection{Model and Training Specifications}
{\bf Whisper} \cite{radford2023robust} is a general-purpose multi-lingual ASR and ST model, a Transformer-based model trained on 680k hours of diverse audio.
We chose the small version as the foundation model for the experiments because it achieves a good balance between performance and cost.
It has 244 million parameters, with the encoder and decoder each consisting of 12 Transformer blocks.
The weight matrices of the attention layers are all 768 by 768, and the MLP layers are 768 by 3072.

For each language-specific or task-specific finetuned model, we use a different, optimal learning rate for each during training, and these models are subsequently used for model merging.
Finetuning involves updating all parameters.
We choose Adam as the optimiser, set the batch size to 8, the accumulation iterations to 4, and train for 10 epochs.
We also select the proportions of low-rank parameters retained by SVP from \{1\%, 2\%, 3\%, 5\%\} and sparse parameters retained by MP from \{10\%, 20\%, 40\%, 60\%\}.
The beam size for decoding is set to 20 across all languages and tasks.
We use Sclite and SacreBLEU tools to score the ASR and ST results, respectively.
See Appendix \ref{sec:appendix_hyper} for more details on hyper-parameter settings. Our experiments are performed on a single RTX 4090 GPU where training on one language and one task with 5 hours of speech data requires 1 hour.

\subsection{Baseline and Merging Methods}
% To validate the effectiveness of our proposed LoRS-Merging method, we compared it with the following commonly used methods.
We use \textbf{Multi-lingual and multi-task training} as the baseline for comparison with model merging methods, where training is conducted on data mixed from both multi-lingual and multi-task sets. To ensure a fair comparison, the same amount of training data is used from each language and each task. Note that for 5 different languages with both ASR and ST tasks, multi-lingual and multi-task training is performed on 10 times more data and hence 10 times more computational resources. 

In addition to LoRS-Merging, we investigate the following model merging methods:

\textbf{Weight Averaging} (WA) merges multiple single models by and unweighted averaging of their weights, $\theta_\text{merged} = \frac{1}{n} \sum_{i=1}^{n} \theta_i$.

\textbf{Task Arithmetic} (TA) uses a scaling factor to weight multiple task vectors estimated on a small development set, $\theta_\text{merged} = \theta_0 + \lambda \sum_{i=1}^{n} \tau_i$.

\textbf{MP-Merging} performs fine-grained magnitude pruning on task vectors to reduce redundancy at the detail level, $\theta_\text{merged} = \theta_0 + \lambda \sum_{i=1}^{n} \text{MP}(\tau_i)$.

\textbf{SVP-Merging} performs coarse-grained singular value pruning on task vectors to reduce redundancy at the structure level, $\theta_\text{merged} = \theta_0 + \lambda \sum_{i=1}^{n} \text{SVP}(\tau_i)$. (see Section \ref{sec:pruning}).

Moreover, we compare methods against the performance of fine-tuning on each language-task combination. This is the topline of all merging methods since the model is completely adapted to a specific language for a specific task with optimised configurations and without any language conflicts.

\section{Evaluation Results and Analysis}

\subsection{Multi-Lingual Model Merging}

\begin{table}[t]
  \caption{Multi-lingual ASR model merging. Finetuned is the topline where the model is finetuned on each language independently, and Avg. averages WER directly.}
  %\vspace*{-3mm} 
  \label{tab:multi-lingual-ASR}
  \centering
  % \setlength{\tabcolsep}{1.5pt}
  \begin{adjustbox}{width=\columnwidth}
  \begin{tabular}{l|ccccc|c}
    \toprule
    \multicolumn{1}{l}{\multirow{2}{*}{\textbf{System}}} & \multicolumn{6}{c}{\textbf{WER$\downarrow$}} \\
    % \cmidrule{2-7}
     & ca & de & es & fr & it & Avg. \\
    \midrule
    Pretrained & 20.6 & 19.6 & 14.7 & 24.5 & 19.4 & 19.88 \\
    Finetuned & 19.5 & 19.7 & 14.4 & 22.1 & 19.2 & 19.05 \\
    \midrule
    Multi-lingual training & 17.1 & 21.8 & 15.1 & 22.6 & 21.9 & 19.69 \\
    \midrule
    Weight Averaging & 19.1 & 19.1 & 14.2 & 24.5 & 20.3 & 19.55 \\
    Task Arithmetic & 19.1 & 18.8 & 13.9 & 24.0 & 19.8 & 19.23 \\
    MP-Merging & 19.4 & 19.3 & 14.0 & 23.8 & 18.1 & 19.03 \\
    SVP-Merging & 19.5 & 19.3 & 14.2 & 23.6 & 18.4 & 19.11 \\
    LoRS-Merging & 19.0 & 18.8 & 13.9 & 23.5 & 18.5 & \textbf{18.85} \\
    \bottomrule
  \end{tabular}
  \end{adjustbox}
\end{table}


\begin{table}[t]
  \caption{Multi-lingual ST model merging. Finetuned is the topline where the model is finetuned on each language independently, and Avg. averages BLEU directly.}
  %\vspace*{-3mm} 
  \label{tab:multi-lingual-ST}
  \centering
  % \setlength{\tabcolsep}{1.5pt}
  \begin{adjustbox}{width=\columnwidth}
  \begin{tabular}{l|ccccc|c}
    \toprule
    \multicolumn{1}{l}{\multirow{2}{*}{\textbf{System}}} & \multicolumn{6}{c}{\textbf{BLEU$\uparrow$}} \\
     & ca & de & es & fr & it & Avg. \\
    \midrule
    Pretrained & 21.1 & 24.1 & 28.6 & 26.8 & 26.8 & 25.48 \\
    Finetuned & 22.6 & 24.6 & 29.2 & 27.2 & 27.3 & 26.18 \\
    \midrule
    Multi-lingual training & 21.4 & 24.4 & 28.8 & 26.8 & 27.2 & 25.72 \\
    \midrule
    Weight Averaging & 22.3 & 24.1 & 28.6 & 27.2 & 26.9 & 25.82 \\
    Task Arithmetic & 22.1 & 24.3 & 28.9 & 27.3 & 26.8 & 25.88 \\
    MP-Merging & 22.1 & 24.7 & 28.9 & 27.3 & 26.9 & 25.98 \\
    SVP-Merging & 22.1 & 24.7 & 29.0 & 27.4 & 26.8 & 26.00 \\
    LoRS-Merging & 22.2 & 24.8 & 29.0 & 27.5 & 26.9 & \textbf{26.08} \\
    \bottomrule
  \end{tabular}
  \end{adjustbox}
  \vspace{-0.3cm}
\end{table}

First, we investigate the merging of finetuned models for different languages on the same task, which corresponds to \textit{multi-lingual single-task} learning.

\textbf{Language knowledge interference yields imbalanced improvements}: Table \ref{tab:multi-lingual-ASR} shows the multi-lingual results of the ASR task with the high-resource language set.
On average, multi-lingual training slightly improves the pretrained model but significantly underperforms the finetuned models and merging methods.
This may be due to negative interference between the knowledge of different languages, leading to gradient conflicts during training \cite{wang2020negative}. From a per-language perspective, it is observed that ca and fr achieve the largest improvements during fine-tuning while still showing significant improvements in multi-lingual training, whereas languages with smaller improvements during finetuning exhibit a substantial performance drop in multi-lingual training, even worse than the pretrained model. This indicates a strong language conflict in multi-lingual training, with ca and fr dominating. Additionally, we observe that the optimal learning rates for finetuned models vary significantly across languages (see Appendix \ref{sec:appendix_hyper}), while the unified learning rate configuration required by multi-lingual training prevents each language from reaching its optimal performance. 

\textbf{Model merging mitigates language conflicts}: 
In contrast, model merging methods show obvious improvements across almost all languages, demonstrating reduced conflict and better stability. Among model merging methods, TA outperforms WA due to its flexible scaling factor.
Both MP-Merging and SVP-Merging further improve the performance of TA by reducing redundancy, and MP-Merging slightly outperforms SVP-Merging due to its finer-grained pruning.
Combining the advantages of SVP and MP, LoRS-Merging achieves the best performance.
% From a per-language perspective, multi-lingual training performs well in ca and fr, but poor in other languages.




Table \ref{tab:multi-lingual-ST} provides the multi-lingual results on ST task with the high-resource language set.
The main conclusion is consistent with the ASR task: model merging methods still significantly outperform multi-lingual training, with LoRS-Merging achieving the best performance.


\subsection{Multi-Task Model Merging}
\begin{table}[t]
    \centering
    \small
    \caption{Multi-task model merging performed on each language independently and WER/BLEU scores are averaged across languages. Finetuned is the topline where the model is finetuned on each language and task combination independently. Per-language results are shown in Appendix \ref{sec:appendix_detail}.}
    \begin{tabular}{lcc}
    \toprule
    \textbf{System}     & \textbf{Avg. WER}$\downarrow$ & \textbf{Avg. BLEU}$\uparrow$ \\
    \midrule
        Pretrained & 19.88 & 25.48 \\
    Finetuned & 19.05 & 26.18 \\
    \midrule
    Multi-task training &  19.00 & 25.90 \\
    \midrule
    Weight Averaging & 18.84 &  26.18 \\
    Task Arithmetic & 18.76 & 26.30 \\
    MP-Merging & 18.62 & 26.40 \\
    SVP-Merging & 18.72 & 26.38 \\
    LoRS-Merging & \textbf{18.45} & \textbf{26.48} \\
    \bottomrule
    \end{tabular}
    \label{tab:multi-task}
    \vspace{-0.3cm}
\end{table}

\begin{table*}[t]
  \caption{Multi-lingual multi-task model merging. Finetuned is the topline where the model is finetuned on each language and task combination independently, and Avg. averages WER or BLEU scores directly.}
  %\vspace*{-3mm} 
  \label{tab:multi-lingual multi-task}
  \centering
  % \setlength{\tabcolsep}{1.5pt}
  \begin{adjustbox}{width=0.95\textwidth}
  \begin{tabular}{l|ccccc|c|ccccc|c}
    \toprule
    \multicolumn{1}{l}{\multirow{2}{*}{\textbf{System}}} & \multicolumn{6}{c}{\textbf{WER$\downarrow$}} & \multicolumn{6}{c}{\textbf{BLEU$\uparrow$}} \\
     & ca & de & es & fr & it & Avg. & ca & de & es & fr & it & Avg. \\
    \midrule
    Pretrained & 20.6 & 19.6 & 14.7 & 24.5 & 19.4 & 19.88 & 21.1 & 24.1 & 28.6 & 26.8 & 26.8 & 25.48 \\
    Finetuned & 19.5 & 19.7 & 14.4 & 22.1 & 19.2 & 19.05 & 22.6 & 24.6 & 29.2 & 27.2 & 27.3 & 26.18 \\
    \midrule
    ML and MT training & 20.5 & 19.7 & 14.6 & 24.5 & 19.4 & 19.86 & 21.3 & 24.3 & 28.3 & 27.1 & 26.9 & 25.58 \\
    \midrule
    ML and MT Task Arithmetic & 18.9 & 19.2 & 14.1 & 23.7 & 18.4 & 18.96 & 22.2 & 24.4 & 29.0 & 27.3 & 26.9 & 25.96 \\
    ML and MT LoRS-Merging & 18.7 & 19.1 & 14.0 & 23.8 & 18.0 & 18.82 & 22.2 & 24.8 & 29.0 & 27.5 & 27.0 & 26.10 \\
    \midrule
    MT training & 17.0 & 19.7 & 14.4 & 24.2 & 19.4 & 19.00 & 22.3 & 24.6 & 28.7 & 27.0 & 26.9 & 25.90 \\
    $\hookrightarrow$ + ML Task Arithmetic & 18.1 & 19.0 & 14.2 & 24.5 & 20.6 & 19.37 & 22.7 & 24.7 & 28.6 & 27.3 & 26.5 & 25.96 \\
    $\hookrightarrow$ + ML LoRS-Merging & 18.1 & 19.0 & 14.1 & 24.2 & 20.3 & 19.23 & 22.4 & 24.5 & 29.1 & 27.6 & 26.7 & 26.06 \\
    \midrule
    ML training & 17.1 & 21.8 & 15.1 & 22.6 & 21.9 & 19.69 & 21.4 & 24.4 & 28.8 & 26.8 & 27.2 & 25.72 \\
    $\hookrightarrow$ + MT Task Arithmetic & 17.1 & 18.5 & 13.3 & 22.7 & 18.0 & 18.00 & 22.6 & 25.0 & 29.2 & 27.5 & 26.9 & 26.24 \\
    $\hookrightarrow$ + MT LoRS-Merging & 16.9 & 18.3 & 13.3 & 22.4 & 17.8 & \textbf{17.82} & 22.8 & 25.2 & 29.3 & 27.6 & 27.0 & \textbf{26.38} \\
    \bottomrule
  \end{tabular}
  \end{adjustbox}
\end{table*}

Next, we merge finetuned models for different tasks (ASR and ST) with the same language which corresponds to \textit{multi-task single-language} learning.

\textbf{ASR and ST tasks for the same language can mutually benefit from each other}: Table \ref{tab:multi-task} presents the multi-task results with the high-resource language set.
In general, multi-task training performs similarly to finetuned models on ASR but is a lot worse on ST. This is likely due to the substantial differences in optimal hyper-parameter configurations between the two tasks.
Model merging methods clearly outperform finetuned models, which not only demonstrates their effectiveness but also shows the mutual benefits between ASR and ST.
In terms of performance gains, the improvement in ASR is greater than in ST. We attribute this to the fact that ASR is inherently simpler than ST and can be viewed as a step in the ST task.
Furthermore, as before, model merging methods combined with pruning further improve performance, and the proposed LoRS-Merging achieves the best performance across the table.


\subsection{Multi-Lingual Multi-Task Model Merging}


Then, we investigate the merging of finetuned models for both different languages and tasks, which correspond to \textit{multi-lingual (ML) and multi-task (MT)} learning.
Specifically, we explore 4 different training and merging settings: 

\textbf{ML and MT training}: Fine-tuning on all languages and both tasks jointly.

\textbf{ML and MT merging}: Fine-tuning on each language for each task separately and merging all.

\textbf{MT training and ML merging}: Fine-tuning both tasks jointly for each language, and merging models from different languages.

\textbf{ML training and MT merging}: Fine-tuning on all languages jointly for each task, and merging models from different tasks.

Table \ref{tab:multi-lingual multi-task} displays the multi-lingual and multi-task results with the high-resource language set.
Multi-lingual and multi-task training shows little improvement over the pretrained model, due to negative interference during training and the use of a unified training configuration for all languages and tasks.
Nevertheless, the performance of multi-lingual and multi-task merging is on par with that of finetuned models, further underscoring the superiority of model merging.
% Multi-task training followed by multi-lingual merging yields slightly lower performance than finetuned models.
% Surprisingly, 
% Moreover, the previously underperforming multi-lingual training shows a significant improvement in performance after multi-task merging.
% Compared to finetuned models, it greatly outperforms on the ASR task and slightly surpasses on the ST task. 
% Unfortunately, we did not observe the same phenomenon on the low-resource language set (see Table \ref{tab:multi-lingual multi-task-2}), but this suggests a potential for achieving better results through the combination of training and merging.
% As a result, LoRS-Merging achieved the best performance when performing ML training followed by MT merging, which consistently outperforms Task Arithmetic. 
% Overall, \textbf{10}\% relative WER reduction and \textbf{4}\% relative BLEU increase are achieved using LoRS-Merging compared to ML and MT training baseline. 
ML training followed by MT merging achieves the best performance, even significantly outperforming finetuned models.
Although we did not observe the same phenomenon on the low-resource language set, this suggests the potential of using a combination of training and merging to achieve better performance.
We provide additional experiments on the low-resource language set in Appendix \ref{sec:appendix_low} to demonstrate the robustness and generalisability of model merging and LoRS-Merging.


\subsection{Effect of Numbers of Languages}


% In this section, we explore the effect of the number of languages on model merging for both ASR and ST tasks.
To further demonstrate the robustness of LoRS-Merging to language selection, experiments are performed using different numbers of languages. Figure \ref{fig:language-number} shows the average performance across all languages and all training runs with possible combinations of 2, 3, 4 or 5 languages.

\begin{figure}[t]
    \centering
    \includegraphics[width=1.0\linewidth]{images/num_languages.pdf}
    \caption{WER and BLEU against the number of languages. Performance is averaged across all languages and all training runs of language combinations.}
    \label{fig:language-number}
    % \vspace{-0.3cm}
\end{figure}

\textbf{LoRS-Merging improvements are consistent across different numbers of languages}: As the number of languages increases, the performance of both TA and LoRS-Merging degrades due to negative interference between languages.
LoRS-Merging consistently outperforms TA in both ASR and ST tasks, and even surpasses the finetuned models in the ASR task.
This is likely due to LoRS-Merging further reducing model redundancy, therefore alleviating negative interference.
Additionally, we observe that the optimal learning rate for the finetuned ASR model is significantly larger compared to the ST task.
This may lead to overfitting, whereas LoRS-Merging improves model generalisation through model merging while reducing language interference, thus outperforming the finetuned models for the ASR task.


\subsection{Effect of Language Data Scale}

\begin{figure}[t]
  \includegraphics[width=0.95\linewidth]{images/datascale.pdf}
  \vspace{-0.2cm}
  \caption{Performance variation against different training data sizes (number of hours for each language) on ASR (top) and ST (bottom) tasks.}
  \label{fig:data_scale}
\end{figure}

We then demonstrate the robustness of merging methods to different training data sizes for both tasks. 
% We examine the effect of language data scale on model merging for both ASR and ST tasks.
Fig. \ref{fig:data_scale} shows the WER (top) and BLEU (bottom) scores for ASR and ST at different data scales, respectively.
As the data scale increases, the performance of multi-lingual training does not always improve.
This may be because the pretrained model already performs well, and the significant language interference and conflict in multi-lingual training hinder the effective improvement of multi-language performance.
Furthermore, the performance loss of model merging increases with data scale, compared to finetuned models.
It can be explained by the fact that larger training data tends to increase the divergence in the optimisation trajectories of different finetuned models, resulting in the breakdown of linear mode connectivity, which leads to a greater performance loss.
Moreover, LoRS-Merging still achieves obvious and stable improvement compared to TA.


% \subsection{Effect of Language Set}

\subsection{Analysis of Model Redundancy}
\begin{figure}[t]
  \includegraphics[width=1.0\linewidth]{images/svp_mp.pdf}
  \caption{Model performance against the retain ratio (1 means to retain all weights and 0 means to prune all weights) in SVP (left) and MP (right) for ASR finetuned models. Three different training data sizes are used.}
  \label{fig:svp_mp}
\end{figure}

Furthermore, we justify the necessity of SVP and MP to remove model redundancy by showing the model performance against the pruning ratio of finetuned models for ASR as shown in Fig. \ref{fig:svp_mp}.
As shown, both SVP and MP significantly improve the performance of finetuned models, indicating the presence of substantial redundancy in the structure and details of the finetuned models, respectively. The model performance reaches the best at a high pruning level, indicating that the redundancy is particularly large for ASR. We observed a much smaller redundancy in ST, which also explains the observation that LoRS-Merging achieves more salient improvement on ASR than ST.
Moreover, redundancy increases with training data, possibly due to the accumulation of gradient noise during training.
MP achieves greater performance gains than SVP, indicating more redundancy at the detail level, which is better addressed by fine-grained MP.


% \subsection{Visualisation Analysis}

\section{Conclusion}
This paper explores model merging for multi-lingual ASR and ST on pre-trained speech models and proposes the LoRS-Merging approach. LoRS-Merging combines low-rank and sparse pruning to retain essential structures, eliminate redundant parameters and mitigate language and task interference. Experiments across 10 languages show that LoRS-Merging effectively alleviates language interference and significantly outperforms multi-lingual multi-task training baselines.

% Experiments across five languages show that LoRS-Merging effectively alleviates language interference, and achieves a 10\% relative WER reduction and a 4\% relative BLEU score improvement compared to multi-lingual multi-task training baselines.

%This paper explores model merging for multi-lingual ASR and ST on pre-trained speech models and proposes the low-rank and sparse merging (LoRS-Merging) approach. LoRS-Merging combines low-rank and sparse pruning to retain essential structures, eliminate redundant parameters and mitigate language and task interference. Experiments across 5 languages show that LoRS-Merging effectively alleviates language interference, and achieves a 10\% relative WER reduction and a 4\% relative BLEU score improvement compared to multi-lingual multi-task training baselines.


\section{Limitations}

There are three main limitations of this work. First, as a common limitation of all model merging methods, the same model structure is required across all tasks and languages. This is less of a concern under the current trend of using the same Transformer structure, but methods need to be developed in the future to accommodate subtle structural differences. Second, reasonably-sized training sets are required for each language, and low-resource languages may suffer from reduced improvements. Third, this work mainly explores the two most popular S2T tasks. Other possible tasks can be explored in future work, including spoken language understanding and speaker adaptation. 

% \section*{Acknowledgments}


% Bibliography entries for the entire Anthology, followed by custom entries
%\bibliography{anthology,custom}
% Custom bibliography entries only
% % This must be in the first 5 lines to tell arXiv to use pdfLaTeX, which is strongly recommended.
\pdfoutput=1
% In particular, the hyperref package requires pdfLaTeX in order to break URLs across lines.

\documentclass[11pt]{article}

% Change "review" to "final" to generate the final (sometimes called camera-ready) version.
% Change to "preprint" to generate a non-anonymous version with page numbers.
\usepackage[final]{acl}

% Standard package includes
\usepackage{times}
\usepackage{latexsym}

% For proper rendering and hyphenation of words containing Latin characters (including in bib files)
\usepackage[T1]{fontenc}
% For Vietnamese characters
% \usepackage[T5]{fontenc}
% See https://www.latex-project.org/help/documentation/encguide.pdf for other character sets

% This assumes your files are encoded as UTF8
\usepackage[utf8]{inputenc}

% This is not strictly necessary, and may be commented out,
% but it will improve the layout of the manuscript,
% and will typically save some space.
\usepackage{microtype}

% This is also not strictly necessary, and may be commented out.
% However, it will improve the aesthetics of text in
% the typewriter font.
\usepackage{inconsolata}

%Including images in your LaTeX document requires adding
%additional package(s)
\usepackage{graphicx}
\usepackage{xcolor}

% If the title and author information does not fit in the area allocated, uncomment the following
%
%\setlength\titlebox{<dim>}
%
% and set <dim> to something 5cm or larger.

\usepackage{booktabs}
\usepackage{hyperref}
\usepackage{multirow}
\usepackage{multicol}
\usepackage[most]{tcolorbox}
\usepackage{adjustbox}
\usepackage{graphicx}
\usepackage{fullpage}
\usepackage{times}
\usepackage{fancyhdr,graphicx,amsmath,amssymb}
%\usepackage[ruled,vlined]{algorithm2e}
\usepackage{algorithm}
\usepackage{algpseudocode}
\usepackage{booktabs}
\usepackage{adjustbox}
\usepackage{url}
\usepackage{hyperref}
\usepackage{amssymb}
\usepackage{marvosym}
\usepackage{multirow}
\usepackage{subcaption}
\DeclareMathOperator*{\argmax}{arg\,max}
\DeclareMathOperator*{\argmin}{arg\,min}


\newtcolorbox{promptbox}[2][]{
  colback=gray!10,
  colframe=gray!50,
  arc=3mm,
  boxrule=1pt,
  left=10pt,
  right=10pt,
  top=8pt,
  bottom=8pt,
  before skip=12pt,
  after skip=12pt,
  fonttitle=\bfseries,
  title=#2,
  #1
}

\title{Quality-Aware Decoding: Unifying Quality Estimation and Decoding}

% Author information can be set in various styles:
% For several authors from the same institution:
% \author{Author 1 \and ... \and Author n \\
%         Address line \\ ... \\ Address line}
% if the names do not fit well on one line use
%         Author 1 \\ {\bf Author 2} \\ ... \\ {\bf Author n} \\
% For authors from different institutions:
% \author{Author 1 \\ Address line \\  ... \\ Address line
%         \And  ... \And
%         Author n \\ Address line \\ ... \\ Address line}
% To start a separate ``row'' of authors use \AND, as in
% \author{Author 1 \\ Address line \\  ... \\ Address line
%         \AND
%         Author 2 \\ Address line \\ ... \\ Address line \And
%         Author 3 \\ Address line \\ ... \\ Address line}

\author{Sai Koneru$^{1}$,
  Matthias Huck$^{2}$,
  Miriam Exel$^{2}$, \textnormal{and}
  Jan Niehues$^{1}$ \\
  $^{1}$ Karlsruhe Institute of Technology \\
  $^{2}$ SAP SE, Dietmar-Hopp-Allee 16, 69190 Walldorf, Germany \\
  \texttt{\{sai.koneru, jan.niehues\}@kit.edu} \\
  \texttt{\{matthias.huck, miriam.exel\}@sap.com}}

%\author{
%  \textbf{First Author\textsuperscript{1}},
%  \textbf{Second Author\textsuperscript{1,2}},
%  \textbf{Third T. Author\textsuperscript{1}},
%  \textbf{Fourth Author\textsuperscript{1}},
%\\
%  \textbf{Fifth Author\textsuperscript{1,2}},
%  \textbf{Sixth Author\textsuperscript{1}},
%  \textbf{Seventh Author\textsuperscript{1}},
%  \textbf{Eighth Author \textsuperscript{1,2,3,4}},
%\\
%  \textbf{Ninth Author\textsuperscript{1}},
%  \textbf{Tenth Author\textsuperscript{1}},
%  \textbf{Eleventh E. Author\textsuperscript{1,2,3,4,5}},
%  \textbf{Twelfth Author\textsuperscript{1}},
%\\
%  \textbf{Thirteenth Author\textsuperscript{3}},
%  \textbf{Fourteenth F. Author\textsuperscript{2,4}},
%  \textbf{Fifteenth Author\textsuperscript{1}},
%  \textbf{Sixteenth Author\textsuperscript{1}},
%\\
%  \textbf{Seventeenth S. Author\textsuperscript{4,5}},
%  \textbf{Eighteenth Author\textsuperscript{3,4}},
%  \textbf{Nineteenth N. Author\textsuperscript{2,5}},
%  \textbf{Twentieth Author\textsuperscript{1}}
%\\
%\\
%  \textsuperscript{1}Affiliation 1,
%  \textsuperscript{2}Affiliation 2,
%  \textsuperscript{3}Affiliation 3,
%  \textsuperscript{4}Affiliation 4,
%  \textsuperscript{5}Affiliation 5
%\\
%  \small{
%    \textbf{Correspondence:} \href{mailto:email@domain}{email@domain}
%  }
%}

\begin{document}
\maketitle
\begin{abstract}
% Neural Machine Translation (NMT) has achieved high-quality translations in many scenarios, pushing the boundaries of tasks such as instruction-following and multimodal translation. 
Quality Estimation (QE) models for Neural Machine Translation (NMT) predict the quality of the hypothesis without having access to the reference.
An emerging research direction in NMT involves the use of QE models, which have demonstrated high correlations with human judgment and can enhance translations through Quality-Aware Decoding. Although several approaches have been proposed based on sampling multiple candidate translations and picking the best candidate, none have integrated these models directly into the decoding process. In this paper, we address this by proposing a novel token-level QE model capable of reliably scoring partial translations. We build a uni-directional QE model for this, as decoder models are inherently trained and efficient on partial sequences. We then present a decoding strategy that integrates the QE model for Quality-Aware decoding and demonstrate that the translation quality improves when compared to the N-best list re-ranking with state-of-the-art QE models (up to $1.39$ XCOMET-XXL $\uparrow$). Finally, we show that our approach provides significant benefits in document translation tasks, where the quality of N-best lists is typically suboptimal\footnote{Code can be found at \url{https://github.com/SAP-samples/quality-aware-decoding-translation}}
\end{abstract}
\section{Introduction}

Large language models (LLMs) have significantly impacted various Natural Language Processing (NLP) tasks \citep{brown2020language, jiang2023mistral, dubey2024llama}, including Neural Machine Translation (NMT). The field of NMT is transitioning from using dedicated encoder-decoder transformers \citep{vaswani2017attention, nllb2024scaling} to leveraging decoder-only LLM-based translation models \citep{kocmi2024findings}. This shift is driven by LLMs' ability to retain knowledge, handle large contexts, and follow instructions, learned during extensive pre-training \citep{xu2024contrastive, alves2024tower}. As a result, LLM-based MT models have achieved state-of-the-art translation quality \citep{kocmi2024findings}.

In parallel, Quality Estimation (QE) has become a well-researched subfield within NMT. QE models are trained to predict the quality of a translation without requiring access to the reference \citep{rei2021references,rei2022cometkiwi}. Interestingly, QE models can achieve performance in assessing translation quality that is comparable to MT evaluation models, which do have access to the reference \citep{zerva2024findings}.

This led to the question: "\textit{Can we integrate QE into the current translation process to improve quality?}" Incorporating QE into NMT offers several benefits. First, having a expert QE model guiding the decoding can further improve the quality. Second, by adapting the QE model with feedback from human annotators, we can generate future translations guided with the newly obtained feedback.

\begin{figure*}[!ht]
\includegraphics[width=\textwidth]{Figures/nbestlist_problem.png}
 \caption{Example from WMT'23 English → German \#ID: 10: The paragraph begins with 'Department of Homeland Security,' which should be translated as 'Ministerium für \textbf{I}nnere Sicherheit.' However, the top 25 beams do not contain the correct translation and begin with an error, making N-best list re-ranking insufficient. Although the top-5 tokens at the decoding contain the correct forms 'Inn' or 'Inner,' the probabilities split among them giving highest mass to the incorrect token 'inn.' Quality-Aware decoding can prevent errors with earlier integration.}
\label{fig:nbestlist}
\end{figure*}


Several approaches have been explored to integrate QE into the translation process. These include re-ranking the N-best list \citep{fernandes2022quality}, applying minimum Bayes risk (MBR) decoding on a quality-filtered N-best list \citep{tomani2024quality}, and training additional models for post-editing based on QE-predicted errors \citep{treviso2024xtower}. However, all these methods operate on fully generated sequences before the QE model can exert influence. Integrating QE earlier in the decoding process, referred in this paper as \textit{Quality-Aware Decoding}, could enhance translation quality and reduce reliance on the N-best list. This is especially relevant when dealing with long inputs as good translations during decoding are likely to be pruned and may need sampling larger number of finished hypothesis. We illustrate this in Figure \ref{fig:nbestlist}.

To achieve this, a QE model capable of predicting the quality of partial translations is required. However, current leading QE models face challenges in this area, as they are typically not trained to predict scores for incomplete hypotheses. \textit{Therefore, developing QE models that can handle partial translations is essential for implementing Quality-Aware Decoding during the translation process}.

In this work, we propose adapting LLM-based MT models to perform QE on partial translations and incorporating this model into the decoding. We create a token-level synthetic QE dataset using WMT Multidimensional Quality Metrics (MQM) data \citep{burchardt2013multidimensional, freitag2024llms}. We then adapt a uni-directional LLM-based MT model to predict whether a token is \textit{Good} or \textit{Bad}. Training QE models on these token-level tasks alleviates the data challenge and allows us to exploit the MQM data while simultaneously making the task easier for the model compared to predicting a score directly.

\begin{figure*}[!ht]
\includegraphics[width=\textwidth]{Figures/annotation_scheme.png}
 \caption{Token-level label annotation scheme using the MQM error tags. \textit{MASK} indicates that this token label will not be used in training to prevent incorrect learning signal.}
\label{fig:annotation}
\end{figure*}

Furthermore, integrating the QE model into NMT during decoding is not trivial, as we need to combine the QE estimates during decoding. Therefore, we use the decoding strategy from \citet{koneru2024plug}, and modify it to incorporate token-level predictions efficiently with the adapted QE model to provide real-time feedback during the decoding process. We summarize our main findings and contributions below.

\begin{itemize}
    \item We present a novel uni-directional QE model which estimates quality on incomplete hypotheses by averaging the probabilities of each token being classified as \textit{Good}. 
    
    %We demonstrate that it achieves improved correlation with human annotations on WMT 23 English $\rightarrow$ German, compared to the log probabilities of the same LLM-based NMT model.

    \item We propose a decoding strategy that combines the token-level QE model on partial hypothesis and the NMT model to perform Quality-Aware Decoding. 
    
    \item We show through experiments that early integration is essential and the translation quality is improved even when compared to re-ranking the N-best list with state-of-the-art QE models.

    \item We highlight the significance of our approach in document translation scenarios, where post-generation QE techniques fall short due to their reliance on the quality of the N-best list, a challenge that becomes more difficult as the input length increases.
\end{itemize}



\section{Quality-Aware Decoding}

The primary objective of this paper is to achieve Quality-Aware Decoding in MT. To accomplish this, it is essential to predict the quality of partial translations and integrate this information during the decoding process. Our approach proposes using one NMT model for generating translations and another adapted NMT model to predict the quality of the candidate translations produced by the first model.

First, we explain why relying solely on the NMT model to predict the quality of a hypothesis is insufficient and why an additional model is necessary. Next, we outline the adaptation of the NMT model for QE on partial translations, detailing the creation of a token-level QE dataset, the modifications made to the NMT model for this task, and the process of estimating the sentence-level quality score. Finally, we describe the algorithm used to incorporate the QE score into the decoding process.

\subsection{Decomposing Decoding: Translation + QE}
NMT models generate a token-by-token sequence and provide the probability of each token at the decoding step. The average of the log-probabilities is often used as a proxy to score the candidate during search. 

While NMT models are capable of generating high-quality translations, using the average log-probabilities of hypotheses as a scoring metric tends to yield poor correlation with actual translation quality \citep{eikema2020map, freitag2020bleu}. In many cases, a translation can continue in several different ways, all of which may be acceptable. If the starting tokens for these continuations differ, the probability mass may be spread across multiple options which is used during the search. However, from a quality perspective, all these continuations could still achieve a high score, as the QE scores are independent and need not sum to $1$.

Therefore, we propose a expert model that focuses on quality to estimate the scores better during decoding and  improve the search space leading to a better hypothesis.


% Therefore, relying solely on the average log-probabilities during decoding is not ideal, as it computes the score independently for each token and does not fully correlate with the overall quality of the current hypothesis.

\subsection{Quality Estimation on Partial Sequences}

% NMT models decode sequences token-by-token. 
To provide a quality score during decoding, the QE model must be capable of handling incomplete sequences. It should not penalize a sequence if there is a potential extension that could lead to a perfect translation.

Estimating the score in this way is not feasible with current QE models, such as COMET \citep{rei2021references}, as they were not trained for this specific task and cannot provide reliable scores in the context of partial translations. Hence, we need to develop a partial QE system.

When building a partial QE system, several factors need to be considered. First, should the model use a uni-directional or bi-directional architecture? A \textbf{uni-directional} model is more efficient, as it allows for caching the hidden states, which can then be used for subsequent steps without re-encoding, unlike a bi-directional model.

Next, we need to decide whether to predict the QE score at the sequence level or at the token level. For \textbf{token-level QE}, we can directly use data from MQM annotations, as we already know which tokens are \textit{Good} or \textit{Bad}. However, for segment-level scoring, we need to consider how to synthetically create the training data. 

% Additionally, COMET models are encoder-only architectures pre-trained on full sentences, rather than partial sentences as required in this case. Moreover, predicting the score of partial translations naturally favors decoder-only models due to their efficiency. New tokens only need to process the preceding sequence, avoiding the need to re-encode the entire sequence. Additionally, this approach simplifies training, as we do not require synthetically shorter samples. 

%  Furthermore, there is no readily available dataset containing partial translations along with their quality scores. Hence, we need to design the adaptation process with a QE model that is uni-directional and exploit already available human annotated data.

% \subsubsection{Token-level Quality Estimation}

Therefore, we decide adapt the uni-directional model into a token-level QE system that predicts whether each token is \textit{Good} or \textit{Bad} (a binary decision) by adding an additional classifier head. This adaptation enables us to estimate the score for a sequence by calculating the average probability that each token is classified as \textit{Good}. We hypothesize that adapting the model in this way, rather than directly predicting the score, provides greater stability, as the last hidden states inherently contain token-level information and do not require mapping the entire sequence to a single score.

For training this model, we leverage the WMT MQM data containing error annotations in MT outputs. We can treat tokens before an error as \textit{Good} and those containing inside an error as \textit{Bad}. Then, we can train in uni-directional manner where each token's label is predicted using only the preceding context in the hypothesis. This is crucial as we only have the preceding context to estimate the quality for partial hypothesis.

\subsubsection{Learning the Right Signal}

\begin{algorithm*}[!t]
\caption{Computing merged score of partial hypothesis with translation and token-level QE models.}
\begin{algorithmic}[1]
\setlength{\baselineskip}{1.2em}
\Procedure{MergeScore}{}
    \State \textbf{Input:}   Hypothesis tokens $h_1, h_2, h_3, \dots, h_{n}$, Translation Model $\mathcal{M}_{NMT}$, QE model $\mathcal{M}_{QE}$, Source sentence $\mathcal{S}$, Re-ranking weight $\alpha$,
    \State \textbf{Output:} $merged\_score$
    \State $Score_{NMT} \gets \frac{1}{n}\sum \log \mathcal{P}(h_1, h_2,\dots, h_{n}|\mathcal{S};\mathcal{M}_{NMT})$ 
    \State $Score_{QE} \gets \frac{1}{n}\sum \log \mathcal{P}(0_{1}, 0_{2},\dots,0_{n} | h_1, h_2,\dots, h_{n},\mathcal{S};\mathcal{M}_{QE})$ 
    \State $merged\_score \gets (\alpha) \times Score_{NMT} + (1 - \alpha) \times Score_{QE}$
\EndProcedure
\end{algorithmic}
\label{alg:joint}
\end{algorithm*}

The straightforward approach to creating labels is to assign $1$ to all tokens within the error span and $0$ otherwise. However, MQM annotations can mark errors from words to phrases, and the starting tokens of an error span may not always be wrong. This is illustrated in Figure \ref{fig:annotation}.

For example, consider the German sentence \textit{"Ich spiele Tennis"} translated by three different NMT systems, each annotated with MQM error labels. In this work, we focus on learning a binary decision: whether an error is present, ignoring error severity.

\textbf{System 1: No error}: The translation \textit{"I play Tennis"} is perfect, and all tokens are labeled as "\textit{Good}."

\textbf{System 2: Partial error}: The translation \textit{"I played Tennis"} has an error in the verb form ("played" instead of "play"). The error is in the token span \textit{"played"}, but not all tokens in this span are incorrect (e.g., "pla" is correct). Assigning a "\textit{Bad}" label to the entire span would lead to incorrect learning. A more refined approach is needed to mark errors accurately at the token level.

\textbf{System 3: Full error}: The translation \textit{"I enjoy Tennis"} contains an error in \textit{"enjoy"}, so all tokens in this span should be labeled as "\textit{Bad}."

It is not trivial to decide when the prefix of an error span is correct/incorrect. To achieve accurate labeling, we propose the following scheme:

\begin{itemize} \item Apply a \texttt{<MASK>} operation to all tokens within the error span. \item Only the last token in the span is assigned the label "\textit{Bad}", as the error is considered complete at the end of the span. \end{itemize}

If the error token is in the middle, we still train the model to predict "\textit{Bad}" in the end and let the model determine which tokens should be part of the error span during inference. This approach ensures that errors are identified without explicitly defining the error span. 

\subsubsection{Sequence-Level Quality Estimation}


After fine-tuning a token-level classification model to predict the quality of the tokens, we still need to map these predictions into a sequence-level score that can be integrated during the decoding process. There are several potential ways to achieve this.

One approach is to simply count how many tokens are classified as \textit{Bad} in the current hypothesis. However, this method has limitations. The number of errors should be normalized based on the length of the hypothesis to account for varying sizes. Additionally, converting the probabilities into a fixed number of error tokens would need to account for different error types according to the MQM format, as each error counts differently.

To avoid such strict scoring schemes, we take a simpler approach. We average the log probabilities of all tokens that are classified as \textit{Good}. This method inherently accounts for the length of the hypothesis, and it provides a score on the scale of log probabilities, which aligns with the decoding process. Therefore, we use this averaged log probability as a proxy metric for the QE score, where a higher score indicates better quality
(\textbf{Line 5} in Algorithm \ref{alg:joint}).

\subsubsection{Fusing Translation and Quality}

We can use a token-level QE system to evaluate the quality of a source and partial hypothesis during decoding. However, integrating these probabilities into all candidates is computationally expensive, as each beam considers extensions equal to the vocabulary size.

To address this, we adopt a simplified decoding strategy from \citet{koneru2024plug}, which ensembles models with different vocabularies. By adapting the same MT model for token-level QE, we simplify the merging process, as the vocabularies match. This restriction is reasonable, as it is also beneficial to leverage the knowledge learned by the specialized MT for token-level QE.

The core idea is to re-rank the top candidates at each decoding step using the QE model. After re-ranking, the translation and QE scores are merged, and the process repeats until the end-of-sentence token is generated, for each beam. This strategy allows us to efficiently incorporate the QE model’s estimate, improving translation quality.

During decoding, at each step, we have scores for $n$ beams and $V$ possible extensions from the vocabulary. In typical beam search, we select the top $n$ extensions and expand the hypothesis. To make the decoding process Quality-aware, we estimate the quality of these extensions. Since estimating all extensions is computationally expensive, we limit the candidates by selecting a specified number of top candidates.

To achieve this, we use a hyper-parameter $topk$, which selects the best $topk$ extensions for each beam. For each of these top $topk$ extensions, we compute a combined score, detailed in Algorithm \ref{alg:joint}. This combined score incorporates both the translation model score and the quality estimation score, ensuring the quality is considered during decoding.

For a top extension at decoding step $n$, let the current tokens be $h_1, h_2, h_3, \dots, h_n$. The NMT model score is computed as the average log probabilities of each token (Line 4). For the token-level QE model, we compute the average probability of each token being classified as '\textit{Good}' (Line 5). The merged score is equal to weighted linear combination of these probabilities, with weight $\alpha$ (Line 6).

Thus, to make the decoding process Quality-Aware, we first train a token-level QE system by adapting the same NMT model to ensure vocabulary matching. We then combine the scores from both models to improve the sequence estimates explored during search.


\begin{table*}[!ht]
\resizebox{2\columnwidth}{!}{
\begin{tabular}{@{}ccccc@{}}
\toprule
\multicolumn{1}{c|}{Model}            & \multicolumn{1}{c|}{Beams}                & \multicolumn{1}{c|}{Re-ranking}              & MetricX ($\downarrow$)     & XCOMET-XXL ($\uparrow$)    \\ \midrule
\multicolumn{5}{c}{\textit{English $\rightarrow$ German}}                                                                                                          \\ \midrule
\multicolumn{1}{c|}{Tower}            & \multicolumn{1}{c|}{5}                    & \multicolumn{1}{c|}{\_}                      & 2.52          & 86.93          \\
\multicolumn{1}{c|}{Tower}            & \multicolumn{1}{c|}{25}                   & \multicolumn{1}{c|}{XCOMET-XL QE}            & 2.37          & 87.79          \\
\multicolumn{1}{c|}{Tower}            & \multicolumn{1}{c|}{25}                   & \multicolumn{1}{c|}{Tower QE} & 2.38          & 87.40          \\
\multicolumn{1}{c|}{Tower + Tower QE} & \multicolumn{1}{c|}{5 (25* for Tower QE)} & \multicolumn{1}{c|}{\_}                      & 2.12          & 88.95          \\
\multicolumn{1}{c|}{Tower + Tower QE} & \multicolumn{1}{c|}{5 (25* for Tower QE)} & \multicolumn{1}{c|}{XCOMET-XL QE}            & \textbf{2.09} & \textbf{89.08} \\ \midrule
\multicolumn{5}{c}{\textit{Chinese $\rightarrow$ English}}                                                                                                         \\ \midrule
\multicolumn{1}{c|}{Tower}            & \multicolumn{1}{c|}{5}                    & \multicolumn{1}{c|}{\_}                      & 2.42          & 88.91          \\
\multicolumn{1}{c|}{Tower}            & \multicolumn{1}{c|}{25}                   & \multicolumn{1}{c|}{XCOMET-XL QE}            & 2.30          & 89.49          \\
\multicolumn{1}{c|}{Tower}            & \multicolumn{1}{c|}{25}                   & \multicolumn{1}{c|}{Tower QE} & 2.32          & 89.51          \\
\multicolumn{1}{c|}{Tower + Tower QE} & \multicolumn{1}{c|}{5 (25* for Tower QE)} & \multicolumn{1}{c|}{\_}                      & 2.26          & 89.82          \\
\multicolumn{1}{c|}{Tower + Tower QE} & \multicolumn{1}{c|}{5 (25* for Tower QE)} & \multicolumn{1}{c|}{XCOMET-XL QE}            & \textbf{2.24} & \textbf{90.00} \\ \bottomrule
\end{tabular}
}
\caption{Translation Quality on WMT23 English $\rightarrow$ German Test set. Both XCOMET and MetricX columns use reference for reporting translation quality where as XCOMET-XL QE does not use for re-ranking.}
\label{tab:qadecoding}
\end{table*}

\begin{table}[!ht]
\resizebox{\columnwidth}{!}{
\centering
\begin{tabular}{@{}c|ccc@{}}
\toprule
                                                                                      & Pearson        & Spearmann      & Kendall        \\ \midrule
COMETQE                                                                               & \textbf{44.41} & 41.29          & 31.19          \\ \midrule
COMETQE-XL                                                                            & 41.23          & \textbf{42.17} & \textbf{31.84} \\ \midrule
Tower Avg. Log Prob                                                                        & 32.32          & 16.74          & 12.77          \\ \midrule
\begin{tabular}[c]{@{}c@{}}Tower QE\end{tabular} & 40.56          & 33.96          & 25.87          \\ \bottomrule
\end{tabular}
}
\caption{Correlation on WMT 23 for English $\rightarrow$ German Test set. The scores are calculated after removing the few sentences labeled for hallucination detection. Best scores according to each coefficient are highlighted in \textbf{bold}.}
\label{tab:correlation}
\end{table}

\section{Experimental Setup}
\paragraph{Datasets:} We focus on two language directions given their availability of MQM data: English $\rightarrow$ German and Chinese $\rightarrow$ English. To train our token-level QE systems, we use the MQM datasets\footnote{https://github.com/google/wmt-mqm-human-evaluation} from WMT \citep{freitag2021experts}. Specifically, we use the datasets until 2022 for training, 2024 for validation, and 2023 for testing \citep{kocmi2024findings}. This setup is consistent with all the other QE metrics, and we do not use any additional data beyond these datasets.
\vspace{-0.1cm}
\paragraph{Models:} 
Our proposed approach achieves Quality-Aware decoding by combining an NMT model with a token-level QE model, where we adapt the same NMT for QE by adding a classification head. We use the state-of-the-art NMT model, Tower 7B\footnote{Unbabel/TowerInstruct-7B-v0.2} \citep{alves2024tower}, which provides high-quality translations and has already been exposed to MQM data during instruction-tuning. This ensures that the gains observed in our approach stem from integrating Quality-Aware decoding into the NMT process, rather than introducing new data. Additional details on training and hyper-parameters are provided in Appendix \ref{sec:training_detail}.
\vspace{-0.1cm}
\paragraph{Metrics:}
For reporting the translation quality, we consistently use XCOMET-XXL\footnote{Unbabel/XCOMET-XXL} \citep{guerreiro2024xcomet} and MetricX\footnote{google/metricx-24-hybrid-xl-v2p6} \citep{juraska2024metricx} \textbf{with the reference}. To compare with N-best list re-ranking, we use the XCOMET-XL QE\footnote{Unbabel/XCOMET-XL} \textbf{without the reference}. This approach allows us to avoid biasing toward a single metric during the re-ranking process and enables us to measure the gains achieved by differently trained metrics. 

\section{Results}



\begin{table*}[!ht]
\centering
\resizebox{2\columnwidth}{!}{
\begin{tabular}{@{}ccccc@{}}
\toprule
\multicolumn{1}{c|}{Model}            & \multicolumn{1}{c|}{Beams}                        & \multicolumn{1}{c|}{Re-ranking}               & MetricX ($\downarrow$) & XCOMET-XXL ($\uparrow$) \\ \midrule
\multicolumn{5}{c}{\textit{English $\rightarrow$ German}}                                                                                                          \\ \midrule
\multicolumn{1}{c|}{Tower}            & \multicolumn{1}{c|}{25}                           & \multicolumn{1}{c|}{XCOMET-XL QE}             & 2.37     & 87.79      \\
\multicolumn{1}{c|}{Tower}            & \multicolumn{1}{c|}{25}                           & \multicolumn{1}{c|}{Tower QE}         & 2.38     & 87.40      \\
\multicolumn{1}{c|}{Tower}            & \multicolumn{1}{c|}{25}                           & \multicolumn{1}{c|}{Tower Distill QE} & 2.38     & 87.39      \\
\multicolumn{1}{c|}{Tower + Tower QE} & \multicolumn{1}{c|}{5 (25* for Tower QE)}         & \multicolumn{1}{c|}{\_}                       & 2.12     & \textbf{88.95}      \\
\multicolumn{1}{c|}{Tower + Tower QE} & \multicolumn{1}{c|}{5 (25* for Tower Distill QE)} & \multicolumn{1}{c|}{\_}                       & \textbf{2.11}     & 88.76      \\ \bottomrule
\end{tabular}
}
\caption{Performance of Unidirectional QE trained with/without distillation on WMT23 English $\rightarrow$ German Test set. Best scores according to each metric are highlighted in \textbf{bold}.}
\label{tab:towerdistill}
\end{table*}


\begin{table*}[!ht]
\centering
\resizebox{2\columnwidth}{!}{
\begin{tabular}{@{}cccccc@{}}
\toprule
\multicolumn{1}{c|}{Model}            & \multicolumn{1}{c|}{Beams}                & \multicolumn{1}{c|}{Re-ranking}       & XCOMET-XL ($\uparrow$)     & \multicolumn{1}{c|}{XCOMET-XXL ($\uparrow$)}     & Impact                                                                                       \\ \midrule
\multicolumn{6}{c}{\textit{Paragraph-Level}}                                                                                                                                                                                                                                    \\ \midrule
\multicolumn{1}{c|}{Tower}            & \multicolumn{1}{c|}{25}                   & \multicolumn{1}{c|}{XCOMET-XL QE}     & \textbf{86.56} & \multicolumn{1}{c|}{87.79}          & \multirow{3}{*}{\begin{tabular}[c]{@{}c@{}}$\delta$ = + 1.16\\ (88.95 - 87.79)\end{tabular}} \\
\multicolumn{1}{c|}{Tower}            & \multicolumn{1}{c|}{25}                   & \multicolumn{1}{c|}{Tower QE} & 85.40          & \multicolumn{1}{c|}{87.40}          &                                                                                              \\
\multicolumn{1}{c|}{Tower + Tower QE} & \multicolumn{1}{c|}{5 (25* for Tower QE)} & \multicolumn{1}{c|}{\_}               & 86.36          & \multicolumn{1}{c|}{\textbf{88.95}} &                                                                                              \\ \midrule
\multicolumn{6}{c}{\textit{Sentence-Level}}                                                                                                                                                                                                                                     \\ \midrule
\multicolumn{1}{c|}{Tower}            & \multicolumn{1}{c|}{25}                   & \multicolumn{1}{c|}{XCOMET-XL QE}     & \textbf{86.42}          & \multicolumn{1}{c|}{87.68}          & \multirow{3}{*}{\begin{tabular}[c]{@{}c@{}}$\delta$ = + 0.38\\ (88.06 - 87.68)\end{tabular}} \\
\multicolumn{1}{c|}{Tower}            & \multicolumn{1}{c|}{25}                   & \multicolumn{1}{c|}{Tower QE} & 85.23          & \multicolumn{1}{c|}{87.41}          &                                                                                              \\
\multicolumn{1}{c|}{Tower + Tower QE} & \multicolumn{1}{c|}{5 (25* for Tower QE)} & \multicolumn{1}{c|}{\_}               & 85.96          & \multicolumn{1}{c|}{\textbf{88.06}}          &                                                                                              \\ \bottomrule
\end{tabular}
}
\caption{Impact of integrating Unidirectional QE during decoding with paragraphs vs sentences on WMT23 English $\rightarrow$ German Test set. $\delta$ denotes the improvement in translation quality from re-ranking N-best list with XCOMET-XL QE to integrating unidirectional Tower QE during the decoding. Best scores according to each metric are highlighted in \textbf{bold}.}
\label{tab:sentvspara}
\end{table*}



We conduct a series of experiments to validate the effectiveness of Quality-Aware decoding and identify the scenarios where it provides the most benefit. First, we evaluate whether our token-level QE model can better estimate sequence quality compared to the log probabilities of the NMT model. Next, we assess the impact of Quality-Aware decoding by comparing it with other approaches to determine if it improves translation quality. We also perform an ablation study to examine whether training the QE model on errors from the same NMT model enhances its performance. Finally, we explore the impact of source sentence length to highlight the limitations of N-best list re-ranking.

\subsection{Quality Estimation Performance}

First, we evaluate the agreement between the Tower-based token-level QE model (\textbf{Tower QE}) and human scores for a given hypothesis. It is only beneficial if we achieve higher correlation than the average of the NMT model log probabilities to show the need to integrate it during decoding. Therefore, we report the correlation with human scores of different models on WMT 23 English $\rightarrow$ German in Table \ref{tab:correlation}. 

We observe that the best-performing systems are the Comet QE models, which predict a single score using the full hypothesis. This is expected, as these models assess quality after the hypothesis is fully generated. In contrast, both log probabilities and Tower QE scores are based on the predicted token of each decoding step, using only the preceding context. Log probabilities perform poorly in this setup, while our proposed model, Tower QE, achieves twice the correlation with human judgments compared to log probabilities, despite scoring token by token with preceding context. This result highlights the potential of integrating our approach into the decoding process.

\subsection{Unified Decoding for NMT}


To validate the effectiveness of our unified decoding approach, we compare it with several baselines in Table \ref{tab:qadecoding}. First, we evaluate whether our approach outperforms generating translations with the NMT model alone. Next, we check if the quality of translations improves compared to N-best list re-ranking. To make the setups comparable, we set $topk$ and $num\_beams$ to $5$ and compare with re-ranking the top $25$ beams using XCOMET-XL. Finally, to demonstrate that re-ranking the N-best list remains a viable and complementary approach, we re-rank the top $5$ beams obtained from Quality-Aware decoding using the same QE model. 

We find that re-ranking with XCOMET-XL and Tower QE yields similar results, indicating that our partial QE model does not over-fit to any specific metric. Furthermore, we observe that the unified decoding approach outperforms N-best list re-ranking across both metrics in both language pairs. For example, the MetricX score improves from $2.37$ to $2.12$ for English $\rightarrow$ German. Note that Tower has already seen this data during instruction-tuning and the improvement is not from new data but from Quality-Aware decoding. Moreover, re-ranking the top $5$ beams obtained from unified decoding with XCOMET-XL leads to a slight further improvement in quality. This highlights the robustness and generalizability of our approach across different evaluation metrics.
%\footnotetext{\href{https://github.com/WMT-QE-Task/wmt-qe-2023-data}{WMT 23 English $\rightarrow$ German QE Data}}

\subsection{Adapting for Tower Errors}

We use the MQM annotations from WMT to train our Tower QE model, which contains error annotations from other systems. However, a viable alternative would be to adapt Tower QE specifically to the errors it typically makes. To maintain a similar data setup, we first generate translations using Tower on these source sentences. Then, we annotate the generated hypotheses with XCOMET-XL using the reference and fine-tune Tower QE on this synthetic dataset, which we refer to as \textbf{Tower Distill QE}. We evaluate the performance of the new distill QE model and report the results in Table \ref{tab:towerdistill}.

We observe that the distilled QE model performs very similarly to the QE model trained on errors from other systems. This indicates that there was no significant benefit in adapting the QE model to the specific errors typically made by Tower. However, further analysis on larger datasets and different domains is needed to fully validate the effectiveness of the distillation approach as the current synthetic data generated is small.

\subsection{Sentence vs Document-level Translation}

From Table \ref{tab:qadecoding}, we observe that the gains for English $\rightarrow$ German (paragraph-level) are much higher than for Chinese $\rightarrow$ English (sentence-level). We hypothesize that this discrepancy arises from the length of the sentences, as the N-best list re-ranking is likely sufficient for shorter sentences. To confirm this, we take the English paragraphs and split them into sentences using a tokenizer while tracking the paragraph IDs. We then perform the entire decoding process similarly, and later join the sentences back using the paragraph IDs before evaluation. We report the results in Table \ref{tab:sentvspara}.

We define the impact as the improvement in translation quality from re-ranking the N-best list with XCOMET-XL QE to integrating Tower QE. Comparing the results at the paragraph level to those at the sentence level, we observe that the impact decreases, which confirms our hypothesis. Additionally, we obtain better scores at the document level, further highlighting the potential benefits of Quality-Aware Decoding.

\section{Related Work}

\textbf{Integrating QE in NMT:} Several advancements have been made in improving QE for NMT over the years \citep{rei2021references, rei2022cometkiwi, blain2023findings, zerva2024findings, guerreiro2024xcomet}. These developments have led to the integration of QE in various ways.
One common approach involves applying QE after generating multiple sequences through techniques such as QE re-ranking \citep{fernandes2022quality, faria2024quest} or Minimum Bayes Risk (MBR) decoding \citep{tomani2024quality}. Another direction focuses on removing noisy data using QE models, followed by fine-tuning on high-quality data \citep{xu2024contrastive, finkelstein2024introducing}. \citet{vernikos2024don} proposes to generate diverse translations as a first step and then combine them. We perform this explicitly by integrating the QE directly into decoding.
Recently, \citet{zhang2024learning} exploited the MQM data by training models to penalize tokens within an error span, improving translation quality. In contrast, our approach adopts a modular framework, where we propose an expert QE model that is trained independently for targeted training. This modular approach aims to improve performance by decomposing the task into separate translation and QE components.

\textbf{Reward Modeling in NLG:}  Quality-Aware decoding shares several similarities with controllable text generation methods, particularly in the use of an additional "Quality/Reward" model that guides the decoding. A well-explored approach for controlling text is altering the decoding with a reward model (Weighted Decoding) \citep{yang2021fudge}. This method modifies the decoding by adjusting token probabilities based on the reward model, allowing for more controlled generation.
Similarly, \citet{deng-raffel-2023-reward} also used a uni-directional reward model, with the aim of maintaining efficiency during generation. This approach minimizes computational complexity while still benefiting from the guiding influence of the reward model. Moreover, recent work by \citet{li-etal-2024-reinforcement} introduced a token-level reinforcement learning-based reward model, providing more fine-grained feedback that enhances control over text generation at a granular level. While similar, the key contribution in our work lies in the development of the first uni-directional QE model for translation. 


\section{Conclusion}
We have shown the importance of Quality-Aware decoding to improve translation quality, rather than relying on post-generation techniques. In this work, we demonstrated how MQM data can be used to build a uni-directional token-level QE model, which is then integrated into the decoding process. Through a series of experiments, we showed that our Quality-Aware decoding approach results in measurable improvements in translation quality. Notably, we did not introduce new training data to the NMT model, and show that the gains stem from Quality-Aware decoding.


\section{Limitations}
While our Quality-Aware decoding improves translation quality, it adds considerable computational complexity to the inference process. Theoretically, this approach would double the time needed to generate a translation and require additional memory to utilize the token-level QE model. One potential solution to mitigate this issue could be to use token-level QE as a reward model for training via Reinforcement Learning.

Additionally, we trained our model on a limited set of human-annotated MQM data. However, current QE models, such as XCOMET, are capable of predicting error tags using the reference with reasonable quality. This suggests that further improvements could be achieved if these models were trained on larger-scale datasets, providing more nuanced feedback and refining translation quality even further.

Lastly, our proposed token-level QE model does not account for error severity. Ideally, it should be able to predict the category of errors, allowing for more nuanced feedback and enabling the model to generate translations with only minor errors when necessary.


% Bibliography entries for the entire Anthology, followed by custom entries
%\bibliography{anthology,custom}
% Custom bibliography entries only
\bibliography{custom}

\appendix

\section{Appendix}
\label{sec:appendix}

% \begin{table*}[!ht]
% \centering
% \begin{tabular}{@{}ccccc@{}}
% \toprule
% \multicolumn{1}{c|}{Model}            & \multicolumn{1}{c|}{Beams}                & \multicolumn{1}{c|}{Re-ranking}              & XCOMET-XL      & XCOMET-XXL     \\ \midrule
% \multicolumn{5}{c}{\textit{English $\rightarrow$ German}}                                                                                                          \\ \midrule
% \multicolumn{1}{c|}{Tower}            & \multicolumn{1}{c|}{5}           & \multicolumn{1}{c|}{\_}                      & 84.93          & 86.93          \\
% \multicolumn{1}{c|}{Tower}            & \multicolumn{1}{c|}{25}                   & \multicolumn{1}{c|}{\textbf{\_}}             & 84.87 & 86.45          \\
% \multicolumn{1}{c|}{Tower MBR}        & \multicolumn{1}{c|}{25}                   & \multicolumn{1}{c|}{\_}                      & 85.23          & 87.09          \\
% \multicolumn{1}{c|}{Tower}            & \multicolumn{1}{c|}{25}                   & \multicolumn{1}{c|}{XCOMET-XL QE}            & 86.56          & 87.79          \\
% \multicolumn{1}{c|}{Tower}            & \multicolumn{1}{c|}{5}                    & \multicolumn{1}{c|}{Tower QE} & 85.34          & 87.33          \\
% \multicolumn{1}{c|}{Tower}            & \multicolumn{1}{c|}{25}                   & \multicolumn{1}{c|}{Tower QE} & 85.40          & 87.40          \\
% \multicolumn{1}{c|}{Tower + Tower QE} & \multicolumn{1}{c|}{5 (25* for Tower QE)} & \multicolumn{1}{c|}{\_}                      & 86.36          & 88.95          \\
% \multicolumn{1}{c|}{Tower + Tower QE} & \multicolumn{1}{c|}{5 (25* for Tower QE)} & \multicolumn{1}{c|}{XCOMET-XL QE}            & \textbf{86.88} & \textbf{89.08} \\ \midrule
% \multicolumn{5}{c}{\textit{Chinese $\rightarrow$ English}}                                                                                                         \\ \midrule
% \multicolumn{1}{c|}{Tower}            & \multicolumn{1}{c|}{5}                    & \multicolumn{1}{c|}{\_}                      & 85.38          & 88.91          \\
% \multicolumn{1}{c|}{Tower}            & \multicolumn{1}{c|}{25}                   & \multicolumn{1}{c|}{\_}                      & 85.29          & 88.71          \\
% \multicolumn{1}{c|}{Tower MBR}        & \multicolumn{1}{c|}{25}                   & \multicolumn{1}{c|}{\_}                      & 86.00          & 89.23          \\
% \multicolumn{1}{c|}{Tower}            & \multicolumn{1}{c|}{25}                   & \multicolumn{1}{c|}{XCOMET-XL QE}            & 87.04          & 89.49          \\
% \multicolumn{1}{c|}{Tower}            & \multicolumn{1}{c|}{5}                    & \multicolumn{1}{c|}{Tower QE} & 85.64          & 89.10          \\
% \multicolumn{1}{c|}{Tower}            & \multicolumn{1}{c|}{25}                   & \multicolumn{1}{c|}{Tower QE} & 85.93          & 89.51          \\
% \multicolumn{1}{c|}{Tower + Tower QE} & \multicolumn{1}{c|}{5 (25* for Tower QE)} & \multicolumn{1}{c|}{\_}                      & 86.01          & 89.82          \\
% \multicolumn{1}{c|}{Tower + Tower QE} & \multicolumn{1}{c|}{5 (25* for Tower QE)} & \multicolumn{1}{c|}{XCOMET-XL QE}            & \textbf{86.67} & \textbf{90.00} \\ \bottomrule
% \end{tabular}
% \caption{COMET scores on WMT23 English $\rightarrow$ German Test set. Both XCOMET metric columns use reference for reporting translation quality and do not when used for re-ranking }
% \end{table*}


\subsection{Training details}
\label{sec:training_detail}

We use the transformers library \citep{wolf-etal-2020-transformers} for training and inference with Tower-Instruct V2.  For adapting Tower to token-level QE, we use LoRA \citep{hulora} based fine-tuning with an additional classifier head. Therefore, we only train the adapters and the weights for classification head.

We add the adapters to the modules \textit{q\_proj,k\_proj,v\_proj,gate\_proj,up\_proj} and \textit{down\_proj}. We set a batch size for each device to 12 initially and enable \textit{auto\_find\_batch\_size} to \textit{True} on 4 NVIDIA RTX A6000 GPU's. For having a  larger batch size during training, we set \textit{gradient\_accumulation\_steps} to 6. We use a \textit{learning\_rate} of $1e^{-5}$. We set the \textit{eval\_steps} to $50$ and \textit{num\_train\_epochs} to $10$. The other parameters are set to default.

Using the cross-entropy loss for token-level QE directly is insufficient due to the fact that the majority of tokens are classified as '\textit{Good}'. Hence, we find that the weighted cross-entropy loss is essential when fine-tuning the model. For the training on human MQM data, we set the weights to $0.05,0.95$ to '\textit{Good}' and '\textit{Bad}' labels respectively. In the case of distilling from XCOMET, we observed more errors. Therefore, we find that setting them $0.2,0.8$ to '\textit{Good}' and '\textit{Bad}' labels respectively provided stable training.

We train on data until WMT'22 for training and use WMT'24 for validation. We calculate the macro '\textit{F1}' on token-level predictions as the validation metric and stop training if it does not improve for 10 consecutive \textit{eval\_steps}.

\subsection{Partial vs Full Sequence Quality Estimation}

We also compare the difference in performance between our proposed token-level QE for partial sequences with Tower trained for full sequence QE. We achieve this by adding a regression head to predict the score at the end-of-sentence token. Hence, the model uses the source and hypothesis to predict the score using regression head at the end.

We fine-tune the model using only direct assesment data \citep{zerva2024findings} (\textbf{Tower Full DA}). Furthermore, we use this as initialisation and continue fine-tuning on the MQM data (\textbf{Tower Full DA + MQM}). We also use LoRA similarly to the previous model with a regression head to adapt the model. We report the scores in Table \ref{tab:correlation_ablation}.

We see that the both Tower QE models based on full sentences outperforms the partial model. However, this is expected as it has seen the entire context and was also trained on larger amounts of data. Nonetheless, the partial model still achieves much higher correlaiton that the log probabilities showcasing its potential for Quality-Aware decoding.

\subsection{Robustness to re-ranking weight}

In our method, we introduce a hyperparameter, $\alpha$, to merge the probabilities from the token-level QE model and the translation model. This section analyzes the impact of $\alpha$ on the final translation quality.

To efficiently evaluate its effect, we re-rank the N-best list using different values of $\alpha$. This approach allows us to estimate the ideal value of $\alpha$ without the need for joint decoding multiple times. If the re-ranking model (in this case, Tower QE) is beneficial, we expect that any $\alpha$ less than 1 will improve translation quality, as it demonstrates that incorporating the probabilities from the QE model is helpful.

We visualize this impact in Figure \ref{fig:mainfigure}. The results show that using an $\alpha$ less than 1 leads to improved translation quality in both scenarios. This indicates that relying entirely on the NMT model does not yield the best results and highlights the importance of the Tower QE model.

Thus, we emphasize that re-ranking the N-best list provides an effective way to tune the value of $\alpha$, and it remains robust to different values.

\begin{figure*}[!htpb]
\begin{promptbox}[title={Tower Translation Prompt}]
    \small
    <|im\_start|>user\\
    Translate the sentence from English into German.\\
    English: \{src\_sent\}\\
    German:\\
    <|im\_end|>\\
    <|im\_start|>assistant
\end{promptbox}

\begin{promptbox}[title={Tower Token-Level QE Prompt}]
    \small
    English:\{src\_sent\}\\
    German: \{tgt\_sent\}
\end{promptbox}
\caption{Prompts used in our experiments for translation and QE model. \{src\_sent\} and \{tgt\_sent\} represent the source and target sentence. We replace the language with Chinese and English when experimenting with that language pair.}
\end{figure*}

\begin{figure*}[!htpb]
    \centering
    % First subfigure
    \begin{subfigure}[b]{0.5\textwidth}
        \centering
        \includegraphics[width=\textwidth]{Figures/alphas_ende_25.png} % Replace with your image path
        \caption{English $\rightarrow$ German}
        \label{fig:subfigure1}
    \end{subfigure}
    
    \vspace{0.5cm} % Adjust space between the two subfigures

    % Second subfigure
    \begin{subfigure}[b]{0.5\textwidth}
        \centering
        \includegraphics[width=\textwidth]{Figures/alphas_zhen_25.png} % Replace with your image path
        \caption{Chinese $\rightarrow$ English}
        \label{fig:subfigure2}
    \end{subfigure}
    
    \caption{Impact of $\alpha$ when re-ranking with token-level Tower QE on WMT'23 Test sets.}
    \label{fig:mainfigure}
\end{figure*}


\begin{table*}[!ht]
\centering
\begin{tabular}{@{}c|ccc@{}}
\toprule
                                                                                      & Pearson        & Spearmann      & Kendall        \\ \midrule
COMETQE                                                                               & \textbf{44.41} & 41.29          & 31.19          \\ \midrule
COMETQE-XL                                                                            & 41.23          & \textbf{42.17} & \textbf{31.84} \\ \midrule
\begin{tabular}[c]{@{}c@{}}COMETQE Scratch\\      Fine-tuned (ours)\end{tabular}      & 36.32          & 33.66          & 25.24          \\ \midrule
Tower Log Prob                                                                        & 32.32          & 16.74          & 12.77          \\ \midrule
\begin{tabular}[c]{@{}c@{}}Tower Partial QE\end{tabular} & 40.56          & 33.96          & 25.87          \\ \midrule
Tower Full DA                                                                        & 33.67          & 36.46          & 27.38          \\ \midrule
Tower Full DA + MQM                                                                 & 32.03          & 40.85          & 30.38          \\ \bottomrule
\end{tabular}
\caption{Full Correlation results on WMT 23 for English $\rightarrow$ German Test set. Partial indicates that the QE model predict scores via token-level where as full indicates predicting the score at the end-of-sentence token. The scores are calculated after removing the few sentences labelled for hallucination detection. Best scores according to each coefficient are highlighted in \textbf{bold}.}
\label{tab:correlation_ablation}
\end{table*}




\end{document}

\bibliography{custom}

\twocolumn[\newpage]

\appendix


\section{Hyper-Parameter Details}
\label{sec:appendix_hyper}

The detailed hyper-parameter settings for each language are shown in Table \ref{tab:ASR-hyperparameter} for ASR and Table \ref{tab:ST-hyperparameter} for ST, respectively.

\begin{table}[h]
  \caption{ASR hyper-parameters for high-resource languages.}
  \label{tab:ASR-hyperparameter}
  \centering
  \setlength{\tabcolsep}{2pt}
  \begin{adjustbox}{width=\columnwidth}
  \begin{tabular}{l|ccccc}
    \toprule
    \multicolumn{1}{l}{\multirow{2}{*}{\textbf{System}}} & \multicolumn{5}{c}{\textbf{ASR}} \\
    % \cmidrule{2-7}
     & ca & de & es & fr & it \\
    \midrule
    \textbf{Finetuned} \\
    learning rate & $1 \times 10^{-6}$ & $5 \times 10^{-8}$ & $1 \times 10^{-7}$ & $1 \times 10^{-6}$ & $5 \times 10^{-6}$ \\
    \midrule
    \textbf{Multi-lingual training} \\
    learning rate & \multicolumn{5}{c}{$1 \times 10^{-5}$} \\
    \midrule
    \textbf{Task Arithmetic} \\
    scaling factor $\lambda$ & \multicolumn{5}{c}{0.15} \\
    \midrule
    \textbf{LoRS-Merging} \\
    scaling factor $\lambda$ & \multicolumn{5}{c}{0.15} \\
    SVP ratio $r$ & 5\% & 3\% & 2\% & 1\% & 1\% \\
    MP ratio $p$ & 40\% & 60\% & 40\% & 10\% & 10\% \\
    \bottomrule
  \end{tabular}
  \end{adjustbox}
\end{table}

\begin{table}[h]
  \caption{ST hyper-parameters for high-resource languages.}
  \label{tab:ST-hyperparameter}
  \centering
  \setlength{\tabcolsep}{2pt}
  \begin{adjustbox}{width=\columnwidth}
  \begin{tabular}{l|ccccc}
    \toprule
    \multicolumn{1}{l}{\multirow{2}{*}{\textbf{System}}} & \multicolumn{5}{c}{\textbf{ST}} \\
    % \cmidrule{2-7}
     & ca & de & es & fr & it \\
    \midrule
    \textbf{Finetuned} \\
    learning rate & $1 \times 10^{-6}$ & $2 \times 10^{-8}$ & $2 \times 10^{-8}$ & $5 \times 10^{-8}$ & $5 \times 10^{-8}$ \\
    \midrule
    \textbf{Multi-lingual training} \\
    learning rate & \multicolumn{5}{c}{$5 \times 10^{-9}$} \\
    \midrule
    \textbf{Task Arithmetic} \\
    scaling factor $\lambda$ & \multicolumn{5}{c}{0.15} \\
    \midrule
    \textbf{LoRS-Merging} \\
    scaling factor $\lambda$ & \multicolumn{5}{c}{0.15} \\
    SVP ratio $r$ & 5\% & 3\% & 5\% & 2\% & 1\% \\
    MP ratio $p$ & 60\% & 40\% & 20\% & 20\% & 20\% \\
    \bottomrule
  \end{tabular}
  \end{adjustbox}
\end{table}


% \newpage


\section{Results of Low-Resource Language Set}
\label{sec:appendix_low}

The results of the low-resource language set are shown in this section. Specifically, Table \ref{tab:multi-lingual-ASR-2} and \ref{tab:multi-lingual-ST-2} show the multi-lingual single task training and merging for ASR and ST respectively.

\begin{table}[ht]
  \caption{Multi-lingual ASR model merging. Finetuned is the topline where the model is finetuned on each language independently, and Avg. averages WER directly.}
  %\vspace*{-3mm} 
  \label{tab:multi-lingual-ASR-2}
  \centering
  % \setlength{\tabcolsep}{1.5pt}
  \begin{adjustbox}{width=\columnwidth}
  \begin{tabular}{l|ccccc|c}
    \toprule
    \multicolumn{1}{l}{\multirow{2}{*}{\textbf{System}}} & \multicolumn{6}{c}{\textbf{WER$\downarrow$}} \\
    % \cmidrule{2-7}
     & id & nl & pt & ru & sv & Avg. \\
    \midrule
    Pretrained & 16.9 & 16.0 & 10.1 & 17.1 & 17.1 & 15.43 \\
    Finetuned & 15.0 & 14.8 & 9.7 & 16.8 & 14.7 & 14.20 \\
    \midrule
    Multi-lingual training & 16.7 & 15.5 & 10.0 & 17.0 & 16.6 & 15.14 \\
    \midrule
    Weight Averaging & 15.7 & 15.2 & 10.1 & 17.1 & 15.8 & 14.77 \\
    Task Arithmetic & 15.7 & 15.1 & 9.9 & 17.0 & 15.8 & 14.69 \\
    MP-Merging & 15.7 & 15.1 & 10.0 & 16.7 & 15.7 & 14.63 \\
    SVP-Merging & 15.7 & 15.1 & 9.9 & 16.9 & 15.7 & 14.65 \\
    LoRS-Merging & 15.7 & 15.1 & 9.7 & 16.8 & 15.6 & \textbf{14.57} \\
    \bottomrule
  \end{tabular}
  \end{adjustbox}
\end{table}


\begin{table}[ht]
  \caption{Multi-lingual ST model merging. Finetuned is the topline where the model is finetuned on each language independently, and Avg. averages BLEU directly.}
  %\vspace*{-3mm} 
  \label{tab:multi-lingual-ST-2}
  \centering
  % \setlength{\tabcolsep}{1.5pt}
  \begin{adjustbox}{width=\columnwidth}
  \begin{tabular}{l|ccccc|c}
    \toprule
    \multicolumn{1}{l}{\multirow{2}{*}{\textbf{System}}} & \multicolumn{6}{c}{\textbf{BLEU$\uparrow$}} \\
     & id & nl & pt & ru & sv & Avg. \\
    \midrule
    Pretrained & 32.5 & 31.6 & 43.3 & 35.5 & 32.1 & 35.00 \\
    Finetuned & 35.2 & 34.0 & 43.8 & 36.7 & 37.6 & 37.46 \\
    \midrule
    Multi-lingual training & 32.3 & 33.2 & 43.5 & 35.4 & 34.3 & 35.74 \\
    \midrule
    Weight Averaging & 33.6 & 32.2 & 43.2 & 35.3 & 34.2 & 35.70 \\
    Task Arithmetic & 33.9 & 32.8 & 43.1 & 35.5 & 34.3 & 35.92 \\
    MP-Merging & 33.8 & 32.8 & 43.5 & 35.8 & 34.0 & 35.98 \\
    SVP-Merging & 33.6 & 32.6 & 43.4 & 35.6 & 34.3 & 35.90 \\
    LoRS-Merging & 33.9 & 32.8 & 43.2 & 35.9 & 34.5 & \textbf{36.06} \\
    \bottomrule
  \end{tabular}
  \end{adjustbox}
\end{table}

Then, Table \ref{tab:multi-task-2} shows the uni-lingual multi-task training and merging performance (c.f. compare to \ref{tab:multi-task} for high-resource languages).

\begin{table*}[ht]
  \caption{Multi-task model merging. Finetuned is the topline where the model is finetuned on each language and task combination independently, and Avg. averages WER or BLEU scores directly.}
  %\vspace*{-3mm} 
  \label{tab:multi-task-2}
  \centering
  % \setlength{\tabcolsep}{1.5pt}
  \begin{adjustbox}{width=0.95\textwidth}
  \begin{tabular}{l|ccccc|c|ccccc|c}
    \toprule
    \multicolumn{1}{l}{\multirow{2}{*}{\textbf{System}}} & \multicolumn{6}{c}{\textbf{WER$\downarrow$}} & \multicolumn{6}{c}{\textbf{BLEU$\uparrow$}} \\
     & id & nl & pt & ru & sv & Avg. & id & nl & pt & ru & sv & Avg. \\
    \midrule
    Pretrained & 16.9 & 16.0 & 10.1 & 17.1 & 17.1 & 15.43 & 32.5 & 31.6 & 43.3 & 35.5 & 32.1 & 35.00 \\
    Finetuned & 15.0 & 14.8 & 9.7 & 16.8 & 14.7 & 14.20 & 35.2 & 34.0 & 43.8 & 36.7 & 37.6 & 37.46 \\
    \midrule
    Multi-task training & 15.4 & 15.0 & 9.3 & 16.6 & 14.3 & 14.12 & 35.3 & 33.7 & 43.6 & 36.2 & 35.8 & 36.92 \\
    \midrule
    Weight Averaging & 14.7 & 14.9 & 9.3 & 16.6 & 13.8 & 13.88 & 35.4 & 33.9 & 44.1 & 36.3 & 35.9 & 37.12 \\
    Task Arithmetic & 14.6 & 14.9 & 9.3 & 16.5 & 14.0 & 13.88 & 35.3 & 33.8 & 44.3 & 36.1 & 36.4 & 37.18 \\
    MP-Merging & 14.4 & 14.7 & 9.4 & 16.5 & 13.8 & 13.78 & 35.7 & 33.9 & 44.3 & 36.1 & 36.1 & 37.22 \\
    SVP-Merging & 14.6 & 14.8 & 9.2 & 16.4 & 13.9 & 13.80 & 35.3 & 33.9 & 44.3 & 36.2 & 36.3 & 37.20 \\
    LoRS-Merging & 14.4 & 14.7 & 9.2 & 16.4 & 13.8 & \textbf{13.72} & 35.6 & 33.9 & 44.3 & 36.3 & 36.4 & \textbf{37.30} \\
    \bottomrule
  \end{tabular}
  \end{adjustbox}
\end{table*}


Last, Table \ref{tab:multi-lingual multi-task-2} shows the results of multi-lingual and multi-task training and merging results for low-resource languages (compare to Table \ref{tab:multi-lingual multi-task} for high-resource languages.). LoRS-Merging achieved the best performance across all merging and training methods in all tables.

\begin{table*}[ht]
  \caption{Multi-lingual multi-task model merging. Finetuned is the topline where the model is finetuned on each language and task combination independently, and Avg. averages WER or BLEU scores directly.}
  %\vspace*{-3mm} 
  \label{tab:multi-lingual multi-task-2}
  \centering
  % \setlength{\tabcolsep}{1.5pt}
  \begin{adjustbox}{width=0.95\textwidth}
  \begin{tabular}{l|ccccc|c|ccccc|c}
    \toprule
    \multicolumn{1}{l}{\multirow{2}{*}{\textbf{System}}} & \multicolumn{6}{c}{\textbf{WER$\downarrow$}} & \multicolumn{6}{c}{\textbf{BLEU$\uparrow$}} \\
     & id & nl & pt & ru & sv & Avg. & id & nl & pt & ru & sv & Avg. \\
    \midrule
    Pretrained & 16.9 & 16.0 & 10.1 & 17.1 & 17.1 & 15.43 & 32.5 & 31.6 & 43.3 & 35.5 & 32.1 & 35.00 \\
    Finetuned & 15.0 & 14.8 & 9.7 & 16.8 & 14.7 & 14.20 & 35.2 & 34.0 & 43.8 & 36.7 & 37.6 & 37.46 \\
    \midrule
    ML and MT training & 16.9 & 15.7 & 9.6 & 17.0 & 16.3 & 15.08 & 32.8 & 32.9 & 43.3 & 35.4 & 32.6 & 35.40 \\
    \midrule
    ML and MT Task Arithmetic & 16.4 & 15.5 & 9.6 & 16.8 & 15.7 & 14.79 & 33.7 & 33.1 & 43.2 & 35.7 & 34.9 & 36.12 \\
    ML and MT LoRS-Merging & 16.1 & 15.5 & 9.5 & 16.8 & 15.7 & 14.72 & 33.7 & 33.2 & 43.5 & 35.8 & 34.9 & \textbf{36.22} \\
    \midrule
    MT training & 15.4 & 15.0 & 9.3 & 16.6 & 14.3 & 14.12 & 35.3 & 33.7 & 43.6 & 36.2 & 35.8 & 36.92 \\
    $\hookrightarrow$ + ML Task Arithmetic & 16.0 & 15.5 & 9.5 & 16.9 & 15.4 & 14.66 & 34.1 & 32.8 & 43.7 & 35.6 & 33.3 & 35.90 \\
    $\hookrightarrow$ + ML LoRS-Merging & 16.1 & 15.3 & 9.4 & 16.8 & 15.3 & \textbf{14.57} & 34.2 & 32.7 & 43.8 & 35.8 & 33.5 & 36.00 \\
    \midrule
    ML training & 16.7 & 15.5 & 10.0 & 17.0 & 16.6 & 15.14 & 32.3 & 33.2 & 43.5 & 35.4 & 34.3 & 35.74 \\
    $\hookrightarrow$ + MT Task Arithmetic & 17.1 & 15.5 & 9.5 & 17.0 & 15.5 & 14.89 & 32.1 & 33.1 & 43.6 & 35.7 & 33.6 & 35.62 \\
    $\hookrightarrow$ + MT LoRS-Merging & 16.9 & 15.5 & 9.4 & 16.8 & 15.5 & 14.80 & 32.6 & 33.2 & 43.6 & 35.9 & 33.6 & 35.78 \\
    \bottomrule
  \end{tabular}
  \end{adjustbox}
\end{table*}

% \newpage

\section{Detailed Results on Multi-task merging}
\label{sec:appendix_detail}

Detailed per-language results of Table \ref{tab:multi-task} are shown in Table \ref{tab:multi-task-d}.

\begin{table*}[ht]
  \caption{Multi-task model merging. Finetuned is the topline where the model is finetuned on each language and task combination independently, and Avg. averages WER or BLEU scores directly.}
  %\vspace*{-3mm} 
  \label{tab:multi-task-d}
  \centering
  % \setlength{\tabcolsep}{1.5pt}
  \begin{adjustbox}{width=0.95\textwidth}
  \begin{tabular}{l|ccccc|c|ccccc|c}
    \toprule
    \multicolumn{1}{l}{\multirow{2}{*}{\textbf{System}}} & \multicolumn{6}{c}{\textbf{WER$\downarrow$}} & \multicolumn{6}{c}{\textbf{BLEU$\uparrow$}} \\
     & ca & de & es & fr & it & Avg. & ca & de & es & fr & it & Avg. \\
    \midrule
    Pretrained & 20.6 & 19.6 & 14.7 & 24.5 & 19.4 & 19.88 & 21.1 & 24.1 & 28.6 & 26.8 & 26.8 & 25.48 \\
    Finetuned & 19.5 & 19.7 & 14.4 & 22.1 & 19.2 & 19.05 & 22.6 & 24.6 & 29.2 & 27.2 & 27.3 & 26.18 \\
    \midrule
    Multi-task training & 17.0 & 19.7 & 14.4 & 24.2 & 19.4 & 19.00 & 22.3 & 24.6 & 28.7 & 27.0 & 26.9 & 25.90 \\
    \midrule
    Weight Averaging & 17.1 & 19.6 & 13.9 & 23.7 & 19.6 & 18.84 & 22.9 & 24.4 & 29.0 & 27.7 & 26.9 & 26.18 \\
    Task Arithmetic & 17.2 & 19.3 & 14.0 & 23.3 & 19.7 & 18.76 & 23.4 & 24.5 & 28.9 & 27.7 & 27.0 & 26.30 \\
    MP-Merging & 17.8 & 19.5 & 14.4 & 23.8 & 17.2 & 18.62 & 23.1 & 24.5 & 29.1 & 27.9 & 27.4 & 26.40 \\
    SVP-Merging & 18.0 & 19.4 & 14.4 & 23.7 & 17.7 & 18.72 & 22.9 & 24.7 & 29.1 & 27.8 & 27.4 & 26.38 \\
    LoRS-Merging & 17.5 & 19.4 & 14.2 & 23.1 & 17.7 & \textbf{18.45} & 23.1 & 24.5 & 29.3 & 27.9 & 27.6 & \textbf{26.48} \\
    \bottomrule
  \end{tabular}
  \end{adjustbox}
\end{table*}


\end{document}
