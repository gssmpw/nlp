\section{Related Work}
~\label{sec:relatedwork}
\subsection{Clustered federated learning}
Clustering in FL can train more targeted models and enhance overall model performance by grouping clients with similar data distributions together. Wang \textit{et al.} proposed an FL mechanism to determine the optimal number of clusters with resource budget in cluster FL~\cite {WangTMC22}. He \textit{et al.} trained a cluster FL model using adaptive pruning, weight compression, and parameter shuffling to reduce the communication overhead of IoT devices~\cite {HeIOT24}. He \textit{et al.} designed a cluster federated learning architecture to cluster drone swarms based on the similarity between optimization directions of participants~\cite {HeTVT23}. Yan \textit{et al.} proposed an iterative clustering FL framework, in which servers dynamically discover clustering structures by continuously performing incremental clustering and clustering in one iteration~\cite{YanTNNLS23}. Morafah \textit{et al.} proposed an algorithm that utilizes the inference similarity of client models to cluster client populations into clusters with jointly trainable data distributions~\cite{MorafahOJCS23}. Xu \textit{et al.} proposed a Clustering FL method, where devices are divided into several clusters based on their data distribution and trained simultaneously~\cite{XuIOT24}. 
Run \textit{et al.} proposed a dynamic adaptive cluster joint learning scheme that utilizes the mutual sensitivity between the model and data as intuition to perform cluster identity estimation, cluster addition, cluster model update, and cluster deletion in the early iterations of FL, in order to find the optimal client cluster partition~\cite{RunKBS23}.
The existing work mainly discusses the framework of clustering FL to enhance FL effectiveness in terms of accuracy. 
%标记:可能需要再次检查文献情况
However, the processing time in FL is a critical factor for satellite networks that requires careful consideration in the framework design.



\subsection{Federated learning in satellite networks}
Recently, the potential of introducing FL in satellites has been widely discussed because of its ability to enable collaborative training while preserving user privacy. Xu \textit{et al.} designed a connectivity-density-aware FL framework for satellite-terrestrial integration, utilizing ground stations as parameter servers. The effectiveness of this method in terms of execution efficiency and accuracy was validated~\cite{XUFGCS24}. Elmahalawy \textit{et al.} used high-altitude platforms as a distributed PS to improve satellite visibility while enhancing FL performance~\cite{ElmahallawySAC24}. Xiong \textit{et al.} proposed an energy-efficient FL scheme for Earth observation applications in LEO satellite systems to minimize training loss and energy consumption~\cite{XiongWCNC24}. Zhou \textit{et al.} proposed a FL framework with partial device involvement. In this framework, the onboard controller strategically selects a portion of devices to upload local model parameters, by which reduces the overall energy consumption on the device side~\cite{ZhouWCNC24}. Chen \textit{et al.} proposed a LEO satellite collaborative FL framework using LEO satellites as parameter servers, which enables FL to achieve high classification accuracy with relatively low communication latency~\cite{ChenPIMRC23}. Zhou \textit{et al.} designed an FL multi-objective optimization algorithm that leverages decomposition and meta deep reinforcement learning. This algorithm enhances both communication training efficiency and local training accuracy by addressing multi-objective optimization challenges, resulting in more efficient uploading and aggregation~\cite{ZhouSAC24}. Xia \textit{et al.} proposed a cross domain joint computation offloading algorithm to optimize the trade-off between information age and energy consumption when serving multiple traffic categories in satellite networks~\cite{XiaPIMRC23}. In resource-constrained satellite environments, the need for implementing FL processes in an energy-efficient manner is heightened. Therefore, it is essential to develop a solution that can optimally select PS while reducing the processing time and energy consumption during the FL process. 

%the demand for realizing FL process in an energy-efficient manner is stronger, so how to choose parameter servers and communication methods is one of the challenges in solving satellite resources.