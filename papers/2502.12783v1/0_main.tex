\documentclass[10pt, conference, letterpaper]{IEEEtran}
\IEEEoverridecommandlockouts
% The preceding line is only needed to identify funding in the first footnote. If that is unneeded, please comment it out.
\usepackage{cite}
\usepackage{amsmath,amssymb,amsfonts}
\usepackage{url}
%\usepackage{algorithmicx}
%\usepackage{algpseudocode}
\usepackage{algorithmic}
\usepackage{algorithm} 
\usepackage[french,boxed,vlined,linesnumbered,inoutnumbered,rightnl,algo2e]{algorithm2e}
%\usepackage[algo2e]{algorithm2e} 
\usepackage{graphicx}
\usepackage{textcomp}
\usepackage{xcolor}
\usepackage{comment}
\usepackage{booktabs}
%\usepackage{subfig}
\usepackage{subcaption}
\usepackage{multirow}
\usepackage{mathrsfs}
\usepackage{threeparttable}
\usepackage{booktabs}
\newcommand*{\note}[1]{\textcolor{red}{#1}}
\newcommand*{\modify}[1]{\textcolor{blue}{#1}}
\newcommand{\algname}{FedHC\xspace}
\renewcommand{\algorithmicrequire}{\textbf{Input:}}
\renewcommand{\algorithmicensure}{\textbf{Output:}}

\renewcommand{\algorithmiccomment}[1]{ $\triangleright$ #1}
\def\BibTeX{{\rm B\kern-.05em{\sc i\kern-.025em b}\kern-.08em
    T\kern-.1667em\lower.7ex\hbox{E}\kern-.125emX}}

\setlength{\abovecaptionskip}{0.05cm}
\begin{document}


%\title{An Adaptive Quantity Approach for Minimizing Cost and Delay in Satellite Clustered Federated Learning}

%\title{An Adaptive Quantization Approach for Cost and Latency Minimization in Satellite Clustered Federated Learning Framework \note{[TENTATIVE]}\\


%\title{An Adaptive Quantization Approach for Multi Objective Optimization in Satellite Clustered Federated Learning Framework}

%\title{Satellite Clustered Federated Learning Framework with Adaptive Quantization \note{[TENTATIVE]}\\

\title{\algname: A Hierarchical Clustered Federated Learning Framework for Satellite Networks \\

%\thanks{Identify applicable funding agency here. If none, delete this.}
}

%\author{Anonymous Authors}

%\begin{comment}
\author{\IEEEauthorblockN{Zhuocheng Liu$^{1}$, Zhishu Shen$^{1}$\textbf{\IEEEauthorrefmark{2}}\thanks{\IEEEauthorrefmark{2} Corresponding author (z\_shen@ieee.org).}, Pan Zhou$^{1}$, Qiushi Zheng$^{2}$, and Jiong Jin$^{2}$}

\IEEEauthorblockA{\textsuperscript{$^{1}$}School of Computer Science and Artificial Intelligence, Wuhan University of Technology, China\\
}


\IEEEauthorblockA{\textsuperscript{$^{2}$}School of Science, Computing and Engineering Technologies, Swinburne University of Technology, Australia\\
}
%\IEEEauthorblockA{E-mail: \{\note{xxx}, \note{xxx}, yjl\}@whut.edu.cn, z\_shen@ieee.org, xxx@xxxx.com, jiongjin@swin.edu.au}
}
%\end{comment}

\maketitle

\begin{abstract}
%With the proliferation of data-driven services, the volume of data that needs to be processed by satellite networks has significantly increased. Federated learning (FL) is well-suited for handling big data processing at distributed, resource-constrained satellite environments. However, ensuring the convergence performance of FL models within dynamically distributed satellite networks is challenging while minimizing FL processing time and energy cost. To this end, we propose a hierarchical clustered federated learning framework named \algname. In this framework, a satellite-clustered parameter server (PS) selection algorithm is deployed at the cluster aggregation stage to partition nearby satellites into distinct clusters, designating the satellite cluster center client as the PS to accelerate the model aggregation. Several communicable cluster PS satellites are then selected through ground stations to aggregate global parameters to support the FL process. Moreover, a meta-learning-driven satellite re-clustering algorithm is introduced to enhance adaptability to dynamic satellite cluster changes. The extensive experiments conducted on satellite networks testbed demonstrate that \algname can significantly reduce processing time (up to 3x) and energy cost (up to 2x) against other comparative methods, while ensuring the accuracy of the model.

With the proliferation of data-driven services, the volume of data that needs to be processed by satellite networks has significantly increased. Federated learning (FL) is well-suited for big data processing in distributed, resource-constrained satellite environments. However, ensuring its convergence performance while minimizing processing time and energy consumption remains a challenge. To this end, we propose a hierarchical clustered federated learning framework, \algname. This framework employs a satellite-clustered parameter server (PS) selection algorithm at the cluster aggregation stage, grouping nearby satellites into distinct clusters and designating a cluster center as the PS to accelerate model aggregation. Several communicable cluster PS satellites are then selected through ground stations to aggregate global parameters, facilitating the FL process. Moreover, a meta-learning-driven satellite re-clustering algorithm is introduced to enhance adaptability to dynamic satellite cluster changes. The extensive experiments on satellite networks testbed demonstrate that \algname can significantly reduce processing time (up to 3x) and energy consumption (up to 2x) compared to other comparative methods while maintaining model accuracy.

%Low Earth Orbit (LEO) satellites need to process vast amounts of communication data from the ground daily. Federated Learning (FL) is well-suited for handling this data at the edge, where resources are limited. FL only requires the transmission of trained model parameters during data processing, reducing communication needs while achieving good accuracy. However, maintaining the convergence performance of FL models in a dynamically distributed satellite architecture, while minimizing communication time and energy consumption, remains a critical challenge.In this work, we propose a novel FL framework called hierarchical clustered quantitative federated learning (\algname)., to address the aforementioned issues while ensuring FL learning accuracy. \algname is built on two stages of operation: satellite cluster aggregation stage and ground station aggregation stage. We use K-means clustering algorithm to divide nearby satellites into different clusters, and specify the client of the satellite cluster center with good communication as the parameter server to accelerate the aggregation speed of the model. Aggregate global model parameters from different satellite clusters to perform FL. After the aggregation process is completed, the ground station selects a communicable cluster parameter server satellite to aggregate global parameters. Conduct model experiments on real satellite distribution data to verify the effectiveness of \algname. The results indicate that \algname can significantly reduce communication time and energy consumption with the same accuracy.
\end{abstract}

\begin{IEEEkeywords}
Satellite networks, hierarchical clustered federated learning, distributed computing%, multi-objective optimization
\end{IEEEkeywords}

\section{Introduction}\label{sec:intro}

In computational finance, Monte Carlo simulations are used extensively to estimate the expected value of financial payoffs based on the solution of stochastic differential equations (SDEs) which model the evolution of stock prices, interest rates, exchange rates and other quantities \cite{glasserman04}.  Monte Carlo methods are very general and flexible, but for high accuracy it requires generating a large number of costly SDE path approximations, which has motivated research into a number of variance reduction or, equivalently, cost reduction techniques. One such method is
Multilevel Monte Carlo (MLMC), which was proposed in \cite{GILES2008} and was adapted for various applications that are summarised in \cite{Giles_overview17} and successfully combined with other methods such as quasi-Monte Carlo methods. The main idea of MLMC is to approximate the payoff using different time stepping resolutions when numerically solving the underlying SDE and to generate an optimal number of samples on each level, such that the overall computational cost is minimised subject to the desired bound on the variance. %, such that the total computational cost is minimised. 
The computational savings come from the fact that most samples are computed on the coarser levels and hence are less expensive while only a few samples from the finest levels are required \cite{GILES2008}.


Among the directions in which the computational cost 
of MLMC methods could further be reduced, an important avenue is the use of lower precision calculations, especially for the first Monte Carlo levels where the targeted accuracy is relatively low. 
 An overview of the research on mixed precision for the standard Monte Carlo (MC) framework is provided in \cite{ChowMixedPrecisionStandardMC} but only a few references study the potential of low precision computation in the MLMC framework \cite{Rounding_error_oliver}. To the best of our knowledge, the only MLMC framework with customised precision in the literature is \cite{brugger2014mixed}, but they use a uniform precision for all operations on each Monte Carlo level instead of optimising 
 the precision of each intermediary variable to reduce as much as possible the cost of path generation.
 
An important motivation for an MLMC framework with variable precision would be performing the low precision computations on reconfigurable hardware devices such as Field Programmable Gate Arrays (FPGAs). FPGAs contain customizable logic blocks and connectors that make it easy to adapt the digital circuit architecture for a specific application, leading to a highly parallel and optimised implementation. Therefore they are successfully exploited in applications that require high speed and have high computational workload, such as signal processing \cite{woods2008fpga}, and real time applications like high frequency trading \cite{HFT1,HFT2}. That is why a number of previous works in hardware architecture design implemented the MLMC algorithm to price financial options using FPGAs as accelerators, which resulted in improved speed and power efficiency compared to full CPU architectures \cite{Schryver2013AMM}. The paper \cite{lindsey2016domain} also proposed 
a Domain Specific Language to automate the configuration of FPGAs for this specific application. However, only \cite{brugger2014mixed} proposed a heuristic to reduce the precision in calculations.

In addition, all aforementioned works considered that the random number generation (RNG) is performed in single or double precision. Yet in most cases an important portion of the workload in the overall MLMC simulation comes from the RNG and in \cite{brugger2014mixed} this limited the total computational savings.
To reduce the cost of MLMC simulations in particular those based on the Geometric Brownian Motion (GBM), \cite{approximateICDF_Oliver, NestedOliver} have proposed to use approximate random numbers that are generated by applying an approximation of the inverse CDF to uniform random numbers. In \cite{NestedOliver}, the authors proposed a way to integrate these lower precision random variables into a \textit{nested} MLMC framework and completed a numerical analysis to bound the resulting error at each MC level by a product of the time step and the error in the random number approximation. The same authors show in \cite{approximateICDF_Oliver} that using approximate random variables reduces the cost of path generation by a factor 7.


In this paper we propose a nested MLMC framework that combines the use of approximate random normal variables and lower precision calculations to reduce the computational cost of MLMC even further than \cite{brugger2014mixed,NestedOliver}. We illustrate the efficiency of our framework in Matlab, after making several assumptions on the cost of operations and size of the errors that we carefully justify. We focus on the case of GBM and use the approximate RNG methods presented in \cite{approximateICDF_Oliver} as well as a new slightly modified method that combines CDF inversion and the central limit theorem. To choose the precision of the variables in the low precision path generation, we introduce a novel method to optimise the bit-widths. This optimisation is performed before the main path generation loop is executed and is based on a linear model of the payoff error  
due to rounding when computing in low precision. The error model relies on algorithmic differentiation in a similar manner to \cite{unifying-bwoptim,bitwidth-AD,ADAPT}. The bit-width optimisation procedure can be performed off-line, so this stage can be excluded from the on-line time complexity of our framework. The user specified desired accuracy is then enforced by calculating on-line the number of samples that need to be generated.

In terms of hardware design, we suggest implementing the low precision path generation on FPGAs and the full-precision ones on a CPU or GPU. 
The FPGA offers enough flexibility to define a separate bit-width for every variable in the low precision path generation, and can be reconfigured periodically to update the bit-widths when the market parameters have changed considerably. 


The paper is organized as follows : \Cref{sec:MLMC} introduces MLMC and nested MLMC to make clear the estimator that is implemented in our framework. Then in \Cref{sec:RNG} we detail the methods that could be used to obtain approximate random normally distributed numbers very cheaply for the low precision path generation. In \Cref{sec:error_model} and \Cref{sec:costModel} we propose an error model and a cost model (resp.) that we then use to formulate the optimisation problem that is solved to obtain the optimal bit-widths of fixed point variables in \Cref{sec:optimisation}. Finally we summarise our results and future directions in \Cref{sec:conclusion}.



%\section{Related Work}
\label{sec:related work}
% In this section, we review the existing literature on point cloud denoising and unsupervised image denoising.
%-------------------------------------------------------------------------
\subsection{Point cloud denoising}

    \subsubsection{Traditional methods}
Traditional approaches to point cloud denoising include statistical methods \cite{computingpointset2003,definingpointset2004,wlop2009HH}, filtering techniques\cite{pointsetsurfaces2001,Robustmoving2005, zaman2017density}, and optimization-based methods \cite{l1sparse2010,clop2014PR,digne2017bilateral,multi-projection2018duan,hu2020featuregraph} . These techniques often rely on handcrafted features and heuristics to distinguish signal from noise. For example, statistical methods may use distribution models to identify and remove outliers. Filtering methods, such as mean or median filters, operate under the assumption that noise is statistically different from the signal. Optimization-based methods formulate denoising as an energy minimization problem, where regularization terms constrain the solution to ensure certain smoothness cirterion or adherence to prior knowledge.

%-------------------------------------------------------------------------
    \subsubsection{Supervised learning based methods}
In recent years, several deep learning-based methods \cite{rakotosaona2020PCN,hermosilla2019TotalDenoising,luo2020DMR,luo_score-based_2021} have been proposed for point cloud denoising. NPD \cite{NPD2019} is the first learning-based point cloud denoising method that directly processes noisy data without requiring noise characteristics or neighboring point definitions. This approach combines local and global information by projecting noisy points onto estimated reference planes, effectively removing noise and enhancing robustness against variations in noise intensity and curvature. PointCleanNet\cite{rakotosaona2020PCN} first removes outlier points and then combines them with residual connectivity to predict the inverse displacement \cite{Guerrero2017PCPNetLL}, and iteratively shifts noisy points to remove noise. Pistilli \etal proposed GPDNet \cite{gpdnet2020}, which is a graph convolutional network to improve denoising robustness at high noise levels. Luo \etal also proposed  DMRDenoise \cite{luo2020DMR}, which filter
points by first downsampling the noisy inputs and reconstructing the local subsurface to perform point upsampling. However, this resampling method is difficult to maintain a good local shape. ScoreDenoise \cite{luo_score-based_2021} is proposed to tackle the aforementioned issues by iteratively updating the point position in implicit gradient fields learned by neural networks. For inference, they follows an iterative procedure with a decaying step size, which stabilizes point movement and prevents over-correction, allowing points to converge gradually toward the underlying geometry. The authors of \cite{de_Silva_Edirimuni_2023_CVPR} proposed IterativePFN - an iterative method that use a novel loss function that utilizes an adaptive ground truth target at each iteration to capture the relationship between intermediate filtering results during training. Zheng \etal proposed a end-to-end network for joint normal filtering-based point cloud denoising \cite{10173632}. They introduce an auxiliary normal filtering task to enhance the network's ability to remove noise while preserving geometric features more accurately.

Supervised methods can achieve impressive results, but are limited by the availability and quality of the training data, as they typically require paired noisy and clean point clouds to train the neural network. The absence of clean data in real-world scenario pose a significant challenge on applicability of these algorithms.

%-------------------------------------------------------------------------
    \subsubsection{Unsupervised learning methods}
Unsupervised learning-based methods for point cloud denoising do not require ground-truth clean data. Instead, these methods leverage the inherent structure or distribution of the point cloud to guide the denoising process. Unsupervised methods show promise in scenarios where clean data is absent or hard to obtain.

Hermosilla \etal first introduced Total Denoising (TotalDn) \cite{hermosilla2019TotalDenoising} as an unsupervised learning approach for point cloud denoising, relying solely on noisy data without requiring clean ground truth. TotalDn approximates the underlying surfaces by regressing points from the distribution of unstructured total noise, utilizing a spatial prior term to refine the estimation of geometry. 

In DMRDenoise \cite{luo2020DMR}, an unsupervised version is proposed which utilizes a loss function that identify local neighborhoods using a probabilistic Gaussian mask on the k-nearest neighbors, which selectively retains points likely to represent the underlying surface. By leveraging an Earth Mover's Distance (EMD) assignment, it achieves a one-to-one correspondence between input and predicted points, aligning them to reduce noise within local neighborhoods.
This approach enhances robustness to noise and adapts well to varied surface geometries. However, the probabilistic masking and EMD calculation add computational complexity, which can slow down inference in dense or noisy point clouds. 

ScoreDenoise \cite{luo_score-based_2021} also introduced an unsupervised version that employs ensemble score function and an adaptive neighborhood-covering loss for model training.  
Score-u is probably the most relevant work to our method. However, the defined score in \cite{luo_score-based_2021} is only an displacement-alike version of the score function and there is no explicit formula relating the estimated score to the final denoising result. Also, the iterative process is computationally expensive, and can suffer from fluctuation, leading to perturbed and unstable solution.

Most recently, Noise4Denoise \cite{noise4Wang2024} method is proposed that use an additional doubly-noisy point cloud to learn noise displacement by comparing the two noise levels. This approach effectively leverages synthetic noise for training, allowing the model to estimate residuals without relying on clean data.
However, in practical applications, noise parameters are often unknown and variable, making it challenging to replicate the exact conditions assumed during training. This reliance on predefined noise characteristics can limit the model's applicability to real-world scenarios where noise distributions may differ significantly from synthetic settings. 
%-------------------------------------------------------------------------
\subsection{Unsupervised image denoising}
Recently unsupervised image denoising has made significant progress. Non-Bayesian methods include PURE \cite{luisier2010image}, SURE \cite{SURE2018} \textit{etc.}, which are based on various unbiased risk estimator under certain noise distribution. Other methods explore self-similarity in natural images \cite{xu2015patch, doi:10.1137/23M1614456} or exploits the statistical properties of noise to achieve denoising effect \cite{gravel2004method}.  

Noise2Noise \cite{2018Noise2NoiseLI} is a pioneering method that does not require clean data, but it requires multiple noisy versions of the same image for training. To address this limitation, methods such as Noise2Void \cite{2018Noise2VoidL}, Noise2Self \cite{2019Noise2SelfBD}, \textit{etc.}, have been developed, which use only a single noisy image. These methods are particularly important for practical applications where obtaining clean images or multiple noisy realizations of the same image is difficult or impossible. Neighbor2Neighbor \cite{huang2021neighbor2neighbor} proposed a two-step method with a a random neighbor sub-sampler that generates training  pairs and a denosing network. Kim \etal proposed Noise2Score\cite{kim_noise2score_2021}, a novel Bayesian framework for self-supervised image denoising without clean data. The core of Noise2Score is the usage of Tweedie's formula, which provides an explicit representation of the denoised image through a score function. Combined with score function estimation, Noise2Score can be applied to image denoising with any exponential family noise. Kim \etal also proposed the Noise Distribution Adaptive Self-Supervised Image Denoising method \cite{kim_noise_2022}, which further extends the application of Noise2Score by combining the Tweedie distribution with score matching. This enables adaptive handling of various noise distributions and dynamically adjusts the denoising process by estimating noise parameters. On the other hand, Xie \etal \cite{scoreXie2024} broadened the denoising scope of Noise2Score by allowing it to handle complex noise models, such as multiplicative and structurally correlated noise, demonstrating broad applicability to diverse noise models.

These development of unsupervised image denoising method motivate us to explore similar ideas in 3D point cloud denoising.




\section{Forestry Crane Simulator} \label{Sec:method}

\subsection{Kinematics}

Figure \ref{fig:crane_scematics} illustrates the schematic of the forestry crane. 
It has eight degrees of freedom (DoFs) $\mathbf{q}^\mathrm{T} = [\mathbf{q}_A^\mathrm{T},\mathbf{q}_U^\mathrm{T}] $ consisting of six actuated DoFs $\mathbf{q}_A^\mathrm{T} = [q_1,q_2,q_3,q_4,q_7,q_8]$ and two unactuated joints $\mathbf{q}_U^\mathrm{T} = [{q}_5,{q}_6]$. Note that there are two pairs of synchronized joints, i.e., the prismatic joint $q_4$ and the revolute joint $q_8$. \corr{In each pair of synchronized joints, the same input is applied to the corresponding actuators; for example, the joint angle $q_8$ at the left- and right-jaw of the grapple in Figure \ref{fig:crane_scematics}. }

%The characteristics of all joints are listed in Table \ref{tab:crane_joints}.  
\begin{figure}
    \centering
    \scalebox{0.8}{
    \includegraphics[trim=5cm 3.5cm 0cm 2cm,clip,scale=0.44]{figures/KinematicChain.pdf}
    }
    \caption{Schematic of the forestry crane \cite{ecker2022iterative}.}
    \label{fig:crane_scematics}
\end{figure}
\iffalse

    \begin{table}
        \caption[abc]{List of the forestry crane joints.}
        \label{tab:crane_joints}
        \begin{center} 
            
                \begin{tabular}{c c c c c}
                    \hline
                    Coordinate & Name & Actuated & Range & Unit\\
                    \hline
                     $q_1$ & Slewing joint & \checkmark & [-3.71,\:3.71] & \si{rad}\\ 
                     $q_2$ & Boom joint & \checkmark & [-1.2,\:1.56] & \si{rad} \\  
                     $q_3$ & Arm joint & \checkmark & [-0.91,\:4.6] & \si{rad} \\
                     $q_4$ & Prismatic joint & \checkmark & [0,\:4.47] & \si{m} \\
                     $q_5$ & Tip joint & \xmark & [-1.57,\:1.57] & \si{rad} \\
                     $q_6$ & Tilt joint & \xmark & [-0.79,\:2.36] & \si{rad} \\
                     $q_7$ & Rotate joint & \checkmark & $ [-\infty,\:\infty]$ & \si{rad}\\
                     $q_8$ & Grapple jaws & \checkmark & [0,\:3] & \si{rad}\\
                    \hline 
                \end{tabular}
        \end{center}        
    \end{table}
\fi    

The kinematics of the forestry crane are described by transformations from a coordinate Frame $\mathcal{F}_i$ attached to joint $i$ to a coordinate frame $\mathcal{F}_{i-1}$ attached to joint $i-1$
\begin{align}
	\vec{H}^i_{i-1} = \begin{bmatrix}
		\vec{R}_{i-1}^i & \vec{d}_{i-1}^i\\
		\transpose{\vec{0}} & 1
	\end{bmatrix}\in\mathcal{SE}(3) \Comma
\end{align}
where $\vec{R}_{i-1}^i\in\mathcal{SO}(3)$ and $\vec{d}_{i-1}^i\in\mathbb{R}^3$ are the three-dimensional rotation matrix and the three-dimensional translation vector, respectively. 
The coordinate frames are illustrated in Figure~\ref{fig:crane_scematics} according to the \textit{Denavit-Hartenberg (DH) convention} \cite{spong:2006}. 
\iffalse

    Note that frame $\mathcal{F}_{11}$ is defined by DH convention w.r.t. frame $\mathcal{F}_{8}$ instead of $\mathcal{F}_{10}$ due to the kinematic structure depicted in Figure \ref{fig:crane_scematics}.
    %, but rather from frame $\mathcal{F}_8$. 
    Hence, homogeneous transformations $\vec{H}_{i-1}^i$, $i=1,\dots,10,12$ and $\vec{H}_{8}^{11}$ can be described using four DH parameters $\theta_i$, $d_i$, $a_i$ and $\alpha_i$ as
    \begin{align}
    	\vec{H}_{i-1}^i = \vec{H}_{Rz}(\theta_i)\vec{H}_{Tz}(d_i)\vec{H}_{Tx}(a_i)\vec{H}_{Rx}(\alpha_i)\Comma
    \end{align}
    where $\vec{H}_{Ri}$ is a pure rotation around the $i$-axis and $\vec{H}_{Ti}$ is a pure translation in direction of the $i$-axis. 
    The transformation from $\mathcal{F}_j$ to $\mathcal{F}_i$, $0\leq i < j$ can be computed using
    \begin{align}
    	\vec{H}_i^j=\begin{cases}
    		\prod_{l=i+1}^j\vec{H}_{l-1}^l&,\text{ for }j\leq 10\\
    		\Big(\prod_{l=i+1}^8\vec{H}_{l-1}^l\Big)\vec{H}_{8}^{11}\vec{H}_{11}^j&,\text{ for }11\leq j\leq 12
    	\end{cases}\Comma
    \end{align}
    where $\prod_{l=i+1}^j\vec{H}_{l-1}^l$ being the identity for $j\leq i$.
    \begin{table}
        \caption{Denavit-Hartenberg parameters of the timber crane.}\label{tab:DHParams}
    	\centering
    	\begin{tabular}{c|cccc}
                \hline
    		$i$ & $\theta_i$ [rad] & $d_i$ [m] & $a_i$ [m] & $\alpha_i$ [rad]\\
    		\hline
    		1   &            $q_1$ & 2.425     & 0.1800      &$\pi/2$\\
    		2   &            $q_2$ & 0         & 3.4931    &0\\
    		3   &            $q_3$ & 0         & -0.3925   &$\pi/2$\\
    		4   &                0 & $q_4$ + 3.157     & 0         &0\\
    		5   &                0 & $q_4$     & 0         &$-\pi/2$\\
    		6   &            $q_5$ & 0         & -0.2130   &$-\pi/2$\\
    		7   &            $q_6$ & 0         & 0         &$-\pi/2$\\
    		8   &            $q_7$ & 0.578     & 0         &0\\
    		9   & $-\pi/2$ & 0         & 0.3402    &$\pi/2$\\
    		10  &           $\pi/2$ & 0         & 0.8566    &0\\
    		11  &  $\pi/2$ & 0         & 0.3248    &$\pi/2$\\
    		12  &           $\pi/2$ & 0         & 0.8566    &0\\
                 \hline
    	\end{tabular}
    \end{table}
\begin{table}[h]
    
    \caption{Denavit-Hartenberg parameters of the forestry crane.}\label{tab:DHParams}
        	\begin{center}
            	\begin{tabular}{c|cccc}
                        \hline
            		$i$ & $\theta_i$ in \SI{}{\radian} & $d_i$ in \SI{}{\meter} & $a_i$ in \SI{}{\meter} & $\alpha_i$ in \SI{}{\radian}\\
            		\hline
            		1   &            $q_1$ & 2.4     & 0.18      &$\pi/2$\\
            		2   &            $q_2$ & 0         & 3.5    &0\\
            		3   &            $q_3$ & 0         & -0.4   &$\pi/2$\\
            		4   &                0 & $q_4$ + 3.1     & 0         &0\\
            		5   &                0 & $q_4$     & 0         &$-\pi/2$\\
            		6   &            $q_5$ & 0         & -0.21   &$-\pi/2$\\
            		7   &            $q_6$ & 0         & 0         &$-\pi/2$\\
            		8   &            $q_7$ & 0.58     & 0         &0\\
                    %9   & $-\pi/2$ & 0         & 0.3402    &$\pi/2$\\
                    \hline
            	\end{tabular}
        	\end{center}
    
    \end{table}
\fi    
    %The DH parameters for the forestry crane are given in Table~\ref{tab:DHParams}. 

\iffalse
    Using the above kinematic relations, the wrist position of the grapple, $\mathbf{d}_g^\mathrm{T} = [g_x,g_y,g_z]$, is taken from  
    %Using this formalism the calculation of the center point coordinates $g_x, g_y$ and $g_z$ of the grapple reads as
    \begin{equation}
        \vec{H}_{0}^{8} = 
        \begin{bmatrix}
            \mathbf{R}_0^8 & \mathbf{d}_g \\
            \mathbf{0}^\mathrm{T} & 1
        \end{bmatrix} \:.
        \label{eq:transformation}
    \end{equation}
\fi    
%In the used scenarios the grapple is already close to the logs and the crane is unfolded, therefore the special hydraulic kinematic is neglected. 
%Instead, all rotary joints are driven with velocity-controlled rotary motors and the telescopic boom is driven by a linear motor.

\subsection{Simulator}
\label{sec: b simulator}

The system dynamics and contacts with the environment are simulated using the open-source MuJoCo \cite{todorov2012mujoco} physics engine. 
An example of the simulated environment is illustrated in Figure \ref{fig: example mujoco}. 
The assembled model of the forestry crane (including the truck) consists of $38$ rigid bodies and $10$ active joints\corr{, including two pairs of synchronized joints}. The total operating weight of the forestry crane is approximately \SI{1981}{\kilo\gram}. 
On standard forestry cranes, hydraulic actuators powered by a pump that is driven by a combustion engine drive the slewing ($q_1$), boom ($q_2$), arm ($q_3$), and prismatic ($q_4$) joints, respectively. 
In order to simplify the model for training purposes, the hydraulic actuators are not explicitly modeled. 
%Instead, two linear motors are modeled for the synchronized joint $q_4$, and rotational motors are utilized to drive other actuated joints. 
\corr{We assume an (ideal) underlying velocity controller, with the reference velocity for the prismatic joint $q_4$ and the reference rotational velocity for the other joints as inputs. }
Thus, in the simulation environment, the grasping controller for the modeled forestry crane is a fine-tuned PID controller with reference velocities for the actuated joints $\mathbf{q}_A$.  

The wood log position is randomized in a reachable region of the forestry crane, depicted as the yellow region in Fig. \ref{fig: example mujoco}. 
The center of this region is approximately \SI{6.5}{\meter} from the crane's base. 
Additionally, logs are modeled as cylinders with a length of $\SI{2.75}{\meter}$, and the log's diameter varies in the range of $[0.3,0.8] \SI{}{\meter}$. 
In order to prevent overfitting during the training process, the slew angle of the crane is varied in the range $[-2\pi/3, -\pi/3] \SI{}{\radian}$. 
The 6-dimensional contact forces between the grapple and the wood log are computed using the signed distance field (SDF) collision primitive \cite{reiner2011interactive}. 
This is particularly important to maintain the robustness of the simulation since the inner and outer jaw of the grapple have curvy shapes. 










\section{Hierarchical Clustered 
Federated Learning } \label{sec:algorithm} 

\subsection{Overview}

\begin{figure}[tb!]
\centerline{\includegraphics[width=1\linewidth]{Figure/Algorithm.png}}
\caption{Overview of the proposed framework \algname.}
\label{fig:overview}
\end{figure}

\begin{algorithm}
\begin{algorithmic}[1]
\caption{\algname for satellite networks}\label{alg:decentralized_FL}
    \REQUIRE{Information of satellite networks, $K$, $C$}
    \ENSURE{Global model $w_{G}$}

\STATE Conduct satellite-clustered parameter server selection algorithm;  \label{line:1}

\STATE \COMMENT{Satellite client initialization model parameters}  \label{line:2}
\FOR {$\forall$ satellites}
     \STATE Initialize global model parameter $w_0$;
\ENDFOR  \label{line:5}

%\STATE All satellites: Initialize global model parameter $w^{(a)}_0$;

\FOR {each FL round $m \in M$}  \label{line:6}
    \STATE \COMMENT{Train local model (in-orbit computing)}
    \FOR {each satellite $i \in C$ in parallel}
        \STATE $w_{m,\lambda+1}^{i} = w_{m,\lambda}^{i} - \eta \nabla \tilde{f}_i(w_{m,\lambda}^{i})$;
    \ENDFOR \label{line:9}
    %\STATE \note{Conduct adaptive weight quantification algorithm to obtain $w{_{m}^{i}}^{\prime}$;}
    \STATE \COMMENT{Aggregate satellite cluster models}
    \FOR {each satellite $i \in C^k$ in the same cluster}  \label{line:12}
        \STATE $w_{m+1} = w_m + \sum_{i \in C^k_i} p_i w_m$;   \label{line:13}
    \STATE \COMMENT{Check if it is necessary to reassemble clusters}  \label{line:14}
        \STATE Calculate dropout rate: $d_r = \frac{C^d}{C^k}$
        \IF {$d_r > Z$}
            \STATE Re-cluster the satellites
        \ENDIF  \label{line:18}
    \ENDFOR
\ENDFOR  
\STATE \COMMENT{Aggregate global model at the ground station}  \label{line:21}
\FOR {each PS $G_{k}, k \in {K}$ in different satellite clusters}
    \STATE $w_{G} =  \sum_{k \in {K}} \frac{D_k}{D} w_m^k$;  
\ENDFOR  \label{line:23}
\RETURN $w_{G}$.
\end{algorithmic}
\end{algorithm}

\figurename~\ref{fig:overview} illustrates the flowchart of our proposed \algname. The clustering FL clustering process in \algname includes two stages: \textbf{satellite cluster aggregation stage} (Step 1-3) and \textbf{ground station aggregation stage} (Step 4). In the satellite cluster aggregation stage, a clustering algorithm is introduced to divide the satellites into distinct clusters. Within each satellite cluster $K_i^a$, the algorithm selects a satellite near the cluster center with strong communication capabilities to act as the PS. The PS is responsible for aggregating model parameters from satellites within its cluster and establishing communication with the corresponding ground stations (Step 1). During the local training process, satellites may dynamically join or leave a cluster, necessitating re-aggregation. To address this issue, MAML is introduced to adjust the initial model parameters of the newly joined satellites, allowing them to better adapt to the tasks of the new cluster (Step 2). This method accelerates the overall convergence of the satellite PS aggregation process. Following each training round, the PS combines parameters from all satellites within its cluster and distributes the updated, aggregated parameters back to them (Step 3).

%This approach facilitates the acceleration of the overall convergence of the satellite PS aggregation process. After each training round, the PS aggregates the parameters of the satellites in its cluster and distributes the \note{Do we need to add ''updated" here?} aggregated parameters to them (Step 3). 

After a specified number of training rounds in the satellite cluster, the ground station aggregation stage starts. In this stage, the ground station communicates with visible satellite clusters to aggregate their model parameters of the respective satellite clusters they are affiliated. Finally, the ground station returns the trained model parameters to the respective satellite clusters (Step 4).

As detailed in Algorithm~\ref{alg:decentralized_FL},  we introduce a satellite-clustered parameter server selection algorithm to partition the original satellite network into distinct satellite clusters based on the satellite network information. For each cluster, the algorithm selects a satellite near the cluster center with robust communication capabilities as the PS (line \ref{line:1}). Then we initialize the global model parameters $w_0$ for all satellite clients within each cluster (lines \ref{line:2}-\ref{line:5}). During each round of FL aggregation, the local satellite client first performs local training to update the global model parameters $w_{m,\lambda+1}^{i}$ after training round $\lambda$ (lines \ref{line:6}-\ref{line:9}). These parameters are then transmitted to their cluster's PS for aggregation. 

After $m$ rounds of training in each satellite cluster, each PS forwards its aggregated parameters to its associated ground station for global aggregation, producing the updated model parameters $w_{m+1}$ (lines \ref{line:12}-\ref{line:13}). During global aggregation, satellite clusters monitor whether the number of dropped-out satellites $C^d$ exceeds a predefined threshold, triggering re-clustering when necessary (lines \ref{line:14}-\ref{line:18}). Finally, ground station broadcasts the global parameters to all affiliated satellites in their clusters, completing the hierarchically clustered FL process (lines \ref{line:21}-\ref{line:23}). 




%As shown in Algorithm~\ref{alg:decentralized_FL}, based on the satellite network information, we introduce a satellite-clustered parameter server selection algorithm to partition the original satellite network into distinct clusters. Within each satellite cluster, the algorithm selects a satellite near the cluster center with strong communication capabilities to serve as the PS (line \ref{line:1}). Then the global model parameters are initialized for all satellite clients within the satellite cluster (lines \ref{line:2}-\ref{line:5}). During each round of FL aggregation, the local satellite client first performs local training and obtains the trained global model parameters $w_{m,\lambda+1}^{i}$ after training round $\lambda$ (lines \ref{line:6}-\ref{line:9}). The satellite clients then transmit their global parameters $w_{m,\lambda+1}^{i} $ to satellite PS for aggregate global model parameters. After completing $m$ rounds of training in each satellite cluster, the global model parameters are transmitted to the ground station associated with the satellite PS for final aggregation, yielding updated aggregated global model parameters $w_{m+1}$ (lines \ref{line:12}-\ref{line:13}). During global aggregation, satellite clusters need to assess whether the number of dropped-out satellites $C^d$ exceeds a critical threshold, necessitating re-clustering (lines \ref{line:14}-\ref{line:18}). Finally, the ground station broadcasts the global parameters to all affiliated satellites in the cluster, completing the hierarchically clustered FL process (lines \ref{line:21}-\ref{line:23}). \note{please double check this paragraph to avoid the meaning changes after my rewrite}

To accelerate the convergence of the global model, we assign weights to clients based on the quality of their model updates. The quality is evaluated using the loss value of the client's local model. Let $L_i$ denote the loss value of the client $i$. The weight $p_i$ assigned to client $i$ is given by:
\begin{equation}
p_i = \frac{\frac{1}{L_i}}{\sum_{ i \in C^k} \frac{1}{L_i}}
\label{eq:p}
\end{equation}

\begin{comment}
The local updates from the satellite clients are then weighted and aggregated according to the weight parameter $p_i$ by:
\begin{equation}
w_{m+1} = w_m + \sum_{i \in U_k} p_i Q_m^i(w_{m+1}^{i} - w_m)
\end{equation}
\end{comment}
%If $p_i$ is 0, the client has not been selected by PS.

\subsection{Satellite-clustered parameter server selection algorithm}

We introduce a satellite-clustered parameter server selection algorithm that partitions the original satellite network topology into a predefined number of clusters $K$, optimizing the clustering process. Our algorithm iteratively refines the cluster centroids and the membership of associated satellites. Initially, $K$ centroids are randomly selected from the satellite location data. These locations are typically derived from geographic coordinates based on the satellite location parameters, i.e., inclination and orbital altitude. Each satellite is assigned to the nearest cluster centroid using the Euclidean distance metric, thereby forming initial clusters. The Euclidean distance between a satellite position vector
$\mathbf{C}^i = \{C_{1}^i, C_{2}^i, \ldots, C_{n}^i \}$ and a centroid $\mathbf{K}^j = \{K_{1}^j, K_{2}^j, \ldots, K_{n}^j \}$ is calculated as:
\begin{equation}
d(\mathbf{C}^i, \mathbf{K}^j) = \sqrt{\sum_{k=1}^{n} (C_{k}^i - K_{k}^j)^2}
\end{equation}

In the next update step, our algorithm recalculates the centroids by computing the mean position of all satellites assigned to each cluster. This process effectively repositions the centroids to more accurately represent the distribution of their associated satellites. For each cluster $K_k^i$, the new centroid $\mathbf{K}^j$ is obtained by:
\begin{equation}
\mathbf{K}^j = \frac{1}{|K^j|} \sum_{\mathbf{C}^i \in K^j} \mathbf{C}^i
\end{equation}
where $|K^j|$ represents the number of satellites in cluster $K^j$. The iterative process continues until the centroids stabilize, indicating their positions no longer change significantly between iterations. This indicates that the algorithm has converged to a local optimum. The convergence criterion is given by:
\begin{equation}
\sum_{j=1}^{|K|} \|\mathbf{K}_{\text{new}}^j - \mathbf{K}_{\text{old}}^j\|^2 < \epsilon
\end{equation}
where $|K|$ represents the number of clusters, and $\epsilon$ is a small positive number indicating stability in centroid positions.
The satellite nearest to the cluster centroid is designated as the PS for the respective cluster.


\subsection{Meta-learning-driven satellite re-clustering algorithm}

In dynamic satellite federated learning, the diverse training objectives of satellite clients, combined with their frequent network participation changes, can hinder model convergence and increase resource consumption. As a result, achieving acceptable performance requires substantial time and a large number of data samples.
%In dynamic satellite federated learning, the diverse training goals of satellite clients, coupled with their frequent network participation changes, can impede model convergence, resulting in heightened resource utilization. This necessitates a significant investment of time and data samples to attain acceptable performance. %the different training objectives of satellite clients, along with their frequent joining or leaving the network, can slow down model convergence, leading to increased resource consumption. This requires a large amount of time and data samples to achieve acceptable results. 

To address this challenge, we propose a satellite re-clustering algorithm based on meta-learning, extending the original satellite-clustered parameter server selection algorithm. When a new satellite joins the network, it inherits model updates from the head node of a specified cluster, rather than starting training from scratch. The core idea of the MAML approach is to identify a set of meta-initialization parameters that enable the model to achieve strong performance with just one or two gradient updates, even with a small number of new task examples.

First, we sample satellite clients from different clusters denoted as $ S = \{S_1, S_2, \dots, S_n\} $. Each satellite client is assigned a task $\textit{TK}_i$, which consists of a dataset $D_i$  and a loss function $L_{S_i}(w)$. The objective is to minimize the loss of the model on the task $\textit{TK}_i$. Then, an inner-loop adaptation is performed for each selected satellite node to fine-tune the global model $w$ by:
\begin{equation}
w_i' = w - \alpha \nabla_{w} L_{S_i}(w)
\end{equation}
where $\alpha$ is the local learning rate. Finally, an outer-loop meta-update is applied to aggregate the model updates from different satellite nodes, updating the global initialization by:
\begin{equation}
w^{t+1} = w^{t} - \beta \sum_{i \in S} \nabla_{w} L_{S_i}(w_i')
\end{equation}
where $\beta$ is the meta-learning rate, $w^{t}$ is the parameter of the current global model.


\section{Experiment}
In this section, we conduct extensive experiments to evaluate the performance of various LLMs on our Hellaswag-Pro benchmark. Our study is guided by three key research questions:
\textbf{RQ1}: How do different LLMs perform across all variants?
\textbf{RQ2}: What is the relative difficulty of different variants?
\textbf{RQ3}: How robust are LLMs to diverse prompts during evaluation?

\subsection{Experiment Setup} 
\subsubsection{Model Selection and Implementation Details}
We select 41 representative commercial and open-source models, including English LLMs, such as GPT-4o, Claude-3.5-Sonnet, Gemini-1.5-Pro,Mistral series, Llama3 series and Chinese LLMs, like Qwen-Max,  Qwen2.5 series, InternLM-2.5 series, Yi-1.5 series, Baichuan-2 series and DeepSeek series.

We integrate both Chinese HellaSwag and HellaSwagPro into the lm-evaluation-harness platform. For the open-source models, we use the default settings of lm-evaluation-harness: do\_sample is set to false and the temperature is set to the default value of the hugging-face library. For the closed-source models, we set the temperature to 0.7. In addition, we set the maximum output length to 1024.

\subsubsection{Prompt Strategy}
Taking into account the influence of language and shot, we design 9 prompting strategies, including Direct, CN-CoT, EN-CoT, CN-XLT and EN-XLT. The last four setups include both zero-shot and few-shot variants.\footnote{
For open-source models, Direct adopts an approach similar to the official implementation of HellaSwag, computing the log-likelihood for each option and selecting the one with the highest log-likelihood. And we report normalized accuracy that accounts for the impact of option length. Other prompting strategies use a generation setup and report accuracy based on exact match.}
\textbf {(1)Direct}: LLMs makes the selection directly without any CoT process.
\textbf{(2)CN-CoT}: LLMs performs CoT in Chinese, regardless of dataset language.
\textbf{(3)EN-CoT}: Similar to CN-CoT, but CoT is conducted in English. 
\textbf{(4)CN-XLT}: LLMs are instructed to first translate English questions and options to Chinese, and then reason in Chinese.
\textbf{(5)EN-XLT}: Similar to CN-XLT, but translates from Chinese dataset to English and reasons in English. 

%\textbf {CN-CoT}: LLMs perform Chinese reasoning and then output the answer and 3 shots are provided.
%\textbf {CN-CoT}: Similar as CNCoTFewShot without any shots.
%\textbf {EN-CoT}: The reasoning process in English is executed and then the answer is output and 3 shots are provided.
%\textbf {CN-XLT}: Inspired by this, we instruct LLMs to translate questions in Chinese and then output the answer after performing reasoning in Chinese too. And 3 shots are provided.
%\textbf {EN-XLT}: Inspired by this, we instruct LLMs to translate questions in Englsih and then output the answer after performing reasoning in Englsih too. Three shots are provided.

\subsubsection{Evaluation metric}

To comprehensively evaluate the robustness of each LLM, we consider four metrics: 
% Original Accuracy (\textbf{OA}), Average Robust Accuracy (\textbf{ARA}), Robust Loss Accuracy (\textbf{RLA}), and  Consistent Robust Accuracy (\textbf{CRA}).
\noindent %
\textbf{- Original Accuracy (OA)} measures accuracy on original problems.
\begin{equation}\label{eq1}
OA=\frac{\sum_{(x, y) \in D} \mathds{1}[L M(x), y]}{|D|}.
\end{equation}
\noindent %
\textbf{- Average Robust Accuracy  (ARA)} represents average accuracy across all variants, gauging overall performance on the robustness tasks.
\begin{equation}\label{eq2}
ARA=\frac{\sum_{\left(x^{\prime}, y^{\prime}\right) \in D_{R}} \mathds{1}\left(L M\left(x^{\prime}, y^{\prime}\right)\right.}{\left|D_{R}\right|}.
\end{equation}

\noindent %
\textbf{- Robust Loss Accuracy (RLA)} is the difference between ARA and OA, indicating performance degradation on robustness data versus original data.
%\begin{tiny}
%\begin{equation}\label{eq3}
%RLA=\frac{\sum_{\left(x^{\prime}, y^{\prime}\right) \in D_{R}} %\mathds{1}\left(L M\left(x^{\prime}, y^{\prime}\right)\right.}{\left|D_{R}\right|}-\frac{\sum_{(x, y) \in D}\mathds{1}[L M(x), y]}{|D|}
%\end{equation}
%\end{tiny}
\begin{equation}\label{eq3}
RLA= OA - ARA.
\end{equation}
\noindent %
\textbf{- Consistent Robust Accuracy (CRA)} shows accuracy when the model correctly answers both original and variant data, reflecting the model do understand the problem.
% consistency in problem-solving.
\begin{equation}\label{eq4}
CRA=\frac{\sum_{x, y, x^{\prime}, y^{\prime}}\mathds{1}[L M(x), y] \cdot \mathds{1}[L M(x^{\prime}), y^{\prime}]}{\left|D_{R}\right|}.
\end{equation}
For all equation above, $D$ denotes the original dataset, where $x$ represents the input question and options, and $y$ represents the correct label, while $D_{R}$ is the robust dataset with $x^{\prime}$ and $y^{\prime}$ representing similar to $x$ and $y$.


\begin{table*}[ht]
\centering
\setlength{\tabcolsep}{5pt}
% \footnotesize
\scalebox{0.6}{
% Please add the following required packages to your document preamble:
% \usepackage{multirow}
% \usepackage[table,xcdraw]{xcolor}
% Beamer presentation requires \usepackage{colortbl} instead of \usepackage[table,xcdraw]{xcolor}
% Please add the following required packages to your document preamble:
% \usepackage{multirow}
% \usepackage[table,xcdraw]{xcolor}
% Beamer presentation requires \usepackage{colortbl} instead of \usepackage[table,xcdraw]{xcolor}
\begin{tabular}{ccccccccccccc}
\hline
\multicolumn{1}{c|}{{ }}& \multicolumn{4}{c|}{Chinese}& \multicolumn{4}{c|}{English}& \multicolumn{4}{c}{AVG}\\ \cline{2-13} 
\multicolumn{1}{c|}{\multirow{-2}{*}{{ Model}}} & { OA(\%)$\uparrow$}& { ARA(\%)$\uparrow$} & {RLA(\%)$\downarrow$}& \multicolumn{1}{l|}{{CRA(\%)$\uparrow$}} & { OA(\%)$\uparrow$}& { ARA(\%)$\uparrow$} & { RLA(\%)$\downarrow$}& \multicolumn{1}{l|}{{CRA(\%)$\uparrow$}} & {OA(\%)$\uparrow$}& { ARA(\%)$\uparrow$} & {RLA(\%)$\downarrow$}& { CRA(\%)$\uparrow$} \\ \hline
\multicolumn{1}{c|}{{ Human}} & 96.41& 97.79& -1.38 & \multicolumn{1}{l|}{92.03}& 95.56& 96.04& -0.48 & \multicolumn{1}{l|}{90.02}& 95.99 & 96.92 & -0.93& 91.03 \\ \hline
\multicolumn{13}{c}{\textit{Close-source LLMs}}\\ 
\multicolumn{1}{c|}{{ GPT-4o}}& { 91.37} & { 81.97} & { 9.40}& \multicolumn{1}{l|}{{ 75.55}} & { \textbf{88.63}} & { \textbf{70.17}} & { \textbf{18.46}} & \multicolumn{1}{l|}{{ \textbf{63.06}}} & { 90.00} & { \textbf{76.07}} & { \textbf{13.93}} & { \textbf{69.31}} \\
\multicolumn{1}{c|}{{ Claude3.5}}& { \textbf{95.37}} & { 80.15} & { 15.22} & \multicolumn{1}{l|}{{ 75.04}} & { 85.11} & { 66.02} & { 19.08} & \multicolumn{1}{l|}{{ 57.20}} & { 90.24} & { 73.09} & { 17.15} & { 66.12} \\
\multicolumn{1}{c|}{{ Gemini-1.5-Pro}}& { 90.62} & { 78.36} & { 12.26} & \multicolumn{1}{l|}{{ 70.48}} & { 87.75} & { 60.74} & { 27.01} & \multicolumn{1}{l|}{{ 58.27}} & { 89.19} & { 69.55} & { 19.63} & { 64.38} \\
\multicolumn{1}{c|}{{ Qwen-Max}}& { 93.50} & { \textbf{84.82}} & { \textbf{8.68}}& \multicolumn{1}{l|}{{ \textbf{78.91}}} & { 87.60} & { 62.61} & { 24.99} & \multicolumn{1}{l|}{{ 59.65}} & { \textbf{90.55}} & { 73.72} & { 16.83} & { 69.28} \\ \hline
\multicolumn{13}{c}{\textit{Chinese open-source LLMs}} \\ 
\multicolumn{1}{c|}{{ Qwen2.5-0.5B}}& { 60.75} & { 45.18} & { \textbf{15.57}} & \multicolumn{1}{l|}{{ 28.70}} & { 49.50} & { 38.21} & { \textbf{11.29}} & \multicolumn{1}{l|}{{ 20.57}} & { 55.13} & { 41.70} & { \textbf{13.43}} & { 24.64} \\
\multicolumn{1}{c|}{{ Qwen2.5-1.5B}}& { 63.25} & { 46.16} & { 17.09} & \multicolumn{1}{l|}{{ 29.89}} & { 56.88} & { 39.57} & { 17.30} & \multicolumn{1}{l|}{{ 23.48}} & { 60.06} & { 42.87} & { 17.20} & { 26.69} \\
\multicolumn{1}{c|}{{ Qwen2.5-3B}}& { 67.50} & { 48.75} & { 18.75} & \multicolumn{1}{l|}{{ 33.79}} & { 61.75} & { 39.98} & { 21.77} & \multicolumn{1}{l|}{{ 25.75}} & { 64.63} & { 44.37} & { 20.26} & { 29.77} \\
\multicolumn{1}{c|}{{ Qwen2.5-7B}}& { 67.63} & { 50.59} & { 17.04} & \multicolumn{1}{l|}{{ 35.62}} & { 65.63} & { 43.93} & { 21.70} & \multicolumn{1}{l|}{{ 30.77}} & { 66.63} & { 47.26} & { 19.37} & { 33.20} \\
\multicolumn{1}{c|}{{ Qwen2.5-14B}} & { 69.00} & { 51.41} & { 17.59} & \multicolumn{1}{l|}{{ 35.84}} & { 68.50} & { 45.20} & { 23.30} & \multicolumn{1}{l|}{{ 32.12}} & { 68.75} & { 48.30} & { 20.45} & { 33.98} \\
\multicolumn{1}{c|}{{ Qwen2.5-32B}} & { 69.75} & { 53.11} & { 16.64} & \multicolumn{1}{l|}{{ 37.54}} & { 70.00} & { 46.10} & { 23.90} & \multicolumn{1}{l|}{{ 32.68}} & { 69.88} & { 49.61} & { 20.27} & { 35.11} \\
\multicolumn{1}{c|}{{ Qwen2.5-72B}} & { \textbf{70.87}} & { \textbf{54.75}} & { 16.12} & \multicolumn{1}{l|}{{ \textbf{39.64}}} & { \textbf{72.00}} & { \textbf{47.75}} & { 24.25} & \multicolumn{1}{l|}{{\textbf{ 35.12}}} & { \textbf{71.44}} & { \textbf{51.25}} & {20.19} & { \textbf{37.38}} \\ \hdashline[0.5pt/5pt]
\multicolumn{1}{c|}{{ Baichuan2-7B}}& { 67.00} & { 46.16} & { 20.84} & \multicolumn{1}{l|}{{ 31.50}} & { 60.62} & { 39.04} & { 21.58} & \multicolumn{1}{l|}{{ 25.21}} & { 63.81} & { 42.60} & { 21.21} & { 28.36} \\
\multicolumn{1}{c|}{{ Baichua2-13B}}& { 69.13} & { 46.98} & { 22.15} & \multicolumn{1}{l|}{{ 33.45}} & { 64.62} & { 38.82} & { 25.80} & \multicolumn{1}{l|}{{ 26.07}} & { 66.88} & { 42.90} & { 23.97} & { 29.76} \\ \hdashline[0.5pt/5pt]
\multicolumn{1}{c|}{{ DeepSeek-7B}} & { 68.13} & { 47.96} & { 20.17} & \multicolumn{1}{l|}{{ 33.30}} & { 63.38} & { 40.39} & { 22.99} & \multicolumn{1}{l|}{{ 26.70}} & { 65.76} & { 44.18} & { 21.58} & { 30.00} \\
\multicolumn{1}{c|}{{ DeepSeek-67B}}& { 71.50} & { 49.21} & { 22.29} & \multicolumn{1}{l|}{{ 35.89}} & { 71.37} & { 40.63} & { 30.75} & \multicolumn{1}{l|}{{ 29.71}} & { 71.44} & { 44.92} & { 26.52} & { 32.80} \\ \hdashline[0.5pt/5pt]
\multicolumn{1}{c|}{{ InternLM2.5-1.8B}}& { 61.62} & { 42.07} & { 19.55} & \multicolumn{1}{l|}{{ 26.99}} & { 55.37} & { 38.46} & { 16.91} & \multicolumn{1}{l|}{{ 22.61}} & { 58.50} & { 40.27} & { 18.23} & { 24.80} \\
\multicolumn{1}{c|}{{ InternLM2.5-7B}}& { 67.25} & { 49.77} & { 17.48} & \multicolumn{1}{l|}{{ 34.57}} & { 69.50} & { 40.89} & { 28.61} & \multicolumn{1}{l|}{{ 29.75}} & { 68.38} & { 45.33} & { 23.04} & { 32.16} \\
\multicolumn{1}{c|}{{ InternLM2.5-20B}} & { 67.37} & { 48.08} & { 19.29} & \multicolumn{1}{l|}{{ 33.21}} & { 73.62} & { 41.11} & { 32.51} & \multicolumn{1}{l|}{{ 31.23}} & { 70.50} & { 44.60} & { 25.90} & { 32.22} \\ \hdashline[0.5pt/5pt]
\multicolumn{1}{c|}{{ Yi-1.5-6B}} & { 67.00} & { 49.59} & { 17.41} & \multicolumn{1}{l|}{{ 34.27}} & { 64.38} & { 39.37} & { 25.01} & \multicolumn{1}{l|}{{ 26.62}} & { 65.69} & { 44.48} & { 21.21} & { 30.45} \\
\multicolumn{1}{c|}{{ Yi-1.5-9B}} & { 68.50} & { 50.18} & { 18.32} & \multicolumn{1}{l|}{{ 35.55}} & { 66.37} & { 39.58} & { 26.79} & \multicolumn{1}{l|}{{ 27.48}} & { 67.44} & { 44.88} & { 22.56} & { 31.52} \\
\multicolumn{1}{c|}{{ Yi-1.5-34B}}& { 71.00} & { 52.23} & { 18.77} & \multicolumn{1}{l|}{{ 38.09}} & { 71.00} & { 40.75} & { 30.25} & \multicolumn{1}{l|}{{ 29.91}} & { 71.00} & { 46.49} & { 24.51} & { 34.00} \\ \hline
\multicolumn{13}{c}{\textit{English open-source LLMs}} \\ 
\multicolumn{1}{c|}{{ Llama3-8B}} & { 59.13} & { 46.62} & { 12.51} & \multicolumn{1}{l|}{{ 28.23}} & { 66.25} & { 40.21} & { 26.04} & \multicolumn{1}{l|}{{ 27.34}} & { 62.69} & { 43.42} & { 19.27} & { 27.79} \\
\multicolumn{1}{c|}{{ Llama3-70B}}& { 65.75} & { 48.63} & { 17.12} & \multicolumn{1}{l|}{{ 32.70}} & { \textbf{72.50}} & { 41.27} & { 31.23} & \multicolumn{1}{l|}{{\textbf{ 30.63}}} & {\textbf{ 69.13}} & { 44.95} & { 24.18} & { 31.67} \\ \hdashline[0.5pt/5pt]
\multicolumn{1}{c|}{{ Mistral-7B-v0.2}} & { 57.75} & { 46.25} & { \textbf{11.50}} & \multicolumn{1}{l|}{{ 27.57}} & { 67.50} & { \textbf{41.52}} & { 25.98} & \multicolumn{1}{l|}{{ 28.93}} & { 62.63} & { 43.88} & { 18.74} & { 28.25} \\
\multicolumn{1}{c|}{{ Mixtral-8x7B-v0.1}} & { 63.62} & { 46.80} & { 16.82} & \multicolumn{1}{l|}{{ 30.82}} & { 69.75} & { 41.21} & { 28.54} & \multicolumn{1}{l|}{{ 29.39}} & { 66.69} & { 44.01} & { 22.68} & { 30.11} \\
\multicolumn{1}{c|}{{ Mixtral-8x22B-v0.1}}& { 66.00} & {\textbf{ 50.73}} & { 15.27} & \multicolumn{1}{l|}{{ \textbf{34.32}}} & { 72.12} & { 41.25} & { 30.87} & \multicolumn{1}{l|}{{ 30.61}} & { 69.06} & { \textbf{45.99}} & { 23.07} & { \textbf{32.47}} \\ \hdashline[0.5pt/5pt]
\multicolumn{1}{c|}{{ Gemma-2-2B}}& { 61.88} & { 45.38} & { 16.51} & \multicolumn{1}{l|}{{ 29.02}} & { 59.62} & { 39.13} & { \textbf{20.50}} & \multicolumn{1}{l|}{{ 24.88}} & { 60.75} & { 42.25} & {\textbf{ 18.50}} & { 26.95} \\
\multicolumn{1}{c|}{{ Gemma-2-9B}}& { \textbf{69.13}} & { 46.75} & { 22.38} & \multicolumn{1}{l|}{{ 33.29}} & { 64.88} & { 39.80} & { 25.08} & \multicolumn{1}{l|}{{ 26.91}} & { 67.01} & { 43.28} & { 23.73} & { 30.10} \\
\multicolumn{1}{c|}{{ Gemma-2-27B}} & { 63.38} & { 48.52} & { 14.86} & \multicolumn{1}{l|}{{ 31.96}} & { 71.88} & { 40.91} & { 30.97} & \multicolumn{1}{l|}{{ 30.25}} & { 67.63} & { 44.71} & { 22.92} & { 31.11} \\ \hline
\end{tabular}
}
\caption{TODO: bolded is not result. Results of existing LLMs on our HellaSwag-Pro dataset using \textbf{Direct} prompt. ``AVG'' indicates the average performance of each model on Chinese and English parts of the dataset.
The best results for each metric in each model category are \textbf{bolded}. }
\label{tab:main experiment.}
\end{table*}

\subsection{Model Performance (RQ1)}
\paragraph{Overall Performance}
Table \ref{tab:main experiment.} provides a comprehensive evaluation of various LLMs across four performance metrics\footnote{The results of instruct/chat models of Qwen2.5, Llama3 and Mixtral latest series are shown in Appendix.}. The main observations are as follow:
\begin{itemize}[leftmargin=*,topsep=0pt]
% \setlength{}{0}
    \item Upon evaluating all available models, we found that all performed well in overall accuracy (e.g., GPT-4 scored 90.00 in AVG OA, Claude 3.5 scored 90.24 in AVG OA). However, all models struggled with variations of the questions, as evidenced by a positive RLA value for each model. In contrast, humans received a negative RLA value, suggesting that the question variants were not more challenging than the originals. This disparity further illustrates that current LLMs lack a true understanding of the reasoning process and can easily be misled by question variants.
    \item When comparing open-source and close-source models, the close-source models demonstrate stronger capabilities in both OA and ARA scores, similar to most existing benchmarks. Overall, the RLA values for close-source models are also smaller, indicating that they are more robust in commonsense reasoning tasks compared to open-source models.
    \item When we compare models within the same series (e.g., Qwen, Llama), we observe that larger models often achieve higher scores on OA, ARA, and CRA. However, they are also more susceptible to variations, i.e., they have higher RLA values, a phenomenon particularly evident in English datasets. We attribute this phenomenon to the fact that larger models, compared to smaller ones, may have memorized more data, allowing them to rely on memorization to solve some problems more easily and making them more prone to the influence of variations~\cite{}.
\end{itemize}
% 1. When evaluating all available models, We find although 
% 2. When comparing the opensource LLMs and close source LLMs, 
% 3. When looking into each serious details
% \noindent
% \textbf{Overall Model Performance.}
% 1. close-source > open-source 2. the large the better 3. all have a performance decline when meeting varients.

% To evaluate the performance of various models, we observed patterns consistent with current mainstream trends: closed-source models generally outperform open-source models across metrics. 
% For instance, the closed-source model GPT-4o achieved scores of 90.00 in OA, 76.07 in ARA, and 69.31 in CRA, whereas the open-source model Qwen2.5-72B scored 71.44, 51.25, and 37.38, respectively. 
% Furthermore, within each model series, performance tends to improve with larger model sizes. 
% Nevertheless, even the strongest closed-source models struggle with variations in questions, as indicated by positive values in RLA for all models. In contrast, human performance yields a negative RLA value, highlighting that current LLMs do not genuinely grasp the reasoning process and are prone to falling into traps set by question variants. 
% This suggests that there is still significant room for improvement in developing models that can robustly understand and reason through complex linguistic challenges.
% It reveals a consistent pattern across Chinese, English, and average scores, with close-sourced LLMs generally outperforming open-sourced models. 
% However, all models exhibit a significant drop in performance when faced with robust variants, as indicated by RLA and CRA. Among closed-source models, GPT-4o demonstrates the highest ARA of 76.07\% in average scores, demonstrating its overwhelming superiority. Among open-sourced models, larger models tend to perform better, with Qwen2.5-72B achieving the highest OA (71.44\%) and ARA (51.25\%) in the average scores. However, even these top performers still struggle with robustness, as evidenced by the substantial RLA of 13.93\% for GPT-4o and 20.19\% for Qwen2.5-72B. Interestingly, some English open-sourced models, such as Llama3-70B and Mixtral-8x22B-v0.1, show competitive performance in English tasks but lag in Chinese tasks, highlighting the importance of language-specific training.

% \noindent
% \textbf{Chinese Models vs English Models.}
% Chinese models generally demonstrate higher OA in Chinese tasks compared to English tasks, with Qwen-Max achieving 93.50\% OA in Chinese versus 87.60\% in English. Conversely, English models tend to perform better in English tasks, exemplified by Llama3-70B's 72.50\% OA in English compared to 65.75\% in Chinese. 
% However, both Chinese and English models exhibit important drops in ARA across languages, indicating challenges in maintaining performance when faced with variations. This trend suggests that while models may excel in their primary language, they struggle with robustness across linguistic boundaries. 
% Notably, larger models tend to achieve higher ARA scores but also experience more substantial RLA, as seen with Qwen2.5-0.5B (41.70\% ARA, 13.43\% RLA in total) and Qwen2.5-72B (51.25\% ARA, 20.19\% RLA in total). 
% This pattern indicates that while increased model size enhances overall performance, it doesn't necessarily improve robustness proportionally. 
% The discrepancy between OA and ARA across languages underscores the need for improved cross-lingual robustness in language models, particularly as they scale in size and capability.


% \noindent
% \textbf{Comparison between Chinese and English datasets.}
% Generally, models demonstrate higher accuracy on the Chinese dataset compared to the English one, as evidenced by the consistently higher OA, ARA and CRA scores. For instance, GPT-4o achieves an OA of 91.37\%, an ARA of 81.97\% , an CRA of 75.55\% on the Chinese dataset, compared to 88.63\% and 70.17\% respectively on the English dataset. This trend is observed across most models, suggesting that the Chinese dataset is easier than English one. Moreover, the RLA values are typically lower for Chinese, indicating smaller performance drops when dealing with robust variants of Chinese questions. For example, Qwen-Max shows an RLA of 8.68\% for Chinese versus 24.99\% for English, highlighting a more consistent performance in Chinese. The CRA scores further reinforce this observation, with models generally maintaining higher consistency in correct answers for both original and variant Chinese questions.
% We attribute this phenomenon to the fact that blablabla

\noindent
\textbf{Reasoning Transferable Capability.}
% 为了进一步
To further analyze whether the model can transfer reasoning ability from the original question to its variant, Figure \ref{consis} presents the distribution of model performance on the original question and variant pairs. For all models, the pairs of (HellaSwag \ding{51} HellaSwag-Pro \ding{55}) occupy a significant proportion, indicating a challenge in transferring reasoning capabilities for current LLMs to more complex scenarios. Looking deeply, closed-source models like GPT-4 and Qwen-Max achieve around a 69\% portion of (HellaSwag \ding{51} HellaSwag-Pro \ding{51}) and a 3\% portion of (HellaSwag \ding{55} HellaSwag-Pro \ding{55}), while in contrast, open-source models struggle with around a 30\% portion of (HellaSwag \ding{51} HellaSwag-Pro \ding{51}) and a 20\% portion of (HellaSwag \ding{55} HellaSwag-Pro \ding{55}), further indicating the robustness of reasoning abilities in closed-source models.
% If a model can get both the original question and the variant right, we consider it to have transferable reasoning ability. Table \ref{consis} presents the distribution of model performance on the original question and variant pairs. Among all models, the pairs of (HellaSwag \ding{51}HellaSwag-Pro \ding{55}) account for a considerable proportion, i 
% The closed-source models like GPT-4o and Qwen-Max achieve around 69\% portion of (HellaSwag \ding{51}HellaSwag-Pro \ding{51}) and 3\% portion of (HellaSwag \ding{55} HellaSwag-Pro \ding{55}), indicating stronger reasoning transfer ability than other models. In contrast, open-source models struggle more, with around 30\% portion of (HellaSwag \ding{51}HellaSwag-Pro \ding{51}) and 20\% portion of (HellaSwag \ding{55} HellaSwag-Pro \ding{55}). 
% A notable trend is observed among the Qwen2.5 series, where increasing model size from 7B to 72B parameters correlates with improved performance on correct answers for both datasets (33.20\% to 37.38\%) and decreased failure rates (17.69\% to 14.7\%). It underscores the importance of model size in commonsense reasoning tasks.

\begin{figure}[t]
\centering
\setlength{\abovecaptionskip}{0.1cm}
\setlength{\belowcaptionskip}{0cm}
\includegraphics[width=\linewidth,scale=1.00]{images/consis.pdf}
\caption{Analysis of the transferable ability of model reasoning based on question pair performance. The green part, where both the original and the variant data are right, represents the transferable performance of model reasoning.}
\label{consis}
\vspace{-15pt}
\end{figure}

\begin{figure*}[ht]
\centering
\setlength{\abovecaptionskip}{0.1cm}
\setlength{\belowcaptionskip}{0cm}
\includegraphics[width=\linewidth,scale=1.00]{images/xing.pdf}
\caption{The impact of different few-shot prompts on model performance. With - as the separator, the first two parts of the legend represent the prompt name, and the third part represents the language of the dataset.}
\label{xing}
\vspace{-15pt}
\end{figure*}

\begin{figure}[ht]
\centering
\setlength{\abovecaptionskip}{0.1cm}
\setlength{\belowcaptionskip}{0cm}
\includegraphics[width=1.05\linewidth,scale=1.05]{images/zhu.pdf}
\caption{The RLA Distribution for 7 variants of commonsense reasoning. Parts below the 0 axis indicate that the model’s performance on the variant is improved compared to the original problem.}
\label{fig:zhu}
\vspace{-15pt}
\end{figure}


\subsection{Variant Analysis (RQ2)}
To further analyze the impact of different variants, we assessed the contribution of each variant to the RLA score. A higher contribution indicates that the model is more likely to make errors in that type. Figure~\ref{fig:zhu} presents the overall results, and the key observations are as follows:
\begin{itemize}[leftmargin=*]
    \item For problem restatement, causal inference, and sentence ordering, these three categories are the least challenging. Almost all models, particularly the close-source and Qwen series models, perform well on these variants, indicating that current LLMs can effectively handle these forms and we do not pay more attention on this kind of varients.
    \item For reverse conversion and critical testing, these two varients each contribute about 10\% to the RLA score. This indicates that current LLMs struggle to fully generalize to these simple scenarios, possibly because these types of questions are not commonly encountered, and reaserchers should pay some attention to this type of varients.
    \item For negative transformation and scenario refinement, this are the two most difficult tasks, with negative transformation being particularly challenging. For almost all models, these two varients accounts for more than 50\% of the RLA score. This may be due to intuitively counterintuitive questions—such as the use of "will not"  or counterfactual scenarios in scenario refinement. These setups are less common in LLM training data and cannot be easily tackled through memory alone. Only those LLMs which truely understand the question could answer the varient correctly, wihch better reflect the true performance of the model.. In the future, researchers should focus more on enhancing LLM's capability to address such types of questions.
\end{itemize}

% 1. Problem restCausal Inference 
% To further analysis the impact of different varients, we further 
% Figure \ref{fig: zhu} presents a comprehensive analysis of various LLMs' performance across different variant types. Negative transformation emerges as the most challenging task for all models, with scores consistently above 50.00\% and peaking at 78.38\% for Gemini-1.5-Pro. Conversely, problem restatement appears to be the least challenging, with most models scoring in the negative range. Intriguingly, smaller models like Qwen2.5-0.5B demonstrate unexpected strengths in certain areas, such as sentence sorting (7.75\%), outperforming some larger counterparts. A detailed analysis of each variant type follows.

% \noindent
% \textbf{Causal inference.} In this category, scores vary widely from -4.73\% for Qwen-Max to 12.25\% for Baichuan2-13B, illustrating differing degrees of sensitivity to causal reasoning among the models. Smaller models, such as Qwen2.5-0.5B and Qwen2.5-1.5B, achieve better scores, indicating relatively stronger robustness in causal reasoning. Conversely, larger models, like Baichuan2-13B, have higher scores, suggesting greater sensitivity to the challenges of inferring causality.

% \noindent
% \textbf{Critical testing.} Larger models, including Qwen2.5-72B and DeepSeek-67B, exhibit higher RLA scores of 30.50\% and 31.37\%, respectively, suggesting increased sensitivity when dealing with incomplete key information. In contrast, GPT-4o achieves the lowest score, highlighting its superior robustness in critical reasoning. This trend indicates that more complex models might struggle to handle incomplete contexts, underscoring potential areas for improvement in sophisticated architectures.

% \noindent
% \textbf{Negative transformation.} This aspect remains consistently challenging for all models, with scores ranging from 48.88\% to 78.38\%. Advanced commercial models like Gemini-1.5-Pro and Claude-3.5 also score higher (78.38\% and 76.43\%, respectively), indicating a prevalent sensitivity issue in reasoning processes when handling negations, irrespective of model size or architecture.

% \noindent
% \textbf{Problem restatement.} The negative values in this category for nearly all models suggest it is not particularly challenging. This is surprising, given that previous models were quite sensitive to sentence representation.

% \noindent
% \textbf{Reverse conversion.} This variation, which involves swapping the roles of the question and answer, seems to specifically impact larger models. For example, Qwen2.5-72B and DeepSeek-67B exhibit higher RLA scores of 24.38\% and 27.43\%, respectively, indicating heightened sensitivity to reverse reasoning compared to their performance on original questions.

% \noindent
% \textbf{Scenario refinement.} The scores range from 16.06\% for Gemma-2-2B to 32.56\% for Qwen2.5-72B, with larger models displaying more sensitivity in adapting to counterfactual predictions. This suggests that larger models may rely more heavily on general commonsense rather than flexibly adapting to specific contexts. Consequently, increased model complexity might adversely affect adaptability to scenario changes, underscoring the need for enhanced flexibility in advanced models.

% \noindent
% \textbf{Sentence sorting.} This category exhibits the most varied results across models. Some larger models like DeepSeek-67B and InternLM2.5-20B display higher scores (26.69\% and 26.68\%), indicating sensitivity, while others like Qwen2.5-72B and Gemini-1.5-Pro excel with lower scores (-9.88\% and -1.07\%, respectively). This suggests that sentence sorting ability may depend more on specific training approaches rather than being solely contingent on model size.


\subsection{Prompt Robustness (RQ3)}
% To investigate how prompt  influence our benchmark, we apply sereral prompt strategy on our datasets and showcase the average performance of all models on different kind of prompt strategies.
% Table~\ref{prompt} illustrates the final results. For both Chinese and English datasets, CN LLMs achieve the highest performance using CN-CoT-Few-Shot, followed closely by EN-CoT-Few-Shot, with overall performance scores of 67.36\% and 67.03\%, respectively. In contrast, English LLMs perform best with the EN-CoT-Few-Shot, reaching 67.55\% on the Chinese dataset and 60.36\% on the English dataset.
% Contrary to previous findings, translating the dataset to the model's advantage language before performing reasoning does not enhance performance. Moreover, Figure~\ref{xing} also shows the similar phenomenon. Conducting CoT reasoning in the model’s advantage language generally leads to better outcomes compared to Direct. Additionally, increasing the number of shots consistently improves performance across most configurations, highlighting the benefits of exposing models to multiple examples. 
To explore the impact of various prompt strategies on our benchmarks, we evaluated several approaches across our datasets and present the average performance of all models using different prompting techniques. Table~\ref{prompt} summarizes the results. For both Chinese and English datasets, Chinese LLMs performed best with the CN-CoT-Few-Shot strategy, followed closely by EN-CoT-Few-Shot, achieving overall scores of 67.36\% and 67.03\%, respectively. Conversely, English LLMs showed optimal performance with the EN-CoT-Few-Shot approach, attaining 67.55\% on the Chinese dataset and 60.36\% on the English dataset.
Besides, translating datasets into the model's native language before reasoning did not enhance performance. This phenomenon is further illustrated in Figure~\ref{xing}. Conducting CoT reasoning in the model's native language generally yields better results compared to direct reasoning. Furthermore, increasing the number of examples (shots) consistently boosts performance across most configurations, emphasizing the advantages of exposing models to multiple examples.
% Overall, the interaction between question language, prompt language, and the number of shots underscores the importance of aligning these factors to optimize task performance and robustness in LLMs.



% Please add the following required packages to your document preamble:
% \usepackage{multirow}
% Please add the following required packages to your document preamble:
% \usepackage{multirow}
\begin{table}[t]
\setlength{\tabcolsep}{8pt}
% \footnotesize
\scalebox{0.65}{
\begin{tabular}{c|l|lll}
\hline
\multicolumn{1}{l|}{Dataset}  & Prompt  & CN LLMs & EN LLMs &  LLMs \\ \hline
\multirow{7}{*}{\begin{tabular}[c]{@{}c@{}}Chinese\\ HellaSwag-Pro\end{tabular}} & Direct  & 48.95& 41.16& 45.06  \\
& CN-CoT-Few  & \textbf{71.04}& 51.90& 61.47  \\
& EN-CoT-Few  & 70.95& \textbf{67.55}& \textbf{69.25}  \\
& EN-XLT-Few  & 41.48& 28.69& 35.09  \\
& CN-CoT-Zero & 44.82& 23.89& 34.36  \\
& EN-CoT-Zero & 45.38& 31.39& 38.39  \\
& EN-XLT-Zero & 28.57& 12.93& 20.75  \\ \hline
\multirow{7}{*}{\begin{tabular}[c]{@{}c@{}}English\\ HellaSwag-Pro\end{tabular}} & Direct  & 47.46& 40.66& 44.06  \\
& CN-CoT-Few  & \textbf{63.67}& 47.24& 55.46  \\
& EN-CoT-Few  & 63.12& \textbf{60.36}& \textbf{61.74}  \\
& CN-XLT-Few  & 48.77& 16.61& 32.69  \\
& CN-CoT-Zero & 34.89& 18.25& 26.57  \\
& EN-CoT-Zero & 42.41& 31.03& 36.72  \\
& CN-XLT-Zero & 16.36& 11.22& 13.79  \\ \hline
\multirow{9}{*}{HellaSwag-Pro}& Direct  & 48.21& 40.91& 44.83  \\
& CN-CoT-Few  & \textbf{67.36}& 49.57& 58.46  \\
& EN-CoT-Few  & 67.03& \textbf{63.95}& \textbf{65.49}  \\
& CN-XLT-Few  & 59.91& 34.26& 47.08  \\
& EN-XLT-Few  & 52.30& 44.52& 48.41  \\
& CN-CoT-Zero & 39.86& 21.07& 30.46  \\
& EN-CoT-Zero & 43.90& 31.21& 37.55  \\
& CN-XLT-Zero & 30.59& 17.55& 24.07  \\
& EN-XLT-Zero & 35.49& 21.98& 28.74  \\ \hline
\end{tabular}
}
\caption{Average ARA of all open-source models on different prompts. CN-LLMs contains 17 LLMs, and EN-LLMs contains 7 LLMs. The bast results for each dataset are \textbf{bolded}.}
\label{prompt}
\end{table}




\section{Concluding Remarks}
In this paper, we proposed a novel approach utilizing multimodal LLMs to generate gesture-aware speech recognition transcripts for patients with language disorders. Our framework integrates verbal speech and iconic gestures, enabling the generation of enriched transcripts that capture the latent meaning conveyed through both modalities. Through extensive experimentation, we demonstrated that the proposed method effectively contextualizes incomplete or disfluent speech by incorporating gesture information, leading to more accurate and meaningful representations of the speaker's intent. These findings highlight the potential of our approach to significantly contribute to the field of speech and language therapy, offering innovative tools that can enhance the quality of life for individuals with language disorders by facilitating better communication and assessment methods.

\subsection{Ethical Statement} 
Our dataset was obtained from AphasiaBank with the approval of the Institutional Review Board (IRB) and adheres to the data sharing guidelines set by TalkBank\footnote{https://talkbank.org/share/ethics.html}. This includes complying with the Ground Rules for all TalkBank databases, which are based on the American Psychological Association Code of Ethics~\cite{american2002ethical}.

\subsection{Limitation \& Future Work} 
%This study represents a preliminary investigation into using multimodal LLMs to generate gesture-aware speech recognition transcripts. 
While the results are promising, we recognize several limitations and outline our plans to extend this work further.

One primary limitation is the absence of a definitive ground truth for quantitative evaluation. Since our model generates transcripts by synthesizing speech and gesture data from scratch, traditional benchmarks, such as comparisons with standard speech recognition outputs, are insufficient. Moreover, existing original transcripts lack gesture annotations, making direct comparisons challenging. In future work, we aim to address this gap by collaborating with certified pathologists to conduct qualitative assessments, such as A-B preference tests, to evaluate the effectiveness of gesture-enriched transcripts in accurately conveying the speaker's intentions.

To support quantitative evaluations, we plan to develop novel metrics that assess transcript quality, including grammar accuracy, semantic consistency, and the integration of multimodal information. Such metrics will provide a more objective basis for assessing our model's performance and facilitate comparisons with other multimodal and unimodal approaches.

Another limitation of this study is its focus on structured gestures from a specific task, the Peanut Butter Sandwich Task. While this task offers a controlled context for testing our approach, it does not encompass the diversity of gestures and communication patterns seen in everyday scenarios. As part of our future work, we plan to expand the scope of our model to include tasks such as the Cinderella Story Recall Task~\cite{bird1996cinderella}, which involves unstructured and complex narrative gestures. This expansion will allow us to evaluate the adaptability and robustness of our model in handling varied linguistic and gestural contexts.

In summary, while this study establishes a strong foundation for gesture-aware speech recognition, we aim to refine and extend our methods through collaborative qualitative evaluations, the development of robust quantitative metrics, and broader task applications. These efforts will ensure that our approach continues to evolve, ultimately contributing to more effective communication tools and interventions for individuals with language disorders.






\begin{comment}
\section*{Acknowledgment}

The preferred spelling of the word ``acknowledgment'' in America is without 
an ``e'' after the ``g''. Avoid the stilted expression ``one of us (R. B. 
G.) thanks $\ldots$''. Instead, try ``R. B. G. thanks$\ldots$''. Put sponsor 
acknowledgments in the unnumbered footnote on the first page.
\end{comment}



\bibliographystyle{ieeetr} 
\bibliography{ref}

\begin{comment}
%表格暂时写在上面,方便对照
\begin{table}[ht]
\centering
\caption{Notations and Descriptions\note{[REMOVE THIS TABLE AT LAST]}}
\begin{tabular}{|c|l|}
\hline
\textbf{Inputs} & \textbf{Description} \\ \hline
$C$ & the set of candidate clients in FL \\ \hline
$K$ & the set of clusters \\ \hline
$D_i$ & Each client $c_i$ holds a local dataset size \\ \hline
$p_k$ & the cluster head of cluster satellite $k$ \\ \hline
$S_k$ & the clients belongs to cluster $k$ \\ \hline
$G$ & ground station number $G$ \\ \hline
$g_{i}^{p_k}$ & the cluster$p_k$ belongs to ground station i  \\ \hline
$t^{cmp}_i$ & computation time of client $i$ for one intra-cluster aggregation \\ \hline
$t^{com}_i$ & communication time of client $i$ for one intra-cluster aggregation \\ \hline
$t^{com}_k$ & communication time of cluster head $l_k$ for one inter-cluster aggregation \\ \hline
$n_k$ & Maximum selected number of clients per cluster per global round \\ \hline
\textbf{Outputs} & \textbf{Description} \\ \hline
$x^k_i$ & whether client $i$ belongs to cluster $k$ \\ \hline
$y^t_i$ & whether or not select client $i$ at global round $t$ \\ \hline
$\tau_{i}^{k}$ & local iterations given to client ci in the k-th communication round \\ \hline
\end{tabular}
\end{table}
\end{comment}

\end{document}
