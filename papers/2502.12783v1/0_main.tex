\documentclass[10pt, conference, letterpaper]{IEEEtran}
\IEEEoverridecommandlockouts
% The preceding line is only needed to identify funding in the first footnote. If that is unneeded, please comment it out.
\usepackage{cite}
\usepackage{amsmath,amssymb,amsfonts}
\usepackage{url}
%\usepackage{algorithmicx}
%\usepackage{algpseudocode}
\usepackage{algorithmic}
\usepackage{algorithm} 
\usepackage[french,boxed,vlined,linesnumbered,inoutnumbered,rightnl,algo2e]{algorithm2e}
%\usepackage[algo2e]{algorithm2e} 
\usepackage{graphicx}
\usepackage{textcomp}
\usepackage{xcolor}
\usepackage{comment}
\usepackage{booktabs}
%\usepackage{subfig}
\usepackage{subcaption}
\usepackage{multirow}
\usepackage{mathrsfs}
\usepackage{threeparttable}
\usepackage{booktabs}
\newcommand*{\note}[1]{\textcolor{red}{#1}}
\newcommand*{\modify}[1]{\textcolor{blue}{#1}}
\newcommand{\algname}{FedHC\xspace}
\renewcommand{\algorithmicrequire}{\textbf{Input:}}
\renewcommand{\algorithmicensure}{\textbf{Output:}}

\renewcommand{\algorithmiccomment}[1]{ $\triangleright$ #1}
\def\BibTeX{{\rm B\kern-.05em{\sc i\kern-.025em b}\kern-.08em
    T\kern-.1667em\lower.7ex\hbox{E}\kern-.125emX}}

\setlength{\abovecaptionskip}{0.05cm}
\begin{document}


%\title{An Adaptive Quantity Approach for Minimizing Cost and Delay in Satellite Clustered Federated Learning}

%\title{An Adaptive Quantization Approach for Cost and Latency Minimization in Satellite Clustered Federated Learning Framework \note{[TENTATIVE]}\\


%\title{An Adaptive Quantization Approach for Multi Objective Optimization in Satellite Clustered Federated Learning Framework}

%\title{Satellite Clustered Federated Learning Framework with Adaptive Quantization \note{[TENTATIVE]}\\

\title{\algname: A Hierarchical Clustered Federated Learning Framework for Satellite Networks \\

%\thanks{Identify applicable funding agency here. If none, delete this.}
}

%\author{Anonymous Authors}

%\begin{comment}
\author{\IEEEauthorblockN{Zhuocheng Liu$^{1}$, Zhishu Shen$^{1}$\textbf{\IEEEauthorrefmark{2}}\thanks{\IEEEauthorrefmark{2} Corresponding author (z\_shen@ieee.org).}, Pan Zhou$^{1}$, Qiushi Zheng$^{2}$, and Jiong Jin$^{2}$}

\IEEEauthorblockA{\textsuperscript{$^{1}$}School of Computer Science and Artificial Intelligence, Wuhan University of Technology, China\\
}


\IEEEauthorblockA{\textsuperscript{$^{2}$}School of Science, Computing and Engineering Technologies, Swinburne University of Technology, Australia\\
}
%\IEEEauthorblockA{E-mail: \{\note{xxx}, \note{xxx}, yjl\}@whut.edu.cn, z\_shen@ieee.org, xxx@xxxx.com, jiongjin@swin.edu.au}
}
%\end{comment}

\maketitle

\begin{abstract}
%With the proliferation of data-driven services, the volume of data that needs to be processed by satellite networks has significantly increased. Federated learning (FL) is well-suited for handling big data processing at distributed, resource-constrained satellite environments. However, ensuring the convergence performance of FL models within dynamically distributed satellite networks is challenging while minimizing FL processing time and energy cost. To this end, we propose a hierarchical clustered federated learning framework named \algname. In this framework, a satellite-clustered parameter server (PS) selection algorithm is deployed at the cluster aggregation stage to partition nearby satellites into distinct clusters, designating the satellite cluster center client as the PS to accelerate the model aggregation. Several communicable cluster PS satellites are then selected through ground stations to aggregate global parameters to support the FL process. Moreover, a meta-learning-driven satellite re-clustering algorithm is introduced to enhance adaptability to dynamic satellite cluster changes. The extensive experiments conducted on satellite networks testbed demonstrate that \algname can significantly reduce processing time (up to 3x) and energy cost (up to 2x) against other comparative methods, while ensuring the accuracy of the model.

With the proliferation of data-driven services, the volume of data that needs to be processed by satellite networks has significantly increased. Federated learning (FL) is well-suited for big data processing in distributed, resource-constrained satellite environments. However, ensuring its convergence performance while minimizing processing time and energy consumption remains a challenge. To this end, we propose a hierarchical clustered federated learning framework, \algname. This framework employs a satellite-clustered parameter server (PS) selection algorithm at the cluster aggregation stage, grouping nearby satellites into distinct clusters and designating a cluster center as the PS to accelerate model aggregation. Several communicable cluster PS satellites are then selected through ground stations to aggregate global parameters, facilitating the FL process. Moreover, a meta-learning-driven satellite re-clustering algorithm is introduced to enhance adaptability to dynamic satellite cluster changes. The extensive experiments on satellite networks testbed demonstrate that \algname can significantly reduce processing time (up to 3x) and energy consumption (up to 2x) compared to other comparative methods while maintaining model accuracy.

%Low Earth Orbit (LEO) satellites need to process vast amounts of communication data from the ground daily. Federated Learning (FL) is well-suited for handling this data at the edge, where resources are limited. FL only requires the transmission of trained model parameters during data processing, reducing communication needs while achieving good accuracy. However, maintaining the convergence performance of FL models in a dynamically distributed satellite architecture, while minimizing communication time and energy consumption, remains a critical challenge.In this work, we propose a novel FL framework called hierarchical clustered quantitative federated learning (\algname)., to address the aforementioned issues while ensuring FL learning accuracy. \algname is built on two stages of operation: satellite cluster aggregation stage and ground station aggregation stage. We use K-means clustering algorithm to divide nearby satellites into different clusters, and specify the client of the satellite cluster center with good communication as the parameter server to accelerate the aggregation speed of the model. Aggregate global model parameters from different satellite clusters to perform FL. After the aggregation process is completed, the ground station selects a communicable cluster parameter server satellite to aggregate global parameters. Conduct model experiments on real satellite distribution data to verify the effectiveness of \algname. The results indicate that \algname can significantly reduce communication time and energy consumption with the same accuracy.
\end{abstract}

\begin{IEEEkeywords}
Satellite networks, hierarchical clustered federated learning, distributed computing%, multi-objective optimization
\end{IEEEkeywords}

\section{Introduction}

Video generation has garnered significant attention owing to its transformative potential across a wide range of applications, such media content creation~\citep{polyak2024movie}, advertising~\citep{zhang2024virbo,bacher2021advert}, video games~\citep{yang2024playable,valevski2024diffusion, oasis2024}, and world model simulators~\citep{ha2018world, videoworldsimulators2024, agarwal2025cosmos}. Benefiting from advanced generative algorithms~\citep{goodfellow2014generative, ho2020denoising, liu2023flow, lipman2023flow}, scalable model architectures~\citep{vaswani2017attention, peebles2023scalable}, vast amounts of internet-sourced data~\citep{chen2024panda, nan2024openvid, ju2024miradata}, and ongoing expansion of computing capabilities~\citep{nvidia2022h100, nvidia2023dgxgh200, nvidia2024h200nvl}, remarkable advancements have been achieved in the field of video generation~\citep{ho2022video, ho2022imagen, singer2023makeavideo, blattmann2023align, videoworldsimulators2024, kuaishou2024klingai, yang2024cogvideox, jin2024pyramidal, polyak2024movie, kong2024hunyuanvideo, ji2024prompt}.


In this work, we present \textbf{\ours}, a family of rectified flow~\citep{lipman2023flow, liu2023flow} transformer models designed for joint image and video generation, establishing a pathway toward industry-grade performance. This report centers on four key components: data curation, model architecture design, flow formulation, and training infrastructure optimization—each rigorously refined to meet the demands of high-quality, large-scale video generation.


\begin{figure}[ht]
    \centering
    \begin{subfigure}[b]{0.82\linewidth}
        \centering
        \includegraphics[width=\linewidth]{figures/t2i_1024.pdf}
        \caption{Text-to-Image Samples}\label{fig:main-demo-t2i}
    \end{subfigure}
    \vfill
    \begin{subfigure}[b]{0.82\linewidth}
        \centering
        \includegraphics[width=\linewidth]{figures/t2v_samples.pdf}
        \caption{Text-to-Video Samples}\label{fig:main-demo-t2v}
    \end{subfigure}
\caption{\textbf{Generated samples from \ours.} Key components are highlighted in \textcolor{red}{\textbf{RED}}.}\label{fig:main-demo}
\end{figure}


First, we present a comprehensive data processing pipeline designed to construct large-scale, high-quality image and video-text datasets. The pipeline integrates multiple advanced techniques, including video and image filtering based on aesthetic scores, OCR-driven content analysis, and subjective evaluations, to ensure exceptional visual and contextual quality. Furthermore, we employ multimodal large language models~(MLLMs)~\citep{yuan2025tarsier2} to generate dense and contextually aligned captions, which are subsequently refined using an additional large language model~(LLM)~\citep{yang2024qwen2} to enhance their accuracy, fluency, and descriptive richness. As a result, we have curated a robust training dataset comprising approximately 36M video-text pairs and 160M image-text pairs, which are proven sufficient for training industry-level generative models.

Secondly, we take a pioneering step by applying rectified flow formulation~\citep{lipman2023flow} for joint image and video generation, implemented through the \ours model family, which comprises Transformer architectures with 2B and 8B parameters. At its core, the \ours framework employs a 3D joint image-video variational autoencoder (VAE) to compress image and video inputs into a shared latent space, facilitating unified representation. This shared latent space is coupled with a full-attention~\citep{vaswani2017attention} mechanism, enabling seamless joint training of image and video. This architecture delivers high-quality, coherent outputs across both images and videos, establishing a unified framework for visual generation tasks.


Furthermore, to support the training of \ours at scale, we have developed a robust infrastructure tailored for large-scale model training. Our approach incorporates advanced parallelism strategies~\citep{jacobs2023deepspeed, pytorch_fsdp} to manage memory efficiently during long-context training. Additionally, we employ ByteCheckpoint~\citep{wan2024bytecheckpoint} for high-performance checkpointing and integrate fault-tolerant mechanisms from MegaScale~\citep{jiang2024megascale} to ensure stability and scalability across large GPU clusters. These optimizations enable \ours to handle the computational and data challenges of generative modeling with exceptional efficiency and reliability.


We evaluate \ours on both text-to-image and text-to-video benchmarks to highlight its competitive advantages. For text-to-image generation, \ours-T2I demonstrates strong performance across multiple benchmarks, including T2I-CompBench~\citep{huang2023t2i-compbench}, GenEval~\citep{ghosh2024geneval}, and DPG-Bench~\citep{hu2024ella_dbgbench}, excelling in both visual quality and text-image alignment. In text-to-video benchmarks, \ours-T2V achieves state-of-the-art performance on the UCF-101~\citep{ucf101} zero-shot generation task. Additionally, \ours-T2V attains an impressive score of \textbf{84.85} on VBench~\citep{huang2024vbench}, securing the top position on the leaderboard (as of 2025-01-25) and surpassing several leading commercial text-to-video models. Qualitative results, illustrated in \Cref{fig:main-demo}, further demonstrate the superior quality of the generated media samples. These findings underscore \ours's effectiveness in multi-modal generation and its potential as a high-performing solution for both research and commercial applications.
%\section{Related Work}
\label{sec:relatedwork}

\subsection{Current AI Tools for Social Service}
\label{subsec:relatedtools}
% the title I feel is quite broad

Harnessing technology for social good has always been a grand challenge in social service \cite{berzin_practice_2015}. As early as the 90s, artificial neural networks and predictive models have been employed as tools for risk assessments, decision-making, and workload management in sectors like child protective services and mental health treatment \cite{fluke_artificial_1989, patterson_application_1999}. The recent rise of generative AI is poised to further advance social service practice, facilitating the automation of administrative tasks, streamlining of paperwork and documentation, optimisation of resource allocation, data analysis, and enhancing client support and interventions \cite{fernando_integration_2023, perron_generative_2023}.

Today, AI solutions are increasingly being deployed in both policy and practice \cite{goldkind_social_2021, hodgson_problematising_2022}. In clinical social work, AI has been used for risk assessments, crisis management, public health initiatives, and education and training for practitioners \cite{asakura_call_2020, gillingham2019can, jacobi_functions_2023, liedgren_use_2016, molala_social_2023, rice_piloting_2018, tambe_artificial_2018}. AI has also been employed for mental health support and therapeutic interventions, with conversational agents serving as on-demand virtual counsellors to provide clinical care and support \cite{lisetti_i_2013, reamer_artificial_2023}.
% commercial solutions include Woebot, which simulates therapeutic conversation, and Wysa, an “emotionally intelligent” AI coach, powered by evidenced-based clinical techniques \cite{reamer_artificial_2023}. 
% Non-clinical AI agents like Replika and companion robots can also provide social support and reduce loneliness amongst individuals \cite{ahmed_humanrobot_2024, chaturvedi_social_2023, pani_can_2024, ta_user_2020}.

Present research largely focuses on \textit{\textbf{AI-based decision support tools}} in social service \cite{james_algorithmic_2023, kawakami2022improving}, especially predictive risk models (PRMs) used to predict social service risks and outcomes \cite{gillingham2019can, van2017predicting}, like the Allegheny Family Screening Tool (AFST), which assesses child abuse risk using data from US public systems \cite{chouldechova_case_2018, vaithianathan2017developing}. Elsewhere, researchers have also piloted PRMs to predict social service needs for the homeless using Medicaid data\cite{erickson_automatic_2018, pourat_easy_2023}, and AI-powered algorithms to promote health interventions for at-risk populations, such as HIV testing among Californian homeless \cite{rice_piloting_2018, yadav_maximizing_2017}.

\subsection{Generative AI and Human-AI Collaboration}
\label{subsec:relatedworkhaicollaboration}
Beyond decision-making algorithms and PRMs, advancements in generative AI, such as large language models (LLMs), open new possibilities for human-AI (HAI) collaboration in social services. 
LLMs have been called "revolutionary" \cite{fui2023generative} and a "seismic shift" \cite{cooper2023examining}, offering "content support" \cite{memmert2023towards} by generating realistic and coherent responses to user inputs \cite{cascella2023evaluating}. Their vastly improved capabilities and ubiquity \cite{cooper2023examining} makes them poised to revolutionise work patterns \cite{fui2023generative}. Generative AI is already used in fields like design, writing, music, \cite{han2024teams, suh2021ai, verheijden2023collaborative, dhillon2024shaping, gero2023social} healthcare, and clinical settings \cite{zhang2023generative, yu2023leveraging, biswas2024intelligent}, with promising results. However, the social service sector has been slower in adopting AI \cite{diez2023artificial, kawakami2023training}.

% Yet, the social service sector is one that could perhaps stand to gain the most from AI technologies. As Goldkind \cite{goldkind_social_2021} writes, social service, as a "values-centred profession with a robust code of ethics" (p. 372), is uniquely placed to inform the development of thoughtful algorithmic policy and practice. 
Social service, however, stands to benefit immensely from generative AI. SSPs work in time-poor environments \cite{tiah_can_2024}, often overwhelmed with tedious administrative work \cite{meilvang_working_2023} and large amounts of paperwork and data processing \cite{singer_ai_2023, tiah_can_2024}. 
% As such, workers often work in time-poor environments and are burdened with information overload and administrative tasks \cite{tiah_can_2024, meilvang_working_2023}. 
Generative AI is well-placed to streamline and automate tasks like formatting case notes, formulating treatment plans and writing progress reports, which can free up valuable time for more meaningful work like client engagement and enhance service quality \cite{fernando_integration_2023, perron_generative_2023, tiah_can_2024, thesocialworkaimentor_ai_nodate}. 

Given the immense potential, there has been emerging research interest in HAI collaboration and teamwork in the Human-Computer Interaction and Computer Supported Cooperative Work space \cite{wang_human-human_2020}. HAI collaboration and interaction has been postulated by researchers to contribute to new forms of HAI symbiosis and augmented intelligence, where algorithmic and human agents work in tandem with one another to perform tasks better than they could accomplish alone by augmenting each other's strengths and capabilities  \cite{dave_augmented_2023, jarrahi_artificial_2018}.

However, compared to the focus on AI decision-making and PRM tools, there is scant research on generative AI and HAI collaboration in the social service sector \cite{wykman_artificial_2023}. This study therefore seeks to fill this critical gap by exploring how SSPs use and interact with a novel generative AI tool, helping to expand our understanding of the new opportunities that HAI collaboration can bring to the social service sector.

\subsection{Challenges in AI Use in Social Service}
\label{subsec:relatedworkaiuse}

% Despite the immense potential of AI systems to augment social work practice, there are multiple challenges with integrating such systems into real-life practice. 
Despite its evident benefits, multiple challenges plague the integration of AI and its vast potential into real-life social service practice.
% Numerous studies have investigated the use of PRMs to help practitioners decide on a course of action for their clients. 
When employing algorithmic decision-making systems, practitioners often experience tension in weighing AI suggestions against their own judgement \cite{kawakami2022improving, saxena2021framework}, being uncertain of how far they should rely on the machine. 
% Despite often being instructed to use the tool as part of evaluating a client, 
Workers are often reluctant to fully embrace AI assessments due to its inability to adequately account for the full context of a case \cite{kawakami2022improving, gambrill2001need}, and lack of clarity and transparency on AI systems and limitations \cite{kawakami2022improving}. Brown et al. \cite{brown2019toward} conducted workshops using hypothetical algorithmic tools 
% to understand service providers' comfort levels with using such tools in their work,
and found similar issues with mistrust and perceived unreliability. Furthermore, introducing AI tools can  create new problems of its own, causing confusion and distrust amongst workers \cite{kawakami2022improving}. Such factors are critical barriers to the acceptance and effective use of AI in the sector.

\citeauthor{meilvang_working_2023} (2023) cites the concept of \textit{boundary work}, which explores the delineation between "monotonous" administrative labour and "professional", "knowledge based" work drawing on core competencies of SSPs. While computers have long been used for bureaucratic tasks like client registration, the introduction of decision support systems like PRMs stirred debate over AI "threatening professional discretion and, as such, the profession itself" \cite{meilvang_working_2023}. Such latent concerns arguably drive the resistance to technology adoption described above. Generative AI is only set to further push this boundary, 
% these concerns are only set to grow in tandem with the vast capabilities of generative and other modern AI systems. Compared to the relatively primitive AI systems in past years, perceived as statistical algorithms \cite{brown2019toward} turning preset inputs like client age and behavioural symptoms \cite{vaithianathan2017developing} into simple numerical outputs indicating various risk scores, modern AI systems are vastly more capable: LLMs 
with its ability to formulate detailed reports and assessments that encroach upon the "core" work of SSPs.
% accept unrestricted and unstructured inputs and return a range of verbose and detailed evaluations according to the user's instructions. 
Introducing these systems exacerbate previously-raised issues such as understanding the limitations and possibilities of AI systems \cite{kawakami2022improving} and risk of overreliance on AI \cite{van2023chatgpt}, and requires a re-examination of where users fall on the algorithmic aversion-bias scale \cite{brown2019toward} and how they detect and react to algorithmic failings \cite{de2020case}. We address these critical issues through an empirical, on-the-ground study that to our knowledge is the first of its kind since the new wave of generative AI.

% W 

% Yet, to date, we have limited knowledge on the real-world impacts and implications of human-AI collaboration, and few studies have investigated practitioners’ experiences working with and using such AI systems in practice, especially within the social work context \cite{kawakami2022improving}. A small number of studies have explored practitioner perspectives on the use of AI in social work, including Kawakami et al. \cite{kawakami2022improving}, who interviewed social workers on their experiences using the AFST; Stapleton et al. \cite{stapleton_imagining_2022}, who conducted design workshops with caseworkers on the use of PRMs in child welfare; and Wassal et al. \cite{wassal_reimagining_2024}, who interviewed UK social work professionals on the use of AI. A common thread from all these studies was a general disregard for the context and users, with many practitioners criticising the failure of past AI tools arising from the lack of participation and involvement of social workers and actual users of such systems in the design and development of algorithmic systems \cite{wassal_reimagining_2024}. Similarly, in a scoping review done on decision-support algorithms in social work, Jacobi \& Christensen \cite{jacobi_functions_2023} reported that the majority of studies reveal limited bottom-up involvement and interaction between social workers, researchers and developers, and that algorithms were rarely developed with consideration of the perspective of social workers.
% so the \cite{yang_unremarkable_2019} and \cite{holten_moller_shifting_2020} are not real-world impacts? real-world means to hear practitioner's voice? I feel this is quite important but i didnt get this point in intro!

% why mentioning 'which have largely focused on existing ADS tools (e.g., AFST)'? i can see our strength is more localized, but without basic knowledge of social work i didnt get what's the 'departure' here orz
% the paragraph is great! do we need to also add one in line 20 21?

\subsection{Designing AI for Social Service through Participatory Design}
\label{subsec:relatedworkpd}
% i think it's important! but maybe not a whole subsection? but i feel the strong connection with practitioners is indeed one of our novelties and need to highlight it, also in intro maybe
% Participatory design (PD) has long been used extensively in HCI \cite{muller1993participatory}, to both design effective solutions for a specific community and gain a deep understanding of that community. Of particular interest here is the rich body of literature on PD in the field of healthcare \cite{donetto2015experience}, which in this regard shares many similarities and concerns with social work. PD has created effective health improvement apps \cite{ryu2017impact}, 

% PD offers researchers the chance to gather detailed user requirements \cite{ryu2017impact}...

Participatory design (PD) is a staple of HCI research \cite{muller1993participatory}, facilitating the design of effective solutions for a specific community while gaining a deep understanding of its stakeholders. The focus in PD of valuing the opinions and perspectives of users as experts \cite{schuler_participatory_1993} 
% In recent years, the tech and social work sectors have awakened to the importance of involving real users in designing and implementing digital technologies, developing human-centred design processes to iteratively design products or technologies through user feedback 
has gained importance in recent years \cite{storer2023reimagining}. Responding to criticisms and failures of past AI tools that have been implemented without adequate involvement and input from actual users, HCI scholars have adopted PD approaches to design predictive tools to better support human decision-making \cite{lehtiniemi_contextual_2023}.
% ; accordingly, in social service, a line of research has begun studying and designing for human-AI collaboration with real-world users (e.g. \cite{holten_moller_shifting_2020, kawakami2022improving, yang_unremarkable_2019}).
Section \ref{subsec:relatedworkaiuse} shows a clear need to better understand SSP perspectives when designing and implementing AI tools in the social sector. 
Yet, PD research in this area has been limited. \citeauthor{yang2019unremarkable} (2019), through field evaluation with clinicians, investigated reasons behind the failure of previous AI-powered decision support tools, allowing them to design a new-and-improved AI decision-support tool that was better aligned with healthcare workers’ workflows. Similarly, \citeauthor{holten_moller_shifting_2020} (2020) ran PD workshops with caseworkers, data scientists and developers in public service systems to identify the expectations and needs that different stakeholders had in using ADS tools.

% Indeed, it is as Wise \cite{wise_intelligent_1998} noted so many years ago on the rise of intelligent agents: “it is perhaps when technologies are new, when their (and our) movements, habits and attitudes seem most awkward and therefore still at the forefront of our thoughts that they are easiest to analyse” (p. 411). 
Building upon this existing body of work, we thus conduct a study to co-design an AI tool \textit{for} and \textit{with} SSPs through participatory workshops and focus group discussions. In the process, we revisit many of the issues mentioned in Section \ref{subsec:relatedworkaiuse}, but in the context of novel generative AI systems, which are fundamentally different from most historical examples of automation technologies \cite{noy2023experimental}. This valuable empirical inquiry occurs at an opportune time when varied expectations about this nascent technology abound \cite{lehtiniemi_contextual_2023}, allowing us to understand how SSPs incorporate AI into their practice, and what AI can (or cannot) do for them. In doing so, we aim to uncover new theoretical and practical insights on what AI can bring to the social service sector, and formulate design implications for developing AI technologies that SSPs find truly meaningful and useful.
% , and drive future technological innovations to transform the social service sector not just within [our country], but also on a global scale.

 % with an on-the-ground study using a real prototype system that reflects the state of AI in current society. With the presumption that AI will continue to be used in social work given the great benefits it brings, we address the pressing need to investigate these issues to ensure that any potential AI systems are designed and implemented in a responsible and effective manner.

% Building upon these works, this study therefore seeks to adopt a participatory design methodology to investigate social workers’ perspectives and attitudes on AI and human-AI collaboration in their social work practice, thus contributing to the nascent body of practitioner-centred HCI research on the use of AI in social work. Yet, in a departure from prior work, which have largely focused on existing ADS tools (e.g., AFST) and were situated in a Western context, our paper also aims to expand the scope by piloting a novel generative AI tool that was designed and developed by the researchers in partnership with a social service agency based in Singapore, with aims of generating more insights on wider use cases of AI beyond what has been previously studied.

% i may think 'While the current lacunae of research on applications of AI in social work may appear to be a limitation, it simultaneously presents an exciting opportunity for further research and exploration \cite{dey_unleashing_2023},' this point is already convincing enough, not sure if we need to quote here
% I like this end! it's a good transition to our study design, do we need to mention the localization in intro as well? like we target at singapore

% Given the increasing prominence and acceptance of AI in modern society, 

% These increased capabilities vastly exacerbate the issues already present with a simpler tool like the AFST: the boundaries and limitations of an LLM system are significantly more difficult to understand and its possible use cases are exponentially greater in scope. 

% Put this in discussion section instead?
% Kawakami et al's work "highlights the importance of studying how collaborative decision-making... impacts how people rely upon and make sense of AI models," They conclude by recommending designing tools that "support workers in understanding the boundaries of [an AI system's] capabilities", and implementing design procedures that "support open cultures for critical discussion around AI decision making". The authors outline critical challenges of implementing AI systems, elucidating factors that may hinder their effectiveness and even negatively affect operations within the organisation.


% Is this needed?:
% talk about the strengths of PD in eliciting user viewpoints and knowledge, in particular when it is a field that is novel or where a certain system has not been used or developed or tested before
\section{Model}
\begin{figure*}[t]
  \centering
  \includegraphics[width=\textwidth]{figures/framework_fig2.pdf}
   \caption{
   The pipeline of our \Model framework. We first generate an initial task instruction using LLMs with in-context learning and sample trajectories aligned with the initial language instructions in the environment. Next, we use the LLM to summarize the sampled trajectories and generate refined task instructions that better match these trajectories. We then modify specific actions within the trajectories to perform new actions in the environment, collecting negative trajectories in the process. Using the refined task instructions, along with both positive and negative trajectories, we train a lightweight reward model to distinguish between matching and non-matching trajectories. The learned reward model can then collaborate with various LLM agents to improve task planning.
   }
   \label{fig:pipeline}
\end{figure*}

In this section, we provide a detailed introduction to our framework, autonomous Agents from automatic Reward Modeling And Planning (\Model). The framework includes automated reward data generation in section~\ref{sec:data}, reward model design in section~\ref{sec:model}, and planning algorithms in section~\ref{sec:plan}.

\subsection{Background}
The planning tasks for LLM agents can be typically formulated as a Partially Observable Markov Decision Process (POMDP): $(\mathcal{X}, \mathcal{S}, \mathcal{A}, \mathcal{O}, \mathcal{T})$, where:
\begin{itemize}
    \item $\mathcal{X}$ is the set of text instructions;
    \item $\mathcal{S}$ is the set of environment states;
    \item $\mathcal{A}$ is the set of available actions at each state;
    \item $\mathcal{O}$ represents the observations available to the agents, including text descriptions and visual information about the environment in our setting;
    \item $\mathcal{T}: \mathcal{S} \times \mathcal{A} \rightarrow \mathcal{S}$ is the transition function of states after taking actions, which is given by the environment in our settings. 
\end{itemize}

Given a task instruction $\mathit{x} \in \mathcal{X}$ and the initial environment state $\mathit{s_0} \in \mathcal{S}$, planning tasks require the LLM agents to propose a sequence of actions ${\{a_n\}_{n=1}^{N}}$ that aim to complete the given task, where $a_n \in \mathcal{A}$ represents the action taken at time step $n$, and $N$ is the total number of actions executed in a trajectory.
Following the $n$-th action, the environment transitions to state $\mathit{s_{n}}$, and the agent receives a new observation $\mathit{o_{n}}$. Based on the accumulated state and action histories, the task evaluator determines whether the task is completed.

An important component of our framework is the learned reward model $\mathcal{R}$, which estimates whether a trajectory $h$ has successfully addressed the task:
\begin{equation}
    r = \mathcal{R}(\mathit{x}, h),
\end{equation}
where $h = \{\{a_n\}_{n=1}^N, \{o_n\}_{n=0}^{N}\}$, $\{a_n\}_{n=1}^N$ are the actions taken in the trajectory, $\{o_n\}_{n=0}^{N}$ are the corresponding environment observations, and $r$ is the predicted reward from the reward model.
By integrating this reward model with LLM agents, we can enhance their performance across various environments using different planning algorithms.

\subsection{ Automatic Reward Data Generation.}
\label{sec:data}
To train a reward model capable of estimating the reward value of history trajectories, we first need to collect a set of training language instructions $\{x_m\}_{m=1}^M$, where $M$ represents the number of instruction goals. Each instruction corresponds to a set of positive trajectories $\{h_m^+\}_{m=1}^M$ that match the instruction goals and a set of negative trajectories $\{h_m^-\}_{m=1}^M$ that fail to meet the task requirements. This process typically involves human annotators and is time-consuming and labor-intensive~\citep{christiano2017deep,rafailov2024direct}. As shown in Fig.~\ref{fig:instruction_generation_sciworld} of the Appendix. we automate data collection by using Large Language Model (LLM) agents to navigate environments and summarize the navigation goals without human labels.

\noindent\textbf{Instruction Synthesis.} The first step in data generation is to propose a task instruction for a given observation. We achieve this using the in-context learning capabilities of LLMs. The prompt for instruction generation is shown in Fig.~\ref{fig:instruction_refinement_sciworld} of the Appendix. Specifically, we provide some few-shot examples in context along with the observation of an environment state to an LLM, asking it to summarize the observation and propose instruction goals. In this way, we collect a set of synthesized language instructions $\{x_m^{raw}\}_{m=1}^M$, where $M$ represents the total number of synthesized instructions.

\noindent\textbf{Trajectory Collection.} Given the synthesized instructions $x_m^{raw}$ and the environment, an LLM-based agent is instructed to take actions and navigate the environment to generate diverse trajectories $\{x_m^{raw}, h_m\}_{m=0}^M$ aimed at accomplishing the task instructions. Here, $h_m$ represents the $m$-th history trajectory, which consists of $N$ actions $\{a_n\}_{n=1}^N$ and $N+1$ environment observations $\{o_n\}_{n=0}^N$.
Due to the limited capabilities of current LLMs, the generated trajectories $h_m$ may not always align well with the synthesized task instructions $x_m$. To address this, we ask the LLM to summarize the completed trajectory $h_m$ and propose a refined goal $x_m^r$. This process results in a set of synthesized demonstrations $\{x_m^r, h_m\}_{m=0}^{M_r}$, where $M_r$ is the number of refined task instructions.

\noindent\textbf{Pairwise Data Construction.} 
To train a reward model capable of distinguishing between good and poor trajectories, we also need trajectories that do not satisfy the task instructions. To create these, we sample additional trajectories that differ from $\{x_m^r, h_m\}$ and do not meet the task requirements by modifying actions in $h_m$ and generating corresponding negative trajectories $\{h_m^-\}$. For clarity, we refer to the refined successful trajectories as $\{x_m, h_m^+\}$ and the unsuccessful ones as $\{x_m, h_m^-\}$. These paired data will be used to train the reward model described in Section~\ref{sec:model}, allowing it to estimate the reward value of any given trajectory in the environment.

\subsection{ Reward Model Design.} 
\label{sec:model}
\noindent\textbf{Reward Model Architectures.}
Theoretically, we can adopt any vision-language model that can take a sequence of visual and text inputs as the backbone for the proposed reward model. In our implementation, we use the recent VILA model~\citep{lin2023vila} as the backbone for reward modeling since it has carefully maintained open-source code, shows strong performance on standard vision-language benchmarks like~\citep{fu2023mme,balanced_vqa_v2,hudson2018gqa}, and support multiple image input. 

The goal of the reward model is to predict a reward score to estimate whether the given trajectory $(x_m, h_m)$  has satisfied the task instruction or not, which is different from the original goal of VILA models that generate a series of text tokens to respond to the task query. To handle this problem, we additionally add a fully-connected layer for the model, which linearly maps the hidden state of the last layer into a scalar value. 

\noindent\textbf{Optimazation Target.}
Given the pairwise data that is automatically synthesized from the environments in Section~\ref{sec:data}, we optimize the reward model by distinguishing the good trajectories $(x_m, h^+_m)$ from bad ones $(x_m, h^-_m)$. Following standard works of reinforcement learning from human feedback~\citep{bradley1952rank,sun2023salmon,sun2023aligning}, we treat the optimization problem of the reward model as a binary classification problem and adopt a cross-entropy loss. Formally, we have 
\begin{equation}
    \mathcal{L(\theta)} = -\mathbf{E}_{(x_m,h_m^+,h_m^-)}[\log\sigma(\mathcal{R}_\theta(x_m, h_m^+)-\mathcal{R}_\theta(x_m, h_m^-))],
\end{equation}
where $\sigma$ is the sigmoid function and $\theta$ are the learnable parameters in the reward model $\mathcal{R}$.
By optimizing this target, the reward model is trained to give higher value scores to the trajectories that are closer to the goal described in the task instruction. 

\subsection{ Planning with Large Vision-Langauge Reward Model.}
After getting the reward model to estimate how well a sampled trajectory match the given task instruction, we are able to combine it with different planning algorithms to improve LLM agents' performance. Here, we summarize the typical algorithms we can adopt in this paper.

\noindent\textbf{Best of N.} This is a simple algorithm that we can adopt the learned reward model to improve the LLM agents' performances. We first prompt the LLM agent to generate $n$ different trajectories independently and choose the one with the highest predicted reward score as the prediction for evaluation. Note that this simple method is previously used in natural language generation~\citep{zhang2024improving} and we adopt it in the context of agent tasks to study the effectiveness of the reward model for agent tasks.

\noindent\textbf{Reflexion.} Reflexion~\citep{shinn2024reflexion} is a planning framework that enables large language models (LLMs) to learn from trial-and-error without additional fine-tuning. Instead of updating model weights, Reflexion agents use verbal feedback derived from task outcomes. This feedback is converted into reflective summaries and stored in an episodic memory buffer, which informs future decisions. Reflexion supports various feedback types and improves performance across decision-making, coding, and reasoning tasks by providing linguistic reinforcement that mimics human self-reflection and learning. %This approach yields significant gains over baseline methods in several benchmarks.

\noindent\textbf{MCTS.} 
We also consider tree search-based planning algorithms like Monte Carlo Tree Search (MCTS)~\citep{coulom2006efficient,silver2017mastering} to find the optimal policy. 
There is a tree structure constructed by the algorithm, where each node represents a state and each edge signifies an action.
Beginning at the initial state of the root node, the algorithm navigates the state space to identify action and state trajectories with high rewards, as predicted by our learned reward model. 

The algorithm tracks 1) the frequency of visits to each node and 2) a value function that records the maximum predicted reward obtained from taking action ${a}$ in state ${s}$.
MCTS would visit and expand nodes with either higher values (as they lead to high predicted reward trajectory) or with smaller visit numbers (as they are under-explored).
We provide more details in the implementation details and the appendix section.


\label{sec:plan}
\section{Hierarchical Clustered 
Federated Learning } \label{sec:algorithm} 

\subsection{Overview}

\begin{figure}[tb!]
\centerline{\includegraphics[width=1\linewidth]{Figure/Algorithm.png}}
\caption{Overview of the proposed framework \algname.}
\label{fig:overview}
\end{figure}

\begin{algorithm}
\begin{algorithmic}[1]
\caption{\algname for satellite networks}\label{alg:decentralized_FL}
    \REQUIRE{Information of satellite networks, $K$, $C$}
    \ENSURE{Global model $w_{G}$}

\STATE Conduct satellite-clustered parameter server selection algorithm;  \label{line:1}

\STATE \COMMENT{Satellite client initialization model parameters}  \label{line:2}
\FOR {$\forall$ satellites}
     \STATE Initialize global model parameter $w_0$;
\ENDFOR  \label{line:5}

%\STATE All satellites: Initialize global model parameter $w^{(a)}_0$;

\FOR {each FL round $m \in M$}  \label{line:6}
    \STATE \COMMENT{Train local model (in-orbit computing)}
    \FOR {each satellite $i \in C$ in parallel}
        \STATE $w_{m,\lambda+1}^{i} = w_{m,\lambda}^{i} - \eta \nabla \tilde{f}_i(w_{m,\lambda}^{i})$;
    \ENDFOR \label{line:9}
    %\STATE \note{Conduct adaptive weight quantification algorithm to obtain $w{_{m}^{i}}^{\prime}$;}
    \STATE \COMMENT{Aggregate satellite cluster models}
    \FOR {each satellite $i \in C^k$ in the same cluster}  \label{line:12}
        \STATE $w_{m+1} = w_m + \sum_{i \in C^k_i} p_i w_m$;   \label{line:13}
    \STATE \COMMENT{Check if it is necessary to reassemble clusters}  \label{line:14}
        \STATE Calculate dropout rate: $d_r = \frac{C^d}{C^k}$
        \IF {$d_r > Z$}
            \STATE Re-cluster the satellites
        \ENDIF  \label{line:18}
    \ENDFOR
\ENDFOR  
\STATE \COMMENT{Aggregate global model at the ground station}  \label{line:21}
\FOR {each PS $G_{k}, k \in {K}$ in different satellite clusters}
    \STATE $w_{G} =  \sum_{k \in {K}} \frac{D_k}{D} w_m^k$;  
\ENDFOR  \label{line:23}
\RETURN $w_{G}$.
\end{algorithmic}
\end{algorithm}

\figurename~\ref{fig:overview} illustrates the flowchart of our proposed \algname. The clustering FL clustering process in \algname includes two stages: \textbf{satellite cluster aggregation stage} (Step 1-3) and \textbf{ground station aggregation stage} (Step 4). In the satellite cluster aggregation stage, a clustering algorithm is introduced to divide the satellites into distinct clusters. Within each satellite cluster $K_i^a$, the algorithm selects a satellite near the cluster center with strong communication capabilities to act as the PS. The PS is responsible for aggregating model parameters from satellites within its cluster and establishing communication with the corresponding ground stations (Step 1). During the local training process, satellites may dynamically join or leave a cluster, necessitating re-aggregation. To address this issue, MAML is introduced to adjust the initial model parameters of the newly joined satellites, allowing them to better adapt to the tasks of the new cluster (Step 2). This method accelerates the overall convergence of the satellite PS aggregation process. Following each training round, the PS combines parameters from all satellites within its cluster and distributes the updated, aggregated parameters back to them (Step 3).

%This approach facilitates the acceleration of the overall convergence of the satellite PS aggregation process. After each training round, the PS aggregates the parameters of the satellites in its cluster and distributes the \note{Do we need to add ''updated" here?} aggregated parameters to them (Step 3). 

After a specified number of training rounds in the satellite cluster, the ground station aggregation stage starts. In this stage, the ground station communicates with visible satellite clusters to aggregate their model parameters of the respective satellite clusters they are affiliated. Finally, the ground station returns the trained model parameters to the respective satellite clusters (Step 4).

As detailed in Algorithm~\ref{alg:decentralized_FL},  we introduce a satellite-clustered parameter server selection algorithm to partition the original satellite network into distinct satellite clusters based on the satellite network information. For each cluster, the algorithm selects a satellite near the cluster center with robust communication capabilities as the PS (line \ref{line:1}). Then we initialize the global model parameters $w_0$ for all satellite clients within each cluster (lines \ref{line:2}-\ref{line:5}). During each round of FL aggregation, the local satellite client first performs local training to update the global model parameters $w_{m,\lambda+1}^{i}$ after training round $\lambda$ (lines \ref{line:6}-\ref{line:9}). These parameters are then transmitted to their cluster's PS for aggregation. 

After $m$ rounds of training in each satellite cluster, each PS forwards its aggregated parameters to its associated ground station for global aggregation, producing the updated model parameters $w_{m+1}$ (lines \ref{line:12}-\ref{line:13}). During global aggregation, satellite clusters monitor whether the number of dropped-out satellites $C^d$ exceeds a predefined threshold, triggering re-clustering when necessary (lines \ref{line:14}-\ref{line:18}). Finally, ground station broadcasts the global parameters to all affiliated satellites in their clusters, completing the hierarchically clustered FL process (lines \ref{line:21}-\ref{line:23}). 




%As shown in Algorithm~\ref{alg:decentralized_FL}, based on the satellite network information, we introduce a satellite-clustered parameter server selection algorithm to partition the original satellite network into distinct clusters. Within each satellite cluster, the algorithm selects a satellite near the cluster center with strong communication capabilities to serve as the PS (line \ref{line:1}). Then the global model parameters are initialized for all satellite clients within the satellite cluster (lines \ref{line:2}-\ref{line:5}). During each round of FL aggregation, the local satellite client first performs local training and obtains the trained global model parameters $w_{m,\lambda+1}^{i}$ after training round $\lambda$ (lines \ref{line:6}-\ref{line:9}). The satellite clients then transmit their global parameters $w_{m,\lambda+1}^{i} $ to satellite PS for aggregate global model parameters. After completing $m$ rounds of training in each satellite cluster, the global model parameters are transmitted to the ground station associated with the satellite PS for final aggregation, yielding updated aggregated global model parameters $w_{m+1}$ (lines \ref{line:12}-\ref{line:13}). During global aggregation, satellite clusters need to assess whether the number of dropped-out satellites $C^d$ exceeds a critical threshold, necessitating re-clustering (lines \ref{line:14}-\ref{line:18}). Finally, the ground station broadcasts the global parameters to all affiliated satellites in the cluster, completing the hierarchically clustered FL process (lines \ref{line:21}-\ref{line:23}). \note{please double check this paragraph to avoid the meaning changes after my rewrite}

To accelerate the convergence of the global model, we assign weights to clients based on the quality of their model updates. The quality is evaluated using the loss value of the client's local model. Let $L_i$ denote the loss value of the client $i$. The weight $p_i$ assigned to client $i$ is given by:
\begin{equation}
p_i = \frac{\frac{1}{L_i}}{\sum_{ i \in C^k} \frac{1}{L_i}}
\label{eq:p}
\end{equation}

\begin{comment}
The local updates from the satellite clients are then weighted and aggregated according to the weight parameter $p_i$ by:
\begin{equation}
w_{m+1} = w_m + \sum_{i \in U_k} p_i Q_m^i(w_{m+1}^{i} - w_m)
\end{equation}
\end{comment}
%If $p_i$ is 0, the client has not been selected by PS.

\subsection{Satellite-clustered parameter server selection algorithm}

We introduce a satellite-clustered parameter server selection algorithm that partitions the original satellite network topology into a predefined number of clusters $K$, optimizing the clustering process. Our algorithm iteratively refines the cluster centroids and the membership of associated satellites. Initially, $K$ centroids are randomly selected from the satellite location data. These locations are typically derived from geographic coordinates based on the satellite location parameters, i.e., inclination and orbital altitude. Each satellite is assigned to the nearest cluster centroid using the Euclidean distance metric, thereby forming initial clusters. The Euclidean distance between a satellite position vector
$\mathbf{C}^i = \{C_{1}^i, C_{2}^i, \ldots, C_{n}^i \}$ and a centroid $\mathbf{K}^j = \{K_{1}^j, K_{2}^j, \ldots, K_{n}^j \}$ is calculated as:
\begin{equation}
d(\mathbf{C}^i, \mathbf{K}^j) = \sqrt{\sum_{k=1}^{n} (C_{k}^i - K_{k}^j)^2}
\end{equation}

In the next update step, our algorithm recalculates the centroids by computing the mean position of all satellites assigned to each cluster. This process effectively repositions the centroids to more accurately represent the distribution of their associated satellites. For each cluster $K_k^i$, the new centroid $\mathbf{K}^j$ is obtained by:
\begin{equation}
\mathbf{K}^j = \frac{1}{|K^j|} \sum_{\mathbf{C}^i \in K^j} \mathbf{C}^i
\end{equation}
where $|K^j|$ represents the number of satellites in cluster $K^j$. The iterative process continues until the centroids stabilize, indicating their positions no longer change significantly between iterations. This indicates that the algorithm has converged to a local optimum. The convergence criterion is given by:
\begin{equation}
\sum_{j=1}^{|K|} \|\mathbf{K}_{\text{new}}^j - \mathbf{K}_{\text{old}}^j\|^2 < \epsilon
\end{equation}
where $|K|$ represents the number of clusters, and $\epsilon$ is a small positive number indicating stability in centroid positions.
The satellite nearest to the cluster centroid is designated as the PS for the respective cluster.


\subsection{Meta-learning-driven satellite re-clustering algorithm}

In dynamic satellite federated learning, the diverse training objectives of satellite clients, combined with their frequent network participation changes, can hinder model convergence and increase resource consumption. As a result, achieving acceptable performance requires substantial time and a large number of data samples.
%In dynamic satellite federated learning, the diverse training goals of satellite clients, coupled with their frequent network participation changes, can impede model convergence, resulting in heightened resource utilization. This necessitates a significant investment of time and data samples to attain acceptable performance. %the different training objectives of satellite clients, along with their frequent joining or leaving the network, can slow down model convergence, leading to increased resource consumption. This requires a large amount of time and data samples to achieve acceptable results. 

To address this challenge, we propose a satellite re-clustering algorithm based on meta-learning, extending the original satellite-clustered parameter server selection algorithm. When a new satellite joins the network, it inherits model updates from the head node of a specified cluster, rather than starting training from scratch. The core idea of the MAML approach is to identify a set of meta-initialization parameters that enable the model to achieve strong performance with just one or two gradient updates, even with a small number of new task examples.

First, we sample satellite clients from different clusters denoted as $ S = \{S_1, S_2, \dots, S_n\} $. Each satellite client is assigned a task $\textit{TK}_i$, which consists of a dataset $D_i$  and a loss function $L_{S_i}(w)$. The objective is to minimize the loss of the model on the task $\textit{TK}_i$. Then, an inner-loop adaptation is performed for each selected satellite node to fine-tune the global model $w$ by:
\begin{equation}
w_i' = w - \alpha \nabla_{w} L_{S_i}(w)
\end{equation}
where $\alpha$ is the local learning rate. Finally, an outer-loop meta-update is applied to aggregate the model updates from different satellite nodes, updating the global initialization by:
\begin{equation}
w^{t+1} = w^{t} - \beta \sum_{i \in S} \nabla_{w} L_{S_i}(w_i')
\end{equation}
where $\beta$ is the meta-learning rate, $w^{t}$ is the parameter of the current global model.


\section{Experiments}
\label{section5}

In this section, we conduct extensive experiments to show that \ourmethod~can significantly speed up the sampling of existing MR Diffusion. To rigorously validate the effectiveness of our method, we follow the settings and checkpoints from \cite{luo2024daclip} and only modify the sampling part. Our experiment is divided into three parts. Section \ref{mainresult} compares the sampling results for different NFE cases. Section \ref{effects} studies the effects of different parameter settings on our algorithm, including network parameterizations and solver types. In Section \ref{analysis}, we visualize the sampling trajectories to show the speedup achieved by \ourmethod~and analyze why noise prediction gets obviously worse when NFE is less than 20.


\subsection{Main results}\label{mainresult}

Following \cite{luo2024daclip}, we conduct experiments with ten different types of image degradation: blurry, hazy, JPEG-compression, low-light, noisy, raindrop, rainy, shadowed, snowy, and inpainting (see Appendix \ref{appd1} for details). We adopt LPIPS \citep{zhang2018lpips} and FID \citep{heusel2017fid} as main metrics for perceptual evaluation, and also report PSNR and SSIM \citep{wang2004ssim} for reference. We compare \ourmethod~with other sampling methods, including posterior sampling \citep{luo2024posterior} and Euler-Maruyama discretization \citep{kloeden1992sde}. We take two tasks as examples and the metrics are shown in Figure \ref{fig:main}. Unless explicitly mentioned, we always use \ourmethod~based on SDE solver, with data prediction and uniform $\lambda$. The complete experimental results can be found in Appendix \ref{appd3}. The results demonstrate that \ourmethod~converges in a few (5 or 10) steps and produces samples with stable quality. Our algorithm significantly reduces the time cost without compromising sampling performance, which is of great practical value for MR Diffusion.


\begin{figure}[!ht]
    \centering
    \begin{minipage}[b]{0.45\textwidth}
        \centering
        \includegraphics[width=1\textwidth, trim=0 20 0 0]{figs/main_result/7_lowlight_fid.pdf}
        \subcaption{FID on \textit{low-light} dataset}
        \label{fig:main(a)}
    \end{minipage}
    \begin{minipage}[b]{0.45\textwidth}
        \centering
        \includegraphics[width=1\textwidth, trim=0 20 0 0]{figs/main_result/7_lowlight_lpips.pdf}
        \subcaption{LPIPS on \textit{low-light} dataset}
        \label{fig:main(b)}
    \end{minipage}
    \begin{minipage}[b]{0.45\textwidth}
        \centering
        \includegraphics[width=1\textwidth, trim=0 20 0 0]{figs/main_result/10_motion_fid.pdf}
        \subcaption{FID on \textit{motion-blurry} dataset}
        \label{fig:main(c)}
    \end{minipage}
    \begin{minipage}[b]{0.45\textwidth}
        \centering
        \includegraphics[width=1\textwidth, trim=0 20 0 0]{figs/main_result/10_motion_lpips.pdf}
        \subcaption{LPIPS on \textit{motion-blurry} dataset}
        \label{fig:main(d)}
    \end{minipage}
    \caption{\textbf{Perceptual evaluations on \textit{low-light} and \textit{motion-blurry} datasets.}}
    \label{fig:main}
\end{figure}

\subsection{Effects of parameter choice}\label{effects}

In Table \ref{tab:ablat_param}, we compare the results of two network parameterizations. The data prediction shows stable performance across different NFEs. The noise prediction performs similarly to data prediction with large NFEs, but its performance deteriorates significantly with smaller NFEs. The detailed analysis can be found in Section \ref{section5.3}. In Table \ref{tab:ablat_solver}, we compare \ourmethod-ODE-d-2 and \ourmethod-SDE-d-2 on the \textit{inpainting} task, which are derived from PF-ODE and reverse-time SDE respectively. SDE-based solver works better with a large NFE, whereas ODE-based solver is more effective with a small NFE. In general, neither solver type is inherently better.


% In Table \ref{tab:hazy}, we study the impact of two step size schedules on the results. On the whole, uniform $\lambda$ performs slightly better than uniform $t$. Our algorithm follows the method of \cite{lu2022dpmsolverplus} to estimate the integral part of the solution, while the analytical part does not affect the error.  Consequently, our algorithm has the same global truncation error, that is $\mathcal{O}\left(h_{max}^{k}\right)$. Note that the initial and final values of $\lambda$ depend on noise schedule and are fixed. Therefore, uniform $\lambda$ scheduling leads to the smallest $h_{max}$ and works better.

\begin{table}[ht]
    \centering
    \begin{minipage}{0.5\textwidth}
    \small
    \renewcommand{\arraystretch}{1}
    \centering
    \caption{Ablation study of network parameterizations on the Rain100H dataset.}
    % \vspace{8pt}
    \resizebox{1\textwidth}{!}{
        \begin{tabular}{cccccc}
			\toprule[1.5pt]
            % \multicolumn{6}{c}{Rainy} \\
            % \cmidrule(lr){1-6}
             NFE & Parameterization      & LPIPS\textdownarrow & FID\textdownarrow &  PSNR\textuparrow & SSIM\textuparrow  \\
            \midrule[1pt]
            \multirow{2}{*}{50}
             & Noise Prediction & \textbf{0.0606}     & \textbf{27.28}   & \textbf{28.89}     & \textbf{0.8615}    \\
             & Data Prediction & 0.0620     & 27.65   & 28.85     & 0.8602    \\
            \cmidrule(lr){1-6}
            \multirow{2}{*}{20}
              & Noise Prediction & 0.1429     & 47.31   & 27.68     & 0.7954    \\
              & Data Prediction & \textbf{0.0635}     & \textbf{27.79}   & \textbf{28.60}     & \textbf{0.8559}    \\
            \cmidrule(lr){1-6}
            \multirow{2}{*}{10}
              & Noise Prediction & 1.376     & 402.3   & 6.623     & 0.0114    \\
              & Data Prediction & \textbf{0.0678}     & \textbf{29.54}   & \textbf{28.09}     & \textbf{0.8483}    \\
            \cmidrule(lr){1-6}
            \multirow{2}{*}{5}
              & Noise Prediction & 1.416     & 447.0   & 5.755     & 0.0051    \\
              & Data Prediction & \textbf{0.0637}     & \textbf{26.92}   & \textbf{28.82}     & \textbf{0.8685}    \\       
            \bottomrule[1.5pt]
        \end{tabular}}
        \label{tab:ablat_param}
    \end{minipage}
    \hspace{0.01\textwidth}
    \begin{minipage}{0.46\textwidth}
    \small
    \renewcommand{\arraystretch}{1}
    \centering
    \caption{Ablation study of solver types on the CelebA-HQ dataset.}
    % \vspace{8pt}
        \resizebox{1\textwidth}{!}{
        \begin{tabular}{cccccc}
			\toprule[1.5pt]
            % \multicolumn{6}{c}{Raindrop} \\     
            % \cmidrule(lr){1-6}
             NFE & Solver Type     & LPIPS\textdownarrow & FID\textdownarrow &  PSNR\textuparrow & SSIM\textuparrow  \\
            \midrule[1pt]
            \multirow{2}{*}{50}
             & ODE & 0.0499     & 22.91   & 28.49     & 0.8921    \\
             & SDE & \textbf{0.0402}     & \textbf{19.09}   & \textbf{29.15}     & \textbf{0.9046}    \\
            \cmidrule(lr){1-6}
            \multirow{2}{*}{20}
              & ODE & 0.0475    & 21.35   & 28.51     & 0.8940    \\
              & SDE & \textbf{0.0408}     & \textbf{19.13}   & \textbf{28.98}    & \textbf{0.9032}    \\
            \cmidrule(lr){1-6}
            \multirow{2}{*}{10}
              & ODE & \textbf{0.0417}    & 19.44   & \textbf{28.94}     & \textbf{0.9048}    \\
              & SDE & 0.0437     & \textbf{19.29}   & 28.48     & 0.8996    \\
            \cmidrule(lr){1-6}
            \multirow{2}{*}{5}
              & ODE & \textbf{0.0526}     & 27.44   & \textbf{31.02}     & \textbf{0.9335}    \\
              & SDE & 0.0529    & \textbf{24.02}   & 28.35     & 0.8930    \\
            \bottomrule[1.5pt]
        \end{tabular}}
        \label{tab:ablat_solver}
    \end{minipage}
\end{table}


% \renewcommand{\arraystretch}{1}
%     \centering
%     \caption{Ablation study of step size schedule on the RESIDE-6k dataset.}
%     % \vspace{8pt}
%         \resizebox{1\textwidth}{!}{
%         \begin{tabular}{cccccc}
% 			\toprule[1.5pt]
%             % \multicolumn{6}{c}{Raindrop} \\     
%             % \cmidrule(lr){1-6}
%              NFE & Schedule      & LPIPS\textdownarrow & FID\textdownarrow &  PSNR\textuparrow & SSIM\textuparrow  \\
%             \midrule[1pt]
%             \multirow{2}{*}{50}
%              & uniform $t$ & 0.0271     & 5.539   & 30.00     & 0.9351    \\
%              & uniform $\lambda$ & \textbf{0.0233}     & \textbf{4.993}   & \textbf{30.19}     & \textbf{0.9427}    \\
%             \cmidrule(lr){1-6}
%             \multirow{2}{*}{20}
%               & uniform $t$ & 0.0313     & 6.000   & 29.73     & 0.9270    \\
%               & uniform $\lambda$ & \textbf{0.0240}     & \textbf{5.077}   & \textbf{30.06}    & \textbf{0.9409}    \\
%             \cmidrule(lr){1-6}
%             \multirow{2}{*}{10}
%               & uniform $t$ & 0.0309     & 6.094   & 29.42     & 0.9274    \\
%               & uniform $\lambda$ & \textbf{0.0246}     & \textbf{5.228}   & \textbf{29.65}     & \textbf{0.9372}    \\
%             \cmidrule(lr){1-6}
%             \multirow{2}{*}{5}
%               & uniform $t$ & 0.0256     & 5.477   & \textbf{29.91}     & 0.9342    \\
%               & uniform $\lambda$ & \textbf{0.0228}     & \textbf{5.174}   & 29.65     & \textbf{0.9416}    \\
%             \bottomrule[1.5pt]
%         \end{tabular}}
%         \label{tab:ablat_schedule}



\subsection{Analysis}\label{analysis}
\label{section5.3}

\begin{figure}[ht!]
    \centering
    \begin{minipage}[t]{0.6\linewidth}
        \centering
        \includegraphics[width=\linewidth, trim=0 20 10 0]{figs/trajectory_a.pdf} %trim左下右上
        \subcaption{Sampling results.}
        \label{fig:traj(a)}
    \end{minipage}
    \begin{minipage}[t]{0.35\linewidth}
        \centering
        \includegraphics[width=\linewidth, trim=0 0 0 0]{figs/trajectory_b.pdf} %trim左下右上
        \subcaption{Trajectory.}
        \label{fig:traj(b)}
    \end{minipage}
    \caption{\textbf{Sampling trajectories.} In (a), we compare our method (with order 1 and order 2) and previous sampling methods (i.e., posterior sampling and Euler discretization) on a motion blurry image. The numbers in parentheses indicate the NFE. In (b), we illustrate trajectories of each sampling method. Previous methods need to take many unnecessary paths to converge. With few NFEs, they fail to reach the ground truth (i.e., the location of $\boldsymbol{x}_0$). Our methods follow a more direct trajectory.}
    \label{fig:traj}
\end{figure}

\textbf{Sampling trajectory.}~ Inspired by the design idea of NCSN \citep{song2019ncsn}, we provide a new perspective of diffusion sampling process. \cite{song2019ncsn} consider each data point (e.g., an image) as a point in high-dimensional space. During the diffusion process, noise is added to each point $\boldsymbol{x}_0$, causing it to spread throughout the space, while the score function (a neural network) \textit{remembers} the direction towards $\boldsymbol{x}_0$. In the sampling process, we start from a random point by sampling a Gaussian distribution and follow the guidance of the reverse-time SDE (or PF-ODE) and the score function to locate $\boldsymbol{x}_0$. By connecting each intermediate state $\boldsymbol{x}_t$, we obtain a sampling trajectory. However, this trajectory exists in a high-dimensional space, making it difficult to visualize. Therefore, we use Principal Component Analysis (PCA) to reduce $\boldsymbol{x}_t$ to two dimensions, obtaining the projection of the sampling trajectory in 2D space. As shown in Figure \ref{fig:traj}, we present an example. Previous sampling methods \citep{luo2024posterior} often require a long path to find $\boldsymbol{x}_0$, and reducing NFE can lead to cumulative errors, making it impossible to locate $\boldsymbol{x}_0$. In contrast, our algorithm produces more direct trajectories, allowing us to find $\boldsymbol{x}_0$ with fewer NFEs.

\begin{figure*}[ht]
    \centering
    \begin{minipage}[t]{0.45\linewidth}
        \centering
        \includegraphics[width=\linewidth, trim=0 0 0 0]{figs/convergence_a.pdf} %trim左下右上
        \subcaption{Sampling results.}
        \label{fig:convergence(a)}
    \end{minipage}
    \begin{minipage}[t]{0.43\linewidth}
        \centering
        \includegraphics[width=\linewidth, trim=0 20 0 0]{figs/convergence_b.pdf} %trim左下右上
        \subcaption{Ratio of convergence.}
        \label{fig:convergence(b)}
    \end{minipage}
    \caption{\textbf{Convergence of noise prediction and data prediction.} In (a), we choose a low-light image for example. The numbers in parentheses indicate the NFE. In (b), we illustrate the ratio of components of neural network output that satisfy the Taylor expansion convergence requirement.}
    \label{fig:converge}
\end{figure*}

\textbf{Numerical stability of parameterizations.}~ From Table 1, we observe poor sampling results for noise prediction in the case of few NFEs. The reason may be that the neural network parameterized by noise prediction is numerically unstable. Recall that we used Taylor expansion in Eq.(\ref{14}), and the condition for the equality to hold is $|\lambda-\lambda_s|<\boldsymbol{R}(s)$. And the radius of convergence $\boldsymbol{R}(t)$ can be calculated by
\begin{equation}
\frac{1}{\boldsymbol{R}(t)}=\lim_{n\rightarrow\infty}\left|\frac{\boldsymbol{c}_{n+1}(t)}{\boldsymbol{c}_n(t)}\right|,
\end{equation}
where $\boldsymbol{c}_n(t)$ is the coefficient of the $n$-th term in Taylor expansion. We are unable to compute this limit and can only compute the $n=0$ case as an approximation. The output of the neural network can be viewed as a vector, with each component corresponding to a radius of convergence. At each time step, we count the ratio of components that satisfy $\boldsymbol{R}_i(s)>|\lambda-\lambda_s|$ as a criterion for judging the convergence, where $i$ denotes the $i$-th component. As shown in Figure \ref{fig:converge}, the neural network parameterized by data prediction meets the convergence criteria at almost every step. However, the neural network parameterized by noise prediction always has components that cannot converge, which will lead to large errors and failed sampling. Therefore, data prediction has better numerical stability and is a more recommended choice.


\paragraph{Summary}
Our findings provide significant insights into the influence of correctness, explanations, and refinement on evaluation accuracy and user trust in AI-based planners. 
In particular, the findings are three-fold: 
(1) The \textbf{correctness} of the generated plans is the most significant factor that impacts the evaluation accuracy and user trust in the planners. As the PDDL solver is more capable of generating correct plans, it achieves the highest evaluation accuracy and trust. 
(2) The \textbf{explanation} component of the LLM planner improves evaluation accuracy, as LLM+Expl achieves higher accuracy than LLM alone. Despite this improvement, LLM+Expl minimally impacts user trust. However, alternative explanation methods may influence user trust differently from the manually generated explanations used in our approach.
% On the other hand, explanations may help refine the trust of the planner to a more appropriate level by indicating planner shortcomings.
(3) The \textbf{refinement} procedure in the LLM planner does not lead to a significant improvement in evaluation accuracy; however, it exhibits a positive influence on user trust that may indicate an overtrust in some situations.
% This finding is aligned with prior works showing that iterative refinements based on user feedback would increase user trust~\cite{kunkel2019let, sebo2019don}.
Finally, the propensity-to-trust analysis identifies correctness as the primary determinant of user trust, whereas explanations provided limited improvement in scenarios where the planner's accuracy is diminished.

% In conclusion, our results indicate that the planner's correctness is the dominant factor for both evaluation accuracy and user trust. Therefore, selecting high-quality training data and optimizing the training procedure of AI-based planners to improve planning correctness is the top priority. Once the AI planner achieves a similar correctness level to traditional graph-search planners, strengthening its capability to explain and refine plans will further improve user trust compared to traditional planners.

\paragraph{Future Research} Future steps in this research include expanding user studies with larger sample sizes to improve generalizability and including additional planning problems per session for a more comprehensive evaluation. Next, we will explore alternative methods for generating plan explanations beyond manual creation to identify approaches that more effectively enhance user trust. 
Additionally, we will examine user trust by employing multiple LLM-based planners with varying levels of planning accuracy to better understand the interplay between planning correctness and user trust. 
Furthermore, we aim to enable real-time user-planner interaction, allowing users to provide feedback and refine plans collaboratively, thereby fostering a more dynamic and user-centric planning process.



\begin{comment}
\section*{Acknowledgment}

The preferred spelling of the word ``acknowledgment'' in America is without 
an ``e'' after the ``g''. Avoid the stilted expression ``one of us (R. B. 
G.) thanks $\ldots$''. Instead, try ``R. B. G. thanks$\ldots$''. Put sponsor 
acknowledgments in the unnumbered footnote on the first page.
\end{comment}



\bibliographystyle{ieeetr} 
\bibliography{ref}

\begin{comment}
%表格暂时写在上面,方便对照
\begin{table}[ht]
\centering
\caption{Notations and Descriptions\note{[REMOVE THIS TABLE AT LAST]}}
\begin{tabular}{|c|l|}
\hline
\textbf{Inputs} & \textbf{Description} \\ \hline
$C$ & the set of candidate clients in FL \\ \hline
$K$ & the set of clusters \\ \hline
$D_i$ & Each client $c_i$ holds a local dataset size \\ \hline
$p_k$ & the cluster head of cluster satellite $k$ \\ \hline
$S_k$ & the clients belongs to cluster $k$ \\ \hline
$G$ & ground station number $G$ \\ \hline
$g_{i}^{p_k}$ & the cluster$p_k$ belongs to ground station i  \\ \hline
$t^{cmp}_i$ & computation time of client $i$ for one intra-cluster aggregation \\ \hline
$t^{com}_i$ & communication time of client $i$ for one intra-cluster aggregation \\ \hline
$t^{com}_k$ & communication time of cluster head $l_k$ for one inter-cluster aggregation \\ \hline
$n_k$ & Maximum selected number of clients per cluster per global round \\ \hline
\textbf{Outputs} & \textbf{Description} \\ \hline
$x^k_i$ & whether client $i$ belongs to cluster $k$ \\ \hline
$y^t_i$ & whether or not select client $i$ at global round $t$ \\ \hline
$\tau_{i}^{k}$ & local iterations given to client ci in the k-th communication round \\ \hline
\end{tabular}
\end{table}
\end{comment}

\end{document}
