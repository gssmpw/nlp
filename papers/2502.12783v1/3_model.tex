\section{System Model}\label{sec:model}



%如图1所示,我们考虑在数量为C的卫星集群网络中,将卫星网络划分为K个区域,每个区域选择一个LEO卫星当做参数服务器,来对集群内的卫星客户端进行联邦学习训练,地面站选择gpk个参数服务器作为它的客户端进行全局聚合

\subsection{Network model}
\figurename~\ref{fig:model} illustrates the assumed satellite network model, which consists of multiple LEO satellites operating at altitudes between 500 and 2000 km. These satellites work in conjunction with ground stations to provide communication services. In this paper, we designate satellites as PS, eliminating the need to download local satellite data to the ground.
The satellite network is divided into $K$ regions (\textbf{clusters}), where each cluster contains $C^k$ satellites (\textbf{clients}). Each ellipse in \figurename~\ref{fig:model} represents an independent satellite cluster. Within each cluster, one satellite is selected as the PS to perform FL training on the satellite clients within the cluster. Once FL training is completed within the satellite clusters, each ground station $G$ selects satellite PS and aggregates their models globally. After global aggregation, the updated model parameters are transmitted back to the corresponding satellite PS, completing the FL process.

%After completing FL training within the satellite cluster, the ground station $g$ selects no more than $K$ satellite PS as its clients for global aggregation, where $p_k$ represents the satellite cluster PS to which the ground station belongs.

We assume that the ground stations operate independently without intercommunication. Each ground station has visibility of all satellites within its assigned cluster, and satellite motion follows a certain periodicity. %and we divide the network topology period into $\sigma $ time slots.
%, each with a duration of $u$. During each time slot the satellite network topology remains constant, allowing it to be treated as a quasi-static topology\cite{DuTAES17}.





\begin{figure}[tb!]
\centerline{\includegraphics[width=1\linewidth, height=0.6\linewidth]{Figure/framework.png}}
\caption{The network model with federated learning.}
\label{fig:model}
\end{figure}



\subsection{Federated learning model}
The objective of satellite FL is to train a shared learning model that enables collaboration among multiple satellite clients without transmitting local private data to a central parameter server. Specifically, the learning objective is to determine suitable model parameters $\mathbf{w}$ to predict the data for input information. We consider a satellite network comprising $C$ clients participating in FL, where each client $i \in C$ possesses a local dataset $D_i$ of size $|D_i|$. 
\begin{comment}
The local loss function of satellite client $i$ on all its data is defined as: 
\begin{equation}
F_i(x) = \frac{1}{|D_i|} \sum_{\Lambda \in D_i} f_i(w; \Lambda)
\end{equation}
where $f_i(w; \Lambda)$ is the loss function calculated by a specific sample $\Lambda$. 
\end{comment}
Therefore, the global loss function can be represented as
\begin{equation}
F(x) = \frac{1}{|D|} \sum_{i \in C} \sum_{\Lambda \in D_i} f_i(w; \Lambda)
\end{equation}
where $f_i(w; \Lambda)$ is the loss function calculated by a specific sample $\Lambda$, and $|D|=\sum_{i \in C}|D_i| $ is the total dataset size across all clients. %The objective of FL is to determine the optimal model parameters $w^* = \arg \min_w L(w)$, that minimize the global loss function. 
The objective of FL is to determine the optimal model parameters $w^*$ that minimizes the global loss function as: 
%\begin{comment}
\begin{equation}
w^* = \arg \min_w L(w)
\end{equation}
%\end{comment}

In each aggregation round $m$, the satellite PS broadcasts the global model \(w_m\) to all participating clients. Each client $i$ performs a local gradient descent to update its local parameter \(w_m^i\). During each FL aggregation round, the clients update their parameters over $\lambda$ epochs of local training. The gradient is then computed using Stochastic gradient descent (SGD) as:
\begin{equation}
    \nabla \tilde f_i(w_{m,\lambda}^{i}) = \nabla  l(w_{m,\lambda}^{i}; \Lambda)
\end{equation} 



For each epoch $\lambda$, the evolution of local model parameter \(w_{m,\lambda}^{i}\) by:
\begin{equation}
w_{m,\lambda+1}^{i} = w_{m,\lambda}^{i} - \eta \nabla \tilde{f}_i(w_{m,\lambda}^{i})
\label{eq:evo}
\end{equation}
where $\eta$ represents the learning rate of model. 

\begin{comment}
and the local gradient is defined as:
\begin{equation}
 \nabla \tilde{f}_i(x_{t,\tau}^i) = \nabla L(x_{t,\tau}^i; \Lambda_{t,\tau}^i)   
\end{equation}
\end{comment}

Upon completing the local update, the client sends the weight $w_m^i$ to the FL processing server. The server subsequently aggregates these weights to create the global model parameter for the next FL round, expressed as:
\begin{equation}
w_{m+1} =  \sum_{i \in C} \frac{D_i}{D} w_m^i
\end{equation}

\subsection{Satellite cluster problem formulation}

\textbf{Processing time analysis:} We assume that at a given time $t_0$, multiple satellite clusters perform local training in parallel, with each client $i$ holding a local dataset  $D_i$. The computing power (i.e. CPU frequency) of client $i$ is denoted by $f_i$ and $Q$  represents the number of CPU cycles required to train a single data sample. Therefore, the computation time for client $i$ is given by $t_{i}^{cmp}= {D_i Q_i}/{f_i}$.
\begin{comment}
\begin{equation}
t_{i}^{cmp}= \frac{D_i Q_i}{f_i}
\end{equation}
\end{comment}

After local training, the client $i$ uploads its local models to the satellite PS for global model aggregation. The achievable transmission rate of client $i$ is expressed as:
\begin{equation}
r_i = B_i \ln \left(1 + \frac{P_0 h_i}{N_0}\right)
\end{equation}
where $N_0$ represents background noise power, $P_0$ represents transmission power, $B_i$ is transmission bandwidth, and $h_i$ is channel gain. We assume that the size of data that each client needs to upload is a constant  $\zeta$, so the communication time of client $i$ is $t_{i}^{com} = {\zeta}/{r_i}$. We further use $T^j_i$ to represent the entire training time of the $i$-th client at communication round $j$ , including computation and communication time as $T_{i}^{j} =  t_{i}^{\text{cmp}} + t_{i}^{\text{com}} $.
\begin{comment}
\begin{equation}
T_{i}^{j} = \tau_{i}^{j} ( t_{i}^{\text{cmp}} + t_{i}^{\text{com}} )
%\nonumber 
        = \tau_{i}^{j}( \frac{D_i Q_i}{f_i} + \frac{\zeta}{B_i \ln(1 + \frac{\rho_i h_i}{N_0})} )
\end{equation}
\end{comment}

In the synchronized FL system, the global model aggregation starts only when the server receives updates from all participating clients. Hence, the training time for the communication round j, indicated by $T_j$, can be calculated by $T_j = \max_{i \in C_j} \{ T_{i}^{j} \}$.
\begin{comment}
\begin{align}
T_j &= \max_{i \in C_j} \{ T_{i}^{j} \}
\end{align}
\end{comment}
It is worth noting that the value of $T_j$ is largely determined by the slowest client in $C_j$.
Beyond the aggregation time within the satellite cluster, the communication time for parameter server $p_k$ to aggregate and broadcast the global model back to its satellite cluster must also be considered. The satellite PS of the ground station is denoted as $g_{p_k}$. The satellite PS of the $i$-th cluster is $K_i^a$. Therefore, the total processing time of FL can be expressed as:
\begin{equation}
T_c = \sum_{K_i^a \in {g_{p_k}} } T_j = \sum_{K_i^a\in {g_{p_k}}} \max_{i \in C_j} \left( \max_{i \in C^k}  \left(t_{i}^{\text{cmp}} + t_{i}^{\text{com}}\right) + t^{\text{com}}_j \right)
\label{eq:time}
\end{equation}


\textbf{Energy consumption analysis}:
The energy consumption in FL can be divided into two parts: the first part is the transmission energy consumption, which arises as satellites upload their local models to the satellite parameter server.  After global aggregation at the ground station, the updated model must be broadcast back to each satellite client. Therefore, the transmission energy consumption for satellite model distribution, denoted as $E_{\text{tr}}$, is defined as:
\begin{equation}
E_{\text{tr}} = \sum_{i \in C} P_0 \frac{|w_i|}{r_i}
\end{equation}
where $|w_i|$ is the size of the global model parameters for the $i$-th client. Subsequently, the aggregation energy \(E_{\text{agg}}\) in the satellite cluster can be derived as:
\begin{equation}
E_{\text{agg}} = \sum_{i \in K} \epsilon_0 f_i t_i^{\text{cmp}}
\end{equation}
where $\epsilon_0$ is the constant coefficient determined by the hard-
ware architecture. Therefore, the total energy consumption is: %can be defined as follows:
\begin{equation}
E_c= \min ( E_{\text{tr}} + E_{\text{agg}}) 
\label{eq:energy}
\end{equation}


\begin{comment}
s.t.
\begin{subequations}

\begin{equation}
x_a \in \{0, 1\}, \ \forall a \in A,
\end{equation}
\begin{equation}
y_v \in \{0, 1\}, \ \forall v \in V,
\end{equation}

\end{subequations}

\begin{equation}
r \in \{v_{\text{l,T}} | l \in L\},
\end{equation}

\begin{equation}
y_r = 1,
\end{equation}

\begin{equation}
x_a \in \{0, 1\}, \ \forall a \in A,
\end{equation}

\begin{equation}
y_v \in \{0, 1\}, \ \forall v \in V,
\end{equation}

\begin{equation}
x_a \leq y_v, \ x_a \leq y_{v'}, \ \forall a = (v, v') \in A.
\end{equation}
\end{comment}

\textbf{Objective function}: The objective of this problem is to minimize both processing time $T_c$ (Equation~\ref{eq:time}) and energy consumption  $E_c$ (Equation~\ref{eq:energy}) generated during the FL process. Given the constraints of satellite communication and computational capacity, the objective function $F(C,K)$ can be formulated as:
\begin{align}
F(C,K) &= \min  (T_c , E_c)
\end{align}

%在处理FL的整个过程中,在满足所有数据不超过,卫星的计算容量上限和通信容量上限的前提下,我们objective function 可以被表示为





\iffalse
s.t.
\begin{subequations}
\begin{equation}
x_a \in \{0, 1\}, \ \forall a \in A,\label{eq:x_a}
\end{equation}
\begin{equation}
y_v \in \{0, 1\}, \ \forall v \in V,\label{eq:y_v}
\end{equation}
\begin{equation}
\Psi \in \{0, 1\}
\end{equation}
\begin{equation}
     0 \leq t_{i} \leq t_{\text{max}},\label{eq:t}
\end{equation}


%\begin{equation}
%     0 \leq p_{\tau_{n,k}} \leq P_{\text{max}},
%\end{equation}
\
\end{subequations}
where constraints~\ref{eq:x_a} and \ref{eq:y_v} confirm that $x_a$ and $y_v$ are binary decision variables, while $\Psi$ is a weight parameter. Constraint~\ref{eq:t} denotes that the transmission delay should not exceed the maximum allowable delay.

\fi