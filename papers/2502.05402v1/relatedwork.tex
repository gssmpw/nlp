\section{Related Work}
With the rise of deep learning, many different studies have focused on a broad range of deep learning techniques and model architectures to accomplish the task of adding color to black and white images \cite{Cheng2015}. These approaches can quickly and easily handle large amounts of data and have demonstrated effective results. Since the first deep learning image colorization method in 2015 \cite{Cheng2015}, the field has rapidly grown. Various approaches have been applied, such as convolutional neural networks \cite{Larsson2016, He2018, Zhang2017, Su2020, Dong2022}, which allow for retaining spatial information, generative adversarial networks (GANs) \cite{Zhang2018, Kuang2020}, which are advantageous for preserving details and high-quality image generation, and transformer networks \cite{Kumar2021} that excel at applying color with context to the entire image. Previous studies have also identified that color fidelity was improved when some original color from the image was retained \cite{FatimaAroosh2021Gitn, Boutarfass2020}. In \cite{FatimaAroosh2021Gitn}, they applied two different automated approaches for choosing the pixel location(s) for color retention. Such approaches included grid-based, where every n-th pixel remained colored, and segment-based, where segments in the images were identified and a pixel within each segment was chosen to retain color. Another recent study of particular interest for our paper is \cite{XiaoYi2022IDCa} which proposes a convolutional, U-net architecture for image recolorization. Hence forth we will refer to this model architecture as \textbf{XiaoNet}. Our study will focus on the marriage of two methods: the use of grid-based, automated color retention like the method described in \cite{FatimaAroosh2021Gitn} with a convolutional U-net architecture derived from the XiaoNet architecture. Furthermore, we aim to find the optimal amount of color-information retention to maximize the image compression while maintaining enough information to make highly accurate predictions using our XiaoNet-like architecture More information on our architecture is available in Section \ref{sec:Model Arch}.