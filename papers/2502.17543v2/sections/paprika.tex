\section{\ours{}}

\begin{table*}[h]
    \caption{\footnotesize Summary of the task groups used by \ours{}. }
    \label{tab:environment_summary}
    \centering
    \resizebox{\textwidth}{!}{%
        \begin{tabular}{c|c c c c c}
        \toprule
            Task Group & \# Train Tasks & \# Test Tasks & Maximum Turns & Env Feedback & Uses COT \\
            \midrule
             Twenty questions & 1499 & 367 & 20 & LLM generated & \ding{55} \\
             Guess my city & 500 & 185 & 20 & LLM generated & \ding{55} \\
             Wordle & 1515 & 800 & 6 & Hardcoded program & \checkmark \\
             Cellular automata & 1000 & 500 & 6 & Hardcoded program & \checkmark \\
             Customer service & 628 & 200 & 20 & LLM generated & \ding{55} \\
             Murder mystery & 203 & 50 & 20 & LLM generated & \ding{55} \\
             Mastermind & 1000 & 500 & 12 & Hardcoded program & \checkmark \\
             Battleship & 1000 & 200 & 20 & Hardcoded program & \checkmark \\
             Minesweeper & 1000 & 200 & 20 & Hardcoded program & \checkmark \\
             Bandit best arm selection & 81 & 1 & 21 & Hardcoded program & \checkmark \\
             \bottomrule
        \end{tabular}
    }
    \vspace{-0.3cm}
\end{table*}

The goal of our paper is to develop a scalable method to instill better strategic exploration and sequential decision-making capabilities into LLMs. Prior works~\citep{krishnamurthy2024largelanguagemodelsexplore} have shown that LLMs can perform poorly on even the simple decision making task of multi-armed bandits. \citet{nie2024evolveevaluatingoptimizingllms} has since then demonstrated that LLMs can be taught to perform better on bandits after fine-tuning them on synthetic trajectories generated by known algorithms such as UCB. However, this idea is limited in scope for three reasons: \textbf{(1)} we want LLMs to perform strategic exploration and decision making in more complex settings, \textbf{(2)} for most tasks, there is no known algorithm like UCB to generate good synthetic trajectories from, \textbf{(3)} it can be infeasible to collect data for all tasks that we care about.

We aim to solve these issues using our method, \ours{}. First, we design a suite of complex decision-making tasks that require strategic information gathering to succeed. Next, we show that in the absence of known good algorithms, existing LLMs can generate trajectories with better decision making behaviors through diversity-encouraging sampling. We then finetune the LLMs to prefer higher performing trajectories (in a fashion similar to STaR~\citep{zelikman2022starbootstrappingreasoningreasoning}) and show that this leads to better decision making abilities at test-time. More importantly, these behaviors often generalize to unseen task groups without additional training. Finally, we propose a general curriculum learning algorithm that can dynamically choose which subset of tasks to train on next to improve data efficiency of such training methods. We next describe each component of \ours{}.


\subsection{Task Design}


The first component of \ours{} is to design a set of task groups that we can evaluate and train LLMs on. The task groups we want should have the following desired properties: \textbf{(1)} they are purely text based, \textbf{(2)} they require multi-turn interaction, where the agents have to both understand prior history in its context and choose actions that maximize the probability of success in the future, \textbf{(3)} they are partially observable, i.e., the observations do not capture the full state or hidden information, so the agents must simultaneously explore to reveal more information and exploit to solve the task efficiently, \textbf{(4)} they are diverse and require different strategies to succeed. 


With these requirements in mind, we design 10 task groups in our paper. On all of them, we employ an LLM as the agent that is given a task it needs to solve through sequential interaction with the task-specific environment, which provides both observations for intermediate timesteps given the agent's actions and also a task reward at the end of an episode. 
For tasks requiring general knowledge about the world to generate intermediate observations, we employ another LLM (typically GPT-4o-mini) as the environment. For tasks that have rule-based observations and rewards, we find that using hardcoded programs as the verifier/observation generator is more reliable than LLMs, similar to~\citet{deepseekai2025deepseekr1incentivizingreasoningcapability}. In order to prevent reward hacking, we also use either another LLM or a hardcoded program as a judge to filter out unsuccessful trajectories that got incorrectly labeled as successful by the task environment (see \cref{section:environment_hacking} for more on environment hacking). We also find that for task groups requiring complex reasoning, letting the agent think using chain-of-thought (COT) prompting~\citep{wei2022chain, kojima2022large} before generating a final answer improves its performance significantly. We provide a brief description of our task groups here, please refer to \cref{tab:environment_summary} for their summary and \cref{section:appendix_environment_design} for more details.

Following prior work~\citep{abdulhai2023lmrl}, we include classic guessing games like \textit{twenty questions} and \textit{guess my city} in our list of task groups. They require guessing a secret topic as quickly as possible by asking a sequence of questions and observing the answers. We also employ \textit{Wordle} and \textit{Mastermind}, where the agent needs to guess a secret 5-letter word and 4-digit code respectively. The environments for these task groups provide feedback in terms of similarity between the guess and the target word/code, and the agent needs to refine their guesses in future turns to maximize information gathering. We design \textit{customer service} and \textit{murder mystery} as dynamic text-based task groups: an LLM plays the role of the task environment, which is provided with the criterion for task success and generates dynamic intermediate observations based on this criterion. 

A desirable capability in LLMs is to code and refine based on interpreter feedback. To simulate this process with a toy case, we design \textit{Cellular Automata}, where the agent needs to make inferences about the transition rule in 1D elementary cellular automata~\citep{wolfram1983statistical, cook2004universality} by observing inputs and outputs. The agent receives the outputs generated from their predicted transition rule and they have to refine their predictions based on it. Next, we incorporate \textit{Minesweeper} and \textit{Battleship} based on classical games, which require the agent to interact with 2D grids to find hidden items within a fixed number of turns and refine their guesses based on per-turn observations. 

Finally, we incorporate a modified version of the multi-armed bandit~\citep{slivkins2024introductionmultiarmedbandits} task group from prior works~\citep{krishnamurthy2024largelanguagemodelsexplore,nie2024evolveevaluatingoptimizingllms} with the following distinctions: \textbf{(1)} we let the agent employ chain-of-thought reasoning before choosing arms so that they can transfer good strategies learned from other tasks, \textbf{(2)} we let the agent interact with the task environment in a multiturn way, \textbf{(3)} instead of reducing regret, we work on the bandit best arm selection~\citep{audibert:hal-00654404,wang2024bestarmidentificationfixed} problem, where we let the agent choose arms and observe rewards for a fixed number of turns and then measure its accuracy in deciding the arm with the highest reward. This is done to reduce computational cost over generating COTs for a large number of turns, since the difference in regret between different models is not meaningful when the number of turns is not large enough.

\subsection{Dataset construction}

In order to learn from these task groups, we must first generate data from them. It is crucial that the data we generate are diverse which would allow the model to learn different strategies without the risk of overfitting. We accomplish this by generating a large number of trajectories at a high temperature with Min-p sampling~\citep{nguyen2024turning}. Min-p sampling works by using an adaptive threshold $p_\text{scaled} \propto p_\text{max}$, where $p_\text{max}$ is the highest probability predicted by the model on the next token, to truncate the vocabulary to tokens that have a probability larger than $p_\text{scaled}$ and sample from them --- this enables us to generate diverse yet coherent trajectories at a higher temperature.


For each task in a set of chosen tasks (e.g., uniformly sampled), we generate $n_\text{sample}$ trajectories and then construct a preference pair $(h_{w}, h_{l})$ where $h_{w}$ is the highest scoring trajectory (trajectory that succeeds and does so at the fewest number of turns) and $h_{l}$ is randomly sampled from the lower scoring (failed  or takes substantially more turns to succeed) trajectories. We choose $h_l$ randomly instead of choosing the worst one to increase the diversity of our dataset. We treat $h_w$ and $h_l$ as proxies for desirable and undesirable behaviors. A dataset $\mathcal{D} = \left\{\left(h^{w}, h^{l}\right)^{(i)}\right\}_{i=1}^N$ is a collection of such trajectory pairs.

\subsection{Optimization}
\label{sec:opt}

\paragraph{Supervised fine-tuning.} If we take the winning episodes as the expert behavior, then we can discard the losing episode and maximize the likelihood of winning episodes:

\begin{align}
    \mathcal{L}_\text{SFT}(\mathcal{D}) = \mathbb{E}_{\mathcal{D}} \left[ \frac{1}{\sum_{t=0}^{|h_w|}|a_t^w|}\sum_{t=0}^{|h_w|} \log \pi_\theta \left(a^w_t \mid h^w_{:t}\right) \right].
\end{align}
where $|a|$ is the number of tokens for the agent response (discarding the environment generation). This is akin to rejection sampling fine-tuning~\citep{gulcehre2023reinforcedselftrainingrestlanguage,dong2023raft,mukobi2023superhfsupervisediterativelearning} seen in prior work.

\paragraph{Direct preference optimization.} A popular approach for finetuning LLMs is DPO~\citep{rafailov2024direct} where one directly optimizes the Bradley-Terry model~\citep{bradley1952rank} for preferences. In our setting, each trajectory consists of multiple rounds of interactions so the original DPO objective does not apply. We instead use a multi-turn version of DPO introduced in ~\citet{rafailov2024rqlanguagemodel}:
\begin{multline}
    \mathcal{L}_\text{DPO}(\gD) = \E_{\gD}\Bigg[\log \sigma\Bigg( 
    \sum_{t=0}^{|h^w|}\beta \log\frac{\pi_\theta(a_t^w \mid h_{:t}^w)}{\pi_\text{ref}(a_t^w \mid h_{:t}^w)} \\
    - \sum_{t=0}^{|h^l|}\beta \log\frac{\pi_\theta(a_t^l \mid h_{:t}^l)}{\pi_\text{ref}(a_t^l \mid h_{:t}^l)}
    \Bigg)\Bigg]
\end{multline}

where $a_t^w$ is the action tokens generated by the model at turn $t$ in the preferred trajectory $h^w$.
$\pi_\text{ref}$ is the reference policy, for which we use the initial model.
The main difference with standard DPO here is that we only calculate the loss on the action tokens --- the log probability ratios of the environment generated tokens are not included in the loss.

We note that we use DPO because it is less compute intensive. DPO allows us to decouple the data collection and policy improvement steps and offload them on different machines. However, in principle, one could also employ online RL with more resources. Following prior work that shows the efficacy of online RL compared to offline algorithms~\citep{xu2024dposuperiorppollm,tajwar2024preferencefinetuningllmsleverage}, we expect doing \ours{} with online RL would lead to even stronger results.

\paragraph{Combining objectives.} 

Finally, prior works have noted DPO having the unintended effect of reducing the probability of preferred trajectories as well, known as unintentional unalignment~\citep{razin2024unintentionalunalignmentlikelihooddisplacement}, which can affect model performance. The RPO objective~\citep{pang2024iterativereasoningpreferenceoptimization}, by combining SFT and DPO loss, has shown promising results in mitigating this issue. Formally, the RPO loss is:

\begin{equation} \label{eq:rpo_formula}
    \mathcal{L}_{\text{RPO}}(\mathcal{D}) = \mathcal{L}_\text{DPO}(\gD) + \alpha \mathcal{L}_\text{SFT}(\gD)
\end{equation}

where $\alpha$ is a hyper-parameter. Following ~\citet{pang2024iterativereasoningpreferenceoptimization}, we set $\alpha$ to be 1.0 for the rest of this paper.


\section{Visual-Language Model based Curriculum} \label{sec:vlm}

\begin{figure*}[t]
  \centering
    \includegraphics[width=0.83\textwidth]{figs/diagram-0915.png}
  \caption{Diagram of the Proposed VLM Algorithms, simulating human decision process on reward scale engineering.
  }
  \label{fig:alog}
  \vspace{-1.5em}
\end{figure*}


While the new formulation makes the quality-critical problem feasible, learning is still hyperparameter sensitive.
To ensure successful trajectories exist and hence can be learned subsequently, we propose a novel Vision-Language Models (VLM) based curriculum learning system, which automatically monitors training metrics and adjusts relative weights of reward terms during the learning process, which resembles the parameter tuning process of human experts.








\subsection{VLM-based Curriculum}

Our learning framework calls the VLM-based curriculum every K steps after the initial M training steps, where K and M are hyper parameters.
The curriculum module auto-adjusts reward weights
for the next cycle with following steps:

\textbf{Step1: Inspection.} In this step, our goal is to collect the initial set of information, which includes success rates, landing pressure profiles, and navigation pressure stats. These stats can be collected by expanding the rollout of the current policy $\pi$ for $N$ times. We maintain the history of the previous information for reference. Once the information is collected, the system checks the pre-defined predicates (e.g., force variance decreased without a significant reduction in navigation success rates) to see if it wants to call the VLM-based hyperparameter tuning.


\textbf{Step2: Update} 
In this step,\added{ there are two large model agents involved: a LLM agent and a VLM agent. } The LLM takes in provided metric \added{from Step 1} and updates reward weights. Depending on the training progress, the LLM could request for different extra information before updating. If the completion rate is low, vision feedback of ending scene summarized by a separate VLM will be provided to describe failure reasons (e.g., no contact, or close to endpoint without finish wiping). If the force metrics require improvements, detailed force percentiles will be provided. This step is desired with multiple purposes: 1) Only providing necessary details into prompts to avoid LLM's catastrophic forgetting on important information. 2) Navigation failures can arise from various scenarios. Leveraging VLM’s semantic capabilities allows us to understand the causes of failures, reducing the need for labor-intensive monitoring and iterative metric development. 3) This hierarchical approach enhances system's extensibility. \added{4) Separating LLM and VLM optimizes reasoning and visual data interpretation respectively.}

\added{The final metrics and extra information will be feed to the LLM. The output from LLM consists of two parts: 1) A 1-2 sentences step-by-step analysis on logs and focus-learning areas; 2) python code for updated reward parameters.} 





\added{Detailed prompts can be found at our website noted in the abstract.} The high-level description is summarized in Algorithm~\ref{algo:vlm-curriculum} and Figure~\ref{fig:alog}.

\makeatletter
\algnewcommand{\LineComment}[1]{\Statex \hskip\ALG@thistlm \(\triangleright\) #1}
\algnewcommand{\IndentLineComment}[1]{\Statex \hskip\ALG@tlm \(\triangleright\) #1}
\makeatother


\begin{algorithm}
\caption{VLM-based Curriculum Learning}
\label{algo:vlm-curriculum}
\begin{algorithmic}[1]
\State \textbf{Data:} \added{pre-trained LLM $L$ and }VLM $V$
\State \textbf{Data:} a \added{RL} policy $\pi$
\State \textbf{Data:} a \added{reward weights} parameter vector $\mathbf{w}$
\State $d \gets$ dict(), $i \gets 0$
\While{not converged}
    \State $\pi \gets$ learn($\pi$, $\mathbf{w}$, K) \Comment{Learn a policy for $K$ steps}
    \LineComment{Step 1. Inspection}
    \State $d \gets$ eval($d$, $\pi$, $i$) \Comment{Eval $\pi$ and update $i$th iter data}

    \If{not $\text{maintain}()$}
        \IndentLineComment{Step 2. Update}
        \State $d \gets$ request\_extra\_info\_if\_needed($V$, $d$, $i$)
        \State $\mathbf{w} \gets$ update\_reward\_params(\replaced{$L$}{$V$}, $d$)
    \EndIf
    \State $i += 1$
\EndWhile
\end{algorithmic}
\end{algorithm}
\vspace{-1.5em}





























\subsection{Implementation details}

\replaced{We used gpt-4~\cite{achiam2023gpt} as LLM and gpt-4-vision-preview~\cite{openai2023gptv} as VLM.}{We implemented GPT-4~\cite{achiam2023gpt} as the VLM.} To ensure thorough exploration of initial parameters, we initiate our module at 300k steps. We evaluate $N=50$ episodes every $K=100k$ steps and invoke the LLM curriculum module unless evaluation metrics meet the maintenance criteria: an improvement in force profiles—defined by a mean force deviation from the target of less than 5N with reduced variance—without significantly compromising the navigation completion rate (a permissible change of less than 15\%). Initially, $\text{W}_T = 1000$ and $\text{W}_q^{\text{max}} = 29$, which is far from the upper limit of the feasible range $0 < \text{W}_T \ll 99\text{W}_q^{\text{max}}$ outlined in Section \ref{sec:bounded-reward-design}. \added{Optionally, researchers can clip the reward weights for each goal do not exceed twice the weight of any other goal for safeguard.} Throughout the fine-tuning process, the ratio consistently remains within this range.

