In this paper, we explored the design space of VR multiscale navigation techniques in dense, homogeneous objects while using the concept of progressive refinement. We proposed PRIMO, an approach that allows users to traverse objects, such as those created through advanced manufacturing, to inspect small regions that have been flagged as potentially having defects. Through a user study, we varied two independent variables of the interaction design to understand the trade-offs and obtain guidelines that could be generalized for guiding practitioners interested in applying multiscale navigation for the inspection of objects that are dense and homogeneous.

Our results showed that navigation time can be minimized by allowing the user to select any arbitrary region within the object and by displaying only the currently focused subvolume. We also found evidence that unstructured navigation can lead to a better overall object understanding than structured navigation. We found qualitative results suggesting that displaying everything may also help with location awareness and overall object understanding, though we were not able to corroborate those with objective data. Based on these results, and giving more weight to spatial awareness than speed, we suggest that an optimal hybrid technique for this domain would use unstructured navigation and would display only the selected subvolume by default, but with the ability to toggle the display of everything.

For future work, we plan to delve deeper into the wayfinding cues that can support users in better understanding the navigations they took at each step. Some candidates include using spatial breadcrumbs to display representations of the previous models the user has been to, spatial trees to organize the division of the object visually, and ghost representations of the hidden objects that could be accessed when needed.