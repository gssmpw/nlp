\paragraph{\textbf{Performance}}
Our first hypothesis stated that \textsc{structured} would be faster than \textsc{unstructured} (\textbf{H1}). Evidence suggests that this hypothesis is false. On the contrary, we found that during navigation, \textsc{unstructured} was 10\% faster than \textsc{structured} on average. Based on the qualitative data, the reason was clear: while in \textsc{structured} participants do have the advantage of already having the object pre-divided into subvolumes and only having to select which one they want, they need to reposition their ray at every level of scale to select the next subvolume. With \textsc{unstructured}, on the other hand, the selection box would show at the intersection between the participant's raycast and the object. This means that if the participant focused on the region of interest at the top level, they would only have to press the confirmation button at each subsequent level of scale. Therefore, even if the original placement of the selection box took longer, they could more quickly navigate down to the targeted location. Obviously, these results might be different if the region of interest was not shown to the user from the top level, and they had to search for it instead.

We also hypothesized that the \textsc{selection} display mode would be faster than \textsc{everything} (\textbf{H2}). Our results support this hypothesis. During navigation, \textsc{selection} display mode was 12\% faster than \textsc{everything} on average. Based on qualitative data, we can attribute this to a couple of reasons. First, \textsc{selection} only showed the focused object, and thus, participants could clearly see and focus on selecting the next subvolume. In \textsc{everything}, however, the other regions outside of the focused/selected volume still caused occlusion. Not only could participants not see the sides of the focused volume, but they would also have to orient their hands awkwardly or physically walk around to get the ray in the right place. This was even stronger when using \textsc{structured} navigation, as the raycast selected the first subdivision it touched, while with \textsc{unstructured} participants could still move their hands in depth to move the selection box in depth. Second, from a pure visualization point of view, displaying the entire object while only allowing for further navigation inside of the focus region made participants confused in cases where they selected the wrong level of scale, but after navigating they could still see the marker for the correct region of interest. In some cases it took them a couple of seconds to realize the mistake before taking action to move back to the higher level and select the other subvolume.

\paragraph{\textbf{Simulator Sickness}}
We hypothesized that \textsc{selection} would lead to less simulator sickness than \textsc{everything} (\textbf{H3}). Our objective results do not support this hypothesis. There was a trend for \textsc{everything} to score higher than \textsc{selection}, but this did not lead to a significant difference due to a high variance in the measure. Only two out of twenty-four participants complained of mild simulator sickness. This is a positive result for the techniques, as each participant completed a large number of trials and still experienced only mild levels of sickness. In the qualitative results, none of the participants commented on a clear difference between the techniques in regards to sickness, dizziness, or disorientation. A few participants did note, however, that the \textsc{everything} display mode created a zooming-in effect with a lot happening visually, where the ``entire world'' was getting larger around them as they shrunk, as opposed to the \textsc{selection} display mode where the object felt like it was getting larger. 

\paragraph{\textbf{Location Awareness}}
Our fourth hypothesis was that \textsc{structured} would lead to better location awareness than \textsc{unstructured} (\textbf{H4}). However, we found no differences between the conditions on the accuracy of answering question 1. From interviews, we got mixed results. Participants who believed \textsc{unstructure} was better argued that they only needed to focus on their target region of interest and then get there faster, having to remember only that one point. Participants who believed \textsc{structured} was better argued that (1) remembering the path they took to get there was easier (because they just had to think about the discrete selection steps, and (2) since the object was already pre-divided, they could think of the octants when trying to pinpoint the focused one. Our lack of significant results may have been due to participants using different strategies to maintain awareness of their location during navigation.

In the fifth hypothesis, we argued that \textsc{everything} would lead to better location awareness than \textsc{selection} (\textbf{H5}). Again, the data for location awareness accuracy do not support this hypothesis. However, our interview data revealed that users perceived it to be true. More than 87\% of participants mentioned one or both of the \textsc{everything} techniques as providing better location awareness. In their comments, they extensively mentioned how using the \textsc{selection} display mode led to them trying to memorize where they were going or which steps they took as they navigated. Some memorization was still needed in the \textsc{everything} mode, but participants could also use the peripheral view of their location in the context of the whole object as they navigated. This suggests that everything may have some benefits for location awareness, but we were not able to measure it objectively because the task was not complicated enough for the memorization used in the \textsc{selection} display mode to be detrimental. We suggest that a future study with a larger number of levels of scale or with greater scale differences between those levels could revisit this hypothesis for confirmation.

\paragraph{\textbf{Overall Object Understanding}}
We hypothesized that \textsc{structured} would provide a better understanding of all defect locations than \textsc{unstructured} (\textbf{H6}). Our objective results do not support this hypothesis. We found that \textsc{unstructured} was 17\% more accurate than \textsc{structured} on question 2, which asked for the locations of all four defects in an object. Although qualitative results showed that participants believed that structuring the navigation process led to a more concise and organized amount of information to remember and combine afterward, the objective measure indicates that the act of placing the selection box in \textsc{unstructured} actually gave people a better cue to remember the defect locations.

Finally, the last hypothesis was that \textsc{everything} would result in a better understanding of all defect locations than \textsc{selection} (\textbf{H7}). Objective measures did not find a difference between the display modes. From the interviews, however, participants mentioned that \textsc{everything} was advantageous because, in some scenarios, they could partially see one defect from a distance while navigating to a different defect region, and that would refresh the location relationship between the defects in their minds. Similarly to H5, we suggest that this should be revisited with a more complicated task, especially one that allows users to navigate from one defect to the next one instead of going back all the way to the top to start the next trial. Alternatively, the task could have more than four defects in each object, leading to higher chances of seeing other defects while making it harder to remember all the defect locations.

\paragraph{Implications} Our findings bring some implications to the domain: in VR, we should give people unstructured ways to navigate when they know where they are going; on the other hand, a structured navigation could still be used for systematic search in an object where the location of defects is unknown, although another study should investigate that; we should prefer to display the entire object to enhance spatial awareness, but we should hide peripheral portions of the object when selecting the next piece to navigate.