% What is the problem we are approaching?
Additive manufacturing enables the fabrication of objects with complex internal geometries \cite{chheang_virtual_2024}. While such objects can have dimensions in meters, such parts are susceptible to defects that can happen at a millimeter scale (too much or too little material across structures \cite{klacansky_virtual_2022}) and can impact the structural integrity of the part. While certain defects can be detected visually, and certain (often destructive) stress tests can be applied to the physical part, smaller defects can only be visualized through CT scans, with a manual check of each layer generated by the scan---which is impractical at scale. The scans can also be visualized in 3D, but scale differences can become a challenge during inspection processes, as it is hard to navigate through an object that is one or more orders of magnitude larger than the defects that must be visually verified. Furthermore, the dense structure of such objects implies that these defects will often be occluded inside of the object. A consequence of these challenges is that currently there are no standard procedures for validating if such products were created according to specifications. 

% What is the VR problem we are approaching?
We investigate the use of virtual reality (VR) technologies to inspect digital twins of the fabricated parts (geometric meshes generated from either CT scans or other replication processes), enabling operators to find and assess such defects in a timely and accurate fashion while raising their understanding of how multiple defects relate and propagate across the object. Such inspections can be characterized as \textit{multiscale navigation} tasks in dense, homogeneous objects. Existing techniques have proposed multiple ways to perform multiscale navigation in VR, either continuously or discretely. These techniques, however, (1) focus mostly on structures with well-defined hierarchical meaning (such as the human body) or (2) happen in open areas (such as a map in a game). They do not account for the nuances of inspecting a dense, homogeneous object, which means defect regions may be hard to see, be hard to reach, and have reduced landmarks to support location awareness.

% How those our impact generalize?
While in this paper we focus on the specific problem of fabricated parts, such inspections can also be relevant in other domains, such as geological analysis of soil, examining the internal defects in composite materials, evaluating the internal defects in sculptures or architectural structures, inspecting the inner workings of machinery for defects or wear, and inspecting biological structures like neural networks or cells. While all these domains still face challenges on obtaining such digital twins with high accuracy, the concepts of how to inspect them are similar to the ones discussed in this paper.

% Why are we approaching this problem?
We approached this problem by applying the concept of progressive refinement to multiscale navigation. Progressive refinement \cite{kopper_rapid_2011} was proposed for selection purposes in cluttered environments, where an initial selection would include multiple objects, which would then be progressively subdivided into groups based on their proximity until a single object could be selected. While traditional navigation techniques define how to move a user through an environment, our progressive refinement navigation is object centered, where users maintain their position and the object is scaled and repositioned based on where they want to focus. While multiscale transitions can induce simulator sickness and disorientation \cite{krekhov_gullivr_2018, piumsomboon_superman_2018, cmentowski_outstanding_2019, abtahi_im_2019}, our approach allows users to select the volume on which they want to focus in a quick, systematic, and effective way.

% What did we design, and what study did we conduct?
We designed an approach called Progressive Refinement for the Inspection of Multiscale Objects (PRIMO). We identified two key characteristics of PRIMO designs: navigation style (\textsc{structured} or \textsc{unstructured}) and display mode (\textsc{selection} vs \textsc{everything}). We conducted a user study to investigate their effects on efficiency, location awareness during navigation, and overall understanding of the defects in the manufactured object. Results indicate that navigation time can be reduced when we let users select any arbitrary region while displaying only the focused subvolume. We also found evidence that displaying the entire object can aid in raising location awareness and overall object understanding. Contributions of this work include (1) the design of a technique that applies the concept of progressive refinement to the domain of multiscale navigation for the inspection of dense, homogeneous objects; and (2) a qualitative and quantitative measurement of the trade-offs of certain design choices on that design.