\section{Related Work}
\subsection{Navigating Across Levels of Scale}
Kopper et al. \cite{kopper_design_2006} introduced techniques for navigating multiscale virtual environments (MSVEs), emphasizing the need to inform users about different levels of scale and provide efficient means of transitioning between them in a \textit{discrete} fashion \cite{al_zayer_virtual_2020}. They propose navigation methods like the magnifying glass and target-based navigation. Moreover, Bacim et al. \cite{bacim_wayfinding_2009} extended this work by focusing on wayfinding aids in MSVEs, introducing techniques like a multiscale version of the world-in-miniature \cite{stoakley_virtual_1995, laviola_hands-free_2001, wingrave_overcoming_2006} and a hierarchically-structured map. Their study reveals the effectiveness of spatial and hierarchical information aids in enhancing user performance and navigation accuracy, further underlining the importance of clear wayfinding mechanisms in multiscale environments. Kouril et al. \cite{kouril_hyperlabels_2021} presented HyperLabels, a technique for navigating hierarchical molecular 3D models. HyperLabels leverages annotations and breadcrumbs to guide users through complex hierarchical structures, enhancing user comprehension and interaction efficiency.

Multiscale navigation has also been used to move through large virtual worlds. Utilizing techniques such as the ones proposed by Pierce and Pausch \cite{pierce_navigation_2004}, users can navigate through large environments by using visible landmarks as points of reference for travel. Krekhov et al. \cite{krekhov_gullivr_2018} proposed GulliVR, a technique that allows users to transition between being a giant or a regular-sized human being while walking through a large terrain. They discussed issues such as changing inter-pupillary distance (IPD) to effectively convey the perception of being a giant. They described the idea of pulling, where the system would help guide users to a point of interest. In a similar setup, Lee et al. \cite{lee_designing_2023} investigated different transitioning techniques when changing level of scale, and showed that having active control improves users' spatial awareness and performance. They also found that zooming straight out, followed by an orbital motion to reorient the user and then zooming in, presented the best spatial orientation, usability, and preference.

Another multiscale navigation approach uses continuous scaling techniques \cite{al_zayer_virtual_2020}. Such techniques gradually manipulate scale and/or speed \cite{mccrae_multiscale_2009, argelaguet_adaptive_2014} to achieve the desired movements, often automatically, with mechanisms such as viewpoint quality \cite{freitag_automatic_2016, mirhosseini_automatic_2017}, distance to surroundings \cite{ware_context_1997, mccrae_multiscale_2009, trindade_improving_2011, carvalho_dynamic_2011, cho_evaluating_2014, cho_multi-scale_2018}, and optical flow \cite{argelaguet_adaptive_2014, argelaguet_giant_2016}.

While these works define the foundation for navigating multiscale environments, the literature is lacking regarding the navigation of dense, homogeneous objects. Existing VR techniques have focused either on how to provide multiscale navigation in structures that have separable components---such as biological cells---where each level of scale has a clear meaning that supports navigation and spatial awareness; or has focused on open environments—such as zooming in and out of a map—where users don’t need to access highly occluded elements inside of a volume. When we consider objects with repeated dense structures, questions of understanding navigation and awareness become essential. Furthermore, most work in the literature has been about exploration or targeted navigation, with little discussion about how to perform systematic multiscale navigations for the purpose of inspection of marked regions of interest.

\subsection{Perception of Multiscale Navigation}
Besides mechanisms for navigation, we must also consider the effects of multiscale navigation on user perception. Piumsomboon et al. \cite{piumsomboon_superman_2018} investigated the effects of scaling a user up versus just moving in the air at normal size. They demonstrated that IPD plays a significant role in altering users' perceptions, particularly suggesting a potential coupling between IPD size and height. Similarly, Cmentowski et al. \cite{cmentowski_outstanding_2019} explored transitions between small and large levels of scale. They show the importance of smooth and fast transformations between perspectives to prevent simulator sickness and maintain spatial orientation. They further uncoupled the user from their avatar during travel mode to enhance spatial orientation and reduce reorientation efforts. Abtahi et al.  \cite{abtahi_im_2019} investigated the impact of perceived walking speed on user experience in large environments. They proposed three techniques: Ground-Level  Scaling, Eye-Level Scaling, and Seven-League Boots. Ground-Level Scaling was found to enhance user embodiment and stride length, while Seven-League Boots, although amplifying user movements, diminished positional accuracy at high gains. They further discussed avoiding scale changes on-the-fly and the use of animations. These findings underscore the importance of considering perceptual aspects in the design of multiscale navigation interfaces to enhance user experience and mitigate potential comfort issues, and the need for smooth transitions in multiscale navigation interfaces to ensure a seamless and immersive user experience.

% ============================================================================