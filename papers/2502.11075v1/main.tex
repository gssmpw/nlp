% This must be in the first 5 lines to tell arXiv to use pdfLaTeX, which is strongly recommended.
\pdfoutput=1
% In particular, the hyperref package requires pdfLaTeX in order to break URLs across lines.

\documentclass[11pt]{article}
\usepackage{enumitem}
\usepackage{booktabs}
\usepackage{float}
\usepackage{color}
% Change "review" to "final" to generate the final (sometimes called camera-ready) version.
% Change to "preprint" to generate a non-anonymous version with page numbers.
%\usepackage[review]{acl}
\usepackage{acl}
\usepackage{multirow}
% Standard package includes
\usepackage{times}
\usepackage{latexsym}
\usepackage{subfig}

\usepackage{amsmath}  
\usepackage{amsfonts}  
\usepackage{amssymb}  
\usepackage{hyperref}


% For proper rendering and hyphenation of words containing Latin characters (including in bib files)
\usepackage[T1]{fontenc}
% For Vietnamese characters
% \usepackage[T5]{fontenc}
% See https://www.latex-project.org/help/documentation/encguide.pdf for other character sets

% This assumes your files are encoded as UTF8
\usepackage[utf8]{inputenc}
\usepackage{xcolor,colortbl}
% This is not strictly necessary, and may be commented out,
% but it will improve the layout of the manuscript,
% and will typically save some space.
\usepackage{microtype}

% This is also not strictly necessary, and may be commented out.
% However, it will improve the aesthetics of text in
% the typewriter font.
\usepackage{inconsolata}

%Including images in your LaTeX document requires adding
%additional package(s)
\usepackage{graphicx}

% for appendix
 
\usepackage{tcolorbox}


%\usepackage{geometry}

% If the title and author information does not fit in the area allocated, uncomment the following
%
%\setlength\titlebox{<dim>}
%
% and set <dim> to something 5cm or larger.

%\title{When Large Language Models Fail: A Study of Retrieval and Reasoning on Numeric Data}
%\title{NumericBench: Exposing the Achilles' Heel of Large Language Models in Numerical Reasoning}
%\title{Exposing Numeracy Gaps in LLMs: A Benchmark for Evaluating Retrieval and Reasoning on Numerical Data}
\title{Exposing Numeracy Gaps: A Benchmark to Evaluate Fundamental Numerical Abilities in Large Language Models}

% Author information can be set in various styles:
% For several authors from the same institution:
% \author{Author 1 \and ... \and Author n \\
%         Address line \\ ... \\ Address line}
% if the names do not fit well on one line use
%         Author 1 \\ {\bf Author 2} \\ ... \\ {\bf Author n} \\
% For authors from different institutions:
% \author{Author 1 \\ Address line \\  ... \\ Address line
%         \And  ... \And
%         Author n \\ Address line \\ ... \\ Address line}
% To start a separate ``row'' of authors use \AND, as in
% \author{Author 1 \\ Address line \\  ... \\ Address line
%         \AND
%         Author 2 \\ Address line \\ ... \\ Address line \And
%         Author 3 \\ Address line \\ ... \\ Address line}

\author{Haoyang LI$^1$, Xuejia CHEN$^1$, Zhanchao XU$^1$, Darian LI$^1$, Nicole HU$^3$,\\ \textbf{Fei TENG$^2$, Yiming LI$^2$, Luyu QIU$^2$, Chen Jason ZHANG$^1$, Qing LI$^1$, Lei CHEN$^2$} \\
  $^1$Hong Kong Polytechnic University, \\$^2$Hong Kong University of Science and Technology, \\$^3$The Chinese University of Hong Kong\\
  \texttt{haoyang-comp.li@polyu.edu.hk}, %\texttt{gresham15437@gmail.com}, \\ \texttt{\{fteng,yliix,lqiuag\}@connect.ust.hk},\texttt{jason-c.zhang@polyu.edu.hk}\\
  \texttt{csqli@comp.polyu.edu.hk}, \texttt{leichen@cse.ust.hk}
}
 
%\author{
%  \textbf{First Author\textsuperscript{1}},
%  \textbf{Second Author\textsuperscript{1,2}},
%  \textbf{Third T. Author\textsuperscript{1}},
%  \textbf{Fourth Author\textsuperscript{1}},
%\\
%  \textbf{Fifth Author\textsuperscript{1,2}},
%  \textbf{Sixth Author\textsuperscript{1}},
%  \textbf{Seventh Author\textsuperscript{1}},
%  \textbf{Eighth Author \textsuperscript{1,2,3,4}},
%\\
%  \textbf{Ninth Author\textsuperscript{1}},
%  \textbf{Tenth Author\textsuperscript{1}},
%  \textbf{Eleventh E. Author\textsuperscript{1,2,3,4,5}},
%  \textbf{Twelfth Author\textsuperscript{1}},
%\\
%  \textbf{Thirteenth Author\textsuperscript{3}},
%  \textbf{Fourteenth F. Author\textsuperscript{2,4}},
%  \textbf{Fifteenth Author\textsuperscript{1}},
%  \textbf{Sixteenth Author\textsuperscript{1}},
%\\
%  \textbf{Seventeenth S. Author\textsuperscript{4,5}},
%  \textbf{Eighteenth Author\textsuperscript{3,4}},
%  \textbf{Nineteenth N. Author\textsuperscript{2,5}},
%  \textbf{Twentieth Author\textsuperscript{1}}
%\\
%\\
%  \textsuperscript{1}Affiliation 1,
%  \textsuperscript{2}Affiliation 2,
%  \textsuperscript{3}Affiliation 3,
%  \textsuperscript{4}Affiliation 4,
%  \textsuperscript{5}Affiliation 5
%\\
%  \small{
%    \textbf{Correspondence:} \href{mailto:email@domain}{email@domain}
%  }
%}

\begin{document}
\maketitle
\begin{abstract}
Large Language Models (LLMs) have demonstrated impressive capabilities in natural language processing tasks, such as text generation and semantic understanding. However, their performance on numerical reasoning tasks, such as basic arithmetic, numerical retrieval, and magnitude comparison, remains surprisingly poor.  This gap arises from their reliance on surface-level statistical patterns rather than understanding numbers as continuous magnitudes. Existing benchmarks primarily focus on either linguistic competence or structured mathematical problem-solving, neglecting fundamental numerical reasoning required in real-world scenarios. To bridge this gap, we propose NumericBench, a comprehensive benchmark to evaluate six fundamental numerical capabilities: number recognition, arithmetic operations, contextual retrieval, comparison, summary, and logical reasoning. NumericBench includes datasets ranging from synthetic number lists to the crawled real-world data, addressing challenges like long contexts, noise, and multi-step reasoning. Extensive experiments on state-of-the-art LLMs, including GPT-4 and DeepSeek, reveal persistent weaknesses in numerical reasoning, highlighting the urgent need to improve numerically-aware language modeling.  
The benchmark is released in:
\url{https://github.com/TreeAI-Lab/NumericBench}.
%The benchmark is released in the  \href{https://zenodo.org/records/14875784?token=eyJhbGciOiJIUzUxMiJ9.eyJpZCI6IjZhYjZlYzYwLWZkMTgtNGU1Ni1iM2I2LWUwNWVlMGIwZmYwZCIsImRhdGEiOnt9LCJyYW5kb20iOiIwNzQxNmU0NWY5ZDkxMzQ4ODVmYjdlZDgyOGJmNjVhYSJ9.m-\_8Owb3TohJ76cGt2Mu4wrpsxMgC4E\_aJG7Q07KHOTlKaxB4kipMY3eBZPaQEIkOv\_iJEkRmlvZr23rLkTMBw}{anonymous hyperlink}.   
%https://github.com/TreeAI-Lab/NumericBench
\end{abstract}


% !TEX root = ../main.tex

Large Language Models (LLMs) have shown remarkable capabilities on numerous tasks in Natural Language Processing (NLP), 
ranging from language understanding to generation \cite{bubeck2023sparks, achiam2023gpt,team2023gemini, dubey2024llama}. The huge success of LLMs comes with important challenges to deploy them due to their massive size and computational costs. For instance,  Llama-3-405B \cite{dubey2024llama} requires 780GB of storage in half precision (FP16) and hence multiple high-end GPUs are needed just for inference. \textit{Model compression} has emerged as an important line of research to reduce the costs associated with deploying these foundation models. In particular, neural network pruning \cite{obd, hassibi1992second, benbaki2023fast}, where model weights are made to be sparse after training, has garnered significant attention. Different sparsity structures (Structured, Semi-Structured and Unstructured) obtained after neural network pruning result in different acceleration schemes. \textit{Structured pruning} removes entire structures such as channels, filters, or attention heads \cite{lebedev2016fast,wen2016learning,voita2019analyzing,el2022data} and readily results in acceleration as model weights dimensions are reduced. \textit{Semi-Structured pruning}, also known as, N:M sparsity \cite{zhou2021learning} requires that at most $N$ out of $M$ consecutive elements are non-zero elements. Modern NVIDIA GPUs provide support for 2:4 sparsity acceleration. \textit{Unstructured pruning} removes individual weights \cite{han2015learning, guo2016dynamic} from the model's weights and requires specialized hardware for acceleration. For instance, DeepSparse \cite{kurtic2022optimal, pmlr-v119-kurtz20a, DBLP:journals/corr/abs-2111-13445} provide CPU inference acceleration for unstructured sparsity.\\
Specializing to LLMs, one-shot pruning~\cite{meng2024alps, frantar2023sparsegpt, sun2023simple, zhang2023dynamic}, where one does a single forward pass on a small amount of calibration data, and prunes the model without expensive fine-tuning/retraining, is of particular interest. This setup requires less hardware requirements. For instance, \citet{meng2024alps} show how to prune an OPT-30B \cite{opt} using a single consumer-level V100 GPU with 32GB of CUDA memory, whereas full fine-tuning of such model using Adam \cite{kingma2014adam} at half-precision requires more than 220GB of CUDA memory.

Although one-shot pruning has desirable computational properties, it can degrade models' predictive and generative performance. To this end, recent work has studied extensions of model pruning to achieve smaller utility drop of model performance from compression. 
% Multiple one-shot methods have been developed in quantization \cite{frantar2022gptq, frantar2023sparsegpt, lin2024awq, behdin2023quantease, dettmers2023spqr} and neural network pruning \cite{frantar2023sparsegpt, meng2024alps, zhang2024oats}, which is closer to this paper's line of research. These one-shot methods do not require retraining--which is extremely expensive for models of the size of Llama-3-405B-- and work as resource-saving techniques that retain the model's performance. 

An interesting compression mechanism in the field of \textit{model compression} is the Sparse plus Low-Rank Matrix-Decomposition problem which aims to approximate model's weights by a sparse component plus a low-rank component~\cite{hintermuller2015robust, candes2011robust, lin2011linearized, 5394889, zhou2011godec, JMLR:v24:21-1130, NIPS2014_443cb001, yu2017compressing, li2023losparse}. Specializing to LLMs,~\citet{zhang2024oats} propose OATS 
%that addresses this type of %compression and 
that outperforms pruning methods for the same compression ratio (number of non-zero elements) on a wide range of LLM evaluation benchmarks (e.g. perplexity in Language generation). 

OATS \cite{zhang2024oats} is however a matrix decomposition algorithm inspired from a pruning algorithm Wanda \cite{sun2023simple}. Wanda has been designed as a relaxation/approximation of another state-of-the-art pruning algorithm SparseGPT \cite{frantar2023sparsegpt}. While Wanda has been found to be extremely useful and efficient in practice, recent work \cite{meng2024alps} show results where Wanda fails for high-sparsity regimes. In this paper, we provide a unified optimization framework to decompose pre-trained model weights into sparse plus low-rank components based on a layer-wise loss function. Our framework is modular and can incorporate different pruning and matrix-decomposition algorithms (developed independently in different contexts).
%under the umbrella of the local %layer-wise reconstruction error; 
Similar to~\cite{meng2024alps} we observe that our optimization-based framework results in models with better model utility-compression tradeoffs. The difference is particularly pronounced for higher compression regimes. 
%especially for higher compression %budgets, where SOTA methods 
% Our numerical results also show similar findings to \citet{meng2024alps} where high-sparsity significantly degrades the performance of approximation-based optimization methods like OATS.

Concurrently, in a different and complementary line of work,~\citet{mozaffari2024slope} have open-sourced highly-specialized CUDA kernels designed for N:M sparse \cite{zhou2021learning} plus low-rank matrix decompositions that result in significant acceleration and memory reduction for the pre-training of LLMs.
We note that our focus here is on improved algorithms for one-shot sparse plus low-rank matrix decompositions for foundation models with billions of parameters which is different from the work of \citet{mozaffari2024slope} that focuses on accelerating the pre-training of LLMs. The designed CUDA kernels \cite{mozaffari2024slope} can be exploited in our setting for faster acceleration and reduced memory footprint during inference.





% \textbf{Summary of approach and contributions:} We propose \ourmethod: an accurate and scalable framework for Sparse plus Low-Rank Matrix Decomposition for LLMs. Following the previous work on one-shot pruning and model compression, we pursue a layerwise approach. In particular, the reconstruction error resulting from compression in the output of each layer is minimized, under the compression constraints (i.e., sparsity and low-rank constraints).

\textbf{Summary of approach.\,\,\,\,} Our framework is coined \ourframework: \underline{H}ardware-\underline{A}ware (Semi-\underline{S}tructured) \underline{S}parse plus \underline{L}ow-rank \underline{E}fficient \& approximation-\underline{free} matrix decomposition for foundation models.

Hardware-aware refers to the fact that we mostly focus on a N:M sparse \cite{zhou2021learning} plus low-rank decomposition, for which acceleration on GPUs is possible, although \ourframework supports any type of sparsity pattern (unstructured, semi-structured, structured) in the sparsity constraint. Approximation-free refers to the fact that we directly minimize the local layer-wise reconstruction error introduced in \cref{eq:matrix-decomposition}, whereas we show prior work minimizes an approximation of this objective.

%Our unified framework introduces a well-posed 
%%As a part of our proposed framework, we consider an 
%%optimization form
We formulate the compression/decomposition task as a clear optimization problem; we minimize a local layer-wise reconstruction objective where the weights are given by the sum of a sparse and low-rank component. 
%%%of dense model weights under the  
%This optimization problem is decoupled into a sparse minimization subproblem and a low-rank minimization subproblem. 
We propose an efficient Alternating-Minimization approach that scales to models with billions of parameters relying on 
two key components: one involving sparse minimization (weight sparsity) and the other involving a low-rank optimization. 
For each of these subproblems 
we discuss how approximations to the optimization task can retrieve prior algorithms.
%the introduced subproblems, 
%we consider approximations to the minimization objective and retrieve different algorithms from related works given different %approximations.

% We provide an efficient and scalable algorithm based on Alternating-Minimization that does not rely on any approximation at the objective minimization level. 
% While \ourframework supports any sparsity pattern (unstructured, semi-structured, structured) in the sparsity constraint, we mostly focus on N:M sparsity \cite{zhou2021learning}, to make the decomposition Hardware-aware, as \citet{mozaffari2024slope} show how to get acceleration on modern GPUs for N:M sparse plus low-rank decomposition.

We note that \ourframework~differs from prior one-shot (sparse) pruning methods~\cite{frantar2023sparsegpt, meng2024alps, benbaki2023fast} as we seek a sparse plus low-rank decompositon of weights.
%%%%%introducing the low-rank component. 
Additionally, it differs from prior one-shot sparse plus low-rank matrix decomposition methods~\cite{zhang2024oats}
%by considering an approximation-free minimization approach of the 
as we directly minimize the local layer-wise reconstruction objective introduced in \cref{eq:matrix-decomposition}.

Our main \textbf{contributions} can be summarized as follows.
\begin{compactitem}
    \item We introduce \ourframework a unified one-shot LLM compression framework that scales to models with billions of parameters where we directly minimize the local layer-wise reconstruction error subject to  a sparse plus low-rank matrix decomposition of the pre-trained dense weights. 
    %    formulates a sparse plus low-rank matrix decomposition as an optimization problem with a local layer-wise reconstruction objective. We discuss approximations of this objective and show that OATS a popular method is recovered in a particular approximation.

    
    \item \ourframework uses an Alternating-Minimization approach that iteratively minimizes a Sparse and a Low-Rank component. \ourframework uses a given pruning method as a plug-in for the subproblem pertaining to the sparse component. Additionally, it uses Gradient-Descent type methods for the subproblem pertaining to the Low-Rank component.
    
    % \item In the subproblem pertaining to the sparse component, a rewrite of the optimization formulation shows that one can use any pruning algorithm, that minimizes the layer-wise reconstruction error, as a plug-in to sparsify the weights. We choose to show results for the algorithm SparseGPT.
    
    % In this pruning subproblem, we also enhance the performance of \ourmethod by exploiting the invariance of the Hessian--of the layer-wise reconstruction error--in each subproblem of the Alternating Minimization procedure, for a given layer. In particular, we use a pre-processing step that computes and stores the Hessian inverse--of the objective--, which is then passed to the deployed pruning algorithm (e.g. SparseGPT). 
    % \item In the subproblem  pertaining to the Low-Rank component, we give a theoretical closed form solution to the subproblem.
    % which does not scale to problems with billions of parameters. 
    % We also present a more tractable first-order optimization method for a reparametrization of the the low-rank problem, which is scalable to models with billions of parameters.
    
    % as $\bfUVt$ and use first-order optimization methods to minimize the layer-wise reconstruction objective.

    \item We discuss how special cases of our framework relying on specific approximations of the objective retrieve popular methods such as OATS, Wanda and MP --- \cite{zhang2024oats, sun2023simple,han2015learning, sze2020efficient}. This provides valuable insights into the underlying connections across different methods. 

    \item \ourframework improves upon state-of-the-art methods for one-shot sparse plus low-rank matrix decomposition. 
    For the Llama3-8B model with a 2:4 sparsity component plus a 64-rank component decomposition, \ourframework reduces the test perplexity by $12\%$ for the WikiText-2 dataset and reduces the gap (compared to the dense model) of the average of eight popular zero-shot tasks by $15\%$ compared to existing methods.
\end{compactitem}




\section{Preliminary and Related Works}
In this section, we first introduce large language models and then present existing benchmarks.


\subsection{Large Language Models}
Large language models (LLMs), such as GPT-4~\citep{achiam2023gpt}, 
DeepSeek~\citep{liu2024deepseek}, 
PaLM~\citep{anil2023palm}, 
and LLaMA~\citep{touvron2023llama}, 
have revolutionized natural language processing (NLP) through their ability to generate coherent text~\citep{cho2019coherentcohesivelongformtext}, 
answer questions~\citep{chen2024analyze}, 
and adapt to diverse tasks~\citep{wang2025graph,jiang2024survey}. 
Their success stems from pretraining on vast text corpora using next-token prediction objectives, 
which enable generalization on tasks requiring semantic understanding, commonsense reasoning, and linguistic creativity.
However, this training paradigm encourages LLMs to prioritize surface-level statistical patterns 
(e.g., lexical co-occurrences, syntactic regularities) rather than numerically grounded reasoning~\citep{bachmann2024pitfalls}. 
Consequently, LLMs treat numbers as discrete tokens rather than continuous magnitudes, inherently limiting their ability to understand exact numerical semantics. 
This leads to errors in numeric retrieval, arithmetic operations, and magnitude comparisons~\citep{qiu2024dissecting}.







\subsection{Benchmarks on Large Language Models}

Existing benchmarks~\cite{li2024survey,chang2024survey,zhao2023survey} for evaluating LLMs primarily fall into two categories, 
i.e., \textit{semantic-oriented} and \textit{math-oriented} benchmarks.
Specifically,
\textit{semantic-oriented} 
benchmarks, 
such as GLUE~\citep{wang2018glue},  
SuperGLUE~\citep{wang2019superglue}, 
SimpleQA~\citep{wei2024measuring}, 
and LongBench~\citep{bai2023longbench},
focus on semantic understanding and linguistic competence, 
testing skills like textual entailment, 
commonsense reasoning, and domain-specific knowledge (e.g., science and law). 
While effective for assessing linguistic proficiency, these benchmarks largely overlook numerical reasoning.
On the other hand,
\textit{math-oriented} benchmarks~\cite{gao2025gllava,li2024forewarned,cobbe2021training}, 
such as MathQA~\citep{amini2019mathqa}, 
GSM8K~\citep{cobbe2021training}, 
and MathBench~\citep{liu2024mathbench},
target mathematical problem-solving (e.g., algebra, calculus) or extractive question-answering with numerical answers. 
However, these datasets emphasize well-formed mathematical problems  in controlled and clean settings.
Consequently,
\textit{math-oriented} benchmarks poorly evaluate numerical retrieval and reasoning in real-world conditions, where noise, and contextual complexity (e.g., multi-step financial workflows or long stock sequences) are common.

Considering that numeric retrieval and reasoning are critical for real-world applications, 
such as finance~\citep{islam2023financebench} and weather forecasting~\cite{zhang2024self}, 
we propose \textbf{\textit{NumericBench}} to systematically evaluate the fundamental numerical abilities of  LLMs on intensive tasks, such as precise value retrieval, dynamic comparisons, and arithmetic-logical reasoning.









\section{NumericBench Generation}
In this section, we present our created  NumericBench, which is specifically designed to evaluate fundamental numerical capabilities of LLMs. 
NumericBench consists of diverse datasets and tasks, 
enabling a systematic and comprehensive evaluation.
We discuss the datasets included in NumericBench, the key abilities it evaluates, and the methodology for benchmark generation.

\begin{table*}[t]
	\caption{NumericBench statistics. R: contextual retrieval, C: comparison, S: summary, L: logical reasoning. The token count is calculated based on tiktoken, which is the tokenizer used by Llama3~\cite{grattafiori2024llama3herdmodels}. The sentences used for token calculation include both the context and the question.}
	\centering
	\renewcommand{\arraystretch}{1.15} % 设置行间距为默认的 1.15 倍
	\setlength{\tabcolsep}{1.5pt} % 将列间距设置为 1pt
\resizebox{\textwidth}{!}{
	\begin{tabular}{c|c|c|c|c}
		\toprule
		\textbf{Data} & \textbf{Format} & \textbf{Questions} & \textbf{\# Instance} & \textbf{Avg Token} \\ \midrule
		
		\multirow{3}{*}{} 
		& \multirow{3}{*}{} 
		& \begin{tabular}[c]{@{}c@{}}R: What is the index of the first occurrence\\ of the number -3095 in the list?\end{tabular} 
		& 1000 & 3704.23 \\ \cline{3-5}
		
		\textbf{\begin{tabular}[c]{@{}c@{}}Number\\ List\end{tabular}}
		& $[69, -1, 6.1, \ldots, 5.7]$
		& \begin{tabular}[c]{@{}c@{}}C: Which index holds the smallest number\\
			 in the list between the indices 20 and 80?\end{tabular} 
		& 1000 & 3685.57  \\ \cline{3-5}
		
		& & \begin{tabular}[c]{@{}c@{}}S: What is the average of the index of\\
			 top 30 largest numbers in the list?\end{tabular} 
		& 1000 & 3654.78 \\ \midrule
		
		\multirow{3}{*}{} 
		& \multirow{3}{*}{
		\begin{tabular}[c]{@{}c@{}}
			\{date: 2024-06-19,\\
			close\_price: 9.79, \\
			open\_price: 9.4, \\
			\ldots \\
			PE\_ratio: 4.5416\}
		\end{tabular}
		} 
		& \begin{tabular}[c]{@{}c@{}}
			R: On which date did the close price\\
			 of stock firstly reach 61.76 yuan?
		\end{tabular}
		& 1000 & 27585.35 \\ \cline{3-5}
		
		\textbf{Stock}
		& 
		& \begin{tabular}[c]{@{}c@{}}
			C: Among the top-45 trading value days, which\\
			 date did the stock have the lowest close price?
		\end{tabular}
		 & 1000 & 27595.40 \\ \cline{3-5}
		
		& & \begin{tabular}[c]{@{}c@{}} 
			S: How many days had the close price higher than\\
			 the open price from 2024-07-31 to 2024-12-13?
		\end{tabular}	
		& 1000 & 27561.29 \\ \midrule
		
		\multirow{3}{*}{} 
		& \multirow{3}{*}{
		\begin{tabular}[c]{@{}c@{}}
			\{date: 2024-07-21,\\
			pressure\_msl: 999.96,\\
			dew\_point\_2m: 26.25,\\
			\ldots \\
			cloud\_cover: 61.5\}
		\end{tabular}
		} 
		& \begin{tabular}[c]{@{}c@{}} 
			R: On which date did the dew point temperature\\
			 at two meters firstly drop below 9.15°C?
		\end{tabular}
		& 1000 & 27359.26 \\ \cline{3-5}
		
		\textbf{Weather}
		& & \begin{tabular}[c]{@{}c@{}} 
			C: On which date did the MSL pressure reach its\\
			highest value when the cloud cover was below 9\%?
		\end{tabular}
		& 1000 & 27368.19 \\ \cline{3-5}
		
		& & \begin{tabular}[c]{@{}c@{}} 
			S: What was the average temperature at two meters\\
			when the relative humidity exceeded 78.56\%?
		\end{tabular}
		& 1000 & 27331.21 \\ \midrule
		
		\textbf{Sequence} 
		& $[0.34, 3, 6, \ldots, 111]$ 
		& L: What is the next number in the sequence? & 500 & 677.57 \\ \midrule
		
		\textbf{\begin{tabular}[c]{@{}c@{}}Arithmetic \\Operation\end{tabular}} 
		& \begin{tabular}[c]{@{}c@{}} 
		$a: 6.755,
		b: -1.225$
		\end{tabular}
		& \begin{tabular}[c]{@{}c@{}} 
		 $Q_{oper}$: What is the result of $a + b$?\\
		 $Q_{context}$: What is the result of $a $ plus $b$?
		 
		\end{tabular}
		& 12000 & 112.00 \\ \midrule
		
		\textbf{\begin{tabular}[c]{@{}c@{}}Mixed-number-string\\ Sequence\end{tabular}} 
		& \begin{tabular}[c]{@{}c@{}} 
		$effV2\ldots x98o7Lo$
		\end{tabular}
		& \begin{tabular}[c]{@{}c@{}} 
		How many numbers are there in the string? Note\\
		that a sequence like 'a243b' counts as a single number.
		\end{tabular}
		& 2000 & 196.53 \\ \bottomrule

	\end{tabular}
}
	\label{tab:data_stat}
	
\end{table*}

 

\subsection{Numeric Dataset Collection}
NumericBench offers a diverse collection of numerical datasets and questions designed to reflect real-world scenarios. 
This variety ensures that LLMs are thoroughly tested on their fundamental  abilities on numerical data.

\noindent\textbf{Number List Dataset.}
The synthetic number list dataset consists of simple collections of numerical values (integer and floats) 
presented as ordered or unordered lists.
Numbers in lists are one of the simplest and most fundamental data representations encountered in real-world scenarios.
Despite their simplicity, retrieving, indexing,  comparison, and summary on numbers can verify the fundamental numerical ability of LLMs. 
This dataset serves as a fundamental dataset of how well LLMs understand numerical values as discrete entities.



\noindent\textbf{Stock Dataset.}
The time-series  stock dataset is crawled from Eastmoney website~\cite{eastmoney}, 
which has eighteen attributes, such as stock close prices, open price,  trading volumes, and price-earnings ratio, over time.
Stock  data reflects dynamic, real-world numerical reasoning challenges that involve trend analysis, comparison, and decision-making under uncertainty,  representing real-world financial workflows.
 




\noindent\textbf{Weather Dataset.}
The weather dataset is crawled from Open-Meteo python API~\citep{openmeteo}, which includes data related to weather metrics, such as temperature, precipitation, humidity, and wind speed. 
The data is presented across various longitude and latitude.
 
 




\noindent\textbf{Numeric Sequence  Dataset.}
The synthetic numeric sequence dataset comprises sequences of numbers generated by arithmetic or geometric progression, complex patterns, or noisy inputs. 
Tasks require identifying patterns, predicting the next number, or reasoning about relationships between numbers.
Numerical sequences test the logical reasoning capabilities of LLMs, requiring pattern recognition and multi-step reasoning. This dataset introduces structured challenges that are common in mathematics and algorithmic reasoning.


 
\noindent\textbf{Arithmetic Operation Dataset.}
The dataset comprises 12,000 pairs of simple numbers, each undergoing addition, subtraction, multiplication, and division operations. Each pair of numbers, $a$ and $b$, consists of $k$-digit integers with three decimal places, where $k \in \{1, 2, \cdots, 6\}$. 
For each value of $k$, there are 2,000 pairs, evenly distributed across the four basic operations (i.e, $+, -,  *, /$), with 500 pairs per operation. 
This dataset is to evaluate the fundamental mathematical operation capabilities of LLMs, simulating the majority of mathematical calculation requirements in real-world scenarios.

\noindent\textbf{Mixed-number-string Sequence Dataset.}
The dataset consists of alphanumeric strings of varying lengths $\{50, 100, 150, 200\}$, each containing a randomized mix of letters and digits. For each string length, 500 samples are generated, resulting in a total of 2,000 samples. Each sample includes a query asking for the count of contiguous numeric sequences within the string, where sequences like "a243b" count as a single number. This dataset is designed to assess the ability of LLMs to identify and count numeric sequences.
 







\subsection{Fundamental Numerical Ability}
NumericBench is designed to comprehensively evaluate six fundamental numerical reasoning abilities of LLMs, which is 
%These three fundamental abilities are 
essential for solving real-world numeric-related tasks.
%such as numeric data summary and financial price analysis.


\noindent\textbf{Contextual Retrieval Ability.}
Contextual retrieval ability evaluates how well LLMs can locate, extract, and identify specific numerical values or their positions within structured or unstructured data. 
This includes tasks like finding a specific number in a list, retrieving values , and indexing numbers based on their order.
For example, as shown in Table~\ref{tab:data_stat}, it evaluates LLMs on tasks such as retrieving stock prices and identifying key values within numerical lists or domain-specific data (e.g., stock market and weather-related information).
This ability is fundamental to numerical reasoning because it forms the foundation for higher-order tasks, such as comparison, aggregation, and logical reasoning. 
 
 



\noindent\textbf{Comparison Ability.}
Comparison ability evaluates how well LLMs can compare numerical values to determine relationships such as greater than, less than, or equal to, and identify trends or differences in datasets. 
Comparison is vital for logical reasoning and decision-making, as many real-world tasks depend on accurate numerical evaluation. 
For instance,  as shown in Table~\ref{tab:data_stat},   comparing prices is essential in stock  for assessing performance, while weather forecasting requires analysis of temperature or precipitation trends over time. 
 



\noindent\textbf{Summary Ability.}
Summary ability assesses the LLM’s capacity to aggregate numerical data into concise insights, such as calculating totals, averages, or other statistical metrics. 
Summarization is critical for condensing large datasets into actionable information, enabling decision-making based on aggregated insights rather than raw data. 
This ability is indispensable in domains like electricity usage analysis, where summarizing hourly or daily consumption helps forecast bills, in business reporting for aggregating sales and revenue data to evaluate performance, 
and in healthcare analytics to monitor trends in patient metrics over time.



\noindent\textbf{Logic Reasoning Ability.}
Logical Reasoning Ability measures the LLM’s ability to perform multi-step operations involving numerical data, 
such as recognizing patterns, inferring rules, and applying arithmetic or geometric reasoning to solve complex problems. Logical reasoning extends beyond simple numerical tasks and reflects the LLM’s capacity for deeper, structured thinking. 
This ability is crucial for algorithm design, where solving problems involving numeric sequences or patterns is essential, in scientific research for identifying relationships and correlations in data.

\noindent\textbf{Arithmetic Operation Ability.}
It reflects the LLM's capacity to perform mathematical calculations accurately. Such ability is essential for tasks involving numerical computations, such as  automated machine learning through LLMs.





\noindent\textbf{Number Recognition  Ability.}
This measures the LLM's proficiency in identifying and interpreting numerical information within a given context. It represents a fundamental requirement for handling numeric-based tasks effectively.




\subsection{NumericBench Generation}
We use the number list, stock, and weather datasets to evaluate the contextual retrieval, comparison, and summary abilities of LLMs. 
Specifically, for each ability and each dataset, we prepare a set of questions designed to assess the corresponding target ability.
As shown in Table~\ref{appx:number_question}, Table~\ref{appx:stock_question}, and Table~\ref{appx:weather_question} in Appendix, there are nine question sets in total, covering three abilities across three datasets. 
When evaluating a specific ability (e.g., contextual retrieval) on a specific dataset (e.g., stock data), we randomly select one question from the corresponding question set for each data instance (e.g., a stock instance) 
and manually label the answer. This approach enables us to generate question-answer pairs for each ability on the number list, stock, and weather datasets.

For arithmetic operations and number counting in the strings dataset, the question format is straightforward, as illustrated in Table~\ref{tab:data_stat}. These questions are designed to evaluate the basic arithmetic operation and number recognition abilities of LLMs.



\section{Experiments}

\subsection{Experiment Setting}

\noindent\textbf{Benchmarks and Evaluated Protocols.}
The statistic of  NumericBench is provided in Table~\ref{tab:data_stat}.
Also,
we set the exact answer for mixed-number-string dataset, 
set
the computed answer to two decimal places for arithmetic datasets, and  set the answer of each question as a single choice (e.g., A, B, or C) for other datasets to reliable evaluate LLMs~\citep{bai2024longbench}.
The evaluation metric is accuracy.

\begin{table*}[t]
	\centering
		\vspace{-2em}
	\setlength\tabcolsep{2pt}
	\footnotesize
	\caption{Evaluation of LLMs on numerical contextual retrieval, comparison, and summary tasks across number list, stock, and weather datasets. 
		Also, * indicates that scores are calculated based on a short subset of outputs, as these models cannot handle  long contexts and exhibit disruption when tested on longer instances.}
	\begin{tabular}{c|ccc|ccc|ccc|c}
		\toprule
		\multirow{2}{*}{\textbf{Model}} & \multicolumn{3}{c|}{\textbf{Retrieval}}                                                         & \multicolumn{3}{c|}{\textbf{Comparison}}                                                           & \multicolumn{3}{c|}{\textbf{Summary}}                                                           & \textbf{Logic}    \\ \cmidrule{2-11} 
		
		& \multicolumn{1}{c}{\textbf{Number}} & \multicolumn{1}{c}{\textbf{Stock}} & \textbf{Weather} & \multicolumn{1}{c}{\textbf{Number}} & \multicolumn{1}{c}{\textbf{Stock}} & \textbf{Weather} & \multicolumn{1}{c}{\textbf{Number}} & \multicolumn{1}{c}{\textbf{Stock}} & \textbf{Weather} & \textbf{Sequence} \\ \midrule
		
		\textbf{\texttt{Random}} & \multicolumn{1}{c}{12.5}                  & \multicolumn{1}{c}{12.5}               &          12.5        & \multicolumn{1}{c}{12.5}                  & \multicolumn{1}{c}{12.5}               &                 12.5 & \multicolumn{1}{c}{12.5}                  & \multicolumn{1}{c}{12.5}               &      12.5            &         12.5          \\ \midrule
		
		
		\textbf{\texttt{Llama-3.1-8B-Inst}}& \multicolumn{1}{c}{22.8}                  & \multicolumn{1}{c}{14.4}               &      13.5          & \multicolumn{1}{c}{19.5}                  & \multicolumn{1}{c}{11.7}               &     13.7            & \multicolumn{1}{c}{18.1}                  & \multicolumn{1}{c}{13.8}               &       13.9*          &       18.2            \\  
		
		\textbf{\texttt{Llama-3.1-70B-Inst}}& \multicolumn{1}{c}{37.3}                  & \multicolumn{1}{c}{17.4}               &     23.0             & \multicolumn{1}{c}{28.3}                  & \multicolumn{1}{c}{15.0}               &     28.7             & \multicolumn{1}{c}{24.7}                  & \multicolumn{1}{c}{16.4}               &       15.2           &     17.8              \\  
		
		\textbf{\texttt{Llama-3.3-70B-Inst}}& \multicolumn{1}{c}{44.4}                  & \multicolumn{1}{c}{19.4}               &      23.1            & \multicolumn{1}{c}{31.5}                  & \multicolumn{1}{c}{13.8}               &       35.8           & \multicolumn{1}{c}{26.3}                  & \multicolumn{1}{c}{16.8}               &   18.0               &     18.6              \\  
		
		\textbf{\texttt{Llama-3.1-405B-Inst}}& \multicolumn{1}{c}{44.6}                  & \multicolumn{1}{c}{26.8}               &          19.8        & \multicolumn{1}{c}{25.1}                  & \multicolumn{1}{c}{14.8}               &     29.8             & \multicolumn{1}{c}{32.9}                  & \multicolumn{1}{c}{17.0}               &    16.1              &     16.6              \\  
		
		\textbf{\texttt{Llama-3.1-Nemotron-70B-Inst}}& \multicolumn{1}{c}{41.6}                  & \multicolumn{1}{c}{19.3}               &        24.9          & \multicolumn{1}{c}{26.6}                  & \multicolumn{1}{c}{13.7}               &      33.6            & \multicolumn{1}{c}{29.4}                  & \multicolumn{1}{c}{16.5}               &     17.0             &     16.4              \\  
		
		\textbf{\texttt{Qwen2.5-7B-Inst}}& \multicolumn{1}{c}{20.2}                  & \multicolumn{1}{c}{17.3}               &    19.6              & \multicolumn{1}{c}{24.8}                  & \multicolumn{1}{c}{17.8}               &      18.8            & \multicolumn{1}{c}{18.5}                  & \multicolumn{1}{c}{11.7}               &     13.8             &    14.4               \\  
		\textbf{\texttt{Qwen2.5-72B-Inst}}& \multicolumn{1}{c}{28.8}                  & \multicolumn{1}{c}{41.4*}               &       12.4*           & \multicolumn{1}{c}{28.0}                  & \multicolumn{1}{c}{26.0*}               &       31.0*           & \multicolumn{1}{c}{31.9}                  & \multicolumn{1}{c}{18.8*}               &        16.4*          &      19.0             \\  
		\textbf{\texttt{GLM-4-Long}}& \multicolumn{1}{c}{26.5}                  & \multicolumn{1}{c}{19.5}               &       8.4           & \multicolumn{1}{c}{18.9}                  & \multicolumn{1}{c}{14.8}               &      21.6            & \multicolumn{1}{c}{20.8}                  & \multicolumn{1}{c}{10.8 }               &      10.5            &        17.6           \\  
		
				\textbf{\texttt{Deepseek-V3}}& \multicolumn{1}{c}{47.2}                  & \multicolumn{1}{c}{47.5}               &       10.9          & \multicolumn{1}{c}{27.0}                  & \multicolumn{1}{c}{22.5}               &       35.8          & \multicolumn{1}{c}{21.8}                  & \multicolumn{1}{c}{13.0}               &       15.1          &   15.8                \\  
		
		\textbf{\texttt{GPT-4o}}& \multicolumn{1}{c}{41.7}                  & \multicolumn{1}{c}{37.5}               &        15.4          & \multicolumn{1}{c}{30.6}                  & \multicolumn{1}{c}{33.0}               &       64.2           & \multicolumn{1}{c}{11.6}                  & \multicolumn{1}{c}{17.4}               &      16.5            &        14.6           \\ 
		

		
		\midrule
		
		\textbf{\texttt{Human Evaluation}}& \multicolumn{1}{c}{\textbf{100}}                  &  \multicolumn{1}{c}{\textbf{100}}               &      \textbf{100}            & \multicolumn{1}{c}{\textbf{100}}                  & \multicolumn{1}{c}{\textbf{100}}               &        \textbf{100}          & \multicolumn{1}{c}{\textbf{100}}                  & \multicolumn{1}{c}{\textbf{100}}               &          \textbf{100}        &               \textbf{52.6}   \\ \bottomrule
	\end{tabular}
	\label{tab:main_experiments}
\end{table*}
\begin{figure*}[t]
		\vspace{-1em}
	\centering 	
	\subfloat[Contextual  Retrieval]	
	{\centering\includegraphics[width=0.33\linewidth]{image/main_fig/retrieval-num-list.pdf}}
	\hfill
	\subfloat[Comparison]
	{\centering\includegraphics[width=0.33\linewidth]{image/main_fig/compare-num-list.pdf}}
	\subfloat[Summary]	
	{\centering\includegraphics[width=0.33\linewidth]{image/main_fig/summary-num-list.pdf}}
	\hfill
	\caption{Evaluation on short and long context on number list.}
	\label{fig:length_number}
	
\end{figure*}
\noindent\textbf{Evaluated Models.}
To comprehensively evaluate the retrieval and reasoning abilities of state-of-the-art and widely-used LLMs on numeric data, 
we benchmark over 10 popular LLMs with our constructed NumericBench, as follows.
\begin{itemize}[leftmargin=*]
	\item \textbf{The Llama Series~\citep{grattafiori2024llama3herdmodels}.} include Llama-3.1-8B Instruct, Llama-3.1-70B Instruct, Llama-3.1-405B Instruct, 
	Llama-3.3-70B-Instruct and Llama-3.1-Nemotron-70B-Instruct.
	%Deepseek-R1~\citep{deepseekai2025deepseekr1incentivizingreasoningcapability}, Deepseek R1-Zero, Deepseek-V3~\citep{liu2024deepseek}, GLM-4-Plus~\citep{glm2024chatglm}, GLM-4-Long~\citep{glm2024chatglm}, Claude Sonnet 3.5, Claude 3.5 Haiku, GPT-4o, GPT-4o-mini, GPT-o3 mini, Gemini 2.0 Pro, Llama-3.1-8B/70B/405B Instruct~\citep{grattafiori2024llama3herdmodels}, Llama-3.3-8B/70B Instruct, Llama-3.1-Nemotron-70B-Instruct, Qwen2.5-7B/72B Instruct, InternLM2.5-20B-Chat
	\item \textbf{The Qwen Series~\citep{qwen2025qwen25technicalreport}.} include the effective Qwen2.5-7B-Instruct and Qwen2.5-72B-Instruct. 
	%	\item \textbf{Math-oriented Models} include DeepSeek-Math-Instruct 7B~\citep{deepseek-math}, MetaMath-Llemma-7B~\citep{azerbayev2023llemma}~\citep{yu2023metamath} Mammoth-7B/14B~\citep{yue2023mammoth}	
	\item \textbf{The GLM Series~\citep{glm2024chatglm}.} We use GLM4-Long to run the benchmark, since it is the commonly used in GLM series.
	% due to the overly high price of GLM4-Plus. 
	
	\item \textbf{The Deepseek Series~\citep{liu2024deepseek}~\citep{deepseekai2025deepseekr1incentivizingreasoningcapability}.} We currently use Deepseek V3 to run the benchmark. 
	Deepseek R1 will be evaluated in the future, since its API is down and unavailable now . 
	\item  \textbf{The GPT Series~\cite{achiam2023gpt}.} We use GPT-4o to run the benchmark. 
\end{itemize}

 
	We attempted to conduct experiments   on various math-oriented LLMs, such as Metamath-Llemma-7B~\citep{yu2023metamath}, Deepseek-Math-7B-instruct~\citep{deepseek-math}, InternLM2-Math-7B~\citep{ying2024internlmmathopenmathlarge} and MAmmoTH-7B~\citep{yue2023mammoth}.
	 However, these models fail during experiments for various reasons such as overly long output sequence length and limited input sequence length. Fail cases are demonstrated in the Figure~\ref{fig:fail_internlm},  Figure~\ref{fig:fail_ds_math},  Figure~\ref{fig:fail_llemma}, and  Figure~\ref{fig:fail_mammoth} in Appendix. 



\subsection{Main Experiments}
\noindent \textbf{Evaluation on Contextual Retrieval, Comparison, Summary, and Logic Reasoning Abilities.}
As shown in Table~\ref{tab:main_experiments}, 
current popular and effective LLMs perform poorly on basic numerical tasks, 
including retrieval, comparison, summarization, and logical reasoning. 
The random baseline for each task is 12.5\%, as there are 8 choices, and the probability of randomly selecting the correct answer is 1/8. 
Human evaluation was conducted by three undergraduate students. 

Firstly,
LLMs particularly struggle with accurately retrieving numerical data.
This limitation arises from LLMs treating numbers as discrete tokens rather than continuous ones, coupled with insufficient exposure to structured numerical datasets during training, which restricts their ability to handle simple numeric retrieval tasks. 
Secondly, LLMs demonstrate weaknesses in recognizing numerical relationships, such as greater-than or less-than comparisons, due to a lack of numerical semantics and underdeveloped arithmetic reasoning capabilities. 
Thirdly,
LLMs also perform poorly in summarizing numerical data (e.g., calculating sums or means), reflecting their inability to execute multi-step numerical operations. 
Similarly, logical reasoning tasks, especially those involving patterns or sequences, are particularly challenging, with all models scoring below 20\%. 
These tasks require multi-step reasoning, pattern recognition, and arithmetic operations, which expose the architectural limitations of current LLMs.









\begin{figure*}[t]
	\vspace{-2em}
	\centering 	
	\subfloat[Contextual  Retrieval]	
	{\centering\includegraphics[width=0.33\linewidth]{image/noisy_dataset_fig/retrieval-noisy-stock.pdf}}
	\hfill
	\subfloat[Comparison]
	{\centering\includegraphics[width=0.33\linewidth]{image/noisy_dataset_fig/compare-noisy-stock.pdf}}
	\subfloat[Summary]	
	{\centering\includegraphics[width=0.33\linewidth]{image/noisy_dataset_fig/summary-noisy-stock.pdf}}
	\hfill
 
	\caption{Evaluation on noisy stock dataset. Due to the input sequence length limit of Qwen2.5-72B-Inst on the API platform, the data containing 6 irrelevant attributes cannot be evaluated using this model.}
	\label{fig:noisy_stock}
	
\end{figure*}




\begin{figure*}[t]
	\vspace{-1em}
	\centering 	
	\subfloat[Accuracy on $Q_{oper}$  (i.e., $a+b$)]	
	{\centering\includegraphics[width=0.32\linewidth]{image/arithmetic_fig/arith_bar.pdf}}
	\hfill
	\subfloat[ $Q_{oper}$  of different digits]
	{\centering\includegraphics[width=0.32\linewidth]{image/arithmetic_fig/arith_plot.pdf}}
	\subfloat[Accuracy on $Q_{context}$  (i.e., $a$ plus $b$)]
	{\centering\includegraphics[width=0.32\linewidth]{image/arithmetic_text_fig/arith_bar.pdf}}
 
	\caption{Evaluation on arithmetic operation.}
	\label{fig:arithmetic_fig}
		\vspace{-1em}
\end{figure*}



 



 
\noindent \textbf{Evaluation on  Different Context Length via Stock and Weather Datasets.}
We evaluate LLMs on varying context lengths.
Specifically, we categorize the contexts of number lists, stock data, and weather data into short and long contexts.
The average token numbers for the short and long contexts across the three datasets are listed in Table~\ref{tab:data_stat_short_long}.
As illustrated in Figure~\ref{fig:length_number}, Figure~\ref{fig:length_stock}, and Figure~\ref{fig:length_weather},
LLMs generally achieve lower accuracy on long contexts compared to short contexts. This is because long contexts require the model to have a stronger ability to capture long-range dependencies.
Furthermore, if an LLM fails to perform well on short contexts, it is unlikely to achieve good results on long contexts. 
It highlights the importance of the inherent capabilities of LLMs in understanding numeric data.



\noindent \textbf{Evaluation on Noisy Context  via Stock and Weather Datasets.}
To evaluate the numerical abilities of LLMs in  noisy contexts, we add $k\in\{2,4,6\}$ irrelevant attributes to each instance in the stock and weather. 
These irrelevant attributes are not used in the user queries.
As shown in Figure~\ref{fig:noisy_stock} and Figure~\ref{fig:noisy_weather} in Appendix, 
as $k$ increases, most LLMs exhibit degraded performance. This indicates that irrelevant context can  affect the LLM's numerical retrieval and reasoning abilities.

 

\noindent \textbf{Evaluation on Arithmetic Operations}
Similarly, 
we evaluate five strong LLMs on arithmetic operations.
Specifically, as illustrated in Figure~\ref{fig:arithmetic_fig}~(a), even for simple arithmetic operations involving two numbers, LLMs fail to achieve 100\% accuracy. 
Moreover, as the number of digits increases shown in Figure~\ref{fig:arithmetic_fig}~(b), the accuracy of LLMs decreases, highlighting their limited ability to handle arithmetic tasks effectively, which is also observed in~\citep{qiu2024dissecting}.
This poor performance stems from how LLMs generate responses. LLMs  predict the highest-order digit  before the lower-order digit~\citep{zhang2024reverse}, contradicting the standard arithmetic logic of progressing from lower- to higher-order digits.
In particular, Figure~\ref{fig:arithmetic_fig}~(a) and (c) shows that LLMs perform similarly on addition, subtraction, and division operations but achieve extremely low accuracy on multiplication tasks.






\noindent \textbf{Evaluation on Number Recognition via Mixed-number-string Dataset.}
We evaluate the number recognition ability of effective LLMs by identifying numbers from mixed-number-string sequences. For this evaluation, we select five  effective LLMs based on Table~\ref{tab:main_experiments}, including DeepSeek-v3, GLM-4-Long, LLaMA3.1-405B, and Qwen2.5-72B.
As shown in Table~\ref{tab:number_counting}, all LLMs achieve extremely low accuracy in counting numbers within strings. Moreover, as the length of the string increases from 50 to 100, the accuracy of the LLMs decreases further.
These results highlight that LLMs are significantly weak at distinguishing numbers from strings. The underlying reason is that current LLMs treat numbers as strings during training. 
This training paradigm inherently limits their ability to understand and process numbers effectively.
Also, the tokenizer can split a single number into multiple tokens, which can negatively affect the numeric meaning of each number.








\begin{table}[]
	\centering
	\small
	\caption{Evaluation on mixed-number-string data with lengths ranging from 50 (i.e., 50 L) to 200.}
	
	%	\footnotesize
	\begin{tabular}{c|cccc}
		\toprule
		\textbf{Model}    & \textbf{50 L} & \textbf{100 L} & \textbf{150 L} & \textbf{200 L} \\ \midrule
		
		
		\textbf{\texttt{LLama3.1-405B }}& 10.8      & 9.2        & 3.2        & 2.2        \\  
		
		\textbf{\texttt{Qwen2.5-72B}}   & 3.0         & 1.2        & 0.6        & 0.2        \\  
		
		\textbf{\texttt{GLM4-Long}  }   & 6.6       & 4.8        & 3.0          & 2.4        \\  
		
		\textbf{\texttt{GPT-4o }}       & 18.2      & 6.4        & 4.0          & 4.2        \\ 
		
		\textbf{\texttt{DeepSeek-V3}}   & 13.2      & 4.0          & 3.2        & 2.0          \\  
		\midrule
		\textbf{\texttt{Human Eval } }   & \textbf{100}      & \textbf{100}        & \textbf{100}         & \textbf{100}        \\ \bottomrule
	\end{tabular}
	\label{tab:number_counting}
\end{table}
\subsection{Discussions on Numeracy Gaps of LLMs}
In summary, extensive experimental results show that current state-of-the-art LLMs perform poorly on six fundamental numerical abilities.
% such as number recognition and arithmetic operations. 
Here we discuss five potential reasons behind their poor performance on numerical tasks.

\noindent \textbf{Tokenizer Limitation.}
LLMs use tokenizers to split input text into smaller units (tokens). Thus,
Numbers are split into chunks as strings, based on statistical patterns in the training data.
For example, $10000$ is split into $100$ and $00$ tokens\footnote{\url{https://gptforwork.com/tools/tokenizer}}.
These tokenizers do not considering  the real meaning of numbers and continuous magnitude of numbers.
Thus, LLMs do not perform well on simple number retrieval and comparison tasks.

\noindent \textbf{Training Corpora Limitation.}
LLMs are trained on extensive corpora, which also limits their ability to understand numerical-related symbols, such as $*$.
For example, the multiplication of 246 and 369 can be denoted as $246*369$.
However, $246*369$ may be interpreted as a password or encrypted text, since $*$ in text strings is often associated with encryption.
As a result, enabling LLMs to accurately interpret arithmetic symbols remains an open problem.


\noindent \textbf{Training Paradigm Limitation.}
The training of LLMs relies on the next-token prediction paradigm, which is inherently misaligned with the logic of numerical computation.
For example, when solving $16 + 56$ with the result being $72$, an LLM will first predict the highest-order digit of the answer (i.e., $7$) before predicting the lower-order digit (i.e., $2$). This approach contradicts the fundamental logic of arithmetic computation, which typically proceeds from the lower-order digit to the higher-order digit.
This discrepancy implies that LLMs effectively need to know the entire result upfront to generate digits sequentially in the correct order. As a result, LLMs struggle to perform well even on simple arithmetic operations.

\noindent \textbf{Positional Embedding Limitation.}
Note that LLMs incorporate positional embeddings for  tokens in sequence inputs. In arithmetic operations like $12 + 26$ and $26 + 12$, the order of the numbers does not affect the result. However, LLMs assign different positional embeddings to the number $12$ in each equation, as its position in the sequence differs. 
This lack of invariance in positional embeddings for numbers can influence the results.
Therefore, how to design the positional embedding that improves numerical ability of LLMs without affecting the text understanding  of LLMs is critical~\cite{mcleish2024transformers,golovneva2024contextual}.



\noindent \textbf{Transformer Architecture Limitation.}
LLMs use Transformer to process input sequence, which rely on pattern recognition rather than explicit algorithmic reasoning.
The computational power of transformers has upper bounds~\cite{merrill2023parallelism}. Considering the complexity of arithmetic operations in real-world applications, it still needs to be theoretically investigated whether transformers can perform well on numerical operations.

\section{Conclusion and Discussion}
\label{sec:conclusion}

We introduce a novel framework for augmenting \emph{any} lossy compressor to preserve the contour tree of a volumetric dataset while maintaining a user-specified global error bound. 
To do this, our framework first imposes topology-informed upper and lower bounds on each data point. 
It then progressively tightens those bounds until the contour tree is preserved. 
We also introduce a novel encoding scheme that efficiently stores individual points with variable precision and maintains these upper and lower bounds. 
When our framework is used to augment state-of-the-art lossy compressors, it is shown to preserve the contour trees of various scientific datasets.
Our augmented compressors also achieve higher compression ratios and reconstruction quality than those obtained by existing topology-preserving compressors in comparable or faster time.
Our framework will benefit from any advancement with lossy compression since it can be used to augment increasingly effective lossy compressors to achieve better topology-preserving compression. 

Our framework is not without limitations. The compression times are longer than the base compressors. This difference gets worse as the topological complexity of the data increases.
However, in some use-cases, topological preservation is preferable to run time.
Regardless, our framework would benefit from more efficient or parallel implementations for the contour/merge tree computation and the encoding scheme. 

\clearpage
\section*{Limitations}
There are two main limitations of this paper.
Firstly, the numerical tasks encountered in real-world scenarios are often far more complex and diverse compared to the six datasets proposed in NumericBench. 
Expanding the scope to include a broader range of numerical reasoning categories, such as traffic, would provide a more comprehensive assessment. 
Nevertheless,  our work can serves as a meaningful  point, highlighting  the current limitations of LLMs in numerical tasks. 
We also analyze the potential reasons why LLMs struggle with numerical reasoning tasks, which can be attributed to the inherent limitations of transformer architectures and the next-token prediction objective.
We
hope it inspires further efforts to address these challenges and develop   more advanced LLMs with enhanced numerical  capabilities.



Secondly, although we evaluate ten state-of-the-art LLMs, several newer LLMs and their variants, such as Claude and GPT-o1 from major companies, are not included in our experiments. 
The reason for this exclusion is the expensive cost of accessing these model APIs. 
In brief, evaluating additional LLM variants across Claude, OpenAI, Mistral and GLM, typically requires a minimum budget of \textbf{\$15,000} US dollars.
Specifically, experiments on the datasets in Table~\ref{tab:main_experiments} require approximately 180 million tokens as inputs, while all left experiments (e.g., noisy contexts) require about 84 million tokens as inputs. 
For 1 million input tokens, Claude 3 Opus costs \$15\footnote{\url{https://www.anthropic.com/pricing\#anthropic-api}}, 
Claude 3.5 Sonnet costs \$3\footnote{\url{https://www.anthropic.com/pricing\#anthropic-api}}, 
OpenAI-o1 costs \$15\footnote{\url{https://openai.com/api/pricing/}}, 
Gemini 1.5 Pro costs \$12.5\footnote{\url{https://ai.google.dev/pricing\#1_5pro}}, 
GLM4-Plus costs \$6.89\footnote{\url{https://bigmodel.cn/pricing}},
Mistral Large 24.11 costs \$2 \footnote{\url{https://mistral.ai/en/products/la-plateforme}},
Mixtral 8x22B costs \$2 \footnote{\url{https://mistral.ai/en/products/la-plateforme}},
and OpenAI o3-mini costs \$1.1\footnote{\url{https://openai.com/api/pricing/}}.

If we conduct experiments above with these top-tier models from major companies, it would cost at least 3960 dollars for Claude 3 Opus, 3960 dollars for OpenAI-o1, 3300 dollars for Gemini 1.5 Pro, 1819 dollars for GLM4-Plus, 792 dollars for Claude Sonnet 3.5, 528 dollars for Mistral Large 24.11, 528 dollars for Mixtral 8x22B and 290 dollars for OpenAI o3-mini, which is beyond our expected total experiment cost. Meantime, there are too many LLM variants in each series.

Also, for models such as OpenAI-o1, 
which require generating really longer outputs for reasoning purposes, the output length is often unpredictable, while the model charges for \$60 per million output tokens, making the experiments even more expensive and difficult to control.
Particularly,
The reason for not using DeepSeek-R1 is that its official API is currently down and unavailable. We plan to include it in our comparisons once the API is restored.
Considering that GPT-4o and DeepSeek-V3 represent  the most state-of-the-art LLM models, we believe our evaluation can reflects the current numerical abilities of leading-edge LLMs.
Therefore, our evaluation highlights the weaknesses of LLMs in numerical abilities and serves as a bridge to inspire further research focused on improving the numerical capabilities of these models.




 

\bibliographystyle{acl_natbib}
\bibliography{acl} 
\appendix
\clearpage
 
\definecolor{exampleblue}{RGB}{0, 114, 188} % Blue for header
\definecolor{exampleborder}{RGB}{0, 114, 188} % Blue for border
\definecolor{redtext}{RGB}{204, 0, 0}         % Red text for emphasis

\section{Appendix}
In this appendix, we provide additional details about the design of \textbf{NumericBench}, along with supplementary experimental results and case studies. The organization of the supplementary materials in this appendix is as follows:

\begin{enumerate}[leftmargin=*]
	
	\item \textbf{Question formats for contextual retrieval, comparison, and summary abilities.}  
As shown in Table~\ref{appx:number_question}, Table~\ref{appx:stock_question}, and Table~\ref{appx:weather_question}, 
we designed diverse question types tailored to each dataset to evaluate the three fundamental numerical abilities of LLMs: contextual retrieval, comparison, and summary. contextual retrieval  assesses the model’s capacity to accurately extract relevant numerical information from complex contexts; comparison tests the ability to analyze and compare numerical values;  Summary evaluates the synthesis of numerical information into concise and meaningful insights for tasks like reporting or trend analysis.


By designing tailored questions for each dataset, we ensure a comprehensive evaluation of LLMs’ numerical reasoning abilities across varying scenarios and complexities.
	\item \textbf{Basic numerical questions answered incorrectly by GPT-4o.}  
	As illustrated in Figure~\ref{fig:number_compare}, Figure~\ref{fig:multiplication}, and Figure~\ref{fig:number_couting}, GPT-4o failed to answer three basic numerical questions correctly. This result is surprising, considering GPT-4o's impressive performance in real-world applications. However, these findings highlight the weak fundamental numerical abilities of LLMs.
	
	\item \textbf{Token counts for short and long contexts.}  
	As shown in Table~\ref{tab:data_stat_short_long}, the token counts of long and short contexts differ significantly. This distinction enables a more thorough evaluation of LLM performance across scenarios involving varying context lengths. Short contexts are designed to test the model's ability to process and understand concise information, focusing on immediate comprehension and reasoning. In contrast, long contexts present a more complex challenge, requiring the model to handle extended sequences of information, maintain coherence over a larger context window, 
	and retrieve relevant details from earlier parts of the input. Such two type length can more comprehensively evaluate LLMs. 
	
	\item \textbf{Additional experimental results on noisy and varying-length contexts.}  
	As shown in Figure~\ref{fig:length_stock} and Figure~\ref{fig:length_weather}, existing LLMs perform poorly on the stock and weather datasets, although they achieve better performance compared to their results on short contexts. 
	Similarly, as shown in Figure~\ref{fig:noisy_weather}, LLMs perform poorly on noisy weather data.
 
 \item \textbf{Real failure cases of math-oriented LLMs.} In this paper, we do not compare existing math-oriented LLMs, such as Metamath-Llemma-7B~\citep{yu2023metamath}, Deepseek-Math-7B-Instruct~\citep{deepseek-math}, InternLM2-Math-7B~\citep{ying2024internlmmathopenmathlarge}, and MAmmoTH-7B~\citep{yue2023mammoth}. 
 This is primarily because these math-oriented LLMs are designed for specialized geometric and structured mathematical problems. They are unable to understand the tasks in NumericBench, fail to follow a correct reasoning process, and directly produce meaningless outputs. These failure cases are illustrated in Figure~\ref{fig:fail_internlm}, Figure~\ref{fig:fail_ds_math}, Figure~\ref{fig:fail_llemma}, and Figure~\ref{fig:fail_mammoth}.
 
\end{enumerate}

\noindent \textbf{The Use of AI Tools.} When writing  this paper, we use Grammarly\footnote{https://www.grammarly.com/} for automated spell checking and use GPT-4o\footnote{https://platform.openai.com/docs/models/gpt-4o} to refine several sentences.


\clearpage
 
 

\begin{table*}[!h]
	\centering
	
	\caption{Question format on number list dataset. R: contextual retrieval, C: comparison, S: summary. In the contextual retrieval task, a number $x$ is randomly selected from the given number list. For the comparison task, the $k$-th largest number is randomly generated within the range of one to the length of the number list. The indices $x$ corresponds to twenty percent of the length of the number list, while $y$ corresponds to eighty percent of the length. The number $z$ is randomly chosen within the range $(\min(\text{list}), \max(\text{list}))$. For the summary task, the top $k$ is set to thirty percent of the length of the number list.}
 
	\renewcommand{\arraystretch}{1.15}  
	\setlength{\tabcolsep}{1.5pt}  
	\begin{tabular}{c|c}
		\toprule
		\textbf{Ability}    & \textbf{Question Format} \\ \midrule
		\textit{\textbf{R}} &  \begin{tabular}[c]{@{}l@{}}
			$Q_0$: What is the index of the first occurrence of the number $x$ in the list?\\
			$Q_1$: What is the index of the last occurrence of the number $x$ in the list?\\
			$Q_2$: What is the number after the first occurrence of the number $x$ in the list?\\
			$Q_3$: What is the number before the last occurrence of the number $x$ in the list?\\
			$Q_4$: What is the index of the first even number in the list?\\
			$Q_5$: What is the index of the first odd number in the list?\\
			$Q_6$: What is the index of the last even number in the list?\\
			$Q_7$: What is the index of the last odd number in the list?
		\end{tabular} \\ \midrule
		\textit{\textbf{C}} &  \begin{tabular}[c]{@{}l@{}}
			$Q_8$: What is the index of the first occurrence of the $k$-th largest number in the given list?\\
			$Q_9$: Which index holds the greatest number in the list between the indices $x$ and $y$?\\
			$Q_{10}$: Which index holds the smallest number in the list between the indices $x$ and $y$?\\
			$Q_{11}$: Which number is closest to $z$ in the list between the indices $x$ and $y$?\\
			$Q_{12}$: Which number is furthest from $z$ in the list between the indices $x$ and $y$?\\
			$Q_{13}$: Which number is the largest among those less than $z$ in the list?\\
			$Q_{14}$: Which number is the smallest among those greater than $z$ in the list?
		\end{tabular} \\ \midrule
		\textit{\textbf{S}} &  \begin{tabular}[c]{@{}l@{}}
			$Q_{15}$: What is the maximum sum of any two consecutive items in the list?\\
			$Q_{16}$: What is the maximum sum of any three consecutive items in the list?\\
			$Q_{17}$: What is the maximum absolute difference between two consecutive items in the list?\\
			$Q_{18}$: What is the sum of the indices of the top $k$ largest numbers in the list?\\
			$Q_{19}$: What is the sum of the indices of the top $k$ smallest numbers in the list?\\
			$Q_{20}$: What is the average of the indices of the top $k$ largest numbers in the list?\\
			$Q_{21}$: What is the average of the indices of the top $k$ smallest numbers in the list?\\
			$Q_{22}$: How many times do numbers consecutively increase for more than five times?\\
			$Q_{23}$: How many times do numbers consecutively decrease for more than five times?\\
			$\cdots \cdots$ \\
		\end{tabular} \\ \bottomrule
	\end{tabular}	
	\label{appx:number_question}
\end{table*}
\clearpage


 

\begin{table*}[]
	
	\caption{Question format on stock dataset. R: contextual retrieval, C: comparison, S: summary. $x$ and $y$ lie within the minimum and maximum ranges of their respective attributes. The top $k$ corresponds to thirty percent of the number list. $date_1$ represents the day at the twentieth percentile of the stock history, while $date_2$ corresponds to the day at the eightieth percentile.}
	\centering
	\renewcommand{\arraystretch}{1.15} % 设置行间距为默认的 1.15 倍
	\setlength{\tabcolsep}{1.5pt} % 将列间距设置为 1pt
	\begin{tabular}{c|c}
		\toprule
		\textbf{Ability}    & \textbf{Question Format} \\ \midrule
		\textit{\textbf{R}} &  \begin{tabular}[c]{@{}l@{}}
			$Q_0$: On which date did the close price of the stock first reach $x$ yuan?\\
			$Q_1$: On which date did the highest price of the stock first reach $x$ yuan?\\
			$Q_2$: On which date did the volume of the stock first reach $x$ lots?\\
			$Q_3$: On which date did the value of the stock first reach $x$ thousand yuan?\\
			$Q_4$: On which date did the price change rate of the stock first reach $x$\%?\\
			$Q_5$: On which date did the price change of the stock first reach $x$ yuan?\\
		\end{tabular} \\ \midrule
		\textit{\textbf{C}} &  \begin{tabular}[c]{@{}l@{}}
			\begin{tabular}[c]{@{}l@{}}
				$Q_6$: On which date did the stock have the highest turnover rate when the close \\price was greater than $x$ yuan?
			\end{tabular}\\
			
			\begin{tabular}[c]{@{}l@{}}
				$Q_7$: On which date did the stock have the highest quantity relative ratio when \\the open price was less than $x$ yuan?
			\end{tabular}\\
			
			\begin{tabular}[c]{@{}l@{}}
				$Q_8$: On which date did the stock have the highest difference between the highest \\and lowest prices when the trading volume exceeded $x$ lots?
			\end{tabular}\\
			
			\begin{tabular}[c]{@{}l@{}}
				$Q_9$: On which date did the stock record the highest daily average price, calculated \\as 'value' divided by 'volume,' when the PE ratio was less than $x$?
			\end{tabular}\\
			
			\begin{tabular}[c]{@{}l@{}}
				$Q_{10}$: Among the top-$k$ trading value days, on which date did the stock have the \\lowest close price?
			\end{tabular}\\
			
			\begin{tabular}[c]{@{}l@{}}
				$Q_{11}$: When the quantity relative ratio exceeded $x$, on which date did the stock \\have the highest sum of the open price and close price?
			\end{tabular}\\
			
			\begin{tabular}[c]{@{}l@{}}
				$Q_{12}$: When the absolute price change rate exceeded $x$\%, on which date did the \\stock have the highest difference between the highest and lowest prices?
			\end{tabular}
		\end{tabular} \\ \midrule
		\textit{\textbf{S}} &  \begin{tabular}[c]{@{}l@{}}
			\begin{tabular}[c]{@{}l@{}}
				$Q_{13}$: How many days had a volume greater than $x$ from $date_1$ to $date_2$?
			\end{tabular}\\
			
			\begin{tabular}[c]{@{}l@{}}
				$Q_{14}$: How many days had the close price higher than the open price from \\$date_1$ to $date_2$?
			\end{tabular}\\
			
			\begin{tabular}[c]{@{}l@{}}
				$Q_{15}$: How many days had a close price higher than the open price, with the \\quantity relative ratio exceeding $x$\%?
			\end{tabular}\\
			
			\begin{tabular}[c]{@{}l@{}}
				$Q_{16}$: How many days had the close price reach $x$ yuan with the absolute price \\change rate exceeding $x$\%?
			\end{tabular}\\
			
			\begin{tabular}[c]{@{}l@{}}
				$Q_{17}$: What was the average trading volume when both the turnover rate \\exceeded $x$\% and the price change rate was greater than $y$\%?
			\end{tabular}\\
			
			\begin{tabular}[c]{@{}l@{}}
				$Q_{18}$: Excluding non-trading days, how many times did the open price of \\the stock rise for three or more consecutive days?
			\end{tabular}\\
			
			\begin{tabular}[c]{@{}l@{}}
				$Q_{19}$: Excluding non-trading days, how many times did the close price of \\the stock rise for three or more consecutive days?
			\end{tabular}\\
			
			\begin{tabular}[c]{@{}l@{}}
				$Q_{20}$: Excluding non-trading days, how many times did the open price and \\close price of the stock both rise for three or more consecutive days?
			\end{tabular}\\
		
			\begin{tabular}[c]{@{}l@{}}
			$\cdots \cdots$
		\end{tabular}
		
		\end{tabular} \\ \bottomrule
	\end{tabular}
\label{appx:stock_question}
\end{table*}
\clearpage
 

\begin{table*}[]
	\centering
	\caption{Question format on weather dataset.  R: contextual retrieval, C: comparison, S: summary. The value of $x$ falls within the minimum and maximum ranges of its respective attribute. $date_1$ represents the day at the twentieth percentile of the stock history, while $date_2$ represents the day at the eightieth percentile.}
	\begin{tabular}{c|c}
		\toprule
		\textbf{Ability}    & \textbf{Question Format} \\ \midrule
		\textit{\textbf{R}} &  \begin{tabular}[c]{@{}l@{}}
			$Q_0$: On which date did the temperature at two meters first reach $x$°C?\\
			$Q_1$: On which date did the relative humidity at two meters first exceed $x$\%?\\
			$Q_2$: On which date did the dew point temperature at two meters first drop below $x$°C?\\
			$Q_3$: On which date did the precipitation first exceed $x$ mm?\\
			$Q_4$: On which date did the sea-level air pressure first exceed $x$ hPa?\\
			$Q_5$: On which date did the cloud cover first reach $x$\%?\\
			$Q_6$: On which date did the wind speed at 10 meters first exceed $x$ m/s?
		\end{tabular} \\ \midrule
		\textit{\textbf{C}} &  \begin{tabular}[c]{@{}l@{}}
			\begin{tabular}[c]{@{}l@{}}
				$Q_7$: On which date did the temperature at two meters reach its highest value \\ 
				when the relative humidity was below $x$\%? 
			\end{tabular} \\
			
			\begin{tabular}[c]{@{}l@{}}
				$Q_8$: On which date did the relative humidity at two meters reach its lowest value \\ 
				when the temperature at two meters was above $x^\circ$C?
			\end{tabular} \\
			
			\begin{tabular}[c]{@{}l@{}}
				$Q_9$: On which date did the difference between the temperature and dew point \\ 
				at two meters reach its maximum when the cloud cover was below $x$\%? 
			\end{tabular} \\
			
			\begin{tabular}[c]{@{}l@{}}
				$Q_{10}$: On which date did the precipitation reach its highest value \\ 
				when the temperature at two meters was below $x^\circ$C? 
			\end{tabular} \\
			
			\begin{tabular}[c]{@{}l@{}}
				$Q_{11}$: On which date did the cloud cover reach its lowest value \\ 
				when the wind speed at 10 meters exceeded $x$ m/s? 
			\end{tabular} \\
			
			\begin{tabular}[c]{@{}l@{}}
				$Q_{12}$: On which date did the wind speed at 10 meters reach its highest value \\ 
				when the precipitation exceeded $x$ mm? 
			\end{tabular} \\
			
			\begin{tabular}[c]{@{}l@{}}
				$Q_{13}$: On which date did the sea-level air pressure reach its highest value \\ 
				when the cloud cover was below $x$\%? 
			\end{tabular}
		\end{tabular} \\ \midrule
		\textit{\textbf{S}} &  \begin{tabular}[c]{@{}l@{}}
			\begin{tabular}[c]{@{}l@{}}
				$Q_{14}$: How many days had a temperature at two meters greater than $x^\circ$C \\from $date_1$ to $date_2$? 
			\end{tabular} \\
			
			\begin{tabular}[c]{@{}l@{}}
				$Q_{15}$: How many days had a relative humidity at two meters exceeding $x$\% \\from $date_1$ to $date_2$? 
			\end{tabular} \\
			
			\begin{tabular}[c]{@{}l@{}}
				$Q_{16}$: How many days had a precipitation greater than $x$ mm from $date_1$ \\to $date_2$? 
			\end{tabular} \\
			
			\begin{tabular}[c]{@{}l@{}}
				$Q_{17}$: What was the average temperature at two meters when the relative \\humidity exceeded $x$\%? 
			\end{tabular} \\
			
			\begin{tabular}[c]{@{}l@{}}
				$Q_{18}$: What was the average wind speed at 10 meters when the precipitation \\exceeded $x$ mm? 
			\end{tabular} \\
			
			\begin{tabular}[c]{@{}l@{}}
				$Q_{19}$: How many times did the temperature at two meters rise for three or more \\consecutive days? 
			\end{tabular} \\
			
			\begin{tabular}[c]{@{}l@{}}
				$Q_{20}$: How many times did the relative humidity at two meters drop for \\three or more consecutive days? 
			\end{tabular} \\
		
					
		\begin{tabular}[c]{@{}l@{}}
			$\cdots \cdots$
		\end{tabular} \\
	
		\end{tabular} \\ \bottomrule
	\end{tabular}
\label{appx:weather_question}
\end{table*}

\clearpage
\begin{figure*}[t]
	\centering	
	\vspace{-1em}
	\frame{
		\includegraphics[width = 0.9\textwidth]{image/intro_example/number_compare.png}
	}
	%	\captionsetup{labelformat=empty}
	%	\addtocounter{figure}{-1}
	\caption{Number comparisons on GPT-4o. The correct answer is -9.11. }
	\label{fig:number_compare}
\end{figure*}

\begin{figure*}[t]
	\centering	
	\vspace{-1em}
	\frame{
		\includegraphics[width = 0.9\textwidth]{image/intro_example/multiplication.png}
	}
	%	\captionsetup{labelformat=empty}
	%	\addtocounter{figure}{-1}
	\caption{Number multiplication on GPT-4o. The correct answer is 102244.12. }
	\label{fig:multiplication}
\end{figure*}
\begin{figure*}[t]
	\centering	
	\vspace{-1em}
	\frame{
		\includegraphics[width = 0.9\textwidth]{image/intro_example/number_counting.jpg}
	}
	%	\captionsetup{labelformat=empty}
	%	\addtocounter{figure}{-1}
	\caption{Number counting on GPT-4o, which is required directly give answer. The correct answer is 4. }
	\label{fig:number_couting}
\end{figure*}

\clearpage


%\subsection{Additional Experiment Results}
%\subsubsection{Additional results on context length evaluation for stock and weather data}\label{appx:sssec:length}
		\begin{figure*}[t]
		\centering 	
		\subfloat[Contextual Retrieval]	
		{\centering\includegraphics[width=0.33\linewidth]{image/main_fig/retrieval-stock.pdf}}
		\hfill
		\subfloat[Comparison]
		{\centering\includegraphics[width=0.33\linewidth]{image/main_fig/compare-stock.pdf}}
		\subfloat[Summary]	
		{\centering\includegraphics[width=0.33\linewidth]{image/main_fig/summary-stock.pdf}}
		\hfill
		%	\subfloat[MUTAG]
		%	{\centering\includegraphics[width=0.25\linewidth, height=3.05cm]{image/g1-4.pdf}}	
		%	\hfill
		%	
		\caption{Evaluation on short and long context on stock dataset. Due to the input sequence length limit of Qwen2.5-72B-Inst on the API platform, the long dataset of all three abilities cannot be evaluated using this model.}
		\label{fig:length_stock}
	\end{figure*}
	
	
	
	\begin{figure*}[t]
		\centering 	
		\subfloat[Contextual Retrieval]	
		{\centering\includegraphics[width=0.33\linewidth]{image/main_fig/retrieval-weather.pdf}}
		\hfill
		\subfloat[Comparison]
		{\centering\includegraphics[width=0.33\linewidth]{image/main_fig/compare-weather.pdf}}
		\subfloat[Summary]	
		{\centering\includegraphics[width=0.33\linewidth]{image/main_fig/summary-weather.pdf}}
		\hfill
		%	\subfloat[MUTAG]
		%	{\centering\includegraphics[width=0.25\linewidth, height=3.05cm]{image/g1-4.pdf}}	
		%	\hfill
		%	
		\caption{Evaluation on short and long context on weather dataset. Due to the input sequence length limit of Qwen2.5-72B-Inst on the API platform, the long dataset of all three abilities cannot be evaluated using this model.}
		\label{fig:length_weather}
		
	\end{figure*}

		\begin{figure*}[t]
		
		\centering 	
		\subfloat[Contextual  Retrieval]	
		{\centering\includegraphics[width=0.33\linewidth]{image/noisy_dataset_fig/retrieval-noisy-weather.pdf}}
		\hfill
		\subfloat[Comparison]
		{\centering\includegraphics[width=0.33\linewidth]{image/noisy_dataset_fig/compare-noisy-weather.pdf}}
		\subfloat[Summary]	
		{\centering\includegraphics[width=0.33\linewidth]{image/noisy_dataset_fig/summary-noisy-weather.pdf}}
		\hfill
		%	\subfloat[MUTAG]
		%	{\centering\includegraphics[width=0.25\linewidth, height=3.05cm]{image/g1-4.pdf}}	
		%	\hfill
		%	
		\caption{Evaluation on  noisy weather dataset. Due to the input sequence length limit of Qwen2.5-72B-Inst on the API platform, the data containing 4 and 6 irrelevant attributes cannot be evaluated using this model.}
		\label{fig:noisy_weather}
		
	\end{figure*}
\clearpage
	
	
	
	
%	\begin{figure*}[t]
%		
%		\centering 	
%		\subfloat[Context Retrieval]	
%		{\centering\includegraphics[width=0.33\linewidth]{image/multi_dataset_fig/retrieval-multi-stock.pdf}}
%		\hfill
%		\subfloat[Comparison]
%		{\centering\includegraphics[width=0.33\linewidth]{image/multi_dataset_fig/compare-multi-stock.pdf}}
%		\subfloat[Summary]	
%		{\centering\includegraphics[width=0.33\linewidth]{image/multi_dataset_fig/summary-multi-stock.pdf}}
%		\hfill
%		%	\subfloat[MUTAG]
%		%	{\centering\includegraphics[width=0.25\linewidth, height=3.05cm]{image/g1-4.pdf}}	
%		%	\hfill
%		%	
%		\caption{Evaluation on multi-turn QA on stock dataset. Due to the input sequence length limit of Qwen2.5-72B-Inst on the API platform, the model cannot generate outputs in the third turn of the conversation. }
%		\label{fig:multurn_stock}
%		
%	\end{figure*}
	
	\begin{table*}[]
		\caption{The average token number on short and long instances for each data.}
		\centering
		\begin{tabular}{c|c|cc|cc}
			\toprule
			\multirow{2}{*}{\textbf{Dataset}}                                               & \multirow{2}{*}{\textbf{Ability}} & \multicolumn{2}{c|}{\textbf{Short}}                            & \multicolumn{2}{c}{\textbf{Long}}                             \\ \cmidrule{3-6} 
			
			&                                   & \multicolumn{1}{c|}{\textbf{\# Instance}} & \textbf{Avg Token} & \multicolumn{1}{c|}{\textbf{\# Instance}} & \textbf{Avg Token} \\ \midrule
			
			\multirow{3}{*}{\textbf{\begin{tabular}[c]{@{}c@{}}Number\\ List\end{tabular}}} & \textit{Contextual Retrieval}

                  & \multicolumn{1}{c|}{500}                     &        809.12     & \multicolumn{1}{c|}{500}                     &         6599.34      \\   
			
			& \textit{Comparison}                        & \multicolumn{1}{c|}{500}                     &     804.86     & \multicolumn{1}{c|}{500}                     &        6566.27      \\ 
			
			& \textit{Summary}



                  & \multicolumn{1}{c|}{500}                     &       822.49      & \multicolumn{1}{c|}{500}                     &       6487.07       \\ \midrule
			
			\multirow{3}{*}{\textbf{Stock}}                                                 & \textit{Contextual Retrieval}

                  & \multicolumn{1}{c|}{500}                     &        18529.07      & \multicolumn{1}{c|}{500}                     &      36641.63     \\  
			& \textit{Comparison}                        & \multicolumn{1}{c|}{500}                     &    18539.58     & \multicolumn{1}{c|}{500}                     &      36651.22      \\ 
			& \textit{Summary}

                  & \multicolumn{1}{c|}{500}                     &      18504.51      & \multicolumn{1}{c|}{500}                     &       36618.07      \\ \midrule
			
			\multirow{3}{*}{\textbf{Weather}}                                               & \textit{Contextual Retrieval}

                  & \multicolumn{1}{c|}{500}                     &        18362.38        & \multicolumn{1}{c|}{500}                     &        36356.13    \\  
			& \textit{Comparison}                        & \multicolumn{1}{c|}{500}                     &        18371.11    & \multicolumn{1}{c|}{500}                     &       36365.27     \\ 
			& \textit{Summary}



                  & \multicolumn{1}{c|}{500}                     &        18334.15     & \multicolumn{1}{c|}{500}                     &        36328.27    \\ \bottomrule
		\end{tabular}
		\label{tab:data_stat_short_long}
	\end{table*}
	\clearpage
%	\begin{figure*}[t]
%		
%		\centering 	
%		\subfloat[Context Retrieval]	
%		{\centering\includegraphics[width=0.33\linewidth]{image/multi_dataset_fig/retrieval-multi-weather.pdf}}
%		\hfill
%		\subfloat[Comparison]
%		{\centering\includegraphics[width=0.33\linewidth]{image/multi_dataset_fig/compare-multi-weather.pdf}}
%		\subfloat[Summary]	
%		{\centering\includegraphics[width=0.33\linewidth]{image/multi_dataset_fig/summary-multi-weather.pdf}}
%		\hfill
%		%	\subfloat[MUTAG]
%		%	{\centering\includegraphics[width=0.25\linewidth, height=3.05cm]{image/g1-4.pdf}}	
%		%	\hfill
%		%	
%		\caption{Evaluation on multi-turn QA on weather dataset.}
%		\label{fig:multurn_weather}
%		
%	\end{figure*}
	

	
	
%\noindent\textbf{Case Study}
	\newtcolorbox[auto counter, number within=section]{examplebox}[2][]{%
		colback=white,        % Background color
		colframe=exampleborder, % Border color
		coltitle=white,       % Header text color
		fonttitle=\bfseries,  % Header font
		colbacktitle=exampleblue, % Header background color
		title=Example~\thetcbcounter: #2, % Title format
	}
	\begin{figure*}
		\begin{examplebox}{InternLM2-Math-7B}
			\textbf{Failure:} \\Model repeatedly output the same number until the output sequence ends.
			
			\vspace{0.2cm}
			\textbf{User:} \\ Question: What is the maximum sum of any three consecutive items in the list? Options: A: 21501.9737231253, B: 21502.8144848529, C: 21501.9642201794, D: 21502.7354388437, E: 21502.0327435643, F: 21501.7526575621, G: 21501.7057960155, H: 21502 Data: [-5279, ....(100 numbers)]
			
			\vspace{0.3cm}
			\textbf{InternLM2-Math-7B Model:} \\
			Step 1: To find the maximum sum of any three consecutive items in the list, we need to identify the three largest numbers in the list and add them together. <PRM +> \\Step 2: We can start by sorting the list in ascending order to make it easier to identify the largest numbers. <PRM +> \\Step 3: After sorting, the list becomes: [-8509, -8509, -8509, -8509, -8509, -8509, -8509, -8509, -8509, -8509, -8509, -8509, -8509, -8509, -8509, -8509, -8509, -8509, -8509, -8509, -8509, -8509, -8509, -8509, -8509, -8509, -8509, -8509...(output too long)
		\end{examplebox}
		\caption{Failure type and conversation record of InternLM2-Math-7B. The model constantly generate the same "-8509" text until the output sequence was too long and truncated, leaving the problem unsolved.}
		\label{fig:fail_internlm}
	\end{figure*}
	
	\begin{figure*}
		\begin{examplebox}{DeepSeek-Math-7B}
				\textbf{Failure:} \\ Model outputs sequence without meaning, probably due to overly long input sequences. 
				
				\vspace{0.2cm}
				\textbf{User:} \\Question: What is the sum of the index of top 300 smallest numbers in the list? Options: A: 146734, B: 146738, C: 146733, D: 146724, E: 146728, F: 146727, G: 146721, H: 146729 Data: [3640, 1560, ...(100 numbers)]
				
				\vspace{0.3cm}
				\textbf{Deepseek-Math-7B Model:} \\
				31,3,9,3,1,1,3,3,3,3,3,3,3,3, 3,3,1,3,3,3,1,3,3,3, 3, 3,3,3, 1, 1,3,3, 3,1,3, 3,3,3, 1, 3,3,3, 3,3,1,3,1,3,91, 3,3,3, 3,31,3,3,3, 3, 339, 3, 3, 311, 1, 3, 1,3,3, 3,3,6,3,3, 3,31,3,3,1, 1, 1, 1, 3,1,1, 3,3, 1,1, 3,1,3,1, 1, 1, 1, 1, 19,1, 1, 1,191,1, 1,1, 19, 31,1,1, 1, 3,1, 1,3, 1133,3, 1, 1,1, 11, 11,3, 31, 3, 3,1,1,3,1, 1,33, 11,3,1911,3, 1,13,3131311119191,1, 3,1,11,1,19,11,1,1,1....(keeps outputting numbers)
		\end{examplebox}
		\caption{Failure type and conversation record of DeepSeek-Math-7B. The model only generates a series of numbers throughout its output, and we're unable to acquire any useful information from the output.}
		\label{fig:fail_ds_math}
	\end{figure*}	

	\begin{figure*}
		\begin{examplebox}{Metamath-Llemma-7B}
			\textbf{Failure:}\\ Model outputs ineffective chain of thoughts, and giving the wrong answer in the wrong format(ground truth is B))
			
			\vspace{0.2cm}
			\textbf{User:}\\ Question: What is the sum of the index of top 30 smallest numbers in the list? Options: A: 1566, B: 1572, C: 1575, D: 1573, E: 1577, F: 1564, G: 1569, H: 1567 Data: [-2568.1542995196, 1035.3645340299...(100 numbers)]
			
			\vspace{0.3cm}
			\textbf{Metamath-Llemma-7B Model:} \\ The data you receive is a list of numbers. To find the sum of the index of the top 30 smallest numbers, we need to find the 30 smallest numbers in the list and then calculate their sum. To find the 30 smallest numbers, we can use the min function in Python. The min function takes a list as input and returns the smallest number in the list. We can use the min function to find the 30 smallest numbers in the list. Now, we need to calculate the sum of these 30 smallest numbers. To do this, we can use the sum function in Python. The sum function takes a list as input and returns the sum of all the numbers in the list. We can use the sum function to calculate the sum of the 30 smallest numbers. Finally, we need to find the index of the sum. To do this, we can use the index function in Python. The index function takes a number as input and returns the index of that number in the list. We can use the index function to find the index of the sum. Therefore, the answer is (correct option). \#\#\#\# 1566 The answer is: 1566
		\end{examplebox}
		\caption{Failure type and conversation record of Metamath-Llemma-7B. The model attempts to use programming language to solve the problem, but it does not provide executable lines of code, only providing vague description of certain functions instead. After outputting the vague solution, the model outputs an answer, which is different from the ground truth and not in the expected format(We expect the model to provide the answer with a capital letter corresponding to the correct option).}
		\label{fig:fail_llemma}
	\end{figure*}

	\begin{figure*}
	\begin{examplebox}{MAmmoTH-13B}
		\textbf{Failure:}\\ Model outputs ineffective chain of thoughts, and giving the wrong answer in the wrong format(ground truth is B))
		
		\vspace{0.2cm}
		\textbf{User:}\\ Question: What is the sum of the index of top 30 smallest numbers in the list? Options: A: 1566, B: 1572, C: 1575, D: 1573, E: 1577, F: 1564, G: 1569, H: 1567 Data: [-2568.1542995196, 1035.3645340299...(100 numbers)]
		
		\vspace{0.3cm}
		\textbf{MAmmoTH-13B Model:}\\  Notice that there are two numbers in your list that have the smallest index: 4346.5174909575 and 8169.6394989672. They have the smallest index of 4346 and 8169 respectively. Since in the options the first number is the index of the smallest number, then the correct option is D.\\             Option: D
	\end{examplebox}
	\caption{Failure type and conversation record of Mammoth-13B. The model fail to understand the problem's instruction, and does not provide reasonable output. Moreover, it provides the wrong answer in incorrect format, just like Metamath-Llemma-7B(We expect the model to provide the answer with a capital letter corresponding to the correct option).}
	\label{fig:fail_mammoth}
	\end{figure*}
\end{document}

 