\section{NumericBench Generation}
In this section, we present our created  NumericBench, which is specifically designed to evaluate fundamental numerical capabilities of LLMs. 
NumericBench consists of diverse datasets and tasks, 
enabling a systematic and comprehensive evaluation.
We discuss the datasets included in NumericBench, the key abilities it evaluates, and the methodology for benchmark generation.

\begin{table*}[t]
	\caption{NumericBench statistics. R: contextual retrieval, C: comparison, S: summary, L: logical reasoning. The token count is calculated based on tiktoken, which is the tokenizer used by Llama3~\cite{grattafiori2024llama3herdmodels}. The sentences used for token calculation include both the context and the question.}
	\centering
	\renewcommand{\arraystretch}{1.15} % 设置行间距为默认的 1.15 倍
	\setlength{\tabcolsep}{1.5pt} % 将列间距设置为 1pt
\resizebox{\textwidth}{!}{
	\begin{tabular}{c|c|c|c|c}
		\toprule
		\textbf{Data} & \textbf{Format} & \textbf{Questions} & \textbf{\# Instance} & \textbf{Avg Token} \\ \midrule
		
		\multirow{3}{*}{} 
		& \multirow{3}{*}{} 
		& \begin{tabular}[c]{@{}c@{}}R: What is the index of the first occurrence\\ of the number -3095 in the list?\end{tabular} 
		& 1000 & 3704.23 \\ \cline{3-5}
		
		\textbf{\begin{tabular}[c]{@{}c@{}}Number\\ List\end{tabular}}
		& $[69, -1, 6.1, \ldots, 5.7]$
		& \begin{tabular}[c]{@{}c@{}}C: Which index holds the smallest number\\
			 in the list between the indices 20 and 80?\end{tabular} 
		& 1000 & 3685.57  \\ \cline{3-5}
		
		& & \begin{tabular}[c]{@{}c@{}}S: What is the average of the index of\\
			 top 30 largest numbers in the list?\end{tabular} 
		& 1000 & 3654.78 \\ \midrule
		
		\multirow{3}{*}{} 
		& \multirow{3}{*}{
		\begin{tabular}[c]{@{}c@{}}
			\{date: 2024-06-19,\\
			close\_price: 9.79, \\
			open\_price: 9.4, \\
			\ldots \\
			PE\_ratio: 4.5416\}
		\end{tabular}
		} 
		& \begin{tabular}[c]{@{}c@{}}
			R: On which date did the close price\\
			 of stock firstly reach 61.76 yuan?
		\end{tabular}
		& 1000 & 27585.35 \\ \cline{3-5}
		
		\textbf{Stock}
		& 
		& \begin{tabular}[c]{@{}c@{}}
			C: Among the top-45 trading value days, which\\
			 date did the stock have the lowest close price?
		\end{tabular}
		 & 1000 & 27595.40 \\ \cline{3-5}
		
		& & \begin{tabular}[c]{@{}c@{}} 
			S: How many days had the close price higher than\\
			 the open price from 2024-07-31 to 2024-12-13?
		\end{tabular}	
		& 1000 & 27561.29 \\ \midrule
		
		\multirow{3}{*}{} 
		& \multirow{3}{*}{
		\begin{tabular}[c]{@{}c@{}}
			\{date: 2024-07-21,\\
			pressure\_msl: 999.96,\\
			dew\_point\_2m: 26.25,\\
			\ldots \\
			cloud\_cover: 61.5\}
		\end{tabular}
		} 
		& \begin{tabular}[c]{@{}c@{}} 
			R: On which date did the dew point temperature\\
			 at two meters firstly drop below 9.15°C?
		\end{tabular}
		& 1000 & 27359.26 \\ \cline{3-5}
		
		\textbf{Weather}
		& & \begin{tabular}[c]{@{}c@{}} 
			C: On which date did the MSL pressure reach its\\
			highest value when the cloud cover was below 9\%?
		\end{tabular}
		& 1000 & 27368.19 \\ \cline{3-5}
		
		& & \begin{tabular}[c]{@{}c@{}} 
			S: What was the average temperature at two meters\\
			when the relative humidity exceeded 78.56\%?
		\end{tabular}
		& 1000 & 27331.21 \\ \midrule
		
		\textbf{Sequence} 
		& $[0.34, 3, 6, \ldots, 111]$ 
		& L: What is the next number in the sequence? & 500 & 677.57 \\ \midrule
		
		\textbf{\begin{tabular}[c]{@{}c@{}}Arithmetic \\Operation\end{tabular}} 
		& \begin{tabular}[c]{@{}c@{}} 
		$a: 6.755,
		b: -1.225$
		\end{tabular}
		& \begin{tabular}[c]{@{}c@{}} 
		 $Q_{oper}$: What is the result of $a + b$?\\
		 $Q_{context}$: What is the result of $a $ plus $b$?
		 
		\end{tabular}
		& 12000 & 112.00 \\ \midrule
		
		\textbf{\begin{tabular}[c]{@{}c@{}}Mixed-number-string\\ Sequence\end{tabular}} 
		& \begin{tabular}[c]{@{}c@{}} 
		$effV2\ldots x98o7Lo$
		\end{tabular}
		& \begin{tabular}[c]{@{}c@{}} 
		How many numbers are there in the string? Note\\
		that a sequence like 'a243b' counts as a single number.
		\end{tabular}
		& 2000 & 196.53 \\ \bottomrule

	\end{tabular}
}
	\label{tab:data_stat}
	
\end{table*}

 

\subsection{Numeric Dataset Collection}
NumericBench offers a diverse collection of numerical datasets and questions designed to reflect real-world scenarios. 
This variety ensures that LLMs are thoroughly tested on their fundamental  abilities on numerical data.

\noindent\textbf{Number List Dataset.}
The synthetic number list dataset consists of simple collections of numerical values (integer and floats) 
presented as ordered or unordered lists.
Numbers in lists are one of the simplest and most fundamental data representations encountered in real-world scenarios.
Despite their simplicity, retrieving, indexing,  comparison, and summary on numbers can verify the fundamental numerical ability of LLMs. 
This dataset serves as a fundamental dataset of how well LLMs understand numerical values as discrete entities.



\noindent\textbf{Stock Dataset.}
The time-series  stock dataset is crawled from Eastmoney website~\cite{eastmoney}, 
which has eighteen attributes, such as stock close prices, open price,  trading volumes, and price-earnings ratio, over time.
Stock  data reflects dynamic, real-world numerical reasoning challenges that involve trend analysis, comparison, and decision-making under uncertainty,  representing real-world financial workflows.
 




\noindent\textbf{Weather Dataset.}
The weather dataset is crawled from Open-Meteo python API~\citep{openmeteo}, which includes data related to weather metrics, such as temperature, precipitation, humidity, and wind speed. 
The data is presented across various longitude and latitude.
 
 




\noindent\textbf{Numeric Sequence  Dataset.}
The synthetic numeric sequence dataset comprises sequences of numbers generated by arithmetic or geometric progression, complex patterns, or noisy inputs. 
Tasks require identifying patterns, predicting the next number, or reasoning about relationships between numbers.
Numerical sequences test the logical reasoning capabilities of LLMs, requiring pattern recognition and multi-step reasoning. This dataset introduces structured challenges that are common in mathematics and algorithmic reasoning.


 
\noindent\textbf{Arithmetic Operation Dataset.}
The dataset comprises 12,000 pairs of simple numbers, each undergoing addition, subtraction, multiplication, and division operations. Each pair of numbers, $a$ and $b$, consists of $k$-digit integers with three decimal places, where $k \in \{1, 2, \cdots, 6\}$. 
For each value of $k$, there are 2,000 pairs, evenly distributed across the four basic operations (i.e, $+, -,  *, /$), with 500 pairs per operation. 
This dataset is to evaluate the fundamental mathematical operation capabilities of LLMs, simulating the majority of mathematical calculation requirements in real-world scenarios.

\noindent\textbf{Mixed-number-string Sequence Dataset.}
The dataset consists of alphanumeric strings of varying lengths $\{50, 100, 150, 200\}$, each containing a randomized mix of letters and digits. For each string length, 500 samples are generated, resulting in a total of 2,000 samples. Each sample includes a query asking for the count of contiguous numeric sequences within the string, where sequences like "a243b" count as a single number. This dataset is designed to assess the ability of LLMs to identify and count numeric sequences.
 







\subsection{Fundamental Numerical Ability}
NumericBench is designed to comprehensively evaluate six fundamental numerical reasoning abilities of LLMs, which is 
%These three fundamental abilities are 
essential for solving real-world numeric-related tasks.
%such as numeric data summary and financial price analysis.


\noindent\textbf{Contextual Retrieval Ability.}
Contextual retrieval ability evaluates how well LLMs can locate, extract, and identify specific numerical values or their positions within structured or unstructured data. 
This includes tasks like finding a specific number in a list, retrieving values , and indexing numbers based on their order.
For example, as shown in Table~\ref{tab:data_stat}, it evaluates LLMs on tasks such as retrieving stock prices and identifying key values within numerical lists or domain-specific data (e.g., stock market and weather-related information).
This ability is fundamental to numerical reasoning because it forms the foundation for higher-order tasks, such as comparison, aggregation, and logical reasoning. 
 
 



\noindent\textbf{Comparison Ability.}
Comparison ability evaluates how well LLMs can compare numerical values to determine relationships such as greater than, less than, or equal to, and identify trends or differences in datasets. 
Comparison is vital for logical reasoning and decision-making, as many real-world tasks depend on accurate numerical evaluation. 
For instance,  as shown in Table~\ref{tab:data_stat},   comparing prices is essential in stock  for assessing performance, while weather forecasting requires analysis of temperature or precipitation trends over time. 
 



\noindent\textbf{Summary Ability.}
Summary ability assesses the LLM’s capacity to aggregate numerical data into concise insights, such as calculating totals, averages, or other statistical metrics. 
Summarization is critical for condensing large datasets into actionable information, enabling decision-making based on aggregated insights rather than raw data. 
This ability is indispensable in domains like electricity usage analysis, where summarizing hourly or daily consumption helps forecast bills, in business reporting for aggregating sales and revenue data to evaluate performance, 
and in healthcare analytics to monitor trends in patient metrics over time.



\noindent\textbf{Logic Reasoning Ability.}
Logical Reasoning Ability measures the LLM’s ability to perform multi-step operations involving numerical data, 
such as recognizing patterns, inferring rules, and applying arithmetic or geometric reasoning to solve complex problems. Logical reasoning extends beyond simple numerical tasks and reflects the LLM’s capacity for deeper, structured thinking. 
This ability is crucial for algorithm design, where solving problems involving numeric sequences or patterns is essential, in scientific research for identifying relationships and correlations in data.

\noindent\textbf{Arithmetic Operation Ability.}
It reflects the LLM's capacity to perform mathematical calculations accurately. Such ability is essential for tasks involving numerical computations, such as  automated machine learning through LLMs.





\noindent\textbf{Number Recognition  Ability.}
This measures the LLM's proficiency in identifying and interpreting numerical information within a given context. It represents a fundamental requirement for handling numeric-based tasks effectively.




\subsection{NumericBench Generation}
We use the number list, stock, and weather datasets to evaluate the contextual retrieval, comparison, and summary abilities of LLMs. 
Specifically, for each ability and each dataset, we prepare a set of questions designed to assess the corresponding target ability.
As shown in Table~\ref{appx:number_question}, Table~\ref{appx:stock_question}, and Table~\ref{appx:weather_question} in Appendix, there are nine question sets in total, covering three abilities across three datasets. 
When evaluating a specific ability (e.g., contextual retrieval) on a specific dataset (e.g., stock data), we randomly select one question from the corresponding question set for each data instance (e.g., a stock instance) 
and manually label the answer. This approach enables us to generate question-answer pairs for each ability on the number list, stock, and weather datasets.

For arithmetic operations and number counting in the strings dataset, the question format is straightforward, as illustrated in Table~\ref{tab:data_stat}. These questions are designed to evaluate the basic arithmetic operation and number recognition abilities of LLMs.


