\documentclass[letterpaper, 10 pt, journal, twoside]{IEEEtran}

\usepackage{graphicx}
\usepackage{amsmath}
\usepackage{amssymb}
\usepackage{amsfonts}
\usepackage{amsthm}
\usepackage[most]{tcolorbox}
\usepackage{xcolor}
\usepackage{booktabs, multirow} 
\usepackage{soul}
\usepackage{xcolor,colortbl}
\usepackage{changepage,threeparttable} 
\usepackage{hyperref}
\usepackage[noadjust]{cite}

\usepackage{tikz}
\usepackage[linesnumbered,ruled,vlined]{algorithm2e}

\usepackage{txfonts}
\usepackage{newtxtext,newtxmath}
\usepackage{newtxtt}
\usepackage{wrapfig}

\usetikzlibrary{tikzmark}


%
\setlength\unitlength{1mm}
\newcommand{\twodots}{\mathinner {\ldotp \ldotp}}
% bb font symbols
\newcommand{\Rho}{\mathrm{P}}
\newcommand{\Tau}{\mathrm{T}}

\newfont{\bbb}{msbm10 scaled 700}
\newcommand{\CCC}{\mbox{\bbb C}}

\newfont{\bb}{msbm10 scaled 1100}
\newcommand{\CC}{\mbox{\bb C}}
\newcommand{\PP}{\mbox{\bb P}}
\newcommand{\RR}{\mbox{\bb R}}
\newcommand{\QQ}{\mbox{\bb Q}}
\newcommand{\ZZ}{\mbox{\bb Z}}
\newcommand{\FF}{\mbox{\bb F}}
\newcommand{\GG}{\mbox{\bb G}}
\newcommand{\EE}{\mbox{\bb E}}
\newcommand{\NN}{\mbox{\bb N}}
\newcommand{\KK}{\mbox{\bb K}}
\newcommand{\HH}{\mbox{\bb H}}
\newcommand{\SSS}{\mbox{\bb S}}
\newcommand{\UU}{\mbox{\bb U}}
\newcommand{\VV}{\mbox{\bb V}}


\newcommand{\yy}{\mathbbm{y}}
\newcommand{\xx}{\mathbbm{x}}
\newcommand{\zz}{\mathbbm{z}}
\newcommand{\sss}{\mathbbm{s}}
\newcommand{\rr}{\mathbbm{r}}
\newcommand{\pp}{\mathbbm{p}}
\newcommand{\qq}{\mathbbm{q}}
\newcommand{\ww}{\mathbbm{w}}
\newcommand{\hh}{\mathbbm{h}}
\newcommand{\vvv}{\mathbbm{v}}

% Vectors

\newcommand{\av}{{\bf a}}
\newcommand{\bv}{{\bf b}}
\newcommand{\cv}{{\bf c}}
\newcommand{\dv}{{\bf d}}
\newcommand{\ev}{{\bf e}}
\newcommand{\fv}{{\bf f}}
\newcommand{\gv}{{\bf g}}
\newcommand{\hv}{{\bf h}}
\newcommand{\iv}{{\bf i}}
\newcommand{\jv}{{\bf j}}
\newcommand{\kv}{{\bf k}}
\newcommand{\lv}{{\bf l}}
\newcommand{\mv}{{\bf m}}
\newcommand{\nv}{{\bf n}}
\newcommand{\ov}{{\bf o}}
\newcommand{\pv}{{\bf p}}
\newcommand{\qv}{{\bf q}}
\newcommand{\rv}{{\bf r}}
\newcommand{\sv}{{\bf s}}
\newcommand{\tv}{{\bf t}}
\newcommand{\uv}{{\bf u}}
\newcommand{\wv}{{\bf w}}
\newcommand{\vv}{{\bf v}}
\newcommand{\xv}{{\bf x}}
\newcommand{\yv}{{\bf y}}
\newcommand{\zv}{{\bf z}}
\newcommand{\zerov}{{\bf 0}}
\newcommand{\onev}{{\bf 1}}

% Matrices

\newcommand{\Am}{{\bf A}}
\newcommand{\Bm}{{\bf B}}
\newcommand{\Cm}{{\bf C}}
\newcommand{\Dm}{{\bf D}}
\newcommand{\Em}{{\bf E}}
\newcommand{\Fm}{{\bf F}}
\newcommand{\Gm}{{\bf G}}
\newcommand{\Hm}{{\bf H}}
\newcommand{\Id}{{\bf I}}
\newcommand{\Jm}{{\bf J}}
\newcommand{\Km}{{\bf K}}
\newcommand{\Lm}{{\bf L}}
\newcommand{\Mm}{{\bf M}}
\newcommand{\Nm}{{\bf N}}
\newcommand{\Om}{{\bf O}}
\newcommand{\Pm}{{\bf P}}
\newcommand{\Qm}{{\bf Q}}
\newcommand{\Rm}{{\bf R}}
\newcommand{\Sm}{{\bf S}}
\newcommand{\Tm}{{\bf T}}
\newcommand{\Um}{{\bf U}}
\newcommand{\Wm}{{\bf W}}
\newcommand{\Vm}{{\bf V}}
\newcommand{\Xm}{{\bf X}}
\newcommand{\Ym}{{\bf Y}}
\newcommand{\Zm}{{\bf Z}}

% Calligraphic

\newcommand{\Ac}{{\cal A}}
\newcommand{\Bc}{{\cal B}}
\newcommand{\Cc}{{\cal C}}
\newcommand{\Dc}{{\cal D}}
\newcommand{\Ec}{{\cal E}}
\newcommand{\Fc}{{\cal F}}
\newcommand{\Gc}{{\cal G}}
\newcommand{\Hc}{{\cal H}}
\newcommand{\Ic}{{\cal I}}
\newcommand{\Jc}{{\cal J}}
\newcommand{\Kc}{{\cal K}}
\newcommand{\Lc}{{\cal L}}
\newcommand{\Mc}{{\cal M}}
\newcommand{\Nc}{{\cal N}}
\newcommand{\nc}{{\cal n}}
\newcommand{\Oc}{{\cal O}}
\newcommand{\Pc}{{\cal P}}
\newcommand{\Qc}{{\cal Q}}
\newcommand{\Rc}{{\cal R}}
\newcommand{\Sc}{{\cal S}}
\newcommand{\Tc}{{\cal T}}
\newcommand{\Uc}{{\cal U}}
\newcommand{\Wc}{{\cal W}}
\newcommand{\Vc}{{\cal V}}
\newcommand{\Xc}{{\cal X}}
\newcommand{\Yc}{{\cal Y}}
\newcommand{\Zc}{{\cal Z}}

% Bold greek letters

\newcommand{\alphav}{\hbox{\boldmath$\alpha$}}
\newcommand{\betav}{\hbox{\boldmath$\beta$}}
\newcommand{\gammav}{\hbox{\boldmath$\gamma$}}
\newcommand{\deltav}{\hbox{\boldmath$\delta$}}
\newcommand{\etav}{\hbox{\boldmath$\eta$}}
\newcommand{\lambdav}{\hbox{\boldmath$\lambda$}}
\newcommand{\epsilonv}{\hbox{\boldmath$\epsilon$}}
\newcommand{\nuv}{\hbox{\boldmath$\nu$}}
\newcommand{\muv}{\hbox{\boldmath$\mu$}}
\newcommand{\zetav}{\hbox{\boldmath$\zeta$}}
\newcommand{\phiv}{\hbox{\boldmath$\phi$}}
\newcommand{\psiv}{\hbox{\boldmath$\psi$}}
\newcommand{\thetav}{\hbox{\boldmath$\theta$}}
\newcommand{\tauv}{\hbox{\boldmath$\tau$}}
\newcommand{\omegav}{\hbox{\boldmath$\omega$}}
\newcommand{\xiv}{\hbox{\boldmath$\xi$}}
\newcommand{\sigmav}{\hbox{\boldmath$\sigma$}}
\newcommand{\piv}{\hbox{\boldmath$\pi$}}
\newcommand{\rhov}{\hbox{\boldmath$\rho$}}
\newcommand{\upsilonv}{\hbox{\boldmath$\upsilon$}}

\newcommand{\Gammam}{\hbox{\boldmath$\Gamma$}}
\newcommand{\Lambdam}{\hbox{\boldmath$\Lambda$}}
\newcommand{\Deltam}{\hbox{\boldmath$\Delta$}}
\newcommand{\Sigmam}{\hbox{\boldmath$\Sigma$}}
\newcommand{\Phim}{\hbox{\boldmath$\Phi$}}
\newcommand{\Pim}{\hbox{\boldmath$\Pi$}}
\newcommand{\Psim}{\hbox{\boldmath$\Psi$}}
\newcommand{\Thetam}{\hbox{\boldmath$\Theta$}}
\newcommand{\Omegam}{\hbox{\boldmath$\Omega$}}
\newcommand{\Xim}{\hbox{\boldmath$\Xi$}}


% Sans Serif small case

\newcommand{\Gsf}{{\sf G}}

\newcommand{\asf}{{\sf a}}
\newcommand{\bsf}{{\sf b}}
\newcommand{\csf}{{\sf c}}
\newcommand{\dsf}{{\sf d}}
\newcommand{\esf}{{\sf e}}
\newcommand{\fsf}{{\sf f}}
\newcommand{\gsf}{{\sf g}}
\newcommand{\hsf}{{\sf h}}
\newcommand{\isf}{{\sf i}}
\newcommand{\jsf}{{\sf j}}
\newcommand{\ksf}{{\sf k}}
\newcommand{\lsf}{{\sf l}}
\newcommand{\msf}{{\sf m}}
\newcommand{\nsf}{{\sf n}}
\newcommand{\osf}{{\sf o}}
\newcommand{\psf}{{\sf p}}
\newcommand{\qsf}{{\sf q}}
\newcommand{\rsf}{{\sf r}}
\newcommand{\ssf}{{\sf s}}
\newcommand{\tsf}{{\sf t}}
\newcommand{\usf}{{\sf u}}
\newcommand{\wsf}{{\sf w}}
\newcommand{\vsf}{{\sf v}}
\newcommand{\xsf}{{\sf x}}
\newcommand{\ysf}{{\sf y}}
\newcommand{\zsf}{{\sf z}}


% mixed symbols

\newcommand{\sinc}{{\hbox{sinc}}}
\newcommand{\diag}{{\hbox{diag}}}
\renewcommand{\det}{{\hbox{det}}}
\newcommand{\trace}{{\hbox{tr}}}
\newcommand{\sign}{{\hbox{sign}}}
\renewcommand{\arg}{{\hbox{arg}}}
\newcommand{\var}{{\hbox{var}}}
\newcommand{\cov}{{\hbox{cov}}}
\newcommand{\Ei}{{\rm E}_{\rm i}}
\renewcommand{\Re}{{\rm Re}}
\renewcommand{\Im}{{\rm Im}}
\newcommand{\eqdef}{\stackrel{\Delta}{=}}
\newcommand{\defines}{{\,\,\stackrel{\scriptscriptstyle \bigtriangleup}{=}\,\,}}
\newcommand{\<}{\left\langle}
\renewcommand{\>}{\right\rangle}
\newcommand{\herm}{{\sf H}}
\newcommand{\trasp}{{\sf T}}
\newcommand{\transp}{{\sf T}}
\renewcommand{\vec}{{\rm vec}}
\newcommand{\Psf}{{\sf P}}
\newcommand{\SINR}{{\sf SINR}}
\newcommand{\SNR}{{\sf SNR}}
\newcommand{\MMSE}{{\sf MMSE}}
\newcommand{\REF}{{\RED [REF]}}

% Markov chain
\usepackage{stmaryrd} % for \mkv 
\newcommand{\mkv}{-\!\!\!\!\minuso\!\!\!\!-}

% Colors

\newcommand{\RED}{\color[rgb]{1.00,0.10,0.10}}
\newcommand{\BLUE}{\color[rgb]{0,0,0.90}}
\newcommand{\GREEN}{\color[rgb]{0,0.80,0.20}}

%%%%%%%%%%%%%%%%%%%%%%%%%%%%%%%%%%%%%%%%%%
\usepackage{hyperref}
\hypersetup{
    bookmarks=true,         % show bookmarks bar?
    unicode=false,          % non-Latin characters in AcrobatÕs bookmarks
    pdftoolbar=true,        % show AcrobatÕs toolbar?
    pdfmenubar=true,        % show AcrobatÕs menu?
    pdffitwindow=false,     % window fit to page when opened
    pdfstartview={FitH},    % fits the width of the page to the window
%    pdftitle={My title},    % title
%    pdfauthor={Author},     % author
%    pdfsubject={Subject},   % subject of the document
%    pdfcreator={Creator},   % creator of the document
%    pdfproducer={Producer}, % producer of the document
%    pdfkeywords={keyword1} {key2} {key3}, % list of keywords
    pdfnewwindow=true,      % links in new window
    colorlinks=true,       % false: boxed links; true: colored links
    linkcolor=red,          % color of internal links (change box color with linkbordercolor)
    citecolor=green,        % color of links to bibliography
    filecolor=blue,      % color of file links
    urlcolor=blue           % color of external links
}
%%%%%%%%%%%%%%%%%%%%%%%%%%%%%%%%%%%%%%%%%%%



\usepackage{balance}

\title{\LARGE \bf
FLoRA: A \underline{F}ramework for \underline{L}earning Sc\underline{o}ring \underline{R}ules in \underline{A}utonomous Driving Planning Systems}


\author{
    Zikang Xiong, Joe Eappen, and Suresh Jagannathan % <-this % stops a space 
\thanks{ Authors are with the Computer Science Department, Purdue University. 
{\tt\small \{xiong84,jeappen,suresh\}@cs.purdue.edu}}
}

\begin{document}
\maketitle

\begin{abstract}
    In autonomous driving systems, motion planning is commonly implemented as a two-stage process: first, a trajectory proposer generates multiple candidate trajectories, then a scoring mechanism selects the most suitable trajectory for execution. For this critical selection stage, rule-based scoring mechanisms are particularly appealing as they can explicitly encode driving preferences, safety constraints, and traffic regulations in a formalized, human-understandable format. However, manually crafting these scoring rules presents significant challenges: the rules often contain complex interdependencies, require careful parameter tuning, and may not fully capture the nuances present in real-world driving data.
    This work introduces FLoRA, a novel framework that bridges this gap by learning interpretable scoring rules represented in temporal logic. Our method features a learnable logic structure that captures nuanced relationships across diverse driving scenarios, optimizing both rules and parameters directly from real-world driving demonstrations collected in NuPlan. Our approach effectively learns to evaluate driving behavior even though the training data only contains positive examples (successful driving demonstrations). Evaluations in closed-loop planning simulations demonstrate that our learned scoring rules outperform existing techniques, including expert designed rules and neural network scoring models, while maintaining interpretability.
    This work introduces a data-driven approach to enhance the scoring mechanism in autonomous driving systems, designed as a plug-in module to seamlessly integrate with various trajectory proposers.  Our video and code are available on \href{https://xiong.zikang.me/FLoRA/}{xiong.zikang.me/FLoRA/}.
\end{abstract}

\section{Introduction}
Backdoor attacks pose a concealed yet profound security risk to machine learning (ML) models, for which the adversaries can inject a stealth backdoor into the model during training, enabling them to illicitly control the model's output upon encountering predefined inputs. These attacks can even occur without the knowledge of developers or end-users, thereby undermining the trust in ML systems. As ML becomes more deeply embedded in critical sectors like finance, healthcare, and autonomous driving \citep{he2016deep, liu2020computing, tournier2019mrtrix3, adjabi2020past}, the potential damage from backdoor attacks grows, underscoring the emergency for developing robust defense mechanisms against backdoor attacks.

To address the threat of backdoor attacks, researchers have developed a variety of strategies \cite{liu2018fine,wu2021adversarial,wang2019neural,zeng2022adversarial,zhu2023neural,Zhu_2023_ICCV, wei2024shared,wei2024d3}, aimed at purifying backdoors within victim models. These methods are designed to integrate with current deployment workflows seamlessly and have demonstrated significant success in mitigating the effects of backdoor triggers \cite{wubackdoorbench, wu2023defenses, wu2024backdoorbench,dunnett2024countering}.  However, most state-of-the-art (SOTA) backdoor purification methods operate under the assumption that a small clean dataset, often referred to as \textbf{auxiliary dataset}, is available for purification. Such an assumption poses practical challenges, especially in scenarios where data is scarce. To tackle this challenge, efforts have been made to reduce the size of the required auxiliary dataset~\cite{chai2022oneshot,li2023reconstructive, Zhu_2023_ICCV} and even explore dataset-free purification techniques~\cite{zheng2022data,hong2023revisiting,lin2024fusing}. Although these approaches offer some improvements, recent evaluations \cite{dunnett2024countering, wu2024backdoorbench} continue to highlight the importance of sufficient auxiliary data for achieving robust defenses against backdoor attacks.

While significant progress has been made in reducing the size of auxiliary datasets, an equally critical yet underexplored question remains: \emph{how does the nature of the auxiliary dataset affect purification effectiveness?} In  real-world  applications, auxiliary datasets can vary widely, encompassing in-distribution data, synthetic data, or external data from different sources. Understanding how each type of auxiliary dataset influences the purification effectiveness is vital for selecting or constructing the most suitable auxiliary dataset and the corresponding technique. For instance, when multiple datasets are available, understanding how different datasets contribute to purification can guide defenders in selecting or crafting the most appropriate dataset. Conversely, when only limited auxiliary data is accessible, knowing which purification technique works best under those constraints is critical. Therefore, there is an urgent need for a thorough investigation into the impact of auxiliary datasets on purification effectiveness to guide defenders in  enhancing the security of ML systems. 

In this paper, we systematically investigate the critical role of auxiliary datasets in backdoor purification, aiming to bridge the gap between idealized and practical purification scenarios.  Specifically, we first construct a diverse set of auxiliary datasets to emulate real-world conditions, as summarized in Table~\ref{overall}. These datasets include in-distribution data, synthetic data, and external data from other sources. Through an evaluation of SOTA backdoor purification methods across these datasets, we uncover several critical insights: \textbf{1)} In-distribution datasets, particularly those carefully filtered from the original training data of the victim model, effectively preserve the model’s utility for its intended tasks but may fall short in eliminating backdoors. \textbf{2)} Incorporating OOD datasets can help the model forget backdoors but also bring the risk of forgetting critical learned knowledge, significantly degrading its overall performance. Building on these findings, we propose Guided Input Calibration (GIC), a novel technique that enhances backdoor purification by adaptively transforming auxiliary data to better align with the victim model’s learned representations. By leveraging the victim model itself to guide this transformation, GIC optimizes the purification process, striking a balance between preserving model utility and mitigating backdoor threats. Extensive experiments demonstrate that GIC significantly improves the effectiveness of backdoor purification across diverse auxiliary datasets, providing a practical and robust defense solution.

Our main contributions are threefold:
\textbf{1) Impact analysis of auxiliary datasets:} We take the \textbf{first step}  in systematically investigating how different types of auxiliary datasets influence backdoor purification effectiveness. Our findings provide novel insights and serve as a foundation for future research on optimizing dataset selection and construction for enhanced backdoor defense.
%
\textbf{2) Compilation and evaluation of diverse auxiliary datasets:}  We have compiled and rigorously evaluated a diverse set of auxiliary datasets using SOTA purification methods, making our datasets and code publicly available to facilitate and support future research on practical backdoor defense strategies.
%
\textbf{3) Introduction of GIC:} We introduce GIC, the \textbf{first} dedicated solution designed to align auxiliary datasets with the model’s learned representations, significantly enhancing backdoor mitigation across various dataset types. Our approach sets a new benchmark for practical and effective backdoor defense.



\section{Related Work}

\subsection{Large 3D Reconstruction Models}
Recently, generalized feed-forward models for 3D reconstruction from sparse input views have garnered considerable attention due to their applicability in heavily under-constrained scenarios. The Large Reconstruction Model (LRM)~\cite{hong2023lrm} uses a transformer-based encoder-decoder pipeline to infer a NeRF reconstruction from just a single image. Newer iterations have shifted the focus towards generating 3D Gaussian representations from four input images~\cite{tang2025lgm, xu2024grm, zhang2025gslrm, charatan2024pixelsplat, chen2025mvsplat, liu2025mvsgaussian}, showing remarkable novel view synthesis results. The paradigm of transformer-based sparse 3D reconstruction has also successfully been applied to lifting monocular videos to 4D~\cite{ren2024l4gm}. \\
Yet, none of the existing works in the domain have studied the use-case of inferring \textit{animatable} 3D representations from sparse input images, which is the focus of our work. To this end, we build on top of the Large Gaussian Reconstruction Model (GRM)~\cite{xu2024grm}.

\subsection{3D-aware Portrait Animation}
A different line of work focuses on animating portraits in a 3D-aware manner.
MegaPortraits~\cite{drobyshev2022megaportraits} builds a 3D Volume given a source and driving image, and renders the animated source actor via orthographic projection with subsequent 2D neural rendering.
3D morphable models (3DMMs)~\cite{blanz19993dmm} are extensively used to obtain more interpretable control over the portrait animation. For example, StyleRig~\cite{tewari2020stylerig} demonstrates how a 3DMM can be used to control the data generated from a pre-trained StyleGAN~\cite{karras2019stylegan} network. ROME~\cite{khakhulin2022rome} predicts vertex offsets and texture of a FLAME~\cite{li2017flame} mesh from the input image.
A TriPlane representation is inferred and animated via FLAME~\cite{li2017flame} in multiple methods like Portrait4D~\cite{deng2024portrait4d}, Portrait4D-v2~\cite{deng2024portrait4dv2}, and GPAvatar~\cite{chu2024gpavatar}.
Others, such as VOODOO 3D~\cite{tran2024voodoo3d} and VOODOO XP~\cite{tran2024voodooxp}, learn their own expression encoder to drive the source person in a more detailed manner. \\
All of the aforementioned methods require nothing more than a single image of a person to animate it. This allows them to train on large monocular video datasets to infer a very generic motion prior that even translates to paintings or cartoon characters. However, due to their task formulation, these methods mostly focus on image synthesis from a frontal camera, often trading 3D consistency for better image quality by using 2D screen-space neural renderers. In contrast, our work aims to produce a truthful and complete 3D avatar representation from the input images that can be viewed from any angle.  

\subsection{Photo-realistic 3D Face Models}
The increasing availability of large-scale multi-view face datasets~\cite{kirschstein2023nersemble, ava256, pan2024renderme360, yang2020facescape} has enabled building photo-realistic 3D face models that learn a detailed prior over both geometry and appearance of human faces. HeadNeRF~\cite{hong2022headnerf} conditions a Neural Radiance Field (NeRF)~\cite{mildenhall2021nerf} on identity, expression, albedo, and illumination codes. VRMM~\cite{yang2024vrmm} builds a high-quality and relightable 3D face model using volumetric primitives~\cite{lombardi2021mvp}. One2Avatar~\cite{yu2024one2avatar} extends a 3DMM by anchoring a radiance field to its surface. More recently, GPHM~\cite{xu2025gphm} and HeadGAP~\cite{zheng2024headgap} have adopted 3D Gaussians to build a photo-realistic 3D face model. \\
Photo-realistic 3D face models learn a powerful prior over human facial appearance and geometry, which can be fitted to a single or multiple images of a person, effectively inferring a 3D head avatar. However, the fitting procedure itself is non-trivial and often requires expensive test-time optimization, impeding casual use-cases on consumer-grade devices. While this limitation may be circumvented by learning a generalized encoder that maps images into the 3D face model's latent space, another fundamental limitation remains. Even with more multi-view face datasets being published, the number of available training subjects rarely exceeds the thousands, making it hard to truly learn the full distibution of human facial appearance. Instead, our approach avoids generalizing over the identity axis by conditioning on some images of a person, and only generalizes over the expression axis for which plenty of data is available. 

A similar motivation has inspired recent work on codec avatars where a generalized network infers an animatable 3D representation given a registered mesh of a person~\cite{cao2022authentic, li2024uravatar}.
The resulting avatars exhibit excellent quality at the cost of several minutes of video capture per subject and expensive test-time optimization.
For example, URAvatar~\cite{li2024uravatar} finetunes their network on the given video recording for 3 hours on 8 A100 GPUs, making inference on consumer-grade devices impossible. In contrast, our approach directly regresses the final 3D head avatar from just four input images without the need for expensive test-time fine-tuning.


\section{Preliminaries}
\label{sec:problem_formulation}

In this section, we formulate the key technical components and our objectives.

\paragraph{Predicate $\Predicate$}
At a certain time point, given all environment information $\mathcal{E} = (M, I, A)$ and a motion plan $\tau$, the differentiable predicate $\Predicate$ is defined as: $\Predicate : (\mathcal{E} \times \tau) \rightarrow [-1, 1]$, which evaluates driving conditions and the ego car's motion plan \footnote{The ego car refers to the vehicle being controlled in a driving scenario.}. Here, $M$ represents the HD map API that provides the functionality to query a drivable area, lane information, routing information and occupancy grid; $I = \{i_1,...,i_K\}$ represents the set of traffic light states, where each $i_k \in \{\mathit{red, yellow, green}\}$; $A = \{a_1,...,a_N\}$ represents the set of other agents, where each $a_i = \{(x_t, y_t, v_t)\}_{t=0}^{T}$ contains the agent's trajectory for the next 4 seconds (i.e., $T = 80$, $20Hz \times 4 s$). The motion plan $\tau = \{(x_t, y_t, v_t)\}_{t=0}^{T}$  consisting of a sequence of positions $(x_t, y_t)$ and speeds $v_t$, where higher-order derivatives (e.g., acceleration, jerk) can be computed through numerical differentiation when needed.
$\Predicate$ maps $\mathcal{E}$ and $\tau$ to a truth confidence value in $[-1,1]$, where $\theta$ are the predicate parameters. Each predicate $\Predicate$ is parameterized by $\theta$, which defines thresholds or constraints specific to that predicate (e.g., safe time-to-collision threshold, comfortable acceleration bounds). When designing the predicate, we ensure that the gradient $\nabla_{\theta} \Predicate$ exists and can be computed. The sign of $\Predicate$ indicates truth. $\Predicate < 0$ implies False; $\Predicate > 0$ implies True. The absolute value $|\Predicate|$ indicates the degree of confidence.
Similar to most existing work \cite{bartocci2022survey}, we explicitly design the predicates and focus this paper on learning logical connections and parameters assuming a given set of predicates. With the predicate defined, we can now move on to discussing how these predicates are combined into logical formulas.

\paragraph{Formula $\mathcal{L}$}
Given a differentiable predicate set $\PredicateSet = \{\PredicateWithIndex{1}, \PredicateWithIndex{2}, \ldots, \PredicateWithIndex{n}\}$, we introduce a $\mathtt{LTL}_f$ logic space \cite{LTLf} (LTL over finite traces) that includes compositions of predicates from $\PredicateSet$ and logic operators. The logic formula  $\mathcal{L}$ can be generated from the following grammar:
\begin{align}
    \mathcal{L} := \Predicate \mid \G\ \mathcal{L} \mid \F\ \mathcal{L} \mid \lnot \mathcal{L} \mid \mathcal{L} \land \mathcal{L}' \mid \mathcal{L} \lor \mathcal{L}'
    \label{eq:logic_formula_syntax}
\end{align}
where $\Predicate \in \mathcal{P}$ is a differentiable predicate, $\G$ and $\F$ are temporal operators representing ``globally'' and ``finally'' respectively, $\lnot$ is logical negation, $\land$ is logical and, and $\lor$ is logical or.  Like most existing work \cite{bartocci2022survey}, the strong ``Until'' ($\mathcal{L}\ \U\ \mathcal{L}'$) is not included because it can be represented using existing logic operators ($\F\ \mathcal{L}' \wedge \G(\mathcal{L} \lor \mathcal{L}')$). Having established the syntax for our logic formulas, we now need a way to evaluate them quantitatively.

\paragraph{Quantitative Evaluation of Formula}
Given a finite input sequence $S = \{(\mathcal{E}_t, \tau_t)\}_{t=0}^T$ sampled at different time points, up to a bounded time $T$, we can evaluate the logic formula $\mathcal{L}$ quantitatively using a set of min and max operators\cite{fainekos2009robustness, deshmukh2017robust}. This evaluation maps the sequence $S$ to a value in $[-1, 1]$, denoted as
$
    \mathcal{L}(S; \boldsymbol{\theta}) \rightarrow [-1, 1].
$
Here, $\boldsymbol{\theta}$ represents all the predicates' parameters.
Specifically, we define the quantitative evaluation of an atomic predicate $\Predicate$ at time $t$ as $\Predicate(\mathcal{E}_t, \tau_t) \in [-1, 1]$. In \eqref{eq:temporal_operators}, the temporal logic operators $\G$ and $\F$ evaluate the formula $\mathcal{L}$ over the entire sequence from time $t$ onwards. $\G \mathcal{L}$ (globally) returns the minimum value of $\mathcal{L}$ over all future time points, which ensures the property holds throughout the sequence if $\G \mathcal{L}$ evaluates to a positive value. $\F \mathcal{L}$ (finally) returns the maximum value, indicating the property is satisfied at least once in the future. The evaluation function $\rho$ is defined as:
\begin{equation}
    \begin{aligned}
        \rho(\G \mathcal{L}, t) & = \min_{t' \geq t} \rho(\mathcal{L}, t') \quad & \rho(\F \mathcal{L}, t) & = \max_{t' \geq t} \rho(\mathcal{L}, t')
    \end{aligned}
    \label{eq:temporal_operators}
\end{equation}
For single time point evaluation, the logical operators and ($\land$), or ($\lor$), and not ($\lnot$) are defined using min, max, and negation operations:
\begin{equation}
    \begin{aligned}
         & \rho(\mathcal{L} \land \mathcal{L}', t) & = & \min\{\rho(\mathcal{L}, t), \rho(\mathcal{L}', t)\} \\
         & \rho(\mathcal{L} \lor \mathcal{L}', t)  & = & \max\{\rho(\mathcal{L}, t), \rho(\mathcal{L}', t)\} \\
         & \rho(\lnot \mathcal{L}, t)              & = & -\rho(\mathcal{L}, t)
    \end{aligned}
    \label{eq:fol_operators}
\end{equation}
All the operations defined by $\rho$ are differentiable, which enables the use of backpropagation in the learning process.  In practice, we use softmin and softmax to approximate min and max operators for a smooth gradient \cite{Leung2020BackpropagationTS}. With the evaluation framework in place, we can now define our overall objective for learning optimal driving rules. For simplicity, we define $\rho(\cdot):= \rho(\cdot, 0)$, meaning evaluate from the initial of input sequence.

\paragraph{Objective}
Our objective is twofold: (1) learn the optimal logic formula $\mathcal{L}^*$, and (2) optimize the parameters $\boldsymbol{\theta}$ of the predicates, which characterize the demonstration data accurately.
Formally, we aim to solve the following problem:
\begin{align}
    \mathcal{L}^*, \boldsymbol{\theta}^* = \argmax_{\mathcal{L} \in \Omega_\PredicateSet, \boldsymbol{\theta}} \mathbb{E}_{S \sim \mathcal{D}^+} [\mathcal{L}(S; \boldsymbol{\theta})]
    \label{eq:objective}
\end{align}
where $\mathcal{D}^+$ represents driving demonstrations, which consist \textit{solely} of correct demonstrations that represent ideal driving behaviors.
\begin{figure}[h]
  \centering
  \includegraphics[width=0.8\linewidth]{figures/pdfs/pipeline.pdf}
  \caption{\textbf{Schematic representation of our DDB framework.} 
  The debiasing process consists of two key steps: (A) \textit{Diffusing the Bias} uses a conditional diffusion model with classifier-free guidance to generate synthetic images that preserve training dataset biases, and (B) employs a \textit{Bias Amplifier} firstly trained on such synthetic data, and subsequently used during inference to extract supervisory bias signals from real images. These signals are used to guide the training process of a target debiased model by designing two \textit{debiasing recipes} (\ie, 2-step and end-to-end methods). 
  }
  \label{fig:pipeline}
\end{figure}
\section{The Approach}
\label{sec:approach}
Our proposed debiasing approach is schematically depicted in Figure~\ref{fig:pipeline}. 
In this section, we provide at first a general description of the problem setting (Sec. ~\ref{sec:problem-formulation}), and then, we illustrate in detail DDB's two main components, which include \textit{bias diffusion} (Sec.~\ref{sec:biasdiff}) and the two \textit{Recipes} for model debiasing (Sec.~\ref{sec:recipes}).
%
\subsection{Problem Setting}
\label{sec:problem-formulation}
Let us consider a general data distribution $p_{\text{data}}$, typically encompassing multiple factors of variation and classes, and to build a dataset of images with the associated labels $~{\dataset = \lbrace(\mathbf{x}_i, y_i)\rbrace_{i=1}^N}$ sampled from such a distribution. Let us also assume that the sampling process to obtain $\dataset$ is not uniform across latent factors of variations, \ie possible biases such as context, appearance, acquisition noise, viewpoint, etc.. 
In this case, data will not faithfully capture the true data distribution ($p_{\text{data}}$) just because of these bias factors. 
%and will likely be biased. 
This phenomenon deeply affects the generalization capabilities of deep neural networks in classifying unseen examples not presenting the same biases.
In the same way, we could consider $\dataset$ as the union between two sets, \ie $\dataset = \udataset \bigcup \bdataset$. Here, the elements of $\udataset$ are uniformly sampled from $p_\text{data}$ and, in $\bdataset$, they are instead sampled from a conditional distribution $p_\text{data}\left(\mathbf{x}, y \: \vert \: b \right)$, with $b \in B$ being some latent factor (bias attribute) from a set of possible attributes $B$, likely to be unknown or merely not annotated, in a realistic setting~\cite{kim2024training}~\footnote{In this context, biased and unbiased samples equivalently refer to bias-aligned and bias conflicting samples.}. 
If $\vert \bdataset \vert \gg \vert \udataset \vert$, optimizing a classification model $f_{\theta}$ over $\dataset$ likely results in biased predictions and poor generalization. This is due to the strong correlation between $b$ and $y$, often called \textit{spurious correlation}, and denoted as $\rho(y, b)$, or just $\rho$ for brevity \cite{kim2021biaswap, Sagawa*2020Distributionally, nahon2023mining}), which is dominating over the true target distribution semantics. 


It is important to notice that data bias is a general problem, not only affecting classification tasks but also impacting several others such as data generation~\cite{d2024openbias}. For instance, given a  Conditional Diffusion 
Probabilistic Models (CDPM) modeled as a neural network $\cdpm$ (with parameters $\phi$) that learns to approximate a conditional distribution $p\left(\mathbf{x} \: \vert \: y \right)$ from $\dataset$, we expect that its generations will be biased, as also stated in~\cite{d2024openbias, kim2024training}. While this is a strong downside for image-generation purposes, in this work, we claim that when $\rho(y, b)$ is very high (\eg $\geq 0.95$, as generally assumed in model debiasing literature \cite{nam2020learning}), a CDPM predominantly learns the biased distribution of a specific class, \ie, $\cdpm \approx p \left(\mathbf{x} \: \vert \: b\right)$ rather than $p \left(\mathbf{x} \: \vert \: y\right)$.
\subsection{Diffusing the Bias}
\label{sec:biasdiff}
In the context of mitigating bias in classification models, the tendency of a CDPM to approximate the per-class biased distribution represents a key feature for training an auxiliary \textit{bias amplified} model.   
\paragraph{The Diffusion Process.}
The diffusion process progressively converts data into noise through a fixed Markov chain of \( T \) steps~\cite{DBLP:conf/nips/HoJA20}. Given a data point \( \mathbf{x}_0 \), the forward process adds Gaussian noise according to a variance schedule \( \{\beta_t\}_{t=1}^T \), resulting in noisy samples \( \mathbf{x}_1, \dots, \mathbf{x}_T \). This forward process can be formulated for any timestep \( t \) as: ~{$q(\mathbf{x}_t | \mathbf{x}_0) = \mathcal{N}(\mathbf{x}_t ; \sqrt{\bar{\alpha}_t} \mathbf{x}_0, (1 - \bar{\alpha}_t) \mathbf{I})$}, 
where \( \bar{\alpha}_t = \prod_{s=1}^t \alpha_s \) with \( \alpha_s = 1 - \beta_s \).
The reverse process then gradually denoises a sample, reparameterizing each step to predict the noise \( \epsilon \) using a model \( \boldsymbol{\epsilon}_\theta \):
\begin{equation}
\label{eq:ddpm_reverse}
\mathbf{x}_{t-1} = \frac{1}{\sqrt{\alpha_t}} \left( \mathbf{x}_t - \frac{\beta_t}{\sqrt{1 - \bar{\alpha}_t}} \boldsymbol{\epsilon}_\theta(\mathbf{x}_t, t) \right) + \sigma_t \mathbf{z},
\end{equation}
\noindent
where \( \mathbf{z} \sim \mathcal{N}(\mathbf{0}, \mathbf{I}) \) and \( \sigma_t = \sqrt{\beta_t} \).
\paragraph{Classifier-Free Guidance for Biased Image Generation.}
In cases where additional context or \textit{conditioning} is available, such as a class label \( y \), diffusion models can use this information to guide the reverse process, generating samples that better reflect the target attributes and semantics. Classifier-Free Guidance (CFG)~\cite{DBLP:journals/corr/abs-2207-12598} introduces a flexible conditioning approach, allowing the model to balance conditional and unconditional outputs without dedicated classifiers.

The CFG technique randomly omits conditioning during training (\eg, with probability \( p_{\text{uncond}} = 0.1 \)), enabling the model to learn both generation modalities. During the sampling process, a guidance scale \( w \) modulates the influence of conditioning. When \( w = 0 \), the model relies solely on the conditional model. As \( w \) increases (\( w \geq 1 \)), the conditioning effect is intensified, potentially resulting in more distinct features linked to \( y \), thereby increasing fidelity to the class while possibly reducing diversity, whereas lower values help to preserve diversity by decreasing the influence of conditioning. The guided noise prediction is given by:
\begin{equation}
\boldsymbol{\epsilon}_{t} = (1 + w) \boldsymbol{\epsilon}_\theta(\mathbf{x}_t, t, y) - w \boldsymbol{\epsilon}_\theta(\mathbf{x}_t, t),
\end{equation}
\noindent
where \( \boldsymbol{\epsilon}_\theta(\mathbf{x}_t, t, y) \) is the noise prediction conditioned on class label \( y \), and \( \boldsymbol{\epsilon}_\theta(\mathbf{x}_t, t) \) is the unconditional noise prediction. This modified noise prediction replaces the standard \( \boldsymbol{\epsilon}_\theta(\mathbf{x}_t, t) \) term in the reverse process formula (Equation \ref{eq:ddpm_reverse}).
In this work, we empirically show how CDPM learns and amplifies the underlying biased distribution when trained on a biased dataset with strong spurious correlations,  allowing bias-aligned image generation. 
\subsection{DDB: Bias Amplifier and Model Debiasing}
\label{sec:recipes}
As stated in Sec.~\ref{sec:rel-work}, a typical unsupervised approach to model debiasing relies on an auxiliary intentionally-biased model, named here as \textit{Bias Amplifier} (BA). This model can be exploited in either 2-step or end-to-end approaches, denoted here as \textit{Recipe I} and \textit{Recipe II}, respectively. 
\subsubsection{Recipe I: 2-step debiasing}
\label{sec:recipe-one}
\begin{figure}[hp]
    \centering    \includegraphics[width=0.6\linewidth]{figures/pdfs/groupdro.pdf}
    \caption{Overview of \textit{Recipe I}'s 2-step debiasing approach.}
    \label{fig:gdro}
\end{figure}
\noindent
The adopted 2-step approach consists in 1) applying the auxiliary model trained on biased generated data to perform a bias pseudo-labeling, hence estimating bias-aligned/bias-conflict split of original actual data, and 2) apply a \textit{bias supervised} method to train a debiased target model for classification. For the latter, we use the group DRO algorithm~\cite {Sagawa*2020Distributionally} (G-DRO) as a proven technique for the pure debiasing step. 
In other words, being in the unsupervised bias scenario where the real bias labels are unknown, we estimate bias pseudo-labels performing an inference step by feeding the trained BA with the original actual training data, and identifying as bias-aligned the correctly classified samples, and as bias-conflicting those misclassified. Among possible strategies to assign bias pseudo-labels, such as feature-clustering~\cite{sohoni2020no} or anomaly detection~\cite{pastore2024lookingmodeldebiasinglens}, we adopt a simple heuristic based on %top of 
the BA misclassifications. 
Specifically, given a sample $(\mathbf{x}_i, y_i, c_i)$ with $c_i$ unknown pseudo-label indicating whether $\mathbf{x}_i$ is bias conflicting or aligned, we estimate bias-conflicting samples as
\begin{equation}\label{eq:gdro-threhsold}
    \hat{c}_i = \mathds{1} \left( \hat{y}_i \neq y_i~\land~\mathcal{L}(\hat{y}_i, y_i) ~>~\mu_n(\mathcal{L}) + \gamma \sigma_n(\mathcal{L}) \right)
\end{equation} 
where $\mathds{1}$ is the indicator function, $\mathcal{L}$ is the CE loss of the BA on the real sample, and $\mu_n$ and $\sigma_n$ represent the average training loss and its standard deviation, respectively, depending on the loss $\mathcal{L}$. Together with the multiplier $\gamma~\in\mathbb{N}$, this condition defines a sort of filter over misclassified samples, considering them as conflicting only if their loss is also higher than the mean loss increased by a quantity corresponding to a certain {z-score} of the per-sample training loss distribution ($~{\mu_n(\mathcal{L}) + \gamma \sigma_n(\mathcal{L})}$ in Eq.~\ref{eq:gdro-threhsold}). 
Once bias pseudo-labels over original training data are obtained, we plug in our estimate as group information for the G-DRO optimization, as schematically depicted in Figure~\ref{fig:gdro}.

The above \textit{filtering} operation refines the plain \textit{error set}, restricting bias-conflicting sample selection to the hardest training samples, with potential benefits for the most difficult correlation settings ($\rho > 0.99$). 
Later in the experimental section, we provide an ablation study comparing different filtering ($\gamma$) configurations and plain error set alternatives. 
\subsubsection{Recipe II: end-to-end debiasing}
\label{sec:recipe-two}
A typical end-to-end debiasing setting includes the joint training of the target debiasing model and one~\cite{nam2020learning} or more~\cite{NEURIPS2022_75004615_LWBC, Lee_Park_Kim_Lee_Choi_Choo_2023} auxiliary intentionally-biased models. Here, we design an end-to-end debiasing procedure, denoted as \textit{Recipe II}, incorporating our BA by customizing a widespread general scheme, introduced in the Learning from Failure (LfF) method~\cite{nam2020learning}.
LfF leverages an intentionally-biased model trained using Generalized CE (GCE) loss to support the simultaneous training of a debiased model adopting the CE loss re-weighted by a per-sample relative difficulty score.
Specifically, we replace the GCE biased model with our Bias Amplifier, which is frozen and only employed in inference to compute its loss function for each original training sample ($\mathcal{L}_\text{bias\_amp}$), as schematically represented in Figure~\ref{fig:LLD}. 
Such loss function is used to obtain %multiplier 
a weighting factor for the target model loss function, defined as $
r = \frac{\mathcal{L}_{\text{Bias\_Amp}}}{\mathcal{L}_{\text{debiasing}} + \mathcal{L}_{\text{Bias\_Amp}}}$. Coarsely speaking, $r$ should be low for bias-aligned and high for bias-conflicting samples.
\begin{figure}[h]
  \centering
\includegraphics[width=.6\linewidth]{figures/pdfs/end2end.pdf}
  \caption{Overview of \textit{Recipe II}'s end-to-end debiasing approach.}
  \label{fig:LLD}
\end{figure}


\section{Experiments}
\label{section5}

In this section, we conduct extensive experiments to show that \ourmethod~can significantly speed up the sampling of existing MR Diffusion. To rigorously validate the effectiveness of our method, we follow the settings and checkpoints from \cite{luo2024daclip} and only modify the sampling part. Our experiment is divided into three parts. Section \ref{mainresult} compares the sampling results for different NFE cases. Section \ref{effects} studies the effects of different parameter settings on our algorithm, including network parameterizations and solver types. In Section \ref{analysis}, we visualize the sampling trajectories to show the speedup achieved by \ourmethod~and analyze why noise prediction gets obviously worse when NFE is less than 20.


\subsection{Main results}\label{mainresult}

Following \cite{luo2024daclip}, we conduct experiments with ten different types of image degradation: blurry, hazy, JPEG-compression, low-light, noisy, raindrop, rainy, shadowed, snowy, and inpainting (see Appendix \ref{appd1} for details). We adopt LPIPS \citep{zhang2018lpips} and FID \citep{heusel2017fid} as main metrics for perceptual evaluation, and also report PSNR and SSIM \citep{wang2004ssim} for reference. We compare \ourmethod~with other sampling methods, including posterior sampling \citep{luo2024posterior} and Euler-Maruyama discretization \citep{kloeden1992sde}. We take two tasks as examples and the metrics are shown in Figure \ref{fig:main}. Unless explicitly mentioned, we always use \ourmethod~based on SDE solver, with data prediction and uniform $\lambda$. The complete experimental results can be found in Appendix \ref{appd3}. The results demonstrate that \ourmethod~converges in a few (5 or 10) steps and produces samples with stable quality. Our algorithm significantly reduces the time cost without compromising sampling performance, which is of great practical value for MR Diffusion.


\begin{figure}[!ht]
    \centering
    \begin{minipage}[b]{0.45\textwidth}
        \centering
        \includegraphics[width=1\textwidth, trim=0 20 0 0]{figs/main_result/7_lowlight_fid.pdf}
        \subcaption{FID on \textit{low-light} dataset}
        \label{fig:main(a)}
    \end{minipage}
    \begin{minipage}[b]{0.45\textwidth}
        \centering
        \includegraphics[width=1\textwidth, trim=0 20 0 0]{figs/main_result/7_lowlight_lpips.pdf}
        \subcaption{LPIPS on \textit{low-light} dataset}
        \label{fig:main(b)}
    \end{minipage}
    \begin{minipage}[b]{0.45\textwidth}
        \centering
        \includegraphics[width=1\textwidth, trim=0 20 0 0]{figs/main_result/10_motion_fid.pdf}
        \subcaption{FID on \textit{motion-blurry} dataset}
        \label{fig:main(c)}
    \end{minipage}
    \begin{minipage}[b]{0.45\textwidth}
        \centering
        \includegraphics[width=1\textwidth, trim=0 20 0 0]{figs/main_result/10_motion_lpips.pdf}
        \subcaption{LPIPS on \textit{motion-blurry} dataset}
        \label{fig:main(d)}
    \end{minipage}
    \caption{\textbf{Perceptual evaluations on \textit{low-light} and \textit{motion-blurry} datasets.}}
    \label{fig:main}
\end{figure}

\subsection{Effects of parameter choice}\label{effects}

In Table \ref{tab:ablat_param}, we compare the results of two network parameterizations. The data prediction shows stable performance across different NFEs. The noise prediction performs similarly to data prediction with large NFEs, but its performance deteriorates significantly with smaller NFEs. The detailed analysis can be found in Section \ref{section5.3}. In Table \ref{tab:ablat_solver}, we compare \ourmethod-ODE-d-2 and \ourmethod-SDE-d-2 on the \textit{inpainting} task, which are derived from PF-ODE and reverse-time SDE respectively. SDE-based solver works better with a large NFE, whereas ODE-based solver is more effective with a small NFE. In general, neither solver type is inherently better.


% In Table \ref{tab:hazy}, we study the impact of two step size schedules on the results. On the whole, uniform $\lambda$ performs slightly better than uniform $t$. Our algorithm follows the method of \cite{lu2022dpmsolverplus} to estimate the integral part of the solution, while the analytical part does not affect the error.  Consequently, our algorithm has the same global truncation error, that is $\mathcal{O}\left(h_{max}^{k}\right)$. Note that the initial and final values of $\lambda$ depend on noise schedule and are fixed. Therefore, uniform $\lambda$ scheduling leads to the smallest $h_{max}$ and works better.

\begin{table}[ht]
    \centering
    \begin{minipage}{0.5\textwidth}
    \small
    \renewcommand{\arraystretch}{1}
    \centering
    \caption{Ablation study of network parameterizations on the Rain100H dataset.}
    % \vspace{8pt}
    \resizebox{1\textwidth}{!}{
        \begin{tabular}{cccccc}
			\toprule[1.5pt]
            % \multicolumn{6}{c}{Rainy} \\
            % \cmidrule(lr){1-6}
             NFE & Parameterization      & LPIPS\textdownarrow & FID\textdownarrow &  PSNR\textuparrow & SSIM\textuparrow  \\
            \midrule[1pt]
            \multirow{2}{*}{50}
             & Noise Prediction & \textbf{0.0606}     & \textbf{27.28}   & \textbf{28.89}     & \textbf{0.8615}    \\
             & Data Prediction & 0.0620     & 27.65   & 28.85     & 0.8602    \\
            \cmidrule(lr){1-6}
            \multirow{2}{*}{20}
              & Noise Prediction & 0.1429     & 47.31   & 27.68     & 0.7954    \\
              & Data Prediction & \textbf{0.0635}     & \textbf{27.79}   & \textbf{28.60}     & \textbf{0.8559}    \\
            \cmidrule(lr){1-6}
            \multirow{2}{*}{10}
              & Noise Prediction & 1.376     & 402.3   & 6.623     & 0.0114    \\
              & Data Prediction & \textbf{0.0678}     & \textbf{29.54}   & \textbf{28.09}     & \textbf{0.8483}    \\
            \cmidrule(lr){1-6}
            \multirow{2}{*}{5}
              & Noise Prediction & 1.416     & 447.0   & 5.755     & 0.0051    \\
              & Data Prediction & \textbf{0.0637}     & \textbf{26.92}   & \textbf{28.82}     & \textbf{0.8685}    \\       
            \bottomrule[1.5pt]
        \end{tabular}}
        \label{tab:ablat_param}
    \end{minipage}
    \hspace{0.01\textwidth}
    \begin{minipage}{0.46\textwidth}
    \small
    \renewcommand{\arraystretch}{1}
    \centering
    \caption{Ablation study of solver types on the CelebA-HQ dataset.}
    % \vspace{8pt}
        \resizebox{1\textwidth}{!}{
        \begin{tabular}{cccccc}
			\toprule[1.5pt]
            % \multicolumn{6}{c}{Raindrop} \\     
            % \cmidrule(lr){1-6}
             NFE & Solver Type     & LPIPS\textdownarrow & FID\textdownarrow &  PSNR\textuparrow & SSIM\textuparrow  \\
            \midrule[1pt]
            \multirow{2}{*}{50}
             & ODE & 0.0499     & 22.91   & 28.49     & 0.8921    \\
             & SDE & \textbf{0.0402}     & \textbf{19.09}   & \textbf{29.15}     & \textbf{0.9046}    \\
            \cmidrule(lr){1-6}
            \multirow{2}{*}{20}
              & ODE & 0.0475    & 21.35   & 28.51     & 0.8940    \\
              & SDE & \textbf{0.0408}     & \textbf{19.13}   & \textbf{28.98}    & \textbf{0.9032}    \\
            \cmidrule(lr){1-6}
            \multirow{2}{*}{10}
              & ODE & \textbf{0.0417}    & 19.44   & \textbf{28.94}     & \textbf{0.9048}    \\
              & SDE & 0.0437     & \textbf{19.29}   & 28.48     & 0.8996    \\
            \cmidrule(lr){1-6}
            \multirow{2}{*}{5}
              & ODE & \textbf{0.0526}     & 27.44   & \textbf{31.02}     & \textbf{0.9335}    \\
              & SDE & 0.0529    & \textbf{24.02}   & 28.35     & 0.8930    \\
            \bottomrule[1.5pt]
        \end{tabular}}
        \label{tab:ablat_solver}
    \end{minipage}
\end{table}


% \renewcommand{\arraystretch}{1}
%     \centering
%     \caption{Ablation study of step size schedule on the RESIDE-6k dataset.}
%     % \vspace{8pt}
%         \resizebox{1\textwidth}{!}{
%         \begin{tabular}{cccccc}
% 			\toprule[1.5pt]
%             % \multicolumn{6}{c}{Raindrop} \\     
%             % \cmidrule(lr){1-6}
%              NFE & Schedule      & LPIPS\textdownarrow & FID\textdownarrow &  PSNR\textuparrow & SSIM\textuparrow  \\
%             \midrule[1pt]
%             \multirow{2}{*}{50}
%              & uniform $t$ & 0.0271     & 5.539   & 30.00     & 0.9351    \\
%              & uniform $\lambda$ & \textbf{0.0233}     & \textbf{4.993}   & \textbf{30.19}     & \textbf{0.9427}    \\
%             \cmidrule(lr){1-6}
%             \multirow{2}{*}{20}
%               & uniform $t$ & 0.0313     & 6.000   & 29.73     & 0.9270    \\
%               & uniform $\lambda$ & \textbf{0.0240}     & \textbf{5.077}   & \textbf{30.06}    & \textbf{0.9409}    \\
%             \cmidrule(lr){1-6}
%             \multirow{2}{*}{10}
%               & uniform $t$ & 0.0309     & 6.094   & 29.42     & 0.9274    \\
%               & uniform $\lambda$ & \textbf{0.0246}     & \textbf{5.228}   & \textbf{29.65}     & \textbf{0.9372}    \\
%             \cmidrule(lr){1-6}
%             \multirow{2}{*}{5}
%               & uniform $t$ & 0.0256     & 5.477   & \textbf{29.91}     & 0.9342    \\
%               & uniform $\lambda$ & \textbf{0.0228}     & \textbf{5.174}   & 29.65     & \textbf{0.9416}    \\
%             \bottomrule[1.5pt]
%         \end{tabular}}
%         \label{tab:ablat_schedule}



\subsection{Analysis}\label{analysis}
\label{section5.3}

\begin{figure}[ht!]
    \centering
    \begin{minipage}[t]{0.6\linewidth}
        \centering
        \includegraphics[width=\linewidth, trim=0 20 10 0]{figs/trajectory_a.pdf} %trim左下右上
        \subcaption{Sampling results.}
        \label{fig:traj(a)}
    \end{minipage}
    \begin{minipage}[t]{0.35\linewidth}
        \centering
        \includegraphics[width=\linewidth, trim=0 0 0 0]{figs/trajectory_b.pdf} %trim左下右上
        \subcaption{Trajectory.}
        \label{fig:traj(b)}
    \end{minipage}
    \caption{\textbf{Sampling trajectories.} In (a), we compare our method (with order 1 and order 2) and previous sampling methods (i.e., posterior sampling and Euler discretization) on a motion blurry image. The numbers in parentheses indicate the NFE. In (b), we illustrate trajectories of each sampling method. Previous methods need to take many unnecessary paths to converge. With few NFEs, they fail to reach the ground truth (i.e., the location of $\boldsymbol{x}_0$). Our methods follow a more direct trajectory.}
    \label{fig:traj}
\end{figure}

\textbf{Sampling trajectory.}~ Inspired by the design idea of NCSN \citep{song2019ncsn}, we provide a new perspective of diffusion sampling process. \cite{song2019ncsn} consider each data point (e.g., an image) as a point in high-dimensional space. During the diffusion process, noise is added to each point $\boldsymbol{x}_0$, causing it to spread throughout the space, while the score function (a neural network) \textit{remembers} the direction towards $\boldsymbol{x}_0$. In the sampling process, we start from a random point by sampling a Gaussian distribution and follow the guidance of the reverse-time SDE (or PF-ODE) and the score function to locate $\boldsymbol{x}_0$. By connecting each intermediate state $\boldsymbol{x}_t$, we obtain a sampling trajectory. However, this trajectory exists in a high-dimensional space, making it difficult to visualize. Therefore, we use Principal Component Analysis (PCA) to reduce $\boldsymbol{x}_t$ to two dimensions, obtaining the projection of the sampling trajectory in 2D space. As shown in Figure \ref{fig:traj}, we present an example. Previous sampling methods \citep{luo2024posterior} often require a long path to find $\boldsymbol{x}_0$, and reducing NFE can lead to cumulative errors, making it impossible to locate $\boldsymbol{x}_0$. In contrast, our algorithm produces more direct trajectories, allowing us to find $\boldsymbol{x}_0$ with fewer NFEs.

\begin{figure*}[ht]
    \centering
    \begin{minipage}[t]{0.45\linewidth}
        \centering
        \includegraphics[width=\linewidth, trim=0 0 0 0]{figs/convergence_a.pdf} %trim左下右上
        \subcaption{Sampling results.}
        \label{fig:convergence(a)}
    \end{minipage}
    \begin{minipage}[t]{0.43\linewidth}
        \centering
        \includegraphics[width=\linewidth, trim=0 20 0 0]{figs/convergence_b.pdf} %trim左下右上
        \subcaption{Ratio of convergence.}
        \label{fig:convergence(b)}
    \end{minipage}
    \caption{\textbf{Convergence of noise prediction and data prediction.} In (a), we choose a low-light image for example. The numbers in parentheses indicate the NFE. In (b), we illustrate the ratio of components of neural network output that satisfy the Taylor expansion convergence requirement.}
    \label{fig:converge}
\end{figure*}

\textbf{Numerical stability of parameterizations.}~ From Table 1, we observe poor sampling results for noise prediction in the case of few NFEs. The reason may be that the neural network parameterized by noise prediction is numerically unstable. Recall that we used Taylor expansion in Eq.(\ref{14}), and the condition for the equality to hold is $|\lambda-\lambda_s|<\boldsymbol{R}(s)$. And the radius of convergence $\boldsymbol{R}(t)$ can be calculated by
\begin{equation}
\frac{1}{\boldsymbol{R}(t)}=\lim_{n\rightarrow\infty}\left|\frac{\boldsymbol{c}_{n+1}(t)}{\boldsymbol{c}_n(t)}\right|,
\end{equation}
where $\boldsymbol{c}_n(t)$ is the coefficient of the $n$-th term in Taylor expansion. We are unable to compute this limit and can only compute the $n=0$ case as an approximation. The output of the neural network can be viewed as a vector, with each component corresponding to a radius of convergence. At each time step, we count the ratio of components that satisfy $\boldsymbol{R}_i(s)>|\lambda-\lambda_s|$ as a criterion for judging the convergence, where $i$ denotes the $i$-th component. As shown in Figure \ref{fig:converge}, the neural network parameterized by data prediction meets the convergence criteria at almost every step. However, the neural network parameterized by noise prediction always has components that cannot converge, which will lead to large errors and failed sampling. Therefore, data prediction has better numerical stability and is a more recommended choice.


\paragraph{Summary}
Our findings provide significant insights into the influence of correctness, explanations, and refinement on evaluation accuracy and user trust in AI-based planners. 
In particular, the findings are three-fold: 
(1) The \textbf{correctness} of the generated plans is the most significant factor that impacts the evaluation accuracy and user trust in the planners. As the PDDL solver is more capable of generating correct plans, it achieves the highest evaluation accuracy and trust. 
(2) The \textbf{explanation} component of the LLM planner improves evaluation accuracy, as LLM+Expl achieves higher accuracy than LLM alone. Despite this improvement, LLM+Expl minimally impacts user trust. However, alternative explanation methods may influence user trust differently from the manually generated explanations used in our approach.
% On the other hand, explanations may help refine the trust of the planner to a more appropriate level by indicating planner shortcomings.
(3) The \textbf{refinement} procedure in the LLM planner does not lead to a significant improvement in evaluation accuracy; however, it exhibits a positive influence on user trust that may indicate an overtrust in some situations.
% This finding is aligned with prior works showing that iterative refinements based on user feedback would increase user trust~\cite{kunkel2019let, sebo2019don}.
Finally, the propensity-to-trust analysis identifies correctness as the primary determinant of user trust, whereas explanations provided limited improvement in scenarios where the planner's accuracy is diminished.

% In conclusion, our results indicate that the planner's correctness is the dominant factor for both evaluation accuracy and user trust. Therefore, selecting high-quality training data and optimizing the training procedure of AI-based planners to improve planning correctness is the top priority. Once the AI planner achieves a similar correctness level to traditional graph-search planners, strengthening its capability to explain and refine plans will further improve user trust compared to traditional planners.

\paragraph{Future Research} Future steps in this research include expanding user studies with larger sample sizes to improve generalizability and including additional planning problems per session for a more comprehensive evaluation. Next, we will explore alternative methods for generating plan explanations beyond manual creation to identify approaches that more effectively enhance user trust. 
Additionally, we will examine user trust by employing multiple LLM-based planners with varying levels of planning accuracy to better understand the interplay between planning correctness and user trust. 
Furthermore, we aim to enable real-time user-planner interaction, allowing users to provide feedback and refine plans collaboratively, thereby fostering a more dynamic and user-centric planning process.


\bibliographystyle{IEEEtran}
\bibliography{ref}


\end{document}


