% This must be in the first 5 lines to tell arXiv to use pdfLaTeX, which is strongly recommended.
\pdfoutput=1
% In particular, the hyperref package requires pdfLaTeX in order to break URLs across lines.

\documentclass[11pt]{article}

% Change "review" to "final" to generate the final (sometimes called camera-ready) version.
% Change to "preprint" to generate a non-anonymous version with page numbers.
\usepackage[preprint]{acl}

% Standard package includes
\usepackage{times}
\usepackage{latexsym}
\usepackage{multirow}
\usepackage{url}
\usepackage{tcolorbox}  % For colored boxes
\usepackage{xcolor}  % For custom colors
\usepackage{amsmath}
\tcbuselibrary{skins}
\usepackage{subcaption}
\usepackage[multiple]{footmisc}
% For proper rendering and hyphenation of words containing Latin characters (including in bib files)
\usepackage[T1]{fontenc}
% For Vietnamese characters
% \usepackage[T5]{fontenc}
% See https://www.latex-project.org/help/documentation/encguide.pdf for other character sets

% This assumes your files are encoded as UTF8
\usepackage[utf8]{inputenc}

% This is not strictly necessary, and may be commented out,
% but it will improve the layout of the manuscript,
% and will typically save some space.
\usepackage{microtype}

% This is also not strictly necessary, and may be commented out.
% However, it will improve the aesthetics of text in
% the typewriter font.
\usepackage{inconsolata}

%Including images in your LaTeX document requires adding
%additional package(s)
\usepackage{graphicx}

\newtcolorbox{mybox}
{
  enhanced jigsaw,
  drop shadow=black!50!white,
  colback=white
}

\newcommand{\smk}[1]{{\color{red}Shoumik: #1}}
\newcommand{\sfeizi}[1]{{\color{blue}Soheil: #1}}

% \newtcolorbox[systembox]{
%   colback=gray!10,  % Light gray background
%   colframe=black!50,  % Dark gray border
%   title=System Prompt,
%   fonttitle=\bfseries,
%   sharp corners
% }

% If the title and author information does not fit in the area allocated, uncomment the following
%
%\setlength\titlebox{<dim>}
%
% and set <dim> to something 5cm or larger.

\title{Almost AI, Almost Human:\\ The Challenge of Detecting AI-Polished Writing}

% Author information can be set in various styles:
% For several authors from the same institution:
% \author{Shoumik Saha \and Soheil Feizi \\
%         Address line \\ ... \\ Address line}
% if the names do not fit well on one line use
%         Author 1 \\ {\bf Author 2} \\ ... \\ {\bf Author n} \\
% For authors from different institutions:
% \author{Author 1 \\ Address line \\  ... \\ Address line
%         \And  ... \And
%         Author n \\ Address line \\ ... \\ Address line}
% To start a separate ``row'' of authors use \AND, as in
% \author{Author 1 \\ Address line \\  ... \\ Address line
%         \AND
%         Author 2 \\ Address line \\ ... \\ Address line \And
%         Author 3 \\ Address line \\ ... \\ Address line}

\author{Shoumik Saha \\
  University of Maryland \\
  College Park, USA \\
  \texttt{smksaha@umd.edu} \\\And
  Soheil Feizi \\
  University of Maryland\\
  College Park, USA \\
  \texttt{sfeizi@umd.edu} \\}

%\author{
%  \textbf{First Author\textsuperscript{1}},
%  \textbf{Second Author\textsuperscript{1,2}},
%  \textbf{Third T. Author\textsuperscript{1}},
%  \textbf{Fourth Author\textsuperscript{1}},
%\\
%  \textbf{Fifth Author\textsuperscript{1,2}},
%  \textbf{Sixth Author\textsuperscript{1}},
%  \textbf{Seventh Author\textsuperscript{1}},
%  \textbf{Eighth Author \textsuperscript{1,2,3,4}},
%\\
%  \textbf{Ninth Author\textsuperscript{1}},
%  \textbf{Tenth Author\textsuperscript{1}},
%  \textbf{Eleventh E. Author\textsuperscript{1,2,3,4,5}},
%  \textbf{Twelfth Author\textsuperscript{1}},
%\\
%  \textbf{Thirteenth Author\textsuperscript{3}},
%  \textbf{Fourteenth F. Author\textsuperscript{2,4}},
%  \textbf{Fifteenth Author\textsuperscript{1}},
%  \textbf{Sixteenth Author\textsuperscript{1}},
%\\
%  \textbf{Seventeenth S. Author\textsuperscript{4,5}},
%  \textbf{Eighteenth Author\textsuperscript{3,4}},
%  \textbf{Nineteenth N. Author\textsuperscript{2,5}},
%  \textbf{Twentieth Author\textsuperscript{1}}
%\\
%\\
%  \textsuperscript{1}Affiliation 1,
%  \textsuperscript{2}Affiliation 2,
%  \textsuperscript{3}Affiliation 3,
%  \textsuperscript{4}Affiliation 4,
%  \textsuperscript{5}Affiliation 5
%\\
%  \small{
%    \textbf{Correspondence:} \href{mailto:email@domain}{email@domain}
%  }
%}

\begin{document}
\maketitle
\begin{abstract}
The growing use of large language models (LLMs) for text generation has led to widespread concerns about AI-generated content detection. However, an overlooked challenge is AI-polished text, where human-written content undergoes subtle refinements using AI tools. This raises a critical question: should minimally polished text be classified as AI-generated? Misclassification can lead to false plagiarism accusations and misleading claims about AI prevalence in online content. In this study, we systematically evaluate eleven state-of-the-art AI-text detectors using our AI-Polished-Text Evaluation \textbf{(APT-Eval) dataset}, which contains $11.7K$ samples refined at varying AI-involvement levels. Our findings reveal that detectors frequently misclassify even minimally polished text as AI-generated, struggle to differentiate between degrees of AI involvement, and exhibit biases against older and smaller models. These limitations highlight the urgent need for more nuanced detection methodologies.

%The widespread adoption of large language models (LLMs) has introduced a growing challenge for AI-text detectors: distinguishing between fully AI-generated text and AI-polished human-written content. While existing detectors are designed to classify text as either human-written or AI-generated, they struggle with content that has been refined by AI tools. This study systematically evaluates the performance of ten state-of-the-art AI-text detectors with AI-polished text. We introduce the AI-Polished-Text Evaluation (APT-Eval) dataset, consisting of 11.7K samples of human-authored text refined at varying levels using different LLMs. Our findings reveal that detectors exhibit alarmingly high false positive rates, frequently misclassifying even minimally polished text as AI-generated. Moreover, most detectors fail to distinguish between minor and major AI polishing and demonstrate biases against older and smaller AI models. These results highlight critical gaps in current AI-text detectors and call for more nuanced approaches to assessing AI-assisted writing.

% This document is a supplement to the general instructions for *ACL authors. It contains instructions for using the \LaTeX{} style files for ACL conferences.
% The document itself conforms to its own specifications, and is therefore an example of what your manuscript should look like.
% These instructions should be used both for papers submitted for review and for final versions of accepted papers.
\end{abstract}

\section{Introduction}
\section{Introduction}


\begin{figure}[t]
\centering
\includegraphics[width=0.6\columnwidth]{figures/evaluation_desiderata_V5.pdf}
\vspace{-0.5cm}
\caption{\systemName is a platform for conducting realistic evaluations of code LLMs, collecting human preferences of coding models with real users, real tasks, and in realistic environments, aimed at addressing the limitations of existing evaluations.
}
\label{fig:motivation}
\end{figure}

\begin{figure*}[t]
\centering
\includegraphics[width=\textwidth]{figures/system_design_v2.png}
\caption{We introduce \systemName, a VSCode extension to collect human preferences of code directly in a developer's IDE. \systemName enables developers to use code completions from various models. The system comprises a) the interface in the user's IDE which presents paired completions to users (left), b) a sampling strategy that picks model pairs to reduce latency (right, top), and c) a prompting scheme that allows diverse LLMs to perform code completions with high fidelity.
Users can select between the top completion (green box) using \texttt{tab} or the bottom completion (blue box) using \texttt{shift+tab}.}
\label{fig:overview}
\end{figure*}

As model capabilities improve, large language models (LLMs) are increasingly integrated into user environments and workflows.
For example, software developers code with AI in integrated developer environments (IDEs)~\citep{peng2023impact}, doctors rely on notes generated through ambient listening~\citep{oberst2024science}, and lawyers consider case evidence identified by electronic discovery systems~\citep{yang2024beyond}.
Increasing deployment of models in productivity tools demands evaluation that more closely reflects real-world circumstances~\citep{hutchinson2022evaluation, saxon2024benchmarks, kapoor2024ai}.
While newer benchmarks and live platforms incorporate human feedback to capture real-world usage, they almost exclusively focus on evaluating LLMs in chat conversations~\citep{zheng2023judging,dubois2023alpacafarm,chiang2024chatbot, kirk2024the}.
Model evaluation must move beyond chat-based interactions and into specialized user environments.



 

In this work, we focus on evaluating LLM-based coding assistants. 
Despite the popularity of these tools---millions of developers use Github Copilot~\citep{Copilot}---existing
evaluations of the coding capabilities of new models exhibit multiple limitations (Figure~\ref{fig:motivation}, bottom).
Traditional ML benchmarks evaluate LLM capabilities by measuring how well a model can complete static, interview-style coding tasks~\citep{chen2021evaluating,austin2021program,jain2024livecodebench, white2024livebench} and lack \emph{real users}. 
User studies recruit real users to evaluate the effectiveness of LLMs as coding assistants, but are often limited to simple programming tasks as opposed to \emph{real tasks}~\citep{vaithilingam2022expectation,ross2023programmer, mozannar2024realhumaneval}.
Recent efforts to collect human feedback such as Chatbot Arena~\citep{chiang2024chatbot} are still removed from a \emph{realistic environment}, resulting in users and data that deviate from typical software development processes.
We introduce \systemName to address these limitations (Figure~\ref{fig:motivation}, top), and we describe our three main contributions below.


\textbf{We deploy \systemName in-the-wild to collect human preferences on code.} 
\systemName is a Visual Studio Code extension, collecting preferences directly in a developer's IDE within their actual workflow (Figure~\ref{fig:overview}).
\systemName provides developers with code completions, akin to the type of support provided by Github Copilot~\citep{Copilot}. 
Over the past 3 months, \systemName has served over~\completions suggestions from 10 state-of-the-art LLMs, 
gathering \sampleCount~votes from \userCount~users.
To collect user preferences,
\systemName presents a novel interface that shows users paired code completions from two different LLMs, which are determined based on a sampling strategy that aims to 
mitigate latency while preserving coverage across model comparisons.
Additionally, we devise a prompting scheme that allows a diverse set of models to perform code completions with high fidelity.
See Section~\ref{sec:system} and Section~\ref{sec:deployment} for details about system design and deployment respectively.



\textbf{We construct a leaderboard of user preferences and find notable differences from existing static benchmarks and human preference leaderboards.}
In general, we observe that smaller models seem to overperform in static benchmarks compared to our leaderboard, while performance among larger models is mixed (Section~\ref{sec:leaderboard_calculation}).
We attribute these differences to the fact that \systemName is exposed to users and tasks that differ drastically from code evaluations in the past. 
Our data spans 103 programming languages and 24 natural languages as well as a variety of real-world applications and code structures, while static benchmarks tend to focus on a specific programming and natural language and task (e.g. coding competition problems).
Additionally, while all of \systemName interactions contain code contexts and the majority involve infilling tasks, a much smaller fraction of Chatbot Arena's coding tasks contain code context, with infilling tasks appearing even more rarely. 
We analyze our data in depth in Section~\ref{subsec:comparison}.



\textbf{We derive new insights into user preferences of code by analyzing \systemName's diverse and distinct data distribution.}
We compare user preferences across different stratifications of input data (e.g., common versus rare languages) and observe which affect observed preferences most (Section~\ref{sec:analysis}).
For example, while user preferences stay relatively consistent across various programming languages, they differ drastically between different task categories (e.g. frontend/backend versus algorithm design).
We also observe variations in user preference due to different features related to code structure 
(e.g., context length and completion patterns).
We open-source \systemName and release a curated subset of code contexts.
Altogether, our results highlight the necessity of model evaluation in realistic and domain-specific settings.






\section{APT(AI-Polished-Text) Eval Dataset}
\section{Dataset}
\label{sec:dataset}

\subsection{Data Collection}

To analyze political discussions on Discord, we followed the methodology in \cite{singh2024Cross-Platform}, collecting messages from politically-oriented public servers in compliance with Discord's platform policies.

Using Discord's Discovery feature, we employed a web scraper to extract server invitation links, names, and descriptions, focusing on public servers accessible without participation. Invitation links were used to access data via the Discord API. To ensure relevance, we filtered servers using keywords related to the 2024 U.S. elections (e.g., Trump, Kamala, MAGA), as outlined in \cite{balasubramanian2024publicdatasettrackingsocial}. This resulted in 302 server links, further narrowed to 81 English-speaking, politics-focused servers based on their names and descriptions.

Public messages were retrieved from these servers using the Discord API, collecting metadata such as \textit{content}, \textit{user ID}, \textit{username}, \textit{timestamp}, \textit{bot flag}, \textit{mentions}, and \textit{interactions}. Through this process, we gathered \textbf{33,373,229 messages} from \textbf{82,109 users} across \textbf{81 servers}, including \textbf{1,912,750 messages} from \textbf{633 bots}. Data collection occurred between November 13th and 15th, covering messages sent from January 1st to November 12th, just after the 2024 U.S. election.

\subsection{Characterizing the Political Spectrum}
\label{sec:timeline}

A key aspect of our research is distinguishing between Republican- and Democratic-aligned Discord servers. To categorize their political alignment, we relied on server names and self-descriptions, which often include rules, community guidelines, and references to key ideologies or figures. Each server's name and description were manually reviewed based on predefined, objective criteria, focusing on explicit political themes or mentions of prominent figures. This process allowed us to classify servers into three categories, ensuring a systematic and unbiased alignment determination.

\begin{itemize}
    \item \textbf{Republican-aligned}: Servers referencing Republican and right-wing and ideologies, movements, or figures (e.g., MAGA, Conservative, Traditional, Trump).  
    \item \textbf{Democratic-aligned}: Servers mentioning Democratic and left-wing ideologies, movements, or figures (e.g., Progressive, Liberal, Socialist, Biden, Kamala).  
    \item \textbf{Unaligned}: Servers with no defined spectrum and ideologies or opened to general political debate from all orientations.
\end{itemize}

To ensure the reliability and consistency of our classification, three independent reviewers assessed the classification following the specified set of criteria. The inter-rater agreement of their classifications was evaluated using Fleiss' Kappa \cite{fleiss1971measuring}, with a resulting Kappa value of \( 0.8191 \), indicating an almost perfect agreement among the reviewers. Disagreements were resolved by adopting the majority classification, as there were no instances where a server received different classifications from all three reviewers. This process guaranteed the consistency and accuracy of the final categorization.

Through this process, we identified \textbf{7 Republican-aligned servers}, \textbf{9 Democratic-aligned servers}, and \textbf{65 unaligned servers}.

Table \ref{tab:statistics} shows the statistics of the collected data. Notably, while Democratic- and Republican-aligned servers had a comparable number of user messages, users in the latter servers were significantly more active, posting more than double the number of messages per user compared to their Democratic counterparts. 
This suggests that, in our sample, Democratic-aligned servers attract more users, but these users were less engaged in text-based discussions. Additionally, around 10\% of the messages across all server categories were posted by bots. 

\subsection{Temporal Data} 

Throughout this paper, we refer to the election candidates using the names adopted by their respective campaigns: \textit{Kamala}, \textit{Biden}, and \textit{Trump}. To examine how the content of text messages evolves based on the political alignment of servers, we divided the 2024 election year into three periods: \textbf{Biden vs Trump} (January 1 to July 21), \textbf{Kamala vs Trump} (July 21 to September 20), and the \textbf{Voting Period} (after September 20). These periods reflect key phases of the election: the early campaign dominated by Biden and Trump, the shift in dynamics with Kamala Harris replacing Joe Biden as the Democratic candidate, and the final voting stage focused on electoral outcomes and their implications. This segmentation enables an analysis of how discourse responds to pivotal electoral moments.

Figure \ref{fig:line-plot} illustrates the distribution of messages over time, highlighting trends in total messages volume and mentions of each candidate. Prior to Biden's withdrawal on July 21, mentions of Biden and Trump were relatively balanced. However, following Kamala's entry into the race, mentions of Trump surged significantly, a trend further amplified by an assassination attempt on him, solidifying his dominance in the discourse. The only instance where Trump’s mentions were exceeded occurred during the first debate, as concerns about Biden’s age and cognitive abilities temporarily shifted the focus. In the final stages of the election, mentions of all three candidates rose, with Trump’s mentions peaking as he emerged as the victor.

\section{AI-text Detectors}
\begin{table}[t]
\centering
\small
\setlength{\tabcolsep}{4pt}
\begin{tabular}{lcccccc}
\toprule
\multirow{2}{*}{\bf Model}& \multicolumn{3}{c}{\bf in-distribution} & \multicolumn{3}{c}{\bf out-of-distribution}\\
& AUC & Acc. & F1 & AUC & Acc. & F1 \\
\midrule
\texttt{RoBERTa}      & $.982$ & $.940$ & $.940$ & $.846$ & $.806$ & $.804$\\
\texttt{DeBERTav3}    & $.971$ & $\mathbf{.954}$ & $\mathbf{.954}$ & $.817$ & $.812$ & $.810$ \\
\texttt{ModernBERT}   & $\mathbf{.986}$ & $.948$ & $.948$ & $\mathbf{.943}$ & $\mathbf{.861}$ & $\mathbf{.860}$ \\
\bottomrule
\end{tabular}
\caption{Machine-generated text detection performance. Accuracy (Acc.) and F1-score are macro-averages.}
\label{tab:ai_detectors}
\end{table}

\section{Key Findings}
\section{Analysis} \label{sec:analysis}
In this section, we provide a comprehensive analysis of Satori. First, we demonstrate that Satori effectively leverages self-reflection to seek better solutions and enhance its overall reasoning performance. Next, we observe that Satori exhibits test-scaling behavior through RL training, where it progressively acquires more tokens to improve its reasoning capabilities. Finally, we conduct ablation studies on various components of Satori's training framework. Additional results are provided in Appendix~\ref{app:results}.



\paragraph{COAT Reasoning v.s. CoT Reasoning.}
\begin{table}[h]
  \begin{center}
  \scriptsize
  \captionsetup{font=small}
  \caption{\textbf{COAT Training v.s. CoT Training.} Qwen-2.5-Math-7B trained with COAT reasoning format (Satori-Qwen-7B) outperforms the same base model but trained with classical CoT reasoning format (Qwen-7B-CoT)}
  \setlength{\tabcolsep}{1.3pt}
  \begin{tabular}{cccccccccc}
    \toprule
    \textbf{Model} & \textbf{GSM8K} & \textbf{MATH500}  &  \textbf{Olym.} & \textbf{AMC2023} & \textbf{AIME2024} \\
    \midrule
    Qwen-2.5-Math-7B-Instruct & 95.2 & 83.6 &41.6& 62.5 &16.7 \\
    Qwen-7B-CoT (SFT+RL) & 93.1 & 84.4  &	42.7 &	60.0 & 10.0 \\
    \midrule
    \textbf{Satori-Qwen-7B}  & 93.2 & 85.6  & 46.6  & 67.5  & 20.0 \\
    \bottomrule
  \end{tabular}
  \label{table:ablation-coat}
  \end{center}
\vspace{-1em}
\end{table}
We begin by conducting an ablation study to demonstrate the benefits of COAT reasoning compared to the classical CoT reasoning. Specifically, starting from the synthesis of demonstration trajectories in the format tuning stage, we ablate the ``reflect'' and  ``explore'' actions, retaining only the ``continue'' actions. Next, we maintain all other training settings, including the same amount of SFT and RL data and consistent hyper-parameters. This results in a typical CoT LLM (Qwen-7B-CoT) without self-reflection or self-exploration capabilities. As shown in Table~\ref{table:ablation-coat}, the performance of Qwen-7B-CoT is suboptimal compared to Satori-Qwen-7B and fails to surpass Qwen-2.5-Math-7B-Instruct, suggesting the advantages of COAT reasoning over CoT reasoning.



\paragraph{Satori Exhibits Self-correction Capability.}
% Please add the following required packages to your document preamble:
% \usepackage{multirow}
\begin{table}[h]
\scriptsize
\captionsetup{font=small}
\caption{\textbf{Satori's Self-correction Capability.} T$\rightarrow$F: negative self-correction; F$\rightarrow$T: positive self-correction.}
\setlength{\tabcolsep}{5pt}
\begin{tabular}{lcccccc}
\toprule
\multirow{3}{*}{\textbf{Model}} & \multicolumn{4}{c}{\textbf{In-Domain}}                                                                            & \multicolumn{2}{c}{\textbf{Out-of-Domain}}              \\ \cmidrule[0.2pt]{2-7} 
                                & \multicolumn{2}{c}{\textbf{MATH500}}                    & \multicolumn{2}{c}{\textbf{OlympiadBench}}              & \multicolumn{2}{c}{\textbf{MMLUProSTEM}}         \\
                                & \textbf{T$\rightarrow$F} & \textbf{F$\rightarrow$T} & \textbf{T$\rightarrow$F} & \textbf{F$\rightarrow$T} & \textbf{T$\rightarrow$F} & \textbf{F$\rightarrow$T} \\ \midrule[0.5pt]
Satori-Qwen-7B-FT                  & 79.4\%                    & 20.6\%                    & 65.6\%                    & 34.4\%                    & 59.2\%                    & 40.8\%                    \\
\textbf{Satori-Qwen-7B}                     & 39.0\%                       & 61.0\%                       & 42.1\%                    & 57.9\%                    & 46.5\%                    & 53.5\%                    \\ \bottomrule
\end{tabular}
\label{table:finegrain-reflect}
\end{table}
We observe that Satori frequently engages in self-reflection during the reasoning process (see demos in Section~\ref{sec:demo}), which occurs in two scenarios: (1) it triggers self-reflection at intermediate reasoning steps, and (2) after completing a problem, it initiates a second attempt through self-reflection. We focus on quantitatively evaluating Satori's self-correction capability in the second scenario. Specifically, we extract responses where the final answer before self-reflection differs from the answer after self-reflection. We then quantify the percentage of responses in which Satori's self-correction is positive (i.e., the solution is corrected from incorrect to correct) or negative (i.e., the solution changes from correct to incorrect). The evaluation results on in-domain datasets (MATH500 and Olympiad) and out-of-domain datasets (MMLUPro) are presented in Table~\ref{table:finegrain-reflect}. First, compared to Satori-Qwen-FT which lacks the RL training stage, Satori-Qwen demonstrates a significantly stronger self-correction capability. Second, we observe that this self-correction capability extends to out-of-domain tasks (MMLUProSTEM). These results suggest that RL plays a crucial role in enhancing the model's true reasoning capabilities.


\paragraph{RL Enables Satori with Test-time Scaling Behavior.}
\begin{figure}[h]
    \centering
    \includegraphics[width=0.5\textwidth]{Figures/rm_shaping_tot_len.pdf}
    \vspace{-2em}
\caption{\textbf{Policy Training Acc. \& Response length v.s. RL Train-time Compute.} Through RL training, Satori learns to improve its reasoning performance through longer thinking.}
\label{fig:test_time_scaling}
\end{figure}
\begin{figure}[h]
    \centering
    \includegraphics[width=0.45\textwidth]{Figures/length_across_levels.pdf}
    \vspace{-1.5em}
\caption{\textbf{Above: Test-time Response Length v.s. Problem Difficulty Level; Below: Test-time Accuracy v.s. Problem Difficulty Level.} Compared to FT model (Satori-Qwen-FT), Satori-Qwen uses more test-time compute to tackle more challenging problems.}
\label{fig:difficulty_level}
\vspace{-1em}
\end{figure}

Next, we aim to explain how reinforcement learning (RL) incentivizes Satori's autoregressive search capability. First, as shown in Figure~\ref{fig:test_time_scaling}, we observe that Satori consistently improves policy accuracy and increases the average length of generated tokens with more RL training-time compute. This suggests that Satori learns to allocate more time to reasoning, thereby solving problems more accurately. One interesting observation is that the response length first decreases from 0 to 200 steps and then increases. Upon a closer investigation of the model response, we observe that in the early stage, our model has not yet learned self-reflection capabilities. During this stage, RL optimization may prioritize the model to find a shot-cut solution without redundant reflection, leading to a temporary reduction in response length. However, in later stage, the model becomes increasingly good at using reflection to self-correct and find a better solution, leading to a longer response length.
 
Additionally, in Figure~\ref{fig:difficulty_level}, we evaluate Satori's test accuracy and response length on MATH datasets across different difficulty levels. Interestingly, through RL training, Satori naturally allocates more test-time compute to tackle more challenging problems, which leads to consistent performance improvements compared to the format-tuned (FT) model.



\paragraph{Large-scale FT v.s. Large-scale RL.}
\begin{table}[h]
  \begin{center}
  \scriptsize
  \captionsetup{font=small}
  \caption{\textbf{Large-scale FT V.S. Large-scale RL} Satori-Qwen (10K FT data + 300K RL data) outperforms same base model Qwen-2.5-Math-7B trained with 300K FT data (w/o RL) across all math and out-of-domain benchmarks.}
  \setlength{\tabcolsep}{1.15pt}
  \vspace{-0.5em}
\begin{tabular}{lccccc}
\toprule
\textbf{(In-domain)}   & \textbf{GSM8K}   & \textbf{MATH500} & \textbf{Olym.} & \textbf{AMC2023} & \textbf{AIME2024} \\ \midrule
Qwen-2.5-Math-7B-Instruct & 95.2 & 83.6                     & 41.6                  & 62.5             & 16.7                 \\
Satori-Qwen-7B-FT (300K)     & 92.3 & 78.2                       & 40.9           & 65.0               & 16.7              \\
\textbf{Satori-Qwen-7B}         & 93.2        & 85.6                     & 46.6           & 67.5             & 20.0                \\ \midrule
\textbf{(Out-of-domain)}  & \textbf{BGQA}    & \textbf{CRUX}  & \textbf{STGQA} & \textbf{TableBench}   & \textbf{STEM}     \\ \midrule
Qwen-2.5-Math-7B-Instruct & 51.3             & 28.0             & 85.3           & 36.3             & 45.2              \\
Satori-Qwen-7B-FT (300K)     & 50.5             & 29.5           & 74.0             & 35.0               & 47.8              \\
\textbf{Satori-Qwen-7B}               & 61.8             & 42.5           & 86.3           & 43.4             & 56.7              \\ \bottomrule
\end{tabular}
  \label{table:ablation-ft-rl}
  \end{center}
\end{table}
We investigate whether scaling up format tuning (FT) can achieve performance gains comparable to RL training. We conduct an ablation study using Qwen-2.5-Math-7B, trained with an equivalent amount of FT data (300K). As shown in Table~\ref{table:ablation-ft-rl}, on the math domain benchmarks, the model trained with large-scale FT (300K) fails to match the performance of the model trained with small-scale FT (10K) and large-scale RL (300K). Additionally, the large-scale FT model performs significantly worse on out-of-domain tasks, demonstrates RL’s advantage in generalization.


\paragraph{Distillation Enables Weak-to-Strong Generalization.} 
\begin{figure}[!t]
    \centering
     \includegraphics[width=0.4\textwidth]
     {Figures/distillation.pdf}
     \vspace{-1.5em}
\caption{\textbf{Format Tuning v.s. Distillation.} Distilling from a Stronger model (Satori-Qwen-7B) to weaker base models (Llama-8B and Granite-8B) are more effective than directly applying format tuning on weaker base models.}
\label{fig:distill}
\vspace{-1em}
\end{figure}
Finally, we investigate whether distilling a stronger reasoning model can enhance the reasoning performance of weaker base models. Specifically, we use Satori-Qwen-7B to generate 240K synthetic data to train weaker base models, Llama-3.1-8B and Granite-3.1-8B. For comparison, we also synthesize 240K FT data (following Section~\ref{subsec:format}) to train the same models. We evaluate the average test accuracy of these models across all math benchmark datasets, with the results presented in Figure~\ref{fig:distill}. The results show that the distilled models outperform the format-tuned models. 

This suggests a new, efficient approach to improve the reasoning capabilities of weaker base models: (1) train a strong reasoning model through small-scale
FT and large-scale RL (our Satori-Qwen-7B) and (2) distill the strong reasoning capabilities of the model into weaker base models. Since RL only requires answer labels as supervision, this approach introduces minimal costs for data synthesis, i.e., the costs induced by a multi-agent data synthesis framework or even more expensive human annotation.




\section{Related Work}
\section{RELATED WORK}
\label{sec:relatedwork}
In this section, we describe the previous works related to our proposal, which are divided into two parts. In Section~\ref{sec:relatedwork_exoplanet}, we present a review of approaches based on machine learning techniques for the detection of planetary transit signals. Section~\ref{sec:relatedwork_attention} provides an account of the approaches based on attention mechanisms applied in Astronomy.\par

\subsection{Exoplanet detection}
\label{sec:relatedwork_exoplanet}
Machine learning methods have achieved great performance for the automatic selection of exoplanet transit signals. One of the earliest applications of machine learning is a model named Autovetter \citep{MCcauliff}, which is a random forest (RF) model based on characteristics derived from Kepler pipeline statistics to classify exoplanet and false positive signals. Then, other studies emerged that also used supervised learning. \cite{mislis2016sidra} also used a RF, but unlike the work by \citet{MCcauliff}, they used simulated light curves and a box least square \citep[BLS;][]{kovacs2002box}-based periodogram to search for transiting exoplanets. \citet{thompson2015machine} proposed a k-nearest neighbors model for Kepler data to determine if a given signal has similarity to known transits. Unsupervised learning techniques were also applied, such as self-organizing maps (SOM), proposed \citet{armstrong2016transit}; which implements an architecture to segment similar light curves. In the same way, \citet{armstrong2018automatic} developed a combination of supervised and unsupervised learning, including RF and SOM models. In general, these approaches require a previous phase of feature engineering for each light curve. \par

%DL is a modern data-driven technology that automatically extracts characteristics, and that has been successful in classification problems from a variety of application domains. The architecture relies on several layers of NNs of simple interconnected units and uses layers to build increasingly complex and useful features by means of linear and non-linear transformation. This family of models is capable of generating increasingly high-level representations \citep{lecun2015deep}.

The application of DL for exoplanetary signal detection has evolved rapidly in recent years and has become very popular in planetary science.  \citet{pearson2018} and \citet{zucker2018shallow} developed CNN-based algorithms that learn from synthetic data to search for exoplanets. Perhaps one of the most successful applications of the DL models in transit detection was that of \citet{Shallue_2018}; who, in collaboration with Google, proposed a CNN named AstroNet that recognizes exoplanet signals in real data from Kepler. AstroNet uses the training set of labelled TCEs from the Autovetter planet candidate catalog of Q1–Q17 data release 24 (DR24) of the Kepler mission \citep{catanzarite2015autovetter}. AstroNet analyses the data in two views: a ``global view'', and ``local view'' \citep{Shallue_2018}. \par


% The global view shows the characteristics of the light curve over an orbital period, and a local view shows the moment at occurring the transit in detail

%different = space-based

Based on AstroNet, researchers have modified the original AstroNet model to rank candidates from different surveys, specifically for Kepler and TESS missions. \citet{ansdell2018scientific} developed a CNN trained on Kepler data, and included for the first time the information on the centroids, showing that the model improves performance considerably. Then, \citet{osborn2020rapid} and \citet{yu2019identifying} also included the centroids information, but in addition, \citet{osborn2020rapid} included information of the stellar and transit parameters. Finally, \citet{rao2021nigraha} proposed a pipeline that includes a new ``half-phase'' view of the transit signal. This half-phase view represents a transit view with a different time and phase. The purpose of this view is to recover any possible secondary eclipse (the object hiding behind the disk of the primary star).


%last pipeline applies a procedure after the prediction of the model to obtain new candidates, this process is carried out through a series of steps that include the evaluation with Discovery and Validation of Exoplanets (DAVE) \citet{kostov2019discovery} that was adapted for the TESS telescope.\par
%



\subsection{Attention mechanisms in astronomy}
\label{sec:relatedwork_attention}
Despite the remarkable success of attention mechanisms in sequential data, few papers have exploited their advantages in astronomy. In particular, there are no models based on attention mechanisms for detecting planets. Below we present a summary of the main applications of this modeling approach to astronomy, based on two points of view; performance and interpretability of the model.\par
%Attention mechanisms have not yet been explored in all sub-areas of astronomy. However, recent works show a successful application of the mechanism.
%performance

The application of attention mechanisms has shown improvements in the performance of some regression and classification tasks compared to previous approaches. One of the first implementations of the attention mechanism was to find gravitational lenses proposed by \citet{thuruthipilly2021finding}. They designed 21 self-attention-based encoder models, where each model was trained separately with 18,000 simulated images, demonstrating that the model based on the Transformer has a better performance and uses fewer trainable parameters compared to CNN. A novel application was proposed by \citet{lin2021galaxy} for the morphological classification of galaxies, who used an architecture derived from the Transformer, named Vision Transformer (VIT) \citep{dosovitskiy2020image}. \citet{lin2021galaxy} demonstrated competitive results compared to CNNs. Another application with successful results was proposed by \citet{zerveas2021transformer}; which first proposed a transformer-based framework for learning unsupervised representations of multivariate time series. Their methodology takes advantage of unlabeled data to train an encoder and extract dense vector representations of time series. Subsequently, they evaluate the model for regression and classification tasks, demonstrating better performance than other state-of-the-art supervised methods, even with data sets with limited samples.

%interpretation
Regarding the interpretability of the model, a recent contribution that analyses the attention maps was presented by \citet{bowles20212}, which explored the use of group-equivariant self-attention for radio astronomy classification. Compared to other approaches, this model analysed the attention maps of the predictions and showed that the mechanism extracts the brightest spots and jets of the radio source more clearly. This indicates that attention maps for prediction interpretation could help experts see patterns that the human eye often misses. \par

In the field of variable stars, \citet{allam2021paying} employed the mechanism for classifying multivariate time series in variable stars. And additionally, \citet{allam2021paying} showed that the activation weights are accommodated according to the variation in brightness of the star, achieving a more interpretable model. And finally, related to the TESS telescope, \citet{morvan2022don} proposed a model that removes the noise from the light curves through the distribution of attention weights. \citet{morvan2022don} showed that the use of the attention mechanism is excellent for removing noise and outliers in time series datasets compared with other approaches. In addition, the use of attention maps allowed them to show the representations learned from the model. \par

Recent attention mechanism approaches in astronomy demonstrate comparable results with earlier approaches, such as CNNs. At the same time, they offer interpretability of their results, which allows a post-prediction analysis. \par



\section{Conclusion}
\section{Conclusion}
In this work, we propose a simple yet effective approach, called SMILE, for graph few-shot learning with fewer tasks. Specifically, we introduce a novel dual-level mixup strategy, including within-task and across-task mixup, for enriching the diversity of nodes within each task and the diversity of tasks. Also, we incorporate the degree-based prior information to learn expressive node embeddings. Theoretically, we prove that SMILE effectively enhances the model's generalization performance. Empirically, we conduct extensive experiments on multiple benchmarks and the results suggest that SMILE significantly outperforms other baselines, including both in-domain and cross-domain few-shot settings.

\section*{Limitations}
\section*{Limitations and Ethical Considerations}

\noindent\textbf{Limitations.} The primary limitation of our work is that it extends only the dataset provided by MUSE and employs DeepSeek-v3 for question generation. 
To mitigate this generalization risk, we have released our code and the generated audit suite, allowing researchers to utilize our framework to create additional audit datasets and evaluate their quality. Meanwhile, this is also our future work to extend our framework to other benchmarks.

\noindent\textbf{Ethical Considerations.} Machine unlearning can be employed to mitigate risks associated with LLMs in terms of privacy, security, bias, and copyright. Our work is dedicated to providing a comprehensive evaluation framework to help researchers better understand the unlearning effectiveness of LLMs, which we believe will have a positive impact on society.

\section*{Acknowledgments}
\section{Acknowledgements}
\label{sec:ack}

XXXX

\iffalse
\section{Document Body}

\subsection{Footnotes}

Footnotes are inserted with the \verb|\footnote| command.\footnote{This is a footnote.}

\subsection{Tables and figures}

See Table~\ref{tab:accents} for an example of a table and its caption.
\textbf{Do not override the default caption sizes.}

\begin{table}
  \centering
  \begin{tabular}{lc}
    \hline
    \textbf{Command} & \textbf{Output} \\
    \hline
    \verb|{\"a}|     & {\"a}           \\
    \verb|{\^e}|     & {\^e}           \\
    \verb|{\`i}|     & {\`i}           \\
    \verb|{\.I}|     & {\.I}           \\
    \verb|{\o}|      & {\o}            \\
    \verb|{\'u}|     & {\'u}           \\
    \verb|{\aa}|     & {\aa}           \\\hline
  \end{tabular}
  \begin{tabular}{lc}
    \hline
    \textbf{Command} & \textbf{Output} \\
    \hline
    \verb|{\c c}|    & {\c c}          \\
    \verb|{\u g}|    & {\u g}          \\
    \verb|{\l}|      & {\l}            \\
    \verb|{\~n}|     & {\~n}           \\
    \verb|{\H o}|    & {\H o}          \\
    \verb|{\v r}|    & {\v r}          \\
    \verb|{\ss}|     & {\ss}           \\
    \hline
  \end{tabular}
  \caption{Example commands for accented characters, to be used in, \emph{e.g.}, Bib\TeX{} entries.}
  \label{tab:accents}
\end{table}

As much as possible, fonts in figures should conform
to the document fonts. See Figure~\ref{fig:experiments} for an example of a figure and its caption.

Using the \verb|graphicx| package graphics files can be included within figure
environment at an appropriate point within the text.
The \verb|graphicx| package supports various optional arguments to control the
appearance of the figure.
You must include it explicitly in the \LaTeX{} preamble (after the
\verb|\documentclass| declaration and before \verb|\begin{document}|) using
\verb|\usepackage{graphicx}|.

\begin{figure}[t]
  \includegraphics[width=\columnwidth]{example-image-golden}
  \caption{A figure with a caption that runs for more than one line.
    Example image is usually available through the \texttt{mwe} package
    without even mentioning it in the preamble.}
  \label{fig:experiments}
\end{figure}

\begin{figure*}[t]
  \includegraphics[width=0.48\linewidth]{example-image-a} \hfill
  \includegraphics[width=0.48\linewidth]{example-image-b}
  \caption {A minimal working example to demonstrate how to place
    two images side-by-side.}
\end{figure*}

\subsection{Hyperlinks}

Users of older versions of \LaTeX{} may encounter the following error during compilation:
\begin{quote}
\verb|\pdfendlink| ended up in different nesting level than \verb|\pdfstartlink|.
\end{quote}
This happens when pdf\LaTeX{} is used and a citation splits across a page boundary. The best way to fix this is to upgrade \LaTeX{} to 2018-12-01 or later.

\subsection{Citations}

\begin{table*}
  \centering
  \begin{tabular}{lll}
    \hline
    \textbf{Output}           & \textbf{natbib command} & \textbf{ACL only command} \\
    \hline
    \citep{Gusfield:97}       & \verb|\citep|           &                           \\
    \citealp{Gusfield:97}     & \verb|\citealp|         &                           \\
    \citet{Gusfield:97}       & \verb|\citet|           &                           \\
    \citeyearpar{Gusfield:97} & \verb|\citeyearpar|     &                           \\
    \citeposs{Gusfield:97}    &                         & \verb|\citeposs|          \\
    \hline
  \end{tabular}
  \caption{\label{citation-guide}
    Citation commands supported by the style file.
    The style is based on the natbib package and supports all natbib citation commands.
    It also supports commands defined in previous ACL style files for compatibility.
  }
\end{table*}

Table~\ref{citation-guide} shows the syntax supported by the style files.
We encourage you to use the natbib styles.
You can use the command \verb|\citet| (cite in text) to get ``author (year)'' citations, like this citation to a paper by \citet{Gusfield:97}.
You can use the command \verb|\citep| (cite in parentheses) to get ``(author, year)'' citations \citep{Gusfield:97}.
You can use the command \verb|\citealp| (alternative cite without parentheses) to get ``author, year'' citations, which is useful for using citations within parentheses (e.g. \citealp{Gusfield:97}).

A possessive citation can be made with the command \verb|\citeposs|.
This is not a standard natbib command, so it is generally not compatible
with other style files.

\subsection{References}

\nocite{Ando2005,andrew2007scalable,rasooli-tetrault-2015}

The \LaTeX{} and Bib\TeX{} style files provided roughly follow the American Psychological Association format.
If your own bib file is named \texttt{custom.bib}, then placing the following before any appendices in your \LaTeX{} file will generate the references section for you:
\begin{quote}
\begin{verbatim}
\bibliography{custom}
\end{verbatim}
\end{quote}

You can obtain the complete ACL Anthology as a Bib\TeX{} file from \url{https://aclweb.org/anthology/anthology.bib.gz}.
To include both the Anthology and your own .bib file, use the following instead of the above.
\begin{quote}
\begin{verbatim}
\bibliography{anthology,custom}
\end{verbatim}
\end{quote}

Please see Section~\ref{sec:bibtex} for information on preparing Bib\TeX{} files.

\subsection{Equations}

An example equation is shown below:
\begin{equation}
  \label{eq:example}
  A = \pi r^2
\end{equation}

Labels for equation numbers, sections, subsections, figures and tables
are all defined with the \verb|\label{label}| command and cross references
to them are made with the \verb|\ref{label}| command.

This an example cross-reference to Equation~\ref{eq:example}.

\subsection{Appendices}

Use \verb|\appendix| before any appendix section to switch the section numbering over to letters. See Appendix~\ref{sec:appendix} for an example.

\section{Bib\TeX{} Files}
\label{sec:bibtex}

Unicode cannot be used in Bib\TeX{} entries, and some ways of typing special characters can disrupt Bib\TeX's alphabetization. The recommended way of typing special characters is shown in Table~\ref{tab:accents}.

Please ensure that Bib\TeX{} records contain DOIs or URLs when possible, and for all the ACL materials that you reference.
Use the \verb|doi| field for DOIs and the \verb|url| field for URLs.
If a Bib\TeX{} entry has a URL or DOI field, the paper title in the references section will appear as a hyperlink to the paper, using the hyperref \LaTeX{} package.

\section*{Limitations}

Since December 2023, a "Limitations" section has been required for all papers submitted to ACL Rolling Review (ARR). This section should be placed at the end of the paper, before the references. The "Limitations" section (along with, optionally, a section for ethical considerations) may be up to one page and will not count toward the final page limit. Note that these files may be used by venues that do not rely on ARR so it is recommended to verify the requirement of a "Limitations" section and other criteria with the venue in question.

\section*{Acknowledgments}

This document has been adapted
by Steven Bethard, Ryan Cotterell and Rui Yan
from the instructions for earlier ACL and NAACL proceedings, including those for
ACL 2019 by Douwe Kiela and Ivan Vuli\'{c},
NAACL 2019 by Stephanie Lukin and Alla Roskovskaya,
ACL 2018 by Shay Cohen, Kevin Gimpel, and Wei Lu,
NAACL 2018 by Margaret Mitchell and Stephanie Lukin,
Bib\TeX{} suggestions for (NA)ACL 2017/2018 from Jason Eisner,
ACL 2017 by Dan Gildea and Min-Yen Kan,
NAACL 2017 by Margaret Mitchell,
ACL 2012 by Maggie Li and Michael White,
ACL 2010 by Jing-Shin Chang and Philipp Koehn,
ACL 2008 by Johanna D. Moore, Simone Teufel, James Allan, and Sadaoki Furui,
ACL 2005 by Hwee Tou Ng and Kemal Oflazer,
ACL 2002 by Eugene Charniak and Dekang Lin,
and earlier ACL and EACL formats written by several people, including
John Chen, Henry S. Thompson and Donald Walker.
Additional elements were taken from the formatting instructions of the \emph{International Joint Conference on Artificial Intelligence} and the \emph{Conference on Computer Vision and Pattern Recognition}.
\fi
% Bibliography entries for the entire Anthology, followed by custom entries
%\bibliography{anthology,custom}
% Custom bibliography entries only
\bibliography{main}

\newpage
\onecolumn

\appendix

\section{Dataset} \label{app:dataset}
\section{Dataset}
\label{sec:dataset}

\subsection{Data Collection}

To analyze political discussions on Discord, we followed the methodology in \cite{singh2024Cross-Platform}, collecting messages from politically-oriented public servers in compliance with Discord's platform policies.

Using Discord's Discovery feature, we employed a web scraper to extract server invitation links, names, and descriptions, focusing on public servers accessible without participation. Invitation links were used to access data via the Discord API. To ensure relevance, we filtered servers using keywords related to the 2024 U.S. elections (e.g., Trump, Kamala, MAGA), as outlined in \cite{balasubramanian2024publicdatasettrackingsocial}. This resulted in 302 server links, further narrowed to 81 English-speaking, politics-focused servers based on their names and descriptions.

Public messages were retrieved from these servers using the Discord API, collecting metadata such as \textit{content}, \textit{user ID}, \textit{username}, \textit{timestamp}, \textit{bot flag}, \textit{mentions}, and \textit{interactions}. Through this process, we gathered \textbf{33,373,229 messages} from \textbf{82,109 users} across \textbf{81 servers}, including \textbf{1,912,750 messages} from \textbf{633 bots}. Data collection occurred between November 13th and 15th, covering messages sent from January 1st to November 12th, just after the 2024 U.S. election.

\subsection{Characterizing the Political Spectrum}
\label{sec:timeline}

A key aspect of our research is distinguishing between Republican- and Democratic-aligned Discord servers. To categorize their political alignment, we relied on server names and self-descriptions, which often include rules, community guidelines, and references to key ideologies or figures. Each server's name and description were manually reviewed based on predefined, objective criteria, focusing on explicit political themes or mentions of prominent figures. This process allowed us to classify servers into three categories, ensuring a systematic and unbiased alignment determination.

\begin{itemize}
    \item \textbf{Republican-aligned}: Servers referencing Republican and right-wing and ideologies, movements, or figures (e.g., MAGA, Conservative, Traditional, Trump).  
    \item \textbf{Democratic-aligned}: Servers mentioning Democratic and left-wing ideologies, movements, or figures (e.g., Progressive, Liberal, Socialist, Biden, Kamala).  
    \item \textbf{Unaligned}: Servers with no defined spectrum and ideologies or opened to general political debate from all orientations.
\end{itemize}

To ensure the reliability and consistency of our classification, three independent reviewers assessed the classification following the specified set of criteria. The inter-rater agreement of their classifications was evaluated using Fleiss' Kappa \cite{fleiss1971measuring}, with a resulting Kappa value of \( 0.8191 \), indicating an almost perfect agreement among the reviewers. Disagreements were resolved by adopting the majority classification, as there were no instances where a server received different classifications from all three reviewers. This process guaranteed the consistency and accuracy of the final categorization.

Through this process, we identified \textbf{7 Republican-aligned servers}, \textbf{9 Democratic-aligned servers}, and \textbf{65 unaligned servers}.

Table \ref{tab:statistics} shows the statistics of the collected data. Notably, while Democratic- and Republican-aligned servers had a comparable number of user messages, users in the latter servers were significantly more active, posting more than double the number of messages per user compared to their Democratic counterparts. 
This suggests that, in our sample, Democratic-aligned servers attract more users, but these users were less engaged in text-based discussions. Additionally, around 10\% of the messages across all server categories were posted by bots. 

\subsection{Temporal Data} 

Throughout this paper, we refer to the election candidates using the names adopted by their respective campaigns: \textit{Kamala}, \textit{Biden}, and \textit{Trump}. To examine how the content of text messages evolves based on the political alignment of servers, we divided the 2024 election year into three periods: \textbf{Biden vs Trump} (January 1 to July 21), \textbf{Kamala vs Trump} (July 21 to September 20), and the \textbf{Voting Period} (after September 20). These periods reflect key phases of the election: the early campaign dominated by Biden and Trump, the shift in dynamics with Kamala Harris replacing Joe Biden as the Democratic candidate, and the final voting stage focused on electoral outcomes and their implications. This segmentation enables an analysis of how discourse responds to pivotal electoral moments.

Figure \ref{fig:line-plot} illustrates the distribution of messages over time, highlighting trends in total messages volume and mentions of each candidate. Prior to Biden's withdrawal on July 21, mentions of Biden and Trump were relatively balanced. However, following Kamala's entry into the race, mentions of Trump surged significantly, a trend further amplified by an assassination attempt on him, solidifying his dominance in the discourse. The only instance where Trump’s mentions were exceeded occurred during the first debate, as concerns about Biden’s age and cognitive abilities temporarily shifted the focus. In the final stages of the election, mentions of all three candidates rose, with Trump’s mentions peaking as he emerged as the victor.

\clearpage
\newpage
\section{AI-text Detectors} \label{app:detector}
We developed our code-base on the framework of RAID \cite{dugan2024raid}, and kept the hyperparameters for all detectors the same as RAID for a fair evaluation. 

Additionally, we identify the threshold corresponding to a $5\%$ false positive rate (FPR) -- or the lowest possible FPR if it exceeds $5\%$. We notice that -- for most detectors, the thresholds for `best accuracy' and `5\% FPR' do not vary much. Moreover, since our primary focus is on misclassification rates for both minor-polished and major-polished texts, optimizing the threshold for overall accuracy is more appropriate than minimizing FPR alone.

%In this work, we consider the threshold that gives the best accuracy. First, we take $100$ evenly spaced thresholds between minimum and maximum prediction by a detector. Then we evaluate the accuracy for each of them and finally select the one with the best accuracy. 
%Besides this, we also find the threshold that yields $5\%$ false positive rate (the lowest FPR if the lowest is more than $5\%$). We notice that -- for most detectors, the thresholds for `best accuracy' and `5\% FPR' do not vary much. Moreover, since in this work, we focus on misclassifying both the minor-polished and major-polished texts, it seems more rational to optimize the threshold over accuracy, rather than only minimizing the FPR.

Table \ref{tab:detector_threshold} shows the threshold that we found by optimizing the accuracy, and used for the evaluation of our APT-Eval dataset. 

\begin{table*}[h!]
\centering
\resizebox{0.7\textwidth}{!}{%
\begin{tabular}{lllll}
                                       & \textbf{Detector}             & \textbf{Threshold} & \textbf{Accuracy} & \textbf{FPR}    \\ \hline
\multirow{4}{*}{\textbf{Model-Based}}  & \textbf{RADAR}                & 0.8989             & 0.8017            & 0.082           \\
                                       & \textbf{RoBERTa (ChatGPT)}    & 0.333              & \textit{0.8617}   & \textit{0.0217} \\
                                       & \textbf{RoBERTa-Base (GPT2)}  & 0.091              & 0.7917            & 0.06            \\
                                       & \textbf{RoBERTa-Large (GPT2)} & 0.0408             & 0.8               & 0.0817          \\ \hline
\multirow{5}{*}{\textbf{Metric-Based}} & \textbf{GLTR}                 & 0.7038             & 0.845             & 0.0683          \\
                                       & \textbf{DetectGPT}            & 0.355              & 0.725             & 0.085           \\
                                       & \textbf{Fast-DetectGPT}       & 0.778              & 0.8317            & 0.06            \\
                                       & \textbf{LLMDet}               & 0.9798             & 0.605             & 0.2383          \\
                                       & \textbf{Binoculars}           & 0.1075             & \textbf{0.88}     & \textbf{0.018}  \\ \hline
\multirow{2}{*}{\textbf{Commercial}}   & \textbf{ZeroGPT}              & 0.2525             & 0.8067            & 0.0367          \\
                                       & \textbf{GPTZero}              & 0.03               & \textit{0.862}             & 0.075           \\ \hline
\end{tabular}%
}
\caption{Detector-based Threshold, Accuracy, and False Positive Rate. The best performance is in bold, and the second best is in italics.}
\label{tab:detector_threshold}
\end{table*}

%We used one NVIDIA RTXA5000 to run the model-based and metric-based detectors (one RTXA6000 GPU for `Binoculars' detector). For commercial detectors, we used their API subscription. 

For computational resources, we employed:
\begin{itemize}
    \item One NVIDIA RTX A5000 GPU for running model-based and metric-based detectors.
    \item One NVIDIA RTX A6000 GPU for the Binoculars detector.
    \item Commercial API subscriptions for ZeroGPT and GPTZero.
\end{itemize}




\newpage
\section{Results and Findings} \label{app:result}
\section{Result} \label{sec:result}

\subsection{Setup}

In this section, we evaluate VB-Com across the following perspectives:
\begin{itemize}
    \item Under what conditions does VB-Com demonstrate superior performance compared to using a single-policy approach?
    \item How does VB-Com outperforms baseline methods in those scenarios?
    \item How well does the proposed return estimator contribute to the composition system?
\end{itemize}

\begin{figure}[h]
\centering{\includegraphics[width=0.5\textwidth]{figures/noise.png}}
\caption{We present four types of perception noises and implement them on heightmaps during evaluation: gaussian noise, \textcolor{red}{forward shifting noise}, \textcolor{green}{lateral shifting noise} and \textcolor{blue}{floating noise}.}
\label{noise}
\end{figure}

\subsubsection{Evaluation Noise}
To simulate situations where the robot encounters perception outliers not present in the simulation, we introduce a quantitative curriculum noise designed to mimic varying levels of perception deficiency. As shown in Fig. \ref{noise}, we focus on four types of noise: (1) \textbf{Gaussian noise}: noise points sampled from a Gaussian distribution, to the original heightmap. The noise level is scaled from 0.0 to 1.0, where the training noise level corresponds to a 0.1 noise level in this scenario. (2) \textbf{Shifting noise}: replacing points in the original heightmap with noise sampled from a Gaussian distribution. The range of replacement points is controlled by the noise level, where a $100\%$ noise level results in a fully noisy heightmap. The shifting direction can either be along the heading direction (red box) or sideways (green box). (3) \textbf{Floating noise}: The heightmap is displaced vertically, either upwards or downwards, the floating noise simulates variations in terrain height. (blue box).

\begin{table}[!ht]
\caption{Terrain Size Scales (m)}
\label{tab:terrains}
\begin{center}
\renewcommand\arraystretch{1.25}
\begin{tabular}{lcccc}
\toprule[1.0pt]
Terrain & Length & Width & Heights\\
\midrule[0.8pt]

Gaps        & $(0.6, 1.2)$ & $(\bm{0.6}, \bm{0.8})$ & $(-1.8, -1.5)$\\  
Hurdles     & $(0.8, 1.0)$ & $(0.1, 0.2)$ & $(\bm{0.2}, \bm{0.4})$\\  
Obstacles   & $(\bm{0.2}, \bm{0.4})$ & $(0.2, 0.4)$ & $(1.4,1.8)$\\  

\bottomrule[1.0pt]
\end{tabular}
\end{center}
\end{table}

\subsubsection{Experiments Setup}
In simulation, we conduct $10 \times 3$ experiments for each method across three types of terrain, replicating the experiments three times to calculate the variance. Each episode involves the robot navigating through 8 goal points, with each goal paired with a corresponding challenging terrain or obstacle. The size of the terrains is set to the maximum curriculum terrain level, as shown in Table \ref{tab:terrains}. The bolded values indicate the primary factors that contribute to the difficulty for the terrain.

\subsubsection{Baselines}
We primarily compare VB-Com with the vision and blind policies operating independently. Additionally, as previous works have shown that robust perceptive locomotion can be learned by incorporating various perception noises during training \cite{miki2022learning}, we add a \textbf{Noisy Perceptive policy baseline} trained using the same noises implemented in the evaluation. This allows us to examine how well the proposed VB-Com policy performs compared to policies that have already seen the evaluation noises. The evaluation noises are introduced to the Noisy Perceptive policy in a curriculum format during training, which evolves with the terrain level.

\begin{figure*}[h]
\centering{\includegraphics[width=\textwidth]{figures/returnsim.png}}
\caption{Illustrations of the variation in estimated return and action phases(0 for $a_b$ and 1 for $a_v$) across three concerned terrains.}
\label{return}
\end{figure*}

\subsection{Example Case}
First, we illustrate how VB-Com operates, specifically when the composition switches to $\pi_b$ and how it effectively controls the robot to traverse the terrain against deficient perception (Fig. \ref{return}). We demonstrate $3$ seconds of the estimated returns, along with the policy composition phase, as the robot walking through the challenging terrain during the simulation experiments at the noise level of $100\%$. Before the robot encounters challenging terrains, we observe that the estimated return $G^e_{\pi_v}(s_t)$ consistently exceeds $G^e_{\pi_b}(s_t)$, as the robot is walking on flat ground with relatively stable motion. This observation aligns with the discussion in Section \ref{subsec:vb-com}, where it was explained that $\pi_v$ benefits from the external state observation and results in a higher return $G_t$. This characteraistic ensures the robot operates at $\pi_b$ while stable motion. 

Once the deficient perception reaches the $100\%$ noise level, the robot will not be aware of the incoming challenging terrains until it collides with them. At this point, we observe that $G^e_{\pi}(s_t)$ drops sharply within several control steps, prompting the switch to the blind policy. This switch allows the robot to respond to the terrain, and once the motion stabilizes, $G^e_{\pi}(s_t)$ returns to a normal level, at which point the vision policy regains control. These cases demonstrate the effectiveness of VB-Com, which responds quickly to deficient perception, but avoids unnecessary switches to the blind policy when it is not needed.


\begin{table*}[!h]
\caption{VB-Com Evaluations}
\label{tab:VB-Com}
\begin{center}
\renewcommand\arraystretch{1.25}
\begin{tabular}{lccccccc}
\toprule[1.0pt]
Noise Level &Method & Goals Completed($\%$) & Rewards & Average Velocity & Fail Rate & Collision Steps($\%$) & Reach Steps\\
\midrule[0.8pt]

% \multirow{4}{*}{Prop Advisor}&0.25& $0.7560$& $0.7964$& $0.7001$ & \multirow{4}{*}{$0.8600$}\\

\multirow{2}{*}{0\% noise} & VB-Com & $84.05 \pm 2.28$ & \bm{$142.07 \pm 4.19$} & $0.71 \pm 0.01$ & \bm{$0.29 \pm 0.01$} & $1.50 \pm 0.14$ & $177.29 \pm 4.66$\\  
                              & Vision & $73.57 \pm 4.97$ & $118.07 \pm 10.42$ & $0.73 \pm 0.01$ & $0.42 \pm 0.07$ & \bm{$1.39 \pm 0.53$} & $204.82 \pm 28.91$\\  \midrule
\multirow{2}{*}{30\% noise} & VB-Com & $82.24 \pm 6.6$ & $132.81 \pm 7.64$ & $0.71 \pm 0.01$ & $0.34 \pm 0.10$ & $2.09 \pm 0.13$ & $178.13 \pm 4.13$\\  
                              & Vision & $72.76 \pm 2.29$ & $115.20 \pm 2.43$ & $0.75 \pm 0.02$ & $0.43 \pm 0.05$ & $2.52 \pm 0.32$ & $195.58 \pm 21.98$\\  \midrule
\multirow{2}{*}{70\% noise} & VB-Com & $82.48 \pm 1.20$ & $132.44 \pm 6.17$ & $0.70 \pm 0.02$ & $0.33 \pm 0.03$ & $2.12 \pm 0.11$ & $184.81 \pm 4.47$\\  
                              & Vision & $55.38 \pm 3.33$ & $58.24 \pm 13.97$ & $0.73 \pm 0.03$ & $0.67 \pm 0.07$ & $6.08 \pm 0.82$ & $190.50 \pm 18.28$\\  \midrule
\multirow{3}{*}{100\% noise} & VB-Com & \bm{$84.81 \pm 6.45$} & $129.99 \pm 9.84$ & $0.72 \pm 0.02$ & \bm{$0.29 \pm 0.08$} & $2.60 \pm 0.68$ & $182.29 \pm 11.47$\\  
                              & Vision & $48.71 \pm 5.60$ & $47.53 \pm 17.55$ & $0.70 \pm 0.06$ & $0.69 \pm 0.06$ & $6.92 \pm 1.36$ & $268.40 \pm 57.11$\\  
                              & Noisy Perceptive & $80.52 \pm 0.91$ & $116.94 \pm 4.07$ & \bm{$0.76 \pm 0.02$} & $0.32 \pm 0.04$ & $3.49 \pm 0.38$ & \bm{$154.98 \pm 4.41$}\\ \midrule
& Blind & $83.76 \pm 1.35$ & $131.29 \pm 3.48$ & $0.70 \pm 0.01$ & $0.33 \pm 0.05$ & $2.57 \pm 0.27$ & $184.08 \pm 1.85$\\  

% Perceptive  & $0.00 \pm 0.00$ & $0.00 \pm 0.00$ & $0.00 \pm 0.00$ & $0.00 \pm 0.00$ & $0.00 \pm 0.00$\\  
% Blind  & $0.00 \pm 0.00$ & $0.00 \pm 0.00$ & $0.00 \pm 0.00$ & $0.00 \pm 0.00$ & $0.00 \pm 0.00$\\  
% Noisy Perceptive & $0.00 \pm 0.00$ & $0.00 \pm 0.00$ & $0.00 \pm 0.00$ & $0.00 \pm 0.00$ & $0.00 \pm 0.00$\\  

\bottomrule[1.0pt]
\end{tabular}
\end{center}
\end{table*}

\subsection{Evaluations on Different Noise Levels}
\textbf{VB-Com achieves robust locomotion performance under different levels of perception deficiency.} As shown in Tab \ref{tab:VB-Com}, performance of the vision policy declines shaprly with the arise of noise level. In addition, since the evaluation experiments set the terrain curriculum to the maximum level, the vision policy struggles even at a $0\%$ noise level: It only achieves around $73\%$ goal-reaching success, with a termination rate exceeding $40\%$. This poor performance is likely due to the severe challenge terrains, such as the farthest range of the heightmap $(0.85m)$ is only $0.05m$ wider than the width of the gaps$(0.8m)$. In contrast, VB-Com achieves a stable higher goal-reaching success against different levels of perception deficiency. In contrast, VB-Com achieves consistently higher goal-reaching success across varying levels of perception deficiency, including both noise and perception range limitations.

Despite the high goal-reaching success, we also include additional metrics to further analyze the performance. The reward values recorded throughout each episode indicate the proposed method’s ability to achieve both goal completion and collision avoidance. These rewards strongly correlate with the robot’s success in reaching the target while minimizing collisions. For instance, VB-Com at the $0\%$ noise level achieves the highest rewards$(142.07)$, although the goal completion rate$(84.05)$ is slightly lower compared to the trail in $100\%$ noise level $(84.81)$. This is because VB-Com switches to the blind policy more often in  $100\%$  noise level, resulting in more frequent collisions and lower rewards obtained. 

The reach steps metrics indicates the smoothness of the policy in overcoming challenging obstacles. Since the switching mechanism requires several steps to respond effectively, VB-Com results in a higher number of reach steps as the noise level increases. This is because, under higher noise conditions, the system needs additional time to transition from the vision policy to the blind policy, which leads to more gradual and controlled responses to terrain challenges.
\begin{figure}[h]
\centering{\includegraphics[width=0.5\textwidth]{figures/noiseevalueate.png}}
\caption{We compare the collision and goal-reaching performances under different noise levels. VB-Com achieves low collisions and high success rates with accurate perception, and its success rate remains high under deficient perception.}
\label{noiseevalueate}
\end{figure}

\begin{figure}[h]
\centering{\includegraphics[width=0.5\textwidth]{figures/terraineval.png}}
\caption{Comparisons between the Noisy Perceptive policy and VB-Com in navigating gaps and hurdles separately.}
\label{terraineval}
\end{figure}


\subsection{Comparisons with Blind Policy}
\textbf{VB-Com achieves less collision than the blind policy when perception becomes less dificient.} As shown in Tab \ref{tab:VB-Com}, the blind policy achieves a relatively high Goals Completed rate $(83.76\%)$, as its performance is unaffected by deficient perception. Therefore, we include an evaluation of the collision performance between VB-Com and the blind policy to further highlight the advantage of the proposed framework. In our evaluations, "Collision Steps" is defined as the ratio of the number of steps during which the robot collision model (Fig \ref{robot}) makes illegal contact with the terrain or obstacles, relative to the total number of steps within an episode.

We can observe from Tab \ref{tab:VB-Com} that the collision steps increase with the noise level for VB-Com. Fig \ref{noiseevalueate} provides a more intuitive illustration: as perception becomes more comprehensive, VB-Com achieves both fewer collisions and better goal-reaching performance. In contrast, the blind policy maintains a high goal-reaching rate but results in more collisions, while the vision policy performs better in avoiding collisions when the perception is accurate and comprehensive. As the noise level increases, the performance of VB-Com begins to resemble that of the blind policy. These results demonstrate the effectiveness of the composition system, which benefits from both sub-policies to achieve better performance in terms of both goal-reaching and minimizing collisions.

\subsection{Comparisons with Noisy Perceptive Training}
\textbf{Compared to policies trained with noisy priors, VB-Com achieves equivalent performance without prior knowledge of the noise, while also demonstrating better training efficiency and the ability to handle more challenging terrain difficulties.} The comparisons (Tab \ref{tab:VB-Com}) with Noisy Perceptive policy show that the Noisy Perceptive policy achieves a relatively high goal completion rate $(80.52\%)$ but exhibits a higher collision step rate $(3.49\%)$. It can be concluded that, as severe noise is introduced during evaluation, the heightmap quickly becomes random noise with the increasing noise level. In response, the Noisy Perceptive policy begins to exhibit behavior similar to that of the blind policy—making contact with obstacles and reacting when the noisy signals overwhelm the external observations.

To further investigate the conditions under which the Noisy Perceptive policy fails to surpass the performance of VB-Com, we evaluate goal-reaching performance under different terrains (Fig. \ref{terraineval}). The results show that VB-Com outperforms the Noisy Perceptive policy in gap terrains, while the Noisy Perceptive policy performs better in hurdle situations, achieving a higher success rate in preventing the robot from being tripped by hurdles. However, recovering from missed gaps requires a quicker response, or the robot risks falling. These results demonstrate that the single-policy method fails to handle such dynamic challenges effectively, highlighting the advantages of the composition in VB-Com.

\begin{figure}[h]
\centering{\includegraphics[width=0.5\textwidth]{figures/trainplot.png}}
\caption{Training curves for terrain levels and the return estimation loss.}
\label{train}
\end{figure}

Moreover, the terrain level rises slowly for the Noisy Perceptive policy (Fig. \ref{train}-(a)), and it fails to reach the maximum level achieved by the vision and blind policies. This is because the policy struggles with the trade-off of whether to trust the external perception, which requires the addition of an extra module to address the challenge. This slow progression highlights the difficulty of handling high levels of perception deficiency, whereas VB-Com can efficiently navigate such situations by leveraging the strengths of both the vision and blind policies.

\begin{table}[!ht]
\caption{Return Estimation Evaluations}
\label{tab:RE}
\begin{center}
\renewcommand\arraystretch{1.25}
\begin{tabular}{lcccc}
\toprule[1.0pt]
Method & Goals Completed($\%$) & Collisions & Reach Steps\\
\midrule[0.8pt]

100-steps) & $78.24 \pm 1.86$ & \bm{$2.49 \pm 0.04$} & $193.7 \pm 3.2$\\  
RE(50-steps)  & \bm{$81.90 \pm 2.81$} & $2.75 \pm 0.17$ & $184.6 \pm 1.4$\\ 
Re(5-steps)   & $69.90 \pm 7.34$ & $5.23 \pm 0.59$ & $192.6 \pm 3.3$\\  
Re(1-step)    & $59.57 \pm 2.00$ & $4.78 \pm 0.16$ & \bm{$167.4 \pm 5.0$}\\  
MC-based      & $74.14 \pm 2.69$ & $4.26 \pm 0.56$ & $192.8 \pm 11.8$\\  

\bottomrule[1.0pt]
\end{tabular}
\end{center}
\end{table}

\subsection{Return Estimator Evaluations}
\textbf{The proposed return estimator achieves accurate and efficient return estimation with accessible states observations.} Since we update the return estimator using temporal difference, we compare it with the Monte Carlo-based search return estimator that estimate the furtuen expected returns with the following regression loss directly: $\mathbb{E}_t[\hat{G}_{\pi_i}^e(s_t) - \sum_{t} ^ {t+T} \gamma^t r(s_t, a_t)]$. As shown in Fig. \ref{train}-(a), the MC-based estimator struggles to converge due to the accumulation of noise. In contrast, the proposed TD-based return estimator within the vision policy convergent stably as it updates alongside the locomotion policy. The results in Tab \ref{tab:RE} further highlight the ineffectiveness of the MC-based return estimator in providing accurate estimations to guide the policy composition. Specifically, the MC-based estimator struggles to respond promptly to collisions with obstacles, this delay in response leads to larger collisions and longer reach steps, as the policy cannot effectively adjust its actions in real-time. 

\textbf{We also evaluate the impact of different switch periods (T), which define the expected return duration during return estimator updates.} While training performance remains consistent across varying periods, we observe that excessively short switch periods can negatively impact system performance. In such cases, the two policies may conflict, resulting in incomplete motion trajectories when traversing the challenging terrains and failures.

\textbf{We observe that training effectiveness is highly dependent on data variance.} For instance, the estimator within vision policy converges the fastest due to its access to more accurate and comprehensive state observations, leading to fewer low-return instances. In contrast, the estimator within Noisy Perceptive and blind policies encounter more collisions and lower returns, causing their loss to degrade more slowly.

\textbf{We demonstrate that the estimated return threhold $G_{th}$ is crucial to the performance of VB-Com.} Tab \ref{tab:TH} evaluates the system's performance under different values of $\alpha$, as well as without $G_{th}$. The results demonstrate that $G_{th}$ is critical for mitigating miscorrection during motion abnormalities, and that a value of $\alpha < 1.0$ ensures a sensitive response to the states that could lead to motion failures.

\begin{table}[!ht]
\caption{Estimated Return Threhold Evaluations}
\label{tab:TH}
\begin{center}
\renewcommand\arraystretch{1.25}
\begin{tabular}{lcccc}
\toprule[1.0pt]
Method & Goals Completed($\%$) & Collisions & Reach Steps\\
\midrule[0.8pt]
 
$\alpha = 2.0$   & $77.10 \pm 4.71$ & $2.63 \pm 0.68$ & $185.11 \pm 7.17$\\ 
$\alpha = 0.5$   & \bm{$85.76 \pm 2.88$} & $2.29 \pm 0.17$ & $186.96 \pm 3.83$\\  
$\alpha = 0.1$   & $84.43 \pm 1.23$ & \bm{$2.10 \pm 0.25$} & $\bm{184.35 \pm 6.27}$\\  
w/o $G_{th}$     & $48.48 \pm 1.28$ & $6.24 \pm 0.41$ & $261.96 \pm 35.63$\\  

\bottomrule[1.0pt]
\end{tabular}
\end{center}
\end{table}



\subsection{Real-World Experiments}

We deploy the proposed system on both the Unitree G1 and Unitree H1 robots and evaluate the performance of the proposed VB-Com method. 
\begin{figure*}[h]
\centering{\includegraphics[width= \textwidth]{figures/hardwarecurve.png}}
\caption{Illustrations of the variation in estimated return under static/dynamic obstacles in hardware experiments.}
\label{hardwarecurve}
\end{figure*}

\subsubsection{Hardware Return Estimations}

We illustrate how VB-Com operates on real robots by plotting $4$ seconds of the estimated return while the robot avoids static (left) and dynamic (right) obstacles (Fig \ref{hardwarecurve}). The results demonstrate that, for static obstacles (a standing person), the elevation map can accurately perceive the obstacle, allowing the robot to plan motions in advance and avoid collisions. Corresponding to this behavior, we observe that the estimated return on the G1 stays a high value, with $\hat{G}^e_{\pi_b}$ slightly lower than $\hat{G}^e_{\pi_v}$, consistent with the scenario where the vision policy continues to operate within VB-Com.

On the other hand, when a person moves towards the robot at high speed, the perception module fails to detect the obstacle, causing a collision, both $\hat{G}^e_{\pi_b}$ and $\hat{G}^e_{\pi_v}$ decline sharply upon collision. However, VB-Com quickly switches to $\pi_b$ to avoid the person, demonstrating the  \textbf{rapid response to collision provided by the proposed return estimation and the successful obstacle avoidance capability of VB-Com under perceptual deficiency}.


\begin{figure}[h]
\centering{\includegraphics[width=0.5\textwidth]{figures/g1avoid.png}}
\caption{ Real-world comparisons of VB-Com, vision, and blind policies in obstacle avoidance on the G1.}
\label{avoid}
\end{figure}

\subsubsection{Avoid Obstacles}
In this section, we make comparisons between VB-Com along with the vision policy and blind policy on G1 (Fig \ref{avoid}), to demonstrate the superior performance of VB-Com in hardware compared with signle policies. In the evaluation scenario, G1 encounters two consecutive obstacles along its path. The second dynamic obstacle obstructs the robot's direction before the elevation map can perceive it. VB-Com enables the robot to avoid the static obstacle without collision and subsequently avoid the dynamic obstacle after it collides with the suddenly appearing obstacle.

In contrast, for the baseline policies, the blind policy makes unnecessary contact with the static obstacles before avoiding them, which damages the environment. As for the vision policy, the robot collides with the obstacle and is unable to avoid it until the newly added obstacle is detected and integrated into the map.

\begin{figure}[h]
\centering{\includegraphics[width=0.5\textwidth]{figures/hurdlegap.png}}
\caption{Hardware demonstrations on the robots traversing gaps and hurldes given deficient perception with VB-Com.}
\label{hurdlegap}
\end{figure}

\subsubsection{Performance Against Deficient Perception}
In this section, we demonstrate the ability of VB-Com to traverse challenging terrains given deficient perception (Fig. \ref{hurdlegap}). We provide zero inputs for the heightmaps to evaluate the performance of VB-Com under perceptual deficiency. We introduce two consecutive hurdles, and the robot successfully recovers after colliding with them by switching to $\pi_b$. Additionally, we demonstrate that VB-Com enables recovery from a missed step on an unobserved gap. In this case, VB-Com saves the robot by performing a larger forward step to traverse the gap without perception, as the blind policy has learned during simulation.




% \section{Appendix}
% \label{sec:appendix}
%\subsection{Lloyd-Max Algorithm}
\label{subsec:Lloyd-Max}
For a given quantization bitwidth $B$ and an operand $\bm{X}$, the Lloyd-Max algorithm finds $2^B$ quantization levels $\{\hat{x}_i\}_{i=1}^{2^B}$ such that quantizing $\bm{X}$ by rounding each scalar in $\bm{X}$ to the nearest quantization level minimizes the quantization MSE. 

The algorithm starts with an initial guess of quantization levels and then iteratively computes quantization thresholds $\{\tau_i\}_{i=1}^{2^B-1}$ and updates quantization levels $\{\hat{x}_i\}_{i=1}^{2^B}$. Specifically, at iteration $n$, thresholds are set to the midpoints of the previous iteration's levels:
\begin{align*}
    \tau_i^{(n)}=\frac{\hat{x}_i^{(n-1)}+\hat{x}_{i+1}^{(n-1)}}2 \text{ for } i=1\ldots 2^B-1
\end{align*}
Subsequently, the quantization levels are re-computed as conditional means of the data regions defined by the new thresholds:
\begin{align*}
    \hat{x}_i^{(n)}=\mathbb{E}\left[ \bm{X} \big| \bm{X}\in [\tau_{i-1}^{(n)},\tau_i^{(n)}] \right] \text{ for } i=1\ldots 2^B
\end{align*}
where to satisfy boundary conditions we have $\tau_0=-\infty$ and $\tau_{2^B}=\infty$. The algorithm iterates the above steps until convergence.

Figure \ref{fig:lm_quant} compares the quantization levels of a $7$-bit floating point (E3M3) quantizer (left) to a $7$-bit Lloyd-Max quantizer (right) when quantizing a layer of weights from the GPT3-126M model at a per-tensor granularity. As shown, the Lloyd-Max quantizer achieves substantially lower quantization MSE. Further, Table \ref{tab:FP7_vs_LM7} shows the superior perplexity achieved by Lloyd-Max quantizers for bitwidths of $7$, $6$ and $5$. The difference between the quantizers is clear at 5 bits, where per-tensor FP quantization incurs a drastic and unacceptable increase in perplexity, while Lloyd-Max quantization incurs a much smaller increase. Nevertheless, we note that even the optimal Lloyd-Max quantizer incurs a notable ($\sim 1.5$) increase in perplexity due to the coarse granularity of quantization. 

\begin{figure}[h]
  \centering
  \includegraphics[width=0.7\linewidth]{sections/figures/LM7_FP7.pdf}
  \caption{\small Quantization levels and the corresponding quantization MSE of Floating Point (left) vs Lloyd-Max (right) Quantizers for a layer of weights in the GPT3-126M model.}
  \label{fig:lm_quant}
\end{figure}

\begin{table}[h]\scriptsize
\begin{center}
\caption{\label{tab:FP7_vs_LM7} \small Comparing perplexity (lower is better) achieved by floating point quantizers and Lloyd-Max quantizers on a GPT3-126M model for the Wikitext-103 dataset.}
\begin{tabular}{c|cc|c}
\hline
 \multirow{2}{*}{\textbf{Bitwidth}} & \multicolumn{2}{|c|}{\textbf{Floating-Point Quantizer}} & \textbf{Lloyd-Max Quantizer} \\
 & Best Format & Wikitext-103 Perplexity & Wikitext-103 Perplexity \\
\hline
7 & E3M3 & 18.32 & 18.27 \\
6 & E3M2 & 19.07 & 18.51 \\
5 & E4M0 & 43.89 & 19.71 \\
\hline
\end{tabular}
\end{center}
\end{table}

\subsection{Proof of Local Optimality of LO-BCQ}
\label{subsec:lobcq_opt_proof}
For a given block $\bm{b}_j$, the quantization MSE during LO-BCQ can be empirically evaluated as $\frac{1}{L_b}\lVert \bm{b}_j- \bm{\hat{b}}_j\rVert^2_2$ where $\bm{\hat{b}}_j$ is computed from equation (\ref{eq:clustered_quantization_definition}) as $C_{f(\bm{b}_j)}(\bm{b}_j)$. Further, for a given block cluster $\mathcal{B}_i$, we compute the quantization MSE as $\frac{1}{|\mathcal{B}_{i}|}\sum_{\bm{b} \in \mathcal{B}_{i}} \frac{1}{L_b}\lVert \bm{b}- C_i^{(n)}(\bm{b})\rVert^2_2$. Therefore, at the end of iteration $n$, we evaluate the overall quantization MSE $J^{(n)}$ for a given operand $\bm{X}$ composed of $N_c$ block clusters as:
\begin{align*}
    \label{eq:mse_iter_n}
    J^{(n)} = \frac{1}{N_c} \sum_{i=1}^{N_c} \frac{1}{|\mathcal{B}_{i}^{(n)}|}\sum_{\bm{v} \in \mathcal{B}_{i}^{(n)}} \frac{1}{L_b}\lVert \bm{b}- B_i^{(n)}(\bm{b})\rVert^2_2
\end{align*}

At the end of iteration $n$, the codebooks are updated from $\mathcal{C}^{(n-1)}$ to $\mathcal{C}^{(n)}$. However, the mapping of a given vector $\bm{b}_j$ to quantizers $\mathcal{C}^{(n)}$ remains as  $f^{(n)}(\bm{b}_j)$. At the next iteration, during the vector clustering step, $f^{(n+1)}(\bm{b}_j)$ finds new mapping of $\bm{b}_j$ to updated codebooks $\mathcal{C}^{(n)}$ such that the quantization MSE over the candidate codebooks is minimized. Therefore, we obtain the following result for $\bm{b}_j$:
\begin{align*}
\frac{1}{L_b}\lVert \bm{b}_j - C_{f^{(n+1)}(\bm{b}_j)}^{(n)}(\bm{b}_j)\rVert^2_2 \le \frac{1}{L_b}\lVert \bm{b}_j - C_{f^{(n)}(\bm{b}_j)}^{(n)}(\bm{b}_j)\rVert^2_2
\end{align*}

That is, quantizing $\bm{b}_j$ at the end of the block clustering step of iteration $n+1$ results in lower quantization MSE compared to quantizing at the end of iteration $n$. Since this is true for all $\bm{b} \in \bm{X}$, we assert the following:
\begin{equation}
\begin{split}
\label{eq:mse_ineq_1}
    \tilde{J}^{(n+1)} &= \frac{1}{N_c} \sum_{i=1}^{N_c} \frac{1}{|\mathcal{B}_{i}^{(n+1)}|}\sum_{\bm{b} \in \mathcal{B}_{i}^{(n+1)}} \frac{1}{L_b}\lVert \bm{b} - C_i^{(n)}(b)\rVert^2_2 \le J^{(n)}
\end{split}
\end{equation}
where $\tilde{J}^{(n+1)}$ is the the quantization MSE after the vector clustering step at iteration $n+1$.

Next, during the codebook update step (\ref{eq:quantizers_update}) at iteration $n+1$, the per-cluster codebooks $\mathcal{C}^{(n)}$ are updated to $\mathcal{C}^{(n+1)}$ by invoking the Lloyd-Max algorithm \citep{Lloyd}. We know that for any given value distribution, the Lloyd-Max algorithm minimizes the quantization MSE. Therefore, for a given vector cluster $\mathcal{B}_i$ we obtain the following result:

\begin{equation}
    \frac{1}{|\mathcal{B}_{i}^{(n+1)}|}\sum_{\bm{b} \in \mathcal{B}_{i}^{(n+1)}} \frac{1}{L_b}\lVert \bm{b}- C_i^{(n+1)}(\bm{b})\rVert^2_2 \le \frac{1}{|\mathcal{B}_{i}^{(n+1)}|}\sum_{\bm{b} \in \mathcal{B}_{i}^{(n+1)}} \frac{1}{L_b}\lVert \bm{b}- C_i^{(n)}(\bm{b})\rVert^2_2
\end{equation}

The above equation states that quantizing the given block cluster $\mathcal{B}_i$ after updating the associated codebook from $C_i^{(n)}$ to $C_i^{(n+1)}$ results in lower quantization MSE. Since this is true for all the block clusters, we derive the following result: 
\begin{equation}
\begin{split}
\label{eq:mse_ineq_2}
     J^{(n+1)} &= \frac{1}{N_c} \sum_{i=1}^{N_c} \frac{1}{|\mathcal{B}_{i}^{(n+1)}|}\sum_{\bm{b} \in \mathcal{B}_{i}^{(n+1)}} \frac{1}{L_b}\lVert \bm{b}- C_i^{(n+1)}(\bm{b})\rVert^2_2  \le \tilde{J}^{(n+1)}   
\end{split}
\end{equation}

Following (\ref{eq:mse_ineq_1}) and (\ref{eq:mse_ineq_2}), we find that the quantization MSE is non-increasing for each iteration, that is, $J^{(1)} \ge J^{(2)} \ge J^{(3)} \ge \ldots \ge J^{(M)}$ where $M$ is the maximum number of iterations. 
%Therefore, we can say that if the algorithm converges, then it must be that it has converged to a local minimum. 
\hfill $\blacksquare$


\begin{figure}
    \begin{center}
    \includegraphics[width=0.5\textwidth]{sections//figures/mse_vs_iter.pdf}
    \end{center}
    \caption{\small NMSE vs iterations during LO-BCQ compared to other block quantization proposals}
    \label{fig:nmse_vs_iter}
\end{figure}

Figure \ref{fig:nmse_vs_iter} shows the empirical convergence of LO-BCQ across several block lengths and number of codebooks. Also, the MSE achieved by LO-BCQ is compared to baselines such as MXFP and VSQ. As shown, LO-BCQ converges to a lower MSE than the baselines. Further, we achieve better convergence for larger number of codebooks ($N_c$) and for a smaller block length ($L_b$), both of which increase the bitwidth of BCQ (see Eq \ref{eq:bitwidth_bcq}).


\subsection{Additional Accuracy Results}
%Table \ref{tab:lobcq_config} lists the various LOBCQ configurations and their corresponding bitwidths.
\begin{table}
\setlength{\tabcolsep}{4.75pt}
\begin{center}
\caption{\label{tab:lobcq_config} Various LO-BCQ configurations and their bitwidths.}
\begin{tabular}{|c||c|c|c|c||c|c||c|} 
\hline
 & \multicolumn{4}{|c||}{$L_b=8$} & \multicolumn{2}{|c||}{$L_b=4$} & $L_b=2$ \\
 \hline
 \backslashbox{$L_A$\kern-1em}{\kern-1em$N_c$} & 2 & 4 & 8 & 16 & 2 & 4 & 2 \\
 \hline
 64 & 4.25 & 4.375 & 4.5 & 4.625 & 4.375 & 4.625 & 4.625\\
 \hline
 32 & 4.375 & 4.5 & 4.625& 4.75 & 4.5 & 4.75 & 4.75 \\
 \hline
 16 & 4.625 & 4.75& 4.875 & 5 & 4.75 & 5 & 5 \\
 \hline
\end{tabular}
\end{center}
\end{table}

%\subsection{Perplexity achieved by various LO-BCQ configurations on Wikitext-103 dataset}

\begin{table} \centering
\begin{tabular}{|c||c|c|c|c||c|c||c|} 
\hline
 $L_b \rightarrow$& \multicolumn{4}{c||}{8} & \multicolumn{2}{c||}{4} & 2\\
 \hline
 \backslashbox{$L_A$\kern-1em}{\kern-1em$N_c$} & 2 & 4 & 8 & 16 & 2 & 4 & 2  \\
 %$N_c \rightarrow$ & 2 & 4 & 8 & 16 & 2 & 4 & 2 \\
 \hline
 \hline
 \multicolumn{8}{c}{GPT3-1.3B (FP32 PPL = 9.98)} \\ 
 \hline
 \hline
 64 & 10.40 & 10.23 & 10.17 & 10.15 &  10.28 & 10.18 & 10.19 \\
 \hline
 32 & 10.25 & 10.20 & 10.15 & 10.12 &  10.23 & 10.17 & 10.17 \\
 \hline
 16 & 10.22 & 10.16 & 10.10 & 10.09 &  10.21 & 10.14 & 10.16 \\
 \hline
  \hline
 \multicolumn{8}{c}{GPT3-8B (FP32 PPL = 7.38)} \\ 
 \hline
 \hline
 64 & 7.61 & 7.52 & 7.48 &  7.47 &  7.55 &  7.49 & 7.50 \\
 \hline
 32 & 7.52 & 7.50 & 7.46 &  7.45 &  7.52 &  7.48 & 7.48  \\
 \hline
 16 & 7.51 & 7.48 & 7.44 &  7.44 &  7.51 &  7.49 & 7.47  \\
 \hline
\end{tabular}
\caption{\label{tab:ppl_gpt3_abalation} Wikitext-103 perplexity across GPT3-1.3B and 8B models.}
\end{table}

\begin{table} \centering
\begin{tabular}{|c||c|c|c|c||} 
\hline
 $L_b \rightarrow$& \multicolumn{4}{c||}{8}\\
 \hline
 \backslashbox{$L_A$\kern-1em}{\kern-1em$N_c$} & 2 & 4 & 8 & 16 \\
 %$N_c \rightarrow$ & 2 & 4 & 8 & 16 & 2 & 4 & 2 \\
 \hline
 \hline
 \multicolumn{5}{|c|}{Llama2-7B (FP32 PPL = 5.06)} \\ 
 \hline
 \hline
 64 & 5.31 & 5.26 & 5.19 & 5.18  \\
 \hline
 32 & 5.23 & 5.25 & 5.18 & 5.15  \\
 \hline
 16 & 5.23 & 5.19 & 5.16 & 5.14  \\
 \hline
 \multicolumn{5}{|c|}{Nemotron4-15B (FP32 PPL = 5.87)} \\ 
 \hline
 \hline
 64  & 6.3 & 6.20 & 6.13 & 6.08  \\
 \hline
 32  & 6.24 & 6.12 & 6.07 & 6.03  \\
 \hline
 16  & 6.12 & 6.14 & 6.04 & 6.02  \\
 \hline
 \multicolumn{5}{|c|}{Nemotron4-340B (FP32 PPL = 3.48)} \\ 
 \hline
 \hline
 64 & 3.67 & 3.62 & 3.60 & 3.59 \\
 \hline
 32 & 3.63 & 3.61 & 3.59 & 3.56 \\
 \hline
 16 & 3.61 & 3.58 & 3.57 & 3.55 \\
 \hline
\end{tabular}
\caption{\label{tab:ppl_llama7B_nemo15B} Wikitext-103 perplexity compared to FP32 baseline in Llama2-7B and Nemotron4-15B, 340B models}
\end{table}

%\subsection{Perplexity achieved by various LO-BCQ configurations on MMLU dataset}


\begin{table} \centering
\begin{tabular}{|c||c|c|c|c||c|c|c|c|} 
\hline
 $L_b \rightarrow$& \multicolumn{4}{c||}{8} & \multicolumn{4}{c||}{8}\\
 \hline
 \backslashbox{$L_A$\kern-1em}{\kern-1em$N_c$} & 2 & 4 & 8 & 16 & 2 & 4 & 8 & 16  \\
 %$N_c \rightarrow$ & 2 & 4 & 8 & 16 & 2 & 4 & 2 \\
 \hline
 \hline
 \multicolumn{5}{|c|}{Llama2-7B (FP32 Accuracy = 45.8\%)} & \multicolumn{4}{|c|}{Llama2-70B (FP32 Accuracy = 69.12\%)} \\ 
 \hline
 \hline
 64 & 43.9 & 43.4 & 43.9 & 44.9 & 68.07 & 68.27 & 68.17 & 68.75 \\
 \hline
 32 & 44.5 & 43.8 & 44.9 & 44.5 & 68.37 & 68.51 & 68.35 & 68.27  \\
 \hline
 16 & 43.9 & 42.7 & 44.9 & 45 & 68.12 & 68.77 & 68.31 & 68.59  \\
 \hline
 \hline
 \multicolumn{5}{|c|}{GPT3-22B (FP32 Accuracy = 38.75\%)} & \multicolumn{4}{|c|}{Nemotron4-15B (FP32 Accuracy = 64.3\%)} \\ 
 \hline
 \hline
 64 & 36.71 & 38.85 & 38.13 & 38.92 & 63.17 & 62.36 & 63.72 & 64.09 \\
 \hline
 32 & 37.95 & 38.69 & 39.45 & 38.34 & 64.05 & 62.30 & 63.8 & 64.33  \\
 \hline
 16 & 38.88 & 38.80 & 38.31 & 38.92 & 63.22 & 63.51 & 63.93 & 64.43  \\
 \hline
\end{tabular}
\caption{\label{tab:mmlu_abalation} Accuracy on MMLU dataset across GPT3-22B, Llama2-7B, 70B and Nemotron4-15B models.}
\end{table}


%\subsection{Perplexity achieved by various LO-BCQ configurations on LM evaluation harness}

\begin{table} \centering
\begin{tabular}{|c||c|c|c|c||c|c|c|c|} 
\hline
 $L_b \rightarrow$& \multicolumn{4}{c||}{8} & \multicolumn{4}{c||}{8}\\
 \hline
 \backslashbox{$L_A$\kern-1em}{\kern-1em$N_c$} & 2 & 4 & 8 & 16 & 2 & 4 & 8 & 16  \\
 %$N_c \rightarrow$ & 2 & 4 & 8 & 16 & 2 & 4 & 2 \\
 \hline
 \hline
 \multicolumn{5}{|c|}{Race (FP32 Accuracy = 37.51\%)} & \multicolumn{4}{|c|}{Boolq (FP32 Accuracy = 64.62\%)} \\ 
 \hline
 \hline
 64 & 36.94 & 37.13 & 36.27 & 37.13 & 63.73 & 62.26 & 63.49 & 63.36 \\
 \hline
 32 & 37.03 & 36.36 & 36.08 & 37.03 & 62.54 & 63.51 & 63.49 & 63.55  \\
 \hline
 16 & 37.03 & 37.03 & 36.46 & 37.03 & 61.1 & 63.79 & 63.58 & 63.33  \\
 \hline
 \hline
 \multicolumn{5}{|c|}{Winogrande (FP32 Accuracy = 58.01\%)} & \multicolumn{4}{|c|}{Piqa (FP32 Accuracy = 74.21\%)} \\ 
 \hline
 \hline
 64 & 58.17 & 57.22 & 57.85 & 58.33 & 73.01 & 73.07 & 73.07 & 72.80 \\
 \hline
 32 & 59.12 & 58.09 & 57.85 & 58.41 & 73.01 & 73.94 & 72.74 & 73.18  \\
 \hline
 16 & 57.93 & 58.88 & 57.93 & 58.56 & 73.94 & 72.80 & 73.01 & 73.94  \\
 \hline
\end{tabular}
\caption{\label{tab:mmlu_abalation} Accuracy on LM evaluation harness tasks on GPT3-1.3B model.}
\end{table}

\begin{table} \centering
\begin{tabular}{|c||c|c|c|c||c|c|c|c|} 
\hline
 $L_b \rightarrow$& \multicolumn{4}{c||}{8} & \multicolumn{4}{c||}{8}\\
 \hline
 \backslashbox{$L_A$\kern-1em}{\kern-1em$N_c$} & 2 & 4 & 8 & 16 & 2 & 4 & 8 & 16  \\
 %$N_c \rightarrow$ & 2 & 4 & 8 & 16 & 2 & 4 & 2 \\
 \hline
 \hline
 \multicolumn{5}{|c|}{Race (FP32 Accuracy = 41.34\%)} & \multicolumn{4}{|c|}{Boolq (FP32 Accuracy = 68.32\%)} \\ 
 \hline
 \hline
 64 & 40.48 & 40.10 & 39.43 & 39.90 & 69.20 & 68.41 & 69.45 & 68.56 \\
 \hline
 32 & 39.52 & 39.52 & 40.77 & 39.62 & 68.32 & 67.43 & 68.17 & 69.30  \\
 \hline
 16 & 39.81 & 39.71 & 39.90 & 40.38 & 68.10 & 66.33 & 69.51 & 69.42  \\
 \hline
 \hline
 \multicolumn{5}{|c|}{Winogrande (FP32 Accuracy = 67.88\%)} & \multicolumn{4}{|c|}{Piqa (FP32 Accuracy = 78.78\%)} \\ 
 \hline
 \hline
 64 & 66.85 & 66.61 & 67.72 & 67.88 & 77.31 & 77.42 & 77.75 & 77.64 \\
 \hline
 32 & 67.25 & 67.72 & 67.72 & 67.00 & 77.31 & 77.04 & 77.80 & 77.37  \\
 \hline
 16 & 68.11 & 68.90 & 67.88 & 67.48 & 77.37 & 78.13 & 78.13 & 77.69  \\
 \hline
\end{tabular}
\caption{\label{tab:mmlu_abalation} Accuracy on LM evaluation harness tasks on GPT3-8B model.}
\end{table}

\begin{table} \centering
\begin{tabular}{|c||c|c|c|c||c|c|c|c|} 
\hline
 $L_b \rightarrow$& \multicolumn{4}{c||}{8} & \multicolumn{4}{c||}{8}\\
 \hline
 \backslashbox{$L_A$\kern-1em}{\kern-1em$N_c$} & 2 & 4 & 8 & 16 & 2 & 4 & 8 & 16  \\
 %$N_c \rightarrow$ & 2 & 4 & 8 & 16 & 2 & 4 & 2 \\
 \hline
 \hline
 \multicolumn{5}{|c|}{Race (FP32 Accuracy = 40.67\%)} & \multicolumn{4}{|c|}{Boolq (FP32 Accuracy = 76.54\%)} \\ 
 \hline
 \hline
 64 & 40.48 & 40.10 & 39.43 & 39.90 & 75.41 & 75.11 & 77.09 & 75.66 \\
 \hline
 32 & 39.52 & 39.52 & 40.77 & 39.62 & 76.02 & 76.02 & 75.96 & 75.35  \\
 \hline
 16 & 39.81 & 39.71 & 39.90 & 40.38 & 75.05 & 73.82 & 75.72 & 76.09  \\
 \hline
 \hline
 \multicolumn{5}{|c|}{Winogrande (FP32 Accuracy = 70.64\%)} & \multicolumn{4}{|c|}{Piqa (FP32 Accuracy = 79.16\%)} \\ 
 \hline
 \hline
 64 & 69.14 & 70.17 & 70.17 & 70.56 & 78.24 & 79.00 & 78.62 & 78.73 \\
 \hline
 32 & 70.96 & 69.69 & 71.27 & 69.30 & 78.56 & 79.49 & 79.16 & 78.89  \\
 \hline
 16 & 71.03 & 69.53 & 69.69 & 70.40 & 78.13 & 79.16 & 79.00 & 79.00  \\
 \hline
\end{tabular}
\caption{\label{tab:mmlu_abalation} Accuracy on LM evaluation harness tasks on GPT3-22B model.}
\end{table}

\begin{table} \centering
\begin{tabular}{|c||c|c|c|c||c|c|c|c|} 
\hline
 $L_b \rightarrow$& \multicolumn{4}{c||}{8} & \multicolumn{4}{c||}{8}\\
 \hline
 \backslashbox{$L_A$\kern-1em}{\kern-1em$N_c$} & 2 & 4 & 8 & 16 & 2 & 4 & 8 & 16  \\
 %$N_c \rightarrow$ & 2 & 4 & 8 & 16 & 2 & 4 & 2 \\
 \hline
 \hline
 \multicolumn{5}{|c|}{Race (FP32 Accuracy = 44.4\%)} & \multicolumn{4}{|c|}{Boolq (FP32 Accuracy = 79.29\%)} \\ 
 \hline
 \hline
 64 & 42.49 & 42.51 & 42.58 & 43.45 & 77.58 & 77.37 & 77.43 & 78.1 \\
 \hline
 32 & 43.35 & 42.49 & 43.64 & 43.73 & 77.86 & 75.32 & 77.28 & 77.86  \\
 \hline
 16 & 44.21 & 44.21 & 43.64 & 42.97 & 78.65 & 77 & 76.94 & 77.98  \\
 \hline
 \hline
 \multicolumn{5}{|c|}{Winogrande (FP32 Accuracy = 69.38\%)} & \multicolumn{4}{|c|}{Piqa (FP32 Accuracy = 78.07\%)} \\ 
 \hline
 \hline
 64 & 68.9 & 68.43 & 69.77 & 68.19 & 77.09 & 76.82 & 77.09 & 77.86 \\
 \hline
 32 & 69.38 & 68.51 & 68.82 & 68.90 & 78.07 & 76.71 & 78.07 & 77.86  \\
 \hline
 16 & 69.53 & 67.09 & 69.38 & 68.90 & 77.37 & 77.8 & 77.91 & 77.69  \\
 \hline
\end{tabular}
\caption{\label{tab:mmlu_abalation} Accuracy on LM evaluation harness tasks on Llama2-7B model.}
\end{table}

\begin{table} \centering
\begin{tabular}{|c||c|c|c|c||c|c|c|c|} 
\hline
 $L_b \rightarrow$& \multicolumn{4}{c||}{8} & \multicolumn{4}{c||}{8}\\
 \hline
 \backslashbox{$L_A$\kern-1em}{\kern-1em$N_c$} & 2 & 4 & 8 & 16 & 2 & 4 & 8 & 16  \\
 %$N_c \rightarrow$ & 2 & 4 & 8 & 16 & 2 & 4 & 2 \\
 \hline
 \hline
 \multicolumn{5}{|c|}{Race (FP32 Accuracy = 48.8\%)} & \multicolumn{4}{|c|}{Boolq (FP32 Accuracy = 85.23\%)} \\ 
 \hline
 \hline
 64 & 49.00 & 49.00 & 49.28 & 48.71 & 82.82 & 84.28 & 84.03 & 84.25 \\
 \hline
 32 & 49.57 & 48.52 & 48.33 & 49.28 & 83.85 & 84.46 & 84.31 & 84.93  \\
 \hline
 16 & 49.85 & 49.09 & 49.28 & 48.99 & 85.11 & 84.46 & 84.61 & 83.94  \\
 \hline
 \hline
 \multicolumn{5}{|c|}{Winogrande (FP32 Accuracy = 79.95\%)} & \multicolumn{4}{|c|}{Piqa (FP32 Accuracy = 81.56\%)} \\ 
 \hline
 \hline
 64 & 78.77 & 78.45 & 78.37 & 79.16 & 81.45 & 80.69 & 81.45 & 81.5 \\
 \hline
 32 & 78.45 & 79.01 & 78.69 & 80.66 & 81.56 & 80.58 & 81.18 & 81.34  \\
 \hline
 16 & 79.95 & 79.56 & 79.79 & 79.72 & 81.28 & 81.66 & 81.28 & 80.96  \\
 \hline
\end{tabular}
\caption{\label{tab:mmlu_abalation} Accuracy on LM evaluation harness tasks on Llama2-70B model.}
\end{table}

%\section{MSE Studies}
%\textcolor{red}{TODO}


\subsection{Number Formats and Quantization Method}
\label{subsec:numFormats_quantMethod}
\subsubsection{Integer Format}
An $n$-bit signed integer (INT) is typically represented with a 2s-complement format \citep{yao2022zeroquant,xiao2023smoothquant,dai2021vsq}, where the most significant bit denotes the sign.

\subsubsection{Floating Point Format}
An $n$-bit signed floating point (FP) number $x$ comprises of a 1-bit sign ($x_{\mathrm{sign}}$), $B_m$-bit mantissa ($x_{\mathrm{mant}}$) and $B_e$-bit exponent ($x_{\mathrm{exp}}$) such that $B_m+B_e=n-1$. The associated constant exponent bias ($E_{\mathrm{bias}}$) is computed as $(2^{{B_e}-1}-1)$. We denote this format as $E_{B_e}M_{B_m}$.  

\subsubsection{Quantization Scheme}
\label{subsec:quant_method}
A quantization scheme dictates how a given unquantized tensor is converted to its quantized representation. We consider FP formats for the purpose of illustration. Given an unquantized tensor $\bm{X}$ and an FP format $E_{B_e}M_{B_m}$, we first, we compute the quantization scale factor $s_X$ that maps the maximum absolute value of $\bm{X}$ to the maximum quantization level of the $E_{B_e}M_{B_m}$ format as follows:
\begin{align}
\label{eq:sf}
    s_X = \frac{\mathrm{max}(|\bm{X}|)}{\mathrm{max}(E_{B_e}M_{B_m})}
\end{align}
In the above equation, $|\cdot|$ denotes the absolute value function.

Next, we scale $\bm{X}$ by $s_X$ and quantize it to $\hat{\bm{X}}$ by rounding it to the nearest quantization level of $E_{B_e}M_{B_m}$ as:

\begin{align}
\label{eq:tensor_quant}
    \hat{\bm{X}} = \text{round-to-nearest}\left(\frac{\bm{X}}{s_X}, E_{B_e}M_{B_m}\right)
\end{align}

We perform dynamic max-scaled quantization \citep{wu2020integer}, where the scale factor $s$ for activations is dynamically computed during runtime.

\subsection{Vector Scaled Quantization}
\begin{wrapfigure}{r}{0.35\linewidth}
  \centering
  \includegraphics[width=\linewidth]{sections/figures/vsquant.jpg}
  \caption{\small Vectorwise decomposition for per-vector scaled quantization (VSQ \citep{dai2021vsq}).}
  \label{fig:vsquant}
\end{wrapfigure}
During VSQ \citep{dai2021vsq}, the operand tensors are decomposed into 1D vectors in a hardware friendly manner as shown in Figure \ref{fig:vsquant}. Since the decomposed tensors are used as operands in matrix multiplications during inference, it is beneficial to perform this decomposition along the reduction dimension of the multiplication. The vectorwise quantization is performed similar to tensorwise quantization described in Equations \ref{eq:sf} and \ref{eq:tensor_quant}, where a scale factor $s_v$ is required for each vector $\bm{v}$ that maps the maximum absolute value of that vector to the maximum quantization level. While smaller vector lengths can lead to larger accuracy gains, the associated memory and computational overheads due to the per-vector scale factors increases. To alleviate these overheads, VSQ \citep{dai2021vsq} proposed a second level quantization of the per-vector scale factors to unsigned integers, while MX \citep{rouhani2023shared} quantizes them to integer powers of 2 (denoted as $2^{INT}$).

\subsubsection{MX Format}
The MX format proposed in \citep{rouhani2023microscaling} introduces the concept of sub-block shifting. For every two scalar elements of $b$-bits each, there is a shared exponent bit. The value of this exponent bit is determined through an empirical analysis that targets minimizing quantization MSE. We note that the FP format $E_{1}M_{b}$ is strictly better than MX from an accuracy perspective since it allocates a dedicated exponent bit to each scalar as opposed to sharing it across two scalars. Therefore, we conservatively bound the accuracy of a $b+2$-bit signed MX format with that of a $E_{1}M_{b}$ format in our comparisons. For instance, we use E1M2 format as a proxy for MX4.

\begin{figure}
    \centering
    \includegraphics[width=1\linewidth]{sections//figures/BlockFormats.pdf}
    \caption{\small Comparing LO-BCQ to MX format.}
    \label{fig:block_formats}
\end{figure}

Figure \ref{fig:block_formats} compares our $4$-bit LO-BCQ block format to MX \citep{rouhani2023microscaling}. As shown, both LO-BCQ and MX decompose a given operand tensor into block arrays and each block array into blocks. Similar to MX, we find that per-block quantization ($L_b < L_A$) leads to better accuracy due to increased flexibility. While MX achieves this through per-block $1$-bit micro-scales, we associate a dedicated codebook to each block through a per-block codebook selector. Further, MX quantizes the per-block array scale-factor to E8M0 format without per-tensor scaling. In contrast during LO-BCQ, we find that per-tensor scaling combined with quantization of per-block array scale-factor to E4M3 format results in superior inference accuracy across models. 


\end{document}
