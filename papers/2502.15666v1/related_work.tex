%Detecting AI-generated text has become increasingly important as language models produce more human-like content. There are statistical metrics like perplexity and n-gram frequency \citep{gehrmann2019gltr, wu2023llmdet, hans2024spotting} to distinguish AI-text from HWT. On the other hand, some approaches use machine learning classifiers, such as fine-tuned BERT and RoBERTa models \citep{hu2023radar, guo-etal-2023-hc3, solaiman2019release}. However, these detection techniques are designed primarily for distinguishing purely human-written (HWT) from fully AI-generated texts.

Detecting AI-generated text is crucial as models become more human-like. Traditional methods use statistical metrics like perplexity and n-gram frequency \citep{gehrmann2019gltr, wu2023llmdet, hans2024spotting}, while others rely on machine learning classifiers like BERT and RoBERTa \citep{hu2023radar, guo-etal-2023-hc3, solaiman2019release}. However, these approaches mainly differentiate pure AI and human-written text.

%Some prior studies have explored paraphrasing as a means to evade AI detectors \citep{sadasivan2023can, krishna2023paraphrasing}, but they do not specifically evaluate the unreliability of detection models in AI-polished text scenarios, which we address. Other works \citep{dugan2023real, zeng2024towards} focus on detecting the boundary between HWT and AI-generated text, treating the sentences as distinct entities. More recent research \citep{gao2024llm, yang2024chatgpt} has investigated LLM-assisted text polishing, but without considering varied degrees of AI involvement. Our work extends this line of research by systematically analyzing AI involvement at multiple levels, using different polishing strategies. We evaluate how current AI-text detectors perform when faced with AI-polished text, identifying potential gaps and limitations in existing detection frameworks.

Prior studies explored paraphrasing to evade AI detectors \citep{sadasivan2023can, krishna2023paraphrasing}, but not AI-polished text. Others focused on AI-human text boundaries \citep{dugan2023real, zeng2024towards}, while recent work examined AI-assisted polishing without varying AI involvement \citep{gao2024llm, yang2024chatgpt}. We extend this by systematically analyzing AI-polished text across multiple levels, assessing detection limitations.