\documentclass{article}

% Recommended, but optional, packages for figures and better typesetting:
\usepackage{microtype}
\usepackage{graphicx}
%\usepackage{subfigure}
\usepackage{subcaption}
\usepackage{multicol, multirow}
\usepackage{makecell}
\usepackage{booktabs} % for professional tables

% hyperref makes hyperlinks in the resulting PDF.
% If your build breaks (sometimes temporarily if a hyperlink spans a page)
% please comment out the following usepackage line and replace
% \usepackage{icml2024} with \usepackage[nohyperref]{icml2024} above.
\usepackage{hyperref}

% Attempt to make hyperref and algorithmic work together better:
\newcommand{\theHalgorithm}{\arabic{algorithm}}

% Use the following line for the initial blind version submitted for review:
\usepackage[accepted]{icml2024}

% For theorems and such
\usepackage{amsmath}
\usepackage{amssymb}
\usepackage{mathtools}
\usepackage{amsthm}
\usepackage{thmtools} 
\usepackage{thm-restate}

% if you use cleveref..
\usepackage[capitalize,noabbrev]{cleveref}

%%%%%%%%%%%%%%%%%%%%%%%%%%%%%%%%
% THEOREMS
%%%%%%%%%%%%%%%%%%%%%%%%%%%%%%%%
\theoremstyle{plain}
\newtheorem{theorem}{Theorem}[section]
\newtheorem{proposition}[theorem]{Proposition}
\newtheorem{lemma}[theorem]{Lemma}
\newtheorem{corollary}[theorem]{Corollary}
\newtheorem{question}[theorem]{Open Question}
\theoremstyle{definition}
\newtheorem{definition}[theorem]{Definition}
\newtheorem{assumption}[theorem]{Assumption}
\theoremstyle{remark}
\newtheorem{remark}[theorem]{Remark}
\newcommand{\ltwo}{$\ell_2$}
\newcommand{\mytexttilde}{\raisebox{0.5ex}{\texttildelow}}

% Todonotes is useful during development; simply uncomment the next line
%    and comment out the line below the next line to turn off comments
%\usepackage[disable,textsize=tiny]{todonotes}
\usepackage[textsize=tiny]{todonotes}
% \usepackage[dvipsnames]{xcolor}


% Therefore, a short form for the running title is supplied here:
\icmltitlerunning{On the Importance of Embedding Norms in Self-Supervised Learning}

% Own packages
\usepackage{listings}
\lstset{
    language=Python,
    basicstyle=\ttfamily\small,
    keywordstyle=\color{blue}\bfseries,
    stringstyle=\color{red},
    commentstyle=\color{green},
    %numbers=left,
    %numberstyle=\tiny\color{gray},
    %stepnumber=1,
    breaklines=true,
    frame=none,
}

\begin{document}

\twocolumn[
\icmltitle{On the Importance of Embedding Norms in Self-Supervised Learning}

% Other title options:
%   - The Forgotten Parameter of Self-Supervised Learning
%   - Why Are We Ignoring the Embedding Norms in Self-Supervised Learning?

% Embedding norm and self-supervised learning should both be in the title

% It is OKAY to include author information, even for blind
% submissions: the style file will automatically remove it for you
% unless you've provided the [accepted] option to the icml2024
% package.

% List of affiliations: The first argument should be a (short)
% identifier you will use later to specify author affiliations
% Academic affiliations should list Department, University, City, Region, Country
% Industry affiliations should list Company, City, Region, Country

% You can specify symbols, otherwise they are numbered in order.
% Ideally, you should not use this facility. Affiliations will be numbered
% in order of appearance and this is the preferred way.
\icmlsetsymbol{equal}{*}

\begin{icmlauthorlist}
\icmlauthor{Andrew Draganov}{au}
\icmlauthor{Sharvaree Vadgama}{uva}
\icmlauthor{Sebastian Damrich}{tub}
\icmlauthor{Jan Niklas Böhm}{tub}
\icmlauthor{Lucas Maes}{mila}
\icmlauthor{Dmitry Kobak}{tub}
\icmlauthor{Erik Bekkers}{uva}
%\icmlauthor{}{sch}
%\icmlauthor{}{sch}
\end{icmlauthorlist}

\icmlaffiliation{au}{Aarhus University, Denmark}
\icmlaffiliation{uva}{AMLab, University of Amsterdam, The Netherlands}
\icmlaffiliation{mila}{Mila, Quebec AI Institute, Canada}
\icmlaffiliation{tub}{Hertie Institute for AI in Brain Health, University of Tübingen, Germany}

\icmlcorrespondingauthor{Andrew Draganov}{draganovandrew@gmail.com}

% You may provide any keywords that you
% find helpful for describing your paper; these are used to populate
% the "keywords" metadata in the PDF but will not be shown in the document
\icmlkeywords{Machine Learning, ICML}

\vskip 0.3in
]

% this must go after the closing bracket ] following \twocolumn[ ...

% This command actually creates the footnote in the first column
% listing the affiliations and the copyright notice.
% The command takes one argument, which is text to display at the start of the footnote.
% The \icmlEqualContribution command is standard text for equal contribution.
% Remove it (just {}) if you do not need this facility.

\printAffiliationsAndNotice{}  % leave blank if no need to mention equal contribution
%\printAffiliationsAndNotice{\icmlEqualContribution} % otherwise use the standard text.

\begin{abstract}
\begin{abstract}
  In this work, we present a novel technique for GPU-accelerated Boolean satisfiability (SAT) sampling. Unlike conventional sampling algorithms that directly operate on conjunctive normal form (CNF), our method transforms the logical constraints of SAT problems by factoring their CNF representations into simplified multi-level, multi-output Boolean functions. It then leverages gradient-based optimization to guide the search for a diverse set of valid solutions. Our method operates directly on the circuit structure of refactored SAT instances, reinterpreting the SAT problem as a supervised multi-output regression task. This differentiable technique enables independent bit-wise operations on each tensor element, allowing parallel execution of learning processes. As a result, we achieve GPU-accelerated sampling with significant runtime improvements ranging from $33.6\times$ to $523.6\times$ over state-of-the-art heuristic samplers. We demonstrate the superior performance of our sampling method through an extensive evaluation on $60$ instances from a public domain benchmark suite utilized in previous studies. 


  
  % Generating a wide range of diverse solutions to logical constraints is crucial in software and hardware testing, verification, and synthesis. These solutions can serve as inputs to test specific functionalities of a software program or as random stimuli in hardware modules. In software verification, techniques like fuzz testing and symbolic execution use this approach to identify bugs and vulnerabilities. In hardware verification, stimulus generation is particularly vital, forming the basis of constrained-random verification. While generating multiple solutions improves coverage and increases the chances of finding bugs, high-throughput sampling remains challenging, especially with complex constraints and refined coverage criteria. In this work, we present a novel technique that enables GPU-accelerated sampling, resulting in high-throughput generation of satisfying solutions to Boolean satisfiability (SAT) problems. Unlike conventional sampling algorithms that directly operate on conjunctive normal form (CNF), our method refines the logical constraints of SAT problems by transforming their CNF into simplified multi-level Boolean expressions. It then leverages gradient-based optimization to guide the search for a diverse set of valid solutions.
  % Our method specifically takes advantage of the circuit structure of refined SAT instances by using GD to learn valid solutions, reinterpreting the SAT problem as a supervised multi-output regression task. This differentiable technique enables independent bit-wise operations on each tensor element, allowing parallel execution of learning processes. As a result, we achieve GPU-accelerated sampling with significant runtime improvements ranging from $10\times$ to $1000\times$ over state-of-the-art heuristic samplers. Specifically, we demonstrate the superior performance of our sampling method through an extensive evaluation on $60$ instances from a public domain benchmark suite utilized in previous studies.

\end{abstract}

\begin{IEEEkeywords}
Boolean Satisfiability, Gradient Descent, Multi-level Circuits, Verification, and Testing.
\end{IEEEkeywords}
\end{abstract}

\section{Introduction}
\label{section:introduction}

% redirection is unique and important in VR
Virtual Reality (VR) systems enable users to embody virtual avatars by mirroring their physical movements and aligning their perspective with virtual avatars' in real time. 
As the head-mounted displays (HMDs) block direct visual access to the physical world, users primarily rely on visual feedback from the virtual environment and integrate it with proprioceptive cues to control the avatar’s movements and interact within the VR space.
Since human perception is heavily influenced by visual input~\cite{gibson1933adaptation}, 
VR systems have the unique capability to control users' perception of the virtual environment and avatars by manipulating the visual information presented to them.
Leveraging this, various redirection techniques have been proposed to enable novel VR interactions, 
such as redirecting users' walking paths~\cite{razzaque2005redirected, suma2012impossible, steinicke2009estimation},
modifying reaching movements~\cite{gonzalez2022model, azmandian2016haptic, cheng2017sparse, feick2021visuo},
and conveying haptic information through visual feedback to create pseudo-haptic effects~\cite{samad2019pseudo, dominjon2005influence, lecuyer2009simulating}.
Such redirection techniques enable these interactions by manipulating the alignment between users' physical movements and their virtual avatar's actions.

% % what is hand/arm redirection, motivation of study arm-offset
% \change{\yj{i don't understand the purpose of this paragraph}
% These illusion-based techniques provide users with unique experiences in virtual environments that differ from the physical world yet maintain an immersive experience. 
% A key example is hand redirection, which shifts the virtual hand’s position away from the real hand as the user moves to enhance ergonomics during interaction~\cite{feuchtner2018ownershift, wentzel2020improving} and improve interaction performance~\cite{montano2017erg, poupyrev1996go}. 
% To increase the realism of virtual movements and strengthen the user’s sense of embodiment, hand redirection techniques often incorporate a complete virtual arm or full body alongside the redirected virtual hand, using inverse kinematics~\cite{hartfill2021analysis, ponton2024stretch} or adjustments to the virtual arm's movement as well~\cite{li2022modeling, feick2024impact}.
% }

% noticeability, motivation of predicting a probability, not a classification
However, these redirection techniques are most effective when the manipulation remains undetected~\cite{gonzalez2017model, li2022modeling}. 
If the redirection becomes too large, the user may not mitigate the conflict between the visual sensory input (redirected virtual movement) and their proprioception (actual physical movement), potentially leading to a loss of embodiment with the virtual avatar and making it difficult for the user to accurately control virtual movements to complete interaction tasks~\cite{li2022modeling, wentzel2020improving, feuchtner2018ownershift}. 
While proprioception is not absolute, users only have a general sense of their physical movements and the likelihood that they notice the redirection is probabilistic. 
This probability of detecting the redirection is referred to as \textbf{noticeability}~\cite{li2022modeling, zenner2024beyond, zenner2023detectability} and is typically estimated based on the frequency with which users detect the manipulation across multiple trials.

% version B
% Prior research has explored factors influencing the noticeability of redirected motion, including the redirection's magnitude~\cite{wentzel2020improving, poupyrev1996go}, direction~\cite{li2022modeling, feuchtner2018ownershift}, and the visual characteristics of the virtual avatar~\cite{ogawa2020effect, feick2024impact}.
% While these factors focus on the avatars, the surrounding virtual environment can also influence the users' behavior and in turn affect the noticeability of redirection.
% One such prominent external influence is through the visual channel - the users' visual attention is constantly distracted by complex visual effects and events in practical VR scenarios.
% Although some prior studies have explored how to leverage user blindness caused by visual distractions to redirect users' virtual hand~\cite{zenner2023detectability}, there remains a gap in understanding how to quantify the noticeability of redirection under visual distractions.

% visual stimuli and gaze behavior
Prior research has explored factors influencing the noticeability of redirected motion, including the redirection's magnitude~\cite{wentzel2020improving, poupyrev1996go}, direction~\cite{li2022modeling, feuchtner2018ownershift}, and the visual characteristics of the virtual avatar~\cite{ogawa2020effect, feick2024impact}.
While these factors focus on the avatars, the surrounding virtual environment can also influence the users' behavior and in turn affect the noticeability of redirection.
This, however, remains underexplored.
One such prominent external influence is through the visual channel - the users' visual attention is constantly distracted by complex visual effects and events in practical VR scenarios.
We thus want to investigate how \textbf{visual stimuli in the virtual environment} affect the noticeability of redirection.
With this, we hope to complement existing works that focus on avatars by incorporating environmental visual influences to enable more accurate control over the noticeability of redirected motions in practical VR scenarios.
% However, in realistic VR applications, the virtual environment often contains complex visual effects beyond the virtual avatar itself. 
% We argue that these visual effects can \textbf{distract users’ visual attention and thus affect the noticeability of redirection offsets}, while current research has yet taken into account.
% For instance, in a VR boxing scenario, a user’s visual attention is likely focused on their opponent rather than on their virtual body, leading to a lower noticeability of redirection offsets on their virtual movements. 
% Conversely, when reaching for an object in the center of their field of view, the user’s attention is more concentrated on the virtual hand’s movement and position to ensure successful interaction, resulting in a higher noticeability of offsets.

Since each visual event is a complex choreography of many underlying factors (type of visual effect, location, duration, etc.), it is extremely difficult to quantify or parameterize visual stimuli.
Furthermore, individuals respond differently to even the same visual events.
Prior neuroscience studies revealed that factors like age, gender, and personality can influence how quickly someone reacts to visual events~\cite{gillon2024responses, gale1997human}. 
Therefore, aiming to model visual stimuli in a way that is generalizable and applicable to different stimuli and users, we propose to use users' \textbf{gaze behavior} as an indicator of how they respond to visual stimuli.
In this paper, we used various gaze behaviors, including gaze location, saccades~\cite{krejtz2018eye}, fixations~\cite{perkhofer2019using}, and the Index of Pupil Activity (IPA)~\cite{duchowski2018index}.
These behaviors indicate both where users are looking and their cognitive activity, as looking at something does not necessarily mean they are attending to it.
Our goal is to investigate how these gaze behaviors stimulated by various visual stimuli relate to the noticeability of redirection.
With this, we contribute a model that allows designers and content creators to adjust the redirection in real-time responding to dynamic visual events in VR.

To achieve this, we conducted user studies to collect users' noticeability of redirection under various visual stimuli.
To simulate realistic VR scenarios, we adopted a dual-task design in which the participants performed redirected movements while monitoring the visual stimuli.
Specifically, participants' primary task was to report if they noticed an offset between the avatar's movement and their own, while their secondary task was to monitor and report the visual stimuli.
As realistic virtual environments often contain complex visual effects, we started with simple and controlled visual stimulus to manage the influencing factors.

% first user study, confirmation study
% collect data under no visual stimuli, different basic visual stimuli
We first conducted a confirmation study (N=16) to test whether applying visual stimuli (opacity-based) actually affects their noticeability of redirection. 
The results showed that participants were significantly less likely to detect the redirection when visual stimuli was presented $(F_{(1,15)}=5.90,~p=0.03)$.
Furthermore, by analyzing the collected gaze data, results revealed a correlation between the proposed gaze behaviors and the noticeability results $(r=-0.43)$, confirming that the gaze behaviors could be leveraged to compute the noticeability.

% data collection study
We then conducted a data collection study to obtain more accurate noticeability results through repeated measurements to better model the relationship between visual stimuli-triggered gaze behaviors and noticeability of redirection.
With the collected data, we analyzed various numerical features from the gaze behaviors to identify the most effective ones. 
We tested combinations of these features to determine the most effective one for predicting noticeability under visual stimuli.
Using the selected features, our regression model achieved a mean squared error (MSE) of 0.011 through leave-one-user-out cross-validation. 
Furthermore, we developed both a binary and a three-class classification model to categorize noticeability, which achieved an accuracy of 91.74\% and 85.62\%, respectively.

% evaluation study
To evaluate the generalizability of the regression model, we conducted an evaluation study (N=24) to test whether the model could accurately predict noticeability with new visual stimuli (color- and scale-based animations).
Specifically, we evaluated whether the model's predictions aligned with participants' responses under these unseen stimuli.
The results showed that our model accurately estimated the noticeability, achieving mean squared errors (MSE) of 0.014 and 0.012 for the color- and scale-based visual stimili, respectively, compared to participants' responses.
Since the tested visual stimuli data were not included in the training, the results suggested that the extracted gaze behavior features capture a generalizable pattern and can effectively indicate the corresponding impact on the noticeability of redirection.

% application
Based on our model, we implemented an adaptive redirection technique and demonstrated it through two applications: adaptive VR action game and opportunistic rendering.
We conducted a proof-of-concept user study (N=8) to compare our adaptive redirection technique with a static redirection, evaluating the usability and benefits of our adaptive redirection technique.
The results indicated that participants experienced less physical demand and stronger sense of embodiment and agency when using the adaptive redirection technique. 
These results demonstrated the effectiveness and usability of our model.

In summary, we make the following contributions.
% 
\begin{itemize}
    \item 
    We propose to use users' gaze behavior as a medium to quantify how visual stimuli influences the noticebility of redirection. 
    Through two user studies, we confirm that visual stimuli significantly influences noticeability and identify key gaze behavior features that are closely related to this impact.
    \item 
    We build a regression model that takes the user's gaze behavioral data as input, then computes the noticeability of redirection.
    Through an evaluation study, we verify that our model can estimate the noticeability with new participants under unseen visual stimuli.
    These findings suggest that the extracted gaze behavior features effectively capture the influence of visual stimuli on noticeability and can generalize across different users and visual stimuli.
    \item 
    We develop an adaptive redirection technique based on our regression model and implement two applications with it.
    With a proof-of-concept study, we demonstrate the effectiveness and potential usability of our regression model on real-world use cases.

\end{itemize}

% \delete{
% Virtual Reality (VR) allows the user to embody a virtual avatar by mirroring their physical movements through the avatar.
% As the user's visual access to the physical world is blocked in tasks involving motion control, they heavily rely on the visual representation of the avatar's motions to guide their proprioception.
% Similar to real-world experiences, the user is able to resolve conflicts between different sensory inputs (e.g., vision and motor control) through multisensory integration, which is essential for mitigating the sensory noise that commonly arises.
% However, it also enables unique manipulations in VR, as the system can intentionally modify the avatar's movements in relation to the user's motions to achieve specific functional outcomes,
% for example, 
% % the manipulations on the avatar's movements can 
% enabling novel interaction techniques of redirected walking~\cite{razzaque2005redirected}, redirected reaching~\cite{gonzalez2022model}, and pseudo haptics~\cite{samad2019pseudo}.
% With small adjustments to the avatar's movements, the user can maintain their sense of embodiment, due to their ability to resolve the perceptual differences.
% % However, a large mismatch between the user and avatar's movements can result in the user losing their sense of embodiment, due to an inability to resolve the perceptual differences.
% }

% \delete{
% However, multisensory integration can break when the manipulation is so intense that the user is aware of the existence of the motion offset and no longer maintains the sense of embodiment.
% Prior research studied the intensity threshold of the offset applied on the avatar's hand, beyond which the embodiment will break~\cite{li2022modeling}. 
% Studies also investigated the user's sensitivity to the offsets over time~\cite{kohm2022sensitivity}.
% Based on the findings, we argue that one crucial factor that affects to what extent the user notices the offset (i.e., \textit{noticeability}) that remains under-explored is whether the user directs their visual attention towards or away from the virtual avatar.
% Related work (e.g., Mise-unseen~\cite{marwecki2019mise}) has showcased applications where adjustments in the environment can be made in an unnoticeable manner when they happen in the area out of the user's visual field.
% We hypothesize that directing the user's visual attention away from the avatar's body, while still partially keeping the avatar within the user's field-of-view, can reduce the noticeability of the offset.
% Therefore, we conduct two user studies and implement a regression model to systematically investigate this effect.
% }

% \delete{
% In the first user study (N = 16), we test whether drawing the user's visual attention away from their body impacts the possibility of them noticing an offset that we apply to their arm motion in VR.
% We adopt a dual-task design to enable the alteration of the user's visual attention and a yes/no paradigm to measure the noticeability of motion offset. 
% The primary task for the user is to perform an arm motion and report when they perceive an offset between the avatar's virtual arm and their real arm.
% In the secondary task, we randomly render a visual animation of a ball turning from transparent to red and becoming transparent again and ask them to monitor and report when it appears.
% We control the strength of the visual stimuli by changing the duration and location of the animation.
% % By changing the time duration and location of the visual animation, we control the strengths of attraction to the users.
% As a result, we found significant differences in the noticeability of the offsets $(F_{(1,15)}=5.90,~p=0.03)$ between conditions with and without visual stimuli.
% Based on further analysis, we also identified the behavioral patterns of the user's gaze (including pupil dilation, fixations, and saccades) to be correlated with the noticeability results $(r=-0.43)$ and they may potentially serve as indicators of noticeability.
% }

% \delete{
% To further investigate how visual attention influences the noticeability, we conduct a data collection study (N = 12) and build a regression model based on the data.
% The regression model is able to calculate the noticeability of the offset applied on the user's arm under various visual stimuli based on their gaze behaviors.
% Our leave-one-out cross-validation results show that the proposed method was able to achieve a mean-squared error (MSE) of 0.012 in the probability regression task.
% }

% \delete{
% To verify the feasibility and extendability of the regression model, we conduct an evaluation study where we test new visual animations based on adjustments on scale and color and invite 24 new participants to attend the study.
% Results show that the proposed method can accurately estimate the noticeability with an MSE of 0.014 and 0.012 in the conditions of the color- and scale-based visual effects.
% Since these animations were not included in the dataset that the regression model was built on, the study demonstrates that the gaze behavioral features we extracted from the data capture a generalizable pattern of the user's visual attention and can indicate the corresponding impact on the noticeability of the offset.
% }

% \delete{
% Finally, we demonstrate applications that can benefit from the noticeability prediction model, including adaptive motion offsets and opportunistic rendering, considering the user's visual attention. 
% We conclude with discussions of our work's limitations and future research directions.
% }

% \delete{
% In summary, we make the following contributions.
% }
% % 
% \begin{itemize}
%     \item 
%     \delete{
%     We quantify the effects of the user's visual attention directed away by stimuli on their noticeability of an offset applied to the avatar's arm motion with respect to the user's physical arm. 
%     Through two user studies, we identified gaze behavioral features that are indicative of the changes in noticeability.
%     }
%     \item 
%     \delete{We build a regression model that takes the user's gaze behavioral data and the offset applied to the arm motion as input, then computes the probability of the user noticing the offset.
%     Through an evaluation study, we verified that the model needs no information about the source attracting the user's visual attention and can be generalizable in different scenarios.
%     }
%     \item 
%     \delete{We demonstrate two applications that potentially benefit from the regression model, including adaptive motion offsets and opportunistic rendering.
%     }

% \end{itemize}

\begin{comment}
However, users will lose the sense of embodiment to the virtual avatars if they notice the offset between the virtual and physical movements.
To address this, researchers have been exploring the noticing threshold of offsets with various magnitudes and proposing various redirection techniques that maintain the sense of embodiment~\cite{}.

However, when users embody virtual avatars to explore virtual environments, they encounter various visual effects and content that can attract their attention~\cite{}.
During this, the user may notice an offset when he observes the virtual movement carefully while ignoring it when the virtual contents attract his attention from the movements.
Therefore, static offset thresholds are not appropriate in dynamic scenarios.

Past research has proposed dynamic mapping techniques that adapted to users' state, such as hand moving speed~\cite{frees2007prism} or ergonomically comfortable poses~\cite{montano2017erg}, but not considering the influence of virtual content.
More specifically, PRISM~\cite{frees2007prism} proposed adjusting the C/D ratio with a non-linear mapping according to users' hand moving speed, but it might not be optimal for various virtual scenarios.
While Erg-O~\cite{montano2017erg} redirected users' virtual hands according to the virtual target's relative position to reduce physical fatigue, neglecting the change of virtual environments. 

Therefore, how to design redirection techniques in various scenarios with different visual attractions remains unknown.
To address this, we investigate how visual attention affects the noticing probability of movement offsets.
Based on our experiments, we implement a computational model that automatically computes the noticing probability of offsets under certain visual attractions.
VR application designers and developers can easily leverage our model to design redirection techniques maintaining the sense of embodiment adapt to the user's visual attention.
We implement a dynamic redirection technique with our model and demonstrate that it effectively reduces the target reaching time without reducing the sense of embodiment compared to static redirection techniques.

% Need to be refined
This paper offers the following contributions.
\begin{itemize}
    \item We investigate how visual attractions affect the noticing probability of redirection offsets.
    \item We construct a computational model to predict the noticing probability of an offset with a given visual background.
    \item We implement a dynamic redirection technique adapting to the visual background. We evaluate the technique and develop three applications to demonstrate the benefits. 
\end{itemize}



First, we conducted a controlled experiment to understand how users perceived the movement offset while subjected to various distractions.
Since hand redirection is one of the most frequently used redirections in VR interactions, we focused on the dynamic arm movements and manually added angular offsets to the' elbow joint~\cite{li2022modeling, gonzalez2022model, zenner2019estimating}. 
We employed flashing spheres in the user's field of view as distractions to attract users' visual attention.
Participants were instructed to report the appearing location of the spheres while simultaneously performing the arm movements and reporting if they perceived an offset during the movement. 
(\zhipeng{Add the results of data collection. Analyze the influence of the distance between the gaze map and the offset.}
We measured the visual attraction's magnitude with the gaze distribution on it.
Results showed that stronger distractions made it harder for users to notice the offset.)
\zhipeng{Need to rewrite. Not sure to use gaze distribution or a metric obtained from the visual content.}
Secondly, we constructed a computational model to predict the noticing probability of offsets with given visual content.
We analyzed the data from the user studies to measure the influence of visual attractions on the noticing probability of offsets.
We built a statistical model to predict the offset's noticing probability with a given visual content.
Based on the model, we implement a dynamic redirection technique to adjust the redirection offset adapted to the user's current field of view.
We evaluated the technique in a target selection task compared to no hand redirection and static hand redirection.
\zhipeng{Add the results of the evaluation.}
Results showed that the dynamic hand redirection technique significantly reduced the target selection time with similar accuracy and a comparable sense of embodiment.
Finally, we implemented three applications to demonstrate the potential benefits of the visual attention adapted dynamic redirection technique.
\end{comment}

% This one modifies arm length, not redirection
% \citeauthor{mcintosh2020iteratively} proposed an adaptation method to iteratively change the virtual avatar arm's length based on the primary tasks' performance~\cite{mcintosh2020iteratively}.



% \zhipeng{TO ADD: what is redirection}
% Redirection enables novel interactions in Virtual Reality, including redirected walking, haptic redirection, and pseudo haptics by introducing an offset to users' movement.
% \zhipeng{TO ADD: extend this sentence}
% The price of this is that users' immersiveness and embodiment in VR can be compromised when they notice the offset and perceive the virtual movement not as theirs~\cite{}.
% \zhipeng{TO ADD: extend this sentence, elaborate how the virtual environment attracts users' attention}
% Meanwhile, the visual content in the virtual environment is abundant and consistently captures users' attention, making it harder to notice the offset~\cite{}.
% While previous studies explored the noticing threshold of the offsets and optimized the redirection techniques to maintain the sense of embodiment~\cite{}, the influence of visual content on the probability of perceiving offsets remains unknown.  
% Therefore, we propose to investigate how users perceive the redirection offset when they are facing various visual attractions.


% We conducted a user study to understand how users notice the shift with visual attractions.
% We used a color-changing ball to attract the user's attention while instructing users to perform different poses with their arms and observe it meanwhile.
% \zhipeng{(Which one should be the primary task? Observe the ball should be the primary one, but if the primary task is too simple, users might allocate more attention on the secondary task and this makes the secondary task primary.)}
% \zhipeng{(We need a good and reasonable dual-task design in which users care about both their pose and the visual content, at least in the evaluation study. And we need to be able to control the visual content's magnitude and saliency maybe?)}
% We controlled the shift magnitude and direction, the user's pose, the ball's size, and the color range.
% We set the ball's color-changing interval as the independent factor.
% We collect the user's response to each shift and the color-changing times.
% Based on the collected data, we constructed a statistical model to describe the influence of visual attraction on the noticing probability.
% \zhipeng{(Are we actually controlling the attention allocation? How do we measure the attracting effect? We need uniform metrics, otherwise it is also hard for others to use our knowledge.)}
% \zhipeng{(Try to use eye gaze? The eye gaze distribution in the last five seconds to decide the attention allocation? Basically constructing a model with eye gaze distribution and noticing probability. But the user's head is moving, so the eye gaze distribution is not aligned well with the current view.)}

% \zhipeng{Saliency and EMD}
% \zhipeng{Gaze is more than just a point: Rethinking visual attention
% analysis using peripheral vision-based gaze mapping}

% Evaluation study(ideal case): based on the visual content, adjusting the redirection magnitude dynamically.

% \zhipeng{(The risk is our model's effect is trivial.)}

% Applications:
% Playing Lego while watching demo videos, we can accelerate the reaching process of bricks, and forbid the redirection during the manipulation.

% Beat saber again: but not make a lot of sense? Difficult game has complicated visual effects, while allows larger shift, but do not need large shift with high difficulty



\section{Related Work}
\label{lit_review}

\begin{highlight}
{

Our research builds upon {\em (i)} Assessing Web Accessibility, {\em (ii)} End-User Accessibility Repair, and {\em (iii)} Developer Tools for Accessibility.

\subsection{Assessing Web Accessibility}
From the earliest attempts to set standards and guidelines, web accessibility has been shaped by a complex interplay of technical challenges, legal imperatives, and educational campaigns. Over the past 25 years, stakeholders have sought to improve digital inclusion by establishing foundational standards~\cite{chisholm2001web, caldwell2008web}, enforcing legal obligations~\cite{sierkowski2002achieving, yesilada2012understanding}, and promoting a broader culture of accessibility awareness among developers~\cite{sloan2006contextual, martin2022landscape, pandey2023blending}. 
Despite these longstanding efforts, systemic accessibility issues persist. According to the 2024 WebAIM Million report~\cite{webaim2024}, 95.9\% of the top one million home pages contained detectable WCAG violations, averaging nearly 57 errors per page. 
These errors take many forms: low color contrast makes the interface difficult for individuals with color deficiency or low vision to read text; missing alternative text leaves users relying on screen readers without crucial visual context; and unlabeled form inputs or empty links and buttons hinder people who navigate with assistive technologies from completing basic tasks. 
Together, these accessibility issues not only limit user access to critical online resources such as healthcare, education, and employment but also result in significant legal risks and lost opportunities for businesses to engage diverse audiences. Addressing these pervasive issues requires systematic methods to identify, measure, and prioritize accessibility barriers, which is the first step toward achieving meaningful improvements.

Prior research has introduced methods blending automation and human evaluation to assess web accessibility. Hybrid approaches like SAMBA combine automated tools with expert reviews to measure the severity and impact of barriers, enhancing evaluation reliability~\cite{brajnik2007samba}. Quantitative metrics, such as Failure Rate and Unified Web Evaluation Methodology, support large-scale monitoring and comparative analysis, enabling cost-effective insights~\cite{vigo2007quantitative, martins2024large}. However, automated tools alone often detect less than half of WCAG violations and generate false positives, emphasizing the need for human interpretation~\cite{freire2008evaluation, vigo2013benchmarking}. Recent progress with large pretrained models like Large Language Models (LLMs)~\cite{dubey2024llama,bai2023qwen} and Large Multimodal Models (LMMs)~\cite{liu2024visual, bai2023qwenvl} offers a promising step forward, automating complex checks like non-text content evaluation and link purposes, achieving higher detection rates than traditional tools~\cite{lopez2024turning, delnevo2024interaction}. Yet, these large models face challenges, including dependence on training data, limited contextual judgment, and the inability to simulate real user experiences. These limitations underscore the necessity of combining models with human oversight for reliable, user-centered evaluations~\cite{brajnik2007samba, vigo2013benchmarking, delnevo2024interaction}. 

Our work builds on these prior efforts and recent advancements by leveraging the capabilities of large pretrained models while addressing their limitations through a developer-centric approach. CodeA11y integrates LLM-powered accessibility assessments, tailored accessibility-aware system prompts, and a dedicated accessibility checker directly into GitHub Copilot---one of the most widely used coding assistants. Unlike standalone evaluation tools, CodeA11y actively supports developers throughout the coding process by reinforcing accessibility best practices, prompting critical manual validations, and embedding accessibility considerations into existing workflows.
% This pervasive shortfall reflects the difficulty of scaling traditional approaches---such as manual audits and automated tools---that either demand immense human effort or lack the nuanced understanding needed to capture real-world user experiences. 
%
% In response, a new wave of AI-driven methods, many powered by large language models (LLMs), is emerging to bridge these accessibility detection and assessment gaps. Early explorations, such as those by Morillo et al.~\cite{morillo2020system}, introduced AI-assisted recommendations capable of automatic corrections, illustrating how computational intelligence can tackle the repetitive, common errors that plague large swaths of the web. Building on this foundation, Huang et al.~\cite{huang2024access} proposed ACCESS, a prompt-engineering framework that streamlines the identification and remediation of accessibility violations, while López-Gil et al.~\cite{lopez2024turning} demonstrated how LLMs can help apply WCAG success criteria more consistently---reducing the reliance on manual effort. Beyond these direct interventions, recent work has also begun integrating user experiences more seamlessly into the evaluation process. For example, Huq et al.~\cite{huq2024automated} translate user transcripts and corresponding issues into actionable test reports, ensuring that accessibility improvements align more closely with authentic user needs.
% However, as these AI-driven solutions evolve, researchers caution against uncritical adoption. Othman et al.~\cite{othman2023fostering} highlight that while LLMs can accelerate remediation, they may also introduce biases or encourage over-reliance on automated processes. Similarly, Delnevo et al.~\cite{delnevo2024interaction} emphasize the importance of contextual understanding and adaptability, pointing to the current limitations of LLM-based systems in serving the full spectrum of user needs. 
% In contrast to this backdrop, our work introduces and evaluates CodeA11y, an LLM-augmented extension for GitHub Copilot that not only mitigates these challenges by providing more consistent guidance and manual validation prompts, but also aligns AI-driven assistance with developers’ workflows, ultimately contributing toward more sustainable propulsion for building accessible web.

% Broader implications of inaccessibility—legal compliance, ethical concerns, and user experience
% A Historical Review of Web Accessibility Using WAVE
% "I tend to view ads almost like a pestilence": On the Accessibility Implications of Mobile Ads for Blind Users

% In the research domain, several methods have been developed to assess and enhance web accessibility. These include incorporating feedback into developer tools~\cite{adesigner, takagi2003accessibility, bigham2010accessibility} and automating the creation of accessibility tests and reports for user interfaces~\cite{swearngin2024towards, taeb2024axnav}. 

% Prior work has also studied accessibility scanners as another avenue of AI to improve web development practices~\cite{}.
% However, a persistent challenge is that developers need to be aware of these tools to utilize them effectively. With recent advancements in LLMs, developers might now build accessible websites with less effort using AI assistants. However, the impact of these assistants on the accessibility of their generated code remains unclear. This study aims to investigate these effects.

\subsection{End-user Accessibility Repair}
In addition to detecting accessibility errors and measuring web accessibility, significant research has focused on fixing these problems.
Since end-users are often the first to notice accessibility problems and have a strong incentive to address them, systems have been developed to help them report or fix these problems.

Collaborative, or social accessibility~\cite{takagi2009collaborative,sato2010social}, enabled these end-user contributions to be scaled through crowd-sourcing.
AccessMonkey~\cite{bigham2007accessmonkey} and Accessibility Commons~\cite{kawanaka2008accessibility} were two examples of repositories that store accessibility-related scripts and metadata, respectively.
Other work has developed browser extensions that leverage crowd-sourced databases to automatically correct reading order, alt-text, color contrast, and interaction-related issues~\cite{sato2009s,huang2015can}.

One drawback of collaborative accessibility approaches is that they cannot fix problems for an ``unseen'' web page on-demand, so many projects aim to automatically detect and improve interfaces without the need for an external source of fixes.
A large body of research has focused on making specific web media (e.g., images~\cite{gleason2019making,guinness2018caption, twitterally, gleason2020making, lee2021image}, design~\cite{potluri2019ai,li2019editing, peng2022diffscriber, peng2023slide}, and videos~\cite{pavel2020rescribe,peng2021say,peng2021slidecho,huh2023avscript}) accessible through a combination of machine learning (ML) and user-provided fixes.
Other work has focused on applying more general fixes across all websites.

Opportunity accessibility addressed a common accessibility problem of most websites: by default, content is often hard to see for people with visual impairments, and many users, especially older adults, do not know how to adjust or enable content zooming~\cite{bigham2014making}.
To this end, a browser script (\texttt{oppaccess.js}) was developed that automatically adjusted the browser's content zoom to maximally enlarge content without introducing adverse side-effects (\textit{e.g.,} content overlap).
While \texttt{oppaccess.js} primarily targeted zoom-related accessibility, recent work aimed to enable larger types of changes, by using LLMs to modify the source code of web pages based on user questions or directives~\cite{li2023using}.

Several efforts have been focused on improving access to desktop and mobile applications, which present additional challenges due to the unavailability of app source code (\textit{e.g.,} HTML).
Prefab is an approach that allows graphical UIs to be modified at runtime by detecting existing UI widgets, then replacing them~\cite{dixon2010prefab}.
Interaction Proxies used these runtime modification strategies to ``repair'' Android apps by replacing inaccessible widgets with improved alternatives~\cite{zhang2017interaction, zhang2018robust}.
The widget detection strategies used by these systems previously relied on a combination of heuristics and system metadata (\textit{e.g.,} the view hierarchy), which are incomplete or missing in the accessible apps.
To this end, ML has been employed to better localize~\cite{chen2020object} and repair UI elements~\cite{chen2020unblind,zhang2021screen,wu2023webui,peng2025dreamstruct}.

In general, end-user solutions to repairing application accessibility are limited due to the lack of underlying code and knowledge of the semantics of the intended content.

\subsection{Developer Tools for Accessibility}
Ultimately, the best solution for ensuring an accessible experience lies with front-end developers. Many efforts have focused on building adequate tooling and support to help developers with ensuring that their UI code complies with accessibility standards.

Numerous automated accessibility testing tools have been created to help developers identify accessibility issues in their code: i) static analysis tools, such as IBM Equal Access Accessibility Checker~\cite{ibm2024toolkit} or Microsoft Accessibility Insights~\cite{accessibilityinsights2024}, scan the UI code's compliance with predefined rules derived from accessibility guidelines; and ii) dynamic or runtime accessibility scanners, such as Chrome Devtools~\cite{chromedevtools2024} or axe-Core Accessibility Engine~\cite{deque2024axe}, perform real-time testing on user interfaces to detect interaction issues not identifiable from the code structure. While these tools greatly reduce the manual effort required for accessibility testing, they are often criticized for their limited coverage. Thus, experts often recommend manually testing with assistive technologies to uncover more complex interaction issues. Prior studies have created accessibility crawlers that either assist in developer testing~\cite{swearngin2024towards,taeb2024axnav} or simulate how assistive technologies interact with UIs~\cite{10.1145/3411764.3445455, 10.1145/3551349.3556905, 10.1145/3544548.3580679}.

Similar to end-user accessibility repair, research has focused on generating fixes to remediate accessibility issues in the UI source code. Initial attempts developed heuristic-based algorithms for fixing specific issues, for instance, by replacing text or background color attributes~\cite{10.1145/3611643.3616329}. More recent work has suggested that the code-understanding capabilities of LLMs allow them to suggest more targeted fixes.
For example, a study demonstrated that prompting ChatGPT to fix identified WCAG compliance issues in source code could automatically resolve a significant number of them~\cite{othman2023fostering}. Researchers have sought to leverage this capability by employing a multi-agent LLM architecture to automatically identify and localize issues in source code and suggest potential code fixes~\cite{mehralian2024automated}.

While the approaches mentioned above focus on assessing UI accessibility of already-authored code (\textit{i.e.,} fixing existing code), there is potential for more proactive approaches.
For example, LLMs are often used by developers to generate UI source code from natural language descriptions or tab completions~\cite{chen2021evaluating,GitHubCopilot,lozhkov2024starcoder,hui2024qwen2,roziere2023code,zheng2023codegeex}, but LLMs frequently produce inaccessible code by default~\cite{10.1145/3677846.3677854,mowar2024tab}, leading to inaccessible output when used by developers without sufficient awareness of accessibility knowledge.
The primary focus of this paper is to design a more accessibility-aware coding assistant that both produces more accessible code without manual intervention (\textit{e.g.,} specific user prompting) and gradually enables developers to implement and improve accessibility of automatically-generated code through IDE UI modifications (\textit{e.g.}, reminder notifications).

}
\end{highlight}



% Work related to this paper includes {\em (i)} Web Accessibility and {\em (ii)} Developer Practices in AI-Assisted Programming.

% \ipstart{Web Accessibility: Practice, Evaluation, and Improvements} Substantial efforts have been made to set accessibility standards~\cite{chisholm2001web, caldwell2008web}, establish legal requirements~\cite{sierkowski2002achieving, yesilada2012understanding}, and promote education and advocacy among developers~\cite{sloan2006contextual, martin2022landscape, pandey2023blending}. In the research domain, several methods have been developed to assess and enhance web accessibility. These include incorporating feedback into developer tools~\cite{adesigner, takagi2003accessibility, bigham2010accessibility} and automating the creation of accessibility tests and reports for user interfaces~\cite{swearngin2024towards, taeb2024axnav}. 
% % Prior work has also studied accessibility scanners as another avenue of AI to improve web development practices~\cite{}.
% However, a persistent challenge is that developers need to be aware of these tools to utilize them effectively. With recent advancements in LLMs, developers might now build accessible websites with less effort using AI assistants. However, the impact of these assistants on the accessibility of their generated code remains unclear. This study aims to investigate these effects.

% \ipstart{Developer Practices in AI-Assisted Programming}
% Recent usability research on AI-assisted development has examined the interaction strategies of developers while using AI coding assistants~\cite{barke2023grounded}.
% They observed developers interacted with these assistants in two modes -- 1) \textit{acceleration mode}: associated with shorter completions and 2) \textit{exploration mode}: associated with long completions.
% Liang {\em et al.} \cite{liang2024large} found that developers are driven to use AI assistants to reduce their keystrokes, finish tasks faster, and recall the syntax of programming languages. On the other hand, developers' reason for rejecting autocomplete suggestions was the need for more consideration of appropriate software requirements. This is because primary research on code generation models has mainly focused on functional correctness while often sidelining non-functional requirements such as latency, maintainability, and security~\cite{singhal2024nofuneval}. Consequently, there have been increasing concerns about the security implications of AI-generated code~\cite{sandoval2023lost}. Similarly, this study focuses on the effectiveness and uptake of code suggestions among developers in mitigating accessibility-related vulnerabilities. 


% ============================= additional rw ============================================
% - Paulina Morillo, Diego Chicaiza-Herrera, and Diego Vallejo-Huanga. 2020. System of Recommendation and Automatic Correction of Web Accessibility Using Artificial Intelligence. In Advances in Usability and User Experience, Tareq Ahram and Christianne Falcão (Eds.). Springer International Publishing, Cham, 479–489
% - Juan-Miguel López-Gil and Juanan Pereira. 2024. Turning manual web accessibility success criteria into automatic: an LLM-based approach. Universal Access in the Information Society (2024). https://doi.org/10.1007/s10209-024-01108-z
% - s
% - Calista Huang, Alyssa Ma, Suchir Vyasamudri, Eugenie Puype, Sayem Kamal, Juan Belza Garcia, Salar Cheema, and Michael Lutz. 2024. ACCESS: Prompt Engineering for Automated Web Accessibility Violation Corrections. arXiv:2401.16450 [cs.HC] https://arxiv.org/abs/2401.16450
% - Syed Fatiul Huq, Mahan Tafreshipour, Kate Kalcevich, and Sam Malek. 2025. Automated Generation of Accessibility Test Reports from Recorded User Transcripts. In Proceedings of the 47th International Conference on Software Engineering (ICSE) (Ottawa, Ontario, Canada). IEEE. https://ics.uci.edu/~seal/publications/2025_ICSE_reca11.pdf To appear in IEEE Xplore
% - Achraf Othman, Amira Dhouib, and Aljazi Nasser Al Jabor. 2023. Fostering websites accessibility: A case study on the use of the Large Language Models ChatGPT for automatic remediation. In Proceedings of the 16th International Conference on PErvasive Technologies Related to Assistive Environments (Corfu, Greece) (PETRA ’23). Association for Computing Machinery, New York, NY, USA, 707–713. https://doi.org/10.1145/3594806.3596542
% - Zsuzsanna B. Palmer and Sushil K. Oswal. 0. Constructing Websites with Generative AI Tools: The Accessibility of Their Workflows and Products for Users With Disabilities. Journal of Business and Technical Communication 0, 0 (0), 10506519241280644. https://doi.org/10.1177/10506519241280644
% ============================= additional rw ============================================
\section{The Properties of SSL Gradients}
\label{sec:theory}

We begin by studying the gradients of the cosine similarity with respect to an arbitrary point $z_i$. Throughout this section, we refer to $\mathbf{Z}$ as any set of points in $\mathbb{R}^d$, with no other assumptions over the distribution. The gradient acting on one of these points has the following structure:

\begin{restatable}{proposition}{cosgrads}[Prop. 3 in \citet{spherical_embeddings};\footnote{\citet{spherical_embeddings} also showed corresponding results under SGD with momentum and Adam optimization \citep{adam}.} proof in \ref{prf:prop_grad_grows}]
    \label{prop:cos_sim_grads}
    Let $\mathbf{Z}$ be a set of points in $\mathbb{R}^d$ and let $z_i$ and $z_j$ be a positive pair in $\mathbf{Z}$. Let $\phi_{ij}$ be the angle between $z_i$ and $z_j$. Then the gradient of $\mathcal{L}_{ij}^\mathcal{A}(\mathbf{Z})$ with respect to $z_i$ is
    \[ \nabla_i^\mathcal{A} = -\frac{1}{\|z_i\|} \left(\mathbf{I}_d - \frac{z_i z_i^\top}{\|z_i\|^2} \right) \frac{z_j}{\|z_j\|} = -\left( \frac{\hat{z}_j}{\|z_i\|} \right)_{\perp z_i} \]
    where $a_{\perp b}$ is the component of $a$ orthogonal to $b$.
    This has magnitude $\|\nabla_i^\mathcal{A}\| = \frac{\sin(\phi_{ij})}{\|z_i\|}$.
\end{restatable}

\noindent This has an easy interpretation: $\mathbf{I}_d - \frac{z_i z_i^\top}{\|z_i\|^2}$ projects the unit vector $\hat{z}_j$ onto the subspace orthogonal to $z_i$. This projected vector is then inversely scaled by $\|z_i\|$. We visualize this in Figure~\ref{fig:fig1}. A similar result holds for the InfoNCE loss:
\begin{restatable}{proposition}{infoncegrads}[Proof in \ref{app:infonce_grads}]
    \label{cor:infonce_grads}
    Let $\mathbf{Z}$ be a set of points in $\mathbb{R}^d$, $z_i$ and $z_j$ be a positive pair in $\mathbf{Z}$, and $\nabla_i^\mathcal{A}$ be as in Prop. \ref{prop:cos_sim_grads}. Then the gradient of $\mathcal{L}_{ij}(\mathbf{Z})$ with respect to $z_i$ is
    \begin{equation}
        \label{eq:infonce_grads}
        \nabla_i = \nabla_i^\mathcal{A} + \frac{1}{\|z_i\|} \cdot \sum_{k \not\sim i} \left( \hat{z}_k \cdot \frac{\text{\emph{ExpSim}}(z_i, z_k)}{S_i} \right)_{\perp z_i}.
    \end{equation}\vspace*{-0.2cm}
\end{restatable}

In essence, because the InfoNCE loss is a function of the cosine similarity, the chain rule implies that its gradients behave similarly to the cosine similarity's. Specifically, just like those of $\mathcal{L}_{ij}^\mathcal{A}$, the gradients of $\mathcal{L}_{ij}^\mathcal{R}$ have the properties that (1) they are inversely scaled by $\|z_i\|$ and (2) they exist in $z_i$'s tangent space. Since the InfoNCE loss is the sum of $\mathcal{L}_{ij}^\mathcal{A}$ and $\mathcal{L}_{ij}^\mathcal{R}$, these properties all extend to the InfoNCE loss as well. Going forward, we refer to any loss function or SSL model as \emph{cosine-similarity-based} (cos.sim.-based) if it exhibits these two properties.

\begin{figure*}
    \centering \hspace*{-0.3cm}
    \resizebox{0.27\linewidth}{!}{%
    \begin{subfigure}{0.27\linewidth}
    \captionsetup{oneside,margin={1cm,0cm}}
    \begin{tikzpicture}
        \node[inner sep=0pt] () at (0, 0) {\includegraphics[width=0.98\linewidth, trim={0cm, 0.23cm, 0cm, 0cm}, clip]{Images/vert_bounds.pdf}};

        % \draw[darkgray] (-2.21, -1.1) -- (-2.21, -1.15);
        % \draw[darkgray] (-1.3, -1.1) -- (-1.3, -1.15);
        % \draw[darkgray] (-0.39, -1.1) -- (-0.39, -1.15);
        
        \fill [white] (-1.2,-1.72) rectangle (-0.6,-1.95);
        \fill [white] (0.2,-1.72) rectangle (0.8,-1.95);
        \fill [white] (1.65,-1.72) rectangle (2.25,-1.95);

        \node () at (-0.87, -1.9) {\small $0$};
        \node () at (0.5, -1.9) {\small $\frac{\pi}{2}$};
        \node () at (1.95, -1.9) {\small $\pi$};

        % \draw[darkgray] (0.9, -1.1) -- (0.9, -1.15);
        % \draw[darkgray] (2.3, -1.1) -- (2.3, -1.15);
        % \draw[darkgray] (3.73, -1.1) -- (3.73, -1.15);
    \end{tikzpicture}
    \caption{}
    \label{fig:convergence_sim}
    \end{subfigure}%
    }
    \,\,\,
    \begin{subfigure}{0.35\linewidth}
    \captionsetup{oneside,margin={1cm,0cm}}
    \includegraphics[width=\linewidth]{Images/density.pdf}
    \caption{}
    \label{fig:density}
    \end{subfigure}
    \,
    \begin{subfigure}{0.35\linewidth}
    \captionsetup{oneside,margin={1cm,0cm}}
    \includegraphics[trim={0cm, 0cm, 0cm, 0.5cm}, clip, width=\linewidth]{Images/embed_norms_sim.pdf}
    \caption{}
    \label{fig:class_imbalance}
    \end{subfigure}
    \caption{Simulations studying the relationship between SSL training and embedding norms. \emph{Left}: Applying cosine similarity gradients to pairs of points converges slower as a function of the points' norm and the sin of their angle. \emph{Middle}: Training via InfoNCE to reconstruct latent classes induces higher norms in dense output regions. \emph{Right}: Training via InfoNCE leads to larger norm for high-frequency classes.}
    \label{fig:confidence_sim}
\end{figure*}

This orthogonality has also been noted in \citet{normface} and \citet{mentions_catch22}. As a direct consequence of this projection onto the tangent plane, applying the cosine similarity or InfoNCE gradients to a point \emph{must grow its magnitude} (visualized in Figure%\ref{fig:growing_embeddings})
~\ref{fig:fig1}, middle and right):
\begin{corollary}[First identified in \citet{normface}; Proof in~\ref{prf:cor_embeddings_grow}]
    \label{cor:embeddings_grow}
    Let $z \in \mathbb{R}^d$ and let $z'$ be the result of applying a step of gradient descent with respect to the cosine similarity (Prop.~\ref{prop:cos_sim_grads}) or InfoNCE (Prop.~\ref{cor:infonce_grads}) to $z$. Then
    $\|z'\| \geq \|z\|$.
\end{corollary}

The results in Propositions~\ref{prop:cos_sim_grads}, \ref{cor:infonce_grads} and Corollary~\ref{cor:embeddings_grow} reveal an inevitable catch-22 for self-supervised learning: we require small embeddings to avoid vanishing gradients but optimizing SSL loss functions grows the embeddings. We refer to this as the \emph{embedding-norm effect}. This effect also holds for the mean squared error between normalized embeddings and, by extension, the triplet loss.

Furthermore, Proposition \ref{prop:cos_sim_grads} has direct implications for convergence rates under the cosine similarity. The gradient's magnitude directly scales the learning rate, since $z_i' = z_i + \gamma \nabla_i = z_i + \left( \gamma \cdot \| \nabla_i \| \right) \hat{\nabla}_i.$ Thus, Proposition \ref{prop:cos_sim_grads} can be interpreted as saying that the embedding norm and the sin of the angle parameterize the model's learning rate. Indeed, both quadratically slow down convergence:

\begin{theorem}[Proof in \ref{prf:thm_convergence_rate}]
\label{thm:convergence_rate}
    Let $z_i$ and $z_j$ be embeddings with equal norm, i.e. $\|z_i\| = \|z_j\| = \rho$. Let $z_i' = z_i + \frac{\gamma}{\rho}(z_j)_{\perp z_i}$ and $z_j' = z_j
    + \frac{\gamma}{\rho}(z_i)_{\perp z_j}$ be the embeddings after maximizing the cosine similarity via a step of gradient descent with learning rate $\gamma$.
    Then the change in cosine similarity is bounded from above by:
        \begin{equation}
            \label{eq:thm_statement}
            \hat{z}_i'^\top \hat{z}_j' - \hat{z}_i^\top \hat{z}_j < \frac{2 \gamma \sin^2 \phi_{ij}}{\rho^2}.
        \end{equation}
\end{theorem}
Put simply, the change in the cosine similarity via a step of gradient descent scales quadratically with the embedding's norm and the sin of the angle to its positive counterpart.
% \begin{figure}
%     \centering
%     \resizebox{\linewidth}{!}{%
%     \begin{tikzpicture}
%         \node[inner sep=0pt] () at (0, 0) {\includegraphics[width=0.98\linewidth]{Images/bounds.pdf}};

%         \draw[darkgray] (-2.21, -1.1) -- (-2.21, -1.15);
%         \draw[darkgray] (-1.3, -1.1) -- (-1.3, -1.15);
%         \draw[darkgray] (-0.39, -1.1) -- (-0.39, -1.15);
        
%         \fill [white] (0.62,-1.38) rectangle (1.2,-1.18);
%         \fill [white] (2,-1.38) rectangle (2.6,-1.18);
%         \fill [white] (3.4,-1.38) rectangle (3.95,-1.18);

%         \node () at (0.9, -1.32) {\tiny $0$};
%         \node () at (2.3, -1.32) {\tiny $\frac{\pi}{2}$};
%         \node () at (3.73, -1.3) {\tiny $\pi$};

%         \draw[darkgray] (0.9, -1.1) -- (0.9, -1.15);
%         \draw[darkgray] (2.3, -1.1) -- (2.3, -1.15);
%         \draw[darkgray] (3.73, -1.1) -- (3.73, -1.15);
%     \end{tikzpicture}%
%     }
%     \caption{}
%     \label{fig:convergence_sim}
% \end{figure}

\section{Simulations}
\label{sec:simulations}

We now present a suite of simulations which allow us to characterize how the parameters in Section \ref{sec:theory} influence SSL training under idealized conditions. Full implementation and experiment details can be found in Appendix \ref{app:simulations}.

\subsection{Effect of the Embedding Norms on SSL Training}
\label{ssec:convergence_simulations}


We start by evaluating to what extent the embedding norms and angles between positive samples slow down convergence. Specifically, we sampled 500 pairs of points directly on $\mathbb{S}^{20}$.
We produce many such sets of samples while varying their mean embedding norms and $\phi_{ij}$ values. We then evaluate Theorem \ref{thm:convergence_rate} by
applying the cosine similarity gradients to all positive pairs of embeddings until convergence.

Figure~\ref{fig:convergence_sim} plots the number of steps until convergence and shows that, although the convergence rate depends on both parameters,
having large embedding norms is \emph{significantly} worse for optimization than having large angles between
positive pairs. In essence, the embedding norm's unbounded nature allows it to induce arbitrarily large slowdowns. Meanwhile, the angle between positive samples only has a non-negligible impact as the angle approaches its upper limit $\pi$. Because it is exponentially unlikely for the angle of \emph{every} positive pair to be close to $\pi$, we ignore the angular component of Theorem \ref{thm:convergence_rate} for the remainder of this paper and relegate its further discussion to Appendix \ref{app:opposite_halves_effect}.

\subsection{Effect of SSL Training on the Embedding Norms}
\label{ssec:confidence_simulations}

We now consider how SSL training affects the embedding norms via a simplified training setting where the data is generated from latent classes. Inspired by \citet{latent_inversion, latent_inversion_2}, consider an SSL dataset as a set of latent class distributions $\{\tilde{\mathcal{Z}}_1, ..., \tilde{\mathcal{Z}}_k\}$, where each $\tilde{\mathcal{Z}}_i$ is a probability distribution on the $d$-dimensional hypersphere $\mathbb{S}^d$. Let the observations $x \in \mathcal{X} \subset \mathbb{R}^D$ be obtained via a generating process $g: \mathbb{S}^{d} \rightarrow \mathbb{R}^D$. That is, our dataset is obtained by randomly choosing a probability distribution $\tilde{\mathcal{Z}}_i$, drawing a sample $\tilde{z}$ from it, and applying $g$ to $\tilde{z}$.
%Under appropriate choices for the distributions and training scheme, \cite{latent_inversion, latent_inversion_2} showed that optimizing a 
Following \citet{latent_inversion, latent_inversion_2}, we are training a neural network $f: \mathcal{X} \to \mathbb{S}^{d}$ via contrastive learning to produce a learned latent embedding $ f(\mathcal{X})$. 

We analyze the relationship of parameterized SSL training to the embedding norms by simulating the above scenario. Specifically, we choose centers %$\{c_1, \ldots c_4\}$ 
for 4 latent classes uniformly at random from $\mathbb{S}^{10}$. We then sample 4K points around these centers and normalize them to the hypersphere.
From this, we produce the dataset via generating process $g: \mathbb{S}^{10} \subset \mathbb{R}^{11} \rightarrow \mathbb{R}^{64}$, where $g$ is given by multiplication by a random matrix. We finally train a 2-layer feedforward network with the supervised InfoNCE loss function\footnote{The supervised InfoNCE loss explicitly chooses positive pairs as those which belong to the same class.} on this dataset.

Figure \ref{fig:density} plots each embedding's magnitude in the learned space as a function of (inverse) density in embedding space. We use the distance to an embedding's $10^\text{th}$ nearest neighbor under the cosine similarity metric as a proxy for inverse density. We see that embeddings in dense regions of the embedding space tend to have higher norm. For instance, there are \emph{no} embeddings which are both in a dense latent region (distance \mytexttilde$0.1$ to the $10^\text{th}$ neighbor) and have small norm (\mytexttilde$6$). This follows from Corollary \ref{cor:embeddings_grow}: dense regions of the embedding space receive the most gradient updates and, consequently, those embeddings will grow the most. We also modify the simulation for Figure \ref{fig:class_imbalance} by providing a class imbalance parameter to the class distribution. Namely, class $1$ now has sample probability \mytexttilde$1/2$, class $2$ has sample probability \mytexttilde$1/4$, and so on. We then see that over the course of training, the mean embedding norms for frequent classes grow to higher values than those for sparse classes.

\paragraph{Takeaways.} Across these experiments, samples which are seen more often have higher embedding norm under the InfoNCE loss. This can occur either due to the network considering these samples to be prototypical (and therefore embedding them in dense regions of the learned space) or being otherwise over-represented in the dataset. We point out that these are precisely the settings in which we expect a network to be confident in the embedding.  We leave a formal quantitative analysis
%it 
as an open question:
%to provide formal bounds which relate an embedding's norm and the model's certainty regarding the corresponding input:

\begin{question}
    What theoretical bounds can be made regarding a sample's embedding norm and (a) the accuracy with which it is classified or (b) the corresponding input's dissimilarity from the training data?
\end{question}

\begin{figure*}[t!]
    \centering\vspace*{-0.7cm}
    \resizebox{0.5\linewidth}{!}{%
    \begin{tikzpicture}
        \clip (-10, -3.8) rectangle + (9, 6.5);
        \node () at (0, 0) {\includegraphics[width=\textwidth, trim={0.6cm, 2.1cm, 0cm, 2cm}, clip]{Images/violin_plot.pdf}};

\node () at (-9.3, 0.) {\small \textcolor{darkgray}{\rotatebox{90}{Normalized Embedding}}};
\node () at (-9., 0.) {\small \textcolor{darkgray}{\rotatebox{90}{Magnitudes}}};

\node () at (-8.6, -1.37) {\small \textcolor{gray}{0}};
\node () at (-8.6, -0.12) {\small \textcolor{gray}{1}};
\node () at (-8.6, 1.1) {\small \textcolor{gray}{2}};

\node () at (-7.99, -1.9) {\scriptsize \textcolor{darkgray}{\rotatebox{90}{Cifar-10}}};
\node () at (-7.74, -1.9) {\scriptsize \textcolor{darkgray}{\rotatebox{90}{Train}}};
\node () at (-7.17, -1.9) {\scriptsize \textcolor{darkgray}{\rotatebox{90}{Cifar-10}}};
\node () at (-6.92, -1.9) {\scriptsize \textcolor{darkgray}{\rotatebox{90}{Test}}};
\node () at (-6.35, -1.9) {\scriptsize \textcolor{darkgray}{\rotatebox{90}{Cifar-100}}};
\node () at (-6.1, -1.9) {\scriptsize \textcolor{darkgray}{\rotatebox{90}{Train}}};
\node () at (-5.55, -1.9) {\scriptsize \textcolor{darkgray}{\rotatebox{90}{Cifar-100}}};
\node () at (-5.3, -1.9) {\scriptsize \textcolor{darkgray}{\rotatebox{90}{Test}}};

\node () at (-7.85, 0.05) {\small \textcolor{blue}{\textbf{1.0}}};
\node () at (-7.05, -0.01) {\small \textcolor{blue}{\textbf{0.93}}};
\node () at (-5.82, -0.65) {\small \textcolor{blue}{\textbf{0.42}}};

\draw[dotted] (-8.15, -2.5) to (-5.05, -2.5);
\node[inner sep=0pt] (simclr) at (-6.6, -2.8) {SimCLR};



\node () at (-3.9, -1.9) {\scriptsize \textcolor{darkgray}{\rotatebox{90}{Cifar-10}}};
\node () at (-3.65, -1.9) {\scriptsize \textcolor{darkgray}{\rotatebox{90}{Train}}};
\node () at (-3.08, -1.9) {\scriptsize \textcolor{darkgray}{\rotatebox{90}{Cifar-10}}};
\node () at (-2.83, -1.9) {\scriptsize \textcolor{darkgray}{\rotatebox{90}{Test}}};
\node () at (-2.28, -1.9) {\scriptsize \textcolor{darkgray}{\rotatebox{90}{Cifar-100}}};
\node () at (-2.03, -1.9) {\scriptsize \textcolor{darkgray}{\rotatebox{90}{Train}}};
\node () at (-1.47, -1.9) {\scriptsize \textcolor{darkgray}{\rotatebox{90}{Cifar-100}}};
\node () at (-1.22, -1.9) {\scriptsize \textcolor{darkgray}{\rotatebox{90}{Test}}};

\node () at (-3.78, 0.05) {\small \textcolor{blue}{\textbf{1.0}}};
\node () at (-2.96, -0.285) {\small \textcolor{blue}{\textbf{0.72}}};
\node () at (-1.71, -0.7) {\small \textcolor{blue}{\textbf{0.37}}};

\draw[dotted] (-4.1, -2.5) to (-0.97, -2.5);
\node[inner sep=0pt] (simsiam) at (-2.5, -2.8) {SimSiam};



\node () at (1.18, -1.9) {\scriptsize \textcolor{darkgray}{\rotatebox{90}{Cifar-10}}};
\node () at (1.43, -1.9) {\scriptsize \textcolor{darkgray}{\rotatebox{90}{Train}}};
\node () at (1.99, -1.9) {\scriptsize \textcolor{darkgray}{\rotatebox{90}{Cifar-10}}};
\node () at (2.24, -1.9) {\scriptsize \textcolor{darkgray}{\rotatebox{90}{Test}}};
\node () at (2.83, -1.9) {\scriptsize \textcolor{darkgray}{\rotatebox{90}{Cifar-100}}};
\node () at (3.08, -1.9) {\scriptsize \textcolor{darkgray}{\rotatebox{90}{Train}}};
\node () at (3.63, -1.9) {\scriptsize \textcolor{darkgray}{\rotatebox{90}{Cifar-100}}};
\node () at (3.89, -1.9) {\scriptsize \textcolor{darkgray}{\rotatebox{90}{Test}}};

\node () at (1.73, -0.27) {\small \textcolor{purple}{\textbf{0.74}}};
\node () at (2.96, 0.05) {\small \textcolor{purple}{\textbf{1.0}}};
\node () at (3.77, -0.05) {\small \textcolor{purple}{\textbf{0.91}}};

\draw[dotted] (1.03, -2.5) to (4.2, -2.5);
\node[inner sep=0pt] (simclr2) at (2.65, -2.8) {SimCLR};



\node () at (5.3, -1.9) {\scriptsize \textcolor{darkgray}{\rotatebox{90}{Cifar-10}}};
\node () at (5.55, -1.9) {\scriptsize \textcolor{darkgray}{\rotatebox{90}{Train}}};
\node () at (6.13, -1.9) {\scriptsize \textcolor{darkgray}{\rotatebox{90}{Cifar-10}}};
\node () at (6.38, -1.9) {\scriptsize \textcolor{darkgray}{\rotatebox{90}{Test}}};
\node () at (6.95, -1.9) {\scriptsize \textcolor{darkgray}{\rotatebox{90}{Cifar-100}}};
\node () at (7.2, -1.9) {\scriptsize \textcolor{darkgray}{\rotatebox{90}{Train}}};
\node () at (7.78, -1.9) {\scriptsize \textcolor{darkgray}{\rotatebox{90}{Cifar-100}}};
\node () at (8.03, -1.9) {\scriptsize \textcolor{darkgray}{\rotatebox{90}{Test}}};

\node () at (5.85, -0.64) {\small \textcolor{purple}{\textbf{0.43}}};
\node () at (7.05, 0.05) {\small \textcolor{purple}{\textbf{1.0}}};
\node () at (7.885, -0.35) {\small \textcolor{purple}{\textbf{0.65}}};

\draw[dotted] (5.05, -2.5) to (8.23, -2.5);
\node[inner sep=0pt] (simsiam2) at (6.6, -2.8) {SimSiam};

\draw[line width=1pt, blue] (-8.25, -3.15) to (-0.85, -3.15);
\node[inner sep=0pt] (trainedcif10) at (-4.6, -3.5) {\textcolor{blue}{Trained on Cifar10}};

\draw[line width=1pt, purple] (0.95, -3.15) to (8.3, -3.15);
\node[inner sep=0pt] (trainedcif10) at (4.6, -3.5) {\textcolor{purple}{Trained on Cifar100}};

    \end{tikzpicture}%
    }
    \hfill
    \resizebox{0.472\linewidth}{!}{%
    \begin{tikzpicture}
        \clip (0.85, -3.8) rectangle + (8.5, 6.5);
        \node () at (0, 0) {\includegraphics[width=\textwidth, trim={0.6cm, 2.1cm, 0cm, 2cm}, clip]{Images/violin_plot.pdf}};

\node () at (-9.3, 0.) {\small \textcolor{darkgray}{\rotatebox{90}{Normalized Embedding}}};
\node () at (-9., 0.) {\small \textcolor{darkgray}{\rotatebox{90}{Magnitudes}}};

\node () at (-8.6, -1.37) {\small \textcolor{gray}{0}};
\node () at (-8.6, -0.12) {\small \textcolor{gray}{1}};
\node () at (-8.6, 1.1) {\small \textcolor{gray}{2}};

\node () at (-7.99, -1.9) {\scriptsize \textcolor{darkgray}{\rotatebox{90}{Cifar-10}}};
\node () at (-7.74, -1.9) {\scriptsize \textcolor{darkgray}{\rotatebox{90}{Train}}};
\node () at (-7.17, -1.9) {\scriptsize \textcolor{darkgray}{\rotatebox{90}{Cifar-10}}};
\node () at (-6.92, -1.9) {\scriptsize \textcolor{darkgray}{\rotatebox{90}{Test}}};
\node () at (-6.35, -1.9) {\scriptsize \textcolor{darkgray}{\rotatebox{90}{Cifar-100}}};
\node () at (-6.1, -1.9) {\scriptsize \textcolor{darkgray}{\rotatebox{90}{Train}}};
\node () at (-5.55, -1.9) {\scriptsize \textcolor{darkgray}{\rotatebox{90}{Cifar-100}}};
\node () at (-5.3, -1.9) {\scriptsize \textcolor{darkgray}{\rotatebox{90}{Test}}};

\node () at (-7.85, 0.05) {\small \textcolor{blue}{\textbf{1.0}}};
\node () at (-7.05, -0.01) {\small \textcolor{blue}{\textbf{0.93}}};
\node () at (-5.82, -0.65) {\small \textcolor{blue}{\textbf{0.42}}};

\draw[dotted] (-8.15, -2.5) to (-5.05, -2.5);
\node[inner sep=0pt] (simclr) at (-6.6, -2.8) {SimCLR};



\node () at (-3.9, -1.9) {\scriptsize \textcolor{darkgray}{\rotatebox{90}{Cifar-10}}};
\node () at (-3.65, -1.9) {\scriptsize \textcolor{darkgray}{\rotatebox{90}{Train}}};
\node () at (-3.08, -1.9) {\scriptsize \textcolor{darkgray}{\rotatebox{90}{Cifar-10}}};
\node () at (-2.83, -1.9) {\scriptsize \textcolor{darkgray}{\rotatebox{90}{Test}}};
\node () at (-2.28, -1.9) {\scriptsize \textcolor{darkgray}{\rotatebox{90}{Cifar-100}}};
\node () at (-2.03, -1.9) {\scriptsize \textcolor{darkgray}{\rotatebox{90}{Train}}};
\node () at (-1.47, -1.9) {\scriptsize \textcolor{darkgray}{\rotatebox{90}{Cifar-100}}};
\node () at (-1.22, -1.9) {\scriptsize \textcolor{darkgray}{\rotatebox{90}{Test}}};

\node () at (-3.78, 0.05) {\small \textcolor{blue}{\textbf{1.0}}};
\node () at (-2.96, -0.285) {\small \textcolor{blue}{\textbf{0.72}}};
\node () at (-1.71, -0.7) {\small \textcolor{blue}{\textbf{0.37}}};

\draw[dotted] (-4.1, -2.5) to (-0.97, -2.5);
\node[inner sep=0pt] (simsiam) at (-2.5, -2.8) {SimSiam};



\node () at (1.18, -1.9) {\scriptsize \textcolor{darkgray}{\rotatebox{90}{Cifar-10}}};
\node () at (1.43, -1.9) {\scriptsize \textcolor{darkgray}{\rotatebox{90}{Train}}};
\node () at (1.99, -1.9) {\scriptsize \textcolor{darkgray}{\rotatebox{90}{Cifar-10}}};
\node () at (2.24, -1.9) {\scriptsize \textcolor{darkgray}{\rotatebox{90}{Test}}};
\node () at (2.83, -1.9) {\scriptsize \textcolor{darkgray}{\rotatebox{90}{Cifar-100}}};
\node () at (3.08, -1.9) {\scriptsize \textcolor{darkgray}{\rotatebox{90}{Train}}};
\node () at (3.63, -1.9) {\scriptsize \textcolor{darkgray}{\rotatebox{90}{Cifar-100}}};
\node () at (3.89, -1.9) {\scriptsize \textcolor{darkgray}{\rotatebox{90}{Test}}};

\node () at (1.73, -0.27) {\small \textcolor{purple}{\textbf{0.74}}};
\node () at (2.96, 0.05) {\small \textcolor{purple}{\textbf{1.0}}};
\node () at (3.77, -0.05) {\small \textcolor{purple}{\textbf{0.91}}};

\draw[dotted] (1.03, -2.5) to (4.2, -2.5);
\node[inner sep=0pt] (simclr2) at (2.65, -2.8) {SimCLR};



\node () at (5.3, -1.9) {\scriptsize \textcolor{darkgray}{\rotatebox{90}{Cifar-10}}};
\node () at (5.55, -1.9) {\scriptsize \textcolor{darkgray}{\rotatebox{90}{Train}}};
\node () at (6.13, -1.9) {\scriptsize \textcolor{darkgray}{\rotatebox{90}{Cifar-10}}};
\node () at (6.38, -1.9) {\scriptsize \textcolor{darkgray}{\rotatebox{90}{Test}}};
\node () at (6.95, -1.9) {\scriptsize \textcolor{darkgray}{\rotatebox{90}{Cifar-100}}};
\node () at (7.2, -1.9) {\scriptsize \textcolor{darkgray}{\rotatebox{90}{Train}}};
\node () at (7.78, -1.9) {\scriptsize \textcolor{darkgray}{\rotatebox{90}{Cifar-100}}};
\node () at (8.03, -1.9) {\scriptsize \textcolor{darkgray}{\rotatebox{90}{Test}}};

\node () at (5.85, -0.64) {\small \textcolor{purple}{\textbf{0.43}}};
\node () at (7.05, 0.05) {\small \textcolor{purple}{\textbf{1.0}}};
\node () at (7.885, -0.35) {\small \textcolor{purple}{\textbf{0.65}}};

\draw[dotted] (5.05, -2.5) to (8.23, -2.5);
\node[inner sep=0pt] (simsiam2) at (6.6, -2.8) {SimSiam};

\draw[line width=1pt, blue] (-8.25, -3.15) to (-0.85, -3.15);
\node[inner sep=0pt] (trainedcif10) at (-4.6, -3.5) {\textcolor{blue}{Trained on Cifar10}};

\draw[line width=1pt, purple] (0.95, -3.15) to (8.3, -3.15);
\node[inner sep=0pt] (trainedcif10) at (4.6, -3.5) {\textcolor{purple}{Trained on Cifar100}};

    \end{tikzpicture}%
    }
    \caption{Violin plot showing the distribution of embedding norms on each dataset split as a function of which dataset the model was trained on. All values are normalized by the training set's mean embedding magnitude. Black bars represent and are labeled by the mean value of each violin. We use the same augmentations for the train and test sets for consistency.}
    \label{fig:in_out_violin}
\end{figure*}
\begin{figure*}
    \centering
    \hspace*{-0.5cm}
    \begin{subfigure}{0.3\linewidth}
    \resizebox{5.5cm}{4.25cm}{\begin{tikzpicture}
        \node (fig) at (0, 0) {\includegraphics[width=\textwidth, trim={1.13cm, 0.85cm, 0cm, 0cm}, clip]{Images/Confidence/knn_accuracy.pdf}};
        
        \node (labels) at (-2.245, -2.15) {\footnotesize \textcolor{gray}{0.0}};
        \node (labels) at (-0.13, -2.15) {\footnotesize \textcolor{gray}{0.4}};
        \node (labels) at (1.98, -2.15) {\footnotesize \textcolor{gray}{0.8}};
        \node () at (-0.05, -2.55) {\footnotesize \textcolor{darkgray}{Relative Embedding Magnitude}};

        %\node (labels) at (-2.8, -1) {\tiny \textcolor{gray}{0.25}};
        \node (labels) at (-2.733, -0.12) {\footnotesize \textcolor{gray}{0.5}};
        %\node (labels) at (-2.8, 0.765) {\tiny \textcolor{gray}{0.75}};
        \node (labels) at (-2.78, 1.66) {\footnotesize \textcolor{gray}{1.0}};
        \node () at (-3.175, -0.1) {\footnotesize \textcolor{darkgray}{\rotatebox{90}{$k$NN Accuracy}}};
    \end{tikzpicture}}
    \end{subfigure}
    \quad\quad
    \begin{subfigure}{0.3\linewidth}
    \resizebox{5.5cm}{4.3cm}{\begin{tikzpicture}
        \node (fig) at (0, 0) {\includegraphics[width=\textwidth, trim={1.13cm, 0.85cm, 0cm, 0cm}, clip]{Images/Confidence/human_agreement.pdf}};

        \node (labels) at (-2.245, -2.15) {\footnotesize \textcolor{gray}{0.0}};
        \node (labels) at (0.33, -2.15) {\footnotesize \textcolor{gray}{0.4}};
        \node () at (-0.05, -2.55) {\small \textcolor{darkgray}{Relative Embedding Magnitude}};

        \node (labels) at (-2.78, -1.1) {\footnotesize \textcolor{gray}{0.6}};
        %\node (labels) at (-2.843, -0.12) {\tiny \textcolor{gray}{0.5}};
        %\node (labels) at (-2.8, 0.765) {\tiny \textcolor{gray}{0.75}};
        \node (labels) at (-2.78, 1.76) {\footnotesize \textcolor{gray}{1.0}};
        \node () at (-3.175, -0.1) {\footnotesize \textcolor{darkgray}{\rotatebox{90}{Human Labeler Accuracy}}};
    \end{tikzpicture}}
    \end{subfigure}
    \quad\quad
    \begin{subfigure}{0.3\linewidth}
        \resizebox{5.5cm}{4.25cm}{\begin{tikzpicture}
        \node (fig) at (0, 0) {\includegraphics[width=\textwidth, trim={1.13cm, 0.85cm, 0cm, 0cm}, clip]{Images/Confidence/human_label_entropy.pdf}};

        \node (labels) at (-2.245, -2.15) {\footnotesize \textcolor{gray}{0.0}};
        \node (labels) at (0.52, -2.15) {\footnotesize \textcolor{gray}{0.4}};
        \node () at (-0.05, -2.55) {\small \textcolor{darkgray}{Relative Embedding Magnitude}};

        \node (labels) at (-2.78, -0.33) {\footnotesize \textcolor{gray}{0.2}};
        %\node (labels) at (-2.843, -0.12) {\tiny \textcolor{gray}{0.5}};
        %\node (labels) at (-2.8, 0.765) {\tiny \textcolor{gray}{0.75}};
        \node (labels) at (-2.78, 1.46) {\footnotesize \textcolor{gray}{0.4}};
        \node () at (-3.175, -0.1) {\footnotesize \textcolor{darkgray}{\rotatebox{90}{Human Labeler Entropy}}};
    \end{tikzpicture}}
    \end{subfigure}
    \caption{Evaluation of how embedding norms correspond to measures of network confidence. In all cases, we normalize the embedding magnitudes by the in-dataset maximum and bin them into twenty buckets. For every bucket with more than $50$ embeddings, we evaluate the corresponding metric. We represent the number of embeddings per bucket using the marker size in the left-hand plot.}
    \label{fig:norm_as_confidence}
\end{figure*}
\section{Embedding Norm as Network Confidence}
\label{sec:norm_confidence}

Given the simulations in Section \ref{ssec:confidence_simulations}, we expect that training cos.sim.-based SSL models should result in common input samples receiving higher norm. We therefore use this section to show the various ways in which embedding norms encode a network's confidence in practice. All model implementation details can be found in Appendix \ref{app:experiment_setup}.

\paragraph{Embedding Norms Encode Novelty.} 
Figure \ref{fig:in_out_violin} demonstrates how embedding norms characterize a sample's ``out-of-distributionness''. On the left side of the figure, we trained SimCLR and SimSiam models on the Cifar-10 train set for 512 epochs and evaluated them across different data splits, normalizing all embedding norms by the Cifar-10 train set mean. The results reveal a clear pattern: embedding norms decrease progressively with increasing distributional distance from the training data. For example, the Cifar-10 test set contains novel but distributionally similar samples and therefore results in only slightly reduced norms. In contrast, the Cifar-100 data splits exhibit substantially smaller norms due to their greater distributional shift. This relationship holds symmetrically when training on Cifar-100 and evaluating on Cifar-10, as seen on the right side of Figure \ref{fig:in_out_violin}.

\paragraph{Embedding Norms Encode Classification Accuracy.} Another measure of a network's confidence in an embedding is the accuracy with which that sample is classified. To this end, we use the same experimental setup as above and train SimCLR and SimSiam on the Cifar-10, Cifar-100, and ImageNet100 datasets.\footnote{We default to the ImageNet-100 split \citep{understanding_contr_learn} at \texttt{huggingface.co/datasets/clane9/imagenet-100}.} We then normalize the embedding magnitudes by the maximum across the dataset and bucket the embeddings into ranges of $0.05$, giving us 20 embedding buckets over the dataset. Figure \ref{fig:norm_as_confidence} (left) then shows the per-bucket accuracy of a $k$-nn classifier which was fit on all the embeddings with respect to the cosine similarity metric. Indeed, we see that the $k$-nn classifier's accuracy shows a clear monotonic trend with the embedding norms across datasets and SSL models. 

\paragraph{Embedding Norms Encode Human Confidence.} Interestingly, not only does the embedding norm provide a measure for the sample's novelty and its classification accuracy, but it also provides a signal for human labelers' confidence and their agreement among one another. Using the Cifar-10-N and Cifar-100-N labels from \citet{cifarN}, where each training sample is labeled by the consensus label over multiple human annotators, Figure \ref{fig:norm_as_confidence} (middle) shows higher embedding norms correspond to more accurate consensus labels. Similarly, the Cifar-10-H dataset from \citet{cifarH} provides $\approx 50$ human predictions for each image from the Cifar-10 test set, allowing us to evaluate the labelers' entropy. Figure \ref{fig:norm_as_confidence} (right) shows that, as the embedding norms grow, the human labels have less entropy and are therefore more likely to agree with one another.

\paragraph{Takeaways.} Under the common assumption that an SSL embedding's direction represents its information, these experiments show that the embedding's norm represents how \emph{confident} the network was in this information. Furthermore, this measure of network confidence is inherent to all cos.sim.-based loss functions and emerges naturally during training. Thus, an SSL latent space looks less like a smooth sphere and more like a spiky ball, with the spikes corresponding to regions of known data samples. This observation has implications for few-shot learning settings, in which one has pre-trained on a large dataset and then wishes to adjust the model to a second, smaller dataset:

\begin{question}
    By using the embedding norm as a measure for a sample's novelty, can one more precisely guide the training process on unseen inputs?
\end{question} 
\section{The Embedding-Norm Effect in Practice}
\label{sec:convergence}

To understand how the embedding-norm effect influences cosine-similarity-based SSL training, we investigate three distinct interventions which should mitigate it. These interventions provide controlled settings to analyze the empirical relationship between embedding norms and SSL training.

\paragraph{Weight Decay}

Our first intervention mechanism---weight decay---is already present in essentially all SSL models. The idea here is that adding a penalty on the weights implicitly regularizes the embedding norms \citep{normface}.
Figure \ref{fig:weight_decay_ablation} demonstrates this effect in SimCLR and SimSiam training: without weight decay ($\gamma=0$), embeddings grow unconstrained, while excessive weight decay ($\gamma=5e\text{-}2$) causes collapse. With appropriate values ($\gamma=1e\text{-}5$ for SimCLR, $\gamma=5e\text{-}4$ for SimSiam), norms decrease gradually, leading to improved $k$-nn accuracy. Interestingly, the embedding norm's impact on the $k$-nn accuracy is much more pronounced in the attraction-only setting. We also see that, even without weight decay, the embeddings can shrink (as occurs for SimSiam). We attribute this to the embeddings being produced by shared weights: although the gradients require all embeddings to grow, a shared weights matrix may not be able to independently update every point's position.

\definecolor{ballblue}{rgb}{0.61, 0.81, 1.0}
\definecolor{azure}{rgb}{0., 0.45, 1}
\definecolor{darkblue}{rgb}{0.0, 0.0, 0.55}

\begin{figure}
    \centering
    \resizebox{\linewidth}{!}{%
    \begin{tikzpicture}
    \node[inner sep=0pt] (img) at (0, 0) {\includegraphics[width=\linewidth, trim={1.05cm, 1in, 0cm, 0.4cm}, clip]{Images/wd_sweep.pdf}};

    \node () at (-2.13, 3) {\large \textcolor{darkgray}{SimCLR}};
    \node () at (2.1, 3) {\large \textcolor{darkgray}{SimSiam}};
    
    \node () at (-4.33, 2.6) {\small \textcolor{darkgray}{40}};
    \node () at (-4.33, 1.5) {\small \textcolor{darkgray}{20}};
    \node () at (-4.25, 0.4) {\small \textcolor{darkgray}{0}};
    \node () at (-4.77, 1.5) {\small \textcolor{darkgray}{\rotatebox{90}{Embedding Norm}}};

    \node () at (-4.33, -0.4) {\small \textcolor{darkgray}{80}};
    \node () at (-4.33, -1.22) {\small \textcolor{darkgray}{60}};
    \node () at (-4.33, -2.07) {\small \textcolor{darkgray}{40}};
    \node () at (-4.77, -1.2) {\small \textcolor{darkgray}{\rotatebox{90}{$k$-nn Accuracy}}};

    \node () at (-3.45, -2.6) {\small \textcolor{darkgray}{2}};
    \node () at (-2.56, -2.6) {\small \textcolor{darkgray}{8}};
    \node () at (-1.67, -2.6) {\small \textcolor{darkgray}{32}};
    \node () at (-0.8, -2.6) {\small \textcolor{darkgray}{128}};
    \node () at (-2.1, -3.05) {\small \textcolor{darkgray}{Train Epoch}};

    \node () at (0.755, -2.6) {\small \textcolor{darkgray}{2}};
    \node () at (1.64, -2.6) {\small \textcolor{darkgray}{8}};
    \node () at (2.51, -2.6) {\small \textcolor{darkgray}{32}};
    \node () at (3.4, -2.6) {\small \textcolor{darkgray}{128}};
    \node () at (2.1, -3.05) {\small \textcolor{darkgray}{Train Epoch}};

    \draw[ballblue, line width=0.07cm] (-4.5, -3.5) -- (-3.9, -3.5);
    \draw[azure, line width=0.07cm] (-1.8, -3.5) -- (-1.2, -3.5);
    \draw[darkblue, line width=0.07cm] (1.5, -3.5) -- (2.1, -3.5);

    \node[inner sep=0pt] () at (-3, -3.52) {\textcolor{darkgray}{\scriptsize No weight decay}};
    \node[inner sep=0pt] () at (-0.02, -3.52) {\textcolor{darkgray}{\scriptsize Standard weight decay}};
    \node[inner sep=0pt] () at (3.07, -3.52) {\textcolor{darkgray}{\scriptsize High weight decay}};

    \end{tikzpicture}%
    }
    \caption{Effect of the weight decay on the embedding norms in SimCLR and SimSiam. Epochs are log-scale and go from 1 to 256. Corresponding $k$-nn accuracies are plotted in the bottom row.}
    \label{fig:weight_decay_ablation}
\end{figure}

\paragraph{Cut-Initialization.}

As shown in Figure \ref{fig:weight_decay_ablation}, weight decay gradually reduces embedding norms over time, but doesn't control them at the start of training. To address this, we propose \emph{cut-initialization} - dividing all network weights by a constant $c$ at initialization. This ensures small embedding norms at initialization which are then kept small via weight decay. We implement this uniformly across all layers for simplicity (Listing \ref{alg:cut_init}). Interestingly, a variant of this can be found in HuggingFace's default image transformer code \cite{pytorch} and we know at least one cos.sim.-based SSL model which uses it \citep{beitv2}.

\begin{figure}
    \begin{lstlisting}[caption={PyTorch code for our cut-initialization layer.}, label={alg:cut_init}, captionpos=b]
@torch.no_grad()
def cut_init(model, c):
    for param in model.parameters():
        param.data = param.data / c
\end{lstlisting}
\end{figure}

We study the interplay between cut-initialization and weight decay values on SimCLR and SimSiam in Table \ref{tbl:wd_cut}. Specifically, we report the $k$-nn classification accuracy after 100 epochs on the CIFAR-100 and ImageNet-100 datasets and see that, in both the contrastive and non-contrastive settings, pairing cut-initialization with weight decay accelerates the training process. As was the case for the weight decay, the difference is more stark in the non-contrastive setting. $c=2$ performed best for SimCLR, providing an additional 2\% in accuracy at the default weight decay, while $c=8$ led to about a 10\% increase for SimSiam. For the remaining experiments, we use $c=3$ for SimCLR and $c=9$ for SimSiam. We report a variant of Figure \ref{fig:weight_decay_ablation} including BYOL experiments in Figure \ref{fig:cut_experiments}. We furthermore show the $k$-nn classifier accuracies at $500$ epochs with and without cut-initialization in Table \ref{tbl:accuracies}. Here we see that pairing SSL models with cut-initialization often helps the model reach higher final accuracies.

We also evaluate cut-initialization in imbalanced data settings. For Cifar-10, we use the exponential split from \cite{dassot}, where class $i$ has $n_i = 5000 \cdot 1.5^{-i}$ samples. Similarly for Cifar-100, the $i$-th class receives $n_i = 500 \cdot 1.034^{-i}$ samples. This way, all classes are represented and both imbalanced datasets contain roughly 15K samples. We also use Flowers102's naturally long-tailed test set for training \citep{flowers}, evaluating on its validation set. Table \ref{tbl:imbalanced_experiments} then shows that, in class-imbalanced settings, pairing SSL training with interventions on the embedding-norm effect can provide double-digit accuracy improvements.

\begin{table}
    \centering
    \captionof{table}{$k$-nn accuracy at epoch 100 for various values of cut-constant $c$ and weight decay $\lambda$. \emph{Left}: SimCLR  on Cifar-100. \emph{Right}: SimSiam on ImageNet-100. Default weight-decay is underlined.}
    \label{tbl:wd_cut}
    \resizebox{\linewidth}{!}{%
    \begin{tabular}{cl cccc cc cccc}
    \toprule
    % & & \multicolumn{10}{c}{Weight Decay $\lambda$} \\
    & & \multicolumn{4}{c}{\large SimCLR} & & & \multicolumn{4}{c}{\large SimSiam} \vspace*{0.2cm}\\
    & & \multicolumn{4}{c}{Weight Decay $\lambda$} & & & \multicolumn{4}{c}{Weight Decay $\lambda$} \\
    & & 1e-8 & \underline{1e-6} & 5e-6 & 1e-5 & & & 5e-5 & 1e-4 & \underline{5e-4} & 1e-3\\
    \cmidrule{3-6} \cmidrule{9-12}
    \multirow{4}{*}{\rotatebox[origin=c]{90}{\makecell{Cut}}} & $c=1$ & 40.8 & 40.5 & 40.9 & 41.5 & & & 36.7 & 38.5 & 40.5 & 44.7 \\
    & $c=2$ & 42.7 & 42.8 & 42.9 & 42.2 & & & 41.1 & 44.0 & 48.8 & 42.1\\
    & $c=4$ & 42.3 & 41.4 & 42.0 & 41.1 & & & 40.2 & 41.2 & 49.8 & 49.1\\
    & $c=8$ & 37.1 & 36.8 & 37.9 & 37.3 & & & 44.4 & 46.4 & 50.2 & 53.2\\
    \bottomrule
    \end{tabular}%
    }
\end{table}


% \begin{table}
%     %\fontsize{10pt}{10pt}\selectfont
%     \centering
%     \begin{tabular}{cl cccc}
%     \midrule \vspace*{0.1cm}
%     & & \multicolumn{4}{c}{Weight Decay $\lambda$} \\
%     & & 5e-5 & 1e-4 & \underline{5e-4} & 1e-3 \\
%     \cmidrule{3-6}
%     \multirow{4}{*}{\rotatebox[origin=c]{90}{\makecell{Cut}}} & $c=1$ & 36.7 & 38.5 & 40.5 & 44.7 \\
%     & $c=2$ & 41.1 & 44.0 & 48.8 & 42.1 \\
%     & $c=4$ & 40.2 & 41.2 & 49.8 & 49.1 \\
%     & $c=8$ & 44.4 & 46.4 & 50.2 & 53.2 \\
%     \bottomrule
%     \end{tabular}
%     \captionof{table}{SimSiam $k$-nn accuracy (epoch 100) on Imagenet-100 for various values of $c$ and $\lambda$. Default weight-decay is underlined.}
%     \label{tbl:simsiam_wd_cut}
% \end{table}


\paragraph{GradScale Layer.}

Perhaps the cleanest way to overcome the embedding-norm effect is to simply rescale the gradient directly. We achieve this using a custom PyTorch \texttt{autograd.Function} which we refer to as \emph{GradScale} (for a full implementation, see Listing~\ref{alg:grad_scaling} in the Appendix). This layer accepts a \emph{power} parameter $p$ and is simply the identity function in the forward pass. However, the backwards pass multiplies each sample $z_i$'s contribution to the gradient by $\|z_i\|^p$. Thus, choosing power $p=0$ gives the default setting while choosing $p=1$ will cancel the gradients' dependence on the embedding norm. We visualize the resulting gradient fields with powers $p=0$ and $p=1$ for a 2D embedding space in Figure~\ref{fig:grad_scaling}. We refer to training with GradScale power $p=1$ simply as GradScale.

Under GradScale, the gradient norms differ from those during default training, necessitating a new learning rate schedule. Traditionally, SSL models are trained with $10$ epochs of linear warmup followed by a cosine-annealing schedule \citep{simclr}. However, the schedule has an implicit division by the embedding norms over the course of training. 
%Thus, the \emph{effective} learning rates are given by $\gamma_{\text{eff}}(t) = \gamma(t) \cdot b \cdot \big(\sum_{i=1}^b \|z_i(t)\|\big)^{-1}$, where $\gamma(t)$ is the learning rate schedule at epoch $t$ and $\frac{1}{b} \cdot \sum_{i=1}^b \|z_i(t)\|$ is a batch's mean embedding norm at epoch $t$. 
We therefore simulate this effective learning rate for the GradScale setting. Namely, we choose base learning rate $\gamma' = \gamma/6$ with 100 linear warmup epochs followed by cosine annealing, where $\gamma$ is the default learning rate. This was the first relatively stable schedule which we found for the $p=1$ setting and we performed no additional tuning.

Table \ref{tbl:accuracies} demonstrates that, when training remains stable, SimCLR's $k$-nn accuracy benefits from GradScale's embedding norm cancellation. Consistent with our cut-initialization experiments, this improvement becomes particularly pronounced on the class-imbalanced datasets in Table \ref{tbl:imbalanced_experiments}: GradScale provides a roughly 5\% accuracy increase across the imbalanced datasets. While these results are promising, we observed that GradScale with $p=1$ failed to converge on Imagenet-100 and for non-contrastive models. This limitation aligns with these models' known sensitivity issues \cite{simsiam_avoid_collapse, BYOL_orthogonality}.

\begin{figure}
    \centering
    \includegraphics[width=\linewidth]{Images/custom_autograd_function_two_settings.pdf}
    \caption{\emph{Left}: the default gradient field of the cosine similarity with respect to the north-pointing line (blue). \emph{Right}: the gradient field using the GradScale layer with $p=1$.}
    \label{fig:grad_scaling}
\end{figure}


\begin{table}
    \centering
    \caption{$k$-nn accuracies at epoch 500 for default, cut-initialized and GradScale training on standard image datasets.}\vspace*{0.1cm}
    \label{tbl:accuracies}
    \resizebox{\linewidth}{!}{%
        \begin{tabular}{cr c c c}
        \toprule
        & & Cifar-10 & Cifar-100 & Imagenet-100 \\
        \cmidrule{3-5}
        % \cmidrule(l{2pt}r{2pt}){3-3} \cmidrule(l{2pt}r{2pt}){4-4} \cmidrule(l{2pt}r{2pt}){5-5}
        \multirow{3}{*}{SimCLR} & Default & 85.2 & 51.9 & 59.4 \\
        & Cut ($c=3$) & 87.0 & 52.6 & 60.9 \\
        & GradScale & 86.5 & 54.0 & 01.0 \\
        \midrule
        % \cmidrule(l{2pt}r{2pt}){3-3} \cmidrule(l{2pt}r{2pt}){4-4} \cmidrule(l{2pt}r{2pt}){5-5}
        \multirow{2}{*}{SimSiam} & Default & 87.0 & 61.1 & 62.0 \\
        & Cut ($c=9$) & 89.0 & 61.5 & 67.2 \\
        % \cmidrule(l{2pt}r{2pt}){3-3} \cmidrule(l{2pt}r{2pt}){4-4} \cmidrule(l{2pt}r{2pt}){5-5} 
        %\multirow{2}{*}{BYOL} & Default & 88.2 & 61.9 & 67.7 \\
        %& Cut ($c=9$) & 88.6 & 61.4 & 68.6 \\
        \bottomrule
        \end{tabular}%
    }
\end{table}

\begin{table}
    \centering
    \caption{$k$-nn accuracies at epoch 500 for default, cut-initialized and GradScale training on class-imbalanced image datasets.}\vspace*{0.1cm}
    \label{tbl:imbalanced_experiments}
    \resizebox{\linewidth}{!}{%
    \begin{tabular}{lrccc}
    \toprule
    & & \makecell{Cifar-10\\Unb.} & \makecell{Cifar-100\\Unb.} & \makecell{Flowers\\Unb.} \\
    \cmidrule{3-5}
    \multirow{3}{*}{SimCLR} & Default & 56.5 & 24.3 & 42.6\\
    & Cut $(c=3)$ & 61.1 & 26.3 & 61.6 \\
    & GradScale & 61.3 & 29.1 & 47.2 \\
    \midrule
    \multirow{2}{*}{SimSiam} & Default & 47.0 & 21.6 & 22.5 \\
    & Cut $(c=9)$ & 61.7 & 31.5 & 39.9 \\
    \bottomrule
    \end{tabular}%
    }
\end{table}

\paragraph{Takeaways.}

These results make it clear that the embedding norm effect impacts SSL training -- particularly in non-contrastive settings -- and can be mitigated using our proposed strategies. The effect appears most detrimental in class-imbalanced settings, aligning with our results on SSL confidence: imbalanced data creates variance in embedding norms, destabilizing training. Nonetheless, there remain questions which are beyond the scope of this work:
\begin{question}
    Why are non-contrastive architectures more sensitive to the embedding-norm effect?
\end{question}
In addition to seeing that the embedding-norm effect is more pronounced in attraction-only settings, we have found that the embeddings can shrink even in the absence of weight decay. We attribute both phenomena to the network's shared weights: while our theory predicts uniform embedding growth, producing these embeddings via a single set of weights creates competition between different regions of the latent space. This competition would be especially relevant for the InfoNCE loss, which enforces uniformity over the latent hypersphere \cite{understanding_contr_learn}.

\begin{question}
    Are there SSL training schemes in which the embedding-norm effect is beneficial?
\end{question}

We have been careful to not describe the embedding-norm effect as a strictly negative phenomenon. Consider the common transfer-learning setting in which prototypical class examples should anchor the learned representation \citep{prototypical_1, prototypical_2}. Our findings suggest the embedding-norm effect naturally supports this goal: prototypical examples should have large embedding norms and consequently would receive smaller gradients.
\section{Discussion}
\omniUIST is capable of tracking a passive tool with an accuracy of roughly 6.9 mm and, at the same time, deliver a maximum force of up to 2 N to the tool. This is enabled by our novel gradient-based approach in 3D position reconstruction that accounts for the force exerted by the electromagnet. 

Over extended periods of time, \omniUIST can comfortably produce a force of 0.615 N without the risk of overheating. In our applications, we show that \omniUIST has the potential for a wide range of usage scenarios, specifically to enrich AR and VR interactions.

\omniUIST is, however, not limited to spatial applications. We believe that \omniUIST can be a valuable addition to desktop interfaces, e.g., navigating through video editing tools or gaming. We plan to broaden \omniUIST's usage scenarios in the future.

The overall tracking performance of \omniUIST suffices for interactive applications such as the ones shown in this paper. The accuracy could be improved by adding more Hall sensors, or optimizing their placement further (e.g., placing them on the outer hull of the device).
Furthermore, a spherical tip on the passive tool that more closely resembles the dipole in our magnetic model could further improve \omniUIST's accuracy. We believe, however, that the design of \omniUIST represents a good balance of cost and complexity of manufacturing, and accuracy.

Our current implementation of \omniUIST and the accompanying tracking and actuation algorithms assumes the presence of a single passive tool. Our method, however, potentially generalizes to tracking multiple passive tools by accounting for the presence of multiple permanent magnets. This poses another interesting challenge: the magnets of multiple tools will interact with each other, i.e., attract and repel each other.The electromagnet will also jointly interact with those tools, leading to challenges in terms of computation and convergence. We believe that our gradient-based optimization can account for such interactions and plan to investigate this in the future.

In developing and testing our applications, we found that \omniUIST's current frame rate of 40 Hz suffices for many interactive scenarios. The frame rate is a trade-off between speed and accuracy. In our tests, decreasing the desired accuracy in our optimization doubled the frame rate, while resulting in errors in the 3D position estimation of more than 1 cm, however. Finding the sweet spot for this trade-off depends on the application. While our applications worked well with 40 Hz and the current accuracy, more intricate actions such as high-precision sculpting might benefit from higher frame rates \textit{and} precision.
Reducing the latency of several system components (e.g., sensor latency, convergence time) is another interesting direction of future research. 

Furthermore, the control strategy we used was fairly naïve, as it only takes the current tool position into account. A model predictive strategy could account for future states, user intent, and optimize to reduce heating. We will explore in the next chapter how model predictive approaches can be used for haptic systems.

Overall, the main benefits of \omniUIST lie in the high accuracy and large force it can produce. It does so without mechanically moving parts, which would be subject to wear.
Such wear is not the case for our device, because it is exclusively based on electromagnetic force. We believe that different form factors of \omniUIST (e.g., body-mounted, larger size) can present interesting directions of future research. \add{A body-mounted version could be interesting for VR applications in which the user moves in 3D space. The larger size could result in more discernible points.}

Additionally, the influence of strength on user perception and factors such as just-noticeable-difference will allow us to characterize the benefits and challenges of \omniUIST, and electromagnetic haptic devices in general.
We believe that \omniUIST opens interesting directions for future research in terms of novel devices, and magnetic actuation and tracking.

\bibliography{references}
\bibliographystyle{icml2024}


%%%%%%%%%%%%%%%%%%%%%%%%%%%%%%%%%%%%%%%%%%%%%%%%%%%%%%%%%%%%%%%%%%%%%%%%%%%%%%%
%%%%%%%%%%%%%%%%%%%%%%%%%%%%%%%%%%%%%%%%%%%%%%%%%%%%%%%%%%%%%%%%%%%%%%%%%%%%%%%
% APPENDIX
%%%%%%%%%%%%%%%%%%%%%%%%%%%%%%%%%%%%%%%%%%%%%%%%%%%%%%%%%%%%%%%%%%%%%%%%%%%%%%%
%%%%%%%%%%%%%%%%%%%%%%%%%%%%%%%%%%%%%%%%%%%%%%%%%%%%%%%%%%%%%%%%%%%%%%%%%%%%%%%
\newpage
\appendix
\onecolumn
\renewcommand{\thefigure}{S\arabic{figure}}
\setcounter{figure}{0}  
\renewcommand{\thetable}{S\arabic{table}}
\setcounter{table}{0} 


\section{Proofs}
\subsection{Proof of Proposition~\ref{prop:cos_sim_grads}}
\label{prf:prop_grad_grows}
\cosgrads*
\begin{proof}
    We are taking the gradient of $\mathcal{L}^\mathcal{A}_i$ as a function of $z_i$. The principal idea is that the gradient has a term with direction $\hat{z}_j$ and a term with direction $-\hat{z}_i$. We then disassemble the vector with direction $\hat{z}_j$ into its component parallel to $z_i$ and its component orthogonal to $z_i$. In doing so, we find that the two terms with direction $z_i$ cancel, leaving only the one with direction orthogonal to $z_i$.
    
    Writing it out fully, we have $\mathcal{L}^\mathcal{A}_i = -z_i^\top z_j / (\|z_i\| \cdot \|z_j\|)$. Taking the gradient amounts to using the quotient rule, with $f = -z_i^\top z_j$ and $g = \|z_i\| \cdot \|z_j\| = \sqrt{z_i^\top z_i} \cdot \sqrt{z_j^\top z_j}$. Taking the derivative of each, we have
    \begin{align*}
        f' &= -\mathbf{z}_j \\
        g' &= \|z_j\| \frac{z_i}{\sqrt{z_i^\top z_i}} = \|z_j\| \frac{\mathbf{z}_i}{\|z_i\|} \\
        \implies \frac{f' g - g' f}{g^2} &= \frac{- \left(\mathbf{z}_j \cdot \|z_i\| \cdot \|z_j\| \right) + \left(\|z_j\| \frac{\mathbf{z}_i}{\|z_i\|} \cdot z_i^\top z_j \right)}{\|z_i\|^2 \cdot \|z_j\|^2} \\
        &= \frac{-\mathbf{z}_j}{\|z_i\| \cdot \|z_j\|} + \frac{\mathbf{z}_i z_i^\top z_j}{\|z_i\|^3 \|z_j\|},
    \end{align*}
    where we use boldface $\mathbf{z}$ to emphasize which direction each term acts along. We now substitute $\cos(\phi_{ij}) = z_i^\top z_j / (\|z_i\| \cdot \|z_j\|)$ in the second term to get
    \begin{equation}
        \label{eq:quotient_rule}
        \frac{f' g - g' f}{g^2} = \frac{-\hat{z}_j}{\|z_i\|} + \frac{\mathbf{z}_i \cos(\phi)}{\|z_i\|^2}
    \end{equation}

    It remains to separate the first term into its sine and cosine components and perform the resulting cancellations. To do this, we take the projection of $\hat{z}_j = \mathbf{z}_j / \|z_j\|$ onto $\mathbf{z}_i$ and onto the plane orthogonal to $\mathbf{z}_i$. The projection of $\hat{z}_j$ onto $\mathbf{z}_i$ is given by
    \[ \cos \phi_{ij} \frac{\mathbf{z}_i}{\|z_i\|} \]
    while the projection of $\mathbf{z}_j / \|z_j\|$ onto the plane orthogonal to $\mathbf{z}_i$ is
    \[ \left( \mathbf{I} - \frac{z_i z_i^\top}{\|z_i\|^2} \right) \frac{\mathbf{z}_j}{\|z_j\|}. \]
    It is easy to assert that these components sum to $\mathbf{z}_j/\|z_j\|$ by replacing the $\cos \phi_{ij}$ by $\frac{z_i^\top z_j}{\|z_i\|\cdot \|z_j\|}$.

    We plug these into Eq.~\ref{eq:quotient_rule} and cancel the first and third term to arrive at the desired value:
    \begin{align*}
        \frac{f' g - g' f}{g^2} = &-\frac{1}{\|z_i\|} \cos \phi \frac{\mathbf{z}_i}{\|z_i\|} \\
        &- \frac{1}{\|z_i\|} \cdot \left( \mathbf{I} - \frac{z_i z_i^\top}{\|z_i\|^2} \right) \frac{\mathbf{z}_j}{\|z_j\|} \\
        &+ \frac{\mathbf{z}_i \cos(\phi)}{\|z_i\|^2} \\
        = &\frac{-1}{\|z_i\|} \cdot \left( \mathbf{I} - \frac{z_i z_i^\top}{\|z_i\|^2} \right) \frac{\mathbf{z}_j}{\|z_j\|}.
    \end{align*}
\end{proof}

We visualize the loss landscape of the cosine similarity function in Figure \ref{fig:cos_sim_surface}. 

\begin{figure}
    \centering
    \begin{subfigure}{0.45\linewidth}
        \centering 
        \includegraphics[width=1\linewidth]{Images/cosine_similarity_surface_with_circles.pdf}
    \end{subfigure}%
    \begin{subfigure}{0.45\linewidth}
        \centering 
        \includegraphics[width=0.8\linewidth]{Images/cosine_similarity_2D_heatmap.pdf}
    \end{subfigure}
    \caption{Cosine similarity with respect to the direction indicated by the blue line. Three circles of radii 0.5, 1, and 2 are superimposed to show that, for higher norms, the cosine similarity is less steep. Left: 3D Surface plot, right: 2D topview plot.}
    \label{fig:cos_sim_surface}
\end{figure}


\subsection{InfoNCE Gradients}
\label{app:infonce_grads}
\infoncegrads*
\begin{proof}
    We are interested in the gradient of $\mathcal{L}_i^\mathcal{R}$ with respect to $z_i$. By the chain rule, we get
    \begin{align*}
        \nabla_i^\mathcal{R} &= -\frac{\sum_{k \not\sim i} \text{ExpSim}(z_i, z_k) \frac{\partial \frac{z_i^\top z_k}{\|z_i\| \cdot \|z_k\|}}{\partial z_i}}{\sum_{k \not\sim i} \text{ExpSim}(z_i, z_k)} \\
        &= -\frac{\sum_{k \not\sim i} \text{ExpSim}(z_i, z_k) \frac{\partial \frac{z_i^\top z_k}{\|z_i\| \cdot \|z_k\|}}{\partial z_i}}{S_i}
    \end{align*}
    It remains to substitute the result of Prop. \ref{prop:cos_sim_grads} for $\partial \frac{z_i^\top z_k}{\|z_i\| \cdot \|z_k\|} / \partial z_i$.

    We sum this this with the gradients of the attractive term to obtain the full InfoNCE gradient, completing the proof.
\end{proof}

We note that the repulsive force is weighted average over a set of unit vectors. Consequently, the repulsive gradient is smaller than the attractive one. Additionally, we point out that these gradients are symmetric: just like positive and negative samples $z_j$ and $z_k$ affect $z_i$, $z_i$ affects $z_j$ and $z_k$.

\subsection{Proof of Corollary~\ref{cor:embeddings_grow}}
\label{prf:cor_embeddings_grow}
\begin{proof}
    First, consider that we applied the cosine similarity's gradients from Proposition~\ref{prop:cos_sim_grads}. Since $z_i$ and $(z_j)_{\perp z_i}$ are orthogonal, $\|z_i'\|_2^2 = \|z_i\|^2 + \frac{\gamma^2}{\|z_i\|^2}\|(z_j)_{\perp z_i}\|^2$. The second term is positive if $\sin \phi_{ij} > 0$.

    The same exact argument holds for the InfoNCE gradients. The gradient is orthogonal to the embedding, so a step of gradient descent can only increase the embedding's magnitude.
\end{proof}

\subsection{Proof of Theorem~\ref{thm:convergence_rate}}
\label{prf:thm_convergence_rate}
We first restate the theorem:

Let $z_i$ and $z_j$ be positive embeddings with equal norm, i.e. $\|z_i\| = \|z_j\| = \rho$. Let $z_i'$ and $z_j'$ be the embeddings after 1 step of gradient descent with learning rate $\gamma$. Then the change in cosine similarity is bounded from above by:
\begin{equation*}
    \hat{z}_i'^\top \hat{z}_j' - \hat{z}_i^\top \hat{z}_j < \frac{\gamma \sin^2 \phi_{ij}}{\rho^2} \left[ 2 - \frac{\gamma \cos \phi}{\rho^2} \right].
\end{equation*}

\noindent We now proceed to the proof:
\begin{proof}
    Let $z_i$ and $z_j$ be two embeddings with equal norm\footnote{We assume the Euclidean distance for all calculations.}, i.e. $\|z_i\| = \|z_j\| = \rho$. We then perform a step of gradient descent to maximize $\hat{z}_i^\top \hat{z}_j$. That is, using the gradients in \ref{prop:cos_sim_grads} and learning rate $\gamma$, we obtain new embeddings $z_i' = z_i + \frac{\gamma}{\|z_i\|} (\hat{z}_j)_{\perp z_i}$ and $z_j' = z_j + \frac{\gamma}{\|z_j\|} (\hat{z}_i)_{\perp z_j}$. Going forward, we write $\delta_{ij} = (\hat{z}_j)_{\perp z_i}$ and $\delta_{ji} = (\hat{z}_i)_{\perp z_j}$, so $z_i' = z_i + \frac{\gamma}{\rho} \delta_{ij}$ and $z_j' = z_j + \frac{\gamma}{\rho} \delta_{ji}$. Notice that since $z_i$ and $\delta_{ij}$ are orthogonal, by the Pythagorean theorem we have $\|z_i'\|^2 = \|z_i\|^2 + \frac{\gamma^2}{\rho^2}\|\delta_{ij}\|^2 \geq \|z_i\|^2$. Lastly, we define $\rho' = \|z_i'\| = \|z_j'\|$.

    We are interested in analyzing $\hat{z}_i'^\top \hat{z}_j' - \hat{z}_i^\top \hat{z}_j$. To this end, we begin by re-framing $\hat{z}_i'^\top \hat{z}_j'$:
    \begin{align*}
        \hat{z}_i'^\top \hat{z}_j' &= \left(\frac{z_i + \frac{\gamma}{\rho} \delta_{ij}}{\rho'}\right)^\top \left(\frac{z_j + \frac{\gamma}{\rho} \delta_{ji}}{\rho'}\right) \\
        &= \frac{1}{\rho'^2}\left[ z_i^\top z_j + \gamma \frac{z_i^\top \delta_{ji}}{\rho'} + \gamma \frac{z_j^\top \delta_{ij}}{\rho'} + \gamma^2 \frac{\delta_{ij}^\top \delta_{ji}}{\rho'^2} \right].
    \end{align*}

    We now consider that, since $\delta_{ij}$ is the projection of $\hat{z}_j$ onto the subspace orthogonal to $z_i$, we have that the angle between $z_i$ and $\delta_{ji}$ is $\pi/2 - \phi_{ij}$. Plugging this in and simplifying, we obtain
    \begin{align*}
        z_i^\top \delta_{ji} &= \|z_i\| \cdot \|\delta_{ji}\| \cos (\pi/2 - \phi_{ij}) \\
        &= \|z_i\| \cdot \|\delta_{ji}\| \sin \phi_{ij} \\
        &= \rho \sin^2 \phi_{ij}.
    \end{align*}
    By symmetry, the same must hold for $z_j^\top \delta_{ij}$.
    
    Similarly, we notice that the angle $\psi_{ij}$ between $\delta_{ij}$ and $\delta_{ji}$ is $\psi_{ij} = \pi - \phi_{ij}$. The reason for this is that we must have a quadrilateral whose four internal angles must sum to $2\pi$, i.e. $\psi_{ij} + \phi_{ij} + 2 \frac{\pi}{2} = 2 \pi$. Thus, we obtain $\delta_{ij}^\top \delta_{ji} = \|\delta_{ij}\| \cdot \|\delta_{ji}\| \cos(\psi) = -\sin^2 \phi_{ij} \cos \phi_{ij}$.

    We plug these back into our equation for $\hat{z}_i'^\top \hat{z}_j'$ and simplify:
    \begin{align*}
        \hat{z}_i'^\top \hat{z}_j' &= \frac{1}{\rho'^2}\left[ z_i^\top z_j + \gamma \frac{z_i^\top \delta_{ji}}{\rho} + \gamma \frac{z_j^\top \delta_{ij}}{\rho} + \gamma^2 \frac{\delta_{ij}^\top \delta_{ji}}{\rho^2} \right] \\
        &= \frac{1}{\rho'^2}\left[ z_i^\top z_j + \gamma \frac{\rho \sin^2 \phi_{ij}}{\rho} + \gamma \frac{\rho \sin^2 \phi_{ij}}{\rho} - \gamma^2 \frac{\sin^2 \phi_{ij} \cos \phi_{ij}}{\rho^2} \right] \\
        &= \frac{1}{\rho'^2}\left[ z_i^\top z_j + 2 \gamma \sin^2 \phi_{ij} - \gamma^2 \frac{\sin^2 \phi_{ij} \cos \phi_{ij}}{\rho^2} \right].
    \end{align*}

    We now consider the original term in question:
    \begin{align*}
        \hat{z}_i'^\top \hat{z}_j' - \hat{z}_i^\top \hat{z}_j &= \frac{1}{\rho'^2}\left[ z_i^\top z_j + 2 \gamma \sin^2 \phi_{ij} - \gamma^2 \frac{\sin^2 \phi_{ij} \cos \phi_{ij}}{\rho^2} \right] - \frac{z_i^\top z_j}{\rho^2} \\
        &\leq \frac{1}{\rho^2}\left[ z_i^\top z_j + 2 \gamma \sin^2 \phi_{ij} - \gamma^2 \frac{\sin^2 \phi_{ij} \cos \phi_{ij}}{\rho^2} \right] - \frac{z_i^\top z_j}{\rho^2} \\
        &= \frac{1}{\rho^2}\left[ 2 \gamma \sin^2 \phi_{ij} - \gamma^2 \frac{\sin^2 \phi_{ij} \cos \phi_{ij}}{\rho^2} \right] \\
        &= \frac{\gamma \sin^2 \phi_{ij}}{\rho^2}\left[ 2 - \frac{\gamma \cos \phi_{ij}}{\rho^2} \right]\\
        &\leq \frac{2 \gamma \sin^2 \phi_{ij}}{\rho^2}
    \end{align*}
    
    This concludes the proof.
\end{proof}

\section{Simulations}
\label{app:simulations}

\subsection{Aparametric Simulations}

For the simulations in Section \ref{ssec:convergence_simulations}, we produced two datasets, $\mathbf{X}_1$ and $\mathbf{X}_2$, independently by randomly sampling points in $\mathbb{R}^20$ from a standard normal distribution and normalizing them to the hypersphere. The $i$-th point in dataset $\mathbf{X}_1$ is the positive counterpart for the $i$-th point in dataset $\mathbf{X}_2$. The first dataset is then set to be static while the second is modified in order to control for the embedding norms and angles between positive pairs.

We optimize the cosine similarity by performing standard gradient descent on the embeddings themselves with learning rate $10$. We consider a dataset ``converged'' when the average cosine similarity between positive pairs exceeds $0.999$.

\paragraph{Controlling for angles.} In order to control for the angle between positive pairs, we use an interpolation value $\alpha \in [-1, 1]$. Let $x_1$ be a static embedding in $\mathbf{X}_1$ and $x_2$ be the embedding in $\mathbf{X}_2$ whose angle we wish to control. In expectation, $\phi(x_1, x_2)$ will be $\pi / 2$. We therefore define the embedding $x_2$ whose angle has been controlled as 
\[ x_2' = x_2 \cdot (1 - |\alpha|) + x_1 \cdot \alpha. \]

In essence, when $\alpha=0$, $x_2' = x_2$. However, when $\alpha=1$ (resp. $\alpha=-1$), $x_2' = x_1$ (resp. $x_2' = -x_1$).

\paragraph{Controlling for embedding norms.} This setting is simpler than the angles between positive pairs. We simply scale $\mathbf{X}_2$ by the desired value.

\subsection{Parametric Simulations}
\label{app:parametric_sim}

We restate the entire implementation for the simulations in Section \ref{ssec:confidence_simulations} for completeness. We choose centers for 4 latent classes $\{c_1, c_2, c_3, c_4\}$ uniformly at random from $\mathbb{S}^{10}$ by randomly sampling vectors from a standard multivariate normal distribution and normalizing them to the hypersphere. We then obtain the latent samples $\tilde{z}$ around center $c_i$ via $z \sim \mathcal{N}(c_i, 0.1 \cdot \mathbf{I})$ and re-normalizing to the hypersphere. For each center, we produce 1K latent samples; these constitute our latent classes. We depict an example of 8 such latent classes (in 3 dimensions) in Figure \ref{fig:orig_latents}. We finally obtain the dataset by generating a random matrix in $\mathbb{R}^{11 \times 64}$ and applying it to the latent samples.

We train the InfoNCE loss via a 2-layer feedforward neural network with the ReLU activation function in the hidden layer. The network's output dimensionality is $\mathbb{R}^{11}$ so that, after normalization, it can reconstruct the original latent classes. We train the network using the supervised InfoNCE loss with a batch size of 128. Each data point's positive pair is simply another data point from the same latent class.

We visualize the learned (unnormalized) embedding space in Figure \ref{fig:learned_latents}.

\begin{figure}
    \centering
    \begin{subfigure}{0.4\linewidth}
    \includegraphics[width=\linewidth]{Images/orig_latents.png}
    \caption{}
    \label{fig:orig_latents}
    \end{subfigure}
    \quad\quad
    \begin{subfigure}{0.4\linewidth}
    \includegraphics[width=\linewidth]{Images/learned_latents.png}
    \caption{}
    \label{fig:learned_latents}
    \end{subfigure}
    \caption{\emph{Left}: A depiction of $8$ latent classes in $3$D obtained via the description in Section \ref{app:parametric_sim}. Dashed lines represent vectors from the origin to the mean of the distribution. \emph{Right}: A depiction of the learned latent space (unnormalized) using the supervised InfoNCE loss after 50 epochs of training.}
    
\end{figure}


\section{Further Discussion and Experiments}
\label{app:experiments}

\subsection{Experimental Setup}
\label{app:experiment_setup}
Unless otherwise stated, we use a ResNet-50 backbone \cite{resnet} and the default settings outlined in the SimCLR \cite{simclr} and SimSiam \cite{simsiam} papers. We use $1$e-$6$ as the default SimCLR weight decay and $5$e-$4$ as the default SimSiam one. When running on Cifar-10 and Cifar-100, we amend the backbone network's first layer as detailed in \citet{simclr}. We use embedding dimensionality $256$ in SimCLR and $2048$ in SimSiam. When reporting embedding norms, we use the projector's output in SimCLR and the predictor's output in SimSiam: these are the spaces where the loss function is applied and therefore where our theory holds.

Due to computational constraints, we run with batch-size 256 in SimCLR. Although each batch is still 256 samples in SimSiam, we simulate larger batch sizes using gradient accumulation. Thus, our default batch-size for SimSiam is 1024. 

\subsection{Opposite-Halves Effects}
\label{app:opposite_halves_effect}

We devote this section of the Appendix to studying the role of the angle between positive samples on the cosine similarity's convergence under gradient descent. Referring back to Figure~\ref{fig:convergence_sim}, we see that the effect is most impactful when the angle between positive embeddings is close to $\pi$, i.e. $\phi_{ij} > \pi - \varepsilon$ for $\varepsilon \rightarrow 0$. The following result shows that this is exceedingly unlikely for a single pair of embeddings in high-dimensional space:
\begin{proposition}
    \label{prop:unlikely_opp_halves}
    Let $x_i, x_j \sim \mathcal{N}(0, \mathbf{I})$ be $d$-dimensional, i.i.d. random variables and let $0 < \varepsilon < 1$. Then \vspace*{-0.1cm}
    \begin{equation}
    \label{eq:opp_halves_unlikely}
    \mathbb{P}\left[ \hat{x}_i^\top \hat{x}_j \geq 1 - \varepsilon \right] \leq \frac{1}{2d(1-\varepsilon)^2}.
    \end{equation}\vspace*{-0.3cm}
\end{proposition}
\begin{proof}
By \citet{distribution_of_cosine_sim}, the cosine similarity between two i.i.d. random variables drawn from $\mathcal{N}(0, \mathbf{I})$ has expected value $\mu = 0$ and variance $\sigma^2 = 1/d$, where $d$ is the dimensionality of the space. We therefore plug these into Chebyshev's inequality:
\begin{align*}
    &\text{Pr} \left[ \left|\frac{x_i^\top x_j}{\|x_i\|\cdot \|x_j\|} - \mu \right|\geq k \sigma \right] \leq \frac{1}{k^2} \\
    \rightarrow & \text{Pr} \left[ \left |\frac{x_i^\top x_j}{\|x_i\|\cdot \|x_j\|} \right |\geq \frac{k}{\sqrt{d}} \right] \leq \frac{1}{k^2}
\end{align*}

\noindent We now choose $k = \sqrt{d}(1 - \varepsilon)$, giving us
\[ \mathbb{P}\left[ \left |\frac{x_i^\top x_j}{\|x_i\| \cdot \|x_j\|}\right | \geq 1 - \varepsilon \right] \leq \frac{1}{d(1-\varepsilon)^2}.\]

It remains to remove the absolute values around the cosine similarity. Since the cosine similarity is symmetric around $0$, the likelihood that its absolute value exceeds $1 - \varepsilon$ is twice the likelihood that its value exceeds $1- \varepsilon$, concluding the proof.

We note that this is actually an extremely optimistic bound since we have not taken into account the fact that the maximum of the cosine similarity is 1.
\end{proof}

The above proposition represents the likelihood that \emph{one} pair of embeddings has large angle between them. It is therefore \emph{exponentially} unlikely for every pair of embeddings in a dataset to have angle close to $\pi$, as we would require Proposition \ref{prop:unlikely_opp_halves} to hold across every pair of embeddings. Thus, the opposite-halves effect is exceedingly unlikely to occur.

\begin{table}
    \centering
    \quad
    \parbox{.47\linewidth}{
        \begin{tabular}{lrcc}
        \toprule
        Model & Dataset \quad\quad & \makecell{Effect Rate\\Epoch 1} & \makecell{Effect Rate\\Epoch 16} \\
        \midrule
        \multirow{2}{*}{SimCLR} & Imagenet-100 & 2\% & 0\%  \\
        & Cifar-100 & 11\% & 1\% \\
        \cmidrule{1-4}
        \multirow{2}{*}{SimSiam} & Imagenet-100 & 26\% & 1\% \\
        & Cifar-100 & 21\% & 0\% \\
        \cmidrule{1-4}
        \multirow{2}{*}{BYOL} & Imagenet-100 & 28\% & 1\% \\
        & Cifar-100 & 20\% & 0\% \\
        \bottomrule
        \end{tabular}
        \captionof{table}{The rate at which embeddings are on opposite sides of the latent space (angle between a positive pair is greater than $\pi / 2$) for various datasets and SSL models.}
        \label{tbl:opposite_halves_effect}
    }
    \hfill
    \parbox{.38\linewidth}{
        \begin{tabular}{cc ccc}
        \toprule
        \multirow{2}{*}{Epoch} & & \multicolumn{3}{c}{Batch Size}\\
        & & 256 & 512 & 1024 \\
        \cmidrule{3-5}
        \multirow{2}{*}{100} & Default & 46.1 & 41.2 & 32.6 \\
        & Cut ($c=9$) & 43.1 & 46.5 & 44.3 \\
        \cmidrule{2-5}
        \multirow{2}{*}{500} & Default & 59.1 & 60.4 & 61.3\\
        & Cut ($c=9$) & 59.4 & 58.9 & 61.5 \\
        \bottomrule
        \end{tabular}
        \captionof{table}{$k$-nn accuracies for SimSiam trained with various batch sizes. We performed training for both the default and cut-initialized variants and reported $k$-nn accuracies at 100 and 500 epochs.}
        \label{tbl:cut_batch_size}
    }
\end{table}

In accordance with this, Table~\ref{tbl:opposite_halves_effect} shows that, after one epoch of training, embeddings have angle greater than $\pi/2$ at a rate of around $5\%$ and $25\%$ for SimCLR and SimSiam/BYOL, respectively. So even if the `strongest' variant of the opposite-halves effect is not occurring, a weaker one may still be. However, very early into training (epoch 16), every method has a rate of effectively 0 for the opposite-halves effect. Furthermore, the rates in Table~\ref{tbl:opposite_half_effect} measure how often $\phi_{ij} > \frac{\pi}{2}$. This is the absolute weakest version of the opposite-halves effect. Thus, while some weak variant of the opposite-halves effect may occur at the beginning of training, it does not have a strong impact on the convergence dynamics and, in either case, disappears quite quickly.

\subsection{Weight Decay}
\label{app:weight_decay}

We evaluate the effect of weight decay in the imbalanced setting in \ref{fig:weight_decay_imbalanced}, which is an analog of Figure \ref{fig:weight_decay_ablation} for the imbalanced Cifar-10 dataset detailed in Section \ref{sec:convergence}. We again see that using weight decay controls for the embedding norms and improves the convergence of both models. In correspondence with the other results on imbalanced training, we find that stronger control over the embedding norms leads to improved convergence: the high weight decay value does not perform as poorly on SimCLR as in Figure \ref{fig:weight_decay_ablation} and, on SimSiam, outperforms the other weight decay options.

\begin{figure}
    \centering
    \begin{tikzpicture}
    \node () at (0, 0) {\includegraphics[width=0.4\linewidth]{Images/wd_sweep_imbalanced.png}};

    \draw[ballblue, line width=0.07cm] (-4, -3.8) -- (-3.4, -3.8);
    \draw[azure, line width=0.07cm] (-1.3, -3.8) -- (-0.7, -3.8);
    \draw[darkblue, line width=0.07cm] (2, -3.8) -- (2.6, -3.8);

    \node () at (-1.15, -3.33) {\small \textcolor{darkgray}{Train Epoch}};
    \node () at (1.95, -3.33) {\small \textcolor{darkgray}{Train Epoch}};

    \node[inner sep=0pt] () at (-2.5, -3.82) {\textcolor{darkgray}{\scriptsize No weight decay}};
    \node[inner sep=0pt] () at (0.48, -3.82) {\textcolor{darkgray}{\scriptsize Standard weight decay}};
    \node[inner sep=0pt] () at (3.57, -3.82) {\textcolor{darkgray}{\scriptsize High weight decay}};
        
        
    \end{tikzpicture}
    \caption{An analog to Figure \ref{fig:weight_decay_ablation} performed on the exponentially imbalanced Cifar-10 dataset. Weight decays are [$0$, $1$e-$5$, $5$e-$2$] for SimCLR and [$0$, $5$e-$4$, $5$e-$2$] for SimSiam. We plot the effective learning rate in the bottom row, calculated in accordance with Section \ref{sec:convergence}.}
    \label{fig:weight_decay_imbalanced}
\end{figure}

\subsection{Cut-Initialization}
\label{app:cut_init}
We plot the effect of the cut constant on the embedding norms and accuracies over training in Figure~\ref{fig:cut_experiments}. To make the effect more apparent, we use weight-decay $\lambda=5e-4$ in all models. We see that dividing the network's weights by $c>1$ leads to immediate convergence improvements in all models. Furthermore, this effect degrades gracefully: as $c > 1$ becomes $c < 1$, the embeddings stay large for longer and, as a result, the convergence is slower. We also see that cut-initialization has a more pronounced effect in attraction-only models -- a trend that remains consistent throughout the experiments.

We also show the relationship between cut-initialization and the network's batch size on SimSiam in Table \ref{tbl:cut_batch_size}. Consistent with the literature, we see that training with large batches provides improvements to training accuracy. However, we note that larger batch sizes also significantly slow down convergence. However, cut-initialization seems to counteract this and accelerate convergence accordingly. Thus, training with cut-initialization and large batches seems to be the most effective method for SSL training (at least in the non-contrastive setting).

\begin{figure}[t!]
    \centering
    \includegraphics[width=0.95\textwidth]{Images/init_experiments.png}
    \caption{The effect of cut-initialization on Cifar10 SSL representations. $x$-axis and embedding norm's $y$-axis are log-scale. $\lambda=5$e$-4$ in all experiments.}
    \label{fig:cut_experiments}
\end{figure}

\section{More details on gradient scaling layer}
\label{app:grad_scaling}

An implementation of our GradScale layer can be found in Listing \ref{alg:grad_scaling_use}.
We note that this layer is purely a PyTorch optimization trick and does not amount to implicitly choosing a different loss function:

\begin{restatable}{proposition}{nopotential}
    \label{prop:no_potential}
    Let $t\in\mathbb{R}^n$ be a unit vector, $p: \mathbb{R}^n\backslash \{0\} \to [-1, 1], z\mapsto t^\top z/\|z\|$ the cosine similarity with respect to $t$, $\alpha \in \mathbb{R}$, and $\sigma: \mathbb{R}^n \to  \mathbb{R}, z\mapsto \|z\|^\alpha$. Then the vector field $\sigma\nabla p$ has a potential $q$, i.e., $\nabla q = \sigma \nabla p$, only for $\alpha=0$.
\end{restatable}

\begin{proof}
    Suppose $\sigma \nabla p$ has potential. Consider two paths with segments $s_1, s_2$ and $s_3, s_4$ going $t \to 2t \to -2t$ and $t \to -t \to -2t$, where the segments $s_1, s_4$ scaling $\pm t \to \pm2t$ are straight lines and the other segments $s_2, s_3$ follow great circles on $S^{n-1}$. By Proposition~\ref{prop:cos_sim_grads}, we know that $\nabla p(z)=0$ for $z\in \mathbb{R}_{\neq 0}\cdot t$. So $\sigma \nabla p$ is zero on $s_1$ and $s_4$. Moreover, we have
    \begin{align}
        \int_{s_2} \sigma \nabla p \,dz &= \int_{s_2} \|z\|^\alpha \nabla p \,dz
        = \int_{s_2} 2^\alpha \nabla p \,dz 
        = 2^\alpha \int_{s_2} \nabla p \,dz 
        = 2^\alpha \big(p(2t) - p(-2t)\big) = 2^{\alpha+1}
    \end{align}
    and similarly 
    \begin{align}
        \int_{s_3} \sigma \nabla p dz = 1^\alpha \cdot 2 = 2.
    \end{align}
    Since we assume the existence of a potential, we can use path independence to conclude 
    \begin{align}
        2^{\alpha+1} &= \int_{s_2} \sigma \nabla p \,dz 
        = \int_{s_1, s_2} \sigma \nabla p \,dz 
        = \int_{s_3, s_4} \sigma \nabla p \,dz 
        = \int_{s_3} \sigma \nabla p \,dz 
        = 2.
    \end{align}
    Thus, $\alpha=0$ and $\sigma$ does not perform any scaling.
\end{proof}




\begin{figure}
    \begin{lstlisting}[caption={PyTorch code for gradient scaling layer}, label={alg:grad_scaling}]
class scale_grad_by_norm(torch.autograd.Function):
    @staticmethod
    def forward(ctx, z, power=0):
        ctx.save_for_backward(z)
        ctx.power = power
        return z
    @staticmethod
    def backward(ctx, grad_output):
        z = ctx.saved_tensors[0]
        power = ctx.power
        norm = torch.linalg.vector_norm(z, dim=-1, keepdim=True)
        return grad_output * norm**power, None
\end{lstlisting}
\end{figure}

\begin{algorithm}[tb]
   \caption{Pytorch-like pseudo-code using the gradient scaling layer}
   \label{alg:grad_scaling_use}
\begin{algorithmic}
   \STATE {\bfseries Input:} Encoder network $model$, gradient scaling power $p$
   \STATE $z = model(batch)$
   \STATE $z = grad\_scaling\_layer.apply(z, p)$
   \STATE $sim = (\frac{z}{\|z\|})^T \frac{z}{\|z\|}$
   \STATE $loss = InfoNCE(sim)$
   \STATE $loss.backward()$
\end{algorithmic}
\end{algorithm}


\section{Additional figures}
We provide a bar plot analogous to Figure \ref{fig:in_out_violin} in Figure \ref{fig:in_out_distribution_norms}.

\begin{figure}
    \centering
    \begin{tikzpicture}   
        \node[inner sep=0pt] (image) at (0,0) {\includegraphics[width=\textwidth]{Images/Confidence/per_class_norms.pdf}};
    \end{tikzpicture}
    \caption{Bar plot which is analogous to Figure \ref{fig:in_out_violin} showing embedding magnitudes on each dataset split as a function of which dataset the model was trained on. All values are normalized by training set's mean embedding magnitude. Normalized means are represented by black bars. We use the same data augmentations for the train and test sets for consistency.}
    \label{fig:in_out_distribution_norms}
\end{figure}

We also show each Cifar-10 class's 10 highest and 10 lowest embedding-norm samples in Figure \ref{fig:cifar_norms}. These are obtained after training default SimCLR on Cifar-10 for 512 epochs. We see that the high-norm class representatives are prototypical examples of the class while the low-norm representatives are obscure and qualitatively difficult to identify. This property was originally shown by \citet{embed_norm_confidence_2}.

\begin{figure}
    \centering
    \includegraphics[width=0.48\linewidth]{Images/high_norm.png}
    \quad
    \includegraphics[width=0.48\linewidth]{Images/low_norm.png}
    \caption{\emph{Left}: highest-norm representatives (top 10) per class. \emph{Right}: lowest-norm representatives (bottom 10) per class. All from default SimCLR trained on Cifar-10.}
    \label{fig:cifar_norms}
\end{figure}


\end{document}


% This document was modified from the file originally made available by
% Pat Langley and Andrea Danyluk for ICML-2K. This version was created
% by Iain Murray in 2018, and modified by Alexandre Bouchard in
% 2019 and 2021 and by Csaba Szepesvari, Gang Niu and Sivan Sabato in 2022.
% Modified again in 2023 and 2024 by Sivan Sabato and Jonathan Scarlett.
% Previous contributors include Dan Roy, Lise Getoor and Tobias
% Scheffer, which was slightly modified from the 2010 version by
% Thorsten Joachims & Johannes Fuernkranz, slightly modified from the
% 2009 version by Kiri Wagstaff and Sam Roweis's 2008 version, which is
% slightly modified from Prasad Tadepalli's 2007 version which is a
% lightly changed version of the previous year's version by Andrew
% Moore, which was in turn edited from those of Kristian Kersting and
% Codrina Lauth. Alex Smola contributed to the algorithmic style files.
