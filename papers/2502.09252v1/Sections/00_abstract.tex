Self-supervised learning (SSL) allows training data representations without a supervised signal and has become an important paradigm in machine learning. Most SSL methods employ the cosine similarity between embedding vectors and hence effectively embed data on a hypersphere. While this seemingly implies that embedding norms cannot play any role in SSL, a few recent works have suggested that embedding norms have properties related to network convergence and confidence. In this paper, we resolve this apparent contradiction and systematically establish the embedding norm's role in SSL training. Using theoretical analysis, simulations, and experiments, we show that embedding norms (i) govern SSL convergence rates and (ii) encode network confidence, with smaller norms corresponding to unexpected samples. 
Additionally, we show that manipulating embedding norms can have large effects on convergence speed.
Our findings demonstrate that SSL embedding norms are integral to understanding and optimizing network behavior.