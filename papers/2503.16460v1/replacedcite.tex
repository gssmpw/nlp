\section{Related Works}
Researchers have begun exploring the use of LLMs in educational applications. The existing literature shows that LLMs can generate worked examples and guide structured problem solving. For example, WorkedGen ____ uses prompt chaining and one-shot learning to produce interactive programming examples. Although user studies indicate that 77\% of students found WorkedGen helpful, such self-reported feedback does not necessarily confirm improved learning outcomes. Similarly, Jamplate ____ harnesses AI-powered templates for idea generation, providing reflection-based scaffolding, but noting a tendency toward reduced critical thinking among students.

Although these studies highlight the potential of LLMs to create structured examples and facilitate reflective engagement, researchers must develop a consistent, stepwise evaluation framework for algebraic or multi-step reasoning tasks. Existing benchmarks, such as GSM8K ____, assess the accuracy of the final answer rather than examining the detailed intermediate steps or the iterative feedback necessary for model-tracing ____ in a typical tutoring context. As a result, there is still a need for a more systematic methodology that tests how effectively LLMs handle multi-step problems and adapt to the pedagogical requirements of a tutoring environment.

Another growing area of research investigates the use of LLMs for tutoring in various domains for non-fluent English speakers. For example, a comparative study of models such as GPT-4, Llama-2-ko-DPO-13B, and eT5-chat reveals trade-offs between individualization and correctness ____. Smaller models provided more personalized interactions, while GPT-4 exhibited greater correctness but less personalized assistance. Tutoring is an immensely personal activity, and both correctness and individualization are needed. These studies demonstrate the need for more investigation into LLM shortcomings in stepwise instruction, as well as how to better integrate LLM into existing intelligent tutoring platforms. Moreover, current studies often prioritize correctness of the final answer, overlooking the quality of intermediate steps that are crucial for meaningful learning ____. For example, in mathematics education, breaking problems down into their steps ensures that students grasp foundational concepts rather than simply arriving at the correct solution. 

The work in this paper aims to fill this gap by introducing a novel method that evaluates LLM performance on a wide range of math questions from college algebra, generated from the Apprentice Tutors platform ____. This platform was designed as a web-based intelligent tutoring platform to support personalized learning in mathematics. The platform supports more than ten tutors including topics like radicals, factoring polynomials, and solving logarithmic equations.

% Additionally, participants in this study manually interacted with LLM's to evaluate their adaptability and pedagogical utility. Importantly, this study goes beyond the correctness by evaluating the quality of step-by-step LLM-generated tutoring guidance, providing an understanding of whether LLMs are suitable for educational training. These contributions address critical literature gaps and pave the way for a more effective integration of LLM into pedagogical technologies.

% REMOVING SECTION TO SAVE SPACE
%