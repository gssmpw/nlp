% !TeX root = ../main.tex

\section{Background}
\label{sec:background}

\subsection{Attribution Tags}

% General intro
Attribution tags link pseudo-anonymous blockchain objects, such as addresses or transactions, to real-world actors or events. They provide additional context, such as the name of a service controlling an address (e.g., btc-e), some form of categorization (e.g., exchange), and any other information that might be useful in forensic investigations. Figure~\ref{fig:tag-example} illustrates an example in which two distinct attribution tags originating from different sources reference the same cryptoasset address \texttt{0x123} and describe the same real-world actor, a well-known cryptoasset exchange. One can observe that, despite describing the same entity, the attribution tag data records the name (\texttt{btc-e} vs. \texttt{btc-e.com}) and categorize the exchange differently (\texttt{Exchange} vs. \texttt{Service}).

\begin{figure}
    \centering
    \includegraphics[width=1\textwidth]{figures/attribution.pdf}
    \caption{\textbf{Attribution Tag Example}. Two attribution tags referencing the same cryptoasset address \emph{0x123} owned by the real-world entity \emph{BTC-e}.}
    \label{fig:tag-example}
\end{figure}

% Data quality issues
Attribution tag data quality issues can occur at various levels~\cite{Haslhofer2010}: technical heterogeneities, such as different data formats, can impede uniform processing; syntactic heterogeneities, like the use of different encoding schemes, can hinder uniform interpretation; and semantic heterogeneities, such as the use of different names to denote the same real-world concept (synonyms, homonyms, hypernyms, etc.), can lead to inconsistent interpretations. In this paper, we primarily focus on resolving semantic interoperability issues, assuming that technical and syntactic issues can be addressed using common data preprocessing procedures.

% About knowledge graphs in general
Introducing knowledge graphs to data management environments has become a common strategy to deal with semantic interoperability issues. A knowledge graph defines nodes representing real-world entities of interest and semantic relationships between these entities~\cite{Hogan2021}. An example is eBay's product knowledge graph, which allows them to identify if two sellers sell the same products, or if the products are related otherwise. 

\begin{figure}
    \centering
    \includegraphics[width=0.9\textwidth]{figures/kg.pdf}
    \caption{\textbf{Linking an Attribution Tag to the Knowledge Graph}. Attribution tag instances are linked to concepts defined in the knowledge graph.}
    \label{fig:knowledge-graph}
\end{figure}

% Linking data records to knowledge graphs
The process of linking text mentions to entities in a knowledge graph is called \emph{entity linking} \cite{Shen2015}. This process can be further broken down into: 1) generating a subset of entities (\emph{candidates}) that are most likely to match, and 2) selecting which, if any, candidate matches to the text mention. Candidate set generation can be achieved by excluding implausible (\emph{filtering}) and clustering similar (\emph{blocking}) entities \cite{Papadakis2020}. The selection of a matching candidate is typically performed using a decision function learned through machine learning. In this paper, the term \emph{entity linking} is used to describe the process of linking tags to a knowledge graph; while our solution also encompasses techniques that stem from the closely related problem of linking database records (record linkage, entity matching), for the sake of simplicity, we will consistently refer to it as entity linking.

\subsection{Related Work}

% General linking
Entity linking approaches where tuples are linked to knowledge base entities have utilized look-up methods \cite{Ritze2015}, embedding comparisons \cite{Deng2020}, and hybrid approaches \cite{Efthymiou2017}. Record-based entity linking has seen advancements using classical machine learning \cite{Konda2016}, deep neural networks \cite{Mudgal2018}, and pre-trained language models \cite{Li2020}. In \cite{Tu2023}, a mixture-of-experts approach was proposed, utilizing the training results from various data integration matching tasks. For blocking, many state-of-the-art approaches use deep learning \cite{Li2020, Thirumuruganathan2021, Wang2023, Brinkmann2024}. A simple tf-idf based blocker can achieve competitive results without training and labeled data as shown in \cite{Paulsen2023}.

% LLM-based linking
More recently, researchers started to examine LLMs on data integration tasks. Studies have evaluated and proposed various models, including GPT-3 \cite{Narayan2022}, Jellyfish variants \cite{Zhang2024}, and others \cite{Peeters2024}, on tasks such as entity linkage, data imputation, and error detection. Different matching strategies for LLMs in detail have also been explored in \cite{Peeters2024, Wang2024}.

% Our contribution
Within the specific domain of cryptoasset investigations, \cite{Gomez2022} described a method for resolving conflicting attribution tags. They use the edit distance to harmonize strings referring to the same entity. The application of LLMs in the context of cryptoassets and blockchains has been explored for multiple tasks, such as detecting anomalous Ethereum transactions \cite{gai_blockchain_2023}, auditing smart contracts \cite{david_you_2023}, and identifying discrepancies between smart contract bytecode and project documentation \cite{gan_defialigner_2024}.

% Our contribution
In this paper, we go beyond this approach, by linking attribution tag datasets to well-defined knowledge graph concepts using an LLM-based approach. 