% !TeX root = ../main.tex

\section{Data}
\label{sec:dataset}

In this section, we present the data that informed the design and implementation of our approach and was used throughout our experiments.

\subsection{GraphSense TagPacks}
\label{subsec:data_gs}

% Attribution tags
GraphSense is an open-source cryptoasset analytics platform that provides a curated collection of over 500,000 publicly available attribution tags. A TagPack is a data structure used to package and share attribution tags~\footnote{https://github.com/graphsense/graphsense-tagpacks}. Each tag corresponds to a blockhain address and includes a label, several optional fields, and categorization information. The categories are drawn from a subset of the \emph{INTERPOL Dark Web and Virtual Assets Taxonomy (DWVA)}\footnote{https://misp-galaxy.org/interpol-dwva/}, a community-driven effort to define common forms of abuse and entities representing real-world actors and services within the broader Darknet and Cryptoasset ecosystems.

% Knowledge graph
In addition, GraphSense provides a curated list of 2,862 actors, each representing a well-defined real-world entity within the cryptoasset ecosystem. These actors encompass a wide range of roles and types in the industry such as centralized exchanges (e.g., Binance), decentralized finance platforms (e.g., Aave), and mixing services (e.g., Tornado Cash). Each actor is assigned a unique ID, a label, one or more categories from the DWVA taxonomy, and optional fields such as a URL or jurisdiction. This enables GraphSense to partially implement a knowledge graph, offering explicit links between entities, which we use as a ground-truth dataset. In total, 378,550 attribution tags contain such actor links.

% Pre-processing
Since many attribution tags share the same labels and actor links and we are only interested in unique records, we filtered out duplicates, leaving 2,570 unique linked attribution tags. These were split into training, validation, and test datasets in a 1:30:69 ratio. The training set was used to create few-shot examples, while the validation set was employed in experiments 1 and 2 to optimize the individual components. The test set was exclusively used in experiment 3 to evaluate our approach end-to-end.

\subsection{WatchYourBack Attribution Tags}
\label{subsec:data_wyb}

We utilize the dataset published in \cite{Gomez2022}. Structurally, these attribution tags are similar to the GraphSense TagPack, as they also include categories and subcategories. However, their taxonomy is not harmonized with the GraphSense Actor Taxonomy.

The dataset consists of tags aggregated from various sources, including GraphSense and other shared datasets. To avoid duplicating tags, we filter out any that are linked to addresses already present in the GraphSense TagPack database. From this filtered set, we manually selected and annotated 126 records, 67 of which contain an actor link.

\subsection{DeFi Rekt Database}
\label{subsec:data_defirekt}

The DeFi Rekt database \cite{Defirekt2024} contains over 3,500 events related to crimes involving cryptoassets. Many of these events involve actors within the ecosystem. Each event includes a title and optional fields such as the date of the event, and funds involved. For our experiments, we randomly sampled 100 records from events where the loss of funds exceeded 100,000 USD. We manually annotated these sampled events with actor links. Out of the 100 sampled records, only 32 contained an actor link.

\subsection{Dataset summary}
\label{subsec:data_summary}

Table~\ref{tab:datasets} summarizes the data we used for design and implementation of our approach and the experiments we conducted. The total number of distinct actors is lower than the sum of individual datasets due to overlapping actors between datasets.

\begin{table}
    \centering
    \caption{Overview of the three attribution tag datasets used in this study.}
    \begin{table}[h] \footnotesize  \centering\resizebox{0.48\textwidth}{!}{\begin{tabular}{c|l|c|c|c}
\toprule
\textbf{Task} & \textbf{Dataset} & \textbf{N-shot} & \multirowcell{\textbf{Train texts} \\ \textbf{for STMD}} & \multirowcell{\textbf{Evaluation} \\ \textbf{texts}} \\
\midrule
\multirow{3}{*}{\multirowcell{Text \\ Summarization}} & CNN/DailyMail & 0 & 2,000 & 2,000 \\
& XSum & 0 & 2,000 & 2,000 \\
& SamSum & 0 & 2,000 & 819 \\
\midrule
\multirow{4}{*}{\multirowcell{QA \\ Long answer}} & PubMedQA & 0 & 2,000 & 2,000 \\
& MedQUAD & 5 & 2,000 & 2,000 \\
& TruthfulQA & 5 & 408 & 409 \\
& GSM8k & 5 & 2,000 & 1,319 \\
\midrule
\multirow{4}{*}{\multirowcell{QA \\ Short answer}} & SciQ & 0 & 5,000 & 1,000 \\
& CoQA & \multirowcell{all preceding \\ questions} & 5,000 & 2,000 \\
& TriviaQA & 5 & 5,000 & 2,000 \\
\midrule
\multirow{1}{*}{\multirowcell{MCQA}} & MMLU & 5 & 5,000 & 2,000 \\
\bottomrule
\end{tabular}
}\caption{\label{tab:dataset_stat} The statistics of the datasets used for evaluation.}
\end{table}
    \label{tab:datasets}
\end{table}

