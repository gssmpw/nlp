
\clearpage
\begin{appendix}
\renewcommand{\thesection}{\appendixname~\arabic{section}}
%
%
\section{}  \label{appendix:cg}
A visual illustration of the candidate set generation performances of the different blocking and filtering techniques can be found in Figure~\ref{fig:cg_results}. We can see that the recall gain from $k=1$ to $k=5$ is significant, while subsequent increases in candidate set size have only a marginal impact.

\begin{figure}
    \centering
    \includegraphics[width=\columnwidth]{figures/cg_results.pdf}
    \caption{\textbf{Choosing Candidate Set Size}. Recall of candidate set generation for different candidate set sizes ($k$). The performance gain for all methods visibly slows down for $k > 5$.}
    \label{fig:cg_results}
\end{figure}
%
%
\section{} \label{appendix:el}
A comprehensive overview of all candidate selector model results across each template can be found in Figure~\ref{fig:cs_result_full}. We can see that the zero-shot performance of the models varies significantly across templates; however, introducing five examples reduces this variance considerably. Interestingly, while GPT-3.5 is ineffective without examples, with five examples it achieves the second-highest F1-score. Also notable is that the zero-shot performance for templates 4, 8, and 9 is lowest, signaling that the output format reminder (OUT) is crucial for good results without examples.

\begin{figure}
    \centering
    \includegraphics[width=\textwidth]{figures/cs_result.pdf}
    \caption{\emph{Candidate Selection Experiment:} Model performances ($\mathbf{F1}$) across each template using no (\emph{Zero-Shot}) and five (\emph{Five-Shot}) examples in the prompt.}
    \label{fig:cs_result_full}
\end{figure}
%
%
\section{} \label{appendix:e2e}

In our experiments, we treat entity linking as a multi-classification problem and measure the performance with the macro F1 score and the accuracy, see Table~\ref{tab:e2e_performance_full}. For the macro F1-score, this means that we calculate each entity's individual F1 score and average them. Consequently, any prediction that is different from the ground truth is treated equally as an error. In a practical setting, we argue that errors predicting no entity matches are less problematic than errors linking the attribution tag to the wrong entity.

\begin{table}
    \centering
    \caption{Full overview of end-to-end entity linking performance including the results of all models the three datasets}
    \begin{tabular*}{\textwidth}{@{\extracolsep{\fill}}lccccccc}
\toprule
\multirow{2}{*}{\textbf{Model}} & \multicolumn{2}{c}{\textbf{GraphSense}} & \multicolumn{2}{c}{\textbf{WatchYourBack}} & \multicolumn{2}{c}{\textbf{DeFi Rekt}} \\
\cmidrule(lr){2-3} \cmidrule(lr){4-5} \cmidrule(lr){6-7}
 & \textbf{F1} & \textbf{Acc.} & \textbf{F1} & \textbf{Acc}. & \textbf{F1} & \textbf{Acc.} \\
\midrule
$\text{BM25}_3$             & 0.718 & 0.789 & 0.495 & 0.516 & 0.393 & 0.510 \\
UnicornPlus                 & 0.667 & 0.445 & 0.558 & 0.651 & 0.352 & 0.670 \\
UnicornPlusFT               & 0.783 & 0.873 & 0.542 & 0.603 & 0.419 & 0.610 \\
\\
GPT4o                       & \textbf{0.853} & \textbf{0.927} & \textbf{0.801} & \textbf{0.873} & \textbf{0.793} & \textbf{0.930} \\
GPT3.5                      & 0.810 & 0.881 & \underline{0.756} & \underline{0.825} & \underline{0.691} & \underline{0.890} \\
%\\
Jellyfish 7B                & 0.799 & 0.914 & 0.634 & 0.746 & 0.432 & 0.640 \\
Jellyfish 13B               & 0.716 & 0.842 & 0.595 & 0.730 & 0.478 & 0.720 \\
Llama 3 8B                  & 0.785 & 0.917 & 0.618 & 0.683 & 0.440 & 0.680 \\
Llama 3 8B-Inst             & 0.814 & \underline{0.921} & 0.625 & 0.746 & 0.516 & 0.710 \\
Mistral 7B                  & 0.745 & 0.887 & 0.467 & 0.452 & 0.327 & 0.330 \\
Mistral 7B-Inst             & \underline{0.821} & 0.918 & 0.692 & 0.786 & 0.547 & 0.770 \\
\bottomrule
\end{tabular*}
    \label{tab:e2e_performance_full}
\end{table}

Figure \ref{fig:error} shows us the error composition of the different models across all three datasets. We distinguish between the \emph{Missed Entity} error from a wrong no-match prediction and the \emph{Wrong Entity} error with a wrong entity predicted. We can see that GPT-4o not only has the best performance in terms of accuracy and macro F1-score but also has less than 1\% of wrong entities predicted, with the majority of the errors coming from missed entities. In contrast, the local LLMs all have more \emph{Wrong Entity} than \emph{Missed Entity} errors. Our best-performing local LLM, Mistral 7B-Instruct, has, despite better overall accuracy, more \emph{Wrong Entity} errors than GPT-3.5.

\begin{figure}
    \centering
    \includegraphics[width=\textwidth]{figures/error.pdf}
    \caption{\textbf{Error analysis.} The number of missed and wrongly predicted actor links for each model in the end-to-end entity linking experiment. The experiment contained 2000 samples from three different datasets.}
    \label{fig:error}
\end{figure}

In Table~\ref{tab:e2e_wrong_actors}, we identify the three actors most frequently misclassified by each model. The results are similar across different models, with a few exceptions. For example, \texttt{maker} appears in the top three only for UnicornPlus, despite being present in 135 different tags. Notably, none of the models correctly classify any of the 20 tags associated with \texttt{0x} as an actor.

\begin{table}
    \centering
    \caption{Overview of the most frequently misclassified actors for each model}
    \begin{tabular*}{\textwidth}{@{\extracolsep{\fill}}llrr}
\toprule
Model & Actor & \# Miss Classified & \# Total Occurences \\
\midrule
\multirow{3}{*}{$\text{BM25}_{3}$} & aave & 121 & 133\\
                                   & curve & 38 & 128 \\
                                   & binance & 33 & 37\\
\addlinespace[0.5em]                                   
\multirow{3}{*}{UnicornPlus} & aave & 133 & 133 \\
                             & maker & 128 & 135\\
                             & synthetix & 122 & 198\\
\addlinespace[0.5em]
\multirow{3}{*}{UnicornPlusFT} & huobi & 51 & 53\\
                               & aave & 35 & 133\\
                               & 0x & 20 & 20\\
\addlinespace[0.5em]
\multirow{3}{*}{GPT4o}  & 0x & 20 & 20\\
                        & aave & 18 & 133\\
                        & feiprotocol & 10 & 27\\
\addlinespace[0.5em]                        
\multirow{3}{*}{GPT3.5} & synthetix & 30 & 198\\
                        & aave & 22 & 133\\
                        & 0x & 20 & 20\\
\addlinespace[0.5em]                        
\multirow{3}{*}{Jellyfish-7B} & aave & 29 & 133\\
                              & 0x & 20 & 20\\
                              & feiprotocol & 10 & 27\\
\addlinespace[0.5em]
\multirow{3}{*}{Jellyfish-13B} & aave & 52 & 133\\
                               & 0x & 20 & 20\\
                               & synthetix & 19 & 198\\
\addlinespace[0.5em]                               
\multirow{3}{*}{Llama 3 8B} & 0x & 20 & 20\\
                            & aave & 20 & 133\\
                            & feiprotocol & 10 & 27\\
\addlinespace[0.5em]                           
\multirow{3}{*}{Llama 3 8B-Inst } & aave & 24 & 133\\
                                  & 0x & 20 & 20\\
                                  & feiprotocol & 11 & 27\\
\addlinespace[0.5em]                             
\multirow{3}{*}{Mistral 7B}  & aave & 23 & 133\\
                             & 0x & 20 & 20\\
                             & curve & 19 & 128\\
\addlinespace[0.5em]
\multirow{3}{*}{Mistral 7B-Inst} & aave & 21 & 133\\
                                 & 0x & 20 & 20\\
                                 & feiprotocol & 10 & 27\\


\bottomrule
\end{tabular*}
    \label{tab:e2e_wrong_actors}
\end{table}

\end{appendix}


\clearpage