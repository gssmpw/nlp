\section{Related Work}
\label{sec:related-work}

\subsection{OS Fingerprinting}
\label{subsec:soa-os-fingerprinting}

\subsubsection{Traditional OS Fingerprinting}
\label{subsubsec:traditional-os-fingerprinting}

OS fingerprinting initially emerged as an approach based solely on the analysis of TCP/IP header fields, such as Time To Live (TTL), the Don't Fragment (DF) flag, Type of Service (ToS) and TCP Window Size \cite{nmaporg_nmap_nodate, lastovicka_passive_2023}. In this context, Nmap was one of the first tools to be developed and remains one of the most widely used, as it employs an active scanning method that compels the target machine to respond, thereby facilitating the identification of its operating system.

Alternatively, passive fingerprinting techniques were subsequently proposed, relying on similar network traffic characteristics but without provoking a response from the target system. Early implementations of this approach, such as p0f and Siphon, emerged in 2000 \cite{zalewski_p0f_nodate, beddoe_siphon_nodate}. As previously discussed (Section~\ref{subsec:os-fingerprinting}), the fundamental difference between the two methods lies in the manner in which network information is collected, rather than in the analytical techniques applied to the traffic for inference.

Traditional OS fingerprinting methods based on TCP/IP headers continued to evolve. Tools like Ettercap and Satori extended earlier approaches \cite{ornaghi_alberto_ettercap_2001, e_kollmann_satori_2018}, while others like NetSleuth and PRADS saw limited longevity \cite{netgrab_netsleuth_2012, eb_fjellskal_prads_2009}. Research expanded into passive OS fingerprinting in large networks, examining features from network flows, as demonstrated by Vymlátil and Matoušek \cite{vymlatil_detection_2014, matousek_gromovs_2014}. These methods proved effective in dynamic environments such as wireless networks, and newer approaches—such as those by Al-Sherari and Osanaiye \cite{al-shehari_improving_2014, osanaiye_tcpip_2015}—combined traditional methods with ML to achieve better accuracy, particularly in unauthorized OS detection and cloud environments.

Modern approaches have shifted towards analysing application layer protocols, encrypted traffic, and specialised traffic types. Hypertext Transfer Protocol (HTTP) banners and User-Agent strings provide more precise OS identification \cite{shah_http_2003}, while encryption complicates traditional methods. Researchers like Muehlstein and Fan \cite{matousek_gromovs_2014, fan_identify_2019} improved accuracy by incorporating Transport Layer Security (TLS) handshake features, and others like Aksoy \cite{aksoy_operating_2016} explored various protocols using ML to optimise fingerprinting. In specific cases, such as smartphone OS identification and Industrial Control System (ICS) devices, timing analysis and advanced algorithms have been used \cite{gurary_operating_2016, shen_hybrid-augmented_2018}. ML has become essential in overcoming traditional limitations, focusing on processing large datasets and achieving higher accuracy in identifying OS in encrypted traffic \cite{beverly_robust_2004, shamsi_faulds_2021, lastovicka_cybersecurity_2018}. A detailed explanation of the works where ML is applied to the OS fingerprinting field is exposed in \Cref{sec:related-work}.

\subsubsection{Machine Learning-based OS Fingerprinting}
\label{subsubsec:ml-based-os-fingerprinting}

The field of OS fingerprinting has evolved significantly with the advent of ML techniques. In this context, several approaches have been proposed to enhance the accuracy and robustness of OS identification. For instance, Lastovicka's research \cite{lastovicka_cybersecurity_2018} explored various classical ML algorithms like Naive Bayes (NB), Decision Trees (DT), k-Nearest Neighbours (kNN), and Support Vector Machines (SVM) for passive OS fingerprinting. In a later study \cite{lastovicka_usingTLS_2020}, Lastovicka et al. expanded this work by employing TLS handshake features, which improved device identification even in encrypted network environments. Similarly, Fan et al. in \cite{fan_identify_2019} employed Gradient Boosting Decision Trees (GBDT) on features extracted from both TLS and TCP/IP headers, achieving high accuracy on a large dataset. This aligns with previous contributions made by our research team \cite{perez-jove_applying_2021, perez-jove_tool_2023}, where we applied a range of classical ML algorithms like NB, Multilayer Perceptron (MLP), DT, and Random Forest (RF) to the Nmap and p0f databases.

Other studies focused on minimalist data requirements and novel feature extraction. Millar et al. \cite{millar_operating_2020} used RF to classify OS types based on IP affiliation graphs, demonstrating resilience to encrypted traffic. Similarly, Barath et al. \cite{barath_use_2021} employed DT, Expectation-Maximization (EM), NB, and Artificial Neural Networks (ANNs) for passive monitoring, further showcasing the potential of various ML techniques for network data analysis. Shamsi et al. \cite{shamsi_faulds_2021} also used a non-parametric EM estimator to improve OS fingerprinting accuracy in noisy data for large-scale network environments and dealing with distortions.

Several works focused on specialised environments or different network protocols. Salah et al. \cite{salah_desktop_2022} focused on IPv6-based fingerprinting using kNN, DT, SVM, and Gaussian Naive Bayes (GNB), whereas Bub et al. \cite{bub_towards_2022} applied DT to identify aged Android devices in home networks. Hulák et al. \cite{hulak_evaluation_2023} compared the performance of DT, RF, and AdaBoost (AB) in passive OS fingerprinting, emphasising the importance of careful data preparation. In a related domain, Zhang et al. \cite{zhang_operating_2022} integrated Active Learning (AL) with SVM, RF, and NB to optimise OS identification, particularly in environments with dynamic network conditions.

The literature review efforts in this field are limited, with Lastovicka et al. \cite{lastovicka_passive_2023} providing the only comprehensive survey of passive OS fingerprinting methods, detailing the transition from traditional techniques to ML-based approaches. 

\subsubsection{Deep Learning-based OS Fingerprinting}
\label{subsubsec:dl-based-os-fingerprinting}

Even though ML techniques have been explored in OS fingerprinting, as previously outlined, there is little research on the application of DL models to this field. For the best of author's knowledge, only two works employed some DL algorithm to solve this network task. Li et al. \cite{li_passive_2023} proposed a combined sampling method paired with a Convolutional Neural Network (CNN) to improve identification accuracy for underrepresented OS types in imbalanced datasets. Hagos et al. \cite{hagos_machine-advanced_2020, hagos_TCPvariant_2021} introduced the TCP variant as a feature in ML models, exploring a mix of traditional ML algorithms, like SVM, RF, kNN, NB, with DL models like Long Short-Term Memory (LSTM). Finally, a preliminary version of this work was presented in \cite{perez_jove_towards_2024}, where the Transformer architecture was applied to the Nmap database.

\subsection{Transformers in Network Traffic Modelling}
\label{subsec:transformers-network-traffic}

Beyond its success in NLP, the Transformer architecture has proved versatile across domains such as computer vision (e.g., the Vision Transformer (ViT) \cite{han_survey_2023}) and network analysis. Although Transformers have not yet been applied directly to OS fingerprinting, they have been adapted for various networking tasks. For example, NetBERT outperforms BERT on network-specific tasks \cite{louis_netbert_2020}, while adaptations of BERT for DNS analysis and the use of Graph Neural Networks for packet sequences have also been explored \cite{le_norbert_2022}. Direct training on network traffic has yielded promising results too, with the Residual 1-D Image Transformer excelling in malware and DDoS detection \cite{barut_r1dit_2023} and the Flow Transformer enhancing anonymity network classification by capturing temporal–spatial dependencies \cite{zhao_flow_2021}. Furthermore, De la Torre Vico et al. \cite{de_la_torre_vico_exploring_2024} have demonstrated the potential of LLMs in analysing network traces for cybersecurity.

Concurrently, interest in Network Traffic Foundational Models (NT-FMs) inspired by large language models is growing. Early work includes ET-BERT, which leverages contextualised datagram representations for encrypted traffic classification \cite{lin_et-bert_2022}, and Ray’s packet-level traffic prediction model \cite{ray_advancing_2022}. Subsequent studies have investigated model generalisation \cite{dietmuller_new_2022} and foundational applications in networking \cite{le_rethinking_2022}. More recent advances include Zhao et al.’s Yet Another Traffic Classifier (YATC) using a masked autoencoder with multi-level flow representations \cite{zhao_yet_2023}, Guthula et al.’s netFound utilising unlabelled traffic for pre-training \cite{guthula_netfound_2023}, and Wang et al.’s Lens capturing temporal–spatial correlations for anomaly detection \cite{wang_lens_2024}. Additional contributions include TrafficGPT for long traffic sequence modelling \cite{qu_trafficgpt_2024}, a graph-based NT-FM by Langendonck et al. for improved scalability \cite{van_langendonck_towards_2024}, the generative pretrained model NetGPT for traffic understanding and generation \cite{meng_netgpt_2023}, and the comprehensive NetBench dataset for evaluating foundational models on traffic tasks \cite{qian_netbench_2024}.
