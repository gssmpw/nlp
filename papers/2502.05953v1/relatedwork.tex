\section{Related Works}
AR intends to place virtual objects into physical world visuals. To achieve this effect, a software that uses virtual reality combines real world elements with virtual objects on real-time images~\cite{Cawood2008}. Azuma~\cite{azuma1997survey} conducted a poll on his work, and defined three attributes that an ideal AR system should have: combination of reality and virtuality, real-time interactivity and 3D registration.

AR is used on many different fields; to create models and visualize historical buildings in architecture, to visualize human body in medicine, to enhance education materials in engineering, and also in entertainment and games~\cite{azuma1997survey,Broschart2014}. Although this technology seems recent, the first prototype accepted as a VR and AR system was developed by Sutherland~\cite{Sutherland1968}. However, this system was quite expensive and even heavy, needed to be hung on ceiling. Nowadays, AR is developing rapidly thanks to the widespread use of personal computers. In 2009, Christian Doppler Laboratory listed the most important milestones in AR technologies. Even though high-quality applications are being developed, there are not many differences between them and the ones listed~\cite{Doppler}.
%%%%
%%%%Figure 1
%%%%
\begin{figure*}[t]
  \centering
  \includegraphics[width=0.95\linewidth]{artoolkitdiagram}
  \caption{ \label{fig:arsystem}
          A general overview of a marker-based AR system~\cite{ARToolKit}.
           }
%\vspace{-0.5cm}
\end{figure*}