\documentclass[twoside]{article}

\usepackage[accepted]{aistats2025}
\usepackage{enumitem}
\usepackage{pifont}
\usepackage{threeparttable}
\usepackage{multirow}
\usepackage{fdsymbol}
\usepackage{caption}
\usepackage{subcaption}
\usepackage{graphicx}
\usepackage{wrapfig}
\usepackage{fdsymbol}
\usepackage{enumitem}
\usepackage{color}
\usepackage{caption}
\usepackage{subcaption}
\usepackage{hyperref}
\usepackage{url}
\usepackage{soul}
\usepackage{amsthm}
\usepackage{pifont}
\usepackage{array}
\usepackage{algorithm}
\usepackage{algorithmic}
\usepackage{booktabs}

\newtheorem{Theorem}{Theorem}[section]
\newtheorem{Definition}{Definition}[section]
\newtheorem{Proposition}{Proposition}[section]
\newtheorem{Assumption}{Assumption}[section]

\theoremstyle{remark}
\newtheorem*{remark}{Remark}
\usepackage[round]{natbib}
\renewcommand{\bibname}{References}
\renewcommand{\bibsection}{\subsubsection*{\bibname}}

\bibliographystyle{apalike}

\begin{document}

\renewcommand{\thefootnote}{\fnsymbol{footnote}}
\runningauthor{J. Zhang, R. Ding, Q. Fu, B. Huang, Z. Deng, Y. Hua, H. Guan, S. Han, D. Zhang}

\twocolumn[

\aistatstitle{Learning Identifiable Structures Helps Avoid Bias in DNN-based Supervised Causal Learning}

\aistatsauthor{ Jiaru Zhang\footnotemark{} \And Rui Ding\footnotemark{} \And Qiang Fu  }
\aistatsaddress{ Shanghai Jiao Tong University \\\texttt{jiaruzhang@sjtu.edu.cn} \And  Microsoft \\\texttt{juding@microsoft.com} \And  Microsoft\\ \texttt{qifu@microsoft.com} } 
\aistatsauthor{Huang Bojun \And Zizhen Deng\And   Yang Hua }
\aistatsaddress{Sony Research \\ \texttt{bojhuang@gmail.com} \And Peking University \\ \texttt{dengzizhen557@outlook.com}  \And Queen’s University Belfast \\ \texttt{y.hua@qub.ac.uk}} 
\aistatsauthor{ Haibing Guan \And Shi Han \And Dongmei Zhang }
\aistatsaddress{Shanghai Jiao Tong University \\ \texttt{hbguan@sjtu.edu.cn}\And Microsoft\\ \texttt{shihan@microsoft.com} \And Microsoft \\ \texttt{dongmeiz@microsoft.com}} 
]

\footnotetext[1]{The work was done during his internship at Microsoft Research Asia.}
\footnotetext[2]{Corresponding author.}

\renewcommand{\thefootnote}{\arabic{footnote}}
\setcounter{footnote}{0}
\begin{abstract}
Causal discovery is a structured prediction task that aims to predict causal relations among variables based on their data samples.
Supervised Causal Learning (SCL) is an emerging paradigm in this field.
Existing Deep Neural Network (DNN)-based methods commonly adopt the “Node-Edge approach”,
in which the model first computes an embedding vector for each variable-node, then uses these variable-wise representations to concurrently and independently predict for each directed causal-edge.
In this paper, we first show that this architecture has some systematic bias that cannot be mitigated regardless of model size and data size. 
We then propose SiCL, a DNN-based SCL method that predicts a skeleton matrix together with a v-tensor (a third-order tensor representing the v-structures). According to the Markov Equivalence Class (MEC) theory, both the skeleton and the v-structures are \emph{identifiable} causal structures under the canonical MEC setting, so predictions about skeleton and v-structures do not suffer from the identifiability limit in causal discovery, thus SiCL can avoid the systematic bias in Node-Edge architecture, and enable consistent estimators for causal discovery. Moreover, SiCL is also equipped with a specially designed pairwise encoder module with a unidirectional attention layer to model both internal and external relationships of pairs of nodes. Experimental results on both synthetic and real-world benchmarks show that SiCL significantly outperforms other DNN-based SCL approaches.
\end{abstract}

\section{Introduction} \label{sec:int}

Causal discovery seeks to infer causal structures from an observational data sample.
Supervised Causal Learning (SCL) \citep{dai2023ml4c,ke2023learning,ma2022ml4s} is an emerging paradigm in this field. The basic idea is to consider causal discovery as a \emph{structured prediction} task, and to train a prediction model using supervised learning techniques. 
At training time, a training dataset comprising a variety of causal mechanisms and their associated data samples is generated. %, either via synthetic generation, or from a simulator if available \citep{lorchamortized}. 
The prediction model is then trained to take such a data sample as input, and to output predictions about the causal mechanism behind the data sample.
%During the inference stage, the causal structure is identified by simply applying the learned model to the target data.  
Compared to traditional rule-based or unsupervised 
%causal discovery 
methods~\citep{glymour2019review}, the SCL method has demonstrated strong empirical performance~\citep{dai2023ml4c,ma2022ml4s}, as well as robustness against sample size and distribution shift \citep{ke2023learning,lorchamortized}.

Deep Neural Network (DNN)-based SCL employs DNN as the prediction model. It allows end-to-end training, removing the need for manual feature engineering. Additionally, it can handle both continuous and discrete data types effectively, and can learn latent representations.
A specific DNN architecture, first introduced by \cite{lorchamortized}, is particularly popular in recent DNN-based SCL works. The model first transforms the given data sample into a set of node-wise feature vectors, each representing an individual variable (corresponding to a node in the associated causal graph).
Based on these node-wise features, the model then outputs a weighted adjacency matrix $A$, where $A_{ij}\in[0,1]$  is an estimated probability for the directed edge $i \rightarrow j$ (meaning that $i$ is a direct cause of $j$). 
% The final DAG is obtained as a Bernoulli sample of $A$.
Finally, the adjacency matrix of an inferred causal graph $G$ is obtained as a Bernoulli sample of $A$, where each entry $G_{ij} \in \{0,1\}$ is sampled \emph{independently}, following probability $A_{ij}$. 
\textcolor{black}{For convenience, we call} such a model architecture as the ``Node-Edge'' \textcolor{black}{architecture}, as the representation is learned for individual nodes and the probability is estimated and sampled for individual directed edges. 

Despite its popularity and encouraging results~\citep{lorchamortized,Zhu2020Causal,dpdag,varamballydiscovering}, we identify two limitations for the Node-Edge approach:


\textcolor{black}{First, the Node-Edge architecture imposes a fundamental bias in the inferred causal relations. Specifically, given an observational data sample $D$, the existence of a directed causal edge $i \rightarrow j$ may \emph{necessarily} depend on the existence of other edges. But the existing Node-Edge models predict each edge separately and independently, so the probability prediction $A_{ij}$ made by such models is only conditioned on the input sample $D$, not on the sampling result of other entries of $A$,  thereby failing to capture the crucial inter-edge dependency in its probability estimation.
}

As a simple example, a Node-Edge model maintaining the possibility of both $G_1: X\rightarrow T \rightarrow Y$ and $G_2: X\leftarrow T \leftarrow Y$ would necessarily have a non-zero probability to output the edges $X\rightarrow T$ and $T \leftarrow Y$, thus cannot rule out the possibility of $G_3: X\rightarrow T \leftarrow Y$, even though $G_3$ is impossible to be the groundtruth causal graph behind a data sample $D$ compatible with $G_1$ and $G_2$~\citep{verma1990equivalence}. 
Crucially, there is no way to tell $G_1$ from $G_2$ based on observational data in general cases~\citep{andersson1997characterization,meek1995strong}. 
It means that for any Node-Edge model to be sound, it has to maintain the possibility of both $G_1$ and $G_2$ (when observing a data sample compatible with any of them), leading to an inevitable error probability to output the impossible graph $G_3$ on the other hand. 


\textcolor{black}{Second, the Node-Edge architecture does not explicitly represent the features about node pairs, which we argue are essential for observational causal discovery.}
For example, a causal edge $X\rightarrow Y$ can exist only if the node pair $\langle X, Y\rangle$ demonstrates \textit{persistent dependency} \citep{ma2022ml4s,spirtes2000causation}, meaning that $X$ and $Y$ remain statistically dependent regardless of conditioning on any subset of other variables. As another example, for causal DAGs, a sufficient condition to determine the causal direction between a persistently dependent node pair $\langle X, Y\rangle$ is that $X$ and $Y$ exhibits \emph{orientation asymmetry}, meaning that there exists a third variable $Z$ such that $X$ is persistently dependent to $Z$ but $Y$ can become independent to $Z$ conditioned on a variable-set $\mathbf{S}\not\ni X$ (or vice versa). A feature like persistent dependency or orientation asymmetry is, in its nature, a collective property of a node pair, but not of any individual node alone. 

To address these limitations, in this paper, we propose a novel DNN-based SCL approach, called Supervised Identifiable Causal Learning (SiCL). 
\textcolor{black}{The neural network in SiCL does not seek to predict the probabilities of directed edges, but tries to predict a skeleton matrix together with v-tensor, a third-order tensor representing the v-structures.}
According to the Markov Equivalence Class (MEC) theory of causal discovery, skeleton and v-structures are \emph{identifiable} causal structures under the canonical MEC setting (while the directed edges are not), so predictions about skeleton and v-structures do not suffer from the (non-)identifiability limit. 
By leveraging this insight, our theory-inspired DNN architecture completely avoids the systematic bias in edge-prediction models as previously discussed, and enables \emph{consistent} neural-estimators\footnote{Recall that a statistical estimator is \emph{consistent} if it converges to the groundtruth given infinite data.} for causal discovery. 
Moreover, SiCL is also equipped with a specially designed pairwise encoder module with a unidirectional attention layer.
With both node features and node-pair features as the layer input, it can model both internal and external relationships of pairs of nodes.
%SiCL is trained on synthetic data, and it shows significant improvement on both skeleton prediction and orientation tasks compared to other DNN-based approaches. 
Experimental results on both synthetic and real-word benchmarks show that \textcolor{black}{SiCL} can effectively address the two \textcolor{black}{above-mentioned limitations}, and the resulted SiCL solution significantly outperforms other DNN-based SCL approaches with more than 50\% performance improvement in terms of SHD (Structural Hamming Distance) on the real-world Sachs data. The codes are publicly available at \url{https://github.com/microsoft/reliableAI/tree/main/causal-kit/SiCL}.


\section{Background and Related Work}\label{sec:bg}
A Causal Graphical Model is defined by a joint probability distribution $P$ over multiple random variables and a DAG $G$. Each node $X_i$ in $G$ represents a variable in $P$, and a directed edge $X_i \rightarrow X_j$ represents a direct cause-effect relation from $X_i$ to $X_j$.
A causal discovery task generally asks to infer about $G$ from an i.i.d. sample of $P$.

%\subsection{Identifiability} 
However, there is a well-known identifiability limit for causal discovery.
In general, the causal DAG is only identifiable up to an equivalence class. 
Studies of this identifiability limit under a canonical assumption setting have led to the well-established MEC theory \citep{frydenberg1990chain,verma1990equivalence}.
We call a causal feature, \textit{MEC-identifiable}, if the value of this feature is invariant among the equivalence class under the canonical MEC assumption setting. 
It is known that such MEC-identifiable features include the skeleton and the set of v-structures, which we briefly present in the following.
% that the causal DAG $G$ is, in general, only identifiable up to its Markov equivalence class (MEC).
% Studies of this identifiability limit have led to a well-established theory \citep{frydenberg1990chain,verma1990equivalence}, which we briefly present in the following.


A \textit{skeleton} $E$ defined over the data distribution $P$ is an undirected graph where an edge exists between $X_i$ and $X_j$ if and only if $X_i$ and $X_j$ are always dependent in $P$, i.e., $\forall Z \subseteq\left\{X_1, X_2, \cdots, X_d\right\} \backslash \left\{X_i, X_j \right\}$, we have $X_i \nperp X_j | Z$.
Under mild assumptions (such as that $P$ is Markovian and faithful to the DAG $G$; see details in Appendix Sec. \ref{sec:da}), 
the skeleton is the same as the corresponding undirected graph of the DAG $G$ \citep{spirtes2000causation}. 
% 
A triple of variables $\langle X, T, Y \rangle$ is an \textit{Unshielded Triple (UT)} if $X$ and $Y$ are both adjacent to $T$ but not adjacent to each other in (the skeleton of) $G$.
It becomes a \textit{v-structure} denoted as $X \rightarrow T \leftarrow Y$ if the directions of the edges are from $X$ and $Y$ to $T$.

% 
Two graphs are Markov equivalent if and only if they have the same skeleton and v-structures. 
The \textit{Markov equivalence class (MEC)} can be represented by a \textit{Completed Partially Directed Acyclic Graph (CPDAG)} consisting of both directed and undirected edges. We use $CPDAG(G)$ to denote the CPDAG derived from $G$.
% \end{Definition}
According to the theorem of Markov completeness \citep{meek1995strong}, 
%assuming causal sufficiency and $P$ is Markovian and faithful w.r.t. $G$, 
we can only identify a causal graph up to its MEC, i.e., the CPDAG, unless additional assumptions are made (see the remark below).
%, for disc)ete data or linear Gaussian data. 
This means that each (un)directed edge in $CPDAG(G)$ indicates a (non)identifiable causal relation.

\textbf{Remark:} The MEC-based identifiability theory is applicable in the general-case setting, when we take into account all possible distributions $P$. 
It is known this identifiability limit could be broken (i.e., an undirected edge in the CPDAG could be oriented) \textit{if} we assume that the data follows some special class of distributions, e.g., linear non-Gaussians, additive noise models, post-nonlinear or location-scale models~\citep{peters2014causal,shimizu2011directlingam,zhang2009identifiability,immer2023identifiability}. 
These assumptions are sometimes hard to verify in practice, so this paper considers the general-case setting.
More discussions on the identifiability and causal assumptions are provided in Appendix Sec. \ref{sec:dica}.


\subsection{Related Work}
In traditional methods of causal discovery, constraint-based methods are mostly related to our work.
%are categorized into constraint-based, score-based and continuous optimization. 
They aim to identify the DAG that is consistent with inter-variable conditional independence constraints. 
These methods first identify the skeleton and then conduct orientation based on v-structure identification \citep{yu2016review}. 
The output is a CPDAG which represents the MEC.
Notable algorithms in this category include PC \citep{spirtes2000causation}, along with variations such as Conservative-PC \citep{ramsey2012adjacency}, PC-stable \citep{colombo2014order}, and Parallel-PC \citep{le2016fast}. 
Compared to constraint-based methods, both our approach and theirs are founded upon the principles of MEC theory for estimating skeleton and v-structures. However, whereas traditional methods rely on symbolic reasoning based on explicit constraints, we employ DNNs to capture the essential causal information intricately linked with these constraints.

Score-based methods aim to find an optimal DAG according to a predefined score function, subject to combinatorial constraints. 
These methods employ specific optimization procedures such as forward-backward search GES \citep{chickering2002optimal}, hill-climbing \citep{koller2009probabilistic}, and integer programming \citep{cussens2011bayesian}.
Continuous optimization methods transform the discrete search procedure into a continuous equality constraint.
NOTEARS \citep{zheng2018dags} formulates the acyclic constraint as a continuous equality constraint and is further extended by DAG-GNN \citep{yu2019dag}, DECI \citep{geffner2022deep} to support non-linear causal relations. 
DECI \citep{geffner2022deep} is a flow-based model which can perform both causal discovery and inference on non-linear additive noise data.
% These methods can be viewed as unsupervised optimization since they do not access additional datasets associated with ground-truth causal relations. 
Recently, ENCO \citep{lippe2021efficient} is proposed as a continuous optimization method where the edge orientation is modeled as a separate parameter to maintain the acyclicity.
It is guaranteed to converge to the correct graph if interventions on all variables are available.
 \textcolor{black}{RL-BIC \citep{Zhu2020Causal} utilizes Reinforcement Learning to search for the optimal DAG.}
These methods can be viewed as unsupervised since they do not access additional datasets associated with ground truth causal relations.
We refer to \cite{glymour2019review,vowels2022d} for a thorough exploration of this literature. 


SCL begins from orienting edges in the bivariate cases under the functional causal model formalism. 
Methods such as RCC \citep{lopez2015randomized} and NCC \citep{lopez2017discovering} have outperformed unsupervised approaches like ANM \citep{hoyer2008nonlinear} or IGCI \citep{janzing2012information}.
For multivariate cases, ML4S \citep{ma2022ml4s} proposes a supervised approach specifically for skeleton learning. 
%It employs an order-based cascade learning procedure and generates training data from vicinal graphs. 
Complementary to ML4S, ML4C \citep{dai2023ml4c} takes both data and skeleton as input and classifies unshielded triples as either v-structures or non-v-structures. 
\cite{petersen2023causal} proposes a SLdisco method, utilizing SCL approach to address some limitations of PC and GES.

DNN-based SCL has emerged as a prominent approach for enabling end-to-end causal learning. 
Two notable works in this line, namely AVICI \citep{lorchamortized} and CSIvA \citep{ke2023learning}, introduced an alternating attention mechanism to enable permutation invariance across samples and variables. Both methods learn individual representation for each node, which is then used to predict directed edges. Among them, AVICI considers the task of predicting DAG from observational data and adopts exactly the Node-Edge architecture, hence suffers from the issues as discussed in Sec. \ref{sec:int}. On the other hand, CSIvA requires additional interventional data as input to identify the full DAG, and applies an autoregressive DNN architecture where edges are predicted sequentially by multiple inference runs. Therefore, this autoregressive approach incurs very high inference cost due to the quadratic number of model runs required (w.r.t. the number of variables in question), as we experimentally verify in Appendix Sec. \ref{sec:auto}. In contrast, the method proposed in this paper only requires a single run of the DNN model. Besides that, our method also differs from both AVICI and CSIvA in terms of the usage of pairwise embedding vectors.


\section{Limitations of the Node-Edge \textcolor{black}{Architecture}} \label{sec:met:lim}

% \subsection{Motivation} 

% \subsubsection{Case Study of Limitation of Bernoulli-sampling adjacency matrix approach}

%\paragraph{Limitation of the Node-Edge approach.} 
\textcolor{black}{The Node-Edge architecture is common and has been adopted to generate the output DAG $G$ in the literature \citep{lorchamortized,Zhu2020Causal,dpdag,varamballydiscovering}.}
% In previous work, the Node-Edge \textcolor{black}{architecture} is adopted to generate the output DAG $G$ \citep{lorchamortized}.
In this \textcolor{black}{architecture}, each entry $G_{ij}$ in the DAG is independently sampled from $A_{ij}$, an entry in the adjacency matrix $A$. This entry $A_{ij}$ represents the probability that $i$ directly causes $j$. 
We introduce a simple yet effective example setting with only three variables $X$, $Y$, and $T$ to reveal its limitation.

\textcolor{black}{Considering a simulator that generates DAGs with equal probability from two causal models: }In model 1, the causal graph is $G_1: X \rightarrow T \rightarrow Y$, and the variables follow $X \sim \mathcal{N} (0, 1)$, $T = X + \mathcal{N}(0, 1)$, $Y = T + \mathcal{N}(0, 1)$.
In model 2, the causal graph is $G_2: X \leftarrow T \leftarrow Y$, and the variables follow $ Y = \mathcal{N}(0, 3)$, $T = \frac{2}{3}Y + \mathcal{N}(0, \frac{2}{3})$, $X = 0.5T + \mathcal{N}(0, 0.5)$.
In this case, data samples coming from both causal models follow the same joint distribution, which makes $G_1$ and $G_2$ inherently indistinguishable (from observational data sample).

More importantly, when the fully-directed causal DAGs are used as the learning target (as the Node-Edge approach does), an optimally trained neural network will predict $0.5$ probabilities on the directions of the two edges $X - T$ and $T - Y$.
As a result, with $0.25$ probability the graph sampling outcome would be $X \rightarrow T \leftarrow Y$ (see Fig. \ref{fig:ps} in the Appendix).
%It is incompatible with the observational data, resulting in a contradictory causal structure.
This error probability is rooted from the fact that the Bernoulli sampling of the edge $X \rightarrow T$ is not conditioned on the sampling result of the edge $T \leftarrow Y$. Consequently, it is a bias that cannot be avoided even if the DNN has perfectly modeled the \emph{marginal probability} of each edge (marginalized over other edges) given input data.





\color{black}
We further find that $0.25$ is not the worst-case error rate yet. 
Formally, for a distribution $Q$ over a set of graphs, we define the graph distribution where the edges are independently sampled from the marginal distribution as $M(Q)$, i.e., for any causal edges $e_1$ and $e_2$, $P_{G\sim Q}(e_1 \in G) = P_{G\sim M(Q)}(e_1 \in G) = P_{G\sim M(Q)}(e_1 \in G | e_2 \in G)$.
In general, a Node-Edge model optimally trained on data samples $D$ coming from the distribution $Q$ will essentially learn to predict $M(Q)$ (when given the same data samples $D$ at test time).
The following proposition shows that for causal graphs with star-shaped skeleton, with a chance of $26.42\%$ the graph sampled from the marginal distribution $M(Q)$ would be incorrect.


\begin{Proposition}
Let $\mathcal{G}_n$ be the set of graphs with $n+1$ nodes where there is a central node $y$ such that (1) every other node is connected to $y$, (2) there is no edge between the other nodes, \textcolor{black}{and} (3) there is at most one edge pointing to $y$. 
We have 
\begin{align}
\sup_n \max_{Q} P_{G \sim M(Q)}(G \nin \mathcal{G}_n) = 
1 - \frac{2}{e} \approx 0.2642.
\end{align}
\label{prop:star}
\end{Proposition}
The proof is provided in Appendix Sec. \ref{sec:mgcs}. 
% It indicates that a Node-Edge model will output $M(Q)$ when optimally trained on a data sample associated with a distribution $Q$.
It indicates that an edge-predicting neural network could suffer from an inevitable error rate of $0.2642$ even if it is perfectly trained.
\color{black}
%\textbf{Remark:} We clarify that our critique is specifically aimed at the limitations of the Node-Edge approach, not the use of an adjacency matrix as a learning target.
%\textcolor{black}{When the final prediction is up to an MEC rather than a fully identifiable DAG, the Bernoulli-sampling adjacency matrix approach results in inconsistency for UTs formed by non-identifiable edges.
%For instance, }our case study shows that the entries in the adjacency matrix are not independent in determining the causal relations, thus the use of independent Bernoulli sampling over the adjacency matrix falls short of adequately representing causal relations. 
In contrast, models that predict skeleton and v-structures would have a theoretical asymptotic guarantee of the consistency under canonical assumption. 
The details, proof and relevant discussions are provided in Appendix Sec. \ref{sec:app:tg}.






\section{The SiCL Method} \label{sec:algo}
In light of the limitations as discussed, we propose a new DNN-based SCL method in this section, named \textbf{SiCL} (\textbf{S}upervised \textbf{i}dentifiable \textbf{C}ausal \textbf{L}earning).

\subsection{Overall Workflow}
\begin{figure*}[htb]
\centering
\includegraphics[width=\linewidth]{figures/new_workflow.pdf}
      % \vspace{-0.05in}
    \caption{The inference workflow of SiCL.}
    \label{fig:ww}
      % \vspace{-0.2in}
      
\end{figure*}
Following the standard DNN-based SCL paradigm, the core inference process is implemented as a DNN. The DNN takes a data sample encoded by a matrix $D\in\mathbb{R}^{n \times d}$ as input, with $d$ being the number of observable variables and $n$ being the number of observations. In contrast to previous \textcolor{black}{Node-Edge} approaches, the SiCL method does not use the DNN to directly predict the causal graph, but instead seeks to predict the skeleton and the v-structures of the causal graph, which amount to the \emph{MEC-identifiable} causal structures as mentioned previously. 

Specifically, our DNN outputs two objects: (1) a skeleton prediction matrix $S \in [0,1]^{d \times d}$ where $S_{ij}$ models the conditional probability  
\textcolor{black}{of the existence of the \emph{undirected} edge $X_i - X_j$, conditioned on the input data sample $D$, }
% $P(undirected~edge~(X_i,X_j)~exists~|~D)$, 
and (2) a v-structure prediction tensor $V \in [0,1]^{d\times d\times d}$ where $V_{ijk}$ models the conditional probability 
% $P(v\text{-}structure~(X_j \rightarrow X_i \leftarrow X_k)~exists~|~D)$. 
\textcolor{black}{of the existence of the v-structure component $X_j \rightarrow X_i \leftarrow X_k$, again conditioned on $D$.}
In our implementation, $S$ and $V$ are generated by two separate sub-networks, called Skeleton Predictor Network (SPN) and V-structure Predictor Network (VPN), respectively. 
%Both the SPN and the VPN utilize a Pairwise Encoder architecture to explicitly model features about node-pairs. 
To further address the limitation of only having node-wise features for Node-Edge models, we propose to equip SPN and VPN with Pairwise Encoder modules to explicitly capture node-pair-wise features.
% Due to the significance of the pairwise relationship \textcolor{black}{between vertices}, we propose a pairwise encoder module to model the pairwise representations.

Based on the skeleton prediction matrix $S$ and v-structure prediction tensor $V$, we infer the skeleton and v-structures of the causal graph; from the two we can determine a unique CPDAG. The CPDAG encodes a Markov equivalence class, from which we can pick up a graph instance as the prediction of the causal DAG (if needed). 
% See Fig. \ref{fig:ww} for a diagram of the overall inference workflow of SiCL.
\textcolor{black}{Figure \ref{fig:ww} provides a diagram of the overall inference workflow of SiCL, and a pseudo-code of the workflow is given by Algorithm \ref{alg:workflow} in Appendix.}

Parameters of the SPN and VPN are trained following the standard supervised learning procedure. In our implementation, we only use synthetic training data, which is relatively easy to obtain, and yet could often lead to strong performance on real-world workloads~\citep{ke2023learning}.


In the following, we elaborate the DNN architecture, the learning targets, as well as the post-processing procedure used in the SiCL method.







\subsection{Feature Extraction} \label{sec:fem}
% \subsection{Pairwise Encoder Module} \label{sec:pem}

\textbf{Input Processing and Node Feature Encoder.} Given input data sample, which is a matrix $D\in\mathbb{R}^{n \times d}$, the input processing module contains a linear layer for continuous input data or an embedding layer for discrete input data, yielding the raw node features $\mathcal{F}^{raw}_{il}$ for each node $i$ in each observation $l$.
% It consists of an input processing module, a node feature encoder module, and a pairwise encoder module sequentially.
% The input processing module contains a linear layer for continuous input data or an embedding layer for discrete input data.
After that, we employ a node feature encoder to further process the raw node features into the final node features $\mathcal{F}_{il}$.
Similar to previous papers \citep{ke2023learning,lorchamortized}, the node feature encoder is a transformer-like network comprising attention layers over the observation dimension and the node dimension alternately, which naturally maintains permutation equivalence across both variable and data dimension \textcolor{black}{because of the intrinsic symmetry of attention operations}.
\textcolor{black}{More details about the node feature encoder are presented in Appendix Sec. \ref{sec:dnf} due to page limit}.

\textbf{Pairwise Encoder.} Given node features $\mathcal{F} \in \mathbb{R}^{d\times n \times h}$ for all the $d$ nodes, the goal of pairwise encoder is to encode their pairwise relationships by $d^2$ pairwise features, represented as a tensor $\mathcal{P} \in \mathbb{R}^{d\times d \times n \times h}$, where $\mathcal{P}_{ijl} \in \mathbb{R}^{h}$ is a pairwise feature corresponding to the node pair $(i,j)$ in observation $l$.
% $$\mathbf{p}_{ij} \in \mathbb{R}^{h}$.}
As argued in Sec. \ref{sec:int}, both ``internal'' information (i.e., the pairwise relationship) and ``external information (e.g., the context of the conditional separation set) of node pairs are needed to capture persistent dependency and orientation asymmetry.
Our pairwise encoder module \textcolor{black}{is designed to model the internal relationship via node feature concatenation and the non-linear mapping by MLP.}
On the other hand, 
we employ attention operations within the pairwise encoder to capture the contextual relationships (including persistent dependency and orientation asymmetry).%, which is a common practice.

More specifically, the pairwise encoder module consists of the following parts (see Appendix Fig. \ref{fig:pem} for diagrammatic illustration):
%context-based
% Given a set of $d$ nodes, $h$-dimensional node features, the goal of the pairwise encoder $PE$ is to encapsulate their pairwise relationships by $d^2$ $h$-dimensional pairwise features.
% \st{Given a set of $d$ nodes, each node is represented by an $h$ dimensional vector.
% The goal of the pairwise encoder is to encapsulate their pairwise relationships by $d^2$ $h$-dimensional pairwise features.}
\begin{enumerate}[leftmargin=*]
    \item \textit{Pairwise Feature Initialization.} The initial step is to concatenate the node features \textcolor{black}{from the previous node feature encoder module} for every pair of nodes.
    Subsequently, we employ a three-layer MLP to convert each concatenated vector $\mathcal{P}_{ijl} \in \mathbb{R}^{2h}$ to an $h$-dimensional raw pairwise feature, i.e., $\mathcal{P}_{ijl}^1 =  \mathrm{MLP}([\mathcal{F}_{il}; \mathcal{F}_{jl}])$. It is designed to capture the intricate relations that exist inside the pairs of nodes.
    \item \textit{Unidirectional Multi-Head Attention.} In order to model the external information, we employ an attention mechanism where the query is composed of the aforementioned $d^2$ $h$-dimensional raw pairwise features, while the keys and values consist of $h$-dimensional features of $d$ individual nodes, i.e., $\mathcal{P}^2 = \mathrm{MultiHeadAttention}(\mathcal{P}^1, \mathcal{F}, \mathcal{F})$.
Note that, this attention operation is unidirectional, which means we only calculate cross attention from raw pairwise features $\mathcal{P}^1$ to node features $F$.
This design is meant to capture \textcolor{black}{both pair-wise and node-wise information (as both are critical to model the causality, as discussed in Sec. \ref{sec:int}) while at the same time to maintain a reasonable computational cost.}
\item \textit{Final Processing.} Following the widely-adopted transformer architecture, we incorporate a residual structure and a dropout layer after the previous part, i.e., $\mathcal{P}^3 = \mathrm{Norm}(\mathcal{P}^1 + \mathcal{P}^2)$.
Finally, we introduce a three-layer MLP to further capture intricate patterns and non-linear relationships between the input embeddings, as well as to more effectively process the information from the attention mechanism: $\mathcal{P} = \mathrm{Norm}(\mathrm{MLP}(\mathcal{P}^3) + \mathcal{P}^3)$. 
% This approach allows for a comprehensive understanding of the complex relationships and leading to more robust and accurate modeling of causal structures.
% For the given dataset matrix $D \in \mathbb{R}^ {n\times d}$ with $n$ observations, the pairwise encoder module yields a tensor of shape $\mathbb{R}^{n \times d \times d \times h}$, containing an $h$-dimension feature vector for each node pair of each observation. 
It yields the final pairwise feature tensor $\mathcal{P} \in \mathbb{R}^{d \times d \times n \times h}$.
% , giving an $h$-dimension feature vector for each node pair of each observation. 
\end{enumerate}

% Now we introduce the whole architecture of feature extractor.
% It consists of an input processing module, a node feature encoder module, and a pairwise encoder module sequentially.
% The input processing module contains a linear layer for continuous input data or an embedding layer for discrete input data.
% Similar to previous papers \citep{ke2023learning,lorchamortized}, the node feature encoder is a transformer-like network comprising attention layers over either the observation dimension or the node dimension alternately.
% It naturally maintains permutation equivariance across both the variable dimension and the data dimension \textcolor{black}{because of the intrinsic symmetry of attention operations}.
% Subsequently, the pairwise encoder module is applied to obtain the pairwise features $\mathcal{P}$. 


\subsection{Learning Targets} \label{sec:met:lic}

% \subsubsection{Learning Identifiable Causal Structures}
% To address the issue, we propose to allow the network model to learn solely the identifiable causal structures in $G$, i.e., its MEC. 
As mentioned above, our learning target is a combination of the skeleton and the set of v-structures, which together represent an MEC. 
Two separate neural (sub-)networks are trained for these two targets.
% As shown in Sec. \ref{sec:bg}, an MEC can be represented by a combination of the skeleton and the set of v-structures, which are our learning targets.
% As discussed in Sec. \ref{sec:bg}, skeletons are entirely identifiable across all data types, and v-structures are also identifiable for Linear-Gaussian continuous data and discrete data. 
% As discussed in Sec. \ref{sec:bg}, skeletons and v-structures are identifiable under our settings. 
% Consequently, our learning target contains the skeleton and the set of v-structures, forming a representation of the MEC. % under the conditions of Linear-Gaussian continuous and discrete data.%, and only the skeleton under other conditions. 
%It avoids unnecessary prediction errors and contradictions.
%For instance, the two different causal structures share the same MEC in the above example, and our model can predict $X - T - Y$ as the MEC representation.
%learn from unified precise labels of skeleton and v-structures.
% This refined approach mitigates the challenges associated with unidentifiable causal structures and enhances the overall performance of the network model.

\textbf{Skeleton Prediction.} 
As the persistent dependency between pairs of nodes determines the existence of edges in the skeleton, the pairwise features correspond to edges in the skeleton naturally.
Therefore, for the skeleton learning task, we initially employ a max-pooling layer over the observation dimension to obtain a single vector $\mathcal{S}_{ij} \in \mathbb{R}^{h}$ for each pair of nodes, i.e., $\mathcal{S}_{ij} = \max_{k} \mathcal{P}_{ijk}$.
Then, a linear layer and a sigmoid function are applied to map the pairwise features to the final prediction of edges, i.e., $S_{ij} = \mathrm{Sigmoid}(\mathrm{Linear}(\mathcal{S}_{ij}))$.
Our learning label, the undirected graph representing the skeleton, can be easily calculated by summing the adjacency of the DAG $G$ and its transpose $G^T$.
% Denoting the combination of max-pooling and linear layer as a skeleton prediction module $SP$, 
Therefore, our learning target for the skeleton prediction task can be formulated as $\min \mathcal{L}(S, G + G^T)$,
where $\mathcal{L}$ is the popularly used binary cross-entropy loss function.

% $FE$ is the feature extractor mentioned above, and $D$ denotes the input data.


\textbf{V-structure Prediction.}
% he orientation asymmetry is significant to judge the existence of v-structures.
A UT $\langle X_i, X_k, X_j \rangle$ is a v-structure when $\exists \mathbf{S}$, such that $X_k \notin \mathbf{S}$ and $X_i \perp X_j | \mathbf{S}$.
Motivated by this, we concatenate the corresponding pairwise features of the pair $\langle X_i, X_j \rangle$ with the node features of $X_k$ as the feature for each UT $\langle X_i, X_k, X_j \rangle$ after a max-pooling along the observation dimension, i.e., $\mathcal{U}_{kij} = [\max_l \mathcal{P}_{ijl};\max_l \mathcal{F}_{kl}]$.
After that, we use a three-layer MLP with a sigmoid function to predict the existence of v-structures among all UTs, i.e., $\mathcal{U}_{kij} = \mathrm{Sigmoid}(\mathrm{MLP}( \mathcal{U}_{kij}))$.
Given a data sample of $d$ nodes, it outputs a third-order tensor of shape $\mathbb{R}^{d \times d \times d}$, namely v-tensor, corresponding to the predictions of the existence of v-structures.
The v-tensor label can be obtained by $\mathcal{V}_{kij} = G_{ik} G_{jk} (1 - G_{ij})(1 - G_{ji})$,
where $\mathcal{V}_{kij}$ indicates the existence of v-structure $X_i \rightarrow X_k \leftarrow X_j$.
Therefore, the learning target for the v-structure prediction task can be formulated as $\min \mathcal{L}_{UT}(\mathcal{U}, \mathcal{V})$,
where $\mathcal{L}_{UT}$ is the binary cross-entropy loss masked by UTs, i.e., we only calculate such loss on the valid UTs.
In our current implementation, the parameters of the feature encoders are fine-tuned from the skeleton prediction task, as the UTs to be classified are obtained from the predicted skeleton and the skeleton prediction can be seen as a general pre-trained task.

Note that neural networks with our learning targets have a theoretical guarantee for correctness in asymptotic sense, as mentioned around the end of Sec. \ref{sec:met:lim}. 

\subsection{Post-Processing} 
\textcolor{black}{Although our method theoretically guarantees asymptotic correctness, conflicts in predicted v-structures might occasionally occur in practice. Therefore,} in the post-processing stage, we apply a straightforward heuristic to resolve the potential conflicts and cycles among predicted v-structures following previous work \citep{dai2023ml4c}.
\textcolor{black}{After that, we use an improved version of Meek rules \citep{meek1995causal,tsagris2019bayesian} to obtain other MEC-identifiable edges without introducing extra cycles.}
Combining the skeleton from the skeleton predictor model with all MEC-identifiable edge directions, we get the CPDAG predictions.

We provide a more detailed description of the post-processing process in Appendix Sec. \ref{app:post}. It is worth noting that our current design of post-processing is a very conservative one, and this module is also non-essential in our whole framework; see Appendix Sec. \ref{app:post} for more discussions and evidences.
\color{black}


\section{Experiments} \label{sec:exp}

\begin{table*}[!tb]
\centering
% \resizebox{\linewidth}{!}{%
\begin{threeparttable}
\caption{\textbf{General comparison of SiCL and other methods}. The average performance results in three runs are reported for SiCL. GES takes more than 24 hours per graph on WS-L-G. SLdicso is unsuitable on non-linear-Gaussian data. \textcolor{black}{Full results on all metrics are provided in Appendix Tab. \ref{tab:epder}}.}
\label{tab:epders}
\begin{tabular}{cccccccccccc}
\toprule
 \multirow{2}{*}{Method} & \multicolumn{2}{c}{WS-L-G}& \multicolumn{2}{c}{SBM-L-G}& \multicolumn{2}{c}{WS-RFF-G}& \multicolumn{2}{c}{SBM-RFF-G}& \multicolumn{2}{c}{ER-CPT-MC} \\
 & s-F1$\uparrow$ & o-F1$\uparrow$ &s-F1$\uparrow$ & o-F1$\uparrow$&s-F1$\uparrow$ & o-F1$\uparrow$&s-F1$\uparrow$ & o-F1$\uparrow$&s-F1$\uparrow$ & o-F1$\uparrow$\\
\midrule
 PC & $30.4 $ & $16.0 $& $58.8$&$35.9$&$36.1$&$16.1$&$57.5$&$34.2$&$82.2$&$40.6$ \\
 GES & * & * & $70.8$& $55.0$&$41.7$&$23.6$&$56.5$&$38.0$&$82.1$&$42.4$\\
 NOTEARS & $33.3 $ & $31.5$&$80.1$&$77.8$&$37.7$&$33.4$&$55.6$&$48.5$&$16.7$&$0.6$ \\
 DAG-GNN & $35.5$ & $32.7$ &$66.2$&$62.5$&$33.2$&$28.9$&$47.1$&$40.6$&$24.8$&$3.7$\\
 GRAN-DAG & $16.6$&$11.7$&$22.6$&$14.4$&$4.7$&$1.1$&$17.4$&$3.8$&$40.8$&$7.3$ \\
 % NOTEARS-MLP &$24.6$&$11.8$&$44.7$&$39.0$&$\mathbf{52.7}$&$\mathbf{47.7}$& $48.2$ &$43.5$& *& * & \\
 GOLEM &$30.0$ &$19.3$&$68.5$&$65.2$&$27.6$&$17.7$&$41.1$&$24.8$&$37.6$&$9.3$& \\
 % GRaSP &&&&&&&&&&$0.0$&$0.0$ \\
 SLdisco &$0.1$ &$0.1$&$1.9$&$1.2$&*&*&*&*&*&* \\
 AVICI & $39.9 $ & $35.8$ & $84.3$ & $81.6$& $47.7$& $45.2$& $76.6$& $72.7$& $76.9$& $57.6$\\
 SiCL & $\mathbf{44.7} $ & $\mathbf{38.5} $& $\mathbf{85.8} $ & $\mathbf{82.7} $ & $\mathbf{51.8}  $ & $ \mathbf{46.3}$ & $ {\mathbf{82.1}}$ & $\mathbf{78.0}$ &$\mathbf{84.2}$ & $\mathbf{59.9}$ \\
\bottomrule
\end{tabular}
\end{threeparttable}
% }
% \vspace{-0.15in}
\end{table*}
In this section, we report the performance of SiCL on both synthetic and real-world benchmarks, followed by an ablation study. 
More results and discussions about \textcolor{black}{time cost}, generality, and acyclicity are deferred to Appendix Sec. \ref{sec:app:exp:e}, due to page limit.

\subsection{Experiment Design}
\textbf{Metrics.} We profile a causal discovery method's performance using the following two tasks:

\emph{Skeleton Prediction}: 
Given a data sample $D$ of $d$ variables, for each variable pair, we want to infer if there exists direct causation between them. The standard metric \textbf{s-F1} (short for \textbf{skeleton-F1}) is used, which considers skeleton prediction as a binary classification task over the $d(d-1)/2$ variable pairs. 
For completion, we also report classification accuracy results. 
For methods with probabilistic outputs, AUC and AUPRC scores are also measured.

% \emph{Graph Prediction} :
% Given a data sample $D$ of $n$ variables, for each variable pair we want to infer if there exists direct causation, and in that case we want to further infer the causal direction. Following ML4C \cite{dai2023ml4c}, we use the following \textbf{orientation-F1} metric for this structured prediction task: A variable pair is considered a positive item if there is direct causation between them, and a prediction about the pair is a positive prediction if at least one causal direction is predicted. A positive prediction is a true positive if it's over a positive item \emph{and the predicted causal direction is correct}. The standard F1 calculation is then applied.

% Another metric, called \textbf{edge-F1} in this paper, was used in some previous works \textcolor{red}{[which?]}, which considers the task as a binary classification problem over the $n^2$ \emph{ordered-pairs} (whether there is a directed edge in the groundtruth causal graph or not). We also measured edge-F1 for completion.

\emph{CPDAG Prediction}: 
% Given a data sample $D$ of $d$ variables, for each variable pair, we want to infer if there exists direct causation between them, in that case we want to infer if the causal direction is identifiable, and in that case we try to infer the causal direction. 
\textcolor{black}{Given a data sample $D$ of $d$ variables, for each pair of variables, we aim to determine if there is direct causality between them. If so, we then assess whether the causal direction is MEC-identifiable, and if it is, we attempt to infer the specific causal direction.}
As this task involves both directed and undirected edge prediction, we use \textbf{SHD} (Structural Hamming Distance) to measure the difference between the true CPDAG and the inferred CPDAG. Besides that, we also measure the \textbf{o-F1} (short for \textbf{orientation-F1}) of the directed sub-graph of the inferred CPDAG (compared against the directed sub-graph of the true CPDAG), which focuses on capturing the inference method's orientation capability in identifying \textit{MEC-identifiable} causal edges. 
% Finally, we also measure \textbf{v-F1}, which is F1 score with the set of v-structures in the true CPDAG as positive items (among all ordered-triples), and v-structures in the inferred CPDAG as positive predictions.
\textcolor{black}{Finally, we calculate the \textbf{v-F1} score, where the F1 score is based on the set of v-structures in the true CPDAG as the positive instances (from all ordered triples), and the v-structures in the inferred CPDAG as the positive predictions.}













\textbf{Testing Data.} 
A testing instance consists of a groundtruth causal graph $G$, the structural equations $f_i$ and noise variables $\epsilon_i$ for each variable $X_i$, and an i.i.d. sample $D$. We use two categories of testing instances in our experiments:

\textit{Analytical Instances}: 
where $G$ is sampled from a DAG distribution $\mathcal{G}$, and $\{f_i,\epsilon_i\}$ sampled from a structural-equation distribution $\mathcal{F}$ and a noise meta-distribution $\mathcal{N}$. 
%It is called an analytical instance because 
%$\mathcal{G}$, $\mathcal{F}$, $\mathcal{N}$ all have analytical forms.
We consider three random graph distributions for $\mathcal{G}$: Watts-Strogatz (WS), Stochastic Block Model (SBM), Erdos-Rényi (ER); and three $\mathcal{F}$'s: random linear (L), Random Fourier Features (RFF), and conditional probability table (CPT). $\mathcal{N}$ is a uniform distribution over Gaussian's for continuous data, and a Dirichlet distribution over Multinomial Categorical distributions for discrete data. 
% We examined five combinations of these variations: \textbf{WS-L-G}, \textbf{SBM-L-G}, \textbf{WS-RFF-G}, \textbf{SBM-RFF-G}, \textbf{ER-CPT-MC}.
\textcolor{black}{We examine five combinations of testing instances: \textbf{WS-L-G}, \textbf{SBM-L-G}, \textbf{WS-RFF-G}, \textbf{SBM-RFF-G}, and \textbf{ER-CPT-MC}.}




\textit{Real-world Instance}:
% The classic dataset \textbf{Sachs} is used. 
% It consists of a data sample recording the concentration levels of 11 phosphorylated proteins in 853 human immune system cells, and of a causal graph over these 11 variables identified by \cite{sachs2005causal} based on expert consensus and experimental biology literature. 
% %\textcolor{red}{[is the concentration levels discretized?]}
The classic dataset \textbf{Sachs} is used \textcolor{black}{to evaluate performance in real-world scenarios}. 
It consists of a data sample recording the concentration levels of 11 phosphorylated proteins in 853 human immune system cells, and of a causal graph over these 11 variables identified by \cite{sachs2005causal} based on expert consensus and biology literature.








\textbf{Algorithms.}
As baselines, we compare with a series of representative unsupervised methods, including \textbf{PC} (using the recent Parallel-PC variation by ~\citet{le2016fast}), 
\textbf{GES}~\citep{chickering2002optimal}, 
\textbf{NOTEARS}~\citep{zheng2018dags}, 
% \textbf{NOTEARS-MLP}~\citep{zheng2018dags},
\textbf{GOLEM}~\citep{ng2020role}, 
\textbf{DAG-GNN}~\citep{yu2019dag},
\textbf{GRANDAG}~\citep{Lachapelle2020Gradient-Based}, \textbf{SLdisco} \citep{petersen2023causal} as well as \textbf{AVICI}~\citep{lorchamortized}, a DNN-based SCL method regarded as current state-of-the-art method.


For our method, besides the full \textbf{SiCL} implementation as described by Sec. \ref{sec:algo}, 
we also implement 
\textbf{SiCL-Node-Edge}, which predicts the causal graph using the node features and can be regarded as equivalent to AVICI, and 
\textbf{SiCL-no-PF}, which skips pairwise feature extraction and predicts the skeleton and v-tensor using node-wise features (see Appendix Fig. \ref{fig:abl}). 
% It is noteworthy that SiCL contains 6 layers on the node feature encoder module while SiCL-no-PF and AVICI contains 8 layers to eliminate any potential bias arising from differences in model size. 
Notably, SiCL contains 2.8M parameters, while SiCL-Node-Edge and SiCL-no-PF contain 3.2M parameters, because SiCL contains fewer layers on the node feature encoder to eliminate potential bias from size difference. 

For DNN-based SCL methods, the DNNs are trained with synthetic data where the causal graphs follow the Erdos-Rényi (ER) and Scale-Free (SF) models and the structural equations and noise variables follow the same distribution type as the corresponding testing data. 
Therefore, the disparities between the causal graph distribution at training and testing time help to examine the generality of SiCL in \textbf{OOD} settings to some extent.
% Our evaluation part is mostly about OOD setting, i.e., the distribution of test set is OOD w.r.t. the distribution of training set.
%For interface mismatches (e.g. CPDAG prediction with a graph prediction algorithm, or discrete-data experiment on an algorithm that originally only accepts continuous data), straightforward adaptations are applied. 
% See Appendix Sec. \ref{sec:app:exp:set} for more details in the experimental setting. 
More details of the experimental setting are presented in Appendix Sec. \ref{sec:app:exp:set}.







\subsection{Results on Synthetic Dataset} \label{sec:exp:gp}



We conduct a comprehensive comparison of SiCL with various baselines in both skeleton prediction and CPDAG prediction tasks.
The main results of metrics skeleton-F1 and orientation-F1 are presented in Tab. \ref{tab:epders}, and results on full metrics are provided in Appendix Tab. \ref{tab:epder}.
% We perform experiments in Tab. \ref{tab:epder} and Appendix Tab. \ref{tab:mc}-\ref{tab:mc2} on both continuous datasets and discrete datasets to evaluate the performance of the competing methods. 
On continuous data, DNN-based SCL methods (i.e., AVICI and SiCL) demonstrate consistent and obvious advantages over traditional approaches.
SiCL consistently outperforms the other methods on both skeleton prediction task and CPDAG prediction task.
On the other hand, some unsupervised methods achieve comparable performance among DNN-based SCL methods on the discrete data ER-CPT-MC. 
Nonetheless, our proposed SiCL emerges as the top performer, further substantiating its superiority in addressing the causal learning problem. 














\subsection{Results on Real-world Dataset} \label{sec:exp:erd}
\begin{table}
\centering
% \vspace{-0.45in}
\caption{Comparison on Sachs dataset.}
% \vspace{-0.05in}
\label{tab:sachs}
    \resizebox{\linewidth}{!}{%
    \begin{threeparttable}
\begin{tabular}{@{}ccccc@{}}
\toprule
 % & \multicolumn{2}{c}{CPDAG metrics} & \multicolumn{2}{c}{Skeleton Metrics} \\
 \multirow{2}{*}{Method} & \multicolumn{2}{c}{Skeleton Prediction }& \multicolumn{2}{c}{CPDAG Prediction} \\
 &  s-F1$\uparrow$ & s-Acc.$\uparrow$ & SHD$\downarrow$ & \#v-struc.$\downarrow$\\ \midrule
 PC& $68.6$& $80.0$ &$19$&$12$\\
 GES & $70.6$ & $81.8$ &$19$&$8$\\
 DAG-GNN & $21.1$ & $72.7$&$15$&$\mathbf{0}$ \\
 NOTEARS & $11.1$ & $70.9$ &$16 $ & $ \mathbf{0}$ \\
  GRAN-DAG & $45.5 $ & $78.2 $ &$ 12 $ & $\mathbf{0}$ \\
  GOLEM & $ 36.4$ & $ 74.5$ &$ 14 $ & $\mathbf{0}$ \\
  % CAM & $87.2$ & $90.9$ & $17$ & $6$\\
AVICI & $66.7 $&$ 83.5$  &$18 $&$ 14$\\ 
SiCL & $\mathbf{71.4}$ & $\mathbf{86.8}$&$\mathbf{6}$&$\mathbf{0}$   \\ 
\bottomrule
\end{tabular}
\end{threeparttable}
}
% \vspace{-0.05in}
\end{table}
To assess the practical applicability of SiCL, we conduct a comparison using the real-world dataset Sachs.
The discretized Sachs data obtained from the bnlearn library\footnote{https://www.bnlearn.com/} is used.
The DNN-based SCL methods are trained on random synthetic graphs, making this also an \textbf{OOD} prediction task.
The results are provided in Tab. \ref{tab:sachs}.

For the skeleton prediction task, SiCL performs the best, albeit with a modest gap (generally 1$\sim$3 scores higher than the runners-up). For the CPDAG prediction task, SiCL performs significantly better than all other methods (reducing SHD from 12 to 6, against the second best). Interestingly, the true causal DAG of the Sachs benchmark actually contains no v-structure, so any predicted v-structure is an error. We see that methods competitive with SiCL in skeleton prediction (AVICI, PC, GES) mistakenly predicted a large number of v-structures on the Sachs data, while SiCL correctly predict zero v-structure.


\begin{table}[!tb]
\centering
% \resizebox{\linewidth}{!}{%
\begin{threeparttable}
\caption{Ablation study of SiCL components. Full metrics are available in Appendix Tab. \ref{tab:fcplg}.}
\label{tab:cplg}
% \vspace{-0.05in}
\begin{tabular}{ccccc}
\toprule
 \multirow{2}{*}{Method} & \multicolumn{2}{c}{WS-L-G}& \multicolumn{2}{c}{SBM-L-G} \\
 & s-F1$\uparrow$ & o-F1$\uparrow$ &s-F1$\uparrow$ & o-F1$\uparrow$\\\midrule
 SiCL-Node-Edge & $39.9$ & $35.8$ & $84.3$ & $81.6$ \\
 SiCL-no-PF &$42.4$ & $37.9$ & $85.5$ & $82.2$ \\ 
 SiCL & $\mathbf{44.7}$ & $\mathbf{38.5}$ &$\mathbf{85.8}$& $\mathbf{82.7}$ \\
\bottomrule
\end{tabular}
\end{threeparttable}
% }
% \vspace{-0.2in}
\end{table}


\subsection{\textcolor{black}{Ablation Study}}


\begin{figure}[t]
    \centering
    % \vspace{-0.4in}
    \includegraphics[width=\linewidth]{figures/Cmp_on_of1_font.pdf}
    \caption{Comparison of SiCL-Node-Edge and SiCL-no-PF in o-F1 trend as observation samples increase on a constructed dataset.}
    \label{fig:cto}
    % \vspace{-0.15in}
\end{figure}
\textbf{Effectiveness of Learning Identifiable Structures.} \label{sec:exp:elis}As discussed in Section \ref{sec:met:lic}, SiCL focuses on learning MEC-identifiable causal structures rather than directly learning the adjacency matrix.
To verify the effectiveness of this idea,
% in practice and eliminate the bias of sampling from specifically designed distribution as much as possible, 
we compare SiCL-Node-Edge with SiCL-no-PF.
% to provide empirical support for the superiority of learning identifiable causal structures.
% evaluate the effectiveness of learning identifiable structures.
% test models on MEC-randomized WS-L-G and SBM-L-G datasets, where the training data is also correspondingly MEC-randomized.
% The comparisons between AVICI and SiCL-no-PF 
These two models share a similar node feature encoder architecture but have different learning targets: SiCL-Node-Edge predicts the adjacency matrix, while the SiCL-no-PF predicts the skeleton and v-tensor.
The results are shown in Table \ref{tab:cplg}.
Consistently, SiCL-no-PF demonstrates higher performance on both skeleton and CPDAG prediction tasks. 
This observation echoes our theoretical conclusion regarding the necessity and benefits of learning identifiable causal structures to improve overall performance.



To further underscore the significance of learning identifiable causal structures (especially in the asymptomatic sense), we conduct a comparative analysis using a specially constructed dataset, which contains six nodes forming an independent v-structure and a UT. 
Figure \ref{fig:cto} illustrates that the orientation F1 scores of CPDAG predictions from SiCL-Node-Edge suffer from an unavoidable error and do not improve with the addition of more observational samples. 
In contrast, predictions from SiCL-no-PF reach perfect accuracy, confirming the value of learning identifiable causal structures.





\textbf{Effectiveness of Pairwise Representation.} To assess the effectiveness of pairwise representation, we compare the full version of SiCL with a variant lacking pairwise features (SiCL-no-PF). As shown in Table \ref{tab:cplg}, the full-version SiCL consistently outperforms SiCL-no-PF in both skeleton prediction and CPDAG prediction tasks. Notably, we have intentionally set the model size of the full-version SiCL (2.8M parameters) to be smaller than that of SiCL-no-PF (3.2M parameters) so as to avoid any potential advantage from increased model complexity brought by the pairwise feature encoder module. The observed performance gains in this case underscore the critical role of pairwise features in identifying causal structures. Additionally, we conduct further comparisons across more diverse settings, with results detailed in Appendix Sec. \ref{sec:mpc}. These results demonstrate even more pronounced improvements in favor of SiCL, reinforcing the importance of pairwise representations in causal discovery.

\section{Conclusion}


We proposed SiCL, a novel DNN-based SCL approach designed to predict the corresponding skeleton and a set of v-structures. We showed that such design do not suffer from the (non-)identifiability limit \textcolor{black}{that exists in current architectures}. Moreover, SiCL is equipped with a pairwise encoder module to explicitly model relationships between node-pairs. 
Experimental results validated the effectiveness of these ideas. %showed that SiCL significantly outperforms other DNN-based SCL approaches. 
% \color{black}
% For limitations and future works, please refer to Appendix \ref{sec:lim}.
% \color{black}

This paper also introduces a few interesting open problems.
The proposed DNN model works in the canonical setting under the classic MEC theory, in which the skeleton and v-structures are the identifiable structure.
It can be an interesting future-work direction to explore how to learn other identifiable causal structure in other assumption settings following the same principle.
Due to the inherent complexity of DNNs, the explanation of the decision mechanism of our model remains an open question. 
Therefore, future work could consider to explore how decisions are made within the networks and provide some insights for traditional methods. 
Moreover, the proposed pairwise encoder modules needs $O(d^3)$ computational complexity, which may restrict its current application to scenarios with huge number of nodes.
Future work could focus on simplifying these operations or exploring features with less complexity (e,g., low rank features) to reduce the overall computational cost.


\documentclass{MITstyle}

%\usepackage[table]{xcolor}
\usepackage{chngcntr}
\usepackage{hyperref}
\usepackage{microtype}

\title{A Lightweight and Extensible Cell Segmentation and Classification Model for Whole Slide Images}

\author{Nikita Shvetsov~$^{1, }$\footnote{Correspondence e-mail: nikita.shvetsov@uit.no}, Thomas K. Kilvaer~$^{2, 3}$, Masoud Tafavvoghi~$^{4}$, Anders Sildnes~$^{1}$, \\ Kajsa Møllersen~$^{4}$, Lill-Tove Rasmussen Busund~$^{5, 6}$, Lars Ailo Bongo~$^{1}$ \\
%
\vspace{1em} % Space between authors and afilliations
%
\normalfont{\small $^{1}$Department of Computer Science, UiT The Arctic University of Norway}\\
\normalfont{\small $^{2}$Department of Oncology, University Hospital of North Norway}\\
\normalfont{\small $^{3}$Department of Clinical Medicine, UiT The Arctic University of Norway}\\
\normalfont{\small $^{4}$Department of Community Medicine, UiT The Arctic University of Norway}\\
\normalfont{\small $^{5}$Department of Medical Biology, UiT The Arctic University of Norway} \\
\normalfont{\small $^{6}$Department of Clinical Pathology, University Hospital of North Norway} %\vspace{2em}
}

\begin{document}
\maketitle

\section*{Abstract}

% \begin{abstract}
% Developing clinically useful cell-level analysis tools in digital pathology remains challenging due to limitations in dataset granularity, inconsistent annotations, computational demands of advanced models, and difficulties in integrating new technologies into clinical workflows. To address these challenges, we propose a multi-faceted solution that enhances data quality, model performance, and usability to create a lightweight and extensible cell segmentation and classification model.

% First, we update data labels by employing a cross-relabeling process that refines the labels of two existing datasets, PanNuke and MoNuSAC, to create a new unified dataset with enhanced granularity, encompassing seven distinct cell types. Second, we leverage the H-Optimus foundation model as a fixed encoder to improve feature representation for simultaneous cell segmentation and classification tasks. Third, to address the computational demands of foundation models, we employ knowledge distillation to reduce model size and complexity while maintaining comparable performance. Finally, to facilitate integration into clinical workflows, we integrate the distilled model into the QuPath software, a widely used open-source platform in digital pathology.

% Our results demonstrate improvements in cell segmentation and classification performance using the H‑Optimus-based model compared to a CNN-based model. Specifically, the average $R^2$ improved from 0.575 to 0.871, and the average $PQ$ score improved from 0.450 to 0.492, indicating better alignment with actual cell counts and enhanced segmentation and classification quality. Furthermore, the distilled student model maintains performance comparable to the larger foundation model while reducing the parameter count by a factor of 48.
% Overall, by reducing computational complexity and integrating it into existing workflows, the proposed approach may significantly impact diagnostic processes, reduce the workload of pathologists, and contribute to improved patient outcomes. Though our approach shows potential enhancements in efficiency and usability of cell segmentation and classification models in digital pathology, extensive validation is needed to deploy these models in clinical practice.
% \end{abstract}

%%% shortened abstract
\begin{abstract}
Developing clinically useful cell-level analysis tools in digital pathology remains challenging due to limitations in dataset granularity, inconsistent annotations, high computational demands, and difficulties integrating new technologies into workflows. To address these issues, we propose a solution that enhances data quality, model performance, and usability by creating a lightweight, extensible cell segmentation and classification model. 

First, we update data labels through cross-relabeling to refine annotations of PanNuke and MoNuSAC, producing a unified dataset with seven distinct cell types. Second, we leverage the H-Optimus foundation model as a fixed encoder to improve feature representation for simultaneous segmentation and classification tasks. Third, to address foundation models' computational demands, we distill knowledge to reduce model size and complexity while maintaining comparable performance. Finally, we integrate the distilled model into QuPath, a widely used open-source digital pathology platform. 

Results demonstrate improved segmentation and classification performance using the H-Optimus-based model compared to a CNN-based model. Specifically, average $R^2$ improved from 0.575 to 0.871, and average $PQ$ score improved from 0.450 to 0.492, indicating better alignment with actual cell counts and enhanced segmentation quality. The distilled model maintains comparable performance while reducing parameter count by a factor of 48. By reducing computational complexity and integrating into workflows, this approach may significantly impact diagnostics, reduce pathologist workload, and improve outcomes. Although the method shows promise, extensive validation is necessary prior to clinical deployment.
\end{abstract}
\clearpage

\section{Introduction}
In digital pathology, accurate segmentation and classification of cells are crucial for many diagnostic, prognostic, and predictive analyses \cite{Jaber_Beziaeva_etal._2019,Lin_Pan_etal._2022,Park_Ock_etal._2022,Shen_Choi_etal._2024}. Nowadays, developments in computational pathology offer multiple solutions \cite{H._Qu_P._Wu_etal._2020,Javed_Mahmood_etal._2020} to utilize cell-level datasets to train machine learning models that solve these problems. The quality and specificity of training datasets are critical for robust and accurate models. Adhering to the principle of "garbage in, garbage out", it is essential to ensure that these datasets are extensively and accurately labeled with distinct classes that reflect the diverse biological characteristics of different cell types. Unfortunately, the number of open-source datasets comprising such high-quality annotations is limited. Existing cell segmentation datasets \cite{Gamper_Koohbanani_etal._2019,Graham_Vu_etal._2019,Verma_Kumar_etal._2021} may offer extensive annotations for certain cell types while providing more general labels for others. For example, in PanNuke, which is one of the largest open-source datasets comprising labeled cells, various types of morphologically and functionally different inflammatory cells like macrophages and lymphocytes are clustered in a broad "inflammatory" class. Consequently, these classes are frequently omitted from analyses or aggregated into broader meta-classes \cite{Gamper_Koohbanani_etal._2020} and likely interfere with other cell classes included in the dataset. This and similar inconsistencies in annotation granularity limit the ability of machine learning models to learn the comprehensive and nuanced features necessary for accurate cell segmentation and classification. To address these challenges, methods for refining and standardizing dataset annotations are essential to enhance the quality of training data.

A complementary approach to mitigate the absence of high-quality training data is the use of foundation models. Foundation models as encoders are defined as large-scale, versatile networks pre-trained on vast, diverse datasets using self-supervised learning, contrasting with convolutional neural network (CNN) pre-trained encoders that rely on supervised learning with labeled data. In practice, foundation models leverage enormous amounts of weakly or unlabeled data from millions of whole slide images (WSIs) and employ self-attention mechanisms to capture long-range dependencies and global context \cite{Chen_Ding_etal._2024,Saillard_Jenatton_etal._2024,Vorontsov_Bozkurt_etal._2024,Xu_Usuyama_etal._2024}. As a consequence, foundation models are able to produce transferable feature representations across different cell types and tissue environments. The feature representations can be leveraged by decoder networks to produce segmentation masks and pixel-level classifications. Because foundation models have comprehensive feature representations, they can be effectively fine-tuned using much smaller amounts of cell-level data compared to the large datasets needed to train models from scratch. Furthermore, foundation models incorporate adversarial training elements or contrastive learning \cite{Chen_Ding_etal._2024,Xu_Usuyama_etal._2024}, enhancing their resilience and adaptability by exposing them to challenging and varied scenarios during training. This may result in more generalizable models, often making them well-suited for diverse and complex tasks in digital pathology.

Despite the inherent advantages of foundation models, their deployment for practical use faces its own obstacles. In particular, they require substantial computational power, financial investments and rigorous testing to ensure reliability and efficacy for a given task \cite{Akkus_Dangott_etal._2022,Dragomir_Cocuz_etal._2022,Go_2022,Jafri_Farooqui_etal._2024}. Moreover, while foundation models enhance feature representation and performance, they depend on the quality of available annotations for decoder fine-tuning and, like any other model, cannot resolve existing inconsistencies or ambiguities in data labels. Therefore, there remains a critical need for solutions that address both data quality and practical deployment considerations.
Further, integrating new technologies into existing clinical workflows often encounters resistance, as it necessitates adjustments to established diagnostic processes. So, there is a need to develop solutions that could be integrated into current practices, minimizing the burden on medical professionals to adopt new tools \cite{King_Williams_etal._2023}.

Existing solutions \cite{Goldsborough_Philps_etal._2024,Hörst_Rempe_etal._2024}, while addressing some aspects of these challenges, fall short in providing a comprehensive approach. To address the data quality and clinical deployment issues, we propose a multi-faceted solution that encompasses data refinement, model optimization, and integration with existing pathology tools (\hyperref[fig:fig1]{Figure 1}). The outcome is a lightweight cell segmentation and classification model that can be integrated into digital pathology workflows for practical clinical use.

\begin{figure}[h!]
    \centering
    \includegraphics[width=\textwidth, height=0.82\textheight, keepaspectratio]{images/Figure_1.pdf}
    \caption{Overview of the proposed solution, including 1) Data refinement using cross-relabeling, 2) Teacher model development and fine tuning, 3) Student model optimization with knowledge distillation and 4) Student model and QuPath integration}
    \label{fig:fig1}
\end{figure}
\clearpage

Our approach begins with preparing the data for the fine-tuning and training of the machine learning models. We create a refined dataset, acquired via cross-relabeling two cell-level datasets, enhancing annotation specificity and consistency of the labeled data. Subsequently, we create a cell segmentation and classification model based on the foundation model. We leverage the foundation model as a fixed encoder and fine-tune a decoder using the refined dataset to improve generalization across diverse tissue- and cell types.
To ensure that the model remains lightweight and deployable in a possibly resource-constrained environment, we employ knowledge distillation to approximate the functionality of the foundation model. Finally, to facilitate the practical application of our model in digital pathology workflows, we integrate it with the QuPath \cite{Bankhead_Loughrey_etal._2017} application. Each methodological component contributes to the overarching goal of enhancing model performance, generalizability, and usability in clinical settings.

The primary contributions of this paper are:
\begin{enumerate}
    \item \textit{Data labels refinement through cross-relabeling:}
    
    We propose a new method for refining labels of cell-level datasets through cross-relabeling. This method employs classification models to re-label broad and ambiguous instances, resulting in a more diverse dataset. Our evaluation demonstrates that these classification models achieve high accuracy on test subsets, indicating the reliability of the method for label refinement.

    \item \textit{Enhanced model performance via foundation models:}
    
    We employ a foundation model as a feature extractor for the cell segmentation and classification task. In comparison with training a CNN model from scratch, the foundation model backbone only needs fine-tuning, which significantly reduces training time, computational resources and data requirements. We show that using a foundation model encoder leads to better performance in cell segmentation and classification networks than using a CNN-based encoder. This improvement may enable the model to generalize more effectively across various tissue types and imaging methods.
    
    \item \textit{Model optimization through knowledge distillation:}
    
    We show that a smaller student model trained using knowledge distillation on the refined dataset obtained via our cross-relabeling approach from a foundation model achieves comparable performance in cell segmentation and quantification tasks. As a result, this model is more suitable for deployment in environments without high-performance computing resources.
    
    \item \textit{Integration with QuPath:}
    
    We integrate the distilled cell segmentation and classification model into QuPath, a widely used open-source digital pathology platform, to accelerate clinical adaptation by enabling pathologists to more easily incorporate advanced computational tools into their existing workflows.
\end{enumerate}

Through these methodological steps, we aim to bridge the gap between advanced machine learning techniques and practical clinical applications, making accurate and efficient digital pathology accessible in a broader range of healthcare settings.

\section{Refining Existing Datasets Using Cross-Relabeling}
To address the limitations of sparse and ambiguous labeling of cell-level datasets, we propose a generalizable cross-relabeling strategy that can be applied to any dataset containing broadly categorized or imprecisely labeled cell types. This approach involves training and subsequently leveraging classification models to refine broad categories into more specific or biologically relevant classes.
When applied to cell-level data, the methodology includes extracting individual cell images from the dataset patches, preprocessing these images to standardize the size and accommodate partial cells, and then training deep learning classifiers capable of distinguishing between the finer cell subtypes within the coarser categories. 
To illustrate our approach, we focus on the PanNuke \cite{Gamper_Koohbanani_etal._2020, Gamper_Koohbanani_etal._2019} and MoNuSAC \cite{Verma_Kumar_etal._2021} datasets that we have used to train models for cell quantification in our previous works \cite{Shvetsov_Grønnesby_etal._2022,Shvetsov_Sildnes_etal._2024}. We find that for better cell differentiation we have to introduce more granular labels. PanNuke includes a broad classification of "inflammatory" cells, encompassing lymphocytes, macrophages, and neutrophils. Each cell type differs significantly in structure, function, and clinical relevance. Conversely, MoNuSAC uses the label "epithelial" for a class that comprises both benign epithelial cells and malignant neoplastic cells. This practice makes it challenging to differentiate between benign and malignant epithelial cells in the dataset, which is a critical distinction when identifying tumor areas within tissue samples. To address these issues, we implement a cross-relabeling strategy as shown in \hyperref[fig:fig2]{Figure 2}. The key components are two classification models: one is trained on singular cell images from PanNuke data to classify the epithelial meta-class into epithelial and neoplastic classes. The other is trained on MoNuSAC to refine the inflammatory class into lymphocytes, neutrophils, and macrophages.

\begin{figure}[h!]
    \centering
    \includegraphics[width=\textwidth]{images/Figure_2.pdf}
    \caption{Refined dataset generation via cross relabeling}
    \label{fig:fig2}
\end{figure}

The refining approach consists of three consecutive steps. The first is the preprocessing step, in which we extract individual cells from both datasets (\hyperref[fig:fig3]{Figure 3}). The specifics of PanNuke and MoNuSAC patch preparation before cell preprocessing are provided in \hyperref[chap:S1]{Appendix S1}.

\begin{figure}[h!]
    \centering
    \includegraphics[width=\textwidth]{images/Figure_3.pdf}
    \caption{Cell instances preprocessing including (1) cell map extraction, (2) bounding box delineation, (3) adjusting cell boxes and (4) cropping and resizing of cell images}
    \label{fig:fig3}
\end{figure}

During preprocessing, we extract cell type maps from the ground truth label mask and calculate bounding boxes around each cell instance. To accommodate partial cells at patch borders, a common issue in cropped patch images, we employ mirror padding and extend the field of view of the cell label by 15 pixels to capture adjacent cells. We then crop and resize the identified regions to $64 \times 64$ pixels using bicubic interpolation.

The preprocessed PanNuke dataset comprises 68,031 neoplastic and 23,207 epithelial cell images, while MoNuSAC comprises  33,104 lymphocytes, 1,252 neutrophils, and 1,695 macrophages, which we subsequently use in training cell classification models and classifying the cell image data \hyperref[fig:S2]{Appendix Figure S2 (1)}. 

The next step is to train two distinct ResNet50-based classifiers tailored to address the specific labeling challenges inherent in each dataset. We use ResNet50 for classification models due to its proven effectiveness for image classification tasks in histopathology \cite{pan2022reviewmachinelearningapproaches}, and its compatibility with small images. For the PanNuke dataset, we design the classifier, trained on MoNuSAC data, to disaggregate the heterogeneous "inflammatory" cell category into distinct subtypes: lymphocytes, macrophages, and neutrophils. Similarly, for the MoNuSAC dataset, the classifier is trained on PanNuke data and distinguishes between benign and malignant epithelial cells within the overarching "epithelial" label. By applying these targeted classifiers to their respective datasets, we assign more specific labels to individual cell instances, thus enabling us to create a unified dataset.
To ensure a balanced representation of classes, we train both models on datasets that had been equalized to match the size of the least represented class. Thus, we obtain datasets comprising 23,207 samples per class for PanNuke and 1,252 samples per class for MoNuSAC data. Next, we partition both of them into training (70\%), validation (20\%), and testing (10\%) subsets. To mitigate the risk of overfitting, we use a single dropout layer with a rate of p=0.5 in both models and data augmentation using randomized color perturbations, rotation, and horizontal and vertical flipping. We employ AdamW optimizer and the cross-entropy loss function for the training criterion.

To evaluate the two trained models, we measure the classification accuracy on the respective test subsets. The accuracies on the test subset for both classifiers are presented in \hyperref[tab:1]{Table 1}. The PanNuke model achieves an average accuracy of 93.57\%, with higher accuracy for neoplastic cells (96.06\%) compared to epithelial cells (86.26\%). The confusion matrix in Figure A3.1 shows that the model predominantly distinguishes accurately between epithelial and neoplastic tissues, with a substantial number of correct classifications and relatively few misclassifications. The MoNuSAC model demonstrates an average accuracy of 98.92\%, excelling in classifying lymphocytes (99.67\%) and macrophages (94.12\%), with lower performance for neutrophils (85.71\%). The confusion matrix in Figure A3.2 shows that the model identifies lymphocytes and performs reasonably well with macrophages and neutrophils.

\begin{table}[h!]
\renewcommand{\arraystretch}{1.5}
  \centering
  \caption{Cell classification results for PanNuke and MoNuSAC trained models (CI 95\%).}
  \label{tab:1}
  \begin{tabular}{|l|c|c|}
   \hline
   %\rowcolor{gray!30}
    Accuracy               & PanNuke model              & MoNuSAC model              \\
    \hline
    Average      & 0.936 (0.931--0.941)         & 0.989 (0.986--0.993)        \\
    \hline
    Neoplastic   & 0.961 (0.956--0.965)         & -                          \\
    \hline
    Epithelial   & 0.863 (0.849--0.877)         & -                          \\
    \hline
    Lymphocytes  & -                          & 0.997 (0.995--0.999)        \\
    \hline
    Neutrophils  & -                          & 0.857 (0.796--0.918)        \\
    \hline
    Macrophages  & -                          & 0.941 (0.906--0.976)        \\
    \hline
  \end{tabular}
\end{table}

Finally, during the last step, we use the model trained on PanNuke data for epithelial cells in MoNuSAC and the model trained on MoNuSAC for the inflammatory cells class in PanNuke. Specifically, we use classifier models to relabel epithelial cells in MoNuSAC and inflammatory cells in PanNuke data. Then we combine cells with refined labels and the rest of the cells in both datasets to create a refined dataset (\hyperref[fig:S2]{Appendix Figure S2 (2)}). The process of relabeling cells and visualizing them on a patch is shown in \hyperref[fig:fig4]{Figure 4}. The cell counts in the refined dataset are provided in \hyperref[tab:S4]{Appendix Table S4}.

\begin{figure}[h!]
    \centering
    \includegraphics[width=\textwidth, height=0.42\textheight, keepaspectratio]{images/Figure_4.pdf}
    \caption{Cell relabeling procedure for epithelial and inflammatory cell classes}
    \label{fig:fig4}
\end{figure}

%\hfill

Relabeling and combining datasets have been explored in a prior study \cite{Parulekar_Kanwat_etal._2023}, where consecutive fine-tuning on multiple datasets was employed to account for hierarchical class label structures. While the method presented in \cite{Parulekar_Kanwat_etal._2023} is intuitive, it often lacks consistency and requires multiple fine-tuning runs, which can be cumbersome and time-consuming. 
In contrast, cross-relabeling simplifies this process by using specialized classification models tailored to each dataset's specific labeling challenges. This approach provides better transparency and produces a unified dataset encompassing seven distinct cell types across multiple tissue samples, enhancing data diversity for further model training or fine-tuning.

Despite these improvements, cross-relabeling does not entirely resolve issues related to poor labeling quality or the amount of labeled data. Specifically, our results show lower accuracies persist for underrepresented classes, such as macrophages, which may stem from a limited sample availability and intrinsic challenges in distinguishing these cells based solely on H\&E staining. Furthermore, while our method enhances label specificity, it relies on the initial quality of the broad labels; thus, any fundamental inaccuracies in the original annotations can propagate through the relabeling process. Addressing the overall problem of limited data labels may require integrating additional data sources or utilizing complementary immunohistochemical staining methods.
Although the reported performance metrics are obtained from evaluations on the native test sets of each dataset, it is important to note that the primary application of these classifiers is to perform cross-relabeling, where a model trained on one dataset (e.g., PanNuke) is applied to another (e.g., MoNuSAC) and vice versa. We acknowledge that a more systematic evaluation of cross-dataset generalization is needed and could be performed in future work.

Overall, the refined dataset produced by our approach can enhance the supervised training or fine-tuning of cell segmentation and classification models, especially those that utilize pre-trained foundation models to improve feature extraction robustness. In addition, these models can detect nuanced classes that enable researchers to conduct more detailed analyses of biological processes in computational pathology.

\section{Foundation models for robust cell segmentation and classification}

Accurate cell segmentation and classification in digital pathology are hindered by limited labeled data and the fact that conventional CNNs are unable to capture global contextual information due to their local receptive field constraints \cite{Gheflati_Rivaz_2022,Yang_Marcus_etal.}. Traditional approaches in cell quantification have predominantly relied on CNN encoders, such as ResNet50, given their proven effectiveness in semantic segmentation tasks \cite{Deshmane_2023,Graham_Vu_etal._2019,Mukasheva_Koishiyeva_etal._2024,Stringer_Wang_etal._2021}. However, approaches that include fine-tuning of pretrained CNNs, data augmentation, and stain normalization to partially increase data variability and address staining differences often fail to achieve the necessary generalization and robustness across diverse tissue types and staining conditions \cite{G._Wang_W._Li_etal._2018,Gao_Bagci_etal._2018,Karim_El_Khoury_Martin_Fockedey_etal._2021}.

To overcome these challenges, we leverage an encoder-decoder network that uses a foundation model as the encoder and a CNN upsampling decoder (\hyperref[fig:fig5]{Figure 5}) for simultaneous cell segmentation and classification in 2D patches extracted from WSIs. Foundation models with transformer-based architectures are viable alternatives to CNN-based encoders \cite{Shamshad_Khan_etal._2023,Sourget_2023}. They enable the creation of more advanced architectures that can decode or transform learned features more effectively \cite{Chen_Duan_etal._2023,Cheng_Misra_etal._2022,Xie_Wang_etal._2021}.

\begin{figure}[h!]
    \centering
    \includegraphics[width=\textwidth]{images/Figure_5.pdf}
    \caption{UNETR-like model with foundational model as backbone}
    \label{fig:fig5}
\end{figure}

By utilizing a transformer-based encoder, we incorporate global contextual information into the feature extraction process, which is a key advantage of such architectures \cite{Chen_Lu_etal._2021}. This foundation model integration facilitates accurate pixel-wise segmentation and classification without the need for extensive encoder training, thereby potentially improving generalization across varied cellular structures and tissue types.
In our implementation, we employ a modified UNETR \cite{Hatamizadeh_Tang_etal._2021} architecture that combines a vision transformer (ViT) \cite{Dosovitskiy_Beyer_etal._2021} encoder with a CNN-based decoder. The encoder utilizes the pretrained H-Optimus foundation model, which contains 1.1 billion parameters and is trained on over 500,000 H\&E stained WSIs \cite{Saillard_Jenatton_etal._2024}. We extract outputs from four evenly spaced transformer blocks $Z_i$, where $i \in [1, 14, 26, 38]$, to serve as residual connections for the CNN decoder. We select these blocks based on our observation that features from non-adjacent levels of the encoder lead to better overall performance on the test subset.

The CNN decoder upsamples the feature representations, acquired from the transformer blocks, to generate an intermediate vector that is handled by two task-specific layers that generate cell segmentation and classification masks. The first task-specific layer is the ‘Cellpose head’,  which is used to delineate cell instances. The layer generates horizontal and vertical gradient maps to form vector fields that are refined through gradient tracking in a post-processing step using the Cellpose algorithm \cite{Stringer_Wang_etal._2021}, known for its efficacy in cell segmentation tasks and generalizability across multiple domains \cite{Pachitariu_Stringer_2022,Stringer_Pachitariu_2024}. The second task-specific layer is the "Cell type head", which assigns labels to individual pixels. In the post-processing step, we determine the output classification label of each segmented cell instance by majority voting over the labeled pixels that comprise the cell in the segmentation map.

To evaluate model performance and measure the impact of adding a foundation model as backbone, we compare it to a ResNet50-based model. ResNet50 is a widely used solution for encoders in segmentation architectures in the medical domain \cite{Deshmane_2023,Graham_Vu_etal._2019,Mukasheva_Koishiyeva_etal._2024,Stringer_Wang_etal._2021}. For the H-Optimus-based model, we utilize frozen weights for the encoder and only fine-tune the decoder to take advantage of the extensive pre-training of the foundation model. For the ResNet50-based model we start with ImageNet \cite{Deng_Dong_etal.} weights and train both encoder and decoder parts. Hyperparameters for the training step are set to be identical, where possible, for comparable evaluation. 
For this evaluation, we deliberately use the PanNuke dataset to provide a standardized and controlled comparison between the H‑Optimus and ResNet50-based models (\hyperref[fig:S2]{Appendix Figure S2 (3)}). Specifically, we use two of the default PanNuke dataset splits (66\%) for training and validation, and reserve the third split (33\%) for testing.

To address the challenge of cell class imbalance in the PanNuke dataset, which is a common characteristic in most cell-level H\&E patch datasets, both models’ training processes employ a weighted loss function comprising cross-entropy and focal loss \cite{Lin_Goyal_etal._2018}. The focal loss component is adjusted with coefficients derived from each cell class' instance frequency, emphasizing learning from underrepresented classes and enhancing the model's sensitivity to rare but significant cellular patterns. The cross-entropy loss is augmented with spectral decoupling regularization \cite{Pezeshki_Kaba_etal._2021,Pohjonen_Stürenberg_etal._2022} and spatially varying label smoothing \cite{Islam_Glocker_2021}, which potentially stabilizes training and improves generalization in case of complex tissue morphologies. For optimization, we employ the \textit{AdamW} \cite{Loshchilov_Hutter_2019} to counter unbalanced class scenarios, with cosine annealing learning rate scheduler.

We utilize the scikit-learn library \cite{Van_der_Walt_Schönberger_etal._2014} and HoVer-Net \cite{Graham_Vu_etal._2019} implementations of $R^2$ (the coefficient of determination) and $PQ$ (panoptic quality) to evaluate our experiments. Complete mathematical formulations and detailed explanations of these metrics are provided in \hyperref[chap:S5]{Appendix S5}. To compute confidence intervals, we use nonparametric bootstrapping, where after calculating the metric on the full sample, we generated 1000 bootstrap replicates by resampling with replacement and then determined the 95\% confidence intervals as the 2.5th and 97.5th percentiles of the resulting empirical distribution.

%\hfill

The model comparisons are summarized in \hyperref[tab:2]{Table 2}. The H‑Optimus-based model achieves higher $R^2$ across all cell classes compared to the ResNet50-based model, which means that its predictions are more closely aligned with the PanNuke cell counts, indicating a stronger correlation with the observed data. Notably, the improvement of $R^2_{dead}$ may be an indicator of better global contextual representations provided by the foundation model backbone. In terms of segmentation and classification quality combined, measured by the PQ score, the H‑Optimus-based model demonstrates notable improvements across most cell classes. Overall, the average $R^2$ improved from 0.575 to 0.871, while the average $PQ$ score improved from 0.450 to 0.492, demonstrating better performance of the H-Optimus-based model.

\begin{table}[h!]
\renewcommand{\arraystretch}{1.5}
  \centering
  \caption{Cell quantification metrics for baseline and proposed models (CI 95\%).}
  \label{tab:2}
  \begin{tabular}{|l|c|c|}
    \hline
    %\rowcolor{gray!30}
    Metric             & Resnet50-based            & H-optimus-based              \\
    \hline
    $R^2_{neoplastic}$    & 0.681 (0.576--0.769)       & \textbf{0.941 (0.917--0.960)} \\
    \hline
    $R^2_{inflammatory}$  & 0.863 (0.778--0.903)       & \textbf{0.949 (0.918--0.966)} \\
    \hline
    $R^2_{connective}$    & 0.600 (0.488--0.698)       & 0.609 (0.436--0.772)          \\
    \hline
    $R^2_{dead}$          & 0.097 (-11.389--0.669)     & 0.925 (0.404--0.982)          \\
    \hline
    $R^2_{epithelial}$    & 0.635 (0.490--0.747)       & \textbf{0.930 (0.886--0.964)} \\
    \hline
    $PQ_{neoplastic}$       & 0.517 (0.499--0.535)       & \textbf{0.589 (0.575--0.604)} \\
    \hline
    $PQ_{inflammatory}$     & 0.455 (0.429--0.482)       & \textbf{0.528 (0.507--0.549)} \\
    \hline
    $PQ_{connective}$       & 0.416 (0.400--0.431)       & \textbf{0.451 (0.436--0.465)} \\
    \hline
    $PQ_{dead}$             & 0.374 (0.342--0.408)       & 0.292 (0.209--0.365)          \\
    \hline
    $PQ_{epithelial}$       & 0.488 (0.460--0.519)       & \textbf{0.599 (0.579--0.618)} \\
    \hline
  \end{tabular}
\end{table}

Our results  show that integrating the H‑Optimus foundation model within the UNETR architecture enhances the model's ability to segment and classify cells across diverse tissues from PanNuke data. The pretrained transformer encoder provides robust feature representations, resulting in higher average $R^2$ and $PQ$ scores compared to the CNN-based model. This leads to more reliable cell quantification and more accurate downstream analysis. Additionally, the streamlined fine-tuning process reduces computational overhead and training time, making the model more adaptable for new data.

Despite these advancements, the foundation model-based approach does not fully resolve all challenges related to cell segmentation and classification. We observe lower metric scores for underrepresented classes in the training data. Furthermore, foundation models typically encompass billions of parameters, resulting in substantial computational and memory requirements. It therefore poses challenges for deployment in resource-constrained environments, limiting their practical applicability in certain clinical settings.

\section{Model optimization via Knowledge Distillation}

To address the limitations posed by the extensive size of foundation models, we implement knowledge distillation — a model compression technique that leverages the teacher-student paradigm \cite{Hinton_Vinyals_etal._2015}. By training a smaller, more efficient student model to replicate the output of a larger, pre-trained teacher model, we retain performance while significantly reducing the model's complexity and resource requirements (\hyperref[fig:fig6]{Figure 6}).

\begin{figure}[h!]
    \centering
    \includegraphics[width=\textwidth, height=0.45\textheight, keepaspectratio]{images/Figure_6.pdf}
    \caption{Knowledge distillation framework for training a student model using a pre-trained teacher}
    \label{fig:fig6}
\end{figure}

We employ knowledge distillation to compress the H‑Optimus-based teacher model into a more efficient student model. The teacher model is the modified UNETR architecture with the H‑Optimus foundation model described in the previous chapter. The student model is based on a UNet architecture augmented with residual connections and incorporates a smaller ViT encoder with 9 million parameters \cite{Steiner_Kolesnikov_etal._2022,Wightman_2019}. 

First, we fine-tune the teacher model using the refined dataset from the cross-relabeling procedure (Section 2). Initially we train the decoder of the teacher model while keeping the encoder weights frozen. We split the refined dataset into train (70\%), validation (20\%) and test (10\%) subsets (\hyperref[fig:S2]{Appendix Figure S2 (4)}). During fine-tuning, we use the train and validation subsets, while leaving the test subset for model evaluation. We set the training procedure and model hyperparameters to be identical to those that were used to demonstrate the utility of foundation models for the simultaneous cell segmentation and classification task.

Next, we perform knowledge distillation from teacher to student using the refined dataset used to fine-tune the teacher model. The student model is trained to replicate the teacher model's outputs. We utilize a specialized loss function that aligns the student's predicted probability distribution with the teacher's, incorporating the teacher's class probability distribution derived from the output. Following the methodology of Hinton et al. \cite{Hinton_Vinyals_etal._2015}, we experiment with various hyperparameter settings for the temperature ($T$) and the balancing coefficients ($\alpha$ and $\beta$) in the loss function. We vary $T$ from 1 to 20 and adjust $\alpha$ and $\beta$ to balance the distillation and student losses. Through iterative tuning and evaluation, we identify that setting $T=14$, $\alpha=0.3$, and $\beta=0.7$ yields a configuration that converges and closely approximates the teacher model's performance during training.

Finally, we assess the performance of both models using the $R^2$ and $PQ$ (defined in \hyperref[chap:S5]{Appendix S5}) on the test set of the refined dataset (\hyperref[tab:3]{Table 3}). We observe that the 95\% confidence intervals overlap for most cell types, so we cannot claim statistically significant performance differences between the teacher and student models. One exception appears in the neoplastic class. The teacher model produces an $R^2$ of 0.919, while the student model shows an $R^2$ of 0.852. In addition, the student model achieves higher $PQ$ values for the neoplastic and connective classes, though the confidence intervals show overlap.

\begin{table}[h!]
\renewcommand{\arraystretch}{1.5}
  \centering
  \caption{Cell quantification metrics for teacher and distilled student models (CI 95\%).}
  \label{tab:3}
  \begin{tabular}{|l|c|c|}
    \hline
    %\rowcolor{gray!30}
    Metric & Teacher & Student \\
    \hline
    $R^2_{neoplastic}$    & \textbf{0.919} (0.898--0.939) & 0.852 (0.800--0.891) \\
    \hline
    $R^2_{lymphocyte}$    & 0.969 (0.956--0.977)         & 0.969 (0.956--0.978) \\
    \hline
    $R^2_{connective}$    & 0.694 (0.548--0.809)         & 0.618 (0.469--0.741) \\
    \hline
    $R^2_{dead}$          & 0.755 (0.400--0.908)         & 0.424 (0.100--0.731) \\
    \hline
    $R^2_{epithelial}$    & 0.922 (0.870--0.958)         & 0.843 (0.738--0.917) \\
    \hline
    $R^2_{macrophage}$    & 0.384 (-0.369--0.724)        & 0.704 (0.352--0.859) \\
    \hline
    $R^2_{neutrofil}$     & 0.854 (0.578--0.929)         & 0.833 (0.502--0.925) \\
    \hline
    $PQ_{neoplastic}$       & 0.581 (0.569--0.593)         & 0.601 (0.588--0.613) \\
    \hline
    $PQ_{lymphocyte}$       & 0.536 (0.520--0.553)         & 0.563 (0.544--0.579) \\
    \hline
    $PQ_{connective}$       & 0.436 (0.421--0.451)         & 0.457 (0.441--0.474) \\
    \hline
    $PQ_{dead}$             & 0.272 (0.235--0.315)         & 0.279 (0.201--0.369) \\
    \hline
    $PQ_{epithelial}$       & 0.522 (0.500--0.545)         & 0.530 (0.506--0.555) \\
    \hline
    $PQ_{macrophage}$       & 0.524 (0.459--0.588)         & 0.474 (0.405--0.543) \\
    \hline
    $PQ_{neutrofil}$        & 0.541 (0.490--0.592)         & 0.565 (0.522--0.607) \\
    \hline
  \end{tabular}
\end{table}


We further decompose the $PQ$ metric into its $SQ$ and $DQ$ components (\hyperref[tab:S6]{Appendix Table S6}). Both models produce nearly identical $SQ$ values, which indicates that they predict instance boundaries with similar precision. Although the student model shows some improvement in $DQ$ scores for certain classes, the confidence intervals overlap and do not confirm a statistically significant difference.

We observe that the student and teacher models yield comparable detection performance despite the student model using a much smaller and simpler architecture. A model with fewer parameters reduces the risk of overfitting when training data are scarce relative to the model’s complexity \cite{Farias_Ludermir_etal._2022}. The knowledge distillation process also encourages the student model to focus on the most generalizable detection features learned from the teacher. These factors enable the student model to achieve similar detection performance across different cell types.

Additionally, considering the model sizes reported in \hyperref[tab:4]{Table 4}, the distilled model achieves a significant reduction compared to the teacher model, with a 48-fold decrease in parameter count and a 5.5-fold reduction in on-disk size. In inference mode, the teacher model requires 16 GB of VRAM for a batch size of 32, while the distilled model only needs 3 GB of VRAM for the same batch size. These reductions make the distilled model significantly more practical for fine-tuning and deployment in resource-constrained environments.

\begin{table}[h!]
\renewcommand{\arraystretch}{1.5}
  \centering
  \caption{Parameter counts and size of teacher and distilled model}
  \label{tab:4}
  \adjustbox{max width=\textwidth}{%
  \begin{tabular}{|l|c|c|c|}
    \hline
    %\rowcolor{gray!30}
    Metric & H-optimus-based (Teacher) & mobileViT-based (Student) & Magnitude of difference \\
    \hline
    Parameters count       & 1,158,917,906   & \textbf{24,093,393}   & \textbf{48x}  \\
    \hline
    Estimated Total Size (MB) & 87,912       & \textbf{15,935}    & \textbf{5.5x} \\
    \hline
  \end{tabular}%
}
\end{table}

%\hfill

With recent advancements in complex network architectures and the use of pretrained encoders to achieve state-of-the-art performance \cite{Baumann_Dislich_etal._2024,Hörst_Rempe_etal._2024} in cell segmentation and classification tasks, model size, computational complexity, and processing times have increased. This limits the scalability and accessibility of these models. As we demonstrate, this may be mitigated using knowledge distillation. Studies in the field of natural language processing have demonstrated the efficacy of knowledge distillation in retaining the capabilities of the teacher model while achieving significant reductions in size and complexity \cite{Huangpu_Gao_2024,Sun_Yu_etal.}. 

We demonstrate the feasibility of knowledge distillation in digital pathology, specifically for cell segmentation and classification tasks. Moreover, we achieve this performance while also significantly reducing the parameter count. In addressing the challenge of knowledge transfer, we found that distillation from a transformer-based model to a smaller transformer is more straightforward than attempting to map transformer features to CNN blocks. In our experiments, using a CNN-based network as a student results in worse cell quantification performance due to the structural constraints of CNN feature space dimensions. 

Although our primary approach relies on a transformer-based student model that performs well, it can be further optimized to incorporate advantages from CNN architectures. For example, employing alternative techniques such as using ViT adapters \cite{Chen_Duan_etal._2023} or $1 \times 1$ convolutions to adjust feature map sizes may be beneficial for harnessing CNN advantages like enhanced local feature extraction. Moreover, if additional performance improvements are desired, the process can be further enhanced by applying supplementary knowledge distillation techniques, such as self-distillation \cite{Zhang_Song_etal._2019} or online distillation \cite{Houyon_Cioppa_etal._2023}.

Despite these promising results, further validation on independent datasets is necessary to fully understand the model's limitations. Underrepresented classes may pose challenges when addressing complex cases. Pathologists need to validate these models to adopt them in clinical settings. While the distilled models are smaller and more deployable, a technological gap persists because pathologists traditionally rely on established methods for inspecting WSIs and diagnosing diseases. Addressing the complexities involved in deploying models for inference and supporting pathologists in adopting new tools is essential for integrating these models into clinical workflows.

\section{Model integration with QuPath}
Digital pathology tools with graphical user interfaces are essential for visualizing and analyzing WSIs. To make our student model useful in clinical pathology workflows, it needs to be integrated into a tool that enables inspecting regions, creating annotations, and providing quantitative analyses of biomarkers. Therefore, we integrate the trained student model from the previous chapter into the QuPath open‑source platform \cite{Bankhead_Loughrey_etal._2017}. QuPath provides the required annotation, visualization, and analysis tools to interpret complex histological data, including workflows for cell segmentation, classification, and quantification (\hyperref[fig:fig7]{Figure 7}). 

\begin{figure}[h!]
    \centering
    \includegraphics[width=\textwidth]{images/Figure_7.pdf}
    \caption{Visualization of model-generated cell quantification annotations (left) and the corresponding unannotated slide (right) in QuPath}
    \label{fig:fig7}
\end{figure}

To identify the regions in a WSI critical for prognosticating tumor development, such as specific tumor areas or border regions without overlapping healthy tissue, the pathologist uses QuPath to outline these regions. Then, the pathologist initiates a cell segmentation and classification script through the QuPath interface for the selected regions. The resulting annotations and quantified cell information are then directly overlaid onto the WSI in the QuPath interface. Additional design and implementation details are in \hyperref[chap:S7]{Appendix S7}. 

Two common approaches for integrating deep learning models into QuPath are Java‑based native QuPath extensions \cite{Goldsborough_Philps_etal._2024} and the execution of RESTful API requests to a model server coupled with handling the response via an extension, as demonstrated in the application of cell segmentation models applied to immunofluorescence images \cite{Sugawara_2023}. While the community is actively working on these integration strategies, there is currently no universal solution that fully addresses all integration and performance requirements.

Extensions may offer better integration with QuPath, allowing slightly improved performance and more widespread usage of the built-in QuPath models, but they lack the flexibility to customize models and modify their behavior. For example, the newest version of QuPath includes models such as StarDist \cite{Weigert_Schmidt} and InstanSeg \cite{Goldsborough_Philps_etal._2024} that can perform cell segmentation. Both models pose limitations when applied to simultaneous cell segmentation and classification. StarDist performs well only on convex, round shapes by design, whereas some neoplastic, inflammatory, and connective cells exhibit complex and non-convex shapes. InstanSeg provides only semantic segmentation without assigning classes to the segmented cells.

%\hfill

In contrast, our approach offers an alternative integration strategy. It utilizes the paquo library to directly interact with QuPath’s internal application programming interface from within Python. This enables data exchange and processing without the need for intermediate conversion steps and provides greater control over model customization, retraining, and the incorporation of custom processing steps.

The integration of our custom model with QuPath underscores its potential to significantly enhance the diagnostic process by reducing the time burden on pathologists and enabling them to focus on more complex interpretative tasks using familiar software. Leveraging a tool that is already well-established among pathologists increases the likelihood of its adoption into daily clinical workflows. The quantitative data generated through the automated workflow is critical for both clinical decision-making and research, facilitating more accurate biomarker analysis, enabling robust statistical evaluations, and supporting hypothesis generation and testing. Additionally, by streamlining cell segmentation and classification, the tool enhances the scalability and reproducibility of pathological assessments, ultimately contributing to improved diagnostic accuracy and patient outcomes.

\section{Conclusion and future work}

In this study, we address critical challenges in digital pathology and tackle the usability and deployment issues of the developed models in standard computing environments without the need for high-performance computing systems. Our multi-faceted approach encompasses data refinement through cross-relabeling, leveraging foundation models for robust cell segmentation and classification, optimizing model performance via knowledge distillation, and integrating the optimized model into the QuPath software for practical application. This approach is used to construct a capable, versatile, and adjustable model for cell segmentation and classification, with enhanced performance and usability.

\begin{sloppypar}
While our approach shows potential in the field of computational pathology, certain limitations persist. 
For example, our implementation currently exhibits lower performance in detecting macrophages. 
This serves as an instance of the broader challenge of accurately identifying complex cell types. In order to address this issue, extending our approach to incorporate additional data sources, exploring alternative modeling approaches, and integrating other imaging modalities such as immunohistochemical staining may help improve detection accuracy. Moreover, although the distilled model reduces computational demands, integrating advanced deep learning models into clinical practice requires addressing technological gaps and potential resistance to adopting new tools within established diagnostic processes.
\end{sloppypar}

Future work could focus on several key areas to refine the proposed approach and facilitate its adoption in clinical environments. Enhancing the cell-relabeling process with additional datasets \cite{Graham_Jahanifar_etal._2021} could improve the representation of underrepresented cell types and enhance overall model performance. Also, incorporating additional data sources, such as multi-modal imaging or complementary staining methods, may address limitations related to cell type differentiation and class imbalance. Exploring other foundation models \cite{Vorontsov_Bozkurt_etal._2024,Zimmermann_Vorontsov_etal._2024} or introducing additional modalities \cite{Ding_Wagner_etal._2024,Vaidya_Zhang_etal._2025} may provide alternative architectures better suited to specific tasks or offer improved efficiency. Implementing more complex knowledge distillation techniques \cite{Houyon_Cioppa_etal._2023,Zhang_Song_etal._2019} could further optimize the model's performance and adaptability. Additionally, deeper integration with QuPath or other digital pathology software could provide pathologists more control over cell quantification analysis directly within the QuPath interface, thereby increasing accessibility and usability. Such enhancements would not only refine model performance but also ensure greater adaptability and scalability within various clinical environments. Finally, extensive validation of the model by pathologists and benchmarking against independent datasets are essential steps toward establishing the model's reliability and fostering confidence in its clinical utility.

\section*{Acknowledgments} 
This work was funded in part by the Research Council of Norway grant no. 309439 SFI Visual Intelligence, and the North Norwegian Health Authority grant no. HNF1521-20.

\bibliographystyle{IEEEtran}
\begin{sloppypar}
\begin{thebibliography}{99}

\bibitem{chaplot2020neural} Chaplot, Devendra Singh, et al. "Neural topological slam for visual navigation." Proceedings of the IEEE/CVF conference on computer vision and pattern recognition. 2020.

\bibitem{maksymets2021thda} Maksymets, Oleksandr, et al. "Thda: Treasure hunt data augmentation for semantic navigation." Proceedings of the IEEE/CVF International Conference on Computer Vision. 2021.

\bibitem{mezghan2022memory} Mezghan, Lina, et al. "Memory-augmented reinforcement learning for image-goal navigation." 2022 IEEE/RSJ International Conference on Intelligent Robots and Systems (IROS). IEEE, 2022.

\bibitem{al2022zero} Al-Halah, Ziad, Santhosh Kumar Ramakrishnan, and Kristen Grauman. "Zero experience required: Plug \& play modular transfer learning for semantic visual navigation." Proceedings of the IEEE/CVF Conference on Computer Vision and Pattern Recognition. 2022.

\bibitem{ye2021auxiliary} Ye, Joel, et al. "Auxiliary tasks and exploration enable objectgoal navigation." Proceedings of the IEEE/CVF international conference on computer vision. 2021.

\bibitem{chaplot2020object} Chaplot, Devendra Singh, et al. "Object goal navigation using goal-oriented semantic exploration." Advances in Neural Information Processing Systems 33 (2020)

\bibitem{ramakrishnan2022poni} Ramakrishnan, Santhosh Kumar, et al. "Poni: Potential functions for objectgoal navigation with interaction-free learning." Proceedings of the IEEE/CVF Conference on Computer Vision and Pattern Recognition. 2022.

\bibitem{ramrakhya2022habitat} Ramrakhya, Ram, et al. "Habitat-web: Learning embodied object-search strategies from human demonstrations at scale." Proceedings of the IEEE/CVF Conference on Computer Vision and Pattern Recognition. 2022.

\bibitem{mousavian2019visual} Mousavian, Arsalan, et al. "Visual representations for semantic target driven navigation." 2019 International Conference on Robotics and Automation (ICRA). IEEE, 2019.

\bibitem{dhariwal2021diffusion} Dhariwal, Prafulla, and Alexander Nichol. "Diffusion models beat gans on image synthesis." Advances in neural information processing systems 34 (2021)

\bibitem{ho2022classifier} Ho, Jonathan, and Tim Salimans. "Classifier-free diffusion guidance." arXiv preprint arXiv:2207.12598 (2022).

\bibitem{nichol2021glide} Nichol, Alex, et al. "Glide: Towards photorealistic image generation and editing with text-guided diffusion models." arXiv preprint arXiv:2112.10741 (2021)

\bibitem{brooks2023instructpix2pix} Brooks, Tim, Aleksander Holynski, and Alexei A. Efros. "Instructpix2pix: Learning to follow image editing instructions." Proceedings of the IEEE/CVF Conference on Computer Vision and Pattern Recognition. 2023.

\bibitem{fu2023guiding} Fu, Tsu-Jui, et al. "Guiding instruction-based image editing via multimodal large language models." arXiv preprint arXiv:2309.17102 (2023).

\bibitem{geng2024instructdiffusion} Geng, Zigang, et al. "Instructdiffusion: A generalist modeling interface for vision tasks." Proceedings of the IEEE/CVF Conference on Computer Vision and Pattern Recognition. 2024.

\bibitem{zhou2024minedreamer} Zhou, Enshen, et al. "Minedreamer: Learning to follow instructions via chain-of-imagination for simulated-world control." arXiv preprint arXiv:2403.12037 (2024).

\bibitem{zhou2023esc} Zhou, Kaiwen, et al. "Esc: Exploration with soft commonsense constraints for zero-shot object navigation." International Conference on Machine Learning. PMLR, 2023.

\bibitem{yu2023l3mvn} Yu, Bangguo, Hamidreza Kasaei, and Ming Cao. "L3mvn: Leveraging large language models for visual target navigation." 2023 IEEE/RSJ International Conference on Intelligent Robots and Systems (IROS). IEEE, 2023.

\bibitem{gadre2023cows} Gadre, Samir Yitzhak, et al. "Cows on pasture: Baselines and benchmarks for language-driven zero-shot object navigation." Proceedings of the IEEE/CVF Conference on Computer Vision and Pattern Recognition. 2023.

\bibitem{shah2023navigation} Shah, Dhruv, et al. "Navigation with large language models: Semantic guesswork as a heuristic for planning." Conference on Robot Learning. PMLR, 2023.

\bibitem{cai2024bridging} Cai, Wenzhe, et al. "Bridging zero-shot object navigation and foundation models through pixel-guided navigation skill." 2024 IEEE International Conference on Robotics and Automation (ICRA). IEEE, 2024.

\bibitem{yu2023co} Yu, Bangguo, Hamidreza Kasaei, and Ming Cao. "Co-NavGPT: Multi-robot cooperative visual semantic navigation using large language models." arXiv preprint arXiv:2310.07937 (2023).

\bibitem{wu2024voronav} Wu, Pengying, et al. "Voronav: Voronoi-based zero-shot object navigation with large language model." arXiv preprint arXiv:2401.02695 (2024).

\bibitem{qin2023mp5} Qin, Yiran, et al. "Mp5: A multi-modal open-ended embodied system in minecraft via active perception." arXiv preprint arXiv:2312.07472 (2023).

\bibitem{du2024learning} Du, Yilun, et al. "Learning universal policies via text-guided video generation." Advances in Neural Information Processing Systems 36 (2024).

\bibitem{ajay2024compositional} Ajay, Anurag, et al. "Compositional foundation models for hierarchical planning." Advances in Neural Information Processing Systems 36 (2024).

\bibitem{liang2024skilldiffuser} Liang, Zhixuan, et al. "Skilldiffuser: Interpretable hierarchical planning via skill abstractions in diffusion-based task execution." Proceedings of the IEEE/CVF Conference on Computer Vision and Pattern Recognition. 2024.

\bibitem{heusel2017gans} Heusel, Martin, et al. "Gans trained by a two time-scale update rule converge to a local nash equilibrium." Advances in neural information processing systems 30 (2017).

\bibitem{zhang2018unreasonable} Zhang, Richard, et al. "The unreasonable effectiveness of deep features as a perceptual metric." Proceedings of the IEEE conference on computer vision and pattern recognition. 2018.

\bibitem{brown2020language} Brown, Tom B. "Language models are few-shot learners." arXiv preprint arXiv:2005.14165 (2020).

\bibitem{podell2023sdxl} Podell, Dustin, et al. "Sdxl: Improving latent diffusion models for high-resolution image synthesis." arXiv preprint arXiv:2307.01952 (2023).

\bibitem{brohan2022rt} Brohan, Anthony, et al. "Rt-1: Robotics transformer for real-world control at scale." arXiv preprint arXiv:2212.06817 (2022).

\bibitem{brohan2023rt} Brohan, Anthony, et al. "Rt-2: Vision-language-action models transfer web knowledge to robotic control." arXiv preprint arXiv:2307.15818 (2023).

\bibitem{li2024manipllm} Li, Xiaoqi, et al. "Manipllm: Embodied multimodal large language model for object-centric robotic manipulation." Proceedings of the IEEE/CVF Conference on Computer Vision and Pattern Recognition. 2024.

\bibitem{shah2023vint} Shah, Dhruv, et al. "ViNT: A foundation model for visual navigation." arXiv preprint arXiv:2306.14846 (2023).

\bibitem{liu2024visual} Liu, Haotian, et al. "Visual instruction tuning." Advances in neural information processing systems 36 (2024).

\bibitem{hu2021lora} Hu, Edward J., et al. "Lora: Low-rank adaptation of large language models." arXiv preprint arXiv:2106.09685 (2021).

\bibitem{qin2023supfusion} Qin, Yiran, et al. "SupFusion: Supervised LiDAR-camera fusion for 3D object detection." Proceedings of the IEEE/CVF International Conference on Computer Vision. 2023.

\bibitem{qin2024worldsimbench} Qin, Yiran, et al. "Worldsimbench: Towards video generation models as world simulators." arXiv preprint arXiv:2410.18072 (2024).

\bibitem{yu2025gamefactory} Yu, Jiwen, et al. "GameFactory: Creating New Games with Generative Interactive Videos." arXiv preprint arXiv:2501.08325 (2025).

\bibitem{zhou2024code} Zhou, Enshen, et al. "Code-as-Monitor: Constraint-aware Visual Programming for Reactive and Proactive Robotic Failure Detection." arXiv preprint arXiv:2412.04455 (2024).

\bibitem{zhang2024ad} Zhang, Zaibin, et al. "AD-H: Autonomous Driving with Hierarchical Agents." arXiv preprint arXiv:2406.03474 (2024).

\bibitem{wang2024toward} Wang, Chaoqun, et al. "Toward Accurate Camera-based 3D Object Detection via Cascade Depth Estimation and Calibration." arXiv preprint arXiv:2402.04883 (2024).

\bibitem{huang2024story3d} Huang, Yuzhou, et al. "Story3d-agent: Exploring 3d storytelling visualization with large language models." arXiv preprint arXiv:2408.11801 (2024).

\bibitem{savinov2018semi} Savinov, Nikolay, Alexey Dosovitskiy, and Vladlen Koltun. "Semi-parametric topological memory for navigation." arXiv preprint arXiv:1803.00653 (2018).

\bibitem{majumdar2022zson} Majumdar, Arjun, et al. "Zson: Zero-shot object-goal navigation using multimodal goal embeddings." Advances in Neural Information Processing Systems 35 (2022): 32340-32352.

\bibitem{yadav2023offline} Yadav, Karmesh, et al. "Offline visual representation learning for embodied navigation." Workshop on Reincarnating Reinforcement Learning at ICLR 2023. 2023.

\bibitem{yadav2023ovrl} Yadav, Karmesh, et al. "Ovrl-v2: A simple state-of-art baseline for imagenav and objectnav." arXiv preprint arXiv:2303.07798 (2023).

\bibitem{sun2024fgprompt} Sun, Xinyu, et al. "FGPrompt: fine-grained goal prompting for image-goal navigation." Advances in Neural Information Processing Systems 36 (2024).

\bibitem{zhu2017target} Zhu, Yuke, et al. "Target-driven visual navigation in indoor scenes using deep reinforcement learning." 2017 IEEE international conference on robotics and automation (ICRA). IEEE, 2017.

\bibitem{koh2024generating} Koh, Jing Yu, Daniel Fried, and Russ R. Salakhutdinov. "Generating images with multimodal language models." Advances in Neural Information Processing Systems 36 (2024).

\bibitem{krantz2022instance} Krantz, Jacob, et al. "Instance-specific image goal navigation: Training embodied agents to find object instances." arXiv preprint arXiv:2211.15876 (2022).

\bibitem{schulman2017proximal} Schulman, John, et al. "Proximal policy optimization algorithms." arXiv preprint arXiv:1707.06347 (2017).

\bibitem{anderson2018evaluation} Anderson, Peter, et al. "On evaluation of embodied navigation agents." arXiv preprint arXiv:1807.06757 (2018).

\bibitem{lin2024navcot} Lin, Bingqian, et al. "NavCoT: Boosting LLM-Based Vision-and-Language Navigation via Learning Disentangled Reasoning." arXiv preprint arXiv:2403.07376 (2024).

\bibitem{NavGPT} Zhou, Gengze, Yicong Hong, and Qi Wu. "Navgpt: Explicit reasoning in vision-and-language navigation with large language models." Proceedings of the AAAI Conference on Artificial Intelligence.

\bibitem{hahn2021no} Hahn, Meera, et al. "No rl, no simulation: Learning to navigate without navigating." Advances in Neural Information Processing Systems 34 (2021): 26661-26673.

\bibitem{li2025t2isafety} Li, Lijun, et al. "T2ISafety: Benchmark for Assessing Fairness, Toxicity, and Privacy in Image Generation." arXiv preprint arXiv:2501.12612 (2025).

\bibitem{an2024agfsync} An, Jingkun, et al. "AGFSync: Leveraging AI-Generated Feedback for Preference Optimization in Text-to-Image Generation." arXiv preprint arXiv:2403.13352 (2024).


\end{thebibliography}
\end{sloppypar}

\clearpage
\beginsupplement
\section*{Appendix}
\renewcommand{\thesubsection}{S\arabic{subsection}}

\subsection{\label{chap:S1}PanNuke and MoNuSAC preprocessing}
The PanNuke dataset comprises a set of 7,901 RGB patches, each with dimensions of $256 \times 256$ pixels, which we set as the standard patch size for our analysis. In contrast, the MoNuSAC dataset encompasses 294 images of heterogeneous dimensions. To standardize the MoNuSAC images with our experiments, we implement a standardization protocol. Specifically, for images exceeding the dimensions of $256 \times 256$ pixels, we segment them into equal-sized patches and apply mirror padding to the remaining portions to avoid information loss at the peripherals. Patches with dimensions less than $128 \times 128$ pixels are excluded from the dataset due to the insufficient resolution to capture relevant cellular details. For patches where either dimension falls between 128 and 256 pixels, we employ upsampling to achieve the standard patch size. As a result, we obtain a total of 2,823 RGB patches derived from the MoNuSAC dataset for subsequent analysis. For additional details on the MoNuSAC data preparation process, refer to the source code \cite{Shvetsov_2025a}.
\clearpage

\subsection{\label{chap:S2}Data usage for the methodology}

\counterwithin{figure}{subsection}
\renewcommand{\thefigure}{S\arabic{subsection}}

\begin{figure}[h!]
    \centering
    \includegraphics[width=\textwidth, height=0.85\textheight, keepaspectratio]{images/A2.pdf}
    \caption{Overview of the methodology for cross-labeling, dataset refinement, and model comparison. (1) Cross-relabeling - training and testing cell classification models, (2) Cross-relabeling - using cell classification models to create refined dataset, (3) Fine-tuning and training models for comparison, (4) Student knowledge distillation with refined dataset}
    \label{fig:S2}
\end{figure}
\clearpage

\subsection{\label{chap:S3}Confusion matrices for classification models}
\counterwithin{figure}{subsection}
\renewcommand{\thefigure}{S\arabic{subsection}.\arabic{figure}}

\begin{figure}[h!]
    \centering
    \includegraphics[width=\textwidth, height=0.4\textheight, keepaspectratio]{images/A3_1.pdf}
    \caption{Confusion matrix for PanNuke trained model}
    \label{fig:S3.1}
\end{figure}

\begin{figure}[h!]
    \centering
    \includegraphics[width=\textwidth, height=0.4\textheight, keepaspectratio]{images/A3_2.pdf}
    \caption{Confusion matrix for MoNuSAC trained model}
    \label{fig:S3.2}
\end{figure}

\clearpage

\subsection{\label{chap:S4}Datasets cell counts}

\counterwithin{table}{subsection}
\renewcommand{\thetable}{S\arabic{subsection}}

\begin{table}[h!]
\renewcommand{\arraystretch}{2.0}
\centering
\caption{\label{tab:S4}Cell counts for PanNuke, MoNuSAC and refined datasets. Numbers in parentheses indicate preprocessed cell counts for cell classifier models training and testing.}
%\adjustbox{max width=\textwidth}{%
\begin{tabular}{|l|c|c|c|}
\hline
%\rowcolor{gray!30}
Cell type & PanNuke & MoNuSAC & Refined \\
\hline
Neoplastic & 77,403 (68,031) & - & 105,451 \\
\hline
Epithelial & 26,572 (23,207) & - & 29,926 \\
\hline
Epithelial (benign and malignant) & - & 31,402 & - \\
\hline
Inflammatory & 32,276 & - & - \\
\hline
Lymphocytes & - & 37,045 (33,104) & 65,275 \\
\hline
Neutrophils & - & 1,355 (1,252) & 3,833 \\
\hline
Macrophage & - & 1,842 (1,695) & 3,410 \\
\hline
Dead & 2,908 & - & 2,908 \\
\hline
Connective & 50,585 & - & 50,585 \\
\hline
\end{tabular}
%
%}
\end{table}



\clearpage

\subsection{\label{chap:S5}Definition of validation metrics}
\counterwithin{equation}{subsection}
\renewcommand{\theequation}{\arabic{equation}}

\subsubsection{\label{chap:S5.1}R\textsuperscript{2}}
The coefficient of determination, denoted as $R^2$, is a statistical measure that represents the proportion of variance in the dependent variable that is predictable from the independent variables. In the context of cell quantification in pathology, $R^2$ is used to assess how well the predicted quantities of different cell types in a patch align with the actual quantities observed in the ground truth data, with higher values representing more accurate quantification. $R^2$ is defined as
\begin{equation*}
R^2 = 1 - \frac{\sum_{i=1}^n (y_i - \hat{y}_i)^2}{\sum_{i=1}^n (y_i - \bar{y})^2},
\end{equation*}
where $y_i$ represents the actual number of cells of a specific type in the $i$-th image, $\hat{y}_i$ represents the predicted number of cells of that type in the $i$-th image, $\bar{y}$ is the mean of the actual numbers across all images, and $n$ is the total number of images in the dataset.

The $R^2$ metric has a range of $(-\infty, 1]$. An $R^2$ of 1 indicates perfect prediction, where all predicted values exactly match the actual values. An $R^2$ of 0 suggests that the model explains none of the variability of the response data around its mean. If $R^2$ is negative, it indicates that the model performs worse than a model that simply predicts the mean of the actual values for all observations.

\subsubsection{\label{chap:S5.2}PQ}
Panoptic Quality ($PQ$) is a comprehensive metric used to evaluate the performance of segmentation models in tasks that require both instance segmentation and classification. $PQ$ provides a single score that encapsulates both the detection accuracy (i.e., how many objects were correctly identified) and the segmentation quality (i.e., how accurately the objects' boundaries were delineated). This metric is particularly useful in multiclass scenarios where each pixel is classified into distinct categories, such as different cell types in pathology images.

$PQ$ is calculated as the product of two terms: Detection Quality ($DQ$) and Segmentation Quality ($SQ$). It can be expressed as
\begin{equation*}
PQ = DQ \cdot SQ,
\end{equation*}
where
\begin{equation*}
DQ = \frac{TP}{TP + 0.5\, FP + 0.5\, FN},
\end{equation*}
\begin{equation*}
SQ = \frac{\sum_{(p, g) \in \mathcal{M}} IoU(p, g)}{TP}.
\end{equation*}
In these formulas, $TP$ denotes the number of correctly matched instances between ground truth and prediction, $FP$ denotes the predicted instances that have no corresponding ground truth, $FN$ denotes the ground truth instances that were not detected, $IoU(p, g)$ is the Intersection over Union for a pair of matched instances $p$ (prediction) and $g$ (ground truth), and $\mathcal{M}$ is the set of matched pairs.

The $PQ$ metric is calculated for each class and is averaged across classes to provide a global performance measure.

The $PQ$ score has a range of $[0, 1.0]$, where a higher score indicates better performance in both detecting and segmenting the instances correctly. A $PQ$ of 1 signifies perfect identification and segmentation of all instances, whereas a $PQ$ of 0 indicates that no instances were correctly identified and segmented.

\clearpage

\subsection{\label{chap:S6}Segmentation and Detection quality metrics for teacher and student models}

\begin{table}[h!]
\renewcommand{\arraystretch}{2.0}
\centering
\caption{Segmentation and detection quality for student and teacher models (CI 95\%)}
\label{tab:S6}
%\adjustbox{max width=\textwidth}{%
\begin{tabular}{|l|c|c|}
\hline
%\rowcolor{gray!30}
Metric & Teacher & Student \\
\hline
$SQ_{neoplastic}$ & 0.819 (0.815--0.823) & 0.824 (0.819--0.828) \\
\hline
$SQ_{lymphocyte}$ & 0.795 (0.788--0.802) & 0.790 (0.783--0.796) \\
\hline
$SQ_{connective}$ & 0.770 (0.762--0.776) & 0.780 (0.772--0.786) \\
\hline
$SQ_{dead}$ & 0.659 (0.623--0.688) & 0.657 (0.624--0.695) \\
\hline
$SQ_{epithelial}$ & 0.780 (0.770--0.790) & 0.788 (0.779--0.797) \\
\hline
$SQ_{macrophage}$ & 0.788 (0.760--0.810) & 0.757 (0.730--0.783) \\
\hline
$SQ_{neutrofil}$ & 0.782 (0.761--0.801) & 0.775 (0.759--0.792) \\
\hline
$DQ_{neoplastic}$ & 0.706 (0.692--0.719) & 0.727 (0.712--0.741) \\
\hline
$DQ_{lymphocyte}$ & 0.675 (0.656--0.698) & 0.713 (0.691--0.734) \\
\hline
$DQ_{connective}$ & 0.566 (0.546--0.584) & 0.583 (0.565--0.602) \\
\hline
$DQ_{dead}$ & 0.410 (0.361--0.465) & 0.435 (0.306--0.561) \\
\hline
$DQ_{epithelial}$ & 0.668 (0.639--0.694) & 0.673 (0.644--0.702) \\
\hline
$DQ_{macrophage}$ & 0.657 (0.583--0.727) & 0.615 (0.531--0.703) \\
\hline
$DQ_{neutrofil}$ & 0.691 (0.625--0.753) & 0.729 (0.679--0.778) \\
\hline
\end{tabular}
%
%}
\end{table}

\clearpage

\subsection{\label{chap:S7}QuPath integration method}
We adopt an integration strategy leveraging the paquo \cite{Bayer_AG} library, a Python package that enables direct interaction with QuPath’s internal API, thereby facilitating seamless data exchange without intermediate conversion steps. The data processing pipeline (\hyperref[fig:S7]{Appendix Figure S7}) begins with the acquisition of WSIs and their associated annotations from QuPath, which are represented as Shapely \cite{Gillies_Wel_etal._2024} polygons. Utilizing paquo, we directly read, create, and modify these annotations and detections within a QuPath project in the Python environment. Images are then cropped using these polygons and processed by cell segmentation and classification models employing standard vision processing toolkits such as OpenCV, pyvips, and PyTorch. Additionally, QuPath employs Groovy scripts to initiate a Python process that starts the entire pipeline from QuPath graphical interface: fetching polygons, extracting images from them, and running deep learning model inference on the cropped images. 
The results are returned to QuPath, leveraging paquo's Python bindings to manipulate QuPath data while minimizing the computational overhead typically associated with cross-environment communication.

\counterwithin{figure}{subsection}
\renewcommand{\thefigure}{S\arabic{subsection}}

\begin{figure}[h!]
    \centering
    \includegraphics[width=\textwidth]{images/A7.pdf}
    \caption{QuPath integration workflow using Python environment}
    \label{fig:S7}
\end{figure}

Compared to traditional workflows that involve exporting annotations as GeoJSON, classifying them in Python, and reimporting them into QuPath, our approach offers several advantages. We eliminate the need to switch between programming languages, providing a cohesive and streamlined development process entirely within QuPath software and removing the necessity to use other tools. Meanwhile, we avoid storing annotations as intermediate JSON files unless required for external use or archiving. By conducting the entire inference and post-processing workflow within the Python environment, we leverage the power and flexibility of Python libraries for image processing and machine learning. This approach also enables adjustments to any set of labels and models, thereby improving its applicability.

%\hfill

The distilled model and QuPath integration code are packaged into a Docker container, enabling streamlined execution with the Docker engine. Detailed integration code and deployment instructions can be found in the GitHub repository \cite{Shvetsov_2025b}.

Despite these benefits, we acknowledge that the paquo library is a proof‑of‑concept project in its early development stage and has not been tested across all versions of QuPath.

\clearpage

\subsection{\label{chap:S8}Data and code availability statement}
All datasets, models, and code used in this study are publicly available and can be obtained from the repositories listed below. 
The PanNuke \cite{Gamper_Koohbanani_etal._2019} and MoNuSAC \cite{Verma_Kumar_etal._2021} datasets are publicly accessible, and download information along with detailed descriptions can be found in their respective articles. Preprocessing scripts for PanNuke and MoNuSAC data, as well as individual cell extraction scripts, are available on GitHub \cite{Shvetsov_2025a}. The H-Optimus foundation model used in our experiments can be downloaded from the HuggingFace repository \cite{hoptimus2024}, and model information is available on GitHub \cite{Saillard_Jenatton_etal._2024}. In addition, the integration code for QuPath and the distilled model packaged in a Docker container are provided in the repository \cite{Shvetsov_2025b}, and paquo Python library is available from the authors GitHub repository \cite{Bayer_AG}.
\clearpage

\end{document}

\bibliography{main}


\appendix
\renewcommand\thesection{A\arabic{section}}
\renewcommand{\thetable}{A\arabic{table}}
\renewcommand{\thefigure}{A\arabic{figure}}
\renewcommand{\thealgorithm}{A\arabic{algorithm}}










%%%%%%%%%%%%%%%%%%%%%%%%%%%%%%%%%%%%%%%%%%%%%%%%%%%%%%%%%%%%
\section*{Checklist}


% %%% BEGIN INSTRUCTIONS %%%
% The checklist follows the references. For each question, choose your answer from the three possible options: Yes, No, Not Applicable.  You are encouraged to include a justification to your answer, either by referencing the appropriate section of your paper or providing a brief inline description (1-2 sentences). 
% Please do not modify the questions.  Note that the Checklist section does not count towards the page limit. Not including the checklist in the first submission won't result in desk rejection, although in such case we will ask you to upload it during the author response period and include it in camera ready (if accepted).

% \textbf{In your paper, please delete this instructions block and only keep the Checklist section heading above along with the questions/answers below.}
% %%% END INSTRUCTIONS %%%


 \begin{enumerate}


 \item For all models and algorithms presented, check if you include:
 \begin{enumerate}
   \item A clear description of the mathematical setting, assumptions, algorithm, and/or model. [Yes] %[Yes/No/Not Applicable]
   \item An analysis of the properties and complexity (time, space, sample size) of any algorithm. [Yes] %[Yes/No/Not Applicable]
   \item (Optional) Anonymized source code, with specification of all dependencies, including external libraries. [Yes] %[Yes/No/Not Applicable]
 \end{enumerate}


 \item For any theoretical claim, check if you include:
 \begin{enumerate}
   \item Statements of the full set of assumptions of all theoretical results. [Yes] %[Yes/No/Not Applicable]
   \item Complete proofs of all theoretical results. [Yes] %[Yes/No/Not Applicable]
   \item Clear explanations of any assumptions. [Yes] %[Yes/No/Not Applicable]
 \end{enumerate}


 \item For all figures and tables that present empirical results, check if you include:
 \begin{enumerate}
   \item The code, data, and instructions needed to reproduce the main experimental results (either in the supplemental material or as a URL). [Yes] %[Yes/No/Not Applicable]
   \item All the training details (e.g., data splits, hyperparameters, how they were chosen). [Yes] %[Yes/No/Not Applicable]
         \item A clear definition of the specific measure or statistics and error bars (e.g., with respect to the random seed after running experiments multiple times). [Yes] %[Yes/No/Not Applicable]
         \item A description of the computing infrastructure used. (e.g., type of GPUs, internal cluster, or cloud provider). [Yes] %[Yes/No/Not Applicable]
 \end{enumerate}

 \item If you are using existing assets (e.g., code, data, models) or curating/releasing new assets, check if you include:
 \begin{enumerate}
   \item Citations of the creator If your work uses existing assets. [Yes] %[Yes/No/Not Applicable]
   \item The license information of the assets, if applicable. [Not Applicable]
   \item New assets either in the supplemental material or as a URL, if applicable. [Not Applicable]
   \item Information about consent from data providers/curators. [Yes]
   \item Discussion of sensible content if applicable, e.g., personally identifiable information or offensive content. [Not Applicable]
 \end{enumerate}

 \item If you used crowdsourcing or conducted research with human subjects, check if you include:
 \begin{enumerate}
   \item The full text of instructions given to participants and screenshots. [Not Applicable]
   \item Descriptions of potential participant risks, with links to Institutional Review Board (IRB) approvals if applicable. [Not Applicable]
   \item The estimated hourly wage paid to participants and the total amount spent on participant compensation. [Not Applicable]
 \end{enumerate}

 \end{enumerate}


\newpage 
\clearpage
\onecolumn

\begin{algorithm}[!tb]
\caption{SiCL Workflow for Predicting Causal Structures}
\label{alg:workflow}
\begin{algorithmic}
% \STATE {\bfseries Input:} test\_data
% \STATE {\bfseries Output:} predicted\_cpdag
% \STATE
\STATE \textbf{Procedure} INFERENCE(data, target)
\STATE Calculate node features with node encoder
\STATE Calculate pairwise features with pairwise encoder following Sec. \ref{sec:fem}
\IF {target is skeleton }
\STATE Calculate skeleton with Sec. \ref{sec:met:lic}
\ELSE
\STATE Calculate v-structures with Sec. \ref{sec:met:lic}
\ENDIF
\STATE \textbf{End Procedure}
\STATE
\STATE \textbf{Procedure} TRAINING\_PHASE()
\STATE  skeleton\_predictor $\leftarrow$ init\_skeleton\_predictor()

\STATE  Sample graphs and corresponding data
\STATE Training the skeleton predictor with INFERENCE(data, skeleton)
\STATE Training the v-structure predictor with INFERENCE(data, v-structure), with feature encoders fine-tuned from skeleton predictor
\STATE \textbf{End Procedure}
\STATE
% \STATE graph\_distribution $\leftarrow$ DEFINE\_GRAPH\_DISTRIBUTION()
\STATE \textbf{Procedure} TESTING\_PHASE(test\_data)
\STATE  Calculate predicted skeleton with the trained skeleton predictor
\STATE  Calculate predicted v-structures with the trained v-structure predictor
\STATE Combine predicted skeleton and v-structures to obtain predicted CPDAG
\STATE \textbf{End Procedure}
\end{algorithmic}
\end{algorithm}

\begin{figure*}[t]
\centering
    \includegraphics[width=\linewidth]{figures/pairwiseencoder.pdf}
    \caption{Illustration of the pairwise encoder module. \textcolor{black}{In Part \ding{172}, it initializes raw pairwise features. In Part \ding{173}, a unidirectional attention is applied to utilized information from node features and pairwise features. In Part \ding{174}, an MLP and residual connection is used to yield final pairwise features.}}
          % \vspace{-0.15in}
    \label{fig:pem}
    
\end{figure*}

\section{Theoretical Guarantee} \label{sec:app:tg}
% In this section, we present the theoretical analysis on the asymptotically correctness of our model. In Sec. \ref{sec:da}, we provide the necessary definitions and our assumptions of the problem. 
% In Sec. \ref{sec:sl} - \ref{sec:ol}, we prove the asymptotically correctness of the neural network model.
% In Sec. \ref{sec:d}, we discuss the practical superiority of the neural network models.
In this section, we delve into the theoretical analysis concerning the asymptotic correctness of our proposed model with respect to the sample size. Sec. \ref{sec:da} lays out the essential definitions and assumptions pertinent to the problem under study. Following this, from Sec. \ref{sec:sl} to \ref{sec:ol}, we rigorously demonstrate the asymptotic correctness of the neural network model. Finally, in Sec. \ref{sec:d}, we engage in a detailed discussion about the practical advantages and superiority of neural network models.

\subsection{Definitions and Assumptions} \label{sec:da}
As outlined in Sec. \ref{sec:bg}, a Causal Graphical Model is defined by a joint probability distribution $P$ over $d$ random variables $X_1, X_2, \cdots, X_{d}$, and a DAG $G$ with $d$ vertices representing the $d$ variables.
An observational dataset $D$ consists of $n$ records and $d$ columns, which represents $n$ instances drawn i.i.d. from $P$. 
In this work, we assume causal sufficiency:
\begin{Assumption}[Causal Sufficiency] \label{ass:cs}
    There are no latent common causes of any of the variables in the graph. 
\end{Assumption}
Moreover, we assume the data distribution $P$ is Markovian to the DAG $G$:
\begin{Assumption}[Markov Factorization Property]\label{ass:mk}
      Given a joint probability distribution $P$ and a DAG $G, P$ is said to satisfy Markov factorization property w.r.t. $G$ if $P:=$ $P\left(X_1, X_2, \cdots, X_d\right)=\prod_{i=1}^d P\left(X_i \mid \mathrm{pa}_i^G\right)$, where $\mathrm{pa}_i^G$ is the parent set of $X_i$ in $G$.
\end{Assumption}
It is noteworthy that the Markov factorization property is equivalent to the Global Markov Property (GMP) \citep{lauritzen1996graphical}, which is
\begin{Definition}[Global Markov Property (GMP)]
    $P$ is said to satisfy GMP (or Markovian) w.r.t. a DAG $G$ if $X \perp_G Y|Z \Rightarrow X \perp Y| Z$. Here $\perp_G$ denotes d-separation, and $\perp$ denotes statistical independence. 
\end{Definition}
GMP indicates that any d-separation in graph $G$ implies conditional independence in distribution $P$. We further assume that $P$ is faithful to $G$ by:
\begin{Assumption}[Faithfulness]\label{ass:f}
Distribution $P$ is faithful w.r.t. a DAG $G$ if $X \perp Y\left|Z \Rightarrow X \perp_G Y\right| Z$.
\end{Assumption}

\begin{Definition}[Canonical Assumption] \label{ass:ca}
    We say our settings satisfy the canonical assumption if the Assumptions \ref{ass:cs} - \ref{ass:f} are all satisfied.
\end{Definition}
We restate the definitions of skeletons, Unshielded Triples (UTs) and v-strucutres as follows.
\begin{Definition}[Skeleton]
    A skeleton $E$ defined over the data distribution $P$ is an undirected graph where an edge exists between $X_i$ and $X_j$ if and only if $X_i$ and $X_j$ are always dependent in $P$, i.e., $\forall Z \subseteq\left\{X_1, X_2, \cdots, X_d\right\} \backslash \left\{X_i, X_j \right\}$, we have $X_i \nperp X_j | Z$.
\end{Definition}
Under our assumptions, the skeleton is the same as the corresponding undirected graph of $G$ \citep{spirtes2000causation}. 
\begin{Definition}[Unshielded Triples (UTs) and V-structures]
A triple of variables $X, T, Y$ is an Unshielded Triple (UT) denoted as $\langle X, T, Y \rangle$, if $X$ and $Y$ are both adjacent to $T$ but not adjacent to each other in the DAG $G$ or the corresponding skeleton.
It becomes a v-structure denoted as $X \rightarrow T \leftarrow Y$, if the directions of the edges are from $X$ and $Y$ to $T$ in $G$.
\end{Definition}
We introduce the definition of separation set as:
\begin{Definition}[Separation Set]
    For a node pair $X_i$ and $X_j$, a node set $Z$ is a separation set if $X_i \perp X_j | Z $. Under faithfulness assumption, a separation set $Z$ is a subset of variables within the vicinity that d-separates $X_i$ and $X_j$.
\end{Definition}

Finally, we assume a neural network can be used as a universal approximator in our settings.
\begin{Assumption}[Universal Approximation Capability]
    A neural network model can be trained to approximate a function under our settings with arbitary accuracy. \label{ass:uac}
\end{Assumption}

\subsection{Skeleton Learning} \label{sec:sl}
In this section, we prove the asymptotic correctness of neural networks on the skeleton prediction task by constructing a perfect model and then approximating it with neural networks. 
For the sake of convenience and brevity in description, we define the skeleton predictor as follows. 
\begin{Definition}[Skeleton Predictor]
    Given observational data $D$, a skeleton predictor is a predicate function with domain as observational data $D$ and predicts the adjacency between each pair of the vertices.
\end{Definition}
Now we restate the Remark from \cite{ma2022ml4s} as the following proposition. 
It proves the existence of a perfect skeleton predictor by viewing the skeleton prediction step of PC \citep{spirtes2000causation} as a skeleton predictor, which is proved to be sound and complete.
\begin{Proposition}[Existence of a Perfect Skeleton Predictor]
There exists a skeleton predictor that always yields the correct skeleton with sufficient samples in $D$. \label{prop:epsp}
\end{Proposition}
\begin{proof}
    We construct a skeleton predictor $SP$ consisting of two parts by viewing PC \citep{spirtes2000causation} as a skeleton predictor. 
    In the first part, it extracts a pairwise feature $\boldsymbol{x}_{i j}$ for each pair of nodes $X_i$ and $X_j$:
    \begin{align}
            % \boldsymbol{x}_{i j}= \left|\left\{Z | Z \subseteq V \backslash\{v_i, v_j\} \wedge  (v_i \perp v_j \mid Z)\right\}\right|, \label{equ:sp1}
        \boldsymbol{x}_{i j}=\min _{Z \subseteq V \backslash\left\{X_i, X_j\right\}}\left\{X_i \sim X_j \mid Z\right\}, \label{equ:sp1}
    \end{align}
    where $\left\{X_i \sim X_j \mid Z\right\} \in [0, 1] $ is a scalar value that measures the conditional dependency between $X_i$ and $X_j$ given a node subset $Z$. 
    % where $v_i \perp v_j \mid Z$ indicates whether the conditional dependency exists between $v_i$ and $v_j$ given a node subset $Z$. 
    % Intuitively, $\boldsymbol{x}_{i j}$ represents the counts of subsets to make $v_i$ and $v_j$ conditional dependency.
    Consequently, $\boldsymbol{x}_{i j} > 0$ indicates the persistent dependency between the two nodes.
    
    In the second part, it predicts the adjacency based on $\boldsymbol{x}_{i j}$:
    \begin{align}
        \left(X_i,  X_j\right)= \begin{cases} 1 \text { (adjacent) } & \boldsymbol{x}_{i j} \neq 0 \\ 0 \text { (non-adjacent) } & \boldsymbol{x}_{i j} = 0\end{cases} \label{equ:sp2}
    \end{align}

    Now we prove that $SP$ always yields the correct skeleton by proving the absence of false positive predictions and false negative predictions. Here, false positive prediction denotes $SP$ predicts a non-adjacent node pair as adjacent and false negative predictions denote $SP$ predicts an adjacent node pair as non-adjacent.

    \begin{itemize}[leftmargin=*]
        \item \textbf{False Positive.} Suppose $X_i, X_j$ are non-adjacent. Under the Markovian assumption, there exists a set of nodes $Z$ such that $\left\{X_i \sim X_j \mid Z\right\} = 0$ and hence $\boldsymbol{x}_{ij} = 0$. According to Eq. (\ref{equ:sp2}), $SP$ will always predicts them as non-adjacent.
        \item \textbf{False Negative}. Suppose $X_i, X_j$ are adjacent. Under the faithfulness assumption, for any $Z \in V \backslash \left\{X_i, X_j\right\}, \left\{X_i \sim X_j \mid Z\right\} > 0$, which implies $\boldsymbol{x}_{ij} > 0$. Therefore, $SP$ always predicts them as adjacent.
    \end{itemize}

    Therefore, $SP$ never yields any false positive predictions or false negative predictions under the Markovian assumption and faithfulness assumption, i.e., it always yields the correct skeleton.
\end{proof}

% \textbf{Remark:} % The constructed perfect skeleton predictor is not a continuous function. 
% The mapping from $\boldsymbol{x}_{ij}$ to adjacency shown in Eq. \ref{equ:sp2} can be continuously approximated. 
% Therefore, there exists arbitrarily approximate continuous function for the perfect skeleton predictor. 
%The intuition behind is that the conditional dependency test in Eq. (\ref{equ:sp1}) can be estimated by some continuous functions like $G^2$ test \citep{agresti2012categorical} and mutual information . 

With the existence of a perfect skeleton predictor, we prove the correctness of neural network models with sufficient samples under our assumptions.
\begin{Theorem}
Under the canonical assumption and the assumption that neural network can be used as a universal approximator (Assumption \ref{ass:uac}),
there exists a neural network model that always predicts the correct skeleton with sufficient samples in $D$.
\end{Theorem}
\begin{proof}
    From Proposition \ref{prop:epsp}, there exists a perfect skeleton predictor that predicts the correct skeleton. 
    Thus, according to the Assumption \ref{ass:uac}, a neural network model can be trained to approximate the perfect skeleton prediction hence predicts the correct skeleton. 
\end{proof}

\subsection{Orientation Learning} \label{sec:ol}
Similarly to the overall thought process in Sec. \ref{sec:sl}, in this section we prove the asymptotic correctness of neural networks on the v-structure prediction task by constructing a perfect model and then approximating it with neural networks. 
\begin{Definition}[V-structure Predictor]
    Given observational data $D$ with sufficient samples from a $BN$ with vertices $V = \{X_1, \dots, X_p\}$, a v-structure predictor is a predicate function with domain as observational data $D$ and predicts existence of the v-structure for each unshielded triple.
\end{Definition}
The following proposition proves the existence of a perfect v-structure predictor by viewing the orientation step of PC \citep{spirtes2000causation} as a v-structure predictor.
\begin{Proposition}[Existence of a Perfect V-structure Predictor]
Under the Markov assumption and faithfulness assumption, there exists skeleton predictor that always yields the correct skeleton. \label{prop:epvp}
\end{Proposition}
\begin{proof}
    We construct a v-structure predictor $VP$ consisting of two parts by viewing PC \citep{spirtes2000causation} as a v-structure predictor. 
    % In the first part, it extracts a feature $\boldsymbol{z}_{ijk}$ for each UT $\langle X_i, X_k, X_j \rangle$:
    % \begin{align}
    %     \boldsymbol{z}_{ijk} = \frac{\left|\left\{ (X_k, Z) | \{ X_i \sim X_j | Z\} = 0 \wedge X_k \in Z \right\}\right|}{\left|\left\{ Z | \{ X_i \sim X_j | Z\} = 0 \right\}\right|},
    % \end{align}
        In the first part, it extracts a boolean feature $\boldsymbol{z}_{kij}$ for each UT $\langle X_i, X_k, X_j \rangle$:
    \begin{align}
        \boldsymbol{z}_{kij} = (X_k \in Z), \text{ where } Z \text{ is called as a sepset, i.e. } X_i\perp Y_j | Z.  \label{equ:vp}
        % \frac{\left|\left\{ (X_k, Z) | \{ X_i \sim X_j | Z\} = 0 \wedge X_k \in Z \right\}\right|}{\left|\left\{ Z | \{ X_i \sim X_j | Z\} = 0 \right\}\right|},
    \end{align}
    \color{black}

% where $\left\{X_i \sim X_j \mid Z\right\} \in [0, 1] $ is a scalar value that measures the conditional dependency between $X_i$ and $X_j$ given a node subset $Z$, and $|\cdot|$ represents the cardinality of a set. 
% Note that the denominator is always positive because the separation set of a UT always exists (See Lemma 4.1 in \cite{dai2023ml4c}).
Note that the sepset $Z$ always exists because the separation set of a UT always exists (See Lemma 4.1 in \cite{dai2023ml4c}).
% % Moreover, $\boldsymbol{z}_{ijk}$ is either $0$ or $1$ under our assumptions and sufficient samples.
% Intuitively, $\boldsymbol{z}_{ijk}$ represents the proportion of supsets of $X_i$ and $X_j$ that include $X_k$.

In the second part, it predicts the v-structures based on $\boldsymbol{z}_{ijk}$:
% \begin{align}
% \langle X_i, X_k, X_j\rangle = \begin{cases} 0 \text { (not v-structure) } & \boldsymbol{z}_{i j k} \neq 0 \\ 1 \text { (v-structure) } & \boldsymbol{z}_{i j k} = 0\end{cases} \label{equ:vp2}
% \end{align}
\begin{align}
\langle X_i, X_k, X_j\rangle = \begin{cases} 0 \text { (not v-structure) } & \boldsymbol{z}_{k i j } = True \\ 1 \text { (v-structure) } & \boldsymbol{z}_{k i j } = False\end{cases} \label{equ:vp2}
\end{align}
Now we prove that $VP$ always yields the correct predictions of v-structures.
According to Theorem 5.1 on p.410 of \cite{spirtes2000causation}, assuming faithfulness and sufficient samples, if a UT $\langle X_i, X_k, X_j \rangle$ is a v-structure, then $X_k$ does not belong to any separation sets of $(X_i, X_j)$; if a UT $\langle X_i, X_k, X_j \rangle$ is not a v-structure, then $X_k$ belongs to every separation sets of $(X_i, X_j)$. Therefore, we have $\boldsymbol{z}_{kij} = False$ if and only if $X_k$ is not in any separation set of $X_i$ and $X_j$, i.e., $\langle X_i, X_k, X_j \rangle$ is a v-structure.
\color{black}
\end{proof}

With the existence of a perfect v-structure predictor, we prove the correctness of neural network models with sufficient samples under our assumptions.
\begin{Theorem}
Under the canonical assumption and the assumption that neural network can be used as a universal approximator (Assumption \ref{ass:uac}), there exists a neural network model that always predicts the correct v-structures with sufficient samples in $D$.
\end{Theorem}
\begin{proof}
    From Proposition \ref{prop:epsp}, there exists a perfect skeleton predictor that predicts the correct v-structures. 
    Thus, according to the Assumption \ref{ass:uac}, a neural network model can be trained to approximate the perfect v-structure predictions hence predicts the correct v-structures. 
\end{proof}
% \begin{Theorem}
% Under our assumptions, neural network models can predict the correct v-structures.
% \end{Theorem}
% \begin{proof}
%     From Proposition \ref{prop:epvp}, there exists a perfect v-structure predictor that predicts the correct v-structure. 
%     Thus, according to the Assumption \ref{ass:uac}, a neural network model can be trained to predict the correct skeleton. 
% \end{proof}

\subsection{Discussion} \label{sec:d}
In the sections above, we prove the asymptotic correctness of neural network models by constructing theoretically perfect predictors.
These predictors both consist of two parts: feature extractors providing features $\boldsymbol{x}_{ij}$ and $\boldsymbol{z}_{ijk}$, and final predictors of adjacency and v-structures.
Even though they have a theoretical guarantee of the correctness with sufficient samples, it is noteworthy that they are hard to be applied practically.
For example, to obtain $\boldsymbol{x}_{ij}$ in Eq. (\ref{equ:sp1}), we need to calculate the conditional dependency between $X_i$ and $X_j$ given every node subset $Z \subseteq V \backslash\left\{X_i, X_j\right\}$.
Leaving aside the fact that the number of $Z$s itself presents factorial complexity, the main issue is that when $Z$ is relatively large, due to the curse of dimensionality, it becomes challenging to find sufficient samples to calculate the conditional dependency. 
This difficulty significantly hampers the ability to apply the constructed prefect predictors in practical scenarios.

Some existing methods can be interpreted as constructing more practical predictors.
Majority-PC (MPC) \citep{colombo2014order} achieves better performance on finite samples by modifying Eq. (\ref{equ:vp}) - (\ref{equ:vp2}) as:
    \begin{align}
        \boldsymbol{z}_{kij} = \frac{\left|\left\{ (X_k, Z) | \{ X_i \sim X_j | Z\} = 0 \wedge X_k \in Z \right\}\right|}{\left|\left\{ Z | \{ X_i \sim X_j | Z\} = 0 \right\}\right|},
    \end{align}
% Note that the denominator is always positive because the separation set of a UT always exists (See Lemma 4.1 in \cite{dai2023ml4c}).
    and
\begin{align}
    \left\langle X_i, X_k, X_j\right\rangle= \begin{cases}0 \text { (not v-structure) } & \boldsymbol{z}_{i j k} > 0.5 \\ 1(\mathrm{v} \text {-structure) } & \boldsymbol{z}_{i j k} \leq 0.5,\end{cases}
\end{align}
    where $\left\{X_i \sim X_j \mid Z\right\} \in [0, 1] $ is a scalar value that measures the conditional dependency between $X_i$ and $X_j$ given a node subset $Z$, and $|\cdot|$ represents the cardinality of a set. 
\color{black}
Due to its more complex classification mechanism, it achieves better performance empirically. 
However, from the machine learning perspective, features from both the PC and MPC predictors are relatively simple.
As supervised causal learning methods, ML4S \citep{ma2022ml4s} and ML4C \citep{dai2023ml4c} provide more systematic featurizations by manual feature engineering and utilization of powerful machine learning models for classification.
% Even though they achieve better performance in practice, the manual feature engineering is complex.
% In out paper, we use neural networks as universal approximator to learn the prediction of identifiable causal structures.
% SCL with NN 也同时在其他地方讨论,如kenan...
While these methods show enhanced practical efficacy, their manual feature engineering processes are complex. 
In our paper, we utilize neural networks as universal approximators for learning the prediction of identifiable causal structures.
It not only simplifies the procedure but also potentially uncovers more nuanced and complex patterns within the data that manual methods might overlook.
It is noteworthy that the benefits of supervised causal learning using neural networks are also discussed elsewhere, as mentioned in SLdisco \citep{petersen2023causal} and CSIvA \citep{ke2023learning}.



\begin{algorithm}[!tb]
\caption{Post-processing}
\label{alg:post}
\begin{algorithmic}
\STATE {\bfseries Input:} weighted skeleton matrix $S$, weighted V-tensor $U$, threshold for skeleton $\tau_s$, threshold for v-structure $\tau_v$
\STATE {\bfseries Output:} predicted oriented edge set $\texttt{oriEdges}$, predicted skeleton $\texttt{skeleton}$
% \STATE
\STATE \textbf{Step 1:}  \\
// Obtaining a predicted skeleton by thresholding. \\
$\texttt{skeleton} = \{(i, j) | max(S_{ij}, S_{ji}) > \tau_s\}$ \\
// Obtaining raw v-structures $\texttt{vstructs}_\texttt{raw}$ by thresholding. \\
$\texttt{vstructs}_\texttt{raw} = \{(i, j, k) | (i, j) \in \texttt{skeleton} \text{ and } (i, k) \in \texttt{skeleton} \text{ and } (j, k) \notin \texttt{skeleton} \text{ and } max(U_{ijk}, U_{ikj}) > \tau_v\}$ \\
\STATE \textbf{Step 2:}  \\
// V-structure conflict resolving: discard any v-structure if there exists another conflicted v-structure with a higher predicted score, following \citep{dai2023ml4c}. \\
$\texttt{vstructs} = \{(i, j, k) \in \texttt{vstructs}_\texttt{raw}|\forall (i', j', k') \in \texttt{vstructs}_\texttt{raw}, (i' \neq k \text{ and }i' \neq j)\text{ or } (k'\neq i \text{ and } j' \neq i) \text{ or } U_{i'j'k'} < U_{ijk}\}$ 
\STATE \textbf{Step 3:}  \\
// Obtaining the predicted directed edge from $\texttt{vstructs}$. \\
$\texttt{oriEdges}_{\texttt{raw}} = \{(j, i)| \exists k, (i, j, k) \in \texttt{vstructs}\}$ \\
//Set a score for each edge with the highest v-structure's score containing this edge. \\
Set $\{p_{ij}\}$ such that $p_{ij} = max_v U_v$ for $v \in \texttt{oriEdges}_{\texttt{raw}}$ and $v \ni (i, j)$. \\
// If there exist any cycles, remove the edge with the smallest score in each cycle. \\
$\texttt{oriEdges} = \{(i, j) \in \texttt{oriEdges}_{\texttt{raw}} |\forall \text{cycle } C, (i, j) \notin C \text{ or } (\exists (i', j') \in C, p_{ij} > p_{i'j'})\}$ \\
\STATE \textbf{Step 4:} \\
// Meek rules: Add edges to $\texttt{oriEdges}$ for directed edges that (1) introducing the edges does not lead to cycles or new v-structures; (2) adding the opposite edges necessarily leads to cycles or new v-structures. \\
$\texttt{oriEdges} = \texttt{oriEdges} \cup \{(i, j) \in \texttt{skeleton}| (i, j) \text{ complies with Meek rules} \}$ \\
\end{algorithmic}
\end{algorithm}

\section{More Discussions on Identifiability and Causal Assumptions} \label{sec:dica}
\textbf{Advocation of Learning Identifiable Structures under All Settings.}
In this paper, we have to work on a concrete setting for demonstration purpose with concrete identifiable causal structures in this paper.
Nonetheless, we want to emphasize that the very concept of identifiability, as well as its ramifications in SCL, is indeed a general issue that is less bound to the issue of ``which causal structures are identifiable under which assumptions". 
The simple fact that in some situations the causal edge cannot be identified -- no matter what feature can be identified in that case -- this identifiability limit has a general effect on SCL. 
Unless the causal graph/edge itself becomes fully invariant/identifiable (a special case that is important but certainly not universally true), the presence of the identifiability limit entails a fundamental bias for a popular SCL model architecture (i.e., Node-Edge) that cannot be mitigated by larger model or bigger data at all. 
This ``identifiability-limit-causes-learning-error" effect is the main thesis of this paper, and we advocate to design neural networks that focus on learning the identifiable features (no matter what those features are). 
In other words, there is nothing stopping one from studying another setting where another feature is identifiable though, and in that case we would also advocate to learn that feature instead of v-structures. 
For example, if we assume canonical MEC assumptions and non-existence of causal-fork and v-structures, the identifiable causal structure becomes a kind of chains.
In that case, one may want to design neural networks that predict about causal chains. 

\textbf{Rationality of Canonical Assumptions.}
In this paper, we choose the canonical setting under the classic MEC theory, in which the skeleton and v-structures are the identifiable structure.
This setting includes the assumptions of the Markov and faithfulness conditions.
Unlike scenario-specific assumptions, such as those tied to a particular data-generating process, these assumptions are classic assumptions about causality that are often adopted as ``postulates" about some general aspects of the world. For example,
\begin{itemize}
    \item \cite{pearl2009causality} argues that stability (faithfulness) stems from the natural improbability of strict equality constraints among parameters, which aligns with the autonomy of causal mechanisms.
\item \cite{spirtes2001causation} support the Causal Faithfulness Condition (CFC) by noting that the exact cancellation of causal paths is highly improbable under natural conditions.
\item \cite{weinberger2018faithfulness} reinforces this argument, proposing that coincidences leading to CFC violations are rare and lack explanatory power, further justifying its adoption within a general modeling framework.
\end{itemize}
These considerations underscore the rationality and generality of the assumptions, making them a natural choice for our analysis.

\section{Details and Discussion about Post-processing} \label{app:post}
For comprehensive clarity, we provide a clear process about the post-processing algorithm in Alg. \ref{alg:post}.

\textbf{Discussion. }
It is worth noting that our design in post-processing is as conservative as possible. In fact, we simply adhere to the conventions in deep learning (i.e., thresholding) to obtain the skeleton and the initial v-structure set. Subsequently, we follow the conventions in constraint-based causal discovery methods to derive the final oriented edges. Therefore, we have not dedicated extensive efforts towards the meticulous design, nor do we intend to emphasize this aspect of our workflow.

The conflicts and cycles are not unique to SiCL; they are, in fact, common issues encountered by all constraint-based algorithms like PC. Moreover, it's worth noting that they never appear if the networks perform perfect. Therefore, the conflict resolving of v-structures and the removal of cycles are designed as fallback mechanisms to ensure the soundness of our workflow, rather than being central elements of our approach. To illustrate it, we experimented with an opposite variant (Intuitively, this is a bad choice) that prioritizes discarding the v-structure with the higher predicted probability. The minimal differences in outcomes between this variant and its counterpart, as detailed in Tab. \ref{tab:crm}, support our viewpoint that the conflict resolution process is of limited significance within our workflow. On the other hand, experimental results presented in Tab. \ref{tab:ccfc} underscore the infrequency of cycles in the predictions, reinforcing the non-essential nature of the cycle removal component.


\begin{table}[tb]
\centering
\begin{threeparttable}
\caption{The o-F1 comparison between the used conflict resolving method with an opposite variant.}
\label{tab:crm}
\begin{tabular}{@{}ccc@{}}
\toprule
Conflict Resolving Method & WS-L-G & SBM-L-G  \\
\midrule
Original Conflict Resolving & $\mathbf{41.1}$&$\mathbf{83.3}$ \\
Opposite Conflict Resolving & $40.7$ & $83.2$ \\
\bottomrule
\end{tabular}
\end{threeparttable}
\end{table}

\color{black}
\section{Details about Node Feature Encoder} \label{sec:dnf}
Motivated by previous approaches \citep{lorchamortized,ke2023learning}, we employ a transformer-like architecture comprising attention layers over either the observation dimension or the node dimension alternately as the node feature encoder.
Concretely, for the raw node features $\mathcal{F} \in \mathbb{R}^{d \times n \times h}$ corresponding to $d$ nodes and $n$ observations, our goal is to capture the correlations between both different nodes and different observations.
Therefore, we utilize two transformer encoder layers over the observation dimension and the node dimension alternatively:
\begin{align}
\begin{aligned}
    \mathcal{F} & \leftarrow TransformerEncoderLayer(\mathcal{F}, \mathcal{F}, \mathcal{F}) \\
    \mathcal{F} & \leftarrow \mathcal{F}.transpose(0, 1) \\
        \mathcal{F} & \leftarrow TransformerEncoderLayer(\mathcal{F}, \mathcal{F}, \mathcal{F})\\
    \mathcal{F} & \leftarrow \mathcal{F}.transpose(0, 1). \\
\end{aligned}
\end{align}
The above operation is repeated multiple times for sufficiently feature encoding.
It yields the final node feature tensor $\mathcal{F} \in \mathcal{R}^{d \times \times h}$.
\color{black}
\section{Illustration of the Case Study in Sec. \ref{sec:met:lim}}
Fig. \ref{fig:ps} presents an illustration for the case study of the Node-Edge approach in Sec. \ref{sec:met:lim}. 
It clearly shows that observational data with the two different parametrized forms follow the same joint distribution:
\begin{align}
    P(\left[X, Y, T\right]) =\mathcal{N}\left([0,0,0],\left[\begin{array}{lll}1 & 1 & 1 \\ 1 & 3 & 2 \\ 1 & 2 & 2\end{array}\right]\right).
\end{align}
Therefore, the observational datasets coming from the two DAGs are inherently indistinguishable.


    
\begin{figure*}[ht]
    \centering
    \includegraphics[width=0.9\linewidth]{figures/ps.pdf}
    \caption{The problem setting to emphasize the limitations of the Node-Edge approach. \textit{Best viewed in color.}}
    \label{fig:ps}
\end{figure*}


\section{Proof and Discussion for Proposition \ref{prop:star}} \label{sec:mgcs}
\color{black}
We first restate the Proposition \ref{prop:star} with more details and provide the proof.
\begin{Proposition}
Let $\mathcal{G}_n$ be the set of graphs with $n+1$ nodes where there is a central node $y$ such that (1) every other node is connected to $y$, (2) there is no edge between the other nodes, (3) there is at most one edge pointing to $y$. 
For any distribution $Q$ over $\mathcal{G}_n$, let $M(Q)$ be another distribution over $\mathcal{G}_n$ such that for any causal edges $e, e'$, $P_{G\sim Q}(e \in G) = P_{G\sim M(Q)}(e \in G) = P_{G\sim M(Q)}(e \in G | e' \in G)$. We have 
\begin{align}\max_{Q} P_{G \sim M(Q)}(G \nin \mathcal{G}_n) = 
1 - \frac{2n-1}{n-1}(1 - \frac{1}{n})^n.
\end{align}
As a corollary, we have 
\begin{align}
\sup_n \max_{Q} P_{G \sim M(Q)}(G \nin \mathcal{G}_n) = 
1 - \frac{2}{e} \approx 0.2642,
\end{align}
\end{Proposition}
\begin{proof}
Denote other nodes except for the central node as $x_i$ where $i \in \left\{1, 2, \dots, n\right\}$. 
In our setting, the set $\mathcal{G}_n$ contains $n + 1$ DAGs
with the same skeleton and no v-structure
: $G_0: y \rightarrow x_i$ for all $x_i$, and $G_i: y \rightarrow x_j$ for all $x_j \neq x_i$ together with $x_i \rightarrow y$.
Denote the sampling probability of DAG $G_i$ from $\mathcal{G}_n$ as $P_i$.
Therefore, the marginal probability of the edge $y \rightarrow x_i$ is $1 - P_i$.

    % As the Node-Edge model $M$ is trained optimally, the prediction of the existence probability of the edge $y \rightarrow x_i$ is $1 - P_i$.
    
    
    If $\exists i$, $P_i = 1$, it means that $\mathcal{G}_n$ only contains the DAG $G_i$. Therefore, $M(Q)$ is equivalent to $Q$ and we have $P_{G \sim M(Q)}(G \nin \mathcal{G}_n) = 0$.

    If $\forall i$, $P_i < 1$, denoting $Q_i = 1 - P_i$ and $P(v)$ as the probability of $G$ containing no v-structures. In other words, $P(v) = P_{G \sim M(Q)}(G \in \mathcal{G}_n)$. We have 
    \begin{align}
        P(v) = \prod_{i=1}^n Q_i + \sum_{j=1}^n \frac{\prod_i^n Q_i}{Q_j} (1 - Q_j)
        &= ( \prod_{i=1}^n Q_i) \cdot (1 + \sum_{j=1}^{n} \frac{1-Q_j}{Q_j}).
    \end{align}
    As $P(v)$ is a probability, we have $P(v) > 0$. Denoting function 
    \begin{align}
    f(Q_1, Q_2, \dots, Q_n) = \log P(v) = \sum_{i=1}^n \log Q_i + \log (1 + \sum_{j=1}^n \frac{1-Q_j}{Q_j}),
    \end{align} we would like to find its minimum s.t. $\sum_i Q_i \geq n - 1$ and $Q_i \in (0, 1]$.

    Define its Lagrange function \begin{align}
        L(Q_1, Q_2, \dots, Q_n, \lambda) = f + \lambda (n-1-\sum_i Q_i).
    \end{align}

    We have 
    \begin{align}
        \frac{\partial L}{\partial \lambda} = n - 1 - \sum_i Q_i,
    \end{align}
    and 
    \begin{align}
        \frac{\partial L}{\partial Q_i} = \frac{1}{Q_i}(1 - \frac{1}{Q_i(1-n + \sum_{k=1}^{n}\frac{1}{Q_k})}) - \lambda.
    \end{align}

    Now we are going to find the extremums for $L(Q_1, Q_2, \dots, Q_n, \lambda)$.
    \begin{enumerate}
        \item[(1)] If $\lambda = 0$, we have $\forall i$, $\frac{\partial f}{\partial Q_i} = 0$, then
        \begin{align}
            \forall i, Q_i = \frac{1}{(1 - n + \sum_{k=1}^{n} \frac{1}{Q_k})}.
        \end{align}
        It indicates that $\forall i, Q_i = 1$, hence $f = 0$ and $P(v) = 1$.
        \item[(2)] If $\lambda \neq 0$, $\exists i$, we have $\forall i$, $\frac{\partial f}{\partial Q_i} = \lambda$ and $\sum_{i=1}^n = n - 1$. In other words, we have 
        \begin{align}
            \forall i, j, \frac{\partial f}{\partial Q_i} = \frac{\partial f}{\partial Q_j} = \lambda.
        \end{align}
        Define function 
        \begin{align}
        h(Q_i) = \frac{\partial f}{\partial Q_i} = \frac{1}{Q_i}(1 - \frac{1}{Q_i(1-n + \sum_{k=1}^{n}\frac{1}{Q_k})}).
        \end{align}
        we can rewrite the function as 
        \begin{align}
            h(Q_i) = \frac{1}{Q_i}(1 - \frac{1}{1 + AQ_i}),
        \end{align}
        where $A = 1 - n + \sum_{k\neq i} \frac{1}{Q_k} \geq 1 - n + \frac{(n-1)^2}{n-1-Q_i} > 0$.
        Therefore, $h(x)$ is a monotonic function in its domain. 

        It indicates that $\forall i, j$, $Q_i = Q_j = \frac{n-1}{n}$, where $P(v) = \frac{2n-1}{n-1}(1 - \frac{1}{n})^n$.
    \end{enumerate}
    Now we are going to list the boundary points for $f$.
    \begin{enumerate}
        \item[(1)] $\forall i$, $Q_i = 1$, it becomes the first extremum point.
        \item[(2)] $\exists i$, $Q_i$ is approaching to $0$. Due to the constraint of $\sum Q_i \geq n - 1$, other $Q$s are approaching to $1$. We have $\lim_{Q_i \rightarrow 0} f = 0$ and $P(v) = 1$.
    \end{enumerate}
    In conclusion, the maximum point of function $f$ is $\forall i$, $Q_i = \frac{n - 1}{n}$, where 
    \begin{align}
        P(v) = \frac{2n-1}{n-1}(1 - \frac{1}{n})^n,
    \end{align}
    and 
        \begin{align}
        P_{G \sim M(Q)}(G \nin \mathcal{G}_n) = 1- P(v) = 1 - \frac{2n-1}{n-1}(1 - \frac{1}{n})^n.
    \end{align}
\end{proof}
\textbf{Discussion. }It is worth noting that $\mathcal{G}_n$ is exactly the MEC of any graph in $\mathcal{G}_n$.
Hence, $P_{G \sim M(Q)}(G \nin \mathcal{G}_n)$ represents the probability that the graph sampled from $M(Q)$ is incorrect.
It indicates that a Node-Edge model could suffer from an inevitable error rate of $0.2642$ though has been perfectly trained to predict $M(Q)$.
\color{black}
% \section{More General Case Study} \label{sec:mgcs}
% % This section provide a more general case study of the inconsistency probability of the Node-Edge approach.
% % Consider a simulator that generates DAGs for nodes $\{y, x_1, x_2, \dots, x_n\}$ from $n + 1$ DAGs with the same skeleton without any v-structure: $G_0: y \rightarrow x_i$ for all $x_i$, and $G_i: y \rightarrow x_j$ for all $x_j \neq x_i$ together with $x_i \rightarrow y$. These DAGs are from the same MEC, and the parametrized forms are designed to yield same joint distribution for data sampled from all DAGs, making them inherently indistinguishable. Under this setting, we have
% This section provide a more general case study of the inconsistency probability of the Node-Edge approach. 
% Concretely, let $\mathcal{G}$ be the set of graphs where there is a central node $y$ such that (1) every other node is connected to $y$, (2) there is no edge between the other nodes, (3) $y$ has at most one edge pointing to it. Denote $\mathcal{G}_n$ as the subset of $\mathcal{G}$ with graphs containing $n$ non-central nodes. 
% Hence, all $G$s from $\mathcal{G}_n$ share a same MEC denoted as $MEC_{\mathcal{G}_n}$, which indicates that there exists a data distribution $D_n$ Markovian compatible with any graph $G \in \mathcal{G}_n$. 
% We have

% \begin{Proposition}
% Denoting $Q_n$ as a distribution over $\mathcal{G}_n$, for any Node-Edge model $M$ optimally trained on $Q_n$ and $D_n$, denoting $M(D_n)$ as the output graph distribution of $M$ with the given input data $D_n$, we have 
% \begin{align}\max_{Q_n} P_{G \sim M(D_n)}(G \nin MEC_{\mathcal{G}_n}) = 
% 1 - \frac{2n-1}{n-1}(1 - \frac{1}{n})^n.
% \end{align}
% As a corollary, we have 
%     \begin{align}
% \sup_n \max_{Q_n} P_{G \sim M(D_n)}(G \nin MEC_{\mathcal{G}_n}) = 
% 1 - \frac{2}{e} \approx 0.2642.
%     \end{align}
% \end{Proposition}
% \color{black}

% % \begin{Proposition}
% % Suppose models with the Node-Edge approach can be trained optimally to predict the DAGs. Denoting the probability of generating DAG $D_i$ as $P_i$, and the probability of yielding inconsistent predictions as $P(inconsistency)$, we have
% %     \begin{align}
% %         \max_{P_0, P_1, \dots, P_n} P(inconsistency) = 
% %         1 - \frac{2n-1}{n-1}(1 - \frac{1}{n})^n, 
% %     \end{align}
% %     under the condition that $P_0=0$ and $\forall i = 1, 2, \dots, n$, $P_i = \frac{1}{n}$. As a corollary, we have
% %     \begin{align}
% %             \sup_{n, P_0, P_1, \dots, P_n} P(inconsistency) = \lim_{n \rightarrow +\infty} \max_{P_0, P_1, \dots, P_n} P(inconsistency) =  1 - \frac{2}{e} \approx 0.2642.
% %     \end{align}

% % \end{Proposition}
% \begin{proof}
% Denote other nodes except for the central node as $x_i$ where $i \in \left\{1, 2, \dots, n\right\}$. 
% In our setting, the distribution $\mathcal{G}_n$ contains $n + 1$ DAGs with the same skeleton without any v-structure: $G_0: y \rightarrow x_i$ for all $x_i$, and $G_i: y \rightarrow x_j$ for all $x_j \neq x_i$ together with $x_i \rightarrow y$.
% Denote the sampling probability of DAG $D_i$ from $\mathcal{G}_n$ as $P_i$.

%     As the Node-Edge model $M$ is trained optimally, the prediction of the existence probability of the edge $y \rightarrow x_i$ is $1 - P_i$.
    
%     If $\exists i$, $P_i = 1$, it means that $\mathcal{G}_n$ only contains the DAG $G_i$. Therefore, the model $M$ only predicts DAG $G_i$ and we have $P_{G \sim M(D_n)}(G \nin MEC_{\mathcal{G}_n}) = P(G_i \nin MEC(G)) = 0$.

%     If $\forall i$, $P_i < 1$, denoting $Q_i = 1 - P_i$ and $P(consistency)$ as the probability of yielding consistent predictions, i.e., $P(consistency) = P_{G \sim M(D_n)}(G \in MEC_{\mathcal{G}_n})$, we have 
%     \begin{align}
%         P(consistency) = \prod_{i=1}^n Q_i + \sum_{j=1}^n \frac{\prod_i^n Q_i}{Q_j} (1 - Q_j)
%         &= ( \prod_{i=1}^n Q_i) \cdot (1 + \sum_{j=1}^{n} \frac{1-Q_j}{Q_j}).
%     \end{align}
%     As $P(consistency)$ is a probability, we have $P(consistency) > 0$. Denoting function 
%     \begin{align}
%     f(Q_1, Q_2, \dots, Q_n) = \log P(consistency) = \sum_{i=1}^n \log Q_i + \log (1 + \sum_{j=1}^n \frac{1-Q_j}{Q_j}),
%     \end{align} we would like to find its minimum s.t. $\sum_i Q_i \geq n - 1$ and $Q_i \in (0, 1]$.

%     Define its Lagrange function \begin{align}
%         L(Q_1, Q_2, \dots, Q_n, \lambda) = f + \lambda (n-1-\sum_i Q_i).
%     \end{align}

%     We have 
%     \begin{align}
%         \frac{\partial L}{\partial \lambda} = n - 1 - \sum_i Q_i,
%     \end{align}
%     and 
%     \begin{align}
%         \frac{\partial L}{\partial Q_i} = \frac{1}{Q_i}(1 - \frac{1}{Q_i(1-n + \sum_{k=1}^{n}\frac{1}{Q_k})}) - \lambda.
%     \end{align}

%     Now we are going to find the extremums for $L(Q_1, Q_2, \dots, Q_n, \lambda)$.
%     \begin{enumerate}
%         \item[(1)] If $\lambda = 0$, we have $\forall i$, $\frac{\partial f}{\partial Q_i} = 0$, then
%         \begin{align}
%             \forall i, Q_i = \frac{1}{(1 - n + \sum_{k=1}^{n} \frac{1}{Q_k})}.
%         \end{align}
%         It indicates that $\forall i, Q_i = 1$, hence $f = 0$ and $P(consistency) = 1$.
%         \item[(2)] If $\lambda \neq 0$, $\exists i$, we have $\forall i$, $\frac{\partial f}{\partial Q_i} = \lambda$ and $\sum_{i=1}^n = n - 1$. In other words, we have 
%         \begin{align}
%             \forall i, j, \frac{\partial f}{\partial Q_i} = \frac{\partial f}{\partial Q_j} = \lambda.
%         \end{align}
%         Define function 
%         \begin{align}
%         h(Q_i) = \frac{\partial f}{\partial Q_i} = \frac{1}{Q_i}(1 - \frac{1}{Q_i(1-n + \sum_{k=1}^{n}\frac{1}{Q_k})}).
%         \end{align}
%         we can rewrite the function as 
%         \begin{align}
%             h(Q_i) = \frac{1}{Q_i}(1 - \frac{1}{1 + AQ_i}),
%         \end{align}
%         where $A = 1 - n + \sum_{k\neq i} \frac{1}{Q_k} \geq 1 - n + \frac{(n-1)^2}{n-1-Q_i} > 0$.
%         Therefore, $h(x)$ is a monotonic function in its domain. 

%         It indicates that $\forall i, j$, $Q_i = Q_j = \frac{n-1}{n}$, where $P(consistency) = \frac{2n-1}{n-1}(1 - \frac{1}{n})^n$.
%     \end{enumerate}
%     Now we are going to list the boundary points for $f$.
%     \begin{enumerate}
%         \item[(1)] $\forall i$, $Q_i = 1$, it becomes the first extremum point.
%         \item[(2)] $\exists i$, $Q_i$ is approaching to $0$. Due to the constraint of $\sum Q_i \geq n - 1$, other $Q$s are approaching to $1$. We have $\lim_{Q_i \rightarrow 0} f = 0$ and $P(consistency) = 1$.
%     \end{enumerate}
%     In conclusion, the maximum point of function $f$ is $\forall i$, $Q_i = \frac{n - 1}{n}$, where 
%     \begin{align}
%         P(consistency) = \frac{2n-1}{n-1}(1 - \frac{1}{n})^n,
%     \end{align}
%     and 
%         \begin{align}
%         P_{G \sim M(D_n)}(G \nin MEC_{\mathcal{G}_n}) = 1- P(consistency) = 1 - \frac{2n-1}{n-1}(1 - \frac{1}{n})^n.
%     \end{align}
% \end{proof}

% \section{More Related Work} \label{sec:app:rw}
% % For the score-based and continuous optimization methods, we refer to our appendix and \citet{glymour2019review,vowels2022d} for a thorough exploration of this literature. 
% Score-based methods aim to find an optimal DAG according to a predefined score function, subject to combinatorial constraints. 
% These methods employ specific optimization procedures such as forward-backward search GES \citep{chickering2002optimal}, hill-climbing \citep{koller2009probabilistic}, and integer programming \citep{cussens2011bayesian}.
% Continuous optimization methods transform the discrete search procedure into a continuous equality constraint.
% NOTEARS \citep{zheng2018dags} formulates the acyclic constraint as a continuous equality constraint and is further extended by DAG-GNN \citep{yu2019dag}, DECI \citep{geffner2022deep} to support non-linear causal relations. 
% DECI \citep{geffner2022deep} is a flow-based model which can perform both causal discovery and inference on non-linear additive noise data.
% % These methods can be viewed as unsupervised optimization since they do not access additional datasets associated with ground-truth causal relations. 
% Recently, ENCO \citep{lippe2021efficient} is proposed as a continuous optimization method where the edge orientation is modeled as a separate parameter to maintain the acyclicity.
% It is guaranteed to converge to the correct graph if interventions on all variables are available.
%  \textcolor{black}{RL-BIC \citep{Zhu2020Causal} utilizes Reinforcement Learning to search for the optimal DAG.}
% These methods can be viewed as unsupervised since they do not access additional datasets associated with ground truth causal relations.


\section{Experimental Settings} \label{sec:app:exp:set}
% \paragraph{Metrics.} In all tables, $\pm$ indicates that the mean value and maximum deviation of three runs with different random seeds are reported.

% For metrics, F1-scores, accuracy and SHD are general metrics for traditional methods and DNN-based methods.
% However, for DNN-based SCL method, AUC and AUPRC are more reasonable metrics because they avoid the influence of threshold selection.

% In the field of skeleton prediction tasks, the F1 score has emerged as a widely adopted metric due to its ability to effectively balance precision and recall \citep{ding2020reliable,ma2022ml4s}. 
% This metric provides a comprehensive evaluation of the model's performance, particularly in cases where the data distribution is imbalanced.
% Accuracy, another commonly used metric, offers a direct measure of the proportion of misclassified edges within the graph. 
% It can also be interpreted as a normalized version of the Structural Hamming Distance (SHD), which has gained popularity in recent years \citep{ma2022ml4s,lorchamortized,ke2023learning}.

% Considering that deep learning models typically output probabilities rather than discrete labels, the Area Under the Receiver Operating Characteristic Curve (AUC) and the Area Under the Precision-Recall Curve (AUPRC)  are also employed as more robust metrics. 
% These metrics take into account all possible decision thresholds, providing a comprehensive evaluation of the model's performance across various operating points. 
% By incorporating these metrics, researchers can gain a deeper understanding of their model's performance, ultimately contributing to the development of more advanced and reliable skeleton prediction systems that are worthy of recognition in top AI conferences.
%For skeleton prediction task, F1 score is a popularly used metric in the domain \cite{ding2020reliable,ma2022ml4s}. 
% Accuracy is a straightforward metric on the number of misclassified edges in the graph, and can also be seen as a normalized version of SHD, which is another popularly used metric \cite{ma2022ml4s,lorchamortized,ke2023learning}. 
% As the output of deep learning models are probabilities instead of single labels, AUC and AUPRC are two more robust metrics as they consider all possible decision thresholds.

% For the CPDAG prediction task, accuracy is used as a comparison metric, which measures the ratio of misclassified edges in the predicted CPDAG. 
% Following the previous paper \citep{dai2023ml4c}, the F1-scores calculated for identifiable edges and v-structures are also provided for a more comprehensive comparison. 
% Following the ML4C approach, v-structure F1 score, Identifiable edges F1, and SHD are also used for CPDAG prediction evaluation.
% For the CPDAG prediction task, we also utilize accuracy as the comparison metric, which counts the ratio of misclassified edges in the predicted CPDAG. 
% Following the previous paper \cite{dai2023ml4c}, we also utilize the F1-score over identifiable edges and v-structures for a more comprehensive comparison. 
% Following ML4C, use v-structure F1 score, Identifiable edges F1, and SHD on CPDAG prediction. 


% \paragraph{Baselines.} To demonstrate the effectiveness and superiority of the proposed framework, several strong baselines representing multiple categories are selected for comparison. These baselines include:
% \begin{enumerate}
%     \item PC: A classic constraint-based causal discovery algorithm based on conditional independence tests. The version with parallelized optimization is selected \citep{le2016fast}.
%     \item GES: A classic score-based greedy equivalence search algorithm \citep{chickering2002optimal}.
%   %  \item DirectLiNGAM: A function-based learning algorithm \cite{shimizu2011directlingam}.
%   \item NOTEARS: A gradient-based algorithm for linear data models \citep{zheng2018dags}.
%     \item DAG-GNN: A continuous optimization algorithm based on graph neural networks \citep{yu2019dag}.
%     % \item NOTEARS-MLP: A gradient-based algorithm for non-linear data models \citep{zheng2018dags}.
%     \item GOLEM: A more efficient version of NOTEARS \citep{ng2020role}.
%     \item GRAN-DAG: A gradient-based algorithm using neural network modeling for non-linear additive noise data \citep{Lachapelle2020Gradient-Based}. 
%     \item AVICI: A powerful deep learning-based supervised causal learning method \citep{lorchamortized}.
%     \color{black}
% \end{enumerate}
\color{black}
\textbf{Baselines.} To demonstrate the effectiveness and superiority of the proposed framework, 
several representative baselines from multiple categories are selected for comparison. 
The PC algorithm is a classic constraint-based causal discovery algorithm based on conditional independence tests, and the version with parallelized optimization is selected \citep{le2016fast}.
GES, a classic score-based greedy equivalence search algorithm, is also included \citep{chickering2002optimal}.
For continuous optimization methods, we compare with NOTEARS \citep{zheng2018dags}, a representative gradient-based optimization method, and GOLEM \citep{ng2020role}, regarded as a more efficient variant of NOTEARS. 
For neural network based optimization algorithms, we compare with DAG-GNN \citep{yu2019dag}, an optimization algorithm based on graph neural networks, and GRAN-DAG, a gradient-based algorithm using neural network modeling \citep{Lachapelle2020Gradient-Based}.
For DNN-based SCL methods, we compare with AVICI, which is the most related work to ours and regarded as the current state-of-the-art method \citep{lorchamortized}.
\color{black}

\textbf{Implementation Details. }The implementation from gCastle \citep{zhang2021gcastle} is utilized for baselines except the SCL methods (i.e., SLdisco and AVICI). 
For PC algorithm, we employ the Fisher-Z transformation with a significance threshold of $0.05$ for conditional independence tests, which is a prevalent choice in statistical analyses and current PC implementations \citep{zhang2021gcastle,zheng2024causal}.
Our criterion for graph selection in GES experiments is the Gaussian Bayesian Information Criterion (BIC), specifically the $l_\infty$-penalized Gaussian likelihood score. It is used in the original paper \citep{chickering2002optimal}, and remains a favored variant in the literature.
For NOTEARS, adhering to the official implementation's settings, we configure NOTEARS with a maximum of 100 dual ascent steps, and an edge dropping threshold of 0.3. For hyperparameters lacking specific default settings, such as the L1 penalty and loss function type, we default to settings used by gCastle \cite{zhang2021gcastle}, employing an L1 penalty of 0.1 and an L2 loss function.
For DAG-GNN, we utilize hyperparameter settings directly from the original implementation, ensuring consistency with established benchmarks.
For GOLEM and GRAN-DAG, we also use the default setting of gCastle \citep{zhang2021gcastle}.
\color{black}
Note that the CSIvA model \citep{ke2023learning} is also a closely related method, but it is not compared due to the unavailability of its relevant codes and its requirement for interventional data as input. 
The original implementation of SLdisco \cite{petersen2023causal} was developed in R. To enhance compatibility with our data generation and evaluation workflows, we reimplemented the model using PyTorch.
The original AVICI model \citep{lorchamortized} does not support discrete data.
Therefore, we use an embedding layer to replace its first linear layer when using AVICI on discrete data.


\textbf{Synthetic Data.} We randomly generate random graphs from multiple random graph models. For continuous data, following previous work \citep{lorchamortized}, Erdős-Rényi (ER) and Scale-free (SF) are utilized as the training graph distribution $p(G)$.
The degree of training graphs in our experiments varies randomly among 1, 2, and 3.
\textcolor{black}{For testing graph distributions, Watts-Strogatz (WS) and Stochastic Block Model (SBM) are used, with parameters consistent with those in the previous paper \citep{lorchamortized}. }
All synthetic graphs for continuous data contain 30 nodes.
\textcolor{black}{The lattice dimension of Watts-Strogatz (WS) graphs is sampled from $\{2, 3\}$, yielding an average degree of about $4.92$. The average degrees of Stochastic Block Model (SBM) graphs are set at 2, following the settings in the aforementioned paper.}
For discrete data, 11-node graphs are used.
SF is utilized as the training graph distribution $p(G)$ and ER is used for testing.
\textcolor{black}{The synthetic training data is generated in real-time, and the training process does not use the same data repeatedly.}
All synthetic test datasets contain 100 graphs, and the average values of the metrics on the 100 graphs are reported to comprehensively reflect the performance.

For the forward sampling process from graph to continuous data, both the linear Gaussian mechanism and general nonlinear mechanism are applied.
Concretely, the Random Fourier Function mechanism is used for the general nonlinear data following the previous paper \citep{lorchamortized}.
% Therefore, it yields two types of continuous datasets. 
In synthesizing discrete datasets, the Bernoulli distribution is used following previous papers \citep{dai2023ml4c,ma2022ml4s}.

\begin{figure*}
    \centering
    \includegraphics[width=\linewidth]{figures/abl.pdf}
    \caption{Illustration of the architecture comparison of Node-Edge models, SiCL-no-PF and SiCL.}
    \label{fig:abl}
\end{figure*}


\textbf{More Implementation Details and Computational Resources. } The two network modules, i.e., the SPN and VPN, are optimized by Adam optimizer with default hyperparameters. Following previous work \citep{lorchamortized}, the training batch size is set as 20. 
All classic algorithms are run on an AMD EPYC 7V13 CPU, and DNN-based methods are run on Nvidia 1080Ti, A40 and A100 GPUs. 
Training SiCL on a 30-node training set with batch size 20 needs about 60GB memory, and training on a 11-node training set needs about 20GB memory.
The learning rate is $3 \times 10^{-4}$.
The batch size is $15$, and the DNN models are trained for $1.2 \times 10^{5}$ batches by default.
\color{black}

\begin{table*}[t]
\centering
    % \resizebox{\columnwidth}{!}{%
\begin{threeparttable}
\caption{\textbf{General comparison of SiCL and other methods}. The average performance results in three runs are provided for SiCL method. GES takes more than 24 hours per graph on WS-L-G, and SLdicso is
unsuitable on non-linear-Gaussian data, hence the results are not included.}
\label{tab:epder}
\begin{tabular}{@{}ccccccccc@{}}
\toprule
\multirow{2}{*}{Dataset} &\multirow{2}{*}{Method} & \multicolumn{4}{c}{Skeleton Prediction }& \multicolumn{3}{c}{CPDAG Prediction} \\
 && s-F1$\uparrow$ & s-Acc.$\uparrow$ & s-AUC$\uparrow$ & s-AUPRC$\uparrow$ &  v-F1$\uparrow$ & o-F1$\uparrow$ & SHD$\downarrow$\\
 \midrule
\multirow{8}{*}{WS-L-G}&PC & $30.4 $&$ 65.6$&N/A&N/A&$ 15.6$&$16.0 $ &$170.4 $\\
% &GES & * &* &*&*&*&*&*\\
&NOTEARS & $33.3 $  & $ 65.1$&N/A&N/A&$ 27.9$&$31.5$ & $159.8$ \\
&DAG-GNN & $35.5$  & $ 55.4$&N/A&N/A&$32.2$&$32.7$ & $193.7 $ \\
&GRAN-DAG& $16.6$ & $62.1$ & N/A & N/A & $11.7$ & $11.7$ & $170.1$ \\
% &NOTEARS-MLP & $24.6$ & $60.3$ & N/A & N/A & $12.2$ & $11.8$ & $190.0$ \\
&GOLEM& $30.0$ & $63.4$ & N/A & N/A & $15.8$ & $19.3$ & $172.7$ \\
&SLdisco & $0.1$ & $66.0$ & $50.2$ & $34.6$ & $0.0$ & $0.1$ & $147.9$ \\
&AVICI & $39.9 $  & $ 74.0$ &$ 71.5$&$ 62.2$&$ 28.2$&$ 35.8$&$119.2$\\
&SiCL & $\mathbf{44.7} $ & $\mathbf{75.3} $&$\mathbf{73.7} $&$\mathbf{65.4} $&$\mathbf{32.0} $&$\mathbf{38.5} $&$\mathbf{116.1} $\\
 \midrule
\multirow{9}{*}{SBM-L-G}&PC & $58.8$&$90.0$&N/A&N/A &$34.8$&$35.9$&$56.4$\\
&GES & $70.8$&$89.4$ &N/A&N/A&$53.9$&$55.0$&$60.3$\\
&NOTEARS & $80.1$  & $94.5$&N/A&N/A&$76.2$&$77.8$ & $26.7$ \\
&DAG-GNN & $66.2$  & $87.4$&N/A&N/A&$60.3$&$62.5$ & $61.0$ \\
&GRAN-GAG & $22.6$  & $85.9$&N/A&N/A&$13.8$&$14.4$ & $64.7$ \\
% &NOTEARS-MLP & $ 44.7$ & $75.4 $ & N/A & N/A & $ 37.9$ & $39.0 $ & $112.5 $ \\
&GOLEM & $68.5$ &$88.5$ &N/A&N/A&$63.5$&$65.2$&$55.1$\\
&SLdisco & $1.9$ & $85.7$ & $56.3$ & $17.6$ & $0.9$ & $1.2$ & $62.6$ \\
&AVICI & $84.3$  & $96.2$ &$98.1$&$92.7$&$79.1$&$81.6$&$17.7$\\
&SiCL &  $\mathbf{85.8}  $  & $\mathbf{96.4}  $& $\mathbf{98.3}$& $\mathbf{93.4} $&$\mathbf{80.6}  $&$\mathbf{82.7}  $&$\mathbf{17.1}  $\\
\midrule
\multirow{9}{*}{WS-RFF-G}&PC & $36.1 $&$69.9$&N/A&N/A &$ 14.8$&$16.1$&$ 156.9$\\
&GES & $ 41.7$&$66.6$ &N/A&N/A&$21.1 $&$23.6$&$174.1$\\
&NOTEARS & $37.7 $  & $64.6 $& N/A & N/A &$30.9 $&$33.4 $ & $164.4 $ \\
&DAG-GNN & $33.2 $  & $ 65.4$&N/A&N/A&$27.0 $&$28.9 $ & $161.1 $ \\
&GRAN-DAG & $4.7 $  & $ 66.7$&N/A&N/A&$0.8 $&$1.1 $ & $146.9 $ \\
% &NOTEARS-MLP & $52.7 $ & $ 40.2$ & N/A & N/A & $ 44.2$ & $ 47.7$ & $282.8 $ \\
 &GOLEM & $27.6$ &$62.4$ &N/A&N/A&$13.8$&$17.7$&$175.8$\\
&AVICI & $47.7 $  & $75.9$ &$ 76.3$&$ 67.6$&$38.7$&$45.2 $&$110.6 $\\
&SiCL & $\mathbf{51.8}  $  & $ \mathbf{77.4}$ &$\mathbf{81.1}$&$ \mathbf{72.9}$&$ \mathbf{40.3}$&$ \mathbf{46.3}$&$ \mathbf{107.0} $\\
\midrule
\multirow{9}{*}{SBM-RFF-G}&PC & $57.5$&$89.3$&N/A&N/A &$32.7$&$ 34.2$&$60.9$\\
&GES & $56.5 $&$84.9$ &N/A&N/A&$37.0$&$38.0$&$82.4$\\
&NOTEARS & $55.6 $  & $86.2 $&N/A&N/A&$ 46.5$&$ 48.5$ & $66.3 $ \\
&DAG-GNN & $ 47.1$  & $82.1 $&N/A&N/A&$39.0$&$40.6$ & $86.2 $ \\
&GRAN-DAG & $17.4$ &$87.4$ & N/A& N/A &$3.2 $&$3.8$&$58.2$\\
% &NOTEARS-MLP & $48.2$ & $71.0$ & N/A & N/A & $41.0$ & $43.5$ & $130.5$ \\
&GOLEM & $31.1$ &$75.7$ & N/A& N/A &$23.0 $&$24.8$&$112.0$\\
&AVICI & $76.6$  & $ 94.5$ &$ 95.4$&$85.7 $&$69.3$&$72.7 $&$ 27.2$\\
&SiCL & $ \mathbf{82.1}$  & $ \mathbf{95.7}$ &$ \mathbf{97.1}$&$ \mathbf{90.7} $&$ \mathbf{75.7} $&$ \mathbf{78.0}$&$ \mathbf{21.9} $\\
\midrule
\multirow{8}{*}{ER-CPT-MC}&PC & $82.2$&$83.0$ &N/A&N/A&$39.2$&$40.6$&$16.4$\\
&GES & $ 82.1$&$81.8$ &N/A&N/A&$40.4$&$42.4$&$17.1$\\
&NOTEARS & $16.7$  & $74.8$&N/A&N/A&$0.2$&$0.6$& $16.1$ \\
&DAG-GNN & $ 24.8$  &  $73.5$&N/A&N/A&$ 3.4$&$3.7 $ & $15.9 $ \\
&GRAN-DAG & $40.8$ &$77.0$ & N/A& N/A &$6.8 $&$7.3$&$15.6$\\
% &NOTEARS-MLP & $ $ & $ $ & N/A & N/A & $ $ & $ $ & $ $ \\
&GOLEM& $37.6$ & $66.4$ & N/A & N/A & $4.6$ & $9.3$ & $21.9$ \\
&AVICI & $76.9$  & $88.4$ & $93.5 $& $87.9 $&$56.6$&$57.6$&$10.2$\\
&SiCL &  $\mathbf{84.2}$  & $\mathbf{90.1}$&$ \mathbf{96.6}$& $\mathbf{94.0} $& $ \mathbf{58.3}$&$\mathbf{59.9}$&$\mathbf{10.1}$\\
\bottomrule
\end{tabular}
\end{threeparttable}
% }
\end{table*}

\begin{table*}[tb]
    \centering
    % \resizebox{\linewidth}{!}{%
\begin{threeparttable}
\caption{Full ablation study results.}
\label{tab:fcplg}
\begin{tabular}{ccccccccc}
\toprule
Dataset &Method & s-F1$\uparrow$ & s-Acc.$\uparrow$ & s-AUC$\uparrow$ & s-AUPRC$\uparrow$ &  v-F1$\uparrow$ & o-F1$\uparrow$ & SHD$\downarrow$\\
 \midrule
\multirow{3}{*}{WS-L-G}& SiCL-Node-Edge & $39.9 $&$ 74.0$&$71.5$&$62.2$&$ 28.2$&$35.8 $ &$119.2 $\\
&SiCL-no-PF & $ 42.4$  & $ 74.4$ &$ 72.8$&$ 63.5$&$ 30.5$&$ 37.9$ &$118.4 $\\
&SiCL & $\mathbf{44.7} $ & $\mathbf{75.3} $&$\mathbf{73.7} $&$\mathbf{65.4} $&$\mathbf{32.0} $&$\mathbf{38.5} $&$\mathbf{116.1} $\\ \hline
\multirow{3}{*}{SBM-L-G}& SiCL-Node-Edge & $ 84.3$&$ 96.2$&$98.1$&$92.7$&$ 79.1$&$81.6 $ &$17.7 $\\
&SiCL-No-PF & $85.5$  & $\mathbf{96.4}$ &$\mathbf{98.3}$&$93.3$&$79.4$&$82.2$&$17.3$\\
&SiCL &  $\mathbf{85.8}$  & $\mathbf{96.4} $& $\mathbf{98.3} $& $\mathbf{93.4} $&$\mathbf{80.6}  $&$\mathbf{82.7}  $&$\mathbf{17.1}$\\
\bottomrule
\end{tabular}
\end{threeparttable}
% }
\end{table*}
\section{Extra Experimental Results} \label{sec:app:exp:e}



\begin{figure*}[t]
     \centering
     \begin{subfigure}[b]{0.45\textwidth}
         \centering
         \includegraphics[width=\textwidth]{figures/v_struc_ws_font.pdf}
         \caption{WS-LG}
         \label{fig:vws}
     \end{subfigure}
     \hfill
     \begin{subfigure}[b]{0.45\textwidth}
         \centering
         \includegraphics[width=\textwidth]{figures/v_struc_sbm_font.pdf}
         \caption{SBM-LG}
         \label{fig:vsbm}
     \end{subfigure}
     \caption{Variation trends of the test performance of the V-structure Prediction Network on WS-LG and SBM-LG during training.}
        \label{fig:tcs}
\end{figure*}



\subsection{Effectiveness of V-structure Prediction Network} Fig. \ref{fig:tcs} illustrates the test performance trends of the v-structure prediction model on SBM and WS random graphs during the training process. In this model, the feature extractor $FE$ is fine-tuned from the skeleton prediction model. The performance increases rapidly and achieves a relatively high level after just a few initial epochs. This suggests that our v-structure prediction network is capable to predict v-structures, and indicates that the pre-trained pairwise features from the skeleton prediction model are both effective and generalizable.



\color{black}
\subsection{More Evidence on Effectiveness of Pairwise Representation}
To further support the effectiveness of using pairwise representation, we present additional experimental results on different training datasets and test datasets, including ER-L-G, SF-L-G, ER-RFF-G, and SF-RFF-G.
% other three test datasets. The graph distributions are Geometric Random Graphs (GRG) and Scale-free graphs with two different parameters (marked as SF1 and SF2). 
% The structural-equation distribution is random linear (L) and noise distribution is Gaussian distribution (G), following other experimental settings.
For models, we compare SiCL with a variant without pairwise representation, i.e., SiCL-no-PF.
% We also implement another variant as a baseline where the alternative attention in the encoder is removed and marked as SiCL-no-AA, and compare it with another variant without pariwise representation, i.e., SiCL-no-AA-no-PF.

The results are provided in Tab. \ref{tab:mpc}. Models with pairwise representation ourperform the corresponding baseline models under almost all comparisons, further verifying the effectiveness of using pairwise representation in models.



\begin{table}[t]\color{black}
\centering
% \resizebox{\linewidth}{!}{%
\begin{threeparttable}
\caption{\color{black}More performance comparison on the effectiveness of pairwise representation.}\label{sec:mpc}
\label{tab:mpc}
\begin{tabular}{ccccccc}
\toprule 
Training Dataset & Test Dataset & Method & s-F1$\uparrow$ & s-AUC$\uparrow$ & s-AUPRC$\uparrow$ & s-Acc.$\uparrow$ \\
\midrule
\multirow{8}{*}{ER-L-G} & \multirow{2}{*}{ER-L-G} & SiCL-no-PF & 75.7 &84.6  & 83.1 & 78.5\\
 &  & SiCL &  80.2 & 89.6 & 90.1 &82.3 \\ \cline{2-7}
 & \multirow{2}{*}{SF-L-G} & SiCL-no-PF & 74.9 &  92.5& 87.3 &84.1 \\
 &  & SiCL & 79.0 & 96.0 & 93.7 & 87.0\\ \cline{2-7}
  & \multirow{2}{*}{ER-RFF-G} & SiCL-no-PF & 49.5 & 60.5 & 49.1 &58.8 \\
 &  & SiCL & 51.0 & 67.0 & 57.6 & 65.2\\ \cline{2-7}
  & \multirow{2}{*}{SF-RFF-G} & SiCL-no-PF & 40.4 & 57.9 & 38.9 & 57.5\\
 &  & SiCL & 46.4 & 71.4 & 53.7 & 69.0\\ \hline
 \multirow{8}{*}{SF-L-G} & \multirow{2}{*}{ER-L-G} & SiCL-no-PF & 64.6 & 77.3 & 68.7 & 70.7\\
 &  & SiCL & 68.0 & 82.1 & 76.4 & 74.3\\ \cline{2-7}
 & \multirow{2}{*}{SF-L-G} & SiCL-no-PF & 88.5 & 96.7 & 95.0 & 91.2\\
 &  & SiCL & 89.7 & 97.9 & 97.0 & 92.4\\ \cline{2-7}
  & \multirow{2}{*}{ER-RFF-G} & SiCL-no-PF & 44.3 & 62.3 & 50.9 & 58.4\\
 &  & SiCL & 47.0 & 66.2 & 55.8 & 63.3\\ \cline{2-7}
  & \multirow{2}{*}{SF-RFF-G} & SiCL-no-PF & 48.1 & 71.6 & 53.9 & 65.8 \\
 &  & SiCL & 56.0 & 79.6 & 64.8 & 74.2\\ \hline
  \multirow{8}{*}{ER-RFF-G} & \multirow{2}{*}{ER-L-G} & SiCL-no-PF & 64.0 & 73.3 & 65.8 & 67.3 \\
 &  & SiCL & 72.0 & 82.0 & 81.1 & 75.2 \\ \cline{2-7}
 & \multirow{2}{*}{SF-L-G} & SiCL-no-PF & 58.1 & 79.0 & 66.8 & 72.8\\
 &  & SiCL & 70.1 & 88.0 & 83.6 & 80.9\\ \cline{2-7}
  & \multirow{2}{*}{ER-RFF-G} & SiCL-no-PF & 63.2 & 74.3 & 67.7 & 71.0\\
 &  & SiCL & 74.8 & 85.7 & 84.5 & 79.7\\ \cline{2-7}
  & \multirow{2}{*}{SF-RFF-G} & SiCL-no-PF & 56.3 & 78.3 & 65.9 & 75.0 \\
 &  & SiCL & 68.2 & 87.0 & 81.5 & 82.1 \\ \hline
  \multirow{8}{*}{SF-RFF-G} & \multirow{2}{*}{ER-L-G} & SiCL-no-PF & 60.3 &  71.2 & 58.7 & 64.7\\
 &  & SiCL & 65.6 & 78.0 & 72.5 & 70.5\\ \cline{2-7}
 & \multirow{2}{*}{SF-L-G} & SiCL-no-PF & 73.6 & 90.5 & 82.9 & 81.0\\
 &  & SiCL & 79.1 & 94.2 & 90.0 & 85.5\\ \cline{2-7}
  & \multirow{2}{*}{ER-RFF-G} & SiCL-no-PF & 57.7 & 71.4 & 60.7 & 66.8\\
 &  & SiCL & 67.2 & 80.5 & 75.8 & 73.9\\ \cline{2-7}
  & \multirow{2}{*}{SF-RFF-G} & SiCL-no-PF & 74.8 & 90.2 & 82.4 & 83.5\\
 &  & SiCL & 80.4 & 94.2 & 90.5 & 87.3\\ 
\bottomrule 
\end{tabular}
\end{threeparttable}
% }
\end{table}

\subsection{Additional Comparison on DAG Prediction}
We provide an additional comparison with the AVICI baseline on the DAG prediction task. Since SiCL predicts CPDAGs and does not directly produce DAG predictions, we corrected the DAG predictions from AVICI using the edge directions inferred from the CPDAGs predicted by SiCL. The results, summarized in the Table \ref{tab:acdp}, demonstrate that incorporating CPDAG-inferred edge directions improves the DAG prediction metrics. This further confirms the effectiveness and generality of our approach, even in tasks focused on DAG metrics. 
\begin{table}[]
    \centering
        \caption{Additional Comparison on DAG Prediction}
    \label{tab:acdp}
    \begin{threeparttable}
    \begin{tabular}{cccccc}
         \toprule  
         Method & Dataset &F1 Score$\uparrow$ & AUC$\uparrow$ & AUPRC$\uparrow$ & Acc.$\uparrow$ \\ \midrule 
         AVICI & \multirow{2}{*}{WS-L-G} &$38.4$ &$86.3$ &$57.7$ &$85.9$ \\
         SiCL-Corrected AVICI && $35.8$&$87.2$ &$60.5$ &$86.2$  \\
         AVICI & \multirow{2}{*}{SBM-L-G}& $78.1$& $95.8$&$80.5$ &$97.3$ \\
         SiCL-Corrected AVICI & &$81.3$&$98.7$ &$90.8$ & $97.8$ \\
         \bottomrule
    \end{tabular}
        \end{threeparttable}
\end{table}


\subsection{Comparison with Autoregressive models on Inference Time Costs} \label{sec:auto}
To validate that the autoregressive models have a relatively high time costs due to the quadratic number of inference runs w.r.t. number of variables, we reproduce the network architecture of a representative autoregressive model, i.e., CSIvA \citep{ke2023learning}, and compare SiCL with it.
We use the same random input for both the models with increasing number of variables.
The results are provided in Fig. \ref{fig:itc}.
The time costs of the autoregressive model show a fast increasing trend and are much more than costs of SiCL, validating the correctness of our analysis.

\begin{figure}
    \centering
    \includegraphics[width=0.5\linewidth]{figures/inference_time_costs_auto_font.pdf}
    \caption{Comparison between an autoregressive model and SiCL on inference time costs.}
    \label{fig:itc}
\end{figure}

\color{black}
\subsection{Training Data Diversity and Model Generalization} We present experimental evidence that highlights the significant contribution of training data diversity to the model's generalization capabilities, even when applied to out-of-distribution (OOD) datasets. 
To illustrate this, we train one SiCL model on a combined dataset of both SF and ER, and another solely on the SF dataset. 
The comparative performance of these models is detailed in Tab. \ref{tab:trainood}. 
The model trained on the combined ER and SF datasets exhibited markedly better performance, not only on the ER dataset but also on the other two OOD datasets, with only a marginal decrease in performance on the SF dataset. 
These findings suggest that enhancing the diversity of the training data correspondingly improves the model’s ability to generalize and maintain robust performance across novel OOD datasets.

\begin{table*}[tb]\color{black}
    \centering
    \caption{\color{black}Comparison of SiCL models with different training data diversity on skeleton prediction.}
    \label{tab:trainood}
    \begin{subtable}{\linewidth}
      \centering
        \caption{\color{black}Model trained on both ER and SF}
        \begin{tabular}{lcccc}
            \toprule
            Test Dataset & s-F1$\uparrow$     & s-AUC$\uparrow$    & s-AUPRC$\uparrow$  & s-Acc.$\uparrow$    \\
            \midrule
            WS-L-G      & 36.3 & 70.6 & 60.6 & 73.3 \\
            SBM-L-G     & 78.1 & 96.8 & 88.1 & 94.8 \\
            ER-L-G      & 80.7 & 96.0 & 89.2 & 94.7 \\
            SF-L-G      & 84.7 & 98.5 & 93.6 & 95.5 \\
            \bottomrule
        \end{tabular}
    \end{subtable}%
    \\
    \begin{subtable}{\linewidth}
      \centering
        \caption{\color{black}Model trained on SF}
        \begin{tabular}{lcccc}
            \toprule
            Test Dataset & s-F1$\uparrow$     & s-AUC$\uparrow$    & s-AUPRC$\uparrow$  & s-Acc.$\uparrow$    \\
            \midrule
            WS-L-G      & 40.1 & 63.0 & 46.1 & 63.5 \\
            SBM-L-G     & 64.3 & 91.7 & 72.9 & 90.9 \\
            ER-L-G      & 67.1 & 90.4 & 73.9 & 90.8 \\
            SF-L-G      & 87.8 & 98.9 & 95.3 & 96.1 \\
            \bottomrule
        \end{tabular}
    \end{subtable}
\end{table*}


\subsection{Varying Amount of Training Graphs}
We present an analysis of how varying the amount of the training graphs influences performance on the skeleton prediction task. The results, depicted in Fig. \ref{fig:ts}, illustrate a clear trend: model performance improves in tandem with the expansion of the training dataset. This trend underscores the potential of our method to achieve even greater accuracy given a more extensive dataset.
\begin{figure*}[!ht]
     \centering
     \begin{subfigure}[b]{0.45\textwidth}
         \centering
    \includegraphics[width=\linewidth]{figures/wstrainingsize_font.pdf}
         \caption{\color{black}WS dataset}
         \label{fig:ts1}
     \end{subfigure}
     \hfill
     \begin{subfigure}[b]{0.45\textwidth}
         \centering
    \includegraphics[width=\linewidth]{figures/sbmtrainingsize_font.pdf}
         \caption{\color{black}SBM dataset}
         \label{fig:ts2}
     \end{subfigure}
         \caption{\color{black}Model performance with varying amount of training graphs.}
        \label{fig:ts}
\end{figure*}

\subsection{Varying Sample Size} We assess SiCL across various quantities of observational samples per graph during testing (100, 200, ..., 1000). The outcomes for both the skeleton prediction task and the CPDAG prediction task are depicted in Fig. \ref{fig:vtss}. It is evident that the model's performance enhances with the augmentation of sample size. These consistent upward trends suggest that SiCL exhibits stability and is not overly sensitive to changes in sample size.

\begin{figure*}[t]
     \centering
     \begin{subfigure}[b]{0.45\textwidth}
         \centering
         \includegraphics[width=\textwidth]{figures/ws1_font.pdf}
         \caption{\color{black}Variation trends of skeleton predicton task performance on WS graph with varying sample sizes.}
         \label{fig:vtss1}
     \end{subfigure}
     \hfill
     \begin{subfigure}[b]{0.45\textwidth}
         \centering
         \includegraphics[width=\textwidth]{figures/ws2_font.pdf}
         \caption{\color{black}Variation trends of CPDAG predicton task performance on WS graph with varying sample sizes.}
         \label{fig:vtss2}
     \end{subfigure}
         \begin{subfigure}[b]{0.45\textwidth}
         \centering
         \includegraphics[width=\textwidth]{figures/sbm1_font.pdf}
         \caption{\color{black}Variation trends of skeleton predicton task performance on SBM graph with varying sample sizes.}
         \label{fig:vtss3}
     \end{subfigure}
     \hfill
     \begin{subfigure}[b]{0.45\textwidth}
         \centering
         \includegraphics[width=\textwidth]{figures/sbm2_font.pdf}
         \caption{\color{black}Variation trends of CPDAG predicton task performance on SBM graph with varying sample sizes.}
         \label{fig:vtss4}
     \end{subfigure}
         \caption{\color{black}Variation trends of performance with varying sample sizes.}
        \label{fig:vtss}
\end{figure*}

\subsection{Varying Edge Density}
We evaluate SiCL over a range of edge densities in the test graphs, utilizing the SBM dataset, as it allows for the direct setting of average edge densities. The findings are presented in Fig. \ref{fig:vted}. It's apparent that the task is becomes more difficult as edge densities increase. However, the performance decline is not abrupt, indicating that SiCL's performance remains relatively stable across various edge densities, thereby confirming its versatility.
\begin{figure*}
     \centering
     \begin{subfigure}[b]{0.45\textwidth}
         \centering
         \includegraphics[width=\textwidth]{figures/sbmskeletondensity_font.pdf}
         \caption{\color{black}Variation trends of skeleton predicton task performance on SBM graph with varying edge densities.}
         \label{fig:vted1}
     \end{subfigure}
     \hfill
     \begin{subfigure}[b]{0.45\textwidth}
         \centering
         \includegraphics[width=\textwidth]{figures/sbmcpdagdensity_font.pdf}
         \caption{\color{black}Variation trends of CPDAG predicton task performance on SBM graph with varying edge densities.}
         \label{fig:vted2}
     \end{subfigure}
         \caption{\color{black}Variation trends of performance with varying edge densities.}
        \label{fig:vted}
\end{figure*}

\subsection{Generality on Testing Graph Sizes}
We offer an analytical perspective on the performance of the SiCL model when applied to larger WS-L-G graphs. 
It is important to highlight that the models were initially trained on graphs comprising 30 vertices, positioning this task within an out-of-distribution setting in terms of graph size. 
To establish a point of reference, we have included results from the PC algorithm as a baseline comparison.
These findings can be examined in Tab. \ref{tab:mpva}.
Despite the OOD conditions, SiCL maintains robust performance, reinforcing its scalability and the model's general applicability across varying graph sizes.


\begin{table}[t]\color{black}
\centering
% \resizebox{\linewidth}{!}{%
\begin{threeparttable}
\caption{\color{black}Performance comparison with varying amounts of graph sizes.}
\label{tab:mpva}
\begin{tabular}{l|ccc|ccc|ccc}
\toprule
Metric & \multicolumn{3}{c|}{s-F1$\uparrow$} & \multicolumn{3}{c|}{v-F1$\uparrow$} & \multicolumn{3}{c}{o-F1$\uparrow$} \\
Size & 50 & 70 & 100 & 50 & 70 & 100 & 50 & 70 & 100 \\
\midrule
PC       & $17.7$ & $14.8$ & $10.6$ & $6.4$ & $5.0$ & $3.7$ & $7.0$ & $5.6$ & $4.0$ \\
SiCL & $\mathbf{41.6}$ & $\mathbf{37.4}$ & $\mathbf{28.3}$ & $\mathbf{34.9}$ & $\mathbf{30.7}$ & $\mathbf{22.6}$ & $\mathbf{37.9}$ & $\mathbf{33.7}$ & $\mathbf{24.8}$ \\
\bottomrule
\end{tabular}
\end{threeparttable}
% }
\end{table}



\subsection{Acyclicity}
% \begin{wraptable}[8]{r}{9cm}
\begin{table}[t]\color{black}
\centering
% \resizebox{\linewidth}{!}{%
\begin{threeparttable}
\caption{\color{black}Count of cycles in the CPDAG predictions without post-processing of removing cycles.}
\label{tab:ccfc}
\begin{tabular}{@{}ccc@{}}
\toprule
Dataset & WS-L-G & SBM-L-G  \\
\midrule
Rate of Graphs with Cycles & $0.66 \pm 0.66 \%$&$0.00 \pm 0.00 \%$ \\
\bottomrule
\end{tabular}
\end{threeparttable}
% }

\end{table}
We provide an empirical evidence supporting of the rarity of cycles in the predictions. The experimental data presented in Tab. \ref{tab:ccfc} corroborates that cycles are infrequently observed in the predicted CPDAGs, even though without any post-processing on removing cycles.
\color{black}



\end{document}
