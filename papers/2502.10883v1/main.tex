\documentclass[twoside]{article}

\usepackage[accepted]{aistats2025}
\usepackage{enumitem}
\usepackage{pifont}
\usepackage{threeparttable}
\usepackage{multirow}
\usepackage{fdsymbol}
\usepackage{caption}
\usepackage{subcaption}
\usepackage{graphicx}
\usepackage{wrapfig}
\usepackage{fdsymbol}
\usepackage{enumitem}
\usepackage{color}
\usepackage{caption}
\usepackage{subcaption}
\usepackage{hyperref}
\usepackage{url}
\usepackage{soul}
\usepackage{amsthm}
\usepackage{pifont}
\usepackage{array}
\usepackage{algorithm}
\usepackage{algorithmic}
\usepackage{booktabs}

\newtheorem{Theorem}{Theorem}[section]
\newtheorem{Definition}{Definition}[section]
\newtheorem{Proposition}{Proposition}[section]
\newtheorem{Assumption}{Assumption}[section]

\theoremstyle{remark}
\newtheorem*{remark}{Remark}
\usepackage[round]{natbib}
\renewcommand{\bibname}{References}
\renewcommand{\bibsection}{\subsubsection*{\bibname}}

\bibliographystyle{apalike}

\begin{document}

\renewcommand{\thefootnote}{\fnsymbol{footnote}}
\runningauthor{J. Zhang, R. Ding, Q. Fu, B. Huang, Z. Deng, Y. Hua, H. Guan, S. Han, D. Zhang}

\twocolumn[

\aistatstitle{Learning Identifiable Structures Helps Avoid Bias in DNN-based Supervised Causal Learning}

\aistatsauthor{ Jiaru Zhang\footnotemark{} \And Rui Ding\footnotemark{} \And Qiang Fu  }
\aistatsaddress{ Shanghai Jiao Tong University \\\texttt{jiaruzhang@sjtu.edu.cn} \And  Microsoft \\\texttt{juding@microsoft.com} \And  Microsoft\\ \texttt{qifu@microsoft.com} } 
\aistatsauthor{Huang Bojun \And Zizhen Deng\And   Yang Hua }
\aistatsaddress{Sony Research \\ \texttt{bojhuang@gmail.com} \And Peking University \\ \texttt{dengzizhen557@outlook.com}  \And Queen’s University Belfast \\ \texttt{y.hua@qub.ac.uk}} 
\aistatsauthor{ Haibing Guan \And Shi Han \And Dongmei Zhang }
\aistatsaddress{Shanghai Jiao Tong University \\ \texttt{hbguan@sjtu.edu.cn}\And Microsoft\\ \texttt{shihan@microsoft.com} \And Microsoft \\ \texttt{dongmeiz@microsoft.com}} 
]

\footnotetext[1]{The work was done during his internship at Microsoft Research Asia.}
\footnotetext[2]{Corresponding author.}

\renewcommand{\thefootnote}{\arabic{footnote}}
\setcounter{footnote}{0}
\begin{abstract}
Causal discovery is a structured prediction task that aims to predict causal relations among variables based on their data samples.
Supervised Causal Learning (SCL) is an emerging paradigm in this field.
Existing Deep Neural Network (DNN)-based methods commonly adopt the “Node-Edge approach”,
in which the model first computes an embedding vector for each variable-node, then uses these variable-wise representations to concurrently and independently predict for each directed causal-edge.
In this paper, we first show that this architecture has some systematic bias that cannot be mitigated regardless of model size and data size. 
We then propose SiCL, a DNN-based SCL method that predicts a skeleton matrix together with a v-tensor (a third-order tensor representing the v-structures). According to the Markov Equivalence Class (MEC) theory, both the skeleton and the v-structures are \emph{identifiable} causal structures under the canonical MEC setting, so predictions about skeleton and v-structures do not suffer from the identifiability limit in causal discovery, thus SiCL can avoid the systematic bias in Node-Edge architecture, and enable consistent estimators for causal discovery. Moreover, SiCL is also equipped with a specially designed pairwise encoder module with a unidirectional attention layer to model both internal and external relationships of pairs of nodes. Experimental results on both synthetic and real-world benchmarks show that SiCL significantly outperforms other DNN-based SCL approaches.
\end{abstract}

\section{Introduction} \label{sec:int}

Causal discovery seeks to infer causal structures from an observational data sample.
Supervised Causal Learning (SCL) \citep{dai2023ml4c,ke2023learning,ma2022ml4s} is an emerging paradigm in this field. The basic idea is to consider causal discovery as a \emph{structured prediction} task, and to train a prediction model using supervised learning techniques. 
At training time, a training dataset comprising a variety of causal mechanisms and their associated data samples is generated. %, either via synthetic generation, or from a simulator if available \citep{lorchamortized}. 
The prediction model is then trained to take such a data sample as input, and to output predictions about the causal mechanism behind the data sample.
%During the inference stage, the causal structure is identified by simply applying the learned model to the target data.  
Compared to traditional rule-based or unsupervised 
%causal discovery 
methods~\citep{glymour2019review}, the SCL method has demonstrated strong empirical performance~\citep{dai2023ml4c,ma2022ml4s}, as well as robustness against sample size and distribution shift \citep{ke2023learning,lorchamortized}.

Deep Neural Network (DNN)-based SCL employs DNN as the prediction model. It allows end-to-end training, removing the need for manual feature engineering. Additionally, it can handle both continuous and discrete data types effectively, and can learn latent representations.
A specific DNN architecture, first introduced by \cite{lorchamortized}, is particularly popular in recent DNN-based SCL works. The model first transforms the given data sample into a set of node-wise feature vectors, each representing an individual variable (corresponding to a node in the associated causal graph).
Based on these node-wise features, the model then outputs a weighted adjacency matrix $A$, where $A_{ij}\in[0,1]$  is an estimated probability for the directed edge $i \rightarrow j$ (meaning that $i$ is a direct cause of $j$). 
% The final DAG is obtained as a Bernoulli sample of $A$.
Finally, the adjacency matrix of an inferred causal graph $G$ is obtained as a Bernoulli sample of $A$, where each entry $G_{ij} \in \{0,1\}$ is sampled \emph{independently}, following probability $A_{ij}$. 
\textcolor{black}{For convenience, we call} such a model architecture as the ``Node-Edge'' \textcolor{black}{architecture}, as the representation is learned for individual nodes and the probability is estimated and sampled for individual directed edges. 

Despite its popularity and encouraging results~\citep{lorchamortized,Zhu2020Causal,dpdag,varamballydiscovering}, we identify two limitations for the Node-Edge approach:


\textcolor{black}{First, the Node-Edge architecture imposes a fundamental bias in the inferred causal relations. Specifically, given an observational data sample $D$, the existence of a directed causal edge $i \rightarrow j$ may \emph{necessarily} depend on the existence of other edges. But the existing Node-Edge models predict each edge separately and independently, so the probability prediction $A_{ij}$ made by such models is only conditioned on the input sample $D$, not on the sampling result of other entries of $A$,  thereby failing to capture the crucial inter-edge dependency in its probability estimation.
}

As a simple example, a Node-Edge model maintaining the possibility of both $G_1: X\rightarrow T \rightarrow Y$ and $G_2: X\leftarrow T \leftarrow Y$ would necessarily have a non-zero probability to output the edges $X\rightarrow T$ and $T \leftarrow Y$, thus cannot rule out the possibility of $G_3: X\rightarrow T \leftarrow Y$, even though $G_3$ is impossible to be the groundtruth causal graph behind a data sample $D$ compatible with $G_1$ and $G_2$~\citep{verma1990equivalence}. 
Crucially, there is no way to tell $G_1$ from $G_2$ based on observational data in general cases~\citep{andersson1997characterization,meek1995strong}. 
It means that for any Node-Edge model to be sound, it has to maintain the possibility of both $G_1$ and $G_2$ (when observing a data sample compatible with any of them), leading to an inevitable error probability to output the impossible graph $G_3$ on the other hand. 


\textcolor{black}{Second, the Node-Edge architecture does not explicitly represent the features about node pairs, which we argue are essential for observational causal discovery.}
For example, a causal edge $X\rightarrow Y$ can exist only if the node pair $\langle X, Y\rangle$ demonstrates \textit{persistent dependency} \citep{ma2022ml4s,spirtes2000causation}, meaning that $X$ and $Y$ remain statistically dependent regardless of conditioning on any subset of other variables. As another example, for causal DAGs, a sufficient condition to determine the causal direction between a persistently dependent node pair $\langle X, Y\rangle$ is that $X$ and $Y$ exhibits \emph{orientation asymmetry}, meaning that there exists a third variable $Z$ such that $X$ is persistently dependent to $Z$ but $Y$ can become independent to $Z$ conditioned on a variable-set $\mathbf{S}\not\ni X$ (or vice versa). A feature like persistent dependency or orientation asymmetry is, in its nature, a collective property of a node pair, but not of any individual node alone. 

To address these limitations, in this paper, we propose a novel DNN-based SCL approach, called Supervised Identifiable Causal Learning (SiCL). 
\textcolor{black}{The neural network in SiCL does not seek to predict the probabilities of directed edges, but tries to predict a skeleton matrix together with v-tensor, a third-order tensor representing the v-structures.}
According to the Markov Equivalence Class (MEC) theory of causal discovery, skeleton and v-structures are \emph{identifiable} causal structures under the canonical MEC setting (while the directed edges are not), so predictions about skeleton and v-structures do not suffer from the (non-)identifiability limit. 
By leveraging this insight, our theory-inspired DNN architecture completely avoids the systematic bias in edge-prediction models as previously discussed, and enables \emph{consistent} neural-estimators\footnote{Recall that a statistical estimator is \emph{consistent} if it converges to the groundtruth given infinite data.} for causal discovery. 
Moreover, SiCL is also equipped with a specially designed pairwise encoder module with a unidirectional attention layer.
With both node features and node-pair features as the layer input, it can model both internal and external relationships of pairs of nodes.
%SiCL is trained on synthetic data, and it shows significant improvement on both skeleton prediction and orientation tasks compared to other DNN-based approaches. 
Experimental results on both synthetic and real-word benchmarks show that \textcolor{black}{SiCL} can effectively address the two \textcolor{black}{above-mentioned limitations}, and the resulted SiCL solution significantly outperforms other DNN-based SCL approaches with more than 50\% performance improvement in terms of SHD (Structural Hamming Distance) on the real-world Sachs data. The codes are publicly available at \url{https://github.com/microsoft/reliableAI/tree/main/causal-kit/SiCL}.


\section{Background and Related Work}\label{sec:bg}
A Causal Graphical Model is defined by a joint probability distribution $P$ over multiple random variables and a DAG $G$. Each node $X_i$ in $G$ represents a variable in $P$, and a directed edge $X_i \rightarrow X_j$ represents a direct cause-effect relation from $X_i$ to $X_j$.
A causal discovery task generally asks to infer about $G$ from an i.i.d. sample of $P$.

%\subsection{Identifiability} 
However, there is a well-known identifiability limit for causal discovery.
In general, the causal DAG is only identifiable up to an equivalence class. 
Studies of this identifiability limit under a canonical assumption setting have led to the well-established MEC theory \citep{frydenberg1990chain,verma1990equivalence}.
We call a causal feature, \textit{MEC-identifiable}, if the value of this feature is invariant among the equivalence class under the canonical MEC assumption setting. 
It is known that such MEC-identifiable features include the skeleton and the set of v-structures, which we briefly present in the following.
% that the causal DAG $G$ is, in general, only identifiable up to its Markov equivalence class (MEC).
% Studies of this identifiability limit have led to a well-established theory \citep{frydenberg1990chain,verma1990equivalence}, which we briefly present in the following.


A \textit{skeleton} $E$ defined over the data distribution $P$ is an undirected graph where an edge exists between $X_i$ and $X_j$ if and only if $X_i$ and $X_j$ are always dependent in $P$, i.e., $\forall Z \subseteq\left\{X_1, X_2, \cdots, X_d\right\} \backslash \left\{X_i, X_j \right\}$, we have $X_i \nperp X_j | Z$.
Under mild assumptions (such as that $P$ is Markovian and faithful to the DAG $G$; see details in Appendix Sec. \ref{sec:da}), 
the skeleton is the same as the corresponding undirected graph of the DAG $G$ \citep{spirtes2000causation}. 
% 
A triple of variables $\langle X, T, Y \rangle$ is an \textit{Unshielded Triple (UT)} if $X$ and $Y$ are both adjacent to $T$ but not adjacent to each other in (the skeleton of) $G$.
It becomes a \textit{v-structure} denoted as $X \rightarrow T \leftarrow Y$ if the directions of the edges are from $X$ and $Y$ to $T$.

% 
Two graphs are Markov equivalent if and only if they have the same skeleton and v-structures. 
The \textit{Markov equivalence class (MEC)} can be represented by a \textit{Completed Partially Directed Acyclic Graph (CPDAG)} consisting of both directed and undirected edges. We use $CPDAG(G)$ to denote the CPDAG derived from $G$.
% \end{Definition}
According to the theorem of Markov completeness \citep{meek1995strong}, 
%assuming causal sufficiency and $P$ is Markovian and faithful w.r.t. $G$, 
we can only identify a causal graph up to its MEC, i.e., the CPDAG, unless additional assumptions are made (see the remark below).
%, for disc)ete data or linear Gaussian data. 
This means that each (un)directed edge in $CPDAG(G)$ indicates a (non)identifiable causal relation.

\textbf{Remark:} The MEC-based identifiability theory is applicable in the general-case setting, when we take into account all possible distributions $P$. 
It is known this identifiability limit could be broken (i.e., an undirected edge in the CPDAG could be oriented) \textit{if} we assume that the data follows some special class of distributions, e.g., linear non-Gaussians, additive noise models, post-nonlinear or location-scale models~\citep{peters2014causal,shimizu2011directlingam,zhang2009identifiability,immer2023identifiability}. 
These assumptions are sometimes hard to verify in practice, so this paper considers the general-case setting.
More discussions on the identifiability and causal assumptions are provided in Appendix Sec. \ref{sec:dica}.


\subsection{Related Work}
In traditional methods of causal discovery, constraint-based methods are mostly related to our work.
%are categorized into constraint-based, score-based and continuous optimization. 
They aim to identify the DAG that is consistent with inter-variable conditional independence constraints. 
These methods first identify the skeleton and then conduct orientation based on v-structure identification \citep{yu2016review}. 
The output is a CPDAG which represents the MEC.
Notable algorithms in this category include PC \citep{spirtes2000causation}, along with variations such as Conservative-PC \citep{ramsey2012adjacency}, PC-stable \citep{colombo2014order}, and Parallel-PC \citep{le2016fast}. 
Compared to constraint-based methods, both our approach and theirs are founded upon the principles of MEC theory for estimating skeleton and v-structures. However, whereas traditional methods rely on symbolic reasoning based on explicit constraints, we employ DNNs to capture the essential causal information intricately linked with these constraints.

Score-based methods aim to find an optimal DAG according to a predefined score function, subject to combinatorial constraints. 
These methods employ specific optimization procedures such as forward-backward search GES \citep{chickering2002optimal}, hill-climbing \citep{koller2009probabilistic}, and integer programming \citep{cussens2011bayesian}.
Continuous optimization methods transform the discrete search procedure into a continuous equality constraint.
NOTEARS \citep{zheng2018dags} formulates the acyclic constraint as a continuous equality constraint and is further extended by DAG-GNN \citep{yu2019dag}, DECI \citep{geffner2022deep} to support non-linear causal relations. 
DECI \citep{geffner2022deep} is a flow-based model which can perform both causal discovery and inference on non-linear additive noise data.
% These methods can be viewed as unsupervised optimization since they do not access additional datasets associated with ground-truth causal relations. 
Recently, ENCO \citep{lippe2021efficient} is proposed as a continuous optimization method where the edge orientation is modeled as a separate parameter to maintain the acyclicity.
It is guaranteed to converge to the correct graph if interventions on all variables are available.
 \textcolor{black}{RL-BIC \citep{Zhu2020Causal} utilizes Reinforcement Learning to search for the optimal DAG.}
These methods can be viewed as unsupervised since they do not access additional datasets associated with ground truth causal relations.
We refer to \cite{glymour2019review,vowels2022d} for a thorough exploration of this literature. 


SCL begins from orienting edges in the bivariate cases under the functional causal model formalism. 
Methods such as RCC \citep{lopez2015randomized} and NCC \citep{lopez2017discovering} have outperformed unsupervised approaches like ANM \citep{hoyer2008nonlinear} or IGCI \citep{janzing2012information}.
For multivariate cases, ML4S \citep{ma2022ml4s} proposes a supervised approach specifically for skeleton learning. 
%It employs an order-based cascade learning procedure and generates training data from vicinal graphs. 
Complementary to ML4S, ML4C \citep{dai2023ml4c} takes both data and skeleton as input and classifies unshielded triples as either v-structures or non-v-structures. 
\cite{petersen2023causal} proposes a SLdisco method, utilizing SCL approach to address some limitations of PC and GES.

DNN-based SCL has emerged as a prominent approach for enabling end-to-end causal learning. 
Two notable works in this line, namely AVICI \citep{lorchamortized} and CSIvA \citep{ke2023learning}, introduced an alternating attention mechanism to enable permutation invariance across samples and variables. Both methods learn individual representation for each node, which is then used to predict directed edges. Among them, AVICI considers the task of predicting DAG from observational data and adopts exactly the Node-Edge architecture, hence suffers from the issues as discussed in Sec. \ref{sec:int}. On the other hand, CSIvA requires additional interventional data as input to identify the full DAG, and applies an autoregressive DNN architecture where edges are predicted sequentially by multiple inference runs. Therefore, this autoregressive approach incurs very high inference cost due to the quadratic number of model runs required (w.r.t. the number of variables in question), as we experimentally verify in Appendix Sec. \ref{sec:auto}. In contrast, the method proposed in this paper only requires a single run of the DNN model. Besides that, our method also differs from both AVICI and CSIvA in terms of the usage of pairwise embedding vectors.


\section{Limitations of the Node-Edge \textcolor{black}{Architecture}} \label{sec:met:lim}

% \subsection{Motivation} 

% \subsubsection{Case Study of Limitation of Bernoulli-sampling adjacency matrix approach}

%\paragraph{Limitation of the Node-Edge approach.} 
\textcolor{black}{The Node-Edge architecture is common and has been adopted to generate the output DAG $G$ in the literature \citep{lorchamortized,Zhu2020Causal,dpdag,varamballydiscovering}.}
% In previous work, the Node-Edge \textcolor{black}{architecture} is adopted to generate the output DAG $G$ \citep{lorchamortized}.
In this \textcolor{black}{architecture}, each entry $G_{ij}$ in the DAG is independently sampled from $A_{ij}$, an entry in the adjacency matrix $A$. This entry $A_{ij}$ represents the probability that $i$ directly causes $j$. 
We introduce a simple yet effective example setting with only three variables $X$, $Y$, and $T$ to reveal its limitation.

\textcolor{black}{Considering a simulator that generates DAGs with equal probability from two causal models: }In model 1, the causal graph is $G_1: X \rightarrow T \rightarrow Y$, and the variables follow $X \sim \mathcal{N} (0, 1)$, $T = X + \mathcal{N}(0, 1)$, $Y = T + \mathcal{N}(0, 1)$.
In model 2, the causal graph is $G_2: X \leftarrow T \leftarrow Y$, and the variables follow $ Y = \mathcal{N}(0, 3)$, $T = \frac{2}{3}Y + \mathcal{N}(0, \frac{2}{3})$, $X = 0.5T + \mathcal{N}(0, 0.5)$.
In this case, data samples coming from both causal models follow the same joint distribution, which makes $G_1$ and $G_2$ inherently indistinguishable (from observational data sample).

More importantly, when the fully-directed causal DAGs are used as the learning target (as the Node-Edge approach does), an optimally trained neural network will predict $0.5$ probabilities on the directions of the two edges $X - T$ and $T - Y$.
As a result, with $0.25$ probability the graph sampling outcome would be $X \rightarrow T \leftarrow Y$ (see Fig. \ref{fig:ps} in the Appendix).
%It is incompatible with the observational data, resulting in a contradictory causal structure.
This error probability is rooted from the fact that the Bernoulli sampling of the edge $X \rightarrow T$ is not conditioned on the sampling result of the edge $T \leftarrow Y$. Consequently, it is a bias that cannot be avoided even if the DNN has perfectly modeled the \emph{marginal probability} of each edge (marginalized over other edges) given input data.





\color{black}
We further find that $0.25$ is not the worst-case error rate yet. 
Formally, for a distribution $Q$ over a set of graphs, we define the graph distribution where the edges are independently sampled from the marginal distribution as $M(Q)$, i.e., for any causal edges $e_1$ and $e_2$, $P_{G\sim Q}(e_1 \in G) = P_{G\sim M(Q)}(e_1 \in G) = P_{G\sim M(Q)}(e_1 \in G | e_2 \in G)$.
In general, a Node-Edge model optimally trained on data samples $D$ coming from the distribution $Q$ will essentially learn to predict $M(Q)$ (when given the same data samples $D$ at test time).
The following proposition shows that for causal graphs with star-shaped skeleton, with a chance of $26.42\%$ the graph sampled from the marginal distribution $M(Q)$ would be incorrect.


\begin{Proposition}
Let $\mathcal{G}_n$ be the set of graphs with $n+1$ nodes where there is a central node $y$ such that (1) every other node is connected to $y$, (2) there is no edge between the other nodes, \textcolor{black}{and} (3) there is at most one edge pointing to $y$. 
We have 
\begin{align}
\sup_n \max_{Q} P_{G \sim M(Q)}(G \nin \mathcal{G}_n) = 
1 - \frac{2}{e} \approx 0.2642.
\end{align}
\label{prop:star}
\end{Proposition}
The proof is provided in Appendix Sec. \ref{sec:mgcs}. 
% It indicates that a Node-Edge model will output $M(Q)$ when optimally trained on a data sample associated with a distribution $Q$.
It indicates that an edge-predicting neural network could suffer from an inevitable error rate of $0.2642$ even if it is perfectly trained.
\color{black}
%\textbf{Remark:} We clarify that our critique is specifically aimed at the limitations of the Node-Edge approach, not the use of an adjacency matrix as a learning target.
%\textcolor{black}{When the final prediction is up to an MEC rather than a fully identifiable DAG, the Bernoulli-sampling adjacency matrix approach results in inconsistency for UTs formed by non-identifiable edges.
%For instance, }our case study shows that the entries in the adjacency matrix are not independent in determining the causal relations, thus the use of independent Bernoulli sampling over the adjacency matrix falls short of adequately representing causal relations. 
In contrast, models that predict skeleton and v-structures would have a theoretical asymptotic guarantee of the consistency under canonical assumption. 
The details, proof and relevant discussions are provided in Appendix Sec. \ref{sec:app:tg}.






\section{The SiCL Method} \label{sec:algo}
In light of the limitations as discussed, we propose a new DNN-based SCL method in this section, named \textbf{SiCL} (\textbf{S}upervised \textbf{i}dentifiable \textbf{C}ausal \textbf{L}earning).

\subsection{Overall Workflow}
\begin{figure*}[htb]
\centering
\includegraphics[width=\linewidth]{figures/new_workflow.pdf}
      % \vspace{-0.05in}
    \caption{The inference workflow of SiCL.}
    \label{fig:ww}
      % \vspace{-0.2in}
      
\end{figure*}
Following the standard DNN-based SCL paradigm, the core inference process is implemented as a DNN. The DNN takes a data sample encoded by a matrix $D\in\mathbb{R}^{n \times d}$ as input, with $d$ being the number of observable variables and $n$ being the number of observations. In contrast to previous \textcolor{black}{Node-Edge} approaches, the SiCL method does not use the DNN to directly predict the causal graph, but instead seeks to predict the skeleton and the v-structures of the causal graph, which amount to the \emph{MEC-identifiable} causal structures as mentioned previously. 

Specifically, our DNN outputs two objects: (1) a skeleton prediction matrix $S \in [0,1]^{d \times d}$ where $S_{ij}$ models the conditional probability  
\textcolor{black}{of the existence of the \emph{undirected} edge $X_i - X_j$, conditioned on the input data sample $D$, }
% $P(undirected~edge~(X_i,X_j)~exists~|~D)$, 
and (2) a v-structure prediction tensor $V \in [0,1]^{d\times d\times d}$ where $V_{ijk}$ models the conditional probability 
% $P(v\text{-}structure~(X_j \rightarrow X_i \leftarrow X_k)~exists~|~D)$. 
\textcolor{black}{of the existence of the v-structure component $X_j \rightarrow X_i \leftarrow X_k$, again conditioned on $D$.}
In our implementation, $S$ and $V$ are generated by two separate sub-networks, called Skeleton Predictor Network (SPN) and V-structure Predictor Network (VPN), respectively. 
%Both the SPN and the VPN utilize a Pairwise Encoder architecture to explicitly model features about node-pairs. 
To further address the limitation of only having node-wise features for Node-Edge models, we propose to equip SPN and VPN with Pairwise Encoder modules to explicitly capture node-pair-wise features.
% Due to the significance of the pairwise relationship \textcolor{black}{between vertices}, we propose a pairwise encoder module to model the pairwise representations.

Based on the skeleton prediction matrix $S$ and v-structure prediction tensor $V$, we infer the skeleton and v-structures of the causal graph; from the two we can determine a unique CPDAG. The CPDAG encodes a Markov equivalence class, from which we can pick up a graph instance as the prediction of the causal DAG (if needed). 
% See Fig. \ref{fig:ww} for a diagram of the overall inference workflow of SiCL.
\textcolor{black}{Figure \ref{fig:ww} provides a diagram of the overall inference workflow of SiCL, and a pseudo-code of the workflow is given by Algorithm \ref{alg:workflow} in Appendix.}

Parameters of the SPN and VPN are trained following the standard supervised learning procedure. In our implementation, we only use synthetic training data, which is relatively easy to obtain, and yet could often lead to strong performance on real-world workloads~\citep{ke2023learning}.


In the following, we elaborate the DNN architecture, the learning targets, as well as the post-processing procedure used in the SiCL method.







\subsection{Feature Extraction} \label{sec:fem}
% \subsection{Pairwise Encoder Module} \label{sec:pem}

\textbf{Input Processing and Node Feature Encoder.} Given input data sample, which is a matrix $D\in\mathbb{R}^{n \times d}$, the input processing module contains a linear layer for continuous input data or an embedding layer for discrete input data, yielding the raw node features $\mathcal{F}^{raw}_{il}$ for each node $i$ in each observation $l$.
% It consists of an input processing module, a node feature encoder module, and a pairwise encoder module sequentially.
% The input processing module contains a linear layer for continuous input data or an embedding layer for discrete input data.
After that, we employ a node feature encoder to further process the raw node features into the final node features $\mathcal{F}_{il}$.
Similar to previous papers \citep{ke2023learning,lorchamortized}, the node feature encoder is a transformer-like network comprising attention layers over the observation dimension and the node dimension alternately, which naturally maintains permutation equivalence across both variable and data dimension \textcolor{black}{because of the intrinsic symmetry of attention operations}.
\textcolor{black}{More details about the node feature encoder are presented in Appendix Sec. \ref{sec:dnf} due to page limit}.

\textbf{Pairwise Encoder.} Given node features $\mathcal{F} \in \mathbb{R}^{d\times n \times h}$ for all the $d$ nodes, the goal of pairwise encoder is to encode their pairwise relationships by $d^2$ pairwise features, represented as a tensor $\mathcal{P} \in \mathbb{R}^{d\times d \times n \times h}$, where $\mathcal{P}_{ijl} \in \mathbb{R}^{h}$ is a pairwise feature corresponding to the node pair $(i,j)$ in observation $l$.
% $$\mathbf{p}_{ij} \in \mathbb{R}^{h}$.}
As argued in Sec. \ref{sec:int}, both ``internal'' information (i.e., the pairwise relationship) and ``external information (e.g., the context of the conditional separation set) of node pairs are needed to capture persistent dependency and orientation asymmetry.
Our pairwise encoder module \textcolor{black}{is designed to model the internal relationship via node feature concatenation and the non-linear mapping by MLP.}
On the other hand, 
we employ attention operations within the pairwise encoder to capture the contextual relationships (including persistent dependency and orientation asymmetry).%, which is a common practice.

More specifically, the pairwise encoder module consists of the following parts (see Appendix Fig. \ref{fig:pem} for diagrammatic illustration):
%context-based
% Given a set of $d$ nodes, $h$-dimensional node features, the goal of the pairwise encoder $PE$ is to encapsulate their pairwise relationships by $d^2$ $h$-dimensional pairwise features.
% \st{Given a set of $d$ nodes, each node is represented by an $h$ dimensional vector.
% The goal of the pairwise encoder is to encapsulate their pairwise relationships by $d^2$ $h$-dimensional pairwise features.}
\begin{enumerate}[leftmargin=*]
    \item \textit{Pairwise Feature Initialization.} The initial step is to concatenate the node features \textcolor{black}{from the previous node feature encoder module} for every pair of nodes.
    Subsequently, we employ a three-layer MLP to convert each concatenated vector $\mathcal{P}_{ijl} \in \mathbb{R}^{2h}$ to an $h$-dimensional raw pairwise feature, i.e., $\mathcal{P}_{ijl}^1 =  \mathrm{MLP}([\mathcal{F}_{il}; \mathcal{F}_{jl}])$. It is designed to capture the intricate relations that exist inside the pairs of nodes.
    \item \textit{Unidirectional Multi-Head Attention.} In order to model the external information, we employ an attention mechanism where the query is composed of the aforementioned $d^2$ $h$-dimensional raw pairwise features, while the keys and values consist of $h$-dimensional features of $d$ individual nodes, i.e., $\mathcal{P}^2 = \mathrm{MultiHeadAttention}(\mathcal{P}^1, \mathcal{F}, \mathcal{F})$.
Note that, this attention operation is unidirectional, which means we only calculate cross attention from raw pairwise features $\mathcal{P}^1$ to node features $F$.
This design is meant to capture \textcolor{black}{both pair-wise and node-wise information (as both are critical to model the causality, as discussed in Sec. \ref{sec:int}) while at the same time to maintain a reasonable computational cost.}
\item \textit{Final Processing.} Following the widely-adopted transformer architecture, we incorporate a residual structure and a dropout layer after the previous part, i.e., $\mathcal{P}^3 = \mathrm{Norm}(\mathcal{P}^1 + \mathcal{P}^2)$.
Finally, we introduce a three-layer MLP to further capture intricate patterns and non-linear relationships between the input embeddings, as well as to more effectively process the information from the attention mechanism: $\mathcal{P} = \mathrm{Norm}(\mathrm{MLP}(\mathcal{P}^3) + \mathcal{P}^3)$. 
% This approach allows for a comprehensive understanding of the complex relationships and leading to more robust and accurate modeling of causal structures.
% For the given dataset matrix $D \in \mathbb{R}^ {n\times d}$ with $n$ observations, the pairwise encoder module yields a tensor of shape $\mathbb{R}^{n \times d \times d \times h}$, containing an $h$-dimension feature vector for each node pair of each observation. 
It yields the final pairwise feature tensor $\mathcal{P} \in \mathbb{R}^{d \times d \times n \times h}$.
% , giving an $h$-dimension feature vector for each node pair of each observation. 
\end{enumerate}

% Now we introduce the whole architecture of feature extractor.
% It consists of an input processing module, a node feature encoder module, and a pairwise encoder module sequentially.
% The input processing module contains a linear layer for continuous input data or an embedding layer for discrete input data.
% Similar to previous papers \citep{ke2023learning,lorchamortized}, the node feature encoder is a transformer-like network comprising attention layers over either the observation dimension or the node dimension alternately.
% It naturally maintains permutation equivariance across both the variable dimension and the data dimension \textcolor{black}{because of the intrinsic symmetry of attention operations}.
% Subsequently, the pairwise encoder module is applied to obtain the pairwise features $\mathcal{P}$. 


\subsection{Learning Targets} \label{sec:met:lic}

% \subsubsection{Learning Identifiable Causal Structures}
% To address the issue, we propose to allow the network model to learn solely the identifiable causal structures in $G$, i.e., its MEC. 
As mentioned above, our learning target is a combination of the skeleton and the set of v-structures, which together represent an MEC. 
Two separate neural (sub-)networks are trained for these two targets.
% As shown in Sec. \ref{sec:bg}, an MEC can be represented by a combination of the skeleton and the set of v-structures, which are our learning targets.
% As discussed in Sec. \ref{sec:bg}, skeletons are entirely identifiable across all data types, and v-structures are also identifiable for Linear-Gaussian continuous data and discrete data. 
% As discussed in Sec. \ref{sec:bg}, skeletons and v-structures are identifiable under our settings. 
% Consequently, our learning target contains the skeleton and the set of v-structures, forming a representation of the MEC. % under the conditions of Linear-Gaussian continuous and discrete data.%, and only the skeleton under other conditions. 
%It avoids unnecessary prediction errors and contradictions.
%For instance, the two different causal structures share the same MEC in the above example, and our model can predict $X - T - Y$ as the MEC representation.
%learn from unified precise labels of skeleton and v-structures.
% This refined approach mitigates the challenges associated with unidentifiable causal structures and enhances the overall performance of the network model.

\textbf{Skeleton Prediction.} 
As the persistent dependency between pairs of nodes determines the existence of edges in the skeleton, the pairwise features correspond to edges in the skeleton naturally.
Therefore, for the skeleton learning task, we initially employ a max-pooling layer over the observation dimension to obtain a single vector $\mathcal{S}_{ij} \in \mathbb{R}^{h}$ for each pair of nodes, i.e., $\mathcal{S}_{ij} = \max_{k} \mathcal{P}_{ijk}$.
Then, a linear layer and a sigmoid function are applied to map the pairwise features to the final prediction of edges, i.e., $S_{ij} = \mathrm{Sigmoid}(\mathrm{Linear}(\mathcal{S}_{ij}))$.
Our learning label, the undirected graph representing the skeleton, can be easily calculated by summing the adjacency of the DAG $G$ and its transpose $G^T$.
% Denoting the combination of max-pooling and linear layer as a skeleton prediction module $SP$, 
Therefore, our learning target for the skeleton prediction task can be formulated as $\min \mathcal{L}(S, G + G^T)$,
where $\mathcal{L}$ is the popularly used binary cross-entropy loss function.

% $FE$ is the feature extractor mentioned above, and $D$ denotes the input data.


\textbf{V-structure Prediction.}
% he orientation asymmetry is significant to judge the existence of v-structures.
A UT $\langle X_i, X_k, X_j \rangle$ is a v-structure when $\exists \mathbf{S}$, such that $X_k \notin \mathbf{S}$ and $X_i \perp X_j | \mathbf{S}$.
Motivated by this, we concatenate the corresponding pairwise features of the pair $\langle X_i, X_j \rangle$ with the node features of $X_k$ as the feature for each UT $\langle X_i, X_k, X_j \rangle$ after a max-pooling along the observation dimension, i.e., $\mathcal{U}_{kij} = [\max_l \mathcal{P}_{ijl};\max_l \mathcal{F}_{kl}]$.
After that, we use a three-layer MLP with a sigmoid function to predict the existence of v-structures among all UTs, i.e., $\mathcal{U}_{kij} = \mathrm{Sigmoid}(\mathrm{MLP}( \mathcal{U}_{kij}))$.
Given a data sample of $d$ nodes, it outputs a third-order tensor of shape $\mathbb{R}^{d \times d \times d}$, namely v-tensor, corresponding to the predictions of the existence of v-structures.
The v-tensor label can be obtained by $\mathcal{V}_{kij} = G_{ik} G_{jk} (1 - G_{ij})(1 - G_{ji})$,
where $\mathcal{V}_{kij}$ indicates the existence of v-structure $X_i \rightarrow X_k \leftarrow X_j$.
Therefore, the learning target for the v-structure prediction task can be formulated as $\min \mathcal{L}_{UT}(\mathcal{U}, \mathcal{V})$,
where $\mathcal{L}_{UT}$ is the binary cross-entropy loss masked by UTs, i.e., we only calculate such loss on the valid UTs.
In our current implementation, the parameters of the feature encoders are fine-tuned from the skeleton prediction task, as the UTs to be classified are obtained from the predicted skeleton and the skeleton prediction can be seen as a general pre-trained task.

Note that neural networks with our learning targets have a theoretical guarantee for correctness in asymptotic sense, as mentioned around the end of Sec. \ref{sec:met:lim}. 

\subsection{Post-Processing} 
\textcolor{black}{Although our method theoretically guarantees asymptotic correctness, conflicts in predicted v-structures might occasionally occur in practice. Therefore,} in the post-processing stage, we apply a straightforward heuristic to resolve the potential conflicts and cycles among predicted v-structures following previous work \citep{dai2023ml4c}.
\textcolor{black}{After that, we use an improved version of Meek rules \citep{meek1995causal,tsagris2019bayesian} to obtain other MEC-identifiable edges without introducing extra cycles.}
Combining the skeleton from the skeleton predictor model with all MEC-identifiable edge directions, we get the CPDAG predictions.

We provide a more detailed description of the post-processing process in Appendix Sec. \ref{app:post}. It is worth noting that our current design of post-processing is a very conservative one, and this module is also non-essential in our whole framework; see Appendix Sec. \ref{app:post} for more discussions and evidences.
\color{black}


\section{Experiments} \label{sec:exp}

\begin{table*}[!tb]
\centering
% \resizebox{\linewidth}{!}{%
\begin{threeparttable}
\caption{\textbf{General comparison of SiCL and other methods}. The average performance results in three runs are reported for SiCL. GES takes more than 24 hours per graph on WS-L-G. SLdicso is unsuitable on non-linear-Gaussian data. \textcolor{black}{Full results on all metrics are provided in Appendix Tab. \ref{tab:epder}}.}
\label{tab:epders}
\begin{tabular}{cccccccccccc}
\toprule
 \multirow{2}{*}{Method} & \multicolumn{2}{c}{WS-L-G}& \multicolumn{2}{c}{SBM-L-G}& \multicolumn{2}{c}{WS-RFF-G}& \multicolumn{2}{c}{SBM-RFF-G}& \multicolumn{2}{c}{ER-CPT-MC} \\
 & s-F1$\uparrow$ & o-F1$\uparrow$ &s-F1$\uparrow$ & o-F1$\uparrow$&s-F1$\uparrow$ & o-F1$\uparrow$&s-F1$\uparrow$ & o-F1$\uparrow$&s-F1$\uparrow$ & o-F1$\uparrow$\\
\midrule
 PC & $30.4 $ & $16.0 $& $58.8$&$35.9$&$36.1$&$16.1$&$57.5$&$34.2$&$82.2$&$40.6$ \\
 GES & * & * & $70.8$& $55.0$&$41.7$&$23.6$&$56.5$&$38.0$&$82.1$&$42.4$\\
 NOTEARS & $33.3 $ & $31.5$&$80.1$&$77.8$&$37.7$&$33.4$&$55.6$&$48.5$&$16.7$&$0.6$ \\
 DAG-GNN & $35.5$ & $32.7$ &$66.2$&$62.5$&$33.2$&$28.9$&$47.1$&$40.6$&$24.8$&$3.7$\\
 GRAN-DAG & $16.6$&$11.7$&$22.6$&$14.4$&$4.7$&$1.1$&$17.4$&$3.8$&$40.8$&$7.3$ \\
 % NOTEARS-MLP &$24.6$&$11.8$&$44.7$&$39.0$&$\mathbf{52.7}$&$\mathbf{47.7}$& $48.2$ &$43.5$& *& * & \\
 GOLEM &$30.0$ &$19.3$&$68.5$&$65.2$&$27.6$&$17.7$&$41.1$&$24.8$&$37.6$&$9.3$& \\
 % GRaSP &&&&&&&&&&$0.0$&$0.0$ \\
 SLdisco &$0.1$ &$0.1$&$1.9$&$1.2$&*&*&*&*&*&* \\
 AVICI & $39.9 $ & $35.8$ & $84.3$ & $81.6$& $47.7$& $45.2$& $76.6$& $72.7$& $76.9$& $57.6$\\
 SiCL & $\mathbf{44.7} $ & $\mathbf{38.5} $& $\mathbf{85.8} $ & $\mathbf{82.7} $ & $\mathbf{51.8}  $ & $ \mathbf{46.3}$ & $ {\mathbf{82.1}}$ & $\mathbf{78.0}$ &$\mathbf{84.2}$ & $\mathbf{59.9}$ \\
\bottomrule
\end{tabular}
\end{threeparttable}
% }
% \vspace{-0.15in}
\end{table*}
In this section, we report the performance of SiCL on both synthetic and real-world benchmarks, followed by an ablation study. 
More results and discussions about \textcolor{black}{time cost}, generality, and acyclicity are deferred to Appendix Sec. \ref{sec:app:exp:e}, due to page limit.

\subsection{Experiment Design}
\textbf{Metrics.} We profile a causal discovery method's performance using the following two tasks:

\emph{Skeleton Prediction}: 
Given a data sample $D$ of $d$ variables, for each variable pair, we want to infer if there exists direct causation between them. The standard metric \textbf{s-F1} (short for \textbf{skeleton-F1}) is used, which considers skeleton prediction as a binary classification task over the $d(d-1)/2$ variable pairs. 
For completion, we also report classification accuracy results. 
For methods with probabilistic outputs, AUC and AUPRC scores are also measured.

% \emph{Graph Prediction} :
% Given a data sample $D$ of $n$ variables, for each variable pair we want to infer if there exists direct causation, and in that case we want to further infer the causal direction. Following ML4C \cite{dai2023ml4c}, we use the following \textbf{orientation-F1} metric for this structured prediction task: A variable pair is considered a positive item if there is direct causation between them, and a prediction about the pair is a positive prediction if at least one causal direction is predicted. A positive prediction is a true positive if it's over a positive item \emph{and the predicted causal direction is correct}. The standard F1 calculation is then applied.

% Another metric, called \textbf{edge-F1} in this paper, was used in some previous works \textcolor{red}{[which?]}, which considers the task as a binary classification problem over the $n^2$ \emph{ordered-pairs} (whether there is a directed edge in the groundtruth causal graph or not). We also measured edge-F1 for completion.

\emph{CPDAG Prediction}: 
% Given a data sample $D$ of $d$ variables, for each variable pair, we want to infer if there exists direct causation between them, in that case we want to infer if the causal direction is identifiable, and in that case we try to infer the causal direction. 
\textcolor{black}{Given a data sample $D$ of $d$ variables, for each pair of variables, we aim to determine if there is direct causality between them. If so, we then assess whether the causal direction is MEC-identifiable, and if it is, we attempt to infer the specific causal direction.}
As this task involves both directed and undirected edge prediction, we use \textbf{SHD} (Structural Hamming Distance) to measure the difference between the true CPDAG and the inferred CPDAG. Besides that, we also measure the \textbf{o-F1} (short for \textbf{orientation-F1}) of the directed sub-graph of the inferred CPDAG (compared against the directed sub-graph of the true CPDAG), which focuses on capturing the inference method's orientation capability in identifying \textit{MEC-identifiable} causal edges. 
% Finally, we also measure \textbf{v-F1}, which is F1 score with the set of v-structures in the true CPDAG as positive items (among all ordered-triples), and v-structures in the inferred CPDAG as positive predictions.
\textcolor{black}{Finally, we calculate the \textbf{v-F1} score, where the F1 score is based on the set of v-structures in the true CPDAG as the positive instances (from all ordered triples), and the v-structures in the inferred CPDAG as the positive predictions.}













\textbf{Testing Data.} 
A testing instance consists of a groundtruth causal graph $G$, the structural equations $f_i$ and noise variables $\epsilon_i$ for each variable $X_i$, and an i.i.d. sample $D$. We use two categories of testing instances in our experiments:

\textit{Analytical Instances}: 
where $G$ is sampled from a DAG distribution $\mathcal{G}$, and $\{f_i,\epsilon_i\}$ sampled from a structural-equation distribution $\mathcal{F}$ and a noise meta-distribution $\mathcal{N}$. 
%It is called an analytical instance because 
%$\mathcal{G}$, $\mathcal{F}$, $\mathcal{N}$ all have analytical forms.
We consider three random graph distributions for $\mathcal{G}$: Watts-Strogatz (WS), Stochastic Block Model (SBM), Erdos-Rényi (ER); and three $\mathcal{F}$'s: random linear (L), Random Fourier Features (RFF), and conditional probability table (CPT). $\mathcal{N}$ is a uniform distribution over Gaussian's for continuous data, and a Dirichlet distribution over Multinomial Categorical distributions for discrete data. 
% We examined five combinations of these variations: \textbf{WS-L-G}, \textbf{SBM-L-G}, \textbf{WS-RFF-G}, \textbf{SBM-RFF-G}, \textbf{ER-CPT-MC}.
\textcolor{black}{We examine five combinations of testing instances: \textbf{WS-L-G}, \textbf{SBM-L-G}, \textbf{WS-RFF-G}, \textbf{SBM-RFF-G}, and \textbf{ER-CPT-MC}.}




\textit{Real-world Instance}:
% The classic dataset \textbf{Sachs} is used. 
% It consists of a data sample recording the concentration levels of 11 phosphorylated proteins in 853 human immune system cells, and of a causal graph over these 11 variables identified by \cite{sachs2005causal} based on expert consensus and experimental biology literature. 
% %\textcolor{red}{[is the concentration levels discretized?]}
The classic dataset \textbf{Sachs} is used \textcolor{black}{to evaluate performance in real-world scenarios}. 
It consists of a data sample recording the concentration levels of 11 phosphorylated proteins in 853 human immune system cells, and of a causal graph over these 11 variables identified by \cite{sachs2005causal} based on expert consensus and biology literature.








\textbf{Algorithms.}
As baselines, we compare with a series of representative unsupervised methods, including \textbf{PC} (using the recent Parallel-PC variation by ~\citet{le2016fast}), 
\textbf{GES}~\citep{chickering2002optimal}, 
\textbf{NOTEARS}~\citep{zheng2018dags}, 
% \textbf{NOTEARS-MLP}~\citep{zheng2018dags},
\textbf{GOLEM}~\citep{ng2020role}, 
\textbf{DAG-GNN}~\citep{yu2019dag},
\textbf{GRANDAG}~\citep{Lachapelle2020Gradient-Based}, \textbf{SLdisco} \citep{petersen2023causal} as well as \textbf{AVICI}~\citep{lorchamortized}, a DNN-based SCL method regarded as current state-of-the-art method.


For our method, besides the full \textbf{SiCL} implementation as described by Sec. \ref{sec:algo}, 
we also implement 
\textbf{SiCL-Node-Edge}, which predicts the causal graph using the node features and can be regarded as equivalent to AVICI, and 
\textbf{SiCL-no-PF}, which skips pairwise feature extraction and predicts the skeleton and v-tensor using node-wise features (see Appendix Fig. \ref{fig:abl}). 
% It is noteworthy that SiCL contains 6 layers on the node feature encoder module while SiCL-no-PF and AVICI contains 8 layers to eliminate any potential bias arising from differences in model size. 
Notably, SiCL contains 2.8M parameters, while SiCL-Node-Edge and SiCL-no-PF contain 3.2M parameters, because SiCL contains fewer layers on the node feature encoder to eliminate potential bias from size difference. 

For DNN-based SCL methods, the DNNs are trained with synthetic data where the causal graphs follow the Erdos-Rényi (ER) and Scale-Free (SF) models and the structural equations and noise variables follow the same distribution type as the corresponding testing data. 
Therefore, the disparities between the causal graph distribution at training and testing time help to examine the generality of SiCL in \textbf{OOD} settings to some extent.
% Our evaluation part is mostly about OOD setting, i.e., the distribution of test set is OOD w.r.t. the distribution of training set.
%For interface mismatches (e.g. CPDAG prediction with a graph prediction algorithm, or discrete-data experiment on an algorithm that originally only accepts continuous data), straightforward adaptations are applied. 
% See Appendix Sec. \ref{sec:app:exp:set} for more details in the experimental setting. 
More details of the experimental setting are presented in Appendix Sec. \ref{sec:app:exp:set}.







\subsection{Results on Synthetic Dataset} \label{sec:exp:gp}



We conduct a comprehensive comparison of SiCL with various baselines in both skeleton prediction and CPDAG prediction tasks.
The main results of metrics skeleton-F1 and orientation-F1 are presented in Tab. \ref{tab:epders}, and results on full metrics are provided in Appendix Tab. \ref{tab:epder}.
% We perform experiments in Tab. \ref{tab:epder} and Appendix Tab. \ref{tab:mc}-\ref{tab:mc2} on both continuous datasets and discrete datasets to evaluate the performance of the competing methods. 
On continuous data, DNN-based SCL methods (i.e., AVICI and SiCL) demonstrate consistent and obvious advantages over traditional approaches.
SiCL consistently outperforms the other methods on both skeleton prediction task and CPDAG prediction task.
On the other hand, some unsupervised methods achieve comparable performance among DNN-based SCL methods on the discrete data ER-CPT-MC. 
Nonetheless, our proposed SiCL emerges as the top performer, further substantiating its superiority in addressing the causal learning problem. 














\subsection{Results on Real-world Dataset} \label{sec:exp:erd}
\begin{table}
\centering
% \vspace{-0.45in}
\caption{Comparison on Sachs dataset.}
% \vspace{-0.05in}
\label{tab:sachs}
    \resizebox{\linewidth}{!}{%
    \begin{threeparttable}
\begin{tabular}{@{}ccccc@{}}
\toprule
 % & \multicolumn{2}{c}{CPDAG metrics} & \multicolumn{2}{c}{Skeleton Metrics} \\
 \multirow{2}{*}{Method} & \multicolumn{2}{c}{Skeleton Prediction }& \multicolumn{2}{c}{CPDAG Prediction} \\
 &  s-F1$\uparrow$ & s-Acc.$\uparrow$ & SHD$\downarrow$ & \#v-struc.$\downarrow$\\ \midrule
 PC& $68.6$& $80.0$ &$19$&$12$\\
 GES & $70.6$ & $81.8$ &$19$&$8$\\
 DAG-GNN & $21.1$ & $72.7$&$15$&$\mathbf{0}$ \\
 NOTEARS & $11.1$ & $70.9$ &$16 $ & $ \mathbf{0}$ \\
  GRAN-DAG & $45.5 $ & $78.2 $ &$ 12 $ & $\mathbf{0}$ \\
  GOLEM & $ 36.4$ & $ 74.5$ &$ 14 $ & $\mathbf{0}$ \\
  % CAM & $87.2$ & $90.9$ & $17$ & $6$\\
AVICI & $66.7 $&$ 83.5$  &$18 $&$ 14$\\ 
SiCL & $\mathbf{71.4}$ & $\mathbf{86.8}$&$\mathbf{6}$&$\mathbf{0}$   \\ 
\bottomrule
\end{tabular}
\end{threeparttable}
}
% \vspace{-0.05in}
\end{table}
To assess the practical applicability of SiCL, we conduct a comparison using the real-world dataset Sachs.
The discretized Sachs data obtained from the bnlearn library\footnote{https://www.bnlearn.com/} is used.
The DNN-based SCL methods are trained on random synthetic graphs, making this also an \textbf{OOD} prediction task.
The results are provided in Tab. \ref{tab:sachs}.

For the skeleton prediction task, SiCL performs the best, albeit with a modest gap (generally 1$\sim$3 scores higher than the runners-up). For the CPDAG prediction task, SiCL performs significantly better than all other methods (reducing SHD from 12 to 6, against the second best). Interestingly, the true causal DAG of the Sachs benchmark actually contains no v-structure, so any predicted v-structure is an error. We see that methods competitive with SiCL in skeleton prediction (AVICI, PC, GES) mistakenly predicted a large number of v-structures on the Sachs data, while SiCL correctly predict zero v-structure.


\begin{table}[!tb]
\centering
% \resizebox{\linewidth}{!}{%
\begin{threeparttable}
\caption{Ablation study of SiCL components. Full metrics are available in Appendix Tab. \ref{tab:fcplg}.}
\label{tab:cplg}
% \vspace{-0.05in}
\begin{tabular}{ccccc}
\toprule
 \multirow{2}{*}{Method} & \multicolumn{2}{c}{WS-L-G}& \multicolumn{2}{c}{SBM-L-G} \\
 & s-F1$\uparrow$ & o-F1$\uparrow$ &s-F1$\uparrow$ & o-F1$\uparrow$\\\midrule
 SiCL-Node-Edge & $39.9$ & $35.8$ & $84.3$ & $81.6$ \\
 SiCL-no-PF &$42.4$ & $37.9$ & $85.5$ & $82.2$ \\ 
 SiCL & $\mathbf{44.7}$ & $\mathbf{38.5}$ &$\mathbf{85.8}$& $\mathbf{82.7}$ \\
\bottomrule
\end{tabular}
\end{threeparttable}
% }
% \vspace{-0.2in}
\end{table}


\subsection{\textcolor{black}{Ablation Study}}


\begin{figure}[t]
    \centering
    % \vspace{-0.4in}
    \includegraphics[width=\linewidth]{figures/Cmp_on_of1_font.pdf}
    \caption{Comparison of SiCL-Node-Edge and SiCL-no-PF in o-F1 trend as observation samples increase on a constructed dataset.}
    \label{fig:cto}
    % \vspace{-0.15in}
\end{figure}
\textbf{Effectiveness of Learning Identifiable Structures.} \label{sec:exp:elis}As discussed in Section \ref{sec:met:lic}, SiCL focuses on learning MEC-identifiable causal structures rather than directly learning the adjacency matrix.
To verify the effectiveness of this idea,
% in practice and eliminate the bias of sampling from specifically designed distribution as much as possible, 
we compare SiCL-Node-Edge with SiCL-no-PF.
% to provide empirical support for the superiority of learning identifiable causal structures.
% evaluate the effectiveness of learning identifiable structures.
% test models on MEC-randomized WS-L-G and SBM-L-G datasets, where the training data is also correspondingly MEC-randomized.
% The comparisons between AVICI and SiCL-no-PF 
These two models share a similar node feature encoder architecture but have different learning targets: SiCL-Node-Edge predicts the adjacency matrix, while the SiCL-no-PF predicts the skeleton and v-tensor.
The results are shown in Table \ref{tab:cplg}.
Consistently, SiCL-no-PF demonstrates higher performance on both skeleton and CPDAG prediction tasks. 
This observation echoes our theoretical conclusion regarding the necessity and benefits of learning identifiable causal structures to improve overall performance.



To further underscore the significance of learning identifiable causal structures (especially in the asymptomatic sense), we conduct a comparative analysis using a specially constructed dataset, which contains six nodes forming an independent v-structure and a UT. 
Figure \ref{fig:cto} illustrates that the orientation F1 scores of CPDAG predictions from SiCL-Node-Edge suffer from an unavoidable error and do not improve with the addition of more observational samples. 
In contrast, predictions from SiCL-no-PF reach perfect accuracy, confirming the value of learning identifiable causal structures.





\textbf{Effectiveness of Pairwise Representation.} To assess the effectiveness of pairwise representation, we compare the full version of SiCL with a variant lacking pairwise features (SiCL-no-PF). As shown in Table \ref{tab:cplg}, the full-version SiCL consistently outperforms SiCL-no-PF in both skeleton prediction and CPDAG prediction tasks. Notably, we have intentionally set the model size of the full-version SiCL (2.8M parameters) to be smaller than that of SiCL-no-PF (3.2M parameters) so as to avoid any potential advantage from increased model complexity brought by the pairwise feature encoder module. The observed performance gains in this case underscore the critical role of pairwise features in identifying causal structures. Additionally, we conduct further comparisons across more diverse settings, with results detailed in Appendix Sec. \ref{sec:mpc}. These results demonstrate even more pronounced improvements in favor of SiCL, reinforcing the importance of pairwise representations in causal discovery.

\section{Conclusion}


We proposed SiCL, a novel DNN-based SCL approach designed to predict the corresponding skeleton and a set of v-structures. We showed that such design do not suffer from the (non-)identifiability limit \textcolor{black}{that exists in current architectures}. Moreover, SiCL is equipped with a pairwise encoder module to explicitly model relationships between node-pairs. 
Experimental results validated the effectiveness of these ideas. %showed that SiCL significantly outperforms other DNN-based SCL approaches. 
% \color{black}
% For limitations and future works, please refer to Appendix \ref{sec:lim}.
% \color{black}

This paper also introduces a few interesting open problems.
The proposed DNN model works in the canonical setting under the classic MEC theory, in which the skeleton and v-structures are the identifiable structure.
It can be an interesting future-work direction to explore how to learn other identifiable causal structure in other assumption settings following the same principle.
Due to the inherent complexity of DNNs, the explanation of the decision mechanism of our model remains an open question. 
Therefore, future work could consider to explore how decisions are made within the networks and provide some insights for traditional methods. 
Moreover, the proposed pairwise encoder modules needs $O(d^3)$ computational complexity, which may restrict its current application to scenarios with huge number of nodes.
Future work could focus on simplifying these operations or exploring features with less complexity (e,g., low rank features) to reduce the overall computational cost.


% This must be in the first 5 lines to tell arXiv to use pdfLaTeX, which is strongly recommended.
\pdfoutput=1
% In particular, the hyperref package requires pdfLaTeX in order to break URLs across lines.

\documentclass[11pt]{article}

% Change "review" to "final" to generate the final (sometimes called camera-ready) version.
% Change to "preprint" to generate a non-anonymous version with page numbers.
\usepackage{acl}

% Standard package includes
\usepackage{times}
\usepackage{latexsym}

% Draw tables
\usepackage{booktabs}
\usepackage{multirow}
\usepackage{xcolor}
\usepackage{colortbl}
\usepackage{array} 
\usepackage{amsmath}

\newcolumntype{C}{>{\centering\arraybackslash}p{0.07\textwidth}}
% For proper rendering and hyphenation of words containing Latin characters (including in bib files)
\usepackage[T1]{fontenc}
% For Vietnamese characters
% \usepackage[T5]{fontenc}
% See https://www.latex-project.org/help/documentation/encguide.pdf for other character sets
% This assumes your files are encoded as UTF8
\usepackage[utf8]{inputenc}

% This is not strictly necessary, and may be commented out,
% but it will improve the layout of the manuscript,
% and will typically save some space.
\usepackage{microtype}
\DeclareMathOperator*{\argmax}{arg\,max}
% This is also not strictly necessary, and may be commented out.
% However, it will improve the aesthetics of text in
% the typewriter font.
\usepackage{inconsolata}

%Including images in your LaTeX document requires adding
%additional package(s)
\usepackage{graphicx}
% If the title and author information does not fit in the area allocated, uncomment the following
%
%\setlength\titlebox{<dim>}
%
% and set <dim> to something 5cm or larger.

\title{Wi-Chat: Large Language Model Powered Wi-Fi Sensing}

% Author information can be set in various styles:
% For several authors from the same institution:
% \author{Author 1 \and ... \and Author n \\
%         Address line \\ ... \\ Address line}
% if the names do not fit well on one line use
%         Author 1 \\ {\bf Author 2} \\ ... \\ {\bf Author n} \\
% For authors from different institutions:
% \author{Author 1 \\ Address line \\  ... \\ Address line
%         \And  ... \And
%         Author n \\ Address line \\ ... \\ Address line}
% To start a separate ``row'' of authors use \AND, as in
% \author{Author 1 \\ Address line \\  ... \\ Address line
%         \AND
%         Author 2 \\ Address line \\ ... \\ Address line \And
%         Author 3 \\ Address line \\ ... \\ Address line}

% \author{First Author \\
%   Affiliation / Address line 1 \\
%   Affiliation / Address line 2 \\
%   Affiliation / Address line 3 \\
%   \texttt{email@domain} \\\And
%   Second Author \\
%   Affiliation / Address line 1 \\
%   Affiliation / Address line 2 \\
%   Affiliation / Address line 3 \\
%   \texttt{email@domain} \\}
% \author{Haohan Yuan \qquad Haopeng Zhang\thanks{corresponding author} \\ 
%   ALOHA Lab, University of Hawaii at Manoa \\
%   % Affiliation / Address line 2 \\
%   % Affiliation / Address line 3 \\
%   \texttt{\{haohany,haopengz\}@hawaii.edu}}
  
\author{
{Haopeng Zhang$\dag$\thanks{These authors contributed equally to this work.}, Yili Ren$\ddagger$\footnotemark[1], Haohan Yuan$\dag$, Jingzhe Zhang$\ddagger$, Yitong Shen$\ddagger$} \\
ALOHA Lab, University of Hawaii at Manoa$\dag$, University of South Florida$\ddagger$ \\
\{haopengz, haohany\}@hawaii.edu\\
\{yiliren, jingzhe, shen202\}@usf.edu\\}



  
%\author{
%  \textbf{First Author\textsuperscript{1}},
%  \textbf{Second Author\textsuperscript{1,2}},
%  \textbf{Third T. Author\textsuperscript{1}},
%  \textbf{Fourth Author\textsuperscript{1}},
%\\
%  \textbf{Fifth Author\textsuperscript{1,2}},
%  \textbf{Sixth Author\textsuperscript{1}},
%  \textbf{Seventh Author\textsuperscript{1}},
%  \textbf{Eighth Author \textsuperscript{1,2,3,4}},
%\\
%  \textbf{Ninth Author\textsuperscript{1}},
%  \textbf{Tenth Author\textsuperscript{1}},
%  \textbf{Eleventh E. Author\textsuperscript{1,2,3,4,5}},
%  \textbf{Twelfth Author\textsuperscript{1}},
%\\
%  \textbf{Thirteenth Author\textsuperscript{3}},
%  \textbf{Fourteenth F. Author\textsuperscript{2,4}},
%  \textbf{Fifteenth Author\textsuperscript{1}},
%  \textbf{Sixteenth Author\textsuperscript{1}},
%\\
%  \textbf{Seventeenth S. Author\textsuperscript{4,5}},
%  \textbf{Eighteenth Author\textsuperscript{3,4}},
%  \textbf{Nineteenth N. Author\textsuperscript{2,5}},
%  \textbf{Twentieth Author\textsuperscript{1}}
%\\
%\\
%  \textsuperscript{1}Affiliation 1,
%  \textsuperscript{2}Affiliation 2,
%  \textsuperscript{3}Affiliation 3,
%  \textsuperscript{4}Affiliation 4,
%  \textsuperscript{5}Affiliation 5
%\\
%  \small{
%    \textbf{Correspondence:} \href{mailto:email@domain}{email@domain}
%  }
%}

\begin{document}
\maketitle
\begin{abstract}
Recent advancements in Large Language Models (LLMs) have demonstrated remarkable capabilities across diverse tasks. However, their potential to integrate physical model knowledge for real-world signal interpretation remains largely unexplored. In this work, we introduce Wi-Chat, the first LLM-powered Wi-Fi-based human activity recognition system. We demonstrate that LLMs can process raw Wi-Fi signals and infer human activities by incorporating Wi-Fi sensing principles into prompts. Our approach leverages physical model insights to guide LLMs in interpreting Channel State Information (CSI) data without traditional signal processing techniques. Through experiments on real-world Wi-Fi datasets, we show that LLMs exhibit strong reasoning capabilities, achieving zero-shot activity recognition. These findings highlight a new paradigm for Wi-Fi sensing, expanding LLM applications beyond conventional language tasks and enhancing the accessibility of wireless sensing for real-world deployments.
\end{abstract}

\section{Introduction}

In today’s rapidly evolving digital landscape, the transformative power of web technologies has redefined not only how services are delivered but also how complex tasks are approached. Web-based systems have become increasingly prevalent in risk control across various domains. This widespread adoption is due their accessibility, scalability, and ability to remotely connect various types of users. For example, these systems are used for process safety management in industry~\cite{kannan2016web}, safety risk early warning in urban construction~\cite{ding2013development}, and safe monitoring of infrastructural systems~\cite{repetto2018web}. Within these web-based risk management systems, the source search problem presents a huge challenge. Source search refers to the task of identifying the origin of a risky event, such as a gas leak and the emission point of toxic substances. This source search capability is crucial for effective risk management and decision-making.

Traditional approaches to implementing source search capabilities into the web systems often rely on solely algorithmic solutions~\cite{ristic2016study}. These methods, while relatively straightforward to implement, often struggle to achieve acceptable performances due to algorithmic local optima and complex unknown environments~\cite{zhao2020searching}. More recently, web crowdsourcing has emerged as a promising alternative for tackling the source search problem by incorporating human efforts in these web systems on-the-fly~\cite{zhao2024user}. This approach outsources the task of addressing issues encountered during the source search process to human workers, leveraging their capabilities to enhance system performance.

These solutions often employ a human-AI collaborative way~\cite{zhao2023leveraging} where algorithms handle exploration-exploitation and report the encountered problems while human workers resolve complex decision-making bottlenecks to help the algorithms getting rid of local deadlocks~\cite{zhao2022crowd}. Although effective, this paradigm suffers from two inherent limitations: increased operational costs from continuous human intervention, and slow response times of human workers due to sequential decision-making. These challenges motivate our investigation into developing autonomous systems that preserve human-like reasoning capabilities while reducing dependency on massive crowdsourced labor.

Furthermore, recent advancements in large language models (LLMs)~\cite{chang2024survey} and multi-modal LLMs (MLLMs)~\cite{huang2023chatgpt} have unveiled promising avenues for addressing these challenges. One clear opportunity involves the seamless integration of visual understanding and linguistic reasoning for robust decision-making in search tasks. However, whether large models-assisted source search is really effective and efficient for improving the current source search algorithms~\cite{ji2022source} remains unknown. \textit{To address the research gap, we are particularly interested in answering the following two research questions in this work:}

\textbf{\textit{RQ1: }}How can source search capabilities be integrated into web-based systems to support decision-making in time-sensitive risk management scenarios? 
% \sq{I mention ``time-sensitive'' here because I feel like we shall say something about the response time -- LLM has to be faster than humans}

\textbf{\textit{RQ2: }}How can MLLMs and LLMs enhance the effectiveness and efficiency of existing source search algorithms? 

% \textit{\textbf{RQ2:}} To what extent does the performance of large models-assisted search align with or approach the effectiveness of human-AI collaborative search? 

To answer the research questions, we propose a novel framework called Auto-\
S$^2$earch (\textbf{Auto}nomous \textbf{S}ource \textbf{Search}) and implement a prototype system that leverages advanced web technologies to simulate real-world conditions for zero-shot source search. Unlike traditional methods that rely on pre-defined heuristics or extensive human intervention, AutoS$^2$earch employs a carefully designed prompt that encapsulates human rationales, thereby guiding the MLLM to generate coherent and accurate scene descriptions from visual inputs about four directional choices. Based on these language-based descriptions, the LLM is enabled to determine the optimal directional choice through chain-of-thought (CoT) reasoning. Comprehensive empirical validation demonstrates that AutoS$^2$-\ 
earch achieves a success rate of 95–98\%, closely approaching the performance of human-AI collaborative search across 20 benchmark scenarios~\cite{zhao2023leveraging}. 

Our work indicates that the role of humans in future web crowdsourcing tasks may evolve from executors to validators or supervisors. Furthermore, incorporating explanations of LLM decisions into web-based system interfaces has the potential to help humans enhance task performance in risk control.






\section{Related Work}
\label{sec:relatedworks}

% \begin{table*}[t]
% \centering 
% \renewcommand\arraystretch{0.98}
% \fontsize{8}{10}\selectfont \setlength{\tabcolsep}{0.4em}
% \begin{tabular}{@{}lc|cc|cc|cc@{}}
% \toprule
% \textbf{Methods}           & \begin{tabular}[c]{@{}c@{}}\textbf{Training}\\ \textbf{Paradigm}\end{tabular} & \begin{tabular}[c]{@{}c@{}}\textbf{$\#$ PT Data}\\ \textbf{(Tokens)}\end{tabular} & \begin{tabular}[c]{@{}c@{}}\textbf{$\#$ IFT Data}\\ \textbf{(Samples)}\end{tabular} & \textbf{Code}  & \begin{tabular}[c]{@{}c@{}}\textbf{Natural}\\ \textbf{Language}\end{tabular} & \begin{tabular}[c]{@{}c@{}}\textbf{Action}\\ \textbf{Trajectories}\end{tabular} & \begin{tabular}[c]{@{}c@{}}\textbf{API}\\ \textbf{Documentation}\end{tabular}\\ \midrule 
% NexusRaven~\citep{srinivasan2023nexusraven} & IFT & - & - & \textcolor{green}{\CheckmarkBold} & \textcolor{green}{\CheckmarkBold} &\textcolor{red}{\XSolidBrush}&\textcolor{red}{\XSolidBrush}\\
% AgentInstruct~\citep{zeng2023agenttuning} & IFT & - & 2k & \textcolor{green}{\CheckmarkBold} & \textcolor{green}{\CheckmarkBold} &\textcolor{red}{\XSolidBrush}&\textcolor{red}{\XSolidBrush} \\
% AgentEvol~\citep{xi2024agentgym} & IFT & - & 14.5k & \textcolor{green}{\CheckmarkBold} & \textcolor{green}{\CheckmarkBold} &\textcolor{green}{\CheckmarkBold}&\textcolor{red}{\XSolidBrush} \\
% Gorilla~\citep{patil2023gorilla}& IFT & - & 16k & \textcolor{green}{\CheckmarkBold} & \textcolor{green}{\CheckmarkBold} &\textcolor{red}{\XSolidBrush}&\textcolor{green}{\CheckmarkBold}\\
% OpenFunctions-v2~\citep{patil2023gorilla} & IFT & - & 65k & \textcolor{green}{\CheckmarkBold} & \textcolor{green}{\CheckmarkBold} &\textcolor{red}{\XSolidBrush}&\textcolor{green}{\CheckmarkBold}\\
% LAM~\citep{zhang2024agentohana} & IFT & - & 42.6k & \textcolor{green}{\CheckmarkBold} & \textcolor{green}{\CheckmarkBold} &\textcolor{green}{\CheckmarkBold}&\textcolor{red}{\XSolidBrush} \\
% xLAM~\citep{liu2024apigen} & IFT & - & 60k & \textcolor{green}{\CheckmarkBold} & \textcolor{green}{\CheckmarkBold} &\textcolor{green}{\CheckmarkBold}&\textcolor{red}{\XSolidBrush} \\\midrule
% LEMUR~\citep{xu2024lemur} & PT & 90B & 300k & \textcolor{green}{\CheckmarkBold} & \textcolor{green}{\CheckmarkBold} &\textcolor{green}{\CheckmarkBold}&\textcolor{red}{\XSolidBrush}\\
% \rowcolor{teal!12} \method & PT & 103B & 95k & \textcolor{green}{\CheckmarkBold} & \textcolor{green}{\CheckmarkBold} & \textcolor{green}{\CheckmarkBold} & \textcolor{green}{\CheckmarkBold} \\
% \bottomrule
% \end{tabular}
% \caption{Summary of existing tuning- and pretraining-based LLM agents with their training sample sizes. "PT" and "IFT" denote "Pre-Training" and "Instruction Fine-Tuning", respectively. }
% \label{tab:related}
% \end{table*}

\begin{table*}[ht]
\begin{threeparttable}
\centering 
\renewcommand\arraystretch{0.98}
\fontsize{7}{9}\selectfont \setlength{\tabcolsep}{0.2em}
\begin{tabular}{@{}l|c|c|ccc|cc|cc|cccc@{}}
\toprule
\textbf{Methods} & \textbf{Datasets}           & \begin{tabular}[c]{@{}c@{}}\textbf{Training}\\ \textbf{Paradigm}\end{tabular} & \begin{tabular}[c]{@{}c@{}}\textbf{\# PT Data}\\ \textbf{(Tokens)}\end{tabular} & \begin{tabular}[c]{@{}c@{}}\textbf{\# IFT Data}\\ \textbf{(Samples)}\end{tabular} & \textbf{\# APIs} & \textbf{Code}  & \begin{tabular}[c]{@{}c@{}}\textbf{Nat.}\\ \textbf{Lang.}\end{tabular} & \begin{tabular}[c]{@{}c@{}}\textbf{Action}\\ \textbf{Traj.}\end{tabular} & \begin{tabular}[c]{@{}c@{}}\textbf{API}\\ \textbf{Doc.}\end{tabular} & \begin{tabular}[c]{@{}c@{}}\textbf{Func.}\\ \textbf{Call}\end{tabular} & \begin{tabular}[c]{@{}c@{}}\textbf{Multi.}\\ \textbf{Step}\end{tabular}  & \begin{tabular}[c]{@{}c@{}}\textbf{Plan}\\ \textbf{Refine}\end{tabular}  & \begin{tabular}[c]{@{}c@{}}\textbf{Multi.}\\ \textbf{Turn}\end{tabular}\\ \midrule 
\multicolumn{13}{l}{\emph{Instruction Finetuning-based LLM Agents for Intrinsic Reasoning}}  \\ \midrule
FireAct~\cite{chen2023fireact} & FireAct & IFT & - & 2.1K & 10 & \textcolor{red}{\XSolidBrush} &\textcolor{green}{\CheckmarkBold} &\textcolor{green}{\CheckmarkBold}  & \textcolor{red}{\XSolidBrush} &\textcolor{green}{\CheckmarkBold} & \textcolor{red}{\XSolidBrush} &\textcolor{green}{\CheckmarkBold} & \textcolor{red}{\XSolidBrush} \\
ToolAlpaca~\cite{tang2023toolalpaca} & ToolAlpaca & IFT & - & 4.0K & 400 & \textcolor{red}{\XSolidBrush} &\textcolor{green}{\CheckmarkBold} &\textcolor{green}{\CheckmarkBold} & \textcolor{red}{\XSolidBrush} &\textcolor{green}{\CheckmarkBold} & \textcolor{red}{\XSolidBrush}  &\textcolor{green}{\CheckmarkBold} & \textcolor{red}{\XSolidBrush}  \\
ToolLLaMA~\cite{qin2023toolllm} & ToolBench & IFT & - & 12.7K & 16,464 & \textcolor{red}{\XSolidBrush} &\textcolor{green}{\CheckmarkBold} &\textcolor{green}{\CheckmarkBold} &\textcolor{red}{\XSolidBrush} &\textcolor{green}{\CheckmarkBold}&\textcolor{green}{\CheckmarkBold}&\textcolor{green}{\CheckmarkBold} &\textcolor{green}{\CheckmarkBold}\\
AgentEvol~\citep{xi2024agentgym} & AgentTraj-L & IFT & - & 14.5K & 24 &\textcolor{red}{\XSolidBrush} & \textcolor{green}{\CheckmarkBold} &\textcolor{green}{\CheckmarkBold}&\textcolor{red}{\XSolidBrush} &\textcolor{green}{\CheckmarkBold}&\textcolor{red}{\XSolidBrush} &\textcolor{red}{\XSolidBrush} &\textcolor{green}{\CheckmarkBold}\\
Lumos~\cite{yin2024agent} & Lumos & IFT  & - & 20.0K & 16 &\textcolor{red}{\XSolidBrush} & \textcolor{green}{\CheckmarkBold} & \textcolor{green}{\CheckmarkBold} &\textcolor{red}{\XSolidBrush} & \textcolor{green}{\CheckmarkBold} & \textcolor{green}{\CheckmarkBold} &\textcolor{red}{\XSolidBrush} & \textcolor{green}{\CheckmarkBold}\\
Agent-FLAN~\cite{chen2024agent} & Agent-FLAN & IFT & - & 24.7K & 20 &\textcolor{red}{\XSolidBrush} & \textcolor{green}{\CheckmarkBold} & \textcolor{green}{\CheckmarkBold} &\textcolor{red}{\XSolidBrush} & \textcolor{green}{\CheckmarkBold}& \textcolor{green}{\CheckmarkBold}&\textcolor{red}{\XSolidBrush} & \textcolor{green}{\CheckmarkBold}\\
AgentTuning~\citep{zeng2023agenttuning} & AgentInstruct & IFT & - & 35.0K & - &\textcolor{red}{\XSolidBrush} & \textcolor{green}{\CheckmarkBold} & \textcolor{green}{\CheckmarkBold} &\textcolor{red}{\XSolidBrush} & \textcolor{green}{\CheckmarkBold} &\textcolor{red}{\XSolidBrush} &\textcolor{red}{\XSolidBrush} & \textcolor{green}{\CheckmarkBold}\\\midrule
\multicolumn{13}{l}{\emph{Instruction Finetuning-based LLM Agents for Function Calling}} \\\midrule
NexusRaven~\citep{srinivasan2023nexusraven} & NexusRaven & IFT & - & - & 116 & \textcolor{green}{\CheckmarkBold} & \textcolor{green}{\CheckmarkBold}  & \textcolor{green}{\CheckmarkBold} &\textcolor{red}{\XSolidBrush} & \textcolor{green}{\CheckmarkBold} &\textcolor{red}{\XSolidBrush} &\textcolor{red}{\XSolidBrush}&\textcolor{red}{\XSolidBrush}\\
Gorilla~\citep{patil2023gorilla} & Gorilla & IFT & - & 16.0K & 1,645 & \textcolor{green}{\CheckmarkBold} &\textcolor{red}{\XSolidBrush} &\textcolor{red}{\XSolidBrush}&\textcolor{green}{\CheckmarkBold} &\textcolor{green}{\CheckmarkBold} &\textcolor{red}{\XSolidBrush} &\textcolor{red}{\XSolidBrush} &\textcolor{red}{\XSolidBrush}\\
OpenFunctions-v2~\citep{patil2023gorilla} & OpenFunctions-v2 & IFT & - & 65.0K & - & \textcolor{green}{\CheckmarkBold} & \textcolor{green}{\CheckmarkBold} &\textcolor{red}{\XSolidBrush} &\textcolor{green}{\CheckmarkBold} &\textcolor{green}{\CheckmarkBold} &\textcolor{red}{\XSolidBrush} &\textcolor{red}{\XSolidBrush} &\textcolor{red}{\XSolidBrush}\\
API Pack~\cite{guo2024api} & API Pack & IFT & - & 1.1M & 11,213 &\textcolor{green}{\CheckmarkBold} &\textcolor{red}{\XSolidBrush} &\textcolor{green}{\CheckmarkBold} &\textcolor{red}{\XSolidBrush} &\textcolor{green}{\CheckmarkBold} &\textcolor{red}{\XSolidBrush}&\textcolor{red}{\XSolidBrush}&\textcolor{red}{\XSolidBrush}\\ 
LAM~\citep{zhang2024agentohana} & AgentOhana & IFT & - & 42.6K & - & \textcolor{green}{\CheckmarkBold} & \textcolor{green}{\CheckmarkBold} &\textcolor{green}{\CheckmarkBold}&\textcolor{red}{\XSolidBrush} &\textcolor{green}{\CheckmarkBold}&\textcolor{red}{\XSolidBrush}&\textcolor{green}{\CheckmarkBold}&\textcolor{green}{\CheckmarkBold}\\
xLAM~\citep{liu2024apigen} & APIGen & IFT & - & 60.0K & 3,673 & \textcolor{green}{\CheckmarkBold} & \textcolor{green}{\CheckmarkBold} &\textcolor{green}{\CheckmarkBold}&\textcolor{red}{\XSolidBrush} &\textcolor{green}{\CheckmarkBold}&\textcolor{red}{\XSolidBrush}&\textcolor{green}{\CheckmarkBold}&\textcolor{green}{\CheckmarkBold}\\\midrule
\multicolumn{13}{l}{\emph{Pretraining-based LLM Agents}}  \\\midrule
% LEMUR~\citep{xu2024lemur} & PT & 90B & 300.0K & - & \textcolor{green}{\CheckmarkBold} & \textcolor{green}{\CheckmarkBold} &\textcolor{green}{\CheckmarkBold}&\textcolor{red}{\XSolidBrush} & \textcolor{red}{\XSolidBrush} &\textcolor{green}{\CheckmarkBold} &\textcolor{red}{\XSolidBrush}&\textcolor{red}{\XSolidBrush}\\
\rowcolor{teal!12} \method & \dataset & PT & 103B & 95.0K  & 76,537  & \textcolor{green}{\CheckmarkBold} & \textcolor{green}{\CheckmarkBold} & \textcolor{green}{\CheckmarkBold} & \textcolor{green}{\CheckmarkBold} & \textcolor{green}{\CheckmarkBold} & \textcolor{green}{\CheckmarkBold} & \textcolor{green}{\CheckmarkBold} & \textcolor{green}{\CheckmarkBold}\\
\bottomrule
\end{tabular}
% \begin{tablenotes}
%     \item $^*$ In addition, the StarCoder-API can offer 4.77M more APIs.
% \end{tablenotes}
\caption{Summary of existing instruction finetuning-based LLM agents for intrinsic reasoning and function calling, along with their training resources and sample sizes. "PT" and "IFT" denote "Pre-Training" and "Instruction Fine-Tuning", respectively.}
\vspace{-2ex}
\label{tab:related}
\end{threeparttable}
\end{table*}

\noindent \textbf{Prompting-based LLM Agents.} Due to the lack of agent-specific pre-training corpus, existing LLM agents rely on either prompt engineering~\cite{hsieh2023tool,lu2024chameleon,yao2022react,wang2023voyager} or instruction fine-tuning~\cite{chen2023fireact,zeng2023agenttuning} to understand human instructions, decompose high-level tasks, generate grounded plans, and execute multi-step actions. 
However, prompting-based methods mainly depend on the capabilities of backbone LLMs (usually commercial LLMs), failing to introduce new knowledge and struggling to generalize to unseen tasks~\cite{sun2024adaplanner,zhuang2023toolchain}. 

\noindent \textbf{Instruction Finetuning-based LLM Agents.} Considering the extensive diversity of APIs and the complexity of multi-tool instructions, tool learning inherently presents greater challenges than natural language tasks, such as text generation~\cite{qin2023toolllm}.
Post-training techniques focus more on instruction following and aligning output with specific formats~\cite{patil2023gorilla,hao2024toolkengpt,qin2023toolllm,schick2024toolformer}, rather than fundamentally improving model knowledge or capabilities. 
Moreover, heavy fine-tuning can hinder generalization or even degrade performance in non-agent use cases, potentially suppressing the original base model capabilities~\cite{ghosh2024a}.

\noindent \textbf{Pretraining-based LLM Agents.} While pre-training serves as an essential alternative, prior works~\cite{nijkamp2023codegen,roziere2023code,xu2024lemur,patil2023gorilla} have primarily focused on improving task-specific capabilities (\eg, code generation) instead of general-domain LLM agents, due to single-source, uni-type, small-scale, and poor-quality pre-training data. 
Existing tool documentation data for agent training either lacks diverse real-world APIs~\cite{patil2023gorilla, tang2023toolalpaca} or is constrained to single-tool or single-round tool execution. 
Furthermore, trajectory data mostly imitate expert behavior or follow function-calling rules with inferior planning and reasoning, failing to fully elicit LLMs' capabilities and handle complex instructions~\cite{qin2023toolllm}. 
Given a wide range of candidate API functions, each comprising various function names and parameters available at every planning step, identifying globally optimal solutions and generalizing across tasks remains highly challenging.



\section{Preliminaries}
\label{Preliminaries}
\begin{figure*}[t]
    \centering
    \includegraphics[width=0.95\linewidth]{fig/HealthGPT_Framework.png}
    \caption{The \ourmethod{} architecture integrates hierarchical visual perception and H-LoRA, employing a task-specific hard router to select visual features and H-LoRA plugins, ultimately generating outputs with an autoregressive manner.}
    \label{fig:architecture}
\end{figure*}
\noindent\textbf{Large Vision-Language Models.} 
The input to a LVLM typically consists of an image $x^{\text{img}}$ and a discrete text sequence $x^{\text{txt}}$. The visual encoder $\mathcal{E}^{\text{img}}$ converts the input image $x^{\text{img}}$ into a sequence of visual tokens $\mathcal{V} = [v_i]_{i=1}^{N_v}$, while the text sequence $x^{\text{txt}}$ is mapped into a sequence of text tokens $\mathcal{T} = [t_i]_{i=1}^{N_t}$ using an embedding function $\mathcal{E}^{\text{txt}}$. The LLM $\mathcal{M_\text{LLM}}(\cdot|\theta)$ models the joint probability of the token sequence $\mathcal{U} = \{\mathcal{V},\mathcal{T}\}$, which is expressed as:
\begin{equation}
    P_\theta(R | \mathcal{U}) = \prod_{i=1}^{N_r} P_\theta(r_i | \{\mathcal{U}, r_{<i}\}),
\end{equation}
where $R = [r_i]_{i=1}^{N_r}$ is the text response sequence. The LVLM iteratively generates the next token $r_i$ based on $r_{<i}$. The optimization objective is to minimize the cross-entropy loss of the response $\mathcal{R}$.
% \begin{equation}
%     \mathcal{L}_{\text{VLM}} = \mathbb{E}_{R|\mathcal{U}}\left[-\log P_\theta(R | \mathcal{U})\right]
% \end{equation}
It is worth noting that most LVLMs adopt a design paradigm based on ViT, alignment adapters, and pre-trained LLMs\cite{liu2023llava,liu2024improved}, enabling quick adaptation to downstream tasks.


\noindent\textbf{VQGAN.}
VQGAN~\cite{esser2021taming} employs latent space compression and indexing mechanisms to effectively learn a complete discrete representation of images. VQGAN first maps the input image $x^{\text{img}}$ to a latent representation $z = \mathcal{E}(x)$ through a encoder $\mathcal{E}$. Then, the latent representation is quantized using a codebook $\mathcal{Z} = \{z_k\}_{k=1}^K$, generating a discrete index sequence $\mathcal{I} = [i_m]_{m=1}^N$, where $i_m \in \mathcal{Z}$ represents the quantized code index:
\begin{equation}
    \mathcal{I} = \text{Quantize}(z|\mathcal{Z}) = \arg\min_{z_k \in \mathcal{Z}} \| z - z_k \|_2.
\end{equation}
In our approach, the discrete index sequence $\mathcal{I}$ serves as a supervisory signal for the generation task, enabling the model to predict the index sequence $\hat{\mathcal{I}}$ from input conditions such as text or other modality signals.  
Finally, the predicted index sequence $\hat{\mathcal{I}}$ is upsampled by the VQGAN decoder $G$, generating the high-quality image $\hat{x}^\text{img} = G(\hat{\mathcal{I}})$.



\noindent\textbf{Low Rank Adaptation.} 
LoRA\cite{hu2021lora} effectively captures the characteristics of downstream tasks by introducing low-rank adapters. The core idea is to decompose the bypass weight matrix $\Delta W\in\mathbb{R}^{d^{\text{in}} \times d^{\text{out}}}$ into two low-rank matrices $ \{A \in \mathbb{R}^{d^{\text{in}} \times r}, B \in \mathbb{R}^{r \times d^{\text{out}}} \}$, where $ r \ll \min\{d^{\text{in}}, d^{\text{out}}\} $, significantly reducing learnable parameters. The output with the LoRA adapter for the input $x$ is then given by:
\begin{equation}
    h = x W_0 + \alpha x \Delta W/r = x W_0 + \alpha xAB/r,
\end{equation}
where matrix $ A $ is initialized with a Gaussian distribution, while the matrix $ B $ is initialized as a zero matrix. The scaling factor $ \alpha/r $ controls the impact of $ \Delta W $ on the model.

\section{HealthGPT}
\label{Method}


\subsection{Unified Autoregressive Generation.}  
% As shown in Figure~\ref{fig:architecture}, 
\ourmethod{} (Figure~\ref{fig:architecture}) utilizes a discrete token representation that covers both text and visual outputs, unifying visual comprehension and generation as an autoregressive task. 
For comprehension, $\mathcal{M}_\text{llm}$ receives the input joint sequence $\mathcal{U}$ and outputs a series of text token $\mathcal{R} = [r_1, r_2, \dots, r_{N_r}]$, where $r_i \in \mathcal{V}_{\text{txt}}$, and $\mathcal{V}_{\text{txt}}$ represents the LLM's vocabulary:
\begin{equation}
    P_\theta(\mathcal{R} \mid \mathcal{U}) = \prod_{i=1}^{N_r} P_\theta(r_i \mid \mathcal{U}, r_{<i}).
\end{equation}
For generation, $\mathcal{M}_\text{llm}$ first receives a special start token $\langle \text{START\_IMG} \rangle$, then generates a series of tokens corresponding to the VQGAN indices $\mathcal{I} = [i_1, i_2, \dots, i_{N_i}]$, where $i_j \in \mathcal{V}_{\text{vq}}$, and $\mathcal{V}_{\text{vq}}$ represents the index range of VQGAN. Upon completion of generation, the LLM outputs an end token $\langle \text{END\_IMG} \rangle$:
\begin{equation}
    P_\theta(\mathcal{I} \mid \mathcal{U}) = \prod_{j=1}^{N_i} P_\theta(i_j \mid \mathcal{U}, i_{<j}).
\end{equation}
Finally, the generated index sequence $\mathcal{I}$ is fed into the decoder $G$, which reconstructs the target image $\hat{x}^{\text{img}} = G(\mathcal{I})$.

\subsection{Hierarchical Visual Perception}  
Given the differences in visual perception between comprehension and generation tasks—where the former focuses on abstract semantics and the latter emphasizes complete semantics—we employ ViT to compress the image into discrete visual tokens at multiple hierarchical levels.
Specifically, the image is converted into a series of features $\{f_1, f_2, \dots, f_L\}$ as it passes through $L$ ViT blocks.

To address the needs of various tasks, the hidden states are divided into two types: (i) \textit{Concrete-grained features} $\mathcal{F}^{\text{Con}} = \{f_1, f_2, \dots, f_k\}, k < L$, derived from the shallower layers of ViT, containing sufficient global features, suitable for generation tasks; 
(ii) \textit{Abstract-grained features} $\mathcal{F}^{\text{Abs}} = \{f_{k+1}, f_{k+2}, \dots, f_L\}$, derived from the deeper layers of ViT, which contain abstract semantic information closer to the text space, suitable for comprehension tasks.

The task type $T$ (comprehension or generation) determines which set of features is selected as the input for the downstream large language model:
\begin{equation}
    \mathcal{F}^{\text{img}}_T =
    \begin{cases}
        \mathcal{F}^{\text{Con}}, & \text{if } T = \text{generation task} \\
        \mathcal{F}^{\text{Abs}}, & \text{if } T = \text{comprehension task}
    \end{cases}
\end{equation}
We integrate the image features $\mathcal{F}^{\text{img}}_T$ and text features $\mathcal{T}$ into a joint sequence through simple concatenation, which is then fed into the LLM $\mathcal{M}_{\text{llm}}$ for autoregressive generation.
% :
% \begin{equation}
%     \mathcal{R} = \mathcal{M}_{\text{llm}}(\mathcal{U}|\theta), \quad \mathcal{U} = [\mathcal{F}^{\text{img}}_T; \mathcal{T}]
% \end{equation}
\subsection{Heterogeneous Knowledge Adaptation}
We devise H-LoRA, which stores heterogeneous knowledge from comprehension and generation tasks in separate modules and dynamically routes to extract task-relevant knowledge from these modules. 
At the task level, for each task type $ T $, we dynamically assign a dedicated H-LoRA submodule $ \theta^T $, which is expressed as:
\begin{equation}
    \mathcal{R} = \mathcal{M}_\text{LLM}(\mathcal{U}|\theta, \theta^T), \quad \theta^T = \{A^T, B^T, \mathcal{R}^T_\text{outer}\}.
\end{equation}
At the feature level for a single task, H-LoRA integrates the idea of Mixture of Experts (MoE)~\cite{masoudnia2014mixture} and designs an efficient matrix merging and routing weight allocation mechanism, thus avoiding the significant computational delay introduced by matrix splitting in existing MoELoRA~\cite{luo2024moelora}. Specifically, we first merge the low-rank matrices (rank = r) of $ k $ LoRA experts into a unified matrix:
\begin{equation}
    \mathbf{A}^{\text{merged}}, \mathbf{B}^{\text{merged}} = \text{Concat}(\{A_i\}_1^k), \text{Concat}(\{B_i\}_1^k),
\end{equation}
where $ \mathbf{A}^{\text{merged}} \in \mathbb{R}^{d^\text{in} \times rk} $ and $ \mathbf{B}^{\text{merged}} \in \mathbb{R}^{rk \times d^\text{out}} $. The $k$-dimension routing layer generates expert weights $ \mathcal{W} \in \mathbb{R}^{\text{token\_num} \times k} $ based on the input hidden state $ x $, and these are expanded to $ \mathbb{R}^{\text{token\_num} \times rk} $ as follows:
\begin{equation}
    \mathcal{W}^\text{expanded} = \alpha k \mathcal{W} / r \otimes \mathbf{1}_r,
\end{equation}
where $ \otimes $ denotes the replication operation.
The overall output of H-LoRA is computed as:
\begin{equation}
    \mathcal{O}^\text{H-LoRA} = (x \mathbf{A}^{\text{merged}} \odot \mathcal{W}^\text{expanded}) \mathbf{B}^{\text{merged}},
\end{equation}
where $ \odot $ represents element-wise multiplication. Finally, the output of H-LoRA is added to the frozen pre-trained weights to produce the final output:
\begin{equation}
    \mathcal{O} = x W_0 + \mathcal{O}^\text{H-LoRA}.
\end{equation}
% In summary, H-LoRA is a task-based dynamic PEFT method that achieves high efficiency in single-task fine-tuning.

\subsection{Training Pipeline}

\begin{figure}[t]
    \centering
    \hspace{-4mm}
    \includegraphics[width=0.94\linewidth]{fig/data.pdf}
    \caption{Data statistics of \texttt{VL-Health}. }
    \label{fig:data}
\end{figure}
\noindent \textbf{1st Stage: Multi-modal Alignment.} 
In the first stage, we design separate visual adapters and H-LoRA submodules for medical unified tasks. For the medical comprehension task, we train abstract-grained visual adapters using high-quality image-text pairs to align visual embeddings with textual embeddings, thereby enabling the model to accurately describe medical visual content. During this process, the pre-trained LLM and its corresponding H-LoRA submodules remain frozen. In contrast, the medical generation task requires training concrete-grained adapters and H-LoRA submodules while keeping the LLM frozen. Meanwhile, we extend the textual vocabulary to include multimodal tokens, enabling the support of additional VQGAN vector quantization indices. The model trains on image-VQ pairs, endowing the pre-trained LLM with the capability for image reconstruction. This design ensures pixel-level consistency of pre- and post-LVLM. The processes establish the initial alignment between the LLM’s outputs and the visual inputs.

\noindent \textbf{2nd Stage: Heterogeneous H-LoRA Plugin Adaptation.}  
The submodules of H-LoRA share the word embedding layer and output head but may encounter issues such as bias and scale inconsistencies during training across different tasks. To ensure that the multiple H-LoRA plugins seamlessly interface with the LLMs and form a unified base, we fine-tune the word embedding layer and output head using a small amount of mixed data to maintain consistency in the model weights. Specifically, during this stage, all H-LoRA submodules for different tasks are kept frozen, with only the word embedding layer and output head being optimized. Through this stage, the model accumulates foundational knowledge for unified tasks by adapting H-LoRA plugins.

\begin{table*}[!t]
\centering
\caption{Comparison of \ourmethod{} with other LVLMs and unified multi-modal models on medical visual comprehension tasks. \textbf{Bold} and \underline{underlined} text indicates the best performance and second-best performance, respectively.}
\resizebox{\textwidth}{!}{
\begin{tabular}{c|lcc|cccccccc|c}
\toprule
\rowcolor[HTML]{E9F3FE} &  &  &  & \multicolumn{2}{c}{\textbf{VQA-RAD \textuparrow}} & \multicolumn{2}{c}{\textbf{SLAKE \textuparrow}} & \multicolumn{2}{c}{\textbf{PathVQA \textuparrow}} &  &  &  \\ 
\cline{5-10}
\rowcolor[HTML]{E9F3FE}\multirow{-2}{*}{\textbf{Type}} & \multirow{-2}{*}{\textbf{Model}} & \multirow{-2}{*}{\textbf{\# Params}} & \multirow{-2}{*}{\makecell{\textbf{Medical} \\ \textbf{LVLM}}} & \textbf{close} & \textbf{all} & \textbf{close} & \textbf{all} & \textbf{close} & \textbf{all} & \multirow{-2}{*}{\makecell{\textbf{MMMU} \\ \textbf{-Med}}\textuparrow} & \multirow{-2}{*}{\textbf{OMVQA}\textuparrow} & \multirow{-2}{*}{\textbf{Avg. \textuparrow}} \\ 
\midrule \midrule
\multirow{9}{*}{\textbf{Comp. Only}} 
& Med-Flamingo & 8.3B & \Large \ding{51} & 58.6 & 43.0 & 47.0 & 25.5 & 61.9 & 31.3 & 28.7 & 34.9 & 41.4 \\
& LLaVA-Med & 7B & \Large \ding{51} & 60.2 & 48.1 & 58.4 & 44.8 & 62.3 & 35.7 & 30.0 & 41.3 & 47.6 \\
& HuatuoGPT-Vision & 7B & \Large \ding{51} & 66.9 & 53.0 & 59.8 & 49.1 & 52.9 & 32.0 & 42.0 & 50.0 & 50.7 \\
& BLIP-2 & 6.7B & \Large \ding{55} & 43.4 & 36.8 & 41.6 & 35.3 & 48.5 & 28.8 & 27.3 & 26.9 & 36.1 \\
& LLaVA-v1.5 & 7B & \Large \ding{55} & 51.8 & 42.8 & 37.1 & 37.7 & 53.5 & 31.4 & 32.7 & 44.7 & 41.5 \\
& InstructBLIP & 7B & \Large \ding{55} & 61.0 & 44.8 & 66.8 & 43.3 & 56.0 & 32.3 & 25.3 & 29.0 & 44.8 \\
& Yi-VL & 6B & \Large \ding{55} & 52.6 & 42.1 & 52.4 & 38.4 & 54.9 & 30.9 & 38.0 & 50.2 & 44.9 \\
& InternVL2 & 8B & \Large \ding{55} & 64.9 & 49.0 & 66.6 & 50.1 & 60.0 & 31.9 & \underline{43.3} & 54.5 & 52.5\\
& Llama-3.2 & 11B & \Large \ding{55} & 68.9 & 45.5 & 72.4 & 52.1 & 62.8 & 33.6 & 39.3 & 63.2 & 54.7 \\
\midrule
\multirow{5}{*}{\textbf{Comp. \& Gen.}} 
& Show-o & 1.3B & \Large \ding{55} & 50.6 & 33.9 & 31.5 & 17.9 & 52.9 & 28.2 & 22.7 & 45.7 & 42.6 \\
& Unified-IO 2 & 7B & \Large \ding{55} & 46.2 & 32.6 & 35.9 & 21.9 & 52.5 & 27.0 & 25.3 & 33.0 & 33.8 \\
& Janus & 1.3B & \Large \ding{55} & 70.9 & 52.8 & 34.7 & 26.9 & 51.9 & 27.9 & 30.0 & 26.8 & 33.5 \\
& \cellcolor[HTML]{DAE0FB}HealthGPT-M3 & \cellcolor[HTML]{DAE0FB}3.8B & \cellcolor[HTML]{DAE0FB}\Large \ding{51} & \cellcolor[HTML]{DAE0FB}\underline{73.7} & \cellcolor[HTML]{DAE0FB}\underline{55.9} & \cellcolor[HTML]{DAE0FB}\underline{74.6} & \cellcolor[HTML]{DAE0FB}\underline{56.4} & \cellcolor[HTML]{DAE0FB}\underline{78.7} & \cellcolor[HTML]{DAE0FB}\underline{39.7} & \cellcolor[HTML]{DAE0FB}\underline{43.3} & \cellcolor[HTML]{DAE0FB}\underline{68.5} & \cellcolor[HTML]{DAE0FB}\underline{61.3} \\
& \cellcolor[HTML]{DAE0FB}HealthGPT-L14 & \cellcolor[HTML]{DAE0FB}14B & \cellcolor[HTML]{DAE0FB}\Large \ding{51} & \cellcolor[HTML]{DAE0FB}\textbf{77.7} & \cellcolor[HTML]{DAE0FB}\textbf{58.3} & \cellcolor[HTML]{DAE0FB}\textbf{76.4} & \cellcolor[HTML]{DAE0FB}\textbf{64.5} & \cellcolor[HTML]{DAE0FB}\textbf{85.9} & \cellcolor[HTML]{DAE0FB}\textbf{44.4} & \cellcolor[HTML]{DAE0FB}\textbf{49.2} & \cellcolor[HTML]{DAE0FB}\textbf{74.4} & \cellcolor[HTML]{DAE0FB}\textbf{66.4} \\
\bottomrule
\end{tabular}
}
\label{tab:results}
\end{table*}
\begin{table*}[ht]
    \centering
    \caption{The experimental results for the four modality conversion tasks.}
    \resizebox{\textwidth}{!}{
    \begin{tabular}{l|ccc|ccc|ccc|ccc}
        \toprule
        \rowcolor[HTML]{E9F3FE} & \multicolumn{3}{c}{\textbf{CT to MRI (Brain)}} & \multicolumn{3}{c}{\textbf{CT to MRI (Pelvis)}} & \multicolumn{3}{c}{\textbf{MRI to CT (Brain)}} & \multicolumn{3}{c}{\textbf{MRI to CT (Pelvis)}} \\
        \cline{2-13}
        \rowcolor[HTML]{E9F3FE}\multirow{-2}{*}{\textbf{Model}}& \textbf{SSIM $\uparrow$} & \textbf{PSNR $\uparrow$} & \textbf{MSE $\downarrow$} & \textbf{SSIM $\uparrow$} & \textbf{PSNR $\uparrow$} & \textbf{MSE $\downarrow$} & \textbf{SSIM $\uparrow$} & \textbf{PSNR $\uparrow$} & \textbf{MSE $\downarrow$} & \textbf{SSIM $\uparrow$} & \textbf{PSNR $\uparrow$} & \textbf{MSE $\downarrow$} \\
        \midrule \midrule
        pix2pix & 71.09 & 32.65 & 36.85 & 59.17 & 31.02 & 51.91 & 78.79 & 33.85 & 28.33 & 72.31 & 32.98 & 36.19 \\
        CycleGAN & 54.76 & 32.23 & 40.56 & 54.54 & 30.77 & 55.00 & 63.75 & 31.02 & 52.78 & 50.54 & 29.89 & 67.78 \\
        BBDM & {71.69} & {32.91} & {34.44} & 57.37 & 31.37 & 48.06 & \textbf{86.40} & 34.12 & 26.61 & {79.26} & 33.15 & 33.60 \\
        Vmanba & 69.54 & 32.67 & 36.42 & {63.01} & {31.47} & {46.99} & 79.63 & 34.12 & 26.49 & 77.45 & 33.53 & 31.85 \\
        DiffMa & 71.47 & 32.74 & 35.77 & 62.56 & 31.43 & 47.38 & 79.00 & {34.13} & {26.45} & 78.53 & {33.68} & {30.51} \\
        \rowcolor[HTML]{DAE0FB}HealthGPT-M3 & \underline{79.38} & \underline{33.03} & \underline{33.48} & \underline{71.81} & \underline{31.83} & \underline{43.45} & {85.06} & \textbf{34.40} & \textbf{25.49} & \underline{84.23} & \textbf{34.29} & \textbf{27.99} \\
        \rowcolor[HTML]{DAE0FB}HealthGPT-L14 & \textbf{79.73} & \textbf{33.10} & \textbf{32.96} & \textbf{71.92} & \textbf{31.87} & \textbf{43.09} & \underline{85.31} & \underline{34.29} & \underline{26.20} & \textbf{84.96} & \underline{34.14} & \underline{28.13} \\
        \bottomrule
    \end{tabular}
    }
    \label{tab:conversion}
\end{table*}

\noindent \textbf{3rd Stage: Visual Instruction Fine-Tuning.}  
In the third stage, we introduce additional task-specific data to further optimize the model and enhance its adaptability to downstream tasks such as medical visual comprehension (e.g., medical QA, medical dialogues, and report generation) or generation tasks (e.g., super-resolution, denoising, and modality conversion). Notably, by this stage, the word embedding layer and output head have been fine-tuned, only the H-LoRA modules and adapter modules need to be trained. This strategy significantly improves the model's adaptability and flexibility across different tasks.


\section{Experiment}
\label{s:experiment}

\subsection{Data Description}
We evaluate our method on FI~\cite{you2016building}, Twitter\_LDL~\cite{yang2017learning} and Artphoto~\cite{machajdik2010affective}.
FI is a public dataset built from Flickr and Instagram, with 23,308 images and eight emotion categories, namely \textit{amusement}, \textit{anger}, \textit{awe},  \textit{contentment}, \textit{disgust}, \textit{excitement},  \textit{fear}, and \textit{sadness}. 
% Since images in FI are all copyrighted by law, some images are corrupted now, so we remove these samples and retain 21,828 images.
% T4SA contains images from Twitter, which are classified into three categories: \textit{positive}, \textit{neutral}, and \textit{negative}. In this paper, we adopt the base version of B-T4SA, which contains 470,586 images and provides text descriptions of the corresponding tweets.
Twitter\_LDL contains 10,045 images from Twitter, with the same eight categories as the FI dataset.
% 。
For these two datasets, they are randomly split into 80\%
training and 20\% testing set.
Artphoto contains 806 artistic photos from the DeviantArt website, which we use to further evaluate the zero-shot capability of our model.
% on the small-scale dataset.
% We construct and publicly release the first image sentiment analysis dataset containing metadata.
% 。

% Based on these datasets, we are the first to construct and publicly release metadata-enhanced image sentiment analysis datasets. These datasets include scenes, tags, descriptions, and corresponding confidence scores, and are available at this link for future research purposes.


% 
\begin{table}[t]
\centering
% \begin{center}
\caption{Overall performance of different models on FI and Twitter\_LDL datasets.}
\label{tab:cap1}
% \resizebox{\linewidth}{!}
{
\begin{tabular}{l|c|c|c|c}
\hline
\multirow{2}{*}{\textbf{Model}} & \multicolumn{2}{c|}{\textbf{FI}}  & \multicolumn{2}{c}{\textbf{Twitter\_LDL}} \\ \cline{2-5} 
  & \textbf{Accuracy} & \textbf{F1} & \textbf{Accuracy} & \textbf{F1}  \\ \hline
% (\rownumber)~AlexNet~\cite{krizhevsky2017imagenet}  & 58.13\% & 56.35\%  & 56.24\%& 55.02\%  \\ 
% (\rownumber)~VGG16~\cite{simonyan2014very}  & 63.75\%& 63.08\%  & 59.34\%& 59.02\%  \\ 
(\rownumber)~ResNet101~\cite{he2016deep} & 66.16\%& 65.56\%  & 62.02\% & 61.34\%  \\ 
(\rownumber)~CDA~\cite{han2023boosting} & 66.71\%& 65.37\%  & 64.14\% & 62.85\%  \\ 
(\rownumber)~CECCN~\cite{ruan2024color} & 67.96\%& 66.74\%  & 64.59\%& 64.72\% \\ 
(\rownumber)~EmoVIT~\cite{xie2024emovit} & 68.09\%& 67.45\%  & 63.12\% & 61.97\%  \\ 
(\rownumber)~ComLDL~\cite{zhang2022compound} & 68.83\%& 67.28\%  & 65.29\% & 63.12\%  \\ 
(\rownumber)~WSDEN~\cite{li2023weakly} & 69.78\%& 69.61\%  & 67.04\% & 65.49\% \\ 
(\rownumber)~ECWA~\cite{deng2021emotion} & 70.87\%& 69.08\%  & 67.81\% & 66.87\%  \\ 
(\rownumber)~EECon~\cite{yang2023exploiting} & 71.13\%& 68.34\%  & 64.27\%& 63.16\%  \\ 
(\rownumber)~MAM~\cite{zhang2024affective} & 71.44\%  & 70.83\% & 67.18\%  & 65.01\%\\ 
(\rownumber)~TGCA-PVT~\cite{chen2024tgca}   & 73.05\%  & 71.46\% & 69.87\%  & 68.32\% \\ 
(\rownumber)~OEAN~\cite{zhang2024object}   & 73.40\%  & 72.63\% & 70.52\%  & 69.47\% \\ \hline
(\rownumber)~\shortname  & \textbf{79.48\%} & \textbf{79.22\%} & \textbf{74.12\%} & \textbf{73.09\%} \\ \hline
\end{tabular}
}
\vspace{-6mm}
% \end{center}
\end{table}
% 

\subsection{Experiment Setting}
% \subsubsection{Model Setting.}
% 
\textbf{Model Setting:}
For feature representation, we set $k=10$ to select object tags, and adopt clip-vit-base-patch32 as the pre-trained model for unified feature representation.
Moreover, we empirically set $(d_e, d_h, d_k, d_s) = (512, 128, 16, 64)$, and set the classification class $L$ to 8.

% 

\textbf{Training Setting:}
To initialize the model, we set all weights such as $\boldsymbol{W}$ following the truncated normal distribution, and use AdamW optimizer with the learning rate of $1 \times 10^{-4}$.
% warmup scheduler of cosine, warmup steps of 2000.
Furthermore, we set the batch size to 32 and the epoch of the training process to 200.
During the implementation, we utilize \textit{PyTorch} to build our entire model.
% , and our project codes are publicly available at https://github.com/zzmyrep/MESN.
% Our project codes as well as data are all publicly available on GitHub\footnote{https://github.com/zzmyrep/KBCEN}.
% Code is available at \href{https://github.com/zzmyrep/KBCEN}{https://github.com/zzmyrep/KBCEN}.

\textbf{Evaluation Metrics:}
Following~\cite{zhang2024affective, chen2024tgca, zhang2024object}, we adopt \textit{accuracy} and \textit{F1} as our evaluation metrics to measure the performance of different methods for image sentiment analysis. 



\subsection{Experiment Result}
% We compare our model against the following baselines: AlexNet~\cite{krizhevsky2017imagenet}, VGG16~\cite{simonyan2014very}, ResNet101~\cite{he2016deep}, CECCN~\cite{ruan2024color}, EmoVIT~\cite{xie2024emovit}, WSCNet~\cite{yang2018weakly}, ECWA~\cite{deng2021emotion}, EECon~\cite{yang2023exploiting}, MAM~\cite{zhang2024affective} and TGCA-PVT~\cite{chen2024tgca}, and the overall results are summarized in Table~\ref{tab:cap1}.
We compare our model against several baselines, and the overall results are summarized in Table~\ref{tab:cap1}.
We observe that our model achieves the best performance in both accuracy and F1 metrics, significantly outperforming the previous models. 
This superior performance is mainly attributed to our effective utilization of metadata to enhance image sentiment analysis, as well as the exceptional capability of the unified sentiment transformer framework we developed. These results strongly demonstrate that our proposed method can bring encouraging performance for image sentiment analysis.

\setcounter{magicrownumbers}{0} 
\begin{table}[t]
\begin{center}
\caption{Ablation study of~\shortname~on FI dataset.} 
% \vspace{1mm}
\label{tab:cap2}
\resizebox{.9\linewidth}{!}
{
\begin{tabular}{lcc}
  \hline
  \textbf{Model} & \textbf{Accuracy} & \textbf{F1} \\
  \hline
  (\rownumber)~Ours (w/o vision) & 65.72\% & 64.54\% \\
  (\rownumber)~Ours (w/o text description) & 74.05\% & 72.58\% \\
  (\rownumber)~Ours (w/o object tag) & 77.45\% & 76.84\% \\
  (\rownumber)~Ours (w/o scene tag) & 78.47\% & 78.21\% \\
  \hline
  (\rownumber)~Ours (w/o unified embedding) & 76.41\% & 76.23\% \\
  (\rownumber)~Ours (w/o adaptive learning) & 76.83\% & 76.56\% \\
  (\rownumber)~Ours (w/o cross-modal fusion) & 76.85\% & 76.49\% \\
  \hline
  (\rownumber)~Ours  & \textbf{79.48\%} & \textbf{79.22\%} \\
  \hline
\end{tabular}
}
\end{center}
\vspace{-5mm}
\end{table}


\begin{figure}[t]
\centering
% \vspace{-2mm}
\includegraphics[width=0.42\textwidth]{fig/2dvisual-linux4-paper2.pdf}
\caption{Visualization of feature distribution on eight categories before (left) and after (right) model processing.}
% 
\label{fig:visualization}
\vspace{-5mm}
\end{figure}

\subsection{Ablation Performance}
In this subsection, we conduct an ablation study to examine which component is really important for performance improvement. The results are reported in Table~\ref{tab:cap2}.

For information utilization, we observe a significant decline in model performance when visual features are removed. Additionally, the performance of \shortname~decreases when different metadata are removed separately, which means that text description, object tag, and scene tag are all critical for image sentiment analysis.
Recalling the model architecture, we separately remove transformer layers of the unified representation module, the adaptive learning module, and the cross-modal fusion module, replacing them with MLPs of the same parameter scale.
In this way, we can observe varying degrees of decline in model performance, indicating that these modules are indispensable for our model to achieve better performance.

\subsection{Visualization}
% 


% % 开始使用minipage进行左右排列
% \begin{minipage}[t]{0.45\textwidth}  % 子图1宽度为45%
%     \centering
%     \includegraphics[width=\textwidth]{2dvisual.pdf}  % 插入图片
%     \captionof{figure}{Visualization of feature distribution.}  % 使用captionof添加图片标题
%     \label{fig:visualization}
% \end{minipage}


% \begin{figure}[t]
% \centering
% \vspace{-2mm}
% \includegraphics[width=0.45\textwidth]{fig/2dvisual.pdf}
% \caption{Visualization of feature distribution.}
% \label{fig:visualization}
% % \vspace{-4mm}
% \end{figure}

% \begin{figure}[t]
% \centering
% \vspace{-2mm}
% \includegraphics[width=0.45\textwidth]{fig/2dvisual-linux3-paper.pdf}
% \caption{Visualization of feature distribution.}
% \label{fig:visualization}
% % \vspace{-4mm}
% \end{figure}



\begin{figure}[tbp]   
\vspace{-4mm}
  \centering            
  \subfloat[Depth of adaptive learning layers]   
  {
    \label{fig:subfig1}\includegraphics[width=0.22\textwidth]{fig/fig_sensitivity-a5}
  }
  \subfloat[Depth of fusion layers]
  {
    % \label{fig:subfig2}\includegraphics[width=0.22\textwidth]{fig/fig_sensitivity-b2}
    \label{fig:subfig2}\includegraphics[width=0.22\textwidth]{fig/fig_sensitivity-b2-num.pdf}
  }
  \caption{Sensitivity study of \shortname~on different depth. }   
  \label{fig:fig_sensitivity}  
\vspace{-2mm}
\end{figure}

% \begin{figure}[htbp]
% \centerline{\includegraphics{2dvisual.pdf}}
% \caption{Visualization of feature distribution.}
% \label{fig:visualization}
% \end{figure}

% In Fig.~\ref{fig:visualization}, we use t-SNE~\cite{van2008visualizing} to reduce the dimension of data features for visualization, Figure in left represents the metadata features before model processing, the features are obtained by embedding through the CLIP model, and figure in right shows the features of the data after model processing, it can be observed that after the model processing, the data with different label categories fall in different regions in the space, therefore, we can conclude that the Therefore, we can conclude that the model can effectively utilize the information contained in the metadata and use it to guide the model for classification.

In Fig.~\ref{fig:visualization}, we use t-SNE~\cite{van2008visualizing} to reduce the dimension of data features for visualization.
The left figure shows metadata features before being processed by our model (\textit{i.e.}, embedded by CLIP), while the right shows the distribution of features after being processed by our model.
We can observe that after the model processing, data with the same label are closer to each other, while others are farther away.
Therefore, it shows that the model can effectively utilize the information contained in the metadata and use it to guide the classification process.

\subsection{Sensitivity Analysis}
% 
In this subsection, we conduct a sensitivity analysis to figure out the effect of different depth settings of adaptive learning layers and fusion layers. 
% In this subsection, we conduct a sensitivity analysis to figure out the effect of different depth settings on the model. 
% Fig.~\ref{fig:fig_sensitivity} presents the effect of different depth settings of adaptive learning layers and fusion layers. 
Taking Fig.~\ref{fig:fig_sensitivity} (a) as an example, the model performance improves with increasing depth, reaching the best performance at a depth of 4.
% Taking Fig.~\ref{fig:fig_sensitivity} (a) as an example, the performance of \shortname~improves with the increase of depth at first, reaching the best performance at a depth of 4.
When the depth continues to increase, the accuracy decreases to varying degrees.
Similar results can be observed in Fig.~\ref{fig:fig_sensitivity} (b).
Therefore, we set their depths to 4 and 6 respectively to achieve the best results.

% Through our experiments, we can observe that the effect of modifying these hyperparameters on the results of the experiments is very weak, and the surface model is not sensitive to the hyperparameters.


\subsection{Zero-shot Capability}
% 

% (1)~GCH~\cite{2010Analyzing} & 21.78\% & (5)~RA-DLNet~\cite{2020A} & 34.01\% \\ \hline
% (2)~WSCNet~\cite{2019WSCNet}  & 30.25\% & (6)~CECCN~\cite{ruan2024color} & 43.83\% \\ \hline
% (3)~PCNN~\cite{2015Robust} & 31.68\%  & (7)~EmoVIT~\cite{xie2024emovit} & 44.90\% \\ \hline
% (4)~AR~\cite{2018Visual} & 32.67\% & (8)~Ours (Zero-shot) & 47.83\% \\ \hline


\begin{table}[t]
\centering
\caption{Zero-shot capability of \shortname.}
\label{tab:cap3}
\resizebox{1\linewidth}{!}
{
\begin{tabular}{lc|lc}
\hline
\textbf{Model} & \textbf{Accuracy} & \textbf{Model} & \textbf{Accuracy} \\ \hline
(1)~WSCNet~\cite{2019WSCNet}  & 30.25\% & (5)~MAM~\cite{zhang2024affective} & 39.56\%  \\ \hline
(2)~AR~\cite{2018Visual} & 32.67\% & (6)~CECCN~\cite{ruan2024color} & 43.83\% \\ \hline
(3)~RA-DLNet~\cite{2020A} & 34.01\%  & (7)~EmoVIT~\cite{xie2024emovit} & 44.90\% \\ \hline
(4)~CDA~\cite{han2023boosting} & 38.64\% & (8)~Ours (Zero-shot) & 47.83\% \\ \hline
\end{tabular}
}
\vspace{-5mm}
\end{table}

% We use the model trained on the FI dataset to test on the artphoto dataset to verify the model's generalization ability as well as robustness to other distributed datasets.
% We can observe that the MESN model shows strong competitiveness in terms of accuracy when compared to other trained models, which suggests that the model has a good generalization ability in the OOD task.

To validate the model's generalization ability and robustness to other distributed datasets, we directly test the model trained on the FI dataset, without training on Artphoto. 
% As observed in Table 3, compared to other models trained on Artphoto, we achieve highly competitive zero-shot performance, indicating that the model has good generalization ability in out-of-distribution tasks.
From Table~\ref{tab:cap3}, we can observe that compared with other models trained on Artphoto, we achieve competitive zero-shot performance, which shows that the model has good generalization ability in out-of-distribution tasks.


%%%%%%%%%%%%
%  E2E     %
%%%%%%%%%%%%


\section{Conclusion}
In this paper, we introduced Wi-Chat, the first LLM-powered Wi-Fi-based human activity recognition system that integrates the reasoning capabilities of large language models with the sensing potential of wireless signals. Our experimental results on a self-collected Wi-Fi CSI dataset demonstrate the promising potential of LLMs in enabling zero-shot Wi-Fi sensing. These findings suggest a new paradigm for human activity recognition that does not rely on extensive labeled data. We hope future research will build upon this direction, further exploring the applications of LLMs in signal processing domains such as IoT, mobile sensing, and radar-based systems.

\section*{Limitations}
While our work represents the first attempt to leverage LLMs for processing Wi-Fi signals, it is a preliminary study focused on a relatively simple task: Wi-Fi-based human activity recognition. This choice allows us to explore the feasibility of LLMs in wireless sensing but also comes with certain limitations.

Our approach primarily evaluates zero-shot performance, which, while promising, may still lag behind traditional supervised learning methods in highly complex or fine-grained recognition tasks. Besides, our study is limited to a controlled environment with a self-collected dataset, and the generalizability of LLMs to diverse real-world scenarios with varying Wi-Fi conditions, environmental interference, and device heterogeneity remains an open question.

Additionally, we have yet to explore the full potential of LLMs in more advanced Wi-Fi sensing applications, such as fine-grained gesture recognition, occupancy detection, and passive health monitoring. Future work should investigate the scalability of LLM-based approaches, their robustness to domain shifts, and their integration with multimodal sensing techniques in broader IoT applications.


% Bibliography entries for the entire Anthology, followed by custom entries
%\bibliography{anthology,custom}
% Custom bibliography entries only
\bibliography{main}
\newpage
\appendix

\section{Experiment prompts}
\label{sec:prompt}
The prompts used in the LLM experiments are shown in the following Table~\ref{tab:prompts}.

\definecolor{titlecolor}{rgb}{0.9, 0.5, 0.1}
\definecolor{anscolor}{rgb}{0.2, 0.5, 0.8}
\definecolor{labelcolor}{HTML}{48a07e}
\begin{table*}[h]
	\centering
	
 % \vspace{-0.2cm}
	
	\begin{center}
		\begin{tikzpicture}[
				chatbox_inner/.style={rectangle, rounded corners, opacity=0, text opacity=1, font=\sffamily\scriptsize, text width=5in, text height=9pt, inner xsep=6pt, inner ysep=6pt},
				chatbox_prompt_inner/.style={chatbox_inner, align=flush left, xshift=0pt, text height=11pt},
				chatbox_user_inner/.style={chatbox_inner, align=flush left, xshift=0pt},
				chatbox_gpt_inner/.style={chatbox_inner, align=flush left, xshift=0pt},
				chatbox/.style={chatbox_inner, draw=black!25, fill=gray!7, opacity=1, text opacity=0},
				chatbox_prompt/.style={chatbox, align=flush left, fill=gray!1.5, draw=black!30, text height=10pt},
				chatbox_user/.style={chatbox, align=flush left},
				chatbox_gpt/.style={chatbox, align=flush left},
				chatbox2/.style={chatbox_gpt, fill=green!25},
				chatbox3/.style={chatbox_gpt, fill=red!20, draw=black!20},
				chatbox4/.style={chatbox_gpt, fill=yellow!30},
				labelbox/.style={rectangle, rounded corners, draw=black!50, font=\sffamily\scriptsize\bfseries, fill=gray!5, inner sep=3pt},
			]
											
			\node[chatbox_user] (q1) {
				\textbf{System prompt}
				\newline
				\newline
				You are a helpful and precise assistant for segmenting and labeling sentences. We would like to request your help on curating a dataset for entity-level hallucination detection.
				\newline \newline
                We will give you a machine generated biography and a list of checked facts about the biography. Each fact consists of a sentence and a label (True/False). Please do the following process. First, breaking down the biography into words. Second, by referring to the provided list of facts, merging some broken down words in the previous step to form meaningful entities. For example, ``strategic thinking'' should be one entity instead of two. Third, according to the labels in the list of facts, labeling each entity as True or False. Specifically, for facts that share a similar sentence structure (\eg, \textit{``He was born on Mach 9, 1941.''} (\texttt{True}) and \textit{``He was born in Ramos Mejia.''} (\texttt{False})), please first assign labels to entities that differ across atomic facts. For example, first labeling ``Mach 9, 1941'' (\texttt{True}) and ``Ramos Mejia'' (\texttt{False}) in the above case. For those entities that are the same across atomic facts (\eg, ``was born'') or are neutral (\eg, ``he,'' ``in,'' and ``on''), please label them as \texttt{True}. For the cases that there is no atomic fact that shares the same sentence structure, please identify the most informative entities in the sentence and label them with the same label as the atomic fact while treating the rest of the entities as \texttt{True}. In the end, output the entities and labels in the following format:
                \begin{itemize}[nosep]
                    \item Entity 1 (Label 1)
                    \item Entity 2 (Label 2)
                    \item ...
                    \item Entity N (Label N)
                \end{itemize}
                % \newline \newline
                Here are two examples:
                \newline\newline
                \textbf{[Example 1]}
                \newline
                [The start of the biography]
                \newline
                \textcolor{titlecolor}{Marianne McAndrew is an American actress and singer, born on November 21, 1942, in Cleveland, Ohio. She began her acting career in the late 1960s, appearing in various television shows and films.}
                \newline
                [The end of the biography]
                \newline \newline
                [The start of the list of checked facts]
                \newline
                \textcolor{anscolor}{[Marianne McAndrew is an American. (False); Marianne McAndrew is an actress. (True); Marianne McAndrew is a singer. (False); Marianne McAndrew was born on November 21, 1942. (False); Marianne McAndrew was born in Cleveland, Ohio. (False); She began her acting career in the late 1960s. (True); She has appeared in various television shows. (True); She has appeared in various films. (True)]}
                \newline
                [The end of the list of checked facts]
                \newline \newline
                [The start of the ideal output]
                \newline
                \textcolor{labelcolor}{[Marianne McAndrew (True); is (True); an (True); American (False); actress (True); and (True); singer (False); , (True); born (True); on (True); November 21, 1942 (False); , (True); in (True); Cleveland, Ohio (False); . (True); She (True); began (True); her (True); acting career (True); in (True); the late 1960s (True); , (True); appearing (True); in (True); various (True); television shows (True); and (True); films (True); . (True)]}
                \newline
                [The end of the ideal output]
				\newline \newline
                \textbf{[Example 2]}
                \newline
                [The start of the biography]
                \newline
                \textcolor{titlecolor}{Doug Sheehan is an American actor who was born on April 27, 1949, in Santa Monica, California. He is best known for his roles in soap operas, including his portrayal of Joe Kelly on ``General Hospital'' and Ben Gibson on ``Knots Landing.''}
                \newline
                [The end of the biography]
                \newline \newline
                [The start of the list of checked facts]
                \newline
                \textcolor{anscolor}{[Doug Sheehan is an American. (True); Doug Sheehan is an actor. (True); Doug Sheehan was born on April 27, 1949. (True); Doug Sheehan was born in Santa Monica, California. (False); He is best known for his roles in soap operas. (True); He portrayed Joe Kelly. (True); Joe Kelly was in General Hospital. (True); General Hospital is a soap opera. (True); He portrayed Ben Gibson. (True); Ben Gibson was in Knots Landing. (True); Knots Landing is a soap opera. (True)]}
                \newline
                [The end of the list of checked facts]
                \newline \newline
                [The start of the ideal output]
                \newline
                \textcolor{labelcolor}{[Doug Sheehan (True); is (True); an (True); American (True); actor (True); who (True); was born (True); on (True); April 27, 1949 (True); in (True); Santa Monica, California (False); . (True); He (True); is (True); best known (True); for (True); his roles in soap operas (True); , (True); including (True); in (True); his portrayal (True); of (True); Joe Kelly (True); on (True); ``General Hospital'' (True); and (True); Ben Gibson (True); on (True); ``Knots Landing.'' (True)]}
                \newline
                [The end of the ideal output]
				\newline \newline
				\textbf{User prompt}
				\newline
				\newline
				[The start of the biography]
				\newline
				\textcolor{magenta}{\texttt{\{BIOGRAPHY\}}}
				\newline
				[The ebd of the biography]
				\newline \newline
				[The start of the list of checked facts]
				\newline
				\textcolor{magenta}{\texttt{\{LIST OF CHECKED FACTS\}}}
				\newline
				[The end of the list of checked facts]
			};
			\node[chatbox_user_inner] (q1_text) at (q1) {
				\textbf{System prompt}
				\newline
				\newline
				You are a helpful and precise assistant for segmenting and labeling sentences. We would like to request your help on curating a dataset for entity-level hallucination detection.
				\newline \newline
                We will give you a machine generated biography and a list of checked facts about the biography. Each fact consists of a sentence and a label (True/False). Please do the following process. First, breaking down the biography into words. Second, by referring to the provided list of facts, merging some broken down words in the previous step to form meaningful entities. For example, ``strategic thinking'' should be one entity instead of two. Third, according to the labels in the list of facts, labeling each entity as True or False. Specifically, for facts that share a similar sentence structure (\eg, \textit{``He was born on Mach 9, 1941.''} (\texttt{True}) and \textit{``He was born in Ramos Mejia.''} (\texttt{False})), please first assign labels to entities that differ across atomic facts. For example, first labeling ``Mach 9, 1941'' (\texttt{True}) and ``Ramos Mejia'' (\texttt{False}) in the above case. For those entities that are the same across atomic facts (\eg, ``was born'') or are neutral (\eg, ``he,'' ``in,'' and ``on''), please label them as \texttt{True}. For the cases that there is no atomic fact that shares the same sentence structure, please identify the most informative entities in the sentence and label them with the same label as the atomic fact while treating the rest of the entities as \texttt{True}. In the end, output the entities and labels in the following format:
                \begin{itemize}[nosep]
                    \item Entity 1 (Label 1)
                    \item Entity 2 (Label 2)
                    \item ...
                    \item Entity N (Label N)
                \end{itemize}
                % \newline \newline
                Here are two examples:
                \newline\newline
                \textbf{[Example 1]}
                \newline
                [The start of the biography]
                \newline
                \textcolor{titlecolor}{Marianne McAndrew is an American actress and singer, born on November 21, 1942, in Cleveland, Ohio. She began her acting career in the late 1960s, appearing in various television shows and films.}
                \newline
                [The end of the biography]
                \newline \newline
                [The start of the list of checked facts]
                \newline
                \textcolor{anscolor}{[Marianne McAndrew is an American. (False); Marianne McAndrew is an actress. (True); Marianne McAndrew is a singer. (False); Marianne McAndrew was born on November 21, 1942. (False); Marianne McAndrew was born in Cleveland, Ohio. (False); She began her acting career in the late 1960s. (True); She has appeared in various television shows. (True); She has appeared in various films. (True)]}
                \newline
                [The end of the list of checked facts]
                \newline \newline
                [The start of the ideal output]
                \newline
                \textcolor{labelcolor}{[Marianne McAndrew (True); is (True); an (True); American (False); actress (True); and (True); singer (False); , (True); born (True); on (True); November 21, 1942 (False); , (True); in (True); Cleveland, Ohio (False); . (True); She (True); began (True); her (True); acting career (True); in (True); the late 1960s (True); , (True); appearing (True); in (True); various (True); television shows (True); and (True); films (True); . (True)]}
                \newline
                [The end of the ideal output]
				\newline \newline
                \textbf{[Example 2]}
                \newline
                [The start of the biography]
                \newline
                \textcolor{titlecolor}{Doug Sheehan is an American actor who was born on April 27, 1949, in Santa Monica, California. He is best known for his roles in soap operas, including his portrayal of Joe Kelly on ``General Hospital'' and Ben Gibson on ``Knots Landing.''}
                \newline
                [The end of the biography]
                \newline \newline
                [The start of the list of checked facts]
                \newline
                \textcolor{anscolor}{[Doug Sheehan is an American. (True); Doug Sheehan is an actor. (True); Doug Sheehan was born on April 27, 1949. (True); Doug Sheehan was born in Santa Monica, California. (False); He is best known for his roles in soap operas. (True); He portrayed Joe Kelly. (True); Joe Kelly was in General Hospital. (True); General Hospital is a soap opera. (True); He portrayed Ben Gibson. (True); Ben Gibson was in Knots Landing. (True); Knots Landing is a soap opera. (True)]}
                \newline
                [The end of the list of checked facts]
                \newline \newline
                [The start of the ideal output]
                \newline
                \textcolor{labelcolor}{[Doug Sheehan (True); is (True); an (True); American (True); actor (True); who (True); was born (True); on (True); April 27, 1949 (True); in (True); Santa Monica, California (False); . (True); He (True); is (True); best known (True); for (True); his roles in soap operas (True); , (True); including (True); in (True); his portrayal (True); of (True); Joe Kelly (True); on (True); ``General Hospital'' (True); and (True); Ben Gibson (True); on (True); ``Knots Landing.'' (True)]}
                \newline
                [The end of the ideal output]
				\newline \newline
				\textbf{User prompt}
				\newline
				\newline
				[The start of the biography]
				\newline
				\textcolor{magenta}{\texttt{\{BIOGRAPHY\}}}
				\newline
				[The ebd of the biography]
				\newline \newline
				[The start of the list of checked facts]
				\newline
				\textcolor{magenta}{\texttt{\{LIST OF CHECKED FACTS\}}}
				\newline
				[The end of the list of checked facts]
			};
		\end{tikzpicture}
        \caption{GPT-4o prompt for labeling hallucinated entities.}\label{tb:gpt-4-prompt}
	\end{center}
\vspace{-0cm}
\end{table*}
% \section{Full Experiment Results}
% \begin{table*}[th]
    \centering
    \small
    \caption{Classification Results}
    \begin{tabular}{lcccc}
        \toprule
        \textbf{Method} & \textbf{Accuracy} & \textbf{Precision} & \textbf{Recall} & \textbf{F1-score} \\
        \midrule
        \multicolumn{5}{c}{\textbf{Zero Shot}} \\
                Zero-shot E-eyes & 0.26 & 0.26 & 0.27 & 0.26 \\
        Zero-shot CARM & 0.24 & 0.24 & 0.24 & 0.24 \\
                Zero-shot SVM & 0.27 & 0.28 & 0.28 & 0.27 \\
        Zero-shot CNN & 0.23 & 0.24 & 0.23 & 0.23 \\
        Zero-shot RNN & 0.26 & 0.26 & 0.26 & 0.26 \\
DeepSeek-0shot & 0.54 & 0.61 & 0.54 & 0.52 \\
DeepSeek-0shot-COT & 0.33 & 0.24 & 0.33 & 0.23 \\
DeepSeek-0shot-Knowledge & 0.45 & 0.46 & 0.45 & 0.44 \\
Gemma2-0shot & 0.35 & 0.22 & 0.38 & 0.27 \\
Gemma2-0shot-COT & 0.36 & 0.22 & 0.36 & 0.27 \\
Gemma2-0shot-Knowledge & 0.32 & 0.18 & 0.34 & 0.20 \\
GPT-4o-mini-0shot & 0.48 & 0.53 & 0.48 & 0.41 \\
GPT-4o-mini-0shot-COT & 0.33 & 0.50 & 0.33 & 0.38 \\
GPT-4o-mini-0shot-Knowledge & 0.49 & 0.31 & 0.49 & 0.36 \\
GPT-4o-0shot & 0.62 & 0.62 & 0.47 & 0.42 \\
GPT-4o-0shot-COT & 0.29 & 0.45 & 0.29 & 0.21 \\
GPT-4o-0shot-Knowledge & 0.44 & 0.52 & 0.44 & 0.39 \\
LLaMA-0shot & 0.32 & 0.25 & 0.32 & 0.24 \\
LLaMA-0shot-COT & 0.12 & 0.25 & 0.12 & 0.09 \\
LLaMA-0shot-Knowledge & 0.32 & 0.25 & 0.32 & 0.28 \\
Mistral-0shot & 0.19 & 0.23 & 0.19 & 0.10 \\
Mistral-0shot-Knowledge & 0.21 & 0.40 & 0.21 & 0.11 \\
        \midrule
        \multicolumn{5}{c}{\textbf{4 Shot}} \\
GPT-4o-mini-4shot & 0.58 & 0.59 & 0.58 & 0.53 \\
GPT-4o-mini-4shot-COT & 0.57 & 0.53 & 0.57 & 0.50 \\
GPT-4o-mini-4shot-Knowledge & 0.56 & 0.51 & 0.56 & 0.47 \\
GPT-4o-4shot & 0.77 & 0.84 & 0.77 & 0.73 \\
GPT-4o-4shot-COT & 0.63 & 0.76 & 0.63 & 0.53 \\
GPT-4o-4shot-Knowledge & 0.72 & 0.82 & 0.71 & 0.66 \\
LLaMA-4shot & 0.29 & 0.24 & 0.29 & 0.21 \\
LLaMA-4shot-COT & 0.20 & 0.30 & 0.20 & 0.13 \\
LLaMA-4shot-Knowledge & 0.15 & 0.23 & 0.13 & 0.13 \\
Mistral-4shot & 0.02 & 0.02 & 0.02 & 0.02 \\
Mistral-4shot-Knowledge & 0.21 & 0.27 & 0.21 & 0.20 \\
        \midrule
        
        \multicolumn{5}{c}{\textbf{Suprevised}} \\
        SVM & 0.94 & 0.92 & 0.91 & 0.91 \\
        CNN & 0.98 & 0.98 & 0.97 & 0.97 \\
        RNN & 0.99 & 0.99 & 0.99 & 0.99 \\
        % \midrule
        % \multicolumn{5}{c}{\textbf{Conventional Wi-Fi-based Human Activity Recognition Systems}} \\
        E-eyes & 1.00 & 1.00 & 1.00 & 1.00 \\
        CARM & 0.98 & 0.98 & 0.98 & 0.98 \\
\midrule
 \multicolumn{5}{c}{\textbf{Vision Models}} \\
           Zero-shot SVM & 0.26 & 0.25 & 0.25 & 0.25 \\
        Zero-shot CNN & 0.26 & 0.25 & 0.26 & 0.26 \\
        Zero-shot RNN & 0.28 & 0.28 & 0.29 & 0.28 \\
        SVM & 0.99 & 0.99 & 0.99 & 0.99 \\
        CNN & 0.98 & 0.99 & 0.98 & 0.98 \\
        RNN & 0.98 & 0.99 & 0.98 & 0.98 \\
GPT-4o-mini-Vision & 0.84 & 0.85 & 0.84 & 0.84 \\
GPT-4o-mini-Vision-COT & 0.90 & 0.91 & 0.90 & 0.90 \\
GPT-4o-Vision & 0.74 & 0.82 & 0.74 & 0.73 \\
GPT-4o-Vision-COT & 0.70 & 0.83 & 0.70 & 0.68 \\
LLaMA-Vision & 0.20 & 0.23 & 0.20 & 0.09 \\
LLaMA-Vision-Knowledge & 0.22 & 0.05 & 0.22 & 0.08 \\

        \bottomrule
    \end{tabular}
    \label{full}
\end{table*}




\end{document}

\bibliography{main}


\appendix
\renewcommand\thesection{A\arabic{section}}
\renewcommand{\thetable}{A\arabic{table}}
\renewcommand{\thefigure}{A\arabic{figure}}
\renewcommand{\thealgorithm}{A\arabic{algorithm}}










%%%%%%%%%%%%%%%%%%%%%%%%%%%%%%%%%%%%%%%%%%%%%%%%%%%%%%%%%%%%
\section*{Checklist}


% %%% BEGIN INSTRUCTIONS %%%
% The checklist follows the references. For each question, choose your answer from the three possible options: Yes, No, Not Applicable.  You are encouraged to include a justification to your answer, either by referencing the appropriate section of your paper or providing a brief inline description (1-2 sentences). 
% Please do not modify the questions.  Note that the Checklist section does not count towards the page limit. Not including the checklist in the first submission won't result in desk rejection, although in such case we will ask you to upload it during the author response period and include it in camera ready (if accepted).

% \textbf{In your paper, please delete this instructions block and only keep the Checklist section heading above along with the questions/answers below.}
% %%% END INSTRUCTIONS %%%


 \begin{enumerate}


 \item For all models and algorithms presented, check if you include:
 \begin{enumerate}
   \item A clear description of the mathematical setting, assumptions, algorithm, and/or model. [Yes] %[Yes/No/Not Applicable]
   \item An analysis of the properties and complexity (time, space, sample size) of any algorithm. [Yes] %[Yes/No/Not Applicable]
   \item (Optional) Anonymized source code, with specification of all dependencies, including external libraries. [Yes] %[Yes/No/Not Applicable]
 \end{enumerate}


 \item For any theoretical claim, check if you include:
 \begin{enumerate}
   \item Statements of the full set of assumptions of all theoretical results. [Yes] %[Yes/No/Not Applicable]
   \item Complete proofs of all theoretical results. [Yes] %[Yes/No/Not Applicable]
   \item Clear explanations of any assumptions. [Yes] %[Yes/No/Not Applicable]
 \end{enumerate}


 \item For all figures and tables that present empirical results, check if you include:
 \begin{enumerate}
   \item The code, data, and instructions needed to reproduce the main experimental results (either in the supplemental material or as a URL). [Yes] %[Yes/No/Not Applicable]
   \item All the training details (e.g., data splits, hyperparameters, how they were chosen). [Yes] %[Yes/No/Not Applicable]
         \item A clear definition of the specific measure or statistics and error bars (e.g., with respect to the random seed after running experiments multiple times). [Yes] %[Yes/No/Not Applicable]
         \item A description of the computing infrastructure used. (e.g., type of GPUs, internal cluster, or cloud provider). [Yes] %[Yes/No/Not Applicable]
 \end{enumerate}

 \item If you are using existing assets (e.g., code, data, models) or curating/releasing new assets, check if you include:
 \begin{enumerate}
   \item Citations of the creator If your work uses existing assets. [Yes] %[Yes/No/Not Applicable]
   \item The license information of the assets, if applicable. [Not Applicable]
   \item New assets either in the supplemental material or as a URL, if applicable. [Not Applicable]
   \item Information about consent from data providers/curators. [Yes]
   \item Discussion of sensible content if applicable, e.g., personally identifiable information or offensive content. [Not Applicable]
 \end{enumerate}

 \item If you used crowdsourcing or conducted research with human subjects, check if you include:
 \begin{enumerate}
   \item The full text of instructions given to participants and screenshots. [Not Applicable]
   \item Descriptions of potential participant risks, with links to Institutional Review Board (IRB) approvals if applicable. [Not Applicable]
   \item The estimated hourly wage paid to participants and the total amount spent on participant compensation. [Not Applicable]
 \end{enumerate}

 \end{enumerate}


\newpage 
\clearpage
\onecolumn

\begin{algorithm}[!tb]
\caption{SiCL Workflow for Predicting Causal Structures}
\label{alg:workflow}
\begin{algorithmic}
% \STATE {\bfseries Input:} test\_data
% \STATE {\bfseries Output:} predicted\_cpdag
% \STATE
\STATE \textbf{Procedure} INFERENCE(data, target)
\STATE Calculate node features with node encoder
\STATE Calculate pairwise features with pairwise encoder following Sec. \ref{sec:fem}
\IF {target is skeleton }
\STATE Calculate skeleton with Sec. \ref{sec:met:lic}
\ELSE
\STATE Calculate v-structures with Sec. \ref{sec:met:lic}
\ENDIF
\STATE \textbf{End Procedure}
\STATE
\STATE \textbf{Procedure} TRAINING\_PHASE()
\STATE  skeleton\_predictor $\leftarrow$ init\_skeleton\_predictor()

\STATE  Sample graphs and corresponding data
\STATE Training the skeleton predictor with INFERENCE(data, skeleton)
\STATE Training the v-structure predictor with INFERENCE(data, v-structure), with feature encoders fine-tuned from skeleton predictor
\STATE \textbf{End Procedure}
\STATE
% \STATE graph\_distribution $\leftarrow$ DEFINE\_GRAPH\_DISTRIBUTION()
\STATE \textbf{Procedure} TESTING\_PHASE(test\_data)
\STATE  Calculate predicted skeleton with the trained skeleton predictor
\STATE  Calculate predicted v-structures with the trained v-structure predictor
\STATE Combine predicted skeleton and v-structures to obtain predicted CPDAG
\STATE \textbf{End Procedure}
\end{algorithmic}
\end{algorithm}

\begin{figure*}[t]
\centering
    \includegraphics[width=\linewidth]{figures/pairwiseencoder.pdf}
    \caption{Illustration of the pairwise encoder module. \textcolor{black}{In Part \ding{172}, it initializes raw pairwise features. In Part \ding{173}, a unidirectional attention is applied to utilized information from node features and pairwise features. In Part \ding{174}, an MLP and residual connection is used to yield final pairwise features.}}
          % \vspace{-0.15in}
    \label{fig:pem}
    
\end{figure*}

\section{Theoretical Guarantee} \label{sec:app:tg}
% In this section, we present the theoretical analysis on the asymptotically correctness of our model. In Sec. \ref{sec:da}, we provide the necessary definitions and our assumptions of the problem. 
% In Sec. \ref{sec:sl} - \ref{sec:ol}, we prove the asymptotically correctness of the neural network model.
% In Sec. \ref{sec:d}, we discuss the practical superiority of the neural network models.
In this section, we delve into the theoretical analysis concerning the asymptotic correctness of our proposed model with respect to the sample size. Sec. \ref{sec:da} lays out the essential definitions and assumptions pertinent to the problem under study. Following this, from Sec. \ref{sec:sl} to \ref{sec:ol}, we rigorously demonstrate the asymptotic correctness of the neural network model. Finally, in Sec. \ref{sec:d}, we engage in a detailed discussion about the practical advantages and superiority of neural network models.

\subsection{Definitions and Assumptions} \label{sec:da}
As outlined in Sec. \ref{sec:bg}, a Causal Graphical Model is defined by a joint probability distribution $P$ over $d$ random variables $X_1, X_2, \cdots, X_{d}$, and a DAG $G$ with $d$ vertices representing the $d$ variables.
An observational dataset $D$ consists of $n$ records and $d$ columns, which represents $n$ instances drawn i.i.d. from $P$. 
In this work, we assume causal sufficiency:
\begin{Assumption}[Causal Sufficiency] \label{ass:cs}
    There are no latent common causes of any of the variables in the graph. 
\end{Assumption}
Moreover, we assume the data distribution $P$ is Markovian to the DAG $G$:
\begin{Assumption}[Markov Factorization Property]\label{ass:mk}
      Given a joint probability distribution $P$ and a DAG $G, P$ is said to satisfy Markov factorization property w.r.t. $G$ if $P:=$ $P\left(X_1, X_2, \cdots, X_d\right)=\prod_{i=1}^d P\left(X_i \mid \mathrm{pa}_i^G\right)$, where $\mathrm{pa}_i^G$ is the parent set of $X_i$ in $G$.
\end{Assumption}
It is noteworthy that the Markov factorization property is equivalent to the Global Markov Property (GMP) \citep{lauritzen1996graphical}, which is
\begin{Definition}[Global Markov Property (GMP)]
    $P$ is said to satisfy GMP (or Markovian) w.r.t. a DAG $G$ if $X \perp_G Y|Z \Rightarrow X \perp Y| Z$. Here $\perp_G$ denotes d-separation, and $\perp$ denotes statistical independence. 
\end{Definition}
GMP indicates that any d-separation in graph $G$ implies conditional independence in distribution $P$. We further assume that $P$ is faithful to $G$ by:
\begin{Assumption}[Faithfulness]\label{ass:f}
Distribution $P$ is faithful w.r.t. a DAG $G$ if $X \perp Y\left|Z \Rightarrow X \perp_G Y\right| Z$.
\end{Assumption}

\begin{Definition}[Canonical Assumption] \label{ass:ca}
    We say our settings satisfy the canonical assumption if the Assumptions \ref{ass:cs} - \ref{ass:f} are all satisfied.
\end{Definition}
We restate the definitions of skeletons, Unshielded Triples (UTs) and v-strucutres as follows.
\begin{Definition}[Skeleton]
    A skeleton $E$ defined over the data distribution $P$ is an undirected graph where an edge exists between $X_i$ and $X_j$ if and only if $X_i$ and $X_j$ are always dependent in $P$, i.e., $\forall Z \subseteq\left\{X_1, X_2, \cdots, X_d\right\} \backslash \left\{X_i, X_j \right\}$, we have $X_i \nperp X_j | Z$.
\end{Definition}
Under our assumptions, the skeleton is the same as the corresponding undirected graph of $G$ \citep{spirtes2000causation}. 
\begin{Definition}[Unshielded Triples (UTs) and V-structures]
A triple of variables $X, T, Y$ is an Unshielded Triple (UT) denoted as $\langle X, T, Y \rangle$, if $X$ and $Y$ are both adjacent to $T$ but not adjacent to each other in the DAG $G$ or the corresponding skeleton.
It becomes a v-structure denoted as $X \rightarrow T \leftarrow Y$, if the directions of the edges are from $X$ and $Y$ to $T$ in $G$.
\end{Definition}
We introduce the definition of separation set as:
\begin{Definition}[Separation Set]
    For a node pair $X_i$ and $X_j$, a node set $Z$ is a separation set if $X_i \perp X_j | Z $. Under faithfulness assumption, a separation set $Z$ is a subset of variables within the vicinity that d-separates $X_i$ and $X_j$.
\end{Definition}

Finally, we assume a neural network can be used as a universal approximator in our settings.
\begin{Assumption}[Universal Approximation Capability]
    A neural network model can be trained to approximate a function under our settings with arbitary accuracy. \label{ass:uac}
\end{Assumption}

\subsection{Skeleton Learning} \label{sec:sl}
In this section, we prove the asymptotic correctness of neural networks on the skeleton prediction task by constructing a perfect model and then approximating it with neural networks. 
For the sake of convenience and brevity in description, we define the skeleton predictor as follows. 
\begin{Definition}[Skeleton Predictor]
    Given observational data $D$, a skeleton predictor is a predicate function with domain as observational data $D$ and predicts the adjacency between each pair of the vertices.
\end{Definition}
Now we restate the Remark from \cite{ma2022ml4s} as the following proposition. 
It proves the existence of a perfect skeleton predictor by viewing the skeleton prediction step of PC \citep{spirtes2000causation} as a skeleton predictor, which is proved to be sound and complete.
\begin{Proposition}[Existence of a Perfect Skeleton Predictor]
There exists a skeleton predictor that always yields the correct skeleton with sufficient samples in $D$. \label{prop:epsp}
\end{Proposition}
\begin{proof}
    We construct a skeleton predictor $SP$ consisting of two parts by viewing PC \citep{spirtes2000causation} as a skeleton predictor. 
    In the first part, it extracts a pairwise feature $\boldsymbol{x}_{i j}$ for each pair of nodes $X_i$ and $X_j$:
    \begin{align}
            % \boldsymbol{x}_{i j}= \left|\left\{Z | Z \subseteq V \backslash\{v_i, v_j\} \wedge  (v_i \perp v_j \mid Z)\right\}\right|, \label{equ:sp1}
        \boldsymbol{x}_{i j}=\min _{Z \subseteq V \backslash\left\{X_i, X_j\right\}}\left\{X_i \sim X_j \mid Z\right\}, \label{equ:sp1}
    \end{align}
    where $\left\{X_i \sim X_j \mid Z\right\} \in [0, 1] $ is a scalar value that measures the conditional dependency between $X_i$ and $X_j$ given a node subset $Z$. 
    % where $v_i \perp v_j \mid Z$ indicates whether the conditional dependency exists between $v_i$ and $v_j$ given a node subset $Z$. 
    % Intuitively, $\boldsymbol{x}_{i j}$ represents the counts of subsets to make $v_i$ and $v_j$ conditional dependency.
    Consequently, $\boldsymbol{x}_{i j} > 0$ indicates the persistent dependency between the two nodes.
    
    In the second part, it predicts the adjacency based on $\boldsymbol{x}_{i j}$:
    \begin{align}
        \left(X_i,  X_j\right)= \begin{cases} 1 \text { (adjacent) } & \boldsymbol{x}_{i j} \neq 0 \\ 0 \text { (non-adjacent) } & \boldsymbol{x}_{i j} = 0\end{cases} \label{equ:sp2}
    \end{align}

    Now we prove that $SP$ always yields the correct skeleton by proving the absence of false positive predictions and false negative predictions. Here, false positive prediction denotes $SP$ predicts a non-adjacent node pair as adjacent and false negative predictions denote $SP$ predicts an adjacent node pair as non-adjacent.

    \begin{itemize}[leftmargin=*]
        \item \textbf{False Positive.} Suppose $X_i, X_j$ are non-adjacent. Under the Markovian assumption, there exists a set of nodes $Z$ such that $\left\{X_i \sim X_j \mid Z\right\} = 0$ and hence $\boldsymbol{x}_{ij} = 0$. According to Eq. (\ref{equ:sp2}), $SP$ will always predicts them as non-adjacent.
        \item \textbf{False Negative}. Suppose $X_i, X_j$ are adjacent. Under the faithfulness assumption, for any $Z \in V \backslash \left\{X_i, X_j\right\}, \left\{X_i \sim X_j \mid Z\right\} > 0$, which implies $\boldsymbol{x}_{ij} > 0$. Therefore, $SP$ always predicts them as adjacent.
    \end{itemize}

    Therefore, $SP$ never yields any false positive predictions or false negative predictions under the Markovian assumption and faithfulness assumption, i.e., it always yields the correct skeleton.
\end{proof}

% \textbf{Remark:} % The constructed perfect skeleton predictor is not a continuous function. 
% The mapping from $\boldsymbol{x}_{ij}$ to adjacency shown in Eq. \ref{equ:sp2} can be continuously approximated. 
% Therefore, there exists arbitrarily approximate continuous function for the perfect skeleton predictor. 
%The intuition behind is that the conditional dependency test in Eq. (\ref{equ:sp1}) can be estimated by some continuous functions like $G^2$ test \citep{agresti2012categorical} and mutual information . 

With the existence of a perfect skeleton predictor, we prove the correctness of neural network models with sufficient samples under our assumptions.
\begin{Theorem}
Under the canonical assumption and the assumption that neural network can be used as a universal approximator (Assumption \ref{ass:uac}),
there exists a neural network model that always predicts the correct skeleton with sufficient samples in $D$.
\end{Theorem}
\begin{proof}
    From Proposition \ref{prop:epsp}, there exists a perfect skeleton predictor that predicts the correct skeleton. 
    Thus, according to the Assumption \ref{ass:uac}, a neural network model can be trained to approximate the perfect skeleton prediction hence predicts the correct skeleton. 
\end{proof}

\subsection{Orientation Learning} \label{sec:ol}
Similarly to the overall thought process in Sec. \ref{sec:sl}, in this section we prove the asymptotic correctness of neural networks on the v-structure prediction task by constructing a perfect model and then approximating it with neural networks. 
\begin{Definition}[V-structure Predictor]
    Given observational data $D$ with sufficient samples from a $BN$ with vertices $V = \{X_1, \dots, X_p\}$, a v-structure predictor is a predicate function with domain as observational data $D$ and predicts existence of the v-structure for each unshielded triple.
\end{Definition}
The following proposition proves the existence of a perfect v-structure predictor by viewing the orientation step of PC \citep{spirtes2000causation} as a v-structure predictor.
\begin{Proposition}[Existence of a Perfect V-structure Predictor]
Under the Markov assumption and faithfulness assumption, there exists skeleton predictor that always yields the correct skeleton. \label{prop:epvp}
\end{Proposition}
\begin{proof}
    We construct a v-structure predictor $VP$ consisting of two parts by viewing PC \citep{spirtes2000causation} as a v-structure predictor. 
    % In the first part, it extracts a feature $\boldsymbol{z}_{ijk}$ for each UT $\langle X_i, X_k, X_j \rangle$:
    % \begin{align}
    %     \boldsymbol{z}_{ijk} = \frac{\left|\left\{ (X_k, Z) | \{ X_i \sim X_j | Z\} = 0 \wedge X_k \in Z \right\}\right|}{\left|\left\{ Z | \{ X_i \sim X_j | Z\} = 0 \right\}\right|},
    % \end{align}
        In the first part, it extracts a boolean feature $\boldsymbol{z}_{kij}$ for each UT $\langle X_i, X_k, X_j \rangle$:
    \begin{align}
        \boldsymbol{z}_{kij} = (X_k \in Z), \text{ where } Z \text{ is called as a sepset, i.e. } X_i\perp Y_j | Z.  \label{equ:vp}
        % \frac{\left|\left\{ (X_k, Z) | \{ X_i \sim X_j | Z\} = 0 \wedge X_k \in Z \right\}\right|}{\left|\left\{ Z | \{ X_i \sim X_j | Z\} = 0 \right\}\right|},
    \end{align}
    \color{black}

% where $\left\{X_i \sim X_j \mid Z\right\} \in [0, 1] $ is a scalar value that measures the conditional dependency between $X_i$ and $X_j$ given a node subset $Z$, and $|\cdot|$ represents the cardinality of a set. 
% Note that the denominator is always positive because the separation set of a UT always exists (See Lemma 4.1 in \cite{dai2023ml4c}).
Note that the sepset $Z$ always exists because the separation set of a UT always exists (See Lemma 4.1 in \cite{dai2023ml4c}).
% % Moreover, $\boldsymbol{z}_{ijk}$ is either $0$ or $1$ under our assumptions and sufficient samples.
% Intuitively, $\boldsymbol{z}_{ijk}$ represents the proportion of supsets of $X_i$ and $X_j$ that include $X_k$.

In the second part, it predicts the v-structures based on $\boldsymbol{z}_{ijk}$:
% \begin{align}
% \langle X_i, X_k, X_j\rangle = \begin{cases} 0 \text { (not v-structure) } & \boldsymbol{z}_{i j k} \neq 0 \\ 1 \text { (v-structure) } & \boldsymbol{z}_{i j k} = 0\end{cases} \label{equ:vp2}
% \end{align}
\begin{align}
\langle X_i, X_k, X_j\rangle = \begin{cases} 0 \text { (not v-structure) } & \boldsymbol{z}_{k i j } = True \\ 1 \text { (v-structure) } & \boldsymbol{z}_{k i j } = False\end{cases} \label{equ:vp2}
\end{align}
Now we prove that $VP$ always yields the correct predictions of v-structures.
According to Theorem 5.1 on p.410 of \cite{spirtes2000causation}, assuming faithfulness and sufficient samples, if a UT $\langle X_i, X_k, X_j \rangle$ is a v-structure, then $X_k$ does not belong to any separation sets of $(X_i, X_j)$; if a UT $\langle X_i, X_k, X_j \rangle$ is not a v-structure, then $X_k$ belongs to every separation sets of $(X_i, X_j)$. Therefore, we have $\boldsymbol{z}_{kij} = False$ if and only if $X_k$ is not in any separation set of $X_i$ and $X_j$, i.e., $\langle X_i, X_k, X_j \rangle$ is a v-structure.
\color{black}
\end{proof}

With the existence of a perfect v-structure predictor, we prove the correctness of neural network models with sufficient samples under our assumptions.
\begin{Theorem}
Under the canonical assumption and the assumption that neural network can be used as a universal approximator (Assumption \ref{ass:uac}), there exists a neural network model that always predicts the correct v-structures with sufficient samples in $D$.
\end{Theorem}
\begin{proof}
    From Proposition \ref{prop:epsp}, there exists a perfect skeleton predictor that predicts the correct v-structures. 
    Thus, according to the Assumption \ref{ass:uac}, a neural network model can be trained to approximate the perfect v-structure predictions hence predicts the correct v-structures. 
\end{proof}
% \begin{Theorem}
% Under our assumptions, neural network models can predict the correct v-structures.
% \end{Theorem}
% \begin{proof}
%     From Proposition \ref{prop:epvp}, there exists a perfect v-structure predictor that predicts the correct v-structure. 
%     Thus, according to the Assumption \ref{ass:uac}, a neural network model can be trained to predict the correct skeleton. 
% \end{proof}

\subsection{Discussion} \label{sec:d}
In the sections above, we prove the asymptotic correctness of neural network models by constructing theoretically perfect predictors.
These predictors both consist of two parts: feature extractors providing features $\boldsymbol{x}_{ij}$ and $\boldsymbol{z}_{ijk}$, and final predictors of adjacency and v-structures.
Even though they have a theoretical guarantee of the correctness with sufficient samples, it is noteworthy that they are hard to be applied practically.
For example, to obtain $\boldsymbol{x}_{ij}$ in Eq. (\ref{equ:sp1}), we need to calculate the conditional dependency between $X_i$ and $X_j$ given every node subset $Z \subseteq V \backslash\left\{X_i, X_j\right\}$.
Leaving aside the fact that the number of $Z$s itself presents factorial complexity, the main issue is that when $Z$ is relatively large, due to the curse of dimensionality, it becomes challenging to find sufficient samples to calculate the conditional dependency. 
This difficulty significantly hampers the ability to apply the constructed prefect predictors in practical scenarios.

Some existing methods can be interpreted as constructing more practical predictors.
Majority-PC (MPC) \citep{colombo2014order} achieves better performance on finite samples by modifying Eq. (\ref{equ:vp}) - (\ref{equ:vp2}) as:
    \begin{align}
        \boldsymbol{z}_{kij} = \frac{\left|\left\{ (X_k, Z) | \{ X_i \sim X_j | Z\} = 0 \wedge X_k \in Z \right\}\right|}{\left|\left\{ Z | \{ X_i \sim X_j | Z\} = 0 \right\}\right|},
    \end{align}
% Note that the denominator is always positive because the separation set of a UT always exists (See Lemma 4.1 in \cite{dai2023ml4c}).
    and
\begin{align}
    \left\langle X_i, X_k, X_j\right\rangle= \begin{cases}0 \text { (not v-structure) } & \boldsymbol{z}_{i j k} > 0.5 \\ 1(\mathrm{v} \text {-structure) } & \boldsymbol{z}_{i j k} \leq 0.5,\end{cases}
\end{align}
    where $\left\{X_i \sim X_j \mid Z\right\} \in [0, 1] $ is a scalar value that measures the conditional dependency between $X_i$ and $X_j$ given a node subset $Z$, and $|\cdot|$ represents the cardinality of a set. 
\color{black}
Due to its more complex classification mechanism, it achieves better performance empirically. 
However, from the machine learning perspective, features from both the PC and MPC predictors are relatively simple.
As supervised causal learning methods, ML4S \citep{ma2022ml4s} and ML4C \citep{dai2023ml4c} provide more systematic featurizations by manual feature engineering and utilization of powerful machine learning models for classification.
% Even though they achieve better performance in practice, the manual feature engineering is complex.
% In out paper, we use neural networks as universal approximator to learn the prediction of identifiable causal structures.
% SCL with NN 也同时在其他地方讨论,如kenan...
While these methods show enhanced practical efficacy, their manual feature engineering processes are complex. 
In our paper, we utilize neural networks as universal approximators for learning the prediction of identifiable causal structures.
It not only simplifies the procedure but also potentially uncovers more nuanced and complex patterns within the data that manual methods might overlook.
It is noteworthy that the benefits of supervised causal learning using neural networks are also discussed elsewhere, as mentioned in SLdisco \citep{petersen2023causal} and CSIvA \citep{ke2023learning}.



\begin{algorithm}[!tb]
\caption{Post-processing}
\label{alg:post}
\begin{algorithmic}
\STATE {\bfseries Input:} weighted skeleton matrix $S$, weighted V-tensor $U$, threshold for skeleton $\tau_s$, threshold for v-structure $\tau_v$
\STATE {\bfseries Output:} predicted oriented edge set $\texttt{oriEdges}$, predicted skeleton $\texttt{skeleton}$
% \STATE
\STATE \textbf{Step 1:}  \\
// Obtaining a predicted skeleton by thresholding. \\
$\texttt{skeleton} = \{(i, j) | max(S_{ij}, S_{ji}) > \tau_s\}$ \\
// Obtaining raw v-structures $\texttt{vstructs}_\texttt{raw}$ by thresholding. \\
$\texttt{vstructs}_\texttt{raw} = \{(i, j, k) | (i, j) \in \texttt{skeleton} \text{ and } (i, k) \in \texttt{skeleton} \text{ and } (j, k) \notin \texttt{skeleton} \text{ and } max(U_{ijk}, U_{ikj}) > \tau_v\}$ \\
\STATE \textbf{Step 2:}  \\
// V-structure conflict resolving: discard any v-structure if there exists another conflicted v-structure with a higher predicted score, following \citep{dai2023ml4c}. \\
$\texttt{vstructs} = \{(i, j, k) \in \texttt{vstructs}_\texttt{raw}|\forall (i', j', k') \in \texttt{vstructs}_\texttt{raw}, (i' \neq k \text{ and }i' \neq j)\text{ or } (k'\neq i \text{ and } j' \neq i) \text{ or } U_{i'j'k'} < U_{ijk}\}$ 
\STATE \textbf{Step 3:}  \\
// Obtaining the predicted directed edge from $\texttt{vstructs}$. \\
$\texttt{oriEdges}_{\texttt{raw}} = \{(j, i)| \exists k, (i, j, k) \in \texttt{vstructs}\}$ \\
//Set a score for each edge with the highest v-structure's score containing this edge. \\
Set $\{p_{ij}\}$ such that $p_{ij} = max_v U_v$ for $v \in \texttt{oriEdges}_{\texttt{raw}}$ and $v \ni (i, j)$. \\
// If there exist any cycles, remove the edge with the smallest score in each cycle. \\
$\texttt{oriEdges} = \{(i, j) \in \texttt{oriEdges}_{\texttt{raw}} |\forall \text{cycle } C, (i, j) \notin C \text{ or } (\exists (i', j') \in C, p_{ij} > p_{i'j'})\}$ \\
\STATE \textbf{Step 4:} \\
// Meek rules: Add edges to $\texttt{oriEdges}$ for directed edges that (1) introducing the edges does not lead to cycles or new v-structures; (2) adding the opposite edges necessarily leads to cycles or new v-structures. \\
$\texttt{oriEdges} = \texttt{oriEdges} \cup \{(i, j) \in \texttt{skeleton}| (i, j) \text{ complies with Meek rules} \}$ \\
\end{algorithmic}
\end{algorithm}

\section{More Discussions on Identifiability and Causal Assumptions} \label{sec:dica}
\textbf{Advocation of Learning Identifiable Structures under All Settings.}
In this paper, we have to work on a concrete setting for demonstration purpose with concrete identifiable causal structures in this paper.
Nonetheless, we want to emphasize that the very concept of identifiability, as well as its ramifications in SCL, is indeed a general issue that is less bound to the issue of ``which causal structures are identifiable under which assumptions". 
The simple fact that in some situations the causal edge cannot be identified -- no matter what feature can be identified in that case -- this identifiability limit has a general effect on SCL. 
Unless the causal graph/edge itself becomes fully invariant/identifiable (a special case that is important but certainly not universally true), the presence of the identifiability limit entails a fundamental bias for a popular SCL model architecture (i.e., Node-Edge) that cannot be mitigated by larger model or bigger data at all. 
This ``identifiability-limit-causes-learning-error" effect is the main thesis of this paper, and we advocate to design neural networks that focus on learning the identifiable features (no matter what those features are). 
In other words, there is nothing stopping one from studying another setting where another feature is identifiable though, and in that case we would also advocate to learn that feature instead of v-structures. 
For example, if we assume canonical MEC assumptions and non-existence of causal-fork and v-structures, the identifiable causal structure becomes a kind of chains.
In that case, one may want to design neural networks that predict about causal chains. 

\textbf{Rationality of Canonical Assumptions.}
In this paper, we choose the canonical setting under the classic MEC theory, in which the skeleton and v-structures are the identifiable structure.
This setting includes the assumptions of the Markov and faithfulness conditions.
Unlike scenario-specific assumptions, such as those tied to a particular data-generating process, these assumptions are classic assumptions about causality that are often adopted as ``postulates" about some general aspects of the world. For example,
\begin{itemize}
    \item \cite{pearl2009causality} argues that stability (faithfulness) stems from the natural improbability of strict equality constraints among parameters, which aligns with the autonomy of causal mechanisms.
\item \cite{spirtes2001causation} support the Causal Faithfulness Condition (CFC) by noting that the exact cancellation of causal paths is highly improbable under natural conditions.
\item \cite{weinberger2018faithfulness} reinforces this argument, proposing that coincidences leading to CFC violations are rare and lack explanatory power, further justifying its adoption within a general modeling framework.
\end{itemize}
These considerations underscore the rationality and generality of the assumptions, making them a natural choice for our analysis.

\section{Details and Discussion about Post-processing} \label{app:post}
For comprehensive clarity, we provide a clear process about the post-processing algorithm in Alg. \ref{alg:post}.

\textbf{Discussion. }
It is worth noting that our design in post-processing is as conservative as possible. In fact, we simply adhere to the conventions in deep learning (i.e., thresholding) to obtain the skeleton and the initial v-structure set. Subsequently, we follow the conventions in constraint-based causal discovery methods to derive the final oriented edges. Therefore, we have not dedicated extensive efforts towards the meticulous design, nor do we intend to emphasize this aspect of our workflow.

The conflicts and cycles are not unique to SiCL; they are, in fact, common issues encountered by all constraint-based algorithms like PC. Moreover, it's worth noting that they never appear if the networks perform perfect. Therefore, the conflict resolving of v-structures and the removal of cycles are designed as fallback mechanisms to ensure the soundness of our workflow, rather than being central elements of our approach. To illustrate it, we experimented with an opposite variant (Intuitively, this is a bad choice) that prioritizes discarding the v-structure with the higher predicted probability. The minimal differences in outcomes between this variant and its counterpart, as detailed in Tab. \ref{tab:crm}, support our viewpoint that the conflict resolution process is of limited significance within our workflow. On the other hand, experimental results presented in Tab. \ref{tab:ccfc} underscore the infrequency of cycles in the predictions, reinforcing the non-essential nature of the cycle removal component.


\begin{table}[tb]
\centering
\begin{threeparttable}
\caption{The o-F1 comparison between the used conflict resolving method with an opposite variant.}
\label{tab:crm}
\begin{tabular}{@{}ccc@{}}
\toprule
Conflict Resolving Method & WS-L-G & SBM-L-G  \\
\midrule
Original Conflict Resolving & $\mathbf{41.1}$&$\mathbf{83.3}$ \\
Opposite Conflict Resolving & $40.7$ & $83.2$ \\
\bottomrule
\end{tabular}
\end{threeparttable}
\end{table}

\color{black}
\section{Details about Node Feature Encoder} \label{sec:dnf}
Motivated by previous approaches \citep{lorchamortized,ke2023learning}, we employ a transformer-like architecture comprising attention layers over either the observation dimension or the node dimension alternately as the node feature encoder.
Concretely, for the raw node features $\mathcal{F} \in \mathbb{R}^{d \times n \times h}$ corresponding to $d$ nodes and $n$ observations, our goal is to capture the correlations between both different nodes and different observations.
Therefore, we utilize two transformer encoder layers over the observation dimension and the node dimension alternatively:
\begin{align}
\begin{aligned}
    \mathcal{F} & \leftarrow TransformerEncoderLayer(\mathcal{F}, \mathcal{F}, \mathcal{F}) \\
    \mathcal{F} & \leftarrow \mathcal{F}.transpose(0, 1) \\
        \mathcal{F} & \leftarrow TransformerEncoderLayer(\mathcal{F}, \mathcal{F}, \mathcal{F})\\
    \mathcal{F} & \leftarrow \mathcal{F}.transpose(0, 1). \\
\end{aligned}
\end{align}
The above operation is repeated multiple times for sufficiently feature encoding.
It yields the final node feature tensor $\mathcal{F} \in \mathcal{R}^{d \times \times h}$.
\color{black}
\section{Illustration of the Case Study in Sec. \ref{sec:met:lim}}
Fig. \ref{fig:ps} presents an illustration for the case study of the Node-Edge approach in Sec. \ref{sec:met:lim}. 
It clearly shows that observational data with the two different parametrized forms follow the same joint distribution:
\begin{align}
    P(\left[X, Y, T\right]) =\mathcal{N}\left([0,0,0],\left[\begin{array}{lll}1 & 1 & 1 \\ 1 & 3 & 2 \\ 1 & 2 & 2\end{array}\right]\right).
\end{align}
Therefore, the observational datasets coming from the two DAGs are inherently indistinguishable.


    
\begin{figure*}[ht]
    \centering
    \includegraphics[width=0.9\linewidth]{figures/ps.pdf}
    \caption{The problem setting to emphasize the limitations of the Node-Edge approach. \textit{Best viewed in color.}}
    \label{fig:ps}
\end{figure*}


\section{Proof and Discussion for Proposition \ref{prop:star}} \label{sec:mgcs}
\color{black}
We first restate the Proposition \ref{prop:star} with more details and provide the proof.
\begin{Proposition}
Let $\mathcal{G}_n$ be the set of graphs with $n+1$ nodes where there is a central node $y$ such that (1) every other node is connected to $y$, (2) there is no edge between the other nodes, (3) there is at most one edge pointing to $y$. 
For any distribution $Q$ over $\mathcal{G}_n$, let $M(Q)$ be another distribution over $\mathcal{G}_n$ such that for any causal edges $e, e'$, $P_{G\sim Q}(e \in G) = P_{G\sim M(Q)}(e \in G) = P_{G\sim M(Q)}(e \in G | e' \in G)$. We have 
\begin{align}\max_{Q} P_{G \sim M(Q)}(G \nin \mathcal{G}_n) = 
1 - \frac{2n-1}{n-1}(1 - \frac{1}{n})^n.
\end{align}
As a corollary, we have 
\begin{align}
\sup_n \max_{Q} P_{G \sim M(Q)}(G \nin \mathcal{G}_n) = 
1 - \frac{2}{e} \approx 0.2642,
\end{align}
\end{Proposition}
\begin{proof}
Denote other nodes except for the central node as $x_i$ where $i \in \left\{1, 2, \dots, n\right\}$. 
In our setting, the set $\mathcal{G}_n$ contains $n + 1$ DAGs
with the same skeleton and no v-structure
: $G_0: y \rightarrow x_i$ for all $x_i$, and $G_i: y \rightarrow x_j$ for all $x_j \neq x_i$ together with $x_i \rightarrow y$.
Denote the sampling probability of DAG $G_i$ from $\mathcal{G}_n$ as $P_i$.
Therefore, the marginal probability of the edge $y \rightarrow x_i$ is $1 - P_i$.

    % As the Node-Edge model $M$ is trained optimally, the prediction of the existence probability of the edge $y \rightarrow x_i$ is $1 - P_i$.
    
    
    If $\exists i$, $P_i = 1$, it means that $\mathcal{G}_n$ only contains the DAG $G_i$. Therefore, $M(Q)$ is equivalent to $Q$ and we have $P_{G \sim M(Q)}(G \nin \mathcal{G}_n) = 0$.

    If $\forall i$, $P_i < 1$, denoting $Q_i = 1 - P_i$ and $P(v)$ as the probability of $G$ containing no v-structures. In other words, $P(v) = P_{G \sim M(Q)}(G \in \mathcal{G}_n)$. We have 
    \begin{align}
        P(v) = \prod_{i=1}^n Q_i + \sum_{j=1}^n \frac{\prod_i^n Q_i}{Q_j} (1 - Q_j)
        &= ( \prod_{i=1}^n Q_i) \cdot (1 + \sum_{j=1}^{n} \frac{1-Q_j}{Q_j}).
    \end{align}
    As $P(v)$ is a probability, we have $P(v) > 0$. Denoting function 
    \begin{align}
    f(Q_1, Q_2, \dots, Q_n) = \log P(v) = \sum_{i=1}^n \log Q_i + \log (1 + \sum_{j=1}^n \frac{1-Q_j}{Q_j}),
    \end{align} we would like to find its minimum s.t. $\sum_i Q_i \geq n - 1$ and $Q_i \in (0, 1]$.

    Define its Lagrange function \begin{align}
        L(Q_1, Q_2, \dots, Q_n, \lambda) = f + \lambda (n-1-\sum_i Q_i).
    \end{align}

    We have 
    \begin{align}
        \frac{\partial L}{\partial \lambda} = n - 1 - \sum_i Q_i,
    \end{align}
    and 
    \begin{align}
        \frac{\partial L}{\partial Q_i} = \frac{1}{Q_i}(1 - \frac{1}{Q_i(1-n + \sum_{k=1}^{n}\frac{1}{Q_k})}) - \lambda.
    \end{align}

    Now we are going to find the extremums for $L(Q_1, Q_2, \dots, Q_n, \lambda)$.
    \begin{enumerate}
        \item[(1)] If $\lambda = 0$, we have $\forall i$, $\frac{\partial f}{\partial Q_i} = 0$, then
        \begin{align}
            \forall i, Q_i = \frac{1}{(1 - n + \sum_{k=1}^{n} \frac{1}{Q_k})}.
        \end{align}
        It indicates that $\forall i, Q_i = 1$, hence $f = 0$ and $P(v) = 1$.
        \item[(2)] If $\lambda \neq 0$, $\exists i$, we have $\forall i$, $\frac{\partial f}{\partial Q_i} = \lambda$ and $\sum_{i=1}^n = n - 1$. In other words, we have 
        \begin{align}
            \forall i, j, \frac{\partial f}{\partial Q_i} = \frac{\partial f}{\partial Q_j} = \lambda.
        \end{align}
        Define function 
        \begin{align}
        h(Q_i) = \frac{\partial f}{\partial Q_i} = \frac{1}{Q_i}(1 - \frac{1}{Q_i(1-n + \sum_{k=1}^{n}\frac{1}{Q_k})}).
        \end{align}
        we can rewrite the function as 
        \begin{align}
            h(Q_i) = \frac{1}{Q_i}(1 - \frac{1}{1 + AQ_i}),
        \end{align}
        where $A = 1 - n + \sum_{k\neq i} \frac{1}{Q_k} \geq 1 - n + \frac{(n-1)^2}{n-1-Q_i} > 0$.
        Therefore, $h(x)$ is a monotonic function in its domain. 

        It indicates that $\forall i, j$, $Q_i = Q_j = \frac{n-1}{n}$, where $P(v) = \frac{2n-1}{n-1}(1 - \frac{1}{n})^n$.
    \end{enumerate}
    Now we are going to list the boundary points for $f$.
    \begin{enumerate}
        \item[(1)] $\forall i$, $Q_i = 1$, it becomes the first extremum point.
        \item[(2)] $\exists i$, $Q_i$ is approaching to $0$. Due to the constraint of $\sum Q_i \geq n - 1$, other $Q$s are approaching to $1$. We have $\lim_{Q_i \rightarrow 0} f = 0$ and $P(v) = 1$.
    \end{enumerate}
    In conclusion, the maximum point of function $f$ is $\forall i$, $Q_i = \frac{n - 1}{n}$, where 
    \begin{align}
        P(v) = \frac{2n-1}{n-1}(1 - \frac{1}{n})^n,
    \end{align}
    and 
        \begin{align}
        P_{G \sim M(Q)}(G \nin \mathcal{G}_n) = 1- P(v) = 1 - \frac{2n-1}{n-1}(1 - \frac{1}{n})^n.
    \end{align}
\end{proof}
\textbf{Discussion. }It is worth noting that $\mathcal{G}_n$ is exactly the MEC of any graph in $\mathcal{G}_n$.
Hence, $P_{G \sim M(Q)}(G \nin \mathcal{G}_n)$ represents the probability that the graph sampled from $M(Q)$ is incorrect.
It indicates that a Node-Edge model could suffer from an inevitable error rate of $0.2642$ though has been perfectly trained to predict $M(Q)$.
\color{black}
% \section{More General Case Study} \label{sec:mgcs}
% % This section provide a more general case study of the inconsistency probability of the Node-Edge approach.
% % Consider a simulator that generates DAGs for nodes $\{y, x_1, x_2, \dots, x_n\}$ from $n + 1$ DAGs with the same skeleton without any v-structure: $G_0: y \rightarrow x_i$ for all $x_i$, and $G_i: y \rightarrow x_j$ for all $x_j \neq x_i$ together with $x_i \rightarrow y$. These DAGs are from the same MEC, and the parametrized forms are designed to yield same joint distribution for data sampled from all DAGs, making them inherently indistinguishable. Under this setting, we have
% This section provide a more general case study of the inconsistency probability of the Node-Edge approach. 
% Concretely, let $\mathcal{G}$ be the set of graphs where there is a central node $y$ such that (1) every other node is connected to $y$, (2) there is no edge between the other nodes, (3) $y$ has at most one edge pointing to it. Denote $\mathcal{G}_n$ as the subset of $\mathcal{G}$ with graphs containing $n$ non-central nodes. 
% Hence, all $G$s from $\mathcal{G}_n$ share a same MEC denoted as $MEC_{\mathcal{G}_n}$, which indicates that there exists a data distribution $D_n$ Markovian compatible with any graph $G \in \mathcal{G}_n$. 
% We have

% \begin{Proposition}
% Denoting $Q_n$ as a distribution over $\mathcal{G}_n$, for any Node-Edge model $M$ optimally trained on $Q_n$ and $D_n$, denoting $M(D_n)$ as the output graph distribution of $M$ with the given input data $D_n$, we have 
% \begin{align}\max_{Q_n} P_{G \sim M(D_n)}(G \nin MEC_{\mathcal{G}_n}) = 
% 1 - \frac{2n-1}{n-1}(1 - \frac{1}{n})^n.
% \end{align}
% As a corollary, we have 
%     \begin{align}
% \sup_n \max_{Q_n} P_{G \sim M(D_n)}(G \nin MEC_{\mathcal{G}_n}) = 
% 1 - \frac{2}{e} \approx 0.2642.
%     \end{align}
% \end{Proposition}
% \color{black}

% % \begin{Proposition}
% % Suppose models with the Node-Edge approach can be trained optimally to predict the DAGs. Denoting the probability of generating DAG $D_i$ as $P_i$, and the probability of yielding inconsistent predictions as $P(inconsistency)$, we have
% %     \begin{align}
% %         \max_{P_0, P_1, \dots, P_n} P(inconsistency) = 
% %         1 - \frac{2n-1}{n-1}(1 - \frac{1}{n})^n, 
% %     \end{align}
% %     under the condition that $P_0=0$ and $\forall i = 1, 2, \dots, n$, $P_i = \frac{1}{n}$. As a corollary, we have
% %     \begin{align}
% %             \sup_{n, P_0, P_1, \dots, P_n} P(inconsistency) = \lim_{n \rightarrow +\infty} \max_{P_0, P_1, \dots, P_n} P(inconsistency) =  1 - \frac{2}{e} \approx 0.2642.
% %     \end{align}

% % \end{Proposition}
% \begin{proof}
% Denote other nodes except for the central node as $x_i$ where $i \in \left\{1, 2, \dots, n\right\}$. 
% In our setting, the distribution $\mathcal{G}_n$ contains $n + 1$ DAGs with the same skeleton without any v-structure: $G_0: y \rightarrow x_i$ for all $x_i$, and $G_i: y \rightarrow x_j$ for all $x_j \neq x_i$ together with $x_i \rightarrow y$.
% Denote the sampling probability of DAG $D_i$ from $\mathcal{G}_n$ as $P_i$.

%     As the Node-Edge model $M$ is trained optimally, the prediction of the existence probability of the edge $y \rightarrow x_i$ is $1 - P_i$.
    
%     If $\exists i$, $P_i = 1$, it means that $\mathcal{G}_n$ only contains the DAG $G_i$. Therefore, the model $M$ only predicts DAG $G_i$ and we have $P_{G \sim M(D_n)}(G \nin MEC_{\mathcal{G}_n}) = P(G_i \nin MEC(G)) = 0$.

%     If $\forall i$, $P_i < 1$, denoting $Q_i = 1 - P_i$ and $P(consistency)$ as the probability of yielding consistent predictions, i.e., $P(consistency) = P_{G \sim M(D_n)}(G \in MEC_{\mathcal{G}_n})$, we have 
%     \begin{align}
%         P(consistency) = \prod_{i=1}^n Q_i + \sum_{j=1}^n \frac{\prod_i^n Q_i}{Q_j} (1 - Q_j)
%         &= ( \prod_{i=1}^n Q_i) \cdot (1 + \sum_{j=1}^{n} \frac{1-Q_j}{Q_j}).
%     \end{align}
%     As $P(consistency)$ is a probability, we have $P(consistency) > 0$. Denoting function 
%     \begin{align}
%     f(Q_1, Q_2, \dots, Q_n) = \log P(consistency) = \sum_{i=1}^n \log Q_i + \log (1 + \sum_{j=1}^n \frac{1-Q_j}{Q_j}),
%     \end{align} we would like to find its minimum s.t. $\sum_i Q_i \geq n - 1$ and $Q_i \in (0, 1]$.

%     Define its Lagrange function \begin{align}
%         L(Q_1, Q_2, \dots, Q_n, \lambda) = f + \lambda (n-1-\sum_i Q_i).
%     \end{align}

%     We have 
%     \begin{align}
%         \frac{\partial L}{\partial \lambda} = n - 1 - \sum_i Q_i,
%     \end{align}
%     and 
%     \begin{align}
%         \frac{\partial L}{\partial Q_i} = \frac{1}{Q_i}(1 - \frac{1}{Q_i(1-n + \sum_{k=1}^{n}\frac{1}{Q_k})}) - \lambda.
%     \end{align}

%     Now we are going to find the extremums for $L(Q_1, Q_2, \dots, Q_n, \lambda)$.
%     \begin{enumerate}
%         \item[(1)] If $\lambda = 0$, we have $\forall i$, $\frac{\partial f}{\partial Q_i} = 0$, then
%         \begin{align}
%             \forall i, Q_i = \frac{1}{(1 - n + \sum_{k=1}^{n} \frac{1}{Q_k})}.
%         \end{align}
%         It indicates that $\forall i, Q_i = 1$, hence $f = 0$ and $P(consistency) = 1$.
%         \item[(2)] If $\lambda \neq 0$, $\exists i$, we have $\forall i$, $\frac{\partial f}{\partial Q_i} = \lambda$ and $\sum_{i=1}^n = n - 1$. In other words, we have 
%         \begin{align}
%             \forall i, j, \frac{\partial f}{\partial Q_i} = \frac{\partial f}{\partial Q_j} = \lambda.
%         \end{align}
%         Define function 
%         \begin{align}
%         h(Q_i) = \frac{\partial f}{\partial Q_i} = \frac{1}{Q_i}(1 - \frac{1}{Q_i(1-n + \sum_{k=1}^{n}\frac{1}{Q_k})}).
%         \end{align}
%         we can rewrite the function as 
%         \begin{align}
%             h(Q_i) = \frac{1}{Q_i}(1 - \frac{1}{1 + AQ_i}),
%         \end{align}
%         where $A = 1 - n + \sum_{k\neq i} \frac{1}{Q_k} \geq 1 - n + \frac{(n-1)^2}{n-1-Q_i} > 0$.
%         Therefore, $h(x)$ is a monotonic function in its domain. 

%         It indicates that $\forall i, j$, $Q_i = Q_j = \frac{n-1}{n}$, where $P(consistency) = \frac{2n-1}{n-1}(1 - \frac{1}{n})^n$.
%     \end{enumerate}
%     Now we are going to list the boundary points for $f$.
%     \begin{enumerate}
%         \item[(1)] $\forall i$, $Q_i = 1$, it becomes the first extremum point.
%         \item[(2)] $\exists i$, $Q_i$ is approaching to $0$. Due to the constraint of $\sum Q_i \geq n - 1$, other $Q$s are approaching to $1$. We have $\lim_{Q_i \rightarrow 0} f = 0$ and $P(consistency) = 1$.
%     \end{enumerate}
%     In conclusion, the maximum point of function $f$ is $\forall i$, $Q_i = \frac{n - 1}{n}$, where 
%     \begin{align}
%         P(consistency) = \frac{2n-1}{n-1}(1 - \frac{1}{n})^n,
%     \end{align}
%     and 
%         \begin{align}
%         P_{G \sim M(D_n)}(G \nin MEC_{\mathcal{G}_n}) = 1- P(consistency) = 1 - \frac{2n-1}{n-1}(1 - \frac{1}{n})^n.
%     \end{align}
% \end{proof}

% \section{More Related Work} \label{sec:app:rw}
% % For the score-based and continuous optimization methods, we refer to our appendix and \citet{glymour2019review,vowels2022d} for a thorough exploration of this literature. 
% Score-based methods aim to find an optimal DAG according to a predefined score function, subject to combinatorial constraints. 
% These methods employ specific optimization procedures such as forward-backward search GES \citep{chickering2002optimal}, hill-climbing \citep{koller2009probabilistic}, and integer programming \citep{cussens2011bayesian}.
% Continuous optimization methods transform the discrete search procedure into a continuous equality constraint.
% NOTEARS \citep{zheng2018dags} formulates the acyclic constraint as a continuous equality constraint and is further extended by DAG-GNN \citep{yu2019dag}, DECI \citep{geffner2022deep} to support non-linear causal relations. 
% DECI \citep{geffner2022deep} is a flow-based model which can perform both causal discovery and inference on non-linear additive noise data.
% % These methods can be viewed as unsupervised optimization since they do not access additional datasets associated with ground-truth causal relations. 
% Recently, ENCO \citep{lippe2021efficient} is proposed as a continuous optimization method where the edge orientation is modeled as a separate parameter to maintain the acyclicity.
% It is guaranteed to converge to the correct graph if interventions on all variables are available.
%  \textcolor{black}{RL-BIC \citep{Zhu2020Causal} utilizes Reinforcement Learning to search for the optimal DAG.}
% These methods can be viewed as unsupervised since they do not access additional datasets associated with ground truth causal relations.


\section{Experimental Settings} \label{sec:app:exp:set}
% \paragraph{Metrics.} In all tables, $\pm$ indicates that the mean value and maximum deviation of three runs with different random seeds are reported.

% For metrics, F1-scores, accuracy and SHD are general metrics for traditional methods and DNN-based methods.
% However, for DNN-based SCL method, AUC and AUPRC are more reasonable metrics because they avoid the influence of threshold selection.

% In the field of skeleton prediction tasks, the F1 score has emerged as a widely adopted metric due to its ability to effectively balance precision and recall \citep{ding2020reliable,ma2022ml4s}. 
% This metric provides a comprehensive evaluation of the model's performance, particularly in cases where the data distribution is imbalanced.
% Accuracy, another commonly used metric, offers a direct measure of the proportion of misclassified edges within the graph. 
% It can also be interpreted as a normalized version of the Structural Hamming Distance (SHD), which has gained popularity in recent years \citep{ma2022ml4s,lorchamortized,ke2023learning}.

% Considering that deep learning models typically output probabilities rather than discrete labels, the Area Under the Receiver Operating Characteristic Curve (AUC) and the Area Under the Precision-Recall Curve (AUPRC)  are also employed as more robust metrics. 
% These metrics take into account all possible decision thresholds, providing a comprehensive evaluation of the model's performance across various operating points. 
% By incorporating these metrics, researchers can gain a deeper understanding of their model's performance, ultimately contributing to the development of more advanced and reliable skeleton prediction systems that are worthy of recognition in top AI conferences.
%For skeleton prediction task, F1 score is a popularly used metric in the domain \cite{ding2020reliable,ma2022ml4s}. 
% Accuracy is a straightforward metric on the number of misclassified edges in the graph, and can also be seen as a normalized version of SHD, which is another popularly used metric \cite{ma2022ml4s,lorchamortized,ke2023learning}. 
% As the output of deep learning models are probabilities instead of single labels, AUC and AUPRC are two more robust metrics as they consider all possible decision thresholds.

% For the CPDAG prediction task, accuracy is used as a comparison metric, which measures the ratio of misclassified edges in the predicted CPDAG. 
% Following the previous paper \citep{dai2023ml4c}, the F1-scores calculated for identifiable edges and v-structures are also provided for a more comprehensive comparison. 
% Following the ML4C approach, v-structure F1 score, Identifiable edges F1, and SHD are also used for CPDAG prediction evaluation.
% For the CPDAG prediction task, we also utilize accuracy as the comparison metric, which counts the ratio of misclassified edges in the predicted CPDAG. 
% Following the previous paper \cite{dai2023ml4c}, we also utilize the F1-score over identifiable edges and v-structures for a more comprehensive comparison. 
% Following ML4C, use v-structure F1 score, Identifiable edges F1, and SHD on CPDAG prediction. 


% \paragraph{Baselines.} To demonstrate the effectiveness and superiority of the proposed framework, several strong baselines representing multiple categories are selected for comparison. These baselines include:
% \begin{enumerate}
%     \item PC: A classic constraint-based causal discovery algorithm based on conditional independence tests. The version with parallelized optimization is selected \citep{le2016fast}.
%     \item GES: A classic score-based greedy equivalence search algorithm \citep{chickering2002optimal}.
%   %  \item DirectLiNGAM: A function-based learning algorithm \cite{shimizu2011directlingam}.
%   \item NOTEARS: A gradient-based algorithm for linear data models \citep{zheng2018dags}.
%     \item DAG-GNN: A continuous optimization algorithm based on graph neural networks \citep{yu2019dag}.
%     % \item NOTEARS-MLP: A gradient-based algorithm for non-linear data models \citep{zheng2018dags}.
%     \item GOLEM: A more efficient version of NOTEARS \citep{ng2020role}.
%     \item GRAN-DAG: A gradient-based algorithm using neural network modeling for non-linear additive noise data \citep{Lachapelle2020Gradient-Based}. 
%     \item AVICI: A powerful deep learning-based supervised causal learning method \citep{lorchamortized}.
%     \color{black}
% \end{enumerate}
\color{black}
\textbf{Baselines.} To demonstrate the effectiveness and superiority of the proposed framework, 
several representative baselines from multiple categories are selected for comparison. 
The PC algorithm is a classic constraint-based causal discovery algorithm based on conditional independence tests, and the version with parallelized optimization is selected \citep{le2016fast}.
GES, a classic score-based greedy equivalence search algorithm, is also included \citep{chickering2002optimal}.
For continuous optimization methods, we compare with NOTEARS \citep{zheng2018dags}, a representative gradient-based optimization method, and GOLEM \citep{ng2020role}, regarded as a more efficient variant of NOTEARS. 
For neural network based optimization algorithms, we compare with DAG-GNN \citep{yu2019dag}, an optimization algorithm based on graph neural networks, and GRAN-DAG, a gradient-based algorithm using neural network modeling \citep{Lachapelle2020Gradient-Based}.
For DNN-based SCL methods, we compare with AVICI, which is the most related work to ours and regarded as the current state-of-the-art method \citep{lorchamortized}.
\color{black}

\textbf{Implementation Details. }The implementation from gCastle \citep{zhang2021gcastle} is utilized for baselines except the SCL methods (i.e., SLdisco and AVICI). 
For PC algorithm, we employ the Fisher-Z transformation with a significance threshold of $0.05$ for conditional independence tests, which is a prevalent choice in statistical analyses and current PC implementations \citep{zhang2021gcastle,zheng2024causal}.
Our criterion for graph selection in GES experiments is the Gaussian Bayesian Information Criterion (BIC), specifically the $l_\infty$-penalized Gaussian likelihood score. It is used in the original paper \citep{chickering2002optimal}, and remains a favored variant in the literature.
For NOTEARS, adhering to the official implementation's settings, we configure NOTEARS with a maximum of 100 dual ascent steps, and an edge dropping threshold of 0.3. For hyperparameters lacking specific default settings, such as the L1 penalty and loss function type, we default to settings used by gCastle \cite{zhang2021gcastle}, employing an L1 penalty of 0.1 and an L2 loss function.
For DAG-GNN, we utilize hyperparameter settings directly from the original implementation, ensuring consistency with established benchmarks.
For GOLEM and GRAN-DAG, we also use the default setting of gCastle \citep{zhang2021gcastle}.
\color{black}
Note that the CSIvA model \citep{ke2023learning} is also a closely related method, but it is not compared due to the unavailability of its relevant codes and its requirement for interventional data as input. 
The original implementation of SLdisco \cite{petersen2023causal} was developed in R. To enhance compatibility with our data generation and evaluation workflows, we reimplemented the model using PyTorch.
The original AVICI model \citep{lorchamortized} does not support discrete data.
Therefore, we use an embedding layer to replace its first linear layer when using AVICI on discrete data.


\textbf{Synthetic Data.} We randomly generate random graphs from multiple random graph models. For continuous data, following previous work \citep{lorchamortized}, Erdős-Rényi (ER) and Scale-free (SF) are utilized as the training graph distribution $p(G)$.
The degree of training graphs in our experiments varies randomly among 1, 2, and 3.
\textcolor{black}{For testing graph distributions, Watts-Strogatz (WS) and Stochastic Block Model (SBM) are used, with parameters consistent with those in the previous paper \citep{lorchamortized}. }
All synthetic graphs for continuous data contain 30 nodes.
\textcolor{black}{The lattice dimension of Watts-Strogatz (WS) graphs is sampled from $\{2, 3\}$, yielding an average degree of about $4.92$. The average degrees of Stochastic Block Model (SBM) graphs are set at 2, following the settings in the aforementioned paper.}
For discrete data, 11-node graphs are used.
SF is utilized as the training graph distribution $p(G)$ and ER is used for testing.
\textcolor{black}{The synthetic training data is generated in real-time, and the training process does not use the same data repeatedly.}
All synthetic test datasets contain 100 graphs, and the average values of the metrics on the 100 graphs are reported to comprehensively reflect the performance.

For the forward sampling process from graph to continuous data, both the linear Gaussian mechanism and general nonlinear mechanism are applied.
Concretely, the Random Fourier Function mechanism is used for the general nonlinear data following the previous paper \citep{lorchamortized}.
% Therefore, it yields two types of continuous datasets. 
In synthesizing discrete datasets, the Bernoulli distribution is used following previous papers \citep{dai2023ml4c,ma2022ml4s}.

\begin{figure*}
    \centering
    \includegraphics[width=\linewidth]{figures/abl.pdf}
    \caption{Illustration of the architecture comparison of Node-Edge models, SiCL-no-PF and SiCL.}
    \label{fig:abl}
\end{figure*}


\textbf{More Implementation Details and Computational Resources. } The two network modules, i.e., the SPN and VPN, are optimized by Adam optimizer with default hyperparameters. Following previous work \citep{lorchamortized}, the training batch size is set as 20. 
All classic algorithms are run on an AMD EPYC 7V13 CPU, and DNN-based methods are run on Nvidia 1080Ti, A40 and A100 GPUs. 
Training SiCL on a 30-node training set with batch size 20 needs about 60GB memory, and training on a 11-node training set needs about 20GB memory.
The learning rate is $3 \times 10^{-4}$.
The batch size is $15$, and the DNN models are trained for $1.2 \times 10^{5}$ batches by default.
\color{black}

\begin{table*}[t]
\centering
    % \resizebox{\columnwidth}{!}{%
\begin{threeparttable}
\caption{\textbf{General comparison of SiCL and other methods}. The average performance results in three runs are provided for SiCL method. GES takes more than 24 hours per graph on WS-L-G, and SLdicso is
unsuitable on non-linear-Gaussian data, hence the results are not included.}
\label{tab:epder}
\begin{tabular}{@{}ccccccccc@{}}
\toprule
\multirow{2}{*}{Dataset} &\multirow{2}{*}{Method} & \multicolumn{4}{c}{Skeleton Prediction }& \multicolumn{3}{c}{CPDAG Prediction} \\
 && s-F1$\uparrow$ & s-Acc.$\uparrow$ & s-AUC$\uparrow$ & s-AUPRC$\uparrow$ &  v-F1$\uparrow$ & o-F1$\uparrow$ & SHD$\downarrow$\\
 \midrule
\multirow{8}{*}{WS-L-G}&PC & $30.4 $&$ 65.6$&N/A&N/A&$ 15.6$&$16.0 $ &$170.4 $\\
% &GES & * &* &*&*&*&*&*\\
&NOTEARS & $33.3 $  & $ 65.1$&N/A&N/A&$ 27.9$&$31.5$ & $159.8$ \\
&DAG-GNN & $35.5$  & $ 55.4$&N/A&N/A&$32.2$&$32.7$ & $193.7 $ \\
&GRAN-DAG& $16.6$ & $62.1$ & N/A & N/A & $11.7$ & $11.7$ & $170.1$ \\
% &NOTEARS-MLP & $24.6$ & $60.3$ & N/A & N/A & $12.2$ & $11.8$ & $190.0$ \\
&GOLEM& $30.0$ & $63.4$ & N/A & N/A & $15.8$ & $19.3$ & $172.7$ \\
&SLdisco & $0.1$ & $66.0$ & $50.2$ & $34.6$ & $0.0$ & $0.1$ & $147.9$ \\
&AVICI & $39.9 $  & $ 74.0$ &$ 71.5$&$ 62.2$&$ 28.2$&$ 35.8$&$119.2$\\
&SiCL & $\mathbf{44.7} $ & $\mathbf{75.3} $&$\mathbf{73.7} $&$\mathbf{65.4} $&$\mathbf{32.0} $&$\mathbf{38.5} $&$\mathbf{116.1} $\\
 \midrule
\multirow{9}{*}{SBM-L-G}&PC & $58.8$&$90.0$&N/A&N/A &$34.8$&$35.9$&$56.4$\\
&GES & $70.8$&$89.4$ &N/A&N/A&$53.9$&$55.0$&$60.3$\\
&NOTEARS & $80.1$  & $94.5$&N/A&N/A&$76.2$&$77.8$ & $26.7$ \\
&DAG-GNN & $66.2$  & $87.4$&N/A&N/A&$60.3$&$62.5$ & $61.0$ \\
&GRAN-GAG & $22.6$  & $85.9$&N/A&N/A&$13.8$&$14.4$ & $64.7$ \\
% &NOTEARS-MLP & $ 44.7$ & $75.4 $ & N/A & N/A & $ 37.9$ & $39.0 $ & $112.5 $ \\
&GOLEM & $68.5$ &$88.5$ &N/A&N/A&$63.5$&$65.2$&$55.1$\\
&SLdisco & $1.9$ & $85.7$ & $56.3$ & $17.6$ & $0.9$ & $1.2$ & $62.6$ \\
&AVICI & $84.3$  & $96.2$ &$98.1$&$92.7$&$79.1$&$81.6$&$17.7$\\
&SiCL &  $\mathbf{85.8}  $  & $\mathbf{96.4}  $& $\mathbf{98.3}$& $\mathbf{93.4} $&$\mathbf{80.6}  $&$\mathbf{82.7}  $&$\mathbf{17.1}  $\\
\midrule
\multirow{9}{*}{WS-RFF-G}&PC & $36.1 $&$69.9$&N/A&N/A &$ 14.8$&$16.1$&$ 156.9$\\
&GES & $ 41.7$&$66.6$ &N/A&N/A&$21.1 $&$23.6$&$174.1$\\
&NOTEARS & $37.7 $  & $64.6 $& N/A & N/A &$30.9 $&$33.4 $ & $164.4 $ \\
&DAG-GNN & $33.2 $  & $ 65.4$&N/A&N/A&$27.0 $&$28.9 $ & $161.1 $ \\
&GRAN-DAG & $4.7 $  & $ 66.7$&N/A&N/A&$0.8 $&$1.1 $ & $146.9 $ \\
% &NOTEARS-MLP & $52.7 $ & $ 40.2$ & N/A & N/A & $ 44.2$ & $ 47.7$ & $282.8 $ \\
 &GOLEM & $27.6$ &$62.4$ &N/A&N/A&$13.8$&$17.7$&$175.8$\\
&AVICI & $47.7 $  & $75.9$ &$ 76.3$&$ 67.6$&$38.7$&$45.2 $&$110.6 $\\
&SiCL & $\mathbf{51.8}  $  & $ \mathbf{77.4}$ &$\mathbf{81.1}$&$ \mathbf{72.9}$&$ \mathbf{40.3}$&$ \mathbf{46.3}$&$ \mathbf{107.0} $\\
\midrule
\multirow{9}{*}{SBM-RFF-G}&PC & $57.5$&$89.3$&N/A&N/A &$32.7$&$ 34.2$&$60.9$\\
&GES & $56.5 $&$84.9$ &N/A&N/A&$37.0$&$38.0$&$82.4$\\
&NOTEARS & $55.6 $  & $86.2 $&N/A&N/A&$ 46.5$&$ 48.5$ & $66.3 $ \\
&DAG-GNN & $ 47.1$  & $82.1 $&N/A&N/A&$39.0$&$40.6$ & $86.2 $ \\
&GRAN-DAG & $17.4$ &$87.4$ & N/A& N/A &$3.2 $&$3.8$&$58.2$\\
% &NOTEARS-MLP & $48.2$ & $71.0$ & N/A & N/A & $41.0$ & $43.5$ & $130.5$ \\
&GOLEM & $31.1$ &$75.7$ & N/A& N/A &$23.0 $&$24.8$&$112.0$\\
&AVICI & $76.6$  & $ 94.5$ &$ 95.4$&$85.7 $&$69.3$&$72.7 $&$ 27.2$\\
&SiCL & $ \mathbf{82.1}$  & $ \mathbf{95.7}$ &$ \mathbf{97.1}$&$ \mathbf{90.7} $&$ \mathbf{75.7} $&$ \mathbf{78.0}$&$ \mathbf{21.9} $\\
\midrule
\multirow{8}{*}{ER-CPT-MC}&PC & $82.2$&$83.0$ &N/A&N/A&$39.2$&$40.6$&$16.4$\\
&GES & $ 82.1$&$81.8$ &N/A&N/A&$40.4$&$42.4$&$17.1$\\
&NOTEARS & $16.7$  & $74.8$&N/A&N/A&$0.2$&$0.6$& $16.1$ \\
&DAG-GNN & $ 24.8$  &  $73.5$&N/A&N/A&$ 3.4$&$3.7 $ & $15.9 $ \\
&GRAN-DAG & $40.8$ &$77.0$ & N/A& N/A &$6.8 $&$7.3$&$15.6$\\
% &NOTEARS-MLP & $ $ & $ $ & N/A & N/A & $ $ & $ $ & $ $ \\
&GOLEM& $37.6$ & $66.4$ & N/A & N/A & $4.6$ & $9.3$ & $21.9$ \\
&AVICI & $76.9$  & $88.4$ & $93.5 $& $87.9 $&$56.6$&$57.6$&$10.2$\\
&SiCL &  $\mathbf{84.2}$  & $\mathbf{90.1}$&$ \mathbf{96.6}$& $\mathbf{94.0} $& $ \mathbf{58.3}$&$\mathbf{59.9}$&$\mathbf{10.1}$\\
\bottomrule
\end{tabular}
\end{threeparttable}
% }
\end{table*}

\begin{table*}[tb]
    \centering
    % \resizebox{\linewidth}{!}{%
\begin{threeparttable}
\caption{Full ablation study results.}
\label{tab:fcplg}
\begin{tabular}{ccccccccc}
\toprule
Dataset &Method & s-F1$\uparrow$ & s-Acc.$\uparrow$ & s-AUC$\uparrow$ & s-AUPRC$\uparrow$ &  v-F1$\uparrow$ & o-F1$\uparrow$ & SHD$\downarrow$\\
 \midrule
\multirow{3}{*}{WS-L-G}& SiCL-Node-Edge & $39.9 $&$ 74.0$&$71.5$&$62.2$&$ 28.2$&$35.8 $ &$119.2 $\\
&SiCL-no-PF & $ 42.4$  & $ 74.4$ &$ 72.8$&$ 63.5$&$ 30.5$&$ 37.9$ &$118.4 $\\
&SiCL & $\mathbf{44.7} $ & $\mathbf{75.3} $&$\mathbf{73.7} $&$\mathbf{65.4} $&$\mathbf{32.0} $&$\mathbf{38.5} $&$\mathbf{116.1} $\\ \hline
\multirow{3}{*}{SBM-L-G}& SiCL-Node-Edge & $ 84.3$&$ 96.2$&$98.1$&$92.7$&$ 79.1$&$81.6 $ &$17.7 $\\
&SiCL-No-PF & $85.5$  & $\mathbf{96.4}$ &$\mathbf{98.3}$&$93.3$&$79.4$&$82.2$&$17.3$\\
&SiCL &  $\mathbf{85.8}$  & $\mathbf{96.4} $& $\mathbf{98.3} $& $\mathbf{93.4} $&$\mathbf{80.6}  $&$\mathbf{82.7}  $&$\mathbf{17.1}$\\
\bottomrule
\end{tabular}
\end{threeparttable}
% }
\end{table*}
\section{Extra Experimental Results} \label{sec:app:exp:e}



\begin{figure*}[t]
     \centering
     \begin{subfigure}[b]{0.45\textwidth}
         \centering
         \includegraphics[width=\textwidth]{figures/v_struc_ws_font.pdf}
         \caption{WS-LG}
         \label{fig:vws}
     \end{subfigure}
     \hfill
     \begin{subfigure}[b]{0.45\textwidth}
         \centering
         \includegraphics[width=\textwidth]{figures/v_struc_sbm_font.pdf}
         \caption{SBM-LG}
         \label{fig:vsbm}
     \end{subfigure}
     \caption{Variation trends of the test performance of the V-structure Prediction Network on WS-LG and SBM-LG during training.}
        \label{fig:tcs}
\end{figure*}



\subsection{Effectiveness of V-structure Prediction Network} Fig. \ref{fig:tcs} illustrates the test performance trends of the v-structure prediction model on SBM and WS random graphs during the training process. In this model, the feature extractor $FE$ is fine-tuned from the skeleton prediction model. The performance increases rapidly and achieves a relatively high level after just a few initial epochs. This suggests that our v-structure prediction network is capable to predict v-structures, and indicates that the pre-trained pairwise features from the skeleton prediction model are both effective and generalizable.



\color{black}
\subsection{More Evidence on Effectiveness of Pairwise Representation}
To further support the effectiveness of using pairwise representation, we present additional experimental results on different training datasets and test datasets, including ER-L-G, SF-L-G, ER-RFF-G, and SF-RFF-G.
% other three test datasets. The graph distributions are Geometric Random Graphs (GRG) and Scale-free graphs with two different parameters (marked as SF1 and SF2). 
% The structural-equation distribution is random linear (L) and noise distribution is Gaussian distribution (G), following other experimental settings.
For models, we compare SiCL with a variant without pairwise representation, i.e., SiCL-no-PF.
% We also implement another variant as a baseline where the alternative attention in the encoder is removed and marked as SiCL-no-AA, and compare it with another variant without pariwise representation, i.e., SiCL-no-AA-no-PF.

The results are provided in Tab. \ref{tab:mpc}. Models with pairwise representation ourperform the corresponding baseline models under almost all comparisons, further verifying the effectiveness of using pairwise representation in models.



\begin{table}[t]\color{black}
\centering
% \resizebox{\linewidth}{!}{%
\begin{threeparttable}
\caption{\color{black}More performance comparison on the effectiveness of pairwise representation.}\label{sec:mpc}
\label{tab:mpc}
\begin{tabular}{ccccccc}
\toprule 
Training Dataset & Test Dataset & Method & s-F1$\uparrow$ & s-AUC$\uparrow$ & s-AUPRC$\uparrow$ & s-Acc.$\uparrow$ \\
\midrule
\multirow{8}{*}{ER-L-G} & \multirow{2}{*}{ER-L-G} & SiCL-no-PF & 75.7 &84.6  & 83.1 & 78.5\\
 &  & SiCL &  80.2 & 89.6 & 90.1 &82.3 \\ \cline{2-7}
 & \multirow{2}{*}{SF-L-G} & SiCL-no-PF & 74.9 &  92.5& 87.3 &84.1 \\
 &  & SiCL & 79.0 & 96.0 & 93.7 & 87.0\\ \cline{2-7}
  & \multirow{2}{*}{ER-RFF-G} & SiCL-no-PF & 49.5 & 60.5 & 49.1 &58.8 \\
 &  & SiCL & 51.0 & 67.0 & 57.6 & 65.2\\ \cline{2-7}
  & \multirow{2}{*}{SF-RFF-G} & SiCL-no-PF & 40.4 & 57.9 & 38.9 & 57.5\\
 &  & SiCL & 46.4 & 71.4 & 53.7 & 69.0\\ \hline
 \multirow{8}{*}{SF-L-G} & \multirow{2}{*}{ER-L-G} & SiCL-no-PF & 64.6 & 77.3 & 68.7 & 70.7\\
 &  & SiCL & 68.0 & 82.1 & 76.4 & 74.3\\ \cline{2-7}
 & \multirow{2}{*}{SF-L-G} & SiCL-no-PF & 88.5 & 96.7 & 95.0 & 91.2\\
 &  & SiCL & 89.7 & 97.9 & 97.0 & 92.4\\ \cline{2-7}
  & \multirow{2}{*}{ER-RFF-G} & SiCL-no-PF & 44.3 & 62.3 & 50.9 & 58.4\\
 &  & SiCL & 47.0 & 66.2 & 55.8 & 63.3\\ \cline{2-7}
  & \multirow{2}{*}{SF-RFF-G} & SiCL-no-PF & 48.1 & 71.6 & 53.9 & 65.8 \\
 &  & SiCL & 56.0 & 79.6 & 64.8 & 74.2\\ \hline
  \multirow{8}{*}{ER-RFF-G} & \multirow{2}{*}{ER-L-G} & SiCL-no-PF & 64.0 & 73.3 & 65.8 & 67.3 \\
 &  & SiCL & 72.0 & 82.0 & 81.1 & 75.2 \\ \cline{2-7}
 & \multirow{2}{*}{SF-L-G} & SiCL-no-PF & 58.1 & 79.0 & 66.8 & 72.8\\
 &  & SiCL & 70.1 & 88.0 & 83.6 & 80.9\\ \cline{2-7}
  & \multirow{2}{*}{ER-RFF-G} & SiCL-no-PF & 63.2 & 74.3 & 67.7 & 71.0\\
 &  & SiCL & 74.8 & 85.7 & 84.5 & 79.7\\ \cline{2-7}
  & \multirow{2}{*}{SF-RFF-G} & SiCL-no-PF & 56.3 & 78.3 & 65.9 & 75.0 \\
 &  & SiCL & 68.2 & 87.0 & 81.5 & 82.1 \\ \hline
  \multirow{8}{*}{SF-RFF-G} & \multirow{2}{*}{ER-L-G} & SiCL-no-PF & 60.3 &  71.2 & 58.7 & 64.7\\
 &  & SiCL & 65.6 & 78.0 & 72.5 & 70.5\\ \cline{2-7}
 & \multirow{2}{*}{SF-L-G} & SiCL-no-PF & 73.6 & 90.5 & 82.9 & 81.0\\
 &  & SiCL & 79.1 & 94.2 & 90.0 & 85.5\\ \cline{2-7}
  & \multirow{2}{*}{ER-RFF-G} & SiCL-no-PF & 57.7 & 71.4 & 60.7 & 66.8\\
 &  & SiCL & 67.2 & 80.5 & 75.8 & 73.9\\ \cline{2-7}
  & \multirow{2}{*}{SF-RFF-G} & SiCL-no-PF & 74.8 & 90.2 & 82.4 & 83.5\\
 &  & SiCL & 80.4 & 94.2 & 90.5 & 87.3\\ 
\bottomrule 
\end{tabular}
\end{threeparttable}
% }
\end{table}

\subsection{Additional Comparison on DAG Prediction}
We provide an additional comparison with the AVICI baseline on the DAG prediction task. Since SiCL predicts CPDAGs and does not directly produce DAG predictions, we corrected the DAG predictions from AVICI using the edge directions inferred from the CPDAGs predicted by SiCL. The results, summarized in the Table \ref{tab:acdp}, demonstrate that incorporating CPDAG-inferred edge directions improves the DAG prediction metrics. This further confirms the effectiveness and generality of our approach, even in tasks focused on DAG metrics. 
\begin{table}[]
    \centering
        \caption{Additional Comparison on DAG Prediction}
    \label{tab:acdp}
    \begin{threeparttable}
    \begin{tabular}{cccccc}
         \toprule  
         Method & Dataset &F1 Score$\uparrow$ & AUC$\uparrow$ & AUPRC$\uparrow$ & Acc.$\uparrow$ \\ \midrule 
         AVICI & \multirow{2}{*}{WS-L-G} &$38.4$ &$86.3$ &$57.7$ &$85.9$ \\
         SiCL-Corrected AVICI && $35.8$&$87.2$ &$60.5$ &$86.2$  \\
         AVICI & \multirow{2}{*}{SBM-L-G}& $78.1$& $95.8$&$80.5$ &$97.3$ \\
         SiCL-Corrected AVICI & &$81.3$&$98.7$ &$90.8$ & $97.8$ \\
         \bottomrule
    \end{tabular}
        \end{threeparttable}
\end{table}


\subsection{Comparison with Autoregressive models on Inference Time Costs} \label{sec:auto}
To validate that the autoregressive models have a relatively high time costs due to the quadratic number of inference runs w.r.t. number of variables, we reproduce the network architecture of a representative autoregressive model, i.e., CSIvA \citep{ke2023learning}, and compare SiCL with it.
We use the same random input for both the models with increasing number of variables.
The results are provided in Fig. \ref{fig:itc}.
The time costs of the autoregressive model show a fast increasing trend and are much more than costs of SiCL, validating the correctness of our analysis.

\begin{figure}
    \centering
    \includegraphics[width=0.5\linewidth]{figures/inference_time_costs_auto_font.pdf}
    \caption{Comparison between an autoregressive model and SiCL on inference time costs.}
    \label{fig:itc}
\end{figure}

\color{black}
\subsection{Training Data Diversity and Model Generalization} We present experimental evidence that highlights the significant contribution of training data diversity to the model's generalization capabilities, even when applied to out-of-distribution (OOD) datasets. 
To illustrate this, we train one SiCL model on a combined dataset of both SF and ER, and another solely on the SF dataset. 
The comparative performance of these models is detailed in Tab. \ref{tab:trainood}. 
The model trained on the combined ER and SF datasets exhibited markedly better performance, not only on the ER dataset but also on the other two OOD datasets, with only a marginal decrease in performance on the SF dataset. 
These findings suggest that enhancing the diversity of the training data correspondingly improves the model’s ability to generalize and maintain robust performance across novel OOD datasets.

\begin{table*}[tb]\color{black}
    \centering
    \caption{\color{black}Comparison of SiCL models with different training data diversity on skeleton prediction.}
    \label{tab:trainood}
    \begin{subtable}{\linewidth}
      \centering
        \caption{\color{black}Model trained on both ER and SF}
        \begin{tabular}{lcccc}
            \toprule
            Test Dataset & s-F1$\uparrow$     & s-AUC$\uparrow$    & s-AUPRC$\uparrow$  & s-Acc.$\uparrow$    \\
            \midrule
            WS-L-G      & 36.3 & 70.6 & 60.6 & 73.3 \\
            SBM-L-G     & 78.1 & 96.8 & 88.1 & 94.8 \\
            ER-L-G      & 80.7 & 96.0 & 89.2 & 94.7 \\
            SF-L-G      & 84.7 & 98.5 & 93.6 & 95.5 \\
            \bottomrule
        \end{tabular}
    \end{subtable}%
    \\
    \begin{subtable}{\linewidth}
      \centering
        \caption{\color{black}Model trained on SF}
        \begin{tabular}{lcccc}
            \toprule
            Test Dataset & s-F1$\uparrow$     & s-AUC$\uparrow$    & s-AUPRC$\uparrow$  & s-Acc.$\uparrow$    \\
            \midrule
            WS-L-G      & 40.1 & 63.0 & 46.1 & 63.5 \\
            SBM-L-G     & 64.3 & 91.7 & 72.9 & 90.9 \\
            ER-L-G      & 67.1 & 90.4 & 73.9 & 90.8 \\
            SF-L-G      & 87.8 & 98.9 & 95.3 & 96.1 \\
            \bottomrule
        \end{tabular}
    \end{subtable}
\end{table*}


\subsection{Varying Amount of Training Graphs}
We present an analysis of how varying the amount of the training graphs influences performance on the skeleton prediction task. The results, depicted in Fig. \ref{fig:ts}, illustrate a clear trend: model performance improves in tandem with the expansion of the training dataset. This trend underscores the potential of our method to achieve even greater accuracy given a more extensive dataset.
\begin{figure*}[!ht]
     \centering
     \begin{subfigure}[b]{0.45\textwidth}
         \centering
    \includegraphics[width=\linewidth]{figures/wstrainingsize_font.pdf}
         \caption{\color{black}WS dataset}
         \label{fig:ts1}
     \end{subfigure}
     \hfill
     \begin{subfigure}[b]{0.45\textwidth}
         \centering
    \includegraphics[width=\linewidth]{figures/sbmtrainingsize_font.pdf}
         \caption{\color{black}SBM dataset}
         \label{fig:ts2}
     \end{subfigure}
         \caption{\color{black}Model performance with varying amount of training graphs.}
        \label{fig:ts}
\end{figure*}

\subsection{Varying Sample Size} We assess SiCL across various quantities of observational samples per graph during testing (100, 200, ..., 1000). The outcomes for both the skeleton prediction task and the CPDAG prediction task are depicted in Fig. \ref{fig:vtss}. It is evident that the model's performance enhances with the augmentation of sample size. These consistent upward trends suggest that SiCL exhibits stability and is not overly sensitive to changes in sample size.

\begin{figure*}[t]
     \centering
     \begin{subfigure}[b]{0.45\textwidth}
         \centering
         \includegraphics[width=\textwidth]{figures/ws1_font.pdf}
         \caption{\color{black}Variation trends of skeleton predicton task performance on WS graph with varying sample sizes.}
         \label{fig:vtss1}
     \end{subfigure}
     \hfill
     \begin{subfigure}[b]{0.45\textwidth}
         \centering
         \includegraphics[width=\textwidth]{figures/ws2_font.pdf}
         \caption{\color{black}Variation trends of CPDAG predicton task performance on WS graph with varying sample sizes.}
         \label{fig:vtss2}
     \end{subfigure}
         \begin{subfigure}[b]{0.45\textwidth}
         \centering
         \includegraphics[width=\textwidth]{figures/sbm1_font.pdf}
         \caption{\color{black}Variation trends of skeleton predicton task performance on SBM graph with varying sample sizes.}
         \label{fig:vtss3}
     \end{subfigure}
     \hfill
     \begin{subfigure}[b]{0.45\textwidth}
         \centering
         \includegraphics[width=\textwidth]{figures/sbm2_font.pdf}
         \caption{\color{black}Variation trends of CPDAG predicton task performance on SBM graph with varying sample sizes.}
         \label{fig:vtss4}
     \end{subfigure}
         \caption{\color{black}Variation trends of performance with varying sample sizes.}
        \label{fig:vtss}
\end{figure*}

\subsection{Varying Edge Density}
We evaluate SiCL over a range of edge densities in the test graphs, utilizing the SBM dataset, as it allows for the direct setting of average edge densities. The findings are presented in Fig. \ref{fig:vted}. It's apparent that the task is becomes more difficult as edge densities increase. However, the performance decline is not abrupt, indicating that SiCL's performance remains relatively stable across various edge densities, thereby confirming its versatility.
\begin{figure*}
     \centering
     \begin{subfigure}[b]{0.45\textwidth}
         \centering
         \includegraphics[width=\textwidth]{figures/sbmskeletondensity_font.pdf}
         \caption{\color{black}Variation trends of skeleton predicton task performance on SBM graph with varying edge densities.}
         \label{fig:vted1}
     \end{subfigure}
     \hfill
     \begin{subfigure}[b]{0.45\textwidth}
         \centering
         \includegraphics[width=\textwidth]{figures/sbmcpdagdensity_font.pdf}
         \caption{\color{black}Variation trends of CPDAG predicton task performance on SBM graph with varying edge densities.}
         \label{fig:vted2}
     \end{subfigure}
         \caption{\color{black}Variation trends of performance with varying edge densities.}
        \label{fig:vted}
\end{figure*}

\subsection{Generality on Testing Graph Sizes}
We offer an analytical perspective on the performance of the SiCL model when applied to larger WS-L-G graphs. 
It is important to highlight that the models were initially trained on graphs comprising 30 vertices, positioning this task within an out-of-distribution setting in terms of graph size. 
To establish a point of reference, we have included results from the PC algorithm as a baseline comparison.
These findings can be examined in Tab. \ref{tab:mpva}.
Despite the OOD conditions, SiCL maintains robust performance, reinforcing its scalability and the model's general applicability across varying graph sizes.


\begin{table}[t]\color{black}
\centering
% \resizebox{\linewidth}{!}{%
\begin{threeparttable}
\caption{\color{black}Performance comparison with varying amounts of graph sizes.}
\label{tab:mpva}
\begin{tabular}{l|ccc|ccc|ccc}
\toprule
Metric & \multicolumn{3}{c|}{s-F1$\uparrow$} & \multicolumn{3}{c|}{v-F1$\uparrow$} & \multicolumn{3}{c}{o-F1$\uparrow$} \\
Size & 50 & 70 & 100 & 50 & 70 & 100 & 50 & 70 & 100 \\
\midrule
PC       & $17.7$ & $14.8$ & $10.6$ & $6.4$ & $5.0$ & $3.7$ & $7.0$ & $5.6$ & $4.0$ \\
SiCL & $\mathbf{41.6}$ & $\mathbf{37.4}$ & $\mathbf{28.3}$ & $\mathbf{34.9}$ & $\mathbf{30.7}$ & $\mathbf{22.6}$ & $\mathbf{37.9}$ & $\mathbf{33.7}$ & $\mathbf{24.8}$ \\
\bottomrule
\end{tabular}
\end{threeparttable}
% }
\end{table}



\subsection{Acyclicity}
% \begin{wraptable}[8]{r}{9cm}
\begin{table}[t]\color{black}
\centering
% \resizebox{\linewidth}{!}{%
\begin{threeparttable}
\caption{\color{black}Count of cycles in the CPDAG predictions without post-processing of removing cycles.}
\label{tab:ccfc}
\begin{tabular}{@{}ccc@{}}
\toprule
Dataset & WS-L-G & SBM-L-G  \\
\midrule
Rate of Graphs with Cycles & $0.66 \pm 0.66 \%$&$0.00 \pm 0.00 \%$ \\
\bottomrule
\end{tabular}
\end{threeparttable}
% }

\end{table}
We provide an empirical evidence supporting of the rarity of cycles in the predictions. The experimental data presented in Tab. \ref{tab:ccfc} corroborates that cycles are infrequently observed in the predicted CPDAGs, even though without any post-processing on removing cycles.
\color{black}



\end{document}
