\subsection{Computing Bounding Boxes of Projected Visible Sub-Charts}
\label{sec:box}

The last step before calling our packing method is generating the target 2D axis-aligned bounding boxes for the visible portions of each projected (parameterized) chart. 
We would like these boxes to tightly bound the visible projected chart portions, or {\em visible sub-charts}, ensuring shading is captured for any visible geometry while also being as small as possible for packing efficiency. Computing bounding boxes by naively projecting chart vertices to the view plane is not only wasteful, but more importantly can produce ill-posed results for charts containing points in front of the near clip plane. We recall that the standard rendering pipeline uses homogeneous coordinates, where each three-dimensional point $p=(x,y,z)$ is represented as a four dimensional vector $p^h=(x,y,z,1)$, and the perspective transformation is represented as a $4\times4$ matrix $P$. To apply a perspective transformation to $p$, we must first compute $p^p=(x^{p},y^{p},z^{p},w^{p}) = P p^h$ and then apply a perspective divide $p' = (x^{p}/w^{p},y^{p}/w^{p},z^{p}/w^{p})$. For points in front of the near plane ($w^{p}\leq0$) the projective divide can produce ill-defined values, projecting them way outside of the screen, or even folding triangles over of the rest of the chart \cite{blinn1978clipping}. Accordingly, our bounding box computation must account for geometry outside the view frustum.

Explicitly clipping charts against the view frustum in a compute shader prior to packing fails to effectively utilize the GPU's fixed-function clip hardware, and also requires keeping track of the clipped geometry for every visible triangle on screen through the rest of the pipeline. As a triangle clipped by a frustum can have up to 7 sides, this is highly time consuming and adds significant complexity to our shading and rasterization steps. We therefore reduce waste and ensure valid bounding boxes by taking inspiration from a classic article by Blinn \shortcite{Blinn:CalculatingScreenCoverage}, which shows that conservative screen-space extents for an object can be found by clamping each post-projection vertex $p^p = (x^p,y^p,z^p,w^p)$ to the range $[-w^p,w^p$], performing the perspective divide by $w^p$, and then computing the bounding box for all resulting points $p'$. This produces results that are correct but conservative, especially when a triangle intersects the near plane. We further observe that while clipping against all planes is complicated and compute-heavy, clipping against {\em one plane} produces either triangles or quads, which allows for straightforward processing by cases (see supplementary for details). Therefore, we clip each triangle of a chart against the near clip plane if it intersects it, and against the frustum plane that reduces the size of the bounding box the most otherwise, prior to using Blinn's coverage estimator. 

The output of this step is a bounding box for each sub-chart in NDC space on the screen, whose coordinates range from $[-1...1]$. We find an integer size for each sub-chart by applying the standard viewport transform which maps $[-1...1,-1...1]$ to $[0...\operatorname{screenWidth},0...\operatorname{screenHeight}]$, and this becomes the target bounding box size for FastAtlas packing.
