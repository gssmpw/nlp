\section{Charting and Atlas Bounding Box Computation for TSS}
FastAtlas packing operates on parametrized charts generated using any existing chartification and parametrization method, and only requires as input the bounding boxes of the parametrized charts. Our end-to-end TSS framework combines it with our new chartification and bounding box computation methods, described below.

\subsection{Chart Computation}
\label{sec:charts}
\label{sec:visible}

To compute our charts, we first perform a depth pre-pass; a second pass then marks visible triangles in a {\em visible triangle (VT) buffer} containing an entry for each triangle in the scene \cite{burns2013visibility,kubisch2014opengl}. Triangles are visible if they cover at least one sample in the output framebuffer.
We then form charts by grouping visible triangles sharing common edges, forming connected components. While numerous algorithms exist for finding connected components on the CPU, we require an algorithm with a parallel, GPU-only implementation. We therefore use a union-find approach \cite{skiena1998algorithm} using the {\em ECL-CC} method of Jaiganesh and Burtscher \shortcite{Jaiganesh:2018}: for each visible triangle $i$, we check if any of its neighbors are visible by querying the VT buffer. If so, we set the $i$'th entry in the buffer to the minimum of the triangle's and its visible neighbors' buffer entries; we repeat until all pairs of visible triangles sharing common edges have the same ID. Once the process converges, we use the IDs to form connected charts. For more details and pseudocode, see the supplemental material.
