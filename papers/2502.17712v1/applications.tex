\section{Applications.}

\begin{figure*}
\includegraphics[width=\linewidth]{fig_supp_dof_v2.pdf}
\caption{Depth of field render using 100\% shading rate (a) and using a combination of 100\% and 12.5\% shading rate (b,c). In (b,c) triangles falling partly inside the focal range are shaded into a 100\% texture atlas; triangles falling outside of the focal range are shaded into a second atlas at 12.5\% shading rate. Using \cite{Neff2022MSA} with the same settings (b) introduces artifacts such as the seams on the awning and the coarse shading on the balcony and menu due to undersampling (highlighted in red). Using FastAtlas (c), the rendered outputs are nearly identical.} 
\label{fig:depthOfField}
\end{figure*}


\begin{figure*}
\includegraphics[width=\linewidth]{fig_supp_fov_v2.pdf}
\caption{Foveated rendering using a 100\% texture atlas (a) and using a combination of 100\% and 25\% shading rate atlases (b,c). In (b,c) triangles falling inside the foveated region are rendered into a high resolution texture atlas; triangles falling outside of the shaded region are rendered into a low-resolution texture atlas and then blurred. (b) Using \cite{Neff2022MSA} with the same setting, the area outside the foveated region has undersampling artifacts on areas such as the wall boundary and shadows (inset). Using FastAtlas (c), we are able to obtain high-quality foveated rendering suitable for VR and AR displays with eye tracking.}
\label{fig:foveated_rendering}
\end{figure*}


\label{sec:applications}
A key advantage of texture-space shading is that shading rate can be decoupled from screen space resolution. This decoupling means that one can generate multiple shading atlases for the same scene and use them to shade different content at different shading resolutions. In particular, one can shade a portion of a scene using a high-resolution atlas, and another portion using a low resolution one. In these scenarios, only a portion of the high-resolution atlas is shaded, significantly reducing computational costs.
Applications that can benefit from this technique are ones where only a portion of a scene requires high quality shading, whereas the rest of the 
scene can be rendered with low resolution or blurry shading. We evaluate the applicability of our method to such scenarios by applying it to two representative examples: depth of field and foveated rendering. 
 
\paragraph*{Depth-of-Field.} 
We modify the depth-of-field method of Bukowski et al. \shortcite{Bukowski2013DepthOfField} to utilize two texture atlases. For the first atlas, we set the target scale to 100\% and use an $8K \times 8K$ atlas that provided enough space to fit all charts at this scale. Our second atlas is allocated to be  $1K \times 1K$ (1/64th the number of original texels). We generate a single packing by computing charts and atlases for the entire visible scene for the $8k$ atlas. 
The packing for the low resolution atlas is generated by simply scaling the coordinates of all charts by factor 0.125 along both axes. When packing the 8K atlas we allocate a gutter size sufficient to ensure that charts continue to not touch if the $8k$ packing is scaled down to fit the $1k$ atlas. We then use this packing to shade both atlases. We shade the entire low-resolution atlas, and only shade a triangle in the $8k$ atlas if it is contained or partly contained within the focal field. We then read from the high-resolution atlas when rasterizing areas in the focal field or its transition region, and use the low resolution one otherwise. The net result is that any shading values outside of the focal field are shaded at a reduced one sixty-fourth shading rate. As  Fig.~\ref{fig:depthOfField}c demonstrates, using our atlases achieves compelling depth-of-field effects while requiring one sixty-fourth the amount of shading for textures outside of the focal field.
We used the exact same setup for \cite{Neff2022MSA} using 8K and 1K atlases. See accompanying video and Fig.~\ref{fig:depthOfField} for a side by side comparison.  

\paragraph*{Foveated Rendering.} We use the same technique for foveated rendering. We generate an $8K$ atlas packing, and then scale it down to create an identical $2K$ atlas packing. We then shade a low-resolution $2K$ atlas for the entire scene, and a matching high-resolution $8K$ atlas with the same packing only for those triangles contained inside the fovea region. The low resolution view is then blurred using a two-pass Gaussian blur and composited with a high-resolution rendering of the foveal region to produce the final image (Fig. \ref{fig:foveated_rendering}c). By providing a fast, robust texture-space shading method, our seamless texture atlases make foveated rendering practical for modern virtual reality systems with eye tracking. We used the exact same setup for \cite{Neff2022MSA} (Fig.~\ref{fig:foveated_rendering}b).

The quality of renders generated using a mix of high and low resolution atlases is clearly contingent on the quality of the shading generated using the low-resolution atlases. Patney et al.~\shortcite{patney2016towards} note that static TSS methods do not provide sufficient quality low resolution results to enable this approach for foveated rendering.
This observation is aligned with our measurements (Tab.~\ref{tab:supp_flip_fixed_atlas}, Tab.~\ref{tab:supp_flip_fixed_sr}) and figures which show that these methods fail to produce adequate quality shading at low resolutions. Shading using the method of \cite{Neff2022MSA} at these resolutions produces undesirable undersampling and highly noticeable visible seams at meshlet boundaries (Figs. \ref{fig:depthOfField}b and \ref{fig:foveated_rendering}b.)  FastAtlas generates atlases of sufficient quality at low resolutions for the needs of these applications.
