\section{Expert ensemble}
The scene, interaction, and behavior expert are designed for different types of anomalies, as identified in Figure~\ref{fig:anomaly-patterns}. An efficient ensemble method is required to build a robust AD system as a driving scenario can be accompanied with any kinds of anomalies. In this work, we adopt a result-level fusion by combining the anomaly scores from all the experts to generate a final anomaly score at time $t$:
\begin{equation}
s_t = e( s_t^\text{ffp}, s_t^\text{str}, s_t^\text{int}, s_t^\text{beh} ),
\end{equation}
where $e$ is the ensemble function. We now present two necessary steps towards the final score during evaluation.

\subsection{Expert score normalization}
The anomaly scores produced by different experts can differ in orders of magnitude due to the completely different mechanisms of score generation. To balance the effect of each score on the final anomaly score, we perform a score normalization step for each expert during evaluation:
\begin{equation}
\label{eq:score-normalization}
\bar{s}_t^{\,\text{exp}} = \frac{s_t^\text{exp} - \mu}{\sigma},
\end{equation}
where $\bar{s}_t^{\,\text{exp}}$ is the normalized expert score, $s_t^\text{exp} \in \{s_t^\text{ffp}, s_t^\text{str}, s_t^\text{int}, s_t^\text{beh}\}$, and $\mu$ and $\sigma$ are the mean and standard deviation of the expert score on the \textit{training data} without any anomalies, respectively. The normalization~(\ref{eq:score-normalization}) differs slightly from the z-score normalization in that $\mu$ and $\sigma$ are computed on the training dataset while $s_t^{\text{exp}}$ comes from a \textit{mutually different} evaluation dataset, which includes both normal and anomalous data. As a result, only $\bar{s}_t^\text{\,exp}$'s obtained from the normal data are expected to have a mean around $0$ and a standard deviation around $1$, while those obtained from the anomalous scenarios are not.

The amount of training data may be limited. Therefore, to estimate a more realistic mean and standard deviation in~(\ref{eq:score-normalization}), we employ a kernel density estimation~\citep{wkeglarczyk2018kernel} to fit the probability density function (pdf) for the normal expert scores. We use a Gaussian kernel and apply the transformation trick~\citep{shalizi2013advanced} to make sure that the estimated pdfs have support on $[0, +\infty]$, as all the expert scores are guaranteed to be positive. The mean and the standard deviation are then computed from the numerical integration on the estimated pdf. We note that only the information derived from the training data is required and thus the computation can be done before model deployment.

In the meantime, we set a threshold for AD for each expert. The threshold is determined by $\tau^\text{exp} = \mathbf{U} (\alpha)$, which is the upper $\alpha$-quantile of the estimated pdf and $\alpha$ is the confidence level, similar to~\citep{feng2022unsupervised}. When evaluating each expert \textit{individually}, we declare an anomaly whenever the unnormalized expert score $s_t^\text{exp}$ is larger than the threshold $\tau^\text{exp}$.

\subsection{Kalman filter based score fusion}
After the score normalization, all the expert scores are brought to a similar range and are ready for the final fusion. However, although informative, the outputs from the experts can be noisy in terms of conveying how likely a type of anomaly is present. In this work, we view each normalized expert score $\bar{s}_t^{\,\text{exp}}$ as a \textit{noise-corrupted} observation of the system state $x_t^\text{exp}$, which reflects the ground truth likelihood of encountering a specific type of anomaly. As mentioned in Section~\ref{sec:overview}, it is possible that each point in time contains more than one anomaly. Therefore, we define the final anomaly score $s_t$ as the addition of the underlying system states for each type of anomaly. Our state vector then becomes $\mathbf{x}_t = [x_t^\text{ffp}, x_t^\text{str}, x_t^\text{int}, x_t^\text{beh}, s_t]^\top \in \mathbb{R}^5$, and the system model has the form of:
\begin{equation}
\label{eq:system-dynamics}
\begin{aligned}
\mathbf{x}_{t+1}
&=
Ax_t + \mathbf{w}_t \\
\mathbf{y}_t
&=
H \mathbf{x}_t + \mathbf{v}_t,
\end{aligned}
\end{equation}
where the observation $\mathbf{y}_t = [\bar{s}_t^\text{\,ffp}, \bar{s}_t^\text{\,str}, \bar{s}_t^\text{\,int}, \bar{s}_t^\text{\,beh}]^\top$ and the state-transition matrix $A$ and the observation matrix $H$ are set respectively as:
\begin{equation}
A
=
\begin{bmatrix}
1 & 0 & 0 & 0 & 0 \\
0 & 1 & 0 & 0 & 0 \\
0 & 0 & 1 & 0 & 0 \\
0 & 0 & 0 & 1 & 0 \\
\frac{1}{4} & \frac{1}{4} & \frac{1}{4} & \frac{1}{4} & 0
\end{bmatrix}, \;
H
=
\begin{bmatrix}
1 & 0 & 0 & 0 & 0 \\
0 & 1 & 0 & 0 & 0 \\
0 & 0 & 1 & 0 & 0 \\
0 & 0 & 0 & 1 & 0
\end{bmatrix}.
\end{equation}
The process noise $\mathbf{w}_t$ and the observation noise $\mathbf{v}_t$ are drawn from two zero mean normal distributions with covariance $Q$ and $R$, respectively. Although the coefficients for the weighted sum in the last row of $A$ can be different, we use identical values for simplicity.

To estimate the final anomaly score, we employ Kalman filter, which has been widely used for estimating the internal state of a system in various application domains~\citep{bewley2016simple,wojke2017simple,sun2021idol}. The state dynamics of our Kalman filter follows the standard framework, such as the one in~\cite{patel2013moving}, and is omitted here for brevity. We set the covariance of the observation noise $R$ to be an identity matrix as a result of the expert score normalization. Using $1$-based indexing, the initial a posteriori estimate covariance matrix $P_{1|1}$ in Kalman filter and the covariance of the process noise $Q$ are both set to be a diagonal matrix filled with $0.1$ for simplicity. The initial guess of the state vector is:
\begin{equation}
\label{eq:kf-initialization}
\begin{aligned}
\hat{\mathbf{x}}_1
&=
[\bar{s}_1^\text{\,ffp}, \bar{s}_1^\text{\,str}, \bar{s}_1^\text{\,int}, \bar{s}_1^\text{\,beh}, s_1]^\top, \\
s_1
&=
0.25(\bar{s}_1^\text{\,ffp} + \bar{s}_1^\text{\,str} + \bar{s}_1^\text{\,int} + \bar{s}_1^\text{\,beh})
\end{aligned}
\end{equation}
The final anomaly score at time $t$ is retrieved as the last element of the state estimate $\hat{\mathbf{x}}_t$ from Kalman filter.

To obtain a threshold for AD for the ensemble, we normalize each $\tau^\text{exp}$ similar to~(\ref{eq:score-normalization}) and perform a weighted sum using the same weights as those in the score fusion: $\tau = 0.25(\bar{\tau}^\text{ffp} + \bar{\tau}^\text{str} + \bar{\tau}^\text{int} + \bar{\tau}^\text{beh})$.