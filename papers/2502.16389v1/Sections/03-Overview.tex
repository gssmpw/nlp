\section{Problem overview}
\label{sec:overview}

Our goal is to develop an AD module that enables an autonomous car to detect anomalous events online in diverse driving scenarios.

\begin{figure*}[t]
  \centering
  \begin{subfigure}[b]{0.35\linewidth}
    \captionsetup{justification=centering}
    \includegraphics[width=\linewidth]{Figures/fig2a_anomaly_patterns.pdf}
    \caption{Common anomaly patterns in driving.}
    \label{fig:anomaly-patterns}
  \end{subfigure}
  \begin{subfigure}[b]{0.64\linewidth}
    \captionsetup{justification=centering}
    \includegraphics[width=\linewidth]{Figures/fig2b_Xen_overview.pdf}
    \caption{System structure overview.}
    \label{fig:Xen-overview}
  \end{subfigure}
  \caption{\textit{Left:} Nodes are possible actors in an on-road anomaly. Each edge represents a type of anomaly happening between the two connected nodes. The edges are grouped by colors into three categories to guide our anomaly detector design. \textit{Right:} Three experts are proposed for detecting different types of anomalies based on the anomaly pattern analysis. Individual expert scores are fused by a Kalman filter to generate a comprehensive final score.}
  \label{fig:problem-overview}
\end{figure*}

To guide our anomaly detector design, an analysis of common anomaly patterns in driving scenarios is presented in Figure~\ref{fig:anomaly-patterns}. In most of the anomalous events on road, there exist three main players, namely the ego car, other road participants (e.g., cars, motorcycles, pedestrians), and the environment (e.g., guardrails, traffic signs, slippery roads), which are represented as the graph nodes. Any two of the nodes can produce a type of anomaly: the edge between the ego car and other road participants includes abnormal cases such as a collision between the ego car and a pedestrian; the edge between the ego car and the environment encompasses accidents such as the ego car being out of control due to slippery roads after raining; the edge connecting other road participants and the environment stands for \textit{non-ego involved individual anomalies} where a \textit{single} agent behaves abnormally, such as a car colliding with a guardrail. In particular, there is a self-loop around the node of other road participants, which represents \textit{non-ego involved interactive anomalies} where anomalous \textit{interactions} happen between other agents, such as two vehicles colliding with each other. We further summarize the two edges with one of the ends being the ego car as \textit{ego involved anomalies}.

Note that the presented graph is general enough to encompass rare moments with more than one anomaly present by activating multiple edges simultaneously. For example, in an event where two cars collide with each other and another car swerves due to loss of control, both \textit{non-ego involved interactive anomalies} and \textit{non-ego involved individual anomalies} edge will be activated. Moreover, two \textit{non-ego involved individual anomalies} edges will be activated if two non-ego cars lose control \textit{independently} in the same time. The AD problem can now be converted into activating relevant graph edges properly and promptly in abnormal events. Note that our AD problem formulation is not limited to automated cars and can be generalized to other robotic applications with proper modifications.

We approach the AD task in autonomous driving using unsupervised learning to avoid the difficulty and cost of collecting large-scale labeled anomaly data for training. Different normal patterns in non-anomalous driving videos have been modeled in prior works to detect out-of-distribution samples during test time for AD. However, learning unimodal normal patterns can only capture a subset of the edges in Figure~\ref{fig:anomaly-patterns}. For example, frame prediction framework is good at detecting \textit{ego involved anomalies} as large unexpected changes in pixel values are often observed in such events but can miss some of the non-ego involved anomalies due to the small occupancy of anomalous objects in the scene~\citep{liu2018future,hasan2016learning}. Object trajectory prediction approach can capture \textit{non-ego involved individual anomalies} well but can miss \textit{non-ego involved interactive anomalies} when the behaviors of both anomaly participants appear smooth individually~\citep{yao2019unsupervised,yao2022dota}. To cover the entire graph in Figure~\ref{fig:anomaly-patterns}, an ensemble of different experts for different edges are necessary.

Formally, we assume that the observations available at time $t$, $\mathbf{o}_t$, consist of the current and past RGB images $\mathbf{o}_t \coloneqq \left( I_1, I_2, \dots, I_t \right)$ with $I_i \in \mathbb{R}^{H \times W \times 3}$ from a forward facing monocular camera. At each time step, the task for the AD module, denoted as the function $g$, is to regress an \textit{anomaly score} $s_t \in \mathbb{R}$ from the current available observations $\mathbf{o}_t$: $s_t = g(\mathbf{o}_t)$. A higher anomaly score indicates a higher probability of an anomaly happening. To ensure the generalization performance of our anomaly detector, we make no assumptions on the consistency of camera intrinsics nor camera poses across videos.

The system structure of Xen is shown in Figure~\ref{fig:Xen-overview}. Compared to the existing AD methods for robotic systems, Xen is able to build upon the holistic analysis of anomaly patterns to design different experts for different types of anomalous events. Specifically, Xen consists of:
\begin{enumerate*}[label=(\arabic*)]
\item
a \textit{scene expert} modeling normal scenes and scene motions at frame level for detecting \textit{ego involved anomalies},
\item
an \textit{interaction expert} modeling normal relative motions between two objects for detecting \textit{non-ego involved interactive anomalies}, and
\item
a \textit{behavior expert} modeling normal object trajectories for detecting \textit{non-ego involved individual anomalies}.
\end{enumerate*}
An ensemble function $e$, which is realized by a Kalman filter, is then applied to fuse the scores from all the experts for enhanced performance:
\begin{equation*}
s_t = e(g_\text{s}(\mathbf{o}_t), \, g_\text{i}(\mathbf{o}_t), \, g_\text{b}(\mathbf{o}_t)),
\end{equation*}
where $g_\text{s}$, $g_\text{i}$, and $g_\text{b}$ are AD experts for modeling normal scenes, interactions, and behaviors, respectively. Such a multi-expert design can also enable the \textit{classification} of an anomaly by comparing the scores from the experts, even though no labels were used during training. Lastly, we modify the common evaluation protocol for AD tasks by removing video-wise anomaly score normalization so that the model performance in the experiments is more realistic as a real-world system (Section~\ref{sec:experiments}).

In the following sections, we describe the design of the three experts $g_\text{s}$, $g_\text{i}$, and $g_\text{b}$ for different anomaly types and the ensemble function $e$ for the improved overall performance than each individual component.