\section{Conclusion}
We present an ensemble method for anomaly detection in autonomous driving using a single monocular camera. We provide a holistic analysis on common anomaly patterns in driving scenarios, which is then used to design experts or modules tuned to detect different types of anomalies. The scene expert aims to capture frame-level abnormal events, which are often accompanied with ego involved anomalies; the interaction expert models normal relative motions between two road participants and raises an alarm whenever anomalous interactions emerge in non-ego involved interactive anomalies; and the behavior expert detects non-ego involved individual anomalies by monitoring inconsistent predictions of future locations for each object. A Kalman filter is developed for expert ensemble, in which the observations are normalized expert scores and the final anomaly score is modeled as one of the states. Our experimental results, with a novel evaluation protocol to reflect realistic model performance, show that the proposed method outperforms various baselines with higher AUC and F1-score in diverse driving scenarios in DoTA. Furthermore, we demonstrate that the multi-expert design can also enable the classification of an on-road anomaly using unsupervised learning. With several possible directions to explore in the future, we hope that this work provides insights on approaching anomaly detection in related areas through anomaly pattern analysis and can serve as a competitive solution to 2D on-road anomaly detection for autonomous driving.