\section{Introduction}
Deep neural networks achieve remarkable success in a wide range of tasks. 
However, this success is challenged by their vulnerability to privacy leakage~\citep{ref_23,ref_24,ref_25}, a critical concern in the era of data-driven algorithms. 
Privacy leakage refers to the unintended disclosure of sensitive information of the training set. 
Several kinds of attacks demonstrate this vulnerability: 
membership inference attacks%determine whether a specific data point is included in the network's training set
~\citep{ref_26, ref_20, ref_28,ref_13}, 
reconstruction attacks% recreate data points from the network's outputs
~\citep{ref_29,ref_30,ref_31}, and
model extraction attacks% reverse engineer a network's architecture or parameters to derive information about the training set
~\citep{ref83,ref84,ref85}. 
Some of these attacks can succeed even in the most challenging scenario, where the attacker has only label-only access to the network~\citep{ref_12,ref_13,ref_14}.    
This label-only access setting is common 
in machine-learning-as-a-service platforms (e.g., Google’s Vertex AI~\citep{ref_103} or Amazon’s ML on AWS~\citep{ref_104}). 
Thus, it is essential to develop privacy-preserving mechanisms.


There are different approaches for mitigating privacy leakage, including privacy-aware searches for a network architecture~\cite{ref_16,ref_17}, integration of regularization terms during training to prevent overfitting~\cite{ref_18, ref_19}, %, thus reducing privacy leakage. 
or federated learning over local datasets~\cite{ref_34,ref_57}. However, these approaches typically do not have a formal privacy guarantee. 
One of the most popular formal guarantees for privacy is 
differential privacy (DP)~\cite{Dwork06}. 
An algorithm taking as input a dataset is DP if, for any dataset,
 the presence or absence of a single data point does not significantly change its output. 
Several training algorithms add noise to guarantee DP~\cite{ref_22,ref_36,ref_37,ref_58,ref_59,ref_60,ref_61,ref_62}.  
Many of them rely on \emph{DP-SGD}~\citep{ref_22,ref_36,ref_37,ref_58}, where noise is added during training to the parameters' gradients.  
%\citet{ref_36,ref_37} developed DP-SGD approaches defending the case where the attacker has an access only to the classification labels. They designed methods that introduce noise only to the outputs of the classifier, rather than to the gradients or parameters directly. This modification aims to provide classifiers with a comparable level of privacy to standard DP-SGD techniques but with potentially higher accuracy.
However, %Although these algorithms provide DP guarantees, 
they have several disadvantages.
Their main disadvantage is that to satisfy DP, they add noise calibrated to protect the privacy of \emph{any} training set. To this end, they add noise protecting the \emph{maximal} privacy leakage, which tends to be high and thus decreases the resulting network's accuracy. 
In many scenarios, network designers care about protecting the privacy of a \emph{given} dataset.
For example, a network designer wishing to protect the privacy of a patients' dataset may not care that the training algorithm protects the privacy of other datasets (e.g., image datasets). 
However, even if they do not care about these datasets, ensuring DP forces the training algorithm
to add significantly higher noise and, consequently, the accuracy of the network that the designer cares about can be substantially reduced. 
%Beyond this disadvantage, DP-SGD have other disadvantages. 
Another disadvantage is that DP-SGD algorithms add noise to a large number of computations, thus %, which leads to decreasing the algorithm's accuracy. 
the amount of added noise is often higher than necessary.  %resulting in a higher than necessary accuracy decrease. 
This is because the noise %is 
%drawn from a distribution depending on privacy hyper-parameters ($\epsilon,\delta$). These hyper-parameters are used to define the algorithm's (i.e., trained network) 
is a function of the user-defined allowed privacy leakage and the algorithm's actual privacy leakage which is commonly overapproximated (e.g., using composition theorems~\cite{ref_62,ref_63,ref_64}).
%Since computing the exact maximal privacy leakage is challenging, it is commonly over-approximated (e.g., using composition theorems \Dana{add cites}),  which consequently results in unnecessary decrease to the algorithm's accuracy.
 %and provide the user a trade-off: the larger the added noise, the higher the privacy of individual training samples, but the lower the network's accuracy.
%Computing the exact maximal privacy leakage is challenging and thus it is commonly over-approximated (e.g., using composition theorems \Dana{add cites}). Because of the over-approximated bound, obtaining a given desired privacy leakage in practice leads to adding more noise than necessary, which consequently results in unnecessary decrease to the algorithm's accuracy.  %Further, DP is a probabilistic property which by design (2) their guarantees are inherently probabilistic, heavily dependent on the amount and type of noise introduced, and the specific training methodologies employed, (3) even after training the classifier with the DP-SGD technique, 
%Also, DP-SGD techniques add noise to the training computations, without differentiating inputs or computations that do not lead to privacy leakage, which consequently results in adding unnecessary noise and an unnecessary decrease in accuracy. %Third, 
%DP-SGD relies on %retraining the network and cannot provide guarantees to pre-trained networks. Thus, 
%specific training algorithm %it is hard to compose DP-SGD with other training techniques to leverage parallel advances in neural network training. 
Also, DP-SGD algorithms are specific algorithms and cannot easily be combined with other training techniques to leverage parallel advances in training. 
%the DP-SGD guarantees do not reveal the malicious attacker queries (classifier's inputs) that can cause privacy leakage. 
This raises the question: 
%\emph{can we provide DP guarantees for a pre-trained neural network with a minor 
%decrease in its accuracy?}
\emph{can we provide privacy guarantees for individuals in a given dataset used by a training algorithm while introducing a minor accuracy decrease to the label predictions of the resulting network?}
\begin{comment}
We propose label-only access to a network that guarantees \emph{Individual Differential Privacy} (\propi)~\citep{ref_88,ref_90} with a minor accuracy decrease.
\propi relaxes the DP requirement to a \emph{given} dataset, which enables to lower the added noise and thus reduce the accuracy decrease. 
\propi has similar properties to DP (e.g., parallel and sequential composition) and common DP noise mechanisms naturally extend to \propi. 
To the best of our knowledge, we are the first to enforce \propi for neural networks.
While we could consider other DP variants, \propi fits best to the above question and unlike other variants, it does not add or change the DP requirement (except for focusing on a given dataset).
For example, individualized privacy assignment~\cite{ref_94,ref_95} requires to assign a privacy budget to every individual in the training set, depending on how much the individual wishes to protect their privacy, which is not always easy to quantify. 
Local differential privacy requires to add randomization to the individuals' data before being released to form the training set~\cite{ref_96,ref_97}. Renyi differential privacy changes the DP requirement based on the Renyi divergence~\cite{ref_98,ref_99}. Homomorphic-encryption approaches require to encrypt the data~\cite{ref_100,ref_101}. 
%Compared to these variants, \emph{Individual Differential Privacy} (\propi)~\citep{ref_88,ref_90} is the closest variant to DP: it merely restricts the DP requirement to a \emph{given} dataset, which enables to lower the added noise and thus reduce the accuracy decrease. 
%\propi has similar properties to DP, including parallel and sequential composition~\citep{ref_88} and common DP noise mechanisms naturally extend to \propi. 
%However, to the best of our knowledge, \propi has not been enforced for neural networks.
\end{comment}

We propose label-only access to a network that guarantees \emph{Individual Differential Privacy} (\propi)~\citep{ref_88,ref_90} with a minor accuracy decrease.
\propi enforces the DP requirement only for individuals in a \emph{given} dataset.
Thanks to this relaxation, it enables the algorithm to introduce a lower noise and thus reduce the accuracy decrease. 
Although several other relaxations of DP have been proposed in order to reduce its accuracy decrease, they target specialized settings, which pose additional requirements. For example, individualized privacy assignment~\cite{ref_94,ref_95} requires to assign different privacy budgets to every individual in the training set, depending on how much the individual wishes to protect their privacy, which is not always given or easy to quantify. 
Local differential privacy~\cite{ref_60,ref_97} requires to add randomization to the individuals' data before being released to form the training set, which assumes that individuals' data is collected by a certain algorithm. Renyi differential privacy~\cite{ref_98,ref_99} changes the DP requirement based on the Renyi divergence. Homomorphic-encryption approaches~\cite{ref_100,ref_101} require to encrypt the data. In contrast, \propi assumes the exact setting
as DP, except that it limits the guarantee to the individuals in a given dataset. Additionally, \propi 
has similar properties to DP (e.g., parallel and sequential composition) and common DP noise mechanisms naturally extend to \propi. 
To the best of our knowledge, we are the first to enforce \propi for neural networks. 

 \begin{figure*}[t]
    \centering
  \includegraphics[width=1\linewidth, trim=0 280 0 0, clip,page=10]{images/figures.pdf}
    \caption{(a)~Illustration of the subspaces of inputs with confidence over or below the \propa on a 2D synthetic training set comprising 7,000 data points. (b)~An overview of \tool:
     given a dataset~$D$, a training algorithm~$\mathcal{T}$, and a classifier~$N$, it computes the \propa of every label (in this example, {\propa}$_{c_0}=0.05$, {\propa}$_{c_1}=0.01$) 
     and returns
     \propi label-only access to $N$. Given an input $x$, this access computes $N(x)$ to identify the predicted label $c$ and the confidence $\mathcal{C}$ in $c$.
      If $\mathcal{C}>$ {\propa}$_c$, it returns $c$; otherwise, it employs a noise mechanism. In this example, if a user submits to the \propi label-only access $x$ where
      $N(x) = (0.6,0.4)$, 
       then $c=c_0$ and $\mathcal{C}=0.6-0.4>0.05$ and thus $c_0$ is returned. If $N(x) = (0.52,0.48)$, then $c=c_0$ and $\mathcal{C}=0.52-0.48\leq 0.05$ and thus a noise mechanism is used to select a class $\widetilde{c}\in\{c_0,c_1\}$ given a probability distribution $(P_0,P_1)$.} %\tool returns the label with the maximal score ($c_1$).}
    \label{fig::intro}
\end{figure*}

%We propose label-only access to a network that guarantees \emph{Individual Differential Privacy} (\propi) and introduces a minor accuracy decrease.
%\propi enforces the DP requirement only for individuals in a given dataset.
%Thanks to this relaxation, it enables the algorithm to introduce a lower noise and thus reduce the accuracy decrease. 
%Although several other relaxations of DP have been proposed, aiming to reduce its accuracy decrease. These approaches assume specific requirements over the original DP property settings, e.g., data encryption in distributed data collection settings~\cite{ref_96,ref_97,ref_101}, different budgets of privacy protection based on different users' requests~\cite{ref_94,ref_95}, or local guarantees that do not generalize to unseen inputs~\cite{ref_8}. In contrast, the \propi property adheres to the original DP property settings but limits the guarantee only to the individuals in the actual training set. 


\sloppy
%We propose label-only access to a network that guarantees \propi and introduces a minor accuracy decrease.
%\propi enforces the DP requirement only for individuals in a given dataset.
%Thanks to this relaxation, it enables the algorithm to introduce a lower noise and thus reduce the accuracy decrease. 
%Although several other relaxations of DP have been proposed, aiming to reduce its accuracy decrease. These approaches assume specific requirements over the original DP property settings, e.g., data encryption in distributed data collection settings~\cite{ref_96,ref_97,ref_101}, different budgets of privacy protection based on different users' requests~\cite{ref_94,ref_95}, or local guarantees that do not generalize to unseen inputs~\cite{ref_8}. In contrast, the \propi property adheres to the original DP property settings but limits the guarantee only to the individuals in the actual training set. 
To naively obtain \propi label-only access to a network, we could add
%While we could naively guarantee \propi by adding 
noise that depends on the maximum privacy leakage of any possible input to the network. %any individual in the given dataset, 
However, this would significantly decrease the accuracy.
  %An algorithm satisfies \propi if, given a dataset, the presence or absence of a single data point does not significantly change the algorithm's output. 
%The \propi property provides a utility-preserving differential privacy framework that relaxes the strict requirements of differential privacy. While differential privacy calibrates added noise to any possible dataset, \propi focuses only on the differential privacy of individuals within the actual dataset. This ensures privacy guarantees for individuals similar to those obtained by differential privacy (but does not extend to groups or unseen data points). Thanks to this relaxation, \propi requires adding a smaller amount of noise compared to DP, thereby reducing the decrease in the resulted network's accuracy.
%We note that safeguarding the privacy of individuals in neural networks is practical mainly due to the focus of privacy attacks on individuals within the training set, such as membership inference or reconstruction of training data points.
 %to the classifier's accuracy. 
Instead, we observe that 
%We employ the \propi property to ensure individual privacy in our classifier settings with label-only access, allowing users to query the classifier by sending inputs and observing the classification results. We observe that, in our setting 
not all inputs to the network lead to an \propi privacy leakage. By identifying these inputs, we can avoid adding noise to their predicted labels and thus avoid unnecessary decrease in accuracy. However, determining whether an input leads to an \propi privacy leakage is challenging since it requires to obtain its classification as determined by a large number of networks: the network trained on the full training set and every network trained on the full training set except for a unique data point. Instead, we propose to overapproximate the set of privacy leaking inputs with the \emph{\propl} (\propa). 
The \propa is the maximal classification confidence of inputs that violate \propi. \emph{Any} input that the network classifies with confidence over this bound does not lead to an \propi violation. 
Namely, such inputs are classified the same regardless if the classifier is trained on the full training set or if any data point is omitted. Given this bound, we can design \propi label-only access to the network by adding noise only to the predicted labels of inputs that the network classifies with confidence smaller or equal to the \propa. 
%Technically, our repair runs an input through the classifier, checks the classification confidence and adds noise if needed
%Given the \propa, our repair is computationally light: given an input, it runs the input through the network and adds noise if the classification confidence is at most the \propa. It is also general to any noise function and is applicable to pre-trained networks, without additional retraining. 
\Cref{fig::intro}(a) visualizes the \propa.
In this example, we consider a binary network classifier trained on a 2D synthetic training set comprising 7,000 data points. 
The figure shows the decision boundary of the classifier along with the decision boundaries of every 
classifier trained identically except that its training set excludes a unique data point (the colored lines in~\Cref{fig::intro}(a)). 
Label-only access to this classifier is \propi
%At high-level, a label-only access to the classifier is \propi 
if all classifiers classify every input the same. 
However, in practice, these classifiers are different. Namely, their decision boundaries are close but not identical. Because the decision boundaries are close, there is a large subspace of inputs that are classified the same by all classifiers (shown by the light blue background). These inputs' classification confidence is higher than the \propa (shown by the dashed green lines), and for them, \propi holds \emph{without added noise}. %We define this space using the minimal confidence bound (shown by the dashed green lines).



Building on this idea, we design
%label-only for neural network classifiers to satisfy \propi with a small accuracy decrease.
%Our system, called 
\tool (for \textbf{L}abel-G\textbf{U}arded \textbf{C}lassifier by \textbf{I}ndividual \textbf{D}ifferential Privacy), a system that computes the \propa of every label and then returns \propi label-only access that wraps a trained network. % and provides users a label-only access. 
Given an input, \tool's label-only access passes the input through the network. If the classification confidence is over the \propa, it returns the predicted label. If not, it selects a label to return using a noise mechanism.  %passes the output through a noise function and returns the noised label.
\Cref{fig::intro}(b) visualizes \tool. 
The main benefits of our \propi label-only access are: (1)~it is applicable to any training algorithm, 
%pre-trained networks, 
(2)~it identifies inputs that cannot lead to privacy leakage and does not add noise to their predicted labels, which leads to a minor decrease in the network's accuracy,
(3)~its noise mechanism can be easily configured with the desired level of privacy guarantee, providing a privacy-accuracy trade-off,
(4)~its privacy guarantee is not tied to a specific privacy attack,  
 and (5)~it is fully automatic.

%We formalize the minimal confidence bound as a novel privacy property, called \emph{\propl} (\propa). Like differential privacy, our property is satisfied when the network's classification remains the same regardless if any entry is present in the training set or not. Unlike differential privacy, our property is deterministic and does not depend on added noise. As such, it characterizes the \emph{region of inputs} in the input space that do not lead to privacy leakage. 
%\Cref{fig::intro}(b) visualizes this region.
%In this example, we consider a binary network classifier trained on a 2D synthetic dataset comprising 7,000 samples. 
%In order for the classifier to be differentially private, it has to classify any input similarly to all other classifiers, each trained in the same way except that their training set excludes a unique data point. 
%However, in practice, these classifiers are different. Namely, their decision boundaries (the colored lines in~\Cref{fig::intro}(b)) are close but not identical. Because the decision boundaries are close, there is a large region of the input space whose inputs are classified the same by all classifiers (shown by the light blue background). This is the region where DP is satisfied \emph{without requiring added noise}. We define this space using the minimal confidence bound (shown by the dashed green lines).
%Inputs that do not satisfy \propa (i.e., their confidence is below the bound) can be exploited by the attacker to leak information about the training set. For these inputs, \tool adds noise.



%Our property extends the standard definition of differential privacy enabling a deterministic distinction between inputs that can cause differential privacy leakage and those that cannot. The main goal of identifying these inputs regions 
%is to actively address and repair privacy breaches and subsequently increase the classifier’s overall level of privacy maintenance (as described in~
 %
%Ideally, a network classifier is differentially private with respect to an individual training sample if it classifies any input within the input domain similarly to another classifier trained under the same training algorithm, but without that specific sample in its training data. 

%To illustrate this, consider~\Cref{fig::intro}(b). In this figure, we show the decision boundaries of a two-classes network classifier trained on a 2D synthetic dataset comprising 7,000 samples (represented by the black line). Additionally, we use the same training algorithm to train another 7,000 classifiers, each differing only by excluding a single training sample from its training data. The decision boundaries of these classifiers are presented by the colored lines. The light-blue region in the figure represents inputs where the differential privacy is maintained, while the light-red region represents inputs where differential privacy is violated. 

%In practice, precisely computing the two distinct input regions — one that maintains differential privacy and the other that violates it — is challenging, primarily due to the highly non-convex nature of decision boundaries in neural networks. 
%Our key idea is to compute an input region that captures all inputs where differential privacy is violated, although it may also include inputs that maintain differential privacy, and another input region at which all inputs maintain differential privacy.
%To that end, we propose the concept of deterministic differentially-private bound.
%We define this bound as a bound in the classification domain. 
%It originates from the decision boundaries of the classifier that has been trained with all the training set samples and extends outward until it encompasses all the classification differences with all other classifiers, each trained by excluding a single sample from their training data (as described in~\Cref{fig::intro}(b); the input area bounded by the green dashed line). 
%The inputs located outside of this bound maintain differential privacy. 
%When the bound is zero, it aligns precisely with the decision boundary of the classifier trained with the complete training dataset. 
%In this scenario, there are inputs beyond this boundary that are susceptible to differential privacy leakage. 
%In contrast, when the bound is exceptionally large, it covers all inputs, encompassing both those that leak differential privacy and those that do not. 

 %Similarly, our approach might overapproximate by including inputs that do not actually lead to privacy leakage within the radius. 

Computing the exact \propa is highly challenging for several reasons. 
First, it is defined over a very large number of classifiers (commonly, several thousand): the classifier trained with the full training set and every classifier trained on the full training set except for a unique data point.
Second, the \propa is a global property requiring to determine for \emph{any} input if it may be classified differently by any of these classifiers and thus violate \propi. Third, it requires to determine the \emph{maximal} classification confidence of any input that violates \propi. % which involves computing and comparing the decision boundaries of all classifiers.
%\Dana{removing the part below, too much emphasis on overapp and its a minor point}
%To reduce the high problem complexity, we observe that \tool can provide DP guarantees given any confidence which is greater or equal to the \propa's bound. This enables \tool to scale the analysis by computing a sound over-approximation to \propa.
 %Note that although the differential privacy bound, may also contain samples where differential privacy is actually maintained, our primary objective remains focused on addressing and repairing privacy violations. 
% This concept is similar to overapproximated verification, shown successful for scaling the analysis of neural network's local robustness~\cite{ref_49,ref_50}. However, unlike overapproximated analysis, which may provide a false-positive and determine that robust neighborhoods are non-robust, our over-approximated analysis may only result in a lower network's accuracy but nonetheless provides a sound repair, making the repaired network DP.
%Our primary goal is to compute the minimal bound, aiming to minimize the overapproximation of the input region associated with privacy violations. 

%Obtaining the minimal differentially-private bound poses significant challenges: (1) computing the classification outputs for any input within an infinite input domain, (2) generating these outputs for a large number of classifiers, which includes the classifier trained with all training inputs and an additional set of classifiers with the size of the training dataset (with each one omitting a different single training sample), and (3) searching within the outputs of all these classifiers to determine the minimal bound that covers all the inputs where the classifiers' decisions differ. This step necessitates comparing the decision boundaries of all classifiers to identify the inputs that could potentially lead to privacy leakage.

%We propose \tool %(\textbf{D}eterministic {D}ifferential-P\textbf{r}ivacy \textbf{G}uarantee \textbf{o}f \textbf{N}etworks),
% a tool specifically designed to compute the minimal differentially-private bound.  
%To compute an effective over-approximated bound for \propa, which is very close to the real \propa's bound,
To compute the tightest \propa, \tool relies on several formal verification techniques. 
First, it encodes the problem as a
mixed-integer linear program (MILP), defined over the network and every network trained on the same training set except for a unique data point. % for expressing the problem of computing the \propa's (minimal) bound given all classifiers. % , a technique that efficiently encodes the network classifier trained with all samples along with other classifiers trained on reduced sample sets. 
%It also allows formalizing the minimization problem over the differentially-private bound.   
Second, to scale the analysis over the immense number of networks, \tool abstracts a set of networks using a \emph{hyper-network}. 
It then restates the MILP to compute the \propa of the network and a hyper-network. %, abstracting a set of networks each trained without a unique data point. 
This enables \tool a simultaneous analysis of multiple, closely related networks. 
Third, to eliminate the abstraction's overapproximation error, \tool 
relies on a novel branch-and-bound (BaB) technique. 
Our BaB branches by refining a hyper-network into several hyper-networks, each
abstracting close networks that are identified by clustering. Our BaB bounds by identifying hyper-networks whose \propa can bound the \propa of other hyper-networks. To keep the number of analyzed hyper-networks minimal, \tool further orders the branching of hyper-networks to increase the chances of bounding as many hyper-networks as possible.
% the classifiers into subgroups based on their weights distance, thereby enabling a simultaneous analysis of multiple, closely related classifiers. 
Fourth, to prune the MILP's search space, \tool bounds the differences of matching neurons in the network and the hyper-network and adds them as linear constraints. %These dependencies are novel and arise from the similarity between classifiers, which are trained by the same algorithm and their training sets differ by a single data point. 
Fifth, neurons in the hyper-network whose differences are small are overapproximated by linear relaxation,
to reduce the MILP's complexity. While this step introduces an overapproximation error, it is small in practice and it is configurable. % employs a similarity encoding strategy, which reduces the number of the boolean variables of the MILP, which govern its complexity. 

%We instantiate \tool with two noise functions: Coin Toss and exponential.  
We evaluate \tool over four data-sensitive datasets (Cryptojacking, Twitter Spam Accounts, Adult Census, and Default of Credit Card Clients) and over fully-connected and convolutional classifiers. %We have implemented two privacy protection schemes based on our differential privacy bounds and 
We compare it to two DP-SGD techniques: the original one~\cite{ref_22}, providing a DP guarantee for all the classifier's parameters, and 
ALIBI~\cite{ref_58}, providing a DP guarantee for the classifier's output. 
Our results indicate that thanks to our formal verification analysis, \tool can provide a perfect \propi guarantee ($\varepsilon=0$) for label-only access with 1.4\% accuracy decrease on average. 
For more relaxed \propi guarantees (higher $\varepsilon$), \tool provides \propi label-only access with 1.2\% accuracy decrease on average. % and up to 1.5\%. 
In contrast, the DP baselines decrease the accuracy on average by 12.7\% to obtain an $\varepsilon$-DP guarantee, which in particular provides $\varepsilon$-\propi guarantee. 
%(except for $\varepsilon=0$, which is impossible for DP algorithms).

In summary, our main contributions are: (1)~formalizing \propi for label-only access to networks, enabling a minor accuracy decrease, (2)~defining the \propa that overapproximates inputs violating \propi, (3)~designing a system called \tool for computing the \propa, relying on several formal verification techniques to scale (constraint solving, hyper-networks, branch-and-bound, 
pruning by bounding the differences of matching neurons, and linear relaxation) (4)~an extensive evaluation showing that \tool provides ~\propi label-only access with less than 1.4\% accuracy decrease. 

 %and up to 45\%
% In some cases, the baselines provide the accuracy of a random classifier.
%c by other approaches require adding a substantial amount of random noise to achieve similar levels of privacy protection, resulting in classifiers with significantly degraded classification ability, akin to random classifiers. 
%Even for the most successful cases of the baselines, \tool obtains a higher accuracy, on average by 7.4\%. %Finally, our ablation study highlights the significance of each component of \tool.






