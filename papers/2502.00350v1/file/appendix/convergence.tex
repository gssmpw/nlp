Early Stop Convergence Mode
In most cases, our agent naturally converges when there are no remaining actions in ASQ. However, in scenarios where the action sequence is lengthy and requires multiple execution steps, we introduce an early stop convergence mode to optimize efficiency.

This mode is controlled by a BERT embedding model, which evaluates the similarity between consecutive observations at each step. Specifically, for two observations, \( O_t \) and \( O_{t+1} \), we compute their cosine similarity using their BERT embeddings:

$$
\cos \theta = \frac{\langle \text{BERT}(O_t), \text{BERT}(O_{t+1}) \rangle}{|\text{BERT}(O_t)| \cdot |\text{BERT}(O_{t+1})|}
$$

If the similarity score exceeds 0.97, the two observations are considered equivalent. 

To ensure stability in the decision-making process, we apply a sliding window mechanism over consecutive observations. Specifically, we require that the similarity condition holds for  $K = 15$ consecutive steps before triggering convergence:
$$
\sum_{i=t}^{t+K-1} \mathbbm{1} (\cos \theta_i > 0.97) = K
$$

Once this condition is met, the agent terminates execution and reaches a conclusion.
