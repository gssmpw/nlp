We follow the design principle of AutoCodeRover's search API while implementing a merged design with default \texttt{file\_path}. For example, in \texttt{search\_class} we have default a \texttt{file\_path} argument equal to None. In this scenario, we leverage LLM to decide whether it needs to add \texttt{file\_path} argument or not based on the given context. 
To guide the agent, we provide the docstrings of the search APIs as part of the system prompt. The detailed API definition and docstring are attached below. 

\begin{lstlisting}[language=Python]
def search_file_contents(
    self, file_name: str, directory_path: str | None = None
) -> str:
    """API to search the file skeleton
        If you want to see the structure of the file, including class and function signatures.
        Be sure to call search_class and search_method_in_class to get the detailed information.

    Args:
        file_name (str): The file name to search. Usage: search_file_contents("example.py"). Do not include the path, only the file name.
        directory_path (str): The directory path to search. Usage: search_file_contents("example.py", "path/to/directory")

    Returns:
       str: If file contents exceed 200 lines, we will return the file skeleton, a string that contains the file path and the file skeleton.
            Otherwise, we will return the file path and the file contents.
    """

def search_class(self, class_name: str, file_path: str = None) -> str:
    """API to search the class in the given repo.

    Args:
        class_name (str): The class name to search.
        file_path (str): The file path to search. If you could make sure the file path, please provide it to avoid ambiguity.
        Leave it as None if you are not sure about the file path.
        Usage: search_class("ModelChoiceField") or search_class("ModelChoiceField", "django/forms/models.py")

    Returns:
        str: The file path and the class content. If the content exceeds 100 lines, we will use class skeleton.
        If not found, return the error message. If multiple classes are found, return the disambiguation message.
        Please call search_method_in_class to get detailed information of the method after skeleton search.
        If the methods don't have docstrings, please make sure use search_method_in_class to get the method signature.
    """

def search_method_in_class(
        self, class_name: str, method_name: str, file_path: str = None
    ) -> str:
    """API to search the method of the class in the given repo.
    Don't try to use this API until you have already tried search_class to get the class info.

    Args:
        class_name (str): The class name to search.
        method_name (str): The method name within the class.
        file_path (str): The file path to search. If you could make sure the file path, please provide it to avoid ambiguity.
        Leave it as None if you are not sure about the file path.
        Usage: search_method_in_class("ModelChoiceField", "to_python") or search_method_in_class("ModelChoiceField", "to_python", "django/forms/models.py")

    Returns:
        str: The file path and the method code snippet. If not found, return the error message.
        If multiple methods are found, return the disambiguation message.
    """

def search_callable(self, query_name: str, file_path: str = None) -> str:
    """API to search the callable definition in the given repo.
    If you are not sure about the query type, please use this API. The query can be a function, class, method or global variable.

    Args:
        query_name (str): The query to search. The format should be only the name.
        file_path (str): The file path to search. If you could make sure the file path, please provide it to avoid ambiguity.
        Leave it as None if you are not sure about the file path.
        Usage: search_callable("ModelChoiceField") or search_callable("ModelChoiceField", "django/forms/models.py")

    Returns:
        str: The file path and the code snippet. If not found, return the error message.
        If multiple matches are found, return the disambiguation message.
    """

def search_source_code(self, file_path: str, source_code: str) -> str:
    """API to search the source code in the file. If you want to search the code snippet in the file.

    Args:
        file_path (str): The file path to search.
        source_code (str): The source code to search.

    Returns:
        str: The file path and the related function/class code snippet.
            If not found, return the error message.
    """


\end{lstlisting}