\section{Related Works}
Quantum circuits have emerged as a critical area of security research due to their importance in the era of quantum computing. Several studies have investigated the challenges and vulnerabilities associated with quantum circuits \cite{ghosh2023primer}. One significant area of research focuses on IP protection for quantum circuits, it mainly addresses concerns similar to those in classical integrated circuit (IC) design, such as IP theft \cite{aboy2022mapping}. Related work has also explored techniques such as quantum logic locking and quantum circuit obfuscation to protect quantum circuits against IP-related attacks \cite{das2023randomized, suresh2021short, topaloglu2023quantum}. These techniques aim to secure the integrity and confidentiality of quantum designs.
Other studies have investigated insertion-based attacks designed to disrupt the quantum computing process. For instance, prior work \cite{roy2024hardware} introduced the concept of Trojan insertion targeting quantum circuits and proposed a CNN-based detection method to identify these malicious modifications \cite{das2023trojannet}. Another study investigated the introduction of adversarial SWAP gates to amplify the computational burden during quantum compilation. \cite{upadhyay2024stealthy}, effectively degrading performance. These efforts have laid a crucial foundation for understanding insertion attacks and their significant impact on quantum circuits.
In this paper, we build upon prior research by introducing a novel approach to controlled Trojan insertion in quantum circuits. Unlike earlier methods involving single-qubit Trojans, our design enables the Trojans to remain dormant until triggered by specific control signal conditions, akin to the behavior of hardware Trojans in classical IC design. This conditional activation significantly improves concealment, making the Trojan harder to detect and more resilient against removal during optimization. For instance, simple gate-based Trojans, such as redundant X-gates, are often eliminated during the compilation process. In contrast, our approach embeds conditional logic directly into the circuit, ensuring the Trojan remains functional and hidden until its activation criteria are met.