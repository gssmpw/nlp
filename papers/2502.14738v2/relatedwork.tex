\section{Related Work}
Misclassification risk and uncertainty quantification for various types of classifiers have been very well studied in the literature \cite{adams1999comparing,pendharkar2017bayesian,hou2013modeling}. In \cite{sensoy2021misclassification}, the authors propose a risk-calibrated classifier to reduce the costs associated with misclassification errors, and empirically show the effectiveness of their algorithm in a  deep learning framework. In \cite{elkan2001foundations}, the authors study cost-sensitive learning for class balancing in order to improve the quality of predictions in decision tree learning methods. In our work, we consider a hypothesis testing (or classification) task in a Bayesian learning framework. 

A subset of the literature has addressed the problem of sequential information gathering within a limited budget \cite{hollinger2013sampling,chen2015sequential}. The authors of  \cite{golovin2010near} study data source selection for a monitoring application, where the sources are selected sequentially in order to estimate certain parameters of an environment. In \cite{ghasemi2019online}, the authors study sequential information gathering under a limited budget for a robotic navigation task. In \cite{gupta2006stochastic}, the authors study sequential sensor scheduling to jointly estimate a process and present a stochastic selection algorithm which is computationally tractable.  In contrast, we consider the scenario where the information set is selected \textit{a priori}, i.e., at \textit{design-time}, and propose an efficient algorithm with guarantees. 

A substantial body of work focuses on the study of submodularity (and/or weak submodularity) properties for efficient greedy techniques with provable guarantees for feature selection in sparse learning \cite{krause2010submodular,chepuri2014sparsity}, sensor selection for estimation \cite{mo2011sensor,hashemi2020randomized,shamaiah2010greedy,krause2007near} \& Kalman filtering \cite{ye2018complexity}, and observation selection for mixed-observable Markov decision processes \cite{bhargav2023complexity}.  Along the lines of these works, we leverage the weak submodularity property of the performance metric and present greedy algorithms with performance guarantees.

Robust sensor selection has been extensively studied in the context of resource-constrained environments where sensors may fail, be removed, or experience adversarial attacks \cite{ye2020complexity,laszka2015resilient,oh2023dynamic}. Early works, such as those by \cite{krause2008robust,kaya2025randomized,laszka2015resilient}, focused on optimizing submodular objectives to achieve near-optimal sensor placement under budget constraints, ensuring reliable performance despite uncertainty. In the robust setting, adversarial or stochastic failures were explored by \cite{tzoumas2018resilient}, where greedy algorithms were developed to guarantee near-optimal performance. The authors of \cite{kaya2025randomized} study robust weak-submodular optimization of a set of weak-submodular functions, where the goal is to maximize the utility of the worst-case objective. Along similar lines as these works, we study the robust information selection problem where the objective is not submodular, and present greedy algorithms with near-optimality guarantees. Furthermore, we present a submodular surrogate metric which enjoys improved performance guarantees for greedy approximation.
\newpage