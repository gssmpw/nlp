\begin{figure*}[t]
\centering
\includegraphics[width=1.0\linewidth]{fig/methodology8.png}
\caption{
\textbf{Overview of our E-3DGS Method.}
We use 3D Gaussians~\cite{3dgs} as the scene representation and assume that initial noisy camera poses are available. We randomly initialize the scene with our frustum-based initialization (Sec.~\ref{sec:frustum_init}) and then optimize the Gaussians and the camera poses jointly (Sec.~\ref{sec:pose_refinement}). To obtain a high-quality reconstruction of both, low-frequency structure and high-frequency detail, we propose a strategy using a large event window from $t_{s_1}$ to $t$ and a small one from $t_{s_2}$ to $t$ (Sec.~\ref{subsec:adaptive_window}). We then define the loss $\mathcal{L}_\mathrm{recon}$ (Sec.~\ref{ssec:Optimization}) between renderings from our model at the current time $t$ (indicated green) and previous times $t_{s_1}$ (indicated orange) and $t_{s_2}$ (indicated red), and the accumulated incoming events $E(t_{s_1},t)$ and $E(t_{s_2},t)$. We regularize the 3D Gaussians with the loss $\mathcal{L}_\mathrm{iso}$ (Sec.~\ref{ssec:IsotropicReg}). 
}
\label{fig:methodology}
\end{figure*}
