


% \todo{modify the table, make it not so similar to neomem (for example, learn from PoM)}

% \todo{modify the description and table, make it not so similar to neomem}

% We build HeteroBox emulation platform on a real CXL memory system, detailed in Table \ref{table:system_config}. 
% The system setup includes a single-socket Intel$^\circledR$ Sapphire-Rapids$^\texttt{TM}$ CPU and a CXL-enabled Intel$^\circledR$ Agilex$^\texttt{TM}$-7 I-Series FPGA acting as CXL memory (CXL 1.1, Type-3 device). 
% The FPGA has dual-channel DDR4-2666 memory with 16GB capacity. 
% The host CPU is equipped with 32GB $\times 2$ dual-channel DDR5-4800 memory.
% We implement HeteroMem based on Linux kernel v6.3. 
% The CXL memory on FPGA are exposed to the software as a CPU-less NUMA node.
% In order to compare the performance of HeteroMem with other baseline memory tiering system, we hack the numa nodes enumeration code and split the CXL memory NUMA node into two fake NUMA nodes, corresponding to the fast memory and slow memory emulated by HeteroBox.
% We use Intel$^\circledR$ FPGA CXL IP (Memory Expander Type 3) and add about 7000 lines of Verilog RTL custom logic code to build HeteroMem, coupled with about 1000 lines C++ simulation environment code and about 1000 lines modification in Linux kernel for HeteroMem driver and fake numa nodes logic.
%For different fast-slow memory ratios, we adjust host memory size by reserving a specific amount of physical memory within the Linux kernel~\cite{memmap}. The default fast-slow ratio is 1:2. We disable CPU's SMT, fix the CPU clock frequency and clear the page cache before running workloads to ensure consistent performance across trials.

% Our FPGA protype of HeteroMem consumes 234.7k ALMs (25.7\%) and 1913 BRAMs (M20K, 14.4\%), no DSPs. The layout of our HeteroMem FPAG prototype is shown in Figure~\ref{fig:implementation}-(a). The red part in the layout is our HeteroMem logic, acting as an abstract layer between Intel CXL hard IP and memory controller. Figure~\ref{fig:implementation}-(b) shows a photo of our implementation platform.

% We build the HeteroBox emulation platform on a real CXL memory system, detailed in Table \ref{table:system_config}. The system setup includes a single-socket Intel$^\circledR$ Sapphire-Rapids$^\texttt{TM}$ CPU and a CXL-enabled Intel$^\circledR$ Agilex$^\texttt{TM}$-7 I-Series FPGA acting as CXL memory (CXL 1.1, Type-3 device). The FPGA is equipped with dual-channel DDR4-2666 memory with a 16GB capacity, while the host CPU has 32GB $\times$ 2 dual-channel DDR5-4800 memory.

\begin{table}[t]
% \vspace{-0.2cm}
\caption{Evaluation System Configuration}
\vspace{-0.4cm}
\label{table:system_config}
\resizebox{0.485\textwidth}{!}{

\begin{tabular}{|l|l|}
\hline
Host CPU   & Single socket Intel$^\circledR$ Xeon 6430 CPU @ 2.10GHz  \\ 
                        \cline{2-2} 
                      & \begin{tabular}[c]{@{}l@{}}32 Cores, hyperthreading disabled. \\ 
                        60MB Shared LLC\end{tabular} \\ 
                        \hline
DDR Memory            & 32GB DDR5 4800MHz x 2         \\ 
                        \hline
CXL Memory & One Intel$^\circledR$ Agilex$^{\texttt{TM}}$ I-Series FPGA Dev Kit @400 MHz \\ 
                        \cline{2-2} 
                      & \begin{tabular}[c]{@{}l@{}}Hard CXL 1.1 IP on PCIe Gen5 x16 \\ 
                        16GB 2-Channel DDR4-2666 DRAM\end{tabular}      \\ 
                        \hline
Software Environment  & Linux kernel version v6.3 \\
                        \hline

\end{tabular}
}
\vspace{-0.4cm}
\end{table}
\begin{tikzpicture}[
	font=\footnotesize,
	node distance=0.3cm
	]
	\node[] (P) at (0,0)  {
		\textbf{Parameter:} Key-Gen()
	};
	\node[] (c) [below = 0cm of P] {
		\begin{minipage}{0.8\columnwidth}
			\begin{minted}[fontsize=\footnotesize,escapeinside=@@,autogobble]{coq}
Parameter KeyGen, (PubKey × SecKey).
			\end{minted}
		\end{minipage}
	};
	
	% Package frame and stuff
	\draw[] (c.north west) |- (P.north) -| (c.north east);
	\draw[] (c.north west) |- (P.south) -| (c.north east);
	\draw[] (c.north east) |- (c.south) -| (c.north west);
	
	%%%%%%%%%%%%%%%%%%%%%%%%%%%%%%%%%%%%%%%
	%%%%%%%%%%%%%%% Sign %%%%%%%%%%%%%%%%%%
	%%%%%%%%%%%%%%%%%%%%%%%%%%%%%%%%%%%%%%%
	
	\node[] (P1) [below = 0.7cm of P]  {
		\textbf{Parameter:} Sign(m,sk)
	};
	\node[] (c1) [below = 0cm of P1] {
		\begin{minipage}{0.8\columnwidth}
			\begin{minted}[fontsize=\footnotesize,escapeinside=@@,autogobble]{coq}
Parameter Sign : @$\forall$@ (sk : SecKey) (m : Message), 
  Signature.
			\end{minted}
		\end{minipage}
	};
	
	% Package frame and stuff
	\draw[] (c1.north west) |- (P1.north) -| (c1.north east);
	\draw[] (c1.north west) |- (P1.south) -| (c1.north east);
	\draw[] (c1.north east) |- (c1.south) -| (c1.north west);
	
	%%%%%%%%%%%%%%%%%%%%%%%%%%%%%%%%%%%%%%%
	%%%%%%%%%%%%%%% Verify %%%%%%%%%%%%%%%%
	%%%%%%%%%%%%%%%%%%%%%%%%%%%%%%%%%%%%%%%
	
	\node[] (P2) [below = 1cm of P1]  {
		\textbf{Parameter:} Verify(m,sig,pk)
	};
	\node[] (c2) [below = 0cm of P2] {
		\begin{minipage}{0.8\columnwidth}
			\begin{minted}[fontsize=\footnotesize,escapeinside=@@,autogobble]{coq}
Parameter Ver_sig : @$\forall$@ (pk : PubKey) 
  (sig : Signature) (m : Message), bool.
			\end{minted}
		\end{minipage}
	};
	
	% Package frame and stuff
	\draw[] (c2.north west) |- (P2.north) -| (c2.north east);
	\draw[] (c2.north west) |- (P2.south) -| (c2.north east);
	\draw[] (c2.north east) |- (c2.south) -| (c2.north west);
	
\end{tikzpicture}
\chapter{Implementation}{\label{ch:implementation}}
In this chapter, we present the implementation of the final product. We start by discussing how the four steps introduced in \hyperref[ch:high_level_approach]{chapter \ref*{ch:high_level_approach}} are integrated. We then outline the main system components of our score follower, presenting each as an independent, self-contained module. We then combine this into an overall system architecture and finally introduce the open-source score renderer used to display the score and evaluate the score follower.       

% \section{Aims and Requirements}
% The overall aim of the score follower was to 


\section{Score Follower Framework Details}
Our score follower conforms to the high-level framework presented in \hyperref[section:score_follower_framework]{section \ref*{section:score_follower_framework}}. In step 1, two score features are extracted from a MIDI file (see \hyperref[subsection:midi]{subsection \ref*{subsection:midi}}), namely MIDI note numbers\footnote{\href{https://inspiredacoustics.com/en/MIDI_note_numbers_and_center_frequencies}{https://inspiredacoustics.com/en/MIDI\_note\_numbers\_and\_center\_frequencies}} (corresponding to pitch) and note onsets (corresponding to duration). In step 2, the audio is streamed (whether from a file or into a microphone) and audioframes that exceed some predefined energy threshold are extracted. Here, audioframes are groups of contiguous audio samples, whose length can be specified by the argument \verb|frame_length|, usually between 800 and 2000 samples. The period between consecutive audioframes can also be defined by the argument \verb|hop_length|, typically between 2000 and 5000 audio samples. In step 3, score following is performed via a `Windowed' Viterbi algorithm (see  \hyperref[subsection:adjusting_viterbi]{subsection \ref*{subsection:adjusting_viterbi}}) which uses the Gaussian Process (GP) log marginal likelihoods (LMLs) for emission probabilities (see \hyperref[section:state_duration_model]{section \ref*{section:state_duration_model}}) and a state duration model for transition probabilities (see \hyperref[section:state_duration_model]{section \ref*{section:state_duration_model}}). Finally, in step 4 we render our results using an adapted version of the open source user interface, \textit{Flippy Qualitative Testbench}.

\section{Following Modes}
Two modes are available to the user: Pre-recorded Mode and Live Mode. The former requires a pre-recorded $\verb|.wav|$ file, whereas the latter takes an input stream of audio via the device's microphone. Note that both modes are still forms of score following, as opposed to score alignment, since in each mode we receive audioframes at the sampling rate, not all at once.\\

Live Mode offers a practical example of a score follower, displaying a score and position marker which a musician can read off while playing. However, this mode is not suitable for evaluation because the input and results cannot be easily replicated. Even ignoring repeatability, Live Mode is not suitable for one-off testing since a musician using this application may be influenced by the movement of the marker. For instance, the performer may speed up if the score follower `gets ahead' or slow down if the position marker lags or `gets lost'. To avoid this, we use Pre-recorded Mode when evaluating the performance of our score follower. Furthermore, Pre-recorded Mode offers the advantage of testing away from the music room, providing the opportunity to evaluate a variety of recordings available online. 

\section{System Architecture}
Our guiding principle for development was to build modular code in order to create a streamlined system where each component performs a specific task independently. This structure facilitates easy testing and debugging. \hyperref[fig:black_box]{Figure \ref*{fig:black_box}} presents a high-level architecture diagram, where each black box abstracts a key component of the score follower. When operating in Pre-recorded Mode, there is the option to stream the recording during run-time, which outputs to the device's speakers (as indicated by the dashed lines).

\begin{figure}[H]
    \centering
    \includegraphics[width=1\textwidth]{figs/Part_4_Implementation_And_Results/black_box.png}
    \caption{Abstracted system architecture diagram displaying inputs in grey, the 4 main components of the score follower in black and the outputs in green.}
    \label{fig:black_box}
\end{figure}

\subsection{Score Preprocessor}
The architecture for the Score Preprocesor is given in \hyperref[fig:score_preprocessor]{Figure \ref*{fig:score_preprocessor}}. First, MIDI note number and note onset times are extracted from each MIDI event. Simultaneous notes can be gathered into states and returned as a time-sorted list of lists called \verb|score|, where each element of the outer list is a list of simultaneous note onsets. Similarly, a list of note durations calculated as the time difference between consecutive states is returned as \verb|times_to_next|. Finally, all covariance matrices are precalculated and stored in a dictionary, where the key of the dictionary is determined by the notes present. This is because the distribution of notes and chords in a score is not random: notes tend to belong to a home \gls{key} and melodies tend to be repeated or related (similar to subject fields in speech processing). Therefore, states tend to be reused often, allowing us to achieve amortised time and space savings (by avoiding repeated calculation of the same covariance matrices). 

\begin{figure}[H]
    \centering
    \includegraphics[width=1\textwidth]{figs/Part_3_Implementation/Stage_2_Alignment/score_preprocessor.png}
    \caption{System architecture diagram representing the Score Preprocessor with inputs in grey, processes in blue and objects in yellow.}
    \label{fig:score_preprocessor}
\end{figure}


\subsection{Audio Preprocessor}
The architecture for the Audio Preprocessor is illustrated in \hyperref[fig:audio_preprocessor]{Figure \ref*{fig:audio_preprocessor}}. In Pre-recorded Mode, the Slicer receives a $\verb|.wav|$ file and returns audioframes separated by the \verb|hop_length|. These audioframes are periodically added to a multiprocessing queue, \verb|AudioFramesQueue|, to simulate real-time score following. In Live Mode, we use the python module \verb|sounddevice| to receive a stream of audio, using a periodic callback function to place audioframes on \verb|AudioFramesQueue|. 

\begin{figure}[H]
    \centering
    \includegraphics[width=1\textwidth]{figs/Part_4_Implementation_And_Results/audio_preprocessor.png}
    \caption{System architecture diagram representing the Audio Preprocessor with inputs in grey, processes in blue and objects in yellow.}
    \label{fig:audio_preprocessor}
\end{figure}

\subsection{Follower and Backend}
The joint Follower and Backend architecture diagram is shown in \hyperref[fig:follwer_and_backend]{Figure \ref*{fig:follwer_and_backend}}. The Viterbi Follower (detailed in \hyperref[subsection:adjusting_viterbi]{section \ref*{subsection:adjusting_viterbi}}) calculates the most probable state in the score, given audioframes continually taken from \verb|AudioFramesQueue|. These states are placed on another multiprocessing queue, the \verb|FollowerOutputQueue|, for the Backend to process and send. This prevents any bottle-necking occurring at the Follower stage. The Backend first sets up a UDP connection and then reads off values from \verb|FollowerOutputQueue|, sending them via UDP packets to the score renderer.

\begin{figure}[H]
    \centering
    \includegraphics[width=1\textwidth]{figs/Part_4_Implementation_And_Results/follower_and_backend.png}
    \caption{System architecture diagram representing the Follower and Backend processes with processes in blue, objects in yellow and outputs in green.}
    \label{fig:follwer_and_backend}
\end{figure}

\subsection{Player}
In Pre-recorded Mode, the Player sets up a new process and begins streaming the recording once the Follower process begins. This provides a baseline for testing purposes, as a trained musician can observe the score position marker and judge how well it matches the music. 

\subsection{Overall System Architecture}
The overall system architecture is presented in \hyperref[fig:overall_system_architecture]{Figure \ref*{fig:overall_system_architecture}}. Since the Follower runs a real-time, time sensitive process, parallelism is employed to reduce the total system latency. We use two \verb|multiprocessing| queues\footnote{\href{https://docs.python.org/3/library/multiprocessing.html}{https://docs.python.org/3/library/multiprocessing.html}} to avoid bottle-necking, which allows us to run 4 concurrent processes (Audio Preprocessor, Follower, Backend, and Audio Player). Hence, this architecture allows the components to run independently of one another to avoid blocking. Furthermore, this allows the system to take advantage of the multiple cores and high computational power offered by most modern machines.  

\begin{figure}[H]
    \centering
    \includegraphics[width=1\textwidth]{figs/Part_4_Implementation_And_Results/overall_score_follower_2.png}
    \caption{System architecture diagram representing the overall score follower running in Pre-recorded mode, with inputs in grey, processes in blue, objects in yellow and outputs in green.}
    \label{fig:overall_system_architecture}
\end{figure}


\section{Rendering Results}{\label{section:renderer}}
To visualise the results of our score follower, we adapted an open source tool for testing different score followers.\footnote{\href{https://github.com/flippy-fyp/flippy-qualitative-testbench/blob/main/README.md}{https://github.com/flippy-fyp/flippy-qualitative-testbench/blob/main/README.md}} \hyperref[fig:flippy_example]{Figure \ref*{fig:flippy_example}} shows the user interface of the score position renderer, where the green bar indicates score position. 

\begin{figure}[H]
    \centering
    \includegraphics{figs/Part_4_Implementation_And_Results/example_renderer.png}
    \caption{Screenshot of the score renderer user interface which displays a score (here we show a keyboard arrangement of \textit{O Haupt voll Blut und Wunden} by Bach). The green marker represents the score follower position.}
    \label{fig:flippy_example}
\end{figure}




We build our HeteroBox emulation platform on a real CXL system, as detailed in Table \ref{table:system_config}. The system setup includes a single-socket Intel$^\circledR$ Sapphire-Rapids$^\texttt{TM}$ CPU paired with a CXL-enabled Intel$^\circledR$ Agilex$^\texttt{TM}$-7 I-Series FPGA that acts as CXL memory (CXL 1.1, Type-3 device). The FPGA is equipped with dual-channel DDR4-2666 memory with a capacity of 16GB, while the host CPU is equipped with 32GB $\times$ 2 dual-channel DDR5-4800 memory.

We implement HeteroMem based on the Linux kernel v6.3. The CXL memory on the FPGA is exposed to the software as a CPU-less NUMA node. To compare the performance of HeteroMem with other baseline memory tiering systems, we modified the NUMA nodes enumeration code and split the CXL memory NUMA node into two fake NUMA nodes, corresponding to the fast memory and slow memory emulated by HeteroBox.
We use Intel$^\circledR$ FPGA CXL IP (Memory Expander Type 3) and add about 7000 lines of Verilog RTL custom logic code to build HeteroBox and HeteroMem, along with approximately 1000 lines of C++ simulation environment code and about 1000 lines of modifications in the Linux kernel for HeteroBox and HeteroMem driver and fake NUMA nodes logic. 
We will open-source all the hardware and software codes after publication.

% \begin{figure}[t]
%   \centering
%   \includegraphics[width=\columnwidth]{fig/mlc_bandwidth_latency.pdf}
%   \vspace{-0.5cm}
%   \caption{Software Profiling Overhead and Inefficient Data Movement}
%   \vspace{-0.6cm}
%   \label{fig:mlc_bandwidth_latency}
% \end{figure}

\begin{figure*} [t]
    \centering
    \includegraphics[width=0.85\linewidth]{fig/mlc_bandwidth_latency_v3.pdf}
    % \vspace{-1.2em}
    \vspace{-0.4cm}
    \caption{Validation of HeteroBox Emulation Platform}
     \label{fig:mlc_bandwidth_latency}
     \vspace{-0.5cm}
\end{figure*}

% Our FPGA prototype of HeteroMem consumes \textcolor{blue}{234.7k ALMs (25.7\%) and 1913 BRAMs (M20K, 14.4\%), with no DSPs.}\xp{remember to change if large cache is avialiable} The layout of our HeteroMem FPGA prototype is shown in Figure~\ref{fig:implementation}-(a). The red part in the layout represents our HeteroMem logic, acting as an abstraction layer between the Intel CXL hard IP and the memory controller. Figure~\ref{fig:implementation}-(b) shows a photo of our implementation platform.
% Our FPGA prototype of HeteroMem consumes 224.4k ALMs (24.6\%) and 3615 BRAMs (M20K, 27.2\%), with no DSPs. The layout of our HeteroMem FPGA prototype is shown in Figure~\ref{fig:implementation}-(a). The red part in the layout represents our HeteroMem logic, acting as an abstraction layer between the Intel CXL hard IP and the memory controller. 
% Figure~\ref{fig:implementation}-(b) shows a photo of our implementation platform. 
% Figure~\ref{fig:implementation}-(c) shows the overview of the system emulated by HeteroBox in the following evaluation. We configure the HeteroBox to emulate two memory regions in the CXL-extended memory space. We allocate the memory of applications in CXL-extended memory, while the OS memory resides in the host memory.
Our FPGA prototype of HeteroMem consumes 224.4k ALMs (24.6\%) and 3615 BRAMs (M20K, 27.2\%), with no usage of DSPs. The layout of our HeteroMem FPGA prototype is illustrated in Figure~\ref{fig:implementation}-(a), where the red part represents our HeteroMem logic. This logic acts as an abstraction layer between Intel CXL hard IP and memory controller. Figure~\ref{fig:implementation}-(b) provides a photo of our implementation platform. Finally, Figure~\ref{fig:implementation}-(c) presents an overview of the system emulated by HeteroBox for the subsequent evaluation. 
We configure HeteroBox to emulate two memory regions in CXL-extended memory space: one fast memory region and one slow memory region. Application memory is allocated in CXL-extended memory space, while OS memory remains in the host memory.




