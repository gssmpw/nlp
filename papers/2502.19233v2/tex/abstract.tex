% The Compute Express Link (CXL) interconnect introduces load-store semantics across a variety of memory types through byte-addressable SerDes links, facilitating a unified memory semantic. 
% This advancement illuminates the potential of alternative memory media beyond traditional DRAM, including Managed DRAM, ReRAM, and 3DXP/Optane, noted for their substantial capacity. However, these innovative memory solutions are hampered by their inherent high latency, significantly impeding performance.
% The Compute Express Link (CXL) interconnect allows.
% To mitigate these latency-induced performance degradations, we introduce HeterBox: a CPU-transparent architecture that synergizes high-capacity, latency-afflicted media with a minimal fraction of DRAM within the same CXL device. HeterBox proficiently intercepts memory requests from the CPU, seamlessly migrating them to the device's DRAM with negligible overhead, thus optimizing latency-sensitive operations.
% Our prototype of HeterBox, realized on a CXL-enabled FPGA platform, undergoes rigorous evaluation against a suite of real-world applications. The results are compelling, demonstrating a performance enhancement of XXX-XXX\% over configurations reliant solely on slower media. Furthermore, HeterBox outperforms existing state-of-the-art memory tiering solutions, marking improvements of XXX-XXX\%. These findings underscore HeterBox's efficacy in harmonizing the disparate characteristics of mixed memory systems, thereby unlocking new avenues for architectural innovation in the era of heterogenous computing.
% The emerging Compute Express Link (CXL) technique enables CPUs to extend  memory via byte-addressable SerDes links. Extended memory via CXL naturally forms complex heterogeneous memory systems. 
% On the one hand, CXL provides load-store semantic to diverse memory types, which may show different latency and bandwidth characteristics. 
% On the other hand, with CXL switch, the CXL memory can be linked to CPU with different hops, rendering different CPU memory access latency. To optimize such a heterogeneous memory system's performance, tiered memory management techniques are widely used. 
% However, due to lack of real world CXL enabled CPU and CXL devices, previous works use remote NUMA node emulation or software simulation to evaluate their performance in the CXL scenario. 
% The current NUMA emulation-based solutions struggle to explore performance under varying latency ratios and are also difficult to apply to hardware-based heterogeneous memory management techniques, while the software simulation-based approaches struggle to handle large workloads and are also challenged when considering the performance of server CPUs at real-scale,  due to their low simulation speeds.
%the current memory tiering technologies face numerous challenges in performance evaluation in the CXL scenario.
%Firstly, the current NUMA emulation-based solutions struggle to explore performance under varying latency ratios and are also difficult to apply to hardware-based heterogeneous memory management techniques. Secondly, software simulation-based approaches, due to their lower simulation speeds, struggle to handle large workloads and are also challenged when considering the performance of server CPUs at real-scale. \zhe{1.Challenge of software-managed h-mem. 2.Challenge of current hardware-managed h-mem. 3. Challenge of simulation}
% However, previous works using two-socket NUMA nodes to enumerate CXL memory fail to characterize the heterogeneity of CXL memory system. 
The Compute Express Link (CXL) technology facilitates the extension of CPU memory through byte-addressable SerDes links and cascaded switches, creating complex heterogeneous memory systems where CPU access to various endpoints differs in latency and bandwidth. Effective tiered memory management is essential for optimizing system performance in such systems. 
% However, existing evaluation methods, including NUMA-based emulation and full-system simulations like GEM5, face limitations in evaluating hardware-based tiered memory management solutions and handling real workloads at scale. A practical, high-fidelity method for  exploring memory tiering solutions is urgently needed.
However, designing an effective memory tiering system for CXL-extended heterogeneous memory faces challenges: 
1) Existing evaluation methods, such as NUMA-based emulation and full-system simulations like GEM5, are limited in assessing hardware-based tiered memory management solutions and handling real-world workloads at scale.
2) Previous memory tiering systems struggle to simultaneously achieve high resolution, low overhead, and high flexibility and compatibility.

% the lack of real-world CXL-enabled CPUs and devices has led 
% previous studies to rely on remote NUMA node emulation or software simulations to evaluate performance in CXL scenarios. NUMA emulation-based solutions struggle to explore performance under varying latency ratios and are challenging to apply to hardware-managed memory tiering systems. Meanwhile, software simulation approaches face difficulties in handling large workloads and assessing server CPU performance at real-scale due to low simulation speeds.

% In this paper, we propose HeteroBox, an emulation platform on real CXL-enabled FPGA to emulate the performance of the heterogeneous memory system extended by CXL link. HeteroBox can be configured to expose a memory space with arbitrary number of regions with different latency attribute to host CPU. 
% Based on the HeteroBox emulation platform, we explore the performance of software-managed and hardware-managed memory tiering system. We found that: (1) Software-managed memory tiering systems suffer from inefficient and inaccurate memory access profiling mechanism and inefficient data movement mechanism. (2) Current hardware-managed memory tiering systems make intrusive modification in CPU, which make them fail to adapt to various CXL extended memory tiering system configure. 
% We propose HeteroMem, a device-side hardware-managed memory tiering system which support multi-tiers. HeteroMem builds an abstraction layer between CPU and device memory, efficiently profiling data hotness and migrating data to fast memory tiers, hiding the heterogeneity at device side from CPU. HeteroMem is totally transparent to CPU and is runtime configurable, which is highly flexible. We build a FPGA prototype of HeteroMem, and the evaluation result shows that, in a set of real world application, HeteroMem achieves 11.8\%-24.3\% improvement compared to several existing memory tiering solutions.

% In this study, we first introduce HeteroBox, a configurable emulation platform leveraging real CXL-enabled FPGAs to emulate the performance of various CXL memory architectures. HeteroBox allows configuration of a memory space with multiple regions, each exhibiting distinct latency and bandwidth characteristics for the host CPU. Using HeteroBox, we assess the performance of both software-managed and hardware-managed memory tiering systems. Building on HeteroBox, we further propose HeteroMem, a hardware-managed memory tiering system that operates on the device side. HeteroMem creates an abstraction layer between the CPU and device memory, effectively monitoring data usage and transferring data to faster memory tiers, thus masking device-side heterogeneity from the CPU. 
% % It is transparent to the CPU and offers runtime configurability for high flexibility. 
% % We have developed an FPGA prototype of HeteroMem, and 
% Evaluations on real-world applications demonstrate that HeteroMem enhances performance by 5.7\% to 17.6\% over existing memory tiering solutions.

In this study, we first introduce HeteroBox, a configurable emulation platform that leverages real CXL-enabled FPGAs to emulate the performance of various CXL memory architectures. HeteroBox allows one to configure a memory space with multiple regions, each exhibiting distinct CPU-access latency and bandwidth.  HeteroBox helps assess the performance of both software-managed and hardware-managed memory tiering systems with high efficiency and fidelity. Based on HeteroBox, we further propose HeteroMem\footnote{The code of HeteroBox and HeteroMem will be released after publication.}, a hardware-managed memory tiering system that operates on the device side. HeteroMem creates an abstraction layer between the CPU and device memory, effectively monitoring data usage and migrating data to faster memory tiers, thus hiding device-side heterogeneity from the CPU. 
% Evaluations using real-world applications demonstrate that HeteroMem enhances performance by 5.1\% $\sim$ 16.2\% over existing memory tiering solutions. 
Evaluations with real-world applications show that HeteroMem delivers high performance while keeping heterogeneous memory management fully transparent to the CPU, achieving a 5.1\% to 16.2\% performance improvement over existing memory tiering solutions. 


%In this paper, we propose HeteroMem, a novel device side hardware managed heterogeneous memory system. HeteroMem provides CPU with an abstraction that the memory extended with CXL has a uniform, average low latency, which is done through adding a abstraction layer between CPU and device. 
%We build an enumeration platform on real CXL-enabled FPGA to enumerate the performance of the heterogeneous memory system extended by CXL link. Based on the enumeration platform, we build a prototype of HeterBox on the CXL-enabled FPGA platform. 
%The evaluation result shows that, in a set of real world application, HeterBox achieves 11.8\%-24.3\% improvement compared to other state of the art memory tiering solutions. 



% The Compute Express Link (CXL) interconnect has provided load-store semantic to diverse memory types via byte-addressable SerDes links. The equal memory semantic shed light on promising media types other than DRAM, such as Managed DRAM, ReRam, 3DXP/Optane, which features large capicity. However, these new medias suffer from high latency and cause significant performance slow down. To address this problem, we propose HeterBox, an CPU-transparent solution which combine slow large capicity media and a small fraction of dram in the same CXL device. HeterBox actively catch the memory accesses from CPU and migrate them to device dram with small overhead. 
% % As a result, HeterBox provides CPU an illusion of near dram latency with much lower cost. 
% We build a prototype of HeterBox on real CXL-enabled FPGA platform. The evaluation result shows that, in a set of real world application, HeterBox achieves XXX-XXX performance improvement compared to slow-media-only configuration, and achieves XXX-XXX improvement compared to other state of the art memory tiering solutions. 

%   This document is intended to serve as a sample for submissions to the 56\textsuperscript{th} IEEE/ACM International Symposium on Microarchitecture\textsuperscript{\textregistered} (MICRO 2023). We provide some guidelines that authors should follow when submitting papers to the conference.  This format is derived from the ACM sig-alternate.cls file, and is used with an objective of keeping the submission version similar to the camera-ready version.

\blfootnote{* Co-first authors.}
\blfootnote{$\dagger$ Corresponding author.}